%%
%% This is file `sample-sigconf-authordraft.tex',
%% generated with the docstrip utility.
%%
%% The original source files were:
%%
%% samples.dtx  (with options: `all,proceedings,bibtex,authordraft')
%% 
%% IMPORTANT NOTICE:
%% 
%% For the copyright see the source file.
%% 
%% Any modified versions of this file must be renamed
%% with new filenames distinct from sample-sigconf-authordraft.tex.
%% 
%% For distribution of the original source see the terms
%% for copying and modification in the file samples.dtx.
%% 
%% This generated file may be distributed as long as the
%% original source files, as listed above, are part of the
%% same distribution. (The sources need not necessarily be
%% in the same archive or directory.)
%%
%%
%% Commands for TeXCount
%TC:macro \cite [option:text,text]
%TC:macro \citep [option:text,text]
%TC:macro \citet [option:text,text]
%TC:envir table 0 1
%TC:envir table* 0 1
%TC:envir tabular [ignore] word
%TC:envir displaymath 0 word
%TC:envir math 0 word
%TC:envir comment 0 0
%%
%%
%% The first command in your LaTeX source must be the \documentclass
%% command.
%%
%% For submission and review of your manuscript please change the
%% command to \documentclass[manuscript, screen, review]{acmart}.
%%
%% When submitting camera ready or to TAPS, please change the command
%% to \documentclass[sigconf]{acmart} or whichever template is required
%% for your publication.
%%
%%
\documentclass[sigconf, nonacm]{acmart}
%\documentclass[sigconf]{acmart}
%\documentclass[sigconf,authordraft]{acmart}
%\documentclass[anonymous,manuscript,review]{acmart}

%%
%% \BibTeX command to typeset BibTeX logo in the docs
\AtBeginDocument{%
  \providecommand\BibTeX{{%
    Bib\TeX}}}

%% Rights management information.  This information is sent to you
%% when you complete the rights form.  These commands have SAMPLE
%% values in them; it is your responsibility as an author to replace
%% the commands and values with those provided to you when you
%% complete the rights form.
%\setcopyright{acmlicensed}
\copyrightyear{2025}
\acmYear{2025}
%\setcopyright{cc}
\setcctype{by}
\acmConference[CHI '25]{CHI Conference on Human Factors in Computing Systems}{April 26-May 1, 2025}{Yokohama, Japan}
\acmBooktitle{CHI Conference on Human Factors in Computing Systems (CHI '25), April 26-May 1, 2025, Yokohama, Japan}
\acmDOI{10.1145/3706598.3713353}
\acmISBN{979-8-4007-1394-1/25/04}


%%
%% Submission ID.
%% Use this when submitting an article to a sponsored event. You'll
%% receive a unique submission ID from the organizers
%% of the event, and this ID should be used as the parameter to this command.
%%\acmSubmissionID{123-A56-BU3}

%%
%% For managing citations, it is recommended to use bibliography
%% files in BibTeX format.
%%
%% You can then either use BibTeX with the ACM-Reference-Format style,
%% or BibLaTeX with the acmnumeric or acmauthoryear sytles, that include
%% support for advanced citation of software artefact from the
%% biblatex-software package, also separately available on CTAN.
%%
%% Look at the sample-*-biblatex.tex files for templates showcasing
%% the biblatex styles.
%%

%%
%% The majority of ACM publications use numbered citations and
%% references.  The command \citestyle{authoryear} switches to the
%% "author year" style.
%%
%% If you are preparing content for an event
%% sponsored by ACM SIGGRAPH, you must use the "author year" style of
%% citations and references.
%% Uncommenting
%% the next command will enable that style.
%%\citestyle{acmauthoryear}
\usepackage{booktabs}
\usepackage{graphicx}
\usepackage{listings}
\usepackage{xcolor} 
\usepackage{caption}
%\usepackage{paralist}
\usepackage{makecell}
\usepackage{url}
\usepackage{longtable}
\usepackage{color}
\usepackage{tikz}
%\usepackage[utf8]{enc}
\usepackage[T1]{fontenc}
\usepackage{lipsum}
\usepackage{stfloats}
\usepackage{tabu} 
\usepackage{tabularx}
\usepackage{multirow}
\usepackage{booktabs}
\usepackage{graphicx}
\usepackage{wrapfig}
\usepackage{hyperref}   
\usepackage{cleveref}
%\usepackage[most]{tcolorbox} 
\usepackage{xcolor}
\usepackage{float}
\usepackage{listings} 
%\usepackage{tcolorbox}




\definecolor{PurpleColor}{RGB}{0,0,0}
\newcommand{\RR}[1]{{\color{PurpleColor}#1}}

\definecolor{PinkColor}{RGB}{0, 0, 0}
\newcommand{\RA}[1]{{\color{PinkColor}#1}}


\definecolor{codegreen}{rgb}{0,0.6,0}
\definecolor{codegray}{rgb}{0.5,0.5,0.5}
\definecolor{codepurple}{rgb}{0.58,0,0.82}
\definecolor{backcolour}{rgb}{0.95,0.95,0.92}

\lstdefinestyle{mystyle}{
  backgroundcolor=\color{backcolour}, commentstyle=\color{codegreen},
  keywordstyle=\color{magenta},
  numberstyle=\tiny\color{codegray},
  stringstyle=\color{codepurple},
  basicstyle=\ttfamily\footnotesize,
  breakatwhitespace=false,         
  breaklines=true,                 
  captionpos=b,                    
  keepspaces=true,                 
  numbers=left,                    
  numbersep=5pt,                  
  showspaces=false,                
  showstringspaces=false,
  showtabs=false,                  
  tabsize=2
}
\lstset{style=mystyle}


%\definecolor{PurpleColor}{RGB}{0,0,0}

%%
%% end of the preamble, start of the body of the document source.
\begin{document}

%%
%% The "title" command has an optional parameter,
%% allowing the author to define a "short title" to be used in page headers.
\title{CPVis: Evidence-based Multimodal Learning Analytics for Evaluation in Collaborative Programming}

%%
%% The "author" command and its associated commands are used to define
%% the authors and their affiliations.
%% Of note is the shared affiliation of the first two authors, and the
%% "authornote" and "authornotemark" commands
%% used to denote shared contribution to the research.
\author{Gefei Zhang}
\affiliation{%
  \institution{Zhejiang University of Technology}
  \city{Hangzhou}
  \state{Zhejiang}
  \country{China}}
\email{gefei@zjut.edu.cn}
\orcid{1234-5678-9012}

\author{Shenming Ji}
\affiliation{%
  \institution{Xi'an Jiaotong-Liverpool University}
  \city{Suzhou}
  \state{Jiangsu}
  \country{China}}
\email{shenming.ji21@student.xjtlu.edu.cn}

\author{Yicao Li}
\affiliation{%
  \institution{Zhejiang University of Technology}
  \city{Hangzhou}
  \state{Zhejiang}
  \country{China}}
\email{yicaoli47@gmail.com}

\author{Jingwei Tang}
\affiliation{%
  \institution{Zhejiang University of Technology}
  \city{Hangzhou}
  \state{Zhejiang}
  \country{China}}
\email{jwtang@zjut.edu.cn}

\author{Jihong Ding}
\affiliation{%
 \institution{Hannan University}
 \city{Haikou}
 \state{Hainan}
 \country{China}}
\email{jhding@hainanu.edu.cn}

\author{Meng Xia}
\affiliation{%
  \institution{Texas A\&M University}
  \city{College Station}
  \state{Texas}
  \country{USA}}
\email{mengxia@tamu.edu}

\author{Guodao Sun}
\affiliation{%
  \institution{Zhejiang University of Technology}
  \city{Hangzhou}
  \state{Zhejiang}
  \country{China}}
\email{guodao@zjut.edu.cn}

\author{Ronghua Liang}
\affiliation{%
  \institution{Zhejiang University of Technology}
  \city{Hangzhou}
  \state{Zhejiang}
  \country{China}}
\email{rhliang@zjut.edu.cn}



%%
%% By default, the full list of authors will be used in the page
%% headers. Often, this list is too long, and will overlap
%% other information printed in the page headers. This command allows
%% the author to define a more concise list
%% of authors' names for this purpose.
\renewcommand{\shortauthors}{Zhang et al.}

%%
%% The abstract is a short summary of the work to be presented in the
%% article.
\begin{abstract}
As programming education becomes more widespread, many college students from non-computer science backgrounds begin learning programming. Collaborative programming emerges as an effective method for instructors to support novice students in developing coding and teamwork abilities. However, due to limited class time and attention, instructors face challenges in monitoring and evaluating the progress and performance of groups or individuals. To address this issue, we collect multimodal data from real-world settings and develop \textit{CPVis}, \RR{an interactive visual analytics system designed to assess student collaboration dynamically.} Specifically, \textit{CPVis} enables instructors to evaluate both group and individual performance efficiently. \textit{CPVis} employs a novel flower-based visual encoding to represent performance and provides time-based views to capture the evolution of collaborative behaviors. A within-subject experiment (N=22), \RR{comparing \textit{CPVis} with two baseline systems, reveals that users gain more insights, find the visualization more intuitive, and report increased confidence in their assessments of collaboration.}
\end{abstract}

%%
%% The code below is generated by the tool at http://dl.acm.org/ccs.cfm.
%% Please copy and paste the code instead of the example below.
%%

\begin{CCSXML}
<ccs2012>
   <concept>
       <concept_id>10010405.10010489.10010492</concept_id>
       <concept_desc>Applied computing~Collaborative learning</concept_desc>
       <concept_significance>500</concept_significance>
       </concept>
 </ccs2012>
\end{CCSXML}

\begin{CCSXML}
<ccs2012>
   <concept>
       <concept_id>10003120.10003145.10003151</concept_id>
       <concept_desc>Human-centered computing~Visualization systems and tools</concept_desc>
       <concept_significance>500</concept_significance>
       </concept>
 </ccs2012>
\end{CCSXML}

\ccsdesc[500]{Human-centered computing~Visualization systems and tools}

\ccsdesc[500]{Applied computing~Collaborative learning}


\begin{comment}



\end{comment}
%%
%% Keywords. The author(s) should pick words that accurately describe
%% the work being presented. Separate the keywords with commas.
\keywords{Group visualization, education visualization, collaborative programming}
%% A "teaser" image appears between the author and affiliation
%% information and the body of the document, and typically spans the
%% page.








% \received{20 February 2007}
% \received[revised]{12 March 2009}
% \received[accepted]{5 June 2009}

%%
%% This command processes the author and affiliation and title
%% information and builds the first part of the formatted document.
\maketitle
%\newtcbox{\inlinecode}{on line, colback=gray!10, colframe=gray!10, boxrule=0.1mm, 
%	rounded corners, fontupper=\bfseries, left=0.1mm, right=0.1mm, top=0.1mm, bottom=0.1mm}

\documentclass[../main.tex]{subfiles}
\graphicspath{{../images/}}
\makeatletter
\def\input@path{{../images/}}
\makeatother
\begin{document}
\section{Introduction}
\begin{figure}
\centering
\begin{tikzpicture}
\node[inner sep=0pt] (ws) at (0, 0) {
\includegraphics[height=.4\textwidth, trim={10cm 0 10cm 0},clip]{world_space.png}};
\node[inner sep=0pt] (cs) at (6,0) {\includegraphics[height=.4\textwidth, trim={10cm 1cm 10cm 4cm},clip]{conf_space.png}};
\end{tikzpicture}
\vspace{-5pt}
\label{fig:pbrm_intro}
\caption{\textbf{Left}: Shows world space obstacles as grey spheres. Robots start and goal configuration is colored red and green, respectively. Configurations along the computed path are colored transparent blue. \textbf{Right:} Mapped world space scenario to configuration space. Obstacle region is the grey mesh. Red spheres are collision-free regions computed by the neural SCDF. The optimized shortest path in the convex corridor is the blue curve.}
\vspace{-25pt}
\end{figure}
Motion planning is the problem of finding a collision-free trajectory that connects a given start and goal configuration. The planning takes place in the configuration space of the robot. For single body robots, like mobile robots or drones, the configuration space and the world space are usually the same. This simplifies the planning, since explicit obstacle representations are available which enables geometrical tools like separating hyperplanes, smallest distance to obstacles etc., to be used when designing motion planning algorithms. For multi-body robots like manipulators, the situation is completely different. The world space obstacles are usually mapped to non-convex regions, and to make the problem even harder, the mapping is usually not known. Forming explicit representations of the obstacle region in the configuration space is usually too expensive or intractable. Despite all of this, sampling based planners are used with great success, which mainly is due to their use of implicit representations of the obstacle region. The basic idea is to construct a graph in the configuration space that covers and connects the collision-free region. From this graph, a path can be extracted that connects a given start and goal configuration. The approach is computationally expensive, since the graph is constructed with the smallest geometrical building block available, points, which represents a collision-check. Furthermore, the extracted paths from the graph are non-smooth and jagged due to the stochastic nature of the approach. This adds an additional post-processing step to the process, where the paths are shortcutted and smoothened, before the path can be used for tracking. Clearly a lot of time is invested to form this graph and produce smooth paths. Thus, if the obstacles start to move, then all of this work is done in no use, since all points that make up this graph need to be re-verified, which is simply too time consuming to be done in real time.
\\\\
In this work, we want to address the existing drawbacks of the sampling based planners. Our main contribution is an improved motion planner where each vertex in the graph covers a collision-free region in the form of a sphere instead of a point and where the edges are formed with neighboring intersecting spheres. This representation has the advantage of instead of returning piecewise linear paths, returning a sequence of overlapping spheres, i.e. a convex corridor, that connects a given start and goal configuration, illustrated in Figure \ref{fig:pbrm_intro}. This convex corridor allows us to use convex optimization to produce smooth trajectories, instead of computationally expensive post-processing methods. The representation further allows us to estimate the coverage of the collision-free space, which gives us awareness and feedback in the offline roadmap construction phase. Finally, our representation is simple to adapt to moving obstacles, simply requery for the new radii and recheck for intersections. 
\\\\
The spherical collision-free regions are formed using a signed distance function (SDF), which is a function that returns the smallest distance from an arbitrary point to the boundary of an obstacle. As the name implies, the distance is signed, thus if the point is inside the obstacle it is negative otherwise positive. If the distance is positive, a sphere with radius equal to the distance is guaranteed to cover a collision-free region. Using an SDF in motion planning is not new, but what is novel about our approach is that we express the distance in the configuration space instead of the world space and by doing so allows us to form these convex collision-free regions. We refer to the resulting SDF as a signed configuration distance function (SCDF). Computing an SCDF analytically is non-trivial, our approach is therefore to parameterize the SCDF with a deep neural network and learn the mapping by supervised learning. Our resulting neural SCDF can compute distances for different parameter values of obstacle shapes and we also show how multiple distances can be combined, thus making our approach flexible.
\section{Related work}
Motion planning algorithms can roughly be divided into three families, grid-based, sampling based and optimization based methods. Grid-based methods (GBM) discretize the planning space from which a graph is then compiled. A standard search method is A$^\star$ \citep{a_star}, which is classified as an \textit{informed} search method, since it employs a heuristic function to speed up the search. A$^\star$ guarantees to return an optimal path at the level of discretization used. GBMs usually discretize the planning space by a regular lattice and this limits the GBMs to problems with low dimensionality due to the curse of dimensionality. Thus, GBMs are usually limited to single-body robots where the degrees of freedom (DOF) are low. To overcome the inherent scaling problem with the GBMs, stochastic methods are usually used for multi-body robots. These methods are termed as sampling-based methods (SBM) and core members within this family are the rapidly-exploring random trees (RRT) \citep{rrt} and the probabilistic roadmap (PRM) \citep{prm}. RRT grows a tree from the start configuration and explores the collision-free region in a rapid way until it is able to connect to the goal region. RRT is usually improved by bi-directional planning \citep{rrt_connect}, i.e. an additional tree is grown from the goal configuration and the trees are tested for connection after any tree has been expanded. RRT is a single-query method, thus it searches for a path from scratch each time it is queried. Contrary to this, PRM is a multi-query method, which solves for multiple queries without starting from scratch. PRM does this by creating a roadmap (graph) that covers the collision-free space as an offline step. The graph is then used to solve for multiple queries. PRMs are used in cases where the environment does not change since the extra offline step is too computationally costly and needs to be re-done if the environment is changed. In our work, we address this inherent issue by using a different roadmap representation. Our vertices in the graph cover a collision-free region in the form of spheres and we form the edges by checking for intersecting spheres. If something in the environment changes, we recompute the spheres radii and recheck the intersections, without relying on collision detection. We use a trained neural network to compute the sphere radius, therefore querying for the radius can be done fast, hence our representation enables the PRM for dynamic environments.
\\\\
In the recent decades, optimization based methods (OBM) \citep{chomp, schulman, itomp, stomp} have been introduced as an alternative to SBM for multi-body robots. Like the SBM, the OBMs scale well to higher dimensional problems and produce smoother motion. It is common to use a SDF in the optimization since it is a smooth function, thus enabling gradient-based methods. However, the standard way of expressing the SDF is in world space. The distance therefore needs to be mapped to the configuration space by the forward kinematics. This mapping makes the optimization problem a non-linear program (NLP), which is computationally expensive to solve. Recently, a different approach has been proposed. In \cite{mp_gcs} motion planning is formulated as a convex optimization problem by using the graph of convex sets framework \citep{gcs}. The underlying idea is to decompose the collision-free space into intersecting convex sets from which a convex optimization problem is formulated. In cases where an explicit representation of the obstacles in the configuration space exists, like for single-body robots, creating collision-free convex regions can be done fast \citep{iris}. For multi-body robots, this is non-trivial. Existing work does this successfully \citep{iris_nlp, iris_c} by an optimization based approach, but the methods are still too time consuming to be used in the presence of moving obstacles. Our approach is instead to use deep learning to learn an SDF expressed in the configuration space. With this, we can query for shortest distances to the collision boundary, which allows us to expand spherical regions which are collision-free. Our approach is fast and therefore enables our suggested roadmap planner to be used in dynamic environments.
\\\\
Recent research has focused on learning collision detection \citep{fk_kernel_distance, diffco, graphdistnet} by predicting the signed distance between the robot links and the surrounding obstacles in the world space. The learned SDF is used in trajectory optimization but since the distance is expressed in the world space, the problem becomes an NLP and therefore takes a long time to solve. We take a novel approach and suggest to instead express the signed distance in the configuration space. This allows us to improve the PRM at the same time as it enables convex optimization for trajectory optimization, which runs faster and is more reliable than NLP solvers. In \cite{cspf} a learned signed distance function in the configuration space is proposed similar to our approach. However, their approach is restricted to point cloud representations, while we propose to represent the obstacles as parameterized geometric shapes, e.g. spheres. Furthermore, we also show how to use our learned SCDF to improve an existing roadmap planner.
\section{Problem formulation}
A robot is located in the world space, $\W \subset \R^3 $. The unique location of the robot is given by its configuration $\q \in \C$, where $\C$ is the configuration space. The set of points covered by the robots bodies at a certain configuration is expressed as $\B(\q) \subset \W$. The robot is surrounded by $\NrObst$ obstacles $\O = \bigcup_{i=1}^{\NrObst} \O_i$, where  $\O_i \subset \W$. The representation of the obstacle in the configuration space is the set $\C\O_i = \{\q \in \C \: |\: \B(\q) \cap \O_i \neq \emptyset \}$. The obstacle space is formed as $\Co = \bigcup_{i=1}^{\NrObst} \C \O_i$. The complement is referred to as the free space, $\Cf = \C \setminus \Co$. The path planning problem is a tuple, ($\Cf$, $\qStart$, $\qGoal$), where we want to connect a query pair, consisting of a start, $\qStart$, and goal configuration, $\qGoal$, with a geometric path, $\q(s): [0, 1] \mapsto \Cf$, such that $\q(0)=\qStart$ and $\q(1)=\qGoal$, or report correctly when such a path does not exist.
\end{document}


\section{Related Work} \label{sec:related}

% \textbf{Adversarial Attack}
\textbf{Attacks on SLAM.} 
%With the rise of machine learning, 
The robustness of computer vision systems is being actively investigated. With the emergence of adversarial images in the digital domain by adding optimized noise directly to images~\cite{szegedy2013intriguing,carlini2017towards}, researchers find that such attacks also exist physically in the real world \cite{eykholt2018robust,song2018physical,zhao2019seeing}. To fill the gap between attacks in the digital and physical worlds, recent studies have demonstrated that attacks on real-world computer vision systems are practical \cite{eykholt2018robust,li2019adversarial,man2020ghostimage,sharif2016accessorize,zhao2019seeing,zhou2018invisible}. However, attacks on traditional computer vision methods such as SLAM are relatively less explored. \cite{yoshida2022adversarial} proposes an attack against the scan matching algorithm in LiDAR-based SLAM, while most SLAMs in AR/VR devices rely on different sensors like RGB/depth cameras and IMUs. \cite{ikram2022perceptual} and \cite{chen2024adversary} mislead visual SLAM by poisoning the images with special patterns, and \cite{wang2021can} causes the camera to fail using infrared light. In our work, we demonstrate attacks on Visual-Inertial SLAM (VI-SLAM) by perturbing the IMU readings, rather than cameras, and showing its impact on XR user experience. 

\textbf{Acoustic Injection Attacks.} Among various physical attacks, acoustic injection attacks are attractive due to their low cost. Son~\etal~\cite{son2015rocking} were the first to introduce acoustic attacks on MEMS gyroscopes, demonstrating how these attacks could lead to sensor denial-of-service and result in drone crashes. WALNUT~\cite{trippel2017walnut} expanded on this by developing output biasing and control attacks that enable precise manipulation of MEMS accelerometer outputs using modulated sound waves. Wang et al.~\cite{wang2017sonic} demonstrated a sonic gun, showcasing the vulnerability of various smart devices (\eg drones and self-balancing vehicles) to acoustic attacks. Tu et al. \cite{tu2018injected} designed side-swing and switching attacks to alter the outputs of MEMS gyroscopes and accelerometers. Furthermore, Ji et al. \cite{ji2021poltergeist} fool the object detectors by applying acoustic attack to the image stabilizers commonly used in modern cameras. However, none of the existing works study the relationship between the acoustic injections and SLAM outputs on recent XR devices. 

% \zijian{Do we need one session about security in AR/VR?}
% \yicheng{TODO}
%\jiasi{cite the AIVR paper (UMass Amherst?) paper is we have not already. They add IMU perturbation but w/o SLAM, iirc} \yicheng{Cited}

\textbf{XR Security and Privacy.} 
%Security and privacy concerns in XR systems have gained significant attention. 
For single-user XR systems, researchers have demonstrated various side-channel attacks to extract sensitive information (\eg keystrokes) through video feeds~\cite{ling2019know}, head movements~\cite{nair2023unique, slocum2023going}, architectural hints~\cite{zhang2023its,shang2020arspy}, power usage~\cite{li2024dangers}, and EM side-channel leakages~\cite{al2021vr}. In multi-user XR systems, Su et al.~\cite{su2024remote} use avatar motion data to infer keystrokes in shared VR environments. Slocum et al.~\cite{slocum2024doesn} reveal vulnerabilities in the shared state frameworks of multi-user AR. Similarly, Lebeck et al.~\cite{lebeck2017securing} highlight risks like deceptive virtual objects and emphasize access control for managing shared physical and virtual spaces. Ruth et al.~\cite{ruth2019secure} further propose a secure multi-user AR framework focusing on content sharing and permissions.
Chandio et al.~\cite{chandio2024stealthy} %introduced a multi-modal spatiotemporal attack that 
simultaneously manipulated visual and inertial sensors to disrupt XR pose estimation. However, their study evaluated the attack using offline datasets and assumed the attacker's capability to manipulate IMU data streams through acoustic means, without real experiments. Ours is the first to demonstrate acoustic injection attacks on recent XR devices, like the Hololens 2, in the real world.
 


\section{Formative Study} 
\label{sec:formative}
We conducted a formative study to explore instructors' challenges in collaborative programming and their visualization needs, with the study protocol approved by our university's Institutional Review Board (IRB).
 
 \subsection{Participants}
We recruited ten participants (five females, age: \(30.6 \pm 6.6\)) with collaborative programming experience, divided into three groups: two educational technology experts (E1, E2) and eight instructors (T1–T8) with an average of 8.14 years of teaching programming. Participants, recruited via snowball sampling from the authors' network, received \$20 as compensation.
 

\subsection{Procedure}\
\RR{The formative study used semi-structured interviews conducted via Zoom, divided into two sections: (A) Questions and Answers and (B) Ratings.}
\textit{A: Questions and Answers.} 
Each participant was independently interviewed to explore the need for collaborative programming analysis. Topics included teaching experience with collaborative programming, class organization, challenges faced, assessment methods, and group and individual performance evaluations. Follow-up questions were asked for clarification or more profound insights. Each session lasted 40–60 minutes and was documented through written notes, audio, and video recordings.
\textit{B: Rating.} 
Participants then rated two aspects of the collaborative programming visual analytics system. They rated feature importance (Q1) on a 1–7 scale (7 being the highest) and ranked features by priority (Q2). 
\RR{More details are in appendix A.}

 
\subsection{Findings} 

\begin{figure*}[htbp]
	\centering
	\includegraphics[width=1\linewidth]{scheme1.pdf}
	\caption{Collaborative programming coding schemes, along with their definitions and examples.}
        \Description{Collaborative programming coding schemes, along with their definitions and examples.}
	\label{fig:scheme}
\end{figure*}



\subsubsection{Evaluation of Groups:}
In large classes, instructors struggled to monitor each group without the help of teaching assistants. They often relied on group presentations (E2, T1, T5), but it was inaccurate. Limited class time forced instructors to focus on solving issues rather than actively monitoring groups.


\textbf{Group Performance:}
Evaluating group work was based on the final code rather than the process, limiting the ability to provide feedback (T7). Some instructors used presentations or technical documentation to streamline assessments (T6). However, evaluating more than 20 groups was still challenging, and instructor assistance was often overlooked during evaluations.


\textbf{Collaboration Evaluation:}
Effective collaboration wasn't just about task completion speed and group dynamics. Some groups (T7) completed tasks quickly due to one member's efforts, not true collaboration. 
\RR{E1 proposed the collaborative problem-solving framework to distinguish task effectiveness from team effectiveness.} Instructors (T6) believed monitoring discussion time helped assess collaboration quality and task difficulty, but off-topic discussions made it hard to evaluate group discussions (E1, T3, T5).


\subsubsection{Evaluation of Students:}

In large classes, assessing individual contributions in group work was challenging. Instructors relied on peer evaluations (T2, T3, T4) and self-reported task distributions (E2), which were often subjective.

\textbf{Individual Performance:}
Instructors typically reviewed code to assess understanding, but measuring individual contributions in group work was hard. Leadership roles often reflected a deeper grasp of concepts (T1), but tracking individual engagement was difficult in large classes.


\textbf{Personalized Feedback:}
Providing personalized feedback was difficult, as group results often masked individual struggles. T3 and T4 noted that group collaboration fostered peer learning but could lead to less engagement from weaker students.
\textit{T8 added that offering personalized feedback in large classes was time-consuming and burdensome.}

 
 
 
\subsection{Design requirement}
\label{dr}
Based on the interview findings, we identified six design requirements (R1–R6) across three levels, summarized as follows:

Support \textbf{\textit{inter-group-level}} to provide a macro perspective, enabling instructors to observe the overall situation of all groups comprehensively and fully understand class-wide dynamics.

\textbf{R1: Displaying the Overall Performance of all Groups.} 
Instructors face challenges in supervising multiple groups simultaneously and shifting focus efficiently. Participants stressed the need for an overview of group performance, allowing instructors to grasp class dynamics and selectively review specific groups.


\RR{\textbf{R2: Comparing Similar and Different Groups.} 
Instructors often compare students' performance to assess their relative standing within the class~\cite{marsh1997making}. Such comparisons enable a more accurate evaluation of group performance and help identify groups excelling or encountering challenges.}


Supporting \textbf{\textit{intra-group-level}} visual exploration to offer a meso perspective, enabling instructors to observe specific groups' performance and gain a comprehensive understanding of group dynamics during the collaborative programming process.


\textbf{ R3: Understand the Dynamics of Programming Problem Solving.}
Analyzing a group's evolving communication patterns and computational thinking during programming tasks provides instructors with deeper insights into students' progress and intermediate learning objectives—details missed in final submissions alone.


\textbf{ R4: Identify Teacher Scaffolding in Collaboration.}
Instructors play a vital role in guiding groups during collaborative programming. Understanding the scaffolding provided and students' responses can help refine instructional strategies, improving the overall effectiveness of collaborative programming.


Supporting \textbf{\textit{individual-student-level}} visual exploration to provide a micro perspective allows instructors to observe each student's performance within a specific group and better understand their role and collaboration.

\textbf{R5: Track Changes in Student Engagement Over Time.} 
Limited classroom time makes it challenging for instructors to monitor individual student engagement in programming tasks.
Tracking and visualizing engagement trends is essential for assessing performance and refining instructional practices.

\textbf{R6: Access Detailed Raw Data.} 
Instructors require access to raw data, such as collaboration videos, conversations, and background information. These details are crucial for validating analysis results and supporting personalized feedback and assessments.


\begin{figure*}
    \centering
    \includegraphics[width=1\linewidth]{bar2.pdf}
    \caption{(a) shows the bar chart of the raw data, (b) presents the results of applying Moving Average Smoothing to reduce anomalies in prediction percentages, and (c) highlights the reduction of visual clutter and emphasizes sequential behavior patterns after merging behaviors of the same category.}
    \label{fig:bar}
    \Description{(a) shows the bar chart of the raw data, (b) presents the results of applying Moving Average Smoothing to reduce anomalies in prediction percentages, and (c) highlights the reduction of visual clutter and emphasizes sequential behavior patterns after merging behaviors of the same category.}
\end{figure*}

\section{Data Collection and Processing}
\label{sec:data}
\RR{In this section, we provided an overview of the data collection context and introduced the collaborative programming performance framework along with its metric quantification methods.}

\subsection{Data Collection}
We collaborated with Professor E1, an expert in programming education, and teaching assistants (TA1 and TA2), experienced in Python, to collect data from E1's Spring 2023 Python course with 66 non-computer science freshmen in 22 groups. Using non-intrusive methods, we recorded group discussions, screen activities (without audio), and code submissions. Session lengths ranged from 10 to 60 minutes based on question completion. 
Due to data quality issues, we selected data from 19 groups (57 students) for analysis.


\subsection{Data Preprocessing}
In collaborative programming analysis, students' spoken content was key to understanding discussion and evaluating collaboration. We used the Faster-Whisper model~\cite{fasterwhisper} for speech recognition and the Pyannote-audio model~\cite{pyannoteaudio} for speaker diarization. 
For groups lacking clear problem-solving strategies, we used Tesseract OCR~\cite{tesseract} to analyze screen recordings and extract key frames through screenshots.

\subsection{Scope of Collaborative Programming Performance Framework}
Evaluating student and group performance in collaborative programming required considering multiple dimensions~\cite{hawlitschek2023empirical}.  
Building on literature and expert input (E1), we proposed the following comprehensive analytical framework to assess performance. 



\subsubsection{Student Performance Assessment}
\label{shema}
Previous research demonstrated that students' skills, backgrounds, and personalities in the classroom vary significantly, affecting their engagement and learning outcomes~\cite{wu2019analysing}. 
Therefore, we focus on each student's \textit{background} (prior academic performance and major), \textit{role transitions}, \textit{behavioral engagement}, and \textit{cognitive engagement}.






\textbf{Problem-solving Categorization:}
Based on previous frameworks~\cite{wu2019analysing}, team theory~\cite{zhao2023analysing}, and collaborative coding processes~\cite{sun2021three}, we developed a coding scheme (Fig.~\ref{fig:scheme}) to capture group problem-solving in collaborative programming. 
The scheme used four color-coded categories to represent discussion types. 
The first three categories followed a hierarchical structure, indicating discussion depth, while the green category focuses on situation awareness and specific behaviors.

Building on the scheme, we used tailored prompts with the ChatGPT-4o model~\cite{gpt4o} to classify behavioral patterns in transcribed dialogue \RR{(More details are in appendix B)}. 
\RR{The model provided a prediction percentage of uncertainty for each classification, improving result interpretability. }
To minimize anomalies, we applied a ``moving window'' technique with Moving Average Smoothing~\cite{chang2022muse}, stabilizing prediction percentages (Fig.\ref{fig:bar}-b). To reduce visual clutter in long time-series data, we aggregated consecutive instances of the same category, averaging prediction percentages (Fig.\ref{fig:bar}-c). These results were displayed in the timeline panel's progress bar, enabling detailed analysis by zooming into specific behavior categories in Sec.~\ref{barchart}. 




\textbf{Roles Extraction:}
We analyzed each speaker's dynamic roles (Driver, Navigator, and Monitor) during programming~\cite{lewis2011pair}. Using ChatGPT-4o and prompts based on the Thought Chain Model~\cite{wei2022chain}, we guided the model through step-by-step reasoning to generate role classifications. Prompts were iterated for clarity, and the model's responses were structured hierarchically and returned in JSON format. Each query was repeated ten times, with the majority result adopted for classification.

\RR{\textbf{Behavioral Engagement:} reflected the level of effort and participation students invested in learning~\cite{fredricks2022measurement}. 
In our study, we focused on the duration and frequency of student speech.} 
We extracted conversation data, excluding irrelevant chat, and divided each conversation into two parts: the first half and the full conversation. We then measured speaking duration, frequency, and degree centrality using co-occurrence networks~\cite{ng1999toward}. For each question, we created and normalized two networks, followed by Non-negative Matrix Factorization (NMF)~\cite{lee2000algorithms} to identify key behavioral patterns for dynamic group comparison.


\RR{\textbf{Cognitive Engagement:} referred to the cognitive investment students made in their learning. We highlighted the role changes and behavior frequencies of students during the collaborative process. }
To capture dynamic changes in student cognitive engagement, we split the dialogue for each question into two segments: the first half and the full dialogue. We extracted the frequency of each speaker's 14 behavioral categories and their roles at each timestamp. After normalizing these features for consistency, we applied NMF to reduce dimensionality and assess each speaker's cognitive engagement.

\begin{figure*}
  \includegraphics[width=\textwidth]{CPVis.pdf}
  \caption{\RR{A screenshot of Group 10 view.} \textit{CPVis} applies multimodal learning analysis to provide instructors with evidence for evaluating group and student performance. It consists of three views:
Filter View (A) Provides an overview and allows group selection. The selected groups appear in the lasso selection area (A2), and the similarity panel (A3) displays the most similar and different groups based on the search (A1a).
Content View (B) Displays group performance, with the B1 panel showing completed codes, the B3a panel illustrating the behavior sequence, and the B3b panel showing student engagement over time.
Detail View (C) Presents the group's collaborative programming video (C1) and raw conversation data (C2).}
  \Description{A screenshot of Group 10 view. \textit{CPVis} applies multimodal learning analysis to provide instructors with evidence for evaluating group and student performance. It consists of three views:
Filter View (A) Provides an overview and allows group selection. The selected groups appear in the lasso selection area (A2), and the similarity panel (A3) displays the most similar and different groups based on the search (A1a).
Content View (B) Displays group performance, with the B1 panel showing completed codes, the B3a panel illustrating the behavior sequence, and the B3b panel showing student engagement over time.
Detail View (C) Presents the group's collaborative programming video (C1) and raw conversation data (C2).}
  \label{fig:teaser}
  \end{figure*}

\subsubsection{Group Performance Assessment}
We evaluated group performance based on three dimensions: code quality, collaborative problem-solving, and teacher scaffolding. 
Through in-depth discussions with domain experts, we assessed how each dimension was valued and measured in the context of our study.




\label{code}
\textbf{Code quality}, reflecting students' mastery of course concepts, was a key metric for evaluating group performance. To assess student submissions, we used ChatGPT-4o~\cite{gpt4o} to evaluate dimensions such as problem-solving, code integrity, accuracy, and algorithmic innovation, scoring each on a 1–5 scale. After refining evaluation prompts, we ran the assessment ten times per submission, averaging the results to ensure consistency and reliability.





\textbf{Collaborative Problem-Solving (CPS):} 
Earlier studies categorized CPS into team effectiveness and task effectiveness~\cite{rosen2020towards}. Team effectiveness was measured by student engagement, while task effectiveness was assessed through code quality. %Our analysis captured problem-solving behaviors by frequency and sequence.
To evaluate CPS, we examined task effectiveness, represented by the average question score (\(\bar{s}\)), and team effectiveness, assessed through the standard deviation of engagement (\(\sigma_e\)) and the average engagement score (\(\bar{e}\)) as shown in Equation \ref{eq:1}. We then used the coefficient of variation (\(CV_e\)) \RR{to account for both engagement variability and engagement}. Finally, the overall collaboration quality was calculated using Equation \ref{eq:2}, combining question performance and engagement balance. 
\begin{equation}
\sigma_e = \sqrt{\frac{1}{n} \sum_{i=1}^{n} (e_i - \bar{e})^2}, \quad CV_e = \frac{\sigma_e}{\bar{e}}
\label{eq:1}
\end{equation}

\begin{equation}
Quality = \bar{s} \cdot (1 - CV_e)
\label{eq:2}
\end{equation}
As shown in Table \ref{table:comparison}, Group 19, despite achieving a respectable average score, exhibited imbalanced engagement, leading to a lower collaboration quality score. In contrast, Group 20 demonstrated more balanced and higher engagement, resulting in a better overall collaboration quality.
\begin{table}[htbp]
\centering
\begin{tabular}{cccccc}
\toprule
\textbf{Group} & \(\bar{s}\) & \textbf{Engagement Levels} & \(\sigma_e\) & \(\text{CV}_e\) & \textbf{CQ} \\
\midrule
Group 19 & \(4.11\) & (10.515, 9.725, 4.575) & \(2.80\) & \(0.24\) & \(2.80\) \\
Group 20 & \(4.14\) & (10.06, 9.32, 8.62) & \(0.73\) & \(0.08\) & \(3.88\) \\
\bottomrule
\end{tabular}
\caption{Comparison of Group 19 and Group 20 on Collaboration Quality (CQ).}
\label{table:comparison}
\end{table}

\textbf{Teacher Scaffolding,} categorized into cognitive (low, medium, high-control) and metacognitive forms~\cite{ouyang2022applying}, reflected the level of support provided to a group and its impact on programming performance. We evaluated four scaffolding dimensions, leveraging GPT-4o for annotation. By using targeted prompts and examples, we improved classification accuracy, while teacher scaffolding was categorized according to the type of support based on a semantic analysis of interactions.






\section{System Design} 
\label{sec:vis}

In this session, we introduce \textit{CPVis} (Fig.~\ref{fig:teaser}), a web-based visual analytics system to assist instructors in evaluating collaborative programming.
\subsection{System Overview}
\textit{CPVis} is a comprehensive system supporting multi-level, progressive analysis, from group-level interactions to individual student performance. Instructors can select specific groups for focused analysis, such as comparative evaluations (R2). Additionally, \textit{CPVis} offers a drill-down feature, enabling an overview of student collaboration and detailed insights into individual performance.
\textit{CPVis} includes four main components:
\textbf{Initial Selection}, instructors can select groups using the lasso tool or search function in the ``Group Overview'' view (Fig.\ref{fig:teaser}-A1). The system displays an overview of the selected groups (Fig.\ref{fig:teaser}-A2), compares similar and different groups (Fig.\ref{fig:teaser}-A3), and synchronizes updates across views (Fig.\ref{fig:teaser}-B).
\textbf{Drill-Down Analysis}, instructors can examine the group's code (Fig.\ref{fig:teaser}-B1) and analyze problem-solving approaches in the Content View. The interaction pattern panel (Fig.\ref{fig:student}-B2b) reveals behavior patterns, while the timeline panel (Fig.\ref{fig:teaser}-B3a) shows activity sequences. The student overview panel (Fig.\ref{fig:student}-B2a) compares individual performance across the class, and the timeline panel (Fig.~\ref{fig:teaser}-B3) highlights engagement and role changes.
\textbf{Multi-Level Interaction}, the system's layered visualization allows instructors to explore and analyze both group and individual behavior, enabling precise assessment of the collaboration process. It also supports side-by-side group comparisons.
\textbf{Detailed Review}, the detail view (Fig.~\ref{fig:teaser}-C) provides original discussion videos and transcripts, enabling in-depth analysis of student conversations and problem-solving processes.
\begin{figure}
	\centering
	\includegraphics[width=1\linewidth]{design.pdf}
	\caption{The iterative design of student glyphs: 1 represents cognitive engagement, and 2 represents behavioral engagement.}
        \Description{The iterative design of student glyphs: 1 represents cognitive engagement, and 2 represents behavioral engagement.}
	\label{fig:designl}
\end{figure}
\subsection{Visual Design}
\begin{figure*}
	\centering
	\includegraphics[width=1\linewidth]{encoding.pdf}
	\caption{The flower metaphor in \textit{CPVis}, along with its visual encoding, color coding, and some samples.}
        \Description{The flower metaphor in \textit{CPVis}, along with its visual encoding, color coding, and some samples.}
	\label{fig:encoding1}
\end{figure*}

We iterated continuously during the glyph design process to optimize the visual representation. Initially, we used a star-shaped design (e.g., radar charts), as shown in Fig.~\ref{fig:designl}. While radar charts effectively displayed behavioral and cognitive engagement, dividing the chart into sections for individual tasks introduced an unnecessary dimension (shape size) that was meaningless and prone to misinterpretation. 
We then shifted to a circular design, encoding behavioral engagement as the radius of a sector and cognitive engagement as the color of the outer ring. However, this design had a significant flaw: color mapping was less intuitive than size mapping, and using color saturation as a visual channel lacked precision. Additionally, both designs struggled to combine individual students into group glyphs intuitively.
\RR{Inspired by previous research~\cite{tausch2014groupgarden,tausch2016comparison,xiong1999peoplegarden}, we introduced a visual design inspired by the flower metaphor, where students are represented as flowers (Fig.~\ref{fig:encoding1}).
The size of the petals represented behavioral engagement, while the size of the stamen indicated cognitive engagement. Three colors were used to represent three different roles, and the varying colors of the leaves symbolized different levels of teacher scaffolding. The number of butterflies reflected the level of collaborative problem-solving ability. As a result, the overall group glyph naturally took the form of a bouquet (R4).}
This approach resolved the issue of merging individual students into group glyphs while enhancing the design's readability and intuitiveness (R1). 
\RR{The final design struck a balance between aesthetics and functionality, effectively conveying the performance of individual students and groups, allowing users to quickly compare similar or different groups (R2).}


\subsection{Filter View}
The Filter View (Fig.~\ref{fig:teaser}-A) serves as the starting point for analysis, featuring an interactive projection panel (Fig.~\ref{fig:teaser}-A1) and a similarity panel (Fig.~\ref{fig:teaser}-A3) to help users filter and explore groups of interest. 
The projection panel displays the distribution of groups in a 2D space to reveal clustering patterns and outliers. We apply the t-SNE algorithm to maximize separation between dissimilar groups, creating clearer clusters. 
To avoid visual clutter caused by group glyphs in dimensionality-reduced views, we follow the approach of Tac-Miner~\cite{wang2021tac}, representing groups as points or rectangles based on whether they received teacher scaffolding, with color coding reflecting prior performance. 
\RR{Additionally, the outer arc represents the duration of the discussion.}
Users can select groups using the lasso tool (Fig.~\ref{fig:teaser}-A1b) or search for specific groups (Fig.~\ref{fig:teaser}-A1a). The similarity panel (Fig.~\ref{fig:teaser}-A2) shows the most similar and dissimilar groups based on Euclidean distance.  
 

 
\subsection{Content View}
Once the search group is selected, users can perform a detailed analysis through the content view (Fig.~\ref{fig:teaser}-B). This view comprises four panels, allowing for a layered exploration of group and individual student performance. 

\subsubsection{Codes Panel}
In the upper left corner of the content view, a control button (Fig.~\ref{fig:teaser}-B1) allows users to toggle between the Codes panel and the Student Projection/Group Pattern panels. Below, users can compare the code quality between the selected group (left) and a comparison group (right). 
For instance, Figure~\ref{fig:teaser}-B1 shows Group 10's answer to Question 5 (left) and Group 18's answer (right). Hint boxes provide two types of feedback: red tips indicating code deficiencies (Fig.~\ref{fig:teaser}-B1a) and pink tips showing that a method was learned from pre-class materials, signaling comprehension, and application. 
Below the code, the ChatGPT-4o score and its rationale are provided, enabling quick, in-depth code evaluation and highlighting areas where groups faced challenges. 


 
\subsubsection{Students Projection (Fig.~\ref{fig:student}-B2a)}

We project students from different groups using t-SNE for clustering, highlighting students with similar performance. 
The three flowers representing the search group are connected with dashed lines to clarify group member distribution and similarity, helping users assess group homogeneity or heterogeneity. 
Only the comparing group's flowers are highlighted to minimize visual clutter. 
\RR{Users can zoom in/out to explore specific students and view detailed background information (e.g., major, grades) by hovering over individual points (Fig.~\ref{fig:student}-B2c).}

\subsubsection{Group Pattern Panel (Fig.~\ref{fig:student}-B2b \& d)}
We use Epistemic Network Analysis (ENA)~\cite{zhao2023analysing} to analyze the dynamic connections between cognitive elements and the collaborative problem-solving behaviors of groups (R3). 
In the Group Pattern Panel, each node represents a behavior in the collaborative problem-solving process, with colors following the coding scheme in Section~\ref{fig:scheme}. Node size indicates behavior frequency, and the intensity of the color reflects the frequency of interactions between behaviors. More significantly, darker nodes represent more frequent behaviors and interactions. 
Users can click on different question buttons to examine dynamic changes in behavior across specific questions or select multiple questions to observe how behaviors evolve during transitions between tasks. Hovering over nodes reveals detailed information about each behavior.
\RR{When comparing two groups, the system displays side-by-side behavior networks.} 
The Group Pattern Panel displays the search group's behavior patterns and compares them to those of the comparison group during the collaborative problem-solving process (Fig.~\ref{fig:student}-B1b).


\subsubsection{Timeline Panel}
\label{barchart}
Users can toggle between the search and comparing groups using the control button in the upper left corner of the Timeline Panel (R5). This panel displays group and individual student performance over time using a filterable bar chart and line chart (Fig.~\ref{fig:teaser}-B3).
In the bar chart, each bar represents a timestamp, with colors indicating different collaborative problem-solving behaviors. The bar height reflects uncertainty, as calculated by ChatGPT-4o. Dashed lines separate different questions for visual clarity. Hovering over a bar reveals the behavior category, predicted certainty, and reasoning behind the label (Fig.~\ref{fig:teaser}-B3e). Clicking on a bar takes users to the relevant conversation content (Fig.~\ref{fig:teaser}-C2a).
The filtering function (Fig.~\ref{fig:teaser}-B3c) allows users to zoom in on specific periods, magnifying the bars for more detailed analysis. 
\RR{The progress bar consolidates behaviors, minimizing visual clutter and highlighting key shifts in temporal behavior in the overview, while also allowing for detailed tracking of group dynamics (Fig.~\ref{fig:group18}-c).}

Three line charts track the behaviors and cognitive engagement of three students in the group (Fig.~\ref{fig:teaser}-B3b). Engagement is calculated at the midpoint and end of each question, and Savitzky–Golay filtering smooths the curves to highlight dynamic changes and trends across questions.
Below the charts, role types are mapped using equally sized rectangles at each timestamp. Users can zoom in on the timeline with the progress bar for detailed analysis. 






\subsection{Detail View}
In the Timeline Panel, users can link to specific conversation content to review group discussions (Fig.~\ref{fig:teaser}-C1) and individual student conversations for each question (Fig.~\ref{fig:teaser}-C2). The playback feature allows users to revisit the original video of collaborative programming sessions, providing a more immersive classroom experience. This feature validates analysis results, offers detailed references, and supports the final step of our analysis workflow (R6).

\begin{figure*}
 	\centering
 	\includegraphics[width=1\linewidth]{Student_panel.pdf}
 	\vspace{-1em}
 	\caption{The panel displayed in the Content View after selecting ``Student Projection \& Group Pattern'' is shown \RR{(A screenshot of the Group 10 \& 18 view)}. On the left (B2a), the projection of all students highlights the searched and compared groups, showing connections between students within the searched group. Hovering over a flower reveals the student's background information (B2c). On the right (B2b), group patterns for two groups across different questions are shown, with clickable question buttons to explore changes in dynamic learning behavior.}
        \Description{The panel displayed in the Content View after selecting ``Student Projection \& Group Pattern'' is shown \RR{(A screenshot of the Group 10 \& 18 view)}. On the left (B2a), the projection of all students highlights the searched and compared groups, showing connections between students within the searched group. Hovering over a flower reveals the student's background information (B2c). On the right (B2b), group patterns for two groups across different questions are shown, with clickable question buttons to explore changes in dynamic learning behavior.}
 	\label{fig:student} \end{figure*}


% \begin{table*}[t]
\centering
\caption{A case study examining \model's response to a query about ``\emph{graders}'' in OpenAI's 12-day edu    cational series demonstrates its video retrieval capabilities and knowledge synthesis.}
\label{tab:case study}
\small
\begin{tabularx}{\textwidth}{X X}

\toprule
\textbf{Query:} Explain the purpose and functionality of 'graders' in the context of reinforcement fine-tuning.\\

\midrule
\textbf{Input Videos:} \href{https://www.youtube.com/playlist?list=PLOXw6I10VTv9lin5AzsHAHCTrC7BdVdEM}{12 Days of OpenAI} (A total of 12 videos) \\

\midrule

\textbf{\model\ Response:} 

\texttt{\#\#\# Understanding Gradation in Reinforcement Fine-Tuning}

\texttt{**Purpose of Graders**}

\texttt{Graders are integral components in the process of reinforcement fine-tuning, mainly utilized to evaluate and improve the performance of machine learning models...} (Omitted for brevity)

\texttt{**Functionality of Graders**}

\texttt{\colorbox{lightgreen}{The grading system operates by comparing a model's outputs against those of}}

\texttt{\colorbox{lightgreen}{verified correct responses. The mechanics of this assessment involve assigning}}

\texttt{\colorbox{lightgreen}{scores to the model's outputs based on their correctness. These scores range}}

\texttt{\colorbox{lightgreen}{from zero to one, where:}}


\texttt{\colorbox{lightblue}{- A score of **0** indicates that the model's answer is completely incorrect.}}

\texttt{\colorbox{lightblue}{- A score of **1** signals that the answer is fully correct.}}

\texttt{- Scores can also be fractional, allowing for partial credit. For instance, \colorbox{lightpink}{a score of **0.7** may denote that the model's suggestion was }}
\texttt{\colorbox{lightpink}{correct but not the highest in a ranked list of possible answers.}} 

\\

\texttt{The process includes:}

\texttt{1. **Input Comparison**: The grader takes both the model’s output and the correct answer, analyzing them to derive a score.}

\texttt{2. **Graded Outputs**: This interaction allows for a dynamic reflection of the model's performance, where the scores directly influence how the model is fine-tuned in subsequent training iterations.}

\texttt{3. **Feedback Loop**: The scores obtained from graders provide feedback that can be used to adjust learning parameters and improve the model’s future responses. This highlights the feedback mechanism that is crucial for reinforcement learning.}

\\

(The remaining content are omitted for brevity.)

\\

\midrule
\textbf{Retrieved Sources by \model:}

\model\ accurately retrieved four continuous video segments from the OpenAI's 12-day show, ranging from 10:00 to 12:00 in Day 2 of the video "\href{https://www.youtube.com/watch?v=yCIYS9fx56U}{Reinforcement Fine-Tuning}." Here, we highlight key moments relevant to the detailed content in the answer. From left to right, these are retrieved moments at timestamps \colorbox{lightgreen}{10:35}, \colorbox{lightblue}{10:39}, and \colorbox{lightpink}{11:10}, which provide informative insights that help \model\ give a comprehensive answer to the query.

\\

\begin{tabular}{ccc}
    {\includegraphics[width=0.3\textwidth]{figs/openai-1.png}} &
    {\includegraphics[width=0.3\textwidth]{figs/openai-2.png}} &
    {\includegraphics[width=0.3\textwidth]{figs/openai-3.png}} \\
\end{tabular}

\\

\bottomrule

\end{tabularx}
\vspace{-0.2in}
\end{table*}
\section{Evaluation}
We provide three sets of insights into this section, organised as \textit{findings (F*)}. We quantitatively study the effect of the adversarial and counterfactual perturbations on the performance of informal reasoners and autoformalisation methods. Then, we dive deeper into method variants. Finally, 
we analyse the nature of formalisation errors made by the models.

\subsection{Robustness Analysis}
\paragraph{\textbf{\emph{F1: Noise perturbations have a stronger effect on formalisation methods than informal \ac{LLM} reasoners.}}}
Table~\ref{tab:distraction_k4_formalisation} shows that, on average, the accuracy of both direct and \ac{CoT} informal reasoning remains between $73\%$ and $74\%$ in the face of added noise. While the autoformalisation method performs similarly to informal reasoners on the original dataset, its performance decreases between $4\%$ and $11\%$. The accuracy drops especially with logical (L) and tautological (T) distractions, whose logical language formats trick the \ac{LLM} into formalizing the noisy clauses. On the other hand, the linguistically complex and more natural sentences of encyclopedic distractions show a minor effect, suggesting that \acp{LLM} successfully avoids formalizing the more complicated sentences.

\paragraph{\textbf{\emph{F2: All \ac{LLM}-based reasoning methods suffer a drop for counterfactual perturbations.}}} % influence .}}}
Table~\ref{tab:distraction_k4_formalisation} shows that counterfactual statements cause a significant decrease in performance for both the informal reasoners and autoformalisation methods of between $12\%$ and $13\%$ on average. 
Moreover, this observation also holds for all tested models, i.e., none are robust towards counterfactual perturbations across every evaluated dimension. Even the strongest model, GPT 4o-mini, yields a performance of 63-68\%, which is relatively close to the random performance of 50\%. The high impact of counterfactual statements (the single ``not'' inserted) could be due to the inability of \acp{LLM} to overwrite prior knowledge with explicitly stated information or memorization of the answers. We study the error sources further in §\ref{subsec:errors}.  

\noindent \paragraph{\textbf{\emph{F3: Introducing multiple noise sentences has an effect only for logical distractions.}}}
We show the impact of introducing between one and four sentences for the two top-performing autoformalisation models in Figure~\ref{fig:length_distraction}. The figure shows similar trends with and without counterfactual perturbations.
As additional logical distractions are introduced, the model performance consistently decreases. Tautological (T) distractions lead to a decline in accuracy with a single disruptive sentence, yet adding more noise does not worsen the outcome. 
The tautological corpus introduces truth constants for all sentences as a persistent unseen logical construct. Given that this leads only to a decrease for a single occurrence, we can assume that a model can consistently handle the same unseen logical construct. In contrast, the logical corpus increases the chance of adding text, requiring new, previously unseen reasoning constructs for each added sentence. The impact of encyclopedic noise remains negligible, generalising F1 to $k$ sentences. Similarly, counterfactual perturbations remain much more effective for all settings, generalising F2.

\begin{table}[!t]
\small
\setlength{\modelspacing}{2pt}
\setlength{\tabcolsep}{1.7pt} % Default value: 6pt
\setlength{\belowrulesep}{4pt}
\begin{threeparttable}
    \centering
    \begin{tabular}{cc l r rrr @{\quad} rrrr}
\toprule
\multirow{2}{*}{} & \multirow{2}{*}{} & Reasoning & \multirow{2}{*}{O} & \multicolumn{3}{c}{Distraction} & \multicolumn{4}{c}{Counterfactual} \\
 & & Format & & E& L & T & $\text{O}_C$ & $\text{E}_C$& $\text{L}_C$ & $\text{T}_C$\\
\midrule
\multirow{6}{*}{\rotatebox{90}{Gemma-2}} & \multirow{3}{*}{\rotatebox{90}{9b}}
   & Informal (direct) & \textbf{0.78} & \textbf{0.80} & \textbf{0.79} & \textbf{0.77} & 0.58 & 0.52 & 0.50 & 0.59 \\
 & & Informal (CoT) & 0.72 & 0.78 & 0.73 & 0.76 & 0.61 & \textbf{0.57} & \textbf{0.60} & \textbf{0.66} \\
 & & Formal (FOL) & 0.62 & 0.58 & 0.52 & 0.53 & \textbf{0.63} & 0.52 & 0.46 & 0.46 \\[\modelspacing]
\cmidrule{2-11}
 & \multirow{3}{*}{\rotatebox{90}{27b}} 
   & Informal (direct) & 0.71 & 0.69 & \textbf{0.66} & \textbf{0.68} & 0.59 & 0.51 & 0.54 & 0.59 \\
 & & Informal (CoT) & 0.66 & 0.65 & 0.64 & 0.63 & 0.62 & 0.58 & \textbf{0.62} & \textbf{0.64} \\
 & & Formal (FOL) & \textbf{0.74} & \textbf{0.74} & 0.61 & 0.61 & \underline{\textbf{0.72}} & \underline{\textbf{0.67}} & 0.58 & 0.51 \\[\modelspacing]
\midrule
\multirow{6}{*}{\rotatebox{90}{Mistral}} & \multirow{3}{*}{\rotatebox{90}{7B}} 
   & Informal (direct) & 0.77 & \textbf{0.77} & 0.75 & \textbf{0.79} & \textbf{0.63} & \textbf{0.54} & \textbf{0.54} & \textbf{0.66} \\
 & & Informal (CoT) & \textbf{0.79} & 0.75 & \textbf{0.77} & 0.78 & 0.55 & 0.52 & \textbf{0.54} & 0.58 \\
 & & Formal (FOL) & 0.62 & 0.58 & 0.54 & 0.57 & 0.50 & \textbf{0.54} & 0.51 & 0.52 \\[\modelspacing]
\cmidrule{2-11}
 & \multirow{3}{*}{\rotatebox{90}{Small}} 
   & Informal (direct) & \textbf{0.77} & \textbf{0.76} & \textbf{0.76} & \textbf{0.75} & 0.61 & 0.51 & 0.56 & 0.59 \\
 & & Informal (CoT) & 0.72 & 0.72 & 0.72 & 0.71 & \textbf{0.62} & \textbf{0.59} & \textbf{0.62} & \textbf{0.68} \\
 & & Formal (FOL) & 0.68 & 0.59 & 0.53 & 0.64 & 0.54 & 0.55 & 0.49 & 0.51 \\[\modelspacing]
\midrule
\multirow{6}{*}{\rotatebox{90}{Llama-3.1}} & \multirow{3}{*}{\rotatebox{90}{8B}} 
   & Informal (direct) & 0.63 & 0.61 & 0.64 & 0.66 & 0.61 & \textbf{0.62} & 0.59 & 0.61 \\
 & & Informal (CoT) & 0.73 & \textbf{0.73} & \textbf{0.71} & \textbf{0.72} & \textbf{0.62} & 0.59 & \textbf{0.61} & \textbf{0.65} \\
 & & Formal (FOL) & \textbf{0.77} & 0.71 & 0.63 & 0.52 & 0.60 & 0.58 & 0.55 & 0.52 \\[\modelspacing]
\cmidrule{2-11}
 & \multirow{3}{*}{\rotatebox{90}{70B}} 
   & Informal (direct) & 0.77 & 0.74 & 0.74 & 0.73 & 0.62 & 0.53 & 0.56 & 0.64 \\
 & & Informal (CoT) & \textbf{0.78} & \textbf{0.75} & \textbf{0.76} & \textbf{0.76} & 0.64 & 0.61 & \textbf{0.66} & \underline{\textbf{0.73}} \\
 & & Formal (FOL) & 0.74 & 0.73 & 0.71 & 0.71 & \textbf{0.66} & \textbf{0.62} & 0.59 & 0.57 \\[\modelspacing]
 \midrule
\multirow{3}{*}{\rotatebox{90}{GPT}} & \multirow{3}{*}{\rotatebox{90}{4o-mini}} 
   & Informal (direct) & 0.78 & 0.77 & 0.79 & 0.79 & 0.64 & 0.61 & 0.61 & 0.63 \\
 & & Informal (CoT) & 0.80 & 0.80 & \underline{\textbf{0.81}} & \underline{\textbf{0.82}} & \textbf{0.68} & \textbf{0.63} & \underline{\textbf{0.68}} & \textbf{0.64} \\
 & & Formal (FOL) & \underline{\textbf{0.84}} & \underline{\textbf{0.82}} & 0.73 & 0.79 & 0.63 & 0.62 & 0.57 & 0.54 \\[\modelspacing]
 \midrule
\multicolumn{2}{c}{\multirow{3}{*}{\textbf{Avg}}} 
 & Informal (direct) & 0.74 & 0.73 & 0.73 & 0.73 & 0.61 & 0.55 & 0.56 & 0.62 \\
 & & Informal (CoT) & 0.74 & 0.74 & 0.73 & 0.74 & 0.62 & 0.58 & 0.62 & 0.65 \\
  & & Formal (FOL) & 0.72 & 0.68 &	0.61 & 0.62 & 0.61 & 0.59 & 0.54 & 0.52 \\
\bottomrule
\end{tabular}
\caption{Accuracies of informal and autoformalisation-based deductive reasoners. The best overall model per dataset is underlined; the best model version is marked in bold.}
\label{tab:distraction_k4_formalisation}
\end{threeparttable}
\end{table} 

\begin{figure}[!t]
    \centering
    \scriptsize
    \begin{tikzpicture}
        \begin{axis}[name=gpt,
            title={GPT-4o-mini},
            width=0.6\linewidth,
            height=0.6\linewidth,
            xlabel={\# Noise sentences},
            ylabel={Accuracy},
            xmin=-0.1, xmax=4.1,
            ymin=0.5, ymax=0.9,
            xtick={1,2,4},
            ytick={0.55, 0.6, 0.65, 0.75, 0.8, 0.85},
            title style={yshift=-0.6em},
            legend style={at={(1,-0.15)},
	           anchor=north,legend columns=-1},
            x label style={at={(axis description cs:1,-0.05)},anchor=north},
            y label style={at={(axis description cs:-0.15,0.5)},anchor=south},
            ymajorgrids=true,
            grid style=dashed,
        ]
            \addplot[color=blue, mark=square,]
                coordinates {
                (0,0.848076939582825)(1,0.823076903820038)(2,0.826923072338104)(4,0.821153819561005)
                };
            \addplot[color=red, mark=triangle,]
                coordinates {
                (0,0.848076939582825)(1,0.817307710647583)(2,0.801923096179962)(4,0.759615361690521)
                };
            \addplot[color=green, mark=diamond,] 
                coordinates {
                (0,0.848076939582825)(1,0.767307698726654)(2,0.769230782985687)(4,0.803846180438995)
                };
            \addplot[color=blue, mark=square*] 
                coordinates {
                (0,0.627777755260468)(1,0.622222244739533)(2,0.600000023841858)(4,0.633333325386047)
                };
            \addplot[color=red, mark=triangle*,] 
                coordinates {
                (0,0.627777755260468)(1,0.611111104488373)(2,0.611111104488373)(4,0.594444453716278)
                };
            \addplot[color=green, mark=diamond*,] 
                coordinates {
                (0,0.627777755260468)(1,0.572222232818604)(2,0.538888871669769)(4,0.555555582046509)
                };
                \legend{E,L,T,$\text{E}_C$, $\text{L}_C$ , $\text{T}_C$}
        \end{axis}

        \begin{axis}[name=llama, at={($(gpt.east)+(0.1cm,0)$)},anchor=west,
            title={Llama 3.1 70b},
            width=0.6\linewidth,
            height=0.6\linewidth,
            xmin=-0.1,, xmax=4.1,
            ymin=0.5, ymax=0.9,
            xtick={1,2,4},
            ytick={0.55, 0.6, 0.65, 0.75, 0.8, 0.85},
            title style={yshift=-0.6em},
            yticklabel=\empty,
            ymajorgrids=true,
            grid style=dashed,
        ]
            \addplot[color=blue, mark=square,]
                coordinates {
                (0,0.838461518287659)(1,0.817307710647583)(2,0.805769205093384)(4,0.817307710647583)
                };
            \addplot[color=red, mark=triangle,]
                coordinates {
                (0,0.838461518287659)(1,0.819230794906616)(2,0.803846180438995)(4,0.771153867244721)
                };
            \addplot[color=green, mark=diamond,]
                coordinates {
                (0,0.838461518287659)(1,0.803846180438995)(2,0.807692289352417)(4,0.805769205093384)
                };
            \addplot[color=blue, mark=square*]
                coordinates {
                (0,0.627777755260468)(1,0.622222244739533)(2,0.577777802944183)(4,0.594444453716278)
                };
            \addplot[color=red, mark=triangle*,]
                coordinates {
                (0,0.627777755260468)(1,0.583333313465118)(2,0.561111092567444)(4,0.577777802944183)
                };
            \addplot[color=green, mark=diamond*,]
                coordinates {
                (0,0.627777755260468)(1,0.627777755260468)(2,0.566666662693024)(4,0.577777802944183)
                };
        \end{axis}
    \end{tikzpicture}
    \caption{Influence of the number of noisy sentences for FOL.}
    \label{fig:length_distraction}
\end{figure}



\subsection{Impact of Method Design}
\paragraph{\textbf{\emph{F4: \ac{CoT} prompting is most impactful when both noise and counterfactual perturbations are applied.}}}
The accuracies for the individual \acp{LLM} in Table~\ref{tab:distraction_k4_formalisation} show that the impact of \ac{CoT} is negligible for noise-only datasets (first four columns). Meanwhile, the benefit from \ac{CoT} is most pronounced in the datasets that combine noise and counterfactual perturbations.
The better-performing informal prompting strategy for a model remains stable for all types of distractions. Still, the decline in performance due to counterfactuals leads to a less consistent preference for a specific prompting style.

\paragraph{\textbf{\emph{F5: The best-performing grammar differs per model and is unstable across data versions.}}}

The evaluation of different logical forms for formal \ac{LLM}-based reasoning in Table~\ref{tab:distraction_k4_logical_form} shows the preference of some models for specific syntactic formats.
Llama 3.1 70B has a considerable improvement of $12\%$ with TPTP syntax on the original set, while Llama 3.1 8B benefits from the R-FOL syntax. However, all grammars show a declining accuracy trend and increased syntax errors for noise perturbations, where the best grammar loses its advantage over the rest. 
When comparing the grammars on the counterfactual partitions, we observe that TPTP is consistently more robust than the standard first-order logic grammar. Here, GPT 4o-mini shows a reduction from $O$ to $O_C$ of $20\%$ for FOL and only $12\%$ for the TPTP grammar. Since this does not correlate with fewer syntax errors, the formalisation in TPTP prevents semantical errors for counterfactual premises. 
A positive reading of these results, especially the minor differences between FOL and R-FOL, is that autoformalisation \acp{LLM} can adapt to the grammar syntax prescribed in the prompt without further loss in performance.

\begin{table}[!t]
\small
\setlength{\modelspacing}{2pt}
\setlength{\tabcolsep}{1.7pt} % Default value: 6pt
\setlength{\belowrulesep}{4pt}
\begin{threeparttable}
    \centering
    \begin{tabular}{cc l r rrr @{\quad} rrrr}
\toprule
\multirow{2}{*}{} & \multirow{2}{*}{} & Grammar & \multirow{2}{*}{O} & \multicolumn{3}{c}{Distraction} & \multicolumn{4}{c}{Counterfactual} \\
 & & Syntax & & E& L & T & $\text{O}_C$ & $\text{E}_C$& $\text{L}_C$ & $\text{T}_C$\\
\midrule
\multirow{6}{*}{\rotatebox{90}{Llama-3.1}} & \multirow{3}{*}{\rotatebox{90}{8B}} 
   & FOL & 0.77 & \textbf{0.71} & 0.61 & \textbf{0.53} & 0.58 & \textbf{0.55} & 0.52 & \textbf{0.56} \\
 & & R-FOL & \textbf{0.78} & 0.69 & \textbf{0.62} & \textbf{0.53} & 0.58 & \textbf{0.55} & \textbf{0.54} & 0.52 \\
 & & TPTP & 0.73 & 0.67 & 0.55 & 0.51 & \textbf{0.68} & 0.54 & 0.46 & 0.51 \\[\modelspacing]
\cmidrule{2-11}
 & \multirow{3}{*}{\rotatebox{90}{70B}} 
   & FOL & 0.76 & 0.73 & 0.71 & \textbf{0.72} & 0.67 & 0.57 & 0.63 & 0.56 \\
 & & R-FOL & 0.76 & 0.73 & 0.67 & 0.71 & 0.64 & 0.57 & 0.53 & 0.64 \\
 & & TPTP & \underline{\textbf{0.88}} & \underline{\textbf{0.84}} & \underline{\textbf{0.81}} & \textbf{0.72} & \underline{\textbf{0.81}} & \underline{\textbf{0.68}} & \underline{\textbf{0.67}} & \underline{\textbf{0.68}} \\[\modelspacing]
\midrule
\multirow{3}{*}{\rotatebox{90}{GPT}} & \multirow{3}{*}{\rotatebox{90}{4o-mini}} 
   & FOL & \textbf{0.84} & \textbf{0.82} & \textbf{0.72} & \underline{\textbf{0.78}} & 0.64 & \textbf{0.63} & \textbf{0.61} & 0.51 \\
 & & R-FOL & \textbf{0.84} & 0.77 & 0.70 & \underline{\textbf{0.78}} & \textbf{0.72} & 0.56 & 0.54 & \textbf{0.63} \\
 & & TPTP & 0.83 & \textbf{0.82} & 0.71 & 0.71 & 0.69 & \textbf{0.63} & 0.57 & 0.57 \\
\bottomrule
\end{tabular}
\caption{Accuracies of different formalisation grammars for autoformalisation.}
\label{tab:distraction_k4_logical_form}
\end{threeparttable}
\end{table} 

\paragraph{\textbf{\emph{F6: Feedback does not help \acp{LLM} self-correct to mitigate robustness issues.}}}
\autoref{tab:distraction_k4_feedback} shows the results with different error recovery mechanisms. The results indicate that no feedback strategy emerges as a winner in the different datasets. 
All feedback variants reduce syntax errors for noise perturbations, but given the lack of a consistent increase in accuracy, the corrected formalisations are most likely to contain semantic errors still. 
The type of feedback message only has a minor influence on correcting syntax errors, whereas Llama 3.1 70b and GPT 4o-mini correct slightly more syntax errors with specific error messages. This finding aligns with \cite{huang2023large}, who also found that \acp{LLM} cannot consistently self-correct their reasoning after receiving relevant feedback.

\begin{table}[!ht]
\small
\setlength{\modelspacing}{2pt}
\setlength{\tabcolsep}{1.7pt} % Default value: 6pt
\setlength{\belowrulesep}{4pt}
\begin{threeparttable}
    \centering
    \begin{tabular}{cc l r rrr @{\quad} rrrr}
\toprule
\multirow{2}{*}{} & \multirow{2}{*}{} & \multirow{2}{*}{Feedback} & \multirow{2}{*}{O} & \multicolumn{3}{c}{Distraction} & \multicolumn{4}{c}{Counterfactual} \\
 & & & & E& L & T & $\text{O}_C$ & $\text{E}_C$& $\text{L}_C$ & $\text{T}_C$\\
\midrule
\multirow{8}{*}{\rotatebox{90}{Llama-3.1}} & \multirow{4}{*}{\rotatebox{90}{8B}} 
   & No recovery & 0.77 & \textbf{0.72} & 0.62 & 0.53 & 0.59 & 0.58 & 0.56 & \textbf{0.56} \\
 & & Error type & \textbf{0.79} & 0.71 & 0.63 & \textbf{0.56} & \textbf{0.66} & 0.54 & 0.52 & 0.51 \\
 & & Error message & 0.78 & 0.71 & \textbf{0.67} & 0.55 & 0.59 & 0.53 & \underline{\textbf{0.64}} & 0.49 \\
 & & Warning & 0.74 & 0.66 & 0.58 & 0.55 & 0.55 & \textbf{0.60} & 0.49 & 0.49 \\[\modelspacing]
\cmidrule{2-11}
 & \multirow{4}{*}{\rotatebox{90}{70B}} 
   & No recovery & \textbf{0.77} & \textbf{0.72} & \textbf{0.73} & 0.71 & \textbf{0.64} & 0.59 & \textbf{0.61} & 0.56 \\
 & & Error type & 0.72 & 0.70 & 0.72 & \textbf{0.73} & 0.62 & 0.56 & 0.60 & 0.58 \\
 & & Error message & 0.71 & 0.70 & \textbf{0.73} & 0.71 & \textbf{0.64} & 0.59 & 0.54 & \underline{\textbf{0.64}} \\
 & & Warning & 0.69 & \textbf{0.72} & 0.72 & 0.72 & 0.62 & \underline{\textbf{0.65}} & \textbf{0.61} & 0.63 \\[\modelspacing]
\midrule
\multirow{4}{*}{\rotatebox{90}{GPT}} & \multirow{4}{*}{\rotatebox{90}{4o-mini}} 
   & No recovery & \underline{\textbf{0.84}} & \underline{\textbf{0.82}} & 0.73 & 0.79 & 0.64 & \textbf{0.62} & 0.56 & \textbf{0.56} \\
 & & Error type & 0.83 & 0.79 & 0.74 & 0.76 & 0.67 & 0.57 & 0.56 & \textbf{0.56} \\
 & & Error message & \underline{\textbf{0.84}} & 0.78 & \underline{\textbf{0.77}} & \underline{\textbf{0.80}} & 0.62 & 0.59 & 0.56 & \textbf{0.56} \\
 & & Warning & \underline{\textbf{0.84}} & 0.75 & 0.73 & 0.76 & \underline{\textbf{0.70}} & 0.61 & \textbf{0.61} & 0.55 \\
 \bottomrule
\end{tabular}
\caption{Accuracies of error recovery strategies.}
\label{tab:distraction_k4_feedback}
\end{threeparttable}
\end{table} 

\subsection{Error Analysis}
\label{subsec:errors}
\paragraph{\textbf{\emph{F7: Autoformalisation increases syntax errors for noise perturbations.}}}
The low performance for noise perturbations correlates with more syntax errors for all models and distraction categories (cf. execution rates in Table~\ref{tab:appendix_k4_formalisation_exec}). The three worst-performing models (both Mistral models, Gemma-2 9b) generate, at best, for $37\%$  and, at worst, for only $4\%$ of the samples, a valid logical form.
Gemma-2 9b and Llama3.1 8b produce more syntax errors than the larger counterparts, suggesting that larger models are more robust towards noise perturbations. 
The accuracy of syntactically valid samples is higher than the informal reasoning methods for most distractions (Table~\ref{tab:appendix_k4_formalisation_vacc}), motivating informal reasoning as a backup strategy for formal reasoning. The error message feedback reveals two common syntax errors: 1) errors by models with an initial low execution rate exhibit issues with the template structure, including using incorrect keywords or adding conversational phrases;
2) perturbation-related errors, the most common of which is using undefined truth constants as part of tautological distractions. 

\paragraph{\textbf{\emph{F8: Autoformalisation increases semantic errors for counterfactuals.}}}
Unlike the introduced noise, counterfactual perturbations do not lead to more syntax errors. The execution rate in Table~\ref{tab:appendix_k4_formalisation_exec} is stable or improves for counterfactuals. However, we see a drop in accuracy for the counterfactual column $\text{O}_C$ in Table~\ref{tab:distraction_k4_formalisation} and can conclude that the number of logical forms with semantic errors has to increase. This suggests that the introduced negation is not correctly formalised. Looking at the warnings generated by the feedback mechanism, for GPT 4o-mini, $161$ warning messages are generated on the unperturbed data. $54$ of these were fixed with a single iteration. Not considering predicates and individuals as part of the context is the most frequent warning across all models. 

\section{Discussion}
\RR{
Our study utilizes an intuitive flower-based visual design and evidence-based collaborative programming process analysis to provide instructors with a clear perspective for evaluating group and individual performance in collaborative programming. In this section, we discuss the lessons learned, the factors contributing to the research outcomes, and how these findings relate to existing works.

\subsection{Flower-Based Visual Design for Intuitive and Useful by Participants}
In large-scale learning analytics, intuitive visualization and interactive features prove to be valuable in assisting instructors with evaluations while reducing their workload~\cite{martinez2020data,fernandez2024data}.
Our study shows that the flower-based visual design effectively helps instructors summarize the performance of students and groups in collaborative programming.
Participants using \textit{CPVis} typically report starting by observing the flower visualization to gain an overview of the group's overall performance and the engagement levels of individual members during collaboration. Our design enables them to make quick assessment judgments and uncover valuable educational insights. 
For instance, students playing the Driver role often exhibit higher engagement levels.
\RA{Previous works use dynamic natural metaphors~\cite{tausch2014groupgarden,tausch2016comparison}, such as blooming flowers, falling leaves, and weather changes, to represent the quality and state of group discussions. However, these metaphors primarily convey overall trends or atmospheres rather than offering a precise and structured representation of multidimensional data, making it difficult for users to extract specific and accurate information efficiently. Moreover, the strong symbolic and emotional nature of their metaphors often leads to subjective interpretations.}
The effectiveness of our design lies in its ability to translate multiple dimensions of process-based learning analytics into visual elements such as colored petals and flower stamens, enabling instructors to quickly interpret multidimensional data and assess both group and individual performance during collaboration.
Furthermore, the flower-based visualization supports hierarchical analysis at both the group and individual levels, allowing instructors to efficiently analyze and compare the performance of multiple groups and students on a large scale.

\subsection{\textit{CPVis} Enhanced Instructors' Confidence in Evaluating Groups and Students}
The study demonstrates that \textit{CPVis} enhances participants' confidence in evaluation outcomes and improves the accuracy of their assessments. In Baseline System 1, participants report that accessing data requires significant time, and evaluating a specific group's performance often necessitates finding similar groups for a relatively fair comparison. 
Such a process demands additional time, causing participants to lose patience and avoid thoroughly examining all the details.
In baseline system 2, participants have to manually browse and process large amounts of student behavior and interaction data, which significantly increases cognitive load and reduces efficiency as they rely on memory to evaluate the performance of different groups.
In comparison, \textit{CPVis} offers significant convenience to participants \RA{by visualizing multidimensional learning analytics data}, allowing them to effortlessly access key information required for evaluations and compare similar groups. By providing both an overall view of multiple groups and detailed comparisons into individual groups, \textit{CPVis} substantially boosts participants' confidence in their evaluation outcomes, as demonstrated in the ratings. 
%This finding aligns with previous research results~\cite{sato2023groupnamics}.
\RA{Clear and intuitive visual analytics systems contribute to improved confidence and efficiency among participants. For instance, Groupnamics helps participants identify groups requiring intervention by visualizing each group's recent vocal activities and discussion statuses in a one-page view, thereby boosting their confidence in decision-making~\cite{sato2023groupnamics}.}
While it is ideal for \textit{CPVis} to support comparisons across an unlimited number of groups, practical limitations related to cognitive load and visual design make this challenging. Future efforts focus on optimizing the evaluation process through visual design, striking a balance between cognitive load and evaluation efficiency, thereby providing effective support for teaching.



\subsection{Theory-driven and LLM-powered Automation Evaluation for Quantifying Collaborative Learning}
Our study utilizes data collection, analysis, and visualization techniques to extract key insights from students' collaborative behaviors and outcomes, providing a deeper understanding of the learning process in collaborative programming. We focus on quantifying complex collaborative learning processes by leveraging LLMs and theoretical frameworks, introducing innovative methods to evaluate collaboration efficiency. 
While collaborative problem-solving is clearly defined in prior research~\cite{rosen2020towards}, achieving a quantitative balance between task performance and team effectiveness remains a significant challenge. To address this, we employ the coefficient of variation as a balancing metric and validate its efficacy using real-world datasets.
By integrating LLMs, \textit{CPVis} automates the annotation of collaborative programming performance, significantly reducing the workload associated with manually labeling large-scale classroom data and offering a novel perspective for automated learning analytics. 
Combining theory-driven metrics and LLM-powered automation provides instructors with robust, multidimensional evidence, enabling them to process and compare extensive student data systematically. 
This empowers instructors to effectively evaluate group and individual behaviors in collaborative programming, identify collaboration patterns, and support evidence-based decision-making. Previous research demonstrates that data-driven analysis helps educational decision-makers~\cite{hou2024codetailor}, such as instructors, uncover hidden learning patterns and deliver personalized guidance. Building on this foundation, \textit{CPVis} further enhances the potential for personalized feedback, enabling instructors to provide precise, data-driven guidance to students.


}


\section{Limitations and Future Work}

\RR{
In this section, we discuss the limitations of the current study and potential future work.
\subsection{Limitation}
Our study has three main limitations.
First, our current analysis is limited to data from a single real-world classroom's collaborative programming discussions, restricting the generalizability of our findings to other contexts. Similarly, our evaluation of \textit{CPVis} relies on a sampled dataset, limiting the study's scope. We hypothesize that participants working with smaller datasets and visualized learning analytics experience reduced cognitive load and find it easier to identify collaboration patterns due to fewer visual elements to process. However, in large-scale collaborative programming classrooms, instructors face the challenge of evaluating more groups and students, which may increase memory load and visual complexity.
Second, the data collected in our study are obtained from real classroom environments, maintaining ecological validity by capturing natural behaviors such as group silence or requests for instructor assistance. However, due to the limitations of non-intrusive equipment, our data lack details such as facial expressions and non-verbal cues. While participants report the comprehensiveness and richness of the learning analytics in the experiment, the absence of these data poses challenges for deeper analysis of emotional expressions and social engagement during collaborative programming. This limitation hinders the provision of a more holistic learning analysis for evaluation purposes.
Additionally, the recorded data are independent and exclude audio information, making it difficult to align screen interactions with dialogue streams. This limitation constrains the exploration of the relationship between collaborative behavior patterns and collaborative problem-solving processes.
Finally, in large-scale collaborative programming classrooms, generating analytics using LLMs requires significant computational time and cost. While feasible for institutions with robust computational resources, this remains a limitation for deploying such tools in real teaching scenarios. Furthermore, in real classrooms, noise from multiple group discussions introduces significant data noise, complicating the automation of learning analytics generation and limiting the accuracy of evaluations for groups and individual students.


\subsection{Future Work}
Without well-structured visualizations, simply presenting multiple data streams poses significant challenges for instructors attempting to interpret these large-scale datasets~\cite{fernandez2024data}.
In this study, we explore the integration and analysis of multimodal data. However, \textit{CPVis} has the potential to further enhance the visualization and perception of multimodal data, enabling instructors to evaluate group and student performance with greater accuracy and reduced cognitive load~\cite{martinez2020data}. 
Our target audience consists of instructors teaching large introductory collaborative programming courses, who require more efficient and intuitive visualizations to understand student performance during collaboration.  
While our use of static 2D visualizations, such as high-dimensional flower glyphs, has been highly regarded by participants for boosting confidence and helping instructors quickly identify key features, we believe there is room for improvement in organizing visualization formats to enhance information transmission efficiency and the users cognitive experience.
For instance, incorporating narrative visualizations further streamlines the process by allowing instructors to generate composite evaluations based on their weighting of different collaboration performance dimensions~\cite{gratzl2013lineup}. 
Narrative visualizations enable instructors to delve into data details, organize learning analytics results along logical paths such as timelines, causality, or categories, and highlight key information~\cite{chen2019designing}. 
This approach mitigates visual overload caused by excessive data, significantly reduces the time and cognitive effort required for evaluation, and ultimately supports instructors in making better decisions and assessments.

\textit{CPVis} requires instructors to spend additional time after class to evaluate collaborative performance. In our study, most participants indicate during follow-up interviews that the extra time spent on evaluating students' collaborative performance is highly valuable for producing comprehensive assessments. They note that providing immediate evaluations during the collaboration process is unrealistic, as final assessments typically need a holistic consideration of task completion and group dynamics after class. However, there is a significant demand for real-time analysis tools to deliver timely, personalized feedback to students and offer appropriate instructional scaffolding during the collaborative process~\cite{tang2024sphere}.
Instructors frequently find themselves overwhelmed by the immediate needs of some students~\cite{yang2023pair}, unintentionally neglecting others. To address this issue, future work could explore the integration of LLMs to enable real-time monitoring and analysis of students' behavioral data—such as code submissions, error logs, and engagement levels. LLMs could automatically detect learning bottlenecks or collaboration issues, providing instant feedback on common problems to students. This would effectively reduce instructors' workload, allowing them to focus on complex or critical issues, and simplify classroom management tasks.
For instance, LLMs could summarize patterns in students' code submissions and generate a ``hotspot report'' identifying recurring issues across the class. They could also provide real-time collaborative performance analytics for different groups, enabling instructors to quickly gain a comprehensive understanding of overall class dynamics. Additionally, LLMs could assist in role allocation within groups, suggest strategies to improve team interactions, and identify potential conflicts or disengagement within collaborative teams.
LLM-powered tools automate evaluations and enable personalized feedback, bridging post-class assessments with in-class scaffolding to enhance teaching and learning in collaborative programming.}
\section*{Conclusion}
This paper aims to enhance our understanding of the computational complexity of computing various Shapley value variants. We found that for various ML models --- including decision trees, regression tree ensembles, weighted automata, and linear regression --- both local and global interventional and baseline SHAP can be computed in polynomial time under HMM modeled distributions. This extends popular algorithms, such as TreeSHAP, beyond their empirical distributional scope. We also establish strict complexity gaps between the various SHAP variants (baseline, interventional, and conditional) and prove the intractability of computing SHAP for tree ensembles and neural networks in simplified scenarios. Overall, we present SHAP as a versatile framework whose complexity depends on four key factors: \begin{inparaenum}[(i)] \item model type, \item SHAP variant, \item distribution modeling approach, \item and local vs. global explanations\end{inparaenum}. We believe this perspective provides deeper insight into the computational complexity of SHAP, paving the way for future work.




%We believe that our framework provides a more intricate understanding of SHAP computation complexity across different models, distributions, and variants, paving the way for further research.

Our work opens promising directions for future research. First, expanding our computational analysis to other SHAP-related metrics, such as asymmetric SHAP~\citep{frye20} and SAGE~\citep{covert2020understanding}, would be valuable. Additionally, we aim to explore more expressive distribution classes and relaxed assumptions beyond those in Section \ref{sec:tractable} while maintaining tractable SHAP computation. Finally, when exact computation is intractable (Section \ref{sec:intractable}), investigating the approximability of SHAP metrics through approximation and parameterized complexity theory~\citep{downey2012parameterized} is an important direction.

%Our work opens several promising avenues for future research on the computational properties of explainable AI methods, with a particular focus on SHAP. First, it would be interesting to broaden the computational analysis conducted in this work to include other popular SHAP-related metrics in the literature, such as asymmetric SHAP \cite{frye20} and SAGE \cite{covert2020understanding}. Also, in the future, we aim to explore more expressive distribution classes and relaxed distributional assumptions—extending beyond those examined in Section \ref{sec:tractable} —that still yield tractable SHAP computation. Finally, when exact computation proves intractable (Section \ref{sec:intractable}), it is worthwhile to theoretically investigate the question of the approximability of computing the SHAP metrics across various configurations, through the lens of approximation and parametrized complexity theory \cite{arora2009computational}.

%This paper aims to deepen our understanding of the computational complexity involved in obtaining different Shapley value variants. We found that for a variety of ML models, including decision trees, tree ensembles for regression, weighted automata, and linear regression models — computing both local and global interventional and baseline SHAP can be done in polynomial time when distributions are modeled by HMMs. This extends the distributional scope of popular algorithms like TreeSHAP, which is limited to empirical distributions. Additionally, we demonstrate a strict complexity gap between SHAP variants, showing that interventional and baseline SHAP can be strictly easier to compute than conditional SHAP. Despite these positive results, we uncovered intractability for various SHAP variants in neural networks and tree ensembles. Finally, we provided generalized complexity relations across SHAP variants. We believe that our framework offers a deeper understanding of the complexity involved in computing SHAP across various variants, models, distributions, as well as in both local and global computations, laying the groundwork for future research.

%%
%% The acknowledgments section is defined using the "acks" environment
%% (and NOT an unnumbered section). This ensures the proper
%% identification of the section in the article metadata, and the
%% consistent spelling of the heading.
\begin{acks}
Meng Xia is the corresponding author.
The work was supported by the National Natural Science Foundation of China, (62422607, 62372411, 62036009) and the Zhejiang Provincial Natural Science Foundation of China.

\end{acks}
%%
%% This is file `sample-sigconf-authordraft.tex',
%% generated with the docstrip utility.
%%
%% The original source files were:
%%
%% samples.dtx  (with options: `all,proceedings,bibtex,authordraft')
%% 
%% IMPORTANT NOTICE:
%% 
%% For the copyright see the source file.
%% 
%% Any modified versions of this file must be renamed
%% with new filenames distinct from sample-sigconf-authordraft.tex.
%% 
%% For distribution of the original source see the terms
%% for copying and modification in the file samples.dtx.
%% 
%% This generated file may be distributed as long as the
%% original source files, as listed above, are part of the
%% same distribution. (The sources need not necessarily be
%% in the same archive or directory.)
%%
%%
%% Commands for TeXCount
%TC:macro \cite [option:text,text]
%TC:macro \citep [option:text,text]
%TC:macro \citet [option:text,text]
%TC:envir table 0 1
%TC:envir table* 0 1
%TC:envir tabular [ignore] word
%TC:envir displaymath 0 word
%TC:envir math 0 word
%TC:envir comment 0 0
%%
%%
%% The first command in your LaTeX source must be the \documentclass
%% command.
%%
%% For submission and review of your manuscript please change the
%% command to \documentclass[manuscript, screen, review]{acmart}.
%%
%% When submitting camera ready or to TAPS, please change the command
%% to \documentclass[sigconf]{acmart} or whichever template is required
%% for your publication.
%%
%%
\documentclass[sigconf, nonacm]{acmart}
%\documentclass[sigconf]{acmart}
%\documentclass[sigconf,authordraft]{acmart}
%\documentclass[anonymous,manuscript,review]{acmart}

%%
%% \BibTeX command to typeset BibTeX logo in the docs
\AtBeginDocument{%
  \providecommand\BibTeX{{%
    Bib\TeX}}}

%% Rights management information.  This information is sent to you
%% when you complete the rights form.  These commands have SAMPLE
%% values in them; it is your responsibility as an author to replace
%% the commands and values with those provided to you when you
%% complete the rights form.
%\setcopyright{acmlicensed}
\copyrightyear{2025}
\acmYear{2025}
%\setcopyright{cc}
\setcctype{by}
\acmConference[CHI '25]{CHI Conference on Human Factors in Computing Systems}{April 26-May 1, 2025}{Yokohama, Japan}
\acmBooktitle{CHI Conference on Human Factors in Computing Systems (CHI '25), April 26-May 1, 2025, Yokohama, Japan}
\acmDOI{10.1145/3706598.3713353}
\acmISBN{979-8-4007-1394-1/25/04}


%%
%% Submission ID.
%% Use this when submitting an article to a sponsored event. You'll
%% receive a unique submission ID from the organizers
%% of the event, and this ID should be used as the parameter to this command.
%%\acmSubmissionID{123-A56-BU3}

%%
%% For managing citations, it is recommended to use bibliography
%% files in BibTeX format.
%%
%% You can then either use BibTeX with the ACM-Reference-Format style,
%% or BibLaTeX with the acmnumeric or acmauthoryear sytles, that include
%% support for advanced citation of software artefact from the
%% biblatex-software package, also separately available on CTAN.
%%
%% Look at the sample-*-biblatex.tex files for templates showcasing
%% the biblatex styles.
%%

%%
%% The majority of ACM publications use numbered citations and
%% references.  The command \citestyle{authoryear} switches to the
%% "author year" style.
%%
%% If you are preparing content for an event
%% sponsored by ACM SIGGRAPH, you must use the "author year" style of
%% citations and references.
%% Uncommenting
%% the next command will enable that style.
%%\citestyle{acmauthoryear}
\usepackage{booktabs}
\usepackage{graphicx}
\usepackage{listings}
\usepackage{xcolor} 
\usepackage{caption}
%\usepackage{paralist}
\usepackage{makecell}
\usepackage{url}
\usepackage{longtable}
\usepackage{color}
\usepackage{tikz}
%\usepackage[utf8]{enc}
\usepackage[T1]{fontenc}
\usepackage{lipsum}
\usepackage{stfloats}
\usepackage{tabu} 
\usepackage{tabularx}
\usepackage{multirow}
\usepackage{booktabs}
\usepackage{graphicx}
\usepackage{wrapfig}
\usepackage{hyperref}   
\usepackage{cleveref}
%\usepackage[most]{tcolorbox} 
\usepackage{xcolor}
\usepackage{float}
\usepackage{listings} 
%\usepackage{tcolorbox}




\definecolor{PurpleColor}{RGB}{0,0,0}
\newcommand{\RR}[1]{{\color{PurpleColor}#1}}

\definecolor{PinkColor}{RGB}{0, 0, 0}
\newcommand{\RA}[1]{{\color{PinkColor}#1}}


\definecolor{codegreen}{rgb}{0,0.6,0}
\definecolor{codegray}{rgb}{0.5,0.5,0.5}
\definecolor{codepurple}{rgb}{0.58,0,0.82}
\definecolor{backcolour}{rgb}{0.95,0.95,0.92}

\lstdefinestyle{mystyle}{
  backgroundcolor=\color{backcolour}, commentstyle=\color{codegreen},
  keywordstyle=\color{magenta},
  numberstyle=\tiny\color{codegray},
  stringstyle=\color{codepurple},
  basicstyle=\ttfamily\footnotesize,
  breakatwhitespace=false,         
  breaklines=true,                 
  captionpos=b,                    
  keepspaces=true,                 
  numbers=left,                    
  numbersep=5pt,                  
  showspaces=false,                
  showstringspaces=false,
  showtabs=false,                  
  tabsize=2
}
\lstset{style=mystyle}


%\definecolor{PurpleColor}{RGB}{0,0,0}

%%
%% end of the preamble, start of the body of the document source.
\begin{document}

%%
%% The "title" command has an optional parameter,
%% allowing the author to define a "short title" to be used in page headers.
\title{CPVis: Evidence-based Multimodal Learning Analytics for Evaluation in Collaborative Programming}

%%
%% The "author" command and its associated commands are used to define
%% the authors and their affiliations.
%% Of note is the shared affiliation of the first two authors, and the
%% "authornote" and "authornotemark" commands
%% used to denote shared contribution to the research.
\author{Gefei Zhang}
\affiliation{%
  \institution{Zhejiang University of Technology}
  \city{Hangzhou}
  \state{Zhejiang}
  \country{China}}
\email{gefei@zjut.edu.cn}
\orcid{1234-5678-9012}

\author{Shenming Ji}
\affiliation{%
  \institution{Xi'an Jiaotong-Liverpool University}
  \city{Suzhou}
  \state{Jiangsu}
  \country{China}}
\email{shenming.ji21@student.xjtlu.edu.cn}

\author{Yicao Li}
\affiliation{%
  \institution{Zhejiang University of Technology}
  \city{Hangzhou}
  \state{Zhejiang}
  \country{China}}
\email{yicaoli47@gmail.com}

\author{Jingwei Tang}
\affiliation{%
  \institution{Zhejiang University of Technology}
  \city{Hangzhou}
  \state{Zhejiang}
  \country{China}}
\email{jwtang@zjut.edu.cn}

\author{Jihong Ding}
\affiliation{%
 \institution{Hannan University}
 \city{Haikou}
 \state{Hainan}
 \country{China}}
\email{jhding@hainanu.edu.cn}

\author{Meng Xia}
\affiliation{%
  \institution{Texas A\&M University}
  \city{College Station}
  \state{Texas}
  \country{USA}}
\email{mengxia@tamu.edu}

\author{Guodao Sun}
\affiliation{%
  \institution{Zhejiang University of Technology}
  \city{Hangzhou}
  \state{Zhejiang}
  \country{China}}
\email{guodao@zjut.edu.cn}

\author{Ronghua Liang}
\affiliation{%
  \institution{Zhejiang University of Technology}
  \city{Hangzhou}
  \state{Zhejiang}
  \country{China}}
\email{rhliang@zjut.edu.cn}



%%
%% By default, the full list of authors will be used in the page
%% headers. Often, this list is too long, and will overlap
%% other information printed in the page headers. This command allows
%% the author to define a more concise list
%% of authors' names for this purpose.
\renewcommand{\shortauthors}{Zhang et al.}

%%
%% The abstract is a short summary of the work to be presented in the
%% article.
\begin{abstract}
As programming education becomes more widespread, many college students from non-computer science backgrounds begin learning programming. Collaborative programming emerges as an effective method for instructors to support novice students in developing coding and teamwork abilities. However, due to limited class time and attention, instructors face challenges in monitoring and evaluating the progress and performance of groups or individuals. To address this issue, we collect multimodal data from real-world settings and develop \textit{CPVis}, \RR{an interactive visual analytics system designed to assess student collaboration dynamically.} Specifically, \textit{CPVis} enables instructors to evaluate both group and individual performance efficiently. \textit{CPVis} employs a novel flower-based visual encoding to represent performance and provides time-based views to capture the evolution of collaborative behaviors. A within-subject experiment (N=22), \RR{comparing \textit{CPVis} with two baseline systems, reveals that users gain more insights, find the visualization more intuitive, and report increased confidence in their assessments of collaboration.}
\end{abstract}

%%
%% The code below is generated by the tool at http://dl.acm.org/ccs.cfm.
%% Please copy and paste the code instead of the example below.
%%

\begin{CCSXML}
<ccs2012>
   <concept>
       <concept_id>10010405.10010489.10010492</concept_id>
       <concept_desc>Applied computing~Collaborative learning</concept_desc>
       <concept_significance>500</concept_significance>
       </concept>
 </ccs2012>
\end{CCSXML}

\begin{CCSXML}
<ccs2012>
   <concept>
       <concept_id>10003120.10003145.10003151</concept_id>
       <concept_desc>Human-centered computing~Visualization systems and tools</concept_desc>
       <concept_significance>500</concept_significance>
       </concept>
 </ccs2012>
\end{CCSXML}

\ccsdesc[500]{Human-centered computing~Visualization systems and tools}

\ccsdesc[500]{Applied computing~Collaborative learning}


\begin{comment}



\end{comment}
%%
%% Keywords. The author(s) should pick words that accurately describe
%% the work being presented. Separate the keywords with commas.
\keywords{Group visualization, education visualization, collaborative programming}
%% A "teaser" image appears between the author and affiliation
%% information and the body of the document, and typically spans the
%% page.








% \received{20 February 2007}
% \received[revised]{12 March 2009}
% \received[accepted]{5 June 2009}

%%
%% This command processes the author and affiliation and title
%% information and builds the first part of the formatted document.
\maketitle
%\newtcbox{\inlinecode}{on line, colback=gray!10, colframe=gray!10, boxrule=0.1mm, 
%	rounded corners, fontupper=\bfseries, left=0.1mm, right=0.1mm, top=0.1mm, bottom=0.1mm}

\documentclass[../main.tex]{subfiles}
\graphicspath{{../images/}}
\makeatletter
\def\input@path{{../images/}}
\makeatother
\begin{document}
\section{Introduction}
\begin{figure}
\centering
\begin{tikzpicture}
\node[inner sep=0pt] (ws) at (0, 0) {
\includegraphics[height=.4\textwidth, trim={10cm 0 10cm 0},clip]{world_space.png}};
\node[inner sep=0pt] (cs) at (6,0) {\includegraphics[height=.4\textwidth, trim={10cm 1cm 10cm 4cm},clip]{conf_space.png}};
\end{tikzpicture}
\vspace{-5pt}
\label{fig:pbrm_intro}
\caption{\textbf{Left}: Shows world space obstacles as grey spheres. Robots start and goal configuration is colored red and green, respectively. Configurations along the computed path are colored transparent blue. \textbf{Right:} Mapped world space scenario to configuration space. Obstacle region is the grey mesh. Red spheres are collision-free regions computed by the neural SCDF. The optimized shortest path in the convex corridor is the blue curve.}
\vspace{-25pt}
\end{figure}
Motion planning is the problem of finding a collision-free trajectory that connects a given start and goal configuration. The planning takes place in the configuration space of the robot. For single body robots, like mobile robots or drones, the configuration space and the world space are usually the same. This simplifies the planning, since explicit obstacle representations are available which enables geometrical tools like separating hyperplanes, smallest distance to obstacles etc., to be used when designing motion planning algorithms. For multi-body robots like manipulators, the situation is completely different. The world space obstacles are usually mapped to non-convex regions, and to make the problem even harder, the mapping is usually not known. Forming explicit representations of the obstacle region in the configuration space is usually too expensive or intractable. Despite all of this, sampling based planners are used with great success, which mainly is due to their use of implicit representations of the obstacle region. The basic idea is to construct a graph in the configuration space that covers and connects the collision-free region. From this graph, a path can be extracted that connects a given start and goal configuration. The approach is computationally expensive, since the graph is constructed with the smallest geometrical building block available, points, which represents a collision-check. Furthermore, the extracted paths from the graph are non-smooth and jagged due to the stochastic nature of the approach. This adds an additional post-processing step to the process, where the paths are shortcutted and smoothened, before the path can be used for tracking. Clearly a lot of time is invested to form this graph and produce smooth paths. Thus, if the obstacles start to move, then all of this work is done in no use, since all points that make up this graph need to be re-verified, which is simply too time consuming to be done in real time.
\\\\
In this work, we want to address the existing drawbacks of the sampling based planners. Our main contribution is an improved motion planner where each vertex in the graph covers a collision-free region in the form of a sphere instead of a point and where the edges are formed with neighboring intersecting spheres. This representation has the advantage of instead of returning piecewise linear paths, returning a sequence of overlapping spheres, i.e. a convex corridor, that connects a given start and goal configuration, illustrated in Figure \ref{fig:pbrm_intro}. This convex corridor allows us to use convex optimization to produce smooth trajectories, instead of computationally expensive post-processing methods. The representation further allows us to estimate the coverage of the collision-free space, which gives us awareness and feedback in the offline roadmap construction phase. Finally, our representation is simple to adapt to moving obstacles, simply requery for the new radii and recheck for intersections. 
\\\\
The spherical collision-free regions are formed using a signed distance function (SDF), which is a function that returns the smallest distance from an arbitrary point to the boundary of an obstacle. As the name implies, the distance is signed, thus if the point is inside the obstacle it is negative otherwise positive. If the distance is positive, a sphere with radius equal to the distance is guaranteed to cover a collision-free region. Using an SDF in motion planning is not new, but what is novel about our approach is that we express the distance in the configuration space instead of the world space and by doing so allows us to form these convex collision-free regions. We refer to the resulting SDF as a signed configuration distance function (SCDF). Computing an SCDF analytically is non-trivial, our approach is therefore to parameterize the SCDF with a deep neural network and learn the mapping by supervised learning. Our resulting neural SCDF can compute distances for different parameter values of obstacle shapes and we also show how multiple distances can be combined, thus making our approach flexible.
\section{Related work}
Motion planning algorithms can roughly be divided into three families, grid-based, sampling based and optimization based methods. Grid-based methods (GBM) discretize the planning space from which a graph is then compiled. A standard search method is A$^\star$ \citep{a_star}, which is classified as an \textit{informed} search method, since it employs a heuristic function to speed up the search. A$^\star$ guarantees to return an optimal path at the level of discretization used. GBMs usually discretize the planning space by a regular lattice and this limits the GBMs to problems with low dimensionality due to the curse of dimensionality. Thus, GBMs are usually limited to single-body robots where the degrees of freedom (DOF) are low. To overcome the inherent scaling problem with the GBMs, stochastic methods are usually used for multi-body robots. These methods are termed as sampling-based methods (SBM) and core members within this family are the rapidly-exploring random trees (RRT) \citep{rrt} and the probabilistic roadmap (PRM) \citep{prm}. RRT grows a tree from the start configuration and explores the collision-free region in a rapid way until it is able to connect to the goal region. RRT is usually improved by bi-directional planning \citep{rrt_connect}, i.e. an additional tree is grown from the goal configuration and the trees are tested for connection after any tree has been expanded. RRT is a single-query method, thus it searches for a path from scratch each time it is queried. Contrary to this, PRM is a multi-query method, which solves for multiple queries without starting from scratch. PRM does this by creating a roadmap (graph) that covers the collision-free space as an offline step. The graph is then used to solve for multiple queries. PRMs are used in cases where the environment does not change since the extra offline step is too computationally costly and needs to be re-done if the environment is changed. In our work, we address this inherent issue by using a different roadmap representation. Our vertices in the graph cover a collision-free region in the form of spheres and we form the edges by checking for intersecting spheres. If something in the environment changes, we recompute the spheres radii and recheck the intersections, without relying on collision detection. We use a trained neural network to compute the sphere radius, therefore querying for the radius can be done fast, hence our representation enables the PRM for dynamic environments.
\\\\
In the recent decades, optimization based methods (OBM) \citep{chomp, schulman, itomp, stomp} have been introduced as an alternative to SBM for multi-body robots. Like the SBM, the OBMs scale well to higher dimensional problems and produce smoother motion. It is common to use a SDF in the optimization since it is a smooth function, thus enabling gradient-based methods. However, the standard way of expressing the SDF is in world space. The distance therefore needs to be mapped to the configuration space by the forward kinematics. This mapping makes the optimization problem a non-linear program (NLP), which is computationally expensive to solve. Recently, a different approach has been proposed. In \cite{mp_gcs} motion planning is formulated as a convex optimization problem by using the graph of convex sets framework \citep{gcs}. The underlying idea is to decompose the collision-free space into intersecting convex sets from which a convex optimization problem is formulated. In cases where an explicit representation of the obstacles in the configuration space exists, like for single-body robots, creating collision-free convex regions can be done fast \citep{iris}. For multi-body robots, this is non-trivial. Existing work does this successfully \citep{iris_nlp, iris_c} by an optimization based approach, but the methods are still too time consuming to be used in the presence of moving obstacles. Our approach is instead to use deep learning to learn an SDF expressed in the configuration space. With this, we can query for shortest distances to the collision boundary, which allows us to expand spherical regions which are collision-free. Our approach is fast and therefore enables our suggested roadmap planner to be used in dynamic environments.
\\\\
Recent research has focused on learning collision detection \citep{fk_kernel_distance, diffco, graphdistnet} by predicting the signed distance between the robot links and the surrounding obstacles in the world space. The learned SDF is used in trajectory optimization but since the distance is expressed in the world space, the problem becomes an NLP and therefore takes a long time to solve. We take a novel approach and suggest to instead express the signed distance in the configuration space. This allows us to improve the PRM at the same time as it enables convex optimization for trajectory optimization, which runs faster and is more reliable than NLP solvers. In \cite{cspf} a learned signed distance function in the configuration space is proposed similar to our approach. However, their approach is restricted to point cloud representations, while we propose to represent the obstacles as parameterized geometric shapes, e.g. spheres. Furthermore, we also show how to use our learned SCDF to improve an existing roadmap planner.
\section{Problem formulation}
A robot is located in the world space, $\W \subset \R^3 $. The unique location of the robot is given by its configuration $\q \in \C$, where $\C$ is the configuration space. The set of points covered by the robots bodies at a certain configuration is expressed as $\B(\q) \subset \W$. The robot is surrounded by $\NrObst$ obstacles $\O = \bigcup_{i=1}^{\NrObst} \O_i$, where  $\O_i \subset \W$. The representation of the obstacle in the configuration space is the set $\C\O_i = \{\q \in \C \: |\: \B(\q) \cap \O_i \neq \emptyset \}$. The obstacle space is formed as $\Co = \bigcup_{i=1}^{\NrObst} \C \O_i$. The complement is referred to as the free space, $\Cf = \C \setminus \Co$. The path planning problem is a tuple, ($\Cf$, $\qStart$, $\qGoal$), where we want to connect a query pair, consisting of a start, $\qStart$, and goal configuration, $\qGoal$, with a geometric path, $\q(s): [0, 1] \mapsto \Cf$, such that $\q(0)=\qStart$ and $\q(1)=\qGoal$, or report correctly when such a path does not exist.
\end{document}


\section{Related Work} \label{sec:related}

% \textbf{Adversarial Attack}
\textbf{Attacks on SLAM.} 
%With the rise of machine learning, 
The robustness of computer vision systems is being actively investigated. With the emergence of adversarial images in the digital domain by adding optimized noise directly to images~\cite{szegedy2013intriguing,carlini2017towards}, researchers find that such attacks also exist physically in the real world \cite{eykholt2018robust,song2018physical,zhao2019seeing}. To fill the gap between attacks in the digital and physical worlds, recent studies have demonstrated that attacks on real-world computer vision systems are practical \cite{eykholt2018robust,li2019adversarial,man2020ghostimage,sharif2016accessorize,zhao2019seeing,zhou2018invisible}. However, attacks on traditional computer vision methods such as SLAM are relatively less explored. \cite{yoshida2022adversarial} proposes an attack against the scan matching algorithm in LiDAR-based SLAM, while most SLAMs in AR/VR devices rely on different sensors like RGB/depth cameras and IMUs. \cite{ikram2022perceptual} and \cite{chen2024adversary} mislead visual SLAM by poisoning the images with special patterns, and \cite{wang2021can} causes the camera to fail using infrared light. In our work, we demonstrate attacks on Visual-Inertial SLAM (VI-SLAM) by perturbing the IMU readings, rather than cameras, and showing its impact on XR user experience. 

\textbf{Acoustic Injection Attacks.} Among various physical attacks, acoustic injection attacks are attractive due to their low cost. Son~\etal~\cite{son2015rocking} were the first to introduce acoustic attacks on MEMS gyroscopes, demonstrating how these attacks could lead to sensor denial-of-service and result in drone crashes. WALNUT~\cite{trippel2017walnut} expanded on this by developing output biasing and control attacks that enable precise manipulation of MEMS accelerometer outputs using modulated sound waves. Wang et al.~\cite{wang2017sonic} demonstrated a sonic gun, showcasing the vulnerability of various smart devices (\eg drones and self-balancing vehicles) to acoustic attacks. Tu et al. \cite{tu2018injected} designed side-swing and switching attacks to alter the outputs of MEMS gyroscopes and accelerometers. Furthermore, Ji et al. \cite{ji2021poltergeist} fool the object detectors by applying acoustic attack to the image stabilizers commonly used in modern cameras. However, none of the existing works study the relationship between the acoustic injections and SLAM outputs on recent XR devices. 

% \zijian{Do we need one session about security in AR/VR?}
% \yicheng{TODO}
%\jiasi{cite the AIVR paper (UMass Amherst?) paper is we have not already. They add IMU perturbation but w/o SLAM, iirc} \yicheng{Cited}

\textbf{XR Security and Privacy.} 
%Security and privacy concerns in XR systems have gained significant attention. 
For single-user XR systems, researchers have demonstrated various side-channel attacks to extract sensitive information (\eg keystrokes) through video feeds~\cite{ling2019know}, head movements~\cite{nair2023unique, slocum2023going}, architectural hints~\cite{zhang2023its,shang2020arspy}, power usage~\cite{li2024dangers}, and EM side-channel leakages~\cite{al2021vr}. In multi-user XR systems, Su et al.~\cite{su2024remote} use avatar motion data to infer keystrokes in shared VR environments. Slocum et al.~\cite{slocum2024doesn} reveal vulnerabilities in the shared state frameworks of multi-user AR. Similarly, Lebeck et al.~\cite{lebeck2017securing} highlight risks like deceptive virtual objects and emphasize access control for managing shared physical and virtual spaces. Ruth et al.~\cite{ruth2019secure} further propose a secure multi-user AR framework focusing on content sharing and permissions.
Chandio et al.~\cite{chandio2024stealthy} %introduced a multi-modal spatiotemporal attack that 
simultaneously manipulated visual and inertial sensors to disrupt XR pose estimation. However, their study evaluated the attack using offline datasets and assumed the attacker's capability to manipulate IMU data streams through acoustic means, without real experiments. Ours is the first to demonstrate acoustic injection attacks on recent XR devices, like the Hololens 2, in the real world.
 


\section{Formative Study} 
\label{sec:formative}
We conducted a formative study to explore instructors' challenges in collaborative programming and their visualization needs, with the study protocol approved by our university's Institutional Review Board (IRB).
 
 \subsection{Participants}
We recruited ten participants (five females, age: \(30.6 \pm 6.6\)) with collaborative programming experience, divided into three groups: two educational technology experts (E1, E2) and eight instructors (T1–T8) with an average of 8.14 years of teaching programming. Participants, recruited via snowball sampling from the authors' network, received \$20 as compensation.
 

\subsection{Procedure}\
\RR{The formative study used semi-structured interviews conducted via Zoom, divided into two sections: (A) Questions and Answers and (B) Ratings.}
\textit{A: Questions and Answers.} 
Each participant was independently interviewed to explore the need for collaborative programming analysis. Topics included teaching experience with collaborative programming, class organization, challenges faced, assessment methods, and group and individual performance evaluations. Follow-up questions were asked for clarification or more profound insights. Each session lasted 40–60 minutes and was documented through written notes, audio, and video recordings.
\textit{B: Rating.} 
Participants then rated two aspects of the collaborative programming visual analytics system. They rated feature importance (Q1) on a 1–7 scale (7 being the highest) and ranked features by priority (Q2). 
\RR{More details are in appendix A.}

 
\subsection{Findings} 

\begin{figure*}[htbp]
	\centering
	\includegraphics[width=1\linewidth]{scheme1.pdf}
	\caption{Collaborative programming coding schemes, along with their definitions and examples.}
        \Description{Collaborative programming coding schemes, along with their definitions and examples.}
	\label{fig:scheme}
\end{figure*}



\subsubsection{Evaluation of Groups:}
In large classes, instructors struggled to monitor each group without the help of teaching assistants. They often relied on group presentations (E2, T1, T5), but it was inaccurate. Limited class time forced instructors to focus on solving issues rather than actively monitoring groups.


\textbf{Group Performance:}
Evaluating group work was based on the final code rather than the process, limiting the ability to provide feedback (T7). Some instructors used presentations or technical documentation to streamline assessments (T6). However, evaluating more than 20 groups was still challenging, and instructor assistance was often overlooked during evaluations.


\textbf{Collaboration Evaluation:}
Effective collaboration wasn't just about task completion speed and group dynamics. Some groups (T7) completed tasks quickly due to one member's efforts, not true collaboration. 
\RR{E1 proposed the collaborative problem-solving framework to distinguish task effectiveness from team effectiveness.} Instructors (T6) believed monitoring discussion time helped assess collaboration quality and task difficulty, but off-topic discussions made it hard to evaluate group discussions (E1, T3, T5).


\subsubsection{Evaluation of Students:}

In large classes, assessing individual contributions in group work was challenging. Instructors relied on peer evaluations (T2, T3, T4) and self-reported task distributions (E2), which were often subjective.

\textbf{Individual Performance:}
Instructors typically reviewed code to assess understanding, but measuring individual contributions in group work was hard. Leadership roles often reflected a deeper grasp of concepts (T1), but tracking individual engagement was difficult in large classes.


\textbf{Personalized Feedback:}
Providing personalized feedback was difficult, as group results often masked individual struggles. T3 and T4 noted that group collaboration fostered peer learning but could lead to less engagement from weaker students.
\textit{T8 added that offering personalized feedback in large classes was time-consuming and burdensome.}

 
 
 
\subsection{Design requirement}
\label{dr}
Based on the interview findings, we identified six design requirements (R1–R6) across three levels, summarized as follows:

Support \textbf{\textit{inter-group-level}} to provide a macro perspective, enabling instructors to observe the overall situation of all groups comprehensively and fully understand class-wide dynamics.

\textbf{R1: Displaying the Overall Performance of all Groups.} 
Instructors face challenges in supervising multiple groups simultaneously and shifting focus efficiently. Participants stressed the need for an overview of group performance, allowing instructors to grasp class dynamics and selectively review specific groups.


\RR{\textbf{R2: Comparing Similar and Different Groups.} 
Instructors often compare students' performance to assess their relative standing within the class~\cite{marsh1997making}. Such comparisons enable a more accurate evaluation of group performance and help identify groups excelling or encountering challenges.}


Supporting \textbf{\textit{intra-group-level}} visual exploration to offer a meso perspective, enabling instructors to observe specific groups' performance and gain a comprehensive understanding of group dynamics during the collaborative programming process.


\textbf{ R3: Understand the Dynamics of Programming Problem Solving.}
Analyzing a group's evolving communication patterns and computational thinking during programming tasks provides instructors with deeper insights into students' progress and intermediate learning objectives—details missed in final submissions alone.


\textbf{ R4: Identify Teacher Scaffolding in Collaboration.}
Instructors play a vital role in guiding groups during collaborative programming. Understanding the scaffolding provided and students' responses can help refine instructional strategies, improving the overall effectiveness of collaborative programming.


Supporting \textbf{\textit{individual-student-level}} visual exploration to provide a micro perspective allows instructors to observe each student's performance within a specific group and better understand their role and collaboration.

\textbf{R5: Track Changes in Student Engagement Over Time.} 
Limited classroom time makes it challenging for instructors to monitor individual student engagement in programming tasks.
Tracking and visualizing engagement trends is essential for assessing performance and refining instructional practices.

\textbf{R6: Access Detailed Raw Data.} 
Instructors require access to raw data, such as collaboration videos, conversations, and background information. These details are crucial for validating analysis results and supporting personalized feedback and assessments.


\begin{figure*}
    \centering
    \includegraphics[width=1\linewidth]{bar2.pdf}
    \caption{(a) shows the bar chart of the raw data, (b) presents the results of applying Moving Average Smoothing to reduce anomalies in prediction percentages, and (c) highlights the reduction of visual clutter and emphasizes sequential behavior patterns after merging behaviors of the same category.}
    \label{fig:bar}
    \Description{(a) shows the bar chart of the raw data, (b) presents the results of applying Moving Average Smoothing to reduce anomalies in prediction percentages, and (c) highlights the reduction of visual clutter and emphasizes sequential behavior patterns after merging behaviors of the same category.}
\end{figure*}

\section{Data Collection and Processing}
\label{sec:data}
\RR{In this section, we provided an overview of the data collection context and introduced the collaborative programming performance framework along with its metric quantification methods.}

\subsection{Data Collection}
We collaborated with Professor E1, an expert in programming education, and teaching assistants (TA1 and TA2), experienced in Python, to collect data from E1's Spring 2023 Python course with 66 non-computer science freshmen in 22 groups. Using non-intrusive methods, we recorded group discussions, screen activities (without audio), and code submissions. Session lengths ranged from 10 to 60 minutes based on question completion. 
Due to data quality issues, we selected data from 19 groups (57 students) for analysis.


\subsection{Data Preprocessing}
In collaborative programming analysis, students' spoken content was key to understanding discussion and evaluating collaboration. We used the Faster-Whisper model~\cite{fasterwhisper} for speech recognition and the Pyannote-audio model~\cite{pyannoteaudio} for speaker diarization. 
For groups lacking clear problem-solving strategies, we used Tesseract OCR~\cite{tesseract} to analyze screen recordings and extract key frames through screenshots.

\subsection{Scope of Collaborative Programming Performance Framework}
Evaluating student and group performance in collaborative programming required considering multiple dimensions~\cite{hawlitschek2023empirical}.  
Building on literature and expert input (E1), we proposed the following comprehensive analytical framework to assess performance. 



\subsubsection{Student Performance Assessment}
\label{shema}
Previous research demonstrated that students' skills, backgrounds, and personalities in the classroom vary significantly, affecting their engagement and learning outcomes~\cite{wu2019analysing}. 
Therefore, we focus on each student's \textit{background} (prior academic performance and major), \textit{role transitions}, \textit{behavioral engagement}, and \textit{cognitive engagement}.






\textbf{Problem-solving Categorization:}
Based on previous frameworks~\cite{wu2019analysing}, team theory~\cite{zhao2023analysing}, and collaborative coding processes~\cite{sun2021three}, we developed a coding scheme (Fig.~\ref{fig:scheme}) to capture group problem-solving in collaborative programming. 
The scheme used four color-coded categories to represent discussion types. 
The first three categories followed a hierarchical structure, indicating discussion depth, while the green category focuses on situation awareness and specific behaviors.

Building on the scheme, we used tailored prompts with the ChatGPT-4o model~\cite{gpt4o} to classify behavioral patterns in transcribed dialogue \RR{(More details are in appendix B)}. 
\RR{The model provided a prediction percentage of uncertainty for each classification, improving result interpretability. }
To minimize anomalies, we applied a ``moving window'' technique with Moving Average Smoothing~\cite{chang2022muse}, stabilizing prediction percentages (Fig.\ref{fig:bar}-b). To reduce visual clutter in long time-series data, we aggregated consecutive instances of the same category, averaging prediction percentages (Fig.\ref{fig:bar}-c). These results were displayed in the timeline panel's progress bar, enabling detailed analysis by zooming into specific behavior categories in Sec.~\ref{barchart}. 




\textbf{Roles Extraction:}
We analyzed each speaker's dynamic roles (Driver, Navigator, and Monitor) during programming~\cite{lewis2011pair}. Using ChatGPT-4o and prompts based on the Thought Chain Model~\cite{wei2022chain}, we guided the model through step-by-step reasoning to generate role classifications. Prompts were iterated for clarity, and the model's responses were structured hierarchically and returned in JSON format. Each query was repeated ten times, with the majority result adopted for classification.

\RR{\textbf{Behavioral Engagement:} reflected the level of effort and participation students invested in learning~\cite{fredricks2022measurement}. 
In our study, we focused on the duration and frequency of student speech.} 
We extracted conversation data, excluding irrelevant chat, and divided each conversation into two parts: the first half and the full conversation. We then measured speaking duration, frequency, and degree centrality using co-occurrence networks~\cite{ng1999toward}. For each question, we created and normalized two networks, followed by Non-negative Matrix Factorization (NMF)~\cite{lee2000algorithms} to identify key behavioral patterns for dynamic group comparison.


\RR{\textbf{Cognitive Engagement:} referred to the cognitive investment students made in their learning. We highlighted the role changes and behavior frequencies of students during the collaborative process. }
To capture dynamic changes in student cognitive engagement, we split the dialogue for each question into two segments: the first half and the full dialogue. We extracted the frequency of each speaker's 14 behavioral categories and their roles at each timestamp. After normalizing these features for consistency, we applied NMF to reduce dimensionality and assess each speaker's cognitive engagement.

\begin{figure*}
  \includegraphics[width=\textwidth]{CPVis.pdf}
  \caption{\RR{A screenshot of Group 10 view.} \textit{CPVis} applies multimodal learning analysis to provide instructors with evidence for evaluating group and student performance. It consists of three views:
Filter View (A) Provides an overview and allows group selection. The selected groups appear in the lasso selection area (A2), and the similarity panel (A3) displays the most similar and different groups based on the search (A1a).
Content View (B) Displays group performance, with the B1 panel showing completed codes, the B3a panel illustrating the behavior sequence, and the B3b panel showing student engagement over time.
Detail View (C) Presents the group's collaborative programming video (C1) and raw conversation data (C2).}
  \Description{A screenshot of Group 10 view. \textit{CPVis} applies multimodal learning analysis to provide instructors with evidence for evaluating group and student performance. It consists of three views:
Filter View (A) Provides an overview and allows group selection. The selected groups appear in the lasso selection area (A2), and the similarity panel (A3) displays the most similar and different groups based on the search (A1a).
Content View (B) Displays group performance, with the B1 panel showing completed codes, the B3a panel illustrating the behavior sequence, and the B3b panel showing student engagement over time.
Detail View (C) Presents the group's collaborative programming video (C1) and raw conversation data (C2).}
  \label{fig:teaser}
  \end{figure*}

\subsubsection{Group Performance Assessment}
We evaluated group performance based on three dimensions: code quality, collaborative problem-solving, and teacher scaffolding. 
Through in-depth discussions with domain experts, we assessed how each dimension was valued and measured in the context of our study.




\label{code}
\textbf{Code quality}, reflecting students' mastery of course concepts, was a key metric for evaluating group performance. To assess student submissions, we used ChatGPT-4o~\cite{gpt4o} to evaluate dimensions such as problem-solving, code integrity, accuracy, and algorithmic innovation, scoring each on a 1–5 scale. After refining evaluation prompts, we ran the assessment ten times per submission, averaging the results to ensure consistency and reliability.





\textbf{Collaborative Problem-Solving (CPS):} 
Earlier studies categorized CPS into team effectiveness and task effectiveness~\cite{rosen2020towards}. Team effectiveness was measured by student engagement, while task effectiveness was assessed through code quality. %Our analysis captured problem-solving behaviors by frequency and sequence.
To evaluate CPS, we examined task effectiveness, represented by the average question score (\(\bar{s}\)), and team effectiveness, assessed through the standard deviation of engagement (\(\sigma_e\)) and the average engagement score (\(\bar{e}\)) as shown in Equation \ref{eq:1}. We then used the coefficient of variation (\(CV_e\)) \RR{to account for both engagement variability and engagement}. Finally, the overall collaboration quality was calculated using Equation \ref{eq:2}, combining question performance and engagement balance. 
\begin{equation}
\sigma_e = \sqrt{\frac{1}{n} \sum_{i=1}^{n} (e_i - \bar{e})^2}, \quad CV_e = \frac{\sigma_e}{\bar{e}}
\label{eq:1}
\end{equation}

\begin{equation}
Quality = \bar{s} \cdot (1 - CV_e)
\label{eq:2}
\end{equation}
As shown in Table \ref{table:comparison}, Group 19, despite achieving a respectable average score, exhibited imbalanced engagement, leading to a lower collaboration quality score. In contrast, Group 20 demonstrated more balanced and higher engagement, resulting in a better overall collaboration quality.
\begin{table}[htbp]
\centering
\begin{tabular}{cccccc}
\toprule
\textbf{Group} & \(\bar{s}\) & \textbf{Engagement Levels} & \(\sigma_e\) & \(\text{CV}_e\) & \textbf{CQ} \\
\midrule
Group 19 & \(4.11\) & (10.515, 9.725, 4.575) & \(2.80\) & \(0.24\) & \(2.80\) \\
Group 20 & \(4.14\) & (10.06, 9.32, 8.62) & \(0.73\) & \(0.08\) & \(3.88\) \\
\bottomrule
\end{tabular}
\caption{Comparison of Group 19 and Group 20 on Collaboration Quality (CQ).}
\label{table:comparison}
\end{table}

\textbf{Teacher Scaffolding,} categorized into cognitive (low, medium, high-control) and metacognitive forms~\cite{ouyang2022applying}, reflected the level of support provided to a group and its impact on programming performance. We evaluated four scaffolding dimensions, leveraging GPT-4o for annotation. By using targeted prompts and examples, we improved classification accuracy, while teacher scaffolding was categorized according to the type of support based on a semantic analysis of interactions.






\section{System Design} 
\label{sec:vis}

In this session, we introduce \textit{CPVis} (Fig.~\ref{fig:teaser}), a web-based visual analytics system to assist instructors in evaluating collaborative programming.
\subsection{System Overview}
\textit{CPVis} is a comprehensive system supporting multi-level, progressive analysis, from group-level interactions to individual student performance. Instructors can select specific groups for focused analysis, such as comparative evaluations (R2). Additionally, \textit{CPVis} offers a drill-down feature, enabling an overview of student collaboration and detailed insights into individual performance.
\textit{CPVis} includes four main components:
\textbf{Initial Selection}, instructors can select groups using the lasso tool or search function in the ``Group Overview'' view (Fig.\ref{fig:teaser}-A1). The system displays an overview of the selected groups (Fig.\ref{fig:teaser}-A2), compares similar and different groups (Fig.\ref{fig:teaser}-A3), and synchronizes updates across views (Fig.\ref{fig:teaser}-B).
\textbf{Drill-Down Analysis}, instructors can examine the group's code (Fig.\ref{fig:teaser}-B1) and analyze problem-solving approaches in the Content View. The interaction pattern panel (Fig.\ref{fig:student}-B2b) reveals behavior patterns, while the timeline panel (Fig.\ref{fig:teaser}-B3a) shows activity sequences. The student overview panel (Fig.\ref{fig:student}-B2a) compares individual performance across the class, and the timeline panel (Fig.~\ref{fig:teaser}-B3) highlights engagement and role changes.
\textbf{Multi-Level Interaction}, the system's layered visualization allows instructors to explore and analyze both group and individual behavior, enabling precise assessment of the collaboration process. It also supports side-by-side group comparisons.
\textbf{Detailed Review}, the detail view (Fig.~\ref{fig:teaser}-C) provides original discussion videos and transcripts, enabling in-depth analysis of student conversations and problem-solving processes.
\begin{figure}
	\centering
	\includegraphics[width=1\linewidth]{design.pdf}
	\caption{The iterative design of student glyphs: 1 represents cognitive engagement, and 2 represents behavioral engagement.}
        \Description{The iterative design of student glyphs: 1 represents cognitive engagement, and 2 represents behavioral engagement.}
	\label{fig:designl}
\end{figure}
\subsection{Visual Design}
\begin{figure*}
	\centering
	\includegraphics[width=1\linewidth]{encoding.pdf}
	\caption{The flower metaphor in \textit{CPVis}, along with its visual encoding, color coding, and some samples.}
        \Description{The flower metaphor in \textit{CPVis}, along with its visual encoding, color coding, and some samples.}
	\label{fig:encoding1}
\end{figure*}

We iterated continuously during the glyph design process to optimize the visual representation. Initially, we used a star-shaped design (e.g., radar charts), as shown in Fig.~\ref{fig:designl}. While radar charts effectively displayed behavioral and cognitive engagement, dividing the chart into sections for individual tasks introduced an unnecessary dimension (shape size) that was meaningless and prone to misinterpretation. 
We then shifted to a circular design, encoding behavioral engagement as the radius of a sector and cognitive engagement as the color of the outer ring. However, this design had a significant flaw: color mapping was less intuitive than size mapping, and using color saturation as a visual channel lacked precision. Additionally, both designs struggled to combine individual students into group glyphs intuitively.
\RR{Inspired by previous research~\cite{tausch2014groupgarden,tausch2016comparison,xiong1999peoplegarden}, we introduced a visual design inspired by the flower metaphor, where students are represented as flowers (Fig.~\ref{fig:encoding1}).
The size of the petals represented behavioral engagement, while the size of the stamen indicated cognitive engagement. Three colors were used to represent three different roles, and the varying colors of the leaves symbolized different levels of teacher scaffolding. The number of butterflies reflected the level of collaborative problem-solving ability. As a result, the overall group glyph naturally took the form of a bouquet (R4).}
This approach resolved the issue of merging individual students into group glyphs while enhancing the design's readability and intuitiveness (R1). 
\RR{The final design struck a balance between aesthetics and functionality, effectively conveying the performance of individual students and groups, allowing users to quickly compare similar or different groups (R2).}


\subsection{Filter View}
The Filter View (Fig.~\ref{fig:teaser}-A) serves as the starting point for analysis, featuring an interactive projection panel (Fig.~\ref{fig:teaser}-A1) and a similarity panel (Fig.~\ref{fig:teaser}-A3) to help users filter and explore groups of interest. 
The projection panel displays the distribution of groups in a 2D space to reveal clustering patterns and outliers. We apply the t-SNE algorithm to maximize separation between dissimilar groups, creating clearer clusters. 
To avoid visual clutter caused by group glyphs in dimensionality-reduced views, we follow the approach of Tac-Miner~\cite{wang2021tac}, representing groups as points or rectangles based on whether they received teacher scaffolding, with color coding reflecting prior performance. 
\RR{Additionally, the outer arc represents the duration of the discussion.}
Users can select groups using the lasso tool (Fig.~\ref{fig:teaser}-A1b) or search for specific groups (Fig.~\ref{fig:teaser}-A1a). The similarity panel (Fig.~\ref{fig:teaser}-A2) shows the most similar and dissimilar groups based on Euclidean distance.  
 

 
\subsection{Content View}
Once the search group is selected, users can perform a detailed analysis through the content view (Fig.~\ref{fig:teaser}-B). This view comprises four panels, allowing for a layered exploration of group and individual student performance. 

\subsubsection{Codes Panel}
In the upper left corner of the content view, a control button (Fig.~\ref{fig:teaser}-B1) allows users to toggle between the Codes panel and the Student Projection/Group Pattern panels. Below, users can compare the code quality between the selected group (left) and a comparison group (right). 
For instance, Figure~\ref{fig:teaser}-B1 shows Group 10's answer to Question 5 (left) and Group 18's answer (right). Hint boxes provide two types of feedback: red tips indicating code deficiencies (Fig.~\ref{fig:teaser}-B1a) and pink tips showing that a method was learned from pre-class materials, signaling comprehension, and application. 
Below the code, the ChatGPT-4o score and its rationale are provided, enabling quick, in-depth code evaluation and highlighting areas where groups faced challenges. 


 
\subsubsection{Students Projection (Fig.~\ref{fig:student}-B2a)}

We project students from different groups using t-SNE for clustering, highlighting students with similar performance. 
The three flowers representing the search group are connected with dashed lines to clarify group member distribution and similarity, helping users assess group homogeneity or heterogeneity. 
Only the comparing group's flowers are highlighted to minimize visual clutter. 
\RR{Users can zoom in/out to explore specific students and view detailed background information (e.g., major, grades) by hovering over individual points (Fig.~\ref{fig:student}-B2c).}

\subsubsection{Group Pattern Panel (Fig.~\ref{fig:student}-B2b \& d)}
We use Epistemic Network Analysis (ENA)~\cite{zhao2023analysing} to analyze the dynamic connections between cognitive elements and the collaborative problem-solving behaviors of groups (R3). 
In the Group Pattern Panel, each node represents a behavior in the collaborative problem-solving process, with colors following the coding scheme in Section~\ref{fig:scheme}. Node size indicates behavior frequency, and the intensity of the color reflects the frequency of interactions between behaviors. More significantly, darker nodes represent more frequent behaviors and interactions. 
Users can click on different question buttons to examine dynamic changes in behavior across specific questions or select multiple questions to observe how behaviors evolve during transitions between tasks. Hovering over nodes reveals detailed information about each behavior.
\RR{When comparing two groups, the system displays side-by-side behavior networks.} 
The Group Pattern Panel displays the search group's behavior patterns and compares them to those of the comparison group during the collaborative problem-solving process (Fig.~\ref{fig:student}-B1b).


\subsubsection{Timeline Panel}
\label{barchart}
Users can toggle between the search and comparing groups using the control button in the upper left corner of the Timeline Panel (R5). This panel displays group and individual student performance over time using a filterable bar chart and line chart (Fig.~\ref{fig:teaser}-B3).
In the bar chart, each bar represents a timestamp, with colors indicating different collaborative problem-solving behaviors. The bar height reflects uncertainty, as calculated by ChatGPT-4o. Dashed lines separate different questions for visual clarity. Hovering over a bar reveals the behavior category, predicted certainty, and reasoning behind the label (Fig.~\ref{fig:teaser}-B3e). Clicking on a bar takes users to the relevant conversation content (Fig.~\ref{fig:teaser}-C2a).
The filtering function (Fig.~\ref{fig:teaser}-B3c) allows users to zoom in on specific periods, magnifying the bars for more detailed analysis. 
\RR{The progress bar consolidates behaviors, minimizing visual clutter and highlighting key shifts in temporal behavior in the overview, while also allowing for detailed tracking of group dynamics (Fig.~\ref{fig:group18}-c).}

Three line charts track the behaviors and cognitive engagement of three students in the group (Fig.~\ref{fig:teaser}-B3b). Engagement is calculated at the midpoint and end of each question, and Savitzky–Golay filtering smooths the curves to highlight dynamic changes and trends across questions.
Below the charts, role types are mapped using equally sized rectangles at each timestamp. Users can zoom in on the timeline with the progress bar for detailed analysis. 






\subsection{Detail View}
In the Timeline Panel, users can link to specific conversation content to review group discussions (Fig.~\ref{fig:teaser}-C1) and individual student conversations for each question (Fig.~\ref{fig:teaser}-C2). The playback feature allows users to revisit the original video of collaborative programming sessions, providing a more immersive classroom experience. This feature validates analysis results, offers detailed references, and supports the final step of our analysis workflow (R6).

\begin{figure*}
 	\centering
 	\includegraphics[width=1\linewidth]{Student_panel.pdf}
 	\vspace{-1em}
 	\caption{The panel displayed in the Content View after selecting ``Student Projection \& Group Pattern'' is shown \RR{(A screenshot of the Group 10 \& 18 view)}. On the left (B2a), the projection of all students highlights the searched and compared groups, showing connections between students within the searched group. Hovering over a flower reveals the student's background information (B2c). On the right (B2b), group patterns for two groups across different questions are shown, with clickable question buttons to explore changes in dynamic learning behavior.}
        \Description{The panel displayed in the Content View after selecting ``Student Projection \& Group Pattern'' is shown \RR{(A screenshot of the Group 10 \& 18 view)}. On the left (B2a), the projection of all students highlights the searched and compared groups, showing connections between students within the searched group. Hovering over a flower reveals the student's background information (B2c). On the right (B2b), group patterns for two groups across different questions are shown, with clickable question buttons to explore changes in dynamic learning behavior.}
 	\label{fig:student} \end{figure*}


% \begin{table*}[t]
\centering
\caption{A case study examining \model's response to a query about ``\emph{graders}'' in OpenAI's 12-day edu    cational series demonstrates its video retrieval capabilities and knowledge synthesis.}
\label{tab:case study}
\small
\begin{tabularx}{\textwidth}{X X}

\toprule
\textbf{Query:} Explain the purpose and functionality of 'graders' in the context of reinforcement fine-tuning.\\

\midrule
\textbf{Input Videos:} \href{https://www.youtube.com/playlist?list=PLOXw6I10VTv9lin5AzsHAHCTrC7BdVdEM}{12 Days of OpenAI} (A total of 12 videos) \\

\midrule

\textbf{\model\ Response:} 

\texttt{\#\#\# Understanding Gradation in Reinforcement Fine-Tuning}

\texttt{**Purpose of Graders**}

\texttt{Graders are integral components in the process of reinforcement fine-tuning, mainly utilized to evaluate and improve the performance of machine learning models...} (Omitted for brevity)

\texttt{**Functionality of Graders**}

\texttt{\colorbox{lightgreen}{The grading system operates by comparing a model's outputs against those of}}

\texttt{\colorbox{lightgreen}{verified correct responses. The mechanics of this assessment involve assigning}}

\texttt{\colorbox{lightgreen}{scores to the model's outputs based on their correctness. These scores range}}

\texttt{\colorbox{lightgreen}{from zero to one, where:}}


\texttt{\colorbox{lightblue}{- A score of **0** indicates that the model's answer is completely incorrect.}}

\texttt{\colorbox{lightblue}{- A score of **1** signals that the answer is fully correct.}}

\texttt{- Scores can also be fractional, allowing for partial credit. For instance, \colorbox{lightpink}{a score of **0.7** may denote that the model's suggestion was }}
\texttt{\colorbox{lightpink}{correct but not the highest in a ranked list of possible answers.}} 

\\

\texttt{The process includes:}

\texttt{1. **Input Comparison**: The grader takes both the model’s output and the correct answer, analyzing them to derive a score.}

\texttt{2. **Graded Outputs**: This interaction allows for a dynamic reflection of the model's performance, where the scores directly influence how the model is fine-tuned in subsequent training iterations.}

\texttt{3. **Feedback Loop**: The scores obtained from graders provide feedback that can be used to adjust learning parameters and improve the model’s future responses. This highlights the feedback mechanism that is crucial for reinforcement learning.}

\\

(The remaining content are omitted for brevity.)

\\

\midrule
\textbf{Retrieved Sources by \model:}

\model\ accurately retrieved four continuous video segments from the OpenAI's 12-day show, ranging from 10:00 to 12:00 in Day 2 of the video "\href{https://www.youtube.com/watch?v=yCIYS9fx56U}{Reinforcement Fine-Tuning}." Here, we highlight key moments relevant to the detailed content in the answer. From left to right, these are retrieved moments at timestamps \colorbox{lightgreen}{10:35}, \colorbox{lightblue}{10:39}, and \colorbox{lightpink}{11:10}, which provide informative insights that help \model\ give a comprehensive answer to the query.

\\

\begin{tabular}{ccc}
    {\includegraphics[width=0.3\textwidth]{figs/openai-1.png}} &
    {\includegraphics[width=0.3\textwidth]{figs/openai-2.png}} &
    {\includegraphics[width=0.3\textwidth]{figs/openai-3.png}} \\
\end{tabular}

\\

\bottomrule

\end{tabularx}
\vspace{-0.2in}
\end{table*}
\section{Evaluation}
We provide three sets of insights into this section, organised as \textit{findings (F*)}. We quantitatively study the effect of the adversarial and counterfactual perturbations on the performance of informal reasoners and autoformalisation methods. Then, we dive deeper into method variants. Finally, 
we analyse the nature of formalisation errors made by the models.

\subsection{Robustness Analysis}
\paragraph{\textbf{\emph{F1: Noise perturbations have a stronger effect on formalisation methods than informal \ac{LLM} reasoners.}}}
Table~\ref{tab:distraction_k4_formalisation} shows that, on average, the accuracy of both direct and \ac{CoT} informal reasoning remains between $73\%$ and $74\%$ in the face of added noise. While the autoformalisation method performs similarly to informal reasoners on the original dataset, its performance decreases between $4\%$ and $11\%$. The accuracy drops especially with logical (L) and tautological (T) distractions, whose logical language formats trick the \ac{LLM} into formalizing the noisy clauses. On the other hand, the linguistically complex and more natural sentences of encyclopedic distractions show a minor effect, suggesting that \acp{LLM} successfully avoids formalizing the more complicated sentences.

\paragraph{\textbf{\emph{F2: All \ac{LLM}-based reasoning methods suffer a drop for counterfactual perturbations.}}} % influence .}}}
Table~\ref{tab:distraction_k4_formalisation} shows that counterfactual statements cause a significant decrease in performance for both the informal reasoners and autoformalisation methods of between $12\%$ and $13\%$ on average. 
Moreover, this observation also holds for all tested models, i.e., none are robust towards counterfactual perturbations across every evaluated dimension. Even the strongest model, GPT 4o-mini, yields a performance of 63-68\%, which is relatively close to the random performance of 50\%. The high impact of counterfactual statements (the single ``not'' inserted) could be due to the inability of \acp{LLM} to overwrite prior knowledge with explicitly stated information or memorization of the answers. We study the error sources further in §\ref{subsec:errors}.  

\noindent \paragraph{\textbf{\emph{F3: Introducing multiple noise sentences has an effect only for logical distractions.}}}
We show the impact of introducing between one and four sentences for the two top-performing autoformalisation models in Figure~\ref{fig:length_distraction}. The figure shows similar trends with and without counterfactual perturbations.
As additional logical distractions are introduced, the model performance consistently decreases. Tautological (T) distractions lead to a decline in accuracy with a single disruptive sentence, yet adding more noise does not worsen the outcome. 
The tautological corpus introduces truth constants for all sentences as a persistent unseen logical construct. Given that this leads only to a decrease for a single occurrence, we can assume that a model can consistently handle the same unseen logical construct. In contrast, the logical corpus increases the chance of adding text, requiring new, previously unseen reasoning constructs for each added sentence. The impact of encyclopedic noise remains negligible, generalising F1 to $k$ sentences. Similarly, counterfactual perturbations remain much more effective for all settings, generalising F2.

\begin{table}[!t]
\small
\setlength{\modelspacing}{2pt}
\setlength{\tabcolsep}{1.7pt} % Default value: 6pt
\setlength{\belowrulesep}{4pt}
\begin{threeparttable}
    \centering
    \begin{tabular}{cc l r rrr @{\quad} rrrr}
\toprule
\multirow{2}{*}{} & \multirow{2}{*}{} & Reasoning & \multirow{2}{*}{O} & \multicolumn{3}{c}{Distraction} & \multicolumn{4}{c}{Counterfactual} \\
 & & Format & & E& L & T & $\text{O}_C$ & $\text{E}_C$& $\text{L}_C$ & $\text{T}_C$\\
\midrule
\multirow{6}{*}{\rotatebox{90}{Gemma-2}} & \multirow{3}{*}{\rotatebox{90}{9b}}
   & Informal (direct) & \textbf{0.78} & \textbf{0.80} & \textbf{0.79} & \textbf{0.77} & 0.58 & 0.52 & 0.50 & 0.59 \\
 & & Informal (CoT) & 0.72 & 0.78 & 0.73 & 0.76 & 0.61 & \textbf{0.57} & \textbf{0.60} & \textbf{0.66} \\
 & & Formal (FOL) & 0.62 & 0.58 & 0.52 & 0.53 & \textbf{0.63} & 0.52 & 0.46 & 0.46 \\[\modelspacing]
\cmidrule{2-11}
 & \multirow{3}{*}{\rotatebox{90}{27b}} 
   & Informal (direct) & 0.71 & 0.69 & \textbf{0.66} & \textbf{0.68} & 0.59 & 0.51 & 0.54 & 0.59 \\
 & & Informal (CoT) & 0.66 & 0.65 & 0.64 & 0.63 & 0.62 & 0.58 & \textbf{0.62} & \textbf{0.64} \\
 & & Formal (FOL) & \textbf{0.74} & \textbf{0.74} & 0.61 & 0.61 & \underline{\textbf{0.72}} & \underline{\textbf{0.67}} & 0.58 & 0.51 \\[\modelspacing]
\midrule
\multirow{6}{*}{\rotatebox{90}{Mistral}} & \multirow{3}{*}{\rotatebox{90}{7B}} 
   & Informal (direct) & 0.77 & \textbf{0.77} & 0.75 & \textbf{0.79} & \textbf{0.63} & \textbf{0.54} & \textbf{0.54} & \textbf{0.66} \\
 & & Informal (CoT) & \textbf{0.79} & 0.75 & \textbf{0.77} & 0.78 & 0.55 & 0.52 & \textbf{0.54} & 0.58 \\
 & & Formal (FOL) & 0.62 & 0.58 & 0.54 & 0.57 & 0.50 & \textbf{0.54} & 0.51 & 0.52 \\[\modelspacing]
\cmidrule{2-11}
 & \multirow{3}{*}{\rotatebox{90}{Small}} 
   & Informal (direct) & \textbf{0.77} & \textbf{0.76} & \textbf{0.76} & \textbf{0.75} & 0.61 & 0.51 & 0.56 & 0.59 \\
 & & Informal (CoT) & 0.72 & 0.72 & 0.72 & 0.71 & \textbf{0.62} & \textbf{0.59} & \textbf{0.62} & \textbf{0.68} \\
 & & Formal (FOL) & 0.68 & 0.59 & 0.53 & 0.64 & 0.54 & 0.55 & 0.49 & 0.51 \\[\modelspacing]
\midrule
\multirow{6}{*}{\rotatebox{90}{Llama-3.1}} & \multirow{3}{*}{\rotatebox{90}{8B}} 
   & Informal (direct) & 0.63 & 0.61 & 0.64 & 0.66 & 0.61 & \textbf{0.62} & 0.59 & 0.61 \\
 & & Informal (CoT) & 0.73 & \textbf{0.73} & \textbf{0.71} & \textbf{0.72} & \textbf{0.62} & 0.59 & \textbf{0.61} & \textbf{0.65} \\
 & & Formal (FOL) & \textbf{0.77} & 0.71 & 0.63 & 0.52 & 0.60 & 0.58 & 0.55 & 0.52 \\[\modelspacing]
\cmidrule{2-11}
 & \multirow{3}{*}{\rotatebox{90}{70B}} 
   & Informal (direct) & 0.77 & 0.74 & 0.74 & 0.73 & 0.62 & 0.53 & 0.56 & 0.64 \\
 & & Informal (CoT) & \textbf{0.78} & \textbf{0.75} & \textbf{0.76} & \textbf{0.76} & 0.64 & 0.61 & \textbf{0.66} & \underline{\textbf{0.73}} \\
 & & Formal (FOL) & 0.74 & 0.73 & 0.71 & 0.71 & \textbf{0.66} & \textbf{0.62} & 0.59 & 0.57 \\[\modelspacing]
 \midrule
\multirow{3}{*}{\rotatebox{90}{GPT}} & \multirow{3}{*}{\rotatebox{90}{4o-mini}} 
   & Informal (direct) & 0.78 & 0.77 & 0.79 & 0.79 & 0.64 & 0.61 & 0.61 & 0.63 \\
 & & Informal (CoT) & 0.80 & 0.80 & \underline{\textbf{0.81}} & \underline{\textbf{0.82}} & \textbf{0.68} & \textbf{0.63} & \underline{\textbf{0.68}} & \textbf{0.64} \\
 & & Formal (FOL) & \underline{\textbf{0.84}} & \underline{\textbf{0.82}} & 0.73 & 0.79 & 0.63 & 0.62 & 0.57 & 0.54 \\[\modelspacing]
 \midrule
\multicolumn{2}{c}{\multirow{3}{*}{\textbf{Avg}}} 
 & Informal (direct) & 0.74 & 0.73 & 0.73 & 0.73 & 0.61 & 0.55 & 0.56 & 0.62 \\
 & & Informal (CoT) & 0.74 & 0.74 & 0.73 & 0.74 & 0.62 & 0.58 & 0.62 & 0.65 \\
  & & Formal (FOL) & 0.72 & 0.68 &	0.61 & 0.62 & 0.61 & 0.59 & 0.54 & 0.52 \\
\bottomrule
\end{tabular}
\caption{Accuracies of informal and autoformalisation-based deductive reasoners. The best overall model per dataset is underlined; the best model version is marked in bold.}
\label{tab:distraction_k4_formalisation}
\end{threeparttable}
\end{table} 

\begin{figure}[!t]
    \centering
    \scriptsize
    \begin{tikzpicture}
        \begin{axis}[name=gpt,
            title={GPT-4o-mini},
            width=0.6\linewidth,
            height=0.6\linewidth,
            xlabel={\# Noise sentences},
            ylabel={Accuracy},
            xmin=-0.1, xmax=4.1,
            ymin=0.5, ymax=0.9,
            xtick={1,2,4},
            ytick={0.55, 0.6, 0.65, 0.75, 0.8, 0.85},
            title style={yshift=-0.6em},
            legend style={at={(1,-0.15)},
	           anchor=north,legend columns=-1},
            x label style={at={(axis description cs:1,-0.05)},anchor=north},
            y label style={at={(axis description cs:-0.15,0.5)},anchor=south},
            ymajorgrids=true,
            grid style=dashed,
        ]
            \addplot[color=blue, mark=square,]
                coordinates {
                (0,0.848076939582825)(1,0.823076903820038)(2,0.826923072338104)(4,0.821153819561005)
                };
            \addplot[color=red, mark=triangle,]
                coordinates {
                (0,0.848076939582825)(1,0.817307710647583)(2,0.801923096179962)(4,0.759615361690521)
                };
            \addplot[color=green, mark=diamond,] 
                coordinates {
                (0,0.848076939582825)(1,0.767307698726654)(2,0.769230782985687)(4,0.803846180438995)
                };
            \addplot[color=blue, mark=square*] 
                coordinates {
                (0,0.627777755260468)(1,0.622222244739533)(2,0.600000023841858)(4,0.633333325386047)
                };
            \addplot[color=red, mark=triangle*,] 
                coordinates {
                (0,0.627777755260468)(1,0.611111104488373)(2,0.611111104488373)(4,0.594444453716278)
                };
            \addplot[color=green, mark=diamond*,] 
                coordinates {
                (0,0.627777755260468)(1,0.572222232818604)(2,0.538888871669769)(4,0.555555582046509)
                };
                \legend{E,L,T,$\text{E}_C$, $\text{L}_C$ , $\text{T}_C$}
        \end{axis}

        \begin{axis}[name=llama, at={($(gpt.east)+(0.1cm,0)$)},anchor=west,
            title={Llama 3.1 70b},
            width=0.6\linewidth,
            height=0.6\linewidth,
            xmin=-0.1,, xmax=4.1,
            ymin=0.5, ymax=0.9,
            xtick={1,2,4},
            ytick={0.55, 0.6, 0.65, 0.75, 0.8, 0.85},
            title style={yshift=-0.6em},
            yticklabel=\empty,
            ymajorgrids=true,
            grid style=dashed,
        ]
            \addplot[color=blue, mark=square,]
                coordinates {
                (0,0.838461518287659)(1,0.817307710647583)(2,0.805769205093384)(4,0.817307710647583)
                };
            \addplot[color=red, mark=triangle,]
                coordinates {
                (0,0.838461518287659)(1,0.819230794906616)(2,0.803846180438995)(4,0.771153867244721)
                };
            \addplot[color=green, mark=diamond,]
                coordinates {
                (0,0.838461518287659)(1,0.803846180438995)(2,0.807692289352417)(4,0.805769205093384)
                };
            \addplot[color=blue, mark=square*]
                coordinates {
                (0,0.627777755260468)(1,0.622222244739533)(2,0.577777802944183)(4,0.594444453716278)
                };
            \addplot[color=red, mark=triangle*,]
                coordinates {
                (0,0.627777755260468)(1,0.583333313465118)(2,0.561111092567444)(4,0.577777802944183)
                };
            \addplot[color=green, mark=diamond*,]
                coordinates {
                (0,0.627777755260468)(1,0.627777755260468)(2,0.566666662693024)(4,0.577777802944183)
                };
        \end{axis}
    \end{tikzpicture}
    \caption{Influence of the number of noisy sentences for FOL.}
    \label{fig:length_distraction}
\end{figure}



\subsection{Impact of Method Design}
\paragraph{\textbf{\emph{F4: \ac{CoT} prompting is most impactful when both noise and counterfactual perturbations are applied.}}}
The accuracies for the individual \acp{LLM} in Table~\ref{tab:distraction_k4_formalisation} show that the impact of \ac{CoT} is negligible for noise-only datasets (first four columns). Meanwhile, the benefit from \ac{CoT} is most pronounced in the datasets that combine noise and counterfactual perturbations.
The better-performing informal prompting strategy for a model remains stable for all types of distractions. Still, the decline in performance due to counterfactuals leads to a less consistent preference for a specific prompting style.

\paragraph{\textbf{\emph{F5: The best-performing grammar differs per model and is unstable across data versions.}}}

The evaluation of different logical forms for formal \ac{LLM}-based reasoning in Table~\ref{tab:distraction_k4_logical_form} shows the preference of some models for specific syntactic formats.
Llama 3.1 70B has a considerable improvement of $12\%$ with TPTP syntax on the original set, while Llama 3.1 8B benefits from the R-FOL syntax. However, all grammars show a declining accuracy trend and increased syntax errors for noise perturbations, where the best grammar loses its advantage over the rest. 
When comparing the grammars on the counterfactual partitions, we observe that TPTP is consistently more robust than the standard first-order logic grammar. Here, GPT 4o-mini shows a reduction from $O$ to $O_C$ of $20\%$ for FOL and only $12\%$ for the TPTP grammar. Since this does not correlate with fewer syntax errors, the formalisation in TPTP prevents semantical errors for counterfactual premises. 
A positive reading of these results, especially the minor differences between FOL and R-FOL, is that autoformalisation \acp{LLM} can adapt to the grammar syntax prescribed in the prompt without further loss in performance.

\begin{table}[!t]
\small
\setlength{\modelspacing}{2pt}
\setlength{\tabcolsep}{1.7pt} % Default value: 6pt
\setlength{\belowrulesep}{4pt}
\begin{threeparttable}
    \centering
    \begin{tabular}{cc l r rrr @{\quad} rrrr}
\toprule
\multirow{2}{*}{} & \multirow{2}{*}{} & Grammar & \multirow{2}{*}{O} & \multicolumn{3}{c}{Distraction} & \multicolumn{4}{c}{Counterfactual} \\
 & & Syntax & & E& L & T & $\text{O}_C$ & $\text{E}_C$& $\text{L}_C$ & $\text{T}_C$\\
\midrule
\multirow{6}{*}{\rotatebox{90}{Llama-3.1}} & \multirow{3}{*}{\rotatebox{90}{8B}} 
   & FOL & 0.77 & \textbf{0.71} & 0.61 & \textbf{0.53} & 0.58 & \textbf{0.55} & 0.52 & \textbf{0.56} \\
 & & R-FOL & \textbf{0.78} & 0.69 & \textbf{0.62} & \textbf{0.53} & 0.58 & \textbf{0.55} & \textbf{0.54} & 0.52 \\
 & & TPTP & 0.73 & 0.67 & 0.55 & 0.51 & \textbf{0.68} & 0.54 & 0.46 & 0.51 \\[\modelspacing]
\cmidrule{2-11}
 & \multirow{3}{*}{\rotatebox{90}{70B}} 
   & FOL & 0.76 & 0.73 & 0.71 & \textbf{0.72} & 0.67 & 0.57 & 0.63 & 0.56 \\
 & & R-FOL & 0.76 & 0.73 & 0.67 & 0.71 & 0.64 & 0.57 & 0.53 & 0.64 \\
 & & TPTP & \underline{\textbf{0.88}} & \underline{\textbf{0.84}} & \underline{\textbf{0.81}} & \textbf{0.72} & \underline{\textbf{0.81}} & \underline{\textbf{0.68}} & \underline{\textbf{0.67}} & \underline{\textbf{0.68}} \\[\modelspacing]
\midrule
\multirow{3}{*}{\rotatebox{90}{GPT}} & \multirow{3}{*}{\rotatebox{90}{4o-mini}} 
   & FOL & \textbf{0.84} & \textbf{0.82} & \textbf{0.72} & \underline{\textbf{0.78}} & 0.64 & \textbf{0.63} & \textbf{0.61} & 0.51 \\
 & & R-FOL & \textbf{0.84} & 0.77 & 0.70 & \underline{\textbf{0.78}} & \textbf{0.72} & 0.56 & 0.54 & \textbf{0.63} \\
 & & TPTP & 0.83 & \textbf{0.82} & 0.71 & 0.71 & 0.69 & \textbf{0.63} & 0.57 & 0.57 \\
\bottomrule
\end{tabular}
\caption{Accuracies of different formalisation grammars for autoformalisation.}
\label{tab:distraction_k4_logical_form}
\end{threeparttable}
\end{table} 

\paragraph{\textbf{\emph{F6: Feedback does not help \acp{LLM} self-correct to mitigate robustness issues.}}}
\autoref{tab:distraction_k4_feedback} shows the results with different error recovery mechanisms. The results indicate that no feedback strategy emerges as a winner in the different datasets. 
All feedback variants reduce syntax errors for noise perturbations, but given the lack of a consistent increase in accuracy, the corrected formalisations are most likely to contain semantic errors still. 
The type of feedback message only has a minor influence on correcting syntax errors, whereas Llama 3.1 70b and GPT 4o-mini correct slightly more syntax errors with specific error messages. This finding aligns with \cite{huang2023large}, who also found that \acp{LLM} cannot consistently self-correct their reasoning after receiving relevant feedback.

\begin{table}[!ht]
\small
\setlength{\modelspacing}{2pt}
\setlength{\tabcolsep}{1.7pt} % Default value: 6pt
\setlength{\belowrulesep}{4pt}
\begin{threeparttable}
    \centering
    \begin{tabular}{cc l r rrr @{\quad} rrrr}
\toprule
\multirow{2}{*}{} & \multirow{2}{*}{} & \multirow{2}{*}{Feedback} & \multirow{2}{*}{O} & \multicolumn{3}{c}{Distraction} & \multicolumn{4}{c}{Counterfactual} \\
 & & & & E& L & T & $\text{O}_C$ & $\text{E}_C$& $\text{L}_C$ & $\text{T}_C$\\
\midrule
\multirow{8}{*}{\rotatebox{90}{Llama-3.1}} & \multirow{4}{*}{\rotatebox{90}{8B}} 
   & No recovery & 0.77 & \textbf{0.72} & 0.62 & 0.53 & 0.59 & 0.58 & 0.56 & \textbf{0.56} \\
 & & Error type & \textbf{0.79} & 0.71 & 0.63 & \textbf{0.56} & \textbf{0.66} & 0.54 & 0.52 & 0.51 \\
 & & Error message & 0.78 & 0.71 & \textbf{0.67} & 0.55 & 0.59 & 0.53 & \underline{\textbf{0.64}} & 0.49 \\
 & & Warning & 0.74 & 0.66 & 0.58 & 0.55 & 0.55 & \textbf{0.60} & 0.49 & 0.49 \\[\modelspacing]
\cmidrule{2-11}
 & \multirow{4}{*}{\rotatebox{90}{70B}} 
   & No recovery & \textbf{0.77} & \textbf{0.72} & \textbf{0.73} & 0.71 & \textbf{0.64} & 0.59 & \textbf{0.61} & 0.56 \\
 & & Error type & 0.72 & 0.70 & 0.72 & \textbf{0.73} & 0.62 & 0.56 & 0.60 & 0.58 \\
 & & Error message & 0.71 & 0.70 & \textbf{0.73} & 0.71 & \textbf{0.64} & 0.59 & 0.54 & \underline{\textbf{0.64}} \\
 & & Warning & 0.69 & \textbf{0.72} & 0.72 & 0.72 & 0.62 & \underline{\textbf{0.65}} & \textbf{0.61} & 0.63 \\[\modelspacing]
\midrule
\multirow{4}{*}{\rotatebox{90}{GPT}} & \multirow{4}{*}{\rotatebox{90}{4o-mini}} 
   & No recovery & \underline{\textbf{0.84}} & \underline{\textbf{0.82}} & 0.73 & 0.79 & 0.64 & \textbf{0.62} & 0.56 & \textbf{0.56} \\
 & & Error type & 0.83 & 0.79 & 0.74 & 0.76 & 0.67 & 0.57 & 0.56 & \textbf{0.56} \\
 & & Error message & \underline{\textbf{0.84}} & 0.78 & \underline{\textbf{0.77}} & \underline{\textbf{0.80}} & 0.62 & 0.59 & 0.56 & \textbf{0.56} \\
 & & Warning & \underline{\textbf{0.84}} & 0.75 & 0.73 & 0.76 & \underline{\textbf{0.70}} & 0.61 & \textbf{0.61} & 0.55 \\
 \bottomrule
\end{tabular}
\caption{Accuracies of error recovery strategies.}
\label{tab:distraction_k4_feedback}
\end{threeparttable}
\end{table} 

\subsection{Error Analysis}
\label{subsec:errors}
\paragraph{\textbf{\emph{F7: Autoformalisation increases syntax errors for noise perturbations.}}}
The low performance for noise perturbations correlates with more syntax errors for all models and distraction categories (cf. execution rates in Table~\ref{tab:appendix_k4_formalisation_exec}). The three worst-performing models (both Mistral models, Gemma-2 9b) generate, at best, for $37\%$  and, at worst, for only $4\%$ of the samples, a valid logical form.
Gemma-2 9b and Llama3.1 8b produce more syntax errors than the larger counterparts, suggesting that larger models are more robust towards noise perturbations. 
The accuracy of syntactically valid samples is higher than the informal reasoning methods for most distractions (Table~\ref{tab:appendix_k4_formalisation_vacc}), motivating informal reasoning as a backup strategy for formal reasoning. The error message feedback reveals two common syntax errors: 1) errors by models with an initial low execution rate exhibit issues with the template structure, including using incorrect keywords or adding conversational phrases;
2) perturbation-related errors, the most common of which is using undefined truth constants as part of tautological distractions. 

\paragraph{\textbf{\emph{F8: Autoformalisation increases semantic errors for counterfactuals.}}}
Unlike the introduced noise, counterfactual perturbations do not lead to more syntax errors. The execution rate in Table~\ref{tab:appendix_k4_formalisation_exec} is stable or improves for counterfactuals. However, we see a drop in accuracy for the counterfactual column $\text{O}_C$ in Table~\ref{tab:distraction_k4_formalisation} and can conclude that the number of logical forms with semantic errors has to increase. This suggests that the introduced negation is not correctly formalised. Looking at the warnings generated by the feedback mechanism, for GPT 4o-mini, $161$ warning messages are generated on the unperturbed data. $54$ of these were fixed with a single iteration. Not considering predicates and individuals as part of the context is the most frequent warning across all models. 

\section{Discussion}
\RR{
Our study utilizes an intuitive flower-based visual design and evidence-based collaborative programming process analysis to provide instructors with a clear perspective for evaluating group and individual performance in collaborative programming. In this section, we discuss the lessons learned, the factors contributing to the research outcomes, and how these findings relate to existing works.

\subsection{Flower-Based Visual Design for Intuitive and Useful by Participants}
In large-scale learning analytics, intuitive visualization and interactive features prove to be valuable in assisting instructors with evaluations while reducing their workload~\cite{martinez2020data,fernandez2024data}.
Our study shows that the flower-based visual design effectively helps instructors summarize the performance of students and groups in collaborative programming.
Participants using \textit{CPVis} typically report starting by observing the flower visualization to gain an overview of the group's overall performance and the engagement levels of individual members during collaboration. Our design enables them to make quick assessment judgments and uncover valuable educational insights. 
For instance, students playing the Driver role often exhibit higher engagement levels.
\RA{Previous works use dynamic natural metaphors~\cite{tausch2014groupgarden,tausch2016comparison}, such as blooming flowers, falling leaves, and weather changes, to represent the quality and state of group discussions. However, these metaphors primarily convey overall trends or atmospheres rather than offering a precise and structured representation of multidimensional data, making it difficult for users to extract specific and accurate information efficiently. Moreover, the strong symbolic and emotional nature of their metaphors often leads to subjective interpretations.}
The effectiveness of our design lies in its ability to translate multiple dimensions of process-based learning analytics into visual elements such as colored petals and flower stamens, enabling instructors to quickly interpret multidimensional data and assess both group and individual performance during collaboration.
Furthermore, the flower-based visualization supports hierarchical analysis at both the group and individual levels, allowing instructors to efficiently analyze and compare the performance of multiple groups and students on a large scale.

\subsection{\textit{CPVis} Enhanced Instructors' Confidence in Evaluating Groups and Students}
The study demonstrates that \textit{CPVis} enhances participants' confidence in evaluation outcomes and improves the accuracy of their assessments. In Baseline System 1, participants report that accessing data requires significant time, and evaluating a specific group's performance often necessitates finding similar groups for a relatively fair comparison. 
Such a process demands additional time, causing participants to lose patience and avoid thoroughly examining all the details.
In baseline system 2, participants have to manually browse and process large amounts of student behavior and interaction data, which significantly increases cognitive load and reduces efficiency as they rely on memory to evaluate the performance of different groups.
In comparison, \textit{CPVis} offers significant convenience to participants \RA{by visualizing multidimensional learning analytics data}, allowing them to effortlessly access key information required for evaluations and compare similar groups. By providing both an overall view of multiple groups and detailed comparisons into individual groups, \textit{CPVis} substantially boosts participants' confidence in their evaluation outcomes, as demonstrated in the ratings. 
%This finding aligns with previous research results~\cite{sato2023groupnamics}.
\RA{Clear and intuitive visual analytics systems contribute to improved confidence and efficiency among participants. For instance, Groupnamics helps participants identify groups requiring intervention by visualizing each group's recent vocal activities and discussion statuses in a one-page view, thereby boosting their confidence in decision-making~\cite{sato2023groupnamics}.}
While it is ideal for \textit{CPVis} to support comparisons across an unlimited number of groups, practical limitations related to cognitive load and visual design make this challenging. Future efforts focus on optimizing the evaluation process through visual design, striking a balance between cognitive load and evaluation efficiency, thereby providing effective support for teaching.



\subsection{Theory-driven and LLM-powered Automation Evaluation for Quantifying Collaborative Learning}
Our study utilizes data collection, analysis, and visualization techniques to extract key insights from students' collaborative behaviors and outcomes, providing a deeper understanding of the learning process in collaborative programming. We focus on quantifying complex collaborative learning processes by leveraging LLMs and theoretical frameworks, introducing innovative methods to evaluate collaboration efficiency. 
While collaborative problem-solving is clearly defined in prior research~\cite{rosen2020towards}, achieving a quantitative balance between task performance and team effectiveness remains a significant challenge. To address this, we employ the coefficient of variation as a balancing metric and validate its efficacy using real-world datasets.
By integrating LLMs, \textit{CPVis} automates the annotation of collaborative programming performance, significantly reducing the workload associated with manually labeling large-scale classroom data and offering a novel perspective for automated learning analytics. 
Combining theory-driven metrics and LLM-powered automation provides instructors with robust, multidimensional evidence, enabling them to process and compare extensive student data systematically. 
This empowers instructors to effectively evaluate group and individual behaviors in collaborative programming, identify collaboration patterns, and support evidence-based decision-making. Previous research demonstrates that data-driven analysis helps educational decision-makers~\cite{hou2024codetailor}, such as instructors, uncover hidden learning patterns and deliver personalized guidance. Building on this foundation, \textit{CPVis} further enhances the potential for personalized feedback, enabling instructors to provide precise, data-driven guidance to students.


}


\section{Limitations and Future Work}

\RR{
In this section, we discuss the limitations of the current study and potential future work.
\subsection{Limitation}
Our study has three main limitations.
First, our current analysis is limited to data from a single real-world classroom's collaborative programming discussions, restricting the generalizability of our findings to other contexts. Similarly, our evaluation of \textit{CPVis} relies on a sampled dataset, limiting the study's scope. We hypothesize that participants working with smaller datasets and visualized learning analytics experience reduced cognitive load and find it easier to identify collaboration patterns due to fewer visual elements to process. However, in large-scale collaborative programming classrooms, instructors face the challenge of evaluating more groups and students, which may increase memory load and visual complexity.
Second, the data collected in our study are obtained from real classroom environments, maintaining ecological validity by capturing natural behaviors such as group silence or requests for instructor assistance. However, due to the limitations of non-intrusive equipment, our data lack details such as facial expressions and non-verbal cues. While participants report the comprehensiveness and richness of the learning analytics in the experiment, the absence of these data poses challenges for deeper analysis of emotional expressions and social engagement during collaborative programming. This limitation hinders the provision of a more holistic learning analysis for evaluation purposes.
Additionally, the recorded data are independent and exclude audio information, making it difficult to align screen interactions with dialogue streams. This limitation constrains the exploration of the relationship between collaborative behavior patterns and collaborative problem-solving processes.
Finally, in large-scale collaborative programming classrooms, generating analytics using LLMs requires significant computational time and cost. While feasible for institutions with robust computational resources, this remains a limitation for deploying such tools in real teaching scenarios. Furthermore, in real classrooms, noise from multiple group discussions introduces significant data noise, complicating the automation of learning analytics generation and limiting the accuracy of evaluations for groups and individual students.


\subsection{Future Work}
Without well-structured visualizations, simply presenting multiple data streams poses significant challenges for instructors attempting to interpret these large-scale datasets~\cite{fernandez2024data}.
In this study, we explore the integration and analysis of multimodal data. However, \textit{CPVis} has the potential to further enhance the visualization and perception of multimodal data, enabling instructors to evaluate group and student performance with greater accuracy and reduced cognitive load~\cite{martinez2020data}. 
Our target audience consists of instructors teaching large introductory collaborative programming courses, who require more efficient and intuitive visualizations to understand student performance during collaboration.  
While our use of static 2D visualizations, such as high-dimensional flower glyphs, has been highly regarded by participants for boosting confidence and helping instructors quickly identify key features, we believe there is room for improvement in organizing visualization formats to enhance information transmission efficiency and the users cognitive experience.
For instance, incorporating narrative visualizations further streamlines the process by allowing instructors to generate composite evaluations based on their weighting of different collaboration performance dimensions~\cite{gratzl2013lineup}. 
Narrative visualizations enable instructors to delve into data details, organize learning analytics results along logical paths such as timelines, causality, or categories, and highlight key information~\cite{chen2019designing}. 
This approach mitigates visual overload caused by excessive data, significantly reduces the time and cognitive effort required for evaluation, and ultimately supports instructors in making better decisions and assessments.

\textit{CPVis} requires instructors to spend additional time after class to evaluate collaborative performance. In our study, most participants indicate during follow-up interviews that the extra time spent on evaluating students' collaborative performance is highly valuable for producing comprehensive assessments. They note that providing immediate evaluations during the collaboration process is unrealistic, as final assessments typically need a holistic consideration of task completion and group dynamics after class. However, there is a significant demand for real-time analysis tools to deliver timely, personalized feedback to students and offer appropriate instructional scaffolding during the collaborative process~\cite{tang2024sphere}.
Instructors frequently find themselves overwhelmed by the immediate needs of some students~\cite{yang2023pair}, unintentionally neglecting others. To address this issue, future work could explore the integration of LLMs to enable real-time monitoring and analysis of students' behavioral data—such as code submissions, error logs, and engagement levels. LLMs could automatically detect learning bottlenecks or collaboration issues, providing instant feedback on common problems to students. This would effectively reduce instructors' workload, allowing them to focus on complex or critical issues, and simplify classroom management tasks.
For instance, LLMs could summarize patterns in students' code submissions and generate a ``hotspot report'' identifying recurring issues across the class. They could also provide real-time collaborative performance analytics for different groups, enabling instructors to quickly gain a comprehensive understanding of overall class dynamics. Additionally, LLMs could assist in role allocation within groups, suggest strategies to improve team interactions, and identify potential conflicts or disengagement within collaborative teams.
LLM-powered tools automate evaluations and enable personalized feedback, bridging post-class assessments with in-class scaffolding to enhance teaching and learning in collaborative programming.}
\section*{Conclusion}
This paper aims to enhance our understanding of the computational complexity of computing various Shapley value variants. We found that for various ML models --- including decision trees, regression tree ensembles, weighted automata, and linear regression --- both local and global interventional and baseline SHAP can be computed in polynomial time under HMM modeled distributions. This extends popular algorithms, such as TreeSHAP, beyond their empirical distributional scope. We also establish strict complexity gaps between the various SHAP variants (baseline, interventional, and conditional) and prove the intractability of computing SHAP for tree ensembles and neural networks in simplified scenarios. Overall, we present SHAP as a versatile framework whose complexity depends on four key factors: \begin{inparaenum}[(i)] \item model type, \item SHAP variant, \item distribution modeling approach, \item and local vs. global explanations\end{inparaenum}. We believe this perspective provides deeper insight into the computational complexity of SHAP, paving the way for future work.




%We believe that our framework provides a more intricate understanding of SHAP computation complexity across different models, distributions, and variants, paving the way for further research.

Our work opens promising directions for future research. First, expanding our computational analysis to other SHAP-related metrics, such as asymmetric SHAP~\citep{frye20} and SAGE~\citep{covert2020understanding}, would be valuable. Additionally, we aim to explore more expressive distribution classes and relaxed assumptions beyond those in Section \ref{sec:tractable} while maintaining tractable SHAP computation. Finally, when exact computation is intractable (Section \ref{sec:intractable}), investigating the approximability of SHAP metrics through approximation and parameterized complexity theory~\citep{downey2012parameterized} is an important direction.

%Our work opens several promising avenues for future research on the computational properties of explainable AI methods, with a particular focus on SHAP. First, it would be interesting to broaden the computational analysis conducted in this work to include other popular SHAP-related metrics in the literature, such as asymmetric SHAP \cite{frye20} and SAGE \cite{covert2020understanding}. Also, in the future, we aim to explore more expressive distribution classes and relaxed distributional assumptions—extending beyond those examined in Section \ref{sec:tractable} —that still yield tractable SHAP computation. Finally, when exact computation proves intractable (Section \ref{sec:intractable}), it is worthwhile to theoretically investigate the question of the approximability of computing the SHAP metrics across various configurations, through the lens of approximation and parametrized complexity theory \cite{arora2009computational}.

%This paper aims to deepen our understanding of the computational complexity involved in obtaining different Shapley value variants. We found that for a variety of ML models, including decision trees, tree ensembles for regression, weighted automata, and linear regression models — computing both local and global interventional and baseline SHAP can be done in polynomial time when distributions are modeled by HMMs. This extends the distributional scope of popular algorithms like TreeSHAP, which is limited to empirical distributions. Additionally, we demonstrate a strict complexity gap between SHAP variants, showing that interventional and baseline SHAP can be strictly easier to compute than conditional SHAP. Despite these positive results, we uncovered intractability for various SHAP variants in neural networks and tree ensembles. Finally, we provided generalized complexity relations across SHAP variants. We believe that our framework offers a deeper understanding of the complexity involved in computing SHAP across various variants, models, distributions, as well as in both local and global computations, laying the groundwork for future research.

%%
%% The acknowledgments section is defined using the "acks" environment
%% (and NOT an unnumbered section). This ensures the proper
%% identification of the section in the article metadata, and the
%% consistent spelling of the heading.
\begin{acks}
Meng Xia is the corresponding author.
The work was supported by the National Natural Science Foundation of China, (62422607, 62372411, 62036009) and the Zhejiang Provincial Natural Science Foundation of China.

\end{acks}
%%
%% This is file `sample-sigconf-authordraft.tex',
%% generated with the docstrip utility.
%%
%% The original source files were:
%%
%% samples.dtx  (with options: `all,proceedings,bibtex,authordraft')
%% 
%% IMPORTANT NOTICE:
%% 
%% For the copyright see the source file.
%% 
%% Any modified versions of this file must be renamed
%% with new filenames distinct from sample-sigconf-authordraft.tex.
%% 
%% For distribution of the original source see the terms
%% for copying and modification in the file samples.dtx.
%% 
%% This generated file may be distributed as long as the
%% original source files, as listed above, are part of the
%% same distribution. (The sources need not necessarily be
%% in the same archive or directory.)
%%
%%
%% Commands for TeXCount
%TC:macro \cite [option:text,text]
%TC:macro \citep [option:text,text]
%TC:macro \citet [option:text,text]
%TC:envir table 0 1
%TC:envir table* 0 1
%TC:envir tabular [ignore] word
%TC:envir displaymath 0 word
%TC:envir math 0 word
%TC:envir comment 0 0
%%
%%
%% The first command in your LaTeX source must be the \documentclass
%% command.
%%
%% For submission and review of your manuscript please change the
%% command to \documentclass[manuscript, screen, review]{acmart}.
%%
%% When submitting camera ready or to TAPS, please change the command
%% to \documentclass[sigconf]{acmart} or whichever template is required
%% for your publication.
%%
%%
\documentclass[sigconf, nonacm]{acmart}
%\documentclass[sigconf]{acmart}
%\documentclass[sigconf,authordraft]{acmart}
%\documentclass[anonymous,manuscript,review]{acmart}

%%
%% \BibTeX command to typeset BibTeX logo in the docs
\AtBeginDocument{%
  \providecommand\BibTeX{{%
    Bib\TeX}}}

%% Rights management information.  This information is sent to you
%% when you complete the rights form.  These commands have SAMPLE
%% values in them; it is your responsibility as an author to replace
%% the commands and values with those provided to you when you
%% complete the rights form.
%\setcopyright{acmlicensed}
\copyrightyear{2025}
\acmYear{2025}
%\setcopyright{cc}
\setcctype{by}
\acmConference[CHI '25]{CHI Conference on Human Factors in Computing Systems}{April 26-May 1, 2025}{Yokohama, Japan}
\acmBooktitle{CHI Conference on Human Factors in Computing Systems (CHI '25), April 26-May 1, 2025, Yokohama, Japan}
\acmDOI{10.1145/3706598.3713353}
\acmISBN{979-8-4007-1394-1/25/04}


%%
%% Submission ID.
%% Use this when submitting an article to a sponsored event. You'll
%% receive a unique submission ID from the organizers
%% of the event, and this ID should be used as the parameter to this command.
%%\acmSubmissionID{123-A56-BU3}

%%
%% For managing citations, it is recommended to use bibliography
%% files in BibTeX format.
%%
%% You can then either use BibTeX with the ACM-Reference-Format style,
%% or BibLaTeX with the acmnumeric or acmauthoryear sytles, that include
%% support for advanced citation of software artefact from the
%% biblatex-software package, also separately available on CTAN.
%%
%% Look at the sample-*-biblatex.tex files for templates showcasing
%% the biblatex styles.
%%

%%
%% The majority of ACM publications use numbered citations and
%% references.  The command \citestyle{authoryear} switches to the
%% "author year" style.
%%
%% If you are preparing content for an event
%% sponsored by ACM SIGGRAPH, you must use the "author year" style of
%% citations and references.
%% Uncommenting
%% the next command will enable that style.
%%\citestyle{acmauthoryear}
\usepackage{booktabs}
\usepackage{graphicx}
\usepackage{listings}
\usepackage{xcolor} 
\usepackage{caption}
%\usepackage{paralist}
\usepackage{makecell}
\usepackage{url}
\usepackage{longtable}
\usepackage{color}
\usepackage{tikz}
%\usepackage[utf8]{enc}
\usepackage[T1]{fontenc}
\usepackage{lipsum}
\usepackage{stfloats}
\usepackage{tabu} 
\usepackage{tabularx}
\usepackage{multirow}
\usepackage{booktabs}
\usepackage{graphicx}
\usepackage{wrapfig}
\usepackage{hyperref}   
\usepackage{cleveref}
%\usepackage[most]{tcolorbox} 
\usepackage{xcolor}
\usepackage{float}
\usepackage{listings} 
%\usepackage{tcolorbox}




\definecolor{PurpleColor}{RGB}{0,0,0}
\newcommand{\RR}[1]{{\color{PurpleColor}#1}}

\definecolor{PinkColor}{RGB}{0, 0, 0}
\newcommand{\RA}[1]{{\color{PinkColor}#1}}


\definecolor{codegreen}{rgb}{0,0.6,0}
\definecolor{codegray}{rgb}{0.5,0.5,0.5}
\definecolor{codepurple}{rgb}{0.58,0,0.82}
\definecolor{backcolour}{rgb}{0.95,0.95,0.92}

\lstdefinestyle{mystyle}{
  backgroundcolor=\color{backcolour}, commentstyle=\color{codegreen},
  keywordstyle=\color{magenta},
  numberstyle=\tiny\color{codegray},
  stringstyle=\color{codepurple},
  basicstyle=\ttfamily\footnotesize,
  breakatwhitespace=false,         
  breaklines=true,                 
  captionpos=b,                    
  keepspaces=true,                 
  numbers=left,                    
  numbersep=5pt,                  
  showspaces=false,                
  showstringspaces=false,
  showtabs=false,                  
  tabsize=2
}
\lstset{style=mystyle}


%\definecolor{PurpleColor}{RGB}{0,0,0}

%%
%% end of the preamble, start of the body of the document source.
\begin{document}

%%
%% The "title" command has an optional parameter,
%% allowing the author to define a "short title" to be used in page headers.
\title{CPVis: Evidence-based Multimodal Learning Analytics for Evaluation in Collaborative Programming}

%%
%% The "author" command and its associated commands are used to define
%% the authors and their affiliations.
%% Of note is the shared affiliation of the first two authors, and the
%% "authornote" and "authornotemark" commands
%% used to denote shared contribution to the research.
\author{Gefei Zhang}
\affiliation{%
  \institution{Zhejiang University of Technology}
  \city{Hangzhou}
  \state{Zhejiang}
  \country{China}}
\email{gefei@zjut.edu.cn}
\orcid{1234-5678-9012}

\author{Shenming Ji}
\affiliation{%
  \institution{Xi'an Jiaotong-Liverpool University}
  \city{Suzhou}
  \state{Jiangsu}
  \country{China}}
\email{shenming.ji21@student.xjtlu.edu.cn}

\author{Yicao Li}
\affiliation{%
  \institution{Zhejiang University of Technology}
  \city{Hangzhou}
  \state{Zhejiang}
  \country{China}}
\email{yicaoli47@gmail.com}

\author{Jingwei Tang}
\affiliation{%
  \institution{Zhejiang University of Technology}
  \city{Hangzhou}
  \state{Zhejiang}
  \country{China}}
\email{jwtang@zjut.edu.cn}

\author{Jihong Ding}
\affiliation{%
 \institution{Hannan University}
 \city{Haikou}
 \state{Hainan}
 \country{China}}
\email{jhding@hainanu.edu.cn}

\author{Meng Xia}
\affiliation{%
  \institution{Texas A\&M University}
  \city{College Station}
  \state{Texas}
  \country{USA}}
\email{mengxia@tamu.edu}

\author{Guodao Sun}
\affiliation{%
  \institution{Zhejiang University of Technology}
  \city{Hangzhou}
  \state{Zhejiang}
  \country{China}}
\email{guodao@zjut.edu.cn}

\author{Ronghua Liang}
\affiliation{%
  \institution{Zhejiang University of Technology}
  \city{Hangzhou}
  \state{Zhejiang}
  \country{China}}
\email{rhliang@zjut.edu.cn}



%%
%% By default, the full list of authors will be used in the page
%% headers. Often, this list is too long, and will overlap
%% other information printed in the page headers. This command allows
%% the author to define a more concise list
%% of authors' names for this purpose.
\renewcommand{\shortauthors}{Zhang et al.}

%%
%% The abstract is a short summary of the work to be presented in the
%% article.
\begin{abstract}
As programming education becomes more widespread, many college students from non-computer science backgrounds begin learning programming. Collaborative programming emerges as an effective method for instructors to support novice students in developing coding and teamwork abilities. However, due to limited class time and attention, instructors face challenges in monitoring and evaluating the progress and performance of groups or individuals. To address this issue, we collect multimodal data from real-world settings and develop \textit{CPVis}, \RR{an interactive visual analytics system designed to assess student collaboration dynamically.} Specifically, \textit{CPVis} enables instructors to evaluate both group and individual performance efficiently. \textit{CPVis} employs a novel flower-based visual encoding to represent performance and provides time-based views to capture the evolution of collaborative behaviors. A within-subject experiment (N=22), \RR{comparing \textit{CPVis} with two baseline systems, reveals that users gain more insights, find the visualization more intuitive, and report increased confidence in their assessments of collaboration.}
\end{abstract}

%%
%% The code below is generated by the tool at http://dl.acm.org/ccs.cfm.
%% Please copy and paste the code instead of the example below.
%%

\begin{CCSXML}
<ccs2012>
   <concept>
       <concept_id>10010405.10010489.10010492</concept_id>
       <concept_desc>Applied computing~Collaborative learning</concept_desc>
       <concept_significance>500</concept_significance>
       </concept>
 </ccs2012>
\end{CCSXML}

\begin{CCSXML}
<ccs2012>
   <concept>
       <concept_id>10003120.10003145.10003151</concept_id>
       <concept_desc>Human-centered computing~Visualization systems and tools</concept_desc>
       <concept_significance>500</concept_significance>
       </concept>
 </ccs2012>
\end{CCSXML}

\ccsdesc[500]{Human-centered computing~Visualization systems and tools}

\ccsdesc[500]{Applied computing~Collaborative learning}


\begin{comment}



\end{comment}
%%
%% Keywords. The author(s) should pick words that accurately describe
%% the work being presented. Separate the keywords with commas.
\keywords{Group visualization, education visualization, collaborative programming}
%% A "teaser" image appears between the author and affiliation
%% information and the body of the document, and typically spans the
%% page.








% \received{20 February 2007}
% \received[revised]{12 March 2009}
% \received[accepted]{5 June 2009}

%%
%% This command processes the author and affiliation and title
%% information and builds the first part of the formatted document.
\maketitle
%\newtcbox{\inlinecode}{on line, colback=gray!10, colframe=gray!10, boxrule=0.1mm, 
%	rounded corners, fontupper=\bfseries, left=0.1mm, right=0.1mm, top=0.1mm, bottom=0.1mm}

\documentclass[../main.tex]{subfiles}
\graphicspath{{../images/}}
\makeatletter
\def\input@path{{../images/}}
\makeatother
\begin{document}
\section{Introduction}
\begin{figure}
\centering
\begin{tikzpicture}
\node[inner sep=0pt] (ws) at (0, 0) {
\includegraphics[height=.4\textwidth, trim={10cm 0 10cm 0},clip]{world_space.png}};
\node[inner sep=0pt] (cs) at (6,0) {\includegraphics[height=.4\textwidth, trim={10cm 1cm 10cm 4cm},clip]{conf_space.png}};
\end{tikzpicture}
\vspace{-5pt}
\label{fig:pbrm_intro}
\caption{\textbf{Left}: Shows world space obstacles as grey spheres. Robots start and goal configuration is colored red and green, respectively. Configurations along the computed path are colored transparent blue. \textbf{Right:} Mapped world space scenario to configuration space. Obstacle region is the grey mesh. Red spheres are collision-free regions computed by the neural SCDF. The optimized shortest path in the convex corridor is the blue curve.}
\vspace{-25pt}
\end{figure}
Motion planning is the problem of finding a collision-free trajectory that connects a given start and goal configuration. The planning takes place in the configuration space of the robot. For single body robots, like mobile robots or drones, the configuration space and the world space are usually the same. This simplifies the planning, since explicit obstacle representations are available which enables geometrical tools like separating hyperplanes, smallest distance to obstacles etc., to be used when designing motion planning algorithms. For multi-body robots like manipulators, the situation is completely different. The world space obstacles are usually mapped to non-convex regions, and to make the problem even harder, the mapping is usually not known. Forming explicit representations of the obstacle region in the configuration space is usually too expensive or intractable. Despite all of this, sampling based planners are used with great success, which mainly is due to their use of implicit representations of the obstacle region. The basic idea is to construct a graph in the configuration space that covers and connects the collision-free region. From this graph, a path can be extracted that connects a given start and goal configuration. The approach is computationally expensive, since the graph is constructed with the smallest geometrical building block available, points, which represents a collision-check. Furthermore, the extracted paths from the graph are non-smooth and jagged due to the stochastic nature of the approach. This adds an additional post-processing step to the process, where the paths are shortcutted and smoothened, before the path can be used for tracking. Clearly a lot of time is invested to form this graph and produce smooth paths. Thus, if the obstacles start to move, then all of this work is done in no use, since all points that make up this graph need to be re-verified, which is simply too time consuming to be done in real time.
\\\\
In this work, we want to address the existing drawbacks of the sampling based planners. Our main contribution is an improved motion planner where each vertex in the graph covers a collision-free region in the form of a sphere instead of a point and where the edges are formed with neighboring intersecting spheres. This representation has the advantage of instead of returning piecewise linear paths, returning a sequence of overlapping spheres, i.e. a convex corridor, that connects a given start and goal configuration, illustrated in Figure \ref{fig:pbrm_intro}. This convex corridor allows us to use convex optimization to produce smooth trajectories, instead of computationally expensive post-processing methods. The representation further allows us to estimate the coverage of the collision-free space, which gives us awareness and feedback in the offline roadmap construction phase. Finally, our representation is simple to adapt to moving obstacles, simply requery for the new radii and recheck for intersections. 
\\\\
The spherical collision-free regions are formed using a signed distance function (SDF), which is a function that returns the smallest distance from an arbitrary point to the boundary of an obstacle. As the name implies, the distance is signed, thus if the point is inside the obstacle it is negative otherwise positive. If the distance is positive, a sphere with radius equal to the distance is guaranteed to cover a collision-free region. Using an SDF in motion planning is not new, but what is novel about our approach is that we express the distance in the configuration space instead of the world space and by doing so allows us to form these convex collision-free regions. We refer to the resulting SDF as a signed configuration distance function (SCDF). Computing an SCDF analytically is non-trivial, our approach is therefore to parameterize the SCDF with a deep neural network and learn the mapping by supervised learning. Our resulting neural SCDF can compute distances for different parameter values of obstacle shapes and we also show how multiple distances can be combined, thus making our approach flexible.
\section{Related work}
Motion planning algorithms can roughly be divided into three families, grid-based, sampling based and optimization based methods. Grid-based methods (GBM) discretize the planning space from which a graph is then compiled. A standard search method is A$^\star$ \citep{a_star}, which is classified as an \textit{informed} search method, since it employs a heuristic function to speed up the search. A$^\star$ guarantees to return an optimal path at the level of discretization used. GBMs usually discretize the planning space by a regular lattice and this limits the GBMs to problems with low dimensionality due to the curse of dimensionality. Thus, GBMs are usually limited to single-body robots where the degrees of freedom (DOF) are low. To overcome the inherent scaling problem with the GBMs, stochastic methods are usually used for multi-body robots. These methods are termed as sampling-based methods (SBM) and core members within this family are the rapidly-exploring random trees (RRT) \citep{rrt} and the probabilistic roadmap (PRM) \citep{prm}. RRT grows a tree from the start configuration and explores the collision-free region in a rapid way until it is able to connect to the goal region. RRT is usually improved by bi-directional planning \citep{rrt_connect}, i.e. an additional tree is grown from the goal configuration and the trees are tested for connection after any tree has been expanded. RRT is a single-query method, thus it searches for a path from scratch each time it is queried. Contrary to this, PRM is a multi-query method, which solves for multiple queries without starting from scratch. PRM does this by creating a roadmap (graph) that covers the collision-free space as an offline step. The graph is then used to solve for multiple queries. PRMs are used in cases where the environment does not change since the extra offline step is too computationally costly and needs to be re-done if the environment is changed. In our work, we address this inherent issue by using a different roadmap representation. Our vertices in the graph cover a collision-free region in the form of spheres and we form the edges by checking for intersecting spheres. If something in the environment changes, we recompute the spheres radii and recheck the intersections, without relying on collision detection. We use a trained neural network to compute the sphere radius, therefore querying for the radius can be done fast, hence our representation enables the PRM for dynamic environments.
\\\\
In the recent decades, optimization based methods (OBM) \citep{chomp, schulman, itomp, stomp} have been introduced as an alternative to SBM for multi-body robots. Like the SBM, the OBMs scale well to higher dimensional problems and produce smoother motion. It is common to use a SDF in the optimization since it is a smooth function, thus enabling gradient-based methods. However, the standard way of expressing the SDF is in world space. The distance therefore needs to be mapped to the configuration space by the forward kinematics. This mapping makes the optimization problem a non-linear program (NLP), which is computationally expensive to solve. Recently, a different approach has been proposed. In \cite{mp_gcs} motion planning is formulated as a convex optimization problem by using the graph of convex sets framework \citep{gcs}. The underlying idea is to decompose the collision-free space into intersecting convex sets from which a convex optimization problem is formulated. In cases where an explicit representation of the obstacles in the configuration space exists, like for single-body robots, creating collision-free convex regions can be done fast \citep{iris}. For multi-body robots, this is non-trivial. Existing work does this successfully \citep{iris_nlp, iris_c} by an optimization based approach, but the methods are still too time consuming to be used in the presence of moving obstacles. Our approach is instead to use deep learning to learn an SDF expressed in the configuration space. With this, we can query for shortest distances to the collision boundary, which allows us to expand spherical regions which are collision-free. Our approach is fast and therefore enables our suggested roadmap planner to be used in dynamic environments.
\\\\
Recent research has focused on learning collision detection \citep{fk_kernel_distance, diffco, graphdistnet} by predicting the signed distance between the robot links and the surrounding obstacles in the world space. The learned SDF is used in trajectory optimization but since the distance is expressed in the world space, the problem becomes an NLP and therefore takes a long time to solve. We take a novel approach and suggest to instead express the signed distance in the configuration space. This allows us to improve the PRM at the same time as it enables convex optimization for trajectory optimization, which runs faster and is more reliable than NLP solvers. In \cite{cspf} a learned signed distance function in the configuration space is proposed similar to our approach. However, their approach is restricted to point cloud representations, while we propose to represent the obstacles as parameterized geometric shapes, e.g. spheres. Furthermore, we also show how to use our learned SCDF to improve an existing roadmap planner.
\section{Problem formulation}
A robot is located in the world space, $\W \subset \R^3 $. The unique location of the robot is given by its configuration $\q \in \C$, where $\C$ is the configuration space. The set of points covered by the robots bodies at a certain configuration is expressed as $\B(\q) \subset \W$. The robot is surrounded by $\NrObst$ obstacles $\O = \bigcup_{i=1}^{\NrObst} \O_i$, where  $\O_i \subset \W$. The representation of the obstacle in the configuration space is the set $\C\O_i = \{\q \in \C \: |\: \B(\q) \cap \O_i \neq \emptyset \}$. The obstacle space is formed as $\Co = \bigcup_{i=1}^{\NrObst} \C \O_i$. The complement is referred to as the free space, $\Cf = \C \setminus \Co$. The path planning problem is a tuple, ($\Cf$, $\qStart$, $\qGoal$), where we want to connect a query pair, consisting of a start, $\qStart$, and goal configuration, $\qGoal$, with a geometric path, $\q(s): [0, 1] \mapsto \Cf$, such that $\q(0)=\qStart$ and $\q(1)=\qGoal$, or report correctly when such a path does not exist.
\end{document}


\section{Related Work} \label{sec:related}

% \textbf{Adversarial Attack}
\textbf{Attacks on SLAM.} 
%With the rise of machine learning, 
The robustness of computer vision systems is being actively investigated. With the emergence of adversarial images in the digital domain by adding optimized noise directly to images~\cite{szegedy2013intriguing,carlini2017towards}, researchers find that such attacks also exist physically in the real world \cite{eykholt2018robust,song2018physical,zhao2019seeing}. To fill the gap between attacks in the digital and physical worlds, recent studies have demonstrated that attacks on real-world computer vision systems are practical \cite{eykholt2018robust,li2019adversarial,man2020ghostimage,sharif2016accessorize,zhao2019seeing,zhou2018invisible}. However, attacks on traditional computer vision methods such as SLAM are relatively less explored. \cite{yoshida2022adversarial} proposes an attack against the scan matching algorithm in LiDAR-based SLAM, while most SLAMs in AR/VR devices rely on different sensors like RGB/depth cameras and IMUs. \cite{ikram2022perceptual} and \cite{chen2024adversary} mislead visual SLAM by poisoning the images with special patterns, and \cite{wang2021can} causes the camera to fail using infrared light. In our work, we demonstrate attacks on Visual-Inertial SLAM (VI-SLAM) by perturbing the IMU readings, rather than cameras, and showing its impact on XR user experience. 

\textbf{Acoustic Injection Attacks.} Among various physical attacks, acoustic injection attacks are attractive due to their low cost. Son~\etal~\cite{son2015rocking} were the first to introduce acoustic attacks on MEMS gyroscopes, demonstrating how these attacks could lead to sensor denial-of-service and result in drone crashes. WALNUT~\cite{trippel2017walnut} expanded on this by developing output biasing and control attacks that enable precise manipulation of MEMS accelerometer outputs using modulated sound waves. Wang et al.~\cite{wang2017sonic} demonstrated a sonic gun, showcasing the vulnerability of various smart devices (\eg drones and self-balancing vehicles) to acoustic attacks. Tu et al. \cite{tu2018injected} designed side-swing and switching attacks to alter the outputs of MEMS gyroscopes and accelerometers. Furthermore, Ji et al. \cite{ji2021poltergeist} fool the object detectors by applying acoustic attack to the image stabilizers commonly used in modern cameras. However, none of the existing works study the relationship between the acoustic injections and SLAM outputs on recent XR devices. 

% \zijian{Do we need one session about security in AR/VR?}
% \yicheng{TODO}
%\jiasi{cite the AIVR paper (UMass Amherst?) paper is we have not already. They add IMU perturbation but w/o SLAM, iirc} \yicheng{Cited}

\textbf{XR Security and Privacy.} 
%Security and privacy concerns in XR systems have gained significant attention. 
For single-user XR systems, researchers have demonstrated various side-channel attacks to extract sensitive information (\eg keystrokes) through video feeds~\cite{ling2019know}, head movements~\cite{nair2023unique, slocum2023going}, architectural hints~\cite{zhang2023its,shang2020arspy}, power usage~\cite{li2024dangers}, and EM side-channel leakages~\cite{al2021vr}. In multi-user XR systems, Su et al.~\cite{su2024remote} use avatar motion data to infer keystrokes in shared VR environments. Slocum et al.~\cite{slocum2024doesn} reveal vulnerabilities in the shared state frameworks of multi-user AR. Similarly, Lebeck et al.~\cite{lebeck2017securing} highlight risks like deceptive virtual objects and emphasize access control for managing shared physical and virtual spaces. Ruth et al.~\cite{ruth2019secure} further propose a secure multi-user AR framework focusing on content sharing and permissions.
Chandio et al.~\cite{chandio2024stealthy} %introduced a multi-modal spatiotemporal attack that 
simultaneously manipulated visual and inertial sensors to disrupt XR pose estimation. However, their study evaluated the attack using offline datasets and assumed the attacker's capability to manipulate IMU data streams through acoustic means, without real experiments. Ours is the first to demonstrate acoustic injection attacks on recent XR devices, like the Hololens 2, in the real world.
 


\section{Formative Study} 
\label{sec:formative}
We conducted a formative study to explore instructors' challenges in collaborative programming and their visualization needs, with the study protocol approved by our university's Institutional Review Board (IRB).
 
 \subsection{Participants}
We recruited ten participants (five females, age: \(30.6 \pm 6.6\)) with collaborative programming experience, divided into three groups: two educational technology experts (E1, E2) and eight instructors (T1–T8) with an average of 8.14 years of teaching programming. Participants, recruited via snowball sampling from the authors' network, received \$20 as compensation.
 

\subsection{Procedure}\
\RR{The formative study used semi-structured interviews conducted via Zoom, divided into two sections: (A) Questions and Answers and (B) Ratings.}
\textit{A: Questions and Answers.} 
Each participant was independently interviewed to explore the need for collaborative programming analysis. Topics included teaching experience with collaborative programming, class organization, challenges faced, assessment methods, and group and individual performance evaluations. Follow-up questions were asked for clarification or more profound insights. Each session lasted 40–60 minutes and was documented through written notes, audio, and video recordings.
\textit{B: Rating.} 
Participants then rated two aspects of the collaborative programming visual analytics system. They rated feature importance (Q1) on a 1–7 scale (7 being the highest) and ranked features by priority (Q2). 
\RR{More details are in appendix A.}

 
\subsection{Findings} 

\begin{figure*}[htbp]
	\centering
	\includegraphics[width=1\linewidth]{scheme1.pdf}
	\caption{Collaborative programming coding schemes, along with their definitions and examples.}
        \Description{Collaborative programming coding schemes, along with their definitions and examples.}
	\label{fig:scheme}
\end{figure*}



\subsubsection{Evaluation of Groups:}
In large classes, instructors struggled to monitor each group without the help of teaching assistants. They often relied on group presentations (E2, T1, T5), but it was inaccurate. Limited class time forced instructors to focus on solving issues rather than actively monitoring groups.


\textbf{Group Performance:}
Evaluating group work was based on the final code rather than the process, limiting the ability to provide feedback (T7). Some instructors used presentations or technical documentation to streamline assessments (T6). However, evaluating more than 20 groups was still challenging, and instructor assistance was often overlooked during evaluations.


\textbf{Collaboration Evaluation:}
Effective collaboration wasn't just about task completion speed and group dynamics. Some groups (T7) completed tasks quickly due to one member's efforts, not true collaboration. 
\RR{E1 proposed the collaborative problem-solving framework to distinguish task effectiveness from team effectiveness.} Instructors (T6) believed monitoring discussion time helped assess collaboration quality and task difficulty, but off-topic discussions made it hard to evaluate group discussions (E1, T3, T5).


\subsubsection{Evaluation of Students:}

In large classes, assessing individual contributions in group work was challenging. Instructors relied on peer evaluations (T2, T3, T4) and self-reported task distributions (E2), which were often subjective.

\textbf{Individual Performance:}
Instructors typically reviewed code to assess understanding, but measuring individual contributions in group work was hard. Leadership roles often reflected a deeper grasp of concepts (T1), but tracking individual engagement was difficult in large classes.


\textbf{Personalized Feedback:}
Providing personalized feedback was difficult, as group results often masked individual struggles. T3 and T4 noted that group collaboration fostered peer learning but could lead to less engagement from weaker students.
\textit{T8 added that offering personalized feedback in large classes was time-consuming and burdensome.}

 
 
 
\subsection{Design requirement}
\label{dr}
Based on the interview findings, we identified six design requirements (R1–R6) across three levels, summarized as follows:

Support \textbf{\textit{inter-group-level}} to provide a macro perspective, enabling instructors to observe the overall situation of all groups comprehensively and fully understand class-wide dynamics.

\textbf{R1: Displaying the Overall Performance of all Groups.} 
Instructors face challenges in supervising multiple groups simultaneously and shifting focus efficiently. Participants stressed the need for an overview of group performance, allowing instructors to grasp class dynamics and selectively review specific groups.


\RR{\textbf{R2: Comparing Similar and Different Groups.} 
Instructors often compare students' performance to assess their relative standing within the class~\cite{marsh1997making}. Such comparisons enable a more accurate evaluation of group performance and help identify groups excelling or encountering challenges.}


Supporting \textbf{\textit{intra-group-level}} visual exploration to offer a meso perspective, enabling instructors to observe specific groups' performance and gain a comprehensive understanding of group dynamics during the collaborative programming process.


\textbf{ R3: Understand the Dynamics of Programming Problem Solving.}
Analyzing a group's evolving communication patterns and computational thinking during programming tasks provides instructors with deeper insights into students' progress and intermediate learning objectives—details missed in final submissions alone.


\textbf{ R4: Identify Teacher Scaffolding in Collaboration.}
Instructors play a vital role in guiding groups during collaborative programming. Understanding the scaffolding provided and students' responses can help refine instructional strategies, improving the overall effectiveness of collaborative programming.


Supporting \textbf{\textit{individual-student-level}} visual exploration to provide a micro perspective allows instructors to observe each student's performance within a specific group and better understand their role and collaboration.

\textbf{R5: Track Changes in Student Engagement Over Time.} 
Limited classroom time makes it challenging for instructors to monitor individual student engagement in programming tasks.
Tracking and visualizing engagement trends is essential for assessing performance and refining instructional practices.

\textbf{R6: Access Detailed Raw Data.} 
Instructors require access to raw data, such as collaboration videos, conversations, and background information. These details are crucial for validating analysis results and supporting personalized feedback and assessments.


\begin{figure*}
    \centering
    \includegraphics[width=1\linewidth]{bar2.pdf}
    \caption{(a) shows the bar chart of the raw data, (b) presents the results of applying Moving Average Smoothing to reduce anomalies in prediction percentages, and (c) highlights the reduction of visual clutter and emphasizes sequential behavior patterns after merging behaviors of the same category.}
    \label{fig:bar}
    \Description{(a) shows the bar chart of the raw data, (b) presents the results of applying Moving Average Smoothing to reduce anomalies in prediction percentages, and (c) highlights the reduction of visual clutter and emphasizes sequential behavior patterns after merging behaviors of the same category.}
\end{figure*}

\section{Data Collection and Processing}
\label{sec:data}
\RR{In this section, we provided an overview of the data collection context and introduced the collaborative programming performance framework along with its metric quantification methods.}

\subsection{Data Collection}
We collaborated with Professor E1, an expert in programming education, and teaching assistants (TA1 and TA2), experienced in Python, to collect data from E1's Spring 2023 Python course with 66 non-computer science freshmen in 22 groups. Using non-intrusive methods, we recorded group discussions, screen activities (without audio), and code submissions. Session lengths ranged from 10 to 60 minutes based on question completion. 
Due to data quality issues, we selected data from 19 groups (57 students) for analysis.


\subsection{Data Preprocessing}
In collaborative programming analysis, students' spoken content was key to understanding discussion and evaluating collaboration. We used the Faster-Whisper model~\cite{fasterwhisper} for speech recognition and the Pyannote-audio model~\cite{pyannoteaudio} for speaker diarization. 
For groups lacking clear problem-solving strategies, we used Tesseract OCR~\cite{tesseract} to analyze screen recordings and extract key frames through screenshots.

\subsection{Scope of Collaborative Programming Performance Framework}
Evaluating student and group performance in collaborative programming required considering multiple dimensions~\cite{hawlitschek2023empirical}.  
Building on literature and expert input (E1), we proposed the following comprehensive analytical framework to assess performance. 



\subsubsection{Student Performance Assessment}
\label{shema}
Previous research demonstrated that students' skills, backgrounds, and personalities in the classroom vary significantly, affecting their engagement and learning outcomes~\cite{wu2019analysing}. 
Therefore, we focus on each student's \textit{background} (prior academic performance and major), \textit{role transitions}, \textit{behavioral engagement}, and \textit{cognitive engagement}.






\textbf{Problem-solving Categorization:}
Based on previous frameworks~\cite{wu2019analysing}, team theory~\cite{zhao2023analysing}, and collaborative coding processes~\cite{sun2021three}, we developed a coding scheme (Fig.~\ref{fig:scheme}) to capture group problem-solving in collaborative programming. 
The scheme used four color-coded categories to represent discussion types. 
The first three categories followed a hierarchical structure, indicating discussion depth, while the green category focuses on situation awareness and specific behaviors.

Building on the scheme, we used tailored prompts with the ChatGPT-4o model~\cite{gpt4o} to classify behavioral patterns in transcribed dialogue \RR{(More details are in appendix B)}. 
\RR{The model provided a prediction percentage of uncertainty for each classification, improving result interpretability. }
To minimize anomalies, we applied a ``moving window'' technique with Moving Average Smoothing~\cite{chang2022muse}, stabilizing prediction percentages (Fig.\ref{fig:bar}-b). To reduce visual clutter in long time-series data, we aggregated consecutive instances of the same category, averaging prediction percentages (Fig.\ref{fig:bar}-c). These results were displayed in the timeline panel's progress bar, enabling detailed analysis by zooming into specific behavior categories in Sec.~\ref{barchart}. 




\textbf{Roles Extraction:}
We analyzed each speaker's dynamic roles (Driver, Navigator, and Monitor) during programming~\cite{lewis2011pair}. Using ChatGPT-4o and prompts based on the Thought Chain Model~\cite{wei2022chain}, we guided the model through step-by-step reasoning to generate role classifications. Prompts were iterated for clarity, and the model's responses were structured hierarchically and returned in JSON format. Each query was repeated ten times, with the majority result adopted for classification.

\RR{\textbf{Behavioral Engagement:} reflected the level of effort and participation students invested in learning~\cite{fredricks2022measurement}. 
In our study, we focused on the duration and frequency of student speech.} 
We extracted conversation data, excluding irrelevant chat, and divided each conversation into two parts: the first half and the full conversation. We then measured speaking duration, frequency, and degree centrality using co-occurrence networks~\cite{ng1999toward}. For each question, we created and normalized two networks, followed by Non-negative Matrix Factorization (NMF)~\cite{lee2000algorithms} to identify key behavioral patterns for dynamic group comparison.


\RR{\textbf{Cognitive Engagement:} referred to the cognitive investment students made in their learning. We highlighted the role changes and behavior frequencies of students during the collaborative process. }
To capture dynamic changes in student cognitive engagement, we split the dialogue for each question into two segments: the first half and the full dialogue. We extracted the frequency of each speaker's 14 behavioral categories and their roles at each timestamp. After normalizing these features for consistency, we applied NMF to reduce dimensionality and assess each speaker's cognitive engagement.

\begin{figure*}
  \includegraphics[width=\textwidth]{CPVis.pdf}
  \caption{\RR{A screenshot of Group 10 view.} \textit{CPVis} applies multimodal learning analysis to provide instructors with evidence for evaluating group and student performance. It consists of three views:
Filter View (A) Provides an overview and allows group selection. The selected groups appear in the lasso selection area (A2), and the similarity panel (A3) displays the most similar and different groups based on the search (A1a).
Content View (B) Displays group performance, with the B1 panel showing completed codes, the B3a panel illustrating the behavior sequence, and the B3b panel showing student engagement over time.
Detail View (C) Presents the group's collaborative programming video (C1) and raw conversation data (C2).}
  \Description{A screenshot of Group 10 view. \textit{CPVis} applies multimodal learning analysis to provide instructors with evidence for evaluating group and student performance. It consists of three views:
Filter View (A) Provides an overview and allows group selection. The selected groups appear in the lasso selection area (A2), and the similarity panel (A3) displays the most similar and different groups based on the search (A1a).
Content View (B) Displays group performance, with the B1 panel showing completed codes, the B3a panel illustrating the behavior sequence, and the B3b panel showing student engagement over time.
Detail View (C) Presents the group's collaborative programming video (C1) and raw conversation data (C2).}
  \label{fig:teaser}
  \end{figure*}

\subsubsection{Group Performance Assessment}
We evaluated group performance based on three dimensions: code quality, collaborative problem-solving, and teacher scaffolding. 
Through in-depth discussions with domain experts, we assessed how each dimension was valued and measured in the context of our study.




\label{code}
\textbf{Code quality}, reflecting students' mastery of course concepts, was a key metric for evaluating group performance. To assess student submissions, we used ChatGPT-4o~\cite{gpt4o} to evaluate dimensions such as problem-solving, code integrity, accuracy, and algorithmic innovation, scoring each on a 1–5 scale. After refining evaluation prompts, we ran the assessment ten times per submission, averaging the results to ensure consistency and reliability.





\textbf{Collaborative Problem-Solving (CPS):} 
Earlier studies categorized CPS into team effectiveness and task effectiveness~\cite{rosen2020towards}. Team effectiveness was measured by student engagement, while task effectiveness was assessed through code quality. %Our analysis captured problem-solving behaviors by frequency and sequence.
To evaluate CPS, we examined task effectiveness, represented by the average question score (\(\bar{s}\)), and team effectiveness, assessed through the standard deviation of engagement (\(\sigma_e\)) and the average engagement score (\(\bar{e}\)) as shown in Equation \ref{eq:1}. We then used the coefficient of variation (\(CV_e\)) \RR{to account for both engagement variability and engagement}. Finally, the overall collaboration quality was calculated using Equation \ref{eq:2}, combining question performance and engagement balance. 
\begin{equation}
\sigma_e = \sqrt{\frac{1}{n} \sum_{i=1}^{n} (e_i - \bar{e})^2}, \quad CV_e = \frac{\sigma_e}{\bar{e}}
\label{eq:1}
\end{equation}

\begin{equation}
Quality = \bar{s} \cdot (1 - CV_e)
\label{eq:2}
\end{equation}
As shown in Table \ref{table:comparison}, Group 19, despite achieving a respectable average score, exhibited imbalanced engagement, leading to a lower collaboration quality score. In contrast, Group 20 demonstrated more balanced and higher engagement, resulting in a better overall collaboration quality.
\begin{table}[htbp]
\centering
\begin{tabular}{cccccc}
\toprule
\textbf{Group} & \(\bar{s}\) & \textbf{Engagement Levels} & \(\sigma_e\) & \(\text{CV}_e\) & \textbf{CQ} \\
\midrule
Group 19 & \(4.11\) & (10.515, 9.725, 4.575) & \(2.80\) & \(0.24\) & \(2.80\) \\
Group 20 & \(4.14\) & (10.06, 9.32, 8.62) & \(0.73\) & \(0.08\) & \(3.88\) \\
\bottomrule
\end{tabular}
\caption{Comparison of Group 19 and Group 20 on Collaboration Quality (CQ).}
\label{table:comparison}
\end{table}

\textbf{Teacher Scaffolding,} categorized into cognitive (low, medium, high-control) and metacognitive forms~\cite{ouyang2022applying}, reflected the level of support provided to a group and its impact on programming performance. We evaluated four scaffolding dimensions, leveraging GPT-4o for annotation. By using targeted prompts and examples, we improved classification accuracy, while teacher scaffolding was categorized according to the type of support based on a semantic analysis of interactions.






\section{System Design} 
\label{sec:vis}

In this session, we introduce \textit{CPVis} (Fig.~\ref{fig:teaser}), a web-based visual analytics system to assist instructors in evaluating collaborative programming.
\subsection{System Overview}
\textit{CPVis} is a comprehensive system supporting multi-level, progressive analysis, from group-level interactions to individual student performance. Instructors can select specific groups for focused analysis, such as comparative evaluations (R2). Additionally, \textit{CPVis} offers a drill-down feature, enabling an overview of student collaboration and detailed insights into individual performance.
\textit{CPVis} includes four main components:
\textbf{Initial Selection}, instructors can select groups using the lasso tool or search function in the ``Group Overview'' view (Fig.\ref{fig:teaser}-A1). The system displays an overview of the selected groups (Fig.\ref{fig:teaser}-A2), compares similar and different groups (Fig.\ref{fig:teaser}-A3), and synchronizes updates across views (Fig.\ref{fig:teaser}-B).
\textbf{Drill-Down Analysis}, instructors can examine the group's code (Fig.\ref{fig:teaser}-B1) and analyze problem-solving approaches in the Content View. The interaction pattern panel (Fig.\ref{fig:student}-B2b) reveals behavior patterns, while the timeline panel (Fig.\ref{fig:teaser}-B3a) shows activity sequences. The student overview panel (Fig.\ref{fig:student}-B2a) compares individual performance across the class, and the timeline panel (Fig.~\ref{fig:teaser}-B3) highlights engagement and role changes.
\textbf{Multi-Level Interaction}, the system's layered visualization allows instructors to explore and analyze both group and individual behavior, enabling precise assessment of the collaboration process. It also supports side-by-side group comparisons.
\textbf{Detailed Review}, the detail view (Fig.~\ref{fig:teaser}-C) provides original discussion videos and transcripts, enabling in-depth analysis of student conversations and problem-solving processes.
\begin{figure}
	\centering
	\includegraphics[width=1\linewidth]{design.pdf}
	\caption{The iterative design of student glyphs: 1 represents cognitive engagement, and 2 represents behavioral engagement.}
        \Description{The iterative design of student glyphs: 1 represents cognitive engagement, and 2 represents behavioral engagement.}
	\label{fig:designl}
\end{figure}
\subsection{Visual Design}
\begin{figure*}
	\centering
	\includegraphics[width=1\linewidth]{encoding.pdf}
	\caption{The flower metaphor in \textit{CPVis}, along with its visual encoding, color coding, and some samples.}
        \Description{The flower metaphor in \textit{CPVis}, along with its visual encoding, color coding, and some samples.}
	\label{fig:encoding1}
\end{figure*}

We iterated continuously during the glyph design process to optimize the visual representation. Initially, we used a star-shaped design (e.g., radar charts), as shown in Fig.~\ref{fig:designl}. While radar charts effectively displayed behavioral and cognitive engagement, dividing the chart into sections for individual tasks introduced an unnecessary dimension (shape size) that was meaningless and prone to misinterpretation. 
We then shifted to a circular design, encoding behavioral engagement as the radius of a sector and cognitive engagement as the color of the outer ring. However, this design had a significant flaw: color mapping was less intuitive than size mapping, and using color saturation as a visual channel lacked precision. Additionally, both designs struggled to combine individual students into group glyphs intuitively.
\RR{Inspired by previous research~\cite{tausch2014groupgarden,tausch2016comparison,xiong1999peoplegarden}, we introduced a visual design inspired by the flower metaphor, where students are represented as flowers (Fig.~\ref{fig:encoding1}).
The size of the petals represented behavioral engagement, while the size of the stamen indicated cognitive engagement. Three colors were used to represent three different roles, and the varying colors of the leaves symbolized different levels of teacher scaffolding. The number of butterflies reflected the level of collaborative problem-solving ability. As a result, the overall group glyph naturally took the form of a bouquet (R4).}
This approach resolved the issue of merging individual students into group glyphs while enhancing the design's readability and intuitiveness (R1). 
\RR{The final design struck a balance between aesthetics and functionality, effectively conveying the performance of individual students and groups, allowing users to quickly compare similar or different groups (R2).}


\subsection{Filter View}
The Filter View (Fig.~\ref{fig:teaser}-A) serves as the starting point for analysis, featuring an interactive projection panel (Fig.~\ref{fig:teaser}-A1) and a similarity panel (Fig.~\ref{fig:teaser}-A3) to help users filter and explore groups of interest. 
The projection panel displays the distribution of groups in a 2D space to reveal clustering patterns and outliers. We apply the t-SNE algorithm to maximize separation between dissimilar groups, creating clearer clusters. 
To avoid visual clutter caused by group glyphs in dimensionality-reduced views, we follow the approach of Tac-Miner~\cite{wang2021tac}, representing groups as points or rectangles based on whether they received teacher scaffolding, with color coding reflecting prior performance. 
\RR{Additionally, the outer arc represents the duration of the discussion.}
Users can select groups using the lasso tool (Fig.~\ref{fig:teaser}-A1b) or search for specific groups (Fig.~\ref{fig:teaser}-A1a). The similarity panel (Fig.~\ref{fig:teaser}-A2) shows the most similar and dissimilar groups based on Euclidean distance.  
 

 
\subsection{Content View}
Once the search group is selected, users can perform a detailed analysis through the content view (Fig.~\ref{fig:teaser}-B). This view comprises four panels, allowing for a layered exploration of group and individual student performance. 

\subsubsection{Codes Panel}
In the upper left corner of the content view, a control button (Fig.~\ref{fig:teaser}-B1) allows users to toggle between the Codes panel and the Student Projection/Group Pattern panels. Below, users can compare the code quality between the selected group (left) and a comparison group (right). 
For instance, Figure~\ref{fig:teaser}-B1 shows Group 10's answer to Question 5 (left) and Group 18's answer (right). Hint boxes provide two types of feedback: red tips indicating code deficiencies (Fig.~\ref{fig:teaser}-B1a) and pink tips showing that a method was learned from pre-class materials, signaling comprehension, and application. 
Below the code, the ChatGPT-4o score and its rationale are provided, enabling quick, in-depth code evaluation and highlighting areas where groups faced challenges. 


 
\subsubsection{Students Projection (Fig.~\ref{fig:student}-B2a)}

We project students from different groups using t-SNE for clustering, highlighting students with similar performance. 
The three flowers representing the search group are connected with dashed lines to clarify group member distribution and similarity, helping users assess group homogeneity or heterogeneity. 
Only the comparing group's flowers are highlighted to minimize visual clutter. 
\RR{Users can zoom in/out to explore specific students and view detailed background information (e.g., major, grades) by hovering over individual points (Fig.~\ref{fig:student}-B2c).}

\subsubsection{Group Pattern Panel (Fig.~\ref{fig:student}-B2b \& d)}
We use Epistemic Network Analysis (ENA)~\cite{zhao2023analysing} to analyze the dynamic connections between cognitive elements and the collaborative problem-solving behaviors of groups (R3). 
In the Group Pattern Panel, each node represents a behavior in the collaborative problem-solving process, with colors following the coding scheme in Section~\ref{fig:scheme}. Node size indicates behavior frequency, and the intensity of the color reflects the frequency of interactions between behaviors. More significantly, darker nodes represent more frequent behaviors and interactions. 
Users can click on different question buttons to examine dynamic changes in behavior across specific questions or select multiple questions to observe how behaviors evolve during transitions between tasks. Hovering over nodes reveals detailed information about each behavior.
\RR{When comparing two groups, the system displays side-by-side behavior networks.} 
The Group Pattern Panel displays the search group's behavior patterns and compares them to those of the comparison group during the collaborative problem-solving process (Fig.~\ref{fig:student}-B1b).


\subsubsection{Timeline Panel}
\label{barchart}
Users can toggle between the search and comparing groups using the control button in the upper left corner of the Timeline Panel (R5). This panel displays group and individual student performance over time using a filterable bar chart and line chart (Fig.~\ref{fig:teaser}-B3).
In the bar chart, each bar represents a timestamp, with colors indicating different collaborative problem-solving behaviors. The bar height reflects uncertainty, as calculated by ChatGPT-4o. Dashed lines separate different questions for visual clarity. Hovering over a bar reveals the behavior category, predicted certainty, and reasoning behind the label (Fig.~\ref{fig:teaser}-B3e). Clicking on a bar takes users to the relevant conversation content (Fig.~\ref{fig:teaser}-C2a).
The filtering function (Fig.~\ref{fig:teaser}-B3c) allows users to zoom in on specific periods, magnifying the bars for more detailed analysis. 
\RR{The progress bar consolidates behaviors, minimizing visual clutter and highlighting key shifts in temporal behavior in the overview, while also allowing for detailed tracking of group dynamics (Fig.~\ref{fig:group18}-c).}

Three line charts track the behaviors and cognitive engagement of three students in the group (Fig.~\ref{fig:teaser}-B3b). Engagement is calculated at the midpoint and end of each question, and Savitzky–Golay filtering smooths the curves to highlight dynamic changes and trends across questions.
Below the charts, role types are mapped using equally sized rectangles at each timestamp. Users can zoom in on the timeline with the progress bar for detailed analysis. 






\subsection{Detail View}
In the Timeline Panel, users can link to specific conversation content to review group discussions (Fig.~\ref{fig:teaser}-C1) and individual student conversations for each question (Fig.~\ref{fig:teaser}-C2). The playback feature allows users to revisit the original video of collaborative programming sessions, providing a more immersive classroom experience. This feature validates analysis results, offers detailed references, and supports the final step of our analysis workflow (R6).

\begin{figure*}
 	\centering
 	\includegraphics[width=1\linewidth]{Student_panel.pdf}
 	\vspace{-1em}
 	\caption{The panel displayed in the Content View after selecting ``Student Projection \& Group Pattern'' is shown \RR{(A screenshot of the Group 10 \& 18 view)}. On the left (B2a), the projection of all students highlights the searched and compared groups, showing connections between students within the searched group. Hovering over a flower reveals the student's background information (B2c). On the right (B2b), group patterns for two groups across different questions are shown, with clickable question buttons to explore changes in dynamic learning behavior.}
        \Description{The panel displayed in the Content View after selecting ``Student Projection \& Group Pattern'' is shown \RR{(A screenshot of the Group 10 \& 18 view)}. On the left (B2a), the projection of all students highlights the searched and compared groups, showing connections between students within the searched group. Hovering over a flower reveals the student's background information (B2c). On the right (B2b), group patterns for two groups across different questions are shown, with clickable question buttons to explore changes in dynamic learning behavior.}
 	\label{fig:student} \end{figure*}


% \begin{table*}[t]
\centering
\caption{A case study examining \model's response to a query about ``\emph{graders}'' in OpenAI's 12-day edu    cational series demonstrates its video retrieval capabilities and knowledge synthesis.}
\label{tab:case study}
\small
\begin{tabularx}{\textwidth}{X X}

\toprule
\textbf{Query:} Explain the purpose and functionality of 'graders' in the context of reinforcement fine-tuning.\\

\midrule
\textbf{Input Videos:} \href{https://www.youtube.com/playlist?list=PLOXw6I10VTv9lin5AzsHAHCTrC7BdVdEM}{12 Days of OpenAI} (A total of 12 videos) \\

\midrule

\textbf{\model\ Response:} 

\texttt{\#\#\# Understanding Gradation in Reinforcement Fine-Tuning}

\texttt{**Purpose of Graders**}

\texttt{Graders are integral components in the process of reinforcement fine-tuning, mainly utilized to evaluate and improve the performance of machine learning models...} (Omitted for brevity)

\texttt{**Functionality of Graders**}

\texttt{\colorbox{lightgreen}{The grading system operates by comparing a model's outputs against those of}}

\texttt{\colorbox{lightgreen}{verified correct responses. The mechanics of this assessment involve assigning}}

\texttt{\colorbox{lightgreen}{scores to the model's outputs based on their correctness. These scores range}}

\texttt{\colorbox{lightgreen}{from zero to one, where:}}


\texttt{\colorbox{lightblue}{- A score of **0** indicates that the model's answer is completely incorrect.}}

\texttt{\colorbox{lightblue}{- A score of **1** signals that the answer is fully correct.}}

\texttt{- Scores can also be fractional, allowing for partial credit. For instance, \colorbox{lightpink}{a score of **0.7** may denote that the model's suggestion was }}
\texttt{\colorbox{lightpink}{correct but not the highest in a ranked list of possible answers.}} 

\\

\texttt{The process includes:}

\texttt{1. **Input Comparison**: The grader takes both the model’s output and the correct answer, analyzing them to derive a score.}

\texttt{2. **Graded Outputs**: This interaction allows for a dynamic reflection of the model's performance, where the scores directly influence how the model is fine-tuned in subsequent training iterations.}

\texttt{3. **Feedback Loop**: The scores obtained from graders provide feedback that can be used to adjust learning parameters and improve the model’s future responses. This highlights the feedback mechanism that is crucial for reinforcement learning.}

\\

(The remaining content are omitted for brevity.)

\\

\midrule
\textbf{Retrieved Sources by \model:}

\model\ accurately retrieved four continuous video segments from the OpenAI's 12-day show, ranging from 10:00 to 12:00 in Day 2 of the video "\href{https://www.youtube.com/watch?v=yCIYS9fx56U}{Reinforcement Fine-Tuning}." Here, we highlight key moments relevant to the detailed content in the answer. From left to right, these are retrieved moments at timestamps \colorbox{lightgreen}{10:35}, \colorbox{lightblue}{10:39}, and \colorbox{lightpink}{11:10}, which provide informative insights that help \model\ give a comprehensive answer to the query.

\\

\begin{tabular}{ccc}
    {\includegraphics[width=0.3\textwidth]{figs/openai-1.png}} &
    {\includegraphics[width=0.3\textwidth]{figs/openai-2.png}} &
    {\includegraphics[width=0.3\textwidth]{figs/openai-3.png}} \\
\end{tabular}

\\

\bottomrule

\end{tabularx}
\vspace{-0.2in}
\end{table*}
\section{Evaluation}
We provide three sets of insights into this section, organised as \textit{findings (F*)}. We quantitatively study the effect of the adversarial and counterfactual perturbations on the performance of informal reasoners and autoformalisation methods. Then, we dive deeper into method variants. Finally, 
we analyse the nature of formalisation errors made by the models.

\subsection{Robustness Analysis}
\paragraph{\textbf{\emph{F1: Noise perturbations have a stronger effect on formalisation methods than informal \ac{LLM} reasoners.}}}
Table~\ref{tab:distraction_k4_formalisation} shows that, on average, the accuracy of both direct and \ac{CoT} informal reasoning remains between $73\%$ and $74\%$ in the face of added noise. While the autoformalisation method performs similarly to informal reasoners on the original dataset, its performance decreases between $4\%$ and $11\%$. The accuracy drops especially with logical (L) and tautological (T) distractions, whose logical language formats trick the \ac{LLM} into formalizing the noisy clauses. On the other hand, the linguistically complex and more natural sentences of encyclopedic distractions show a minor effect, suggesting that \acp{LLM} successfully avoids formalizing the more complicated sentences.

\paragraph{\textbf{\emph{F2: All \ac{LLM}-based reasoning methods suffer a drop for counterfactual perturbations.}}} % influence .}}}
Table~\ref{tab:distraction_k4_formalisation} shows that counterfactual statements cause a significant decrease in performance for both the informal reasoners and autoformalisation methods of between $12\%$ and $13\%$ on average. 
Moreover, this observation also holds for all tested models, i.e., none are robust towards counterfactual perturbations across every evaluated dimension. Even the strongest model, GPT 4o-mini, yields a performance of 63-68\%, which is relatively close to the random performance of 50\%. The high impact of counterfactual statements (the single ``not'' inserted) could be due to the inability of \acp{LLM} to overwrite prior knowledge with explicitly stated information or memorization of the answers. We study the error sources further in §\ref{subsec:errors}.  

\noindent \paragraph{\textbf{\emph{F3: Introducing multiple noise sentences has an effect only for logical distractions.}}}
We show the impact of introducing between one and four sentences for the two top-performing autoformalisation models in Figure~\ref{fig:length_distraction}. The figure shows similar trends with and without counterfactual perturbations.
As additional logical distractions are introduced, the model performance consistently decreases. Tautological (T) distractions lead to a decline in accuracy with a single disruptive sentence, yet adding more noise does not worsen the outcome. 
The tautological corpus introduces truth constants for all sentences as a persistent unseen logical construct. Given that this leads only to a decrease for a single occurrence, we can assume that a model can consistently handle the same unseen logical construct. In contrast, the logical corpus increases the chance of adding text, requiring new, previously unseen reasoning constructs for each added sentence. The impact of encyclopedic noise remains negligible, generalising F1 to $k$ sentences. Similarly, counterfactual perturbations remain much more effective for all settings, generalising F2.

\begin{table}[!t]
\small
\setlength{\modelspacing}{2pt}
\setlength{\tabcolsep}{1.7pt} % Default value: 6pt
\setlength{\belowrulesep}{4pt}
\begin{threeparttable}
    \centering
    \begin{tabular}{cc l r rrr @{\quad} rrrr}
\toprule
\multirow{2}{*}{} & \multirow{2}{*}{} & Reasoning & \multirow{2}{*}{O} & \multicolumn{3}{c}{Distraction} & \multicolumn{4}{c}{Counterfactual} \\
 & & Format & & E& L & T & $\text{O}_C$ & $\text{E}_C$& $\text{L}_C$ & $\text{T}_C$\\
\midrule
\multirow{6}{*}{\rotatebox{90}{Gemma-2}} & \multirow{3}{*}{\rotatebox{90}{9b}}
   & Informal (direct) & \textbf{0.78} & \textbf{0.80} & \textbf{0.79} & \textbf{0.77} & 0.58 & 0.52 & 0.50 & 0.59 \\
 & & Informal (CoT) & 0.72 & 0.78 & 0.73 & 0.76 & 0.61 & \textbf{0.57} & \textbf{0.60} & \textbf{0.66} \\
 & & Formal (FOL) & 0.62 & 0.58 & 0.52 & 0.53 & \textbf{0.63} & 0.52 & 0.46 & 0.46 \\[\modelspacing]
\cmidrule{2-11}
 & \multirow{3}{*}{\rotatebox{90}{27b}} 
   & Informal (direct) & 0.71 & 0.69 & \textbf{0.66} & \textbf{0.68} & 0.59 & 0.51 & 0.54 & 0.59 \\
 & & Informal (CoT) & 0.66 & 0.65 & 0.64 & 0.63 & 0.62 & 0.58 & \textbf{0.62} & \textbf{0.64} \\
 & & Formal (FOL) & \textbf{0.74} & \textbf{0.74} & 0.61 & 0.61 & \underline{\textbf{0.72}} & \underline{\textbf{0.67}} & 0.58 & 0.51 \\[\modelspacing]
\midrule
\multirow{6}{*}{\rotatebox{90}{Mistral}} & \multirow{3}{*}{\rotatebox{90}{7B}} 
   & Informal (direct) & 0.77 & \textbf{0.77} & 0.75 & \textbf{0.79} & \textbf{0.63} & \textbf{0.54} & \textbf{0.54} & \textbf{0.66} \\
 & & Informal (CoT) & \textbf{0.79} & 0.75 & \textbf{0.77} & 0.78 & 0.55 & 0.52 & \textbf{0.54} & 0.58 \\
 & & Formal (FOL) & 0.62 & 0.58 & 0.54 & 0.57 & 0.50 & \textbf{0.54} & 0.51 & 0.52 \\[\modelspacing]
\cmidrule{2-11}
 & \multirow{3}{*}{\rotatebox{90}{Small}} 
   & Informal (direct) & \textbf{0.77} & \textbf{0.76} & \textbf{0.76} & \textbf{0.75} & 0.61 & 0.51 & 0.56 & 0.59 \\
 & & Informal (CoT) & 0.72 & 0.72 & 0.72 & 0.71 & \textbf{0.62} & \textbf{0.59} & \textbf{0.62} & \textbf{0.68} \\
 & & Formal (FOL) & 0.68 & 0.59 & 0.53 & 0.64 & 0.54 & 0.55 & 0.49 & 0.51 \\[\modelspacing]
\midrule
\multirow{6}{*}{\rotatebox{90}{Llama-3.1}} & \multirow{3}{*}{\rotatebox{90}{8B}} 
   & Informal (direct) & 0.63 & 0.61 & 0.64 & 0.66 & 0.61 & \textbf{0.62} & 0.59 & 0.61 \\
 & & Informal (CoT) & 0.73 & \textbf{0.73} & \textbf{0.71} & \textbf{0.72} & \textbf{0.62} & 0.59 & \textbf{0.61} & \textbf{0.65} \\
 & & Formal (FOL) & \textbf{0.77} & 0.71 & 0.63 & 0.52 & 0.60 & 0.58 & 0.55 & 0.52 \\[\modelspacing]
\cmidrule{2-11}
 & \multirow{3}{*}{\rotatebox{90}{70B}} 
   & Informal (direct) & 0.77 & 0.74 & 0.74 & 0.73 & 0.62 & 0.53 & 0.56 & 0.64 \\
 & & Informal (CoT) & \textbf{0.78} & \textbf{0.75} & \textbf{0.76} & \textbf{0.76} & 0.64 & 0.61 & \textbf{0.66} & \underline{\textbf{0.73}} \\
 & & Formal (FOL) & 0.74 & 0.73 & 0.71 & 0.71 & \textbf{0.66} & \textbf{0.62} & 0.59 & 0.57 \\[\modelspacing]
 \midrule
\multirow{3}{*}{\rotatebox{90}{GPT}} & \multirow{3}{*}{\rotatebox{90}{4o-mini}} 
   & Informal (direct) & 0.78 & 0.77 & 0.79 & 0.79 & 0.64 & 0.61 & 0.61 & 0.63 \\
 & & Informal (CoT) & 0.80 & 0.80 & \underline{\textbf{0.81}} & \underline{\textbf{0.82}} & \textbf{0.68} & \textbf{0.63} & \underline{\textbf{0.68}} & \textbf{0.64} \\
 & & Formal (FOL) & \underline{\textbf{0.84}} & \underline{\textbf{0.82}} & 0.73 & 0.79 & 0.63 & 0.62 & 0.57 & 0.54 \\[\modelspacing]
 \midrule
\multicolumn{2}{c}{\multirow{3}{*}{\textbf{Avg}}} 
 & Informal (direct) & 0.74 & 0.73 & 0.73 & 0.73 & 0.61 & 0.55 & 0.56 & 0.62 \\
 & & Informal (CoT) & 0.74 & 0.74 & 0.73 & 0.74 & 0.62 & 0.58 & 0.62 & 0.65 \\
  & & Formal (FOL) & 0.72 & 0.68 &	0.61 & 0.62 & 0.61 & 0.59 & 0.54 & 0.52 \\
\bottomrule
\end{tabular}
\caption{Accuracies of informal and autoformalisation-based deductive reasoners. The best overall model per dataset is underlined; the best model version is marked in bold.}
\label{tab:distraction_k4_formalisation}
\end{threeparttable}
\end{table} 

\begin{figure}[!t]
    \centering
    \scriptsize
    \begin{tikzpicture}
        \begin{axis}[name=gpt,
            title={GPT-4o-mini},
            width=0.6\linewidth,
            height=0.6\linewidth,
            xlabel={\# Noise sentences},
            ylabel={Accuracy},
            xmin=-0.1, xmax=4.1,
            ymin=0.5, ymax=0.9,
            xtick={1,2,4},
            ytick={0.55, 0.6, 0.65, 0.75, 0.8, 0.85},
            title style={yshift=-0.6em},
            legend style={at={(1,-0.15)},
	           anchor=north,legend columns=-1},
            x label style={at={(axis description cs:1,-0.05)},anchor=north},
            y label style={at={(axis description cs:-0.15,0.5)},anchor=south},
            ymajorgrids=true,
            grid style=dashed,
        ]
            \addplot[color=blue, mark=square,]
                coordinates {
                (0,0.848076939582825)(1,0.823076903820038)(2,0.826923072338104)(4,0.821153819561005)
                };
            \addplot[color=red, mark=triangle,]
                coordinates {
                (0,0.848076939582825)(1,0.817307710647583)(2,0.801923096179962)(4,0.759615361690521)
                };
            \addplot[color=green, mark=diamond,] 
                coordinates {
                (0,0.848076939582825)(1,0.767307698726654)(2,0.769230782985687)(4,0.803846180438995)
                };
            \addplot[color=blue, mark=square*] 
                coordinates {
                (0,0.627777755260468)(1,0.622222244739533)(2,0.600000023841858)(4,0.633333325386047)
                };
            \addplot[color=red, mark=triangle*,] 
                coordinates {
                (0,0.627777755260468)(1,0.611111104488373)(2,0.611111104488373)(4,0.594444453716278)
                };
            \addplot[color=green, mark=diamond*,] 
                coordinates {
                (0,0.627777755260468)(1,0.572222232818604)(2,0.538888871669769)(4,0.555555582046509)
                };
                \legend{E,L,T,$\text{E}_C$, $\text{L}_C$ , $\text{T}_C$}
        \end{axis}

        \begin{axis}[name=llama, at={($(gpt.east)+(0.1cm,0)$)},anchor=west,
            title={Llama 3.1 70b},
            width=0.6\linewidth,
            height=0.6\linewidth,
            xmin=-0.1,, xmax=4.1,
            ymin=0.5, ymax=0.9,
            xtick={1,2,4},
            ytick={0.55, 0.6, 0.65, 0.75, 0.8, 0.85},
            title style={yshift=-0.6em},
            yticklabel=\empty,
            ymajorgrids=true,
            grid style=dashed,
        ]
            \addplot[color=blue, mark=square,]
                coordinates {
                (0,0.838461518287659)(1,0.817307710647583)(2,0.805769205093384)(4,0.817307710647583)
                };
            \addplot[color=red, mark=triangle,]
                coordinates {
                (0,0.838461518287659)(1,0.819230794906616)(2,0.803846180438995)(4,0.771153867244721)
                };
            \addplot[color=green, mark=diamond,]
                coordinates {
                (0,0.838461518287659)(1,0.803846180438995)(2,0.807692289352417)(4,0.805769205093384)
                };
            \addplot[color=blue, mark=square*]
                coordinates {
                (0,0.627777755260468)(1,0.622222244739533)(2,0.577777802944183)(4,0.594444453716278)
                };
            \addplot[color=red, mark=triangle*,]
                coordinates {
                (0,0.627777755260468)(1,0.583333313465118)(2,0.561111092567444)(4,0.577777802944183)
                };
            \addplot[color=green, mark=diamond*,]
                coordinates {
                (0,0.627777755260468)(1,0.627777755260468)(2,0.566666662693024)(4,0.577777802944183)
                };
        \end{axis}
    \end{tikzpicture}
    \caption{Influence of the number of noisy sentences for FOL.}
    \label{fig:length_distraction}
\end{figure}



\subsection{Impact of Method Design}
\paragraph{\textbf{\emph{F4: \ac{CoT} prompting is most impactful when both noise and counterfactual perturbations are applied.}}}
The accuracies for the individual \acp{LLM} in Table~\ref{tab:distraction_k4_formalisation} show that the impact of \ac{CoT} is negligible for noise-only datasets (first four columns). Meanwhile, the benefit from \ac{CoT} is most pronounced in the datasets that combine noise and counterfactual perturbations.
The better-performing informal prompting strategy for a model remains stable for all types of distractions. Still, the decline in performance due to counterfactuals leads to a less consistent preference for a specific prompting style.

\paragraph{\textbf{\emph{F5: The best-performing grammar differs per model and is unstable across data versions.}}}

The evaluation of different logical forms for formal \ac{LLM}-based reasoning in Table~\ref{tab:distraction_k4_logical_form} shows the preference of some models for specific syntactic formats.
Llama 3.1 70B has a considerable improvement of $12\%$ with TPTP syntax on the original set, while Llama 3.1 8B benefits from the R-FOL syntax. However, all grammars show a declining accuracy trend and increased syntax errors for noise perturbations, where the best grammar loses its advantage over the rest. 
When comparing the grammars on the counterfactual partitions, we observe that TPTP is consistently more robust than the standard first-order logic grammar. Here, GPT 4o-mini shows a reduction from $O$ to $O_C$ of $20\%$ for FOL and only $12\%$ for the TPTP grammar. Since this does not correlate with fewer syntax errors, the formalisation in TPTP prevents semantical errors for counterfactual premises. 
A positive reading of these results, especially the minor differences between FOL and R-FOL, is that autoformalisation \acp{LLM} can adapt to the grammar syntax prescribed in the prompt without further loss in performance.

\begin{table}[!t]
\small
\setlength{\modelspacing}{2pt}
\setlength{\tabcolsep}{1.7pt} % Default value: 6pt
\setlength{\belowrulesep}{4pt}
\begin{threeparttable}
    \centering
    \begin{tabular}{cc l r rrr @{\quad} rrrr}
\toprule
\multirow{2}{*}{} & \multirow{2}{*}{} & Grammar & \multirow{2}{*}{O} & \multicolumn{3}{c}{Distraction} & \multicolumn{4}{c}{Counterfactual} \\
 & & Syntax & & E& L & T & $\text{O}_C$ & $\text{E}_C$& $\text{L}_C$ & $\text{T}_C$\\
\midrule
\multirow{6}{*}{\rotatebox{90}{Llama-3.1}} & \multirow{3}{*}{\rotatebox{90}{8B}} 
   & FOL & 0.77 & \textbf{0.71} & 0.61 & \textbf{0.53} & 0.58 & \textbf{0.55} & 0.52 & \textbf{0.56} \\
 & & R-FOL & \textbf{0.78} & 0.69 & \textbf{0.62} & \textbf{0.53} & 0.58 & \textbf{0.55} & \textbf{0.54} & 0.52 \\
 & & TPTP & 0.73 & 0.67 & 0.55 & 0.51 & \textbf{0.68} & 0.54 & 0.46 & 0.51 \\[\modelspacing]
\cmidrule{2-11}
 & \multirow{3}{*}{\rotatebox{90}{70B}} 
   & FOL & 0.76 & 0.73 & 0.71 & \textbf{0.72} & 0.67 & 0.57 & 0.63 & 0.56 \\
 & & R-FOL & 0.76 & 0.73 & 0.67 & 0.71 & 0.64 & 0.57 & 0.53 & 0.64 \\
 & & TPTP & \underline{\textbf{0.88}} & \underline{\textbf{0.84}} & \underline{\textbf{0.81}} & \textbf{0.72} & \underline{\textbf{0.81}} & \underline{\textbf{0.68}} & \underline{\textbf{0.67}} & \underline{\textbf{0.68}} \\[\modelspacing]
\midrule
\multirow{3}{*}{\rotatebox{90}{GPT}} & \multirow{3}{*}{\rotatebox{90}{4o-mini}} 
   & FOL & \textbf{0.84} & \textbf{0.82} & \textbf{0.72} & \underline{\textbf{0.78}} & 0.64 & \textbf{0.63} & \textbf{0.61} & 0.51 \\
 & & R-FOL & \textbf{0.84} & 0.77 & 0.70 & \underline{\textbf{0.78}} & \textbf{0.72} & 0.56 & 0.54 & \textbf{0.63} \\
 & & TPTP & 0.83 & \textbf{0.82} & 0.71 & 0.71 & 0.69 & \textbf{0.63} & 0.57 & 0.57 \\
\bottomrule
\end{tabular}
\caption{Accuracies of different formalisation grammars for autoformalisation.}
\label{tab:distraction_k4_logical_form}
\end{threeparttable}
\end{table} 

\paragraph{\textbf{\emph{F6: Feedback does not help \acp{LLM} self-correct to mitigate robustness issues.}}}
\autoref{tab:distraction_k4_feedback} shows the results with different error recovery mechanisms. The results indicate that no feedback strategy emerges as a winner in the different datasets. 
All feedback variants reduce syntax errors for noise perturbations, but given the lack of a consistent increase in accuracy, the corrected formalisations are most likely to contain semantic errors still. 
The type of feedback message only has a minor influence on correcting syntax errors, whereas Llama 3.1 70b and GPT 4o-mini correct slightly more syntax errors with specific error messages. This finding aligns with \cite{huang2023large}, who also found that \acp{LLM} cannot consistently self-correct their reasoning after receiving relevant feedback.

\begin{table}[!ht]
\small
\setlength{\modelspacing}{2pt}
\setlength{\tabcolsep}{1.7pt} % Default value: 6pt
\setlength{\belowrulesep}{4pt}
\begin{threeparttable}
    \centering
    \begin{tabular}{cc l r rrr @{\quad} rrrr}
\toprule
\multirow{2}{*}{} & \multirow{2}{*}{} & \multirow{2}{*}{Feedback} & \multirow{2}{*}{O} & \multicolumn{3}{c}{Distraction} & \multicolumn{4}{c}{Counterfactual} \\
 & & & & E& L & T & $\text{O}_C$ & $\text{E}_C$& $\text{L}_C$ & $\text{T}_C$\\
\midrule
\multirow{8}{*}{\rotatebox{90}{Llama-3.1}} & \multirow{4}{*}{\rotatebox{90}{8B}} 
   & No recovery & 0.77 & \textbf{0.72} & 0.62 & 0.53 & 0.59 & 0.58 & 0.56 & \textbf{0.56} \\
 & & Error type & \textbf{0.79} & 0.71 & 0.63 & \textbf{0.56} & \textbf{0.66} & 0.54 & 0.52 & 0.51 \\
 & & Error message & 0.78 & 0.71 & \textbf{0.67} & 0.55 & 0.59 & 0.53 & \underline{\textbf{0.64}} & 0.49 \\
 & & Warning & 0.74 & 0.66 & 0.58 & 0.55 & 0.55 & \textbf{0.60} & 0.49 & 0.49 \\[\modelspacing]
\cmidrule{2-11}
 & \multirow{4}{*}{\rotatebox{90}{70B}} 
   & No recovery & \textbf{0.77} & \textbf{0.72} & \textbf{0.73} & 0.71 & \textbf{0.64} & 0.59 & \textbf{0.61} & 0.56 \\
 & & Error type & 0.72 & 0.70 & 0.72 & \textbf{0.73} & 0.62 & 0.56 & 0.60 & 0.58 \\
 & & Error message & 0.71 & 0.70 & \textbf{0.73} & 0.71 & \textbf{0.64} & 0.59 & 0.54 & \underline{\textbf{0.64}} \\
 & & Warning & 0.69 & \textbf{0.72} & 0.72 & 0.72 & 0.62 & \underline{\textbf{0.65}} & \textbf{0.61} & 0.63 \\[\modelspacing]
\midrule
\multirow{4}{*}{\rotatebox{90}{GPT}} & \multirow{4}{*}{\rotatebox{90}{4o-mini}} 
   & No recovery & \underline{\textbf{0.84}} & \underline{\textbf{0.82}} & 0.73 & 0.79 & 0.64 & \textbf{0.62} & 0.56 & \textbf{0.56} \\
 & & Error type & 0.83 & 0.79 & 0.74 & 0.76 & 0.67 & 0.57 & 0.56 & \textbf{0.56} \\
 & & Error message & \underline{\textbf{0.84}} & 0.78 & \underline{\textbf{0.77}} & \underline{\textbf{0.80}} & 0.62 & 0.59 & 0.56 & \textbf{0.56} \\
 & & Warning & \underline{\textbf{0.84}} & 0.75 & 0.73 & 0.76 & \underline{\textbf{0.70}} & 0.61 & \textbf{0.61} & 0.55 \\
 \bottomrule
\end{tabular}
\caption{Accuracies of error recovery strategies.}
\label{tab:distraction_k4_feedback}
\end{threeparttable}
\end{table} 

\subsection{Error Analysis}
\label{subsec:errors}
\paragraph{\textbf{\emph{F7: Autoformalisation increases syntax errors for noise perturbations.}}}
The low performance for noise perturbations correlates with more syntax errors for all models and distraction categories (cf. execution rates in Table~\ref{tab:appendix_k4_formalisation_exec}). The three worst-performing models (both Mistral models, Gemma-2 9b) generate, at best, for $37\%$  and, at worst, for only $4\%$ of the samples, a valid logical form.
Gemma-2 9b and Llama3.1 8b produce more syntax errors than the larger counterparts, suggesting that larger models are more robust towards noise perturbations. 
The accuracy of syntactically valid samples is higher than the informal reasoning methods for most distractions (Table~\ref{tab:appendix_k4_formalisation_vacc}), motivating informal reasoning as a backup strategy for formal reasoning. The error message feedback reveals two common syntax errors: 1) errors by models with an initial low execution rate exhibit issues with the template structure, including using incorrect keywords or adding conversational phrases;
2) perturbation-related errors, the most common of which is using undefined truth constants as part of tautological distractions. 

\paragraph{\textbf{\emph{F8: Autoformalisation increases semantic errors for counterfactuals.}}}
Unlike the introduced noise, counterfactual perturbations do not lead to more syntax errors. The execution rate in Table~\ref{tab:appendix_k4_formalisation_exec} is stable or improves for counterfactuals. However, we see a drop in accuracy for the counterfactual column $\text{O}_C$ in Table~\ref{tab:distraction_k4_formalisation} and can conclude that the number of logical forms with semantic errors has to increase. This suggests that the introduced negation is not correctly formalised. Looking at the warnings generated by the feedback mechanism, for GPT 4o-mini, $161$ warning messages are generated on the unperturbed data. $54$ of these were fixed with a single iteration. Not considering predicates and individuals as part of the context is the most frequent warning across all models. 

\section{Discussion}
\RR{
Our study utilizes an intuitive flower-based visual design and evidence-based collaborative programming process analysis to provide instructors with a clear perspective for evaluating group and individual performance in collaborative programming. In this section, we discuss the lessons learned, the factors contributing to the research outcomes, and how these findings relate to existing works.

\subsection{Flower-Based Visual Design for Intuitive and Useful by Participants}
In large-scale learning analytics, intuitive visualization and interactive features prove to be valuable in assisting instructors with evaluations while reducing their workload~\cite{martinez2020data,fernandez2024data}.
Our study shows that the flower-based visual design effectively helps instructors summarize the performance of students and groups in collaborative programming.
Participants using \textit{CPVis} typically report starting by observing the flower visualization to gain an overview of the group's overall performance and the engagement levels of individual members during collaboration. Our design enables them to make quick assessment judgments and uncover valuable educational insights. 
For instance, students playing the Driver role often exhibit higher engagement levels.
\RA{Previous works use dynamic natural metaphors~\cite{tausch2014groupgarden,tausch2016comparison}, such as blooming flowers, falling leaves, and weather changes, to represent the quality and state of group discussions. However, these metaphors primarily convey overall trends or atmospheres rather than offering a precise and structured representation of multidimensional data, making it difficult for users to extract specific and accurate information efficiently. Moreover, the strong symbolic and emotional nature of their metaphors often leads to subjective interpretations.}
The effectiveness of our design lies in its ability to translate multiple dimensions of process-based learning analytics into visual elements such as colored petals and flower stamens, enabling instructors to quickly interpret multidimensional data and assess both group and individual performance during collaboration.
Furthermore, the flower-based visualization supports hierarchical analysis at both the group and individual levels, allowing instructors to efficiently analyze and compare the performance of multiple groups and students on a large scale.

\subsection{\textit{CPVis} Enhanced Instructors' Confidence in Evaluating Groups and Students}
The study demonstrates that \textit{CPVis} enhances participants' confidence in evaluation outcomes and improves the accuracy of their assessments. In Baseline System 1, participants report that accessing data requires significant time, and evaluating a specific group's performance often necessitates finding similar groups for a relatively fair comparison. 
Such a process demands additional time, causing participants to lose patience and avoid thoroughly examining all the details.
In baseline system 2, participants have to manually browse and process large amounts of student behavior and interaction data, which significantly increases cognitive load and reduces efficiency as they rely on memory to evaluate the performance of different groups.
In comparison, \textit{CPVis} offers significant convenience to participants \RA{by visualizing multidimensional learning analytics data}, allowing them to effortlessly access key information required for evaluations and compare similar groups. By providing both an overall view of multiple groups and detailed comparisons into individual groups, \textit{CPVis} substantially boosts participants' confidence in their evaluation outcomes, as demonstrated in the ratings. 
%This finding aligns with previous research results~\cite{sato2023groupnamics}.
\RA{Clear and intuitive visual analytics systems contribute to improved confidence and efficiency among participants. For instance, Groupnamics helps participants identify groups requiring intervention by visualizing each group's recent vocal activities and discussion statuses in a one-page view, thereby boosting their confidence in decision-making~\cite{sato2023groupnamics}.}
While it is ideal for \textit{CPVis} to support comparisons across an unlimited number of groups, practical limitations related to cognitive load and visual design make this challenging. Future efforts focus on optimizing the evaluation process through visual design, striking a balance between cognitive load and evaluation efficiency, thereby providing effective support for teaching.



\subsection{Theory-driven and LLM-powered Automation Evaluation for Quantifying Collaborative Learning}
Our study utilizes data collection, analysis, and visualization techniques to extract key insights from students' collaborative behaviors and outcomes, providing a deeper understanding of the learning process in collaborative programming. We focus on quantifying complex collaborative learning processes by leveraging LLMs and theoretical frameworks, introducing innovative methods to evaluate collaboration efficiency. 
While collaborative problem-solving is clearly defined in prior research~\cite{rosen2020towards}, achieving a quantitative balance between task performance and team effectiveness remains a significant challenge. To address this, we employ the coefficient of variation as a balancing metric and validate its efficacy using real-world datasets.
By integrating LLMs, \textit{CPVis} automates the annotation of collaborative programming performance, significantly reducing the workload associated with manually labeling large-scale classroom data and offering a novel perspective for automated learning analytics. 
Combining theory-driven metrics and LLM-powered automation provides instructors with robust, multidimensional evidence, enabling them to process and compare extensive student data systematically. 
This empowers instructors to effectively evaluate group and individual behaviors in collaborative programming, identify collaboration patterns, and support evidence-based decision-making. Previous research demonstrates that data-driven analysis helps educational decision-makers~\cite{hou2024codetailor}, such as instructors, uncover hidden learning patterns and deliver personalized guidance. Building on this foundation, \textit{CPVis} further enhances the potential for personalized feedback, enabling instructors to provide precise, data-driven guidance to students.


}


\section{Limitations and Future Work}

\RR{
In this section, we discuss the limitations of the current study and potential future work.
\subsection{Limitation}
Our study has three main limitations.
First, our current analysis is limited to data from a single real-world classroom's collaborative programming discussions, restricting the generalizability of our findings to other contexts. Similarly, our evaluation of \textit{CPVis} relies on a sampled dataset, limiting the study's scope. We hypothesize that participants working with smaller datasets and visualized learning analytics experience reduced cognitive load and find it easier to identify collaboration patterns due to fewer visual elements to process. However, in large-scale collaborative programming classrooms, instructors face the challenge of evaluating more groups and students, which may increase memory load and visual complexity.
Second, the data collected in our study are obtained from real classroom environments, maintaining ecological validity by capturing natural behaviors such as group silence or requests for instructor assistance. However, due to the limitations of non-intrusive equipment, our data lack details such as facial expressions and non-verbal cues. While participants report the comprehensiveness and richness of the learning analytics in the experiment, the absence of these data poses challenges for deeper analysis of emotional expressions and social engagement during collaborative programming. This limitation hinders the provision of a more holistic learning analysis for evaluation purposes.
Additionally, the recorded data are independent and exclude audio information, making it difficult to align screen interactions with dialogue streams. This limitation constrains the exploration of the relationship between collaborative behavior patterns and collaborative problem-solving processes.
Finally, in large-scale collaborative programming classrooms, generating analytics using LLMs requires significant computational time and cost. While feasible for institutions with robust computational resources, this remains a limitation for deploying such tools in real teaching scenarios. Furthermore, in real classrooms, noise from multiple group discussions introduces significant data noise, complicating the automation of learning analytics generation and limiting the accuracy of evaluations for groups and individual students.


\subsection{Future Work}
Without well-structured visualizations, simply presenting multiple data streams poses significant challenges for instructors attempting to interpret these large-scale datasets~\cite{fernandez2024data}.
In this study, we explore the integration and analysis of multimodal data. However, \textit{CPVis} has the potential to further enhance the visualization and perception of multimodal data, enabling instructors to evaluate group and student performance with greater accuracy and reduced cognitive load~\cite{martinez2020data}. 
Our target audience consists of instructors teaching large introductory collaborative programming courses, who require more efficient and intuitive visualizations to understand student performance during collaboration.  
While our use of static 2D visualizations, such as high-dimensional flower glyphs, has been highly regarded by participants for boosting confidence and helping instructors quickly identify key features, we believe there is room for improvement in organizing visualization formats to enhance information transmission efficiency and the users cognitive experience.
For instance, incorporating narrative visualizations further streamlines the process by allowing instructors to generate composite evaluations based on their weighting of different collaboration performance dimensions~\cite{gratzl2013lineup}. 
Narrative visualizations enable instructors to delve into data details, organize learning analytics results along logical paths such as timelines, causality, or categories, and highlight key information~\cite{chen2019designing}. 
This approach mitigates visual overload caused by excessive data, significantly reduces the time and cognitive effort required for evaluation, and ultimately supports instructors in making better decisions and assessments.

\textit{CPVis} requires instructors to spend additional time after class to evaluate collaborative performance. In our study, most participants indicate during follow-up interviews that the extra time spent on evaluating students' collaborative performance is highly valuable for producing comprehensive assessments. They note that providing immediate evaluations during the collaboration process is unrealistic, as final assessments typically need a holistic consideration of task completion and group dynamics after class. However, there is a significant demand for real-time analysis tools to deliver timely, personalized feedback to students and offer appropriate instructional scaffolding during the collaborative process~\cite{tang2024sphere}.
Instructors frequently find themselves overwhelmed by the immediate needs of some students~\cite{yang2023pair}, unintentionally neglecting others. To address this issue, future work could explore the integration of LLMs to enable real-time monitoring and analysis of students' behavioral data—such as code submissions, error logs, and engagement levels. LLMs could automatically detect learning bottlenecks or collaboration issues, providing instant feedback on common problems to students. This would effectively reduce instructors' workload, allowing them to focus on complex or critical issues, and simplify classroom management tasks.
For instance, LLMs could summarize patterns in students' code submissions and generate a ``hotspot report'' identifying recurring issues across the class. They could also provide real-time collaborative performance analytics for different groups, enabling instructors to quickly gain a comprehensive understanding of overall class dynamics. Additionally, LLMs could assist in role allocation within groups, suggest strategies to improve team interactions, and identify potential conflicts or disengagement within collaborative teams.
LLM-powered tools automate evaluations and enable personalized feedback, bridging post-class assessments with in-class scaffolding to enhance teaching and learning in collaborative programming.}
\section*{Conclusion}
This paper aims to enhance our understanding of the computational complexity of computing various Shapley value variants. We found that for various ML models --- including decision trees, regression tree ensembles, weighted automata, and linear regression --- both local and global interventional and baseline SHAP can be computed in polynomial time under HMM modeled distributions. This extends popular algorithms, such as TreeSHAP, beyond their empirical distributional scope. We also establish strict complexity gaps between the various SHAP variants (baseline, interventional, and conditional) and prove the intractability of computing SHAP for tree ensembles and neural networks in simplified scenarios. Overall, we present SHAP as a versatile framework whose complexity depends on four key factors: \begin{inparaenum}[(i)] \item model type, \item SHAP variant, \item distribution modeling approach, \item and local vs. global explanations\end{inparaenum}. We believe this perspective provides deeper insight into the computational complexity of SHAP, paving the way for future work.




%We believe that our framework provides a more intricate understanding of SHAP computation complexity across different models, distributions, and variants, paving the way for further research.

Our work opens promising directions for future research. First, expanding our computational analysis to other SHAP-related metrics, such as asymmetric SHAP~\citep{frye20} and SAGE~\citep{covert2020understanding}, would be valuable. Additionally, we aim to explore more expressive distribution classes and relaxed assumptions beyond those in Section \ref{sec:tractable} while maintaining tractable SHAP computation. Finally, when exact computation is intractable (Section \ref{sec:intractable}), investigating the approximability of SHAP metrics through approximation and parameterized complexity theory~\citep{downey2012parameterized} is an important direction.

%Our work opens several promising avenues for future research on the computational properties of explainable AI methods, with a particular focus on SHAP. First, it would be interesting to broaden the computational analysis conducted in this work to include other popular SHAP-related metrics in the literature, such as asymmetric SHAP \cite{frye20} and SAGE \cite{covert2020understanding}. Also, in the future, we aim to explore more expressive distribution classes and relaxed distributional assumptions—extending beyond those examined in Section \ref{sec:tractable} —that still yield tractable SHAP computation. Finally, when exact computation proves intractable (Section \ref{sec:intractable}), it is worthwhile to theoretically investigate the question of the approximability of computing the SHAP metrics across various configurations, through the lens of approximation and parametrized complexity theory \cite{arora2009computational}.

%This paper aims to deepen our understanding of the computational complexity involved in obtaining different Shapley value variants. We found that for a variety of ML models, including decision trees, tree ensembles for regression, weighted automata, and linear regression models — computing both local and global interventional and baseline SHAP can be done in polynomial time when distributions are modeled by HMMs. This extends the distributional scope of popular algorithms like TreeSHAP, which is limited to empirical distributions. Additionally, we demonstrate a strict complexity gap between SHAP variants, showing that interventional and baseline SHAP can be strictly easier to compute than conditional SHAP. Despite these positive results, we uncovered intractability for various SHAP variants in neural networks and tree ensembles. Finally, we provided generalized complexity relations across SHAP variants. We believe that our framework offers a deeper understanding of the complexity involved in computing SHAP across various variants, models, distributions, as well as in both local and global computations, laying the groundwork for future research.

%%
%% The acknowledgments section is defined using the "acks" environment
%% (and NOT an unnumbered section). This ensures the proper
%% identification of the section in the article metadata, and the
%% consistent spelling of the heading.
\begin{acks}
Meng Xia is the corresponding author.
The work was supported by the National Natural Science Foundation of China, (62422607, 62372411, 62036009) and the Zhejiang Provincial Natural Science Foundation of China.

\end{acks}
%%
%% This is file `sample-sigconf-authordraft.tex',
%% generated with the docstrip utility.
%%
%% The original source files were:
%%
%% samples.dtx  (with options: `all,proceedings,bibtex,authordraft')
%% 
%% IMPORTANT NOTICE:
%% 
%% For the copyright see the source file.
%% 
%% Any modified versions of this file must be renamed
%% with new filenames distinct from sample-sigconf-authordraft.tex.
%% 
%% For distribution of the original source see the terms
%% for copying and modification in the file samples.dtx.
%% 
%% This generated file may be distributed as long as the
%% original source files, as listed above, are part of the
%% same distribution. (The sources need not necessarily be
%% in the same archive or directory.)
%%
%%
%% Commands for TeXCount
%TC:macro \cite [option:text,text]
%TC:macro \citep [option:text,text]
%TC:macro \citet [option:text,text]
%TC:envir table 0 1
%TC:envir table* 0 1
%TC:envir tabular [ignore] word
%TC:envir displaymath 0 word
%TC:envir math 0 word
%TC:envir comment 0 0
%%
%%
%% The first command in your LaTeX source must be the \documentclass
%% command.
%%
%% For submission and review of your manuscript please change the
%% command to \documentclass[manuscript, screen, review]{acmart}.
%%
%% When submitting camera ready or to TAPS, please change the command
%% to \documentclass[sigconf]{acmart} or whichever template is required
%% for your publication.
%%
%%
\documentclass[sigconf, nonacm]{acmart}
%\documentclass[sigconf]{acmart}
%\documentclass[sigconf,authordraft]{acmart}
%\documentclass[anonymous,manuscript,review]{acmart}

%%
%% \BibTeX command to typeset BibTeX logo in the docs
\AtBeginDocument{%
  \providecommand\BibTeX{{%
    Bib\TeX}}}

%% Rights management information.  This information is sent to you
%% when you complete the rights form.  These commands have SAMPLE
%% values in them; it is your responsibility as an author to replace
%% the commands and values with those provided to you when you
%% complete the rights form.
%\setcopyright{acmlicensed}
\copyrightyear{2025}
\acmYear{2025}
%\setcopyright{cc}
\setcctype{by}
\acmConference[CHI '25]{CHI Conference on Human Factors in Computing Systems}{April 26-May 1, 2025}{Yokohama, Japan}
\acmBooktitle{CHI Conference on Human Factors in Computing Systems (CHI '25), April 26-May 1, 2025, Yokohama, Japan}
\acmDOI{10.1145/3706598.3713353}
\acmISBN{979-8-4007-1394-1/25/04}


%%
%% Submission ID.
%% Use this when submitting an article to a sponsored event. You'll
%% receive a unique submission ID from the organizers
%% of the event, and this ID should be used as the parameter to this command.
%%\acmSubmissionID{123-A56-BU3}

%%
%% For managing citations, it is recommended to use bibliography
%% files in BibTeX format.
%%
%% You can then either use BibTeX with the ACM-Reference-Format style,
%% or BibLaTeX with the acmnumeric or acmauthoryear sytles, that include
%% support for advanced citation of software artefact from the
%% biblatex-software package, also separately available on CTAN.
%%
%% Look at the sample-*-biblatex.tex files for templates showcasing
%% the biblatex styles.
%%

%%
%% The majority of ACM publications use numbered citations and
%% references.  The command \citestyle{authoryear} switches to the
%% "author year" style.
%%
%% If you are preparing content for an event
%% sponsored by ACM SIGGRAPH, you must use the "author year" style of
%% citations and references.
%% Uncommenting
%% the next command will enable that style.
%%\citestyle{acmauthoryear}
\usepackage{booktabs}
\usepackage{graphicx}
\usepackage{listings}
\usepackage{xcolor} 
\usepackage{caption}
%\usepackage{paralist}
\usepackage{makecell}
\usepackage{url}
\usepackage{longtable}
\usepackage{color}
\usepackage{tikz}
%\usepackage[utf8]{enc}
\usepackage[T1]{fontenc}
\usepackage{lipsum}
\usepackage{stfloats}
\usepackage{tabu} 
\usepackage{tabularx}
\usepackage{multirow}
\usepackage{booktabs}
\usepackage{graphicx}
\usepackage{wrapfig}
\usepackage{hyperref}   
\usepackage{cleveref}
%\usepackage[most]{tcolorbox} 
\usepackage{xcolor}
\usepackage{float}
\usepackage{listings} 
%\usepackage{tcolorbox}




\definecolor{PurpleColor}{RGB}{0,0,0}
\newcommand{\RR}[1]{{\color{PurpleColor}#1}}

\definecolor{PinkColor}{RGB}{0, 0, 0}
\newcommand{\RA}[1]{{\color{PinkColor}#1}}


\definecolor{codegreen}{rgb}{0,0.6,0}
\definecolor{codegray}{rgb}{0.5,0.5,0.5}
\definecolor{codepurple}{rgb}{0.58,0,0.82}
\definecolor{backcolour}{rgb}{0.95,0.95,0.92}

\lstdefinestyle{mystyle}{
  backgroundcolor=\color{backcolour}, commentstyle=\color{codegreen},
  keywordstyle=\color{magenta},
  numberstyle=\tiny\color{codegray},
  stringstyle=\color{codepurple},
  basicstyle=\ttfamily\footnotesize,
  breakatwhitespace=false,         
  breaklines=true,                 
  captionpos=b,                    
  keepspaces=true,                 
  numbers=left,                    
  numbersep=5pt,                  
  showspaces=false,                
  showstringspaces=false,
  showtabs=false,                  
  tabsize=2
}
\lstset{style=mystyle}


%\definecolor{PurpleColor}{RGB}{0,0,0}

%%
%% end of the preamble, start of the body of the document source.
\begin{document}

%%
%% The "title" command has an optional parameter,
%% allowing the author to define a "short title" to be used in page headers.
\title{CPVis: Evidence-based Multimodal Learning Analytics for Evaluation in Collaborative Programming}

%%
%% The "author" command and its associated commands are used to define
%% the authors and their affiliations.
%% Of note is the shared affiliation of the first two authors, and the
%% "authornote" and "authornotemark" commands
%% used to denote shared contribution to the research.
\author{Gefei Zhang}
\affiliation{%
  \institution{Zhejiang University of Technology}
  \city{Hangzhou}
  \state{Zhejiang}
  \country{China}}
\email{gefei@zjut.edu.cn}
\orcid{1234-5678-9012}

\author{Shenming Ji}
\affiliation{%
  \institution{Xi'an Jiaotong-Liverpool University}
  \city{Suzhou}
  \state{Jiangsu}
  \country{China}}
\email{shenming.ji21@student.xjtlu.edu.cn}

\author{Yicao Li}
\affiliation{%
  \institution{Zhejiang University of Technology}
  \city{Hangzhou}
  \state{Zhejiang}
  \country{China}}
\email{yicaoli47@gmail.com}

\author{Jingwei Tang}
\affiliation{%
  \institution{Zhejiang University of Technology}
  \city{Hangzhou}
  \state{Zhejiang}
  \country{China}}
\email{jwtang@zjut.edu.cn}

\author{Jihong Ding}
\affiliation{%
 \institution{Hannan University}
 \city{Haikou}
 \state{Hainan}
 \country{China}}
\email{jhding@hainanu.edu.cn}

\author{Meng Xia}
\affiliation{%
  \institution{Texas A\&M University}
  \city{College Station}
  \state{Texas}
  \country{USA}}
\email{mengxia@tamu.edu}

\author{Guodao Sun}
\affiliation{%
  \institution{Zhejiang University of Technology}
  \city{Hangzhou}
  \state{Zhejiang}
  \country{China}}
\email{guodao@zjut.edu.cn}

\author{Ronghua Liang}
\affiliation{%
  \institution{Zhejiang University of Technology}
  \city{Hangzhou}
  \state{Zhejiang}
  \country{China}}
\email{rhliang@zjut.edu.cn}



%%
%% By default, the full list of authors will be used in the page
%% headers. Often, this list is too long, and will overlap
%% other information printed in the page headers. This command allows
%% the author to define a more concise list
%% of authors' names for this purpose.
\renewcommand{\shortauthors}{Zhang et al.}

%%
%% The abstract is a short summary of the work to be presented in the
%% article.
\begin{abstract}
As programming education becomes more widespread, many college students from non-computer science backgrounds begin learning programming. Collaborative programming emerges as an effective method for instructors to support novice students in developing coding and teamwork abilities. However, due to limited class time and attention, instructors face challenges in monitoring and evaluating the progress and performance of groups or individuals. To address this issue, we collect multimodal data from real-world settings and develop \textit{CPVis}, \RR{an interactive visual analytics system designed to assess student collaboration dynamically.} Specifically, \textit{CPVis} enables instructors to evaluate both group and individual performance efficiently. \textit{CPVis} employs a novel flower-based visual encoding to represent performance and provides time-based views to capture the evolution of collaborative behaviors. A within-subject experiment (N=22), \RR{comparing \textit{CPVis} with two baseline systems, reveals that users gain more insights, find the visualization more intuitive, and report increased confidence in their assessments of collaboration.}
\end{abstract}

%%
%% The code below is generated by the tool at http://dl.acm.org/ccs.cfm.
%% Please copy and paste the code instead of the example below.
%%

\begin{CCSXML}
<ccs2012>
   <concept>
       <concept_id>10010405.10010489.10010492</concept_id>
       <concept_desc>Applied computing~Collaborative learning</concept_desc>
       <concept_significance>500</concept_significance>
       </concept>
 </ccs2012>
\end{CCSXML}

\begin{CCSXML}
<ccs2012>
   <concept>
       <concept_id>10003120.10003145.10003151</concept_id>
       <concept_desc>Human-centered computing~Visualization systems and tools</concept_desc>
       <concept_significance>500</concept_significance>
       </concept>
 </ccs2012>
\end{CCSXML}

\ccsdesc[500]{Human-centered computing~Visualization systems and tools}

\ccsdesc[500]{Applied computing~Collaborative learning}


\begin{comment}



\end{comment}
%%
%% Keywords. The author(s) should pick words that accurately describe
%% the work being presented. Separate the keywords with commas.
\keywords{Group visualization, education visualization, collaborative programming}
%% A "teaser" image appears between the author and affiliation
%% information and the body of the document, and typically spans the
%% page.








% \received{20 February 2007}
% \received[revised]{12 March 2009}
% \received[accepted]{5 June 2009}

%%
%% This command processes the author and affiliation and title
%% information and builds the first part of the formatted document.
\maketitle
%\newtcbox{\inlinecode}{on line, colback=gray!10, colframe=gray!10, boxrule=0.1mm, 
%	rounded corners, fontupper=\bfseries, left=0.1mm, right=0.1mm, top=0.1mm, bottom=0.1mm}

\input{introduction}
\input{related}
\input{formative}
\input{data}
\input{vis}
% \input{texfiles/case}
\input{evaluation}
\input{dis}
\input{conclusion}

%%
%% The acknowledgments section is defined using the "acks" environment
%% (and NOT an unnumbered section). This ensures the proper
%% identification of the section in the article metadata, and the
%% consistent spelling of the heading.
\begin{acks}
Meng Xia is the corresponding author.
The work was supported by the National Natural Science Foundation of China, (62422607, 62372411, 62036009) and the Zhejiang Provincial Natural Science Foundation of China.

\end{acks}
\input{cpvis.bbl}
%%
%% The next two lines define the bibliography style to be used, and
%% the bibliography file.
%\bibliographystyle{ACM-Reference-Format}
%\bibliography{cpvisref}


%%
%% If your work has an appendix, this is the place to put it.
\appendix
\RR{
\newpage
\section{DETAILS OF FORMATIVE STUDY}
\subsection{Background Information of Participants}
The table~\ref{tab:info} presents information about the participants in our formative study.
}


%Lorem ipsum dolor sit amet, consectetur adipiscing elit. Morbi

\RR{
\subsection{Findings of the Semi-structured Interviews}

\begin{table*}[hb]
\renewcommand{\arraystretch}{1.5}
\centering
\begin{tabular}{m{1.5cm} m{1cm} m{1cm} m{1.5cm} m{3.5cm} m{2.5cm} m{3.5cm}}
\hline
Participants & Gender & Age & Teaching experience & Students taught & Whether the class is large-scale & Whether the students know each other before \\ \hline
E1 & Female & 45 & 15 & University Non-Computer Science Majors & Yes & No \\
E2 & Female & 30 & 5 & University Non-Computer Science Majors & Yes & No \\
TA1 & Male & 24 & 1 & University Non-Computer Science Majors & Yes & No \\
TA2 & Female & 24 & 1 & University Non-Computer Science Majors & Yes & No \\
T1 & Male & 31 & 3 & University Non-Computer Science Majors & Yes & No \\
T2 & Male & 29 & 5 & University Computer Science Majors & Yes & Yes \\
T3 & Male & 36 & 8 & University Computer Science Majors & No & Yes \\
T4 & Female & 26 & 3 & Vocational High School Non-Computer Majors & Yes & Yes \\
T5 & Female & 26 & 3 & Elementary School Students & No & Yes \\
T6 & Male & 35 & 7 & University Non-Computer Science Majors & Yes & No \\ \hline
\end{tabular}
\caption{The detailed background statistical information of the participants in the Formative Study.}
\label{tab:info}
\end{table*}

\textbf{\textit{Motivations:}}
Participants unanimously agreed that collaborative programming is an important teaching strategy for programming education. However, different teachers had varying focuses and practices regarding collaborative programming. 
Teachers specializing in computer science tended to have students complete tasks independently in foundational programming courses (e.g., Python, Java) while implementing collaborative programming for more complex projects. 
For example, \textit{T3 stated, ``I consider collaborative learning only when the programming project requires different students' creative input, innovation, exploration, and discovery of new content, or when the workload is too large for one person to complete.''}
In contrast, teachers of non-computer science courses held different views. 
E2 pointed out that for beginners who are new to programming languages, even a fundamental programming problem can be daunting.
Collaborative programming can significantly alleviate this issue and improve students' efficiency. 
Group members can work together to complete a programming task, which motivates students, allows them to learn from each other, and reduces the teaching burden (T4).
Additionally, in collaborative programming classes, teachers focus on students' cognitive skills (e.g., knowledge acquisition) and non-cognitive skills (e.g., computational thinking, problem-solving abilities). Most participants believed that these two aspects do not conflict and are part of the overall teaching goals. 
\textit{T5 stated that her course design primarily focuses on developing non-cognitive skills, particularly enhancing students' abilities during the collaboration process. }
T4 believed that cognitive skills can be supplemented with extra time or individual tutoring, but problem-solving and collaboration skills are challenging to develop outside of class. 
Furthermore, E1, T4, and T3 mentioned that quantifying the improvement of students' non-cognitive skills is very challenging.
On one hand, teachers and assistants are busy answering students' questions, making it challenging to observe students' performance. 
On the other hand, non-cognitive skills are not as quickly assessed through tests as knowledge-based content.

Despite the unanimous recognition of the importance of collaborative programming, participants shared several challenges they faced in practice, summarized as follows:

\textit{\textbf{Group Formation:}}
For group size, most participants preferred group sizes of 2-3 students, noting that larger groups (up to 5 members) could decrease the quality of collaboration and make it difficult for students to find their roles. T4 favored pair programming but often had to form groups of three due to large class sizes. T1 and T3 typically recommended groups of 4-5 students, given the complexity and workload of their programming projects, which smaller groups might struggle to complete.
For the group formation method, participants had various approaches to forming groups. T2, T3, T4, and T5, whose students come from fixed classes, generally allowed students to create their own groups. In contrast, T6, T1, E1, and E2 taught general education courses open to all students across the university. These students came from different majors and did not know each other, making self-selection difficult. E2 conducted preliminary assessments of students and formed balanced groups based on their prior knowledge. However, determining how to create groups remains a significant challenge, as different criteria can lead to varying outcomes. Arbitrarily formed groups might result in imbalanced teams, where students' abilities are insufficient to complete the tasks.

\textit{\textbf{Evaluating the Learning Objectives:}}
Evaluating whether the teacher has achieved the learning objectives is an essential criterion for assessing the quality of collaborative programming courses. All participants stated that they generally meet the learning objectives and sometimes even exceed expectations. T4 mentioned that at the start of the collaborative course, she assigned a group leader to each group to take on a leadership role. While she initially spent a significant amount of time guiding the group leaders and facilitating student collaboration, by the second half of the semester, the group leaders were typically able to function as ``little teachers,'' helping group members complete tasks. This allowed the teacher to free up time and energy for other responsibilities.



%Etiam commodo feugiat nisl pulvinar pellentesque. Etiam auctor sodales
\subsection{Results of Rating and Ranking in Formative Study}
Table~\ref{table:feature} presents the metrics participants rated and ranked for the collaborative programming visual analytics system, as discussed in Section 3.2.}
\begin{table}[htbp]
\renewcommand{\arraystretch}{1.5}
\begin{tabular}{p{8cm}}
\hline
\textbf{Features}  \\ \hline
F1: Change in student engagement  \\
F2: Change in student behavior \\
F3: Analysis of student emotional state \\
F4: Student background statistics \\
F5: Analysis of student programming habits \\
F6: Change in role distribution \\
F7: Role assignment suggestions within the group \\
F8: Group performance overview \\
F9: Comparison of group performances \\
F10: Analysis of similarity in group performance  \\
F11: Generation of personalized suggestions \\
F12: View of raw data \\
F13: Programming task evaluation \\
F14: Visualization of problem-solving path \\
F15: Real-time feedback feature \\
F16: Students performance overview \\ \hline
\end{tabular}
\caption{Features of Collaborative Programming Visualization Systems Extracted from Semi-structured Interviews.}
\label{table:feature}
\end{table}
\RR{The raw rating data is displayed in Table~\ref{table:rating}.
The results show that participants rated most of the features mentioned highly, except for F3, F4, F5, F7, and F15.}
\begin{table*}[htbp]
\renewcommand{\arraystretch}{1.5}
\centering
\small
\begin{tabular}{l c c c c c c c c c c c c}
\hline
\textbf{Feature} & \textbf{T6} & \textbf{T5} & \textbf{E1} & \textbf{E2} & \textbf{TA1} & \textbf{TA2} & \textbf{T1} & \textbf{T2} & \textbf{T3} & \textbf{T4} & \textbf{Mean} & \textbf{SD} \\ \hline
F1 & 5 & 6 & 5 & 6 & 7 & 6 & 7 & 5 & 6 & 7 & 6 & 0.82 \\
F2 & 7 & 6 & 5 & 5 & 4 & 7 & 6 & 6 & 6 & 6 & 5.8 & 0.92 \\
F3 & 3 & 4 & 3 & 4 & 4 & 5 & 6 & 4 & 3 & 4 & 4 & 0.94 \\
F4 & 2 & 4 & 2 & 2 & 2 & 3 & 3 & 4 & 6 & 5 & 3.3 & 1.42 \\
F5 & 4 & 3 & 6 & 5 & 4 & 3 & 4 & 4 & 3 & 5 & 4.1 & 0.99 \\
F6 & 4 & 6 & 7 & 3 & 7 & 5 & 4 & 6 & 6 & 7 & 5.5 & 1.43 \\
F7 & 2 & 4 & 3 & 4 & 2 & 4 & 4 & 3 & 4 & 3 & 3.3 & 0.82 \\
F8 & 7 & 6 & 7 & 7 & 7 & 7 & 7 & 7 & 7 & 7 & 6.9 & 0.32 \\
F9 & 6 & 6 & 5 & 7 & 5 & 4 & 7 & 6 & 5 & 7 & 5.8 & 1.03 \\
F10 & 7 & 6 & 5 & 6 & 5 & 5 & 5 & 5 & 6 & 7 & 5.7 & 0.82 \\
F11 & 7 & 6 & 5 & 5 & 5 & 5 & 6 & 7 & 7 & 7 & 6 & 0.94 \\
F12 & 3 & 5 & 5 & 5 & 5 & 5 & 6 & 7 & 4 & 3 & 4.8 & 1.23 \\
F13 & 7 & 6 & 7 & 7 & 7 & 7 & 7 & 7 & 6 & 6 & 6.7 & 0.48 \\
F14 & 6 & 7 & 6 & 6 & 6 & 5 & 4 & 5 & 6 & 6 & 5.7 & 0.82 \\
F15 & 2 & 3 & 4 & 3 & 2 & 5 & 6 & 7 & 7 & 4 & 4.3 & 1.89 \\
F16 & 7 & 7 & 6 & 6 & 5 & 7 & 7 & 5 & 6 & 6 & 6.2 & 0.79 \\ \hline
\end{tabular}
\caption{Mean and SD Results of the Ratings for F1-F16 by Ten Participants.}
\label{table:rating}
\end{table*}
\RR{The raw ranking data is displayed in Table~\ref{table:rank}.
Each feature's average ranking is listed (Fig.~\ref{fig:rank}), with lower numbers indicating that the evaluators more highly value the feature. For example, the average ranking of ``Students Performance Overview'' is 4.8, making it one of the most valued features. On the other hand, ``Role assignment suggestions within the group'' has an average ranking of 13.4, which is ranked lower by the evaluators.}



\begin{table*}[htbp]

\renewcommand{\arraystretch}{1.5}
\centering
\small
\begin{tabular}{l c c c c c c c c c c c c}
\hline
\textbf{Feature} & \textbf{T6} & \textbf{T5} & \textbf{E1} & \textbf{E2} & \textbf{TA1} & \textbf{TA2} & \textbf{T1} & \textbf{T2} & \textbf{T3} & \textbf{T4} & \textbf{Mean} & \textbf{SD} \\ \hline
F1 & 7 & 1 & 7 & 9 & 12 & 2 & 1 & 11 & 1 & 4 & 5.5 & 4.28 \\
F2 & 6 & 3 & 6 & 10 & 11 & 3 & 3 & 10 & 3 & 5 & 6 & 3.23 \\
F3 & 12 & 8 & 16 & 14 & 16 & 12 & 13 & 9 & 4 & 13 & 11.7 & 3.74 \\
F4 & 16 & 15 & 15 & 13 & 14 & 13 & 12 & 6 & 6 & 14 & 12.4 & 3.57 \\
F5 & 14 & 13 & 14 & 15 & 15 & 15 & 14 & 12 & 2 & 15 & 12.9 & 3.96 \\
F6 & 7 & 2 & 5 & 8 & 10 & 4 & 11 & 8 & 7 & 6 & 6.8 & 2.70 \\
F7 & 15 & 14 & 13 & 16 & 13 & 14 & 15 & 13 & 5 & 16 & 13.4 & 3.17 \\
F8 & 8 & 6 & 4 & 1 & 3 & 11 & 8 & 7 & 8 & 2 & 5.8 & 3.19 \\
F9 & 5 & 5 & 12 & 7 & 9 & 6 & 9 & 14 & 16 & 7 & 9 & 3.83 \\
F10 & 10 & 4 & 3 & 3 & 4 & 5 & 10 & 5 & 10 & 8 & 6.2 & 2.97 \\
F11 & 9 & 7 & 2 & 4 & 5 & 7 & 7 & 4 & 9 & 9 & 6.3 & 2.45 \\
F12 & 11 & 10 & 11 & 11 & 6 & 10 & 16 & 15 & 15 & 10 & 11.5 & 3.03 \\
F13 & 3 & 9 & 1 & 5 & 1 & 8 & 6 & 16 & 11 & 3 & 6.3 & 4.79 \\
F14 & 4 & 11 & 8 & 6 & 7 & 9 & 5 & 3 & 14 & 11 & 7.8 & 3.49 \\
F15 & 13 & 16 & 10 & 12 & 8 & 16 & 2 & 1 & 12 & 12 & 10.2 & 5.18 \\
F16 & 2 & 12 & 9 & 2 & 2 & 1 & 4 & 2 & 13 & 1 & 4.8 & 4.69 \\
\hline
\end{tabular}
\caption{Mean and SD Results of the Rankings for F1-F16 by Ten Participants.}
\label{table:rank}
\end{table*}

\RR{\textbf{Spearman Correlation Matrix:}
To assess the consistency between rankings, we used Spearman's Rank Correlation to calculate the correlation of rankings between each participant., The Spearman correlation matrix shows the consistency of the rankings between participants (Fig.~\ref{fig:myplot}). The correlation coefficient ranges from -1 to 1:
- A value closer to 1 indicates that the two evaluators' rankings are more consistent.
- A value closer to -1 suggests that the rankings are more inconsistent.
- A value near 0 implies no significant correlation between the two evaluators' rankings. For example, the correlation between participant 0 and participant 9 is 0.809, indicating
relatively consistent rankings. In contrast, the correlation between evaluator seven and evaluator 8 is - 0.150, suggesting their rankings differ considerably.}
\begin{figure}[htbp]
	\centering
	\includegraphics[width=0.8\linewidth]{rank.png}
	\caption{Ranking Results of F1-F16.}
        \Description{Ranking Results of F1-F16.}
	\label{fig:rank}
\end{figure}


\begin{figure}[htbp]
    \centering
    \includegraphics[width=0.8\linewidth]{myplot.png}
    \caption{Spearman's Rank Correlation Results Between Participants.}
    \Description{Spearman's Rank Correlation Results Between Participants.}
    \label{fig:myplot}
\end{figure}

\RR{\textbf{Friedman Test Results:}
We conducted a Friedman test to assess whether there are significant differences in rankings among different participants. The Friedman test compares the ranking data of multiple related samples to determine whether there are significant differences in rankings under different conditions. The test statistic is 60.796, and the p-value is 1.84e-07. This very small p-value (< 0.05) indicates that:

- There are significant differences in the rankings of different features. The evaluators' ratings are not random or similar but show clear preferences for specific features over others.

In conclusion, after comprehensively analyzing the results for each feature, our system focuses on implementing features \textbf{F1, F2, F6, F8-14, and F16}.
}
\newpage
\RR{
\section{PROMPTS}
Below are the prompts used to label data.
\subsection{Python Code Evaluation}
\textbf{Prompt:}

I would like you to play the role of a teacher who teaches a Python programming class, and you will be provided a question statement and a Python code, which is the student's answer to the question. 
Regarding the Python code, you need to accomplish two tasks. 
Here are the scoring criteria. Please mark each point according to the scoring criteria and explain the reason. Meanwhile, you should give your final score.
If the score of each aspect is not 5, please point out the demerits of the code. Also, note that you don't need to give the advised code. }


\begin{lstlisting}
criteria = {
    "Problem-solving Approach (5%)": {
        "Excellent (5)": "Shows an effective problem-solving approach, effectively addressing key challenges in the task",
        "Good (4)": "Shows a good problem-solving approach, with clear attempts to address challenges in the task",
        "Fair (3)": "Shows some effort in problem-solving, but lacks clarity or effectiveness in addressing challenges in the task",
        "Poor (2)": "Shows limited problem-solving efforts, with unclear or ineffective attempts to address challenges in the task",
        "Bad (1)": "Demonstrates no effective problem-solving approach, unable to address the task"
    },
    "Code Integrity (35%)": {
        "Excellent (5)": "The code is well-structured, organized, readable, and it effectively implements the desired functionality",
        "Good (4)": "The code structure and organization are sufficient, and it implements the function, though readability could be enhanced",
        "Fair (3)": "The code implements 80% of the function but lacks consistency or clarity",
        "Poor (2)": "The code implements 60% of the function but lacks consistency or clarity",
        "Bad (1)": "The code failed to implement 40% of the function and lacks consistency or clarity"
    },
    "Code Accuracy (35%)": {
        "Excellent (5)": "Code exhibits an excellent level of accuracy, producing correct results under various conditions and inputs",
        "Good (4)": "Code exhibits good accuracy, producing correct results under most conditions and inputs",
        "Fair (3)": "Code accuracy is acceptable, but there may be some occasional errors or inconsistencies in the results",
        "Poor (2)": "Code accuracy is poor, with frequent errors or inconsistencies in the results",
        "Bad (1)": "Code accuracy is severely lacking, with a high frequency of errors or inconsistencies in the results"
    },
    "Algorithm Innovation (25%)": {
        "Excellent (5)": "Shows innovative and creative approaches, showcasing originality and ingenuity in algorithm design and implementation",
        "Good (4)": "Shows basic innovation in algorithm design and implementation, but lacks originality or significant creativity",
        "Fair (3)": "Demonstrates limited innovation in algorithm design and implementation, with little originality or creativity demonstrated",
        "Poor (2)": "Shows little innovation in algorithm design and implementation, with no originality or creativity demonstrated",
        "Bad (1)": "Shows no innovation in algorithm design and implementation; only capable of imitating from simple examples"
    }
}
\end{lstlisting} 



\RR{
\textbf{Task:} Task 1, you need to extract the key ideas of the code. Give me a paragraph outlining the thought process of this code. Task 2, Grade the code according to the scoring criteria, and don't round the final score.

\textbf{Input Example:}
}
\begin{lstlisting} 
Question: Existing list a=[49, 38 , 65 , 97 , 76 , 13 , 27 , 55 , 4]
Requirement: Try to write a program in Python that sorts the data elements in a from smallest to largest and prints out the new sorted list a.
Answer:
a=[49,38,65,97,76,13,27,55,4]
a.sort()
print(a)       
\end{lstlisting} 
\RR{
\textbf{Output Example:}}
\begin{lstlisting}
{
    "Key ideas": "The provided code aims to sort a given list of integers in ascending order. The process follows these key steps:\n\t1. Define a list named `a` containing a series of integers.\n\t2. Call the `sort()` method on the list `a`, which sorts the list in place in ascending order.\n\t3. Print the sorted list to display the numbers from smallest to largest.\nThis approach efficiently utilizes Python's built-in sorting capabilities and achieves the task as specified in the question.",
    "Score": "4.55 / 5",
    "Details": [
        {
            "Problem-solving Approach (5%)": {
                "Score": "Excellent (5)",
                "Explanation": "The code presents a clear and effective approach to sorting the list using Python's built-in functionality, which is suitable for the problem at hand."
            },
            "Demerits": null
        },
        {
            "Code Integrity (35%)": {
                "Score": "Excellent (5)",
                "Explanation": "The code is well-structured, organized, and highly readable. It makes effective use of the `sort()` method while maintaining clarity. There are no unnecessary complexities in the implementation."
            },
            "Demerits": null
        },
        {
            "Code Accuracy (35%)": {
                "Score": "Excellent (5)",
                "Explanation": "The code accurately sorts the list `a` and produces the correct output without any errors. The use of the `sort()` method guarantees that the order will be correct."
            },
            "Demerits": null
        },
        {
            "Algorithm Innovation (25%)": {
                "Score": "Fair (3)",
                "Explanation": "While the implementation is effective and clear, it does not demonstrate any innovative or creative algorithmic approach since it relies on the built-in `sort()` method without any modifications or enhancements."
            },
            "Demerits": "There is a lack of originality in the sorting technique, as it could have showcased a custom sorting algorithm, which would demonstrate deeper understanding and application of algorithms."
        }
    ]
}
\end{lstlisting} 
\RR{\subsection{Collaborative Behavioral Analysis}
\textbf{Prompt:}}
\RR{
You are a teacher teaching programming class and will be provided some conversation files belonging to one group but in different questions period, including question number, timestamp for each speaker and corresponding conversation recorded in the context of a collaborative programming course. It will be your job to find students' communication behaviors for all sentences they said and record them in a JSON format file, besides, show the prediction percentage of your response after each behavior by using the scaffold.
Notice again, you should show all the sentences in the JSON file. If you can't classify some of them into category, indicate them into one category with the most similar meaning and point out the percentage. 
Specifically, each question is encircled by a ``{ },'' and you need to show details information in it.
%The out format of your each response should be like, each one should strictly follow the format:


\textbf{Output Format:}}
\begin{lstlisting}
{
    "Question": "Question Number",
    "Conversations": [
        {
            "Speaker": "",
            "Timestamp": "",
            "Content": "",
            "Behavior Category": "",
            "Prediction Percentage": "",
            "Explanation": ""
        },
        {
            "Speaker": "",
            "Timestamp": "",
            "Content": "",
            "Behavior Category": "",
            "Prediction Percentage": "",
            "Explanation": ""
        }
    ]
}
\end{lstlisting}

\RR{
\textbf{Input Example:}}
\begin{lstlisting}
{
    "role": "user",
    "content": """
        Question1
        1.00 2.90 0303 This one is better done
        2.90 13.90 0302 This one compares This first question is so good that it doesn't need doesn't need to be discussed too much, what else could be wrong with it? There's really no problem with it.
        13.90 14.90 0303 Don't don't laugh eh?
        21.00 24.30 0303 Good, the first question is not too difficult, successfully completed
        34.30 40.40 0302 Whatever, just mess around with the name, just send this when the time comes.
        40.40 43.40 0302 Good success, perfect, success.
         """
}
\end{lstlisting}


\RR{
\textbf{Output Example:}}
\begin{lstlisting}
{
    "Question": "Question1",
    "Conversations": [
        {
            "Speaker": "0303",
            "Timestamp": "1.00-2.90",
            "Content": "This is better done",
            "Behavior Category": "Project understanding",
            "Prediction Percentage": "100%",
            "Explanation": "Member expresses that the task is easy, indicating understanding of the project."
        },
        {
            "Speaker": "0302",
            "Timestamp": "2.90-13.90",
            "Content": "This comparison of this first question is so good that it doesn't need to be without too much discussion, what could possibly be wrong with it? There really is no problem with it.",
            "Behavior Category": "Project understanding",
            "Prediction Percentage": "90%",
            "Explanation": "Member emphasizes that the first task is simple and questions if there are any concerns, showing clarity in project understanding."
        },
        {
            "Speaker": "0303",
            "Timestamp": "13.90-14.90",
            "Content": "Don't laugh.",
            "Behavior Category": "Unrelated chat",
            "Prediction Percentage": "60%",
            "Explanation": "This is unrelated humor, not pertinent to the task, indicating a digression from the activity."
        },
        {
            "Speaker": "0303",
            "Timestamp": "21.00-24.30",
            "Content": "Okay, the first question was not too difficult and went well",
            "Behavior Category": "Acknowledgement",
            "Prediction Percentage": "100%",
            "Explanation": "The member acknowledges successful completion of the first task."
        },
        {
            "Speaker": "0302",
            "Timestamp": "34.30-40.40",
            "Content": "Whatever, just make up a name, and when the time comes, just send this.",
            "Behavior Category": "Unrelated chat",
            "Prediction Percentage": "50%",
            "Explanation": "Casual discussion about naming, which strays from focused collaboration."
        },
        {
            "Speaker": "0302",
            "Timestamp": "40.40-43.40",
            "Content": "So successful, perfect, successful.",
            "Behavior Category": "Acknowledgement",
            "Prediction Percentage": "80%",
            "Explanation": "Member acknowledges success with positive feedback."
        }
    ]
}
\end{lstlisting}
\RR{
\subsection{Student Role Analysis}
\textbf{Prompt:}}

\RR{
You are a teacher teaching programming class and will be provided some conversation files belonging to one group but in different questions period, including question number, timestamp for each speaker and corresponding conversation recorded in the context of a collaborative programming course. Your jobYour job will be to find students' planning solutions behaviors related to the question for all sentences they said. 
Here is some example: ``This question could go like this...'', ``Combine A and B'', ``Notice the function.'', etc. Specifically, you should indicate each speaker's sentences of planning solutions corresponding timestamp. Besides, there are some misunderstanding sentences that are not planning solutions: ``I'm just messing around with names.'', ``That's good. That's good.'' etc. You need to find valuable comments that contribute or drive the problem-solving process. Notice that you only need to indicate the sentences about planning solutions or providing insights.
Let me explain ``Navigator'', ``Driver'' and ``Monitor''. Each group only have three members, ``Navigator'' is the speaker who's sentence is about planning solutions, ``Driver'' is the member who responsible for coding, this role is non-changeable. I will tell you who are the ``Driver'' in each file's first line. Besides, ``Monitor'' is the role who is neither 
``Navigator'' nor ``Driver''. If the role of ``Driver'' is also planning solutions, then the role of ``Driver'' should be changed to ``Navigator'', and ``Driver'' is None for this sentence. Significantly, if a sentence is not about planning solutions, you also need to list/indicate it. The Navigator should be None, but fill out the speakers of ``Monitors'' and ``Drivers''.

\textbf{Output Format:}}
\begin{lstlisting}
{
    "Question": "QuestionX",
    "Conversations": [
        {
            "Timestamp": "XXX",
            "Content": "XXX",
            "Navigator": "XXX",
            "Other_Roles": [
                {
                    "Monitors": ["XXX"],
                    "Driver": ["XXX"]
                }
            ]
        },
        {
            "Timestamp": "XXX",
            "Content": "XXX",
            "Navigator": "XXX",
            "Other_Roles": [
                {
                    "Monitors": ["XXX"],
                    "Driver": ["XXX"]
                }
            ]
        }
    ]
}
\end{lstlisting}

\RR{
\textbf{Input Example:}}
\begin{lstlisting}
"""
    Question2 Driver: 0302
    46.70 49.40 0303 Question two, question two, continue with question two.
    58.10 60.90 0302 What is the title of the second question?
    60.90 62.20 0303, and the question is.
    75.20 77.90 0301, also in descending order.
    78.90 80.80 0302 That's okay; that should be the template for the first question.
    80.80 81.80 0301 AB merge and then
    81.80 87.30 0303 Right, right, the second question would also work, yes
    91.30 98.50 0302 Right, this one should be noted that this ijk he this exclamation point should be an introduction to the.
"""
\end{lstlisting}

\RR{
\textbf{Output Example:}}
\begin{lstlisting}
{
    "Question": "Question2",
    "Conversations": [
        {
            "Timestamp": "46.70-49.40",
            "Content": "Question two. Question two. Go on to question two.",
            "Navigator": "0303",
            "Other_Roles": [
                {
                    "Monitors": ["0301"],
                    "Drivers": ["0302"]
                }
            ]
        },
        {
            "Timestamp": "58.10-60.90",
            "Content": "What is the title of the second question?",
            "Navigator": "0302",
            "Other_Roles": [
                {
                    "Monitors": ["0301", "0303"],
                    "Drivers": ["None"]
                }
            ]
        },
        {
            "Timestamp": "60.90-62.20",
            "Content": "The question is.",
            "Navigator": "0303",
            "Other_Roles": [
                {
                    "Monitors": ["0301"],
                    "Drivers": ["0302"]
                }
            ]
        },
        {
            "Timestamp": "75.20-77.90",
            "Content": "It's also sorted from childhood to adulthood.",
            "Navigator": "0301",
            "Other_Roles": [
                {
                    "Monitors": ["0303"],
                    "Drivers": ["0302"]
                }
            ]
        },
        {
            "Timestamp": "78.90-80.80",
            "Content": "That's okay. That would be the template for the first question.",
            "Navigator": "0302",
            "Other_Roles": [
                {
                    "Monitors": ["0301", "0303"],
                    "Drivers": ["None"]
                }
            ]
        },
        {
            "Timestamp": "80.80-81.80",
            "Content": "AB is merged and then",
            "Navigator": "0301",
            "Other_Roles": [
                {
                    "Monitors": ["0303"],
                    "Drivers": ["0302"]
                }
            ]
        },
        {
            "Timestamp": "81.80-87.30",
            "Content": "Yeah, yeah, the second question would work, too. Yeah.",
            "Navigator": "0303",
            "Other_Roles": [
                {
                    "Monitors": ["0301"],
                    "Drivers": ["0302"]
                }
            ]
        },
        {
            "Timestamp": "91.30-98.50",
            "Content": "Yes, this one should be noted that this ijk he this exclamation point is supposed to be an introduction.",
            "Navigator": "0302",
            "Other_Roles": [
                {
                    "Monitors": ["0301", "0303"],
                    "Drivers": ["None"]
                }
            ]
        }
    ]
}
\end{lstlisting}

\RR{
\subsection{Teacher Scaffold Analysis}

\textbf{Prompt:}

You are a teacher teaching programming class and you will be provided a conversation file including timestamp for each speaker and corresponding content recorded in the context of a collaborative programming course. It will be your job to find instructors' assitance category based on the following scaffold.
Analyze the different levels of scaffolding used by instructors during group learning based on the following categories:
Low-control cognitive scaffolding (CS-L): The instructor raises open-ended questions that elicit group thinking without providing new information. This method encourages critical thinking but leaves the group to figure out the details.      
Medium-control cognitive scaffolding (CS-M): The instructor provides hints or clues to help groups solve cognitive problems. This method supports problem-solving but maintains some cognitive challenge.       
High-control cognitive scaffolding (CS-H): The instructor directly provides answers or demonstrates tasks (such as programming) using tools like computers. This method offers direct guidance but may limit students' independent problem-solving.      
Metacognitive scaffolding (MS): The instructor monitors and regulates the group's learning goals and collaborative processes, helping to manage group dynamics and learning strategies.

\textbf{Input Example:}}
\begin{lstlisting}
"""
    131.0 133.0 0000 What's the problem? Oh, I see. 
    134.0 135.0 0000 3 Hmm, yes, exactly. 
    139.0 142.0 0000 Hmm, I, I, where is that program? 
    143.0 196.0 0000 Then here, you need to indent first, right? Wherever you don't finish, you must add a colon, okay? Hmm, like in the if statement, you're correct about that. I, I, it should be I != K, right? Hmm, and and means "and", K should not be equal to J, right? And another one, I should also not be equal to K, right? Isn't it true that I, J, and K should all be different from each other? So if this condition is met, what happens next? Then you calculate something, calculate an S. The S would be equal to I * 100 + J * 10 + K, right? Is that correct? But don't forget to print it out, so print what? print this S, and that's it.
"""
\end{lstlisting}
\RR{
\textbf{Output Example:}}

\begin{lstlisting}
{
    "Speaker": "0000",
    "Timestamp": "131.0-133.0",
    "Content": "What is the question? Oh, oh, yes.",
    "Behavior Category": "Metacognitive scaffolding",
    "Prediction Percentage": "80%",
    "Explanation": "Requests clarification about the third question.",
},
{
    "Speaker": "0000",
    "Timestamp": "134.0-135.0",
    "Content": "Three digits, yes, yes.",
    "Behavior Category": "Metacognitive scaffolding",
    "Prediction Percentage": "90%",
    "Explanation": "Acknowledges the provided information.",
},
{
    "Speaker": "0000",
    "Timestamp": "139.0-142.0",
    "Content": "Hmm, where is your program? Oh.",
    "Behavior Category": "Metacognitive scaffolding",
    "Prediction Percentage": "90%",
    "Explanation": "Teacher responds by asking about the student's program location.",
},
{
    "Speaker": "0000",
    "Timestamp": "143.0-196.0",
    "Content": "Then you start with if here, make sure to indent, and always use a colon after each if. Yes, if i is not equal to k, right? And also k is not equal to j. I also can't be equal to k, right? So, if i, j, and k are not equal to each other, then what? You calculate something, calculate an s, which equals i times 100, plus j times 10, plus k. Right? But you have to remember to print it out. Print s, and that's it.",
    "Behavior Category": "High-control cognitive scaffolding",
    "Prediction Percentage": "100%",
    "Explanation": "Teacher provides detailed explanation and coding instructions.",
}
         
\end{lstlisting}

\end{document}
\endinput
%%
%% End of file `sample-sigconf-authordraft.tex'.

%%
%% The next two lines define the bibliography style to be used, and
%% the bibliography file.
%\bibliographystyle{ACM-Reference-Format}
%\bibliography{cpvisref}


%%
%% If your work has an appendix, this is the place to put it.
\appendix
\RR{
\newpage
\section{DETAILS OF FORMATIVE STUDY}
\subsection{Background Information of Participants}
The table~\ref{tab:info} presents information about the participants in our formative study.
}


%Lorem ipsum dolor sit amet, consectetur adipiscing elit. Morbi

\RR{
\subsection{Findings of the Semi-structured Interviews}

\begin{table*}[hb]
\renewcommand{\arraystretch}{1.5}
\centering
\begin{tabular}{m{1.5cm} m{1cm} m{1cm} m{1.5cm} m{3.5cm} m{2.5cm} m{3.5cm}}
\hline
Participants & Gender & Age & Teaching experience & Students taught & Whether the class is large-scale & Whether the students know each other before \\ \hline
E1 & Female & 45 & 15 & University Non-Computer Science Majors & Yes & No \\
E2 & Female & 30 & 5 & University Non-Computer Science Majors & Yes & No \\
TA1 & Male & 24 & 1 & University Non-Computer Science Majors & Yes & No \\
TA2 & Female & 24 & 1 & University Non-Computer Science Majors & Yes & No \\
T1 & Male & 31 & 3 & University Non-Computer Science Majors & Yes & No \\
T2 & Male & 29 & 5 & University Computer Science Majors & Yes & Yes \\
T3 & Male & 36 & 8 & University Computer Science Majors & No & Yes \\
T4 & Female & 26 & 3 & Vocational High School Non-Computer Majors & Yes & Yes \\
T5 & Female & 26 & 3 & Elementary School Students & No & Yes \\
T6 & Male & 35 & 7 & University Non-Computer Science Majors & Yes & No \\ \hline
\end{tabular}
\caption{The detailed background statistical information of the participants in the Formative Study.}
\label{tab:info}
\end{table*}

\textbf{\textit{Motivations:}}
Participants unanimously agreed that collaborative programming is an important teaching strategy for programming education. However, different teachers had varying focuses and practices regarding collaborative programming. 
Teachers specializing in computer science tended to have students complete tasks independently in foundational programming courses (e.g., Python, Java) while implementing collaborative programming for more complex projects. 
For example, \textit{T3 stated, ``I consider collaborative learning only when the programming project requires different students' creative input, innovation, exploration, and discovery of new content, or when the workload is too large for one person to complete.''}
In contrast, teachers of non-computer science courses held different views. 
E2 pointed out that for beginners who are new to programming languages, even a fundamental programming problem can be daunting.
Collaborative programming can significantly alleviate this issue and improve students' efficiency. 
Group members can work together to complete a programming task, which motivates students, allows them to learn from each other, and reduces the teaching burden (T4).
Additionally, in collaborative programming classes, teachers focus on students' cognitive skills (e.g., knowledge acquisition) and non-cognitive skills (e.g., computational thinking, problem-solving abilities). Most participants believed that these two aspects do not conflict and are part of the overall teaching goals. 
\textit{T5 stated that her course design primarily focuses on developing non-cognitive skills, particularly enhancing students' abilities during the collaboration process. }
T4 believed that cognitive skills can be supplemented with extra time or individual tutoring, but problem-solving and collaboration skills are challenging to develop outside of class. 
Furthermore, E1, T4, and T3 mentioned that quantifying the improvement of students' non-cognitive skills is very challenging.
On one hand, teachers and assistants are busy answering students' questions, making it challenging to observe students' performance. 
On the other hand, non-cognitive skills are not as quickly assessed through tests as knowledge-based content.

Despite the unanimous recognition of the importance of collaborative programming, participants shared several challenges they faced in practice, summarized as follows:

\textit{\textbf{Group Formation:}}
For group size, most participants preferred group sizes of 2-3 students, noting that larger groups (up to 5 members) could decrease the quality of collaboration and make it difficult for students to find their roles. T4 favored pair programming but often had to form groups of three due to large class sizes. T1 and T3 typically recommended groups of 4-5 students, given the complexity and workload of their programming projects, which smaller groups might struggle to complete.
For the group formation method, participants had various approaches to forming groups. T2, T3, T4, and T5, whose students come from fixed classes, generally allowed students to create their own groups. In contrast, T6, T1, E1, and E2 taught general education courses open to all students across the university. These students came from different majors and did not know each other, making self-selection difficult. E2 conducted preliminary assessments of students and formed balanced groups based on their prior knowledge. However, determining how to create groups remains a significant challenge, as different criteria can lead to varying outcomes. Arbitrarily formed groups might result in imbalanced teams, where students' abilities are insufficient to complete the tasks.

\textit{\textbf{Evaluating the Learning Objectives:}}
Evaluating whether the teacher has achieved the learning objectives is an essential criterion for assessing the quality of collaborative programming courses. All participants stated that they generally meet the learning objectives and sometimes even exceed expectations. T4 mentioned that at the start of the collaborative course, she assigned a group leader to each group to take on a leadership role. While she initially spent a significant amount of time guiding the group leaders and facilitating student collaboration, by the second half of the semester, the group leaders were typically able to function as ``little teachers,'' helping group members complete tasks. This allowed the teacher to free up time and energy for other responsibilities.



%Etiam commodo feugiat nisl pulvinar pellentesque. Etiam auctor sodales
\subsection{Results of Rating and Ranking in Formative Study}
Table~\ref{table:feature} presents the metrics participants rated and ranked for the collaborative programming visual analytics system, as discussed in Section 3.2.}
\begin{table}[htbp]
\renewcommand{\arraystretch}{1.5}
\begin{tabular}{p{8cm}}
\hline
\textbf{Features}  \\ \hline
F1: Change in student engagement  \\
F2: Change in student behavior \\
F3: Analysis of student emotional state \\
F4: Student background statistics \\
F5: Analysis of student programming habits \\
F6: Change in role distribution \\
F7: Role assignment suggestions within the group \\
F8: Group performance overview \\
F9: Comparison of group performances \\
F10: Analysis of similarity in group performance  \\
F11: Generation of personalized suggestions \\
F12: View of raw data \\
F13: Programming task evaluation \\
F14: Visualization of problem-solving path \\
F15: Real-time feedback feature \\
F16: Students performance overview \\ \hline
\end{tabular}
\caption{Features of Collaborative Programming Visualization Systems Extracted from Semi-structured Interviews.}
\label{table:feature}
\end{table}
\RR{The raw rating data is displayed in Table~\ref{table:rating}.
The results show that participants rated most of the features mentioned highly, except for F3, F4, F5, F7, and F15.}
\begin{table*}[htbp]
\renewcommand{\arraystretch}{1.5}
\centering
\small
\begin{tabular}{l c c c c c c c c c c c c}
\hline
\textbf{Feature} & \textbf{T6} & \textbf{T5} & \textbf{E1} & \textbf{E2} & \textbf{TA1} & \textbf{TA2} & \textbf{T1} & \textbf{T2} & \textbf{T3} & \textbf{T4} & \textbf{Mean} & \textbf{SD} \\ \hline
F1 & 5 & 6 & 5 & 6 & 7 & 6 & 7 & 5 & 6 & 7 & 6 & 0.82 \\
F2 & 7 & 6 & 5 & 5 & 4 & 7 & 6 & 6 & 6 & 6 & 5.8 & 0.92 \\
F3 & 3 & 4 & 3 & 4 & 4 & 5 & 6 & 4 & 3 & 4 & 4 & 0.94 \\
F4 & 2 & 4 & 2 & 2 & 2 & 3 & 3 & 4 & 6 & 5 & 3.3 & 1.42 \\
F5 & 4 & 3 & 6 & 5 & 4 & 3 & 4 & 4 & 3 & 5 & 4.1 & 0.99 \\
F6 & 4 & 6 & 7 & 3 & 7 & 5 & 4 & 6 & 6 & 7 & 5.5 & 1.43 \\
F7 & 2 & 4 & 3 & 4 & 2 & 4 & 4 & 3 & 4 & 3 & 3.3 & 0.82 \\
F8 & 7 & 6 & 7 & 7 & 7 & 7 & 7 & 7 & 7 & 7 & 6.9 & 0.32 \\
F9 & 6 & 6 & 5 & 7 & 5 & 4 & 7 & 6 & 5 & 7 & 5.8 & 1.03 \\
F10 & 7 & 6 & 5 & 6 & 5 & 5 & 5 & 5 & 6 & 7 & 5.7 & 0.82 \\
F11 & 7 & 6 & 5 & 5 & 5 & 5 & 6 & 7 & 7 & 7 & 6 & 0.94 \\
F12 & 3 & 5 & 5 & 5 & 5 & 5 & 6 & 7 & 4 & 3 & 4.8 & 1.23 \\
F13 & 7 & 6 & 7 & 7 & 7 & 7 & 7 & 7 & 6 & 6 & 6.7 & 0.48 \\
F14 & 6 & 7 & 6 & 6 & 6 & 5 & 4 & 5 & 6 & 6 & 5.7 & 0.82 \\
F15 & 2 & 3 & 4 & 3 & 2 & 5 & 6 & 7 & 7 & 4 & 4.3 & 1.89 \\
F16 & 7 & 7 & 6 & 6 & 5 & 7 & 7 & 5 & 6 & 6 & 6.2 & 0.79 \\ \hline
\end{tabular}
\caption{Mean and SD Results of the Ratings for F1-F16 by Ten Participants.}
\label{table:rating}
\end{table*}
\RR{The raw ranking data is displayed in Table~\ref{table:rank}.
Each feature's average ranking is listed (Fig.~\ref{fig:rank}), with lower numbers indicating that the evaluators more highly value the feature. For example, the average ranking of ``Students Performance Overview'' is 4.8, making it one of the most valued features. On the other hand, ``Role assignment suggestions within the group'' has an average ranking of 13.4, which is ranked lower by the evaluators.}



\begin{table*}[htbp]

\renewcommand{\arraystretch}{1.5}
\centering
\small
\begin{tabular}{l c c c c c c c c c c c c}
\hline
\textbf{Feature} & \textbf{T6} & \textbf{T5} & \textbf{E1} & \textbf{E2} & \textbf{TA1} & \textbf{TA2} & \textbf{T1} & \textbf{T2} & \textbf{T3} & \textbf{T4} & \textbf{Mean} & \textbf{SD} \\ \hline
F1 & 7 & 1 & 7 & 9 & 12 & 2 & 1 & 11 & 1 & 4 & 5.5 & 4.28 \\
F2 & 6 & 3 & 6 & 10 & 11 & 3 & 3 & 10 & 3 & 5 & 6 & 3.23 \\
F3 & 12 & 8 & 16 & 14 & 16 & 12 & 13 & 9 & 4 & 13 & 11.7 & 3.74 \\
F4 & 16 & 15 & 15 & 13 & 14 & 13 & 12 & 6 & 6 & 14 & 12.4 & 3.57 \\
F5 & 14 & 13 & 14 & 15 & 15 & 15 & 14 & 12 & 2 & 15 & 12.9 & 3.96 \\
F6 & 7 & 2 & 5 & 8 & 10 & 4 & 11 & 8 & 7 & 6 & 6.8 & 2.70 \\
F7 & 15 & 14 & 13 & 16 & 13 & 14 & 15 & 13 & 5 & 16 & 13.4 & 3.17 \\
F8 & 8 & 6 & 4 & 1 & 3 & 11 & 8 & 7 & 8 & 2 & 5.8 & 3.19 \\
F9 & 5 & 5 & 12 & 7 & 9 & 6 & 9 & 14 & 16 & 7 & 9 & 3.83 \\
F10 & 10 & 4 & 3 & 3 & 4 & 5 & 10 & 5 & 10 & 8 & 6.2 & 2.97 \\
F11 & 9 & 7 & 2 & 4 & 5 & 7 & 7 & 4 & 9 & 9 & 6.3 & 2.45 \\
F12 & 11 & 10 & 11 & 11 & 6 & 10 & 16 & 15 & 15 & 10 & 11.5 & 3.03 \\
F13 & 3 & 9 & 1 & 5 & 1 & 8 & 6 & 16 & 11 & 3 & 6.3 & 4.79 \\
F14 & 4 & 11 & 8 & 6 & 7 & 9 & 5 & 3 & 14 & 11 & 7.8 & 3.49 \\
F15 & 13 & 16 & 10 & 12 & 8 & 16 & 2 & 1 & 12 & 12 & 10.2 & 5.18 \\
F16 & 2 & 12 & 9 & 2 & 2 & 1 & 4 & 2 & 13 & 1 & 4.8 & 4.69 \\
\hline
\end{tabular}
\caption{Mean and SD Results of the Rankings for F1-F16 by Ten Participants.}
\label{table:rank}
\end{table*}

\RR{\textbf{Spearman Correlation Matrix:}
To assess the consistency between rankings, we used Spearman's Rank Correlation to calculate the correlation of rankings between each participant., The Spearman correlation matrix shows the consistency of the rankings between participants (Fig.~\ref{fig:myplot}). The correlation coefficient ranges from -1 to 1:
- A value closer to 1 indicates that the two evaluators' rankings are more consistent.
- A value closer to -1 suggests that the rankings are more inconsistent.
- A value near 0 implies no significant correlation between the two evaluators' rankings. For example, the correlation between participant 0 and participant 9 is 0.809, indicating
relatively consistent rankings. In contrast, the correlation between evaluator seven and evaluator 8 is - 0.150, suggesting their rankings differ considerably.}
\begin{figure}[htbp]
	\centering
	\includegraphics[width=0.8\linewidth]{rank.png}
	\caption{Ranking Results of F1-F16.}
        \Description{Ranking Results of F1-F16.}
	\label{fig:rank}
\end{figure}


\begin{figure}[htbp]
    \centering
    \includegraphics[width=0.8\linewidth]{myplot.png}
    \caption{Spearman's Rank Correlation Results Between Participants.}
    \Description{Spearman's Rank Correlation Results Between Participants.}
    \label{fig:myplot}
\end{figure}

\RR{\textbf{Friedman Test Results:}
We conducted a Friedman test to assess whether there are significant differences in rankings among different participants. The Friedman test compares the ranking data of multiple related samples to determine whether there are significant differences in rankings under different conditions. The test statistic is 60.796, and the p-value is 1.84e-07. This very small p-value (< 0.05) indicates that:

- There are significant differences in the rankings of different features. The evaluators' ratings are not random or similar but show clear preferences for specific features over others.

In conclusion, after comprehensively analyzing the results for each feature, our system focuses on implementing features \textbf{F1, F2, F6, F8-14, and F16}.
}
\newpage
\RR{
\section{PROMPTS}
Below are the prompts used to label data.
\subsection{Python Code Evaluation}
\textbf{Prompt:}

I would like you to play the role of a teacher who teaches a Python programming class, and you will be provided a question statement and a Python code, which is the student's answer to the question. 
Regarding the Python code, you need to accomplish two tasks. 
Here are the scoring criteria. Please mark each point according to the scoring criteria and explain the reason. Meanwhile, you should give your final score.
If the score of each aspect is not 5, please point out the demerits of the code. Also, note that you don't need to give the advised code. }


\begin{lstlisting}
criteria = {
    "Problem-solving Approach (5%)": {
        "Excellent (5)": "Shows an effective problem-solving approach, effectively addressing key challenges in the task",
        "Good (4)": "Shows a good problem-solving approach, with clear attempts to address challenges in the task",
        "Fair (3)": "Shows some effort in problem-solving, but lacks clarity or effectiveness in addressing challenges in the task",
        "Poor (2)": "Shows limited problem-solving efforts, with unclear or ineffective attempts to address challenges in the task",
        "Bad (1)": "Demonstrates no effective problem-solving approach, unable to address the task"
    },
    "Code Integrity (35%)": {
        "Excellent (5)": "The code is well-structured, organized, readable, and it effectively implements the desired functionality",
        "Good (4)": "The code structure and organization are sufficient, and it implements the function, though readability could be enhanced",
        "Fair (3)": "The code implements 80% of the function but lacks consistency or clarity",
        "Poor (2)": "The code implements 60% of the function but lacks consistency or clarity",
        "Bad (1)": "The code failed to implement 40% of the function and lacks consistency or clarity"
    },
    "Code Accuracy (35%)": {
        "Excellent (5)": "Code exhibits an excellent level of accuracy, producing correct results under various conditions and inputs",
        "Good (4)": "Code exhibits good accuracy, producing correct results under most conditions and inputs",
        "Fair (3)": "Code accuracy is acceptable, but there may be some occasional errors or inconsistencies in the results",
        "Poor (2)": "Code accuracy is poor, with frequent errors or inconsistencies in the results",
        "Bad (1)": "Code accuracy is severely lacking, with a high frequency of errors or inconsistencies in the results"
    },
    "Algorithm Innovation (25%)": {
        "Excellent (5)": "Shows innovative and creative approaches, showcasing originality and ingenuity in algorithm design and implementation",
        "Good (4)": "Shows basic innovation in algorithm design and implementation, but lacks originality or significant creativity",
        "Fair (3)": "Demonstrates limited innovation in algorithm design and implementation, with little originality or creativity demonstrated",
        "Poor (2)": "Shows little innovation in algorithm design and implementation, with no originality or creativity demonstrated",
        "Bad (1)": "Shows no innovation in algorithm design and implementation; only capable of imitating from simple examples"
    }
}
\end{lstlisting} 



\RR{
\textbf{Task:} Task 1, you need to extract the key ideas of the code. Give me a paragraph outlining the thought process of this code. Task 2, Grade the code according to the scoring criteria, and don't round the final score.

\textbf{Input Example:}
}
\begin{lstlisting} 
Question: Existing list a=[49, 38 , 65 , 97 , 76 , 13 , 27 , 55 , 4]
Requirement: Try to write a program in Python that sorts the data elements in a from smallest to largest and prints out the new sorted list a.
Answer:
a=[49,38,65,97,76,13,27,55,4]
a.sort()
print(a)       
\end{lstlisting} 
\RR{
\textbf{Output Example:}}
\begin{lstlisting}
{
    "Key ideas": "The provided code aims to sort a given list of integers in ascending order. The process follows these key steps:\n\t1. Define a list named `a` containing a series of integers.\n\t2. Call the `sort()` method on the list `a`, which sorts the list in place in ascending order.\n\t3. Print the sorted list to display the numbers from smallest to largest.\nThis approach efficiently utilizes Python's built-in sorting capabilities and achieves the task as specified in the question.",
    "Score": "4.55 / 5",
    "Details": [
        {
            "Problem-solving Approach (5%)": {
                "Score": "Excellent (5)",
                "Explanation": "The code presents a clear and effective approach to sorting the list using Python's built-in functionality, which is suitable for the problem at hand."
            },
            "Demerits": null
        },
        {
            "Code Integrity (35%)": {
                "Score": "Excellent (5)",
                "Explanation": "The code is well-structured, organized, and highly readable. It makes effective use of the `sort()` method while maintaining clarity. There are no unnecessary complexities in the implementation."
            },
            "Demerits": null
        },
        {
            "Code Accuracy (35%)": {
                "Score": "Excellent (5)",
                "Explanation": "The code accurately sorts the list `a` and produces the correct output without any errors. The use of the `sort()` method guarantees that the order will be correct."
            },
            "Demerits": null
        },
        {
            "Algorithm Innovation (25%)": {
                "Score": "Fair (3)",
                "Explanation": "While the implementation is effective and clear, it does not demonstrate any innovative or creative algorithmic approach since it relies on the built-in `sort()` method without any modifications or enhancements."
            },
            "Demerits": "There is a lack of originality in the sorting technique, as it could have showcased a custom sorting algorithm, which would demonstrate deeper understanding and application of algorithms."
        }
    ]
}
\end{lstlisting} 
\RR{\subsection{Collaborative Behavioral Analysis}
\textbf{Prompt:}}
\RR{
You are a teacher teaching programming class and will be provided some conversation files belonging to one group but in different questions period, including question number, timestamp for each speaker and corresponding conversation recorded in the context of a collaborative programming course. It will be your job to find students' communication behaviors for all sentences they said and record them in a JSON format file, besides, show the prediction percentage of your response after each behavior by using the scaffold.
Notice again, you should show all the sentences in the JSON file. If you can't classify some of them into category, indicate them into one category with the most similar meaning and point out the percentage. 
Specifically, each question is encircled by a ``{ },'' and you need to show details information in it.
%The out format of your each response should be like, each one should strictly follow the format:


\textbf{Output Format:}}
\begin{lstlisting}
{
    "Question": "Question Number",
    "Conversations": [
        {
            "Speaker": "",
            "Timestamp": "",
            "Content": "",
            "Behavior Category": "",
            "Prediction Percentage": "",
            "Explanation": ""
        },
        {
            "Speaker": "",
            "Timestamp": "",
            "Content": "",
            "Behavior Category": "",
            "Prediction Percentage": "",
            "Explanation": ""
        }
    ]
}
\end{lstlisting}

\RR{
\textbf{Input Example:}}
\begin{lstlisting}
{
    "role": "user",
    "content": """
        Question1
        1.00 2.90 0303 This one is better done
        2.90 13.90 0302 This one compares This first question is so good that it doesn't need doesn't need to be discussed too much, what else could be wrong with it? There's really no problem with it.
        13.90 14.90 0303 Don't don't laugh eh?
        21.00 24.30 0303 Good, the first question is not too difficult, successfully completed
        34.30 40.40 0302 Whatever, just mess around with the name, just send this when the time comes.
        40.40 43.40 0302 Good success, perfect, success.
         """
}
\end{lstlisting}


\RR{
\textbf{Output Example:}}
\begin{lstlisting}
{
    "Question": "Question1",
    "Conversations": [
        {
            "Speaker": "0303",
            "Timestamp": "1.00-2.90",
            "Content": "This is better done",
            "Behavior Category": "Project understanding",
            "Prediction Percentage": "100%",
            "Explanation": "Member expresses that the task is easy, indicating understanding of the project."
        },
        {
            "Speaker": "0302",
            "Timestamp": "2.90-13.90",
            "Content": "This comparison of this first question is so good that it doesn't need to be without too much discussion, what could possibly be wrong with it? There really is no problem with it.",
            "Behavior Category": "Project understanding",
            "Prediction Percentage": "90%",
            "Explanation": "Member emphasizes that the first task is simple and questions if there are any concerns, showing clarity in project understanding."
        },
        {
            "Speaker": "0303",
            "Timestamp": "13.90-14.90",
            "Content": "Don't laugh.",
            "Behavior Category": "Unrelated chat",
            "Prediction Percentage": "60%",
            "Explanation": "This is unrelated humor, not pertinent to the task, indicating a digression from the activity."
        },
        {
            "Speaker": "0303",
            "Timestamp": "21.00-24.30",
            "Content": "Okay, the first question was not too difficult and went well",
            "Behavior Category": "Acknowledgement",
            "Prediction Percentage": "100%",
            "Explanation": "The member acknowledges successful completion of the first task."
        },
        {
            "Speaker": "0302",
            "Timestamp": "34.30-40.40",
            "Content": "Whatever, just make up a name, and when the time comes, just send this.",
            "Behavior Category": "Unrelated chat",
            "Prediction Percentage": "50%",
            "Explanation": "Casual discussion about naming, which strays from focused collaboration."
        },
        {
            "Speaker": "0302",
            "Timestamp": "40.40-43.40",
            "Content": "So successful, perfect, successful.",
            "Behavior Category": "Acknowledgement",
            "Prediction Percentage": "80%",
            "Explanation": "Member acknowledges success with positive feedback."
        }
    ]
}
\end{lstlisting}
\RR{
\subsection{Student Role Analysis}
\textbf{Prompt:}}

\RR{
You are a teacher teaching programming class and will be provided some conversation files belonging to one group but in different questions period, including question number, timestamp for each speaker and corresponding conversation recorded in the context of a collaborative programming course. Your jobYour job will be to find students' planning solutions behaviors related to the question for all sentences they said. 
Here is some example: ``This question could go like this...'', ``Combine A and B'', ``Notice the function.'', etc. Specifically, you should indicate each speaker's sentences of planning solutions corresponding timestamp. Besides, there are some misunderstanding sentences that are not planning solutions: ``I'm just messing around with names.'', ``That's good. That's good.'' etc. You need to find valuable comments that contribute or drive the problem-solving process. Notice that you only need to indicate the sentences about planning solutions or providing insights.
Let me explain ``Navigator'', ``Driver'' and ``Monitor''. Each group only have three members, ``Navigator'' is the speaker who's sentence is about planning solutions, ``Driver'' is the member who responsible for coding, this role is non-changeable. I will tell you who are the ``Driver'' in each file's first line. Besides, ``Monitor'' is the role who is neither 
``Navigator'' nor ``Driver''. If the role of ``Driver'' is also planning solutions, then the role of ``Driver'' should be changed to ``Navigator'', and ``Driver'' is None for this sentence. Significantly, if a sentence is not about planning solutions, you also need to list/indicate it. The Navigator should be None, but fill out the speakers of ``Monitors'' and ``Drivers''.

\textbf{Output Format:}}
\begin{lstlisting}
{
    "Question": "QuestionX",
    "Conversations": [
        {
            "Timestamp": "XXX",
            "Content": "XXX",
            "Navigator": "XXX",
            "Other_Roles": [
                {
                    "Monitors": ["XXX"],
                    "Driver": ["XXX"]
                }
            ]
        },
        {
            "Timestamp": "XXX",
            "Content": "XXX",
            "Navigator": "XXX",
            "Other_Roles": [
                {
                    "Monitors": ["XXX"],
                    "Driver": ["XXX"]
                }
            ]
        }
    ]
}
\end{lstlisting}

\RR{
\textbf{Input Example:}}
\begin{lstlisting}
"""
    Question2 Driver: 0302
    46.70 49.40 0303 Question two, question two, continue with question two.
    58.10 60.90 0302 What is the title of the second question?
    60.90 62.20 0303, and the question is.
    75.20 77.90 0301, also in descending order.
    78.90 80.80 0302 That's okay; that should be the template for the first question.
    80.80 81.80 0301 AB merge and then
    81.80 87.30 0303 Right, right, the second question would also work, yes
    91.30 98.50 0302 Right, this one should be noted that this ijk he this exclamation point should be an introduction to the.
"""
\end{lstlisting}

\RR{
\textbf{Output Example:}}
\begin{lstlisting}
{
    "Question": "Question2",
    "Conversations": [
        {
            "Timestamp": "46.70-49.40",
            "Content": "Question two. Question two. Go on to question two.",
            "Navigator": "0303",
            "Other_Roles": [
                {
                    "Monitors": ["0301"],
                    "Drivers": ["0302"]
                }
            ]
        },
        {
            "Timestamp": "58.10-60.90",
            "Content": "What is the title of the second question?",
            "Navigator": "0302",
            "Other_Roles": [
                {
                    "Monitors": ["0301", "0303"],
                    "Drivers": ["None"]
                }
            ]
        },
        {
            "Timestamp": "60.90-62.20",
            "Content": "The question is.",
            "Navigator": "0303",
            "Other_Roles": [
                {
                    "Monitors": ["0301"],
                    "Drivers": ["0302"]
                }
            ]
        },
        {
            "Timestamp": "75.20-77.90",
            "Content": "It's also sorted from childhood to adulthood.",
            "Navigator": "0301",
            "Other_Roles": [
                {
                    "Monitors": ["0303"],
                    "Drivers": ["0302"]
                }
            ]
        },
        {
            "Timestamp": "78.90-80.80",
            "Content": "That's okay. That would be the template for the first question.",
            "Navigator": "0302",
            "Other_Roles": [
                {
                    "Monitors": ["0301", "0303"],
                    "Drivers": ["None"]
                }
            ]
        },
        {
            "Timestamp": "80.80-81.80",
            "Content": "AB is merged and then",
            "Navigator": "0301",
            "Other_Roles": [
                {
                    "Monitors": ["0303"],
                    "Drivers": ["0302"]
                }
            ]
        },
        {
            "Timestamp": "81.80-87.30",
            "Content": "Yeah, yeah, the second question would work, too. Yeah.",
            "Navigator": "0303",
            "Other_Roles": [
                {
                    "Monitors": ["0301"],
                    "Drivers": ["0302"]
                }
            ]
        },
        {
            "Timestamp": "91.30-98.50",
            "Content": "Yes, this one should be noted that this ijk he this exclamation point is supposed to be an introduction.",
            "Navigator": "0302",
            "Other_Roles": [
                {
                    "Monitors": ["0301", "0303"],
                    "Drivers": ["None"]
                }
            ]
        }
    ]
}
\end{lstlisting}

\RR{
\subsection{Teacher Scaffold Analysis}

\textbf{Prompt:}

You are a teacher teaching programming class and you will be provided a conversation file including timestamp for each speaker and corresponding content recorded in the context of a collaborative programming course. It will be your job to find instructors' assitance category based on the following scaffold.
Analyze the different levels of scaffolding used by instructors during group learning based on the following categories:
Low-control cognitive scaffolding (CS-L): The instructor raises open-ended questions that elicit group thinking without providing new information. This method encourages critical thinking but leaves the group to figure out the details.      
Medium-control cognitive scaffolding (CS-M): The instructor provides hints or clues to help groups solve cognitive problems. This method supports problem-solving but maintains some cognitive challenge.       
High-control cognitive scaffolding (CS-H): The instructor directly provides answers or demonstrates tasks (such as programming) using tools like computers. This method offers direct guidance but may limit students' independent problem-solving.      
Metacognitive scaffolding (MS): The instructor monitors and regulates the group's learning goals and collaborative processes, helping to manage group dynamics and learning strategies.

\textbf{Input Example:}}
\begin{lstlisting}
"""
    131.0 133.0 0000 What's the problem? Oh, I see. 
    134.0 135.0 0000 3 Hmm, yes, exactly. 
    139.0 142.0 0000 Hmm, I, I, where is that program? 
    143.0 196.0 0000 Then here, you need to indent first, right? Wherever you don't finish, you must add a colon, okay? Hmm, like in the if statement, you're correct about that. I, I, it should be I != K, right? Hmm, and and means "and", K should not be equal to J, right? And another one, I should also not be equal to K, right? Isn't it true that I, J, and K should all be different from each other? So if this condition is met, what happens next? Then you calculate something, calculate an S. The S would be equal to I * 100 + J * 10 + K, right? Is that correct? But don't forget to print it out, so print what? print this S, and that's it.
"""
\end{lstlisting}
\RR{
\textbf{Output Example:}}

\begin{lstlisting}
{
    "Speaker": "0000",
    "Timestamp": "131.0-133.0",
    "Content": "What is the question? Oh, oh, yes.",
    "Behavior Category": "Metacognitive scaffolding",
    "Prediction Percentage": "80%",
    "Explanation": "Requests clarification about the third question.",
},
{
    "Speaker": "0000",
    "Timestamp": "134.0-135.0",
    "Content": "Three digits, yes, yes.",
    "Behavior Category": "Metacognitive scaffolding",
    "Prediction Percentage": "90%",
    "Explanation": "Acknowledges the provided information.",
},
{
    "Speaker": "0000",
    "Timestamp": "139.0-142.0",
    "Content": "Hmm, where is your program? Oh.",
    "Behavior Category": "Metacognitive scaffolding",
    "Prediction Percentage": "90%",
    "Explanation": "Teacher responds by asking about the student's program location.",
},
{
    "Speaker": "0000",
    "Timestamp": "143.0-196.0",
    "Content": "Then you start with if here, make sure to indent, and always use a colon after each if. Yes, if i is not equal to k, right? And also k is not equal to j. I also can't be equal to k, right? So, if i, j, and k are not equal to each other, then what? You calculate something, calculate an s, which equals i times 100, plus j times 10, plus k. Right? But you have to remember to print it out. Print s, and that's it.",
    "Behavior Category": "High-control cognitive scaffolding",
    "Prediction Percentage": "100%",
    "Explanation": "Teacher provides detailed explanation and coding instructions.",
}
         
\end{lstlisting}

\end{document}
\endinput
%%
%% End of file `sample-sigconf-authordraft.tex'.

%%
%% The next two lines define the bibliography style to be used, and
%% the bibliography file.
%\bibliographystyle{ACM-Reference-Format}
%\bibliography{cpvisref}


%%
%% If your work has an appendix, this is the place to put it.
\appendix
\RR{
\newpage
\section{DETAILS OF FORMATIVE STUDY}
\subsection{Background Information of Participants}
The table~\ref{tab:info} presents information about the participants in our formative study.
}


%Lorem ipsum dolor sit amet, consectetur adipiscing elit. Morbi

\RR{
\subsection{Findings of the Semi-structured Interviews}

\begin{table*}[hb]
\renewcommand{\arraystretch}{1.5}
\centering
\begin{tabular}{m{1.5cm} m{1cm} m{1cm} m{1.5cm} m{3.5cm} m{2.5cm} m{3.5cm}}
\hline
Participants & Gender & Age & Teaching experience & Students taught & Whether the class is large-scale & Whether the students know each other before \\ \hline
E1 & Female & 45 & 15 & University Non-Computer Science Majors & Yes & No \\
E2 & Female & 30 & 5 & University Non-Computer Science Majors & Yes & No \\
TA1 & Male & 24 & 1 & University Non-Computer Science Majors & Yes & No \\
TA2 & Female & 24 & 1 & University Non-Computer Science Majors & Yes & No \\
T1 & Male & 31 & 3 & University Non-Computer Science Majors & Yes & No \\
T2 & Male & 29 & 5 & University Computer Science Majors & Yes & Yes \\
T3 & Male & 36 & 8 & University Computer Science Majors & No & Yes \\
T4 & Female & 26 & 3 & Vocational High School Non-Computer Majors & Yes & Yes \\
T5 & Female & 26 & 3 & Elementary School Students & No & Yes \\
T6 & Male & 35 & 7 & University Non-Computer Science Majors & Yes & No \\ \hline
\end{tabular}
\caption{The detailed background statistical information of the participants in the Formative Study.}
\label{tab:info}
\end{table*}

\textbf{\textit{Motivations:}}
Participants unanimously agreed that collaborative programming is an important teaching strategy for programming education. However, different teachers had varying focuses and practices regarding collaborative programming. 
Teachers specializing in computer science tended to have students complete tasks independently in foundational programming courses (e.g., Python, Java) while implementing collaborative programming for more complex projects. 
For example, \textit{T3 stated, ``I consider collaborative learning only when the programming project requires different students' creative input, innovation, exploration, and discovery of new content, or when the workload is too large for one person to complete.''}
In contrast, teachers of non-computer science courses held different views. 
E2 pointed out that for beginners who are new to programming languages, even a fundamental programming problem can be daunting.
Collaborative programming can significantly alleviate this issue and improve students' efficiency. 
Group members can work together to complete a programming task, which motivates students, allows them to learn from each other, and reduces the teaching burden (T4).
Additionally, in collaborative programming classes, teachers focus on students' cognitive skills (e.g., knowledge acquisition) and non-cognitive skills (e.g., computational thinking, problem-solving abilities). Most participants believed that these two aspects do not conflict and are part of the overall teaching goals. 
\textit{T5 stated that her course design primarily focuses on developing non-cognitive skills, particularly enhancing students' abilities during the collaboration process. }
T4 believed that cognitive skills can be supplemented with extra time or individual tutoring, but problem-solving and collaboration skills are challenging to develop outside of class. 
Furthermore, E1, T4, and T3 mentioned that quantifying the improvement of students' non-cognitive skills is very challenging.
On one hand, teachers and assistants are busy answering students' questions, making it challenging to observe students' performance. 
On the other hand, non-cognitive skills are not as quickly assessed through tests as knowledge-based content.

Despite the unanimous recognition of the importance of collaborative programming, participants shared several challenges they faced in practice, summarized as follows:

\textit{\textbf{Group Formation:}}
For group size, most participants preferred group sizes of 2-3 students, noting that larger groups (up to 5 members) could decrease the quality of collaboration and make it difficult for students to find their roles. T4 favored pair programming but often had to form groups of three due to large class sizes. T1 and T3 typically recommended groups of 4-5 students, given the complexity and workload of their programming projects, which smaller groups might struggle to complete.
For the group formation method, participants had various approaches to forming groups. T2, T3, T4, and T5, whose students come from fixed classes, generally allowed students to create their own groups. In contrast, T6, T1, E1, and E2 taught general education courses open to all students across the university. These students came from different majors and did not know each other, making self-selection difficult. E2 conducted preliminary assessments of students and formed balanced groups based on their prior knowledge. However, determining how to create groups remains a significant challenge, as different criteria can lead to varying outcomes. Arbitrarily formed groups might result in imbalanced teams, where students' abilities are insufficient to complete the tasks.

\textit{\textbf{Evaluating the Learning Objectives:}}
Evaluating whether the teacher has achieved the learning objectives is an essential criterion for assessing the quality of collaborative programming courses. All participants stated that they generally meet the learning objectives and sometimes even exceed expectations. T4 mentioned that at the start of the collaborative course, she assigned a group leader to each group to take on a leadership role. While she initially spent a significant amount of time guiding the group leaders and facilitating student collaboration, by the second half of the semester, the group leaders were typically able to function as ``little teachers,'' helping group members complete tasks. This allowed the teacher to free up time and energy for other responsibilities.



%Etiam commodo feugiat nisl pulvinar pellentesque. Etiam auctor sodales
\subsection{Results of Rating and Ranking in Formative Study}
Table~\ref{table:feature} presents the metrics participants rated and ranked for the collaborative programming visual analytics system, as discussed in Section 3.2.}
\begin{table}[htbp]
\renewcommand{\arraystretch}{1.5}
\begin{tabular}{p{8cm}}
\hline
\textbf{Features}  \\ \hline
F1: Change in student engagement  \\
F2: Change in student behavior \\
F3: Analysis of student emotional state \\
F4: Student background statistics \\
F5: Analysis of student programming habits \\
F6: Change in role distribution \\
F7: Role assignment suggestions within the group \\
F8: Group performance overview \\
F9: Comparison of group performances \\
F10: Analysis of similarity in group performance  \\
F11: Generation of personalized suggestions \\
F12: View of raw data \\
F13: Programming task evaluation \\
F14: Visualization of problem-solving path \\
F15: Real-time feedback feature \\
F16: Students performance overview \\ \hline
\end{tabular}
\caption{Features of Collaborative Programming Visualization Systems Extracted from Semi-structured Interviews.}
\label{table:feature}
\end{table}
\RR{The raw rating data is displayed in Table~\ref{table:rating}.
The results show that participants rated most of the features mentioned highly, except for F3, F4, F5, F7, and F15.}
\begin{table*}[htbp]
\renewcommand{\arraystretch}{1.5}
\centering
\small
\begin{tabular}{l c c c c c c c c c c c c}
\hline
\textbf{Feature} & \textbf{T6} & \textbf{T5} & \textbf{E1} & \textbf{E2} & \textbf{TA1} & \textbf{TA2} & \textbf{T1} & \textbf{T2} & \textbf{T3} & \textbf{T4} & \textbf{Mean} & \textbf{SD} \\ \hline
F1 & 5 & 6 & 5 & 6 & 7 & 6 & 7 & 5 & 6 & 7 & 6 & 0.82 \\
F2 & 7 & 6 & 5 & 5 & 4 & 7 & 6 & 6 & 6 & 6 & 5.8 & 0.92 \\
F3 & 3 & 4 & 3 & 4 & 4 & 5 & 6 & 4 & 3 & 4 & 4 & 0.94 \\
F4 & 2 & 4 & 2 & 2 & 2 & 3 & 3 & 4 & 6 & 5 & 3.3 & 1.42 \\
F5 & 4 & 3 & 6 & 5 & 4 & 3 & 4 & 4 & 3 & 5 & 4.1 & 0.99 \\
F6 & 4 & 6 & 7 & 3 & 7 & 5 & 4 & 6 & 6 & 7 & 5.5 & 1.43 \\
F7 & 2 & 4 & 3 & 4 & 2 & 4 & 4 & 3 & 4 & 3 & 3.3 & 0.82 \\
F8 & 7 & 6 & 7 & 7 & 7 & 7 & 7 & 7 & 7 & 7 & 6.9 & 0.32 \\
F9 & 6 & 6 & 5 & 7 & 5 & 4 & 7 & 6 & 5 & 7 & 5.8 & 1.03 \\
F10 & 7 & 6 & 5 & 6 & 5 & 5 & 5 & 5 & 6 & 7 & 5.7 & 0.82 \\
F11 & 7 & 6 & 5 & 5 & 5 & 5 & 6 & 7 & 7 & 7 & 6 & 0.94 \\
F12 & 3 & 5 & 5 & 5 & 5 & 5 & 6 & 7 & 4 & 3 & 4.8 & 1.23 \\
F13 & 7 & 6 & 7 & 7 & 7 & 7 & 7 & 7 & 6 & 6 & 6.7 & 0.48 \\
F14 & 6 & 7 & 6 & 6 & 6 & 5 & 4 & 5 & 6 & 6 & 5.7 & 0.82 \\
F15 & 2 & 3 & 4 & 3 & 2 & 5 & 6 & 7 & 7 & 4 & 4.3 & 1.89 \\
F16 & 7 & 7 & 6 & 6 & 5 & 7 & 7 & 5 & 6 & 6 & 6.2 & 0.79 \\ \hline
\end{tabular}
\caption{Mean and SD Results of the Ratings for F1-F16 by Ten Participants.}
\label{table:rating}
\end{table*}
\RR{The raw ranking data is displayed in Table~\ref{table:rank}.
Each feature's average ranking is listed (Fig.~\ref{fig:rank}), with lower numbers indicating that the evaluators more highly value the feature. For example, the average ranking of ``Students Performance Overview'' is 4.8, making it one of the most valued features. On the other hand, ``Role assignment suggestions within the group'' has an average ranking of 13.4, which is ranked lower by the evaluators.}



\begin{table*}[htbp]

\renewcommand{\arraystretch}{1.5}
\centering
\small
\begin{tabular}{l c c c c c c c c c c c c}
\hline
\textbf{Feature} & \textbf{T6} & \textbf{T5} & \textbf{E1} & \textbf{E2} & \textbf{TA1} & \textbf{TA2} & \textbf{T1} & \textbf{T2} & \textbf{T3} & \textbf{T4} & \textbf{Mean} & \textbf{SD} \\ \hline
F1 & 7 & 1 & 7 & 9 & 12 & 2 & 1 & 11 & 1 & 4 & 5.5 & 4.28 \\
F2 & 6 & 3 & 6 & 10 & 11 & 3 & 3 & 10 & 3 & 5 & 6 & 3.23 \\
F3 & 12 & 8 & 16 & 14 & 16 & 12 & 13 & 9 & 4 & 13 & 11.7 & 3.74 \\
F4 & 16 & 15 & 15 & 13 & 14 & 13 & 12 & 6 & 6 & 14 & 12.4 & 3.57 \\
F5 & 14 & 13 & 14 & 15 & 15 & 15 & 14 & 12 & 2 & 15 & 12.9 & 3.96 \\
F6 & 7 & 2 & 5 & 8 & 10 & 4 & 11 & 8 & 7 & 6 & 6.8 & 2.70 \\
F7 & 15 & 14 & 13 & 16 & 13 & 14 & 15 & 13 & 5 & 16 & 13.4 & 3.17 \\
F8 & 8 & 6 & 4 & 1 & 3 & 11 & 8 & 7 & 8 & 2 & 5.8 & 3.19 \\
F9 & 5 & 5 & 12 & 7 & 9 & 6 & 9 & 14 & 16 & 7 & 9 & 3.83 \\
F10 & 10 & 4 & 3 & 3 & 4 & 5 & 10 & 5 & 10 & 8 & 6.2 & 2.97 \\
F11 & 9 & 7 & 2 & 4 & 5 & 7 & 7 & 4 & 9 & 9 & 6.3 & 2.45 \\
F12 & 11 & 10 & 11 & 11 & 6 & 10 & 16 & 15 & 15 & 10 & 11.5 & 3.03 \\
F13 & 3 & 9 & 1 & 5 & 1 & 8 & 6 & 16 & 11 & 3 & 6.3 & 4.79 \\
F14 & 4 & 11 & 8 & 6 & 7 & 9 & 5 & 3 & 14 & 11 & 7.8 & 3.49 \\
F15 & 13 & 16 & 10 & 12 & 8 & 16 & 2 & 1 & 12 & 12 & 10.2 & 5.18 \\
F16 & 2 & 12 & 9 & 2 & 2 & 1 & 4 & 2 & 13 & 1 & 4.8 & 4.69 \\
\hline
\end{tabular}
\caption{Mean and SD Results of the Rankings for F1-F16 by Ten Participants.}
\label{table:rank}
\end{table*}

\RR{\textbf{Spearman Correlation Matrix:}
To assess the consistency between rankings, we used Spearman's Rank Correlation to calculate the correlation of rankings between each participant., The Spearman correlation matrix shows the consistency of the rankings between participants (Fig.~\ref{fig:myplot}). The correlation coefficient ranges from -1 to 1:
- A value closer to 1 indicates that the two evaluators' rankings are more consistent.
- A value closer to -1 suggests that the rankings are more inconsistent.
- A value near 0 implies no significant correlation between the two evaluators' rankings. For example, the correlation between participant 0 and participant 9 is 0.809, indicating
relatively consistent rankings. In contrast, the correlation between evaluator seven and evaluator 8 is - 0.150, suggesting their rankings differ considerably.}
\begin{figure}[htbp]
	\centering
	\includegraphics[width=0.8\linewidth]{rank.png}
	\caption{Ranking Results of F1-F16.}
        \Description{Ranking Results of F1-F16.}
	\label{fig:rank}
\end{figure}


\begin{figure}[htbp]
    \centering
    \includegraphics[width=0.8\linewidth]{myplot.png}
    \caption{Spearman's Rank Correlation Results Between Participants.}
    \Description{Spearman's Rank Correlation Results Between Participants.}
    \label{fig:myplot}
\end{figure}

\RR{\textbf{Friedman Test Results:}
We conducted a Friedman test to assess whether there are significant differences in rankings among different participants. The Friedman test compares the ranking data of multiple related samples to determine whether there are significant differences in rankings under different conditions. The test statistic is 60.796, and the p-value is 1.84e-07. This very small p-value (< 0.05) indicates that:

- There are significant differences in the rankings of different features. The evaluators' ratings are not random or similar but show clear preferences for specific features over others.

In conclusion, after comprehensively analyzing the results for each feature, our system focuses on implementing features \textbf{F1, F2, F6, F8-14, and F16}.
}
\newpage
\RR{
\section{PROMPTS}
Below are the prompts used to label data.
\subsection{Python Code Evaluation}
\textbf{Prompt:}

I would like you to play the role of a teacher who teaches a Python programming class, and you will be provided a question statement and a Python code, which is the student's answer to the question. 
Regarding the Python code, you need to accomplish two tasks. 
Here are the scoring criteria. Please mark each point according to the scoring criteria and explain the reason. Meanwhile, you should give your final score.
If the score of each aspect is not 5, please point out the demerits of the code. Also, note that you don't need to give the advised code. }


\begin{lstlisting}
criteria = {
    "Problem-solving Approach (5%)": {
        "Excellent (5)": "Shows an effective problem-solving approach, effectively addressing key challenges in the task",
        "Good (4)": "Shows a good problem-solving approach, with clear attempts to address challenges in the task",
        "Fair (3)": "Shows some effort in problem-solving, but lacks clarity or effectiveness in addressing challenges in the task",
        "Poor (2)": "Shows limited problem-solving efforts, with unclear or ineffective attempts to address challenges in the task",
        "Bad (1)": "Demonstrates no effective problem-solving approach, unable to address the task"
    },
    "Code Integrity (35%)": {
        "Excellent (5)": "The code is well-structured, organized, readable, and it effectively implements the desired functionality",
        "Good (4)": "The code structure and organization are sufficient, and it implements the function, though readability could be enhanced",
        "Fair (3)": "The code implements 80% of the function but lacks consistency or clarity",
        "Poor (2)": "The code implements 60% of the function but lacks consistency or clarity",
        "Bad (1)": "The code failed to implement 40% of the function and lacks consistency or clarity"
    },
    "Code Accuracy (35%)": {
        "Excellent (5)": "Code exhibits an excellent level of accuracy, producing correct results under various conditions and inputs",
        "Good (4)": "Code exhibits good accuracy, producing correct results under most conditions and inputs",
        "Fair (3)": "Code accuracy is acceptable, but there may be some occasional errors or inconsistencies in the results",
        "Poor (2)": "Code accuracy is poor, with frequent errors or inconsistencies in the results",
        "Bad (1)": "Code accuracy is severely lacking, with a high frequency of errors or inconsistencies in the results"
    },
    "Algorithm Innovation (25%)": {
        "Excellent (5)": "Shows innovative and creative approaches, showcasing originality and ingenuity in algorithm design and implementation",
        "Good (4)": "Shows basic innovation in algorithm design and implementation, but lacks originality or significant creativity",
        "Fair (3)": "Demonstrates limited innovation in algorithm design and implementation, with little originality or creativity demonstrated",
        "Poor (2)": "Shows little innovation in algorithm design and implementation, with no originality or creativity demonstrated",
        "Bad (1)": "Shows no innovation in algorithm design and implementation; only capable of imitating from simple examples"
    }
}
\end{lstlisting} 



\RR{
\textbf{Task:} Task 1, you need to extract the key ideas of the code. Give me a paragraph outlining the thought process of this code. Task 2, Grade the code according to the scoring criteria, and don't round the final score.

\textbf{Input Example:}
}
\begin{lstlisting} 
Question: Existing list a=[49, 38 , 65 , 97 , 76 , 13 , 27 , 55 , 4]
Requirement: Try to write a program in Python that sorts the data elements in a from smallest to largest and prints out the new sorted list a.
Answer:
a=[49,38,65,97,76,13,27,55,4]
a.sort()
print(a)       
\end{lstlisting} 
\RR{
\textbf{Output Example:}}
\begin{lstlisting}
{
    "Key ideas": "The provided code aims to sort a given list of integers in ascending order. The process follows these key steps:\n\t1. Define a list named `a` containing a series of integers.\n\t2. Call the `sort()` method on the list `a`, which sorts the list in place in ascending order.\n\t3. Print the sorted list to display the numbers from smallest to largest.\nThis approach efficiently utilizes Python's built-in sorting capabilities and achieves the task as specified in the question.",
    "Score": "4.55 / 5",
    "Details": [
        {
            "Problem-solving Approach (5%)": {
                "Score": "Excellent (5)",
                "Explanation": "The code presents a clear and effective approach to sorting the list using Python's built-in functionality, which is suitable for the problem at hand."
            },
            "Demerits": null
        },
        {
            "Code Integrity (35%)": {
                "Score": "Excellent (5)",
                "Explanation": "The code is well-structured, organized, and highly readable. It makes effective use of the `sort()` method while maintaining clarity. There are no unnecessary complexities in the implementation."
            },
            "Demerits": null
        },
        {
            "Code Accuracy (35%)": {
                "Score": "Excellent (5)",
                "Explanation": "The code accurately sorts the list `a` and produces the correct output without any errors. The use of the `sort()` method guarantees that the order will be correct."
            },
            "Demerits": null
        },
        {
            "Algorithm Innovation (25%)": {
                "Score": "Fair (3)",
                "Explanation": "While the implementation is effective and clear, it does not demonstrate any innovative or creative algorithmic approach since it relies on the built-in `sort()` method without any modifications or enhancements."
            },
            "Demerits": "There is a lack of originality in the sorting technique, as it could have showcased a custom sorting algorithm, which would demonstrate deeper understanding and application of algorithms."
        }
    ]
}
\end{lstlisting} 
\RR{\subsection{Collaborative Behavioral Analysis}
\textbf{Prompt:}}
\RR{
You are a teacher teaching programming class and will be provided some conversation files belonging to one group but in different questions period, including question number, timestamp for each speaker and corresponding conversation recorded in the context of a collaborative programming course. It will be your job to find students' communication behaviors for all sentences they said and record them in a JSON format file, besides, show the prediction percentage of your response after each behavior by using the scaffold.
Notice again, you should show all the sentences in the JSON file. If you can't classify some of them into category, indicate them into one category with the most similar meaning and point out the percentage. 
Specifically, each question is encircled by a ``{ },'' and you need to show details information in it.
%The out format of your each response should be like, each one should strictly follow the format:


\textbf{Output Format:}}
\begin{lstlisting}
{
    "Question": "Question Number",
    "Conversations": [
        {
            "Speaker": "",
            "Timestamp": "",
            "Content": "",
            "Behavior Category": "",
            "Prediction Percentage": "",
            "Explanation": ""
        },
        {
            "Speaker": "",
            "Timestamp": "",
            "Content": "",
            "Behavior Category": "",
            "Prediction Percentage": "",
            "Explanation": ""
        }
    ]
}
\end{lstlisting}

\RR{
\textbf{Input Example:}}
\begin{lstlisting}
{
    "role": "user",
    "content": """
        Question1
        1.00 2.90 0303 This one is better done
        2.90 13.90 0302 This one compares This first question is so good that it doesn't need doesn't need to be discussed too much, what else could be wrong with it? There's really no problem with it.
        13.90 14.90 0303 Don't don't laugh eh?
        21.00 24.30 0303 Good, the first question is not too difficult, successfully completed
        34.30 40.40 0302 Whatever, just mess around with the name, just send this when the time comes.
        40.40 43.40 0302 Good success, perfect, success.
         """
}
\end{lstlisting}


\RR{
\textbf{Output Example:}}
\begin{lstlisting}
{
    "Question": "Question1",
    "Conversations": [
        {
            "Speaker": "0303",
            "Timestamp": "1.00-2.90",
            "Content": "This is better done",
            "Behavior Category": "Project understanding",
            "Prediction Percentage": "100%",
            "Explanation": "Member expresses that the task is easy, indicating understanding of the project."
        },
        {
            "Speaker": "0302",
            "Timestamp": "2.90-13.90",
            "Content": "This comparison of this first question is so good that it doesn't need to be without too much discussion, what could possibly be wrong with it? There really is no problem with it.",
            "Behavior Category": "Project understanding",
            "Prediction Percentage": "90%",
            "Explanation": "Member emphasizes that the first task is simple and questions if there are any concerns, showing clarity in project understanding."
        },
        {
            "Speaker": "0303",
            "Timestamp": "13.90-14.90",
            "Content": "Don't laugh.",
            "Behavior Category": "Unrelated chat",
            "Prediction Percentage": "60%",
            "Explanation": "This is unrelated humor, not pertinent to the task, indicating a digression from the activity."
        },
        {
            "Speaker": "0303",
            "Timestamp": "21.00-24.30",
            "Content": "Okay, the first question was not too difficult and went well",
            "Behavior Category": "Acknowledgement",
            "Prediction Percentage": "100%",
            "Explanation": "The member acknowledges successful completion of the first task."
        },
        {
            "Speaker": "0302",
            "Timestamp": "34.30-40.40",
            "Content": "Whatever, just make up a name, and when the time comes, just send this.",
            "Behavior Category": "Unrelated chat",
            "Prediction Percentage": "50%",
            "Explanation": "Casual discussion about naming, which strays from focused collaboration."
        },
        {
            "Speaker": "0302",
            "Timestamp": "40.40-43.40",
            "Content": "So successful, perfect, successful.",
            "Behavior Category": "Acknowledgement",
            "Prediction Percentage": "80%",
            "Explanation": "Member acknowledges success with positive feedback."
        }
    ]
}
\end{lstlisting}
\RR{
\subsection{Student Role Analysis}
\textbf{Prompt:}}

\RR{
You are a teacher teaching programming class and will be provided some conversation files belonging to one group but in different questions period, including question number, timestamp for each speaker and corresponding conversation recorded in the context of a collaborative programming course. Your jobYour job will be to find students' planning solutions behaviors related to the question for all sentences they said. 
Here is some example: ``This question could go like this...'', ``Combine A and B'', ``Notice the function.'', etc. Specifically, you should indicate each speaker's sentences of planning solutions corresponding timestamp. Besides, there are some misunderstanding sentences that are not planning solutions: ``I'm just messing around with names.'', ``That's good. That's good.'' etc. You need to find valuable comments that contribute or drive the problem-solving process. Notice that you only need to indicate the sentences about planning solutions or providing insights.
Let me explain ``Navigator'', ``Driver'' and ``Monitor''. Each group only have three members, ``Navigator'' is the speaker who's sentence is about planning solutions, ``Driver'' is the member who responsible for coding, this role is non-changeable. I will tell you who are the ``Driver'' in each file's first line. Besides, ``Monitor'' is the role who is neither 
``Navigator'' nor ``Driver''. If the role of ``Driver'' is also planning solutions, then the role of ``Driver'' should be changed to ``Navigator'', and ``Driver'' is None for this sentence. Significantly, if a sentence is not about planning solutions, you also need to list/indicate it. The Navigator should be None, but fill out the speakers of ``Monitors'' and ``Drivers''.

\textbf{Output Format:}}
\begin{lstlisting}
{
    "Question": "QuestionX",
    "Conversations": [
        {
            "Timestamp": "XXX",
            "Content": "XXX",
            "Navigator": "XXX",
            "Other_Roles": [
                {
                    "Monitors": ["XXX"],
                    "Driver": ["XXX"]
                }
            ]
        },
        {
            "Timestamp": "XXX",
            "Content": "XXX",
            "Navigator": "XXX",
            "Other_Roles": [
                {
                    "Monitors": ["XXX"],
                    "Driver": ["XXX"]
                }
            ]
        }
    ]
}
\end{lstlisting}

\RR{
\textbf{Input Example:}}
\begin{lstlisting}
"""
    Question2 Driver: 0302
    46.70 49.40 0303 Question two, question two, continue with question two.
    58.10 60.90 0302 What is the title of the second question?
    60.90 62.20 0303, and the question is.
    75.20 77.90 0301, also in descending order.
    78.90 80.80 0302 That's okay; that should be the template for the first question.
    80.80 81.80 0301 AB merge and then
    81.80 87.30 0303 Right, right, the second question would also work, yes
    91.30 98.50 0302 Right, this one should be noted that this ijk he this exclamation point should be an introduction to the.
"""
\end{lstlisting}

\RR{
\textbf{Output Example:}}
\begin{lstlisting}
{
    "Question": "Question2",
    "Conversations": [
        {
            "Timestamp": "46.70-49.40",
            "Content": "Question two. Question two. Go on to question two.",
            "Navigator": "0303",
            "Other_Roles": [
                {
                    "Monitors": ["0301"],
                    "Drivers": ["0302"]
                }
            ]
        },
        {
            "Timestamp": "58.10-60.90",
            "Content": "What is the title of the second question?",
            "Navigator": "0302",
            "Other_Roles": [
                {
                    "Monitors": ["0301", "0303"],
                    "Drivers": ["None"]
                }
            ]
        },
        {
            "Timestamp": "60.90-62.20",
            "Content": "The question is.",
            "Navigator": "0303",
            "Other_Roles": [
                {
                    "Monitors": ["0301"],
                    "Drivers": ["0302"]
                }
            ]
        },
        {
            "Timestamp": "75.20-77.90",
            "Content": "It's also sorted from childhood to adulthood.",
            "Navigator": "0301",
            "Other_Roles": [
                {
                    "Monitors": ["0303"],
                    "Drivers": ["0302"]
                }
            ]
        },
        {
            "Timestamp": "78.90-80.80",
            "Content": "That's okay. That would be the template for the first question.",
            "Navigator": "0302",
            "Other_Roles": [
                {
                    "Monitors": ["0301", "0303"],
                    "Drivers": ["None"]
                }
            ]
        },
        {
            "Timestamp": "80.80-81.80",
            "Content": "AB is merged and then",
            "Navigator": "0301",
            "Other_Roles": [
                {
                    "Monitors": ["0303"],
                    "Drivers": ["0302"]
                }
            ]
        },
        {
            "Timestamp": "81.80-87.30",
            "Content": "Yeah, yeah, the second question would work, too. Yeah.",
            "Navigator": "0303",
            "Other_Roles": [
                {
                    "Monitors": ["0301"],
                    "Drivers": ["0302"]
                }
            ]
        },
        {
            "Timestamp": "91.30-98.50",
            "Content": "Yes, this one should be noted that this ijk he this exclamation point is supposed to be an introduction.",
            "Navigator": "0302",
            "Other_Roles": [
                {
                    "Monitors": ["0301", "0303"],
                    "Drivers": ["None"]
                }
            ]
        }
    ]
}
\end{lstlisting}

\RR{
\subsection{Teacher Scaffold Analysis}

\textbf{Prompt:}

You are a teacher teaching programming class and you will be provided a conversation file including timestamp for each speaker and corresponding content recorded in the context of a collaborative programming course. It will be your job to find instructors' assitance category based on the following scaffold.
Analyze the different levels of scaffolding used by instructors during group learning based on the following categories:
Low-control cognitive scaffolding (CS-L): The instructor raises open-ended questions that elicit group thinking without providing new information. This method encourages critical thinking but leaves the group to figure out the details.      
Medium-control cognitive scaffolding (CS-M): The instructor provides hints or clues to help groups solve cognitive problems. This method supports problem-solving but maintains some cognitive challenge.       
High-control cognitive scaffolding (CS-H): The instructor directly provides answers or demonstrates tasks (such as programming) using tools like computers. This method offers direct guidance but may limit students' independent problem-solving.      
Metacognitive scaffolding (MS): The instructor monitors and regulates the group's learning goals and collaborative processes, helping to manage group dynamics and learning strategies.

\textbf{Input Example:}}
\begin{lstlisting}
"""
    131.0 133.0 0000 What's the problem? Oh, I see. 
    134.0 135.0 0000 3 Hmm, yes, exactly. 
    139.0 142.0 0000 Hmm, I, I, where is that program? 
    143.0 196.0 0000 Then here, you need to indent first, right? Wherever you don't finish, you must add a colon, okay? Hmm, like in the if statement, you're correct about that. I, I, it should be I != K, right? Hmm, and and means "and", K should not be equal to J, right? And another one, I should also not be equal to K, right? Isn't it true that I, J, and K should all be different from each other? So if this condition is met, what happens next? Then you calculate something, calculate an S. The S would be equal to I * 100 + J * 10 + K, right? Is that correct? But don't forget to print it out, so print what? print this S, and that's it.
"""
\end{lstlisting}
\RR{
\textbf{Output Example:}}

\begin{lstlisting}
{
    "Speaker": "0000",
    "Timestamp": "131.0-133.0",
    "Content": "What is the question? Oh, oh, yes.",
    "Behavior Category": "Metacognitive scaffolding",
    "Prediction Percentage": "80%",
    "Explanation": "Requests clarification about the third question.",
},
{
    "Speaker": "0000",
    "Timestamp": "134.0-135.0",
    "Content": "Three digits, yes, yes.",
    "Behavior Category": "Metacognitive scaffolding",
    "Prediction Percentage": "90%",
    "Explanation": "Acknowledges the provided information.",
},
{
    "Speaker": "0000",
    "Timestamp": "139.0-142.0",
    "Content": "Hmm, where is your program? Oh.",
    "Behavior Category": "Metacognitive scaffolding",
    "Prediction Percentage": "90%",
    "Explanation": "Teacher responds by asking about the student's program location.",
},
{
    "Speaker": "0000",
    "Timestamp": "143.0-196.0",
    "Content": "Then you start with if here, make sure to indent, and always use a colon after each if. Yes, if i is not equal to k, right? And also k is not equal to j. I also can't be equal to k, right? So, if i, j, and k are not equal to each other, then what? You calculate something, calculate an s, which equals i times 100, plus j times 10, plus k. Right? But you have to remember to print it out. Print s, and that's it.",
    "Behavior Category": "High-control cognitive scaffolding",
    "Prediction Percentage": "100%",
    "Explanation": "Teacher provides detailed explanation and coding instructions.",
}
         
\end{lstlisting}

\end{document}
\endinput
%%
%% End of file `sample-sigconf-authordraft.tex'.

%%
%% The next two lines define the bibliography style to be used, and
%% the bibliography file.
%\bibliographystyle{ACM-Reference-Format}
%\bibliography{cpvisref}


%%
%% If your work has an appendix, this is the place to put it.
\appendix
\RR{
\newpage
\section{DETAILS OF FORMATIVE STUDY}
\subsection{Background Information of Participants}
The table~\ref{tab:info} presents information about the participants in our formative study.
}


%Lorem ipsum dolor sit amet, consectetur adipiscing elit. Morbi

\RR{
\subsection{Findings of the Semi-structured Interviews}

\begin{table*}[hb]
\renewcommand{\arraystretch}{1.5}
\centering
\begin{tabular}{m{1.5cm} m{1cm} m{1cm} m{1.5cm} m{3.5cm} m{2.5cm} m{3.5cm}}
\hline
Participants & Gender & Age & Teaching experience & Students taught & Whether the class is large-scale & Whether the students know each other before \\ \hline
E1 & Female & 45 & 15 & University Non-Computer Science Majors & Yes & No \\
E2 & Female & 30 & 5 & University Non-Computer Science Majors & Yes & No \\
TA1 & Male & 24 & 1 & University Non-Computer Science Majors & Yes & No \\
TA2 & Female & 24 & 1 & University Non-Computer Science Majors & Yes & No \\
T1 & Male & 31 & 3 & University Non-Computer Science Majors & Yes & No \\
T2 & Male & 29 & 5 & University Computer Science Majors & Yes & Yes \\
T3 & Male & 36 & 8 & University Computer Science Majors & No & Yes \\
T4 & Female & 26 & 3 & Vocational High School Non-Computer Majors & Yes & Yes \\
T5 & Female & 26 & 3 & Elementary School Students & No & Yes \\
T6 & Male & 35 & 7 & University Non-Computer Science Majors & Yes & No \\ \hline
\end{tabular}
\caption{The detailed background statistical information of the participants in the Formative Study.}
\label{tab:info}
\end{table*}

\textbf{\textit{Motivations:}}
Participants unanimously agreed that collaborative programming is an important teaching strategy for programming education. However, different teachers had varying focuses and practices regarding collaborative programming. 
Teachers specializing in computer science tended to have students complete tasks independently in foundational programming courses (e.g., Python, Java) while implementing collaborative programming for more complex projects. 
For example, \textit{T3 stated, ``I consider collaborative learning only when the programming project requires different students' creative input, innovation, exploration, and discovery of new content, or when the workload is too large for one person to complete.''}
In contrast, teachers of non-computer science courses held different views. 
E2 pointed out that for beginners who are new to programming languages, even a fundamental programming problem can be daunting.
Collaborative programming can significantly alleviate this issue and improve students' efficiency. 
Group members can work together to complete a programming task, which motivates students, allows them to learn from each other, and reduces the teaching burden (T4).
Additionally, in collaborative programming classes, teachers focus on students' cognitive skills (e.g., knowledge acquisition) and non-cognitive skills (e.g., computational thinking, problem-solving abilities). Most participants believed that these two aspects do not conflict and are part of the overall teaching goals. 
\textit{T5 stated that her course design primarily focuses on developing non-cognitive skills, particularly enhancing students' abilities during the collaboration process. }
T4 believed that cognitive skills can be supplemented with extra time or individual tutoring, but problem-solving and collaboration skills are challenging to develop outside of class. 
Furthermore, E1, T4, and T3 mentioned that quantifying the improvement of students' non-cognitive skills is very challenging.
On one hand, teachers and assistants are busy answering students' questions, making it challenging to observe students' performance. 
On the other hand, non-cognitive skills are not as quickly assessed through tests as knowledge-based content.

Despite the unanimous recognition of the importance of collaborative programming, participants shared several challenges they faced in practice, summarized as follows:

\textit{\textbf{Group Formation:}}
For group size, most participants preferred group sizes of 2-3 students, noting that larger groups (up to 5 members) could decrease the quality of collaboration and make it difficult for students to find their roles. T4 favored pair programming but often had to form groups of three due to large class sizes. T1 and T3 typically recommended groups of 4-5 students, given the complexity and workload of their programming projects, which smaller groups might struggle to complete.
For the group formation method, participants had various approaches to forming groups. T2, T3, T4, and T5, whose students come from fixed classes, generally allowed students to create their own groups. In contrast, T6, T1, E1, and E2 taught general education courses open to all students across the university. These students came from different majors and did not know each other, making self-selection difficult. E2 conducted preliminary assessments of students and formed balanced groups based on their prior knowledge. However, determining how to create groups remains a significant challenge, as different criteria can lead to varying outcomes. Arbitrarily formed groups might result in imbalanced teams, where students' abilities are insufficient to complete the tasks.

\textit{\textbf{Evaluating the Learning Objectives:}}
Evaluating whether the teacher has achieved the learning objectives is an essential criterion for assessing the quality of collaborative programming courses. All participants stated that they generally meet the learning objectives and sometimes even exceed expectations. T4 mentioned that at the start of the collaborative course, she assigned a group leader to each group to take on a leadership role. While she initially spent a significant amount of time guiding the group leaders and facilitating student collaboration, by the second half of the semester, the group leaders were typically able to function as ``little teachers,'' helping group members complete tasks. This allowed the teacher to free up time and energy for other responsibilities.



%Etiam commodo feugiat nisl pulvinar pellentesque. Etiam auctor sodales
\subsection{Results of Rating and Ranking in Formative Study}
Table~\ref{table:feature} presents the metrics participants rated and ranked for the collaborative programming visual analytics system, as discussed in Section 3.2.}
\begin{table}[htbp]
\renewcommand{\arraystretch}{1.5}
\begin{tabular}{p{8cm}}
\hline
\textbf{Features}  \\ \hline
F1: Change in student engagement  \\
F2: Change in student behavior \\
F3: Analysis of student emotional state \\
F4: Student background statistics \\
F5: Analysis of student programming habits \\
F6: Change in role distribution \\
F7: Role assignment suggestions within the group \\
F8: Group performance overview \\
F9: Comparison of group performances \\
F10: Analysis of similarity in group performance  \\
F11: Generation of personalized suggestions \\
F12: View of raw data \\
F13: Programming task evaluation \\
F14: Visualization of problem-solving path \\
F15: Real-time feedback feature \\
F16: Students performance overview \\ \hline
\end{tabular}
\caption{Features of Collaborative Programming Visualization Systems Extracted from Semi-structured Interviews.}
\label{table:feature}
\end{table}
\RR{The raw rating data is displayed in Table~\ref{table:rating}.
The results show that participants rated most of the features mentioned highly, except for F3, F4, F5, F7, and F15.}
\begin{table*}[htbp]
\renewcommand{\arraystretch}{1.5}
\centering
\small
\begin{tabular}{l c c c c c c c c c c c c}
\hline
\textbf{Feature} & \textbf{T6} & \textbf{T5} & \textbf{E1} & \textbf{E2} & \textbf{TA1} & \textbf{TA2} & \textbf{T1} & \textbf{T2} & \textbf{T3} & \textbf{T4} & \textbf{Mean} & \textbf{SD} \\ \hline
F1 & 5 & 6 & 5 & 6 & 7 & 6 & 7 & 5 & 6 & 7 & 6 & 0.82 \\
F2 & 7 & 6 & 5 & 5 & 4 & 7 & 6 & 6 & 6 & 6 & 5.8 & 0.92 \\
F3 & 3 & 4 & 3 & 4 & 4 & 5 & 6 & 4 & 3 & 4 & 4 & 0.94 \\
F4 & 2 & 4 & 2 & 2 & 2 & 3 & 3 & 4 & 6 & 5 & 3.3 & 1.42 \\
F5 & 4 & 3 & 6 & 5 & 4 & 3 & 4 & 4 & 3 & 5 & 4.1 & 0.99 \\
F6 & 4 & 6 & 7 & 3 & 7 & 5 & 4 & 6 & 6 & 7 & 5.5 & 1.43 \\
F7 & 2 & 4 & 3 & 4 & 2 & 4 & 4 & 3 & 4 & 3 & 3.3 & 0.82 \\
F8 & 7 & 6 & 7 & 7 & 7 & 7 & 7 & 7 & 7 & 7 & 6.9 & 0.32 \\
F9 & 6 & 6 & 5 & 7 & 5 & 4 & 7 & 6 & 5 & 7 & 5.8 & 1.03 \\
F10 & 7 & 6 & 5 & 6 & 5 & 5 & 5 & 5 & 6 & 7 & 5.7 & 0.82 \\
F11 & 7 & 6 & 5 & 5 & 5 & 5 & 6 & 7 & 7 & 7 & 6 & 0.94 \\
F12 & 3 & 5 & 5 & 5 & 5 & 5 & 6 & 7 & 4 & 3 & 4.8 & 1.23 \\
F13 & 7 & 6 & 7 & 7 & 7 & 7 & 7 & 7 & 6 & 6 & 6.7 & 0.48 \\
F14 & 6 & 7 & 6 & 6 & 6 & 5 & 4 & 5 & 6 & 6 & 5.7 & 0.82 \\
F15 & 2 & 3 & 4 & 3 & 2 & 5 & 6 & 7 & 7 & 4 & 4.3 & 1.89 \\
F16 & 7 & 7 & 6 & 6 & 5 & 7 & 7 & 5 & 6 & 6 & 6.2 & 0.79 \\ \hline
\end{tabular}
\caption{Mean and SD Results of the Ratings for F1-F16 by Ten Participants.}
\label{table:rating}
\end{table*}
\RR{The raw ranking data is displayed in Table~\ref{table:rank}.
Each feature's average ranking is listed (Fig.~\ref{fig:rank}), with lower numbers indicating that the evaluators more highly value the feature. For example, the average ranking of ``Students Performance Overview'' is 4.8, making it one of the most valued features. On the other hand, ``Role assignment suggestions within the group'' has an average ranking of 13.4, which is ranked lower by the evaluators.}



\begin{table*}[htbp]

\renewcommand{\arraystretch}{1.5}
\centering
\small
\begin{tabular}{l c c c c c c c c c c c c}
\hline
\textbf{Feature} & \textbf{T6} & \textbf{T5} & \textbf{E1} & \textbf{E2} & \textbf{TA1} & \textbf{TA2} & \textbf{T1} & \textbf{T2} & \textbf{T3} & \textbf{T4} & \textbf{Mean} & \textbf{SD} \\ \hline
F1 & 7 & 1 & 7 & 9 & 12 & 2 & 1 & 11 & 1 & 4 & 5.5 & 4.28 \\
F2 & 6 & 3 & 6 & 10 & 11 & 3 & 3 & 10 & 3 & 5 & 6 & 3.23 \\
F3 & 12 & 8 & 16 & 14 & 16 & 12 & 13 & 9 & 4 & 13 & 11.7 & 3.74 \\
F4 & 16 & 15 & 15 & 13 & 14 & 13 & 12 & 6 & 6 & 14 & 12.4 & 3.57 \\
F5 & 14 & 13 & 14 & 15 & 15 & 15 & 14 & 12 & 2 & 15 & 12.9 & 3.96 \\
F6 & 7 & 2 & 5 & 8 & 10 & 4 & 11 & 8 & 7 & 6 & 6.8 & 2.70 \\
F7 & 15 & 14 & 13 & 16 & 13 & 14 & 15 & 13 & 5 & 16 & 13.4 & 3.17 \\
F8 & 8 & 6 & 4 & 1 & 3 & 11 & 8 & 7 & 8 & 2 & 5.8 & 3.19 \\
F9 & 5 & 5 & 12 & 7 & 9 & 6 & 9 & 14 & 16 & 7 & 9 & 3.83 \\
F10 & 10 & 4 & 3 & 3 & 4 & 5 & 10 & 5 & 10 & 8 & 6.2 & 2.97 \\
F11 & 9 & 7 & 2 & 4 & 5 & 7 & 7 & 4 & 9 & 9 & 6.3 & 2.45 \\
F12 & 11 & 10 & 11 & 11 & 6 & 10 & 16 & 15 & 15 & 10 & 11.5 & 3.03 \\
F13 & 3 & 9 & 1 & 5 & 1 & 8 & 6 & 16 & 11 & 3 & 6.3 & 4.79 \\
F14 & 4 & 11 & 8 & 6 & 7 & 9 & 5 & 3 & 14 & 11 & 7.8 & 3.49 \\
F15 & 13 & 16 & 10 & 12 & 8 & 16 & 2 & 1 & 12 & 12 & 10.2 & 5.18 \\
F16 & 2 & 12 & 9 & 2 & 2 & 1 & 4 & 2 & 13 & 1 & 4.8 & 4.69 \\
\hline
\end{tabular}
\caption{Mean and SD Results of the Rankings for F1-F16 by Ten Participants.}
\label{table:rank}
\end{table*}

\RR{\textbf{Spearman Correlation Matrix:}
To assess the consistency between rankings, we used Spearman's Rank Correlation to calculate the correlation of rankings between each participant., The Spearman correlation matrix shows the consistency of the rankings between participants (Fig.~\ref{fig:myplot}). The correlation coefficient ranges from -1 to 1:
- A value closer to 1 indicates that the two evaluators' rankings are more consistent.
- A value closer to -1 suggests that the rankings are more inconsistent.
- A value near 0 implies no significant correlation between the two evaluators' rankings. For example, the correlation between participant 0 and participant 9 is 0.809, indicating
relatively consistent rankings. In contrast, the correlation between evaluator seven and evaluator 8 is - 0.150, suggesting their rankings differ considerably.}
\begin{figure}[htbp]
	\centering
	\includegraphics[width=0.8\linewidth]{rank.png}
	\caption{Ranking Results of F1-F16.}
        \Description{Ranking Results of F1-F16.}
	\label{fig:rank}
\end{figure}


\begin{figure}[htbp]
    \centering
    \includegraphics[width=0.8\linewidth]{myplot.png}
    \caption{Spearman's Rank Correlation Results Between Participants.}
    \Description{Spearman's Rank Correlation Results Between Participants.}
    \label{fig:myplot}
\end{figure}

\RR{\textbf{Friedman Test Results:}
We conducted a Friedman test to assess whether there are significant differences in rankings among different participants. The Friedman test compares the ranking data of multiple related samples to determine whether there are significant differences in rankings under different conditions. The test statistic is 60.796, and the p-value is 1.84e-07. This very small p-value (< 0.05) indicates that:

- There are significant differences in the rankings of different features. The evaluators' ratings are not random or similar but show clear preferences for specific features over others.

In conclusion, after comprehensively analyzing the results for each feature, our system focuses on implementing features \textbf{F1, F2, F6, F8-14, and F16}.
}
\newpage
\RR{
\section{PROMPTS}
Below are the prompts used to label data.
\subsection{Python Code Evaluation}
\textbf{Prompt:}

I would like you to play the role of a teacher who teaches a Python programming class, and you will be provided a question statement and a Python code, which is the student's answer to the question. 
Regarding the Python code, you need to accomplish two tasks. 
Here are the scoring criteria. Please mark each point according to the scoring criteria and explain the reason. Meanwhile, you should give your final score.
If the score of each aspect is not 5, please point out the demerits of the code. Also, note that you don't need to give the advised code. }


\begin{lstlisting}
criteria = {
    "Problem-solving Approach (5%)": {
        "Excellent (5)": "Shows an effective problem-solving approach, effectively addressing key challenges in the task",
        "Good (4)": "Shows a good problem-solving approach, with clear attempts to address challenges in the task",
        "Fair (3)": "Shows some effort in problem-solving, but lacks clarity or effectiveness in addressing challenges in the task",
        "Poor (2)": "Shows limited problem-solving efforts, with unclear or ineffective attempts to address challenges in the task",
        "Bad (1)": "Demonstrates no effective problem-solving approach, unable to address the task"
    },
    "Code Integrity (35%)": {
        "Excellent (5)": "The code is well-structured, organized, readable, and it effectively implements the desired functionality",
        "Good (4)": "The code structure and organization are sufficient, and it implements the function, though readability could be enhanced",
        "Fair (3)": "The code implements 80% of the function but lacks consistency or clarity",
        "Poor (2)": "The code implements 60% of the function but lacks consistency or clarity",
        "Bad (1)": "The code failed to implement 40% of the function and lacks consistency or clarity"
    },
    "Code Accuracy (35%)": {
        "Excellent (5)": "Code exhibits an excellent level of accuracy, producing correct results under various conditions and inputs",
        "Good (4)": "Code exhibits good accuracy, producing correct results under most conditions and inputs",
        "Fair (3)": "Code accuracy is acceptable, but there may be some occasional errors or inconsistencies in the results",
        "Poor (2)": "Code accuracy is poor, with frequent errors or inconsistencies in the results",
        "Bad (1)": "Code accuracy is severely lacking, with a high frequency of errors or inconsistencies in the results"
    },
    "Algorithm Innovation (25%)": {
        "Excellent (5)": "Shows innovative and creative approaches, showcasing originality and ingenuity in algorithm design and implementation",
        "Good (4)": "Shows basic innovation in algorithm design and implementation, but lacks originality or significant creativity",
        "Fair (3)": "Demonstrates limited innovation in algorithm design and implementation, with little originality or creativity demonstrated",
        "Poor (2)": "Shows little innovation in algorithm design and implementation, with no originality or creativity demonstrated",
        "Bad (1)": "Shows no innovation in algorithm design and implementation; only capable of imitating from simple examples"
    }
}
\end{lstlisting} 



\RR{
\textbf{Task:} Task 1, you need to extract the key ideas of the code. Give me a paragraph outlining the thought process of this code. Task 2, Grade the code according to the scoring criteria, and don't round the final score.

\textbf{Input Example:}
}
\begin{lstlisting} 
Question: Existing list a=[49, 38 , 65 , 97 , 76 , 13 , 27 , 55 , 4]
Requirement: Try to write a program in Python that sorts the data elements in a from smallest to largest and prints out the new sorted list a.
Answer:
a=[49,38,65,97,76,13,27,55,4]
a.sort()
print(a)       
\end{lstlisting} 
\RR{
\textbf{Output Example:}}
\begin{lstlisting}
{
    "Key ideas": "The provided code aims to sort a given list of integers in ascending order. The process follows these key steps:\n\t1. Define a list named `a` containing a series of integers.\n\t2. Call the `sort()` method on the list `a`, which sorts the list in place in ascending order.\n\t3. Print the sorted list to display the numbers from smallest to largest.\nThis approach efficiently utilizes Python's built-in sorting capabilities and achieves the task as specified in the question.",
    "Score": "4.55 / 5",
    "Details": [
        {
            "Problem-solving Approach (5%)": {
                "Score": "Excellent (5)",
                "Explanation": "The code presents a clear and effective approach to sorting the list using Python's built-in functionality, which is suitable for the problem at hand."
            },
            "Demerits": null
        },
        {
            "Code Integrity (35%)": {
                "Score": "Excellent (5)",
                "Explanation": "The code is well-structured, organized, and highly readable. It makes effective use of the `sort()` method while maintaining clarity. There are no unnecessary complexities in the implementation."
            },
            "Demerits": null
        },
        {
            "Code Accuracy (35%)": {
                "Score": "Excellent (5)",
                "Explanation": "The code accurately sorts the list `a` and produces the correct output without any errors. The use of the `sort()` method guarantees that the order will be correct."
            },
            "Demerits": null
        },
        {
            "Algorithm Innovation (25%)": {
                "Score": "Fair (3)",
                "Explanation": "While the implementation is effective and clear, it does not demonstrate any innovative or creative algorithmic approach since it relies on the built-in `sort()` method without any modifications or enhancements."
            },
            "Demerits": "There is a lack of originality in the sorting technique, as it could have showcased a custom sorting algorithm, which would demonstrate deeper understanding and application of algorithms."
        }
    ]
}
\end{lstlisting} 
\RR{\subsection{Collaborative Behavioral Analysis}
\textbf{Prompt:}}
\RR{
You are a teacher teaching programming class and will be provided some conversation files belonging to one group but in different questions period, including question number, timestamp for each speaker and corresponding conversation recorded in the context of a collaborative programming course. It will be your job to find students' communication behaviors for all sentences they said and record them in a JSON format file, besides, show the prediction percentage of your response after each behavior by using the scaffold.
Notice again, you should show all the sentences in the JSON file. If you can't classify some of them into category, indicate them into one category with the most similar meaning and point out the percentage. 
Specifically, each question is encircled by a ``{ },'' and you need to show details information in it.
%The out format of your each response should be like, each one should strictly follow the format:


\textbf{Output Format:}}
\begin{lstlisting}
{
    "Question": "Question Number",
    "Conversations": [
        {
            "Speaker": "",
            "Timestamp": "",
            "Content": "",
            "Behavior Category": "",
            "Prediction Percentage": "",
            "Explanation": ""
        },
        {
            "Speaker": "",
            "Timestamp": "",
            "Content": "",
            "Behavior Category": "",
            "Prediction Percentage": "",
            "Explanation": ""
        }
    ]
}
\end{lstlisting}

\RR{
\textbf{Input Example:}}
\begin{lstlisting}
{
    "role": "user",
    "content": """
        Question1
        1.00 2.90 0303 This one is better done
        2.90 13.90 0302 This one compares This first question is so good that it doesn't need doesn't need to be discussed too much, what else could be wrong with it? There's really no problem with it.
        13.90 14.90 0303 Don't don't laugh eh?
        21.00 24.30 0303 Good, the first question is not too difficult, successfully completed
        34.30 40.40 0302 Whatever, just mess around with the name, just send this when the time comes.
        40.40 43.40 0302 Good success, perfect, success.
         """
}
\end{lstlisting}


\RR{
\textbf{Output Example:}}
\begin{lstlisting}
{
    "Question": "Question1",
    "Conversations": [
        {
            "Speaker": "0303",
            "Timestamp": "1.00-2.90",
            "Content": "This is better done",
            "Behavior Category": "Project understanding",
            "Prediction Percentage": "100%",
            "Explanation": "Member expresses that the task is easy, indicating understanding of the project."
        },
        {
            "Speaker": "0302",
            "Timestamp": "2.90-13.90",
            "Content": "This comparison of this first question is so good that it doesn't need to be without too much discussion, what could possibly be wrong with it? There really is no problem with it.",
            "Behavior Category": "Project understanding",
            "Prediction Percentage": "90%",
            "Explanation": "Member emphasizes that the first task is simple and questions if there are any concerns, showing clarity in project understanding."
        },
        {
            "Speaker": "0303",
            "Timestamp": "13.90-14.90",
            "Content": "Don't laugh.",
            "Behavior Category": "Unrelated chat",
            "Prediction Percentage": "60%",
            "Explanation": "This is unrelated humor, not pertinent to the task, indicating a digression from the activity."
        },
        {
            "Speaker": "0303",
            "Timestamp": "21.00-24.30",
            "Content": "Okay, the first question was not too difficult and went well",
            "Behavior Category": "Acknowledgement",
            "Prediction Percentage": "100%",
            "Explanation": "The member acknowledges successful completion of the first task."
        },
        {
            "Speaker": "0302",
            "Timestamp": "34.30-40.40",
            "Content": "Whatever, just make up a name, and when the time comes, just send this.",
            "Behavior Category": "Unrelated chat",
            "Prediction Percentage": "50%",
            "Explanation": "Casual discussion about naming, which strays from focused collaboration."
        },
        {
            "Speaker": "0302",
            "Timestamp": "40.40-43.40",
            "Content": "So successful, perfect, successful.",
            "Behavior Category": "Acknowledgement",
            "Prediction Percentage": "80%",
            "Explanation": "Member acknowledges success with positive feedback."
        }
    ]
}
\end{lstlisting}
\RR{
\subsection{Student Role Analysis}
\textbf{Prompt:}}

\RR{
You are a teacher teaching programming class and will be provided some conversation files belonging to one group but in different questions period, including question number, timestamp for each speaker and corresponding conversation recorded in the context of a collaborative programming course. Your jobYour job will be to find students' planning solutions behaviors related to the question for all sentences they said. 
Here is some example: ``This question could go like this...'', ``Combine A and B'', ``Notice the function.'', etc. Specifically, you should indicate each speaker's sentences of planning solutions corresponding timestamp. Besides, there are some misunderstanding sentences that are not planning solutions: ``I'm just messing around with names.'', ``That's good. That's good.'' etc. You need to find valuable comments that contribute or drive the problem-solving process. Notice that you only need to indicate the sentences about planning solutions or providing insights.
Let me explain ``Navigator'', ``Driver'' and ``Monitor''. Each group only have three members, ``Navigator'' is the speaker who's sentence is about planning solutions, ``Driver'' is the member who responsible for coding, this role is non-changeable. I will tell you who are the ``Driver'' in each file's first line. Besides, ``Monitor'' is the role who is neither 
``Navigator'' nor ``Driver''. If the role of ``Driver'' is also planning solutions, then the role of ``Driver'' should be changed to ``Navigator'', and ``Driver'' is None for this sentence. Significantly, if a sentence is not about planning solutions, you also need to list/indicate it. The Navigator should be None, but fill out the speakers of ``Monitors'' and ``Drivers''.

\textbf{Output Format:}}
\begin{lstlisting}
{
    "Question": "QuestionX",
    "Conversations": [
        {
            "Timestamp": "XXX",
            "Content": "XXX",
            "Navigator": "XXX",
            "Other_Roles": [
                {
                    "Monitors": ["XXX"],
                    "Driver": ["XXX"]
                }
            ]
        },
        {
            "Timestamp": "XXX",
            "Content": "XXX",
            "Navigator": "XXX",
            "Other_Roles": [
                {
                    "Monitors": ["XXX"],
                    "Driver": ["XXX"]
                }
            ]
        }
    ]
}
\end{lstlisting}

\RR{
\textbf{Input Example:}}
\begin{lstlisting}
"""
    Question2 Driver: 0302
    46.70 49.40 0303 Question two, question two, continue with question two.
    58.10 60.90 0302 What is the title of the second question?
    60.90 62.20 0303, and the question is.
    75.20 77.90 0301, also in descending order.
    78.90 80.80 0302 That's okay; that should be the template for the first question.
    80.80 81.80 0301 AB merge and then
    81.80 87.30 0303 Right, right, the second question would also work, yes
    91.30 98.50 0302 Right, this one should be noted that this ijk he this exclamation point should be an introduction to the.
"""
\end{lstlisting}

\RR{
\textbf{Output Example:}}
\begin{lstlisting}
{
    "Question": "Question2",
    "Conversations": [
        {
            "Timestamp": "46.70-49.40",
            "Content": "Question two. Question two. Go on to question two.",
            "Navigator": "0303",
            "Other_Roles": [
                {
                    "Monitors": ["0301"],
                    "Drivers": ["0302"]
                }
            ]
        },
        {
            "Timestamp": "58.10-60.90",
            "Content": "What is the title of the second question?",
            "Navigator": "0302",
            "Other_Roles": [
                {
                    "Monitors": ["0301", "0303"],
                    "Drivers": ["None"]
                }
            ]
        },
        {
            "Timestamp": "60.90-62.20",
            "Content": "The question is.",
            "Navigator": "0303",
            "Other_Roles": [
                {
                    "Monitors": ["0301"],
                    "Drivers": ["0302"]
                }
            ]
        },
        {
            "Timestamp": "75.20-77.90",
            "Content": "It's also sorted from childhood to adulthood.",
            "Navigator": "0301",
            "Other_Roles": [
                {
                    "Monitors": ["0303"],
                    "Drivers": ["0302"]
                }
            ]
        },
        {
            "Timestamp": "78.90-80.80",
            "Content": "That's okay. That would be the template for the first question.",
            "Navigator": "0302",
            "Other_Roles": [
                {
                    "Monitors": ["0301", "0303"],
                    "Drivers": ["None"]
                }
            ]
        },
        {
            "Timestamp": "80.80-81.80",
            "Content": "AB is merged and then",
            "Navigator": "0301",
            "Other_Roles": [
                {
                    "Monitors": ["0303"],
                    "Drivers": ["0302"]
                }
            ]
        },
        {
            "Timestamp": "81.80-87.30",
            "Content": "Yeah, yeah, the second question would work, too. Yeah.",
            "Navigator": "0303",
            "Other_Roles": [
                {
                    "Monitors": ["0301"],
                    "Drivers": ["0302"]
                }
            ]
        },
        {
            "Timestamp": "91.30-98.50",
            "Content": "Yes, this one should be noted that this ijk he this exclamation point is supposed to be an introduction.",
            "Navigator": "0302",
            "Other_Roles": [
                {
                    "Monitors": ["0301", "0303"],
                    "Drivers": ["None"]
                }
            ]
        }
    ]
}
\end{lstlisting}

\RR{
\subsection{Teacher Scaffold Analysis}

\textbf{Prompt:}

You are a teacher teaching programming class and you will be provided a conversation file including timestamp for each speaker and corresponding content recorded in the context of a collaborative programming course. It will be your job to find instructors' assitance category based on the following scaffold.
Analyze the different levels of scaffolding used by instructors during group learning based on the following categories:
Low-control cognitive scaffolding (CS-L): The instructor raises open-ended questions that elicit group thinking without providing new information. This method encourages critical thinking but leaves the group to figure out the details.      
Medium-control cognitive scaffolding (CS-M): The instructor provides hints or clues to help groups solve cognitive problems. This method supports problem-solving but maintains some cognitive challenge.       
High-control cognitive scaffolding (CS-H): The instructor directly provides answers or demonstrates tasks (such as programming) using tools like computers. This method offers direct guidance but may limit students' independent problem-solving.      
Metacognitive scaffolding (MS): The instructor monitors and regulates the group's learning goals and collaborative processes, helping to manage group dynamics and learning strategies.

\textbf{Input Example:}}
\begin{lstlisting}
"""
    131.0 133.0 0000 What's the problem? Oh, I see. 
    134.0 135.0 0000 3 Hmm, yes, exactly. 
    139.0 142.0 0000 Hmm, I, I, where is that program? 
    143.0 196.0 0000 Then here, you need to indent first, right? Wherever you don't finish, you must add a colon, okay? Hmm, like in the if statement, you're correct about that. I, I, it should be I != K, right? Hmm, and and means "and", K should not be equal to J, right? And another one, I should also not be equal to K, right? Isn't it true that I, J, and K should all be different from each other? So if this condition is met, what happens next? Then you calculate something, calculate an S. The S would be equal to I * 100 + J * 10 + K, right? Is that correct? But don't forget to print it out, so print what? print this S, and that's it.
"""
\end{lstlisting}
\RR{
\textbf{Output Example:}}

\begin{lstlisting}
{
    "Speaker": "0000",
    "Timestamp": "131.0-133.0",
    "Content": "What is the question? Oh, oh, yes.",
    "Behavior Category": "Metacognitive scaffolding",
    "Prediction Percentage": "80%",
    "Explanation": "Requests clarification about the third question.",
},
{
    "Speaker": "0000",
    "Timestamp": "134.0-135.0",
    "Content": "Three digits, yes, yes.",
    "Behavior Category": "Metacognitive scaffolding",
    "Prediction Percentage": "90%",
    "Explanation": "Acknowledges the provided information.",
},
{
    "Speaker": "0000",
    "Timestamp": "139.0-142.0",
    "Content": "Hmm, where is your program? Oh.",
    "Behavior Category": "Metacognitive scaffolding",
    "Prediction Percentage": "90%",
    "Explanation": "Teacher responds by asking about the student's program location.",
},
{
    "Speaker": "0000",
    "Timestamp": "143.0-196.0",
    "Content": "Then you start with if here, make sure to indent, and always use a colon after each if. Yes, if i is not equal to k, right? And also k is not equal to j. I also can't be equal to k, right? So, if i, j, and k are not equal to each other, then what? You calculate something, calculate an s, which equals i times 100, plus j times 10, plus k. Right? But you have to remember to print it out. Print s, and that's it.",
    "Behavior Category": "High-control cognitive scaffolding",
    "Prediction Percentage": "100%",
    "Explanation": "Teacher provides detailed explanation and coding instructions.",
}
         
\end{lstlisting}

\end{document}
\endinput
%%
%% End of file `sample-sigconf-authordraft.tex'.
