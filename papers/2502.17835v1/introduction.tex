\section{Introduction} 
\label{sec:introduction}
Computing and programming have become integral components of the foundational curricula in many countries, spurring the rapid expansion of introductory programming courses, \RR{which are frequently taught in large-scale classrooms with hundreds of students~\cite{zhang2024cflow,hawlitschek2023empirical}.
To manage these large classes,} instructors employ diverse teaching methods, such as collaborative learning and flipped classrooms, to coordinate better and manage students~\cite{yan2024evidence}.  
In computer-supported collaborative learning (CSCL), \RR{students collaborate in groups, engaging in communication and interactions to perform high-level cognitive tasks}, such as learning programming languages, solving programming problems, and enhancing logical reasoning and critical thinking skills~\cite{ouyang2022applying}.

\RR{
Although research highlights the benefits of CSCL\cite{ouyang2022applying}, instructors still face challenges in practice. In large collaborative programming classes, they must balance guiding multiple groups with fostering student engagement. This makes it difficult to closely monitor group dynamics and individual performance, and to offer real-time feedback during discussions\cite{mangaroska2022exploring}. Additionally, assessments often focus on final solutions rather than the collaborative programming process~\cite{schulz2023exploring,yan2019pensieve}, limiting instructors' ability to provide feedback on techniques or evaluate individual engagement, thus complicating the assessment of collaborative programming~\cite{schulz2023exploring}.
}
Most current research focuses on evaluating general group discussions~\cite{niemantsverdriet2017recurring}, analyzing multi-person dialogue to extract discussion topics and participant behavior patterns. These studies propose tools such as context-based narrative meeting dashboards~\cite{samrose2021meetingcoach} and visual discourse analysis reports~\cite{sevastjanova2021visinreport}, significantly improving the efficiency of group discussion evaluation. However, evaluating collaborative programming in large classrooms requires more comprehensive and multi-level analysis.

Collaborative programming is a dynamic process involving the gradual evolution of problem-solving, meaning-making, and knowledge construction~\cite{ma2022detecting}. 
\RR{Within collaborative programming groups, students often switch between roles, such as the ``Driver,'' who writes the code; the ``Navigator,'' who provides ideas; and the ``Monitor,'' who checks each line of code for issues~\cite{lewis2011pair}. These role dynamics and group interactions significantly impact students' engagement.}
Moreover, effective evaluation requires considering not only students' role changes but also group-level dynamics and collaboration. Therefore, studying students' engagement, behavior patterns, and collaborative problem-solving skills from a process-oriented perspective is essential for understanding the complexities of collaborative programming.
\RR{To address this pressing need, we aim to develop a visual analysis system that harnesses multimodal data and advanced visualization techniques, empowering instructors with comprehensive insights into collaborative programming and enabling them to evaluate group and individual student performance with greater efficiency.}

However, developing such a system involves overcoming three key challenges. 
First, extracting potential problem-solving patterns from unstructured data and analyzing their dynamic changes over time is challenging. In addition to semantic analysis of speech, \RR{raw data such as screen recordings and behavioral videos in programming education complicate the analysis, while students' diverse and constantly changing roles in collaborative processes make it difficult to capture and identify role transitions accurately.
Second, the visual design must balance intuitive representation and complex data encoding (e.g., multidimensional, temporal, and comparative) while presenting individual students and groups cohesively rather than in isolation.}  
Third, instructors need to combine individual student changes with group patterns to gain deep insights into the dynamics of collaborative problem-solving.
\RR{To address the first challenge, we leverage large language models (LLMs) for fine-grained semantic analysis and error detection of unstructured data, enabling the extraction of students' dynamically changing roles while reducing the time required for manual annotation~\cite{hou2024codetailor}. }
Additionally, we construct a learning analytics framework to evaluate the performance of groups and individuals in collaborative programming.  
For the second challenge, we design novel flower-based visual metaphors to present student and group performance intuitively. We also use timeline visualizations to reveal dynamic changes in key features and patterns.  
For the third challenge, we introduce \textit{CPVis}, an interactive visual analytics system that provides instructors with perspectives ranging from overviews to detailed insights, enabling the exploration of multidimensional learning dynamics in collaborative programming. 

The contributions of this study are summarized as follows:  

\begin{itemize}
	\item We propose an innovative set of visual metaphors to create profiles for students and groups, facilitating the evaluation and comparison of different groups and individual students in collaborative programming.


	\item We develop \textit{CPVis}, an interactive visual analytics system designed for instructors to assess students' problem-solving processes and collaborative programming dynamics, \RR{facilitating quantitative assessment through the use of LLMs }and supporting the exploration of collaborative programming performance from the holistic (group level) to the detailed (individual level).
    
    
	\item We conduct quantitative, case, and user studies, demonstrating that \textit{CPVis}' multimodal learning analytics are intuitive and effective, enabling instructors to quickly and efficiently evaluate collaborative programming. 


 
\end{itemize}














