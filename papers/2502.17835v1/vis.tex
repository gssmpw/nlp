


\section{System Design} 
\label{sec:vis}

In this session, we introduce \textit{CPVis} (Fig.~\ref{fig:teaser}), a web-based visual analytics system to assist instructors in evaluating collaborative programming.
\subsection{System Overview}
\textit{CPVis} is a comprehensive system supporting multi-level, progressive analysis, from group-level interactions to individual student performance. Instructors can select specific groups for focused analysis, such as comparative evaluations (R2). Additionally, \textit{CPVis} offers a drill-down feature, enabling an overview of student collaboration and detailed insights into individual performance.
\textit{CPVis} includes four main components:
\textbf{Initial Selection}, instructors can select groups using the lasso tool or search function in the ``Group Overview'' view (Fig.\ref{fig:teaser}-A1). The system displays an overview of the selected groups (Fig.\ref{fig:teaser}-A2), compares similar and different groups (Fig.\ref{fig:teaser}-A3), and synchronizes updates across views (Fig.\ref{fig:teaser}-B).
\textbf{Drill-Down Analysis}, instructors can examine the group's code (Fig.\ref{fig:teaser}-B1) and analyze problem-solving approaches in the Content View. The interaction pattern panel (Fig.\ref{fig:student}-B2b) reveals behavior patterns, while the timeline panel (Fig.\ref{fig:teaser}-B3a) shows activity sequences. The student overview panel (Fig.\ref{fig:student}-B2a) compares individual performance across the class, and the timeline panel (Fig.~\ref{fig:teaser}-B3) highlights engagement and role changes.
\textbf{Multi-Level Interaction}, the system's layered visualization allows instructors to explore and analyze both group and individual behavior, enabling precise assessment of the collaboration process. It also supports side-by-side group comparisons.
\textbf{Detailed Review}, the detail view (Fig.~\ref{fig:teaser}-C) provides original discussion videos and transcripts, enabling in-depth analysis of student conversations and problem-solving processes.
\begin{figure}
	\centering
	\includegraphics[width=1\linewidth]{design.pdf}
	\caption{The iterative design of student glyphs: 1 represents cognitive engagement, and 2 represents behavioral engagement.}
        \Description{The iterative design of student glyphs: 1 represents cognitive engagement, and 2 represents behavioral engagement.}
	\label{fig:designl}
\end{figure}
\subsection{Visual Design}
\begin{figure*}
	\centering
	\includegraphics[width=1\linewidth]{encoding.pdf}
	\caption{The flower metaphor in \textit{CPVis}, along with its visual encoding, color coding, and some samples.}
        \Description{The flower metaphor in \textit{CPVis}, along with its visual encoding, color coding, and some samples.}
	\label{fig:encoding1}
\end{figure*}

We iterated continuously during the glyph design process to optimize the visual representation. Initially, we used a star-shaped design (e.g., radar charts), as shown in Fig.~\ref{fig:designl}. While radar charts effectively displayed behavioral and cognitive engagement, dividing the chart into sections for individual tasks introduced an unnecessary dimension (shape size) that was meaningless and prone to misinterpretation. 
We then shifted to a circular design, encoding behavioral engagement as the radius of a sector and cognitive engagement as the color of the outer ring. However, this design had a significant flaw: color mapping was less intuitive than size mapping, and using color saturation as a visual channel lacked precision. Additionally, both designs struggled to combine individual students into group glyphs intuitively.
\RR{Inspired by previous research~\cite{tausch2014groupgarden,tausch2016comparison,xiong1999peoplegarden}, we introduced a visual design inspired by the flower metaphor, where students are represented as flowers (Fig.~\ref{fig:encoding1}).
The size of the petals represented behavioral engagement, while the size of the stamen indicated cognitive engagement. Three colors were used to represent three different roles, and the varying colors of the leaves symbolized different levels of teacher scaffolding. The number of butterflies reflected the level of collaborative problem-solving ability. As a result, the overall group glyph naturally took the form of a bouquet (R4).}
This approach resolved the issue of merging individual students into group glyphs while enhancing the design's readability and intuitiveness (R1). 
\RR{The final design struck a balance between aesthetics and functionality, effectively conveying the performance of individual students and groups, allowing users to quickly compare similar or different groups (R2).}


\subsection{Filter View}
The Filter View (Fig.~\ref{fig:teaser}-A) serves as the starting point for analysis, featuring an interactive projection panel (Fig.~\ref{fig:teaser}-A1) and a similarity panel (Fig.~\ref{fig:teaser}-A3) to help users filter and explore groups of interest. 
The projection panel displays the distribution of groups in a 2D space to reveal clustering patterns and outliers. We apply the t-SNE algorithm to maximize separation between dissimilar groups, creating clearer clusters. 
To avoid visual clutter caused by group glyphs in dimensionality-reduced views, we follow the approach of Tac-Miner~\cite{wang2021tac}, representing groups as points or rectangles based on whether they received teacher scaffolding, with color coding reflecting prior performance. 
\RR{Additionally, the outer arc represents the duration of the discussion.}
Users can select groups using the lasso tool (Fig.~\ref{fig:teaser}-A1b) or search for specific groups (Fig.~\ref{fig:teaser}-A1a). The similarity panel (Fig.~\ref{fig:teaser}-A2) shows the most similar and dissimilar groups based on Euclidean distance.  
 

 
\subsection{Content View}
Once the search group is selected, users can perform a detailed analysis through the content view (Fig.~\ref{fig:teaser}-B). This view comprises four panels, allowing for a layered exploration of group and individual student performance. 

\subsubsection{Codes Panel}
In the upper left corner of the content view, a control button (Fig.~\ref{fig:teaser}-B1) allows users to toggle between the Codes panel and the Student Projection/Group Pattern panels. Below, users can compare the code quality between the selected group (left) and a comparison group (right). 
For instance, Figure~\ref{fig:teaser}-B1 shows Group 10's answer to Question 5 (left) and Group 18's answer (right). Hint boxes provide two types of feedback: red tips indicating code deficiencies (Fig.~\ref{fig:teaser}-B1a) and pink tips showing that a method was learned from pre-class materials, signaling comprehension, and application. 
Below the code, the ChatGPT-4o score and its rationale are provided, enabling quick, in-depth code evaluation and highlighting areas where groups faced challenges. 


 
\subsubsection{Students Projection (Fig.~\ref{fig:student}-B2a)}

We project students from different groups using t-SNE for clustering, highlighting students with similar performance. 
The three flowers representing the search group are connected with dashed lines to clarify group member distribution and similarity, helping users assess group homogeneity or heterogeneity. 
Only the comparing group's flowers are highlighted to minimize visual clutter. 
\RR{Users can zoom in/out to explore specific students and view detailed background information (e.g., major, grades) by hovering over individual points (Fig.~\ref{fig:student}-B2c).}

\subsubsection{Group Pattern Panel (Fig.~\ref{fig:student}-B2b \& d)}
We use Epistemic Network Analysis (ENA)~\cite{zhao2023analysing} to analyze the dynamic connections between cognitive elements and the collaborative problem-solving behaviors of groups (R3). 
In the Group Pattern Panel, each node represents a behavior in the collaborative problem-solving process, with colors following the coding scheme in Section~\ref{fig:scheme}. Node size indicates behavior frequency, and the intensity of the color reflects the frequency of interactions between behaviors. More significantly, darker nodes represent more frequent behaviors and interactions. 
Users can click on different question buttons to examine dynamic changes in behavior across specific questions or select multiple questions to observe how behaviors evolve during transitions between tasks. Hovering over nodes reveals detailed information about each behavior.
\RR{When comparing two groups, the system displays side-by-side behavior networks.} 
The Group Pattern Panel displays the search group's behavior patterns and compares them to those of the comparison group during the collaborative problem-solving process (Fig.~\ref{fig:student}-B1b).


\subsubsection{Timeline Panel}
\label{barchart}
Users can toggle between the search and comparing groups using the control button in the upper left corner of the Timeline Panel (R5). This panel displays group and individual student performance over time using a filterable bar chart and line chart (Fig.~\ref{fig:teaser}-B3).
In the bar chart, each bar represents a timestamp, with colors indicating different collaborative problem-solving behaviors. The bar height reflects uncertainty, as calculated by ChatGPT-4o. Dashed lines separate different questions for visual clarity. Hovering over a bar reveals the behavior category, predicted certainty, and reasoning behind the label (Fig.~\ref{fig:teaser}-B3e). Clicking on a bar takes users to the relevant conversation content (Fig.~\ref{fig:teaser}-C2a).
The filtering function (Fig.~\ref{fig:teaser}-B3c) allows users to zoom in on specific periods, magnifying the bars for more detailed analysis. 
\RR{The progress bar consolidates behaviors, minimizing visual clutter and highlighting key shifts in temporal behavior in the overview, while also allowing for detailed tracking of group dynamics (Fig.~\ref{fig:group18}-c).}

Three line charts track the behaviors and cognitive engagement of three students in the group (Fig.~\ref{fig:teaser}-B3b). Engagement is calculated at the midpoint and end of each question, and Savitzky–Golay filtering smooths the curves to highlight dynamic changes and trends across questions.
Below the charts, role types are mapped using equally sized rectangles at each timestamp. Users can zoom in on the timeline with the progress bar for detailed analysis. 






\subsection{Detail View}
In the Timeline Panel, users can link to specific conversation content to review group discussions (Fig.~\ref{fig:teaser}-C1) and individual student conversations for each question (Fig.~\ref{fig:teaser}-C2). The playback feature allows users to revisit the original video of collaborative programming sessions, providing a more immersive classroom experience. This feature validates analysis results, offers detailed references, and supports the final step of our analysis workflow (R6).

\begin{figure*}
 	\centering
 	\includegraphics[width=1\linewidth]{Student_panel.pdf}
 	\vspace{-1em}
 	\caption{The panel displayed in the Content View after selecting ``Student Projection \& Group Pattern'' is shown \RR{(A screenshot of the Group 10 \& 18 view)}. On the left (B2a), the projection of all students highlights the searched and compared groups, showing connections between students within the searched group. Hovering over a flower reveals the student's background information (B2c). On the right (B2b), group patterns for two groups across different questions are shown, with clickable question buttons to explore changes in dynamic learning behavior.}
        \Description{The panel displayed in the Content View after selecting ``Student Projection \& Group Pattern'' is shown \RR{(A screenshot of the Group 10 \& 18 view)}. On the left (B2a), the projection of all students highlights the searched and compared groups, showing connections between students within the searched group. Hovering over a flower reveals the student's background information (B2c). On the right (B2b), group patterns for two groups across different questions are shown, with clickable question buttons to explore changes in dynamic learning behavior.}
 	\label{fig:student} \end{figure*}

