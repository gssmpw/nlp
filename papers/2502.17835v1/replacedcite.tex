\section{Related Work}
\label{sec:related}
\RR{In this section, we discuss the relevant research, including Programming  Education and Evaluation, Multimodal Learning Analysis in Collaborative Programming, and Visual Analysis of Collaborative Behaviors in Meeting.}

\subsection{Programming Education and Evaluation}
As technology advances rapidly, programming skills are a significant driving force behind social and economic development. 
Whether in artificial intelligence____, big data____, or the Internet of Things____, programming plays a crucial role in these cutting-edge technologies.
The importance of programming education increases globally, with programming being integrated into the foundational curricula of many countries____. 
Instructors supplement programming education by incorporating lab courses that encourage coding practice____. \RR{Among the various teaching strategies employed in these lab courses, collaborative programming emerges as a commonly used and effective method for teaching novices, fostering teamwork and practical coding skills____.}

Though programming education is widespread, evaluating programming skills in a large classroom remains challenging, particularly in assessing students' code____. 
Researchers develop tools that leverage LLMs to help instructors analyze code more effectively____. 
These tools provide insights into students' code____ and facilitate code to identify patterns____. 
For example, VizProg uses CodeBERT to cluster solutions and visualize coding progress in real-time____, while CFlow combines LLMs for semantic annotation and interactive views to identify patterns in courses____. 
\RR{However, existing tools largely assessed code fragments from collaborative programming in isolation, failing to consider the collaborative behaviors and dynamics among students____.}
In contrast, \textit{CPVis} analyzes final code submissions and students' collaborative behaviors, enabling instructors to provide detailed, evidence-based feedback that enhances students' metacognitive awareness and addresses deficiencies in their learning approaches.

\subsection{Multimodal Learning Analysis in Collaborative Programming}
In collaborative programming, students work in groups to complete programming tasks, share code modifications, and review each other's work____. 
This method aims to improve students' programming efficiency, code quality, and teamwork skills through collaboration____. 
To understand students' interactions and engagement, multimodal data such as conversations and screen recordings from the collaborative programming process are typically collected____.
Multimodal Learning Analytics (MMLA) provides new insights into students' learning by analyzing various data streams generated during the learning process, such as speech, facial expressions, and gestures____. 
For instance, multimodal analysis in collaborative learning is pioneered by quantifying learners' nonverbal behaviors through video segmentation to predict academic performance____.
Various types of multimodal data are increasingly used to analyze collaborative problem-solving processes in learning environments____. 
\RR{For example, Mangaroska et al.____ combine multimodal data (camera, electroencephalogram, eye-tracking) with cognitive load theory and affective dynamics models to analyze problem-solving in collaborative learning.}
Although these studies explored the analysis of problem-solving processes, they are limited to extracting relevant metrics from multimodal data, \RR{lacking the capability to dynamically analyze students' problem-solving processes over time in a narrative form. }

Collaborative programming is a complex form of collaborative learning, involving various factors such as role transitions during programming____ and interaction patterns____. 
\RR{For instance, Lewis et al. accelerate the speed at which students complete exercises by intervening in the roles students play in collaborative programming____.
However, existing research fails to address the interdependent development of other key dimensions such as teacher scaffolding____ and learning engagement analysis________, focusing instead on the impact of a single dimension on the collaborative programming process____.}
These limitations highlight the need for designing an innovative visualization framework to address the unique challenges of tracking multiple metrics over time, clearly representing interaction patterns in collaborative programming to help instructors better understand and evaluate the collaborative problem-solving process.

\subsection{Visual Analysis of Collaborative Behaviors in Meeting}
Collaborative programming discussion is a form of multi-person collaboration, and we focus on analyzing collaborative behavior during meetings.
In the visualization community, collaborative behavior analysis is explored across various applications, including face-to-face____ and online discussions____, as well as real-time____ and post-meeting analysis____. 
\RR{These two dimensions can be combined to form a larger design space.} For instance, integrating face-to-face discussions with real-time analysis reduces environmental distractions and fosters more natural, seamless interactions____. 


Some real-time analysis systems____, such as ClassBeacons, effectively visualize instructors' time and attention allocation through light objects on students' desks, streamlining issue resolution during student discussions____. 
Real-time analysis systems focus on summarizing ongoing discussions without disrupting their flow____. Similarly, Groupnamics provides an overview of parallel groups in online classrooms, helping identify those needing intervention____.
However, post-analysis offers a distinct advantage for task evaluation, as it provides a broader perspective to understand better and assess the outcomes of collaboration.
Some post-analysis visualization systems, such as ConToVi, NEREx, and Meeting Mediator____, focus on speaker behavior patterns over time but do not fully capture the evolution of speakers' problem-solving abilities across different discussion topics. \RR{Moreover, these systems primarily summarize single-session dialogues, lacking the ability to analyze multiple groups or discussions simultaneously.}

Studies analyzing group dialogues in classrooms typically focus on engagement and comprehension but often lack detailed visual insights into the underlying problem-solving logic, strategies, or intentions____.
\RR{In this work, we focus on the post-analysis of face-to-face discussions, emphasizing the collaborative problem-solving process across multiple groups. 
\textit{CPVis} incorporates narrative-based visualizations to depict and compare the evolution of dialogue, behavior, and engagement, providing instructors with richer evidence to assess group dynamics and individual engagement more effectively.}