We prove the following proposition.

\existenceEpsComplete*

We let $\kappa=2^{\aleph_0}$ and let $S$ denote $S_d^\kappa$.
Our goal is to define a (chromatic) $k$-blowup $\B$ of $\A$ which is $k$-wise $\eps$-complete and recognises $W$.
For now, we discuss general memory, and explain below how the proof is (very easily) adapted to the chromatic case.

We start with the extraction lemma mentioned in the overview, then we will present the definition of $\B$ and then prove its correctness.

\subsection{A combinatorial lemma: Extracting homogeneous subtrees}

As an important part of our proof, we will take the graph $S$, whose set of vertices is $\kappa^{\d_\odd}$, where $\d_\odd=\{1,3,\dots,d-1\}$, and extract from it a large enough subgraph which is homogeneous.
This requires a combinatorial lemma which we now describe.

By a slight abuse of terminology (since this is not consistent with the definition of trees from the preliminaries), we use the terminology ``signature trees'' to refer to subsets of $\kappa^{\d_\odd}$.
Elements of the subsets should be thought of as leaves of the tree, while their (non-proper) prefixes correspond to nodes.
More precisely, a node of level $x$, where $x$ is an even priority from $\d$, in a signature tree $T$ is a tuple $s_{< x} \in \kappa^{x_\odd}$ such that there exists $s_{> x}$ satisfying $s_{< x} s_{> x} \in T$.

The subtree rooted at a node $s_{<x}$ of level $x$ is defined to be the tree $\{s_{>x} \mid s_{<x}s_{>x} \in T\}$.
We say that a tree $T$ is everywhere cofinal if for each node $s_{< x}$, the subtree rooted at $x$ is cofinal in $\kappa^{(d-x)_\odd}$.
An inner labelling of a tree $T$ by $L$ is a map $\lambda$ assigning a label in $L$ to every node in $T$.
We are now ready to state the extraction lemma
We let $\kappa=2^{\aleph_0}$.

\begin{lemma}\label{lem:everywhere_cofinal}
Let $\lambda$ be an inner labelling of $\kappa^{d_\odd}$ by $L$, where $L$ is countable.
There is an everywhere cofinal tree $T$ such that at every level $x$, $\lambda$ is constant over nodes of level $x$.
\end{lemma}
We say that a labelling as in the conclusion of the lemma is constant per level.

\begin{proof}
    We prove the lemma by induction on $d$.
    For $d=0$ there is nothing to prove since $d_\odd$ is empty; let $d\geq 1$ and assume the result known for $d-2$.
    For each node $s$ at level $2$, apply the induction hypothesis on the subtree of $\kappa^{d_\odd}$ rooted at $s_{<2}$,
    which gives an everywhere cofinal tree $T'$ and a labelling $\lambda'$ which is constant per level over $T'$.
    Let $\ell'_0,\ell'_2,\dots,\ell'_{d-2}$ denote the constant values of $\lambda'$ on the corresponding levels of $T'$, and define the new labelling of $s$ to be the tuple $(\lambda(s),\ell'_0,\dots,\ell'_{d-2})$.

    Now since there are at most countably-many new labellings, and there are $\kappa=2^{\aleph_0}$ nodes at level $2$, there is a new labelling $\ell$ such that cofinaly-many nodes have this new labelling.
    We conclude by taking $T$ to be the union of $\{s\} \times T'_s$, where $s$ ranges over nodes at level $2$ with the new labelling $\ell$, and $T'_s$ are the corresponding everywhere cofinal trees.
\end{proof}

\subsection{Definition of $\B$}

Consider the cascade product $\A \casc S$. 
By Lemma~\ref{lem:cascade_products}, it satisfies $W$ and is well-founded.
Moreover it has size $\leq \kappa$, so by our assumption on $W$, we may apply Theorem~\ref{thm:structuration}.
%Since there are transitions $s\re \eps s'$ whenever $s \geq s'$ in $S$, we also have transitions $(q,s) \re \eps (q,s')$ in $\A \casc S$ when $s \geq s'$.
%We close it by $\eps$-transitivity, meaning that we add transitions $(q,s) \re c (q',s')$, for $c \in \Sigmaeps$, whenever $(q,s) \re{\eps^* c \eps^*} (q',s')$; the obtained graph $\overline{\A \casc S}$ still satisfies $W$.
This yields a $k$-blowup $G$ of ${\A \casc S}$ which is $k$-wise $\eps$-complete.
Let us write $V(G) = V(\A) \times \{1,\dots,k\} \times V(S)$.
We close $G$ by transitivity, meaning that we add transitions $(q,m,s) \re c (q',m',s')$, for $c \in \Sigmaeps$, whenever $(q,m,s) \re{\eps^* c \eps^*} (q',m,s')$; the obtained graph $\overline{G}$ still satisfies $W$.

Here comes the important definition: say that $(q,m)$ strongly $c$-dominates $(q',m')$ at node $s_{< x}$ if
\[
    \exists s_{> x} \forall s'_{> x} \qquad (q,m,s_{< x} s_{> x}) \re c (q',m',s_{< x} s'_{> x}) \tin \overline{G},
\]
and that $(q,m)$ weakly $c$-dominates $(q',m')$ at $s_{< x}$ if
\[
    \forall s'_{> x} \exists s_{> x} \qquad (q,m,s_{< x} s_{> x}) \re c (q',m',s_{< x} s'_{> x}) \tin \overline{G},
\]
where $c \in \Sigmaeps$.
Note that strong domination implies weak domination.
The type of a node $s_{< x}$ is the information, for each $q,q',m,m'$ and $c$, of whether $(q,m)$ strongly or weakly (or not at all) $c$-dominates $(q',m')$.
This gives finitely-many possibilities for fixed $q,q',m,m'$ and $c$, and therefore there are in total a countable number of possible types.
Thus Lemma~\ref{lem:everywhere_cofinal} yields a tree $T \subseteq \kappa^{\alpha}$ which is everywhere cofinal and such that for all $x$, nodes at level $x$ in $T$ all have the same type $t_x$.

We are now ready to define $\B$.
We put $V(\B) = V(\A) \times \{1,\dots,k\}$, and for each even $x \in \d$ and $c \in \Sigmaeps$, we define transitions by
\[
    \begin{array}{lcrl}
    (q,m) &\re{c:x}& (q',m') & \tif (q,m) \text{ strongly $c$-dominates } (q',m') \tin t_x \\
    (q,m) &\re{c:x+1}& (q',m') & \tif (q,m) \text{ weakly $c$-dominates } (q',m')\tin t_x.
    \end{array}
\]
Here is the main lemma, which proves the direct implication in Theorem~\ref{thm:main-charac}.

\begin{lemma}\label{lem:main_lemma_charac}
Automaton $\B$ is a $k$-blowup of $\A$, it is $k$-wise $\eps$-complete and it recognises $W$.
\end{lemma}
The remainder of the section is devoted to proving Lemma~\ref{lem:main_lemma_charac}.

\subsection{Correctness of $\B$: Proof of Lemma~\ref{lem:main_lemma_charac}}

There are a few things to show.
The interesting argument is the one that shows that $\B$ recognises $W$ (Lemma~\ref{lem:language-containement} below).
We should also prove there is no accepting run over words in $\Sigma^*\eps^\omega$, which will be done below as part of Lemma~\ref{lem:language-containement}.

\subparagraph*{$\B$ is a $k$-blowup of $\A$.}

We should prove the following.

\begin{claim}
    For all transitions $q \re{a:y} q'$ in $\A$, and any $m \in \{1,\dots,k\}$ there is some $m' \in\{1,\dots,k\}$ such that $(q,m) \re{a:y} (q',m')$ in $\B$.
\end{claim}

\begin{claimproof}
Let $q \re{a:y} q'$ be a transition in $\A$ and let $m \in \{1,\dots,k\}$.
Since $G$ is a $k$-blowup of $\A \times S$, for all edges $s\re y s'$ in $S$ there is $m' \in \{1,\dots,k\}$ such that $(q,m,s) \re a (q',m',s') \tin G$.
Although both proofs are similar, we distinguish two cases.
\begin{itemize}
    \item If $y=x$ is even.
    We prove that for all nodes $s_{< x}$ at level $x$, $(q,m)$ strongly $c$-dominates $(q',m')$ for some $m'$.
    Therefore the same is true in $t_x$ which implies the wanted result.
    We let $s_{> x}=0_{> x}$, the zero sequence in $\kappa^{(d-x)_\odd}$.
    Now for all $s'_{>x} \in \kappa^{(d-x)_\odd}$, it holds that $s_{< x} s_{> x} = s_{< x} 0_{> x} \re x s_{< x} s'_{>x}$ in $S$, so there is $m'$ such that $(q,m,s_{< x} s_{>x}) \re a (q,m',s_{< x} s'_{>x})$ in $G$ and thus also in $\overline G$; in this case say that $m'$ is good for $s'_{> x}$.
    
    Now we claim that if $m'$ is good for $\tilde s'_{> x} \geq s'_{> x}$, then it is also good for $s'_{> x}$.
    Indeed, we have $(q',s_{<x}\tilde s'_{> x}) \re \eps (q',s_{<x}s'_{> x})$ in $\A \times S$ therefore since $\eps$-transitions preserve the memory state in $G$ (Theorem~\ref{thm:structuration}) we have $(q',m',s_{<x} \tilde s'_{> x}) \re \eps (q',m',s_{<x} s'_{>x})$ in $G$ thus $(q,m,s_{<x}s_{> x}) \re a (q',m',s_{<x}s'_{> x})$ in $\overline{G}$.

    Therefore the sets $M'_{s'_{> x}}$ of $m'$ which are good for $s'_{> x}$ form a decreasing chain of non-empty subsets of $\{1,\dots,k\}$ and thus their intersection is non-empty: there is some $m'$ which is good for all $s'_{> x}$, as required.

    \item If $y=x+1$ is odd. 
    We now prove that for all nodes $s_{< x}$ at level $x$, $(q,m)$ weakly $c$-dominates $(q',m')$ for some $m'$.
    Let $s'_{> x} \in \kappa^{(d-x)_\odd}$.
    Then for any $s_{> x}$ such that $s_x > s'_x$, it holds that $s_{< x} s_{> x} \re {x+1} s'_{< x} s_{>x}$ in $S$, so there is some $m'$ such that $(q,m,s_{< x} s_{>x}) \re a (q,m',s_{< x} s'_{>x})$ in $G$.
    Hence there is some $m'$ such that, for cofinitely many $s_{> x}$, $(q,m,s_{< x} s_{>x}) \re a (q,m',s_{< x} s'_{>x})$ is an edge in $G$ and thus also in $\overline G$; say that such an $m'$ is good for $s'_{> x}$.

    Now, we claim that if $m'$ is good for $\tilde s'_{> x} \geq s'_{> x}$, then it is also good for $s'_{> x}$.
    Indeed, as in the first case, we have $(q',m',\tilde s_{<x}s'_{> x}) \re \eps (q',m',s_{<x}s'_{> x})$ in $G$ thus for cofinitely many $s_{> x}$ we have $(q,m,s_{< x}s_{>x}) \re a (q',m',s_{<x} s'_{> x})$ in $\overline{G}$.

    We conclude just as above. \claimqedhere 
\end{itemize}
\end{claimproof}


\subparagraph*{$\B$ is $k$-wise $\eps$-complete.}

We should prove the following claim.

\begin{claim}
    For every even $x$, memory state $m$ and states $q,q'$, either $(q,m) \re{\eps:x} (q',m)$ or $(q',m) \re{\eps:x+1} (q,m)$.
\end{claim}

\begin{claimproof}
    Assume that $(q,m) \re{\eps:x+1} (q',m)$ is not a transition in $\B$.
    Consider a node $s_{<x}$ at level $x$ in $T$: it has type $t_x$ and thus $(q,m)$ does not weakly $\eps$-dominate $(q',m)$ at $s_{<x}$.
    This rewrites as
    \[
        \exists s'_{> x} \forall s_{> x} \qquad (q,m,s_{< x} s_{>x}) \re \eps (q',m,s_{< x} s'_{>x}) \text{ is not an edge in } \overline G.
    \]
    Now since $\overline G$ is $\eps$-complete, we get 
    \[
        \exists s'_{> x} \forall s_{> x} \qquad (q',m,s_{< x} s'_{>x}) \re \eps (q,m,s_{< x} s_{>x}) \tin \overline G,
    \]
    therefore $(q',m)$ strongly $\eps$-dominates $(q,m)$ at $s_{<x}$, which concludes.
\end{claimproof}


\subparagraph*{$\B$ recognises $W$.}

We now turn to the more involved part.
We start by proving the following two technical lemmas.

\begin{lemma}\label{lem:technical1}
Assume that $(q,m) \re{c:y} (q',m') \tin \B$ for some $y$.
There is a map $f:T \to T$ such that for all $s \in T$, 
\begin{itemize}
    \item there is an edge $(q,m,f(s)) \re c (q',m',s)$ in $\overline G$; and
    \item it holds that $f(s)_{< y} = s_{< y}$.
\end{itemize}
\end{lemma}

\begin{proof}
Let $x+1$ be the smallest odd priority $\geq y$.
Since strong domination implies weak domination, and $(q,m) \re{c:y} (q',m')$, we have in any case that $(q,m)$ weakly $c$-dominates $(q',m')$ in $t_{x}$.
This means that for any $s'=s'_{< x}s'_{> x} \in T$, there exists $s_{> x}$ such that $(q,m,s'_{< x}s_{> x}) \re c (q',m',s'_{< x}s'_{> x}) \tin \overline G$.
Now since $T$ is everywhere cofinal, there exists $\tilde s_{> x} \geq s_{> x}$ such that $s'_{<x}\tilde s_{> x} \in T$, and we let $f(s')=s'_{<x}\tilde s_{> x}$.
Clearly $f(s')_{< y} = s'_{< x} = s'_{< y}$, and also, we have $(q,m,s'_{<x}\tilde s_{> x} ) \re \eps (q,m,s'_{<x}s_{> x})\tin G$ so the result follows from $\eps$-transitivity.
\end{proof}

\begin{lemma}\label{lem:technical2}
    Assume that $(q,m) \re{c:x} (q',m') \tin \B$ for some even $x$, and let $s_{< x}$ be a node at level $x$ in $T$.
    There is $s_{> x} \in \kappa^{(d-x)_\odd}$ such that $s_{< x}s_{> x} \in T$ and for any $s'_{> x} \in \kappa^{(d-x)_\odd}$, $(q,m,s_{< x}s_{> x}) \re c (q',m',s_{<x}s'_{> x})$ in $\overline G$.
\end{lemma}

\begin{proof}
    From the definition of strong domination, there is $\tilde s_{>x}$ such that for all $s'_>x$, it holds that $(q,m,s_{<x} \tilde s_{>x}) \re c (q',m',s_{<x} s'_{>x})$ in $\overline G$. By everywhere confinality of $T$, there is $s_{>x} 
    \geq \tilde s_{>x}$ such that $s_{<x} s_{>x} \in T$.
    We conclude (as previously) using $\eps$-transitivity.
\end{proof}

We are now ready for the main argument.

\begin{lemma}\label{lem:language-containement}
    The language of $\B$ is contained in $W$.
    Moreover, there is no accepting run labelled by words in $\Sigma^* \eps^\omega$.
\end{lemma}

\begin{proof}
    Take an accepting run
    \[
        (q_0,m_0) \re{c_0:y_0} (q_1,m_1) \re{c_1:y_1} \dots
    \]
    in $\B$.
    Let $x=\liminf_i y_i$ (it is even since the run is accepting), and let $i_0$ be such that $y_i \geq x$ for $i \geq i_0$.

    As explained in the general overview, our goal will be to endow each $(q_i,m_i)$ with some $s_i \in T$ such that for all $i$, $(q_i,m_i,s_i) \re{c_i} (q_{i+1},m_{i+1},s_{i+1}) \tin \overline G$.
    This implies the result since $\overline G$ satisfies $W$, and since it does not have paths labelled by words in $\Sigma^* \eps^\omega$ by well-foundedness.
    We pick an arbitrary node $s_{< x}$ at level $x$ in $T$ and proceed as follows:
    \begin{itemize}
        \item for each $i\geq i_0$ such that $y_i = x$, we let $s_{> x}$ be obtained from Lemma~\ref{lem:technical2} and set $s_i = s_{< x} s_{> x}$;
        \item for any other $i$, we proceed by backwards induction (see Figure~\ref{fig:backtrack}) and let $s_i=f(s_{i+1})$, where $f$ is obtained by applying Lemma~\ref{lem:technical1} to transition $(q_i,m_i) \re{c_i:y_i} (q_{i+1},m_{i+1})$.
    \end{itemize}

    \begin{figure}[h]
        \begin{center}
            \includegraphics[width=0.9\linewidth]{Figures/backtrack.pdf}
        \end{center}
        \caption{The run in $\B$ (in black), and the order in which the $s_i$'s are computed (in red).}\label{fig:backtrack}
    \end{figure}

    For $i$'s as in the second item, it follows from Lemma~\ref{lem:technical1} that $(q_i,m_i,s_i) \re{c_i} (q_{i+1},m_{i+1},s_{i+1})$, and moreover, assuming $i \geq i_0$, that $(s_i)_{< x} = (s_{i+1})_{< x}$.
    Thus for all $i \geq i_0$ we have $(s_i)_{< x} = s_{< x}$ hence we also have, by Lemma~\ref{lem:technical2}, that $(q_i,m_i,s_i) \re{c_i} (q_{i+1},m_{i+1},s_{i+1}) \tin \overline G$ for $i$'s as in the first item. 
\end{proof}

\po{todo: figure}