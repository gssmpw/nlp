We now give a proof of Theorem~\ref{thm:structuration}.

\structuration*

The idea is to use choice arenas, which were introduced in Ohlmann's PhD thesis~\cite[Section 3.2 in Chapter 3]{Ohlmann21PhD} for positionality, and then adapted to memory in~\cite{CO24Positional}.

\begin{proof}
    Let $H$ be the game defined as follows.
    The set of vertices is $V(G) \sqcup \powne{V(G)}$, where $\powne{V(G)}$ is the set of non-empty subsets $X$ of $V(G)$, partitioned into $V_\Adam = V(G)$ and $V_\Eve=\powne{V(G)}$.
    The initial vertex is $v_0$, the one of $G$.
    Then the edges are given by taking those of $G$, and then adding $v \rer \eps X$ whenever $v \in X$.
    The objective is $W$.

    In words, when playing in $H$, Adam follows a path of his choice in $G$, except that at any point, he may choose a set $X$ containing the current vertex $v$, and allow Eve to continue the game from any vertex of her choice in $X$.
    In some sense, Adam can hide the current vertex; this is especially true if Eve is required to play with finite memory.

    Since $G$ satisfies $W$, Eve wins, simply by going back to the previous vertex $v$ every time Adam picks an edge $v \re \eps X$.
    (Formally, the corresponding winning strategy has vertices $V(G) \sqcup \{(v,X) \mid v \in X\}$, projection $\pi(v)=v$ and $\pi(v,X)=X$, edges $E(G) \cup \{v \rer \eps (v,X) \mid v \in X\}$, and initial vertex $v_0$.)
    Therefore by our assumption on $W$, there is a (chromatic) winning strategy $S$ with memory $k$, i.e. $V(S)=V(H) \times \{1,\dots,k\}$.
    
    Now we define $G'$ by $V(G')=V(G) \times \{1,\dots,k\}$, initial vertex $(v_0,m_0)$, the initial vertex of $S$, and with the edges from $E(S) \cap (V(G') \times (\Sigma \cup \{\eps\}) \times V(G'))$, together with edges $(v,m) \re \eps (v',m)$ whenever there is $X \ni v,v'$ such that $(X,m) \re \eps (v',m) \tin S$.

    We prove that $G'$ satisfies the conclusion of the theorem, except for well-foundedness which is dealt with below.

    \begin{description}
        \item[$G'$ is a $k$-blowup of $G$.]
        This is because $S$ is a strategy, and vertices in $V(G)$ belong to Adam in $H$.
        Therefore, for each $(v,m) \in V(G')$, and each edge $v \re c v'$ in $G$, $v \re c v'$ is also an edge in $H$, thus there is $m'$ such that $(v,m) \re c (v',m')$ is an edge in $G'$.
        \item[In the chromatic case, $G'$ is chromatic.]
        This is because $S$ is chromatic, therefore by definition of $G'$, it is chromatic with the same chromatic update function.
        \item[$G'$ is $k$-wise $\eps$-complete.] Since it is a graph, we should prove that for each $v,v'$ and each $m$, either $(v,m) \re \eps (v',m)$ or $(v',m) \re \eps (v,m)$ in $G$.
        This follows from applying the definition of $G'$ to $X=\{v,v'\}$, since either $(X,m) \re \eps v$ or $(X,m) \re \eps v'$ is an edge in $S$ (because $S$ is without dead-ends).
        \item[$G'$ satisfies $W$.] This is because $S$ is a winning strategy, and every path in $G'$ corresponds to a path in $S$, by replacing each edge $(v,m)\re \eps(v',m)$ by $(v,m) \re \eps X \re \eps (v',m)$.
    \end{description}

    There remains a slight technical difficulty, which is that $G'$ may have $\re \eps$-cycles (in fact, it even has all $\eps$-self-loops, by applying the definition to $X$ being a singleton).
    However for each memory state $m$, the relation $\re \eps$ has the property that for every subset of states $X$, there is $v \in X$ (which should be seen as a minimal element) such that every $v' \in X$ satisfies $(v',m) \re \eps (v,m)$ in $G'$.

    Therefore for each $m$, and each $X\subseteq V(G)$ such that $X \times \{m\}$ is a strongly connected component for $\re \eps$ in $G'$, we pick an arbitrary strict well-order $\re{}$ over $X$ which extends the $\eps$-edges already present in $G$ over $X$.
    This is possible because $G$ is assumed $G$ to be well-founded (and it is necessary so that the obtained graph remains a $k$-blowup of $G$).
    Finally, we rewire $\eps$-edges over $X \times \{m\}$ so that they correspond to $\re{}$; it is not hard to see that the above points are not broken by this construction.
\end{proof}