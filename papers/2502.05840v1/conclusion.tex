
We characterised objectives in $\BCSigma$ with memory (or chromatic memory) $\leq k$ as those recognised by a well-identified class of automata.
In particular, this gives the first known characterisation of (chromatic) memory for $\omega$-regular objectives, and proves that it is decidable (in fact even in $\NP$).
We also settled (a strengthening of) Kopczyński's conjecture for $\BCSigma$ objectives.
We now discuss some directions for future work.

\subparagraph*{Memory in finite games.}
This paper focuses on games over potentially infinite game graphs. 
%Several results on memory from the litterature concern finite games~\cite{Kop08Thesis,BRORV22}, but no characterisation for 
%Vandenhove conjectured that, for $\oo$-regular languages, the memory over finite games coincides with the unrestricted memory~\cite[Conjecture~9.1.2]{Vandenhove23Thesis}.
A wide body of literature studies the memory over finite game graphs~\cite{Kop08Thesis,BRORV22}; we believe that, for $\oo$-regular objectives, both notions should coincide, and our results should characterise the memory of $\oo$-regular objectives over finite games too. 
More precisely, we believe that in item~(\ref{item:memory-small-games}) in Theorem~\ref{thm:main-charac}, it suffices to assume that $W$ has memory $\leq k$ over games of size $f(n)$, for some finite bound $f(n)$, where $n$ is the size of a deterministic automaton representing $W$. %, when $W$ is $\oo$-regular.

\begin{question}[{Version of~\cite[Conjecture~9.1.2]{Vandenhove23Thesis}}]\label{quest:ptime}
	Show that if an $\oo$-regular objective has memory $\leq k$ over finite game graphs, then it has memory $\leq k$ over all game graphs.
\end{question}

The hypothesis on $\oo$-regularity is necessary, as this statement already fails in the case of positionality and closed objectives~\cite{CFH14}.
We believe that one should be able to adapt the proof of Proposition~\ref{prop:existenceEpsComplete} to obtain this result, but some new ideas seem to be required.
As a follow-up question one could investigate the bound $f(n)$: can it be assumed polynomial?


\subparagraph*{Exact complexity of computation of memory.} We established that computing the (chromatic) memory of an $\omega$-regular objective is in $\NP$.
In fact, computing the chromatic memory is $\NP$-hard already for simple classes of objectives, such as Muller~\cite{Casares22Chromatic} or safety ones~\cite{BFRV23Regular}.
However, no such hardness results are known for non-chromatic memory.

\begin{question}\label{quest:ptime}
	Given a deterministic parity automaton $\A$ and a number $k$, can we decide whether the memory of $L(\A)$ is $\leq k$ in polynomial time?
\end{question}

This question is open already for the simpler case of regular open objectives (that is, those recognised by reachability automata).


\subparagraph{Assymetric 1-to-2-player lifts.} A celebrated result of Gimbert and Zielonka~\cite{GZ05} states that if for an objective $W$ both players can play optimally using positional strategies in finite games where all vertices belong to one player, then $W$ is bipositonal over finite games. This result has been extended in two orthogonal directions: to objectives where both players require finite chromatic memory~\cite{BRORV22,BRV23} (symmetric lift for memory), and to $\oo$-regular objectives where Eve can play positionally in $1$-player games~\cite{CO24Positional} (asymmetric lift for positionality).
In this work, we have not provided an asymmetric lift for memory, as in most cases no such result can hold.
For  $\BCSigma$ objectives, it is known to fail already for positional objectives~\cite[Section~7]{GK22Submixing}.
For non-chromatic memory, it cannot hold for $\oo$-regular objectives neither, because of the example described below.

\begin{restatable}{proposition}{counterexampleLift}
\label{prop:1-2-player-counterexample}
	Let $\Sigma_n = \{1,\dots, n\}$. For every $n$, the objective 
	\[W_n = \{w\in \SS_n^\oo \mid w \text{ contains two different letters infinitely often}\} \] 
	 has memory $2$ over games where Eve controls all vertices and memory $n$ over arbitrary~games.
\end{restatable}
\begin{proof}
	The fact that $W_n$ has memory $n$ follows from~\cite{DJW1997memory}. We include the proof that it has memory $2$ over games controlled by Eve in Appendix~\ref{app:no-lift}.
\end{proof}
In his PhD thesis, Vandenhove conjectures that an assymetric lift for chromatic memory holds for $\oo$-regular objectives~\cite[Conjecture~9.1.2]{Vandenhove23Thesis}. This question remains open.

\begin{question}
	Is there an $\oo$-regular objective  with chromatic memory $k$ over games where Eve controls all vertices and chromatic memory $k'>k$ over arbitrary~games?
\end{question}


\subparagraph*{Further decidability results for memory.} As mentioned in the introduction, many extensions of $\oo$-automata (including deterministic $\oo$-Turing machines and unambiguous $\oo$-petri nets~\cite{FSJLS22}) compute languages that are in $\BCSigma$.
We believe that our characterisation may lead to decidability results regarding the memory of objectives represented by these models.

\subparagraph*{Objectives in $\Delta_0^3$.} Some of the questions answered in this work in the case of $\BCSigma$ objectives are open in full generality, for instance, the generalised Kopczy\'nski's conjecture. A reasonable next step would be to consider the class $\bdelta 3 = \bsigma 3 \cap \bpi 3$.
Objectives in $\bdelta 3$ are those recognised by max-parity automata using infinitely many priorities~\cite{Skrzypczak13Colorings}.
Our methods seem appropriate to tackle this class, however, we have been unable to extend the extraction lemma (Lemma~\ref{lem:everywhere_cofinal}) used in the proof of Section~\ref{sec:existence_automata}.



