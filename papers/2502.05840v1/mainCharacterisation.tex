We state our main characterisation theorem and its decidability consequences for $\oo$-regular languages.
We assume that the alphabet $\Sigma$ is countable, therefore automata can also be taken with countable sets of states.


\begin{theorem}[Main characterisation]\label{thm:main-charac}
Let $W$ be a $\BCSigma$ objective and let $k \in \N$.
The following are equivalent:
\begin{enumerate}[(i.)]
    \item\label{item:memory-small-games} $W$ has memory $\leq k$ (resp. chromatic memory $\leq k$) on games of size $\leq 2^{2^{\aleph_0}}$.
    \item\label{item:existence-automata} For any automaton $\A$ recognising $W$, there is a (chromatic) $k$-blowup $\B$ of $\A$ which is $k$-wise $\eps$-complete and recognises $W$.
    \item\label{item:existence-det-automata} There is a deterministic (chromatic) $k$-automaton $\A$ which is $k$-wise $\eps$-completable and recognises $W$. If $W$ is recognised by a deterministic automaton of size $n$, then $\A$ can be taken of size $kn$.
    \item\label{item:existence-universal-graph} For every cardinal $\kappa$, there is a (chromatic)  $(\kappa,W)$-universal graph which is well-founded and $k$-wise $\eps$-complete.
    \item\label{item:memory-arbitrary-games} $W$ has (chromatic) memory $\leq k$ on arbitrary games.
\end{enumerate}
\end{theorem}

For $\omega$-regular $W$ given by a deterministic automaton $\B$ of size $n$, this allows to compute the (chromatic) memory in $\NP$: guess a deterministic automaton $\A$ of size $\leq kn$ and a (chromatic) $k$-wise $\eps$-completion $\A^\eps$, and check if  $L(\B) \subseteq L(\A)$ and if $L(\A^\eps) \subseteq L(\B)$, which can be done in polynomial time, since $\A$ and $\B$ are deterministic~\cite{ClarkeDK93Unified}.
Prior to our work, neither computing the memory nor the chromatic memory was known to be decidable (although the chromatic memory over finite graphs can be computed in exponential time~\cite{Kop08Thesis}).

\begin{corollary}[Decidability in $\NP$]\label{thm:NP-computation-memory}
    Given an integer $k$ and a deterministic automaton $\A$, the problem of deciding if $L(\A)$ has (chromatic) memory $\leq k$ belongs to $\NP$.
\end{corollary}

Our main contribution is the implication from (\ref{item:memory-small-games}) to (\ref{item:existence-automata}), which is the object of Section~\ref{sec:existence_automata}.
We proceed in Section~\ref{sec:existence-det-automata} to show that (\ref{item:existence-automata}) implies (\ref{item:existence-det-automata}) which is straightforward.
The implication (\ref{item:existence-det-automata}) $\implies$ (\ref{item:existence-universal-graph}) is adapted from~\cite{CO24Positional} and presented in Section~\ref{sec:existence-universal-graphs}.
Finally, the implication (\ref{item:existence-universal-graph}) $\implies$ (\ref{item:memory-arbitrary-games}) is the result of~\cite{CO25LMCS} (Theorem~\ref{thm:universal_graphs}), and the remaining one is trivial.

\subsection{Existence of $k$-wise $\eps$-complete automata: Proof overview}\label{sec:existence_automata}

We start with the more challenging and innovative implication: how to obtain a $k$-wise $\eps$-complete automaton given an objective in $\BC(\bsigma 2)$ with memory $k$ (that is, (\ref{item:memory-small-games}) $\implies$ (\ref{item:existence-automata})).
In this Section, we give a detail overview of the proof. 
Full details are given in Appendix~\ref{app:existence-eps-automata}.

\begin{restatable}{proposition}{existenceEpsComplete}
    \label{prop:existenceEpsComplete}
    Let $W$ be an objective recognised by an automaton $\A$, and assume that $W$ has (chromatic) memory $\leq k$ on games of size $\leq 2^{2^{\aleph_0}}$.
    Then there is a (chromatic) $k$-blowup $\B$ of $\A$ recognising $W$ which is $k$-wise $\eps$-complete.
\end{restatable}

We assume that $W$ has memory $\leq k$ on games of size $\leq \kappa$; we discuss the chromatic case at the end of the section.
Let $\A$ be an automaton recognising $W$; we aim to construct a $k$-blowup $\B$ of $\A$ which is $\eps$-complete.
We let $S$ denote $S^\kappa_d$, the $(\kappa,\Parity_{d})$-universal graph defined above.


%\subparagraph*{Detailed proof overview.}
We consider the cascade product $\A \casc S$; this is a $(\Sigma \cup \{\eps\})$-graph which intuitively encodes all possible accepting behaviours in $\A$.
Then we apply the structuration result (Theorem~\ref{thm:structuration}) to $\A \casc S$ which yields a $k$-blowup $G$ of $\A \casc S$ which is well-founded and $k$-wise $\eps$-complete (as a graph).
Stated differently, up to blowing the graph $\A \casc S$ into $k$ copies, we have been able to endow it with many $\eps$-transitions, so that over each copy, $\re \eps$ defines a well-order.
Note that the states of $G$ are of the form $(q,m,s)$, with $q\in V(\A)$, $m\in\{1,\dots,k\}$ and $s\in V(S)$.

The states of $\B$ will be $V(\B)=V(\A) \times \{1,\dots,k\}$.
The challenge lies in defining the transitions in $\B$, based on those of $G$.


Given a state $(q,m)\in V(\B)$ and a transition $q \re {c:y} q'$ in $\A$, where $c \in \Sigma \cup \{\eps\}$, by applying the definitions we get transitions of the form $(q,m,s) \re c (q',m',s')$ in $G$, for different values of $m'$, whenever $s \re y s'$ in $S$.
We will therefore define transition $(q,m) \re{c:y} (q',m')$ in $\B^\eps$ if $m'$ matches suitably many transitions as above; for now, we postpone the precise definition.

We should then verify that the obtained automaton $\B$:
\begin{itemize}
    \item is a $k$-blowup of $\A$,
    \item is $\eps$-complete, and
    \item recognises $W$.
\end{itemize}
The first two items above state that $\B$ should have many transitions: at least those inherited from $\A$, and in addition a number of $\eps$-transitions.
This creates a tension with the third item, which states that even with all these added transitions, the automaton $\B$ should not accept too many words.

Let us focus on the third item for now, which will lead to a correct definition for $\B$.
Take an accepting run
\[
    (q_0,m_0) \re{c_0:y_0} (q_1,m_1) \re{c_1,y_1} \dots
\]
in $\B$, where $x = \liminf_i y_i$ is even; for the sake of simplicity, assume that all $y_i$'s are $\geq x$.
We should show that its labelling $w=c_0c_1\dots$ belongs to $W$.
To this end, we will decorate the run with labels $s_0,s_1,\dots \in S$ so that
\[
    (q_0,m_0,s_0) \re{c_0} (q_1,m_1,s_1) \re{c_1} \dots
\]
defines a path in $G$, which concludes since $G$ satisfies $W$.

Recall that the elements of $S$ are tuples of ordinals $<\kappa$ indexed by $d/2$ odd priorities. We use $s_{<x}$ (resp. $s_>x$)to refer to a tuple indexed by priorities up to $x-1$ (resp. from $x+1$).
%  the prefix up to position $y$.

To construct the $s_i$'s, we fix a well chosen prefix $s_{<x}\in \kappa^{x_\odd}$ which will be constant, and proceed as follows.

\begin{enumerate}[(a.)]
    \item If $y_i=x$, then we set $s_i=s_{<x} s_{>x}$, for some $s_{>x}$ which depends only on $c_i$.
    \item If $y_i<x$, then we set $s_i=s_{<x} s_{>x}$, for some $s_{>x}$ which depends on $c_i$ as well as $s_{i+1}$.
\end{enumerate}

At this stage the reader may be worried that the backward induction underlying the above definition is not well-founded; however, since the first case occurs infinitely often, the backward induction from the second item is only performed over finite blocks (see also Figure~\ref{fig:backtrack} in Appendix~\ref{app:existence-eps-automata}).

This leads to the following definition for $\B$, where $x$ is an even priority:
\[
    \begin{array}{rcl}
     (q,m) \re{c:x} (q',m') \tin \B  &\iff&  \exists s_{<x} \exists s_{>x} \forall s'_{>x}, (q,m,s_{<x}s_{>x}) \re{c} (q',m',s_{<x} s'_{>x}) \tin G, \\
     (q,m) \re{c:x+1} (q',m') \tin \B  &\iff&  \exists s_{<x} \forall s'_{>x} \exists s_{>x}, (q,m,s_{<x}s_{>x}) \re{c} (q',m',s_{<x} s'_{>x}) \tin G.
    \end{array}
\]

The first line corresponds to point (a.) above, where $s_{>x}$ can be chosen independently of $s'_{>x}$, whereas the second line corresponds to point (b.), since the choice of $s_{>x}$ is conditioned on the value of $s'_{>x}$.
(For priorities $>x+1$, we may apply either the first or second line, depending on the parity, to get the required conclusion.)

The remaining issue is that the choice of the fixed common prefix $s_{<x}$ should be made uniformly, regardless of the transition.
This is achieved thanks to an adequate extraction lemma (which extends the pigeonhole principle to the case at hands), which finds a large enough subset $T$ of $\kappa^{\d_\odd}$, so that transitions $(q,m,s)\re{}(q',m',s')$ are similar for different choices of $s,s' \in T$.
This ensures that $s_{<x}$ can be chosen uniformly.

There remains to verify that $\B$ is indeed a $k$-blowup of $\A$ and that it is $\eps$-complete, which will follow easily from the definitions and $\eps$-completeness of $G$ (this is because, after removing ``$\exists s_{<x}$'' from the definition above, the second line resemble the negation of the first).

For the chromatic case, the proof is exactly the same, we should simply check that if $G$ is chromatic (which is guaranteed by Theorem~\ref{thm:structuration}), then the so is the obtained automaton $\B$.

\subsection{Existence of deterministic $\eps$-completable automata}\label{sec:existence-det-automata}

We now prove the implication from (\ref{item:existence-automata}) to (\ref{item:existence-det-automata}) in Theorem~\ref{thm:main-charac}.
Take $\A$ to be a deterministic automaton recognising $W$, and let $\B^\eps$ be the obtained $k$-blowup of $\A$ which is (chromatically) $k$-wise $\eps$-complete and recognises $W$.
Then let $\B$ be obtained from $\B^\eps$ by only keeping, for each state $(q,m) \in V(\B)$ and each transition $q \re{a:y} q'$ in $\A$, a single transition of the form $(q,m) \re {a:y} (q',m')$ chosen arbitrarily.
Note that $\B$ is a $k$-blowup of $\A$, so we have: $L(\A) \subseteq L(\B) \subseteq L(\B^\eps)$.
We conclude that $\B$ is a deterministic (chromatically) $k$-wise $\eps$-completable automaton recognising $W$, as required.


\subsection{From deterministic $\eps$-completable automata to universal graphs}\label{sec:existence-universal-graphs}

We now prove the implication from (\ref{item:existence-det-automata}) to (\ref{item:existence-universal-graph}) in Theorem~\ref{thm:main-charac}.
This result was already proved in~\cite[Prop.~5.30]{CO24Arxiv} for the case of $k=1$; extending to greater values for $k$ presents no difficulty.

Let $\B$ be a deterministic\footnote{For the purpose of this proof, history-determinism would be sufficient (we refer to~\cite{BL23SurveyHD} for the definition and context on history-determinism).} (chromatic) $k$-wise $\eps$-completable automaton recognising $W$ and $\B^\eps$ be an $\eps$-completion.
Fix a cardinal $\kappa$ and let $S$ denote $S_d^\kappa$.
Consider the cascade product $U = \B^\eps \casc S$.
We claim that $U$ is (chromatic) $k$-wise $\eps$-complete and $(\kappa,W)$-universal.
Universality follows from the facts that $\B$ is deterministic and $S$ is $(\kappa,\Parity_d)$-universal.

\begin{claim}
    The graph $U = \B^\eps \casc S$ is $(\kappa,W)$-universal.
\end{claim}

\begin{claimproof}
    Take a tree $T$ of size $<\kappa$ which satisfies $W$.
    We should show that $T \to U$.
    Since $\B$ is deterministic, we can define a labelling $\rho\colon V(T) \to V(\B)$ by mapping $t_0$ to $q_0$ and $t \mapsto q$ if the run of $\B$ on the finite word labelling the path $t_0\re{}t$ ends in $q$. Then, any infinite path from $t_0$ in $T$ is mapped to a run in $\B$ that is accepting (since $T$ satisfies $W$).
    Therefore the tree $T'$ obtained by taking $T$ and replacing edge-labels by the priorities appearing in their $\rho$-images satisfies $\Parity_d$ and has size $<\kappa$, so there is a morphism $\mu:T \to S$.
    It is a direct check that $(\rho,\mu) \colon V(T) \to V(\B) \times V(S) = V(U)$ indeed defines a morphism.
\end{claimproof}

Showing that $U$ is $k$-wise $\eps$-complete is slightly trickier.

\begin{claim}
    The graph $U = \B^\eps \casc S$ is well-founded and (chromatic) $k$-wise $\eps$-complete.
\end{claim}

\begin{claimproof}
    Well-foundedness of $U$ follows directly from Lemma~\ref{lem:cascade_products}.
    Let us write $B_1,\dots,B_k$ for the $k$ parts of $\B^\eps$; the $k$ parts of $U$ will be $B_1 \times V(S),\dots, B_k \times V(S)$.
    If applicable, chromaticity is a direct check.
    Let $(b,s),(b',s') \in V(U)$ be in the same part, i.e.~$b,b' \in B_i$ for some $i$.
    Let $x_0$ be the minimal even priority such that 
    $b \re{\eps:x_0} b'$ in $\B^\eps$ 
    (if such an $x$ does not exist then $x_0=d+2$).
    Then let $y_0$ be the minimal odd priority such that $s'_{\leq y_0} > s_{\leq y_0}$ (as previously, if $s=s'$ then we let $y_0=d+1$).
    We distinguish two cases.
    \begin{enumerate}[(1)]
        \item If $x_0 < y_0$. Then we have $b \re{\eps:x_0} b'$ in $\B^\eps$ and $s_{<x_0} = s'_{<x_0}$ which gives $s \re{x_0} s'$ in $S$.
        Thus we get $(b,s) \re{\eps} (b',s')$ in $U$.
        \item If $y_0 < x_0$. Then $s'_{\leq y_0} > s_{\leq y_0}$, which gives $s' \re{y_0} s$ in $S$.
        Since $\B^\eps$ is $k$-wise $\eps$-completable and by definition of $x_0$, we also have $b' \re{\eps:y_0} b$ in $\B^\eps$, therefore $(b',s') \re{\eps} (b,s)$ in $U$.
    \end{enumerate}

    We conclude that either $(b,s)\re{\eps}(b',s')$ or $(b',s') \re{\eps} (b,s)$, as required.
\end{claimproof}
