\subsection*{Context: Strategy complexity in infinite duration games}

We study infinite duration games on graphs in which two players, called Eve and Adam, interact by  moving a token along the edges of a (potentially infinite) edge-coloured directed graph. Each vertex belongs to one player, who chooses where to move next during a play. This interaction goes on for an infinite duration, producing an infinite path in the graph. The winner is determined according to a language of infinite sequences of colours~$W$, called the objective of the game; Eve aims to produce a path coloured by a sequence in $W$, while Adam tries to prevent this.
This model is widespread for its use in verification and synthesis~\cite{HandbookModelChecking2018}.

% \subparagraph{$\oo$-regular and $\BCSigma$ languages.} A central class of objectives is given by $\oo$-regular languages. These languages are those recognised by finite deterministic parity automata, by non-deterministic B\"uchi automata, or definable in monadic second order logic~\cite{Buchi1962decision, McNaughton1966Testing, Mostowski1984RegularEF}. 
% The results in this work apply not only to $\oo$-regular languages, but to a larger class: $\BCSigma$ languages. These are boolean combinations of languages in $\bsigma 2$ (countable unions of closed languages), or equivalently, those recognised by deterministic parity automata with infinite many states. This class includes typical non-$\omega$-regular objectives such as energy or mean-payoff objectives, as well as...
% \ac{citations?}
% \po{it could be that we also include other classes of languages which correspond to infinite (but finitely representable) parity automata. I am not familiar with the literature, but maybe things like omega-pushdown, or vass, or parikh stuff, or timed stuff, idk. Also we could mention quantitative automata, I guess some reasonable ones are in $\BC(\bsigma 2)$} 

% A crucial parameter, both for the algorithmic resolution of games and for the conciseness of the obtained controller,  is the complexity of strategies required to win in those games. 
% A seminal result of B\"uchi and Landweber~\cite{BL69Strategies} states that in \po{finite} games where the objective is an $\oo$-regular language, the winner has a winning strategy that can be implemented by a finite automaton.
% A common measure of the complexity of a strategy is the size of such an automaton.
% The memory of an objective $W$ is the minimal $k$ such that whenever Eve wins a game with objective $W$, she has a winning strategy implemented by an automaton with $\leq k$ states. For $\oo$-regular objectives, this is always finite~\cite{BL69Strategies,Gurevich1982trees}. 
% Depending on whether these automata are allowed to read the edges of the game graph or are only allowed to read their colours, we speak of general or chromatic memory.

In order to achieve their goal, players use strategies, which are representations of the course of all possible plays together with instructions on how to act in each scenario.
In this work, we are interested in optimal strategies for Eve, that is, strategies that guarantee a victory whenever this is possible. More precisely, we are interested in the complexity of such strategies, or in other words, in the succinctness of the representation of the space of plays.

\subparagraph*{Positionality.}
The simplest strategies are those that assign in advance an outgoing edge to each vertex owned by Eve, and always play along this edge, disregarding all the other features of the play.
All the information required to implement such a strategy appears in the game graph itself.
Objectives for which such strategies are sufficient to play optimally are called positional (or memoryless).
Understanding positionality has been the object of a long line of research. The landmark results of Gimbert and Zielonka~\cite{GZ05} and Colcombet and Niwinski~\cite{CN06} gave a good understanding of which objectives are bi-positional, %; unfortunately these results apply only for bi-positionality, 
i.e.~positional for both players.

More recently, Ohlmann proposed to use universal graphs as a tool for studying positionality (taking  Eve's point of view)~\cite{Ohlmann23}.
This led to many advances in the study of positionality~\cite{BCRV24HalfJournal,OS24Sigma2}, and most notably, a characterisation of positional $\omega$-regular objectives by Casares and Ohlmann~\cite{CO24Positional}, together with a polynomial time decision procedure (and some other important corollaries, more discussion below).

\subparagraph*{Strategies with memory.} 
However, in many scenarios, playing optimally requires distinguishing different plays that end in the same vertex.
A seminal result of B\"uchi and Landweber~\cite{BL69Strategies} states that in finite games where the objective is an $\oo$-regular language, the winner has a winning strategy that can be implemented by a finite automaton processing the edges of the game; this result was later extended to infinite game graphs by Gurevich and Harrington~\cite{Gurevich1982trees}.
Here, the states of the automaton are interpreted as memory states of the strategy, and a natural measure of the complexity of a strategy is the number of such states.
More precisely, the memory of an objective $W$ is the minimal $k$ such that whenever Eve wins a game with objective $W$, she has a winning strategy with $k$ states of memory.
For $\oo$-regular objectives, this is always finite~\cite{BL69Strategies,Gurevich1982trees}, while the case of positionality discussed above corresponds to memory $k=1$.

\subparagraph*{Chromatic versus general memory.}
In the special case where these automata are only allowed to read the colours on the edges of the game graph, we speak of chromatic memory.
In his PhD thesis, Kopczy\'nski showed that, for prefix-independent $\omega$-regular objectives and over finite game graphs, the chromatic memory can be computed in exponential time~\cite[Theorem~8.14]{Kop08Thesis}.
Recently, it was shown that computing the chromatic memory of some restricted subclasses of $\oo$-regular objectives is in fact $\NP$-complete: for Muller objectives~\cite{Casares22Chromatic} and for topologically open or closed objectives~\cite{BFRV23Regular}.
%Their characterisation results for chromatic memory apply more generally to any open or closed objective, not necessarily $\omega$-regular.

For the more natural model of not-necessarily chromatic memory (which we will simply call memory\footnote{In the literature, this is sometimes called general memory, or chaotic memory.}), results are sparser. Notable ones include the characterisation for memory of Muller objectives by Dziembowski, Jurdzi\'nski, and Walukiewicz~\cite{DJW1997memory}, or the memory of closed objectives~\cite{CFH14}.
However, these are all rather restricted classes of $\omega$-regular objectives.
Prior to this work, even computing the memory of open $\omega$-regular objectives (sometimes called regular reachability objectives) was not known to be decidable.

%\subparagraph{Finite-to-infinite lift.} In general, the memory of an objective may differ if we consider only finite game graphs or arbitrary ones (a well-known such example is the mean-payoff objective\footnote{Ohlmann and Skrzypczak recently showed that if defined as $\{w \in [-N,N]^\omega \mid \limsup w < 0\}$, the mean-payoff objective is even positional over infinite games~\cite{OS24Sigma2}. However, the complement of this objective is only positional on finite arenas.}).
%In his PhD thesis, Vandenhove conjectured that this is not the case for $\oo$-regular objectives~\cite[Conjecture~9.1.2]{Vandenhove23Thesis}\footnote{Formally, the conjecture is stated for arena-independent memories (a slightly different setting).}: their memory is the same over finite and infinite games. 
%This conjecture has recently been proved in the case of positional (memoryless) objectives, i.e., those with memory equal to one~\cite[Theorem~3.4]{CO24Positional}.


\subparagraph{Unions of objectives.} The driving question in Kopczyński's PhD thesis~\cite{Kop08Thesis} is whether prefix-independent positional objectives are closed under union, which has become known as Kopczyński's conjecture. Recently, Kozachinskiy~\cite{Kozachinskiy24EnergyGroups} disproved this conjecture, but only for positionality over finite game graphs, and using non-$\oo$-regular objectives. In fact, the conjecture is now known to hold for $\oo$-regular objectives~\cite{CO24Positional} and $\bsigma 2$ objectives~\cite{OS24Sigma2}. 
Casares and Ohlmann proposed a generalisation of this conjecture from positional objectives to objectives requiring memory~\cite[Conjecture~7.1]{CO25LMCS} (see also~\cite[Proposition~8.11]{Kop08Thesis}):

\begin{conjecture}[{Generalised Kopczyński's conjecture}]\label{conj:Kopcz-Union-Memory}
	Let $W_1,W_2\subseteq \SS^\oo$ be two prefix-independent objectives with memory $k_1$ and $k_2$, respectively. Then $W_1\cup W_2$ has memory at most $k_1 \cdot k_2$.
\end{conjecture}

Using the characterisation of~\cite{DJW1997memory}, it is not hard to verify that the conjecture holds for Muller objectives.

\subparagraph*{$\BCSigma$ languages.}

The results in this work apply not only to $\oo$-regular languages, but to the broader class of $\BCSigma$ languages.
These are boolean combinations of languages in $\bsigma 2$ (countable unions of closed languages), or equivalently, recognised by deterministic parity automata with infinitely many states.
This class includes typical non-$\omega$-regular examples such as energy or mean-payoff objectives, but also broader classes such as unambiguous $\oo$-petri nets~\cite{FSJLS22} and deterministic $\oo$-Turing machines (Turing machines with a Muller condition).


\subsection*{Contributions}

Our main contribution is a characterisation of $\BCSigma$ objectives with memory $\leq k$, stated in Theorem~\ref{thm:main-charac}.
It captures both the memory and the chromatic memory of objectives over infinite game graphs.
The characterisation is based on the notion of $k$-wise $\epsilon$-completable automata, which are parity automata with states partitioned in $k$ chains, where each chain is endowed with a tight hierarchical structure encoded in the $\eps$-transitions of the automaton.

From this characterisation, we derive the following corollaries:
\begin{enumerate}
	\item \bfDescript{Decidability in $\NP$.} Given a deterministic parity automaton $\A$, the memory (resp. chromatic memory) of $L(\A)$  can be computed in $\NP$.% (Theorem~\ref{??}).
	
	% \item \bfDescript{Finite-to-infinite lift.} If an $\oo$-regular objective has memory (resp. chromatic memory) $\leq k$ over finite games, then it the same is true over arbitrary games. % (Theorem~\ref{??}).
	% This settles a conjecture of Vandenhove~\cite[Conjecture~9.1.2]{Vandenhove23Thesis} in the affirmative.

	\item \bfDescript{Generalised Kopczyński's conjecture.} We establish (and strengthen) Conjecture~\ref{conj:Kopcz-Union-Memory} in the case of $\BCSigma$ objectives: if $W_1$ and $W_2$ are $\BCSigma$ objectives with memory $k_1$ and
	 $k_2$, and one of them is prefix-increasing, then the memory of $W_1 \cup W_2$ is $\leq k_1\cdot k_2$.%  (Theorem~\ref{??}).
\end{enumerate}


\subparagraph{Main technical tools: Universal graphs and $\eps$-completable automata.}
As mentioned above, Ohlmann proposed a characterisation of positionality by means of universal graphs~\cite{Ohlmann23}.
In 2023, Casares and Ohlmann extended this characterisation to objectives with memory $\leq k$ by considering partially ordered universal graphs~\cite{CO25LMCS}.
Until now, universal graphs have been mainly used to show that certain objectives have memory $\leq k$ (usually for $k=1$); this is done by constructing a universal graph for the objective.
One technical novelty of this work is to exploit both directions of the characterisation, as we also rely on the existence of universal graphs to obtain decidability results.

Our characterisation is based on the notion of $k$-wise $\eps$-completable automata, which extends the key notion of~\cite{CO24Positional} from positionality to finite memory.  

\subparagraph*{Comparison with~\cite{CO24Positional}.}
In 2024, Casares and Ohlmann characterised positional $\oo$-regular objectives~\cite{CO24Positional}, establishing decidability of positionality in polynomial time, and settling Kopczyński's conjecture for $\oo$-regular objectives.
%The proof of Casares and Ohlmann requires intricate combinatorial arguments about parity automata.
Although the current paper generalises the notions and some of the results from~\cite{CO24Positional} to the case of memory, as well as potentially infinite automata, the proof techniques are significantly different: while~\cite{CO24Positional} is based on intricate successive transformations of parity automata, ours is based on an extraction method in the infinite and manipulates ordinal numbers.
Though somewhat less elementary, our proof is notably shorter, and probably easier to read.

When instantiated to the case of memory $1$, we thus extend Kopczyński's conjecture from $\omega$-regular objectives to positional $\BCSigma$ objectives.
This also gives an easier proof of decidability of positionality in polynomial time\footnote{It is explained in~\cite[Theorem~5.3]{CO24Positional} how decidability of positionality in polynomial time can be derived, with reasonable efforts, from Kopczy\'nski's conjecture.} for $\omega$-regular objectives.
However, some of the results of~\cite{CO24Positional} are not recovered with our methods: the finite-to-infinite and 1-to-2-player lifts, as well as closure of positionality under addition of neutral letters.