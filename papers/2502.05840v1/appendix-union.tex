We let $\A^\eps$ be the automaton defined in Section~\ref{sec:union} (i.e. a version of the cascade product of $\A_{1}^\eps \times \A_{2}^\eps$ with~$\T$). In this appendix we prove:
\AepsCompleteUnion*

%Before going on to the proof of Lemma~\ref{lem:A-eps-complete-union}, 
First, we need a few remarks on the structure of $\eps$-complete automata.
We write $q \rer{\eps:x+1} q'$ to denote the conjunction of $q \re{\eps:x+1} q' $ and $q' \re{\eps:x+1} q$.
Given two states $q,q'$ in the same part of a $k$-wise $\eps$-complete automaton, we call breakpoint priority of $q$ and $q'$ the least even $\xbreak$ such that $q \rer{\eps:\xbreak+1} q'$ does not hold.
Note that this is a property of the unordered pair $\{q,q'\}$.
Observe also that by the definition of $\eps$-completeness, we have either $q \re{\eps:\xbreak} q'$ or $q' \re{\eps:\xbreak} q$. %, where $\xbreak$ is the breakpoint priority.
Moreover, assuming that $q \re{\eps:\xbreak} q'$, we also get that there can be no $q' \re{\eps:x} q$, for even $x$, otherwise we would accept some run labelled by $\Sigma^* \eps^\omega$; therefore for even $x \geq \xbreak$ we also have $q \re{\eps:x} q'$.
To sum up, if $\xbreak$ is the breakpoint priority of $\{q,q'\}$ and $q \re{\eps:\xbreak} q'$, then: 
\begin{itemize}
    \item $q\rer{\eps:y} q'$ for all odd $y<\xbreak$;
    \item $q\re{\eps:x} q'$ for all $x\geq \xbreak$; and
    \item there is no transition $q'\re{\eps:x} q$ for even $x$.
\end{itemize}
Finally, observe that in $\A_2^\eps$, since there are no transitions with priority $0$ (and therefore $\re{\eps:1}$ connects every ordered pair of states), breakpoint priorities are always $\geq 2$.


We are now ready to prove Lemma~\ref{lem:A-eps-complete-union}

\begin{proof}[Proof of Lemma~\ref{lem:A-eps-complete-union}]
    %For $l=1,2$, let $V(\A_l) = Q^1_l,\dots,Q_l^{k_l}$ be the parts making $\A_l^\eps$ $k_l$-wise $\eps$-complete.
    %This induces a partition of the states of $\A^\eps$ into the $k_1k_2$ parts  $Q_1^i \times Q_2^j\times V(\T)$.
    First observe that the $k_1$ parts of $\A_1^\eps$ and the $k_2$ parts of $\A_2^\eps$ naturally induce a partition of the states of $\A^\eps$ into $k_1k_2$ parts.
    Let $r=(q_{1},q_{2},\tau)$ and $r'=(q'_{1}, q'_{2},\tau')$ be two states in the same part of $\A^\eps$, that is, $q_{l}$ and $q'_{l}$ are in the same part in $\A_{l}^\eps$, for $l=1,2$.
    %, and likewise $q_{2}$ and $q'_{2}$ are in the same part in $\A_{2}^\eps$.
    We will show that for some even output priority $\xbreak$, it holds that:
    \begin{enumerate}[1.]
        \item\label{item:small-odd} %for every even $x < \xbreak$, 
        $r \rer{\eps:\xbreak-1} r'$; and
        \item\label{item:large-even} %for every even $x \geq \xbreak$, 
        either $r \re{\eps:\xbreak} r'$ or $r' \re{\eps:\xbreak} r$.
    \end{enumerate}
    Note that since $r\re{\eps:t} r'$ in $\A^\eps$ implies $r \re{\eps:t'} r'$ for all $t'\cleq t$, the two points above imply that for every even $x < \xbreak$ we have $r \rer{\eps:x+1} r'$ and for every even $x \geq \xbreak$, 
    either $r \re{\eps:\xbreak} r'$ or $r' \re{\eps:\xbreak} r$.
    Therefore, this will prove that $\A^\eps$ is $(k_{1}k_{2})$-wise $\eps$-complete.

    Let $\xbreak_{1}$ and $\xbreak_{2}$ denote the breakpoint priorities of $\{q_{1},q'_{1}\}$, $\{q_{2}, q'_{2}\}$ in $\A_1^\eps$ and $\A_2^\eps$, respectively (even and $\geq 2$).
    Let $i_T$ be the largest index such that $\tau[..i_T] = \tau'[..i_T]$, with $i_T=0$ if $\tau[1]\neq \tau'[1]$.
    %We let $\xbreak_T = \tau[i_T]+1$ (even) if $i_T>0$, and $\xbreak_T = 0$ if $i_T=0$.
    We distinguish two cases, depending on whether some $\xbreak_l-1$ appears in  $\tau[..i_T]$.

    \begin{description}
        \item[a) $i_T < \indtau{\xbreak_1}, \indtau{\xbreak_2}$.]
        Note that, in particular, $0<\xbreak_{1}, \xbreak_2$. %.< d= d_1 + d_2$ and $\tau \neq \tau'$.
        
        We show that in this case, we can set $\xbreak=2i_T+2$. We prove the two points above:
        \begin{itemize}
            \item[\ref{item:small-odd}.] We will find odd priorities $y_{1}$ and $y_{2}$ such that
            \[
            \underbrace{q_{1} \re{\eps:y_{1}} q'_{1}}_{\tin \A_{1}}\tand  \underbrace{q_{2} \re{\eps:y_{2}} q'_{2}}_{\tin \A_{2}} \tand \underbrace{\tau \re{(y_{1}, y_{2}):\xbreak-1} \tau'}_{\tin \overline \T},
            \]
            which gives the wanted result when applied symmetrically in the other direction.
            Consider the element $\tau[i_T+1]$ (odd), and assume w.l.o.g. that it belongs to $\done$. We let $y_1 = \tau[i_T+1]$.
            %Let $x_1 = \tau[i+1]+1$ (even).
            Note that $1\leq y_1< \xbreak_1$ , as $i_T+1\leq \indtau{\xbreak_1}$, and therefore we have $q_1 \rer{\eps:y_1} q'_1$. 
            We let $y_2 = \xbreak_2 - 1$;
            by definition of $\xbreak_2$ we have $q_2 \rer{\eps:y_2} q'_2$.
            %Now $\xbreak_2$ appears after $x$, which is at position $x+2$ in $\tau$, and therefore 
            As $\indextau{\tau}{y_2} = \indextau{\tau}{\xbreak_2} > \indextau{\tau}{\xbreak_1} $ we have the third wanted transition $\tau \re{(y_1,y_2):2i_T + 1} \tau'$ in $\overline \T$, as wanted.
            By flipping $\tau$ and $\tau'$ and applying the same reasoning, we get the transition $r' \re{\eps:\xbreak-1} r$, as required (note that we also have $i_T < \indextau{\tau'}{\xbreak_1}, \indextau{\tau'}{\xbreak_2}$ by definition of $i_T$).
            
            \item[\ref{item:large-even}.] We assume w.l.o.g. that $\tau' < \tau$.
            By definition of $i_T$, we have that $\tau' <_{i_T+1} \tau$. 
            %which amounts to saying that $\tau[i_T+1]\in \dtwo$, whereas $\tau'[i_T+1]\in \done$.
            Let $z = \tau[i_T+1]$ (which belong to $\dtwoodd$ if $\tau' <_{i_T+1} \tau$).
            As $z\leq \xbreak_2$, we have $q_2 \re{\eps:z} q'_2$. Also, $q_1 \re{\eps:\xbreak_1-1} q'_1$.
            Using point (3) of the definition of $\overline \T$, we have $\tau \re{(\xbreak_1-1,z):2i_T+2} \tau'$.
            We conclude that $r \re{\eps:\xbreak} r'$.
        \end{itemize}
        \item[ b) Either $\indtau{\xbreak_1}\leq i_T$ or $\indtau{\xbreak_2}\leq i_T$.]
        We assume w.l.o.g. that $\indtau{\xbreak_1} < \indtau{\xbreak_2}$.
        We let $\xbreak = 2 \indtau{\xbreak_1}$; note that $\tau \re {(\xbreak_1,\xbreak_2-1):\xbreak}$ and  $\tau \re {(\xbreak_1-1,\xbreak_2-1):\xbreak-1}$ in $\T$.
        We verify the two cases highlighted above.
        \begin{itemize}
            \item[\ref{item:small-odd}.]
            We have $q_l \rer{\eps:\xbreak_l-1} q_l$ for $l=1,2$.
            As $\tau[..i_T] = \tau'[..i_T]$, thanks to rule (1) in the definition of $\overline \T$, we have $\tau \rer{(\xbreak_1-1,\xbreak_2-1):\xbreak-1} \tau'$, and so we get $r \rer{\eps:\xbreak-1} r'$, as required.
            \item[\ref{item:large-even}.] We have one of the transitions $q_1 \re{\eps:\xbreak_1} q_1'$ or $q_1' \re{\eps:\xbreak_1} q_1$; assume we are in the first case. Since $q_2 \rer{\eps: \xbreak_2-1} q'_2$, we also have $\tau \re{(\xbreak_1,\xbreak_2-1):x} \tau'$, so we conclude that $r \re{\eps:x} r'$.\qedhere
        \end{itemize}
    \end{description}
\end{proof}

This concludes the proof of Theorem~\ref{thm:union}.