% !TEX root = main.tex

In this section, we establish a strong form of the generalised Kopczyński conjecture for $\BCSigma$ objectives.
An objective $W \subseteq \Sigma^\omega$ is prefix-increasing\footnote{In other papers~\cite{Ohlmann21PhD,Ohlmann23,CO25LMCS} this notion is called prefix-decreasing, as Eve is seen as a ``minimiser'' player who aims to minimise some quantity.}
 if for all $a \in \Sigma$ and $w \in \Sigma^\omega$, it holds that if $w \in W$ then $aw \in W$.
In words, one remains in $W$ when adding a finite prefix to a word of $w$.
Examples of prefix-increasing objectives include prefix-independent and closed objectives.
\ac{Other interesting clases? }


\begin{theorem}\label{thm:union}
    Let $W_1, W_2 \subseteq \SS^\oo$ be two $\BC(\bsigma 2)$ objectives over the same alphabet, such that $W_{2}$ is prefix-increasing.
    Assume that $W_{1}$ has memory $\leq k_1$ and $W_{2}$ has memory $\leq k_2$.
    Then $W_1 \cup W_2$ has memory $\leq k_1 k_2$.
\end{theorem}

\begin{remark}
The assumption that one of the two objectives is prefix-increasing is indeed required: for instance if $W_1=aa(a+b)^\omega$ and $W_{2}=bb(a+b)^\omega$, which are positional but not prefix-increasing, the union $(aa+bb)(a+b)^\omega$ is not positional (it has memory $2$).

\end{remark}
\begin{remark}
    The bound $k_1k_2$ in Theorem~\ref{thm:union} is tight: For every $k_1, k_2$, there are objectives $W_1$, $W_2$ with memories $k_1, k_2$ respectively, such that $W_1 \cup W_2$ has memory exactly $k_1k_2$. 
    One such example is as follows: let $\SS = \{a_1,\dots, a_{k_1}, b_1,\dots, b_{k_2}\}$ and 
    $W_1 = \{ w \mid w \text{ contains at } \text{least } \text{two } \text{different } a_i \text{ infinitely often} \}$ and $W_2 = \{ w \mid w$ contains at least two  $b_i$ infinitely often$\}$. 
    We can see that $W_1$, $W_2$ and $W_1\cup W_2$ have memory, respectively, $k_1, k_2$, and  $k_1\cdot k_2$ by building the Zielonka tree of these objectives and applying~\cite[Thms.~6,~14]{DJW1997memory}.
\end{remark}

The rest of the section is devoted to the proof of Theorem~\ref{thm:union}.
We explicit a construction for the union of two parity automata, inspired from the Zielonka tree of the union of two parity conditions, and show that it is $(k_1 k_2)$-wise $\eps$-completable.



\subparagraph{Union of parity languages.} We give an explicit construction of a deterministic parity automaton $\mathcal{T}$ recognising the union of two parity languages, which may be of independent interest.  This corresponds to the automaton given by the Zielonka tree of the union (a reduced and structured version of the LAR).

We let $\done = \{0, 1,\dots,d_1\}$  and $\dtwo= \{1^*,\dots,d_2^*\}$.
Let $\doneodd$ and $\dtwoodd$ denote the restrictions of $\done$ and $\dtwo$ to odd elements.
By a slight abuse of notation, we sometimes treat elements in $\dtwo$ as natural numbers (e.g. when comparing them).\\

\textit{Alphabet.} 
The input alphabet is $\done \times \dtwo$, and we write letters as $(y,z)$. 
The index of $\T$  is $d_1 + d_2$, we use the letter $t$ for its output priorities.\\


\textit{States}. 
States are given by interleavings of two strictly increasing sequences of $\doneodd$ and $\dtwoodd$. Any state can be taken as initial. For instance, for $d_1=d_2=6$, an example of a state is:\\[-4mm]
\[  \hspace{15mm} \tau = \langle \tikzmark{tau1}1,\tikzmark{tau2}3,\tikzmark{tau3}1^*,\tikzmark{tau4}5, \tikzmark{tau5}3^*, \tikzmark{tau6}5^* \rangle .\]

\begin{tikzpicture}[remember picture, overlay] %%Magie d'Aliaume
    \foreach \lbl in {1,...,6} {
        \node[below, xshift = 1.4mm, yshift = -1mm, scale=0.7, color = lipicsGray] at (pic cs:tau\lbl) {$\lbl$};
    }
\end{tikzpicture}

% \begin{tikzpicture}
%     \centering
%     \node[] at (0,0.2) {};
%     \node[] at (4,-0.2) {$\tau =$};
%     \node[] at (6,0) {$\langle 1,3,1^*, 5, 3^*, 5^* \rangle$};
%     \node[scale=0.7] at (6,-0.4) {$1 \hspace{0.35cm} 2 \hspace{0.35cm}3 \hspace{0.42cm} 4 \hspace{0.35cm} 5 \hspace{0.42cm} 6$};
%     \node[] at (7.5,-0.2) {.};
% \end{tikzpicture}

\vspace{-2.5mm}

We use $\tau$ to denote such a sequence, which we index from $1$ to $(d_1+d_2)/2$, and write $\tau[i]$ for its $i$-th element.
For $x\in \done$, $x$ odd,  we let $\indtau{x} = i$ for the index such that $\tau[i]=x$. For $x\geq 2$ even, we let $\indtau{x}$ be the index $i$ such that $\tau[i]=x-1$. We let $\indtau{0} = 0$. We use the same notation for $y\in \dtwo$, $y\geq 1$. For example, in the state above, $\indtau{2^*}=3$.

The intuition is that a state stores a local order of importance between input priorities.\\

%Formally, we let $\leq_\tau$ denote the linear order over ?? induced by $\tau$, and $\min_\tau$ the corresponding minimum function.
\textit{Transitions.}
Let $\tau$ be a state and $(y,z)$ an input letter. We define the transition $\tau \re{(y,z):t}\tau'$ as follows:

Let $i = \min \{ \indtau{y}, \indtau{z}\}$. 
In the following, we assume $i = \indtau{y}$ (the definition for $i = \indtau{z}$ is symmetric). 
We let $t = 2i$, if $y$ even, and  $t = 2i-1$ if $y$ is odd.
If $y$ is even, we let $\tau' = \tau$.
If $y$ is odd, let $i'$ be the smallest index $i<i'$ such that $\tau[i']\in \dtwo$, and let $\tau'$ be the sequence obtained by inserting $\tau[i']$ on the left of $\tau[i]$ (or $\tau' = \tau$ if no such index $i'$ exists). 
Formally,
\[ \tau'[j] = \tau[j] \tfor j<i \tand i'<j,  \quad \tau'[i] = \tau[i'] \quad \tand \quad \tau'[j] = \tau[j-1] \tfor i<j\leq i' .\]

For example, for the state above, we have:
$ \langle 1,3,1^*, 5, 3^*, 5^* \rangle  \re{ (3,2^*):3} \langle 1, 1^*, 3, 5, 3^*, 5^* \rangle. $\\
%We let $\outtau(x,y)$ be the output priority of the transition $\tau \re{(x,y)}$.
\begin{lemma}\label{lem:language-of-T}
    The automaton $\T$ recognises the language \[L = \{w \in (\done \times \dtwo)^\omega \mid \pi_1(w) \in \Parity_{d_1} \ \mathrm{or} \ \pi_2(w) \in \Parity_{d_2}\}.\]
\end{lemma}
\begin{proof}
    We show that $L\subseteq L(\T)$, the other inclusion is similar (and implied by Lemma~\ref{lem:overline-T} below).
    Let $(y_1,z_1)(y_2,z_2)\dots \in W$, and assume w.l.o.g. that  $y_1y_2\dots \in \Parity_{d_1}$.
    Let $\ymin = \liminf y_i$, which is even, and let $n_0$ be so that for all $n \geq n_0$ we have $\ymin \leq y_n$.
    Let
    \[
        \tau_0 \re{(y_1,z_1):t_1} \tau_1 \re{(y_2,z_2):t_2} \dots
    \]
    denote the corresponding run in $\T$.
    Let $i_n = \indextau{\tau_n}{\ymin}$ be the index where $\ymin-1$ appears in $\tau_n$. 
    Note that the sequence $(i_n)_{n\geq n_0}$ is decreasing. %, as the index of $\xmin$'s position can only increase if an odd priority smaller than $\xmin$ is read in the $\done$ component. 
    Let $n_1$ be the moment where this sequence stabilises, i.e., $i_n = i_{n_1}$ for $n\geq n_1$.
    By definition of the transitions of $\T$, for $n\geq n_1$ all output priorities are $\geq 2i_{n_1}$, and priority $2i_{n_1}$ is produced every time that a letter $(\ymin, z_n)$ is read.
    We conclude that $\T$ accepts $w$.
\end{proof}


\subparagraph{$0$-freeness of automata for prefix-increasing objectives.} The fact that $W_2$ is prefix-increasing will be used via the following lemma. It recasts the fact that we can add $\re{\eps:1}$ transitions everywhere to automata recognising prefix-increasing objectives.

\begin{lemma}\label{lem:prefix-increasing}
    Let $W$ a be prefix-increasing objective with memory $\leq k$.
    There exists a deterministic $k$-wise $\eps$-completable automaton $\A$ recognising $W$ and an $\eps$-completion $\A^\eps$ of $\A$ such that $\A^\eps$ does not have any transition with priority $0$.
\end{lemma}

\begin{proof}
    First, take any automaton $\A_0$ recognising $W$.
    Then let $\A_1$ be obtained by shifting every priority from $\A_0$ by $2$.
    Clearly $\A_1$ also recognises $W$ and does not have any transition with priority $0$.
    Then apply Proposition~\ref{prop:existenceEpsComplete} to get a $k$-blowup $\A$ of $\A_1$ which is $k$-wise $\eps$-completable, and let $\A^\eps$ denote the corresponding $\eps$-completion.
    Since $\A$ has no transition with priority $0$, the only possible such transitions in $\A^\eps$ are $\eps$-transition.
    Then remove all $\eps$-transitions with priority $0$ in $\A^\eps$, and add $\re{\eps:1}$ transitions between all pairs of states in $\A^\eps$ (in both directions).

    Clearly, the obtained automaton $\tilde \A^\eps$ is $\eps$-complete and has no transition with priority $0$.
    There remains to prove that it recognises $W$.
    Take an accepting run in $\tilde \A^\eps$ and observe that the priority $1$ is only seen finitely often.
    Hence from some moment on, the run coincides with a run in $\A^\eps$.
    We conclude since $W$ is prefix-increasing.
\end{proof}



\subparagraph{Main proof: $\eps$-completion of the product.}
We now proceed with the proof of Theorem~\ref{thm:union}.
Using Theorem~\ref{thm:main-charac}, for $l =1,2$, we take  deterministic $k_l$-wise $\eps$-completable automata $\A_l$ of index $d_l$ recognising $W_l$, and its $\eps$-completion $\A_{l}^\eps$. 
For $l=2$, we assume thanks to Lemma~\ref{lem:prefix-increasing} that $\A_{2}^\eps$ does not have any transition with priority $0$.


We consider the product $\A=(\A_{1} \times \A_{2}) \casc \T$, with states $V(\A_1) \times V(\A_2) \times V(\T)$ and transitions $(q_1,q_2,\tau) \re{a:t} (q'_1,q'_2,\tau')$ if $q_1\re{a:y}q_1'$ in $\A_1$, $q_2\re{a:z}q_2'$ in $\A_2$, and $\tau \re{(y,z):t} \tau'$ in~$\T$. 
The correctness of such a construction is folklore.\footnote{$\A_1\times \A_2$ can be seen as a Muller automaton with acceptance condition the union of two parity languages. The composition with $\T$ yields a correct parity automaton, as $\T$ recognises the acceptance condition.}


\begin{claim}
    The automaton $\A$ is deterministic and recognises $W=W_{1} \cup W_{2}$.
\end{claim}


Therefore, there only remains to show the following lemma.

\begin{lemma}\label{lem:product-with-T-epsCompl}
    The automaton $\A=(\A_{1} \times \A_{2}) \casc \T$ is $(k_1k_2)$-wise $\eps$-completable.
\end{lemma}

The $\eps$-completion of $\A$ will be a variant of a product of the form $\A_1^\eps \times \A_2^\eps \casc \overline \T$, where $\overline \T$ is a non-deterministic extension of $\T$ with more transitions, but which still recognises the same language.\\

\textit{The automaton $\overline \T$.}
Intuitively, we obtain $\overline \T$ by allowing to reconfigure the elements of index $>i$ in a state $\tau$ by paying an odd priority $2i-1$, as well as allowing to move elements of $\done$ to the left. We precise this idea next.

We order the states of $\T$ lexicographically, where we assume that $x < y$ for $x\in \done$ and $y\in \dtwo$.
Formally, we let $\tau < \tau'$ if for the first position $j$ where $\tau$ and $\tau'$ differ, $\tau[j]\in \done$ (and therefore necessarily $\tau'[j]\in \dtwo$).
We let $\tau[..i]$ be the prefix of $\tau$ up to (and including) $\tau[i]$.
We write $\tau <_i \tau'$ if $\tau[..i] < \tau'[..i]$. % and moreover $\tau[..i] \neq \tau'[..i]$. %$j\leq i$, and $\tau' \leq_i \tau$ for its negation.

%We describe the transitions in $\overline{\T}$. 
Let $\tau \re{(y,z):t}$ be a transition in $\T$ as above, and $i_0 = \min \{ \indtau{y}, \indtau{z}\}$ be the index determining $t$ (i.e. $t\in \{ 2i_0-1, 2i\}$). The automaton $\overline \T$ contains a transition $\tau \re{(y,z):t'} \tau'$ if:

\begin{enumerate}
    \item $t' = t$ is odd, and $\tau' \leq_{i_0-1} \tau$; or
    \item $t' = t$ is even, and $\tau' \leq_{i_0} \tau$; or
    \item $t' \in \{2i'-1, 2i'\}$ for some $i' \leq i_0$ and $\tau' <_{i'} \tau$. (Note that, if $t$ is odd, this includes all (possibly even) $t'\leq t+1$.)
\end{enumerate}

In words, we are allowed to output a small (i.e. important) priority when following a strict decrease on sufficiently small components in $\tau$. 
%, and we may reset large coordinates of $\tau$ when the priority corresponding to the input (relative to the ordering prescribed by $\tau$) is read.
Note that transitions in $\T$ also belong to $\overline \T$ thanks to the rules (1) and (2).

% \begin{itemize}
%     \item  $\tau \re{(x,y):z'} \tau'$, for $z' \cleq  z$ and all $\tau'$ such that $\tau'[j] = \tau[j]$, for $j<i_0$.
%     \item $\tau \re{(x,y):2i-1} \tau'$, for all $\tau'$ such that $\tau'[j] = \tau[j]$, for $j<i$,
%     \item if $\tau[i]\in \dtwo$, $\tau \re{(x,y):z} \tau'$, for all $\tau'$ such that $\tau'[j] = \tau[j]$, for $j<i$, $\tau'[i]\in \done$ and $z\cleq 2i$.    
% \end{itemize}

\begin{lemma}\label{lem:overline-T}
    The automaton $\overline \T$ recognises the same language as $\T$.
\end{lemma}
\begin{proof}
    It is clear that $L(\T) \subseteq L(\overline \T)$. We show the other inclusion.
    %First, note that if $t$ is even and $\tau \re{\_ : t} \tau'$ in $\overline \T$, then $\tau' \leq_{t/2-1} \tau$.

    Consider
    \[\tau_0 \re{(y_1,z_1):t_1} \tau_1 \re{(y_2,z_2):t_2} \dots , \quad \text{ an accepting run in } \overline \T.\]
    %Without loss of generality and for the sake of simplicity, we may assume that if $\tau_i \re{(y_i,y_i):z}$ in $\T$, then $z\cleq z_i$. 

    Let $\tmin=\liminf t_1t_2\dots$, which is even.
    Let $\imin = \tmin/2$ be the index responsible for producing priority $\tmin$.
    Let $n_0$ be such that $t_n \geq \tmin$ for all $n \geq n_0$.
    Observe that for all $n \geq n_0$, we have $\tau_n \geq_{(\imin-1)} \tau_{n+1}$, and therefore there is $n_1 \geq n_0$ such that the prefix $\tau_n[..\imin-1]$ is the same for all $n \geq n_1$.
    In fact, the prefix $\tau_n[..\imin]$ must be constant too, as we can only modify $\tau_n[\imin]$ using rule (1.) and that would output priority $\tmin -1$.
    Assume w.l.o.g. that $\tau_n[\imin] = y\in \doneodd$.
    Now, for each $n \geq n_1$ it must be that $y_i\geq y+1$ and it must be that $y_i = y + 1$ each time that priority $\tmin$ is produced. 
    Therefore, $\liminf y_1y_2\dots = y+1$ is even.     
\end{proof}


\textit{The $\eps$-completion.} 
We define $\A^\eps$ as a version of the cascade product of $\A_{1}^\eps \times \A_{2}^\eps$ with~$\T$, in which $\eps$-transitions are also allowed to use the transitions of $\overline \T$.
We let $\cleq$ denote the preference ordering over priorities, given by
$ 1 \cleq 3 \cleq \dots\cleq d-1 \cleq d \cleq  \dots 2 \cleq 0 \text{ ($d$ even)}$.
The transitions in $\A^\eps$ are defined as follows:
\begin{itemize}
    \item For $a\in \SS$:
    $(q_1,q_2,\tau) \re{a:t} (q'_1,q'_2,\tau')$ if this transition appears in $(\A_{1} \times \A_{2})\casc \T$. %$q_1\re{a:y_1}q_1'$ in $\A_1$, $q_2\re{a:y_2}q_2'$ in $\A_2$, and $\tau \re{(y_1,y_2):y} \tau'$ in $\T$.
    \item $(q_1,q_2,\tau) \re{\eps:t} (q'_1,q'_2,\tau')$ if $q_1\re{\eps:y}q_1'$ in $\A_1^\eps$, $q_2\re{\eps:z}q_2'$ in $\A_2^\eps$, and $\tau \re{(y,z):t'} \tau'$ in $\overline \T$ with $t' \cleq t$.
\end{itemize}

Note that the condition $t' \cleq t$ simply allows to output a less favorable priority, so it does not create extra accepting runs.
By definition, $\A^\eps$ has been obtained by adding $\eps$-transitions to $\A$.
It is a folklore result that composition of non-deterministic automata also preserves the language recognised, so this construction is correct.
\begin{claim}
    The automaton $\A^\eps$ recognises $W$.
\end{claim}

%Therefore, in order to prove  and conclude by using Theorem~\ref{thm:aut-to-univ-graph}, there remains to show that $\A^\eps$ is $(k_{1}k_{2})$-wise $\eps$-complete. 
Therefore, we there only remains to prove the following lemma.

\begin{restatable}{lemma}{AepsCompleteUnion}\label{lem:A-eps-complete-union}
     The automaton $\A^\eps$ is $(k_{1}k_{2})$-wise $\eps$-complete. 
 \end{restatable}

The formal proof of this statement is presented in Appendix~\ref{app:unon}.
The $k_1$ parts of $\A_1^\eps$ and the $k_2$ parts of $\A_2^\eps$ naturally induce a partition of the states of $\A^\eps$ into $k_1k_2$ parts.
Then, given two states $r = (q_1,q_2,\tau)$ and $r' = (q_1',q_2', \tau')$ in the same part of $\A^\eps$, we consider the longest common prefix of $\tau$ and $\tau'$. We perform a case analysis: Depending on the priorities of $\done$ or $\dtwo$ appearing in this prefix, and the transitions $q_l \re{\eps:x} q_l'$ of the automata $\A_1^\eps$, $\A_2^\eps$, we will find transitions $r\re{\eps:x} r'$ or $r'\re{\eps:x+1} r$ for all even $x$.
