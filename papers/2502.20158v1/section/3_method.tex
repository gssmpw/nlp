\section{Method}
\label{sec:method}

\begin{figure}[t]
    \centering
    \scriptsize
    \begin{tcolorbox}[colback=blue!5!white, colframe=blue!40!black, title=Task I, width=\linewidth]
        
Right after the above code block, you need to predict and fill in the blank \_\_\_\_\_ in the following assertion statement:

\texttt{```}python
\{ cloze\_question\_with\_blank \}
\texttt{```}

Use the context and the provided prefix in the unit test method to infer the missing part of the assertion.


    \end{tcolorbox}
    \begin{tcolorbox}[colback=blue!5!white, colframe=blue!40!black, title=Task II, width=\linewidth]

Here is the checklist:
    
    The generated test code should contain ALL of the following calls:
    
    - \{target\_name\}, a Python \{type\} defined in file \{file\_name\}.

\end{tcolorbox}

    \begin{tcolorbox}[colback=blue!5!white, colframe=blue!40!black, title=Task III, width=\linewidth]
\#\# Coverage Requirement

Please ensure that the following lines of code from the specified files are included and tested. Complete code snippets have already been provided earlier. When referring to coverage, it pertains to pytest coverage. If your code calls a function or method that indirectly utilizes the specified lines, this will also be considered as covering those lines.

\#\#\# \{block.file\_path\}

\texttt{```}python
\{ block.code\_snippet \}
\texttt{```}

This includes \{block.line\_count\} lines from line numbers \{block.start\_line\_number\} to \{ block.end\_line\_number \}.

[more code blocks if available]

\#\# Summary

In summary, there are \{num\_files\} files and \{num\_code\} code segments that need to be covered. 
Please make an effort to enhance the coverage of these lines, either directly or indirectly, to ensure thorough testing with pytest.
    \end{tcolorbox}

    \caption{Task specific prompt template for Task I, II and III. For the complete prompts, check Sec~\ref{app:prompt}.}
    \label{fig:prompt_box}
\end{figure}
\subsection{Preliminaries}
\noindent {\bf Action recognition with CLIP-based video learner.}
% 
Pre-training on large-scale image-text data based on contrastive learning, CLIP learns separate uni-modal encoders for image and text, embedding them into a joint feature space, respectively.
% 
Consider a CLIP-based video learner with a ViT architecture~\cite{dosovitskiy2020image}, that incorporates temporal modeling for video understanding~\cite{wu2023revisiting,wang2021actionclip,weng2023open,zhu2023orthogonal,yang2023aim,wasim2023vita}.
% 
Next, we present the standard vision-only fine-tuning paradigm that applies such a video learner $f_{\theta_v}$ with a frozen text encoder $f_{\theta_t}$ to action recognition. 
% 
Specifically, given a video clip $V_i$, and a candidate action label $T_j\in \mathcal{Z}_{tr}$ described in predefined textual templates (\eg, ``\textit{a video of \{action\}}'') from the training set $\mathcal{D}_{tr}$, the similarity is calculated as:
\begin{equation}
    s_{i,j}=\frac{\left \langle v_i,t_j \right \rangle }{\left \| v_i \right \| \left \| t_j \right \| },v_i=f_{\theta_v}(V_i),t_j=f_{\theta_t}(T_j),
\end{equation}
where the training objective is to maximize it of the matched $V_i$ and $T_j$, or to minimize it otherwise. The loss function is implemented by the cross-entropy loss in~\cite{wu2023revisiting,chen2021empirical,radford2021learning} as:
\begin{equation}
\label{eq:ce}
    \mathcal{L}_{CE} = -\frac{1}{B}\sum_{i}^{B}\sum_{k}^{K}y_{i,k}\log \left ( \frac{\exp(s_{i,k})}{\sum_{j}^{K}\exp (s_{i,j})}  \right ),  
\end{equation}
where $B$ and $K$ denote the minibatch size and the number of all known classes, respectively. If the $i$-th video belongs to the $k$-th class, $y_{i,k}$ equals $1$; otherwise, $y_{i,k}$ equals $0$.
% 
In OVAR, the trained video learner should achieve good generalization on test data with the class label $T_i\in \mathcal{Z}_{te}$, where $\mathcal{Z}_{te} \cap \mathcal{Z}_{tr}=\emptyset $.

\noindent {\bf Model-agnostic meta-learning (MAML).}
MAML~\cite{finn2017model} is a gradient-based meta-optimization framework designed for few-shot learning, which aims to learn good initialization such that a few gradient steps will lead to fast learning on new tasks.
% 
Formally, consider a model $f_\theta$ with parameters $\theta$, MAML learns a set of initial weight values, which will serve as a good starting point for fast adaptation to a new task $\mathcal{T}_i$, sampled from a task distribution $p(\mathcal{T})$.
% 
When adapting to the task $\mathcal{T}_i$, the fast weights $\theta_i'$ are computed \textit{w.r.t.} examples from $\mathcal{T}_i$ though single inner-loop update as:
\begin{equation}
    \theta_i'=\theta-\alpha \nabla_\theta \mathcal{L}_{\mathcal{T}_i}(f_\theta),
\end{equation}
where $\alpha$ denotes the step size for inner loops. Then, the model with fast weights $f_{\theta_i'}$ is evaluated on new samples from the same task $\mathcal{T}_i$, to act as the feedback (\ie, loss gradients) to adapt to current task $\mathcal{T}_i$ to optimize the initialization $\theta$ for generalization as:
\begin{equation}
\label{eq:maml}
    \theta \leftarrow \theta-\beta \nabla_\theta \sum_{\mathcal{T}_i}\mathcal{L}_{\mathcal{T}_i}(f_{\theta_i'}).
\end{equation}
where $\beta$ is the step size for outer loops. 
Computationally, due to the additional backward propagation burden of the gradient by gradient update, MAML presents a first-order approximation, FOMAML, by dropping the backward pass. 
% Additionally, Reptile~\cite{nichol2018first} algorithm defines the update as:
% \begin{equation}
% \label{eq:reptile}
%     \theta \leftarrow \theta+ \epsilon(\theta_i'-\theta),
% \end{equation}
% where $(\theta-\theta_i')/\alpha$ can serve as the gradient term with scaling factor $\epsilon$ for updating $\theta$ for adaptation to task $\mathcal{T}_i$.

\subsection{Open-MeDe}
% 【(—)说现有】
As discussed above, the standard fine-tuning paradigm can cause the video learner to overfit to the known classes during training, leading to poor zero-shot capabilities. Also, CLIP regularization-based approaches face challenges in achieving robust generalization due to the excessive reliance on superficial static cues in videos.
% 【(+)说解决】
To tackle these issues, we draw upon the philosophy and methodology from meta-learning, and propose Open-MeDe framework, which is illustrated in~\cref{fig:overview}, to enhance both know-to-open generalizing and image-to-video debiasing simultaneously.

\subsubsection{Cross-batch meta-optimization}
% 【(+)说方法】
Our Open-MeDe framework primarily adopts a cross-batch meta-optimization scheme (in~\cref{fig:overview}(a)) to enhance the video learner via \textit{meta training and testing}, enabling it to acquire generalizable, video-specific knowledge instead of overly exploiting static biases. 
% 【说划分】
Note that we neither sample from a distribution of $N$-way $K$-shot tasks as done in few-shot MAML nor deliberately split the training set into support and query sets as Meta-ZSL~\cite{liu2021task,verma2020meta} suggested. Instead, our support and query examples are constructed effortlessly and arbitrarily by the default training data sampler.
% 【说cross-batch好处】
In effect, we consider this arbitrariness a blessing for building the natural ``\textit{known-to-open generalization task}'', since the known biases in \textit{meta training} data do not hold in \textit{meta testing} data due to different inherent label distributions across batches.
% 【说meta-training】
A known-to-open task can be created by extending the original gradient step into two consecutive mini-batches in one pass, with the current batch acting as support data and the subsequent batch as query data.
Specifically, in line with the episode-wise training akin to MAML, we first train the learner within an inner loop (\ie, \textit{meta training}), where the fast weights are obtained through a single gradient step for each support batch.
% 【说meta-testing】
Following this adaptation, in the outer loop, query videos are sampled to evaluate the generalization performance of the adapted learner with fast weights (\ie, \textit{meta testing}).
% 【说优化】
In this work, our framework further updates the fast weights of the learner based on the evaluation performance during \textit{meta testing}, which then provides feedback for the task to derive more generalizable optimization for the learner.

\noindent {\bf Meta training.}
At each training iteration, we first utilize each support batch $\mathcal{S}=\{V_i,T_i\}^B$ from the task $\mathcal{T}_i$ to train the video learner $f_{\theta}$ (with parameters $\theta$), via one standard gradient step. The inner loop update is governed by the loss on the support batch as:
\begin{equation}
\label{eq:spt}
     \mathcal{L}_{\mathcal{T}_i}^\mathcal{S}(\theta) =\mathcal{L}(f_{\theta }(\mathcal{S})),
\end{equation}
where $\mathcal{L}(\cdot)$ refers to the loss function (\eg, the cross-entropy loss $\mathcal{L}_{CE}$ \wrt Eq.~\eqref{eq:ce}). Then, we make a temporary copy for the original parameters $\theta$ and update the intermediate parameters for fast weights as follows:
\begin{equation}
\label{eq:spt_update}
    \theta'_i = \theta-\alpha \nabla_{\theta} \mathcal{L}_{\mathcal{T}_i}^{\mathcal{S}}(\theta),
\end{equation}
where $\alpha$ denotes the learning rate for \textit{meta training}. Intuitively, this step simulates a direct update to train the learner to obtain class-specific knowledge of the support data.

\noindent {\bf Meta testing.}
After meta training on the support batch, we then scheme a virtual testing process, leveraging the query batch $\mathcal{Q}=\{V_i, T_i\}^B$, where $\mathcal{S}\cap \mathcal{Q}=\emptyset$,  to evaluate the generalization performance of the base learner $f_{\theta_i'}$. Formally, we measure the known-to-open performance on $\mathcal{T}_i$ by calculating the class-specific loss concerning the query data as:
\begin{equation}
\label{eq:qry}
    \mathcal{L}_{\mathcal{T}_i}^\mathcal{Q}(\theta'_i) =\mathcal{L}(f_{\theta_i' }(\mathcal{Q})).
\end{equation}
Here, the formulation closely relates to the standard fine-tuning process, which aims to obtain decent class-specific performance for all training batches. Differently, this step merely evaluates the intermediary base learner for its known-to-open generalizability on each task, due to the original parameters $\theta$ remaining immune to the task-specific updates.
% 
Hence, it can be used to provide feedback on \textit{what video-specific knowledge should be learned in the sense that the learner can derive the robust generalization across different class distributions against inherent known and static biases} in the following meta-optimization.

\noindent {\bf Meta-optimization.}
As mentioned above, the intuition behind our approach is that the virtual evaluation during meta testing can provide useful feedback to encourage the learning of more robust representations for fast known-to-open generalization after \textit{meta training} on the support data (\ie, $\theta'_i \leftarrow \theta$).
Note that original MAML approaches focus on optimizing parameters for a strong initialization, enabling quick adaptation to new tasks with minimal gradient updates. Conversely, open-vocabulary recognition requires zero-shot capabilities, where no further adaptation can be applied for new tasks. Therefore, class-specific knowledge should be strengthened in terms of global optimization.
To this end, within the outer loop, the parameters of the learner are optimized to minimize the class-specific errors for the support data and the adaptation cost for the query data simultaneously. 
The combination of both Eq.~\eqref{eq:spt} and Eq.~\eqref{eq:qry} is used to carry out the outer loop update, thus the objective for meta-optimization can be defined as:
\begin{equation}
\label{eq:maml*}
\begin{aligned}
\min _{\theta}\mathcal{L}_{\mathcal{T}_i}(\theta )=& \min _{\theta}\left( \mathcal{L}_{\mathcal{T}_i}^\mathcal{S} (\theta) +\mathcal{L}_{\mathcal{T}_i}^\mathcal{Q}(\theta_i') \right)  \\
= & \min _{\theta} \left(\mathcal{L}_{\mathcal{T}_i}^\mathcal{S} (\theta)+\mathcal{L}_{\mathcal{T}_i}^\mathcal{Q}(\theta-\alpha \nabla_{\theta} \mathcal{L}_{\mathcal{T}_i}^{\mathcal{S}}(\theta))\right).
\end{aligned}
\end{equation}
Here, the first term refers to the class-specific knowledge learned on the support batch, while the second term provides the known-to-open generalization feedback based on $\theta_i'$ towards robust representation learning \wrt the task $\mathcal{T}_i$.
The optimizing process of the parameter $\theta$ can be given by:
\begin{equation}
\label{eq:maml**}
    \theta \leftarrow \theta - \beta \nabla_{\theta} \sum _{i = 1}^{N}\left( \mathcal{L}_{\mathcal{T} _i}^\mathcal{S} (\theta) + \mathcal{L}_{\mathcal{T} _i}^\mathcal{Q} \left ( \theta - \alpha \nabla_{\theta} \mathcal{L}_{\mathcal{T} _i}^\mathcal{S} (\theta) \right) \right),
\end{equation}
where $N$ is the batch size of the task for meta-optimization.
Since the MAML meta-gradient update needs to differentiate through the optimization process (\ie, a gradient by a gradient), it’s not an ideal solution where we need to optimize a large number of tasks during the training phase. 
Therefore, we opt for the one-step update approximation by dropping the backward pass of $\theta \gets \theta'_i$, where Eq.~\eqref{eq:maml**} can be rewritten as:
\begin{equation}
\label{eq:FOMAML}
    \theta \leftarrow \theta - \beta  \sum _{i=1}^{N}\left (\nabla_{\theta}\mathcal{L}_{\mathcal{T} _i}^\mathcal{S} (\theta) +\delta \nabla_{\theta'_i} \mathcal{L}_{\mathcal{T}_i}^{\mathcal{Q}}(\theta'_i)\right),
\end{equation}
where $\beta$ and $\delta$ are the learning rates for meta-optimization.
With the genuine update of the learner in Eq.~\eqref{eq:FOMAML} without CLIP regularization, we can optimize a parallel or batch version that evaluates on $N$ known-to-open tasks of different class distributions (\ie, class-specific knowledge), which encourages to learn more generalizable features against known and static biases.

\begin{table*}[!ht]
\centering
\caption{Performance comparison (Top1-Acc (\%)) with the CLIP-adapted methods using ViT-B/16 under the in-context base-to-novel setting. We also report the harmonic mean (HM) of base and novel recognition accuracy. The \textbf{best} and the \underline{second-best} results are highlighted. $\ast$ and $\dagger$ denote the results reproduced with our implementation using frozen text learners.}
\label{tab:B2N}
\resizebox{0.8\textwidth}{!}{
\begin{tabular}{lcccccccccccc}
\toprule
\multirow{2}{*}{Method} &
  \multicolumn{3}{c}{K400} &
  \multicolumn{3}{c}{HMDB} &
  \multicolumn{3}{c}{UCF} &
  \multicolumn{3}{c}{SSv2} \\ \cmidrule(lr){2-4} \cmidrule{5-7} \cmidrule{8-10} \cmidrule{11-13} 
 &
  Base &
  Novel &
  \multicolumn{1}{c|}{HM} &
  Base &
  Novel &
  \multicolumn{1}{c|}{HM} &
  Base &
  Novel &
  \multicolumn{1}{c|}{HM} &
  Base &
  Novel &
  HM \\ \cmidrule{1-13}
Frozen CLIP~\cite{radford2021learning} &
  62.3 &
  53.4 &
  \multicolumn{1}{c|}{57.5} &
  53.3 &
  46.8 &
  \multicolumn{1}{c|}{49.8} &
  78.5 &
  63.6 &
  \multicolumn{1}{c|}{70.3} &
  4.9 &
  5.3 &
  5.1 \\
ActionCLIP~\cite{wang2021actionclip} &
  61.0 &
  46.2 &
  \multicolumn{1}{c|}{52.6} &
  69.1 &
  37.3 &
  \multicolumn{1}{c|}{48.5} &
  90.1 &
  58.1 &
  \multicolumn{1}{c|}{70.7} &
  13.3 &
  10.1 &
  11.5 \\
X-CLIP~\cite{ni2022expanding} &
  74.1 &
  56.4 &
  \multicolumn{1}{c|}{64.0} &
  69.4 &
  45.5 &
  \multicolumn{1}{c|}{55.0} &
  89.9 &
  58.9 &
  \multicolumn{1}{c|}{71.2} &
  8.5 &
  6.6 &
  7.4 \\
VPT~\cite{ju2022prompting} &
  69.7 &
  37.6 &
  \multicolumn{1}{c|}{48.8} &
  46.2 &
  16.0 &
  \multicolumn{1}{c|}{23.8} &
  90.5 &
  40.4 &
  \multicolumn{1}{c|}{55.8} &
  8.3 &
  5.3 &
  6.4 \\
ST-Adapter~\cite{pan2022st} &
  74.6 &
  62.0 &
  \multicolumn{1}{c|}{67.3} &
  65.3 &
  48.9 &
  \multicolumn{1}{c|}{55.9} &
  85.5 &
  76.8 &
  \multicolumn{1}{c|}{80.9} &
  9.3 &
  8.4 &
  8.8 \\
ViFi-CLIP~\cite{rasheed2023fine} &
  \underline{76.4} &
  61.1 &
  \multicolumn{1}{c|}{67.9} &
  \textbf{73.8} &
  \underline{53.3} &
  \multicolumn{1}{c|}{\underline{61.9}} &
  92.9 &
  67.7 &
  \multicolumn{1}{c|}{78.3} &
  \underline{16.2} &
  \underline{12.1} &
  \underline{13.9} \\
Open-VCLIP $\ast$~\cite{weng2023open} &
  76.3 &
  \underline{62.3} &
  \multicolumn{1}{c|}{\underline{68.6}} &
  70.2 &
  50.2 &
  \multicolumn{1}{c|}{58.5} &
  \underline{94.6} &
  \underline{77.2} &
  \multicolumn{1}{c|}{\underline{85.0}} &
  15.9 &
  10.8 &
  12.9 \\
FROSTER $\dagger$~\cite{huang2024froster}&
  76.0 &
  61.9 &
  \multicolumn{1}{c|}{68.3} &
  70.0 &
  49.9 &
  \multicolumn{1}{c|}{58.3} &
  94.3 &
  76.9 &
  \multicolumn{1}{c|}{84.7} &
  15.5 &
  10.3 &
  12.4 \\
\rowcolor{red!10} 
\textbf{Open-MeDe} &
  \textbf{77.2} &
  \textbf{63.8} &
  \multicolumn{1}{c|}{\textbf{69.9}} &
  \underline{73.6} &
  \textbf{56.4} &
  \multicolumn{1}{c|}{\textbf{63.9}} &
  \textbf{94.9} &
  \textbf{78.5} &
  \multicolumn{1}{c|}{\textbf{85.9}} &
  \textbf{17.1} &
  \textbf{12.3} &
  \textbf{14.3} \\ \bottomrule
\end{tabular}
}
\end{table*}
\vspace{-0.3em}
\subsubsection{Gaussian self-ensemble stabilization}
% 【(—)说传统】
Typically, training the video learner for longer iterations to gain specialization on the supervised tasks comes with the risk of diminished plasticity and generalizability. 
Model patching~\cite{ilharco2022patching,wortsman2022robust,shu2023clipood,weng2023open} of weight ensembling has been shown to improve both the performance and generalization.
% 【(*)说谨慎利用】
Given that the fine-tuning videos are limited in class-specific knowledge, while the open-vocabulary tasks are unconstrained, the static generalizable flexibility derived from large-scale I-VL pre-training should be scrupulously exploited to enhance the adaptation of the video learner while minimizing the impact of static bias.
% 【(+)说GWA】
Therefore, we further incorporate self-ensemble stabilization tailored to the video learner over its optimization trajectory, which utilizes the knowledge from previous training iterations for a generalizable solution.
% 
In a fine-tuning procedure of $R$ epochs with $l$ step length for each, the learner's optimization trajectory is represented by $\{\theta_t\}_{t=0}^R$, where $\theta_0$ is the pre-trained weights. The self-ensemble averages the weights of the learner as:
\begin{equation}
    \theta_{\texttt{WA}}= \sum_{t=0}^{R}\frac{w_t}{\sum_{i=0}^{R}w_i}\cdot \theta_t,
\end{equation}
where $w_t$ specifies the weight contributed by the parameters at $t$-th epoch.
% 【(—)说早期】
Intuitively, during the early fine-tuning epochs (\ie, at a smaller epoch $t$), the video learner lacks the maturity to effectively capture video-specific knowledge while still retaining substantial static-related orientation from large-scale pre-training, which introduces vulnerable information for temporal understanding. 
% 【(—)说晚期】
Conversely, the parameters at the last few epochs (\ie, at a larger epoch $t$) have integrated more video-specific knowledge, highly featuring the supervised downstream task distribution, whereas the plasticity of the unconstrained zero-shot capability is not guaranteed.
% 【(—)说动机】
As both sides degrade the final open-vocabulary generalizability, we aim to weaken the contribution of the parameters near the initial and terminal epochs by employing a distribution prior, resulting in a generic optima for robust generalization.

% \section{With what \emph{data}?}\label{sec:data}
\section{How do we \emph{train models}?}\label{sec:model_training}

% Which scales? How? 
%  - [DONE] Too small may not hold
%  - [DONE] number of models can affect confidence intervals
%  - [DONE] counting size also matters data (tokens vs bits) and flop counts kaplan vs hoffman CITE; non-embedding vs embedding 
%  - DITCH? how you scale architecture matters; for eg - clark shows scaling N and E is needed to separate; enc and decoder scaling can be different for architectures; in the 
%  - [DONE] 6ND is a common approximation - does not hold for very long context (these days 128k to 1M)


In order to fit a scaling law, one needs to train a range of models across multiple orders of magnitude in model size and/or dataset size. Researchers must first decide the range and distribution of $N$ and $D$ values for their training runs, in order to achieve stable convergence to a solution with high confidence, while limiting the total compute budget of all experiments. Many papers did not specify the number of data points used to fit each scaling law; those that did range from 4 to several hundred, but most used fewer than 50 data points. The specific $N$ and $D$ values also skew the optimization process towards a certain range of $N/D$ ratios, which may be too narrow to include the true optimum. Some approaches, such as using IsoFLOPs \citep{hoffmann2022training}, additionally dictate rules for choosing $N$ and $D$ values. Moreover, using a minimum $N$ or $D$ value may result in outlier values that may need to be dropped \citep{porian2024resolving,shin2023scaling,henighan2020scaling}. We investigate this choice in Section \S\ref{sec:repl-model_training}

The definition of $N$, $D$, or compute cost $C$ can affect the results of a scaling study. For example, if a study studies variation in tokenizers, a definition of training data size based on character count may be more appropriate than one based on token count \citep{tao2024scaling}. The inclusion or exclusion of embedding layer compute and parameters, may also skew the results of a study - a major factor in the different in optimal ratios determined by \cite{kaplan2020scaling} and \cite{hoffmann2022training} has been attributed to not factoring embedding FLOPs into the final compute cost \citep{pearce2024reconcilingkaplanchinchillascaling, porian2024resolving}. Given the increase in extremely long context models (128k-1M) \cite{reid2024gemini}, the commonly used training FLOPs approximation $C = 6 ND$ (see Appendix \ref{app:full-details}) may not hold for such models, given the additional cost proportional to the context length and model dimension - \citet{bi2024deepseek} introduce a new terms non-embedding FLOPs/token to account for this.

% \ml{we've gotta decide where this discussion should go }

% \luke{I agree we could probably cut some of the next few paragraphs for space if needed. The last paragraph ends well though I think.}
% \srk{ DITCH? how you scale architecture matters; for eg - clark shows scaling N and E is needed to separate; enc and decoder scaling can be different for architectures}

 % - counting size also matters data (tokens vs bits) and flop counts kaplan vs hoffman CITE; non-embedding vs embedding 

% The goal of scaling laws is generally to extrapolate findings to larger compute budgets. It is unclear which $N$ and $D$ values should be included in the data in order to predict loss at a larger scale. 


% \srk{discuss outliers being dropped and therefore, need to be sure about the scale of training}



% The scaling law form identifies the relevant variables (e.g., $N$ and $D$). However, there remain many decisions affecting the way in which data is selected to for scaling law optimization.

%  - [DONE] knowing data composition matters because knowing data quality can change exponent across different studies ofc CITE


% Moreover, hyperparams can matter
%  - [DONE] For example, learning rate schedule can changes results CITE
%  - [DONE] batch size can change
%  - [DONE] optimal hparams change with scale so determining those matters
%  - [DONE] embedding size has been shown to matter - part of scaling law 



Scaling law fit depends on the performance of each individual checkpoint, which is highly dependent on factors such as training data source, architecture and hyperparameter choice. \citet{bansal2022data} and \citet{goyal2024scaling}, for instance, discuss the effect of data quality and composition on power law exponents and constants. Repeating data has also been found to yield different scaling patterns in large language models \citep{muennighoff2024scaling,goyal2024scaling}. 

Researchers have also studied the effect of architecture choice on scaling - \citet{hestness2017deep} find that architectural improvements only shift the irreducible loss, while \citet{poli2024mechanistic} suggest that these improvements may be more significant. The way in which a model is scaled can also affect results. Within the same architecture family, \citet{clark2022unified} show that increasing the number of experts in a routed language model has diminishing returns beyond a point, while \citet{ghorbani2021scaling} find that scaling the encoder and decoder have different effects on model performance. Scaling embedding size can also drastically change scaling trends \citep{tao2024scaling}.

The optimal hyperparameters to train a model changes with scale. Changing batch size, for example, can change model performance \cite{mccandlish2018empirical, kaplan2020scaling}. Optimal learning rate is another hyperparameter shown to change with scale, though techniques such as those proposed in Tensor Programs series of papers \citep{yang2022tensor} can keep this factor constant with simple changes to initialization. More specifically, changing the learning rate schedule from a cosine decay to a constant learning rate with a cooldown (or even changing the learning rate hyperparameters) has been found to greatly affect the results of scaling laws studies \citep{hu2024minicpm,porian2024resolving,hagele2024scaling, hoffmann2022training}.

% \ml{add paragraph about max model params}

% \srk{talk abt }

% - LR \citep{hu2024minicpm}.
% - batch size

% Some factors, such as training data distribution, do not have a single optimal value and may simply be held constant for all experiments, but they change the absolute value of all power law parameters, e.g. it is not meaningful to compare power laws fit to experiments on different training data distributions. Other factors, such as batch size, may have optimal values depending on other hyperparameters, data budget, architecture, etc. It is sometimes possible to fix these factors in relation to others (e.g., scale batch size with the parameter count). There are other factors for which there is no smooth interpolation between points. For example, the width and depth of a transformer are limited to integer values, and further limited by the convention of using widths equal to small multiples of powers of 2.

% \srk{we should concretely discuss learning rate schedules and cite miniCPM, the LR schedule rate paper}
One common motivation for fitting a scaling laws is extrapolation to higher compute budgets. However, there is no consensus on the orders of magnitude up that one can project a scaling law and still find it accurate, nor on the breadth of compute budgets that should be covered by the data. We find that the range of model size $N$ and dataset size $D$ greatly varies, with the maximum value of $N$ in each paper ranging from 10M parameters to around 7B and that of $D$ being as large as 400B tokens. 
% \textcolor{blue}{
For most papers we survey, the scales are relatively modest: 13 of 51 papers train models beyond 2B parameters; most only train models smaller than 1B parameters.
% }
% \srk{idk what this refers to: Some papers show projected results at as much as 6 or 7 orders of magnitude greater than the compute budgets they experimented with. } 
It has been shown, with some controversy \cite{schaeffer2023emergent}, that scaling to significantly larger scales can result in new abilities that did not appear in smaller models \citep{wei2022emergent}. Forecasting limits to extrapolation and the appearance of new abilities at new scales is an open question.

% 
% - while SOTA models go till 100s of B; 
% - scaling laws often only go until 1b
% - it remains unclear what the corrcet scale is

% \srk{mention emergent abilities; and counterpt}

% \srk{Training data [DONE] and downstream metrics chosen may completely change exponent?}
% 【(+)说GWA】
Driven by~\cite{khattak2023self} in prompt learning, we perform Gaussian Weight Average (GWA) based on model patching, as shown in~\cref{fig:overview}(b), which assigns the parameters with lower weights at initial epochs, higher weights at middle epochs, and relatively lower weights at final epochs.
% 【(+)】
Given a  Gaussian distribution $w_t\sim \mathcal{N}(\mu,\sigma^2)$ defined over the epochs, we sample the weight values for the parameters $\theta_t$ as its corresponding probability in the distribution as:
\begin{equation}
    w_t=\frac{1}{\sqrt{2 \pi} \sigma} e^{-\frac{(t-\mu)^{2}}{2 \sigma^{2}}}, t=1,\dots,R.
\end{equation}
Here, we exclude the integration of CLIP weights $\theta_0$ for the purpose of static debiasing. $\mu$ and $\sigma^2$ are hyper-parameters for the distribution, and in practice, we determine the value of $\mu$ according to the epoch number. Then, we perform normalization towards the weights of total epochs to achieve $\sum_{t=1}^{R}w_t=1$. 
We also formulate GWA as a moving average to avoid increasing the storage cost of saving multiple snapshots of the parameters by updating the average of current learner $\theta_t$ on the fly (\ie, at epoch $t$) as:
\begin{equation}
\label{eq:GWA}
    \theta_{\texttt{GWA}}\gets \frac{\sum_{i=1}^{t-1} w_{i}}{\sum_{i=1}^{t} w_{i}} \cdot \theta_{\texttt{GWA}}+\frac{w_{t}}{\sum_{i=1}^{t} w_{i}} \cdot \theta_{t} .
\end{equation}


\subsection{Algorithm overview}
We present the overall training procedure of the proposed model-agnostic Open-MeDe in~\cref{alg:training}. The video learner is fine-tuned on training videos based on our cross-batch meta-optimization scheme cost-effectively. And the Gaussian self-ensemble stabilization is performed on the video learner via our GWA for robust generalization under open-vocabulary settings.

