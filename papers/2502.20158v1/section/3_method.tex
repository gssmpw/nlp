\section{Method}
\label{sec:method}
\begin{figure*}[t]
    \centering
    \begin{tikzpicture}[scale=0.85] % Added scale factor to fit double columns
        % Define colors
        \definecolor{orangetext}{RGB}{255,99,71}
        \definecolor{bluetext}{RGB}{30,144,255}
        
       % Left figure
        \node at (6,3) {\includegraphics[width=0.7\textwidth]{Figures/2_1_attn-sums.pdf}};
        
        % Right figure
        \node at (15.5,3) {\includegraphics[width=0.28\textwidth]{Figures/2_2_Meta-Llama-3-8B-Instruct_L17H21.pdf}};
        
         % Vertical dashed line between figures
        \draw[dashed, line width=0.8pt] (12.7,-0.5) -- (12.7,6.8);
        
        % Split the left text into two lines
        \node[anchor=west, text width=10cm, font=\tiny] at (0,6.7) {More heads become \textbf{\textcolor{orangetext}{less}} focused on this region.};
        \node[anchor=east, text width=10cm, align=right, font=\tiny] at (12.6,6.7) {More heads become \textbf{\textcolor{bluetext}{more}} focused on this region.};
        
        % Title for the right figure
        \node[anchor=west, font=\scriptsize] at (13.5,6.8) {L17H21 of Llama-3-8B-Instruct};
        
        % Arrows - moved up to match new text position
        \draw[{Triangle[fill=orangetext, length=1.2mm, width=1.2mm]}-, line width=1pt, orangetext] (-0.1,6.5) -- (5.1,6.5);
        \draw[-{Triangle[fill=orangetext, length=1.2mm, width=1.2mm]}, line width=1pt, bluetext] (7.3,6.5) -- (12.5,6.5);
    \end{tikzpicture}
    \caption{\textbf{Left:} Attention distributions across different LLMs demonstrate that their attentions shift systematically from the \textit{instruction} to the \textit{template} region when processing harmful inputs. \textbf{Right:} Attention heatmaps (17th-layer, 21st-head) from Llama-3-8B-Instruct consistently illustrate this distinct pattern.}
    \label{fig:attn_shift}
\end{figure*}
\subsection{Preliminaries}
\noindent {\bf Action recognition with CLIP-based video learner.}
% 
Pre-training on large-scale image-text data based on contrastive learning, CLIP learns separate uni-modal encoders for image and text, embedding them into a joint feature space, respectively.
% 
Consider a CLIP-based video learner with a ViT architecture~\cite{dosovitskiy2020image}, that incorporates temporal modeling for video understanding~\cite{wu2023revisiting,wang2021actionclip,weng2023open,zhu2023orthogonal,yang2023aim,wasim2023vita}.
% 
Next, we present the standard vision-only fine-tuning paradigm that applies such a video learner $f_{\theta_v}$ with a frozen text encoder $f_{\theta_t}$ to action recognition. 
% 
Specifically, given a video clip $V_i$, and a candidate action label $T_j\in \mathcal{Z}_{tr}$ described in predefined textual templates (\eg, ``\textit{a video of \{action\}}'') from the training set $\mathcal{D}_{tr}$, the similarity is calculated as:
\begin{equation}
    s_{i,j}=\frac{\left \langle v_i,t_j \right \rangle }{\left \| v_i \right \| \left \| t_j \right \| },v_i=f_{\theta_v}(V_i),t_j=f_{\theta_t}(T_j),
\end{equation}
where the training objective is to maximize it of the matched $V_i$ and $T_j$, or to minimize it otherwise. The loss function is implemented by the cross-entropy loss in~\cite{wu2023revisiting,chen2021empirical,radford2021learning} as:
\begin{equation}
\label{eq:ce}
    \mathcal{L}_{CE} = -\frac{1}{B}\sum_{i}^{B}\sum_{k}^{K}y_{i,k}\log \left ( \frac{\exp(s_{i,k})}{\sum_{j}^{K}\exp (s_{i,j})}  \right ),  
\end{equation}
where $B$ and $K$ denote the minibatch size and the number of all known classes, respectively. If the $i$-th video belongs to the $k$-th class, $y_{i,k}$ equals $1$; otherwise, $y_{i,k}$ equals $0$.
% 
In OVAR, the trained video learner should achieve good generalization on test data with the class label $T_i\in \mathcal{Z}_{te}$, where $\mathcal{Z}_{te} \cap \mathcal{Z}_{tr}=\emptyset $.

\noindent {\bf Model-agnostic meta-learning (MAML).}
MAML~\cite{finn2017model} is a gradient-based meta-optimization framework designed for few-shot learning, which aims to learn good initialization such that a few gradient steps will lead to fast learning on new tasks.
% 
Formally, consider a model $f_\theta$ with parameters $\theta$, MAML learns a set of initial weight values, which will serve as a good starting point for fast adaptation to a new task $\mathcal{T}_i$, sampled from a task distribution $p(\mathcal{T})$.
% 
When adapting to the task $\mathcal{T}_i$, the fast weights $\theta_i'$ are computed \textit{w.r.t.} examples from $\mathcal{T}_i$ though single inner-loop update as:
\begin{equation}
    \theta_i'=\theta-\alpha \nabla_\theta \mathcal{L}_{\mathcal{T}_i}(f_\theta),
\end{equation}
where $\alpha$ denotes the step size for inner loops. Then, the model with fast weights $f_{\theta_i'}$ is evaluated on new samples from the same task $\mathcal{T}_i$, to act as the feedback (\ie, loss gradients) to adapt to current task $\mathcal{T}_i$ to optimize the initialization $\theta$ for generalization as:
\begin{equation}
\label{eq:maml}
    \theta \leftarrow \theta-\beta \nabla_\theta \sum_{\mathcal{T}_i}\mathcal{L}_{\mathcal{T}_i}(f_{\theta_i'}).
\end{equation}
where $\beta$ is the step size for outer loops. 
Computationally, due to the additional backward propagation burden of the gradient by gradient update, MAML presents a first-order approximation, FOMAML, by dropping the backward pass. 
% Additionally, Reptile~\cite{nichol2018first} algorithm defines the update as:
% \begin{equation}
% \label{eq:reptile}
%     \theta \leftarrow \theta+ \epsilon(\theta_i'-\theta),
% \end{equation}
% where $(\theta-\theta_i')/\alpha$ can serve as the gradient term with scaling factor $\epsilon$ for updating $\theta$ for adaptation to task $\mathcal{T}_i$.

\subsection{Open-MeDe}
% 【(—)说现有】
As discussed above, the standard fine-tuning paradigm can cause the video learner to overfit to the known classes during training, leading to poor zero-shot capabilities. Also, CLIP regularization-based approaches face challenges in achieving robust generalization due to the excessive reliance on superficial static cues in videos.
% 【(+)说解决】
To tackle these issues, we draw upon the philosophy and methodology from meta-learning, and propose Open-MeDe framework, which is illustrated in~\cref{fig:overview}, to enhance both know-to-open generalizing and image-to-video debiasing simultaneously.

\subsubsection{Cross-batch meta-optimization}
% 【(+)说方法】
Our Open-MeDe framework primarily adopts a cross-batch meta-optimization scheme (in~\cref{fig:overview}(a)) to enhance the video learner via \textit{meta training and testing}, enabling it to acquire generalizable, video-specific knowledge instead of overly exploiting static biases. 
% 【说划分】
Note that we neither sample from a distribution of $N$-way $K$-shot tasks as done in few-shot MAML nor deliberately split the training set into support and query sets as Meta-ZSL~\cite{liu2021task,verma2020meta} suggested. Instead, our support and query examples are constructed effortlessly and arbitrarily by the default training data sampler.
% 【说cross-batch好处】
In effect, we consider this arbitrariness a blessing for building the natural ``\textit{known-to-open generalization task}'', since the known biases in \textit{meta training} data do not hold in \textit{meta testing} data due to different inherent label distributions across batches.
% 【说meta-training】
A known-to-open task can be created by extending the original gradient step into two consecutive mini-batches in one pass, with the current batch acting as support data and the subsequent batch as query data.
Specifically, in line with the episode-wise training akin to MAML, we first train the learner within an inner loop (\ie, \textit{meta training}), where the fast weights are obtained through a single gradient step for each support batch.
% 【说meta-testing】
Following this adaptation, in the outer loop, query videos are sampled to evaluate the generalization performance of the adapted learner with fast weights (\ie, \textit{meta testing}).
% 【说优化】
In this work, our framework further updates the fast weights of the learner based on the evaluation performance during \textit{meta testing}, which then provides feedback for the task to derive more generalizable optimization for the learner.

\noindent {\bf Meta training.}
At each training iteration, we first utilize each support batch $\mathcal{S}=\{V_i,T_i\}^B$ from the task $\mathcal{T}_i$ to train the video learner $f_{\theta}$ (with parameters $\theta$), via one standard gradient step. The inner loop update is governed by the loss on the support batch as:
\begin{equation}
\label{eq:spt}
     \mathcal{L}_{\mathcal{T}_i}^\mathcal{S}(\theta) =\mathcal{L}(f_{\theta }(\mathcal{S})),
\end{equation}
where $\mathcal{L}(\cdot)$ refers to the loss function (\eg, the cross-entropy loss $\mathcal{L}_{CE}$ \wrt Eq.~\eqref{eq:ce}). Then, we make a temporary copy for the original parameters $\theta$ and update the intermediate parameters for fast weights as follows:
\begin{equation}
\label{eq:spt_update}
    \theta'_i = \theta-\alpha \nabla_{\theta} \mathcal{L}_{\mathcal{T}_i}^{\mathcal{S}}(\theta),
\end{equation}
where $\alpha$ denotes the learning rate for \textit{meta training}. Intuitively, this step simulates a direct update to train the learner to obtain class-specific knowledge of the support data.

\noindent {\bf Meta testing.}
After meta training on the support batch, we then scheme a virtual testing process, leveraging the query batch $\mathcal{Q}=\{V_i, T_i\}^B$, where $\mathcal{S}\cap \mathcal{Q}=\emptyset$,  to evaluate the generalization performance of the base learner $f_{\theta_i'}$. Formally, we measure the known-to-open performance on $\mathcal{T}_i$ by calculating the class-specific loss concerning the query data as:
\begin{equation}
\label{eq:qry}
    \mathcal{L}_{\mathcal{T}_i}^\mathcal{Q}(\theta'_i) =\mathcal{L}(f_{\theta_i' }(\mathcal{Q})).
\end{equation}
Here, the formulation closely relates to the standard fine-tuning process, which aims to obtain decent class-specific performance for all training batches. Differently, this step merely evaluates the intermediary base learner for its known-to-open generalizability on each task, due to the original parameters $\theta$ remaining immune to the task-specific updates.
% 
Hence, it can be used to provide feedback on \textit{what video-specific knowledge should be learned in the sense that the learner can derive the robust generalization across different class distributions against inherent known and static biases} in the following meta-optimization.

\noindent {\bf Meta-optimization.}
As mentioned above, the intuition behind our approach is that the virtual evaluation during meta testing can provide useful feedback to encourage the learning of more robust representations for fast known-to-open generalization after \textit{meta training} on the support data (\ie, $\theta'_i \leftarrow \theta$).
Note that original MAML approaches focus on optimizing parameters for a strong initialization, enabling quick adaptation to new tasks with minimal gradient updates. Conversely, open-vocabulary recognition requires zero-shot capabilities, where no further adaptation can be applied for new tasks. Therefore, class-specific knowledge should be strengthened in terms of global optimization.
To this end, within the outer loop, the parameters of the learner are optimized to minimize the class-specific errors for the support data and the adaptation cost for the query data simultaneously. 
The combination of both Eq.~\eqref{eq:spt} and Eq.~\eqref{eq:qry} is used to carry out the outer loop update, thus the objective for meta-optimization can be defined as:
\begin{equation}
\label{eq:maml*}
\begin{aligned}
\min _{\theta}\mathcal{L}_{\mathcal{T}_i}(\theta )=& \min _{\theta}\left( \mathcal{L}_{\mathcal{T}_i}^\mathcal{S} (\theta) +\mathcal{L}_{\mathcal{T}_i}^\mathcal{Q}(\theta_i') \right)  \\
= & \min _{\theta} \left(\mathcal{L}_{\mathcal{T}_i}^\mathcal{S} (\theta)+\mathcal{L}_{\mathcal{T}_i}^\mathcal{Q}(\theta-\alpha \nabla_{\theta} \mathcal{L}_{\mathcal{T}_i}^{\mathcal{S}}(\theta))\right).
\end{aligned}
\end{equation}
Here, the first term refers to the class-specific knowledge learned on the support batch, while the second term provides the known-to-open generalization feedback based on $\theta_i'$ towards robust representation learning \wrt the task $\mathcal{T}_i$.
The optimizing process of the parameter $\theta$ can be given by:
\begin{equation}
\label{eq:maml**}
    \theta \leftarrow \theta - \beta \nabla_{\theta} \sum _{i = 1}^{N}\left( \mathcal{L}_{\mathcal{T} _i}^\mathcal{S} (\theta) + \mathcal{L}_{\mathcal{T} _i}^\mathcal{Q} \left ( \theta - \alpha \nabla_{\theta} \mathcal{L}_{\mathcal{T} _i}^\mathcal{S} (\theta) \right) \right),
\end{equation}
where $N$ is the batch size of the task for meta-optimization.
Since the MAML meta-gradient update needs to differentiate through the optimization process (\ie, a gradient by a gradient), it’s not an ideal solution where we need to optimize a large number of tasks during the training phase. 
Therefore, we opt for the one-step update approximation by dropping the backward pass of $\theta \gets \theta'_i$, where Eq.~\eqref{eq:maml**} can be rewritten as:
\begin{equation}
\label{eq:FOMAML}
    \theta \leftarrow \theta - \beta  \sum _{i=1}^{N}\left (\nabla_{\theta}\mathcal{L}_{\mathcal{T} _i}^\mathcal{S} (\theta) +\delta \nabla_{\theta'_i} \mathcal{L}_{\mathcal{T}_i}^{\mathcal{Q}}(\theta'_i)\right),
\end{equation}
where $\beta$ and $\delta$ are the learning rates for meta-optimization.
With the genuine update of the learner in Eq.~\eqref{eq:FOMAML} without CLIP regularization, we can optimize a parallel or batch version that evaluates on $N$ known-to-open tasks of different class distributions (\ie, class-specific knowledge), which encourages to learn more generalizable features against known and static biases.

\begin{table*}[!ht]
\centering
\caption{Performance comparison (Top1-Acc (\%)) with the CLIP-adapted methods using ViT-B/16 under the in-context base-to-novel setting. We also report the harmonic mean (HM) of base and novel recognition accuracy. The \textbf{best} and the \underline{second-best} results are highlighted. $\ast$ and $\dagger$ denote the results reproduced with our implementation using frozen text learners.}
\label{tab:B2N}
\resizebox{0.8\textwidth}{!}{
\begin{tabular}{lcccccccccccc}
\toprule
\multirow{2}{*}{Method} &
  \multicolumn{3}{c}{K400} &
  \multicolumn{3}{c}{HMDB} &
  \multicolumn{3}{c}{UCF} &
  \multicolumn{3}{c}{SSv2} \\ \cmidrule(lr){2-4} \cmidrule{5-7} \cmidrule{8-10} \cmidrule{11-13} 
 &
  Base &
  Novel &
  \multicolumn{1}{c|}{HM} &
  Base &
  Novel &
  \multicolumn{1}{c|}{HM} &
  Base &
  Novel &
  \multicolumn{1}{c|}{HM} &
  Base &
  Novel &
  HM \\ \cmidrule{1-13}
Frozen CLIP~\cite{radford2021learning} &
  62.3 &
  53.4 &
  \multicolumn{1}{c|}{57.5} &
  53.3 &
  46.8 &
  \multicolumn{1}{c|}{49.8} &
  78.5 &
  63.6 &
  \multicolumn{1}{c|}{70.3} &
  4.9 &
  5.3 &
  5.1 \\
ActionCLIP~\cite{wang2021actionclip} &
  61.0 &
  46.2 &
  \multicolumn{1}{c|}{52.6} &
  69.1 &
  37.3 &
  \multicolumn{1}{c|}{48.5} &
  90.1 &
  58.1 &
  \multicolumn{1}{c|}{70.7} &
  13.3 &
  10.1 &
  11.5 \\
X-CLIP~\cite{ni2022expanding} &
  74.1 &
  56.4 &
  \multicolumn{1}{c|}{64.0} &
  69.4 &
  45.5 &
  \multicolumn{1}{c|}{55.0} &
  89.9 &
  58.9 &
  \multicolumn{1}{c|}{71.2} &
  8.5 &
  6.6 &
  7.4 \\
VPT~\cite{ju2022prompting} &
  69.7 &
  37.6 &
  \multicolumn{1}{c|}{48.8} &
  46.2 &
  16.0 &
  \multicolumn{1}{c|}{23.8} &
  90.5 &
  40.4 &
  \multicolumn{1}{c|}{55.8} &
  8.3 &
  5.3 &
  6.4 \\
ST-Adapter~\cite{pan2022st} &
  74.6 &
  62.0 &
  \multicolumn{1}{c|}{67.3} &
  65.3 &
  48.9 &
  \multicolumn{1}{c|}{55.9} &
  85.5 &
  76.8 &
  \multicolumn{1}{c|}{80.9} &
  9.3 &
  8.4 &
  8.8 \\
ViFi-CLIP~\cite{rasheed2023fine} &
  \underline{76.4} &
  61.1 &
  \multicolumn{1}{c|}{67.9} &
  \textbf{73.8} &
  \underline{53.3} &
  \multicolumn{1}{c|}{\underline{61.9}} &
  92.9 &
  67.7 &
  \multicolumn{1}{c|}{78.3} &
  \underline{16.2} &
  \underline{12.1} &
  \underline{13.9} \\
Open-VCLIP $\ast$~\cite{weng2023open} &
  76.3 &
  \underline{62.3} &
  \multicolumn{1}{c|}{\underline{68.6}} &
  70.2 &
  50.2 &
  \multicolumn{1}{c|}{58.5} &
  \underline{94.6} &
  \underline{77.2} &
  \multicolumn{1}{c|}{\underline{85.0}} &
  15.9 &
  10.8 &
  12.9 \\
FROSTER $\dagger$~\cite{huang2024froster}&
  76.0 &
  61.9 &
  \multicolumn{1}{c|}{68.3} &
  70.0 &
  49.9 &
  \multicolumn{1}{c|}{58.3} &
  94.3 &
  76.9 &
  \multicolumn{1}{c|}{84.7} &
  15.5 &
  10.3 &
  12.4 \\
\rowcolor{red!10} 
\textbf{Open-MeDe} &
  \textbf{77.2} &
  \textbf{63.8} &
  \multicolumn{1}{c|}{\textbf{69.9}} &
  \underline{73.6} &
  \textbf{56.4} &
  \multicolumn{1}{c|}{\textbf{63.9}} &
  \textbf{94.9} &
  \textbf{78.5} &
  \multicolumn{1}{c|}{\textbf{85.9}} &
  \textbf{17.1} &
  \textbf{12.3} &
  \textbf{14.3} \\ \bottomrule
\end{tabular}
}
\end{table*}
\vspace{-0.3em}
\subsubsection{Gaussian self-ensemble stabilization}
% 【(—)说传统】
Typically, training the video learner for longer iterations to gain specialization on the supervised tasks comes with the risk of diminished plasticity and generalizability. 
Model patching~\cite{ilharco2022patching,wortsman2022robust,shu2023clipood,weng2023open} of weight ensembling has been shown to improve both the performance and generalization.
% 【(*)说谨慎利用】
Given that the fine-tuning videos are limited in class-specific knowledge, while the open-vocabulary tasks are unconstrained, the static generalizable flexibility derived from large-scale I-VL pre-training should be scrupulously exploited to enhance the adaptation of the video learner while minimizing the impact of static bias.
% 【(+)说GWA】
Therefore, we further incorporate self-ensemble stabilization tailored to the video learner over its optimization trajectory, which utilizes the knowledge from previous training iterations for a generalizable solution.
% 
In a fine-tuning procedure of $R$ epochs with $l$ step length for each, the learner's optimization trajectory is represented by $\{\theta_t\}_{t=0}^R$, where $\theta_0$ is the pre-trained weights. The self-ensemble averages the weights of the learner as:
\begin{equation}
    \theta_{\texttt{WA}}= \sum_{t=0}^{R}\frac{w_t}{\sum_{i=0}^{R}w_i}\cdot \theta_t,
\end{equation}
where $w_t$ specifies the weight contributed by the parameters at $t$-th epoch.
% 【(—)说早期】
Intuitively, during the early fine-tuning epochs (\ie, at a smaller epoch $t$), the video learner lacks the maturity to effectively capture video-specific knowledge while still retaining substantial static-related orientation from large-scale pre-training, which introduces vulnerable information for temporal understanding. 
% 【(—)说晚期】
Conversely, the parameters at the last few epochs (\ie, at a larger epoch $t$) have integrated more video-specific knowledge, highly featuring the supervised downstream task distribution, whereas the plasticity of the unconstrained zero-shot capability is not guaranteed.
% 【(—)说动机】
As both sides degrade the final open-vocabulary generalizability, we aim to weaken the contribution of the parameters near the initial and terminal epochs by employing a distribution prior, resulting in a generic optima for robust generalization.
We compare the performance of agents trained on data from the InSTA pipeline to agents trained on human demonstrations from WebLINX \citep{WebLINX} and Mind2Web \citep{Mind2Web}, two recent and popular benchmarks for web navigation. Recent works that mix synthetic data with real data control the real data sampling probability in the batch $p_{\text{real}}$ independently from data size \citep{DAFusion}. We employ $p_{\text{real}} = 0.5$ in few-shot experiments and $p_{\text{real}} = 0.8$ otherwise. Shown in Figure~\ref{fig:data-statistics}, our data have a wide spread in performance, so we apply several filtering rules to select high-quality training data. First, we require the evaluator to return \texttt{conf} = 1 that the task was successfully completed, and that the agent was on the right track (this selects data where the actions are reliable, and directly caused the task to be solved). Second, we filter data where the trajectory contains at least three actions. Third, we remove data where the agent encountered any type of server error, was presented with a captcha, or was blocked at any point. These steps produce $7,463$ high-quality demonstrations in which agents successfully completed tasks on diverse websites. We sample 500 demonstrations uniformly at random from this pool to create a diverse test set, and employ the remaining $6,963$ demonstrations to train agents on a mix of real and synthetic data.

\subsection{Improving Data-Efficiency}
\label{sec:few-shot}

\begin{wrapfigure}{r}{0.48\textwidth}
    \centering
    \vspace{-0.8cm}
    \includegraphics[width=\linewidth]{assets/few_shot_results_weblinx_mind2web.pdf}
    \vspace{-0.3cm}
    \caption{\small \textbf{Data from InSTA improves efficiency.} Language model agents trained on mixtures of our data and human demonstrations scale faster than agents trained on human data. In a setting with 32 human actions, adding our data improves \textit{Step Accuracy} by +89.5\% relative to human data for Mind2Web, and +122.1\% relative to human data for WebLINX.}
    \vspace{-0.2cm}
    \label{fig:few-shot-results}
\end{wrapfigure}

In a data-limited setting derived from WebLINX \citep{WebLINX} and Mind2Web \citep{Mind2Web}, agents trained on our data \textit{scale faster with increasing data size} than human data alone. Without requiring laborious human annotations, the data produced by our pipeline leads to improvements on Mind2Web that range from +89.5\% in \textit{Step Accuracy} (the rate at which the correct element is selected and the correct action is performed on that element) with 32 human actions, to +77.5\% with 64 human actions, +13.8\% with 128 human actions, and +12.1\% with 256 human actions. For WebLINX, our data improves by +122.1\% with 32 human actions, and +24.6\% with 64 human actions, and +6.2\% for 128 human actions. Adding our data is comparable in performance gained to doubling the amount of human data from 32 to 64 actions. Performance on the original test sets for Mind2Web and WebLINX appears to saturate as the amount of human data increases, but these benchmark only test agent capabilities for a limited set of 150 popular sites.

\subsection{Improving Generalization} 
\label{sec:generalization}

\begin{wrapfigure}{r}{0.48\textwidth}
    \centering
    \vspace{-1.0cm}
    \includegraphics[width=\linewidth]{assets/diverse_results_weblinx_mind2web.pdf}
    \vspace{-0.3cm}
    \caption{\small \textbf{Our data improves generalization.} We train agents with all human data from the WebLINX and Mind2Web training sets, and resulting agents struggle to generalize to more diverse test data. Adding our data improves generalization by +149.0\% for WebLINX, and +156.3\% for Mind2Web.}
    \vspace{-0.3cm}
    \label{fig:generalization-results}
\end{wrapfigure}

To understand how agents trained on data from our pipeline generalize to diverse real-world sites, we construct a more diverse test set than Mind2Web and WebLINX using 500 held-out demonstrations produced by our pipeline. Shown in Figure~\ref{fig:generalization-results}, we train agents using all human data in the training sets for WebLINX and Mind2Web, and compare the performance with agents trained on 80\% human data, and 20\% data from our pipeline. Agents trained with our data achieve comparable performance to agents trained purely on human data on the official test sets for the WebLINX and Mind2Web benchmarks, suggesting that when enough human data are available, synthetic data may not be necessary. However, when evaluated in a more diverse test set that includes 500 sites not considered by existing benchmarks, agents trained purely on existing human data struggle to generalize. Training with our data improves generalization to these sites by +149.0\% for WebLINX agents, and +156.3\% for Mind2Web agents, with the largest gains in generalization \textit{Step Accuracy} appearing for harder tasks.
% 【(+)说GWA】
Driven by~\cite{khattak2023self} in prompt learning, we perform Gaussian Weight Average (GWA) based on model patching, as shown in~\cref{fig:overview}(b), which assigns the parameters with lower weights at initial epochs, higher weights at middle epochs, and relatively lower weights at final epochs.
% 【(+)】
Given a  Gaussian distribution $w_t\sim \mathcal{N}(\mu,\sigma^2)$ defined over the epochs, we sample the weight values for the parameters $\theta_t$ as its corresponding probability in the distribution as:
\begin{equation}
    w_t=\frac{1}{\sqrt{2 \pi} \sigma} e^{-\frac{(t-\mu)^{2}}{2 \sigma^{2}}}, t=1,\dots,R.
\end{equation}
Here, we exclude the integration of CLIP weights $\theta_0$ for the purpose of static debiasing. $\mu$ and $\sigma^2$ are hyper-parameters for the distribution, and in practice, we determine the value of $\mu$ according to the epoch number. Then, we perform normalization towards the weights of total epochs to achieve $\sum_{t=1}^{R}w_t=1$. 
We also formulate GWA as a moving average to avoid increasing the storage cost of saving multiple snapshots of the parameters by updating the average of current learner $\theta_t$ on the fly (\ie, at epoch $t$) as:
\begin{equation}
\label{eq:GWA}
    \theta_{\texttt{GWA}}\gets \frac{\sum_{i=1}^{t-1} w_{i}}{\sum_{i=1}^{t} w_{i}} \cdot \theta_{\texttt{GWA}}+\frac{w_{t}}{\sum_{i=1}^{t} w_{i}} \cdot \theta_{t} .
\end{equation}


\subsection{Algorithm overview}
We present the overall training procedure of the proposed model-agnostic Open-MeDe in~\cref{alg:training}. The video learner is fine-tuned on training videos based on our cross-batch meta-optimization scheme cost-effectively. And the Gaussian self-ensemble stabilization is performed on the video learner via our GWA for robust generalization under open-vocabulary settings.

