\section{Appendix}
\subsection{Full AUC Results}
\label{appendix:auc_results}
% Please add the following required packages to your document preamble:
% \usepackage{booktabs}
% \usepackage{multirow}
% \usepackage[table,xcdraw]{xcolor}
% Beamer presentation requires \usepackage{colortbl} instead of \usepackage[table,xcdraw]{xcolor}
\begin{table}[H]
\centering
\begin{tabular}{@{}cccc|ccc@{}}
\toprule
Category & Dataset & Direct & Hybrid SC & Elo Rating & TrueSkill & Bradley-Terry \\ \midrule
 & GPQA & 0.441 & 0.457 & 0.454 & 0.460 & \cellcolor[HTML]{BAEFFE}\textbf{0.466} \\
 & MedQA & 0.904 & \cellcolor[HTML]{BAEFFE}\textbf{0.920} & 0.893 & 0.900 & 0.901 \\
 & OBQA & 0.979 & 0.984 & 0.984 & 0.984 & \cellcolor[HTML]{BAEFFE}\textbf{0.985} \\
 & Physics & 0.909 & 0.911 & 0.925 & \cellcolor[HTML]{BAEFFE}\textbf{0.928} & 0.927 \\
 & Algebra & 0.802 & \cellcolor[HTML]{BAEFFE}\textbf{0.811} & 0.806 & 0.805 & 0.804 \\
 & Chem & 0.832 & 0.840 & \cellcolor[HTML]{BAEFFE}\textbf{0.864} & 0.854 & 0.850 \\
\multirow{-7}{*}{STEM} & Security & 0.920 & 0.930 & \cellcolor[HTML]{BAEFFE}\textbf{0.934} & 0.930 & 0.917 \\ \midrule
 & Law & 0.789 & 0.809 & 0.799 & 0.815 & \cellcolor[HTML]{BAEFFE}\textbf{0.816} \\
 & Ethics & 0.956 & 0.964 & \cellcolor[HTML]{BAEFFE}\textbf{0.971} & 0.970 & 0.966 \\
 & Econ & 0.798 & 0.822 & 0.825 & \cellcolor[HTML]{BAEFFE}\textbf{0.829} & 0.819 \\
\multirow{-4}{*}{Social Sciences} & Policy & 0.991 & \cellcolor[HTML]{BAEFFE}\textbf{0.995} & 0.982 & 0.980 & 0.982 \\ \midrule
 & TQA & 0.880 & 0.889 & \cellcolor[HTML]{BAEFFE}\textbf{0.918} & 0.917 & 0.917 \\
 & CSQA & 0.885 & \cellcolor[HTML]{BAEFFE}\textbf{0.887} & 0.873 & 0.869 & 0.872 \\
\multirow{-3}{*}{\begin{tabular}[c]{@{}c@{}}Commonsense\\ Reasoning\end{tabular}} & SIQA & 0.861 & \cellcolor[HTML]{BAEFFE}\textbf{0.897} & 0.861 & 0.856 & 0.863 \\ \midrule
 & Average & 0.853 & \cellcolor[HTML]{BAEFFE}\textbf{0.865} & 0.863 & 0.864 & 0.863 \\ \bottomrule
\end{tabular}
\caption{\textbf{Claude 3.5 Sonnet AUCs All Methods.} We show the dataset-level results for Claude 3.5 Sonnet, for the Direct and Hybrid SC absolute confidence baselines and for relative confidence estimation with different rank aggregation methods (Elo Rating, TrueSkill, Bradley-Terry). Relative confidences outperform absolute confidence baselines for 9 out of 14 datasets across STEM, social science, and commonsense reasoning. On average, relative confidences closely match the performance of the best absolute confidence methods (only 0.1\% lower AUC than self-consistency prompting).}
\label{tab:claude_auc_results}
\end{table}
% Please add the following required packages to your document preamble:
% \usepackage{booktabs}
% \usepackage{multirow}
% \usepackage[table,xcdraw]{xcolor}
% Beamer presentation requires \usepackage{colortbl} instead of \usepackage[table,xcdraw]{xcolor}
\begin{table}[H]
\centering
\begin{tabular}{@{}cccc|ccc@{}}
\toprule
Category & Dataset & Direct & Hybrid SC & Elo Rating & TrueSkill & Bradley-Terry \\ \midrule
 & GPQA & 0.395 & \cellcolor[HTML]{BAEFFE}\textbf{0.424} & 0.410 & 0.409 & 0.413 \\
 & MedQA & 0.794 & \cellcolor[HTML]{BAEFFE}\textbf{0.831} & 0.725 & 0.786 & 0.792 \\
 & OBQA & 0.955 & 0.959 & 0.979 & \cellcolor[HTML]{BAEFFE}\textbf{0.985} & \cellcolor[HTML]{BAEFFE}\textbf{0.985} \\
 & Physics & 0.878 & 0.922 & 0.927 & \cellcolor[HTML]{BAEFFE}\textbf{0.936} & 0.934 \\
 & Algebra & 0.603 & 0.612 & 0.629 & \cellcolor[HTML]{BAEFFE}\textbf{0.651} & 0.640 \\
 & Chem & 0.717 & 0.762 & 0.806 & \cellcolor[HTML]{BAEFFE}\textbf{0.851} & 0.842 \\
\multirow{-7}{*}{STEM} & Security & \cellcolor[HTML]{BAEFFE}\textbf{0.868} & 0.863 & 0.850 & 0.824 & 0.837 \\ \midrule
 & Law & 0.695 & 0.727 & 0.766 & 0.776 & \cellcolor[HTML]{BAEFFE}\textbf{0.778} \\
 & Ethics & 0.903 & 0.910 & 0.949 & 0.954 & \cellcolor[HTML]{BAEFFE}\textbf{0.957} \\
 & Econ & 0.684 & 0.734 & \cellcolor[HTML]{BAEFFE}\textbf{0.747} & 0.732 & 0.736 \\
\multirow{-4}{*}{Social Sciences} & Policy & 0.961 & 0.957 & \cellcolor[HTML]{BAEFFE}\textbf{0.980} & 0.979 & 0.978 \\ \midrule
 & TQA & 0.876 & \cellcolor[HTML]{BAEFFE}\textbf{0.891} & 0.861 & 0.853 & 0.854 \\
 & CSQA & 0.835 & \cellcolor[HTML]{BAEFFE}\textbf{0.889} & 0.860 & 0.869 & 0.872 \\
\multirow{-3}{*}{\begin{tabular}[c]{@{}c@{}}Commonsense\\ Reasoning\end{tabular}} & SIQA & 0.854 & \cellcolor[HTML]{BAEFFE}\textbf{0.874} & 0.840 & 0.854 & 0.848 \\ \midrule
 & Average & 0.787 & 0.811 & 0.809 & 0.818 & \cellcolor[HTML]{BAEFFE}\textbf{0.819} \\ \bottomrule
\end{tabular}
\caption{\textbf{Gemini 1.5 Pro AUCs All Methods.} We show the dataset-level AUC results for Gemini 1.5 Pro. On average, relative confidence estimation with Bradley-Terry leads to the best AUC with a 3.2\% improvement over direct prompting and a 0.8\% improvement over self-consistency prompting.
}
\label{tab:gemini_auc_results}
\end{table}
% Please add the following required packages to your document preamble:
% \usepackage{booktabs}
% \usepackage{multirow}
% \usepackage[table,xcdraw]{xcolor}
% Beamer presentation requires \usepackage{colortbl} instead of \usepackage[table,xcdraw]{xcolor}
\begin{table}[]
\centering
\begin{tabular}{@{}cccc|ccc@{}}
\toprule
Category & Dataset & Direct & Hybrid SC & Elo Rating & TrueSkill & Bradley-Terry \\ \midrule
 & GPQA & 0.393 & 0.383 & 0.377 & \cellcolor[HTML]{BAEFFE}\textbf{0.404} & 0.394 \\
 & MedQA & 0.841 & \cellcolor[HTML]{BAEFFE}\textbf{0.893} & 0.870 & 0.875 & 0.864 \\
 & OBQA & 0.966 & 0.979 & \cellcolor[HTML]{BAEFFE}\textbf{0.990} & \cellcolor[HTML]{BAEFFE}\textbf{0.990} & 0.989 \\
 & Physics & 0.818 & 0.851 & 0.908 & \cellcolor[HTML]{BAEFFE}\textbf{0.918} & 0.917 \\
 & Algebra & 0.587 & 0.650 & 0.642 & 0.651 & \cellcolor[HTML]{BAEFFE}\textbf{0.663} \\
 & Chem & 0.682 & 0.774 & 0.797 & \cellcolor[HTML]{BAEFFE}\textbf{0.805} & 0.795 \\
\multirow{-7}{*}{STEM} & Security & 0.911 & 0.916 & \cellcolor[HTML]{BAEFFE}\textbf{0.933} & 0.927 & 0.922 \\ \midrule
 & Law & 0.716 & 0.741 & 0.722 & 0.753 & \cellcolor[HTML]{BAEFFE}\textbf{0.754} \\
 & Ethics & 0.870 & \cellcolor[HTML]{BAEFFE}\textbf{0.915} & 0.908 & 0.914 & 0.911 \\
 & Econ & 0.634 & 0.638 & 0.717 & 0.714 & \cellcolor[HTML]{BAEFFE}\textbf{0.725} \\
\multirow{-4}{*}{Social Sciences} & Policy & 0.959 & \cellcolor[HTML]{BAEFFE}\textbf{0.973} & 0.970 & 0.971 & 0.971 \\ \midrule
 & TQA & 0.892 & \cellcolor[HTML]{BAEFFE}\textbf{0.926} & 0.865 & 0.869 & 0.872 \\
 & CSQA & 0.831 & \cellcolor[HTML]{BAEFFE}\textbf{0.868} & 0.835 & 0.841 & 0.837 \\
\multirow{-3}{*}{\begin{tabular}[c]{@{}c@{}}Commonsense\\ Reasoning\end{tabular}} & SIQA & 0.851 & 0.872 & 0.886 & \cellcolor[HTML]{BAEFFE}\textbf{0.888} & 0.887 \\ \midrule
 & Average & 0.782 & 0.813 & 0.816 & \cellcolor[HTML]{BAEFFE}\textbf{0.823} & 0.821 \\ \bottomrule
\end{tabular}
\caption{\textbf{GPT-4 AUCs All Methods.} For GPT-4, relative confidences with TrueSkill lead to the best average AUC with a 4.1\% improvement over direct prompting and a 1.0\% improvement over self-consistency.}
\label{tab:gpt4_auc_results}
\end{table}

\subsection{Average AUROC Results}
\label{appendix:auroc_results}
% Please add the following required packages to your document preamble:
% \usepackage{booktabs}
% \usepackage[table,xcdraw]{xcolor}
% Beamer presentation requires \usepackage{colortbl} instead of \usepackage[table,xcdraw]{xcolor}
\begin{table}[H]
\centering
\begin{tabular}{@{}ccc|ccc@{}}
\toprule
Model & Direct & Hybrid SC & Elo Rating & TrueSkill & Bradley-Terry \\ \midrule
Llama 3.1 405B & 0.575 & 0.774 & 0.849 & \cellcolor[HTML]{BAEFFE}\textbf{0.856} & 0.852 \\
GPT-4 & 0.642 & \cellcolor[HTML]{BAEFFE}\textbf{0.730} & 0.708 & 0.719 & 0.713 \\
Gemini 1.5 Pro & 0.627 & 0.700 & 0.689 & \cellcolor[HTML]{BAEFFE}\textbf{0.713} & 0.712 \\
GPT-4o & 0.698 & \cellcolor[HTML]{BAEFFE}\textbf{0.774} & 0.762 & 0.763 & 0.758 \\
Claude 3.5 Sonnet & 0.685 & \cellcolor[HTML]{BAEFFE}\textbf{0.726} & 0.711 & 0.713 & 0.713 \\ \midrule
Average Across Models & 0.645 & 0.741 & 0.744 & \cellcolor[HTML]{BAEFFE}\textbf{0.753} & 0.749 \\ \bottomrule
\end{tabular}
\caption{\textbf{Model AUROCs.} Relative confidences with TrueSkill lead to the best average AUROC for 2 out of 5 models, and a 10.8\% gain over direct prompting and a 1.2\% gain over self-consistency across all models.}
\label{tab:avg_auroc_results}
\end{table}

\subsection{Hyperparameters}
\label{appendix:hyperparameters}
Following are the hyperparameters involved for each rank aggregation method of relative confidence estimation.

\textbf{Elo rating.} initial scores, $K$, $\#$ iterations

\textbf{TrueSkill.} $\mu$, $\sigma$, $\beta$, $\tau$

\textbf{Bradley-Terry.} maximum $\#$ iterations, $\lambda$ for regularization

We use the following fixed set of hyperparameters for datasets which do not have a sufficient validation set for hyperparameter tuning of a hundred examples or more beyond their test set.

\begin{table}[H]
\centering
\begin{tabular}{@{}ccc|cccc|cc@{}}
\toprule
\multicolumn{3}{c|}{Elo Rating} & \multicolumn{4}{c|}{TrueSkill} & \multicolumn{2}{c}{Bradley-Terry} \\ \midrule
Initial Score & \textit{K} & \# iterations & $\mu$ & $\sigma$ & $\beta$ & $\tau$ & max \# iterations & $\lambda$ \\ \midrule
1000 & 400 & 1 & 25.0 & $\frac{\mu}{3.0}$ & $\frac{\mu}{6.0}$ & $\frac{\mu}{300.0}$ & 5 & 0.01 \\ \bottomrule
\end{tabular}
\caption{\textbf{Rank Aggregation Hyperparameter Values.}}
\label{tab:rank_agg_hyperparams}
\end{table}

For the datasets which have a hundred or more examples in their train or validation sets, we select a hundred examples to use for tuning the following hyperparameters to achieve the best AUC on this held-out set. 

\begin{table}[H]
\centering
\begin{tabular}{@{}c|cc@{}}
\toprule
Algorithm & Parameter & Values \\ \midrule
Elo Rating & \# iters & {[}1-20{]} \\ \midrule
\multirow{3}{*}{TrueSkill} & $\sigma$ & {[}$\frac{\mu}{3.0}$, $\frac{\mu}{2.5}$, $\frac{\mu}{2.2}$, $\frac{\mu}{2.0}${]} \\
 & $\beta$ & {[}$\frac{\mu}{6.0}$, $\frac{\mu}{5.0}$, $\frac{\mu}{4.0}$, $\frac{\mu}{3.0}${]} \\
 & $\tau$ & {[}$\frac{\mu}{300.0}$, $\frac{\mu}{250.0}$, $\frac{\mu}{200.0}$, $\frac{\mu}{150.0}${]} \\ \midrule
Bradley-Terry & max \# iters & {[}1-20{]} \\ \bottomrule
\end{tabular}
\caption{\textbf{Rank Aggregation Hyperparameter Ranges.}}
\end{table}

\subsection{Prompts}
\label{appendix:prompts}
\begin{center}
{\textbf{Linguistic Confidence Prompt}}
  \fbox{\parbox{0.8\textwidth}{%
    Answer the following question to the best of your ability, and provide a score between 0 and 1 to indicate the confidence you have in your answer. Confidence scores closer to 0 indicate you have less confidence in your answer, while scores closer to 1 indicate you have more confidence in your answer. You must answer the question with one of the valid choices. You must provide only a single answer. \\\\
    Question: This is a question\\
    (A) first answer\\
    (B) second answer\\
    (C) third answer\\
    (D) fourth answer\\
    (E) fifth answer\\
    Answer: (D)\\
    Confidence: 0.4\\\\
    Question: This is another question\\
    (A) first answer\\
    (B) second answer\\
    (C) third answer\\
    (D) fourth answer\\
    (E) fifth answer\\
    Answer: (A)\\
    Confidence: 0.7
  }}
\end{center}

\begin{center}
{\textbf{CoT Relative Confidence Prompt}}
  \fbox{\parbox{0.8\textwidth}{%
Here are two questions and your answers to those questions. Which question are you more confident in answering correctly and why? Respond in the following format: `I am more confident that I correctly answered question <your selected question>, because <your reasoning>.'   
  }}
\end{center}


\begin{center}
{\textbf{Difficulty Prompt}}
  \fbox{\parbox{0.8\textwidth}{%
Here are two questions. Which question is more difficult? Respond in the following format: `<your selected question> is more difficult.' 
  }}
\end{center}


