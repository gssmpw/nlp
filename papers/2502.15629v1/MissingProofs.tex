\section{Missing Proofs}\label{sec:missing-proofs}
\remove{
\subsection{Proving \cref{lem:distance-I}}\label{sec:missing-proofs:distance-I}

We make use of the following fact.

\begin{fact}[Proposition 3.28 in \cite{HaitnerMST22}]\label{fact:I}
	Let $R$ be uniform random variable over $\{0,1\}^n$, and let $I$ be uniform random variable over $\mathcal{I}\subseteq[n]$, independent of $R$, then $SD(R|_{R_I=0},R|_{R_I=1})\leq1/\sqrt{\size{\cI}}$.
\end{fact}
We now prove \cref{lem:distance-I}, restated below.

\begin{lemma}[Restatement of \cref{lem:distance-I}]
    \distanceILemma
\end{lemma}
\begin{proof}
    Assume towards a contradiction that there exist $f$ and $\alpha$ such that \cref{eq:f-alpha} does not hold.
    Namely, for $m = 1/\alpha^2$, there exist more than $2m$ indices $i \in [n]$ with $$\size{\:\ex{f(R) \mid R_i = 0}-\ex{f(R) \mid R_i = 1}\:}\geq 1/\sqrt{m}.$$
    This implies that there exist $b \in \oo$ and more than $m$ indices $i \in [n]$ with $$\ex{f(R) \mid R_i = b}-\ex{f(R) \mid R_i = 1-b} \geq 1/\sqrt{m}$$ (denote this set by $\cI$). 
    Thus, we deduce for $I \la \cI$ that
    \begin{align}\label{eq:big-R_I-gap}
        \size{\ex{f(R) \mid R_I = 0}-\ex{f(R) \mid R_I = 1}} 
        &\geq \ex{f(R) \mid R_I = b}-\ex{f(R) \mid R_I = 1-b}\\
        &\geq 1/\sqrt{m}.\nonumber
    \end{align}
    On the other hand, note that
    \begin{align*}
        SD(f(R)|_{R_I=0},f(R)|_{R_I=1})
        \leq SD(R|_{R_I=0},R|_{R_I=1})
        \leq 1/\sqrt{\size{\cI}}
        < 1/\sqrt{m},
    \end{align*}
    where the second inequality holds by \cref{fact:I}.
    Thus, we conclude that
    \begin{align*}
        \size{\ex{f(R) \mid R_I = 0}-\ex{f(R) \mid R_I = 1}}
        &\leq SD(f(R)|_{R_I=0},f(R)|_{R_I=1}) \cdot \sup_{r \in \zo^n, s \in \zo^*}(\size{f_s(r)})\\
        &< 1/\sqrt{m},
    \end{align*}
    where $f_s(r)$ denotes the function $f$ when fixing its random coins to $s$. 
    The above inequality contradicts \cref{eq:big-R_I-gap}, concluding the proof of the lemma.
\end{proof}
}

\subsection{Proving \cref{prop:hard-to-guess-inf,prop:hard-to-guess-comp}}\label{sec:missing-proofs:hard-to-guess}

We make use of the following claim.

\begin{claim}\label{claim:X-star}
    Let $X \la \oo$ and let $X^*$ be a random variable over $\set{-1,1,\bot}$ (correlated with $X$) such that for every $b,b' \in \oo$:
    \begin{align*}
        \pr{X^* = b \mid X = b'} \leq e^{\eps}\cdot \pr{X^* = b \mid X = -b'} + \delta.
    \end{align*}
    Then
    \begin{align*}
        \pr{X^* = X} \leq e^{\eps}\cdot \pr{X^* = -X} + \delta.
    \end{align*}
\end{claim}
\begin{proof}
    Compute
    \begin{align*}
        \pr{X^* = X } 
        &= \frac12 \cdot \pr{X^* = -1 \mid X = -1} + \frac12 \cdot \pr{X^* = 1 \mid X = 1}\\
        &\leq  \frac12 \cdot \paren{e^{\eps}\cdot \pr{X^* = -1 \mid X = 1} + \delta} + \frac12 \cdot \paren{e^{\eps}\cdot \pr{X^* = 1 \mid X = -1} + \delta}\\
        &= e^{\eps} \cdot \paren{\frac12 \cdot \pr{X^* = -1 \mid X = 1} + \frac12 \cdot \pr{X^* = 1 \mid X = -1}} + \delta\\
        &= e^{\eps} \cdot \pr{X^*_i = -X_i} + \delta.
    \end{align*}
\end{proof}


We next prove \cref{prop:hard-to-guess-inf}, restated below.

\begin{proposition}[Restatement of \cref{prop:hard-to-guess-inf}]
    \propHardToGuessInf
\end{proposition}
\begin{proof}
    %Note that it is enough to prove the claim just for deterministic $g$'s because if the statement does not hold for a specific randomized $g$, then there exists a fixing of its randomness that it does not hold under this fixing. 
    
    Fix $b,b' \in \oo$ and $i \in [n]$. By \cref{claim:X-star}, it is sufficient to prove that
    \begin{align}\label{eq:X_i^*-goal}
        \pr{X_i^* = b \mid X_i = b'} \leq e^{\eps}\cdot \pr{X_i^* = b \mid X_i = -b'} + \delta.
    \end{align}
    For $x_{-i} = (x_1,\ldots,x_{i-1},x_{i+1},\ldots,x_n) \in \oo^{n-1}$, define the function
    \begin{align*}
        h_{x_{-i}}(y) = g(i,x_{-i},y).
    \end{align*}
    Since $f$ is $(\eps,\delta)$-\DP, for any $x_{-i} \in \oo^{n-1}$ it holds that
    \begin{align*}
        \pr{h_{x_{-i}}(f(x_1,\ldots,x_{i-1}, b', x_{i+1},\ldots,x_n)) = b} \leq e^{\eps}\cdot \pr{h_{x_{-i}}(f(x_1,\ldots,x_{i-1}, -b', x_{i+1},\ldots,x_n)) = b} + \delta.
    \end{align*}
    Thus,
    \begin{align*}
        \pr{X_i^* = b \mid X_i = b'}
        &= \eex{x_{-i} \la \oo^{n-1}}{\pr{g(i,x_{-i}, f(x_1,\ldots,x_{i-1}, b', x_{i+1},\ldots,x_n)) = b }}\\
        &= \eex{x_{-i} \la \oo^{n-1}}{\pr{h_{x_{-i}}(f(x_1,\ldots,x_{i-1}, b', x_{i+1},\ldots,x_n)) = b}}\\
        &\leq e^{\eps}\cdot \eex{x_{-i} \la \oo^{n-1}}{\pr{h_{x_{-i}}(f(x_1,\ldots,x_{i-1}, -b', x_{i+1},\ldots,x_n)) = b}} + \delta\\
        &= e^{\eps}\cdot\pr{X_i^* = b \mid X_i = -b'}+ \delta.
    \end{align*}
\end{proof}


We next prove \cref{prop:hard-to-guess-comp}, restated below.

\begin{proposition}[Restatement of \cref{prop:hard-to-guess-comp}]
    \propHardToGuessComp
\end{proposition}
\begin{proof}
    
    In the following, fix $b,b' \in \oo$, and for $\pk \in \bbN$, $i \in [n(\pk)]$ and $x_{-i} \in \oo^{n(\pk)-1}$, define 
    \begin{align*}
        h_{\pk}^{i,x_{-i}}(y) = b\cdot g_{\pk}(i,x_{-i},y).
    \end{align*}
    Note that for any ensemble $\set{(i,x_{-i})_{\pk} \in [n(\pk)]\times \oo^{n(\pk)-1}}_{\pk \in \bbN}$, the circuit family \\$\set{h_{\pk}^{(i, x_{-i})_{\pk}} = g_{\pk}((i,x_{-i})_{\pk},\cdot)}_{\pk \in \bbN}$ has poly-size. 
    Since $f = \set{f_{\pk}}_{\pk \in \bbN}$ is $(\eps,\delta)$-\CDP, then for large enough $\pk$, the following holds for every $i \in [n(\pk)]$ and $x_{-i} \in \oo^{n(\pk)-1}$:
    \begin{align*}
        \lefteqn{\pr{h_{\pk}^{i, x_{-i}}(f_{\pk}(x_1,\ldots,x_{i-1}, b', x_{i+1},\ldots,x_n)) = 1}}\\
        &\leq e^{\eps(\pk)}\cdot \pr{h_{\pk}^{i,x_{-i}}(f(x_1,\ldots,x_{i-1}, -b', x_{i+1},\ldots,x_n)) = 1} + \delta(\pk),
    \end{align*}
    as otherwise, there would exist an ensemble $\set{(i,x_{-i})_{\pk} \in [n(\pk)]\times \oo^{n(\pk)-1}}_{\pk \in \cS}$ for an infinite set $\cS \subseteq \bbN$ such that the circuit family 
    $\set{h_{\pk}^{(i, x_{-i})_{\pk}}}_{\pk \in \cS}$ violates the $(\eps,\delta)$-\CDP property of $f$. 

    In the following, fix such large enough $\pk$ and $i \in [n]$ for $n = n(\kappa)$, let $X = (X_1,\ldots,X_{n}) \la \oo^{n}$ and $X_i^* = g_{\pk}(i,X_{-i},f_{\pk}(X_1,\ldots,X_n))$, and compute
    \begin{align*}
        \lefteqn{\pr{X_i^* = b \mid X_i = b'}}\\
        &= \eex{x_{-i} \la \oo^{n-1}}{\pr{g_{\pk}(i,x_{-i}, f_{\pk}(x_1,\ldots,x_{i-1}, b', x_{i+1},\ldots,x_n)) = b }}\\
        &= \eex{x_{-i} \la \oo^{n-1}}{\pr{h_{\pk}^{i,x_{-i}}(f(x_1,\ldots,x_{i-1}, b', x_{i+1},\ldots,x_n)) = 1}}\\
        &\leq e^{\eps(\pk)}\cdot \eex{x_{-i} \la \oo^{n-1}}{\pr{h_{\pk}^{i,x_{-i}}(f(x_1,\ldots,x_{i-1}, -b', x_{i+1},\ldots,x_n)) = 1}} + \delta(\pk)\\
        &= e^{\eps(\pk)}\cdot \eex{x_{-i} \la \oo^{n-1}}{\pr{g_{\pk}(i,x_{-i}, f_{\pk}(x_1,\ldots,x_{i-1}, -b', x_{i+1},\ldots,x_n)) = b}} + \delta(\pk)\\
        &= e^{\eps(\pk)}\cdot\pr{X_i^* = b \mid X_i = -b'}+\delta(\pk).
    \end{align*}
    Since the above holds for any $b, b' \in \oo$, we conclude by \cref{claim:X-star} that 
    \begin{align*}
        \pr{X_i^* = X_i} \leq e^{\eps(\pk)}\cdot \pr{X_i^* = -X_i} + \delta(\pk),
    \end{align*}
    as required.
\end{proof}

\subsection{Proving \cref{lemma:property1:prediction}}\label{sec:prediction-lemma}

To prove \cref{lemma:property1:prediction}, we use the following lemma that measures the distance between two uniform stings conditioned on a random index $i$ either being fixed to $0$ or to $1$.

\def\distanceILemma{
    Let $R \la \zo^n$. For any (randomized) function $F:\{0,1\}^n\rightarrow \{0,1\}$ and $\alpha > 0$, it holds that
    \begin{align}\label{eq:f-alpha}
        \ppr{i \la [n]}{\size{\:\ex{F(R) \mid R_i = 0}-\ex{F(R) \mid R_i = 1}\:}\geq \alpha} \leq \frac{2}{n \alpha^2},
    \end{align}
    where the expectations are taken over $R$ and the randomness of $F$.
}

\begin{lemma}\label{lem:distance-I}
    \distanceILemma
\end{lemma}

The proof of \cref{lem:distance-I} uses the following fact.

\begin{fact}[Proposition 3.28 in \cite{HaitnerMST22}]\label{fact:I}
	Let $R$ be uniform random variable over $\{0,1\}^n$, and let $I$ be uniform random variable over $\mathcal{I}\subseteq[n]$, independent of $R$, then $SD(R|_{R_I=0},R|_{R_I=1})\leq1/\sqrt{\size{\cI}}$.
\end{fact}

We first prove \cref{lem:distance-I} using \cref{fact:I}.

\begin{proof}[Proof of \cref{lem:distance-I}]
    Assume towards a contradiction that there exist $F$ and $\alpha$ such that \cref{eq:f-alpha} does not hold.
    Namely, for $m = 1/\alpha^2$, there exist more than $2m$ indices $i \in [n]$ with $$\size{\:\ex{F(R) \mid R_i = 0}-\ex{F(R) \mid R_i = 1}\:}\geq 1/\sqrt{m}.$$
    This implies that there exist $b \in \oo$ and more than $m$ indices $i \in [n]$ with $$\ex{F(R) \mid R_i = b}-\ex{F(R) \mid R_i = 1-b} \geq 1/\sqrt{m}$$ (denote this set by $\cI$). 
    Thus, we deduce for $I \la \cI$ that
    \begin{align}\label{eq:big-R_I-gap}
        \size{\ex{F(R) \mid R_I = 0}-\ex{F(R) \mid R_I = 1}} 
        &\geq \ex{F(R) \mid R_I = b}-\ex{F(R) \mid R_I = 1-b}\\
        &\geq 1/\sqrt{m}.\nonumber
    \end{align}
    On the other hand, note that
    \begin{align*}
        SD(F(R)|_{R_I=0},F(R)|_{R_I=1})
        \leq SD(R|_{R_I=0},R|_{R_I=1})
        \leq 1/\sqrt{\size{\cI}}
        < 1/\sqrt{m},
    \end{align*}
    where the second inequality holds by \cref{fact:I}.
    Thus, we conclude that
    \begin{align*}
        \size{\ex{F(R) \mid R_I = 0}-\ex{F(R) \mid R_I = 1}}
        &\leq SD(F(R)|_{R_I=0},F(R)|_{R_I=1}) \cdot \sup_{r \in \zo^n, s \in \zo^*}(\size{F_s(r)})\\
        &< 1/\sqrt{m},
    \end{align*}
    where $F_s(r)$ denotes the function $F$ when fixing its random coins to $s$. 
    The above inequality contradicts \cref{eq:big-R_I-gap}, concluding the proof of the lemma.
\end{proof}

Using \cref{lem:distance-I}, we now prove \cref{lemma:property1:prediction}, restated below.

\begin{lemma}[Restatement of \cref{lemma:property1:prediction}]
    \PredictionLemma
\end{lemma}
\begin{proof}
	
	Let $\gamma \in (0,1)$ and $n \in \bbN$ and consider the following algorithm $\Gc = \Gc_{\gamma}$:
	\begin{algorithm}
		\item Inputs: $i \in [n]$,  $z_{-i} = (z_1,\ldots,z_{i-1},z_{i+1},\ldots,z_n) \in \oo^{n-1}$ and $w \in \zo^*$.
		\item Parameter: $\gamma \in (0,1)$.
		\item Oracle: $F \colon \zo^n \times \oo^{\leq n} \times \zo^* \rightarrow \zo$.
		\item Operation:~
		\begin{enumerate}
			\item For $b \in \oo$:
			\begin{enumerate}
				\item Let $z^b = (z_1,\ldots,z_{i-1},b,z_{i+1},\ldots,z_n)$.
				\item Estimate $\mu^b \eqdef \eex{r \la \zo^n \mid r_i = 1, \: F}{F(r,z^b_{r}, w)}$ as follows:
				\begin{itemize}
					\item Sample $r_1,\ldots,r_{s} \la \set{r \in \zo^n \colon r_i = 1}$, for $s = \ceil{\frac{128 \log (12n)}{\gamma^2}}$, and then sample $\tilde{\mu}^b \la \frac1{s} \sum_{j=1}^s F(r_j,z^b_{r_j}, w)$ (using $s$ oracle calls to $F$).
				\end{itemize}
			\end{enumerate}
			\item Estimate $\mu^* \eqdef \eex{r \la \zo^n \mid r_i = 0, \: f}{f(r,z^1_{r}, w)}$ as follows:
			\begin{itemize}
				\item Sample $r_1,\ldots,r_{s}  \la \set{r \in \zo^n \colon r_i = 0}$, for $s = \ceil{\frac{128 \log (12n)}{\gamma^2}}$, and then sample $\tilde{\mu}^* \la \frac1{s} \sum_{j=1}^s F(r_j,z^1_{r_j}, w)$ (using $s$ oracle calls to $F$).
			\end{itemize}
			\item If exists $b \in \oo$ s.t. $\size{\tilde{\mu}^b - \tilde{\mu}^*} < \gamma/4$ and $\size{\tilde{\mu}^{-b} - \tilde{\mu}^*} > \gamma/4$, output $b$.
			\item Otherwise, output $\bot$.
		\end{enumerate}
	\end{algorithm}
	In the following, fix a pair of (jointly distributed) random variables $(Z,W) \in \oo^n \times \zo^*$ and a randomized function  $F \colon \zo^n \times \oo^{\leq n} \times \zo^* \rightarrow \oo$ that satisfy 
	\begin{align*}
		\size{\ex{F(R,Z_{R},W) - F(R,Z^{(I)}_{R},W)}} \geq \gamma,
	\end{align*}
	for $R \la \zo^n$ and $I \la [n]$ that are sampled independently. 
	Our goal is to prove that 
	
	\begin{align}\label{eq:predic-lemma:UB}
		\pr{\Gc^F(I,Z_{-I}, W) = -Z_I} \leq O\paren{\frac{1}{\gamma^2 n}},
	\end{align}
	and 
	\begin{align}\label{eq:predic-lemma:LB}
		\pr{\Gc^F(I,Z_{-I}, W) = Z_I} \geq \Omega(\gamma) - O\paren{\frac{1}{\gamma^2 n}}.
	\end{align}

	Note that
	\begin{align}\label{eq:D}
		\text{For every random variable }D \in [-1,1]\text{ with }\size{\ex{D}} \geq \gamma >0: \quad \pr{\size{D} > \gamma/2} > \gamma/2,
	\end{align}
	as otherwise, $\size{\ex{D}} \leq \ex{\size{D}} \leq 1\cdot \frac{\gamma}{2} + \frac{\gamma}{2}(1-\frac{\gamma}{2}) < \gamma$.
	
	
	By applying \cref{eq:D} with $D = \eex{r \la \zo^n, F}{F(r,Z_{r},W) - f(r,Z^{(I)}_{r},W)}$, it holds that
	\begin{align}\label{eq:main-lemma:good}
		\ppr{(i,z,w) \la (I,Z,W)}{\size{\eex{r \la \zo^n, F}{F(r,z_{r},w) - F(r,z^{(i)}_{r},w)}} > \gamma/2} > \gamma/2.
	\end{align}
	On the other hand, for every fixing of $(z,w) \in \Supp(Z,W)$, we can apply \cref{lem:distance-I} with the function $F_{z,w}(r) = F(r,z_{r},w)$ and with $\alpha =\frac{\gamma}{16}$ to obtain that
	\begin{align*}
		\ppr{i\la I}{\: \size{\eex{r \la \zo^n \mid r_i = 0, \: F}{F(r,z_{r},w)} - \eex{r \la \zo^n \mid r_i = 1, \: F}{F(r,z_{r},w)} \:} \geq \frac{\gamma}{16}} \leq \frac{512}{n \gamma^2}.
	\end{align*}
	But since the above holds for every fixing of $(z,w)$, then in particular it holds that
	\begin{align}\label{eq:main-lemma:bad}
		\ppr{(i,z,w) \la (I,Z,W)}{\: \size{\eex{r \la \zo^n \mid r_i = 0, \: F}{F(r,z_{r},w)} - \eex{r \la \zo^n \mid r_i = 1, \: F}{F(r,z_{r},w)} \:} \geq \frac{\gamma}{16}} \leq \frac{512}{n \gamma^2}.
	\end{align}
	
	We next prove \cref{eq:predic-lemma:UB,eq:predic-lemma:LB} using \cref{eq:main-lemma:good,eq:main-lemma:bad}.
	
	In the following, for a triplet $t = (i,z,w) \in \Supp(I,Z,W)$, consider a random execution of $\Gc^F(i,z_{-i},w)$. For $x \in \set{-1,1,*}$, let $\mu^x_t$ be the value of $\mu^x$ in the execution, and let $M^x_t$ be the (random variable of the) value of $\tilde{\mu}^x$ in the execution. Note that by definition, it holds that $\mu_{i,z,w}^{z_i} = \eex{r \la \zo^n, F}{F(r,z_{r},w)}$ and $\mu_{i,z,w}^{-z_i} = \eex{r \la \zo^n, F}{F(r,z^{(i)}_{r},w)}$. Therefore, \cref{eq:main-lemma:good} is equivalent to 
	\begin{align}\label{eq:main-lemma:good2}
		\ppr{t=(i,z,w) \la (I,Z,W)}{\size{\mu_t^{z_i} - \mu_t^{-z_i}} > \gamma/2} > \gamma/2.
	\end{align}
	Furthermore, note that $\mu_{i,z,w}^* \eqdef \eex{r \la \zo^n \mid r_i = 0, \: F}{F(r,z^1_{r}, w)} = \eex{r \la \zo^n \mid r_i = 0, \: F}{F(r,z_{r}, w)}$ and that $\mu_{i,z,w}^{z_i} = \eex{r \la \zo^n \mid r_i = 1, \: F}{F(r,z^{z_i}_{r}, w)}$. Therefore, \cref{eq:main-lemma:bad} is equivalent to 
	\begin{align}\label{eq:main-lemma:bad2}
		\ppr{t = (i,z,w) \la (I,Z,W)}{\: \size{\mu_t^* - \mu_t^{z_i}}\geq \frac{\gamma}{16}}  \leq \frac{512}{n \gamma^2}.
	\end{align}
	
	We next prove the lemma using \cref{eq:main-lemma:good2,eq:main-lemma:bad2}.
	
	Note that by Hoeffding's inequality, for every $t = (i,z,w) \in \Supp(I,Z,W)$ and  $x \in \set{-1,1,*}$ it holds that $\pr{\size{M_t^x - \mu_t^x} \geq \frac{\gamma}{16}} \leq 2\cdot e^{-2 s \paren{\frac{\gamma}{16}}^2} \leq \frac1{6n}$, which yields that for every fixing of $t = (i,z,w) \in \Supp(I,Z,W)$, w.p.\ at least $1-\frac1{2n}$ we have for all $x \in \set{-1,1,*}$ that $\size{M_t^x - \mu_t^x} < \frac{\gamma}{16}$ (denote this event by $E_t$).
	
	The proof of \cref{eq:predic-lemma:UB} holds by the following calculation:
	\begin{align*}
		\lefteqn{\pr{\Gc^F(I,Z_{-I},W) = -Z_I}}\\
		&= \eex{(i,z,w) \la (I,Z,W)}{\pr{\Gc^F(i,z_{-i},w) = -z_i}}\\
		&= \eex{t = (i,z,w) \la (I,Z,W)}{\pr{\set{\size{M_t^* - M_t^{z_i}} > \gamma/4} \land \set{\size{M_t^* - M_t^{-z_i}} < \gamma/4}}}\\
		&\leq \eex{t =(i,z,w) \la (I,Z,W)}{\pr{\size{M_t^* - M_t^{z_i}} > \gamma/4}}\\
		&\leq \eex{t = (i,z,w) \la (I,Z,W)}{\pr{\size{M_t^* - M_t^{z_i}} > \gamma/4 \mid E_t}} + \frac{1}{2n}\\
		&\leq \ppr{t = (i,z,w) \la (I,Z,W)}{\size{\mu_t^* -  \mu_t^{z_i}} \geq \frac{\gamma}{16}} + \frac{1}{2n}\\
		&\leq \frac{512}{n \gamma^2} + \frac1{2n},
	\end{align*}
	The second inequality holds since $\pr{\neg E_t} \leq \frac1{2n}$ for every $t$. The penultimate inequality holds since conditioned on $E_t$, it holds that $\size{M_t^* - \mu_t^*} \leq \frac{\gamma}{16}$ and $\size{M_t^{z_i} - \mu_t^{z_i}} \leq \frac{\gamma}{16}$, which implies that $\size{M_t^* - M_t^{z_i}} > \gamma/4 \: \implies \: \size{\mu_t^* -  \mu_t^{z_i}} \geq \frac{\gamma}{4} - 2\cdot \frac{\gamma}{16} > \frac{\gamma}{16}$. The last inequality holds by \cref{eq:main-lemma:bad2}.
	
	It is left to prove \cref{eq:predic-lemma:LB}. 
	Observe that \cref{eq:main-lemma:good2,eq:main-lemma:bad2} imply with probability at least $\gamma/2 - \frac{512}{n \gamma^2}$ over $t = (i,z,w)\in (I,Z,W)$, we have $\size{\mu_t^* -  \mu_t^{z_i}} \leq  \frac{\gamma}{16}$ and $\size{\mu_t^{z_i}- \mu_t^{-z_i}} \geq\frac{\gamma}{2}$. Therefore, conditioned on event $E_t$ (i.e., $\forall x \in \set{-1,1,*}: \: \size{M_t^{x} - \mu_t^{x}} \leq \frac{\gamma}{16}$), we have for such triplets $t = (i,z,w)$ that 
	\begin{align*}
		\size{M_t^* - M_t^{-z_i}} 
		& \geq \size{\mu_t^{z_i} - \mu_t^{-z_i}} - \size{\mu_t^{z_i} - \mu_t^*} - \size{M_t^*  - \mu_t^*} -  \size{M_t^{-z_i} - \mu_t^{-z_i}}\\
		&\geq \frac{\gamma}{2} - 3\cdot \frac{\gamma}{16}\\
		&> \gamma/4,
	\end{align*}
	and 
	\begin{align*}
		\size{M_t^* - M_t^{z_i}}
		&\leq  \size{M_t^* - \mu_t^*} + \size{\mu_t^* - \mu_t^{z_i}} + \size{\mu_t^{z_i} - M_t^{z_i}} \\
		&\leq 3\cdot \frac{\gamma}{16}\\
		&< \gamma/4.
	\end{align*}
	
	Thus, we conclude that
	\begin{align*}
		\lefteqn{\pr{\Gc^F(I,Z_{-I},W) = Z_I}}\\
		&= \eex{(i,z,w) \la (I,Z,W)}{\pr{\Gc^F(i,z_{-i},w) = z_i}}\\
		&= \eex{t = (i,z,w) \la (I,Z,W)}{\pr{\set{\size{M_t^* - M_t^{-z_i}} > \gamma/4} \land \set{\size{M_t^* - M_t^{z_i}} < \gamma/4}}}\\
		&\geq \paren{1 - \frac1{2n}}\cdot \eex{t = (i,z,w) \la (I,Z,W)}{\pr{\set{\size{M_t^* - M_t^{-z_i}} > \gamma/4} \land \set{\size{M_t^* - M_t^{z_i}} < \gamma/4} \mid E_t}}\\
		&\geq \paren{1 - \frac1{2n}}\cdot \paren{\gamma/2 - \frac{512}{n \gamma^2}}\\
		&\geq \gamma/4 - \frac{512}{n \gamma^2},
	\end{align*}
	which proves \cref{eq:predic-lemma:LB}. The first inequality holds since $\pr{E_t} \geq 1-\frac1{2n}$ for every $t = (i,z,w)$, and the second one holds by the observation above.
	
\end{proof}