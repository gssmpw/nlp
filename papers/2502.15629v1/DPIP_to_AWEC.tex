\newcommand{\GenRand}{\MathAlgX{GenRand}}
\newcommand{\GenView}{\MathAlgX{GenView}}

\section{\AWEC From \DP Inner Product (Proof of \cref{lemma:DPIP-to-AWEC})}\label{sec:DPIP_to_WAEC}

In this section, we show how to implement AWEC (\cref{def:AWEC}) from an $(\eps,\delta)$-DP channel that is accurate enough for the inner product functionality. We do that using a \ppt protocol and a constructive security proof.



%$C = ((X,U = (U',O)),(Y,V = (V',O)))$ with independent $X,Y \la \oo^n$ that is 
%$\ell(n)$-accurate for the inner product functionality (i.e., $\ppr{C}{\size{O-\ip{X,Y}} \leq \ell(n)} \geq 0.99$), for $\ell(n) = \tilde{\Theta}(n^{1/6})$. Importantly, our security proof is constructive. 


%\Cnote{we can set $m=O(\frac{(n\cdot\frac{e^\varepsilon+1}{e^\varepsilon})^{2/5}}{\log^{2/5}(n\cdot\frac{e^\varepsilon+1}{e^\varepsilon})})$, $k=O(m^{1/2})=O(\frac{n^{1/5}}{\log^{1/5}n})$, $\ell=O(k^{1/2})=O(\frac{n^{1/10}}{\log^{1/10}n})$}





The following protocol is used to prove \cref{lemma:DPIP-to-AWEC}.

\begin{protocol}[Protocol $\Pi = (\Ac,\Bc)$]\label{protocol:DPIP-to-AWEC}
	\item Oracle access: An $(\eps,\delta)$-DP channel $C =((X,U),(Y,V))$ with $X,Y \la \oo^n$\remove{ (i.e., \emph{uniform} channel)} that is $(\ell,0.999)$-accurate for the \emph{inner-product} functionality.
	\item Operation:
	\begin{enumerate}
		%\item  $\Ac$ samples $x\gets \mon$, and   $\Bc$ samples $y\gets \mon$.
		
		\item Sample $(x,u), (y,v) \la C$. $\Ac$ gets $(x,u)$ and $\Bc$ gets $(y,v)$. 
		
		\item  $\Ac$ samples  $r \la \zo^n$ and sends $(r,x_{r} = \set{x_i \colon r_i =1})$ to $\Bc$.
		
		\item $\Bc$ samples a random bit $b \la \zo$ and acts as follows:
		
		\begin{enumerate}
			
			\item If $b=0$, it sends $y_{-r}= \set{y_i \colon r_i =0}$ to $\Ac$ and outputs $o_B = \out(v) - \ip{x_r, y_r}$.
			
			\item~\label{step: add noise} Otherwise ($b=1$), it performs the following steps:\label{B_steps_in_abort}
			
			\begin{itemize}
				\item Sample $k$ uniformly random indices $i_1,\ldots,i_k \la [n]$, where $k = \floor{e^{\lambda_1 \eps} \cdot \lambda_2 \cdot \ell^2}$ for constants $\lambda_1,\lambda_2>0$ (to be determined later by the analysis in \cref{eq:lamdas}).
				\item Compute $\ty = (\ty_1,\ldots,\ty_n)$ where $\ty_i \la \begin{cases} \cU_{\oo} & i \in \set{i_1,\ldots,i_k} \\ y_i & \text{otherwise} \end{cases}$,
				\item Send $\ty_{-r}= \set{\ty_i \colon r_i =0}$ to $\Ac$, and output $o_B = \perp$.
				%\item Output $o_B = \perp$.
			\end{itemize}
			
			
		\end{enumerate}
		
		\item  Denote by $\hy_{-r}$ the message $\Ac$ received from $\Bc$. Then $\Ac$ outputs $o_A =  \ip{x_{-r}, \hy_{-r}}$.
		
		
	\end{enumerate}
\end{protocol}

In the following, let $\Pi$ be \cref{protocol:DPIP-to-AWEC},
let $C =((X,U),(Y,V))$ be a uniform channel that is $(\ell,0.999)$-accurate for the \emph{inner-product} functionality, let $\tC = ((O_A,V_A),(O_B,V_B))$ be the channel that is induced by $\Pi^C$ (i.e., the parties' outputs and views in the execution of $\Pi$ with oracle access to $C$), and let $R$, $\tY$, $\hY$, $I_1,\ldots,I_k$ be the random variables of the values of $r, \ty, \hy, i_1,\ldots,i_k$ in the execution of $\Pi^C$ (recall that the view of $\Bc$ in the execution is $V_B = (Y, V, R, X_R, I_1,\ldots,I_k, \tY)$, and the view of $\Ac$ is $V_A = (X, U, R, \hY_{-R})$).

Recall that to prove \cref{lemma:DPIP-to-AWEC}, we need to prove that the channel $\tC$ satisfies the accuracy and secrecy properties of \AWEC (\cref{def:AWEC}), and in addition, the secrecy guarantees are constructive as stated in Properties \ref{item:privacy-of-Y}-\ref{item:privacy-of-X} of Lemma~\ref{lemma:DPIP-to-AWEC} (i.e., a violation of at least one secrecy guaranty results with an efficient privacy attack on the DP channel $C$).

We first start with the easy part, which is to prove the accuracy guarantee of $\tC$.

\begin{claim}[Accuracy of $\tC$]
	It holds that
	\begin{align*}
		\pr{\size{O_A - O_B} > \ell \mid O_B \neq \bot} < 0.001. 
	\end{align*}
\end{claim}
\begin{proof}
    Compute
    \begin{align*}
        \pr{\size{O_A - O_B} > \ell \mid O_B \neq \bot}
        &= \pr{\size{\ip{X_{-R},Y_{-R}} - \paren{\out(V) - \ip{X_R,Y_R}}} > \ell}\\
        &= \pr{\size{\out(V) - \ip{X,Y}} > \ell}\\
        &< 0.001,
    \end{align*}
    as required. The first equality holds since conditioned on $O_B \neq \bot$ it holds that $O_A = \ip{X_{-R},Y_{-R}}$, and the inequality holds since $C$ is an $(\ell,0.999)$-accurate for the inner-product functionality. 
\end{proof}

We next move to prove the secrecy guarantees of $\tC$ in a constructive manner. 
In \cref{sec:proving-prop1} we prove property~\ref{item:privacy-of-Y}, and in \cref{sec:proving-prop2} we prove property~\ref{item:privacy-of-X}.



%The following claim (proven in \cref{sec:proving-prop1}) captures the first secrecy guarantee where party $\Ac$ (almost) cannot distinguish if $\Bc$ aborts (i.e., $O_B = \perp$) or not, as otherwise, such a distinguisher $\Ac$ can be used to construct an efficient attack $\Act$ that violates the privacy guarantee of the DP channel $C$.

\remove{
	Finally, the following claim (proven in \cref{sec:property 2}) captures the second secrecy guarantee where party $\Bc$ cannot estimate $O_A$ too well, as otherwise, such an estimator $\Bc$ can be used to construct an efficient attack $\Bct$ that violates the privacy guarantee of the DP channel $C$.
	
	
	\begin{claim}[Property 2 of Lemma~\ref{lemma:DPIP-to-AWEC}]~\label{clm:property 2}
		There exists an oracle-aided \ppt algorithm $\Bct$ such that for every algorithm $\Bc$ violating the AWEC secrecy property~\ref{AWEC:item:B} of $\tilde{C}$, i.e.
		\begin{align*}
			\pr{\size{\Bc(V_B) - O_A} \leq 1000\ell  \mid O_B=\bot} > q,
		\end{align*}
		where $\frac{131072\ell_2^2}{q^2k}\leq\frac{1}{e^{\varepsilon}+1}$,
		the for $X_i^* = \Bct^{\Bc}(i,\: X_{-i}, \: Y, \: V)$  it holds that
		\begin{align*}
			\eex{i \la [n]}{\pr{X_i^* = X_i \mid X_i^* \neq \bot}} > \frac{e^{\eps}}{e^{\eps}+1}.
		\end{align*}
	\end{claim}
	
	\Enote{Remove:}
	
	
	\begin{proof}
		We remind that $\tilde{C}$ is an AWEC channel constructed by a differentially private inner-product protocol. Let $\mathcal{R}=\{i|r_i=0\}$ and $\cI = \set{i_1,\dots,i_k} \cap \cR$ \remove{$\mathcal{I}=\{i|i\in\{i_1,\dots,i_k\},i\in \mathcal{R}\}$}. Then $O_A=o-\ip{x_{\mathcal{R}\setminus \mathcal{I}},\hy_{\mathcal{R}\setminus \mathcal{I}}}-\ip{x_{ \mathcal{I}},\hy_{ \mathcal{I}}}$. It suffices to prove party $\Bc$ cannot predict $\ip{x_{ \mathcal{I}},\hy_{ \mathcal{I}}}$ with high accuracy. We put the details in Section~\ref{sec:property 2}  
	\end{proof}
}




\subsection{B's Security: Proving Property~\ref{item:privacy-of-Y} of \cref{lemma:DPIP-to-AWEC}}\label{sec:proving-prop1}

Let $\Ac$ be an algorithm that violates the AWEC secrecy property~\ref{AWEC:item:A} of $\tilde{C}$ (the channel of $\Pi^C$). Namely,

\begin{align*}
	\size{\pr{\Ac(V_A) = 1 \mid O_B \neq \perp} - \pr{\Ac(V_A) = 1 \mid O_B = \perp}} > \frac1{1000}.
\end{align*}

%for $k \in \Theta\paren{\frac{n^{1/4} \paren{\frac{1}{e^{\eps}+1} - \delta}}{\log^{1/4} n}}$ and $\gamma = \frac1{100 k}$.
Recall that $V_A = (X,U,R,\hY_{-R})$ where $\hY = \begin{cases} Y & O_B \neq \perp \\ \tY & O_B = \perp\end{cases}$ and that $\tY$ is obtained from $Y$ by planting uniformly random $\oo$ values in the random locations $I_1,\ldots, I_k$ of $Y$ (the value of $k$ will be determined later by the analysis).

Therefore, the above inequality is equivalent to 

\begin{align*}%\label{eq:A-pr}
	\size{\pr{\Ac(X,U,R,Y_{-R}) = 1} - \pr{\Ac(X,U,R,\tY_{-R}) = 1}} > \frac1{1000}.
\end{align*}

We assume \wlg that $\Ac$ outputs a $\zo$ bit, so the above inequality can be written as

\begin{align}\label{eq:property1:assump}
	\size{\ex{\Ac(X,U,R,Y_{-R}) - \Ac(X,U,R,\tY_{-R})}} > \frac1{1000}.
\end{align}


In the following, for $j \in \set{0,\ldots,k}$, let $\tY^{j} = (\tY^{j}_1,\ldots, \tY^{j}_n)$ where $\tY^{j}_i = \begin{cases} \tY_i & i \in \set{I_1,\ldots,I_j} \\ Y_i & i \notin \set{I_1,\ldots,I_j}\end{cases}$.
Note that $\tY^{0} = Y$ and $\tY^{k} = \tY$. Therefore,

\begin{align}\label{eq:property1:hybrid}
    \lefteqn{\size{\eex{j \la [k]}{\ex{\Ac(X,U,R,Y^{(j-1)}_{-R}) - \Ac(X,U,R,\tY^{(j)}_{-R})}}}}\\
    &= \frac1k \cdot \size{\sum_{j=1}^k \ex{\Ac(X,U,R,Y^{(j-1)}_{-R})} - \ex{\Ac(X,U,R,Y^{(j)}_{-R})}}\nonumber\\
    &= \frac1k \cdot \size{\ex{\Ac(X,U,R,Y_{-R}) - \Ac(X,U,R,\tY_{-R})}}\nonumber\\
    &> \frac1{1000 k},\nonumber
\end{align}
where the inequality holds by \cref{eq:property1:assump}.

In the following, let $J \la [n]$ (sampled independently of the other random variables defined above), let $Z = (Z_1,\ldots,Z_n) = \tY^{J-1}$ and let $Z^{(i)} = (Z_1,\ldots,Z_{i-1}, -Z_i, Z_{i+1},\ldots, Z_n)$. It holds that

\begin{align}\label{eq:property1:exp}
	\lefteqn{\size{\ex{\Ac(X,U,R,Z_{-R})} - \ex{\Ac(X,U,R,Z^{(I_{J})}_{-R})}}}\\
        &= \frac12 \cdot \size{\ex{\Ac(X,U,R,Y^{(J-1)}_{-R}) - \Ac(X,U,R,\tY^{(J)}_{-R})}}\nonumber\\
        &= \frac12\cdot \size{\eex{j \la [k]}{\ex{\Ac(X,U,R,Y^{(j-1)}_{-R}) - \Ac(X,U,R,\tY^{(j)}_{-R})}}}\nonumber\\
        &> \frac1{2000 k},\nonumber
\end{align}
where the first equality holds since $\tY^{J}$ is equal to $Z$ w.p. $1/2$ (happens when $\tY_{I_{J}} = Y_{I_{J}}$)  and otherwise is equal to $Z^{(I_{J})}$ (happens when $\tY_{I_{J}} \neq Y_{I_{J}}$), and the last inequality holds by \cref{eq:property1:hybrid}.

%For the ease of notation, since $R \la \oo^n$, by symmetry we can assume \wlg that 
%\begin{align}\label{eq:property1:exp}
%	\size{\ex{\Ac(X,U,R,Z_{R})} - \ex{\Ac(X,U,R,Z^{(I_{J})}_{R})}} > \frac1{2000 k}.
%\end{align}



We next use the following lemma, proven in \cref{sec:prediction-lemma} (recall that a proof overview appears in \cref{sec:overview:prediction-lemma}).

\def\PredictionLemma{
    For every $\gamma \in (0,1)$ and $n \in \bbN$, there exists an oracle aided (randomized) algorithm $\Gc = \Gc_{\gamma} \colon [n] \times \zo^{n-1} \rightarrow \set{-1,1,\perp}$ that runs in time $\poly(n,1/\gamma)$ such that the following holds: 
	
	Let $(Z,W) \in \oo^n \times \zo^*$ be jointly distributed random variables, let $R \la \zo^n$ and $I \la [n]$ (sampled independently), 
	and let $F$ be a (randomized) function that satisfies
	\begin{align*}
		\size{\ex{F(R,Z_{R},W) - F(R,Z^{(I)}_{R},W)}} \geq \gamma,
	\end{align*}
	for $Z^{(I)} = (Z_1,\ldots, Z_{I-1}, -Z_I, Z_{I+1},\ldots, Z_n)$. Then
	\begin{enumerate}
		\item $\pr{\Gc^F(I,Z_{-I}, W) = -Z_I} \leq O\paren{\frac{1}{\gamma^2 n}}$, and\label{property1:prediction:bad}
		
		\item $\pr{\Gc^F(I,Z_{-I}, W) = Z_I} \geq \Omega(\gamma) - O\paren{\frac{1}{\gamma^2 n}}$.\label{property1:prediction:good}
	\end{enumerate}
}

\begin{lemma}\label{lemma:property1:prediction}
    \PredictionLemma
\end{lemma}

We now ready to finalize the proof of Property~\ref{item:privacy-of-Y} of \cref{lemma:DPIP-to-AWEC} using \cref{eq:property1:exp,lemma:property1:prediction}.

\begin{proof}
	
	Consider the following oracle-aided algorithm $\Act$:
	
	\begin{algorithm}[Algorithm $\Act$]\label{alg:Act}
		\item Inputs: $i \in [n]$, $y_{-i} \in \oo^{n-1}$, $x \in \oo^n$ and $u \in \zo^*$.
		\item Oracle: Deterministic algorithm $\Ac$.%and $\Gc$ from \cref{lemma:property1:prediction}.
		\item Operation:~
		\begin{enumerate}
			\item Sample $j \la [k]$ and $i_1,\ldots,i_{j-1} \la [n]$.
			
			\item Sample $z_{-i} = (z_1,\ldots,z_{i-1}, z_{i+1},\ldots, z_n)$, where for $t \in [n]\setminus \set{i}$: $$z_{t} \la \begin{cases} \cU_{\oo} & t \in \set{i_1,\ldots,i_{j-1}} \\ y_t & \text{otherwise} \end{cases}.$$\label{step:z-i}
			
			\item Output $y_i^* \la \Gc^{F}(i,z_{-i},w)$ for $w=(x,u)$, where $\Gc = \Gc_{\frac1{2000 k}}$ is the algorithm from \cref{lemma:property1:prediction}, and $F(r,z_r,w) \eqdef \Ac(w,(1-r_1,\ldots,1-r_n), z_r)$.
		\end{enumerate}
	\end{algorithm}

	Note that by \cref{eq:property1:exp} it holds that
	\begin{align*}
		\size{\ex{F(R,Z_{R}, W)} - \ex{F(R,Z_{R}^{I},W)}} > \frac1{2000 k}
	\end{align*}
	for $W = (X,U)$, $I = I_J$ and the function $F(r,z_r,w=(x,u)) \eqdef \Ac(w,(1-r_1,\ldots,1-r_n), z_r)$. Thus, \cref{lemma:property1:prediction} implies that
	\begin{align}\label{eq:G:LB}
		\pr{\Gc^{F}(I,Z_{-I}, V) = -Z_I} \leq O\paren{\frac{k^2}{n}}
	\end{align}
	and
	\begin{align}\label{eq:G:UB}
		\pr{\Gc^{F}(I,Z_{-I}, V) = Z_I} \geq \Omega(1/k) - O\paren{\frac{k^2}{n}}.
	\end{align} 
	
	Recall that we denote $Y_i^* = \Act^{\Ac}(i,Y_{-i},X,U)$. We next lower-bound $\eex{i \la [n]}{\pr{Y_i^* = Y_i}}$ and upper-bound $\eex{i \la [n]}{\pr{Y_i^* = -Y_i}}$.
	The first bound hold by the following calculation
	\begin{align}\label{eq:Y_i*-LB}
		\eex{i \la [n]}{\pr{Y_i^* = Y_i}}
		&\geq 0.9 \cdot \eex{i \la [n]}{\pr{Y_i^* = Z_i}}\\
		&= 0.9 \cdot \eex{i \la [n]}{\pr{\Gc^{F}(i,Z_{-i},W) = Z_i}}\nonumber\\
		&\geq \Omega\paren{\frac1k} - O\paren{\frac{k^2}{n}}.\nonumber
	\end{align}
	The first inequality holds since $i \notin \set{I_1,\ldots,I_{J-1}} \implies Z_i = Y_i$ and $\pr{i \notin \set{I_1,\ldots,I_{J-1}}} \geq \paren{1 - \frac1n}^{k-1} \geq 0.9$ (recall that $k \in o(n)$). The equality holds since, conditioned on $X=x,U=u,Y_{-i} = y_{-i}$ (the inputs of $\Act$), the value of $z_{-i}$ that is sampled in \stepref{step:z-i} of $\Act$ is distributed the same as $Z_{-i}|_{X=x,U=u,Y_{-i} = y_{-i}}$, and therefore, $Y_i^* \equiv \Gc^{F}(i,Z_{-i},W)$ for $W = (X,U)$. The last inequality holds by \cref{eq:G:UB}.
	
	On the other hand, we have that
	\begin{align}\label{eq:Y_i*-UB}
		\eex{i \la [n]}{\pr{Y_i^* = -Y_i}}
		&\leq \eex{i \la [n]}{\pr{Y_i^* = - Y_i \mid Y_i = Z_i} + \pr{Z_i \neq Y_i}}\\
		&\leq \eex{i \la [n]}{\pr{Y_i^* = - Z_i}} + \frac{k}{2n}\nonumber\\
		&=  \eex{i \la [n]}{\pr{\Gc^{F}(i,Z_{-i}) = - Z_i}} + \frac{k}{2n}\nonumber\\
		&\leq O\paren{\frac{k^2}{n}}.\nonumber
	\end{align}
	The second inequality holds since for any fixing $Y_{-i}=y_{-i},X=x,U=u$, we have $Y_i^*= \Act^{\Ac}(i,y_{-i},x,u)$ which is independent of $(Y_i,Z_i)$, and also since $i \notin \set{I_1,\ldots,I_{J-1}} \implies Z_i = Y_i$ which yields that
	\begin{align*}
		\pr{Z_i \neq Y_i} \leq 1 - \pr{i \notin \set{I_1,\ldots,I_{J-1}}} \leq 1 - \paren{1 - \frac1n}^{k} \leq 1 - e^{\frac{k}{2n}} \leq \frac{k}{2n}. 
	\end{align*}
	The equality in \cref{eq:Y_i*-UB} holds since, conditioned on $X=x,U=u,Y_{-i} = y_{-i}$ (the inputs of $\Act$), the value of $z_{-i}$ that is sampled in \stepref{step:z-i} of $\Act$ is distributed the same as $Z_{-i}|_{X=x,U=u,Y_{-i} = y_{-i}}$, and therefore, $Y_i^* \equiv \Gc^{F}(i,Z_{-i},W)$ for $W = (X,U)$. The last inequality in \cref{eq:Y_i*-UB} holds by \cref{eq:G:LB}.
	
	By \cref{eq:Y_i*-LB,eq:Y_i*-UB}, assuming that $\delta \leq 1/n$ where $n$ is large enough, 
	there exists a constant $c > 0$ such that if $k \leq c \cdot (e^{-\eps} n)^{1/3}$ then $\eex{i \la [n]}{\pr{Y_i^* = Y_i}} > e^{\eps} \cdot \eex{i \la [n]}{\pr{Y_i^* \neq Y_i}} + \delta$, as required.
	Recall that $k = \ceil{e^{\lambda_1 \eps} \lambda_2 \cdot \ell^2}$, where $\lambda_1,\lambda_2$ are the constants from \cref{eq:lamdas}. Hence, we can set a bound of $e^{-c_1 \eps} c_2 \cdot n^{1/6}$ on $\ell$ for $c_1 = \lambda_1/2 + 1/6$ and $c_2 = \sqrt{c/\lambda_2}$ to guarantee that $k \leq c \cdot (e^{-\eps} n)^{1/3}$.
	
\end{proof}

\remove{
\subsubsection{Proving \cref{lemma:property1:prediction}}\label{sec:prediction-lemma}
\begin{proof}
	
	Let $0 < \gamma \leq 0.01$ and consider the following algorithm $\Gc = \Gc_{\gamma}$:
	\begin{algorithm}
		\item Inputs: $i \in [n]$,  $z_{-i} = (z_1,\ldots,z_{i-1},z_{i+1},\ldots,z_n) \in \oo^{n-1}$ and $w \in \zo^*$.
		\item Parameter: $0 < \gamma \leq 0.01$.
		\item Oracle: $f \colon \zo^n \times \oo^{\leq n} \times \zo^* \rightarrow \zo$.
		\item Operation:~
		\begin{enumerate}
			\item For $b \in \oo$:
			\begin{enumerate}
				\item Let $z^b = (z_1,\ldots,z_{i-1},b,z_{i+1},\ldots,z_n)$.
				\item Estimate $\mu^b \eqdef \eex{r \la \zo^n \mid r_i = 1, \: f}{f(r,z^b_{r}, w)}$ as follows:
				\begin{itemize}
					\item Sample $r_1,\ldots,r_{s} \la \set{r \in \zo \colon r_i = 1}$, for $s = \ceil{\frac{128 \log (6n)}{\gamma^2}}$, and then sample $\tilde{\mu}^b \la \frac1{s} \sum_{j=1}^s f(r_j,z^b_{r_j}, w)$ (using $s$ calls to the oracle $f$).
				\end{itemize}
			\end{enumerate}
			\item Estimate $\mu^* \eqdef \eex{r \la \zo^n \mid r_i = 0, \: f}{f(r,z^1_{r}, w)}$ as follows:
			\begin{itemize}
				\item Sample $r_1,\ldots,r_{s}  \la \set{r \in \zo \colon r_i = 0}$, for $s = \ceil{\frac{128 \log (6n)}{\gamma^2}}$, and then sample $\tilde{\mu}^* \la \frac1{s} \sum_{j=1}^s f(r_j,z^1_{r_j}, w)$ (using $s$ calls to the oracle $f$).
			\end{itemize}
			\item If exists $b \in \oo$ s.t. $\size{\tilde{\mu}^b - \tilde{\mu}^*} < \gamma/4$ and $\size{\tilde{\mu}^{-b} - \tilde{\mu}^*} > \gamma/4$, output $b$.
			\item Otherwise, output $\bot$.
		\end{enumerate}
	\end{algorithm}
	In the following, fix a pair of (jointly distributed) random variables $(Z,W) \in \oo^n \times \zo^*$ and a randomized function  $f \colon \zo^n \times \oo^{\leq n} \times \zo^* \rightarrow \oo$ that satisfy 
	\begin{align*}
		\size{\ex{f(R,Z_{R},W) - f(R,Z^{(I)}_{R},W)}} \geq \gamma,
	\end{align*}
	for $R \la \zo^n$ and $I \la [n]$ that are sampled independently. 
	Our goal is to prove that 
	
	\begin{align}\label{eq:predic-lemma:UB}
		\pr{\Gc^f(I,Z_{-I}, W) = -Z_I} \leq O\paren{\frac{1}{\gamma^2 n}},
	\end{align}
	and 
	\begin{align}\label{eq:predic-lemma:LB}
		\pr{\Gc^f(I,Z_{-I}, W) = Z_I} \geq \Omega(\gamma) - O\paren{\frac{1}{\gamma^2 n}}.
	\end{align}

	Note that
	\begin{align}\label{eq:D}
		\text{For every random variable }D \in [-1,1]\text{ with }\size{\ex{D}} \geq \gamma >0: \quad \pr{\size{D} > \gamma/2} > \gamma/2,
	\end{align}
	as otherwise, $\size{\ex{D}} \leq \ex{\size{D}} \leq 1\cdot \frac{\gamma}{2} + \frac{\gamma}{2}(1-\frac{\gamma}{2}) < \gamma$.
	
	
	By applying \cref{eq:D} with $D = \eex{r \la \zo^n, f}{f(r,Z_{r},W) - f(r,Z^{(I)}_{r},W)}$, it holds that
	\begin{align}\label{eq:main-lemma:good}
		\ppr{(i,z,w) \la (I,Z,W)}{\size{\eex{r \la \zo^n, f}{f(r,z_{r},w) - f(r,z^{(i)}_{r},w)}} > \gamma/2} > \gamma/2.
	\end{align}
	On the other hand, for every fixing of $(z,w) \in \Supp(Z,W)$, we can apply \cref{lem:distance-I} with the function $f_{z,w}(r) = f(r,z_{r},w)$ and with $\alpha =\frac{\gamma}{16}$ to obtain that
	\begin{align*}
		\ppr{i\la I}{\: \size{\eex{r \la \zo^n \mid r_i = 0, \: f}{f(r,z_{r},w)} - \eex{r \la \zo^n \mid r_i = 1, \: f}{f(r,z_{r},w)} \:} \geq \frac{\gamma}{16}} \leq \frac{512}{n \gamma^2}.
	\end{align*}
	But since the above holds for every fixing of $(z,w)$, then in particular it holds that
	\begin{align}\label{eq:main-lemma:bad}
		\ppr{(i,z,w) \la (I,Z,W)}{\: \size{\eex{r \la \zo^n \mid r_i = 0, \: f}{f(r,z_{r},w)} - \eex{r \la \zo^n \mid r_i = 1, \: f}{f(r,z_{r},w)} \:} \geq \frac{\gamma}{16}} \leq \frac{512}{n \gamma^2}.
	\end{align}
	
	We next prove \cref{eq:predic-lemma:UB,eq:predic-lemma:LB} using \cref{eq:main-lemma:good,eq:main-lemma:bad}.
	
	In the following, for a triplet $t = (i,z,w) \in \Supp(I,Z,W)$, consider a random execution of $\Gc^f(i,z_{-i},w)$. For $x \in \set{-1,1,*}$, let $\mu^x_t$ be the value of $\mu^x$ in the execution, and let $M^x_t$ be the (random variable of the) value of $\tilde{\mu}^x$ in the execution. Note that by definition, it holds that $\mu_{i,z,w}^{z_i} = \eex{r \la \zo^n, f}{f(r,z_{r},w)}$ and $\mu_{i,z,w}^{-z_i} = \eex{r \la \zo^n, f}{f(r,z^{(i)}_{r},w)}$. Therefore, \cref{eq:main-lemma:good} is equivalent to 
	\begin{align}\label{eq:main-lemma:good2}
		\ppr{t=(i,z,w) \la (I,Z,W)}{\size{\mu_t^{z_i} - \mu_t^{-z_i}} > \gamma/2} > \gamma/2.
	\end{align}
	Furthermore, note that $\mu_{i,z,w}^* \eqdef \eex{r \la \zo^n \mid r_i = 0, \: f}{f(r,z^1_{r}, w)} = \eex{r \la \zo^n \mid r_i = 0, \: f}{f(r,z_{r}, w)}$ and that $\mu_{i,z,w}^{z_i} = \eex{r \la \zo^n \mid r_i = 1, \: f}{f(r,z^{z_i}_{r}, w)}$. Therefore, \cref{eq:main-lemma:bad} is equivalent to 
	\begin{align}\label{eq:main-lemma:bad2}
		\ppr{t = (i,z,w) \la (I,Z,W)}{\: \size{\mu_t^* - \mu_t^{z_i}}\geq \frac{\gamma}{16}}  \leq \frac{512}{n \gamma^2}.
	\end{align}
	
	We next prove the lemma using \cref{eq:main-lemma:good2,eq:main-lemma:bad2}.
	
	Note that by Hoeffding's inequality, for every $t = (i,z,w) \in \Supp(I,Z,W)$ and  $x \in \set{-1,1,*}$ it holds that $\pr{\size{M_t^x - \mu_t^x} \geq \frac{\gamma}{16}} \leq 2\cdot e^{-2 s \paren{\frac{\gamma}{16}}^2} \leq \frac1{3n}$, which yields that for every fixing of $t = (i,z,w) \in \Supp(I,Z,W)$, w.p.\ at least $1-1/n$ we have for all $x \in \set{-1,1,*}$ that $\size{M_t^x - \mu_t^x} < \frac{\gamma}{16}$ (denote this event by $E_t$).
	
	The proof of \cref{eq:predic-lemma:UB} holds by the following calculation:
	\begin{align*}
		\lefteqn{\pr{\Gc^f(I,Z_{-I},W) = -Z_I}}\\
		&= \eex{(i,z,w) \la (I,Z,W)}{\pr{\Gc^f(i,z_{-i},w) = -z_i}}\\
		&= \eex{t = (i,z,w) \la (I,Z,W)}{\pr{\set{\size{M_t^* - M_t^{z_i}} > \gamma/4} \land \set{\size{M_t^* - M_t^{-z_i}} < \gamma/4}}}\\
		&\leq \eex{t =(i,z,w) \la (I,Z,W)}{\pr{\size{M_t^* - M_t^{z_i}} > \gamma/4}}\\
		&\leq \eex{t = (i,z,w) \la (I,Z,W)}{\pr{\size{M_t^* - M_t^{z_i}} > \gamma/4 \mid E_t}} + \frac{1}{n}\\
		&\leq \ppr{t = (i,z,w) \la (I,Z,W)}{\size{\mu_t^* -  \mu_t^{z_i}} \geq \frac{\gamma}{16}} + \frac{1}{n}\\
		&\leq \frac{512}{n \gamma^2} + \frac1n,
	\end{align*}
	The second inequality holds since $\pr{\neg E_t} \leq 1/n$ for every $t$. The penultimate inequality holds since conditioned on $E_t$, it holds that $\size{M_t^* - \mu_t^*} \leq \frac{\gamma}{16}$ and $\size{M_t^{z_i} - \mu_t^{z_i}} \leq \frac{\gamma}{16}$, which implies that $\size{M_t^* - M_t^{z_i}} > \gamma/4 \: \implies \: \size{\mu_t^* -  \mu_t^{z_i}} \geq \frac{\gamma}{4} - 2\cdot \frac{\gamma}{16} > \frac{\gamma}{16}$. The last inequality holds by \cref{eq:main-lemma:bad2}.
	
	It is left to prove \cref{eq:predic-lemma:LB}. 
	Observe that \cref{eq:main-lemma:good2,eq:main-lemma:bad2} imply with probability at least $\gamma/2 - \frac{512}{n \gamma^2}$ over $t = (i,z,w)\in (I,Z,W)$, we have $\size{\mu_t^* -  \mu_t^{z_i}} \leq  \frac{\gamma}{16}$ and $\size{\mu_t^{z_i}- \mu_t^{-z_i}} \geq\frac{\gamma}{2}$. Therefore, conditioned on event $E_t$ (i.e., $\forall x \in \set{-1,1,*}: \: \size{M_t^{x} - \mu_t^{x}} \leq \frac{\gamma}{16}$), we have for such triplets $t = (i,z,w)$ that 
	\begin{align*}
		\size{M_t^* - M_t^{-z_i}} 
		& \geq \size{\mu_t^{z_i} - \mu_t^{-z_i}} - \size{\mu_t^{z_i} - \mu_t^*} - \size{M_t^*  - \mu_t^*} -  \size{M_t^{-z_i} - \mu_t^{-z_i}}\\
		&\geq \frac{\gamma}{2} - 3\cdot \frac{\gamma}{16}\\
		&> \gamma/4,
	\end{align*}
	and 
	\begin{align*}
		\size{M_t^* - M_t^{z_i}}
		&\leq  \size{M_t^* - \mu_t^*} + \size{\mu_t^* - \mu_t^{z_i}} + \size{\mu_t^{z_i} - M_t^{z_i}} \\
		&\leq 3\cdot \frac{\gamma}{16}\\
		&< \gamma/4.
	\end{align*}
	
	Thus, we conclude that
	\begin{align*}
		\lefteqn{\pr{\Gc^f(I,Z_{-I},W) = Z_I}}\\
		&= \eex{(i,z,w) \la (I,Z,W)}{\pr{\Gc^f(i,z_{-i},w) = z_i}}\\
		&= \eex{t = (i,z,w) \la (I,Z,W)}{\pr{\set{\size{M_t^* - M_t^{-z_i}} > \gamma/4} \land \set{\size{M_t^* - M_t^{z_i}} < \gamma/4}}}\\
		&\geq \paren{1 - \frac1{n}}\cdot \eex{t = (i,z,w) \la (I,Z,W)}{\pr{\set{\size{M_t^* - M_t^{-z_i}} > \gamma/4} \land \set{\size{M_t^* - M_t^{z_i}} < \gamma/4} \mid E_t}}\\
		&\geq \paren{1 - \frac1{n}}\cdot \paren{\gamma/2 - \frac{512}{n \gamma^2}}\\
		&\geq \gamma/4 - \frac{1024}{n \gamma^2},
	\end{align*}
	which proves \cref{eq:predic-lemma:LB}. The first inequality holds since $\pr{E_t} \geq 1-1/n$ for every $t = (i,z,w)$, and the second one holds by the observation above.
	
\end{proof}
}

\subsection{A's Security: Proving Property~\ref{item:privacy-of-X} of \cref{lemma:DPIP-to-AWEC}}\label{sec:proving-prop2}

Let $\Bc$ be an algorithm that violates the AWEC secrecy property~\ref{AWEC:item:B} of $\tilde{C} = ((V_A,O_A),(V_B,O_B))$ --- the channel of $\Pi^{C = ((X,U),(Y,V))}$ (\cref{protocol:DPIP-to-AWEC}). Namely,

\begin{align}\label{eq:violating-B}
	\pr{\size{\Bc(V_B) - O_A} \leq 1000 \ell \mid O_B=\bot} > \frac1{1000}.
\end{align}

Recall that $V_B = (Y,V,R,X_{R},I_1,\ldots,I_k, \tY)$ where $I_1,\ldots,I_k \la [n]$ are the indices that $\Bc$ chooses at Step~\ref{B_steps_in_abort}, and $\tY = (\tY_1,\ldots,\tY_n)$ where $\tY_i \la \oo$ for $i \in \set{I_1,\ldots,I_k}$ and otherwise $\tY_i = Y_i$. Furthermore, conditioned on $O_B=\bot$, recall that $O_A = \ip{X_{-R}, \tY_{-R}}$. Therefore, \cref{eq:violating-B} is equivalent to 
\begin{align}\label{eq:Bc-guarantee}
	%\pr{\size{\Bc(Y,V,R,X_{R},I_1,\ldots,I_k, \tY) - \ip{X_{-R}, \tY_{-R}}} \leq 1000 \ell } > \frac1{100}.
	\pr{\size{\Bc(V_B) - \ip{X_{-R}, \tY_{-R}}} \leq 1000 \ell } > \frac1{1000}.
\end{align}


In the following, define the random variable $H$ to be the first $m = \ceil{k/4}$ indices of $R^0 \cap \set{I_1,\ldots,I_k}$ for $R^0 = \set{i \colon R_i = 0}$, where we let $H = \emptyset$ if the size of the intersection is smaller than $m$.
Since $R \la \zo^n$, Hoeffding's inequality implies that $\pr{\size{R^0} \geq 0.4 n} \geq 1 - e^{-\Omega(n)}$. Since $I_1,\ldots,I_k \la [n]$ (independent of $R$), then again by Hoeffding's inequality we obtain that $\pr{\size{H} = \ceil{k/4} \mid \size{R^0} \geq 0.4 n} \geq 1 - e^{-\Omega(k)}$, which yields that
\begin{align*}
	\pr{H \neq \emptyset} = \pr{\size{H} = \ceil{k/4}} \geq 1 - e^{-\Omega(k)} - e^{-\Omega(n)} \geq 1-\frac1{10000}.
\end{align*}
Therefore, by the union bound,
\begin{align}\label{eq:good-and-not-empty-H}
	\lefteqn{\pr{\set{\size{\Bc(V_B) - \ip{X_{-R}, \tY_{-R}}} \leq 1000 \ell} \land \set{H \neq \emptyset}}}\\
	&= 1- \pr{\set{\size{\Bc(V_B) - \ip{X_{-R}, \tY_{-R}}} > 1000 \ell} \lor \set{H = \emptyset}}\nonumber\\
	&\geq 1- \pr{\set{\size{\Bc(V_B) - \ip{X_{-R}, \tY_{-R}}} > 1000 \ell}} - \pr{ \set{H = \emptyset}}\nonumber\\
    &\geq 1- \paren{1- \frac1{1000}} - \frac1{10000} \nonumber\\
     &\geq \frac1{2000}.\nonumber
\end{align}
%\begin{align}\label{eq:good-and-not-empty-H}
%	\lefteqn{\pr{\set{\size{\Bc(V_B) - \ip{X_{-R}, \tY_{-R}}} \leq 1000 \ell} \land \set{H \neq \emptyset}}}\\
%	&= \pr{H \neq \emptyset} \cdot \pr{\size{\Bc(V_B) - \ip{X_{-R}, \tY_{-R}}} \leq 1000 \ell \mid H \neq \emptyset }\nonumber\\
%	&\geq \pr{H \neq \emptyset}\cdot \paren{\pr{\size{\Bc(V_B) - \ip{X_{-R}, \tY_{-R}}} \leq 1000 \ell} - \pr{H = \emptyset}}\nonumber\\
%	&\geq \paren{1-\frac1{10000}}\cdot \paren{\frac1{1000} - \frac1{10000}}\nonumber\\
%	&> \frac1{2000}.\nonumber
%\end{align}


In the following, let $d$ be the number of random coins that $\Bc$ uses, and for $\psi \in \zo^d$ let $\Bc_{\psi}$ be algorithm $\Bc$ when fixing its random coins to $\psi$.
Let $\Psi \la \zo^d$,  $Z = X_H$,  $T' = (Y,V,R,I_1,\ldots,I_k, H,\tY_{-\cH}, X_{-\cH})$,  $T= (\Psi, T')$ and $S = \tY_H$. Note  that conditioned on $H \neq \perp$, $S$ is a uniformly random string in $\oo^m$, independent of $Z$ and $T$, and note that $V_B$ is a deterministic function of $(\tY_{H}, T)$ (because $X_R$, which is part of $V_B$, is a sub-string of $X_{-\cH}$).
\cref{eq:good-and-not-empty-H} yields that w.p. $1/2000$ over $z \la Z$,  $t = (\psi, t'=(y,v,r,i_1,\ldots,i_k,\cH,x_{-\cH},\ty_{-\cH})) \la T$ and $s \la \oo^m$,
the following holds for $\bar{\cH} =  \set{i \in [n] \colon r_i = 0} \setminus \cH\:$:
\begin{align*}
	\size{\Bc_{\psi}(s,t') - \ip{x_{\bar{\cH}}, \ty_{\bar{\cH}}} - \ip{z, s}}\leq 1000 \ell.
\end{align*}

By denoting $f(s,t=(\psi,t')) = \Bc_{\psi}(s,t') + \ip{x_{\bar{\cH}}, \ty_{\bar{\cH}}}$,\footnote{Note that $f(s,t)$ is well-defined because $t$ contains $x_{\bar{\cH}}$ and $\ty_{\bar{\cH}}$ (sub-strings of $x_{-\cH}$ and $\ty_{-\cH}$, respectively).} the above observation is equivalent to

\begin{align}\label{eq:our-good-f}
	\ppr{(z,t) \la (Z,T), \: s \la \oo^m}{\size{f(s,t) - \ip{z, s}} \leq 1000 \ell} \geq \frac1{2000}.
\end{align}

In the following, let $\cD$ be the joint distribution of $(Z,T)$, which is equivalent to the output of $\GenView^{C}()$ defined below in \cref{alg:GenView}.

\begin{algorithm}[$\GenRand$]\label{alg:GenRand}
	~
	\begin{enumerate}
            \item Sample $\psi \la \zo^d$.
		\item Sample $r \la \zo^n$ and $i_1,\ldots,i_k \la [n]$.
		\item Let $\cH$ be the first $m=\ceil{k/4}$ indices of $\set{i \in [n] \colon r_i = 0} \cap  \set{i_1,\ldots,i_k}$, where $\cH = \emptyset$ if the intersection size is less than $m$.
		\item Let $\bar{\cH} = \set{i \in [n] \colon r_i = 0} \setminus \cH$.
		\item Sample $\ty_{\bar{\cH}} \la \oo^{\size{\bar{\cH}}}$.
		\item Output $(\psi,r,i_1,\ldots,i_k,\cH,\ty_{\bar{\cH}})$.
	\end{enumerate}
\end{algorithm}



\begin{algorithm}[$\GenView$]\label{alg:GenView}
	\item Oracle: A channel $C = ((X,U),(Y,V))$.
	\item Operation:~
	\begin{enumerate}
		\item Sample $((x,u),(y,v)) \la C$.
		\item Sample $(\psi,r,i_1,\ldots,i_k,\cH,\ty_{\bar{\cH}}) \la \GenRand()$ (\cref{alg:GenRand}).
		\item Output $z = x_{\cH}$ and $t = (\psi, t')$ for $t'=(y,v,r,i_1,\ldots,i_k,\cH,\ty_{-\cH},x_{-\cH})$, where $\ty_i = y_i$ for $i \in [n]\setminus (\cH \cup \bar{\cH})$.
	\end{enumerate}
\end{algorithm}



We now can use the following reconstruction result from \cite{HaitnerMST22}:


\begin{fact}[Follows by Theorem 4.6 in \cite{HaitnerMST22}]\label{fact:prev-rec}
	There exists constants $\eta_1,\eta_2 > 0$ and a \ppt algorithm $\Dist$ such that the following holds for large enough $m \in \bbN$: Let $\eps \geq 0$ and $a \geq \log m$, and let $\cD$ be a distribution over $\oo^m \times \Sigma^*$. Then for every function $f$ that satisfies 
	\begin{align*}
		\ppr{(z,t) \la \cD, \: s \la \oo^m}{f(s,t) - \ip{z,s} \leq a} \geq e^{\eta_1 \cdot \eps}\cdot \eta_2 \cdot a/\sqrt{m},
	\end{align*}
	it holds that
	\begin{align*}
		\ppr{(z,t) \la \cD, \: j \la [m]}{\Dist^{\cD,f}(j, z, t) = 1} > e^{\eps}\cdot \ppr{(z,t) \la \cD, \: j \la [m]}{\Dist^{\cD,f}(j ,z^{(j)}, t) = 1} + \frac1m,
	\end{align*}
	where $z^{(j)} = (z_1,\ldots,z_{j-1},-z_j,z_{j+1},\ldots,z_m)$.\footnote{Theorem 4.6 in \cite{HaitnerMST22} actually considered a harder setting where $z$ is the coordinate-wise product of two vectors $x,y \in \oo^n$, and $f$ only guarantees accuracy when in addition to $s$ and $t$, it also gets $x_{s} = \set{x_i \colon s_i = 1}$ and $y_{-s} = \set{y_i \colon s_i = -1}$ as inputs.}
	%\Enote{explain the differences from Theorem 4.6 in \cite{HaitnerMST22}.}%\Nnote{Maybe for after the deadline: I guess we don't really need this theorem and can use the proof of the easy case. Maybe we want to write the simple proof}
\end{fact}



%Note that the output of the following $\GenView^{C}()$ (\cref{alg:GenView}) is distributed the same as the joint distribution of $(Z,T)$. 
Now, we would like to apply \cref{fact:prev-rec} with $\cD = \GenView^C()$ and $a = 1000 \ell$. To do that, \cref{fact:prev-rec} yields that we need to choose $k$ such that $\frac{e^{\eta_1 \cdot \eps}\cdot \eta_2  \cdot 1000\ell}{\ceil{k/4}} \leq \frac1{2000}$, which holds by choosing $k = \floor{e^{\lambda_1 \eps}\cdot \lambda_2 \cdot \ell^2}$ with 
\begin{align}\label{eq:lamdas}
	\lambda_1 = \eta_1 \text{ and }\lambda_2 = 10^7 \eta_2 + 1,
\end{align}
where $\eta_1,\eta_2$ are the constants from \cref{fact:prev-rec}.

We deduce from \cref{fact:prev-rec,eq:our-good-f} that 

\begin{align}\label{eq:Dist-gap}
	\ppr{(z,t) \la \cD, \: j \la [m]}{\Dist^{\cD,f}(j, z, t) = 1} > e^{\eps}\cdot \ppr{(z,t) \la \cD, \: j \la [m]}{\Dist^{\cD,f}(j ,z^{(j)}, t) = 1} + \frac1m.
\end{align}


We now ready to describe our algorithm $\Bct$ that satisfies Property~\ref{item:privacy-of-X} of \cref{lemma:DPIP-to-AWEC}.


\begin{algorithm}[Algorithm $\Bct$]\label{alg:Bct}
	\item Inputs: $i \in [n]$, $x_{-i} = (x_1,\ldots,x_{i-1},x_{i+1},\ldots,x_n) \in \oo^{n-1}$, $y \in \oo^n$ and $v \in \zo^*$.
	\item Oracles: A channel $C = ((X,U),(Y,V))$ and an algorithm $\Bc$.
	\item Operation:~
	\begin{enumerate}
		\item Sample $(\psi, r,i_1,\ldots,i_k,\cH,\ty_{\bar{\cH}}) \la \GenRand()$ (\cref{alg:GenRand}).
		\item If $i \notin \cH$, output $\bot$.
		\item Otherwise:
		\begin{enumerate}
			\item Let $t = (y,v,r,i_1,\ldots,i_k,\cH,\ty_{-\cH},x_{-\cH})$ where $\ty_{i'} = y_{i'}$ for $i' \in [n]\setminus (\cH \cup \bar{\cH})$.
			\item For $b \in \oo$: Let $x^b = (x_1,\ldots,x_{i-1},b, x_{i+1},\ldots,x_n)$ and $z^b = x^b_{\cH} \in \oo^m$, where $m = \ceil{k/4}$.
			\item Let $j \in [m]$ be the index such that $z^1_j \neq z^{-1}_j$.
			\item Sample $b \la \oo$ and $o \la \Dist^{\GenView^C, f}(j,z^b,t)$ where $\GenView^C$ is \cref{alg:GenView} with oracle access to $C$, and $f(s,t = (\psi,t')) = \Bc_{\psi}(s,t') + \ip{x_{\bar{\cH}}, \ty_{\bar{\cH}}}$. 
			\color{gray}{\# Recall that $x_{\bar{\cH}}$ is a sub-string of $x_{-\cH}$ (part of $t$) and that $ \ty_{\bar{\cH}}$ is a sub-string of $\ty_{-\cH}$ (also part of $t$).}
			\color{black}{\item If $o = 1$, output $b$. Otherwise, output $\bot$.}
		\end{enumerate}
	\end{enumerate}
\end{algorithm}

\begin{proof}[Proof of Property~\ref{item:privacy-of-X} of \cref{lemma:DPIP-to-AWEC} using \cref{alg:Bct}]
	
In the following, let $\cD = \GenView^C()$ and $((X,U),(Y,V)) \la C$. Consider a random execution of $\Bct^{\Bc, C}(I,\: X_{-I}, \: Y, \: V)$ for $I \la [n]$, and let $T, H, B, O, Z, J$ be the values of $\:\: t, h,\cH,o, z, j$ in the execution. 
Let $p = \ppr{(z,t) \la \cD, \: j \la [m]}{\Dist^{\cD,f}(j ,z^{(j)}, t) = 1}$, and note that conditioned on $I \in H$, $J$ is distributed uniformly over $[m]$. Therefore,
$$\pr{\Dist^{\cD,f}(J, Z^{(J)}, T) = 1 \mid I \in H} = p$$ and $$\pr{\Dist^{\cD,f}(J, Z, T) = 1 \mid I \in H} = \ppr{(z,t) \la \cD, \: j \la [m]}{\Dist^{\cD,f}(j ,z, t) = 1} \geq e^\eps p + \frac1m,$$ where the inequality holds by \cref{eq:Dist-gap}.
Thus, the following holds for $X^*_i = \Bct^{\Bc, C}(i,\: X_{-i}, \: Y, \: V)$:


\begin{align*}
	\eex{i \la [n]}{\pr{X^*_i = -X_i}}
	&= \pr{X_I^* = -X_I}\\
	&=\pr{\set{I \in H} \land \set{B = -X_I} \land \set{O=1}}\\
	&= \paren{\frac{m}{n}\cdot \pr{H \neq \emptyset}} \cdot \frac12 \cdot  \pr{O=1 \mid \set{I \in H}\land \set{B = -X_I}}\\
	&= \frac{m}{2n} \cdot \pr{H \neq \emptyset} \cdot \pr{\Dist^{\cD,f}(J, Z^{(J)}, T) = 1 \mid I \in H}\\
	&= \frac{m}{2n} \cdot \pr{H \neq \emptyset} \cdot p,
\end{align*}
 and 

\begin{align*}
	\eex{i \la [n]}{\pr{X^*_i = X_i}}
	&=\pr{\set{I \in H} \land \set{B = X_I} \land \set{O=1}}\\
	&= \paren{\frac{m}{n}\cdot \pr{H \neq \emptyset}} \cdot \frac12 \cdot  \pr{O=1 \mid \set{I \in H}\land \set{B = X_I}}\\
	&= \frac{m}{2n} \cdot \pr{H \neq \emptyset} \cdot\pr{\Dist^{\cD,f}(J, Z, T) = 1 \mid I \in H}\\
	&> \frac{m}{2n}  \cdot \pr{H \neq \emptyset} \cdot (e^{\eps} p + 1/m)\\
	&= e^{\eps} \cdot \paren{\frac{m}{2n}  \cdot \pr{H \neq \emptyset} \cdot p} + \frac{1}{2n}\cdot \pr{H \neq \emptyset},\\
	&>  e^{\eps} \cdot \eex{i \la [n]}{\pr{X^*_i = -X_i}} + \delta,
\end{align*}
which concludes the proof. The last inequality holds since $\pr{H \neq \emptyset} \geq 1-\frac1{10000}$ and $\delta \leq \frac1{3n}$.

\end{proof}

