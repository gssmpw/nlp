\section{Related Works}
\label{sec:related-works}

\textbf{Computational differential privacy.} Computational differential privacy ($\CDP$) can be defined using two primary approaches.
The more flexible and widely used approach is the \emph{indistinguishability-based} definition, which limits the distinguished $g$ in \cref{def:intro:DP}, to those that are computationally efficient. The second approach, known as the \emph{simulation-based} definition, requires that the output distribution of the mechanism $f$ be computationally indistinguishable from that of an (information-theoretic) differentially private mechanism. Various relationships between these and other privacy definitions have been established in ____ (a more updated survey is provided in ____). We remark that our result holds for the weaker definition of indistinguishability-based $\CDP$, which makes it stronger. As a corollary, we also get the equivalence of the power of the definitions for the inner-product task we consider (in our accuracy regime).\Nnote{(maybe put it as n open questions for other two-party tasks?}

\textbf{\CDP in the centralized model.} In the single-party scenario (\ie the centralized model), computational and information-theoretic differential privacy appear to be more closely aligned. Specifically, ____ demonstrated that a broad class of $\CDP$ mechanisms can be converted into an information-theoretic $\DP$ mechanism. On the other hand, ____ showed that under certain non-standard and very-strong cryptographic assumptions, there exist somewhat contrived tasks that can be efficiently solved with $\CDP$, but remain infeasible (____) or impossible (____) under information-theoretic $\DP$. It still remains open whether such separations exist under more standard cryptographic assumptions and for more natural tasks.

\textbf{\CDP in the local model.} At the other end of the spectrum, the \emph{local model} is highly relevant in practical applications. In this setting, each of the (typically many) participants holds a single data element. Protocols achieving information-theoretic $\DP$ in this model often rely on randomized response, which has been shown to be optimal for counting functions, including inner product, as proven by ____. In contrast, local $\CDP$ protocols can leverage secure multiparty computation to simulate any efficient single-party mechanism, thereby demonstrating a fundamental gap between the power of $\CDP$ and information-theoretic $\DP$.\Nnote{We should be more careful here - there are MPC protocol for 3 players with information-theoretic security} We remark that there exist other approaches to bridge this by relaxing the distributed model, e.g., using a trusted shuffler ____ (see ____ for a survey on this model), or partially trusted servers that enable more accurate estimations using weaker and more practical cryptographic tools than secure multi-party computation (e.g., ____).

\textbf{Two-Party \CDP.} In the two-party (or small-party) setting, the complexity of $\CDP$ protocols is much less clear. Prior works have maintly focused on Boolean functionalities, where each party holds a single sensitive bit, and the objective is to compute a Boolean function over these bits while preserving privacy (\eg XOR). ____ established that, for any non-trivial Boolean functionality, there is a fundamental gap in accuracy between what can be achieved in the centralized and distributed settings. Moreover, any $\CDP$ protocol that surpasses this accuracy gap would imply the existence of one-way functions. Later, ____ demonstrated that an accurate enough $\CDP$ protocol for the XOR function would inherently imply an oblivious transfer protocol. Building on this, ____ proved that any meaningful $\eps$-\CDP two-party protocol for XOR necessarily implies an (infinitely-often) key agreement protocol. ____ refined and extended the results of ____, showing that any non-trivial $\CDP$ protocol for XOR implies oblivious transfer.

Beyond Boolean functionalities, the complexity of $\CDP$ protocols for more general tasks such as computing low-sensitivity many-bit functions like the inner product remains largely unexplored. The only exceptions are the work of ____, who applied a generic reduction to the impossibility result of ____, concluding that no accurate $\CDP$ protocol for inner product can exist in the \textit{random oracle model}, and more recently, the work of ____ who showed that any non-trivial $\CDP$ inner-product protocol implies a key-agreement protocol. But none of these results close the gap to $\OT$, and thus, our result provide the first tight characterization (with respect to assumptions) of a natural, non-boolean functionality.