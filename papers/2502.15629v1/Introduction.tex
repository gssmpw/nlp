\section{Introduction}


Differential privacy \cite{DMNS06} is a mathematical privacy guarantee designed to facilitate statistical analyses of databases while ensuring strong guarantees against the leakage of individual-level information. 
Formally, 
\begin{definition}[Differential Privacy]\label{def:intro:DP}
	A randomized function (``mechanism'') $\Mc \colon \cX^n \rightarrow \cY$ is $(\eps,\delta)$-differentially private (in short, $(\eps,\delta)$-$\DP$) if for every $x,x' \in \cX^n$ that differ on only one entry, and any algorithm $\Dc\colon \cY \rightarrow \zo$ (``distinguisher''), it holds that
	\begin{align}\label{eq:intro:DP}
		\pr{\Dc(\Mc(x)) = 1} \leq e^{\eps}\cdot \pr{\Dc(\Mc(x')) = 1} + \delta.
	\end{align}
        When $\delta = 0$, we omit it from the notation.
\end{definition}

Much of the research in differential privacy operates within the central model. This model assumes that data owners are willing to share their data with a central server, which applies a differentially private mechanism and releases the results to the (potentially adversarial) world. However, in many scenarios, there is no such trusted server, and the data is distributed among several parties who wish to perform a joint computation while preserving each party's data from the eyes of the others.
But constructing useful multiparty DP protocols is harder than constructing them in the centralized model because now we need to develop a $\DP$ \emph{interactive protocol}, which means that not only the output should be a differentially private function of the data (as in the centralized model), but also each party's view in the interaction should be a differentially private function of the other parties' data (\cite{DN04,BNO08}).

Perhaps the most fundamental and natural example that best illustrates the hardness of distributed $\DP$ is the inner-product functionality of two binary datasets. 
Namely, two parties $\Ac$ and $\Bc$ hold datasets $x \in \oo^n$ and $y \in \oo^n$ (respectively), and wish to estimate the inner product over the integers $\ip{x,y} = \sum_{i=1}^n x_i \cdot y_i$ (the correlation between the datasets) using a $\DP$ protocol.
We know that there exists a simple randomized response $\eps$-$\DP$ protocol that estimates the inner-product up to additive error $O_{\eps}(\sqrt{n})$ (\cite{McGregorMPRTV10,MPRV09}), and this is essentially the best error rate we can achieve. This was shown by \cite{McGregorMPRTV10,HaitnerMST22} who showed that any $\eps$-$\DP$ protocol for the inner-product, even when allowing any $\delta \in o(1/n)$, must have an additive error of $\Omega_{\eps}(\sqrt{n})$. Note that this is in complete contrast to the centralized model, where the simple $\eps$-$\DP$ Laplace mechanism $\Mc(x,y) = \ip{x,y} + \Lap(2/\eps)$ has only additive error $O_{\eps}(1)$. 

To overcome such accuracy gaps between the models, \cite{BNO08,MPRV09} considered protocols that only guarantee a computational analog of differential privacy (denote by $\CDP$), which is a relaxation of \cref{def:intro:DP} where \cref{eq:intro:DP} only holds for computationally efficient (\ppt) algorithms $\Dc$, and not necessarily for any algorithm $\Dc$. This relaxation is very powerful. For example, if we assume the existence of the cryptographic primitive \emph{oblivious transfer} (in short, $\OT$, \cite{Rabin81}), then in any distributed setting (even for more than two parties), we can achieve the accuracy of a centralize $\DP$ mechanism by emulating it using \emph{secure multiparty computation} (\cite{BNO08,DKMMN06}). In particular, using $\OT$, two parties can securely emulate the (centralized) Laplace mechanism to achieve an $\eps$-$\CDP$ protocol with additive error $O_{\eps}(1)$. This leads to the following fundamental question:

\begin{question}
	Can we construct a two-party $\CDP$ protocol that estimates the inner product with a small additive error using a cryptographic primitive that is weaker than $\OT$? 
    %Can we construct a two-party $\CDP$ protocol that estimates the inner product with a constant, or even sub-polynomial additive error, using a cryptographic primitive that is weaker than $\OT$? 
\end{question}



\cite{HaitnerMST22} only made partial progress on this matter, showing that any $\CDP$ protocol with non-trivial accuracy guarantee of $o_{\eps}(\sqrt{n})$ must use a cryptographic primitive that is at least as strong as \emph{key agreement} (in short, $\KA$).  But although $\KA$ and $\OT$ are both public-key cryptography assumptions, it is known that $\KA$ is strictly weaker than $\OT$ (at least under black-box reductions \cite{GertnerKMRV00}).
In fact, even for $\CDP$ protocols that estimate the inner-product with very high accuracy of $O_{\eps}(1)$, it was previously unknown whether $\OT$ is indeed necessary.


\subsection{Our Result}

In this work, we prove the first equivalence between $\OT$ and $\CDP$ two-party protocols that estimate the inner-product with small additive error, and in particular, additive error up to $O_{\eps}(n^{1/6})$.

\begin{theorem}[Main result, informal]\label{thm:intro:main}
    An $\paren{\eps,\frac1{3n}}$-$\CDP$ two-party protocol that estimates the inner product over $\oo^n \times \oo^n$ up to an additive error $n^{1/6}/(c\cdot e^{c \eps})$ with probability $0.999$ (for some universal constant $c > 0$), can be used to construct an $\OT$ protocol.\footnote{We remark that \cref{thm:intro:main} is proven via a uniform and black-box reduction. Thus, although the security of $\CDP$ and $\OT$ is commonly defined with respect to non-uniform adversaries, \cref{thm:intro:main} also implies a reduction from $\OT$ that is secure against uniform adversaries to such mildly accurate inner-product protocols that are $\CDP$ with respect to uniform distinguishers.}
\end{theorem} 

In particular, \cref{thm:intro:main} implies that without assuming the existence of $\OT$, it is impossible to construct a $\CDP$ protocol for the inner-product with a constant, or even sub-polynomial, additive error.

As a direct corollary of \cref{thm:intro:main}, we obtain that a protocol with a small polynomial additive error is equivalent to a protocol with a constant error. That is because using such a mildly accurate protocol, we can construct by \cref{thm:intro:main} an oblivious transfer, and then use it to perform a secure two-party computation of the Laplace mechanism.

\begin{corollary}
There exists a universal constant $c>0$ such that the following holds. 
    Assume that an $\paren{\eps,\frac1{3n}}$-$\CDP$ two-party protocol that estimates the inner product over $\oo^n \times \oo^n$ up to an additive error $n^{1/6}/(c\cdot e^{c \eps})$ with probability $0.999$ exists. Then there exists an $\paren{\eps,\negl(n)}$-$\CDP$ protocol that estimates the inner product up to an additive error $20/\eps$ with probability $0.999$.
\end{corollary}


We remark that \cref{thm:intro:main} is the first equivalence result between $\OT$ to any $\CDP$ protocol that estimates a natural, non-Boolean functionality.
Prior works only provide such equivalences for $\CDP$ protocols that estimate the (Boolean) XOR functionality (\cite{GKMPS16,HMSS19}), which is less interesting in the context of $\DP$, as even in the centralized model, any $\DP$ algorithm for estimating XOR must incur an error close to one-half.
In contrast, the inner product has a significantly larger gap between the accuracy achievable in two-party $\DP$ and in $\CDP$.
%Recall that with $\OT$, we can construct a $\CDP$ protocol that estimates the inner product with additive error $O_{\eps}(1)$. By our \cref{thm:intro:main}, even the much relaxed task of estimating the error up to $O_{\eps}(n^{1/6})$ still requires $\OT$.

A proof overview of \cref{thm:intro:main} is given in \cref{sec:overview}.

\paragraph{Comparison with \cite{HaitnerMST22}.}
We remark that our result is not strictly stronger than \cite{HaitnerMST22}. First, \cite{HaitnerMST22} covered the whole non-trivial accuracy regime up to $O_{\eps}(\sqrt{n})$ while our result is only limited to $O_{\eps}(n^{1/6})$. Second, \cite{HaitnerMST22}'s result holds even for low confidence regimes, while \cref{thm:intro:main} requires high accuracy confidence of $0.999$. But in the natural high-accuracy high-confidence regimes, our result has several advantages beyond the necessity of a stronger primitive. First, \cite{HaitnerMST22}'s reduction only works for $\CDP$ two-party protocols that provide the inner-product estimation within the transcript, while our reduction in \cref{thm:intro:main} just assumes that (at least) one of the parties sees the estimation. Second, and perhaps surprisingly, the proof of \cref{thm:intro:main} is conceptually simpler, even though it provides reduction from a stronger primitive. We provide more details in \cref{sec:overview} (see \cref{remark:OTvsKA}). 

\remove{
\Nnote{the paragraph below is hard to follow, need to think what to do with it - maybe move it to be a high-level remark after the protocol is the proof overview? I'm not sure we want to say the words SV, condenser, extractor, ...}
Very roughly, in the secrecy analysis of \cite{HaitnerMST22}'s $\KA$ protocol, they had to prove (in a constructive way) that the inner product of a strong Santha Vazirani (SV) source \cite{McGregorMPRTV10} with a uniformly random seed is a good \emph{condenser}, even when the seed and the source are \emph{dependent}. When the seed and the source are \emph{independent}, \cite{McGregorMPRTV10} already proved that the inner product is not only a good condenser for strong SV sources, but also a good \emph{extractor} for (standard) SV sources \cite{SV87}, which apart of their non-constructive technique and the independency assumption, is a stronger result. However, \cite{HaitnerMST22}'s dependency between the source and the seed resulted in a significant technical challenge, which made the proof conceptually harder. In contrast, in our $\OT$ security analysis, since we only provide secrecy guarantees from the point of views of the parties (and not from the point of view of an external observer that only sees the transcript), we bypass this dependency issue. See more details in \cref{sec:overview}.
}



%This result implies that such a protocol cannot be constructed in a fully black-box manner from a symmetric-key primitive.


\subsection{Open Questions}

In this paper, we show that a $\paren{\eps,\frac1{3n}}$-$\CDP$ protocol that estimates the inner product over $\oo^n \times \oo^n$ up to an additive error $O_{\eps}(n^{1/6})$ is equivalent to oblivious transfer ($\OT$). For the inner product functionality, it still left open to close the gap up to any non-trivial accuracy of $O_{\eps}(\sqrt{n})$. \cite{HaitnerMST22} closed this gap w.r.t.\ a weaker notion of $\CDP$ \emph{against external observer}\footnote{A protocol is $\CDP$ against external observer, if the transcript of the execution is a $\CDP$ function of the input datasets, but not necessarily the views of the parties as in the standard notion.} by showing that key agreement ($\KA$) is necessary and sufficient for any non-trivial accuracy guarantee, which in particular implies that $\KA$ is also necessary for the standard notion of $\CDP$. But since $\KA$ is strictly weaker than $\OT$ (under block-box reductions), we still do not know what is the right answer for additive errors in the range between $n^{1/6}$ and $\sqrt{n}$.

Beyond characterizing the inner-product functionality, the main challenge is to extend this understanding to other $\CDP$ distributed computations. For some functionalities, \eg Hamming distance, we have a simple reduction to the inner-product functionality. But finding a more general characterization that captures more (or even all) functionalities, remains open. 



\subsection{Paper Organization}
In \cref{sec:overview}, we give a high-level proof of \cref{thm:intro:main}\remove{, and in \cref{sec:related-works} we discuss additional related works}. Notations, definitions and general statements used throughout the paper are given in \cref{sec:preliminaries}. Our main formal theorems are stated in \cref{sec:main_theorems}, and the oblivious transfer protocol together with its
security proof are given in
\cref{sec:DPIP_to_WAEC}. 
In \cref{sec:related-works} we discuss additional related works.
The proof of our main result relies on a technical tool that is proven in \cref{sec:AWEC-to-WEC}. Other missing proofs appear in \cref{sec:missing-proofs}.


