\def \IsDraft{} % set for draft mode


%%%%%%%%%%%%%%%%%%%%%%%%%%%%%%%%%%%%%%%%%%%%%%%%%%%%%%%%%%%%%%%%%%%%%%%%%%%%%%%%
% FLAGS
%

\newif\ifdraft
%\drafttrue
\draftfalse

\newif\ifinc
%\inctrue
\incfalse

\newif\ifans
%\anstrue
\ansfalse

\newif\ifsample
%\sampletrue
\samplefalse

\newif\iffull
\fulltrue 
%\fullfalse 
%%%%%%%%%%%%%%%%%%%%%%%%%%%%%%%%%%%%%%%%%%%%%%%%%%%%%%%%%%%%%%%%%%%%%%%%%%%%%%%%
 

\providecommand{\remove}[1]{}




\iffull
	\documentclass[11pt,letterpaper]{article}
\else
	\documentclass[letter,envcountsame,envcountsect]{llncs}
    
\fi




%%%%%%%%%%%%
% Packages %
%%%%%%%%%%%%

\iffull
 	\usepackage{fullpage}
 	\usepackage[margin=4mm, small, labelfont=bf]{caption}
	\ifdraft 
		\usepackage[draft,notes=true,later=false]{dtrt} 
	\else
		\usepackage[notes=true,later=false]{dtrt} 
    \fi
 
\usepackage[natbib=true,backend=bibtex,backref=true,style=alphabetic,maxnames=2,maxalphanames=2,maxbibnames=99, sorting=anyt]{biblatex}
\else
\usepackage[notes=xxx,llncssubsub]{dtrt}

\usepackage[style=alphabetic,firstinits,backend=bibtex,maxalphanames=2,minalphanames=2,maxbibnames=6,natbib=true]{biblatex}


\fi

\bibliography{crypto}

\iffull
\usepackage{titlesec}
\usepackage{amsthm}
\else
\let\proof\relax
\let\endproof\relax
\usepackage{amsthm}
\fi


\remove{
\usepackage[
pagebackref,backref=true,
hyperfootnotes=false,
colorlinks=true,
urlcolor=externallinkcolor,
linkcolor=internallinkcolor,
filecolor=externallinkcolor,
citecolor=internallinkcolor,
breaklinks=true,
pdfstartview=FitH,
pdfpagelayout=OneColumn]{hyperref}
}

%\usepackage{aliascnt}


%\usepackage[backend=bibtex,style=ieee-alphabetic,natbib=true,backref=true]{biblatex}
%\addbibresource{crypto.bib}


%\usepackage[numbers]{natbib} % \citet{foo} -> Foo et al. P




\usepackage{mathtools}
\usepackage{bbm}
%\usepackage{xcolor}
\usepackage{amsthm,amsfonts,amsmath,amssymb,color,amsthm,boxedminipage,url,xparse}
\definecolor{internallinkcolor}{rgb}{0,.5,0}
\usepackage{amsthm}
% in order to remove warnings such as "destination with the same identifier (name{corollary.1.3}) has been already used, duplicate ignored" the order of packages should be hyperref, amsthm, cleveref
%\usepackage{enumerate,paralist}
%\usepackage{enumitem}

\usepackage{xspace}
\usepackage{nicefrac}

\usepackage{cleveref}


\usepackage{amssymb}
\usepackage{amsmath}
\usepackage{bbm}
\usepackage{paralist}
\usepackage{enumitem}

\setlist[itemize]{leftmargin=*}
\setlist[enumerate]{leftmargin=*}

\usepackage{mathtools}
\usepackage{paralist}


%%%%%%%%%%
% Macros %
 %%%%%%%%%%




%
\setlength\unitlength{1mm}
\newcommand{\twodots}{\mathinner {\ldotp \ldotp}}
% bb font symbols
\newcommand{\Rho}{\mathrm{P}}
\newcommand{\Tau}{\mathrm{T}}

\newfont{\bbb}{msbm10 scaled 700}
\newcommand{\CCC}{\mbox{\bbb C}}

\newfont{\bb}{msbm10 scaled 1100}
\newcommand{\CC}{\mbox{\bb C}}
\newcommand{\PP}{\mbox{\bb P}}
\newcommand{\RR}{\mbox{\bb R}}
\newcommand{\QQ}{\mbox{\bb Q}}
\newcommand{\ZZ}{\mbox{\bb Z}}
\newcommand{\FF}{\mbox{\bb F}}
\newcommand{\GG}{\mbox{\bb G}}
\newcommand{\EE}{\mbox{\bb E}}
\newcommand{\NN}{\mbox{\bb N}}
\newcommand{\KK}{\mbox{\bb K}}
\newcommand{\HH}{\mbox{\bb H}}
\newcommand{\SSS}{\mbox{\bb S}}
\newcommand{\UU}{\mbox{\bb U}}
\newcommand{\VV}{\mbox{\bb V}}


\newcommand{\yy}{\mathbbm{y}}
\newcommand{\xx}{\mathbbm{x}}
\newcommand{\zz}{\mathbbm{z}}
\newcommand{\sss}{\mathbbm{s}}
\newcommand{\rr}{\mathbbm{r}}
\newcommand{\pp}{\mathbbm{p}}
\newcommand{\qq}{\mathbbm{q}}
\newcommand{\ww}{\mathbbm{w}}
\newcommand{\hh}{\mathbbm{h}}
\newcommand{\vvv}{\mathbbm{v}}

% Vectors

\newcommand{\av}{{\bf a}}
\newcommand{\bv}{{\bf b}}
\newcommand{\cv}{{\bf c}}
\newcommand{\dv}{{\bf d}}
\newcommand{\ev}{{\bf e}}
\newcommand{\fv}{{\bf f}}
\newcommand{\gv}{{\bf g}}
\newcommand{\hv}{{\bf h}}
\newcommand{\iv}{{\bf i}}
\newcommand{\jv}{{\bf j}}
\newcommand{\kv}{{\bf k}}
\newcommand{\lv}{{\bf l}}
\newcommand{\mv}{{\bf m}}
\newcommand{\nv}{{\bf n}}
\newcommand{\ov}{{\bf o}}
\newcommand{\pv}{{\bf p}}
\newcommand{\qv}{{\bf q}}
\newcommand{\rv}{{\bf r}}
\newcommand{\sv}{{\bf s}}
\newcommand{\tv}{{\bf t}}
\newcommand{\uv}{{\bf u}}
\newcommand{\wv}{{\bf w}}
\newcommand{\vv}{{\bf v}}
\newcommand{\xv}{{\bf x}}
\newcommand{\yv}{{\bf y}}
\newcommand{\zv}{{\bf z}}
\newcommand{\zerov}{{\bf 0}}
\newcommand{\onev}{{\bf 1}}

% Matrices

\newcommand{\Am}{{\bf A}}
\newcommand{\Bm}{{\bf B}}
\newcommand{\Cm}{{\bf C}}
\newcommand{\Dm}{{\bf D}}
\newcommand{\Em}{{\bf E}}
\newcommand{\Fm}{{\bf F}}
\newcommand{\Gm}{{\bf G}}
\newcommand{\Hm}{{\bf H}}
\newcommand{\Id}{{\bf I}}
\newcommand{\Jm}{{\bf J}}
\newcommand{\Km}{{\bf K}}
\newcommand{\Lm}{{\bf L}}
\newcommand{\Mm}{{\bf M}}
\newcommand{\Nm}{{\bf N}}
\newcommand{\Om}{{\bf O}}
\newcommand{\Pm}{{\bf P}}
\newcommand{\Qm}{{\bf Q}}
\newcommand{\Rm}{{\bf R}}
\newcommand{\Sm}{{\bf S}}
\newcommand{\Tm}{{\bf T}}
\newcommand{\Um}{{\bf U}}
\newcommand{\Wm}{{\bf W}}
\newcommand{\Vm}{{\bf V}}
\newcommand{\Xm}{{\bf X}}
\newcommand{\Ym}{{\bf Y}}
\newcommand{\Zm}{{\bf Z}}

% Calligraphic

\newcommand{\Ac}{{\cal A}}
\newcommand{\Bc}{{\cal B}}
\newcommand{\Cc}{{\cal C}}
\newcommand{\Dc}{{\cal D}}
\newcommand{\Ec}{{\cal E}}
\newcommand{\Fc}{{\cal F}}
\newcommand{\Gc}{{\cal G}}
\newcommand{\Hc}{{\cal H}}
\newcommand{\Ic}{{\cal I}}
\newcommand{\Jc}{{\cal J}}
\newcommand{\Kc}{{\cal K}}
\newcommand{\Lc}{{\cal L}}
\newcommand{\Mc}{{\cal M}}
\newcommand{\Nc}{{\cal N}}
\newcommand{\nc}{{\cal n}}
\newcommand{\Oc}{{\cal O}}
\newcommand{\Pc}{{\cal P}}
\newcommand{\Qc}{{\cal Q}}
\newcommand{\Rc}{{\cal R}}
\newcommand{\Sc}{{\cal S}}
\newcommand{\Tc}{{\cal T}}
\newcommand{\Uc}{{\cal U}}
\newcommand{\Wc}{{\cal W}}
\newcommand{\Vc}{{\cal V}}
\newcommand{\Xc}{{\cal X}}
\newcommand{\Yc}{{\cal Y}}
\newcommand{\Zc}{{\cal Z}}

% Bold greek letters

\newcommand{\alphav}{\hbox{\boldmath$\alpha$}}
\newcommand{\betav}{\hbox{\boldmath$\beta$}}
\newcommand{\gammav}{\hbox{\boldmath$\gamma$}}
\newcommand{\deltav}{\hbox{\boldmath$\delta$}}
\newcommand{\etav}{\hbox{\boldmath$\eta$}}
\newcommand{\lambdav}{\hbox{\boldmath$\lambda$}}
\newcommand{\epsilonv}{\hbox{\boldmath$\epsilon$}}
\newcommand{\nuv}{\hbox{\boldmath$\nu$}}
\newcommand{\muv}{\hbox{\boldmath$\mu$}}
\newcommand{\zetav}{\hbox{\boldmath$\zeta$}}
\newcommand{\phiv}{\hbox{\boldmath$\phi$}}
\newcommand{\psiv}{\hbox{\boldmath$\psi$}}
\newcommand{\thetav}{\hbox{\boldmath$\theta$}}
\newcommand{\tauv}{\hbox{\boldmath$\tau$}}
\newcommand{\omegav}{\hbox{\boldmath$\omega$}}
\newcommand{\xiv}{\hbox{\boldmath$\xi$}}
\newcommand{\sigmav}{\hbox{\boldmath$\sigma$}}
\newcommand{\piv}{\hbox{\boldmath$\pi$}}
\newcommand{\rhov}{\hbox{\boldmath$\rho$}}
\newcommand{\upsilonv}{\hbox{\boldmath$\upsilon$}}

\newcommand{\Gammam}{\hbox{\boldmath$\Gamma$}}
\newcommand{\Lambdam}{\hbox{\boldmath$\Lambda$}}
\newcommand{\Deltam}{\hbox{\boldmath$\Delta$}}
\newcommand{\Sigmam}{\hbox{\boldmath$\Sigma$}}
\newcommand{\Phim}{\hbox{\boldmath$\Phi$}}
\newcommand{\Pim}{\hbox{\boldmath$\Pi$}}
\newcommand{\Psim}{\hbox{\boldmath$\Psi$}}
\newcommand{\Thetam}{\hbox{\boldmath$\Theta$}}
\newcommand{\Omegam}{\hbox{\boldmath$\Omega$}}
\newcommand{\Xim}{\hbox{\boldmath$\Xi$}}


% Sans Serif small case

\newcommand{\Gsf}{{\sf G}}

\newcommand{\asf}{{\sf a}}
\newcommand{\bsf}{{\sf b}}
\newcommand{\csf}{{\sf c}}
\newcommand{\dsf}{{\sf d}}
\newcommand{\esf}{{\sf e}}
\newcommand{\fsf}{{\sf f}}
\newcommand{\gsf}{{\sf g}}
\newcommand{\hsf}{{\sf h}}
\newcommand{\isf}{{\sf i}}
\newcommand{\jsf}{{\sf j}}
\newcommand{\ksf}{{\sf k}}
\newcommand{\lsf}{{\sf l}}
\newcommand{\msf}{{\sf m}}
\newcommand{\nsf}{{\sf n}}
\newcommand{\osf}{{\sf o}}
\newcommand{\psf}{{\sf p}}
\newcommand{\qsf}{{\sf q}}
\newcommand{\rsf}{{\sf r}}
\newcommand{\ssf}{{\sf s}}
\newcommand{\tsf}{{\sf t}}
\newcommand{\usf}{{\sf u}}
\newcommand{\wsf}{{\sf w}}
\newcommand{\vsf}{{\sf v}}
\newcommand{\xsf}{{\sf x}}
\newcommand{\ysf}{{\sf y}}
\newcommand{\zsf}{{\sf z}}


% mixed symbols

\newcommand{\sinc}{{\hbox{sinc}}}
\newcommand{\diag}{{\hbox{diag}}}
\renewcommand{\det}{{\hbox{det}}}
\newcommand{\trace}{{\hbox{tr}}}
\newcommand{\sign}{{\hbox{sign}}}
\renewcommand{\arg}{{\hbox{arg}}}
\newcommand{\var}{{\hbox{var}}}
\newcommand{\cov}{{\hbox{cov}}}
\newcommand{\Ei}{{\rm E}_{\rm i}}
\renewcommand{\Re}{{\rm Re}}
\renewcommand{\Im}{{\rm Im}}
\newcommand{\eqdef}{\stackrel{\Delta}{=}}
\newcommand{\defines}{{\,\,\stackrel{\scriptscriptstyle \bigtriangleup}{=}\,\,}}
\newcommand{\<}{\left\langle}
\renewcommand{\>}{\right\rangle}
\newcommand{\herm}{{\sf H}}
\newcommand{\trasp}{{\sf T}}
\newcommand{\transp}{{\sf T}}
\renewcommand{\vec}{{\rm vec}}
\newcommand{\Psf}{{\sf P}}
\newcommand{\SINR}{{\sf SINR}}
\newcommand{\SNR}{{\sf SNR}}
\newcommand{\MMSE}{{\sf MMSE}}
\newcommand{\REF}{{\RED [REF]}}

% Markov chain
\usepackage{stmaryrd} % for \mkv 
\newcommand{\mkv}{-\!\!\!\!\minuso\!\!\!\!-}

% Colors

\newcommand{\RED}{\color[rgb]{1.00,0.10,0.10}}
\newcommand{\BLUE}{\color[rgb]{0,0,0.90}}
\newcommand{\GREEN}{\color[rgb]{0,0.80,0.20}}

%%%%%%%%%%%%%%%%%%%%%%%%%%%%%%%%%%%%%%%%%%
\usepackage{hyperref}
\hypersetup{
    bookmarks=true,         % show bookmarks bar?
    unicode=false,          % non-Latin characters in AcrobatÕs bookmarks
    pdftoolbar=true,        % show AcrobatÕs toolbar?
    pdfmenubar=true,        % show AcrobatÕs menu?
    pdffitwindow=false,     % window fit to page when opened
    pdfstartview={FitH},    % fits the width of the page to the window
%    pdftitle={My title},    % title
%    pdfauthor={Author},     % author
%    pdfsubject={Subject},   % subject of the document
%    pdfcreator={Creator},   % creator of the document
%    pdfproducer={Producer}, % producer of the document
%    pdfkeywords={keyword1} {key2} {key3}, % list of keywords
    pdfnewwindow=true,      % links in new window
    colorlinks=true,       % false: boxed links; true: colored links
    linkcolor=red,          % color of internal links (change box color with linkbordercolor)
    citecolor=green,        % color of links to bibliography
    filecolor=blue,      % color of file links
    urlcolor=blue           % color of external links
}
%%%%%%%%%%%%%%%%%%%%%%%%%%%%%%%%%%%%%%%%%%%



\newcommand{\Dist}{\MathAlgX{Dist}}
\newcommand{\Pred}{\MathAlgX{Pred}}
\newcommand{\Rec}{\MathAlgX{Rec}}

\newcommand{\EveDP}{\MathAlgX{\Dist}}
\newcommand{\RecBit}{\MathAlgX{\Rec}}
\newcommand{\tRecBit}{\MathAlgX{\widetilde{\Rec}}}
%\newcommand{\AlgEstimateBit}{\MathAlgX{\Rec}}
\newcommand{\AlgEstimateBit}{\MathAlgX{EstBit}}

\newcommand{\KLD}[2]{\KL({#1}||{#2})}

\newcommand{\Bn}{\cB_n}
\newcommand{\Cn}{\cC_n}
\newcommand{\Dn}{\cD_n}

\newcommand{\In}{I_n}
\newcommand{\Xn}{X_n}
\newcommand{\Yn}{Y_n}
\newcommand{\Zn}{Z_n}
\renewcommand{\Rn}{\cR_n}
\newcommand{\Pn}{\Pc_n}
\newcommand{\ensm}[1]{\set{#1}_{n\in \N}}


\newcommand{\HH}{H}



\renewcommand{\nu}{n.u.\xspace}
\newcommand{\nupt}{\nu-poly-time\xspace}
%\newcommand{\nuptm}{\nupt algorithm\xspace}

%\newcommand{\NBPE}{NBPE\xspace}
%\newcommand{\nuNBPE}{$\nu$-\NBPE}


\newcommand{\Inote}[1]{\authnote{Iftach}{#1}}
\newcommand{\Jnote}[1]{\authnote{Jad}{#1}}
\newcommand{\Nnote}[1]{\authnote{Noam}{#1}}
\newcommand{\Enote}[1]{\authnote{Eliad}{#1}}
\newcommand{\Cnote}[1]{\authnote{Chao}{#1}}


\title{
Mildly Accurate Computationally Differentially Private Inner Product Protocols Imply Oblivious Transfer
}
%Oblivious Transfer and Accurate Computationally Differentially Private Inner Product Protocols are Equivalent

\iffull
\author{
Iftach Haitner\thanks{Stellar Development Foundation and Tel Aviv University. {\tt iftachh@tauex.tau.ac.il}.} 
%Research supported by Israel Science Foundation grant 666/19.
\and
Noam Mazor\thanks{Tel Aviv University. {\tt noammaz@gmail.com}. Research partly supported by NSF CNS-2149305, AFOSR
Award FA9550-23-1-0312 and AFOSR Award FA9550-23-1-0387 and ISF Award 2338/23.}
\and
Jad Silbak\thanks{Northeastern University. {\tt jadsilbak@gmail.com}. Research supported by the Khoury College Distinguished Post-doctoral Fellowship.}
\and
Eliad Tsfadia\thanks{Georgetown University. {\tt eliadtsfadia@gmail.com}. Research supported by a gift to Georgetown University.}
\and
Chao Yan\thanks{Georgetown University. {\tt cy399@georgetown.edu}. Research supported by a gift to Georgetown University.}
}



%\else
%\author{}
\date{}
%\institute{}
\fi

\begin{document}

\maketitle



\begin{abstract}

In distributed differential privacy, multiple parties collaboratively analyze their combined data while protecting the privacy of each party's data from the eyes of the others. Interestingly, for certain fundamental two-party functions like inner product and Hamming distance, the accuracy of distributed solutions significantly lags behind what can be achieved in the centralized model. However, under computational differential privacy, these limitations can be circumvented using oblivious transfer via secure multi-party computation.
Yet, no results show that oblivious transfer is indeed necessary for accurately estimating a non-Boolean functionality. 
In particular, for the inner-product functionality, it was previously unknown whether oblivious transfer is necessary even for the best possible constant additive error.

In this work, we prove that any computationally differentially private protocol that estimates the inner product over $\oo^n \times \oo^n$ up to an additive error of $O(n^{1/6})$, can be used to construct oblivious transfer. In particular, our result implies that protocols with sub-polynomial accuracy are equivalent to oblivious transfer. In this accuracy regime, our result improves upon \citeauthor*{HaitnerMST22} [STOC '22] who showed that a key-agreement protocol is necessary.


\keywords{differential privacy; inner product; oblivious transfer}
\end{abstract}

\iffull
\Tableofcontents
\fi

\section{Introduction}

Large language models (LLMs) have achieved remarkable success in automated math problem solving, particularly through code-generation capabilities integrated with proof assistants~\citep{lean,isabelle,POT,autoformalization,MATH}. Although LLMs excel at generating solution steps and correct answers in algebra and calculus~\citep{math_solving}, their unimodal nature limits performance in plane geometry, where solution depends on both diagram and text~\citep{math_solving}. 

Specialized vision-language models (VLMs) have accordingly been developed for plane geometry problem solving (PGPS)~\citep{geoqa,unigeo,intergps,pgps,GOLD,LANS,geox}. Yet, it remains unclear whether these models genuinely leverage diagrams or rely almost exclusively on textual features. This ambiguity arises because existing PGPS datasets typically embed sufficient geometric details within problem statements, potentially making the vision encoder unnecessary~\citep{GOLD}. \cref{fig:pgps_examples} illustrates example questions from GeoQA and PGPS9K, where solutions can be derived without referencing the diagrams.

\begin{figure}
    \centering
    \begin{subfigure}[t]{.49\linewidth}
        \centering
        \includegraphics[width=\linewidth]{latex/figures/images/geoqa_example.pdf}
        \caption{GeoQA}
        \label{fig:geoqa_example}
    \end{subfigure}
    \begin{subfigure}[t]{.48\linewidth}
        \centering
        \includegraphics[width=\linewidth]{latex/figures/images/pgps_example.pdf}
        \caption{PGPS9K}
        \label{fig:pgps9k_example}
    \end{subfigure}
    \caption{
    Examples of diagram-caption pairs and their solution steps written in formal languages from GeoQA and PGPS9k datasets. In the problem description, the visual geometric premises and numerical variables are highlighted in green and red, respectively. A significant difference in the style of the diagram and formal language can be observable. %, along with the differences in formal languages supported by the corresponding datasets.
    \label{fig:pgps_examples}
    }
\end{figure}



We propose a new benchmark created via a synthetic data engine, which systematically evaluates the ability of VLM vision encoders to recognize geometric premises. Our empirical findings reveal that previously suggested self-supervised learning (SSL) approaches, e.g., vector quantized variataional auto-encoder (VQ-VAE)~\citep{unimath} and masked auto-encoder (MAE)~\citep{scagps,geox}, and widely adopted encoders, e.g., OpenCLIP~\citep{clip} and DinoV2~\citep{dinov2}, struggle to detect geometric features such as perpendicularity and degrees. 

To this end, we propose \geoclip{}, a model pre-trained on a large corpus of synthetic diagram–caption pairs. By varying diagram styles (e.g., color, font size, resolution, line width), \geoclip{} learns robust geometric representations and outperforms prior SSL-based methods on our benchmark. Building on \geoclip{}, we introduce a few-shot domain adaptation technique that efficiently transfers the recognition ability to real-world diagrams. We further combine this domain-adapted GeoCLIP with an LLM, forming a domain-agnostic VLM for solving PGPS tasks in MathVerse~\citep{mathverse}. 
%To accommodate diverse diagram styles and solution formats, we unify the solution program languages across multiple PGPS datasets, ensuring comprehensive evaluation. 

In our experiments on MathVerse~\citep{mathverse}, which encompasses diverse plane geometry tasks and diagram styles, our VLM with a domain-adapted \geoclip{} consistently outperforms both task-specific PGPS models and generalist VLMs. 
% In particular, it achieves higher accuracy on tasks requiring geometric-feature recognition, even when critical numerical measurements are moved from text to diagrams. 
Ablation studies confirm the effectiveness of our domain adaptation strategy, showing improvements in optical character recognition (OCR)-based tasks and robust diagram embeddings across different styles. 
% By unifying the solution program languages of existing datasets and incorporating OCR capability, we enable a single VLM, named \geovlm{}, to handle a broad class of plane geometry problems.

% Contributions
We summarize the contributions as follows:
We propose a novel benchmark for systematically assessing how well vision encoders recognize geometric premises in plane geometry diagrams~(\cref{sec:visual_feature}); We introduce \geoclip{}, a vision encoder capable of accurately detecting visual geometric premises~(\cref{sec:geoclip}), and a few-shot domain adaptation technique that efficiently transfers this capability across different diagram styles (\cref{sec:domain_adaptation});
We show that our VLM, incorporating domain-adapted GeoCLIP, surpasses existing specialized PGPS VLMs and generalist VLMs on the MathVerse benchmark~(\cref{sec:experiments}) and effectively interprets diverse diagram styles~(\cref{sec:abl}).

\iffalse
\begin{itemize}
    \item We propose a novel benchmark for systematically assessing how well vision encoders recognize geometric premises, e.g., perpendicularity and angle measures, in plane geometry diagrams.
	\item We introduce \geoclip{}, a vision encoder capable of accurately detecting visual geometric premises, and a few-shot domain adaptation technique that efficiently transfers this capability across different diagram styles.
	\item We show that our final VLM, incorporating GeoCLIP-DA, effectively interprets diverse diagram styles and achieves state-of-the-art performance on the MathVerse benchmark, surpassing existing specialized PGPS models and generalist VLM models.
\end{itemize}
\fi

\iffalse

Large language models (LLMs) have made significant strides in automated math word problem solving. In particular, their code-generation capabilities combined with proof assistants~\citep{lean,isabelle} help minimize computational errors~\citep{POT}, improve solution precision~\citep{autoformalization}, and offer rigorous feedback and evaluation~\citep{MATH}. Although LLMs excel in generating solution steps and correct answers for algebra and calculus~\citep{math_solving}, their uni-modal nature limits performance in domains like plane geometry, where both diagrams and text are vital.

Plane geometry problem solving (PGPS) tasks typically include diagrams and textual descriptions, requiring solvers to interpret premises from both sources. To facilitate automated solutions for these problems, several studies have introduced formal languages tailored for plane geometry to represent solution steps as a program with training datasets composed of diagrams, textual descriptions, and solution programs~\citep{geoqa,unigeo,intergps,pgps}. Building on these datasets, a number of PGPS specialized vision-language models (VLMs) have been developed so far~\citep{GOLD, LANS, geox}.

Most existing VLMs, however, fail to use diagrams when solving geometry problems. Well-known PGPS datasets such as GeoQA~\citep{geoqa}, UniGeo~\citep{unigeo}, and PGPS9K~\citep{pgps}, can be solved without accessing diagrams, as their problem descriptions often contain all geometric information. \cref{fig:pgps_examples} shows an example from GeoQA and PGPS9K datasets, where one can deduce the solution steps without knowing the diagrams. 
As a result, models trained on these datasets rely almost exclusively on textual information, leaving the vision encoder under-utilized~\citep{GOLD}. 
Consequently, the VLMs trained on these datasets cannot solve the plane geometry problem when necessary geometric properties or relations are excluded from the problem statement.

Some studies seek to enhance the recognition of geometric premises from a diagram by directly predicting the premises from the diagram~\citep{GOLD, intergps} or as an auxiliary task for vision encoders~\citep{geoqa,geoqa-plus}. However, these approaches remain highly domain-specific because the labels for training are difficult to obtain, thus limiting generalization across different domains. While self-supervised learning (SSL) methods that depend exclusively on geometric diagrams, e.g., vector quantized variational auto-encoder (VQ-VAE)~\citep{unimath} and masked auto-encoder (MAE)~\citep{scagps,geox}, have also been explored, the effectiveness of the SSL approaches on recognizing geometric features has not been thoroughly investigated.

We introduce a benchmark constructed with a synthetic data engine to evaluate the effectiveness of SSL approaches in recognizing geometric premises from diagrams. Our empirical results with the proposed benchmark show that the vision encoders trained with SSL methods fail to capture visual \geofeat{}s such as perpendicularity between two lines and angle measure.
Furthermore, we find that the pre-trained vision encoders often used in general-purpose VLMs, e.g., OpenCLIP~\citep{clip} and DinoV2~\citep{dinov2}, fail to recognize geometric premises from diagrams.

To improve the vision encoder for PGPS, we propose \geoclip{}, a model trained with a massive amount of diagram-caption pairs.
Since the amount of diagram-caption pairs in existing benchmarks is often limited, we develop a plane diagram generator that can randomly sample plane geometry problems with the help of existing proof assistant~\citep{alphageometry}.
To make \geoclip{} robust against different styles, we vary the visual properties of diagrams, such as color, font size, resolution, and line width.
We show that \geoclip{} performs better than the other SSL approaches and commonly used vision encoders on the newly proposed benchmark.

Another major challenge in PGPS is developing a domain-agnostic VLM capable of handling multiple PGPS benchmarks. As shown in \cref{fig:pgps_examples}, the main difficulties arise from variations in diagram styles. 
To address the issue, we propose a few-shot domain adaptation technique for \geoclip{} which transfers its visual \geofeat{} perception from the synthetic diagrams to the real-world diagrams efficiently. 

We study the efficacy of the domain adapted \geoclip{} on PGPS when equipped with the language model. To be specific, we compare the VLM with the previous PGPS models on MathVerse~\citep{mathverse}, which is designed to evaluate both the PGPS and visual \geofeat{} perception performance on various domains.
While previous PGPS models are inapplicable to certain types of MathVerse problems, we modify the prediction target and unify the solution program languages of the existing PGPS training data to make our VLM applicable to all types of MathVerse problems.
Results on MathVerse demonstrate that our VLM more effectively integrates diagrammatic information and remains robust under conditions of various diagram styles.

\begin{itemize}
    \item We propose a benchmark to measure the visual \geofeat{} recognition performance of different vision encoders.
    % \item \sh{We introduce geometric CLIP (\geoclip{} and train the VLM equipped with \geoclip{} to predict both solution steps and the numerical measurements of the problem.}
    \item We introduce \geoclip{}, a vision encoder which can accurately recognize visual \geofeat{}s and a few-shot domain adaptation technique which can transfer such ability to different domains efficiently. 
    % \item \sh{We develop our final PGPS model, \geovlm{}, by adapting \geoclip{} to different domains and training with unified languages of solution program data.}
    % We develop a domain-agnostic VLM, namely \geovlm{}, by applying a simple yet effective domain adaptation method to \geoclip{} and training on the refined training data.
    \item We demonstrate our VLM equipped with GeoCLIP-DA effectively interprets diverse diagram styles, achieving superior performance on MathVerse compared to the existing PGPS models.
\end{itemize}

\fi 

% \vspace{-0.2cm}
\section{System Overview}
The mmE-Loc enhances the mmWave radar with an event camera to achieve accurate and low-latency drone ground localization, allowing the drone to rapidly adjust its location state and perform a precise landing.
% Given the paramount importance of safety in commercial drone operations, mmE-Loc can collaborate with RTK or visual markers to ensure precise landings.
Given the critical importance of safety in commercial drone operations, mmE-Loc can work in conjunction with RTK or visual markers to ensure precise landing performance.
In this section, we mathematically introduce the problem that mmE-Loc tries to address and provide an overview of the system design.

% mmE-Loc leverages the integration of mmWave radar and event camera data to achieve accurate, low-latency drone ground localization, allowing the drone to quickly adjust its position and perform precise landings. Given the critical importance of safety in commercial drone operations, mmE-Loc can also work in conjunction with RTK or visual markers to ensure precise landing performance.

% enabling a reliable and accurate localization service for drones (\fig \ref{overview}).

\vspace{-0.3cm}
\subsection{Problem Formulation}
In this section, we illustrate key variables in mmE-Loc and introduce the system's inputs and outputs.

% \begin{figure}[t]
%     \setlength{\abovecaptionskip}{-0.1cm} % height above Figure X caption
%     \setlength{\belowcaptionskip}{-0.3cm}
%     \setlength{\subfigcapskip}{-0.25cm}
%     \centering
%         \includegraphics[width=1\columnwidth]{Figs/reference.png}
%         \vspace{-0.2cm}
%     \caption{Illustration of reference systems and essential variables in mmE-Loc.}
%     \label{reference}
%     % \vspace{-0.6cm}
% \end{figure} 

\textbf{Reference systems.} \label{3.2}
% As shown in \fig \ref{reference}, t
There are four reference (\aka, coordinate) systems in mmE-Loc: 
$(i)$ the Event camera reference system $\mathtt{E}$; 
$(ii)$ the Radar reference system $\mathtt{R}$; 
$(iii)$ the Object reference system $\mathtt{O}$;
$(iv)$ the Drone reference system $\mathtt{D}$.
Note that a drone can be considered as an object.
For clarity, before an object is identified as a drone, we utilize $\mathtt{O}$. 
Once confirmed as a drone, we use $\mathtt{D}$ for the drone and continue using $\mathtt{O}$ for other objects.
Throughout the operation of system, $\mathtt{E}$ and $\mathtt{R}$ remain stationary and are rigidly attached together, while $\mathtt{O}$ and $\mathtt{D}$ undergo changes in accordance with movement of the object and the drone, respectively. 
The transformation from $\mathtt{R}$ to $\mathtt{E}$ can be readily obtained from calibration \cite{wang2023vital}. 

\textbf{Goal of mmE-Loc.}
The goal of mmE-Loc is to determine 3D location of the drone, defined as $t_{\mathtt{ED}}$, the translation from coordinate system $\mathtt{D}$ to $\mathtt{E}$.
Specifically, mmE-Loc optimizes and reports 3D location of drone $(l_x, l_y, l_z)$ at each timestamp $i$ with input from event stream and radar sample.
% where $t^i_{\mathtt{ED}}$ represents the translation vector from $\mathtt{D}$ to $\mathtt{E}^3$. 
$t_{\mathtt{ED}}$ and ($l_x$, $l_y$, $l_z$) are equivalent representations of the drone’s location and can be inter-converted with Rodrigues’ formula \cite{min2021joint}. 
The former representation is adopted in the paper, as it is commonly used in drone flight control systems.

\begin{figure*}[t]
    \setlength{\abovecaptionskip}{0.3cm} % height above Figure X caption
    \setlength{\belowcaptionskip}{-0.2cm}
    \setlength{\subfigcapskip}{-0.25cm}
    \centering
        \includegraphics[width=1.85\columnwidth]{Figs/trackingmodel.png}
        \vspace{-0.3cm}
    \caption{Illustration of tracking models in Consistency-instructed Collaborative Tracking algorithm.}
    \label{CCT}
    \vspace{-0.2cm}
\end{figure*} 

% We first briefly describe and illustrate some essential variables in mmE-Loc and the problem of localizing the landing drone,which is the 3D location ($l_x$, $l_y$, $l_z$).
% \todo{As shown in \fig )}, there are three reference (\aka, coordinate) systems in mmE-Loc: $(i)$ the Event camera reference system \mathtt{E}; $(ii)$ the Radar reference system \mathtt{R}; $(iii)$ the Drone reference system \mathtt{D}.


% \subsection{mmE-Loc: System goals}
% \noindent $\bullet$ \textbf{How to detect and track the drones from noisy sensing results at a high frequency?}
% Environmental changes often introduce irrelevant information in the sensing results from the event camera and mmWave radar, hindering the system's ability to identify signals changes caused by the drones to be tracked. 
% Additionally, modern flight control loops operate at frequencies exceeding 400 Hz, requiring the system to detect and track drones rapidly. 
% However, traditional noise filtering, object detection and tracking algorithms have high time complexities, resulting in precision and tracking frequency bottlenecks. 
% \todo{Explanation in figures or tables.}

% \noindent $\bullet$ \textbf{How to fuse two types of heterogeneous data to precisely localize drones at a high frequency?} 
% Once the drone is detected, accurate 3D spatial localization of it is essential, which is more time-consuming than detection and tracking due to additional processing. 
% Moreover, event cameras provide asynchronous event streams, while mmWave radar generates sparse point clouds with relatively low spatial resolution. 
% Previous fusion methods (\eg extended Kalman filters and particle filters) often suffer from significant cumulative drift error and lengthy processing times, rendering them inadequate for high-frequency and high-precision tracking.
% \todo{Explanation in figures or tables.}

\vspace{-0.3cm}
\subsection{Overview}
% mmE-Loc is a localization system designed for high-frequency and precise localization of drones, enabling them to rapidly adjust their location state and achieve accurate landing. 
As illustrated in \fig \ref{overview}, mmE-Loc comprises two key modules: 
% $(i)$ \textit{CCT} (\textbf{C}onsistency-instructed \textbf{C}ollaborative \textbf{T}racking) for noise filtering and drone detection and 
% $(ii)$ \textit{GAO} (\textbf{G}raph-informed \textbf{A}daptive \textbf{O}ptimization) for localization of drones from the integrated event stream and mmWave 3D point cloud. \todo{Fix the figure.}

\noindent $\bullet$ 
The \textit{CCT} (\textbf{C}onsistency-instructed \textbf{C}ollaborative \textbf{T}racking) for noise filtering, drone detection, and preliminary localization of the drone.
This module utilizes time-synchronized event streams and mmWave radar measurements as inputs. 
Subsequently, the \textit{Radar Tracking Model} processes radar measurements to generate a sparse 3D point cloud. 
Meanwhile, the \textit{Event Tracking Model} takes into the stream of asynchronous events for event filtering, drone detection, and tracking. 
% Finally, \textit{Consistency-instructed Measurements Filter} aligns the results of the both tracking models leverage the consistency between both modalities, and then utilizes periodic micro motion of drone to extract drone-related measurements,  eliminate noise and error detections, and roughly localize the drone.
Finally, \textit{Consistency-instructed Measurements Filter} aligns the outputs of both tracking models by leveraging \textit{temporal-consistency} between the two modalities. 
It then utilizes the drone's periodic micro-motion to extract drone-specific measurements and achieve drone preliminary localization.

\noindent $\bullet$
The \textit{GAJO} (\textbf{G}raph-informed \textbf{A}daptive \textbf{J}oint \textbf{O}ptimization) for fine localization and trajectory optimization of the drone.
% GAJO initially derives preliminary estimates of drone motion and location, using radar measurements and event camera tracking results, respectively. 
% GAJO includes a carefully designed \textit{factor graph-based optimization} method, which 通过深入挖掘事件相机与radar的潜力, jointly fuses and refines the outputs from both the \textit{Event Tracking Model} and \textit{Radar Tracking Model} to accurately determine the location of drone. 
Based on the operational principles of two sensors and their respective noise distributions, \textit{GAJO} incorporates a meticulously designed \textit{factor graph-based optimization} method. 
This module employs the \textit{spatial-complementarity} from both modalities to unleash the potential of event camera and mmWave radar in drone ground localization.
Specifically, \textit{GAJO} jointly fuses the preliminary location estimation from the \textit{Event Tracking Model} and the \textit{Radar Tracking Model} and adaptively refines them, determining the fine location of drone with $ms$-level processing time.
% Finally, to expedite the process of the \textit{factor graph-based location optimization} method, we transform factor graph into a Bayes tree through elimination and optimize the drone's location adaptively via a \textit{Incremental Optimization method}. 
% !TEX root =  ../main.tex
\section{Background on causality and abstraction}\label{sec:preliminaries}

This section provides the notation and key concepts related to causal modeling and abstraction theory.

\spara{Notation.} The set of integers from $1$ to $n$ is $[n]$.
The vectors of zeros and ones of size $n$ are $\zeros_n$ and $\ones_n$.
The identity matrix of size $n \times n$ is $\identity_n$. The Frobenius norm is $\frob{\mathbf{A}}$.
The set of positive definite matrices over $\reall^{n\times n}$ is $\pd^n$. The Hadamard product is $\odot$.
Function composition is $\circ$.
The domain of a function is $\dom{\cdot}$ and its kernel $\ker$.
Let $\mathcal{M}(\mathcal{X}^n)$ be the set of Borel measures over $\mathcal{X}^n \subseteq \reall^n$. Given a measure $\mu^n \in \mathcal{M}(\mathcal{X}^n)$ and a measurable map $\varphi^{\V}$, $\mathcal{X}^n \ni \mathbf{x} \overset{\varphi^{\V}}{\longmapsto} \V^\top \mathbf{x} \in \mathcal{X}^m$, we denote by $\varphi^{\V}_{\#}(\mu^n) \coloneqq \mu^n(\varphi^{\V^{-1}}(\mathbf{x}))$ the pushforward measure $\mu^m \in \mathcal{M}(\mathcal{X}^m)$. 


We now present the standard definition of SCM.

\begin{definition}[SCM, \citealp{pearl2009causality}]\label{def:SCM}
A (Markovian) structural causal model (SCM) $\scm^n$ is a tuple $\langle \myendogenous, \myexogenous, \myfunctional, \zeta^\myexogenous \rangle$, where \emph{(i)} $\myendogenous = \{X_1, \ldots, X_n\}$ is a set of $n$ endogenous random variables; \emph{(ii)} $\myexogenous =\{Z_1,\ldots,Z_n\}$ is a set of $n$ exogenous variables; \emph{(iii)} $\myfunctional$ is a set of $n$ functional assignments such that $X_i=f_i(\parents_i, Z_i)$, $\forall \; i \in [n]$, with $ \parents_i \subseteq \myendogenous \setminus \{ X_i\}$; \emph{(iv)} $\zeta^\myexogenous$ is a product probability measure over independent exogenous variables $\zeta^\myexogenous=\prod_{i \in [n]} \zeta^i$, where $\zeta^i=P(Z_i)$. 
\end{definition}
A Markovian SCM induces a directed acyclic graph (DAG) $\mathcal{G}_{\scm^n}$ where the nodes represent the variables $\myendogenous$ and the edges are determined by the structural functions $\myfunctional$; $ \parents_i$ constitutes then the parent set for $X_i$. Furthermore, we can recursively rewrite the set of structural function $\myfunctional$ as a set of mixing functions $\mymixing$ dependent only on the exogenous variables (cf. \cref{app:CA}). A key feature for studying causality is the possibility of defining interventions on the model:
\begin{definition}[Hard intervention, \citealp{pearl2009causality}]\label{def:intervention}
Given SCM $\scm^n = \langle \myendogenous, \myexogenous, \myfunctional, \zeta^\myexogenous \rangle$, a (hard) intervention $\iota = \operatorname{do}(\myendogenous^{\iota} = \mathbf{x}^{\iota})$, $\myendogenous^{\iota}\subseteq \myendogenous$,
is an operator that generates a new post-intervention SCM $\scm^n_\iota = \langle \myendogenous, \myexogenous, \myfunctional_\iota, \zeta^\myexogenous \rangle$ by replacing each function $f_i$ for $X_i\in\myendogenous^{\iota}$ with the constant $x_i^\iota\in \mathbf{x}^\iota$. 
Graphically, an intervention mutilates $\mathcal{G}_{\mathsf{M}^n}$ by removing all the incoming edges of the variables in $\myendogenous^{\iota}$.
\end{definition}

Given multiple SCMs describing the same system at different levels of granularity, CA provides the definition of an $\alpha$-abstraction map to relate these SCMs:
\begin{definition}[$\abst$-abstraction, \citealp{rischel2020category}]\label{def:abstraction}
Given low-level $\mathsf{M}^\ell$ and high-level $\mathsf{M}^h$ SCMs, an $\abst$-abstraction is a triple $\abst = \langle \Rset, \amap, \alphamap{} \rangle$, where \emph{(i)} $\Rset \subseteq \datalow$ is a subset of relevant variables in $\mathsf{M}^\ell$; \emph{(ii)} $\amap: \Rset \rightarrow \datahigh$ is a surjective function between the relevant variables of $\mathsf{M}^\ell$ and the endogenous variables of $\mathsf{M}^h$; \emph{(iii)} $\alphamap{}: \dom{\Rset} \rightarrow \dom{\datahigh}$ is a modular function $\alphamap{} = \bigotimes_{i\in[n]} \alphamap{X^h_i}$ made up by surjective functions $\alphamap{X^h_i}: \dom{\amap^{-1}(X^h_i)} \rightarrow \dom{X^h_i}$ from the outcome of low-level variables $\amap^{-1}(X^h_i) \in \datalow$ onto outcomes of the high-level variables $X^h_i \in \datahigh$.
\end{definition}
Notice that an $\abst$-abstraction simultaneously maps variables via the function $\amap$ and values through the function $\alphamap{}$. The definition itself does not place any constraint on these functions, although a common requirement in the literature is for the abstraction to satisfy \emph{interventional consistency} \cite{rubenstein2017causal,rischel2020category,beckers2019abstracting}. An important class of such well-behaved abstractions is \emph{constructive linear abstraction}, for which the following properties hold. By constructivity, \emph{(i)} $\abst$ is interventionally consistent; \emph{(ii)} all low-level variables are relevant $\Rset=\datalow$; \emph{(iii)} in addition to the map $\alphamap{}$ between endogenous variables, there exists a map ${\alphamap{}}_U$ between exogenous variables satisfying interventional consistency \cite{beckers2019abstracting,schooltink2024aligning}. By linearity, $\alphamap{} = \V^\top \in \reall^{h \times \ell}$ \cite{massidda2024learningcausalabstractionslinear}. \cref{app:CA} provides formal definitions for interventional consistency, linear and constructive abstraction.
\section{Oblivious Transfer from $\DP$ Inner Product}\label{sec:main_theorems}

In this section, we state our main theorems. We first state and prove the information theoretic case in \cref{sec:DP TO OT IT}, and then prove the computational case in \cref{sec:DP TO OT Comp}.

\subsection{Information-Theoretic Case}\label{sec:DP TO OT IT}
The following is our main theorem (for the information theoretic case) which shows that given a sufficiently accurate and private $\DP$ channel, we can construct a statistically secure semi-honest oblivious transfer protocol.

\begin{theorem}\label{thm:DPIP-to-OT}
There exist constants $c_1,c_2>0$ and an oracle-aided \ppt protocol $\Pi$ such that the following holds for large enough $n \in \bbN$ and for 
$\eps \leq \log^{0.9} n$, $\delta \leq \frac1{3n}$, and $\ell \leq e^{-c_1  \eps}  c_2\cdot n^{1/6}$:
    Let $C = ((X,U),(Y,V))$ be an $(\eps,\delta)$-\DP channel with independent $X,Y \la \oo^n$ that is $(\ell,0.999)$-accurate for the inner-product functionality (i.e., $\ppr{C}{\size{O-\ip{X,Y}} \leq \ell} \geq 0.999$).
    Then $\Pi^C$ is a semi-honest statistically secure $\OT$ protocol.
\end{theorem}

To prove Theorem \ref{thm:DPIP-to-OT}, we will make use of the following technical Lemma.


\begin{lemma}\label{lemma:DPIP-to-AWEC}
        There exists a \ppt protocol $\Gamma=(\Ac,\Bc)$ such that for every $c_1,c_2,n,\eps,\delta,\ell$ and $C$ as in \cref{thm:DPIP-to-OT},
        the channel $\tilde{C}$ induced by the execution of $\Gamma^C$ is $(\ell, \alpha=0.001, p=0.001, q=0.001)$-\AWEC (\cref{def:AWEC}).
\remove{
	There exist constant $c_1,c_2 > 0$ and an oracle-aided \ppt protocol $\Gamma=(\Ac,\Bc)$ such that the following holds for large enough $n \in \bbN$ and for 
	$\eps \leq \log^{0.9} n$, $\delta \leq \frac1{3n}$, and $\ell \leq e^{-c_1  \eps}  c_2\cdot n^{1/6}$:
	Let $C = ((X,U),(Y,V))$ be an $(\eps,\delta)$-\DP channel with independent $X,Y \la \oo^n$ that is $(\ell,0.999)$-accurate for the inner-product functionality (i.e., $\ppr{C}{\size{O-\ip{X,Y}} \leq \ell} \geq 0.999$). Then the channel $\tilde{C}$ induced by the execution of $\Gamma^C$ is $(\ell, \alpha=0.001, p=0.001, q=0.001)$-\AWEC (\cref{def:AWEC}). 
}
    Furthermore, the proof is constructive in a black-box manner:
	\begin{enumerate}
		\item There exists an oracle-aided \ppt algorithm $\Act$ such that for every channel $C = ((X,U),(Y,V))$ and algorithm $\Ac$ violating the \AWEC secrecy property~\ref{AWEC:item:A} of $\tilde{C}$ (the channel of $\Gamma^C$), the following holds for $Y^*_i = \Act^{\Ac}(i,\: Y_{-i}, \: X, \: U)$:\label{item:privacy-of-Y}
		\begin{align*}
			\eex{i \la [n]}{\pr{Y^*_i = Y_i }} > e^{\eps} \cdot \eex{i \la [n]}{\pr{Y^*_i = -Y_i }} + \delta.
		\end{align*}
		
		%\Enote{The above two items suffice for breaking $(\eps,\delta)$-DP because if we let $p = \pr{Y^*_i = Y_i}$ and $q = \pr{Y^*_i = -Y_i}$ then $p \leq e^{\eps} q + \delta$, and if $q \geq e^{-\eps} \delta$ then we get that
			%$$\frac{p}{p+q} \leq \frac{e^{\eps} q + \delta}{e^{\eps} q + \delta + q} = \frac{1}{1 + \frac{1}{e^{\eps} + \delta/p}} \leq  \frac{1}{1 + \frac{1}{2\cdot e^{\eps} }} \leq \frac{2\cdot e^{\eps}}{2\cdot e^{\eps}+1}.$$}
		
		%\item \Enote{We should use Theorem 5.1 from our previous paper (or Theorem 6.2 directly).}
		
		
		\item There exists an oracle-aided \ppt algorithm $\Bct$ such that for every channel $C = ((X,U),(Y,V))$ and algorithm $\Bc$ violating the \AWEC secrecy property~\ref{AWEC:item:B} of $\tilde{C}$ (the channel of $\Gamma^C$),  the following holds for $X^*_i = \Bct^{\Bc, C}(i,\: X_{-i}, \: Y, \: V)$:\label{item:privacy-of-X}
		\begin{align*}
			\eex{i \la [n]}{\pr{X^*_i = X_i }} > e^{\eps} \cdot \eex{i \la [n]}{\pr{X^*_i = -X_i }} + \delta.
		\end{align*} 
	\end{enumerate}
\end{lemma}

The proof of \cref{lemma:DPIP-to-AWEC} is given in \cref{sec:DPIP_to_WAEC}. But first we use \cref{lemma:DPIP-to-AWEC} to prove \cref{thm:DPIP-to-OT}. 

\begin{proof}[Proof of \cref{thm:DPIP-to-OT}]
By \cref{lemma:DPIP-to-AWEC} there exists a \ppt protocol $\Gamma\coloneqq\Gamma^C$ such that channel $\tilde{C}\coloneqq C_{\Gamma(\pk)}$ is a $(\ell, \alpha=0.001, p=0.001, q=0.001)$-\AWEC. By \cref{lemma:AWEC-to-WEC} (and \cref{prop:hard-to-guess-inf}) there exists a \ppt protocol $\Lambda\coloneqq\Lambda^{\tilde{C}}$ such that the channel $\hhC=C_{\Lambda(\pk)}$ is a $(\alpha'=0.002,\: p' = 0.001 ,\:  q' = 1/2 + 0.022))$-\WEC. Since $44(\eps'+p')<1-q'$, by  \cref{thm:WEC TO OT IT} there exists a \ppt protocol $\Pi$, such that $\Pi^{\hhC}$ is a semi-honest statistically secure \OT, concluding the proof.
\end{proof}


\subsection{Computational Case}\label{sec:DP TO OT Comp}


In this section, we state and prove our results for the computational case. We show that a \CDP (computational differential private) protocol that estimates the inner product well implies a semi-honest computationally secure oblivious transfer protocol. 


\begin{theorem}[Restatement of \cref{thm:intro:main}]\label{thm:CDPIP-to-OT}
There exist constant $c_1,c_2 > 0$ and an oracle-aided \ppt protocol $\Pi$ such that the following holds for large enough $n \in \bbN$ and for 
$\eps \leq \log^{0.9} n$, $\delta \leq \frac1{3n}$, and $\ell \leq e^{-c_1  \eps}  c_2\cdot n^{1/6}$:
    Let $\Psi$ be an $(\eps,\delta)$-\CDP protocol that is $(\ell,0.999)$-accurate for the inner-product functionality. 
    Then $\Pi^\Psi$ is a semi-honest computationally secure oblivious transfer protocol.
\end{theorem}

By the result of \cite{GoldreichMW87}, in the computational setting, we can ``compile" any semi-honest computational oblivious transfer protocol into a protocol that is secure against any \ppt (malicious) adversary (assuming one-way functions). We state this formally in the following corollary.

\begin{corollary}\label{cor:mal OT}
   Let $\eps,\delta,\ell$ be as in \cref{thm:CDPIP-to-OT}. If there exists a protocol $\Psi$ that is $(\eps,\delta)$-\CDP  and is $(\ell,0.999)$-accurate for the inner-product functionality, then there exists a computationally secure oblivious transfer protocol. 
\end{corollary}
\begin{proof}[Proof of \cref{cor:mal OT}]
By \cref{thm:CDPIP-to-OT}, there exists a semi-honest computational secure oblivious transfer protocol $\Pi$. Note that by \cite{ImpagliazzoLu89}, $\Pi$ implies the existence of one-way functions and by \cite{GoldreichMW87}, using the one-way function, we can compile $\Pi$ into a (computational) \OT secure against arbitrary adversaries.
\end{proof}


\paragraph{Proof of \cref{thm:CDPIP-to-OT}.}

 In order to use \cref{lemma:DPIP-to-AWEC}, and similar to \cite{HaitnerMST22}, we first convert the protocol $\Psi$ into a (no input) protocol such that the \CDP-channel it induces, is uniform and accurately estimates the inner-product functionality. Such a transformation is simply the following protocol that invokes $\Psi$ over uniform inputs, and each party locally outputs its input. 
\begin{protocol}[$\hPsi^{\Psi} = (\hAc,\hBc)$]\label{prot:EDPtoSV}
	\item Common input: $1^\kappa$.
	%	\item Oracle: protocol $\Psi=(\Ac,\Bc)$.
	\item Operation:
	\begin{enumerate}
		
		\item $\hAc$ samples $x \gets \oo^{\pn(\kappa)}$ and $\hBc$ samples $y\gets \oo^{\pn(\kappa)}$. 
		
		\item The parties interact in \remove{a random execution protocol }$\Psi(1^\kappa)$, with $\hAc$ playing the role of $\Ac$ with private input $x$, and $\hBc$ playing the role of $\Bc$ with private input $y$.
		
		\item $\hAc$ locally outputs $x$ and $\hBc$ locally outputs $y$. 
	\end{enumerate}
\end{protocol}

Let $C$ be the channel ensemble induced by $\hPsi$, letting its designated output (the function $\out$) be the designated output of the embedded execution of $\Psi$. The following fact is immediate by definition.

\begin{proposition}\label{prop:EDP to SV}
	The channel ensemble $C$ is $(\eps,\delta)$-$\CDP$, and has the same accuracy for computing the inner product as protocol $\Psi$ has.
\end{proposition}


We first prove the computational version of \cref{lemma:DPIP-to-AWEC}.
\begin{claim}\label{claim:Comp DP to AWEC}
    There exist constant $c_1,c_2 > 0$ and an oracle-aided \ppt protocol $\Gamma=(\Ac,\Bc)$ such that the following holds for large enough $n \in \bbN$ and for 
	$\eps \leq \log^{0.9} n$, $\delta \leq \frac1{3n}$, and $\ell \leq e^{-c_1  \eps}  c_2\cdot n^{1/6}$:
    Let $\Psi$ be an $(\eps,\delta)$-\CDP protocol that is $(\ell,0.999)$-accurate for the inner-product functionality. Then the protocol $\Gamma^{\Psi}$ is an  $(\ell, \alpha=0.001, p=0.001, q=0.001)$-\CompAWEC protocol.
\end{claim}
\begin{proof}[Proof of \cref{claim:Comp DP to AWEC}]
Recall that $\Psi$ is an $(\eps,\delta)$-$\DP$ protocol that is $(\ell,0.999)$-acccurate for the inner-product functionality for 
	$\eps \leq \log^{0.9} n$, $\delta \leq \frac1{3n}$, and $\ell \leq e^{-c_1  \eps}  c_2\cdot n^{1/6}$. By \cref{prop:EDP to SV}, $\hPsi\coloneqq\hPsi^\Psi$ induces a channel ensemble $C=\set{C_\pk=((X_\pk,U_\pk),(Y_\pk,V_\pk))}_{\pk\in\N}$ that is $(\eps,\delta)$-\CDP and $(\ell,0.999)$-acccurate for the inner-product functionality. Let $\Gamma$ be the \ppt protocol guarrented by \cref{lemma:DPIP-to-AWEC}, we now claim that $\Gamma^{\Psi}(\pk)\coloneqq\Gamma^{C_\pk}(\pk)$ is an $(\ell, \alpha=0.001, p=0.001, q=0.001)$-\CompAWEC protocol. %That is, the ensemble $\tC=\set{\tC_\pk=(X,V)(}_{\pk\in\N}$
Assume towards contradiction that this does not hold, then by \cref{lemma:DPIP-to-AWEC}, there exists a \ppt $\Dc$, such that for infinity many $\pk\in\N$, $\Dc$ contradicts one of the two secrecy properties of $\CompAWEC$. Fix such $\pk\in\N$ and omit it from the notation when clear from the context, and without loss of generality, assume that $\Dc$ contradicts the first secrecy property of $\CompAWEC$ (the case where $D$ contradicts the second property is essentially identical). By the first item of \cref{lemma:DPIP-to-AWEC}, there exists a \ppt algorithm $\tilde{\Ac}$ such that 
$Y^*_i = \Act^{\Dc}(i,\: Y_{-i}, \: X, \: U)$  and it holds that:
\begin{align*}
			\eex{i \la [n]}{\pr{Y^*_i = Y_i }} > e^{\eps} \cdot \eex{i \la [n]}{\pr{Y^*_i = -Y_i }} + \delta.
\end{align*}
Thus, we get a contradiction since by \cref{prop:hard-to-guess-comp}, algorithm $\Act^{\Dc}$ breaks the \CDP property of $C$.  
\end{proof}


\begin{proof}[Proof of \cref{thm:CDPIP-to-OT}]
Recall that $\Psi$ is an $(\eps,\delta)$-$\DP$ protocol that is $(\ell,0.999)$-acccurate for the inner-product functionality. By \cref{claim:Comp DP to AWEC} there exists a \ppt protocol $\Gamma$ such that $\Gamma\coloneqq\Gamma^\Psi$ is an $(\ell, \alpha=0.001, p=0.001, q=0.001)$-\CompAWEC protocol. By \cref{cor:CompAWEC to CompWEC}, there exists a \ppt protocol $\Lambda$ such that $\Lambda\coloneqq \Lambda^\Gamma$ is $(\epsilon'=\epsilon+0.001,\: p' = p ,\:  q' = 1/2 + 2(q+0.01))$-\CompWEC. Finally, since $44(\eps'+p')<1-q'$ by  \cref{thm:WEC TO OT Comp} there exists a \ppt protocol $\Pi$, such that $\Pi^{\Lambda}$ is a semi-honest computationally secure \OT, as required.
\end{proof}
\newcommand{\GenRand}{\MathAlgX{GenRand}}
\newcommand{\GenView}{\MathAlgX{GenView}}

\section{\AWEC From \DP Inner Product (Proof of \cref{lemma:DPIP-to-AWEC})}\label{sec:DPIP_to_WAEC}

In this section, we show how to implement AWEC (\cref{def:AWEC}) from an $(\eps,\delta)$-DP channel that is accurate enough for the inner product functionality. We do that using a \ppt protocol and a constructive security proof.



%$C = ((X,U = (U',O)),(Y,V = (V',O)))$ with independent $X,Y \la \oo^n$ that is 
%$\ell(n)$-accurate for the inner product functionality (i.e., $\ppr{C}{\size{O-\ip{X,Y}} \leq \ell(n)} \geq 0.99$), for $\ell(n) = \tilde{\Theta}(n^{1/6})$. Importantly, our security proof is constructive. 


%\Cnote{we can set $m=O(\frac{(n\cdot\frac{e^\varepsilon+1}{e^\varepsilon})^{2/5}}{\log^{2/5}(n\cdot\frac{e^\varepsilon+1}{e^\varepsilon})})$, $k=O(m^{1/2})=O(\frac{n^{1/5}}{\log^{1/5}n})$, $\ell=O(k^{1/2})=O(\frac{n^{1/10}}{\log^{1/10}n})$}





The following protocol is used to prove \cref{lemma:DPIP-to-AWEC}.

\begin{protocol}[Protocol $\Pi = (\Ac,\Bc)$]\label{protocol:DPIP-to-AWEC}
	\item Oracle access: An $(\eps,\delta)$-DP channel $C =((X,U),(Y,V))$ with $X,Y \la \oo^n$\remove{ (i.e., \emph{uniform} channel)} that is $(\ell,0.999)$-accurate for the \emph{inner-product} functionality.
	\item Operation:
	\begin{enumerate}
		%\item  $\Ac$ samples $x\gets \mon$, and   $\Bc$ samples $y\gets \mon$.
		
		\item Sample $(x,u), (y,v) \la C$. $\Ac$ gets $(x,u)$ and $\Bc$ gets $(y,v)$. 
		
		\item  $\Ac$ samples  $r \la \zo^n$ and sends $(r,x_{r} = \set{x_i \colon r_i =1})$ to $\Bc$.
		
		\item $\Bc$ samples a random bit $b \la \zo$ and acts as follows:
		
		\begin{enumerate}
			
			\item If $b=0$, it sends $y_{-r}= \set{y_i \colon r_i =0}$ to $\Ac$ and outputs $o_B = \out(v) - \ip{x_r, y_r}$.
			
			\item~\label{step: add noise} Otherwise ($b=1$), it performs the following steps:\label{B_steps_in_abort}
			
			\begin{itemize}
				\item Sample $k$ uniformly random indices $i_1,\ldots,i_k \la [n]$, where $k = \floor{e^{\lambda_1 \eps} \cdot \lambda_2 \cdot \ell^2}$ for constants $\lambda_1,\lambda_2>0$ (to be determined later by the analysis in \cref{eq:lamdas}).
				\item Compute $\ty = (\ty_1,\ldots,\ty_n)$ where $\ty_i \la \begin{cases} \cU_{\oo} & i \in \set{i_1,\ldots,i_k} \\ y_i & \text{otherwise} \end{cases}$,
				\item Send $\ty_{-r}= \set{\ty_i \colon r_i =0}$ to $\Ac$, and output $o_B = \perp$.
				%\item Output $o_B = \perp$.
			\end{itemize}
			
			
		\end{enumerate}
		
		\item  Denote by $\hy_{-r}$ the message $\Ac$ received from $\Bc$. Then $\Ac$ outputs $o_A =  \ip{x_{-r}, \hy_{-r}}$.
		
		
	\end{enumerate}
\end{protocol}

In the following, let $\Pi$ be \cref{protocol:DPIP-to-AWEC},
let $C =((X,U),(Y,V))$ be a uniform channel that is $(\ell,0.999)$-accurate for the \emph{inner-product} functionality, let $\tC = ((O_A,V_A),(O_B,V_B))$ be the channel that is induced by $\Pi^C$ (i.e., the parties' outputs and views in the execution of $\Pi$ with oracle access to $C$), and let $R$, $\tY$, $\hY$, $I_1,\ldots,I_k$ be the random variables of the values of $r, \ty, \hy, i_1,\ldots,i_k$ in the execution of $\Pi^C$ (recall that the view of $\Bc$ in the execution is $V_B = (Y, V, R, X_R, I_1,\ldots,I_k, \tY)$, and the view of $\Ac$ is $V_A = (X, U, R, \hY_{-R})$).

Recall that to prove \cref{lemma:DPIP-to-AWEC}, we need to prove that the channel $\tC$ satisfies the accuracy and secrecy properties of \AWEC (\cref{def:AWEC}), and in addition, the secrecy guarantees are constructive as stated in Properties \ref{item:privacy-of-Y}-\ref{item:privacy-of-X} of Lemma~\ref{lemma:DPIP-to-AWEC} (i.e., a violation of at least one secrecy guaranty results with an efficient privacy attack on the DP channel $C$).

We first start with the easy part, which is to prove the accuracy guarantee of $\tC$.

\begin{claim}[Accuracy of $\tC$]
	It holds that
	\begin{align*}
		\pr{\size{O_A - O_B} > \ell \mid O_B \neq \bot} < 0.001. 
	\end{align*}
\end{claim}
\begin{proof}
    Compute
    \begin{align*}
        \pr{\size{O_A - O_B} > \ell \mid O_B \neq \bot}
        &= \pr{\size{\ip{X_{-R},Y_{-R}} - \paren{\out(V) - \ip{X_R,Y_R}}} > \ell}\\
        &= \pr{\size{\out(V) - \ip{X,Y}} > \ell}\\
        &< 0.001,
    \end{align*}
    as required. The first equality holds since conditioned on $O_B \neq \bot$ it holds that $O_A = \ip{X_{-R},Y_{-R}}$, and the inequality holds since $C$ is an $(\ell,0.999)$-accurate for the inner-product functionality. 
\end{proof}

We next move to prove the secrecy guarantees of $\tC$ in a constructive manner. 
In \cref{sec:proving-prop1} we prove property~\ref{item:privacy-of-Y}, and in \cref{sec:proving-prop2} we prove property~\ref{item:privacy-of-X}.



%The following claim (proven in \cref{sec:proving-prop1}) captures the first secrecy guarantee where party $\Ac$ (almost) cannot distinguish if $\Bc$ aborts (i.e., $O_B = \perp$) or not, as otherwise, such a distinguisher $\Ac$ can be used to construct an efficient attack $\Act$ that violates the privacy guarantee of the DP channel $C$.

\remove{
	Finally, the following claim (proven in \cref{sec:property 2}) captures the second secrecy guarantee where party $\Bc$ cannot estimate $O_A$ too well, as otherwise, such an estimator $\Bc$ can be used to construct an efficient attack $\Bct$ that violates the privacy guarantee of the DP channel $C$.
	
	
	\begin{claim}[Property 2 of Lemma~\ref{lemma:DPIP-to-AWEC}]~\label{clm:property 2}
		There exists an oracle-aided \ppt algorithm $\Bct$ such that for every algorithm $\Bc$ violating the AWEC secrecy property~\ref{AWEC:item:B} of $\tilde{C}$, i.e.
		\begin{align*}
			\pr{\size{\Bc(V_B) - O_A} \leq 1000\ell  \mid O_B=\bot} > q,
		\end{align*}
		where $\frac{131072\ell_2^2}{q^2k}\leq\frac{1}{e^{\varepsilon}+1}$,
		the for $X_i^* = \Bct^{\Bc}(i,\: X_{-i}, \: Y, \: V)$  it holds that
		\begin{align*}
			\eex{i \la [n]}{\pr{X_i^* = X_i \mid X_i^* \neq \bot}} > \frac{e^{\eps}}{e^{\eps}+1}.
		\end{align*}
	\end{claim}
	
	\Enote{Remove:}
	
	
	\begin{proof}
		We remind that $\tilde{C}$ is an AWEC channel constructed by a differentially private inner-product protocol. Let $\mathcal{R}=\{i|r_i=0\}$ and $\cI = \set{i_1,\dots,i_k} \cap \cR$ \remove{$\mathcal{I}=\{i|i\in\{i_1,\dots,i_k\},i\in \mathcal{R}\}$}. Then $O_A=o-\ip{x_{\mathcal{R}\setminus \mathcal{I}},\hy_{\mathcal{R}\setminus \mathcal{I}}}-\ip{x_{ \mathcal{I}},\hy_{ \mathcal{I}}}$. It suffices to prove party $\Bc$ cannot predict $\ip{x_{ \mathcal{I}},\hy_{ \mathcal{I}}}$ with high accuracy. We put the details in Section~\ref{sec:property 2}  
	\end{proof}
}




\subsection{B's Security: Proving Property~\ref{item:privacy-of-Y} of \cref{lemma:DPIP-to-AWEC}}\label{sec:proving-prop1}

Let $\Ac$ be an algorithm that violates the AWEC secrecy property~\ref{AWEC:item:A} of $\tilde{C}$ (the channel of $\Pi^C$). Namely,

\begin{align*}
	\size{\pr{\Ac(V_A) = 1 \mid O_B \neq \perp} - \pr{\Ac(V_A) = 1 \mid O_B = \perp}} > \frac1{1000}.
\end{align*}

%for $k \in \Theta\paren{\frac{n^{1/4} \paren{\frac{1}{e^{\eps}+1} - \delta}}{\log^{1/4} n}}$ and $\gamma = \frac1{100 k}$.
Recall that $V_A = (X,U,R,\hY_{-R})$ where $\hY = \begin{cases} Y & O_B \neq \perp \\ \tY & O_B = \perp\end{cases}$ and that $\tY$ is obtained from $Y$ by planting uniformly random $\oo$ values in the random locations $I_1,\ldots, I_k$ of $Y$ (the value of $k$ will be determined later by the analysis).

Therefore, the above inequality is equivalent to 

\begin{align*}%\label{eq:A-pr}
	\size{\pr{\Ac(X,U,R,Y_{-R}) = 1} - \pr{\Ac(X,U,R,\tY_{-R}) = 1}} > \frac1{1000}.
\end{align*}

We assume \wlg that $\Ac$ outputs a $\zo$ bit, so the above inequality can be written as

\begin{align}\label{eq:property1:assump}
	\size{\ex{\Ac(X,U,R,Y_{-R}) - \Ac(X,U,R,\tY_{-R})}} > \frac1{1000}.
\end{align}


In the following, for $j \in \set{0,\ldots,k}$, let $\tY^{j} = (\tY^{j}_1,\ldots, \tY^{j}_n)$ where $\tY^{j}_i = \begin{cases} \tY_i & i \in \set{I_1,\ldots,I_j} \\ Y_i & i \notin \set{I_1,\ldots,I_j}\end{cases}$.
Note that $\tY^{0} = Y$ and $\tY^{k} = \tY$. Therefore,

\begin{align}\label{eq:property1:hybrid}
    \lefteqn{\size{\eex{j \la [k]}{\ex{\Ac(X,U,R,Y^{(j-1)}_{-R}) - \Ac(X,U,R,\tY^{(j)}_{-R})}}}}\\
    &= \frac1k \cdot \size{\sum_{j=1}^k \ex{\Ac(X,U,R,Y^{(j-1)}_{-R})} - \ex{\Ac(X,U,R,Y^{(j)}_{-R})}}\nonumber\\
    &= \frac1k \cdot \size{\ex{\Ac(X,U,R,Y_{-R}) - \Ac(X,U,R,\tY_{-R})}}\nonumber\\
    &> \frac1{1000 k},\nonumber
\end{align}
where the inequality holds by \cref{eq:property1:assump}.

In the following, let $J \la [n]$ (sampled independently of the other random variables defined above), let $Z = (Z_1,\ldots,Z_n) = \tY^{J-1}$ and let $Z^{(i)} = (Z_1,\ldots,Z_{i-1}, -Z_i, Z_{i+1},\ldots, Z_n)$. It holds that

\begin{align}\label{eq:property1:exp}
	\lefteqn{\size{\ex{\Ac(X,U,R,Z_{-R})} - \ex{\Ac(X,U,R,Z^{(I_{J})}_{-R})}}}\\
        &= \frac12 \cdot \size{\ex{\Ac(X,U,R,Y^{(J-1)}_{-R}) - \Ac(X,U,R,\tY^{(J)}_{-R})}}\nonumber\\
        &= \frac12\cdot \size{\eex{j \la [k]}{\ex{\Ac(X,U,R,Y^{(j-1)}_{-R}) - \Ac(X,U,R,\tY^{(j)}_{-R})}}}\nonumber\\
        &> \frac1{2000 k},\nonumber
\end{align}
where the first equality holds since $\tY^{J}$ is equal to $Z$ w.p. $1/2$ (happens when $\tY_{I_{J}} = Y_{I_{J}}$)  and otherwise is equal to $Z^{(I_{J})}$ (happens when $\tY_{I_{J}} \neq Y_{I_{J}}$), and the last inequality holds by \cref{eq:property1:hybrid}.

%For the ease of notation, since $R \la \oo^n$, by symmetry we can assume \wlg that 
%\begin{align}\label{eq:property1:exp}
%	\size{\ex{\Ac(X,U,R,Z_{R})} - \ex{\Ac(X,U,R,Z^{(I_{J})}_{R})}} > \frac1{2000 k}.
%\end{align}



We next use the following lemma, proven in \cref{sec:prediction-lemma} (recall that a proof overview appears in \cref{sec:overview:prediction-lemma}).

\def\PredictionLemma{
    For every $\gamma \in (0,1)$ and $n \in \bbN$, there exists an oracle aided (randomized) algorithm $\Gc = \Gc_{\gamma} \colon [n] \times \zo^{n-1} \rightarrow \set{-1,1,\perp}$ that runs in time $\poly(n,1/\gamma)$ such that the following holds: 
	
	Let $(Z,W) \in \oo^n \times \zo^*$ be jointly distributed random variables, let $R \la \zo^n$ and $I \la [n]$ (sampled independently), 
	and let $F$ be a (randomized) function that satisfies
	\begin{align*}
		\size{\ex{F(R,Z_{R},W) - F(R,Z^{(I)}_{R},W)}} \geq \gamma,
	\end{align*}
	for $Z^{(I)} = (Z_1,\ldots, Z_{I-1}, -Z_I, Z_{I+1},\ldots, Z_n)$. Then
	\begin{enumerate}
		\item $\pr{\Gc^F(I,Z_{-I}, W) = -Z_I} \leq O\paren{\frac{1}{\gamma^2 n}}$, and\label{property1:prediction:bad}
		
		\item $\pr{\Gc^F(I,Z_{-I}, W) = Z_I} \geq \Omega(\gamma) - O\paren{\frac{1}{\gamma^2 n}}$.\label{property1:prediction:good}
	\end{enumerate}
}

\begin{lemma}\label{lemma:property1:prediction}
    \PredictionLemma
\end{lemma}

We now ready to finalize the proof of Property~\ref{item:privacy-of-Y} of \cref{lemma:DPIP-to-AWEC} using \cref{eq:property1:exp,lemma:property1:prediction}.

\begin{proof}
	
	Consider the following oracle-aided algorithm $\Act$:
	
	\begin{algorithm}[Algorithm $\Act$]\label{alg:Act}
		\item Inputs: $i \in [n]$, $y_{-i} \in \oo^{n-1}$, $x \in \oo^n$ and $u \in \zo^*$.
		\item Oracle: Deterministic algorithm $\Ac$.%and $\Gc$ from \cref{lemma:property1:prediction}.
		\item Operation:~
		\begin{enumerate}
			\item Sample $j \la [k]$ and $i_1,\ldots,i_{j-1} \la [n]$.
			
			\item Sample $z_{-i} = (z_1,\ldots,z_{i-1}, z_{i+1},\ldots, z_n)$, where for $t \in [n]\setminus \set{i}$: $$z_{t} \la \begin{cases} \cU_{\oo} & t \in \set{i_1,\ldots,i_{j-1}} \\ y_t & \text{otherwise} \end{cases}.$$\label{step:z-i}
			
			\item Output $y_i^* \la \Gc^{F}(i,z_{-i},w)$ for $w=(x,u)$, where $\Gc = \Gc_{\frac1{2000 k}}$ is the algorithm from \cref{lemma:property1:prediction}, and $F(r,z_r,w) \eqdef \Ac(w,(1-r_1,\ldots,1-r_n), z_r)$.
		\end{enumerate}
	\end{algorithm}

	Note that by \cref{eq:property1:exp} it holds that
	\begin{align*}
		\size{\ex{F(R,Z_{R}, W)} - \ex{F(R,Z_{R}^{I},W)}} > \frac1{2000 k}
	\end{align*}
	for $W = (X,U)$, $I = I_J$ and the function $F(r,z_r,w=(x,u)) \eqdef \Ac(w,(1-r_1,\ldots,1-r_n), z_r)$. Thus, \cref{lemma:property1:prediction} implies that
	\begin{align}\label{eq:G:LB}
		\pr{\Gc^{F}(I,Z_{-I}, V) = -Z_I} \leq O\paren{\frac{k^2}{n}}
	\end{align}
	and
	\begin{align}\label{eq:G:UB}
		\pr{\Gc^{F}(I,Z_{-I}, V) = Z_I} \geq \Omega(1/k) - O\paren{\frac{k^2}{n}}.
	\end{align} 
	
	Recall that we denote $Y_i^* = \Act^{\Ac}(i,Y_{-i},X,U)$. We next lower-bound $\eex{i \la [n]}{\pr{Y_i^* = Y_i}}$ and upper-bound $\eex{i \la [n]}{\pr{Y_i^* = -Y_i}}$.
	The first bound hold by the following calculation
	\begin{align}\label{eq:Y_i*-LB}
		\eex{i \la [n]}{\pr{Y_i^* = Y_i}}
		&\geq 0.9 \cdot \eex{i \la [n]}{\pr{Y_i^* = Z_i}}\\
		&= 0.9 \cdot \eex{i \la [n]}{\pr{\Gc^{F}(i,Z_{-i},W) = Z_i}}\nonumber\\
		&\geq \Omega\paren{\frac1k} - O\paren{\frac{k^2}{n}}.\nonumber
	\end{align}
	The first inequality holds since $i \notin \set{I_1,\ldots,I_{J-1}} \implies Z_i = Y_i$ and $\pr{i \notin \set{I_1,\ldots,I_{J-1}}} \geq \paren{1 - \frac1n}^{k-1} \geq 0.9$ (recall that $k \in o(n)$). The equality holds since, conditioned on $X=x,U=u,Y_{-i} = y_{-i}$ (the inputs of $\Act$), the value of $z_{-i}$ that is sampled in \stepref{step:z-i} of $\Act$ is distributed the same as $Z_{-i}|_{X=x,U=u,Y_{-i} = y_{-i}}$, and therefore, $Y_i^* \equiv \Gc^{F}(i,Z_{-i},W)$ for $W = (X,U)$. The last inequality holds by \cref{eq:G:UB}.
	
	On the other hand, we have that
	\begin{align}\label{eq:Y_i*-UB}
		\eex{i \la [n]}{\pr{Y_i^* = -Y_i}}
		&\leq \eex{i \la [n]}{\pr{Y_i^* = - Y_i \mid Y_i = Z_i} + \pr{Z_i \neq Y_i}}\\
		&\leq \eex{i \la [n]}{\pr{Y_i^* = - Z_i}} + \frac{k}{2n}\nonumber\\
		&=  \eex{i \la [n]}{\pr{\Gc^{F}(i,Z_{-i}) = - Z_i}} + \frac{k}{2n}\nonumber\\
		&\leq O\paren{\frac{k^2}{n}}.\nonumber
	\end{align}
	The second inequality holds since for any fixing $Y_{-i}=y_{-i},X=x,U=u$, we have $Y_i^*= \Act^{\Ac}(i,y_{-i},x,u)$ which is independent of $(Y_i,Z_i)$, and also since $i \notin \set{I_1,\ldots,I_{J-1}} \implies Z_i = Y_i$ which yields that
	\begin{align*}
		\pr{Z_i \neq Y_i} \leq 1 - \pr{i \notin \set{I_1,\ldots,I_{J-1}}} \leq 1 - \paren{1 - \frac1n}^{k} \leq 1 - e^{\frac{k}{2n}} \leq \frac{k}{2n}. 
	\end{align*}
	The equality in \cref{eq:Y_i*-UB} holds since, conditioned on $X=x,U=u,Y_{-i} = y_{-i}$ (the inputs of $\Act$), the value of $z_{-i}$ that is sampled in \stepref{step:z-i} of $\Act$ is distributed the same as $Z_{-i}|_{X=x,U=u,Y_{-i} = y_{-i}}$, and therefore, $Y_i^* \equiv \Gc^{F}(i,Z_{-i},W)$ for $W = (X,U)$. The last inequality in \cref{eq:Y_i*-UB} holds by \cref{eq:G:LB}.
	
	By \cref{eq:Y_i*-LB,eq:Y_i*-UB}, assuming that $\delta \leq 1/n$ where $n$ is large enough, 
	there exists a constant $c > 0$ such that if $k \leq c \cdot (e^{-\eps} n)^{1/3}$ then $\eex{i \la [n]}{\pr{Y_i^* = Y_i}} > e^{\eps} \cdot \eex{i \la [n]}{\pr{Y_i^* \neq Y_i}} + \delta$, as required.
	Recall that $k = \ceil{e^{\lambda_1 \eps} \lambda_2 \cdot \ell^2}$, where $\lambda_1,\lambda_2$ are the constants from \cref{eq:lamdas}. Hence, we can set a bound of $e^{-c_1 \eps} c_2 \cdot n^{1/6}$ on $\ell$ for $c_1 = \lambda_1/2 + 1/6$ and $c_2 = \sqrt{c/\lambda_2}$ to guarantee that $k \leq c \cdot (e^{-\eps} n)^{1/3}$.
	
\end{proof}

\remove{
\subsubsection{Proving \cref{lemma:property1:prediction}}\label{sec:prediction-lemma}
\begin{proof}
	
	Let $0 < \gamma \leq 0.01$ and consider the following algorithm $\Gc = \Gc_{\gamma}$:
	\begin{algorithm}
		\item Inputs: $i \in [n]$,  $z_{-i} = (z_1,\ldots,z_{i-1},z_{i+1},\ldots,z_n) \in \oo^{n-1}$ and $w \in \zo^*$.
		\item Parameter: $0 < \gamma \leq 0.01$.
		\item Oracle: $f \colon \zo^n \times \oo^{\leq n} \times \zo^* \rightarrow \zo$.
		\item Operation:~
		\begin{enumerate}
			\item For $b \in \oo$:
			\begin{enumerate}
				\item Let $z^b = (z_1,\ldots,z_{i-1},b,z_{i+1},\ldots,z_n)$.
				\item Estimate $\mu^b \eqdef \eex{r \la \zo^n \mid r_i = 1, \: f}{f(r,z^b_{r}, w)}$ as follows:
				\begin{itemize}
					\item Sample $r_1,\ldots,r_{s} \la \set{r \in \zo \colon r_i = 1}$, for $s = \ceil{\frac{128 \log (6n)}{\gamma^2}}$, and then sample $\tilde{\mu}^b \la \frac1{s} \sum_{j=1}^s f(r_j,z^b_{r_j}, w)$ (using $s$ calls to the oracle $f$).
				\end{itemize}
			\end{enumerate}
			\item Estimate $\mu^* \eqdef \eex{r \la \zo^n \mid r_i = 0, \: f}{f(r,z^1_{r}, w)}$ as follows:
			\begin{itemize}
				\item Sample $r_1,\ldots,r_{s}  \la \set{r \in \zo \colon r_i = 0}$, for $s = \ceil{\frac{128 \log (6n)}{\gamma^2}}$, and then sample $\tilde{\mu}^* \la \frac1{s} \sum_{j=1}^s f(r_j,z^1_{r_j}, w)$ (using $s$ calls to the oracle $f$).
			\end{itemize}
			\item If exists $b \in \oo$ s.t. $\size{\tilde{\mu}^b - \tilde{\mu}^*} < \gamma/4$ and $\size{\tilde{\mu}^{-b} - \tilde{\mu}^*} > \gamma/4$, output $b$.
			\item Otherwise, output $\bot$.
		\end{enumerate}
	\end{algorithm}
	In the following, fix a pair of (jointly distributed) random variables $(Z,W) \in \oo^n \times \zo^*$ and a randomized function  $f \colon \zo^n \times \oo^{\leq n} \times \zo^* \rightarrow \oo$ that satisfy 
	\begin{align*}
		\size{\ex{f(R,Z_{R},W) - f(R,Z^{(I)}_{R},W)}} \geq \gamma,
	\end{align*}
	for $R \la \zo^n$ and $I \la [n]$ that are sampled independently. 
	Our goal is to prove that 
	
	\begin{align}\label{eq:predic-lemma:UB}
		\pr{\Gc^f(I,Z_{-I}, W) = -Z_I} \leq O\paren{\frac{1}{\gamma^2 n}},
	\end{align}
	and 
	\begin{align}\label{eq:predic-lemma:LB}
		\pr{\Gc^f(I,Z_{-I}, W) = Z_I} \geq \Omega(\gamma) - O\paren{\frac{1}{\gamma^2 n}}.
	\end{align}

	Note that
	\begin{align}\label{eq:D}
		\text{For every random variable }D \in [-1,1]\text{ with }\size{\ex{D}} \geq \gamma >0: \quad \pr{\size{D} > \gamma/2} > \gamma/2,
	\end{align}
	as otherwise, $\size{\ex{D}} \leq \ex{\size{D}} \leq 1\cdot \frac{\gamma}{2} + \frac{\gamma}{2}(1-\frac{\gamma}{2}) < \gamma$.
	
	
	By applying \cref{eq:D} with $D = \eex{r \la \zo^n, f}{f(r,Z_{r},W) - f(r,Z^{(I)}_{r},W)}$, it holds that
	\begin{align}\label{eq:main-lemma:good}
		\ppr{(i,z,w) \la (I,Z,W)}{\size{\eex{r \la \zo^n, f}{f(r,z_{r},w) - f(r,z^{(i)}_{r},w)}} > \gamma/2} > \gamma/2.
	\end{align}
	On the other hand, for every fixing of $(z,w) \in \Supp(Z,W)$, we can apply \cref{lem:distance-I} with the function $f_{z,w}(r) = f(r,z_{r},w)$ and with $\alpha =\frac{\gamma}{16}$ to obtain that
	\begin{align*}
		\ppr{i\la I}{\: \size{\eex{r \la \zo^n \mid r_i = 0, \: f}{f(r,z_{r},w)} - \eex{r \la \zo^n \mid r_i = 1, \: f}{f(r,z_{r},w)} \:} \geq \frac{\gamma}{16}} \leq \frac{512}{n \gamma^2}.
	\end{align*}
	But since the above holds for every fixing of $(z,w)$, then in particular it holds that
	\begin{align}\label{eq:main-lemma:bad}
		\ppr{(i,z,w) \la (I,Z,W)}{\: \size{\eex{r \la \zo^n \mid r_i = 0, \: f}{f(r,z_{r},w)} - \eex{r \la \zo^n \mid r_i = 1, \: f}{f(r,z_{r},w)} \:} \geq \frac{\gamma}{16}} \leq \frac{512}{n \gamma^2}.
	\end{align}
	
	We next prove \cref{eq:predic-lemma:UB,eq:predic-lemma:LB} using \cref{eq:main-lemma:good,eq:main-lemma:bad}.
	
	In the following, for a triplet $t = (i,z,w) \in \Supp(I,Z,W)$, consider a random execution of $\Gc^f(i,z_{-i},w)$. For $x \in \set{-1,1,*}$, let $\mu^x_t$ be the value of $\mu^x$ in the execution, and let $M^x_t$ be the (random variable of the) value of $\tilde{\mu}^x$ in the execution. Note that by definition, it holds that $\mu_{i,z,w}^{z_i} = \eex{r \la \zo^n, f}{f(r,z_{r},w)}$ and $\mu_{i,z,w}^{-z_i} = \eex{r \la \zo^n, f}{f(r,z^{(i)}_{r},w)}$. Therefore, \cref{eq:main-lemma:good} is equivalent to 
	\begin{align}\label{eq:main-lemma:good2}
		\ppr{t=(i,z,w) \la (I,Z,W)}{\size{\mu_t^{z_i} - \mu_t^{-z_i}} > \gamma/2} > \gamma/2.
	\end{align}
	Furthermore, note that $\mu_{i,z,w}^* \eqdef \eex{r \la \zo^n \mid r_i = 0, \: f}{f(r,z^1_{r}, w)} = \eex{r \la \zo^n \mid r_i = 0, \: f}{f(r,z_{r}, w)}$ and that $\mu_{i,z,w}^{z_i} = \eex{r \la \zo^n \mid r_i = 1, \: f}{f(r,z^{z_i}_{r}, w)}$. Therefore, \cref{eq:main-lemma:bad} is equivalent to 
	\begin{align}\label{eq:main-lemma:bad2}
		\ppr{t = (i,z,w) \la (I,Z,W)}{\: \size{\mu_t^* - \mu_t^{z_i}}\geq \frac{\gamma}{16}}  \leq \frac{512}{n \gamma^2}.
	\end{align}
	
	We next prove the lemma using \cref{eq:main-lemma:good2,eq:main-lemma:bad2}.
	
	Note that by Hoeffding's inequality, for every $t = (i,z,w) \in \Supp(I,Z,W)$ and  $x \in \set{-1,1,*}$ it holds that $\pr{\size{M_t^x - \mu_t^x} \geq \frac{\gamma}{16}} \leq 2\cdot e^{-2 s \paren{\frac{\gamma}{16}}^2} \leq \frac1{3n}$, which yields that for every fixing of $t = (i,z,w) \in \Supp(I,Z,W)$, w.p.\ at least $1-1/n$ we have for all $x \in \set{-1,1,*}$ that $\size{M_t^x - \mu_t^x} < \frac{\gamma}{16}$ (denote this event by $E_t$).
	
	The proof of \cref{eq:predic-lemma:UB} holds by the following calculation:
	\begin{align*}
		\lefteqn{\pr{\Gc^f(I,Z_{-I},W) = -Z_I}}\\
		&= \eex{(i,z,w) \la (I,Z,W)}{\pr{\Gc^f(i,z_{-i},w) = -z_i}}\\
		&= \eex{t = (i,z,w) \la (I,Z,W)}{\pr{\set{\size{M_t^* - M_t^{z_i}} > \gamma/4} \land \set{\size{M_t^* - M_t^{-z_i}} < \gamma/4}}}\\
		&\leq \eex{t =(i,z,w) \la (I,Z,W)}{\pr{\size{M_t^* - M_t^{z_i}} > \gamma/4}}\\
		&\leq \eex{t = (i,z,w) \la (I,Z,W)}{\pr{\size{M_t^* - M_t^{z_i}} > \gamma/4 \mid E_t}} + \frac{1}{n}\\
		&\leq \ppr{t = (i,z,w) \la (I,Z,W)}{\size{\mu_t^* -  \mu_t^{z_i}} \geq \frac{\gamma}{16}} + \frac{1}{n}\\
		&\leq \frac{512}{n \gamma^2} + \frac1n,
	\end{align*}
	The second inequality holds since $\pr{\neg E_t} \leq 1/n$ for every $t$. The penultimate inequality holds since conditioned on $E_t$, it holds that $\size{M_t^* - \mu_t^*} \leq \frac{\gamma}{16}$ and $\size{M_t^{z_i} - \mu_t^{z_i}} \leq \frac{\gamma}{16}$, which implies that $\size{M_t^* - M_t^{z_i}} > \gamma/4 \: \implies \: \size{\mu_t^* -  \mu_t^{z_i}} \geq \frac{\gamma}{4} - 2\cdot \frac{\gamma}{16} > \frac{\gamma}{16}$. The last inequality holds by \cref{eq:main-lemma:bad2}.
	
	It is left to prove \cref{eq:predic-lemma:LB}. 
	Observe that \cref{eq:main-lemma:good2,eq:main-lemma:bad2} imply with probability at least $\gamma/2 - \frac{512}{n \gamma^2}$ over $t = (i,z,w)\in (I,Z,W)$, we have $\size{\mu_t^* -  \mu_t^{z_i}} \leq  \frac{\gamma}{16}$ and $\size{\mu_t^{z_i}- \mu_t^{-z_i}} \geq\frac{\gamma}{2}$. Therefore, conditioned on event $E_t$ (i.e., $\forall x \in \set{-1,1,*}: \: \size{M_t^{x} - \mu_t^{x}} \leq \frac{\gamma}{16}$), we have for such triplets $t = (i,z,w)$ that 
	\begin{align*}
		\size{M_t^* - M_t^{-z_i}} 
		& \geq \size{\mu_t^{z_i} - \mu_t^{-z_i}} - \size{\mu_t^{z_i} - \mu_t^*} - \size{M_t^*  - \mu_t^*} -  \size{M_t^{-z_i} - \mu_t^{-z_i}}\\
		&\geq \frac{\gamma}{2} - 3\cdot \frac{\gamma}{16}\\
		&> \gamma/4,
	\end{align*}
	and 
	\begin{align*}
		\size{M_t^* - M_t^{z_i}}
		&\leq  \size{M_t^* - \mu_t^*} + \size{\mu_t^* - \mu_t^{z_i}} + \size{\mu_t^{z_i} - M_t^{z_i}} \\
		&\leq 3\cdot \frac{\gamma}{16}\\
		&< \gamma/4.
	\end{align*}
	
	Thus, we conclude that
	\begin{align*}
		\lefteqn{\pr{\Gc^f(I,Z_{-I},W) = Z_I}}\\
		&= \eex{(i,z,w) \la (I,Z,W)}{\pr{\Gc^f(i,z_{-i},w) = z_i}}\\
		&= \eex{t = (i,z,w) \la (I,Z,W)}{\pr{\set{\size{M_t^* - M_t^{-z_i}} > \gamma/4} \land \set{\size{M_t^* - M_t^{z_i}} < \gamma/4}}}\\
		&\geq \paren{1 - \frac1{n}}\cdot \eex{t = (i,z,w) \la (I,Z,W)}{\pr{\set{\size{M_t^* - M_t^{-z_i}} > \gamma/4} \land \set{\size{M_t^* - M_t^{z_i}} < \gamma/4} \mid E_t}}\\
		&\geq \paren{1 - \frac1{n}}\cdot \paren{\gamma/2 - \frac{512}{n \gamma^2}}\\
		&\geq \gamma/4 - \frac{1024}{n \gamma^2},
	\end{align*}
	which proves \cref{eq:predic-lemma:LB}. The first inequality holds since $\pr{E_t} \geq 1-1/n$ for every $t = (i,z,w)$, and the second one holds by the observation above.
	
\end{proof}
}

\subsection{A's Security: Proving Property~\ref{item:privacy-of-X} of \cref{lemma:DPIP-to-AWEC}}\label{sec:proving-prop2}

Let $\Bc$ be an algorithm that violates the AWEC secrecy property~\ref{AWEC:item:B} of $\tilde{C} = ((V_A,O_A),(V_B,O_B))$ --- the channel of $\Pi^{C = ((X,U),(Y,V))}$ (\cref{protocol:DPIP-to-AWEC}). Namely,

\begin{align}\label{eq:violating-B}
	\pr{\size{\Bc(V_B) - O_A} \leq 1000 \ell \mid O_B=\bot} > \frac1{1000}.
\end{align}

Recall that $V_B = (Y,V,R,X_{R},I_1,\ldots,I_k, \tY)$ where $I_1,\ldots,I_k \la [n]$ are the indices that $\Bc$ chooses at Step~\ref{B_steps_in_abort}, and $\tY = (\tY_1,\ldots,\tY_n)$ where $\tY_i \la \oo$ for $i \in \set{I_1,\ldots,I_k}$ and otherwise $\tY_i = Y_i$. Furthermore, conditioned on $O_B=\bot$, recall that $O_A = \ip{X_{-R}, \tY_{-R}}$. Therefore, \cref{eq:violating-B} is equivalent to 
\begin{align}\label{eq:Bc-guarantee}
	%\pr{\size{\Bc(Y,V,R,X_{R},I_1,\ldots,I_k, \tY) - \ip{X_{-R}, \tY_{-R}}} \leq 1000 \ell } > \frac1{100}.
	\pr{\size{\Bc(V_B) - \ip{X_{-R}, \tY_{-R}}} \leq 1000 \ell } > \frac1{1000}.
\end{align}


In the following, define the random variable $H$ to be the first $m = \ceil{k/4}$ indices of $R^0 \cap \set{I_1,\ldots,I_k}$ for $R^0 = \set{i \colon R_i = 0}$, where we let $H = \emptyset$ if the size of the intersection is smaller than $m$.
Since $R \la \zo^n$, Hoeffding's inequality implies that $\pr{\size{R^0} \geq 0.4 n} \geq 1 - e^{-\Omega(n)}$. Since $I_1,\ldots,I_k \la [n]$ (independent of $R$), then again by Hoeffding's inequality we obtain that $\pr{\size{H} = \ceil{k/4} \mid \size{R^0} \geq 0.4 n} \geq 1 - e^{-\Omega(k)}$, which yields that
\begin{align*}
	\pr{H \neq \emptyset} = \pr{\size{H} = \ceil{k/4}} \geq 1 - e^{-\Omega(k)} - e^{-\Omega(n)} \geq 1-\frac1{10000}.
\end{align*}
Therefore, by the union bound,
\begin{align}\label{eq:good-and-not-empty-H}
	\lefteqn{\pr{\set{\size{\Bc(V_B) - \ip{X_{-R}, \tY_{-R}}} \leq 1000 \ell} \land \set{H \neq \emptyset}}}\\
	&= 1- \pr{\set{\size{\Bc(V_B) - \ip{X_{-R}, \tY_{-R}}} > 1000 \ell} \lor \set{H = \emptyset}}\nonumber\\
	&\geq 1- \pr{\set{\size{\Bc(V_B) - \ip{X_{-R}, \tY_{-R}}} > 1000 \ell}} - \pr{ \set{H = \emptyset}}\nonumber\\
    &\geq 1- \paren{1- \frac1{1000}} - \frac1{10000} \nonumber\\
     &\geq \frac1{2000}.\nonumber
\end{align}
%\begin{align}\label{eq:good-and-not-empty-H}
%	\lefteqn{\pr{\set{\size{\Bc(V_B) - \ip{X_{-R}, \tY_{-R}}} \leq 1000 \ell} \land \set{H \neq \emptyset}}}\\
%	&= \pr{H \neq \emptyset} \cdot \pr{\size{\Bc(V_B) - \ip{X_{-R}, \tY_{-R}}} \leq 1000 \ell \mid H \neq \emptyset }\nonumber\\
%	&\geq \pr{H \neq \emptyset}\cdot \paren{\pr{\size{\Bc(V_B) - \ip{X_{-R}, \tY_{-R}}} \leq 1000 \ell} - \pr{H = \emptyset}}\nonumber\\
%	&\geq \paren{1-\frac1{10000}}\cdot \paren{\frac1{1000} - \frac1{10000}}\nonumber\\
%	&> \frac1{2000}.\nonumber
%\end{align}


In the following, let $d$ be the number of random coins that $\Bc$ uses, and for $\psi \in \zo^d$ let $\Bc_{\psi}$ be algorithm $\Bc$ when fixing its random coins to $\psi$.
Let $\Psi \la \zo^d$,  $Z = X_H$,  $T' = (Y,V,R,I_1,\ldots,I_k, H,\tY_{-\cH}, X_{-\cH})$,  $T= (\Psi, T')$ and $S = \tY_H$. Note  that conditioned on $H \neq \perp$, $S$ is a uniformly random string in $\oo^m$, independent of $Z$ and $T$, and note that $V_B$ is a deterministic function of $(\tY_{H}, T)$ (because $X_R$, which is part of $V_B$, is a sub-string of $X_{-\cH}$).
\cref{eq:good-and-not-empty-H} yields that w.p. $1/2000$ over $z \la Z$,  $t = (\psi, t'=(y,v,r,i_1,\ldots,i_k,\cH,x_{-\cH},\ty_{-\cH})) \la T$ and $s \la \oo^m$,
the following holds for $\bar{\cH} =  \set{i \in [n] \colon r_i = 0} \setminus \cH\:$:
\begin{align*}
	\size{\Bc_{\psi}(s,t') - \ip{x_{\bar{\cH}}, \ty_{\bar{\cH}}} - \ip{z, s}}\leq 1000 \ell.
\end{align*}

By denoting $f(s,t=(\psi,t')) = \Bc_{\psi}(s,t') + \ip{x_{\bar{\cH}}, \ty_{\bar{\cH}}}$,\footnote{Note that $f(s,t)$ is well-defined because $t$ contains $x_{\bar{\cH}}$ and $\ty_{\bar{\cH}}$ (sub-strings of $x_{-\cH}$ and $\ty_{-\cH}$, respectively).} the above observation is equivalent to

\begin{align}\label{eq:our-good-f}
	\ppr{(z,t) \la (Z,T), \: s \la \oo^m}{\size{f(s,t) - \ip{z, s}} \leq 1000 \ell} \geq \frac1{2000}.
\end{align}

In the following, let $\cD$ be the joint distribution of $(Z,T)$, which is equivalent to the output of $\GenView^{C}()$ defined below in \cref{alg:GenView}.

\begin{algorithm}[$\GenRand$]\label{alg:GenRand}
	~
	\begin{enumerate}
            \item Sample $\psi \la \zo^d$.
		\item Sample $r \la \zo^n$ and $i_1,\ldots,i_k \la [n]$.
		\item Let $\cH$ be the first $m=\ceil{k/4}$ indices of $\set{i \in [n] \colon r_i = 0} \cap  \set{i_1,\ldots,i_k}$, where $\cH = \emptyset$ if the intersection size is less than $m$.
		\item Let $\bar{\cH} = \set{i \in [n] \colon r_i = 0} \setminus \cH$.
		\item Sample $\ty_{\bar{\cH}} \la \oo^{\size{\bar{\cH}}}$.
		\item Output $(\psi,r,i_1,\ldots,i_k,\cH,\ty_{\bar{\cH}})$.
	\end{enumerate}
\end{algorithm}



\begin{algorithm}[$\GenView$]\label{alg:GenView}
	\item Oracle: A channel $C = ((X,U),(Y,V))$.
	\item Operation:~
	\begin{enumerate}
		\item Sample $((x,u),(y,v)) \la C$.
		\item Sample $(\psi,r,i_1,\ldots,i_k,\cH,\ty_{\bar{\cH}}) \la \GenRand()$ (\cref{alg:GenRand}).
		\item Output $z = x_{\cH}$ and $t = (\psi, t')$ for $t'=(y,v,r,i_1,\ldots,i_k,\cH,\ty_{-\cH},x_{-\cH})$, where $\ty_i = y_i$ for $i \in [n]\setminus (\cH \cup \bar{\cH})$.
	\end{enumerate}
\end{algorithm}



We now can use the following reconstruction result from \cite{HaitnerMST22}:


\begin{fact}[Follows by Theorem 4.6 in \cite{HaitnerMST22}]\label{fact:prev-rec}
	There exists constants $\eta_1,\eta_2 > 0$ and a \ppt algorithm $\Dist$ such that the following holds for large enough $m \in \bbN$: Let $\eps \geq 0$ and $a \geq \log m$, and let $\cD$ be a distribution over $\oo^m \times \Sigma^*$. Then for every function $f$ that satisfies 
	\begin{align*}
		\ppr{(z,t) \la \cD, \: s \la \oo^m}{f(s,t) - \ip{z,s} \leq a} \geq e^{\eta_1 \cdot \eps}\cdot \eta_2 \cdot a/\sqrt{m},
	\end{align*}
	it holds that
	\begin{align*}
		\ppr{(z,t) \la \cD, \: j \la [m]}{\Dist^{\cD,f}(j, z, t) = 1} > e^{\eps}\cdot \ppr{(z,t) \la \cD, \: j \la [m]}{\Dist^{\cD,f}(j ,z^{(j)}, t) = 1} + \frac1m,
	\end{align*}
	where $z^{(j)} = (z_1,\ldots,z_{j-1},-z_j,z_{j+1},\ldots,z_m)$.\footnote{Theorem 4.6 in \cite{HaitnerMST22} actually considered a harder setting where $z$ is the coordinate-wise product of two vectors $x,y \in \oo^n$, and $f$ only guarantees accuracy when in addition to $s$ and $t$, it also gets $x_{s} = \set{x_i \colon s_i = 1}$ and $y_{-s} = \set{y_i \colon s_i = -1}$ as inputs.}
	%\Enote{explain the differences from Theorem 4.6 in \cite{HaitnerMST22}.}%\Nnote{Maybe for after the deadline: I guess we don't really need this theorem and can use the proof of the easy case. Maybe we want to write the simple proof}
\end{fact}



%Note that the output of the following $\GenView^{C}()$ (\cref{alg:GenView}) is distributed the same as the joint distribution of $(Z,T)$. 
Now, we would like to apply \cref{fact:prev-rec} with $\cD = \GenView^C()$ and $a = 1000 \ell$. To do that, \cref{fact:prev-rec} yields that we need to choose $k$ such that $\frac{e^{\eta_1 \cdot \eps}\cdot \eta_2  \cdot 1000\ell}{\ceil{k/4}} \leq \frac1{2000}$, which holds by choosing $k = \floor{e^{\lambda_1 \eps}\cdot \lambda_2 \cdot \ell^2}$ with 
\begin{align}\label{eq:lamdas}
	\lambda_1 = \eta_1 \text{ and }\lambda_2 = 10^7 \eta_2 + 1,
\end{align}
where $\eta_1,\eta_2$ are the constants from \cref{fact:prev-rec}.

We deduce from \cref{fact:prev-rec,eq:our-good-f} that 

\begin{align}\label{eq:Dist-gap}
	\ppr{(z,t) \la \cD, \: j \la [m]}{\Dist^{\cD,f}(j, z, t) = 1} > e^{\eps}\cdot \ppr{(z,t) \la \cD, \: j \la [m]}{\Dist^{\cD,f}(j ,z^{(j)}, t) = 1} + \frac1m.
\end{align}


We now ready to describe our algorithm $\Bct$ that satisfies Property~\ref{item:privacy-of-X} of \cref{lemma:DPIP-to-AWEC}.


\begin{algorithm}[Algorithm $\Bct$]\label{alg:Bct}
	\item Inputs: $i \in [n]$, $x_{-i} = (x_1,\ldots,x_{i-1},x_{i+1},\ldots,x_n) \in \oo^{n-1}$, $y \in \oo^n$ and $v \in \zo^*$.
	\item Oracles: A channel $C = ((X,U),(Y,V))$ and an algorithm $\Bc$.
	\item Operation:~
	\begin{enumerate}
		\item Sample $(\psi, r,i_1,\ldots,i_k,\cH,\ty_{\bar{\cH}}) \la \GenRand()$ (\cref{alg:GenRand}).
		\item If $i \notin \cH$, output $\bot$.
		\item Otherwise:
		\begin{enumerate}
			\item Let $t = (y,v,r,i_1,\ldots,i_k,\cH,\ty_{-\cH},x_{-\cH})$ where $\ty_{i'} = y_{i'}$ for $i' \in [n]\setminus (\cH \cup \bar{\cH})$.
			\item For $b \in \oo$: Let $x^b = (x_1,\ldots,x_{i-1},b, x_{i+1},\ldots,x_n)$ and $z^b = x^b_{\cH} \in \oo^m$, where $m = \ceil{k/4}$.
			\item Let $j \in [m]$ be the index such that $z^1_j \neq z^{-1}_j$.
			\item Sample $b \la \oo$ and $o \la \Dist^{\GenView^C, f}(j,z^b,t)$ where $\GenView^C$ is \cref{alg:GenView} with oracle access to $C$, and $f(s,t = (\psi,t')) = \Bc_{\psi}(s,t') + \ip{x_{\bar{\cH}}, \ty_{\bar{\cH}}}$. 
			\color{gray}{\# Recall that $x_{\bar{\cH}}$ is a sub-string of $x_{-\cH}$ (part of $t$) and that $ \ty_{\bar{\cH}}$ is a sub-string of $\ty_{-\cH}$ (also part of $t$).}
			\color{black}{\item If $o = 1$, output $b$. Otherwise, output $\bot$.}
		\end{enumerate}
	\end{enumerate}
\end{algorithm}

\begin{proof}[Proof of Property~\ref{item:privacy-of-X} of \cref{lemma:DPIP-to-AWEC} using \cref{alg:Bct}]
	
In the following, let $\cD = \GenView^C()$ and $((X,U),(Y,V)) \la C$. Consider a random execution of $\Bct^{\Bc, C}(I,\: X_{-I}, \: Y, \: V)$ for $I \la [n]$, and let $T, H, B, O, Z, J$ be the values of $\:\: t, h,\cH,o, z, j$ in the execution. 
Let $p = \ppr{(z,t) \la \cD, \: j \la [m]}{\Dist^{\cD,f}(j ,z^{(j)}, t) = 1}$, and note that conditioned on $I \in H$, $J$ is distributed uniformly over $[m]$. Therefore,
$$\pr{\Dist^{\cD,f}(J, Z^{(J)}, T) = 1 \mid I \in H} = p$$ and $$\pr{\Dist^{\cD,f}(J, Z, T) = 1 \mid I \in H} = \ppr{(z,t) \la \cD, \: j \la [m]}{\Dist^{\cD,f}(j ,z, t) = 1} \geq e^\eps p + \frac1m,$$ where the inequality holds by \cref{eq:Dist-gap}.
Thus, the following holds for $X^*_i = \Bct^{\Bc, C}(i,\: X_{-i}, \: Y, \: V)$:


\begin{align*}
	\eex{i \la [n]}{\pr{X^*_i = -X_i}}
	&= \pr{X_I^* = -X_I}\\
	&=\pr{\set{I \in H} \land \set{B = -X_I} \land \set{O=1}}\\
	&= \paren{\frac{m}{n}\cdot \pr{H \neq \emptyset}} \cdot \frac12 \cdot  \pr{O=1 \mid \set{I \in H}\land \set{B = -X_I}}\\
	&= \frac{m}{2n} \cdot \pr{H \neq \emptyset} \cdot \pr{\Dist^{\cD,f}(J, Z^{(J)}, T) = 1 \mid I \in H}\\
	&= \frac{m}{2n} \cdot \pr{H \neq \emptyset} \cdot p,
\end{align*}
 and 

\begin{align*}
	\eex{i \la [n]}{\pr{X^*_i = X_i}}
	&=\pr{\set{I \in H} \land \set{B = X_I} \land \set{O=1}}\\
	&= \paren{\frac{m}{n}\cdot \pr{H \neq \emptyset}} \cdot \frac12 \cdot  \pr{O=1 \mid \set{I \in H}\land \set{B = X_I}}\\
	&= \frac{m}{2n} \cdot \pr{H \neq \emptyset} \cdot\pr{\Dist^{\cD,f}(J, Z, T) = 1 \mid I \in H}\\
	&> \frac{m}{2n}  \cdot \pr{H \neq \emptyset} \cdot (e^{\eps} p + 1/m)\\
	&= e^{\eps} \cdot \paren{\frac{m}{2n}  \cdot \pr{H \neq \emptyset} \cdot p} + \frac{1}{2n}\cdot \pr{H \neq \emptyset},\\
	&>  e^{\eps} \cdot \eex{i \la [n]}{\pr{X^*_i = -X_i}} + \delta,
\end{align*}
which concludes the proof. The last inequality holds since $\pr{H \neq \emptyset} \geq 1-\frac1{10000}$ and $\delta \leq \frac1{3n}$.

\end{proof}


%\section{\WEC from \AWEC (Proof of \cref{lemma:AWEC-to-WEC})}\label{sec:AWEC-to-WEC}


In this section, we prove \cref{lemma:AWEC-to-WEC} and  show how to implement \WEC (\cref{def:WEC}) from \AWEC (\cref{def:AWEC}) using a \ppt protocol. Crucially, the security proof is constructive, so that it could be used in the computational case as well (see, \cref{cor:CompAWEC to CompWEC}).


%Find a place to put this 


To prove \cref{lemma:AWEC-to-WEC}, we will need the following easy version of Goldriech-Levin \cite{GoldreichL89}.
\begin{lemma}
\label{prel:gl:weak:prob}
There exists a \ppt oracle-aided  algorithm $\Dec$ such that the following holds. Let $n\in N$ be a number, $x\in \zn$, and %$f\colon \zn \to \zn$ be a (possibly randomized) function, 
 and let $\Pred$ be an algorithm such that
\begin{align*}
\ppr{ r\gets \zn}{\Pred(r)=\GL(x,r)} > 3/4+0.001,
\end{align*}
 where $\GL(x,r)\eqdef \langle x,r \rangle$ is the Goldreich-Levin predicate. 
Then $\pr{\Dec^\Pred(1^n)=x}=1-\negl(n)$.
\end{lemma}
\begin{proof}[Proof of \cref{prel:gl:weak:prob}]
We use $\Pred$ to decode each bit of $x$ separately. For every $i$, let $e_i$ be the vector that has $1$ in the $i$-th entry, and $0$'s everywhere else. Observe that, for a uniformly chosen $R\gets \zn$, 
$$\pr{\Pred(R)=\GL(x,R) \land \Pred(R\xor e_i)=\GL(x,R\xor e_i)}\ge 1/2+0.001.$$
Thus,
$$\pr{\Pred(R)\xor\Pred(R\xor e_i) =\GL(x,R) \xor \GL(x,R\xor e_i)}\ge 1/2+0.001.$$
By linearity of the inner product we get that,
$$\pr{\Pred(R)\xor\Pred(R\xor e_i) =x_i}\ge 1/2+0.001.$$
Let $\Dec$ be the algorithm that for every $i$, computes $\Pred(R)\xor\Pred(R\xor e_i)$  for $n$ random values of $R$, and let $x'_i$ to be the majority of the outputs. Then, $\Dec$ outputs $x'=x'_1,\dots,x'_n$. By Chernoff bound, $x'_i$ is equal to $x_i$ with all but negligible probability. By the union bound, the above is true for all $i$'s simultaneously  with all but negligible probability, as we wanted to show.
\end{proof}
We are now ready to prove \cref{lemma:AWEC-to-WEC}.
\begin{proof}[Proof of \cref{lemma:AWEC-to-WEC}]
We now define the protocol $\Lambda^C$ as follows:
\begin{protocol}[$\Lambda^C=(\Ac,\Bc)$]
\item Oracle access: A channel $C =((\OA,\VA),(\OB,\VB))$.
	\item Operation:
	\begin{enumerate}
            \item Sample $((\oA,\vA),(\oB,\vB))\from C$. $\Ac$ gets $(\oA,\vA)$ and $\Bc$ gets $(\oB,\vB)$. 
		
			\item $\Ac$ chooses a random offset $s\gets [1000\ell]$ and sends it to $\Bc$. Let $\oA'=\ceil{\frac{\oA+s}{1000\ell}}$ and $\oB'=\ceil{\frac{\oB+s}{1000\ell}}$ (if $\oB=\bot$, let $\oB'=\bot$). 
        \item $\Ac$ chooses a random vector $r\from \zo^{\log(n)}$ and sends it to $\Bc$. Let $\hoA=\langle \oA',r \rangle$ and $\hoB=\langle \oB',r \rangle$ (if $\oB'=\bot$, let $\hoB=\bot$).
        \item $\Ac$ outputs $\hoA$ and $\Bc$ outputs $\hoB$.
        \end{enumerate}
\end{protocol}
Let $\tilde{C}$ be the channel induces by $\Lambda^C=(\Ac,\Bc)$ defined above. 
Let $\OA,\VA,\OB,\VB,S,R,\OA',\OB',\hOA,\hOB$ be the random variables that corresponds to the value of $\oA,\vA,\oB,\vB,s,r,\oA',\oB',\hoA,\hoB$ in a random execution of the above protocol. Denote $\hVA=(\VA,S,R)$ and $\hVB=(\VB,S,R)$ and note that $(\hVA,\hOA),(\hVB,\hOB)$ defines the channels $\tilde{C}$.



We now prove that if $C=((\OA,\VA),(\OB,\VB))$ is an $(\ell,\alpha,p,q)$-\AWEC then $\tilde{C}$ is an $(\alpha'=\alpha+0.001,\: p' = p ,\:  q' = 1/2 + 2.001q)$-\WEC.

\paragraph{Agreement:} If $\size{\OA-\OB}\le \ell$, then $\ppr{S}{\ceil{\frac{\OA+S}{1000\ell}}\ne \ceil{\frac{\OB+s}{1000\ell}}}\le 1/1000$. Thus, 
\begin{align*}
    \pr{\hOA\ne \hOB\mid \hOB\ne \bot} &\leq\pr{|\OA - \OB|\leq \ell\mid \OB\ne \bot}+1/1000\\
    &= \eps+1/1000=\eps'
\end{align*}

\paragraph{$\Bc$'s privacy:} Recall that the view of $\Ac$ in the above protocol is $\hVA=(\VA,S,R)$, and note that $S,R$ are independent of $\OB$. Since, $\hOB=\bot$ iff $\OB=\bot$ it follows that it follows that for every algorithm $\Dc$:
    \begin{align*}
    &\size{{\pr{\Dc(\hVA) = 1 \mid \hOB \neq \bot} - \pr{\Dc(\hVA) = 1 \mid \hOB = \bot}} }\\
   %  &=\size{{\pr{\Dc(\hVA) = 1 \mid \OB \neq \bot} - \pr{\Dc(\hVA) = 1 \mid \OB = \bot}} }\\
    &=\size{{\pr{\Dc(\VA,S,R) = 1 \mid \OB \neq \bot} - \pr{\Dc(\VA,S,R) = 1 \mid \OB = \bot}} }\\
    &=\size{\pr{\Dc(\VA) = 1 \mid \OB \neq \bot} - \pr{\Dc(\VA) = 1 \mid \OB = \bot}} \le p=p'.
    \end{align*}
\paragraph{$\Ac$’s privacy:}  Assume towards a contradiction that there exists an algorithm $\Dc$ such that

\begin{align}\label{eq:avrging}
\pr{\Dc(\hVB)=\hOA  \mid \hOB\neq\bot} \ge\frac{1+q'}{2}=  3/4+q+0.01.
\end{align}
Let $\cG=\set{(\vB,s)\colon \ppr{R}{\Dc(\vB,s,R)=\hOA\mid\VB=\vB,S=s,\hOB\neq \bot}\ge 3/4+0.001}$, and first note that $\ppr{\VB,S}{(\VB,S)\in \cG\mid \hOB\neq \bot}\ge q+0.009$.
Indeed, otherwise it holds that
\begin{align*}
&\pr{\Dc(\hVB)=\hOA  \mid \hOB\neq\bot}\\
& = \ppr{\VB,S}{(\VB,S)\in \cG\mid \hOB\neq \bot} + \ppr{\VB,S}{(\VB,S)\notin \cG\mid \hOB\neq \bot}\cdot(3/4+ 0.001)\\
&< (q+0.009) + 1\cdot (3/4+ 0.001)\\
&= 3/4+ q+0.01
\end{align*}
in contradiction to \cref{eq:avrging}.
   % $$\ppr{\VB,S}{\ppr{R}{\Dc(\VB,S,R)=\hOA }\ge 3/4+\delta\cdot\alpha\mid \hOB=\bot}\ge \delta(1-\alpha).$$ 
    Next, by \cref{prel:gl:weak:prob} there exists an algorithm $\Dc'$ such that 
    $$
    \pr{\Dc'(\VB,S)=O'_A \mid (\VB,S)\in\cG, \OB'\neq \bot}\ge 1-o(1)
    $$ 
    Which implies that, 
    \begin{align*}
    \pr{\Dc'(\VB,S)=O'_A \mid  \OB'\neq \bot}
    &\ge \pr{\Dc'(\VB,S)=O'_A \mid (\VB,S)\in\cG, \OB'\neq \bot}\cdot \pr{\Dc'(\VB,S)\in\cG \mid\hOB\neq \bot}\\
    &\ge (q+0.009)(1-o(1))\\
    &\ge q.
    \end{align*}
    %for every fixing of $\VB,S$ to values $\vB,s$ such that 
    % $$
    % \ppr{R}{\Dc(\vB,s,R)=\hOA\mid\VB=\vB,S=s,\hOB= \bot}\ge 3/4+\delta\cdot\alpha
    % $$
    % it holds that 
    % it holds that $\Dc'$
    % there exists an algorithm $\Dc'$
    % it holds that $$\pr{\Dc'(\VB,S)=O'_A }\ge (1-\alpha)$$ ($\Dc'$ is the GL reconstruction), 
    Since by definition $\size{(O'_A\cdot 1000\ell-S)-\OA}\le 1000\ell$, and $S$ is independent of $\VB$, it follows that there exists an algorithm $\Dc''$ such that  
    $$\pr{\size{\Dc''(\VB)-\OA}\le 1000\ell\mid \OB\neq \bot}\ge \delta(1-2\alpha)> q.$$
Contradicting the fact that $C$ is an $(\ell,\alpha,p,q)$-\AWEC.
\end{proof}
% \section{From CDP-IP to OT}\label{sec:CDPIP_to_OT}

% In this section we state and prove our results for the computational case: \CDP (computational differential private) protocols that estimate the inner product well. For such protocols, we prove the following result.


% \begin{theorem}\label{thm:DPIP-to-OT}
% There exist constant $c_1,c_2 > 0$ and an oracle-aided \ppt protocol $\Pi$ such that the following holds for large enough $n \in \bbN$ and for 
% $\eps \leq \log^{0.9} n$, $\delta \leq \frac1{3n}$, and $\ell = e^{-c_1  \eps}  c_2\cdot n^{1/8}$:
%     Let $\Lambda$ be an $(\eps,\delta)$-DP protocol that is $(\ell,0.999)$-accurate for the inner-product functionality. 
%     Then $\Pi^\Lambda$ is a computational oblivious transfer protocol.
% \end{theorem}









% \begin{lemma}[Main lemma,  the computational case]\label{lem:KAProtocol:Comp}
% 	There exists a constant $c>0$ such that the following holds: Let $C = \set{C_\kappa}_{\kappa\in \N}$ be an $n$-size, $\eps$-\CDP channel ensemble, that is $(,)$-accurate for the inner-product functionality, and let $\Pi$ be according to \ref{??}. Then $???$ is an Oblivious transfer protocol. 
% \end{lemma}



% \begin{theorem}
%     \Enote{State our final result.}
% \end{theorem}


\printbibliography

\appendix

Here, we discuss significant related works and potential countermeasures relevant to our Thor attack.

\paragrabf{Hertzbleed \cite{wang2022hertzbleed}.}
Hertzbleed leverages dynamic voltage and frequency scaling (DVFS) to transform power side-channel attacks into timing attacks. By exploiting the timing differences caused by frequency variations, even remote attacks become feasible. For instance, in an attack against Supersingular Isogeny Key Encapsulation (SIKE), they managed to recover 378 bits of the private key within 36 hours. Although this attack shares similarities with our work in exploiting frequency changes, DVFS can be managed by the CPU core and disabled in BIOS settings to mitigate the Hertzbleed attack. However, in our case, disabling DVFS is not a viable countermeasure since AMX, an on-chip accelerator, manages its power and frequency independently.

\paragrabf{Collide+Power \cite{kogler2023collide+}.}
The Collide+Power research focuses on the power leakage of the memory hierarchy via Running Average Power Limit (RAPL). If RAPL is unavailable, monitoring can be done through a throttling side-channel, albeit requiring more measurements. They demonstrated two types of attacks: Meltdown-style and Microarchitectural Data Sampling (MDS)-style. In Meltdown-style, targeting a shared cache among two processes on different cores, theoretically, one bit can be leaked in 99.95 days with power limit control or 2.86 years with stress-induced throttling. In MDS-style, where both victim and attacker run on different logical cores of the same physical core, data can be leaked from the L1/L2 cache at a rate of 4.82 bits per hour. Disabling simultaneous multithreading can mitigate MDS-style attacks. Generally, RAPL being a privileged interface is not accessible to unprivileged attackers, and throttling can be disabled by turning off DVFS.

\paragrabf{Platypus \cite{Lipp2021Platypus}.}
The Platypus attack reconstructed 509 RSA key bits using RAPL MSRs within Intel SGX enclaves. However, this attack vector has been mitigated by making RAPL a privileged interface.

\paragrabf{Neural Network Specific Attacks.}
Several studies have attacked neural network accelerators using power side channels. In the work by Wei et al. \cite{wei2018know}, an FPGA-based convolutional neural network accelerator was attacked, requiring physical access to recover the model's input image with up to 89\% accuracy. Effective mitigations include masking and random scheduling, although masking introduces significant overheads, as demonstrated in MaskedNet \cite{dubey2020maskednet}, increasing latency and area costs by 2.8x and 2.3x, respectively.

Open DNN Box \cite{Xiang2019OpenDB} inferred the weight sparsity of neural network models with 96.5\% accuracy on average. CSI NN \cite{236204} used power and electromagnetic traces to infer information about weights and architecture in fully connected neural networks. DeepEM \cite{Yu2020DeepEMDN} and DeepSniffer \cite{Hu2020DeepSnifferAD} collected electromagnetic traces to glean architectural information, with DeepEM specifically targeting binarized neural networks. Cache Telepathy \cite{244042}, GANRED \cite{Liu2020GANREDGR}, and DeepRecon \cite{Hong2018SecurityAO} employed well-known cache side channels like Flush+Reload and Prime+Probe to gather neural network insights. For these cache attacks, the attacker runs locally, and the presence of a shared cache is necessary. In contrast, our Thor attack doesn't need physical access or shared cache. It introduces a novel, data-dependent timing side-channel vulnerability specific to Intel AMX accelerators.

In the work by Gongye et al. \cite{9218707}, they attacked DNNs using a floating-point timing side channel to obtain weights and biases. They took advantage of the drastically different execution times for floating-point multiplication and addition in certain scenarios, such as when dealing with subnormal values, to launch their attack. However, with modern accelerators like Intel AMX, this floating-point timing vulnerability has been eliminated. Now, the execution time for tile multiplication remains constant, even for special cases like zero inputs. Specifically, the latency is fixed at 52 cycles and the throughput is 16. Despite this, to attack DNNs with more than one layer, cache monitoring or physical access is still required to measure the execution time of each layer.

% \paragrabf{Potential Countermeasures.}
% Eliminating the cooldown state could defend against Thor but at a high power cost since Intel AMX is an energy-intensive accelerator designed for AI tasks. Keeping AMX continuously active would be power-prohibitive.

% Masking is a proven countermeasure for protecting AI model parameters against power side-channel attacks and could be adapted for future AMX versions despite the performance overhead. Additionally, machine learning models should incorporate techniques to detect unusual usage patterns, which can help identify and thwart attacks attempting to infer parameter values using methods similar to our AMX-type attack. One well-known countermeasure to these timing attacks is to coarsen the timer. By reducing the timer's precision, it becomes much harder for attackers to measure the subtle differences in execution times that they rely on for their exploits.

% In summary, while various countermeasures exist for different attack vectors, protecting against Thor on Intel AMX accelerators requires novel approaches to power and frequency management alongside traditional techniques.



\section{\cite{HaitnerMST22}'s Protocol Cannot Imply $\OT$}\label{appendix:HaitnerMST22}

In this section, we show that for some carefully chosen $\CDP$ and accurate protocol $\Pi$, the joint view of the parties in \cite{HaitnerMST22}'s protocol (\cref{protocol:HaitnerMST22}) can be \emph{simulated} using a trivial protocol, without using $\Pi$ at all.

\begin{protocol}[\cite{HaitnerMST22}'s protocol]\label{protocol:HaitnerMST22}
    \item Oracle: An accurate $\CDP$ protocol $\Pi$ for the inner-product.
    \item Operation:~
    \begin{enumerate}
        \item $\Ac$ and $\Bc$ choose random inputs $x\in \set{-1,1}^n$ and $y\in \set{-1,1}^n$, respectively.
        \item The parties interact using $\Pi$ to get approximation $z$ of $\langle x,y \rangle$.
        \item $\Ac$ chooses a random string $r \gets \zn$, and sends $r, x_r=\set{x_i \colon r_i=1}$ to $\Bc$. $\Bc$ replies with $y_{- r}=\set{x_i \colon r_i=0}$.
        \item Finally, $\Ac$ computes and outputs $\out_\Ac=\langle x_{-r},y_{-r} \rangle$, and $\Bc$ outputs $\out_\Bc=z-\langle x_{r},y_{r} \rangle$.
\end{enumerate}
\end{protocol}

To see this, assume that $\Pi$ is a protocol that on inputs $x$ and $y$, outputs $z=\langle x,y \rangle+e_\Ac+e_\Bc$, where $e_\Ac$ and $e_\Bc$ are independent samples from the $\Lap(2/\eps)$ distribution. Moreover, assume that $\Pi$ reveals $e_\Ac$ to $\Ac$ and $e_\Bc$ to $\Bc$ (and nothing else).  Such a protocol is indeed differential private, and it can be implemented using secure multi-party computation. Moreover, as we show in this work, such a protocol can be used to construct OT. However, when composed with the $\KA$ protocol of \cite{HaitnerMST22}, the resulting protocol can be simulated trivially.\footnote{More formally, and using the definition given in \cref{sec:protocol}, we claim that the channel induced by executing \cref{protocol:HaitnerMST22} with oracle access to the channel $$\Pi=\set{((x,(\langle x,y \rangle + e_A+e_B,e_A)),(y,(\langle x,y \rangle + e_A+e_B,e_B)))}_{x,y\gets \oo^n, e_A,e_B\gets \Lap(2/\epsilon)}$$ is a trivial channel.} 

Indeed, note that the view of $\Ac$ in $\Pi$ only contains $x,z$ and  $e_A$, while the view of $\Bc$ only contains $y,z$ and $e_B$. In this case, the view of $\Ac$ in the $\KA$ protocol of \cite{HaitnerMST22} contains $x,z, e_A, r$ and $y_{-r}$, while the view of $\Bc$ contains $y,z, e_B, r$ and $y_{r}$.
We next explain how to simulate this view without using $\Pi$. Consider the following protocol $\Pi'$ that  simulates the $\KA$ protocol in a reverse order:

\begin{protocol}[The simulation $\Pi'$]\label{protocol:trivial}
    \item Operation:~
    \begin{enumerate}
        \item $\Ac$ and $\Bc$ choose random inputs $x\in \set{-1,1}^n$ and $y\in \set{-1,1}^n$, respectively.
               \item $\Ac$ chooses a random string $r \gets \zn$, and sends $r, x_r=\set{x_i \colon r_i=1}$ to $\Bc$. $\Bc$ replies with $y_{- r}=\set{x_i \colon r_i=0}$.
               \item $\Ac$ samples $e_A\gets \Lap(2/\eps)$ and sends $z_A=\langle x_{r},y_r\rangle + e_A$  to $\Bc$. $\Bc$ samples $e_B\gets \Lap(2/\eps)$ and sends $z_B=\langle x_{-r},y_{-r}\rangle + e_B$  to $\Ac$.
        \item The output of the protocol is $z=z_A+z_B$.
\end{enumerate}
\end{protocol}
Clearly, \cref{protocol:trivial} is a trivial protocol and does not use any cryptographic assumptions. However, the views of $\Ac$ and $\Bc$ in $\Pi'$ contain all the information learned by the parties in the $\KA$ protocol ($(x,z,e_A,r,y_{-r})$ and $(y,z,e_B,r,x_{r})$ respectively). Moreover, we claim that the parties in $\Pi'$ do not learn any information that the parties in the $\KA$ protocol did not learn. Indeed, the only new value learned by $\Bc$, $z_A,$ can be also computed by $\Bc$ in the $\KA$ protocol by computing $z-e_B-\langle x_r,y_r \rangle$. Similarly, $z_B$ can be already computed by $\Ac$. 

We note that every protocol that uses only a communication channel cannot be used to construct $\OT$ unless $\OT$ already exists, and similarly, every protocol that uses  $\Pi'$ as a subroutine (in a black-box manner) cannot be used to construct $\OT$. Since \cref{protocol:HaitnerMST22} (that is, the views and outputs of the parties when running \cref{protocol:HaitnerMST22}) are the same as $\Pi'$, we conclude that \cref{protocol:HaitnerMST22} could not be used to construct $\OT$ in a black-box manner.  
%we conclude that \cref{protocol:HaitnerMST22} (that is, the views and outputs of the parties when running \cref{protocol:HaitnerMST22}) cannot be used in a black-box manner to construct $\OT$ unless $\OT$ already exists.

%\Nnote{Since every protocol that uses $\Pi'$ as a subroutine (in a black-box manner) cannot be used to construct OT unless $\OT$ already exists, we conclude that \cref{protocol:HaitnerMST22} (that is, the views and outputs of the parties when running \cref{protocol:HaitnerMST22}) cannot be used in a black-box manner to construct \OT (unless OT already exists).}



\section{\WEC from \AWEC (Proof of \cref{lemma:AWEC-to-WEC})}\label{sec:AWEC-to-WEC}


In this section, we prove \cref{lemma:AWEC-to-WEC} and  show how to implement \WEC (\cref{def:WEC}) from \AWEC (\cref{def:AWEC}) using a \ppt protocol. Crucially, the security proof is constructive, so that it could be used in the computational case as well (see, \cref{cor:CompAWEC to CompWEC}).


%Find a place to put this 


To prove \cref{lemma:AWEC-to-WEC}, we will need the following easy version of Goldriech-Levin \cite{GoldreichL89}.
\begin{lemma}
\label{prel:gl:weak:prob}
There exists a \ppt oracle-aided  algorithm $\Dec$ such that the following holds. Let $n\in N$ be a number, $x\in \zn$, and %$f\colon \zn \to \zn$ be a (possibly randomized) function, 
 and let $\Pred$ be an algorithm such that
\begin{align*}
\ppr{ r\gets \zn}{\Pred(r)=\GL(x,r)} > 3/4+0.001,
\end{align*}
 where $\GL(x,r)\eqdef \langle x,r \rangle$ is the Goldreich-Levin predicate. 
Then $\pr{\Dec^\Pred(1^n)=x}=1-\negl(n)$.
\end{lemma}
\begin{proof}[Proof of \cref{prel:gl:weak:prob}]
We use $\Pred$ to decode each bit of $x$ separately. For every $i$, let $e_i$ be the vector that has $1$ in the $i$-th entry, and $0$'s everywhere else. Observe that, for a uniformly chosen $R\gets \zn$, 
$$\pr{\Pred(R)=\GL(x,R) \land \Pred(R\xor e_i)=\GL(x,R\xor e_i)}\ge 1/2+0.001.$$
Thus,
$$\pr{\Pred(R)\xor\Pred(R\xor e_i) =\GL(x,R) \xor \GL(x,R\xor e_i)}\ge 1/2+0.001.$$
By linearity of the inner product we get that,
$$\pr{\Pred(R)\xor\Pred(R\xor e_i) =x_i}\ge 1/2+0.001.$$
Let $\Dec$ be the algorithm that for every $i$, computes $\Pred(R)\xor\Pred(R\xor e_i)$  for $n$ random values of $R$, and let $x'_i$ to be the majority of the outputs. Then, $\Dec$ outputs $x'=x'_1,\dots,x'_n$. By Chernoff bound, $x'_i$ is equal to $x_i$ with all but negligible probability. By the union bound, the above is true for all $i$'s simultaneously  with all but negligible probability, as we wanted to show.
\end{proof}
We are now ready to prove \cref{lemma:AWEC-to-WEC}.
\begin{proof}[Proof of \cref{lemma:AWEC-to-WEC}]
We now define the protocol $\Lambda^C$ as follows:
\begin{protocol}[$\Lambda^C=(\Ac,\Bc)$]
\item Oracle access: A channel $C =((\OA,\VA),(\OB,\VB))$.
	\item Operation:
	\begin{enumerate}
            \item Sample $((\oA,\vA),(\oB,\vB))\from C$. $\Ac$ gets $(\oA,\vA)$ and $\Bc$ gets $(\oB,\vB)$. 
		
			\item $\Ac$ chooses a random offset $s\gets [1000\ell]$ and sends it to $\Bc$. Let $\oA'=\ceil{\frac{\oA+s}{1000\ell}}$ and $\oB'=\ceil{\frac{\oB+s}{1000\ell}}$ (if $\oB=\bot$, let $\oB'=\bot$). 
        \item $\Ac$ chooses a random vector $r\from \zo^{\log(n)}$ and sends it to $\Bc$. Let $\hoA=\langle \oA',r \rangle$ and $\hoB=\langle \oB',r \rangle$ (if $\oB'=\bot$, let $\hoB=\bot$).
        \item $\Ac$ outputs $\hoA$ and $\Bc$ outputs $\hoB$.
        \end{enumerate}
\end{protocol}
Let $\tilde{C}$ be the channel induces by $\Lambda^C=(\Ac,\Bc)$ defined above. 
Let $\OA,\VA,\OB,\VB,S,R,\OA',\OB',\hOA,\hOB$ be the random variables that corresponds to the value of $\oA,\vA,\oB,\vB,s,r,\oA',\oB',\hoA,\hoB$ in a random execution of the above protocol. Denote $\hVA=(\VA,S,R)$ and $\hVB=(\VB,S,R)$ and note that $(\hVA,\hOA),(\hVB,\hOB)$ defines the channels $\tilde{C}$.



We now prove that if $C=((\OA,\VA),(\OB,\VB))$ is an $(\ell,\alpha,p,q)$-\AWEC then $\tilde{C}$ is an $(\alpha'=\alpha+0.001,\: p' = p ,\:  q' = 1/2 + 2.001q)$-\WEC.

\paragraph{Agreement:} If $\size{\OA-\OB}\le \ell$, then $\ppr{S}{\ceil{\frac{\OA+S}{1000\ell}}\ne \ceil{\frac{\OB+s}{1000\ell}}}\le 1/1000$. Thus, 
\begin{align*}
    \pr{\hOA\ne \hOB\mid \hOB\ne \bot} &\leq\pr{|\OA - \OB|\leq \ell\mid \OB\ne \bot}+1/1000\\
    &= \eps+1/1000=\eps'
\end{align*}

\paragraph{$\Bc$'s privacy:} Recall that the view of $\Ac$ in the above protocol is $\hVA=(\VA,S,R)$, and note that $S,R$ are independent of $\OB$. Since, $\hOB=\bot$ iff $\OB=\bot$ it follows that it follows that for every algorithm $\Dc$:
    \begin{align*}
    &\size{{\pr{\Dc(\hVA) = 1 \mid \hOB \neq \bot} - \pr{\Dc(\hVA) = 1 \mid \hOB = \bot}} }\\
   %  &=\size{{\pr{\Dc(\hVA) = 1 \mid \OB \neq \bot} - \pr{\Dc(\hVA) = 1 \mid \OB = \bot}} }\\
    &=\size{{\pr{\Dc(\VA,S,R) = 1 \mid \OB \neq \bot} - \pr{\Dc(\VA,S,R) = 1 \mid \OB = \bot}} }\\
    &=\size{\pr{\Dc(\VA) = 1 \mid \OB \neq \bot} - \pr{\Dc(\VA) = 1 \mid \OB = \bot}} \le p=p'.
    \end{align*}
\paragraph{$\Ac$’s privacy:}  Assume towards a contradiction that there exists an algorithm $\Dc$ such that

\begin{align}\label{eq:avrging}
\pr{\Dc(\hVB)=\hOA  \mid \hOB\neq\bot} \ge\frac{1+q'}{2}=  3/4+q+0.01.
\end{align}
Let $\cG=\set{(\vB,s)\colon \ppr{R}{\Dc(\vB,s,R)=\hOA\mid\VB=\vB,S=s,\hOB\neq \bot}\ge 3/4+0.001}$, and first note that $\ppr{\VB,S}{(\VB,S)\in \cG\mid \hOB\neq \bot}\ge q+0.009$.
Indeed, otherwise it holds that
\begin{align*}
&\pr{\Dc(\hVB)=\hOA  \mid \hOB\neq\bot}\\
& = \ppr{\VB,S}{(\VB,S)\in \cG\mid \hOB\neq \bot} + \ppr{\VB,S}{(\VB,S)\notin \cG\mid \hOB\neq \bot}\cdot(3/4+ 0.001)\\
&< (q+0.009) + 1\cdot (3/4+ 0.001)\\
&= 3/4+ q+0.01
\end{align*}
in contradiction to \cref{eq:avrging}.
   % $$\ppr{\VB,S}{\ppr{R}{\Dc(\VB,S,R)=\hOA }\ge 3/4+\delta\cdot\alpha\mid \hOB=\bot}\ge \delta(1-\alpha).$$ 
    Next, by \cref{prel:gl:weak:prob} there exists an algorithm $\Dc'$ such that 
    $$
    \pr{\Dc'(\VB,S)=O'_A \mid (\VB,S)\in\cG, \OB'\neq \bot}\ge 1-o(1)
    $$ 
    Which implies that, 
    \begin{align*}
    \pr{\Dc'(\VB,S)=O'_A \mid  \OB'\neq \bot}
    &\ge \pr{\Dc'(\VB,S)=O'_A \mid (\VB,S)\in\cG, \OB'\neq \bot}\cdot \pr{\Dc'(\VB,S)\in\cG \mid\hOB\neq \bot}\\
    &\ge (q+0.009)(1-o(1))\\
    &\ge q.
    \end{align*}
    %for every fixing of $\VB,S$ to values $\vB,s$ such that 
    % $$
    % \ppr{R}{\Dc(\vB,s,R)=\hOA\mid\VB=\vB,S=s,\hOB= \bot}\ge 3/4+\delta\cdot\alpha
    % $$
    % it holds that 
    % it holds that $\Dc'$
    % there exists an algorithm $\Dc'$
    % it holds that $$\pr{\Dc'(\VB,S)=O'_A }\ge (1-\alpha)$$ ($\Dc'$ is the GL reconstruction), 
    Since by definition $\size{(O'_A\cdot 1000\ell-S)-\OA}\le 1000\ell$, and $S$ is independent of $\VB$, it follows that there exists an algorithm $\Dc''$ such that  
    $$\pr{\size{\Dc''(\VB)-\OA}\le 1000\ell\mid \OB\neq \bot}\ge \delta(1-2\alpha)> q.$$
Contradicting the fact that $C$ is an $(\ell,\alpha,p,q)$-\AWEC.
\end{proof}
\section{Missing Proofs}\label{sec:missing-proofs}
\remove{
\subsection{Proving \cref{lem:distance-I}}\label{sec:missing-proofs:distance-I}

We make use of the following fact.

\begin{fact}[Proposition 3.28 in \cite{HaitnerMST22}]\label{fact:I}
	Let $R$ be uniform random variable over $\{0,1\}^n$, and let $I$ be uniform random variable over $\mathcal{I}\subseteq[n]$, independent of $R$, then $SD(R|_{R_I=0},R|_{R_I=1})\leq1/\sqrt{\size{\cI}}$.
\end{fact}
We now prove \cref{lem:distance-I}, restated below.

\begin{lemma}[Restatement of \cref{lem:distance-I}]
    \distanceILemma
\end{lemma}
\begin{proof}
    Assume towards a contradiction that there exist $f$ and $\alpha$ such that \cref{eq:f-alpha} does not hold.
    Namely, for $m = 1/\alpha^2$, there exist more than $2m$ indices $i \in [n]$ with $$\size{\:\ex{f(R) \mid R_i = 0}-\ex{f(R) \mid R_i = 1}\:}\geq 1/\sqrt{m}.$$
    This implies that there exist $b \in \oo$ and more than $m$ indices $i \in [n]$ with $$\ex{f(R) \mid R_i = b}-\ex{f(R) \mid R_i = 1-b} \geq 1/\sqrt{m}$$ (denote this set by $\cI$). 
    Thus, we deduce for $I \la \cI$ that
    \begin{align}\label{eq:big-R_I-gap}
        \size{\ex{f(R) \mid R_I = 0}-\ex{f(R) \mid R_I = 1}} 
        &\geq \ex{f(R) \mid R_I = b}-\ex{f(R) \mid R_I = 1-b}\\
        &\geq 1/\sqrt{m}.\nonumber
    \end{align}
    On the other hand, note that
    \begin{align*}
        SD(f(R)|_{R_I=0},f(R)|_{R_I=1})
        \leq SD(R|_{R_I=0},R|_{R_I=1})
        \leq 1/\sqrt{\size{\cI}}
        < 1/\sqrt{m},
    \end{align*}
    where the second inequality holds by \cref{fact:I}.
    Thus, we conclude that
    \begin{align*}
        \size{\ex{f(R) \mid R_I = 0}-\ex{f(R) \mid R_I = 1}}
        &\leq SD(f(R)|_{R_I=0},f(R)|_{R_I=1}) \cdot \sup_{r \in \zo^n, s \in \zo^*}(\size{f_s(r)})\\
        &< 1/\sqrt{m},
    \end{align*}
    where $f_s(r)$ denotes the function $f$ when fixing its random coins to $s$. 
    The above inequality contradicts \cref{eq:big-R_I-gap}, concluding the proof of the lemma.
\end{proof}
}

\subsection{Proving \cref{prop:hard-to-guess-inf,prop:hard-to-guess-comp}}\label{sec:missing-proofs:hard-to-guess}

We make use of the following claim.

\begin{claim}\label{claim:X-star}
    Let $X \la \oo$ and let $X^*$ be a random variable over $\set{-1,1,\bot}$ (correlated with $X$) such that for every $b,b' \in \oo$:
    \begin{align*}
        \pr{X^* = b \mid X = b'} \leq e^{\eps}\cdot \pr{X^* = b \mid X = -b'} + \delta.
    \end{align*}
    Then
    \begin{align*}
        \pr{X^* = X} \leq e^{\eps}\cdot \pr{X^* = -X} + \delta.
    \end{align*}
\end{claim}
\begin{proof}
    Compute
    \begin{align*}
        \pr{X^* = X } 
        &= \frac12 \cdot \pr{X^* = -1 \mid X = -1} + \frac12 \cdot \pr{X^* = 1 \mid X = 1}\\
        &\leq  \frac12 \cdot \paren{e^{\eps}\cdot \pr{X^* = -1 \mid X = 1} + \delta} + \frac12 \cdot \paren{e^{\eps}\cdot \pr{X^* = 1 \mid X = -1} + \delta}\\
        &= e^{\eps} \cdot \paren{\frac12 \cdot \pr{X^* = -1 \mid X = 1} + \frac12 \cdot \pr{X^* = 1 \mid X = -1}} + \delta\\
        &= e^{\eps} \cdot \pr{X^*_i = -X_i} + \delta.
    \end{align*}
\end{proof}


We next prove \cref{prop:hard-to-guess-inf}, restated below.

\begin{proposition}[Restatement of \cref{prop:hard-to-guess-inf}]
    \propHardToGuessInf
\end{proposition}
\begin{proof}
    %Note that it is enough to prove the claim just for deterministic $g$'s because if the statement does not hold for a specific randomized $g$, then there exists a fixing of its randomness that it does not hold under this fixing. 
    
    Fix $b,b' \in \oo$ and $i \in [n]$. By \cref{claim:X-star}, it is sufficient to prove that
    \begin{align}\label{eq:X_i^*-goal}
        \pr{X_i^* = b \mid X_i = b'} \leq e^{\eps}\cdot \pr{X_i^* = b \mid X_i = -b'} + \delta.
    \end{align}
    For $x_{-i} = (x_1,\ldots,x_{i-1},x_{i+1},\ldots,x_n) \in \oo^{n-1}$, define the function
    \begin{align*}
        h_{x_{-i}}(y) = g(i,x_{-i},y).
    \end{align*}
    Since $f$ is $(\eps,\delta)$-\DP, for any $x_{-i} \in \oo^{n-1}$ it holds that
    \begin{align*}
        \pr{h_{x_{-i}}(f(x_1,\ldots,x_{i-1}, b', x_{i+1},\ldots,x_n)) = b} \leq e^{\eps}\cdot \pr{h_{x_{-i}}(f(x_1,\ldots,x_{i-1}, -b', x_{i+1},\ldots,x_n)) = b} + \delta.
    \end{align*}
    Thus,
    \begin{align*}
        \pr{X_i^* = b \mid X_i = b'}
        &= \eex{x_{-i} \la \oo^{n-1}}{\pr{g(i,x_{-i}, f(x_1,\ldots,x_{i-1}, b', x_{i+1},\ldots,x_n)) = b }}\\
        &= \eex{x_{-i} \la \oo^{n-1}}{\pr{h_{x_{-i}}(f(x_1,\ldots,x_{i-1}, b', x_{i+1},\ldots,x_n)) = b}}\\
        &\leq e^{\eps}\cdot \eex{x_{-i} \la \oo^{n-1}}{\pr{h_{x_{-i}}(f(x_1,\ldots,x_{i-1}, -b', x_{i+1},\ldots,x_n)) = b}} + \delta\\
        &= e^{\eps}\cdot\pr{X_i^* = b \mid X_i = -b'}+ \delta.
    \end{align*}
\end{proof}


We next prove \cref{prop:hard-to-guess-comp}, restated below.

\begin{proposition}[Restatement of \cref{prop:hard-to-guess-comp}]
    \propHardToGuessComp
\end{proposition}
\begin{proof}
    
    In the following, fix $b,b' \in \oo$, and for $\pk \in \bbN$, $i \in [n(\pk)]$ and $x_{-i} \in \oo^{n(\pk)-1}$, define 
    \begin{align*}
        h_{\pk}^{i,x_{-i}}(y) = b\cdot g_{\pk}(i,x_{-i},y).
    \end{align*}
    Note that for any ensemble $\set{(i,x_{-i})_{\pk} \in [n(\pk)]\times \oo^{n(\pk)-1}}_{\pk \in \bbN}$, the circuit family \\$\set{h_{\pk}^{(i, x_{-i})_{\pk}} = g_{\pk}((i,x_{-i})_{\pk},\cdot)}_{\pk \in \bbN}$ has poly-size. 
    Since $f = \set{f_{\pk}}_{\pk \in \bbN}$ is $(\eps,\delta)$-\CDP, then for large enough $\pk$, the following holds for every $i \in [n(\pk)]$ and $x_{-i} \in \oo^{n(\pk)-1}$:
    \begin{align*}
        \lefteqn{\pr{h_{\pk}^{i, x_{-i}}(f_{\pk}(x_1,\ldots,x_{i-1}, b', x_{i+1},\ldots,x_n)) = 1}}\\
        &\leq e^{\eps(\pk)}\cdot \pr{h_{\pk}^{i,x_{-i}}(f(x_1,\ldots,x_{i-1}, -b', x_{i+1},\ldots,x_n)) = 1} + \delta(\pk),
    \end{align*}
    as otherwise, there would exist an ensemble $\set{(i,x_{-i})_{\pk} \in [n(\pk)]\times \oo^{n(\pk)-1}}_{\pk \in \cS}$ for an infinite set $\cS \subseteq \bbN$ such that the circuit family 
    $\set{h_{\pk}^{(i, x_{-i})_{\pk}}}_{\pk \in \cS}$ violates the $(\eps,\delta)$-\CDP property of $f$. 

    In the following, fix such large enough $\pk$ and $i \in [n]$ for $n = n(\kappa)$, let $X = (X_1,\ldots,X_{n}) \la \oo^{n}$ and $X_i^* = g_{\pk}(i,X_{-i},f_{\pk}(X_1,\ldots,X_n))$, and compute
    \begin{align*}
        \lefteqn{\pr{X_i^* = b \mid X_i = b'}}\\
        &= \eex{x_{-i} \la \oo^{n-1}}{\pr{g_{\pk}(i,x_{-i}, f_{\pk}(x_1,\ldots,x_{i-1}, b', x_{i+1},\ldots,x_n)) = b }}\\
        &= \eex{x_{-i} \la \oo^{n-1}}{\pr{h_{\pk}^{i,x_{-i}}(f(x_1,\ldots,x_{i-1}, b', x_{i+1},\ldots,x_n)) = 1}}\\
        &\leq e^{\eps(\pk)}\cdot \eex{x_{-i} \la \oo^{n-1}}{\pr{h_{\pk}^{i,x_{-i}}(f(x_1,\ldots,x_{i-1}, -b', x_{i+1},\ldots,x_n)) = 1}} + \delta(\pk)\\
        &= e^{\eps(\pk)}\cdot \eex{x_{-i} \la \oo^{n-1}}{\pr{g_{\pk}(i,x_{-i}, f_{\pk}(x_1,\ldots,x_{i-1}, -b', x_{i+1},\ldots,x_n)) = b}} + \delta(\pk)\\
        &= e^{\eps(\pk)}\cdot\pr{X_i^* = b \mid X_i = -b'}+\delta(\pk).
    \end{align*}
    Since the above holds for any $b, b' \in \oo$, we conclude by \cref{claim:X-star} that 
    \begin{align*}
        \pr{X_i^* = X_i} \leq e^{\eps(\pk)}\cdot \pr{X_i^* = -X_i} + \delta(\pk),
    \end{align*}
    as required.
\end{proof}

\subsection{Proving \cref{lemma:property1:prediction}}\label{sec:prediction-lemma}

To prove \cref{lemma:property1:prediction}, we use the following lemma that measures the distance between two uniform stings conditioned on a random index $i$ either being fixed to $0$ or to $1$.

\def\distanceILemma{
    Let $R \la \zo^n$. For any (randomized) function $F:\{0,1\}^n\rightarrow \{0,1\}$ and $\alpha > 0$, it holds that
    \begin{align}\label{eq:f-alpha}
        \ppr{i \la [n]}{\size{\:\ex{F(R) \mid R_i = 0}-\ex{F(R) \mid R_i = 1}\:}\geq \alpha} \leq \frac{2}{n \alpha^2},
    \end{align}
    where the expectations are taken over $R$ and the randomness of $F$.
}

\begin{lemma}\label{lem:distance-I}
    \distanceILemma
\end{lemma}

The proof of \cref{lem:distance-I} uses the following fact.

\begin{fact}[Proposition 3.28 in \cite{HaitnerMST22}]\label{fact:I}
	Let $R$ be uniform random variable over $\{0,1\}^n$, and let $I$ be uniform random variable over $\mathcal{I}\subseteq[n]$, independent of $R$, then $SD(R|_{R_I=0},R|_{R_I=1})\leq1/\sqrt{\size{\cI}}$.
\end{fact}

We first prove \cref{lem:distance-I} using \cref{fact:I}.

\begin{proof}[Proof of \cref{lem:distance-I}]
    Assume towards a contradiction that there exist $F$ and $\alpha$ such that \cref{eq:f-alpha} does not hold.
    Namely, for $m = 1/\alpha^2$, there exist more than $2m$ indices $i \in [n]$ with $$\size{\:\ex{F(R) \mid R_i = 0}-\ex{F(R) \mid R_i = 1}\:}\geq 1/\sqrt{m}.$$
    This implies that there exist $b \in \oo$ and more than $m$ indices $i \in [n]$ with $$\ex{F(R) \mid R_i = b}-\ex{F(R) \mid R_i = 1-b} \geq 1/\sqrt{m}$$ (denote this set by $\cI$). 
    Thus, we deduce for $I \la \cI$ that
    \begin{align}\label{eq:big-R_I-gap}
        \size{\ex{F(R) \mid R_I = 0}-\ex{F(R) \mid R_I = 1}} 
        &\geq \ex{F(R) \mid R_I = b}-\ex{F(R) \mid R_I = 1-b}\\
        &\geq 1/\sqrt{m}.\nonumber
    \end{align}
    On the other hand, note that
    \begin{align*}
        SD(F(R)|_{R_I=0},F(R)|_{R_I=1})
        \leq SD(R|_{R_I=0},R|_{R_I=1})
        \leq 1/\sqrt{\size{\cI}}
        < 1/\sqrt{m},
    \end{align*}
    where the second inequality holds by \cref{fact:I}.
    Thus, we conclude that
    \begin{align*}
        \size{\ex{F(R) \mid R_I = 0}-\ex{F(R) \mid R_I = 1}}
        &\leq SD(F(R)|_{R_I=0},F(R)|_{R_I=1}) \cdot \sup_{r \in \zo^n, s \in \zo^*}(\size{F_s(r)})\\
        &< 1/\sqrt{m},
    \end{align*}
    where $F_s(r)$ denotes the function $F$ when fixing its random coins to $s$. 
    The above inequality contradicts \cref{eq:big-R_I-gap}, concluding the proof of the lemma.
\end{proof}

Using \cref{lem:distance-I}, we now prove \cref{lemma:property1:prediction}, restated below.

\begin{lemma}[Restatement of \cref{lemma:property1:prediction}]
    \PredictionLemma
\end{lemma}
\begin{proof}
	
	Let $\gamma \in (0,1)$ and $n \in \bbN$ and consider the following algorithm $\Gc = \Gc_{\gamma}$:
	\begin{algorithm}
		\item Inputs: $i \in [n]$,  $z_{-i} = (z_1,\ldots,z_{i-1},z_{i+1},\ldots,z_n) \in \oo^{n-1}$ and $w \in \zo^*$.
		\item Parameter: $\gamma \in (0,1)$.
		\item Oracle: $F \colon \zo^n \times \oo^{\leq n} \times \zo^* \rightarrow \zo$.
		\item Operation:~
		\begin{enumerate}
			\item For $b \in \oo$:
			\begin{enumerate}
				\item Let $z^b = (z_1,\ldots,z_{i-1},b,z_{i+1},\ldots,z_n)$.
				\item Estimate $\mu^b \eqdef \eex{r \la \zo^n \mid r_i = 1, \: F}{F(r,z^b_{r}, w)}$ as follows:
				\begin{itemize}
					\item Sample $r_1,\ldots,r_{s} \la \set{r \in \zo^n \colon r_i = 1}$, for $s = \ceil{\frac{128 \log (12n)}{\gamma^2}}$, and then sample $\tilde{\mu}^b \la \frac1{s} \sum_{j=1}^s F(r_j,z^b_{r_j}, w)$ (using $s$ oracle calls to $F$).
				\end{itemize}
			\end{enumerate}
			\item Estimate $\mu^* \eqdef \eex{r \la \zo^n \mid r_i = 0, \: f}{f(r,z^1_{r}, w)}$ as follows:
			\begin{itemize}
				\item Sample $r_1,\ldots,r_{s}  \la \set{r \in \zo^n \colon r_i = 0}$, for $s = \ceil{\frac{128 \log (12n)}{\gamma^2}}$, and then sample $\tilde{\mu}^* \la \frac1{s} \sum_{j=1}^s F(r_j,z^1_{r_j}, w)$ (using $s$ oracle calls to $F$).
			\end{itemize}
			\item If exists $b \in \oo$ s.t. $\size{\tilde{\mu}^b - \tilde{\mu}^*} < \gamma/4$ and $\size{\tilde{\mu}^{-b} - \tilde{\mu}^*} > \gamma/4$, output $b$.
			\item Otherwise, output $\bot$.
		\end{enumerate}
	\end{algorithm}
	In the following, fix a pair of (jointly distributed) random variables $(Z,W) \in \oo^n \times \zo^*$ and a randomized function  $F \colon \zo^n \times \oo^{\leq n} \times \zo^* \rightarrow \oo$ that satisfy 
	\begin{align*}
		\size{\ex{F(R,Z_{R},W) - F(R,Z^{(I)}_{R},W)}} \geq \gamma,
	\end{align*}
	for $R \la \zo^n$ and $I \la [n]$ that are sampled independently. 
	Our goal is to prove that 
	
	\begin{align}\label{eq:predic-lemma:UB}
		\pr{\Gc^F(I,Z_{-I}, W) = -Z_I} \leq O\paren{\frac{1}{\gamma^2 n}},
	\end{align}
	and 
	\begin{align}\label{eq:predic-lemma:LB}
		\pr{\Gc^F(I,Z_{-I}, W) = Z_I} \geq \Omega(\gamma) - O\paren{\frac{1}{\gamma^2 n}}.
	\end{align}

	Note that
	\begin{align}\label{eq:D}
		\text{For every random variable }D \in [-1,1]\text{ with }\size{\ex{D}} \geq \gamma >0: \quad \pr{\size{D} > \gamma/2} > \gamma/2,
	\end{align}
	as otherwise, $\size{\ex{D}} \leq \ex{\size{D}} \leq 1\cdot \frac{\gamma}{2} + \frac{\gamma}{2}(1-\frac{\gamma}{2}) < \gamma$.
	
	
	By applying \cref{eq:D} with $D = \eex{r \la \zo^n, F}{F(r,Z_{r},W) - f(r,Z^{(I)}_{r},W)}$, it holds that
	\begin{align}\label{eq:main-lemma:good}
		\ppr{(i,z,w) \la (I,Z,W)}{\size{\eex{r \la \zo^n, F}{F(r,z_{r},w) - F(r,z^{(i)}_{r},w)}} > \gamma/2} > \gamma/2.
	\end{align}
	On the other hand, for every fixing of $(z,w) \in \Supp(Z,W)$, we can apply \cref{lem:distance-I} with the function $F_{z,w}(r) = F(r,z_{r},w)$ and with $\alpha =\frac{\gamma}{16}$ to obtain that
	\begin{align*}
		\ppr{i\la I}{\: \size{\eex{r \la \zo^n \mid r_i = 0, \: F}{F(r,z_{r},w)} - \eex{r \la \zo^n \mid r_i = 1, \: F}{F(r,z_{r},w)} \:} \geq \frac{\gamma}{16}} \leq \frac{512}{n \gamma^2}.
	\end{align*}
	But since the above holds for every fixing of $(z,w)$, then in particular it holds that
	\begin{align}\label{eq:main-lemma:bad}
		\ppr{(i,z,w) \la (I,Z,W)}{\: \size{\eex{r \la \zo^n \mid r_i = 0, \: F}{F(r,z_{r},w)} - \eex{r \la \zo^n \mid r_i = 1, \: F}{F(r,z_{r},w)} \:} \geq \frac{\gamma}{16}} \leq \frac{512}{n \gamma^2}.
	\end{align}
	
	We next prove \cref{eq:predic-lemma:UB,eq:predic-lemma:LB} using \cref{eq:main-lemma:good,eq:main-lemma:bad}.
	
	In the following, for a triplet $t = (i,z,w) \in \Supp(I,Z,W)$, consider a random execution of $\Gc^F(i,z_{-i},w)$. For $x \in \set{-1,1,*}$, let $\mu^x_t$ be the value of $\mu^x$ in the execution, and let $M^x_t$ be the (random variable of the) value of $\tilde{\mu}^x$ in the execution. Note that by definition, it holds that $\mu_{i,z,w}^{z_i} = \eex{r \la \zo^n, F}{F(r,z_{r},w)}$ and $\mu_{i,z,w}^{-z_i} = \eex{r \la \zo^n, F}{F(r,z^{(i)}_{r},w)}$. Therefore, \cref{eq:main-lemma:good} is equivalent to 
	\begin{align}\label{eq:main-lemma:good2}
		\ppr{t=(i,z,w) \la (I,Z,W)}{\size{\mu_t^{z_i} - \mu_t^{-z_i}} > \gamma/2} > \gamma/2.
	\end{align}
	Furthermore, note that $\mu_{i,z,w}^* \eqdef \eex{r \la \zo^n \mid r_i = 0, \: F}{F(r,z^1_{r}, w)} = \eex{r \la \zo^n \mid r_i = 0, \: F}{F(r,z_{r}, w)}$ and that $\mu_{i,z,w}^{z_i} = \eex{r \la \zo^n \mid r_i = 1, \: F}{F(r,z^{z_i}_{r}, w)}$. Therefore, \cref{eq:main-lemma:bad} is equivalent to 
	\begin{align}\label{eq:main-lemma:bad2}
		\ppr{t = (i,z,w) \la (I,Z,W)}{\: \size{\mu_t^* - \mu_t^{z_i}}\geq \frac{\gamma}{16}}  \leq \frac{512}{n \gamma^2}.
	\end{align}
	
	We next prove the lemma using \cref{eq:main-lemma:good2,eq:main-lemma:bad2}.
	
	Note that by Hoeffding's inequality, for every $t = (i,z,w) \in \Supp(I,Z,W)$ and  $x \in \set{-1,1,*}$ it holds that $\pr{\size{M_t^x - \mu_t^x} \geq \frac{\gamma}{16}} \leq 2\cdot e^{-2 s \paren{\frac{\gamma}{16}}^2} \leq \frac1{6n}$, which yields that for every fixing of $t = (i,z,w) \in \Supp(I,Z,W)$, w.p.\ at least $1-\frac1{2n}$ we have for all $x \in \set{-1,1,*}$ that $\size{M_t^x - \mu_t^x} < \frac{\gamma}{16}$ (denote this event by $E_t$).
	
	The proof of \cref{eq:predic-lemma:UB} holds by the following calculation:
	\begin{align*}
		\lefteqn{\pr{\Gc^F(I,Z_{-I},W) = -Z_I}}\\
		&= \eex{(i,z,w) \la (I,Z,W)}{\pr{\Gc^F(i,z_{-i},w) = -z_i}}\\
		&= \eex{t = (i,z,w) \la (I,Z,W)}{\pr{\set{\size{M_t^* - M_t^{z_i}} > \gamma/4} \land \set{\size{M_t^* - M_t^{-z_i}} < \gamma/4}}}\\
		&\leq \eex{t =(i,z,w) \la (I,Z,W)}{\pr{\size{M_t^* - M_t^{z_i}} > \gamma/4}}\\
		&\leq \eex{t = (i,z,w) \la (I,Z,W)}{\pr{\size{M_t^* - M_t^{z_i}} > \gamma/4 \mid E_t}} + \frac{1}{2n}\\
		&\leq \ppr{t = (i,z,w) \la (I,Z,W)}{\size{\mu_t^* -  \mu_t^{z_i}} \geq \frac{\gamma}{16}} + \frac{1}{2n}\\
		&\leq \frac{512}{n \gamma^2} + \frac1{2n},
	\end{align*}
	The second inequality holds since $\pr{\neg E_t} \leq \frac1{2n}$ for every $t$. The penultimate inequality holds since conditioned on $E_t$, it holds that $\size{M_t^* - \mu_t^*} \leq \frac{\gamma}{16}$ and $\size{M_t^{z_i} - \mu_t^{z_i}} \leq \frac{\gamma}{16}$, which implies that $\size{M_t^* - M_t^{z_i}} > \gamma/4 \: \implies \: \size{\mu_t^* -  \mu_t^{z_i}} \geq \frac{\gamma}{4} - 2\cdot \frac{\gamma}{16} > \frac{\gamma}{16}$. The last inequality holds by \cref{eq:main-lemma:bad2}.
	
	It is left to prove \cref{eq:predic-lemma:LB}. 
	Observe that \cref{eq:main-lemma:good2,eq:main-lemma:bad2} imply with probability at least $\gamma/2 - \frac{512}{n \gamma^2}$ over $t = (i,z,w)\in (I,Z,W)$, we have $\size{\mu_t^* -  \mu_t^{z_i}} \leq  \frac{\gamma}{16}$ and $\size{\mu_t^{z_i}- \mu_t^{-z_i}} \geq\frac{\gamma}{2}$. Therefore, conditioned on event $E_t$ (i.e., $\forall x \in \set{-1,1,*}: \: \size{M_t^{x} - \mu_t^{x}} \leq \frac{\gamma}{16}$), we have for such triplets $t = (i,z,w)$ that 
	\begin{align*}
		\size{M_t^* - M_t^{-z_i}} 
		& \geq \size{\mu_t^{z_i} - \mu_t^{-z_i}} - \size{\mu_t^{z_i} - \mu_t^*} - \size{M_t^*  - \mu_t^*} -  \size{M_t^{-z_i} - \mu_t^{-z_i}}\\
		&\geq \frac{\gamma}{2} - 3\cdot \frac{\gamma}{16}\\
		&> \gamma/4,
	\end{align*}
	and 
	\begin{align*}
		\size{M_t^* - M_t^{z_i}}
		&\leq  \size{M_t^* - \mu_t^*} + \size{\mu_t^* - \mu_t^{z_i}} + \size{\mu_t^{z_i} - M_t^{z_i}} \\
		&\leq 3\cdot \frac{\gamma}{16}\\
		&< \gamma/4.
	\end{align*}
	
	Thus, we conclude that
	\begin{align*}
		\lefteqn{\pr{\Gc^F(I,Z_{-I},W) = Z_I}}\\
		&= \eex{(i,z,w) \la (I,Z,W)}{\pr{\Gc^F(i,z_{-i},w) = z_i}}\\
		&= \eex{t = (i,z,w) \la (I,Z,W)}{\pr{\set{\size{M_t^* - M_t^{-z_i}} > \gamma/4} \land \set{\size{M_t^* - M_t^{z_i}} < \gamma/4}}}\\
		&\geq \paren{1 - \frac1{2n}}\cdot \eex{t = (i,z,w) \la (I,Z,W)}{\pr{\set{\size{M_t^* - M_t^{-z_i}} > \gamma/4} \land \set{\size{M_t^* - M_t^{z_i}} < \gamma/4} \mid E_t}}\\
		&\geq \paren{1 - \frac1{2n}}\cdot \paren{\gamma/2 - \frac{512}{n \gamma^2}}\\
		&\geq \gamma/4 - \frac{512}{n \gamma^2},
	\end{align*}
	which proves \cref{eq:predic-lemma:LB}. The first inequality holds since $\pr{E_t} \geq 1-\frac1{2n}$ for every $t = (i,z,w)$, and the second one holds by the observation above.
	
\end{proof}


\end{document}
