\def \IsDraft{} % set for draft mode


%%%%%%%%%%%%%%%%%%%%%%%%%%%%%%%%%%%%%%%%%%%%%%%%%%%%%%%%%%%%%%%%%%%%%%%%%%%%%%%%
% FLAGS
%

\newif\ifdraft
%\drafttrue
\draftfalse

\newif\ifinc
%\inctrue
\incfalse

\newif\ifans
%\anstrue
\ansfalse

\newif\ifsample
%\sampletrue
\samplefalse

\newif\iffull
\fulltrue 
%\fullfalse 
%%%%%%%%%%%%%%%%%%%%%%%%%%%%%%%%%%%%%%%%%%%%%%%%%%%%%%%%%%%%%%%%%%%%%%%%%%%%%%%%
 

\providecommand{\remove}[1]{}




\iffull
	\documentclass[11pt,letterpaper]{article}
\else
	\documentclass[letter,envcountsame,envcountsect]{llncs}
    
\fi




%%%%%%%%%%%%
% Packages %
%%%%%%%%%%%%

\iffull
 	\usepackage{fullpage}
 	\usepackage[margin=4mm, small, labelfont=bf]{caption}
	\ifdraft 
		\usepackage[draft,notes=true,later=false]{dtrt} 
	\else
		\usepackage[notes=true,later=false]{dtrt} 
    \fi
 
\usepackage[natbib=true,backend=bibtex,backref=true,style=alphabetic,maxnames=2,maxalphanames=2,maxbibnames=99, sorting=anyt]{biblatex}
\else
\usepackage[notes=xxx,llncssubsub]{dtrt}

\usepackage[style=alphabetic,firstinits,backend=bibtex,maxalphanames=2,minalphanames=2,maxbibnames=6,natbib=true]{biblatex}


\fi

\bibliography{crypto}

\iffull
\usepackage{titlesec}
\usepackage{amsthm}
\else
\let\proof\relax
\let\endproof\relax
\usepackage{amsthm}
\fi


\remove{
\usepackage[
pagebackref,backref=true,
hyperfootnotes=false,
colorlinks=true,
urlcolor=externallinkcolor,
linkcolor=internallinkcolor,
filecolor=externallinkcolor,
citecolor=internallinkcolor,
breaklinks=true,
pdfstartview=FitH,
pdfpagelayout=OneColumn]{hyperref}
}

%\usepackage{aliascnt}


%\usepackage[backend=bibtex,style=ieee-alphabetic,natbib=true,backref=true]{biblatex}
%\addbibresource{crypto.bib}


%\usepackage[numbers]{natbib} % \citet{foo} -> Foo et al. P




\usepackage{mathtools}
\usepackage{bbm}
%\usepackage{xcolor}
\usepackage{amsthm,amsfonts,amsmath,amssymb,color,amsthm,boxedminipage,url,xparse}
\definecolor{internallinkcolor}{rgb}{0,.5,0}
\usepackage{amsthm}
% in order to remove warnings such as "destination with the same identifier (name{corollary.1.3}) has been already used, duplicate ignored" the order of packages should be hyperref, amsthm, cleveref
%\usepackage{enumerate,paralist}
%\usepackage{enumitem}

\usepackage{xspace}
\usepackage{nicefrac}

\usepackage{cleveref}


\usepackage{amssymb}
\usepackage{amsmath}
\usepackage{bbm}
\usepackage{paralist}
\usepackage{enumitem}

\setlist[itemize]{leftmargin=*}
\setlist[enumerate]{leftmargin=*}

\usepackage{mathtools}
\usepackage{paralist}


%%%%%%%%%%
% Macros %
 %%%%%%%%%%



\newcommand{\thought}[1]{{\color[rgb]{0.2,0.39,0.66}(#1)}}
\newcommand{\todo}[1]{{\color[rgb]{1.0,0.0,0.0}(#1)}}
\newcommand{\hsh}[1]{{\color{green!50!black} Henrik: #1}}
\newcommand{\st}[1]{{\color{red!50!black} Sebastian: #1}}

\newcommand{\ulm}[1]{_{\scaleto{\mathrm{#1}}{3pt}}}
\newcommand\at[2]{\left.#1\right|_{#2}}











\newtheorem{assumption}{Assumption}

\DeclareMathOperator*{\argmax}{arg\,max}
\DeclareMathOperator*{\argmin}{arg\,min}

\newcommand{\swname}[1]{\texttt{#1}}
\newcommand{\ie}{i\/.\/e\/.,\/~}
\newcommand{\eg}{e\/.\/g\/.,\/~}
\newcommand{\cf}{cf\/.\/~}

\newcommand{\fig}{Fig\/.\/~}
\newcommand{\defn}{Def\/.\/~}
\newcommand{\sect}{Sec\/.\/~}
\newcommand{\tabl}{Tab\/.\/~}
\newcommand{\algo}{Algorithm~}
\newcommand{\theo}{Theorem~}

\newcommand{\bnnl}{3 hidden layers}
\newcommand{\bnnn}{50 neurons}
\newcommand{\bnna}{tanh activations}

\newcommand{\capt}[1]{\mdseries{\emph{#1}}}

\newcommand{\videolink}{at \url{https://youtu.be/_d7AqTRjz6g}}
\newcommand{\codelink}{\url{https://github.com/wheelbot/mini-wheelbot}}

\newcommand{\fakepar}[1]{\vspace{0mm}\noindent\textbf{#1.}}

\newcommand{\needref}{\textcolor{red}{[REF]}}

\newcommand{\plotfontsize}{9pt}


\newcommand{\Dist}{\MathAlgX{Dist}}
\newcommand{\Pred}{\MathAlgX{Pred}}
\newcommand{\Rec}{\MathAlgX{Rec}}

\newcommand{\EveDP}{\MathAlgX{\Dist}}
\newcommand{\RecBit}{\MathAlgX{\Rec}}
\newcommand{\tRecBit}{\MathAlgX{\widetilde{\Rec}}}
%\newcommand{\AlgEstimateBit}{\MathAlgX{\Rec}}
\newcommand{\AlgEstimateBit}{\MathAlgX{EstBit}}

\newcommand{\KLD}[2]{\KL({#1}||{#2})}

\newcommand{\Bn}{\cB_n}
\newcommand{\Cn}{\cC_n}
\newcommand{\Dn}{\cD_n}

\newcommand{\In}{I_n}
\newcommand{\Xn}{X_n}
\newcommand{\Yn}{Y_n}
\newcommand{\Zn}{Z_n}
\renewcommand{\Rn}{\cR_n}
\newcommand{\Pn}{\Pc_n}
\newcommand{\ensm}[1]{\set{#1}_{n\in \N}}


\newcommand{\HH}{H}



\renewcommand{\nu}{n.u.\xspace}
\newcommand{\nupt}{\nu-poly-time\xspace}
%\newcommand{\nuptm}{\nupt algorithm\xspace}

%\newcommand{\NBPE}{NBPE\xspace}
%\newcommand{\nuNBPE}{$\nu$-\NBPE}


\newcommand{\Inote}[1]{\authnote{Iftach}{#1}}
\newcommand{\Jnote}[1]{\authnote{Jad}{#1}}
\newcommand{\Nnote}[1]{\authnote{Noam}{#1}}
\newcommand{\Enote}[1]{\authnote{Eliad}{#1}}
\newcommand{\Cnote}[1]{\authnote{Chao}{#1}}


\title{
Mildly Accurate Computationally Differentially Private Inner Product Protocols Imply Oblivious Transfer
}
%Oblivious Transfer and Accurate Computationally Differentially Private Inner Product Protocols are Equivalent

\iffull
\author{
Iftach Haitner\thanks{Stellar Development Foundation and Tel Aviv University. {\tt iftachh@tauex.tau.ac.il}.} 
%Research supported by Israel Science Foundation grant 666/19.
\and
Noam Mazor\thanks{Tel Aviv University. {\tt noammaz@gmail.com}. Research partly supported by NSF CNS-2149305, AFOSR
Award FA9550-23-1-0312 and AFOSR Award FA9550-23-1-0387 and ISF Award 2338/23.}
\and
Jad Silbak\thanks{Northeastern University. {\tt jadsilbak@gmail.com}. Research supported by the Khoury College Distinguished Post-doctoral Fellowship.}
\and
Eliad Tsfadia\thanks{Georgetown University. {\tt eliadtsfadia@gmail.com}. Research supported by a gift to Georgetown University.}
\and
Chao Yan\thanks{Georgetown University. {\tt cy399@georgetown.edu}. Research supported by a gift to Georgetown University.}
}



%\else
%\author{}
\date{}
%\institute{}
\fi

\begin{document}

\maketitle



\begin{abstract}

In distributed differential privacy, multiple parties collaboratively analyze their combined data while protecting the privacy of each party's data from the eyes of the others. Interestingly, for certain fundamental two-party functions like inner product and Hamming distance, the accuracy of distributed solutions significantly lags behind what can be achieved in the centralized model. However, under computational differential privacy, these limitations can be circumvented using oblivious transfer via secure multi-party computation.
Yet, no results show that oblivious transfer is indeed necessary for accurately estimating a non-Boolean functionality. 
In particular, for the inner-product functionality, it was previously unknown whether oblivious transfer is necessary even for the best possible constant additive error.

In this work, we prove that any computationally differentially private protocol that estimates the inner product over $\oo^n \times \oo^n$ up to an additive error of $O(n^{1/6})$, can be used to construct oblivious transfer. In particular, our result implies that protocols with sub-polynomial accuracy are equivalent to oblivious transfer. In this accuracy regime, our result improves upon \citeauthor*{HaitnerMST22} [STOC '22] who showed that a key-agreement protocol is necessary.


\keywords{differential privacy; inner product; oblivious transfer}
\end{abstract}

\iffull
\Tableofcontents
\fi

% 
% 
The widespread integration of communication networks and smart devices in modern control systems has increased the vulnerability of industrial systems to online cyber-attacks, e.g., Industroyer, Blackenergy, etc \citep{osti_1505628}.
% Modern control systems have seen a large push to include communication networks and smart devices to increase performance, made possible by improvements in communication device cost and energy consumption. This trend has been coupled with the usage of open-standard communication protocols among industrial control systems, making them vulnerable to online cyber-attacks such as Industroyer, Blackenergy, etc \citep{osti_1505628}. 
To counter this, methods have been developed to improve security by achieving attack detection, mitigation, and monitoring, among others \citep{sandberg2022secure}. This paper focuses on active attack diagnosis to mitigate stealthy attacks. 
%
%\subsection{Literature review}

Active diagnosis techniques rely on the inclusion of additional moduli to control systems
% inclusion within the control system of additional moduli 
to alter the behavior of the system compared to information known by the attacker. 
For instance, the concept of additive watermarking was introduced in \cite{mo2015physical}, where noise signals of known mean and variance are added at the plant and compensated for it at the controller. 
This compensation, however, is not exact, causing some performance degradation. Thus, trade-offs between performance and detectability  are necessary \citep{zhu2023detection}.
% A later work \citep{zhu2023detection} designs the watermark signal by trading performance for detection. Thus, although additive watermarking serves as a good detection scheme, they endure performance losses even in the nominal case. 

In encrypted control \citep{darup2021encrypted}, the sensor data is encrypted, sent to the controller, and then operated on directly. Encrypted input signals are sent back to the plant for decryption. Although encryption is widespread in IT security, in control systems it presents some concerns, such as the introduction of time delays \citep{stabile2024verifiable}, while it may present inherent weaknesses \citep{alisic2023model}.
% they are not preferred as they introduce time delays \citep{stabile2024verifiable} which can cause instability, and some encryption schemes can be very weak  \citep{alisic2023model}. 

In moving target defense \citep{griffioen2020moving}, the plant is augmented with fictitious dynamics, known to the controller. The plant output is transmitted to the controller along with the fictitious states over a network under attack. 
The additional measurements then aide in the detection of attacks. 
This comes at the cost of higher communication bandwidth needs, which increases rapidly with the dimension of the augmented systems.
% Since the dynamics of the fictitious dynamics are exactly known to the controller, the attack is detected easily. However, when the scale of the system increases, the communication bandwidth used by moving the target defense approach increases rapidly. 

Other recently proposed works include two-way coding \citep{fang2019two}, a weak encryuption technique, and dynamic masking \citep{abdalmoaty2023privacy}, which enhances privacy as well as security, have been shown to be effective against zero-dynamics attacks.
% Two-way coding \citep{fang2019two} and dynamic masking \citep{abdalmoaty2023privacy} are other recently proposed approaches. Two-way coding is another form of weak encryption technique whilst dynamic masking proposes an architecture that enhances both privacy and security. These schemes are shown to be effective against zero dynamics attacks but remain to be studied for other classes of attacks. 
% Recent extensions include \citep{mukherjee2021secure,ramos2024privacy}.
% Some other works which are related are \citep{mukherjee2021secure}, an extension of \cite{fang2019two}. The work \citep{ramos2024privacy} is an extension of moving target defense for multi-agent systems. 
Furthermore, filtering techniques for attack detection are proposed by \cite{murguia2020security,hashemi2022codesign,escudero2023safety}, while not focusing on stealthy attacks.
% The works \citep{murguia2020security,hashemi2022codesign,escudero2023safety} develop filtering techniques to guarantee safety, without being focused on stealthy covert attacks.

Multiplicative watermarking (mWM) has been proposed by the authors as a diagnosis technique \citep{ferrari2020switching}. mWM consists of a pair of filters on each communication channel between the plant and its controller; the scheme is affine to weak encryption, whereby ``encoding'' and ``decoding'' are done by changing signals' dynamic characteristics through inverse pairs of filters. This enables original signals to be recovered exactly, and thus does not lead to performance degradation.
% A multiplicative watermark is an affine to a weak encryption technique, through which the signal is ``encoded'' by a filter, changing its dynamic behavior. The use of inverse pairs means that the original signal can be recovered, through ``decoding'' via an inverse filter. As such, differently to techniques based on additive watermarking, no performance is lost due to the injection of noise, and there are no bandwidth limitations.

%\subsection{Contributions}
One of the critical features of multiplicative watermarking is that to detect stealthy attacks, the mWM filter parameters must be switched over time. In this paper, an algorithm to optimally design the mWM parameters after a switching event is presented, enhancing detection performance, without changing the switching time.
% This is done without changing the switching time, which is taken as given.

\textcolor{black}{
To formalize the filter design problem, we suppose the defender is interested in optimal performance against adversaries injecting covert attacks with matched system parameters \citep{smith2015covert}, including the mWM parameters prior to the switch. This scenario represents a worst case where malicious agents can take full control of the system while remaining undetected.
Thus, the attack strategy is explicitly included within the formulation of the closed-loop system, and the mWM filters are chosen by solving an optimization problem minimizing the attack-energy-constrained output-to-output gain (AEC-OOG) \citep{anand2023risk}, a variation of the output-to-output gain proposed in  \cite{teixeira2015strategic}.
}
The main contributions of this paper are:
% We consider an adversary injecting a covert attack with matched system parameters \citep{smith2015covert}, i.e., an attacker with full knowledge of the control system parameters, including those of the mWM filters before the switch. This scenario is taken as a worst case, as it has been shown that this class of attacks can be made stealthy. To quantitatively define a cost, the output-to-output gain (OOG) \citep{teixeira2015strategic} is leveraged,
% a metric introduced to evaluate the impact of an additive attack in a control system. %Specifically, OOG evaluates the worst-case performance loss that an attacker injecting an undetectable attack can obtain. 
% Here, the maximum performance loss caused by a stealthy adversary with limited energy is taken, the attack-energy-constrained OOG (AEC-OOG) \citep{anand2023risk}. The main contributions of this paper are:
\begin{enumerate}
%[label=\alph*.]
\item The problem of optimally designing the switching mWM filters is formulated as an optimization problem, with the AEC-OOG is taken as the objective;%where the AEC-OOG is taken as the impact metric; 
\item The worst-case scenario of a covert attack with exact knowledge of plant and mWM filter parameters is embedded within the design problem;
% The optimization problem is defined to incorporate the worst-case scenario of a covert attack with exact knowledge of plant and mWM filter parameters;
\item The feasibility of the optimization problem is shown to be dependent only on stability conditions; 
\item A solution scheme is proposed to promote randomization of the mWM filter parameters such that an eavesdropping adversary cannot remain stealthy.
\end{enumerate} 

This builds on the results of \cite{ferrari2020switching}, where the focus was on the design of the switching protocols, rather than the parameters themselves.
Compared to previous work \citep{gallo2021design}, this paper introduces an optimization problem which is always feasible (thanks to the use of AEC-OOG in the objective), while also considering a more sophisticated class of covert attacks, where the presence of watermark is known to the adversary. 
Moreover, this paper poses a different objective than \citep{zhang2023hybrid}; indeed, while \citep{zhang2023hybrid} provided a design strategy to ensure certain privacy properties, in this paper we address the problem of optimal parameter design following a switching event.


%\subsection{Organization}
The rest of the paper is organized as follows. 
After formulating the problem in Section~\ref{sec:PF}, we propose our design algorithm in Section~\ref{sec:main}, and analyze its properties. It is then evaluated through a numerical example in Section~\ref{sec:NE}, and concluding remarks are given Section~\ref{sec:Con}.
% We provide the problem background in Section~\ref{sec:PF}. We formulate the design problem in Section~\ref{sec:main}, together with an analysis of its properties. The proposed algorithm is evaluated through a numerical example in Section \ref{sec:NE}. Concluding remarks are offered in Section \ref{sec:Con}.
\section{System Overview}
\label{sec:overview}

In this section, we present the control system architecture of the proposed framework, shown in Fig. \ref{fig:controlArchi}. 
Empirically, humanoid kino-dynamics MPC explicitly optimizes the joint states through kinematics constraints \cite{gu2025humanoid}, while traditional centroidal-dynamics MPC often requires subsequent inverse kinematics solver or whole-body control for motion execution. Both approaches employ nonlinear approaches to solve the optimization problem. In our framework, we proposed a Gait-Net-augmented sequential CMPC algorithm that translates the original nonlinear problem into convex sequential subproblems. With the additional assistance of Gait-Net, we reduce the optimization variable and mimic a natural step duration decision in each iteration. 

The control framework converts user commands and contact sequence into joint space references $\{\mathbf q_k^\text{ref} \in \mathbb R^{6+n_j},\: \dot{\mathbf q}_k^\text{ref} \in \mathbb R^{6+n_j}\}^h_{k = 0}$ and foot location reference trajectory $\{\bm p_f^\text{ref}\in \mathbb R^{3n_i}\}^h_{k = 0}$, where $n_j$ is the number of joints, $n_i$ is the number of contact/foot, and $h$ is a finite number of horizon. These joint-space trajectories, along with joint-space feedback states, are then translated into spatial momenta $\bm h\in \mathbb R^6$ and their primitive, the centroidal pose $\bm H\in \mathbb R^6$, which are the state variables used in the Gait-Net-augmented kino-dynamic MPC. Within the MPC, we break down the nonlinear dynamics constraints into sequential CMPC subproblems that can be solved through QP solvers. In each sequential iteration $j$, the Gait-Net predicts and updates the MPC sampling time $dt$ towards convergence and enables variable-frequency walking.
The spatial momentum and pose trajectories are updated at each iteration to reflect the kinematic configuration based on the iterative solution of $dt$, CoM location $\bm p_c \in \mathbb R^3$, and foot locations $\bm p_f\in \mathbb R^{3n_i}$,
providing a kinematically feasible reference. Once the terminal condition is met in the custom sequential solver, the control inputs are then mapped to motor commands in low-level control, which incorporates standard techniques such as inverse kinematics, contact Jacobian mapping, and joint-PD swing leg control \cite{di2018dynamic}. Notably, the full Gait-Net-augmented Kino-dynamic MPC is run at the beginning of each footstep to determine the step duration, the rest of the duration will incorporate the kino-dynamic MPC with the same MPC $dt$ throughout this very footstep. 


 


\section{Preliminaries}\label{sec:preliminaries}



%We denote by $(\Ac(x_\Ac),\Bc(x_\Bc))(z)$ a random execution of $\pi$ with private inputs $(x_\Ac,y_\Ac)$, and common input $z$.

%\Jnote{Move to DP}
% At the end of such an execution, the protocol outputs a public transcript denoted by the random variable $\trans_\pi(x_\Ac,x_\Ac,z)$ we denotes the common as $\out(\trans_\pi(x_\Ac,x_\Ac,z)$, and each party $\Pc \in \set{\Ac,\Bc}$ obtains his view denoted $\view^\Pc_\pi(x_\Ac,x_\Bc,z)$, which may also contain a ``local output'' \Jnote{Local} $\out^\Pc(x_\Ac,x_\Bc,z)$ (if the protocol specifies such an output). \Jnote{Common output, and parties output}


\subsection{Distributions and Random Variables}\label{sec:prelim:dist}
The support of a distribution $P$ over a finite set $\cS$ is defined by $\Supp(P) \eqdef \set{x\in \cS: P(x)>0}$. For a distribution or a random variable $D$, let $d\from D$ denote that $d$ was sampled according to $D$. Similarly,  for a set $\cS$, let $x \from \cS$ denote that $x$ is drawn uniformly from $\cS$, and denote by $\cU_{\cS}$ the uniform distribution over $\cS$. For a finite set $\cX$ and a distribution $C_X$ over $\cX$, we use the capital letter $X$ to denote the random variable that takes values in $\cX$ and is sampled according to $C_X$. The {\sf statistical distance} (\aka {\sf~variation distance}) of two distributions $P$ and $Q$ over a discrete domain $\cX$ is defined by $\sdist{P}{Q} \eqdef \max_{\cS\subseteq \cX} \size{P(\cS)-Q(\cS)} = \frac{1}{2} \sum_{x \in \cS}\size{P(x)-Q(x)}$. 
For a vector $x = (x_1,\ldots,x_n)$ and index $i\in [n]$, we let $x_{-i} = (x_1,\ldots,x_{i-1},x_{i+1},\ldots,x_n)$ and $x^{(i)} = (x_1,\ldots,x_{i-1}, -x_i, x_{i+1},\ldots,x_n)$, for a set $\cS \subseteq [n]$ we let $x_{\cS} = (x_i)_{i \in \cS}$ and $x_{-\cS} = (x_i)_{i \in [n]\setminus \cS}$, and for a vector $r \in \zo^n$ we let $x_r = (x_i)_{\set{i \colon r_i = 1}}$ and $x_{-r} = (x_i)_{\set{i \colon r_i = 0}}$.

%For $n \in \N$ we let $U_n$ be the uniform distribution over $\oo^n$, and let $S_n$ be the distribution induces by the sum of $n$ i.i.d.\ random variables, each is distributed according to $U_1$. Let $\cN(0,1)$ be the standard normal distribution.
%For a distribution $\cD$ and a function $f$, we define by $f(\cD)$ the distribution that is induced by the output of $f(x)$ for $x \from \cD$. 





% \begin{theorem}[\cite{McGregorMPRTV10}]\label{thm:sv-extracotr}
% 	\Enote{Remove if not needed}
% 	There is a constant $c$ to make the following holds. Let $X$ be an $\alpha$-SV source on $\{0,1\}^n$, let $Y$ be a source on $\{0,1\}^n$ with min-entropy at least $\beta n$ (independent from $X$), and let $Z=\ip{X,Y}\mbox{mod m}$ for some $m\in\mathbb{N}$. Then for every $\delta\in[0,1]$, the random variable $(Y,Z)$ is $\delta$-close to $(Y,U)$ where $U$ is uniform on $\mathbb{Z}_m$ and independent of $Y$, provided that
% 	$$
% 	n\geq c\cdot\frac{m^2}{\alpha\beta}\cdot\log(\frac{m}{\beta})\cdot\log(\frac{m}{\delta}).
% 	$$
% \end{theorem}



\Enote{I removed the definition of DP since it already appears in the intro}
\remove{
\subsection{Differential Privacy}\label{sec:prelim:DP}
We use the following standard definition of (information theoretic) differential privacy, due to \citet{DMNS06}. For notational convenience, we focus on databases over $\oo$.
\begin{definition}[Differentially private mechanisms]\label{def:mech}
	A randomized function $f\colon\oo^n\mapsto \zs$ is an {\sf $n$-size, $(\eps,\delta)$-differentially private mechanism} (denoted $(\eps,\delta)$-\DP) if for every neighboring $w,w'\in \oo^n$ and every function $g\colon \zs\mapsto \zo$, it holds that 
	$$
	\pr{g(f(w))=1}\leq e^{\eps}\cdot \pr{g(f(w'))=1} +\delta.
	$$ 	
	If $\delta=0$, we omit it from the notation.
\end{definition}
}


\subsubsection{Computational Differential Privacy}
There are several ways for defining computational differential privacy (see \cref{sec:related-works}). We use the most relaxed version due to \cite{BNO08}. For notational convenience, we focus on databases over $\oo$.
\begin{definition}[Computational differentially private mechanisms]\label{def:ComMech}
	A randomized function ensemble $f=\set{f_\pk\colon\oo^{n(\pk)}\mapsto \zs}$ is an {\sf $n$-size, $(\eps,\delta)$-computationally differentially private} (denoted $(\eps,\delta)$-$\CDP$) if for every poly-size circuit family $\set{\Ac_\pk}_{\pk\in \N}$, the following holds for every large enough $\pk$ and every neighboring $w,w'\in\oo^{n(\pk)}$:
	$$
	\pr{\Ac_\pk(f_\pk(w))=1}\leq e^{\eps(\pk)}\cdot \pr{\Ac_\pk(f_\pk(w'))=1} +\delta(\pk).
	$$ 
	If $\delta(\pk) = \negl(\pk)$, we omit it from the notation. 
\end{definition}



\subsubsection{Two-Party Differential Privacy}\label{sec:DP}
In this section we formally define distributed differential privacy mechanism (\ie protocols). %For the ease of notation, we consider protocol with no common input.

\begin{definition}\label{def:DP}%\Nnote{fix security parameter}
	A two-party protocol $\Pi=(\Ac,\Bc)$ is {\sf $(\eps,\delta)$-differentially private}, denoted $(\eps,\delta)$-$\DP$, if the following holds for every algorithm $\Dc$: let $\V^\Pc(x,y)(\pk)$ be the view of party $\Pc$ in a random execution of $\Pi(x,y)(1^\pk)$. Then for every $\pk,n \in \N$, $x\in \oo^n$ and neighboring $y,y'\in\oo^n$:
	\begin{align*}
	\pr{\Dc(V^\Ac(x,y)(\pk))=1}\le e^{\eps(\pk)}\cdot \pr{\Dc(V^\Ac (x,y')(\pk))=1}+\delta(\pk),
	\end{align*} 
	and for every $y\in \oo^n$ and neighboring $x,x'\in\oo^{n}$:
	\begin{align*}
	\pr{\Dc(V^\Bc(x,y)(\pk))=1}\le e^{\eps(\pk)}\cdot \pr{\Dc(V^\Bc (x',y)(\pk))=1}+\delta(\pk).
	\end{align*} 	
	Protocol $\Pi$ is {\sf $(\eps,\delta)$-computational differentially private}, denoted $(\eps,\delta)$-$\CDP$, if the above inequalities only hold for a non-uniform \ppt $\Dc$ and large enough $\pk$. We omit $\delta = \negl(\pk)$ from the notation. \footnote{Note that define we give for two-party differentially private protocols is a semi-honest definition, in which we ask for the security to hold when the parties interact in an honest execution of the protocol. Since we are proving a lower bound, starting from this weaker guarantee (as opposed to security against malicious players), yields a stronger result.}
\end{definition}
%We omit $\delta$ from the notation if $\delta$ is a negligible function of $n$.

%\Enote{simulation-based}
\begin{remark}[The definition for computational differential privacy we use]\label{rem:comDPChannel} 
	An alternative, stronger definition of computational differential privacy, known as simulation-based computational differential privacy, requires that the distribution of each party’s view be computationally indistinguishable from a distribution that ensures privacy in an information-theoretic sense. \cref{def:DP} is a weaker notion in comparison. Consequently, establishing a lower bound for a protocol that satisfies this weaker guarantee (as we do in this work) yields a stronger result.%Actually, our lower bound only requires the privacy to hold against \emph{uniform} external observer.
	%\Nnote{Maybe add: When only interesting in \Dp against external observer, the two definitions can be achieve using key-agreement and (single-party) \Dp mechanism. }
\end{remark}




\subsection{Useful Claims}
\remove{
In this section, we state generic lemmas and propositions that we will use later in our proofs.

The following lemma which we prove in \cref{sec:missing-proofs:distance-I}, measures the distance between two uniform stings conditioned one a random index $i$ either being fixed to $0$ or to $1$.

\def\distanceILemma{
    Let $R \la \zo^n$. For any (randomized) function $f:\{0,1\}^n\rightarrow \{0,1\}$ and $\alpha > 0$, it holds that
    \begin{align}\label{eq:f-alpha}
        \ppr{i \la [n]}{\size{\:\ex{f(R) \mid R_i = 0}-\ex{f(R) \mid R_i = 1}\:}\geq \alpha} \leq \frac{2}{n \alpha^2},
    \end{align}
    where the expectations are taken over $R$ and the randomness of $f$.
}

\begin{lemma}\label{lem:distance-I}
    \distanceILemma
\end{lemma}
}

The following two propositions state that given the output of a differentially private function, it is not possible to predict well even a random index (even if all other indexes are leaked). The first proposition handles the information-theoretic case and the second handles the computation case. Both propositions are proven in \cref{sec:missing-proofs:hard-to-guess}. 

\def\propHardToGuessInf{
    Let $f\colon \oo^n \rightarrow \cY$ be an $(\eps,\delta)$-\DP function, let $g \colon [n] \times \oo^{n-1} \times \cY \rightarrow \set{-1,1,\bot}$ be a (randomized) function, and let $X = (X_1,\ldots,X_n) \la \oo^n$. Then the following holds for every $i \in [n]$ where $X_i^* = g(i,X_{-i},f(X_1,\ldots,X_n))$:
    \begin{align*}
        \pr{X_i^* = X_i} \leq e^{\eps}\cdot \pr{X_i^* = -X_i} + \delta.
    \end{align*}
}

\begin{proposition}\label{prop:hard-to-guess-inf}
    \propHardToGuessInf
\end{proposition}


\def\propHardToGuessComp{
    Let $f = \set{f_{\pk} \colon \oo^{n(\pk)} \rightarrow \zo^{m(\pk)}}_{\pk \in \bbN}$ be an $(\eps,\delta)$-\CDP function ensemble, and let $\set{g_{\pk}}_{\pk \in \bbN}$ be a poly-size circuit family. Then, for large enough $\pk$ and $X = (X_1,\ldots,X_{n(\pk)}) \la \oo^{n(\pk)}$, the following holds for every $i \in [n(\pk)]$ where $X_i^* = g_{\pk}(i,X_{-i},f_{\pk}(X_1,\ldots,X_n))$:
    \begin{align*}
        \pr{X_i^* = X_i} \leq e^{\eps(\pk)}\cdot \pr{X_i^* = -X_i} + \delta(\pk).
    \end{align*}
}

\begin{proposition}\label{prop:hard-to-guess-comp}
    \propHardToGuessComp
\end{proposition}





\remove{
\Enote{Chao's old statement:}
\begin{lemma}\label{lem:distance-I-old}
        Let $R \la \zo^n$. 
	For any function $f:\{0,1\}^n\rightarrow \{0,1\}$ and $\alpha<0.01$, it holds that
	$$
	\Pr_{i\la[n]}\left[\: \size{\:\mathbb{E}[f(R) \mid R_i = 0]-\mathbb{E}[f(R) \mid R_i = 1]\:}\geq \alpha\right]\leq \frac{2+2\log(\frac{1}{\alpha})}{n\alpha^2}.
	$$
\end{lemma}
\begin{proof}
	Define $S_1=\{r \in \zo^n \colon f(r)=1\}$. Then for any $i\in[n]$, we have
	$$
	\begin{array}{rl}
		\size{\mathbb{E}[f(R) \mid R_i = 0]-\mathbb{E}[f(R) \mid R_i = 1]}
		&=\size{\Pr[R\in S_1|R_i=0]-\Pr[R\in S_1|R_i=1]}\\
		&=\size{\frac{\Pr[R_i=0|R\in S_1]\cdot\Pr[R\in S_1]}{\Pr[R_i=0]}-\frac{\Pr[R_i=1|R\in S_1]\cdot\Pr[R\in S_1]}{\Pr[R_i=1]}}\\
		&=\frac{2\size{S_1}}{2^n}\size{\Pr[R_i=0|R\in S_1]-\Pr[R_i=1|R\in S_1]}
	\end{array}
	$$
	When $|S_1|\leq \alpha\cdot 2^{n-1}$, we have $\size{\mathbb{E}[f(R) \mid R_i = 0]-\mathbb{E}[f(R) \mid R_i = 1]}\leq\frac{2\size{S_1}}{2^n}\leq \alpha$ for any $i\in[n]$. Hence, in the following, we assume $|S_1|> \alpha\cdot 2^{n-1}$.

	%Define $I_{bad}=\{i|\size{\Pr[R_i=0|R\in S_1]-\Pr[R_i=1|R\in S_1]}>2\alpha\}$ and $k=\size{I_{bad}}$, then for any $i\notin I_{bad}$, we have 
    %$$
    %\begin{array}{rl}
    %    2\alpha&\geq \size{\Pr[R_i=0|R\in S_1]-\Pr[R_i=1|R\in S_1]}\\
    %    &=\size{\frac{\Pr[R\in S_1|R_i=0]\cdot\Pr[R_i=0]}{\Pr[R\in S_1]}-\frac{\Pr[R\in S_1|R_i=1]\cdot\Pr[R_i=1]}{\Pr[R\in S_1]}}\\
    %    &=\size{\Pr[R\in S_1|R_i=0]-\Pr[R\in S_1|R_i=1]}\cdot\frac{1}{2\Pr[R\in S_1]}\\
    %    &\geq \size{\mathbb{E}[f(R) \mid R_i = 0]-\mathbb{E}[f(R) \mid R_i = 1]}\cdot \frac{1}{2},
    %\end{array}
    %$$ 
    %where the last inequality is because $\Pr[R\in S_1]\leq 1$. So that $\size{\mathbb{E}}[f(R) \mid R_i = 0]-\mathbb{E}[f(R) \mid R_i = 1]\leq %4\alpha$.
    Define $I_{bad}=\{i \colon \size{\Pr[R_i=0|R\in S_1]-\Pr[R_i=1|R\in S_1]} \geq 2\alpha\}$ and $k=\size{I_{bad}}$, and denote $I_{bad}=\{i_1,\dots,i_k\}$. Define $(X_{i_1}, \ldots X_{i_k}) = (R_{i_1},\dots,R_{i_k})\mid_{R \in S_1}$. 
    Consider the min-entropy
	$$
	\begin{array}{rl}
		H_{min}(X_{i_1},\dots,X_{i_k})&\leq H(X_{i_1},\dots,X_{i_k})\\
		&\leq \sum_{j=1}^k H(X_{i_j})\\
		&\leq k\cdot \left(-(\frac{1}{2}+2\alpha)\cdot\log(\frac{1}{2}+2\alpha)-(\frac{1}{2}-2\alpha)\cdot\log(\frac{1}{2}-2\alpha)\right)\\
            &=k\cdot \left(-(\frac{1}{2}+2\alpha)\cdot(\log(1+4\alpha)-1)-(\frac{1}{2}-2\alpha)\cdot(\log(1-4\alpha)-1)\right)\\
            &=k\cdot \left(1-(\frac{1}{2}+2\alpha)\cdot\log(1+4\alpha)-(\frac{1}{2}-2\alpha)\cdot\log(1-4\alpha)\right),
		
	\end{array}
	$$
	where $H_{min}(Y)$ is the minimum entropy of $Y$ and $H(Y)$ is the Shannon entropy of $Y$.\Enote{add to preliminaries.}
        The third inequality holds since by the definition of $I_{bad}$, for every $j \in [k]$ it holds that $\size{\pr{X_{i_j} = 1}-\pr{X_{i_j} = 0}} > 2\alpha$, and therefore $H(X_{i_j}) \leq H(1/2 + 2\alpha)$\Enote{define}.
	
	Therefore, there exists $b_1,\dots,b_k\in\{0,1\}$, such that 
	
	\begin{align}\label{eq:min-entropy-result}
		\Pr\left[(R_{i_1},\ldots,R_{i_k}) = (b_1,\ldots,b_k) \mid R\in S_1\right]
		&= \pr{(X_{i_1},\ldots,X_{i_k}) = (b_1,\ldots,b_k)}\\
		&= 2^{-H_{min}(X_{i_1},\dots,X_{i_k})}\nonumber\\
		&\geq 2^{k\cdot \left(-1+(\frac{1}{2}+2\alpha)\cdot\log(1+4\alpha)+(\frac{1}{2}-2\alpha)\cdot\log(1-4\alpha)\right)}.\nonumber
	\end{align}
	
	Let $S_{bad}=\{r \in \zo^n  \colon \set{(r_{i_1},\ldots,r_{i_k}) = (b_1,\ldots,b_k)} \land \set{r\in S_1}\}$.
	It holds that
	\begin{align*}
		|S_{bad}|
		&= \size{S_1} \cdot \Pr\left[(R_{i_1},\ldots,R_{i_k}) = (b_1,\ldots,b_k) \mid R\in S_1\right]\\
		&\geq \alpha\cdot 2^{n-1}\cdot2^{k\cdot \left(-1+(\frac{1}{2}+2\alpha)\cdot\log(1+4\alpha)+(\frac{1}{2}-2\alpha)\cdot\log(1-4\alpha)\right)},
	\end{align*} 
	where the inequality holds by \cref{eq:min-entropy-result} and since $\size{S_1} \geq \alpha\cdot 2^{n-1}$.
	Notice that any string in $S_{bad}$ depends on at most $n-k$ bits. It implies that $|S_{bad}|\leq 2^{n-k}$. Therefore, we have
	$$
	\begin{array}{rl}
		&2^{n-k}\geq \alpha\cdot 2^{n-1}\cdot2^{k\cdot \left(-1+(\frac{1}{2}+2\alpha)\cdot\log(1+4\alpha)+(\frac{1}{2}-2\alpha)\cdot\log(1-4\alpha)\right)} \\
		\Rightarrow& n-k \geq \log \alpha+n-1+k\cdot \left(-1+(\frac{1}{2}+2\alpha)\cdot\log(1+4\alpha)+(\frac{1}{2}-2\alpha)\cdot\log(1-4\alpha)\right)\\
		\Rightarrow& 1-\log \alpha \geq k\cdot((\frac{1}{2}+2\alpha)\cdot\log(1+4\alpha)+(\frac{1}{2}-2\alpha)\cdot\log(1-4\alpha))\\
		\Rightarrow& 1-\log \alpha \geq k\cdot(4\alpha\cdot\log(1+4\alpha)+(\frac{1}{2}-2\alpha)\cdot\log(1-16\alpha^2))\\
        \Rightarrow& 1-\log\alpha \geq k\cdot(15.9\alpha^2-8\alpha^2+32\alpha^3)=k\cdot(7.9\alpha^2+32\alpha^3)>0.5k\alpha^2\\
		\Rightarrow& k\leq \frac{2-2\log \alpha}{\alpha^2} = \frac{2+2\log (1/\alpha)}{\alpha^2},
	\end{array}
	$$
	Where the third transition holds since 
	\begin{align*}
		\lefteqn{(\frac{1}{2}+2\alpha)\cdot\log(1+4\alpha)+(\frac{1}{2}-2\alpha)\cdot\log(1-4\alpha)}\\
		&= 4\alpha\cdot\log(1+4\alpha) + (\frac{1}{2}-2\alpha)\paren{\log(1+4\alpha)+\log(1-4\alpha)}\\
		&= 4\alpha\cdot\log(1+4\alpha)+(\frac{1}{2}-2\alpha)\cdot\log(1-16\alpha^2),
	\end{align*}
	and the forth transition holds since $4\alpha\cdot\log(1+4\alpha)+(\frac{1}{2}-2\alpha)\cdot\log(1-16\alpha^2) > 15.9\alpha^2-8\alpha^2+32\alpha^3$ for $\alpha < 0.01$.
	Thus, we conclude that 
	$$
	\Pr_{i\la[n]}\left[\size{\mathbb{E}[f(R) \mid R_i=0]-\mathbb{E}[f(R) \mid R_i = 1]}\geq \alpha\right]\leq \frac{k}{n}\leq \frac{2+2\log (1/\alpha)}{n\alpha^2}.
	$$
\end{proof}
}


\subsection{Channels and Two-Party Protocols}\label{sec:protocol}

\paragraph{Channels.}A channel is simply a distribution of a pair of tuples defined as follows. 
\begin{definition}[Channels]\label{def:channel} A {\sf channel} $C_{(X,U)(Y,V)}$ of size $\isize$ over alphabet $\Sigma$ is a probability distribution over $(\Sigma^\isize \times\zo^\ast) \times(\Sigma^\isize \times\zo^\ast)$. The ensemble $C_{(X,U)(Y,V)}= \set{C_{(X_\pk,U_\pk)(Y_\pk,V_\pk)}}_{\pk\in \N}$ is an $\isize$-size channel ensemble, if for every $\pk\in \N$, $C_{(X_\pk,U_\pk)(Y_\pk,V_\pk)}$ is an $\isize(\pk)$-size channel. %We denote a channel of size one by a \emph{single-bit} channel. 
We refer to $X$ and $Y$ as the {\sf local outputs}, and to $U$ and $V$ as the {\sf views}.	
\end{definition}

We view a  channel as the experiment in which there are two parties $\Ac$ and $\Bc$.  Party $\Ac$ receives ``output'' $X$ and ``view'' $U$, and party $\Bc$ receives ``output'' $Y$ and ``view'' $V$. Unless stated otherwise, the channels we consider are over the alphabet $\Sigma = \oo$. We naturally identify channels with the distribution that characterizes their output.








\subsubsection{Two-Party Protocols}

A two-party protocol $\Pi=(\Ac,\Bc)$ is \ppt if the running time of both parties is polynomial in their input length. We let $\Pi(x,y)(z)$ or $(\Ac(x),\Bc(y))(z)$ denote a random execution of $\Pi$ on a common input $z$, and private inputs $x,y$.%We assume \wlg that a protocol has a common output (part of its transcript).\Jnote{This is not really the case we consider in this paper..}

\begin{definition}[Oracle-aided protocols]\label{def:ChannelAidedProtocol}
	In a two-party protocol $\Pi$ with oracle access to a {\sf protocol} $\Psi$, denoted $\Pi^\Psi$, the parties make use of the \textit{next-message function} of $\Psi$.\footnote{The function that on a partial view of one of the parties, returns its next message.} In a two-party protocol $\Pi$ with oracle access to a {\sf channel} $C_{Z W}$, denoted $\Pi^C$, the parties can jointly invoke $C$ for several times. In each call, an independent pair $(z,w)$ is sampled according to $C_{Z W}$, one party gets $z$, the other gets $w$.
\end{definition}


\begin{definition}[The channel of a protocol]\label{def:ChannlOfProtocol}
	For a no-input two-party protocol $\Pi= (\Ac,\Bc)$, we associate the channel $C_\Pi$, defined by $\C_\Pi= C_{(X, U),(Y, V)}$, where $X$ and $Y$ are the local outputs of $\Ac$ and $\Bc$ (respectively) and
	$U$ and $V$ are the local views of $\Ac$ and $\Bc$ (respectively).
    
	For a two-party protocol $\Pi$ that gets a security parameter $1^\pk$ as its (only, common) input, we associate the channel ensemble $ \set{C_{\Pi(1^\pk)}}_{\pk\in \N}$. 
\end{definition}

\begin{definition}[$(\alpha,\gamma)$-Accurate channel]\label{def:accurate-func}
	A channel $C = C_{(X, U),(Y, V)}$ is {\sf $(\alpha,\gamma)$-accurate for the function $f$}, if $\ppr{C}{\size{\out(V)-f(X,Y)}\leq \alpha}\ge \gamma$, where $\out(V)$ is the designated output.
    A channel ensemble $C_{(X, U),(Y, V)}= \set{C_{(X_\pk, U_\pk),(Y_\pk, V_\pk)}}_{\pk\in \N}$ is  $(\alpha,\gamma)$-accurate for  $f$ if $C_{(X_\pk, U_\pk),(Y_\pk, V_\pk)}$ is $(\alpha(\pk),\gamma(\pk))$-accurate for $f$, for every $\pk \in \N$.
\end{definition}

\subsubsection{Differentially Private Channels}\label{sec:DPChannel}
Differentially private channels are naturally defined as follows:
\begin{definition}[Differentially private channels]\label{def:DPChannel}
	An $n$-size channel $C = C_{(X, U),(Y, V)}$ with $X, Y$ over $\oo^n$ 
	is {\sf$(\eps,\delta)$-differentially private} (denoted $(\eps,\delta)$-$\DP$) if for every $x \in \Supp(X)$ there exists an $n$-size $(\eps,\delta)$-$\DP$ mechanisms $\Mc_x$ such that $(X,Y,U) \equiv (X,Y,\Mc_X(Y))$, and for every $y \in \Supp(Y)$ there exists an $n$-size $(\eps,\delta)$-$\DP$ mechanisms $\Mc_y'$ such that $(X,Y,V) \equiv (X,Y,\Mc_Y'(X))$. In addition, we say that the channel is \emph{uniform} if $X$ and $Y$ are independent random variables uniformly distributed in $\oo^n$. 
\end{definition}

\begin{definition}[Computational differentially private channels]\label{def:CDPChannel}
	An $n$-size channel ensemble $C = \set{C_{(X_\pk, U_\pk),(Y_\pk, V_\pk)}}_{\pk\in\N}$ with $X_\pk, Y_\pk$ over $\oo^n$ 
	is {\sf$(\eps,\delta)$-computationally differentially private} (denoted $(\eps,\delta)$-$\CDP$) if for every ensemble $\set{x_\pk \in \Supp(X_\pk)}_{\pk\in\N}$ there exists an $n$-size $(\eps,\delta)$-\CDP mechanisms ensemble $\set{\Mc_{x_\pk}}_{\pk\in\N}$ such that $(X_\pk,Y_\pk,U_\pk) \equiv (X_\pk,Y_\pk,\Mc_{X_\pk}(Y_\pk))$, for every $\pk\in\N$, and for every ensemble $\set{y_\pk \in \Supp(Y_\pk)}_{\pk\in\N}$ there exists an $n$-size $(\eps,\delta)$-$\CDP$ mechanisms ensemble $\set{\Mc'_{y_\pk}}_{\pk\in\N}$ such that $(X_\pk,Y_\pk,V_\pk) \equiv (X_\pk,Y_\pk,\Mc_{Y_\pk}'(X_\pk))$ for every $\pk\in \N$. In addition, we say that the channel is \emph{uniform} if $X_\pk$ and $Y_\pk$ are independent random variables uniformly distributed in $\{\pm 1\}^n$ for all $\pk\in\N$.
\end{definition}




% \begin{lemma}~\label{lem:dp-sv-source}
% 	Let $P$ be an $\varepsilon$-DP randomized protocol. Let $X$ and $Y$ be independent random variables uniformly distributed in $\{\pm 1\}^n$ and let random variable $\Pi(X,Y)$ denote the transcript of running $P(X,y)$. Then for every $\pi\in Supp(\Pi)$, the random variables corresponding to the inputs conditioned on transcript $\pi$, $X_\pi$ and $Y_\pi$, are independent $e^{-\varepsilon}$-strong SV source.
% \end{lemma}





\subsubsection{Weak Erasure Channel (\WEC)}

\begin{definition}[\WEC]\label{def:WEC}
	A channel $((O_A,V_A), (O_B,V_B))$ with $O_A \in \set{0,1}$ and $O_B \in \set{0,1,\bot}$ is a {\sf weak erasure channel}, denoted $(\alpha,p,q)$-$\WEC$, if:
	\begin{itemize}
		%\item $O_A\in \set{-1,1}$ and $O_B\in \set{-1,1,\bot}$.
		\item Random erasure: $\pr{O_B = \perp} = 1/2$.
		
		\item Agreement: $\pr{O_A\ne O_B\mid O_B\ne \bot}\le \alpha$.
		
		\item Secrecy:
		
		\begin{enumerate}
			\item For every algorithm $\Dc$ it holds that\label{WEC:item:A}
			\begin{align*}
				%\size{\pr{\Ac(O_A,V_A) = 1 \mid O_B \neq \perp} - \pr{\Ac(O_A,V_A) = 1 \mid O_B = \perp}} \le p
				\size{\pr{\Dc(V_A) = 1 \mid O_B \neq \perp} - \pr{\Dc(V_A) = 1 \mid O_B = \perp}} \le p
			\end{align*}
			(Alice doesn't know if $O_B = \perp$.)
			
			\item For every algorithm $\Dc$ it holds that\label{WEC:item:B}
			\begin{align*}
				\pr{\Dc(V_B) = O_A \mid O_B=\bot} \leq \frac{1+q}{2}.
			\end{align*}
			(i.e., if $O_B=\bot$, Bob don't know what is the value of $O_A$).
			
			%\item $SD((O_A U|O_B=\bot),(O_A U|O_B\ne \bot))\le p$ (The sender don't know if $O_B=\bot$).
			
			%\item $SD(V O_A|O_B=\bot,V(-O_A)|O_B=\bot)\le q$ (If $O_B=\bot$, Bob don't know what the value of $O_A$).
		\end{enumerate}
	\end{itemize}
   We say that a channel ensemble $C=\set{C_\pk}_{\pk\in N}$ is a {\sf computational weak erasure channel}, denoted $(\alpha,p,q)$-\CompWEC, if for every \ppt algorithm $\Dc$ and every sufficiently large $\pk\in\N$, $C_\pk$ satisfies the properties stated in the items above, where the secrecy property holds with respect to a \ppt algorithm $\Dc$. A protocol $\Lambda$ is said to be $(\alpha,p,q)$-$\CompWEC$, if the ensemble induces by the protocol (that is, $C=\set{C_{\Lambda(\pk)}}_{\pk\in\N}$) is $(\alpha,p,q)$-$\CompWEC$.  
\end{definition}



\subsubsection{Approximate Weak Erasure Channel (\AWEC)}\label{sec:AWEC}

\begin{definition}[\AWEC]\label{def:AWEC}
	A channel $C = ((O_A,V_A), (O_B,V_B))$ over $([-n,n] \times \zo^*) \times (([-n,n] \cup \bot)  \times \zo^*)$ is an {\sf approximate weak erasure channel}, denoted $(\ell,\alpha,p,q)$-\AWEC if:
	\begin{itemize}
		
		\item Random erasure: $\pr{O_B = \perp} = 1/2$.
		
		\item Accuracy: $\pr{\size{O_A - O_B} > \ell \mid O_B \ne \bot}\le \alpha$.
		
		\item Secrecy:
		
		\begin{enumerate}
			\item For every algorithm $\Dc$ it holds that\label{AWEC:item:A}
			\begin{align*}
				%\size{\pr{\Ac(O_A,V_A) = 1 \mid O_B \neq \perp} - \pr{\Ac(O_A,V_A) = 1 \mid O_B = \perp}} \le p
				\size{\pr{\Dc(V_A) = 1 \mid O_B \neq \perp} - \pr{\Dc(V_A) = 1 \mid O_B = \perp}} \le p
			\end{align*}
			(Alice doesn't know if $O_B=\bot$).
			
			\item For every algorithm $\Dc$ it holds that\label{AWEC:item:B}
			\begin{align*}
				\pr{\size{\Dc(V_B) - O_A} \leq 1000 \ell \mid O_B=\bot} \leq q.
			\end{align*}
			(i.e., if $O_B=\bot$, Bob can't estimate the value of $O_A$ with error $\leq 1000 \ell$).
		\end{enumerate}
	\end{itemize}
     We say that a channel ensemble $C=\set{C_\pk}_{\pk\in N}$ is a {\sf computational approximate weak erasure channel}, denoted $(\ell,\alpha,p,q)$-\CompAWEC, if for every \ppt algorithm $\Dc$ and every sufficiently large $\pk\in\N$, $C_\pk$ satisfies the properties stated in the items above. A protocol $\Gamma$ is said to be $(\ell,\alpha,p,q)$-$\CompAWEC$, if the ensemble induced by the protocol (that is, $C=\set{C_{\Gamma(\pk)}}_{\pk\in\N}$) is $(\ell,\alpha,p,q)$-$\CompAWEC$.  
\end{definition}

We will make use of the following lemma, which shows that for some choices of the parameters, \AWEC implies \WEC. The lemma is proven in \cref{sec:AWEC-to-WEC}.

\begin{lemma}\label{lemma:AWEC-to-WEC}
	For every $\ell> 0$, there exists a \ppt protocol $\Lambda = (\Pc_1,\Pc_2)$ such that given an oracle access to an $(\ell,\alpha,p,q)$-\AWEC $C$, the channel $\tilde{C}$ induced by $\Lambda^C$ is $(\alpha'=\alpha+0.001,\: p' = p ,\:  q' = 1/2 + 2(q+0.01))$-\WEC.
	Furthermore, the proof is constructive in a black-box manner:
	\begin{enumerate}
		\item There exists an oracle-aided \ppt algorithm $\Ec_1$ such that for every channel $C = ((\OA,\VA), (\OB,\VB))$ and algorithm $\Dc$ violating the \WEC secrecy property~\ref{WEC:item:A} of $\tilde{C}$, algorithm $\Ec_1^{\Dc}$ violates the \AWEC secrecy property~\ref{AWEC:item:A} of $C$.
		
		\item There exists an oracle-aided \ppt algorithm $\Ec_2$ such that for every channel $C = ((\OA,\VA), (\OB,\VB))$ and algorithm $\Dc$ violating the \WEC secrecy property~\ref{WEC:item:B} of $\tilde{C}$, algorithm $\Ec_2^{\Dc}$ violates the \AWEC secrecy property~\ref{AWEC:item:B} of $C$.
	\end{enumerate}
\end{lemma}

Since \cref{lemma:AWEC-to-WEC} is constructive, the following is an immediate corollary.
\begin{corollary}\label{cor:CompAWEC to CompWEC}
There exists an oracle aided \ppt protocol $\Lambda$, such that given a protocol $\Gamma$ that induces $(\ell,\alpha,p,q)$-\CompAWEC, it holds that $\Lambda^\Gamma$ is $(\alpha'=\alpha+0.001,\: p' = p ,\:  q' = 1/2 + 2(q+0.01))$-\CompWEC.  
\end{corollary}
\begin{proof}[Proof of \ref{cor:CompAWEC to CompWEC}]
Let $\Lambda$ be the \ppt algorithm guaranteed  by Lemma \ref{lemma:AWEC-to-WEC}. Given an $(\ell,\alpha,p,q)$-\CompAWEC protocol $\Gamma$, we define $\Lambda(\pk)={\Lambda^{\Gamma(\pk)}(\pk)}$. Assume towards a contradiction that $\Lambda$ is not a $(\alpha',p',q')$-\CompWEC. It follows that there exists a \ppt $\Dc$ that for infinity many $\pk\in\N$ contradicts one of the \WEC secrecy properties of channel ensemble $\set{C_{\Lambda(\pk)}}_{\pk\in\N}$. Fix $\pk\in\N$ for which this holds. By Lemma \ref{lemma:AWEC-to-WEC}, there exists a \ppt $\Ec^\Dc$ that for every such $\pk$  contradicts one of the secrecy properties of the channel $C_{\Gamma(\pk)}$. This implies that for infinity many $\pk\in\N$, $\Ec^\Dc$  contradict the secrecy of the channel ensemble $\set{C_{\Gamma(\pk)}}_{\pk\in\N}$, which is a contradiction since this would means that $\Gamma$ is not a $(\ell,\alpha,p,q)$-\CompAWEC.       
\end{proof}



\subsection{Oblivious Transfer (\OT)}

\paragraph{Secure Computation.}
We use the standard notion of securely computing a functionality, \cf  \cite{Goldreich04}.
\begin{definition}[Secure computation]\label{def:SFE}
	A two-party protocol {\sf securely computes a functionality $f$}, if it does so according to the real/ideal paradigm.   We add the term perfectly/statistically/computationally/non-uniform computationally, if the simulator's output is  perfect/statistical/computationally indistinguishable/  non-uniformly indistinguishable from  the real distribution.  The protocol have the above notions of security {\sf against semi-honest  adversaries}, if its security only  guaranteed to holds against an adversary that follows the prescribed protocol.   Finally, for the case of perfectly secure computation, we naturally apply the above notion also to the non-asymptotic case: the protocol with no security parameter perfectly  compute a functionality $f$.
	
	A two-party protocol {\sf securely computes a functionality ensemble $f$ with oracle to a channel $C$}, if it does so according to the above definition when the parties have access to a trusted party computing $C$. All the above adjectives naturally extend to this setting.
\end{definition}

\paragraph{Oblivious Transfer.}
The (one-out-of-two) oblivious transfer functionality is defined as follows.
\begin{definition}[oblivious transfer functionality $f_{\OT}$]\label{def:OTfunc}
	The oblivious transfer functionality over $\zo \times (\zs)^2$ is defined by  $f_{\OT} (i,(\sigma_0,\sigma_1)) = (\perp,\sigma_i)$.
\end{definition}
A protocol is $\ast$ secure OT,   for \\$\ast\in \set{\text{semi-honest statistically/computationally/computationally non-uniform}}$, if it  compute the $f_{\OT}$  functionality with $\ast$ security.





% \begin{definition}[Computational oblivious transfer, semi-honest model]
% A protocol $\Pi=(\Ac,\Bc)$ is a semi-honest 1-out-of-2 computational oblivious transfer (comp-OT) protocol if the following holds. Given a common input $1^{\pk}$, the parties $\Ac$ and $\Bc$ run the protocol $\Pi(1^\pk)$ (in an honest manner) and    
% $\Ac$ outputs $X=(m_1,m_2)\in \zo\times\zo$ and has a view $U$ and $\Bc$ outputs $Y=(i,\hat{m})\in\zo\times\zo$ and has a view $V$, and the following properties are satisfied:
% \begin{enumerate}
%     \item \textbf{Correctness:} 
%     $\pr{\hat{m}\neq m_i}<\negl(\pk).$ 
    
%     \item \textbf{A's Privacy:} For every \ppt $\Dc$ and every sufficiently large $\pk$:
%     $\pr{\Dc(V)=m_{i-1}}<(1+\negl(\pk))/2$
    
%     \item \textbf{B's Privacy:} For every \ppt $\Dc$ and every sufficiently large $\pk$:
%     $\pr{\Dc(U)=i}<(1+\negl(\pk))/2$  
% \end{enumerate}
% \end{definition}

We make use of the following useful results by Wullschleger on oblivious transfer amplification from weak channels.
\begin{theorem}[\cite{Wullschleger09}, from \WEC to statistically secure \OT]\label{thm:WEC TO OT IT}
    There exists an oracle aided protocol $\Pi$ such that the following holds: Given a $(\alpha,p,q)$-\WEC $C$, if $44(\alpha+p)\le 1-q$ then $\Pi^{C}(1^\pk)$ is a semi-honest statistically secure \OT.
\end{theorem}

The following computational version of \cref{thm:WEC TO OT IT} is implicit in \cite{Wullschleger09} and is based on the computational proof explicitly stated in \cite{Wul07} (see Section 6 in \cite{Wullschleger09} for discussion).   

\begin{theorem}[\cite{Wullschleger09,   Wul07}, from \CompWEC to computinally secure \OT]\label{thm:WEC TO OT Comp}
    There exists an oracle aided protocol $\Pi$ such that the following holds: Given a $(\alpha,p,q)$-\CompWEC protocol $\Lambda$, if $44(\alpha+p)\le 1-q$ then $\Pi^{\Lambda}$ is a semi-honest computational secure \OT.
\end{theorem}



% \begin{definition}[Computational 1-out-of-2 Oblivious Transfer, semi-honest model]
% A protocol $\Pi=(\Ac,\Bc)$ is a semi-honest 1-out-of-2 $(\eps,\alpha,\beta)$-oblivious transfer (OT) protocol if the following holds. 

% The parties $\Ac$ and $\Bc$ run the protocol (in an honest manner) and    
% $\Ac$ outputs $X=(m_1,m_2)\in \zo\times\zo$ and has a view $U$ and $\Bc$ outputs $Y=(i,\hat{m})\in\zo\times\zo$ and has a view $V$, and following properties are satisfied:
% \begin{enumerate}
%     \item \textbf{Correctness:} 
%     $\pr{\hat{m}\neq m_i}<\eps.$ 
    
%     \item \textbf{A's Privacy:} For every adversary $\Dc$:
%     $\pr{\Dc(V)=m_{i-1}}<(1+\alpha)/2$
    
%     \item \textbf{B's Privacy:} For every adversary $\Dc$: $\pr{\Dc(U)=i}<(1+\beta)/2$  
% \end{enumerate}
% \end{definition}
\section{Oblivious Transfer from $\DP$ Inner Product}\label{sec:main_theorems}

In this section, we state our main theorems. We first state and prove the information theoretic case in \cref{sec:DP TO OT IT}, and then prove the computational case in \cref{sec:DP TO OT Comp}.

\subsection{Information-Theoretic Case}\label{sec:DP TO OT IT}
The following is our main theorem (for the information theoretic case) which shows that given a sufficiently accurate and private $\DP$ channel, we can construct a statistically secure semi-honest oblivious transfer protocol.

\begin{theorem}\label{thm:DPIP-to-OT}
There exist constants $c_1,c_2>0$ and an oracle-aided \ppt protocol $\Pi$ such that the following holds for large enough $n \in \bbN$ and for 
$\eps \leq \log^{0.9} n$, $\delta \leq \frac1{3n}$, and $\ell \leq e^{-c_1  \eps}  c_2\cdot n^{1/6}$:
    Let $C = ((X,U),(Y,V))$ be an $(\eps,\delta)$-\DP channel with independent $X,Y \la \oo^n$ that is $(\ell,0.999)$-accurate for the inner-product functionality (i.e., $\ppr{C}{\size{O-\ip{X,Y}} \leq \ell} \geq 0.999$).
    Then $\Pi^C$ is a semi-honest statistically secure $\OT$ protocol.
\end{theorem}

To prove Theorem \ref{thm:DPIP-to-OT}, we will make use of the following technical Lemma.


\begin{lemma}\label{lemma:DPIP-to-AWEC}
        There exists a \ppt protocol $\Gamma=(\Ac,\Bc)$ such that for every $c_1,c_2,n,\eps,\delta,\ell$ and $C$ as in \cref{thm:DPIP-to-OT},
        the channel $\tilde{C}$ induced by the execution of $\Gamma^C$ is $(\ell, \alpha=0.001, p=0.001, q=0.001)$-\AWEC (\cref{def:AWEC}).
\remove{
	There exist constant $c_1,c_2 > 0$ and an oracle-aided \ppt protocol $\Gamma=(\Ac,\Bc)$ such that the following holds for large enough $n \in \bbN$ and for 
	$\eps \leq \log^{0.9} n$, $\delta \leq \frac1{3n}$, and $\ell \leq e^{-c_1  \eps}  c_2\cdot n^{1/6}$:
	Let $C = ((X,U),(Y,V))$ be an $(\eps,\delta)$-\DP channel with independent $X,Y \la \oo^n$ that is $(\ell,0.999)$-accurate for the inner-product functionality (i.e., $\ppr{C}{\size{O-\ip{X,Y}} \leq \ell} \geq 0.999$). Then the channel $\tilde{C}$ induced by the execution of $\Gamma^C$ is $(\ell, \alpha=0.001, p=0.001, q=0.001)$-\AWEC (\cref{def:AWEC}). 
}
    Furthermore, the proof is constructive in a black-box manner:
	\begin{enumerate}
		\item There exists an oracle-aided \ppt algorithm $\Act$ such that for every channel $C = ((X,U),(Y,V))$ and algorithm $\Ac$ violating the \AWEC secrecy property~\ref{AWEC:item:A} of $\tilde{C}$ (the channel of $\Gamma^C$), the following holds for $Y^*_i = \Act^{\Ac}(i,\: Y_{-i}, \: X, \: U)$:\label{item:privacy-of-Y}
		\begin{align*}
			\eex{i \la [n]}{\pr{Y^*_i = Y_i }} > e^{\eps} \cdot \eex{i \la [n]}{\pr{Y^*_i = -Y_i }} + \delta.
		\end{align*}
		
		%\Enote{The above two items suffice for breaking $(\eps,\delta)$-DP because if we let $p = \pr{Y^*_i = Y_i}$ and $q = \pr{Y^*_i = -Y_i}$ then $p \leq e^{\eps} q + \delta$, and if $q \geq e^{-\eps} \delta$ then we get that
			%$$\frac{p}{p+q} \leq \frac{e^{\eps} q + \delta}{e^{\eps} q + \delta + q} = \frac{1}{1 + \frac{1}{e^{\eps} + \delta/p}} \leq  \frac{1}{1 + \frac{1}{2\cdot e^{\eps} }} \leq \frac{2\cdot e^{\eps}}{2\cdot e^{\eps}+1}.$$}
		
		%\item \Enote{We should use Theorem 5.1 from our previous paper (or Theorem 6.2 directly).}
		
		
		\item There exists an oracle-aided \ppt algorithm $\Bct$ such that for every channel $C = ((X,U),(Y,V))$ and algorithm $\Bc$ violating the \AWEC secrecy property~\ref{AWEC:item:B} of $\tilde{C}$ (the channel of $\Gamma^C$),  the following holds for $X^*_i = \Bct^{\Bc, C}(i,\: X_{-i}, \: Y, \: V)$:\label{item:privacy-of-X}
		\begin{align*}
			\eex{i \la [n]}{\pr{X^*_i = X_i }} > e^{\eps} \cdot \eex{i \la [n]}{\pr{X^*_i = -X_i }} + \delta.
		\end{align*} 
	\end{enumerate}
\end{lemma}

The proof of \cref{lemma:DPIP-to-AWEC} is given in \cref{sec:DPIP_to_WAEC}. But first we use \cref{lemma:DPIP-to-AWEC} to prove \cref{thm:DPIP-to-OT}. 

\begin{proof}[Proof of \cref{thm:DPIP-to-OT}]
By \cref{lemma:DPIP-to-AWEC} there exists a \ppt protocol $\Gamma\coloneqq\Gamma^C$ such that channel $\tilde{C}\coloneqq C_{\Gamma(\pk)}$ is a $(\ell, \alpha=0.001, p=0.001, q=0.001)$-\AWEC. By \cref{lemma:AWEC-to-WEC} (and \cref{prop:hard-to-guess-inf}) there exists a \ppt protocol $\Lambda\coloneqq\Lambda^{\tilde{C}}$ such that the channel $\hhC=C_{\Lambda(\pk)}$ is a $(\alpha'=0.002,\: p' = 0.001 ,\:  q' = 1/2 + 0.022))$-\WEC. Since $44(\eps'+p')<1-q'$, by  \cref{thm:WEC TO OT IT} there exists a \ppt protocol $\Pi$, such that $\Pi^{\hhC}$ is a semi-honest statistically secure \OT, concluding the proof.
\end{proof}


\subsection{Computational Case}\label{sec:DP TO OT Comp}


In this section, we state and prove our results for the computational case. We show that a \CDP (computational differential private) protocol that estimates the inner product well implies a semi-honest computationally secure oblivious transfer protocol. 


\begin{theorem}[Restatement of \cref{thm:intro:main}]\label{thm:CDPIP-to-OT}
There exist constant $c_1,c_2 > 0$ and an oracle-aided \ppt protocol $\Pi$ such that the following holds for large enough $n \in \bbN$ and for 
$\eps \leq \log^{0.9} n$, $\delta \leq \frac1{3n}$, and $\ell \leq e^{-c_1  \eps}  c_2\cdot n^{1/6}$:
    Let $\Psi$ be an $(\eps,\delta)$-\CDP protocol that is $(\ell,0.999)$-accurate for the inner-product functionality. 
    Then $\Pi^\Psi$ is a semi-honest computationally secure oblivious transfer protocol.
\end{theorem}

By the result of \cite{GoldreichMW87}, in the computational setting, we can ``compile" any semi-honest computational oblivious transfer protocol into a protocol that is secure against any \ppt (malicious) adversary (assuming one-way functions). We state this formally in the following corollary.

\begin{corollary}\label{cor:mal OT}
   Let $\eps,\delta,\ell$ be as in \cref{thm:CDPIP-to-OT}. If there exists a protocol $\Psi$ that is $(\eps,\delta)$-\CDP  and is $(\ell,0.999)$-accurate for the inner-product functionality, then there exists a computationally secure oblivious transfer protocol. 
\end{corollary}
\begin{proof}[Proof of \cref{cor:mal OT}]
By \cref{thm:CDPIP-to-OT}, there exists a semi-honest computational secure oblivious transfer protocol $\Pi$. Note that by \cite{ImpagliazzoLu89}, $\Pi$ implies the existence of one-way functions and by \cite{GoldreichMW87}, using the one-way function, we can compile $\Pi$ into a (computational) \OT secure against arbitrary adversaries.
\end{proof}


\paragraph{Proof of \cref{thm:CDPIP-to-OT}.}

 In order to use \cref{lemma:DPIP-to-AWEC}, and similar to \cite{HaitnerMST22}, we first convert the protocol $\Psi$ into a (no input) protocol such that the \CDP-channel it induces, is uniform and accurately estimates the inner-product functionality. Such a transformation is simply the following protocol that invokes $\Psi$ over uniform inputs, and each party locally outputs its input. 
\begin{protocol}[$\hPsi^{\Psi} = (\hAc,\hBc)$]\label{prot:EDPtoSV}
	\item Common input: $1^\kappa$.
	%	\item Oracle: protocol $\Psi=(\Ac,\Bc)$.
	\item Operation:
	\begin{enumerate}
		
		\item $\hAc$ samples $x \gets \oo^{\pn(\kappa)}$ and $\hBc$ samples $y\gets \oo^{\pn(\kappa)}$. 
		
		\item The parties interact in \remove{a random execution protocol }$\Psi(1^\kappa)$, with $\hAc$ playing the role of $\Ac$ with private input $x$, and $\hBc$ playing the role of $\Bc$ with private input $y$.
		
		\item $\hAc$ locally outputs $x$ and $\hBc$ locally outputs $y$. 
	\end{enumerate}
\end{protocol}

Let $C$ be the channel ensemble induced by $\hPsi$, letting its designated output (the function $\out$) be the designated output of the embedded execution of $\Psi$. The following fact is immediate by definition.

\begin{proposition}\label{prop:EDP to SV}
	The channel ensemble $C$ is $(\eps,\delta)$-$\CDP$, and has the same accuracy for computing the inner product as protocol $\Psi$ has.
\end{proposition}


We first prove the computational version of \cref{lemma:DPIP-to-AWEC}.
\begin{claim}\label{claim:Comp DP to AWEC}
    There exist constant $c_1,c_2 > 0$ and an oracle-aided \ppt protocol $\Gamma=(\Ac,\Bc)$ such that the following holds for large enough $n \in \bbN$ and for 
	$\eps \leq \log^{0.9} n$, $\delta \leq \frac1{3n}$, and $\ell \leq e^{-c_1  \eps}  c_2\cdot n^{1/6}$:
    Let $\Psi$ be an $(\eps,\delta)$-\CDP protocol that is $(\ell,0.999)$-accurate for the inner-product functionality. Then the protocol $\Gamma^{\Psi}$ is an  $(\ell, \alpha=0.001, p=0.001, q=0.001)$-\CompAWEC protocol.
\end{claim}
\begin{proof}[Proof of \cref{claim:Comp DP to AWEC}]
Recall that $\Psi$ is an $(\eps,\delta)$-$\DP$ protocol that is $(\ell,0.999)$-acccurate for the inner-product functionality for 
	$\eps \leq \log^{0.9} n$, $\delta \leq \frac1{3n}$, and $\ell \leq e^{-c_1  \eps}  c_2\cdot n^{1/6}$. By \cref{prop:EDP to SV}, $\hPsi\coloneqq\hPsi^\Psi$ induces a channel ensemble $C=\set{C_\pk=((X_\pk,U_\pk),(Y_\pk,V_\pk))}_{\pk\in\N}$ that is $(\eps,\delta)$-\CDP and $(\ell,0.999)$-acccurate for the inner-product functionality. Let $\Gamma$ be the \ppt protocol guarrented by \cref{lemma:DPIP-to-AWEC}, we now claim that $\Gamma^{\Psi}(\pk)\coloneqq\Gamma^{C_\pk}(\pk)$ is an $(\ell, \alpha=0.001, p=0.001, q=0.001)$-\CompAWEC protocol. %That is, the ensemble $\tC=\set{\tC_\pk=(X,V)(}_{\pk\in\N}$
Assume towards contradiction that this does not hold, then by \cref{lemma:DPIP-to-AWEC}, there exists a \ppt $\Dc$, such that for infinity many $\pk\in\N$, $\Dc$ contradicts one of the two secrecy properties of $\CompAWEC$. Fix such $\pk\in\N$ and omit it from the notation when clear from the context, and without loss of generality, assume that $\Dc$ contradicts the first secrecy property of $\CompAWEC$ (the case where $D$ contradicts the second property is essentially identical). By the first item of \cref{lemma:DPIP-to-AWEC}, there exists a \ppt algorithm $\tilde{\Ac}$ such that 
$Y^*_i = \Act^{\Dc}(i,\: Y_{-i}, \: X, \: U)$  and it holds that:
\begin{align*}
			\eex{i \la [n]}{\pr{Y^*_i = Y_i }} > e^{\eps} \cdot \eex{i \la [n]}{\pr{Y^*_i = -Y_i }} + \delta.
\end{align*}
Thus, we get a contradiction since by \cref{prop:hard-to-guess-comp}, algorithm $\Act^{\Dc}$ breaks the \CDP property of $C$.  
\end{proof}


\begin{proof}[Proof of \cref{thm:CDPIP-to-OT}]
Recall that $\Psi$ is an $(\eps,\delta)$-$\DP$ protocol that is $(\ell,0.999)$-acccurate for the inner-product functionality. By \cref{claim:Comp DP to AWEC} there exists a \ppt protocol $\Gamma$ such that $\Gamma\coloneqq\Gamma^\Psi$ is an $(\ell, \alpha=0.001, p=0.001, q=0.001)$-\CompAWEC protocol. By \cref{cor:CompAWEC to CompWEC}, there exists a \ppt protocol $\Lambda$ such that $\Lambda\coloneqq \Lambda^\Gamma$ is $(\epsilon'=\epsilon+0.001,\: p' = p ,\:  q' = 1/2 + 2(q+0.01))$-\CompWEC. Finally, since $44(\eps'+p')<1-q'$ by  \cref{thm:WEC TO OT Comp} there exists a \ppt protocol $\Pi$, such that $\Pi^{\Lambda}$ is a semi-honest computationally secure \OT, as required.
\end{proof}
\newcommand{\GenRand}{\MathAlgX{GenRand}}
\newcommand{\GenView}{\MathAlgX{GenView}}

\section{\AWEC From \DP Inner Product (Proof of \cref{lemma:DPIP-to-AWEC})}\label{sec:DPIP_to_WAEC}

In this section, we show how to implement AWEC (\cref{def:AWEC}) from an $(\eps,\delta)$-DP channel that is accurate enough for the inner product functionality. We do that using a \ppt protocol and a constructive security proof.



%$C = ((X,U = (U',O)),(Y,V = (V',O)))$ with independent $X,Y \la \oo^n$ that is 
%$\ell(n)$-accurate for the inner product functionality (i.e., $\ppr{C}{\size{O-\ip{X,Y}} \leq \ell(n)} \geq 0.99$), for $\ell(n) = \tilde{\Theta}(n^{1/6})$. Importantly, our security proof is constructive. 


%\Cnote{we can set $m=O(\frac{(n\cdot\frac{e^\varepsilon+1}{e^\varepsilon})^{2/5}}{\log^{2/5}(n\cdot\frac{e^\varepsilon+1}{e^\varepsilon})})$, $k=O(m^{1/2})=O(\frac{n^{1/5}}{\log^{1/5}n})$, $\ell=O(k^{1/2})=O(\frac{n^{1/10}}{\log^{1/10}n})$}





The following protocol is used to prove \cref{lemma:DPIP-to-AWEC}.

\begin{protocol}[Protocol $\Pi = (\Ac,\Bc)$]\label{protocol:DPIP-to-AWEC}
	\item Oracle access: An $(\eps,\delta)$-DP channel $C =((X,U),(Y,V))$ with $X,Y \la \oo^n$\remove{ (i.e., \emph{uniform} channel)} that is $(\ell,0.999)$-accurate for the \emph{inner-product} functionality.
	\item Operation:
	\begin{enumerate}
		%\item  $\Ac$ samples $x\gets \mon$, and   $\Bc$ samples $y\gets \mon$.
		
		\item Sample $(x,u), (y,v) \la C$. $\Ac$ gets $(x,u)$ and $\Bc$ gets $(y,v)$. 
		
		\item  $\Ac$ samples  $r \la \zo^n$ and sends $(r,x_{r} = \set{x_i \colon r_i =1})$ to $\Bc$.
		
		\item $\Bc$ samples a random bit $b \la \zo$ and acts as follows:
		
		\begin{enumerate}
			
			\item If $b=0$, it sends $y_{-r}= \set{y_i \colon r_i =0}$ to $\Ac$ and outputs $o_B = \out(v) - \ip{x_r, y_r}$.
			
			\item~\label{step: add noise} Otherwise ($b=1$), it performs the following steps:\label{B_steps_in_abort}
			
			\begin{itemize}
				\item Sample $k$ uniformly random indices $i_1,\ldots,i_k \la [n]$, where $k = \floor{e^{\lambda_1 \eps} \cdot \lambda_2 \cdot \ell^2}$ for constants $\lambda_1,\lambda_2>0$ (to be determined later by the analysis in \cref{eq:lamdas}).
				\item Compute $\ty = (\ty_1,\ldots,\ty_n)$ where $\ty_i \la \begin{cases} \cU_{\oo} & i \in \set{i_1,\ldots,i_k} \\ y_i & \text{otherwise} \end{cases}$,
				\item Send $\ty_{-r}= \set{\ty_i \colon r_i =0}$ to $\Ac$, and output $o_B = \perp$.
				%\item Output $o_B = \perp$.
			\end{itemize}
			
			
		\end{enumerate}
		
		\item  Denote by $\hy_{-r}$ the message $\Ac$ received from $\Bc$. Then $\Ac$ outputs $o_A =  \ip{x_{-r}, \hy_{-r}}$.
		
		
	\end{enumerate}
\end{protocol}

In the following, let $\Pi$ be \cref{protocol:DPIP-to-AWEC},
let $C =((X,U),(Y,V))$ be a uniform channel that is $(\ell,0.999)$-accurate for the \emph{inner-product} functionality, let $\tC = ((O_A,V_A),(O_B,V_B))$ be the channel that is induced by $\Pi^C$ (i.e., the parties' outputs and views in the execution of $\Pi$ with oracle access to $C$), and let $R$, $\tY$, $\hY$, $I_1,\ldots,I_k$ be the random variables of the values of $r, \ty, \hy, i_1,\ldots,i_k$ in the execution of $\Pi^C$ (recall that the view of $\Bc$ in the execution is $V_B = (Y, V, R, X_R, I_1,\ldots,I_k, \tY)$, and the view of $\Ac$ is $V_A = (X, U, R, \hY_{-R})$).

Recall that to prove \cref{lemma:DPIP-to-AWEC}, we need to prove that the channel $\tC$ satisfies the accuracy and secrecy properties of \AWEC (\cref{def:AWEC}), and in addition, the secrecy guarantees are constructive as stated in Properties \ref{item:privacy-of-Y}-\ref{item:privacy-of-X} of Lemma~\ref{lemma:DPIP-to-AWEC} (i.e., a violation of at least one secrecy guaranty results with an efficient privacy attack on the DP channel $C$).

We first start with the easy part, which is to prove the accuracy guarantee of $\tC$.

\begin{claim}[Accuracy of $\tC$]
	It holds that
	\begin{align*}
		\pr{\size{O_A - O_B} > \ell \mid O_B \neq \bot} < 0.001. 
	\end{align*}
\end{claim}
\begin{proof}
    Compute
    \begin{align*}
        \pr{\size{O_A - O_B} > \ell \mid O_B \neq \bot}
        &= \pr{\size{\ip{X_{-R},Y_{-R}} - \paren{\out(V) - \ip{X_R,Y_R}}} > \ell}\\
        &= \pr{\size{\out(V) - \ip{X,Y}} > \ell}\\
        &< 0.001,
    \end{align*}
    as required. The first equality holds since conditioned on $O_B \neq \bot$ it holds that $O_A = \ip{X_{-R},Y_{-R}}$, and the inequality holds since $C$ is an $(\ell,0.999)$-accurate for the inner-product functionality. 
\end{proof}

We next move to prove the secrecy guarantees of $\tC$ in a constructive manner. 
In \cref{sec:proving-prop1} we prove property~\ref{item:privacy-of-Y}, and in \cref{sec:proving-prop2} we prove property~\ref{item:privacy-of-X}.



%The following claim (proven in \cref{sec:proving-prop1}) captures the first secrecy guarantee where party $\Ac$ (almost) cannot distinguish if $\Bc$ aborts (i.e., $O_B = \perp$) or not, as otherwise, such a distinguisher $\Ac$ can be used to construct an efficient attack $\Act$ that violates the privacy guarantee of the DP channel $C$.

\remove{
	Finally, the following claim (proven in \cref{sec:property 2}) captures the second secrecy guarantee where party $\Bc$ cannot estimate $O_A$ too well, as otherwise, such an estimator $\Bc$ can be used to construct an efficient attack $\Bct$ that violates the privacy guarantee of the DP channel $C$.
	
	
	\begin{claim}[Property 2 of Lemma~\ref{lemma:DPIP-to-AWEC}]~\label{clm:property 2}
		There exists an oracle-aided \ppt algorithm $\Bct$ such that for every algorithm $\Bc$ violating the AWEC secrecy property~\ref{AWEC:item:B} of $\tilde{C}$, i.e.
		\begin{align*}
			\pr{\size{\Bc(V_B) - O_A} \leq 1000\ell  \mid O_B=\bot} > q,
		\end{align*}
		where $\frac{131072\ell_2^2}{q^2k}\leq\frac{1}{e^{\varepsilon}+1}$,
		the for $X_i^* = \Bct^{\Bc}(i,\: X_{-i}, \: Y, \: V)$  it holds that
		\begin{align*}
			\eex{i \la [n]}{\pr{X_i^* = X_i \mid X_i^* \neq \bot}} > \frac{e^{\eps}}{e^{\eps}+1}.
		\end{align*}
	\end{claim}
	
	\Enote{Remove:}
	
	
	\begin{proof}
		We remind that $\tilde{C}$ is an AWEC channel constructed by a differentially private inner-product protocol. Let $\mathcal{R}=\{i|r_i=0\}$ and $\cI = \set{i_1,\dots,i_k} \cap \cR$ \remove{$\mathcal{I}=\{i|i\in\{i_1,\dots,i_k\},i\in \mathcal{R}\}$}. Then $O_A=o-\ip{x_{\mathcal{R}\setminus \mathcal{I}},\hy_{\mathcal{R}\setminus \mathcal{I}}}-\ip{x_{ \mathcal{I}},\hy_{ \mathcal{I}}}$. It suffices to prove party $\Bc$ cannot predict $\ip{x_{ \mathcal{I}},\hy_{ \mathcal{I}}}$ with high accuracy. We put the details in Section~\ref{sec:property 2}  
	\end{proof}
}




\subsection{B's Security: Proving Property~\ref{item:privacy-of-Y} of \cref{lemma:DPIP-to-AWEC}}\label{sec:proving-prop1}

Let $\Ac$ be an algorithm that violates the AWEC secrecy property~\ref{AWEC:item:A} of $\tilde{C}$ (the channel of $\Pi^C$). Namely,

\begin{align*}
	\size{\pr{\Ac(V_A) = 1 \mid O_B \neq \perp} - \pr{\Ac(V_A) = 1 \mid O_B = \perp}} > \frac1{1000}.
\end{align*}

%for $k \in \Theta\paren{\frac{n^{1/4} \paren{\frac{1}{e^{\eps}+1} - \delta}}{\log^{1/4} n}}$ and $\gamma = \frac1{100 k}$.
Recall that $V_A = (X,U,R,\hY_{-R})$ where $\hY = \begin{cases} Y & O_B \neq \perp \\ \tY & O_B = \perp\end{cases}$ and that $\tY$ is obtained from $Y$ by planting uniformly random $\oo$ values in the random locations $I_1,\ldots, I_k$ of $Y$ (the value of $k$ will be determined later by the analysis).

Therefore, the above inequality is equivalent to 

\begin{align*}%\label{eq:A-pr}
	\size{\pr{\Ac(X,U,R,Y_{-R}) = 1} - \pr{\Ac(X,U,R,\tY_{-R}) = 1}} > \frac1{1000}.
\end{align*}

We assume \wlg that $\Ac$ outputs a $\zo$ bit, so the above inequality can be written as

\begin{align}\label{eq:property1:assump}
	\size{\ex{\Ac(X,U,R,Y_{-R}) - \Ac(X,U,R,\tY_{-R})}} > \frac1{1000}.
\end{align}


In the following, for $j \in \set{0,\ldots,k}$, let $\tY^{j} = (\tY^{j}_1,\ldots, \tY^{j}_n)$ where $\tY^{j}_i = \begin{cases} \tY_i & i \in \set{I_1,\ldots,I_j} \\ Y_i & i \notin \set{I_1,\ldots,I_j}\end{cases}$.
Note that $\tY^{0} = Y$ and $\tY^{k} = \tY$. Therefore,

\begin{align}\label{eq:property1:hybrid}
    \lefteqn{\size{\eex{j \la [k]}{\ex{\Ac(X,U,R,Y^{(j-1)}_{-R}) - \Ac(X,U,R,\tY^{(j)}_{-R})}}}}\\
    &= \frac1k \cdot \size{\sum_{j=1}^k \ex{\Ac(X,U,R,Y^{(j-1)}_{-R})} - \ex{\Ac(X,U,R,Y^{(j)}_{-R})}}\nonumber\\
    &= \frac1k \cdot \size{\ex{\Ac(X,U,R,Y_{-R}) - \Ac(X,U,R,\tY_{-R})}}\nonumber\\
    &> \frac1{1000 k},\nonumber
\end{align}
where the inequality holds by \cref{eq:property1:assump}.

In the following, let $J \la [n]$ (sampled independently of the other random variables defined above), let $Z = (Z_1,\ldots,Z_n) = \tY^{J-1}$ and let $Z^{(i)} = (Z_1,\ldots,Z_{i-1}, -Z_i, Z_{i+1},\ldots, Z_n)$. It holds that

\begin{align}\label{eq:property1:exp}
	\lefteqn{\size{\ex{\Ac(X,U,R,Z_{-R})} - \ex{\Ac(X,U,R,Z^{(I_{J})}_{-R})}}}\\
        &= \frac12 \cdot \size{\ex{\Ac(X,U,R,Y^{(J-1)}_{-R}) - \Ac(X,U,R,\tY^{(J)}_{-R})}}\nonumber\\
        &= \frac12\cdot \size{\eex{j \la [k]}{\ex{\Ac(X,U,R,Y^{(j-1)}_{-R}) - \Ac(X,U,R,\tY^{(j)}_{-R})}}}\nonumber\\
        &> \frac1{2000 k},\nonumber
\end{align}
where the first equality holds since $\tY^{J}$ is equal to $Z$ w.p. $1/2$ (happens when $\tY_{I_{J}} = Y_{I_{J}}$)  and otherwise is equal to $Z^{(I_{J})}$ (happens when $\tY_{I_{J}} \neq Y_{I_{J}}$), and the last inequality holds by \cref{eq:property1:hybrid}.

%For the ease of notation, since $R \la \oo^n$, by symmetry we can assume \wlg that 
%\begin{align}\label{eq:property1:exp}
%	\size{\ex{\Ac(X,U,R,Z_{R})} - \ex{\Ac(X,U,R,Z^{(I_{J})}_{R})}} > \frac1{2000 k}.
%\end{align}



We next use the following lemma, proven in \cref{sec:prediction-lemma} (recall that a proof overview appears in \cref{sec:overview:prediction-lemma}).

\def\PredictionLemma{
    For every $\gamma \in (0,1)$ and $n \in \bbN$, there exists an oracle aided (randomized) algorithm $\Gc = \Gc_{\gamma} \colon [n] \times \zo^{n-1} \rightarrow \set{-1,1,\perp}$ that runs in time $\poly(n,1/\gamma)$ such that the following holds: 
	
	Let $(Z,W) \in \oo^n \times \zo^*$ be jointly distributed random variables, let $R \la \zo^n$ and $I \la [n]$ (sampled independently), 
	and let $F$ be a (randomized) function that satisfies
	\begin{align*}
		\size{\ex{F(R,Z_{R},W) - F(R,Z^{(I)}_{R},W)}} \geq \gamma,
	\end{align*}
	for $Z^{(I)} = (Z_1,\ldots, Z_{I-1}, -Z_I, Z_{I+1},\ldots, Z_n)$. Then
	\begin{enumerate}
		\item $\pr{\Gc^F(I,Z_{-I}, W) = -Z_I} \leq O\paren{\frac{1}{\gamma^2 n}}$, and\label{property1:prediction:bad}
		
		\item $\pr{\Gc^F(I,Z_{-I}, W) = Z_I} \geq \Omega(\gamma) - O\paren{\frac{1}{\gamma^2 n}}$.\label{property1:prediction:good}
	\end{enumerate}
}

\begin{lemma}\label{lemma:property1:prediction}
    \PredictionLemma
\end{lemma}

We now ready to finalize the proof of Property~\ref{item:privacy-of-Y} of \cref{lemma:DPIP-to-AWEC} using \cref{eq:property1:exp,lemma:property1:prediction}.

\begin{proof}
	
	Consider the following oracle-aided algorithm $\Act$:
	
	\begin{algorithm}[Algorithm $\Act$]\label{alg:Act}
		\item Inputs: $i \in [n]$, $y_{-i} \in \oo^{n-1}$, $x \in \oo^n$ and $u \in \zo^*$.
		\item Oracle: Deterministic algorithm $\Ac$.%and $\Gc$ from \cref{lemma:property1:prediction}.
		\item Operation:~
		\begin{enumerate}
			\item Sample $j \la [k]$ and $i_1,\ldots,i_{j-1} \la [n]$.
			
			\item Sample $z_{-i} = (z_1,\ldots,z_{i-1}, z_{i+1},\ldots, z_n)$, where for $t \in [n]\setminus \set{i}$: $$z_{t} \la \begin{cases} \cU_{\oo} & t \in \set{i_1,\ldots,i_{j-1}} \\ y_t & \text{otherwise} \end{cases}.$$\label{step:z-i}
			
			\item Output $y_i^* \la \Gc^{F}(i,z_{-i},w)$ for $w=(x,u)$, where $\Gc = \Gc_{\frac1{2000 k}}$ is the algorithm from \cref{lemma:property1:prediction}, and $F(r,z_r,w) \eqdef \Ac(w,(1-r_1,\ldots,1-r_n), z_r)$.
		\end{enumerate}
	\end{algorithm}

	Note that by \cref{eq:property1:exp} it holds that
	\begin{align*}
		\size{\ex{F(R,Z_{R}, W)} - \ex{F(R,Z_{R}^{I},W)}} > \frac1{2000 k}
	\end{align*}
	for $W = (X,U)$, $I = I_J$ and the function $F(r,z_r,w=(x,u)) \eqdef \Ac(w,(1-r_1,\ldots,1-r_n), z_r)$. Thus, \cref{lemma:property1:prediction} implies that
	\begin{align}\label{eq:G:LB}
		\pr{\Gc^{F}(I,Z_{-I}, V) = -Z_I} \leq O\paren{\frac{k^2}{n}}
	\end{align}
	and
	\begin{align}\label{eq:G:UB}
		\pr{\Gc^{F}(I,Z_{-I}, V) = Z_I} \geq \Omega(1/k) - O\paren{\frac{k^2}{n}}.
	\end{align} 
	
	Recall that we denote $Y_i^* = \Act^{\Ac}(i,Y_{-i},X,U)$. We next lower-bound $\eex{i \la [n]}{\pr{Y_i^* = Y_i}}$ and upper-bound $\eex{i \la [n]}{\pr{Y_i^* = -Y_i}}$.
	The first bound hold by the following calculation
	\begin{align}\label{eq:Y_i*-LB}
		\eex{i \la [n]}{\pr{Y_i^* = Y_i}}
		&\geq 0.9 \cdot \eex{i \la [n]}{\pr{Y_i^* = Z_i}}\\
		&= 0.9 \cdot \eex{i \la [n]}{\pr{\Gc^{F}(i,Z_{-i},W) = Z_i}}\nonumber\\
		&\geq \Omega\paren{\frac1k} - O\paren{\frac{k^2}{n}}.\nonumber
	\end{align}
	The first inequality holds since $i \notin \set{I_1,\ldots,I_{J-1}} \implies Z_i = Y_i$ and $\pr{i \notin \set{I_1,\ldots,I_{J-1}}} \geq \paren{1 - \frac1n}^{k-1} \geq 0.9$ (recall that $k \in o(n)$). The equality holds since, conditioned on $X=x,U=u,Y_{-i} = y_{-i}$ (the inputs of $\Act$), the value of $z_{-i}$ that is sampled in \stepref{step:z-i} of $\Act$ is distributed the same as $Z_{-i}|_{X=x,U=u,Y_{-i} = y_{-i}}$, and therefore, $Y_i^* \equiv \Gc^{F}(i,Z_{-i},W)$ for $W = (X,U)$. The last inequality holds by \cref{eq:G:UB}.
	
	On the other hand, we have that
	\begin{align}\label{eq:Y_i*-UB}
		\eex{i \la [n]}{\pr{Y_i^* = -Y_i}}
		&\leq \eex{i \la [n]}{\pr{Y_i^* = - Y_i \mid Y_i = Z_i} + \pr{Z_i \neq Y_i}}\\
		&\leq \eex{i \la [n]}{\pr{Y_i^* = - Z_i}} + \frac{k}{2n}\nonumber\\
		&=  \eex{i \la [n]}{\pr{\Gc^{F}(i,Z_{-i}) = - Z_i}} + \frac{k}{2n}\nonumber\\
		&\leq O\paren{\frac{k^2}{n}}.\nonumber
	\end{align}
	The second inequality holds since for any fixing $Y_{-i}=y_{-i},X=x,U=u$, we have $Y_i^*= \Act^{\Ac}(i,y_{-i},x,u)$ which is independent of $(Y_i,Z_i)$, and also since $i \notin \set{I_1,\ldots,I_{J-1}} \implies Z_i = Y_i$ which yields that
	\begin{align*}
		\pr{Z_i \neq Y_i} \leq 1 - \pr{i \notin \set{I_1,\ldots,I_{J-1}}} \leq 1 - \paren{1 - \frac1n}^{k} \leq 1 - e^{\frac{k}{2n}} \leq \frac{k}{2n}. 
	\end{align*}
	The equality in \cref{eq:Y_i*-UB} holds since, conditioned on $X=x,U=u,Y_{-i} = y_{-i}$ (the inputs of $\Act$), the value of $z_{-i}$ that is sampled in \stepref{step:z-i} of $\Act$ is distributed the same as $Z_{-i}|_{X=x,U=u,Y_{-i} = y_{-i}}$, and therefore, $Y_i^* \equiv \Gc^{F}(i,Z_{-i},W)$ for $W = (X,U)$. The last inequality in \cref{eq:Y_i*-UB} holds by \cref{eq:G:LB}.
	
	By \cref{eq:Y_i*-LB,eq:Y_i*-UB}, assuming that $\delta \leq 1/n$ where $n$ is large enough, 
	there exists a constant $c > 0$ such that if $k \leq c \cdot (e^{-\eps} n)^{1/3}$ then $\eex{i \la [n]}{\pr{Y_i^* = Y_i}} > e^{\eps} \cdot \eex{i \la [n]}{\pr{Y_i^* \neq Y_i}} + \delta$, as required.
	Recall that $k = \ceil{e^{\lambda_1 \eps} \lambda_2 \cdot \ell^2}$, where $\lambda_1,\lambda_2$ are the constants from \cref{eq:lamdas}. Hence, we can set a bound of $e^{-c_1 \eps} c_2 \cdot n^{1/6}$ on $\ell$ for $c_1 = \lambda_1/2 + 1/6$ and $c_2 = \sqrt{c/\lambda_2}$ to guarantee that $k \leq c \cdot (e^{-\eps} n)^{1/3}$.
	
\end{proof}

\remove{
\subsubsection{Proving \cref{lemma:property1:prediction}}\label{sec:prediction-lemma}
\begin{proof}
	
	Let $0 < \gamma \leq 0.01$ and consider the following algorithm $\Gc = \Gc_{\gamma}$:
	\begin{algorithm}
		\item Inputs: $i \in [n]$,  $z_{-i} = (z_1,\ldots,z_{i-1},z_{i+1},\ldots,z_n) \in \oo^{n-1}$ and $w \in \zo^*$.
		\item Parameter: $0 < \gamma \leq 0.01$.
		\item Oracle: $f \colon \zo^n \times \oo^{\leq n} \times \zo^* \rightarrow \zo$.
		\item Operation:~
		\begin{enumerate}
			\item For $b \in \oo$:
			\begin{enumerate}
				\item Let $z^b = (z_1,\ldots,z_{i-1},b,z_{i+1},\ldots,z_n)$.
				\item Estimate $\mu^b \eqdef \eex{r \la \zo^n \mid r_i = 1, \: f}{f(r,z^b_{r}, w)}$ as follows:
				\begin{itemize}
					\item Sample $r_1,\ldots,r_{s} \la \set{r \in \zo \colon r_i = 1}$, for $s = \ceil{\frac{128 \log (6n)}{\gamma^2}}$, and then sample $\tilde{\mu}^b \la \frac1{s} \sum_{j=1}^s f(r_j,z^b_{r_j}, w)$ (using $s$ calls to the oracle $f$).
				\end{itemize}
			\end{enumerate}
			\item Estimate $\mu^* \eqdef \eex{r \la \zo^n \mid r_i = 0, \: f}{f(r,z^1_{r}, w)}$ as follows:
			\begin{itemize}
				\item Sample $r_1,\ldots,r_{s}  \la \set{r \in \zo \colon r_i = 0}$, for $s = \ceil{\frac{128 \log (6n)}{\gamma^2}}$, and then sample $\tilde{\mu}^* \la \frac1{s} \sum_{j=1}^s f(r_j,z^1_{r_j}, w)$ (using $s$ calls to the oracle $f$).
			\end{itemize}
			\item If exists $b \in \oo$ s.t. $\size{\tilde{\mu}^b - \tilde{\mu}^*} < \gamma/4$ and $\size{\tilde{\mu}^{-b} - \tilde{\mu}^*} > \gamma/4$, output $b$.
			\item Otherwise, output $\bot$.
		\end{enumerate}
	\end{algorithm}
	In the following, fix a pair of (jointly distributed) random variables $(Z,W) \in \oo^n \times \zo^*$ and a randomized function  $f \colon \zo^n \times \oo^{\leq n} \times \zo^* \rightarrow \oo$ that satisfy 
	\begin{align*}
		\size{\ex{f(R,Z_{R},W) - f(R,Z^{(I)}_{R},W)}} \geq \gamma,
	\end{align*}
	for $R \la \zo^n$ and $I \la [n]$ that are sampled independently. 
	Our goal is to prove that 
	
	\begin{align}\label{eq:predic-lemma:UB}
		\pr{\Gc^f(I,Z_{-I}, W) = -Z_I} \leq O\paren{\frac{1}{\gamma^2 n}},
	\end{align}
	and 
	\begin{align}\label{eq:predic-lemma:LB}
		\pr{\Gc^f(I,Z_{-I}, W) = Z_I} \geq \Omega(\gamma) - O\paren{\frac{1}{\gamma^2 n}}.
	\end{align}

	Note that
	\begin{align}\label{eq:D}
		\text{For every random variable }D \in [-1,1]\text{ with }\size{\ex{D}} \geq \gamma >0: \quad \pr{\size{D} > \gamma/2} > \gamma/2,
	\end{align}
	as otherwise, $\size{\ex{D}} \leq \ex{\size{D}} \leq 1\cdot \frac{\gamma}{2} + \frac{\gamma}{2}(1-\frac{\gamma}{2}) < \gamma$.
	
	
	By applying \cref{eq:D} with $D = \eex{r \la \zo^n, f}{f(r,Z_{r},W) - f(r,Z^{(I)}_{r},W)}$, it holds that
	\begin{align}\label{eq:main-lemma:good}
		\ppr{(i,z,w) \la (I,Z,W)}{\size{\eex{r \la \zo^n, f}{f(r,z_{r},w) - f(r,z^{(i)}_{r},w)}} > \gamma/2} > \gamma/2.
	\end{align}
	On the other hand, for every fixing of $(z,w) \in \Supp(Z,W)$, we can apply \cref{lem:distance-I} with the function $f_{z,w}(r) = f(r,z_{r},w)$ and with $\alpha =\frac{\gamma}{16}$ to obtain that
	\begin{align*}
		\ppr{i\la I}{\: \size{\eex{r \la \zo^n \mid r_i = 0, \: f}{f(r,z_{r},w)} - \eex{r \la \zo^n \mid r_i = 1, \: f}{f(r,z_{r},w)} \:} \geq \frac{\gamma}{16}} \leq \frac{512}{n \gamma^2}.
	\end{align*}
	But since the above holds for every fixing of $(z,w)$, then in particular it holds that
	\begin{align}\label{eq:main-lemma:bad}
		\ppr{(i,z,w) \la (I,Z,W)}{\: \size{\eex{r \la \zo^n \mid r_i = 0, \: f}{f(r,z_{r},w)} - \eex{r \la \zo^n \mid r_i = 1, \: f}{f(r,z_{r},w)} \:} \geq \frac{\gamma}{16}} \leq \frac{512}{n \gamma^2}.
	\end{align}
	
	We next prove \cref{eq:predic-lemma:UB,eq:predic-lemma:LB} using \cref{eq:main-lemma:good,eq:main-lemma:bad}.
	
	In the following, for a triplet $t = (i,z,w) \in \Supp(I,Z,W)$, consider a random execution of $\Gc^f(i,z_{-i},w)$. For $x \in \set{-1,1,*}$, let $\mu^x_t$ be the value of $\mu^x$ in the execution, and let $M^x_t$ be the (random variable of the) value of $\tilde{\mu}^x$ in the execution. Note that by definition, it holds that $\mu_{i,z,w}^{z_i} = \eex{r \la \zo^n, f}{f(r,z_{r},w)}$ and $\mu_{i,z,w}^{-z_i} = \eex{r \la \zo^n, f}{f(r,z^{(i)}_{r},w)}$. Therefore, \cref{eq:main-lemma:good} is equivalent to 
	\begin{align}\label{eq:main-lemma:good2}
		\ppr{t=(i,z,w) \la (I,Z,W)}{\size{\mu_t^{z_i} - \mu_t^{-z_i}} > \gamma/2} > \gamma/2.
	\end{align}
	Furthermore, note that $\mu_{i,z,w}^* \eqdef \eex{r \la \zo^n \mid r_i = 0, \: f}{f(r,z^1_{r}, w)} = \eex{r \la \zo^n \mid r_i = 0, \: f}{f(r,z_{r}, w)}$ and that $\mu_{i,z,w}^{z_i} = \eex{r \la \zo^n \mid r_i = 1, \: f}{f(r,z^{z_i}_{r}, w)}$. Therefore, \cref{eq:main-lemma:bad} is equivalent to 
	\begin{align}\label{eq:main-lemma:bad2}
		\ppr{t = (i,z,w) \la (I,Z,W)}{\: \size{\mu_t^* - \mu_t^{z_i}}\geq \frac{\gamma}{16}}  \leq \frac{512}{n \gamma^2}.
	\end{align}
	
	We next prove the lemma using \cref{eq:main-lemma:good2,eq:main-lemma:bad2}.
	
	Note that by Hoeffding's inequality, for every $t = (i,z,w) \in \Supp(I,Z,W)$ and  $x \in \set{-1,1,*}$ it holds that $\pr{\size{M_t^x - \mu_t^x} \geq \frac{\gamma}{16}} \leq 2\cdot e^{-2 s \paren{\frac{\gamma}{16}}^2} \leq \frac1{3n}$, which yields that for every fixing of $t = (i,z,w) \in \Supp(I,Z,W)$, w.p.\ at least $1-1/n$ we have for all $x \in \set{-1,1,*}$ that $\size{M_t^x - \mu_t^x} < \frac{\gamma}{16}$ (denote this event by $E_t$).
	
	The proof of \cref{eq:predic-lemma:UB} holds by the following calculation:
	\begin{align*}
		\lefteqn{\pr{\Gc^f(I,Z_{-I},W) = -Z_I}}\\
		&= \eex{(i,z,w) \la (I,Z,W)}{\pr{\Gc^f(i,z_{-i},w) = -z_i}}\\
		&= \eex{t = (i,z,w) \la (I,Z,W)}{\pr{\set{\size{M_t^* - M_t^{z_i}} > \gamma/4} \land \set{\size{M_t^* - M_t^{-z_i}} < \gamma/4}}}\\
		&\leq \eex{t =(i,z,w) \la (I,Z,W)}{\pr{\size{M_t^* - M_t^{z_i}} > \gamma/4}}\\
		&\leq \eex{t = (i,z,w) \la (I,Z,W)}{\pr{\size{M_t^* - M_t^{z_i}} > \gamma/4 \mid E_t}} + \frac{1}{n}\\
		&\leq \ppr{t = (i,z,w) \la (I,Z,W)}{\size{\mu_t^* -  \mu_t^{z_i}} \geq \frac{\gamma}{16}} + \frac{1}{n}\\
		&\leq \frac{512}{n \gamma^2} + \frac1n,
	\end{align*}
	The second inequality holds since $\pr{\neg E_t} \leq 1/n$ for every $t$. The penultimate inequality holds since conditioned on $E_t$, it holds that $\size{M_t^* - \mu_t^*} \leq \frac{\gamma}{16}$ and $\size{M_t^{z_i} - \mu_t^{z_i}} \leq \frac{\gamma}{16}$, which implies that $\size{M_t^* - M_t^{z_i}} > \gamma/4 \: \implies \: \size{\mu_t^* -  \mu_t^{z_i}} \geq \frac{\gamma}{4} - 2\cdot \frac{\gamma}{16} > \frac{\gamma}{16}$. The last inequality holds by \cref{eq:main-lemma:bad2}.
	
	It is left to prove \cref{eq:predic-lemma:LB}. 
	Observe that \cref{eq:main-lemma:good2,eq:main-lemma:bad2} imply with probability at least $\gamma/2 - \frac{512}{n \gamma^2}$ over $t = (i,z,w)\in (I,Z,W)$, we have $\size{\mu_t^* -  \mu_t^{z_i}} \leq  \frac{\gamma}{16}$ and $\size{\mu_t^{z_i}- \mu_t^{-z_i}} \geq\frac{\gamma}{2}$. Therefore, conditioned on event $E_t$ (i.e., $\forall x \in \set{-1,1,*}: \: \size{M_t^{x} - \mu_t^{x}} \leq \frac{\gamma}{16}$), we have for such triplets $t = (i,z,w)$ that 
	\begin{align*}
		\size{M_t^* - M_t^{-z_i}} 
		& \geq \size{\mu_t^{z_i} - \mu_t^{-z_i}} - \size{\mu_t^{z_i} - \mu_t^*} - \size{M_t^*  - \mu_t^*} -  \size{M_t^{-z_i} - \mu_t^{-z_i}}\\
		&\geq \frac{\gamma}{2} - 3\cdot \frac{\gamma}{16}\\
		&> \gamma/4,
	\end{align*}
	and 
	\begin{align*}
		\size{M_t^* - M_t^{z_i}}
		&\leq  \size{M_t^* - \mu_t^*} + \size{\mu_t^* - \mu_t^{z_i}} + \size{\mu_t^{z_i} - M_t^{z_i}} \\
		&\leq 3\cdot \frac{\gamma}{16}\\
		&< \gamma/4.
	\end{align*}
	
	Thus, we conclude that
	\begin{align*}
		\lefteqn{\pr{\Gc^f(I,Z_{-I},W) = Z_I}}\\
		&= \eex{(i,z,w) \la (I,Z,W)}{\pr{\Gc^f(i,z_{-i},w) = z_i}}\\
		&= \eex{t = (i,z,w) \la (I,Z,W)}{\pr{\set{\size{M_t^* - M_t^{-z_i}} > \gamma/4} \land \set{\size{M_t^* - M_t^{z_i}} < \gamma/4}}}\\
		&\geq \paren{1 - \frac1{n}}\cdot \eex{t = (i,z,w) \la (I,Z,W)}{\pr{\set{\size{M_t^* - M_t^{-z_i}} > \gamma/4} \land \set{\size{M_t^* - M_t^{z_i}} < \gamma/4} \mid E_t}}\\
		&\geq \paren{1 - \frac1{n}}\cdot \paren{\gamma/2 - \frac{512}{n \gamma^2}}\\
		&\geq \gamma/4 - \frac{1024}{n \gamma^2},
	\end{align*}
	which proves \cref{eq:predic-lemma:LB}. The first inequality holds since $\pr{E_t} \geq 1-1/n$ for every $t = (i,z,w)$, and the second one holds by the observation above.
	
\end{proof}
}

\subsection{A's Security: Proving Property~\ref{item:privacy-of-X} of \cref{lemma:DPIP-to-AWEC}}\label{sec:proving-prop2}

Let $\Bc$ be an algorithm that violates the AWEC secrecy property~\ref{AWEC:item:B} of $\tilde{C} = ((V_A,O_A),(V_B,O_B))$ --- the channel of $\Pi^{C = ((X,U),(Y,V))}$ (\cref{protocol:DPIP-to-AWEC}). Namely,

\begin{align}\label{eq:violating-B}
	\pr{\size{\Bc(V_B) - O_A} \leq 1000 \ell \mid O_B=\bot} > \frac1{1000}.
\end{align}

Recall that $V_B = (Y,V,R,X_{R},I_1,\ldots,I_k, \tY)$ where $I_1,\ldots,I_k \la [n]$ are the indices that $\Bc$ chooses at Step~\ref{B_steps_in_abort}, and $\tY = (\tY_1,\ldots,\tY_n)$ where $\tY_i \la \oo$ for $i \in \set{I_1,\ldots,I_k}$ and otherwise $\tY_i = Y_i$. Furthermore, conditioned on $O_B=\bot$, recall that $O_A = \ip{X_{-R}, \tY_{-R}}$. Therefore, \cref{eq:violating-B} is equivalent to 
\begin{align}\label{eq:Bc-guarantee}
	%\pr{\size{\Bc(Y,V,R,X_{R},I_1,\ldots,I_k, \tY) - \ip{X_{-R}, \tY_{-R}}} \leq 1000 \ell } > \frac1{100}.
	\pr{\size{\Bc(V_B) - \ip{X_{-R}, \tY_{-R}}} \leq 1000 \ell } > \frac1{1000}.
\end{align}


In the following, define the random variable $H$ to be the first $m = \ceil{k/4}$ indices of $R^0 \cap \set{I_1,\ldots,I_k}$ for $R^0 = \set{i \colon R_i = 0}$, where we let $H = \emptyset$ if the size of the intersection is smaller than $m$.
Since $R \la \zo^n$, Hoeffding's inequality implies that $\pr{\size{R^0} \geq 0.4 n} \geq 1 - e^{-\Omega(n)}$. Since $I_1,\ldots,I_k \la [n]$ (independent of $R$), then again by Hoeffding's inequality we obtain that $\pr{\size{H} = \ceil{k/4} \mid \size{R^0} \geq 0.4 n} \geq 1 - e^{-\Omega(k)}$, which yields that
\begin{align*}
	\pr{H \neq \emptyset} = \pr{\size{H} = \ceil{k/4}} \geq 1 - e^{-\Omega(k)} - e^{-\Omega(n)} \geq 1-\frac1{10000}.
\end{align*}
Therefore, by the union bound,
\begin{align}\label{eq:good-and-not-empty-H}
	\lefteqn{\pr{\set{\size{\Bc(V_B) - \ip{X_{-R}, \tY_{-R}}} \leq 1000 \ell} \land \set{H \neq \emptyset}}}\\
	&= 1- \pr{\set{\size{\Bc(V_B) - \ip{X_{-R}, \tY_{-R}}} > 1000 \ell} \lor \set{H = \emptyset}}\nonumber\\
	&\geq 1- \pr{\set{\size{\Bc(V_B) - \ip{X_{-R}, \tY_{-R}}} > 1000 \ell}} - \pr{ \set{H = \emptyset}}\nonumber\\
    &\geq 1- \paren{1- \frac1{1000}} - \frac1{10000} \nonumber\\
     &\geq \frac1{2000}.\nonumber
\end{align}
%\begin{align}\label{eq:good-and-not-empty-H}
%	\lefteqn{\pr{\set{\size{\Bc(V_B) - \ip{X_{-R}, \tY_{-R}}} \leq 1000 \ell} \land \set{H \neq \emptyset}}}\\
%	&= \pr{H \neq \emptyset} \cdot \pr{\size{\Bc(V_B) - \ip{X_{-R}, \tY_{-R}}} \leq 1000 \ell \mid H \neq \emptyset }\nonumber\\
%	&\geq \pr{H \neq \emptyset}\cdot \paren{\pr{\size{\Bc(V_B) - \ip{X_{-R}, \tY_{-R}}} \leq 1000 \ell} - \pr{H = \emptyset}}\nonumber\\
%	&\geq \paren{1-\frac1{10000}}\cdot \paren{\frac1{1000} - \frac1{10000}}\nonumber\\
%	&> \frac1{2000}.\nonumber
%\end{align}


In the following, let $d$ be the number of random coins that $\Bc$ uses, and for $\psi \in \zo^d$ let $\Bc_{\psi}$ be algorithm $\Bc$ when fixing its random coins to $\psi$.
Let $\Psi \la \zo^d$,  $Z = X_H$,  $T' = (Y,V,R,I_1,\ldots,I_k, H,\tY_{-\cH}, X_{-\cH})$,  $T= (\Psi, T')$ and $S = \tY_H$. Note  that conditioned on $H \neq \perp$, $S$ is a uniformly random string in $\oo^m$, independent of $Z$ and $T$, and note that $V_B$ is a deterministic function of $(\tY_{H}, T)$ (because $X_R$, which is part of $V_B$, is a sub-string of $X_{-\cH}$).
\cref{eq:good-and-not-empty-H} yields that w.p. $1/2000$ over $z \la Z$,  $t = (\psi, t'=(y,v,r,i_1,\ldots,i_k,\cH,x_{-\cH},\ty_{-\cH})) \la T$ and $s \la \oo^m$,
the following holds for $\bar{\cH} =  \set{i \in [n] \colon r_i = 0} \setminus \cH\:$:
\begin{align*}
	\size{\Bc_{\psi}(s,t') - \ip{x_{\bar{\cH}}, \ty_{\bar{\cH}}} - \ip{z, s}}\leq 1000 \ell.
\end{align*}

By denoting $f(s,t=(\psi,t')) = \Bc_{\psi}(s,t') + \ip{x_{\bar{\cH}}, \ty_{\bar{\cH}}}$,\footnote{Note that $f(s,t)$ is well-defined because $t$ contains $x_{\bar{\cH}}$ and $\ty_{\bar{\cH}}$ (sub-strings of $x_{-\cH}$ and $\ty_{-\cH}$, respectively).} the above observation is equivalent to

\begin{align}\label{eq:our-good-f}
	\ppr{(z,t) \la (Z,T), \: s \la \oo^m}{\size{f(s,t) - \ip{z, s}} \leq 1000 \ell} \geq \frac1{2000}.
\end{align}

In the following, let $\cD$ be the joint distribution of $(Z,T)$, which is equivalent to the output of $\GenView^{C}()$ defined below in \cref{alg:GenView}.

\begin{algorithm}[$\GenRand$]\label{alg:GenRand}
	~
	\begin{enumerate}
            \item Sample $\psi \la \zo^d$.
		\item Sample $r \la \zo^n$ and $i_1,\ldots,i_k \la [n]$.
		\item Let $\cH$ be the first $m=\ceil{k/4}$ indices of $\set{i \in [n] \colon r_i = 0} \cap  \set{i_1,\ldots,i_k}$, where $\cH = \emptyset$ if the intersection size is less than $m$.
		\item Let $\bar{\cH} = \set{i \in [n] \colon r_i = 0} \setminus \cH$.
		\item Sample $\ty_{\bar{\cH}} \la \oo^{\size{\bar{\cH}}}$.
		\item Output $(\psi,r,i_1,\ldots,i_k,\cH,\ty_{\bar{\cH}})$.
	\end{enumerate}
\end{algorithm}



\begin{algorithm}[$\GenView$]\label{alg:GenView}
	\item Oracle: A channel $C = ((X,U),(Y,V))$.
	\item Operation:~
	\begin{enumerate}
		\item Sample $((x,u),(y,v)) \la C$.
		\item Sample $(\psi,r,i_1,\ldots,i_k,\cH,\ty_{\bar{\cH}}) \la \GenRand()$ (\cref{alg:GenRand}).
		\item Output $z = x_{\cH}$ and $t = (\psi, t')$ for $t'=(y,v,r,i_1,\ldots,i_k,\cH,\ty_{-\cH},x_{-\cH})$, where $\ty_i = y_i$ for $i \in [n]\setminus (\cH \cup \bar{\cH})$.
	\end{enumerate}
\end{algorithm}



We now can use the following reconstruction result from \cite{HaitnerMST22}:


\begin{fact}[Follows by Theorem 4.6 in \cite{HaitnerMST22}]\label{fact:prev-rec}
	There exists constants $\eta_1,\eta_2 > 0$ and a \ppt algorithm $\Dist$ such that the following holds for large enough $m \in \bbN$: Let $\eps \geq 0$ and $a \geq \log m$, and let $\cD$ be a distribution over $\oo^m \times \Sigma^*$. Then for every function $f$ that satisfies 
	\begin{align*}
		\ppr{(z,t) \la \cD, \: s \la \oo^m}{f(s,t) - \ip{z,s} \leq a} \geq e^{\eta_1 \cdot \eps}\cdot \eta_2 \cdot a/\sqrt{m},
	\end{align*}
	it holds that
	\begin{align*}
		\ppr{(z,t) \la \cD, \: j \la [m]}{\Dist^{\cD,f}(j, z, t) = 1} > e^{\eps}\cdot \ppr{(z,t) \la \cD, \: j \la [m]}{\Dist^{\cD,f}(j ,z^{(j)}, t) = 1} + \frac1m,
	\end{align*}
	where $z^{(j)} = (z_1,\ldots,z_{j-1},-z_j,z_{j+1},\ldots,z_m)$.\footnote{Theorem 4.6 in \cite{HaitnerMST22} actually considered a harder setting where $z$ is the coordinate-wise product of two vectors $x,y \in \oo^n$, and $f$ only guarantees accuracy when in addition to $s$ and $t$, it also gets $x_{s} = \set{x_i \colon s_i = 1}$ and $y_{-s} = \set{y_i \colon s_i = -1}$ as inputs.}
	%\Enote{explain the differences from Theorem 4.6 in \cite{HaitnerMST22}.}%\Nnote{Maybe for after the deadline: I guess we don't really need this theorem and can use the proof of the easy case. Maybe we want to write the simple proof}
\end{fact}



%Note that the output of the following $\GenView^{C}()$ (\cref{alg:GenView}) is distributed the same as the joint distribution of $(Z,T)$. 
Now, we would like to apply \cref{fact:prev-rec} with $\cD = \GenView^C()$ and $a = 1000 \ell$. To do that, \cref{fact:prev-rec} yields that we need to choose $k$ such that $\frac{e^{\eta_1 \cdot \eps}\cdot \eta_2  \cdot 1000\ell}{\ceil{k/4}} \leq \frac1{2000}$, which holds by choosing $k = \floor{e^{\lambda_1 \eps}\cdot \lambda_2 \cdot \ell^2}$ with 
\begin{align}\label{eq:lamdas}
	\lambda_1 = \eta_1 \text{ and }\lambda_2 = 10^7 \eta_2 + 1,
\end{align}
where $\eta_1,\eta_2$ are the constants from \cref{fact:prev-rec}.

We deduce from \cref{fact:prev-rec,eq:our-good-f} that 

\begin{align}\label{eq:Dist-gap}
	\ppr{(z,t) \la \cD, \: j \la [m]}{\Dist^{\cD,f}(j, z, t) = 1} > e^{\eps}\cdot \ppr{(z,t) \la \cD, \: j \la [m]}{\Dist^{\cD,f}(j ,z^{(j)}, t) = 1} + \frac1m.
\end{align}


We now ready to describe our algorithm $\Bct$ that satisfies Property~\ref{item:privacy-of-X} of \cref{lemma:DPIP-to-AWEC}.


\begin{algorithm}[Algorithm $\Bct$]\label{alg:Bct}
	\item Inputs: $i \in [n]$, $x_{-i} = (x_1,\ldots,x_{i-1},x_{i+1},\ldots,x_n) \in \oo^{n-1}$, $y \in \oo^n$ and $v \in \zo^*$.
	\item Oracles: A channel $C = ((X,U),(Y,V))$ and an algorithm $\Bc$.
	\item Operation:~
	\begin{enumerate}
		\item Sample $(\psi, r,i_1,\ldots,i_k,\cH,\ty_{\bar{\cH}}) \la \GenRand()$ (\cref{alg:GenRand}).
		\item If $i \notin \cH$, output $\bot$.
		\item Otherwise:
		\begin{enumerate}
			\item Let $t = (y,v,r,i_1,\ldots,i_k,\cH,\ty_{-\cH},x_{-\cH})$ where $\ty_{i'} = y_{i'}$ for $i' \in [n]\setminus (\cH \cup \bar{\cH})$.
			\item For $b \in \oo$: Let $x^b = (x_1,\ldots,x_{i-1},b, x_{i+1},\ldots,x_n)$ and $z^b = x^b_{\cH} \in \oo^m$, where $m = \ceil{k/4}$.
			\item Let $j \in [m]$ be the index such that $z^1_j \neq z^{-1}_j$.
			\item Sample $b \la \oo$ and $o \la \Dist^{\GenView^C, f}(j,z^b,t)$ where $\GenView^C$ is \cref{alg:GenView} with oracle access to $C$, and $f(s,t = (\psi,t')) = \Bc_{\psi}(s,t') + \ip{x_{\bar{\cH}}, \ty_{\bar{\cH}}}$. 
			\color{gray}{\# Recall that $x_{\bar{\cH}}$ is a sub-string of $x_{-\cH}$ (part of $t$) and that $ \ty_{\bar{\cH}}$ is a sub-string of $\ty_{-\cH}$ (also part of $t$).}
			\color{black}{\item If $o = 1$, output $b$. Otherwise, output $\bot$.}
		\end{enumerate}
	\end{enumerate}
\end{algorithm}

\begin{proof}[Proof of Property~\ref{item:privacy-of-X} of \cref{lemma:DPIP-to-AWEC} using \cref{alg:Bct}]
	
In the following, let $\cD = \GenView^C()$ and $((X,U),(Y,V)) \la C$. Consider a random execution of $\Bct^{\Bc, C}(I,\: X_{-I}, \: Y, \: V)$ for $I \la [n]$, and let $T, H, B, O, Z, J$ be the values of $\:\: t, h,\cH,o, z, j$ in the execution. 
Let $p = \ppr{(z,t) \la \cD, \: j \la [m]}{\Dist^{\cD,f}(j ,z^{(j)}, t) = 1}$, and note that conditioned on $I \in H$, $J$ is distributed uniformly over $[m]$. Therefore,
$$\pr{\Dist^{\cD,f}(J, Z^{(J)}, T) = 1 \mid I \in H} = p$$ and $$\pr{\Dist^{\cD,f}(J, Z, T) = 1 \mid I \in H} = \ppr{(z,t) \la \cD, \: j \la [m]}{\Dist^{\cD,f}(j ,z, t) = 1} \geq e^\eps p + \frac1m,$$ where the inequality holds by \cref{eq:Dist-gap}.
Thus, the following holds for $X^*_i = \Bct^{\Bc, C}(i,\: X_{-i}, \: Y, \: V)$:


\begin{align*}
	\eex{i \la [n]}{\pr{X^*_i = -X_i}}
	&= \pr{X_I^* = -X_I}\\
	&=\pr{\set{I \in H} \land \set{B = -X_I} \land \set{O=1}}\\
	&= \paren{\frac{m}{n}\cdot \pr{H \neq \emptyset}} \cdot \frac12 \cdot  \pr{O=1 \mid \set{I \in H}\land \set{B = -X_I}}\\
	&= \frac{m}{2n} \cdot \pr{H \neq \emptyset} \cdot \pr{\Dist^{\cD,f}(J, Z^{(J)}, T) = 1 \mid I \in H}\\
	&= \frac{m}{2n} \cdot \pr{H \neq \emptyset} \cdot p,
\end{align*}
 and 

\begin{align*}
	\eex{i \la [n]}{\pr{X^*_i = X_i}}
	&=\pr{\set{I \in H} \land \set{B = X_I} \land \set{O=1}}\\
	&= \paren{\frac{m}{n}\cdot \pr{H \neq \emptyset}} \cdot \frac12 \cdot  \pr{O=1 \mid \set{I \in H}\land \set{B = X_I}}\\
	&= \frac{m}{2n} \cdot \pr{H \neq \emptyset} \cdot\pr{\Dist^{\cD,f}(J, Z, T) = 1 \mid I \in H}\\
	&> \frac{m}{2n}  \cdot \pr{H \neq \emptyset} \cdot (e^{\eps} p + 1/m)\\
	&= e^{\eps} \cdot \paren{\frac{m}{2n}  \cdot \pr{H \neq \emptyset} \cdot p} + \frac{1}{2n}\cdot \pr{H \neq \emptyset},\\
	&>  e^{\eps} \cdot \eex{i \la [n]}{\pr{X^*_i = -X_i}} + \delta,
\end{align*}
which concludes the proof. The last inequality holds since $\pr{H \neq \emptyset} \geq 1-\frac1{10000}$ and $\delta \leq \frac1{3n}$.

\end{proof}


%\section{\WEC from \AWEC (Proof of \cref{lemma:AWEC-to-WEC})}\label{sec:AWEC-to-WEC}


In this section, we prove \cref{lemma:AWEC-to-WEC} and  show how to implement \WEC (\cref{def:WEC}) from \AWEC (\cref{def:AWEC}) using a \ppt protocol. Crucially, the security proof is constructive, so that it could be used in the computational case as well (see, \cref{cor:CompAWEC to CompWEC}).


%Find a place to put this 


To prove \cref{lemma:AWEC-to-WEC}, we will need the following easy version of Goldriech-Levin \cite{GoldreichL89}.
\begin{lemma}
\label{prel:gl:weak:prob}
There exists a \ppt oracle-aided  algorithm $\Dec$ such that the following holds. Let $n\in N$ be a number, $x\in \zn$, and %$f\colon \zn \to \zn$ be a (possibly randomized) function, 
 and let $\Pred$ be an algorithm such that
\begin{align*}
\ppr{ r\gets \zn}{\Pred(r)=\GL(x,r)} > 3/4+0.001,
\end{align*}
 where $\GL(x,r)\eqdef \langle x,r \rangle$ is the Goldreich-Levin predicate. 
Then $\pr{\Dec^\Pred(1^n)=x}=1-\negl(n)$.
\end{lemma}
\begin{proof}[Proof of \cref{prel:gl:weak:prob}]
We use $\Pred$ to decode each bit of $x$ separately. For every $i$, let $e_i$ be the vector that has $1$ in the $i$-th entry, and $0$'s everywhere else. Observe that, for a uniformly chosen $R\gets \zn$, 
$$\pr{\Pred(R)=\GL(x,R) \land \Pred(R\xor e_i)=\GL(x,R\xor e_i)}\ge 1/2+0.001.$$
Thus,
$$\pr{\Pred(R)\xor\Pred(R\xor e_i) =\GL(x,R) \xor \GL(x,R\xor e_i)}\ge 1/2+0.001.$$
By linearity of the inner product we get that,
$$\pr{\Pred(R)\xor\Pred(R\xor e_i) =x_i}\ge 1/2+0.001.$$
Let $\Dec$ be the algorithm that for every $i$, computes $\Pred(R)\xor\Pred(R\xor e_i)$  for $n$ random values of $R$, and let $x'_i$ to be the majority of the outputs. Then, $\Dec$ outputs $x'=x'_1,\dots,x'_n$. By Chernoff bound, $x'_i$ is equal to $x_i$ with all but negligible probability. By the union bound, the above is true for all $i$'s simultaneously  with all but negligible probability, as we wanted to show.
\end{proof}
We are now ready to prove \cref{lemma:AWEC-to-WEC}.
\begin{proof}[Proof of \cref{lemma:AWEC-to-WEC}]
We now define the protocol $\Lambda^C$ as follows:
\begin{protocol}[$\Lambda^C=(\Ac,\Bc)$]
\item Oracle access: A channel $C =((\OA,\VA),(\OB,\VB))$.
	\item Operation:
	\begin{enumerate}
            \item Sample $((\oA,\vA),(\oB,\vB))\from C$. $\Ac$ gets $(\oA,\vA)$ and $\Bc$ gets $(\oB,\vB)$. 
		
			\item $\Ac$ chooses a random offset $s\gets [1000\ell]$ and sends it to $\Bc$. Let $\oA'=\ceil{\frac{\oA+s}{1000\ell}}$ and $\oB'=\ceil{\frac{\oB+s}{1000\ell}}$ (if $\oB=\bot$, let $\oB'=\bot$). 
        \item $\Ac$ chooses a random vector $r\from \zo^{\log(n)}$ and sends it to $\Bc$. Let $\hoA=\langle \oA',r \rangle$ and $\hoB=\langle \oB',r \rangle$ (if $\oB'=\bot$, let $\hoB=\bot$).
        \item $\Ac$ outputs $\hoA$ and $\Bc$ outputs $\hoB$.
        \end{enumerate}
\end{protocol}
Let $\tilde{C}$ be the channel induces by $\Lambda^C=(\Ac,\Bc)$ defined above. 
Let $\OA,\VA,\OB,\VB,S,R,\OA',\OB',\hOA,\hOB$ be the random variables that corresponds to the value of $\oA,\vA,\oB,\vB,s,r,\oA',\oB',\hoA,\hoB$ in a random execution of the above protocol. Denote $\hVA=(\VA,S,R)$ and $\hVB=(\VB,S,R)$ and note that $(\hVA,\hOA),(\hVB,\hOB)$ defines the channels $\tilde{C}$.



We now prove that if $C=((\OA,\VA),(\OB,\VB))$ is an $(\ell,\alpha,p,q)$-\AWEC then $\tilde{C}$ is an $(\alpha'=\alpha+0.001,\: p' = p ,\:  q' = 1/2 + 2.001q)$-\WEC.

\paragraph{Agreement:} If $\size{\OA-\OB}\le \ell$, then $\ppr{S}{\ceil{\frac{\OA+S}{1000\ell}}\ne \ceil{\frac{\OB+s}{1000\ell}}}\le 1/1000$. Thus, 
\begin{align*}
    \pr{\hOA\ne \hOB\mid \hOB\ne \bot} &\leq\pr{|\OA - \OB|\leq \ell\mid \OB\ne \bot}+1/1000\\
    &= \eps+1/1000=\eps'
\end{align*}

\paragraph{$\Bc$'s privacy:} Recall that the view of $\Ac$ in the above protocol is $\hVA=(\VA,S,R)$, and note that $S,R$ are independent of $\OB$. Since, $\hOB=\bot$ iff $\OB=\bot$ it follows that it follows that for every algorithm $\Dc$:
    \begin{align*}
    &\size{{\pr{\Dc(\hVA) = 1 \mid \hOB \neq \bot} - \pr{\Dc(\hVA) = 1 \mid \hOB = \bot}} }\\
   %  &=\size{{\pr{\Dc(\hVA) = 1 \mid \OB \neq \bot} - \pr{\Dc(\hVA) = 1 \mid \OB = \bot}} }\\
    &=\size{{\pr{\Dc(\VA,S,R) = 1 \mid \OB \neq \bot} - \pr{\Dc(\VA,S,R) = 1 \mid \OB = \bot}} }\\
    &=\size{\pr{\Dc(\VA) = 1 \mid \OB \neq \bot} - \pr{\Dc(\VA) = 1 \mid \OB = \bot}} \le p=p'.
    \end{align*}
\paragraph{$\Ac$’s privacy:}  Assume towards a contradiction that there exists an algorithm $\Dc$ such that

\begin{align}\label{eq:avrging}
\pr{\Dc(\hVB)=\hOA  \mid \hOB\neq\bot} \ge\frac{1+q'}{2}=  3/4+q+0.01.
\end{align}
Let $\cG=\set{(\vB,s)\colon \ppr{R}{\Dc(\vB,s,R)=\hOA\mid\VB=\vB,S=s,\hOB\neq \bot}\ge 3/4+0.001}$, and first note that $\ppr{\VB,S}{(\VB,S)\in \cG\mid \hOB\neq \bot}\ge q+0.009$.
Indeed, otherwise it holds that
\begin{align*}
&\pr{\Dc(\hVB)=\hOA  \mid \hOB\neq\bot}\\
& = \ppr{\VB,S}{(\VB,S)\in \cG\mid \hOB\neq \bot} + \ppr{\VB,S}{(\VB,S)\notin \cG\mid \hOB\neq \bot}\cdot(3/4+ 0.001)\\
&< (q+0.009) + 1\cdot (3/4+ 0.001)\\
&= 3/4+ q+0.01
\end{align*}
in contradiction to \cref{eq:avrging}.
   % $$\ppr{\VB,S}{\ppr{R}{\Dc(\VB,S,R)=\hOA }\ge 3/4+\delta\cdot\alpha\mid \hOB=\bot}\ge \delta(1-\alpha).$$ 
    Next, by \cref{prel:gl:weak:prob} there exists an algorithm $\Dc'$ such that 
    $$
    \pr{\Dc'(\VB,S)=O'_A \mid (\VB,S)\in\cG, \OB'\neq \bot}\ge 1-o(1)
    $$ 
    Which implies that, 
    \begin{align*}
    \pr{\Dc'(\VB,S)=O'_A \mid  \OB'\neq \bot}
    &\ge \pr{\Dc'(\VB,S)=O'_A \mid (\VB,S)\in\cG, \OB'\neq \bot}\cdot \pr{\Dc'(\VB,S)\in\cG \mid\hOB\neq \bot}\\
    &\ge (q+0.009)(1-o(1))\\
    &\ge q.
    \end{align*}
    %for every fixing of $\VB,S$ to values $\vB,s$ such that 
    % $$
    % \ppr{R}{\Dc(\vB,s,R)=\hOA\mid\VB=\vB,S=s,\hOB= \bot}\ge 3/4+\delta\cdot\alpha
    % $$
    % it holds that 
    % it holds that $\Dc'$
    % there exists an algorithm $\Dc'$
    % it holds that $$\pr{\Dc'(\VB,S)=O'_A }\ge (1-\alpha)$$ ($\Dc'$ is the GL reconstruction), 
    Since by definition $\size{(O'_A\cdot 1000\ell-S)-\OA}\le 1000\ell$, and $S$ is independent of $\VB$, it follows that there exists an algorithm $\Dc''$ such that  
    $$\pr{\size{\Dc''(\VB)-\OA}\le 1000\ell\mid \OB\neq \bot}\ge \delta(1-2\alpha)> q.$$
Contradicting the fact that $C$ is an $(\ell,\alpha,p,q)$-\AWEC.
\end{proof}
% \section{From CDP-IP to OT}\label{sec:CDPIP_to_OT}

% In this section we state and prove our results for the computational case: \CDP (computational differential private) protocols that estimate the inner product well. For such protocols, we prove the following result.


% \begin{theorem}\label{thm:DPIP-to-OT}
% There exist constant $c_1,c_2 > 0$ and an oracle-aided \ppt protocol $\Pi$ such that the following holds for large enough $n \in \bbN$ and for 
% $\eps \leq \log^{0.9} n$, $\delta \leq \frac1{3n}$, and $\ell = e^{-c_1  \eps}  c_2\cdot n^{1/8}$:
%     Let $\Lambda$ be an $(\eps,\delta)$-DP protocol that is $(\ell,0.999)$-accurate for the inner-product functionality. 
%     Then $\Pi^\Lambda$ is a computational oblivious transfer protocol.
% \end{theorem}









% \begin{lemma}[Main lemma,  the computational case]\label{lem:KAProtocol:Comp}
% 	There exists a constant $c>0$ such that the following holds: Let $C = \set{C_\kappa}_{\kappa\in \N}$ be an $n$-size, $\eps$-\CDP channel ensemble, that is $(,)$-accurate for the inner-product functionality, and let $\Pi$ be according to \ref{??}. Then $???$ is an Oblivious transfer protocol. 
% \end{lemma}



% \begin{theorem}
%     \Enote{State our final result.}
% \end{theorem}


\printbibliography

\appendix

A wealth of research exists looking at the effects of AI companions on humans, for example \citet{Brandtzaeg2022AIfriend, xie2022attachment}. Our paper instead focuses on evaluating the biases and stereotypes that chatbots perpetuate as it becomes increasingly important to mitigate their impacts.

Metrics play a crucial role in assessing {LLM}s, and a range of papers have produced quantitative evaluations of these models \citep{nangia-etal-2020-crows, dhamalabold2021, bellem2024are, wan2023biasasker}. Through the lens of gender, extensive work has been done on creating a metric for occupational bias \citep{kirk2024box, rudinger-etal-2018-wino}. \citet{bai2024measuring} is one of few papers that focus on more underlying gender biases in that it studies implicit (unintentional, automatic) rather than explicit (intentional, deliberate) bias. It does this by using the Implicit Association Test (IAT), commonly used for human biases, and modifies it to {LLM}s.

\subsection{Persona Bias in LLMs}

Research into {AI} personas find that, generally, the design and implementation of personas result in models reflecting existing human biases, as shown by \citet{cheng-etal-2023-marked}. They generated personas with different ethnicities and genders and then had the LLM describe itself in that personas voice. This output is compared to the unmarked default persona descriptions, i.e., White and Man, by finding words that statistically distinguish the two groups and comparing the generated descriptions to human-created ones. The results show that models positively stereotype and assume resilience in marked groups much more heavily than unmarked ones and much more often than humans do. \citet{wan-etal-2023-stochastic} aimed to categorise and measure ‘persona biases’ by creating a UniversalPersona dataset of generic and specific personas. These personas are measured against harmful expression (offensiveness, toxic continuation, and regard) and harmful agreement metrics (stereotype and toxic agreement). Findings show that models have fairness issues when taking on the role of a persona. This work is a continuation of that by \citet{deshpande-etal-2023-toxicity}, which shows that assigning a specific persona can increase toxicity up to six-fold. 

To uncover more implicit bias, \citet{gupta2024bias} evaluates the unintended effects of persona assignment by measuring the reasoning capability of different models on different tasks. The results are clear; although ChatGPT will unilaterally reply that there is no difference in the maths problem-solving skills between a physically-abled and disabled person, when adopting the identity of a physically-disabled person, it outputs that because of its disability, it is unable to perform calculations. The work by \citet{plaza2024angry} evaluates a more inferred bias that assumes women are more emotional than men, which {LLM}s seem to agree with; sadness is overwhelmingly linked with women, anger with men.

To date, no work has studied how assigning gendered personas to a model with an implied relationship with its user impacts model responses. Not acknowledging the user's role disregards the topic of sycophancy -- where {LLM}s may echo the opinions of the users they interact with. \citet{huang2024trustllm} and \citet{xu2024earthflatbecauseinvestigating} show that assigning the user a persona and then prompting the model with questions tends to have the model giving responses that would align with the user's persona. However, there is a research gap in how sycophancy may change when assigning a persona to the model system. The role of sycophancy is an essential question when focusing on {AI} companions, as the relationship between user and model is, at its core, intertwined \citep{sharma2023understandingsycophancylanguagemodels}.


\section{\cite{HaitnerMST22}'s Protocol Cannot Imply $\OT$}\label{appendix:HaitnerMST22}

In this section, we show that for some carefully chosen $\CDP$ and accurate protocol $\Pi$, the joint view of the parties in \cite{HaitnerMST22}'s protocol (\cref{protocol:HaitnerMST22}) can be \emph{simulated} using a trivial protocol, without using $\Pi$ at all.

\begin{protocol}[\cite{HaitnerMST22}'s protocol]\label{protocol:HaitnerMST22}
    \item Oracle: An accurate $\CDP$ protocol $\Pi$ for the inner-product.
    \item Operation:~
    \begin{enumerate}
        \item $\Ac$ and $\Bc$ choose random inputs $x\in \set{-1,1}^n$ and $y\in \set{-1,1}^n$, respectively.
        \item The parties interact using $\Pi$ to get approximation $z$ of $\langle x,y \rangle$.
        \item $\Ac$ chooses a random string $r \gets \zn$, and sends $r, x_r=\set{x_i \colon r_i=1}$ to $\Bc$. $\Bc$ replies with $y_{- r}=\set{x_i \colon r_i=0}$.
        \item Finally, $\Ac$ computes and outputs $\out_\Ac=\langle x_{-r},y_{-r} \rangle$, and $\Bc$ outputs $\out_\Bc=z-\langle x_{r},y_{r} \rangle$.
\end{enumerate}
\end{protocol}

To see this, assume that $\Pi$ is a protocol that on inputs $x$ and $y$, outputs $z=\langle x,y \rangle+e_\Ac+e_\Bc$, where $e_\Ac$ and $e_\Bc$ are independent samples from the $\Lap(2/\eps)$ distribution. Moreover, assume that $\Pi$ reveals $e_\Ac$ to $\Ac$ and $e_\Bc$ to $\Bc$ (and nothing else).  Such a protocol is indeed differential private, and it can be implemented using secure multi-party computation. Moreover, as we show in this work, such a protocol can be used to construct OT. However, when composed with the $\KA$ protocol of \cite{HaitnerMST22}, the resulting protocol can be simulated trivially.\footnote{More formally, and using the definition given in \cref{sec:protocol}, we claim that the channel induced by executing \cref{protocol:HaitnerMST22} with oracle access to the channel $$\Pi=\set{((x,(\langle x,y \rangle + e_A+e_B,e_A)),(y,(\langle x,y \rangle + e_A+e_B,e_B)))}_{x,y\gets \oo^n, e_A,e_B\gets \Lap(2/\epsilon)}$$ is a trivial channel.} 

Indeed, note that the view of $\Ac$ in $\Pi$ only contains $x,z$ and  $e_A$, while the view of $\Bc$ only contains $y,z$ and $e_B$. In this case, the view of $\Ac$ in the $\KA$ protocol of \cite{HaitnerMST22} contains $x,z, e_A, r$ and $y_{-r}$, while the view of $\Bc$ contains $y,z, e_B, r$ and $y_{r}$.
We next explain how to simulate this view without using $\Pi$. Consider the following protocol $\Pi'$ that  simulates the $\KA$ protocol in a reverse order:

\begin{protocol}[The simulation $\Pi'$]\label{protocol:trivial}
    \item Operation:~
    \begin{enumerate}
        \item $\Ac$ and $\Bc$ choose random inputs $x\in \set{-1,1}^n$ and $y\in \set{-1,1}^n$, respectively.
               \item $\Ac$ chooses a random string $r \gets \zn$, and sends $r, x_r=\set{x_i \colon r_i=1}$ to $\Bc$. $\Bc$ replies with $y_{- r}=\set{x_i \colon r_i=0}$.
               \item $\Ac$ samples $e_A\gets \Lap(2/\eps)$ and sends $z_A=\langle x_{r},y_r\rangle + e_A$  to $\Bc$. $\Bc$ samples $e_B\gets \Lap(2/\eps)$ and sends $z_B=\langle x_{-r},y_{-r}\rangle + e_B$  to $\Ac$.
        \item The output of the protocol is $z=z_A+z_B$.
\end{enumerate}
\end{protocol}
Clearly, \cref{protocol:trivial} is a trivial protocol and does not use any cryptographic assumptions. However, the views of $\Ac$ and $\Bc$ in $\Pi'$ contain all the information learned by the parties in the $\KA$ protocol ($(x,z,e_A,r,y_{-r})$ and $(y,z,e_B,r,x_{r})$ respectively). Moreover, we claim that the parties in $\Pi'$ do not learn any information that the parties in the $\KA$ protocol did not learn. Indeed, the only new value learned by $\Bc$, $z_A,$ can be also computed by $\Bc$ in the $\KA$ protocol by computing $z-e_B-\langle x_r,y_r \rangle$. Similarly, $z_B$ can be already computed by $\Ac$. 

We note that every protocol that uses only a communication channel cannot be used to construct $\OT$ unless $\OT$ already exists, and similarly, every protocol that uses  $\Pi'$ as a subroutine (in a black-box manner) cannot be used to construct $\OT$. Since \cref{protocol:HaitnerMST22} (that is, the views and outputs of the parties when running \cref{protocol:HaitnerMST22}) are the same as $\Pi'$, we conclude that \cref{protocol:HaitnerMST22} could not be used to construct $\OT$ in a black-box manner.  
%we conclude that \cref{protocol:HaitnerMST22} (that is, the views and outputs of the parties when running \cref{protocol:HaitnerMST22}) cannot be used in a black-box manner to construct $\OT$ unless $\OT$ already exists.

%\Nnote{Since every protocol that uses $\Pi'$ as a subroutine (in a black-box manner) cannot be used to construct OT unless $\OT$ already exists, we conclude that \cref{protocol:HaitnerMST22} (that is, the views and outputs of the parties when running \cref{protocol:HaitnerMST22}) cannot be used in a black-box manner to construct \OT (unless OT already exists).}



\section{\WEC from \AWEC (Proof of \cref{lemma:AWEC-to-WEC})}\label{sec:AWEC-to-WEC}


In this section, we prove \cref{lemma:AWEC-to-WEC} and  show how to implement \WEC (\cref{def:WEC}) from \AWEC (\cref{def:AWEC}) using a \ppt protocol. Crucially, the security proof is constructive, so that it could be used in the computational case as well (see, \cref{cor:CompAWEC to CompWEC}).


%Find a place to put this 


To prove \cref{lemma:AWEC-to-WEC}, we will need the following easy version of Goldriech-Levin \cite{GoldreichL89}.
\begin{lemma}
\label{prel:gl:weak:prob}
There exists a \ppt oracle-aided  algorithm $\Dec$ such that the following holds. Let $n\in N$ be a number, $x\in \zn$, and %$f\colon \zn \to \zn$ be a (possibly randomized) function, 
 and let $\Pred$ be an algorithm such that
\begin{align*}
\ppr{ r\gets \zn}{\Pred(r)=\GL(x,r)} > 3/4+0.001,
\end{align*}
 where $\GL(x,r)\eqdef \langle x,r \rangle$ is the Goldreich-Levin predicate. 
Then $\pr{\Dec^\Pred(1^n)=x}=1-\negl(n)$.
\end{lemma}
\begin{proof}[Proof of \cref{prel:gl:weak:prob}]
We use $\Pred$ to decode each bit of $x$ separately. For every $i$, let $e_i$ be the vector that has $1$ in the $i$-th entry, and $0$'s everywhere else. Observe that, for a uniformly chosen $R\gets \zn$, 
$$\pr{\Pred(R)=\GL(x,R) \land \Pred(R\xor e_i)=\GL(x,R\xor e_i)}\ge 1/2+0.001.$$
Thus,
$$\pr{\Pred(R)\xor\Pred(R\xor e_i) =\GL(x,R) \xor \GL(x,R\xor e_i)}\ge 1/2+0.001.$$
By linearity of the inner product we get that,
$$\pr{\Pred(R)\xor\Pred(R\xor e_i) =x_i}\ge 1/2+0.001.$$
Let $\Dec$ be the algorithm that for every $i$, computes $\Pred(R)\xor\Pred(R\xor e_i)$  for $n$ random values of $R$, and let $x'_i$ to be the majority of the outputs. Then, $\Dec$ outputs $x'=x'_1,\dots,x'_n$. By Chernoff bound, $x'_i$ is equal to $x_i$ with all but negligible probability. By the union bound, the above is true for all $i$'s simultaneously  with all but negligible probability, as we wanted to show.
\end{proof}
We are now ready to prove \cref{lemma:AWEC-to-WEC}.
\begin{proof}[Proof of \cref{lemma:AWEC-to-WEC}]
We now define the protocol $\Lambda^C$ as follows:
\begin{protocol}[$\Lambda^C=(\Ac,\Bc)$]
\item Oracle access: A channel $C =((\OA,\VA),(\OB,\VB))$.
	\item Operation:
	\begin{enumerate}
            \item Sample $((\oA,\vA),(\oB,\vB))\from C$. $\Ac$ gets $(\oA,\vA)$ and $\Bc$ gets $(\oB,\vB)$. 
		
			\item $\Ac$ chooses a random offset $s\gets [1000\ell]$ and sends it to $\Bc$. Let $\oA'=\ceil{\frac{\oA+s}{1000\ell}}$ and $\oB'=\ceil{\frac{\oB+s}{1000\ell}}$ (if $\oB=\bot$, let $\oB'=\bot$). 
        \item $\Ac$ chooses a random vector $r\from \zo^{\log(n)}$ and sends it to $\Bc$. Let $\hoA=\langle \oA',r \rangle$ and $\hoB=\langle \oB',r \rangle$ (if $\oB'=\bot$, let $\hoB=\bot$).
        \item $\Ac$ outputs $\hoA$ and $\Bc$ outputs $\hoB$.
        \end{enumerate}
\end{protocol}
Let $\tilde{C}$ be the channel induces by $\Lambda^C=(\Ac,\Bc)$ defined above. 
Let $\OA,\VA,\OB,\VB,S,R,\OA',\OB',\hOA,\hOB$ be the random variables that corresponds to the value of $\oA,\vA,\oB,\vB,s,r,\oA',\oB',\hoA,\hoB$ in a random execution of the above protocol. Denote $\hVA=(\VA,S,R)$ and $\hVB=(\VB,S,R)$ and note that $(\hVA,\hOA),(\hVB,\hOB)$ defines the channels $\tilde{C}$.



We now prove that if $C=((\OA,\VA),(\OB,\VB))$ is an $(\ell,\alpha,p,q)$-\AWEC then $\tilde{C}$ is an $(\alpha'=\alpha+0.001,\: p' = p ,\:  q' = 1/2 + 2.001q)$-\WEC.

\paragraph{Agreement:} If $\size{\OA-\OB}\le \ell$, then $\ppr{S}{\ceil{\frac{\OA+S}{1000\ell}}\ne \ceil{\frac{\OB+s}{1000\ell}}}\le 1/1000$. Thus, 
\begin{align*}
    \pr{\hOA\ne \hOB\mid \hOB\ne \bot} &\leq\pr{|\OA - \OB|\leq \ell\mid \OB\ne \bot}+1/1000\\
    &= \eps+1/1000=\eps'
\end{align*}

\paragraph{$\Bc$'s privacy:} Recall that the view of $\Ac$ in the above protocol is $\hVA=(\VA,S,R)$, and note that $S,R$ are independent of $\OB$. Since, $\hOB=\bot$ iff $\OB=\bot$ it follows that it follows that for every algorithm $\Dc$:
    \begin{align*}
    &\size{{\pr{\Dc(\hVA) = 1 \mid \hOB \neq \bot} - \pr{\Dc(\hVA) = 1 \mid \hOB = \bot}} }\\
   %  &=\size{{\pr{\Dc(\hVA) = 1 \mid \OB \neq \bot} - \pr{\Dc(\hVA) = 1 \mid \OB = \bot}} }\\
    &=\size{{\pr{\Dc(\VA,S,R) = 1 \mid \OB \neq \bot} - \pr{\Dc(\VA,S,R) = 1 \mid \OB = \bot}} }\\
    &=\size{\pr{\Dc(\VA) = 1 \mid \OB \neq \bot} - \pr{\Dc(\VA) = 1 \mid \OB = \bot}} \le p=p'.
    \end{align*}
\paragraph{$\Ac$’s privacy:}  Assume towards a contradiction that there exists an algorithm $\Dc$ such that

\begin{align}\label{eq:avrging}
\pr{\Dc(\hVB)=\hOA  \mid \hOB\neq\bot} \ge\frac{1+q'}{2}=  3/4+q+0.01.
\end{align}
Let $\cG=\set{(\vB,s)\colon \ppr{R}{\Dc(\vB,s,R)=\hOA\mid\VB=\vB,S=s,\hOB\neq \bot}\ge 3/4+0.001}$, and first note that $\ppr{\VB,S}{(\VB,S)\in \cG\mid \hOB\neq \bot}\ge q+0.009$.
Indeed, otherwise it holds that
\begin{align*}
&\pr{\Dc(\hVB)=\hOA  \mid \hOB\neq\bot}\\
& = \ppr{\VB,S}{(\VB,S)\in \cG\mid \hOB\neq \bot} + \ppr{\VB,S}{(\VB,S)\notin \cG\mid \hOB\neq \bot}\cdot(3/4+ 0.001)\\
&< (q+0.009) + 1\cdot (3/4+ 0.001)\\
&= 3/4+ q+0.01
\end{align*}
in contradiction to \cref{eq:avrging}.
   % $$\ppr{\VB,S}{\ppr{R}{\Dc(\VB,S,R)=\hOA }\ge 3/4+\delta\cdot\alpha\mid \hOB=\bot}\ge \delta(1-\alpha).$$ 
    Next, by \cref{prel:gl:weak:prob} there exists an algorithm $\Dc'$ such that 
    $$
    \pr{\Dc'(\VB,S)=O'_A \mid (\VB,S)\in\cG, \OB'\neq \bot}\ge 1-o(1)
    $$ 
    Which implies that, 
    \begin{align*}
    \pr{\Dc'(\VB,S)=O'_A \mid  \OB'\neq \bot}
    &\ge \pr{\Dc'(\VB,S)=O'_A \mid (\VB,S)\in\cG, \OB'\neq \bot}\cdot \pr{\Dc'(\VB,S)\in\cG \mid\hOB\neq \bot}\\
    &\ge (q+0.009)(1-o(1))\\
    &\ge q.
    \end{align*}
    %for every fixing of $\VB,S$ to values $\vB,s$ such that 
    % $$
    % \ppr{R}{\Dc(\vB,s,R)=\hOA\mid\VB=\vB,S=s,\hOB= \bot}\ge 3/4+\delta\cdot\alpha
    % $$
    % it holds that 
    % it holds that $\Dc'$
    % there exists an algorithm $\Dc'$
    % it holds that $$\pr{\Dc'(\VB,S)=O'_A }\ge (1-\alpha)$$ ($\Dc'$ is the GL reconstruction), 
    Since by definition $\size{(O'_A\cdot 1000\ell-S)-\OA}\le 1000\ell$, and $S$ is independent of $\VB$, it follows that there exists an algorithm $\Dc''$ such that  
    $$\pr{\size{\Dc''(\VB)-\OA}\le 1000\ell\mid \OB\neq \bot}\ge \delta(1-2\alpha)> q.$$
Contradicting the fact that $C$ is an $(\ell,\alpha,p,q)$-\AWEC.
\end{proof}
\section{Missing Proofs}\label{sec:missing-proofs}
\remove{
\subsection{Proving \cref{lem:distance-I}}\label{sec:missing-proofs:distance-I}

We make use of the following fact.

\begin{fact}[Proposition 3.28 in \cite{HaitnerMST22}]\label{fact:I}
	Let $R$ be uniform random variable over $\{0,1\}^n$, and let $I$ be uniform random variable over $\mathcal{I}\subseteq[n]$, independent of $R$, then $SD(R|_{R_I=0},R|_{R_I=1})\leq1/\sqrt{\size{\cI}}$.
\end{fact}
We now prove \cref{lem:distance-I}, restated below.

\begin{lemma}[Restatement of \cref{lem:distance-I}]
    \distanceILemma
\end{lemma}
\begin{proof}
    Assume towards a contradiction that there exist $f$ and $\alpha$ such that \cref{eq:f-alpha} does not hold.
    Namely, for $m = 1/\alpha^2$, there exist more than $2m$ indices $i \in [n]$ with $$\size{\:\ex{f(R) \mid R_i = 0}-\ex{f(R) \mid R_i = 1}\:}\geq 1/\sqrt{m}.$$
    This implies that there exist $b \in \oo$ and more than $m$ indices $i \in [n]$ with $$\ex{f(R) \mid R_i = b}-\ex{f(R) \mid R_i = 1-b} \geq 1/\sqrt{m}$$ (denote this set by $\cI$). 
    Thus, we deduce for $I \la \cI$ that
    \begin{align}\label{eq:big-R_I-gap}
        \size{\ex{f(R) \mid R_I = 0}-\ex{f(R) \mid R_I = 1}} 
        &\geq \ex{f(R) \mid R_I = b}-\ex{f(R) \mid R_I = 1-b}\\
        &\geq 1/\sqrt{m}.\nonumber
    \end{align}
    On the other hand, note that
    \begin{align*}
        SD(f(R)|_{R_I=0},f(R)|_{R_I=1})
        \leq SD(R|_{R_I=0},R|_{R_I=1})
        \leq 1/\sqrt{\size{\cI}}
        < 1/\sqrt{m},
    \end{align*}
    where the second inequality holds by \cref{fact:I}.
    Thus, we conclude that
    \begin{align*}
        \size{\ex{f(R) \mid R_I = 0}-\ex{f(R) \mid R_I = 1}}
        &\leq SD(f(R)|_{R_I=0},f(R)|_{R_I=1}) \cdot \sup_{r \in \zo^n, s \in \zo^*}(\size{f_s(r)})\\
        &< 1/\sqrt{m},
    \end{align*}
    where $f_s(r)$ denotes the function $f$ when fixing its random coins to $s$. 
    The above inequality contradicts \cref{eq:big-R_I-gap}, concluding the proof of the lemma.
\end{proof}
}

\subsection{Proving \cref{prop:hard-to-guess-inf,prop:hard-to-guess-comp}}\label{sec:missing-proofs:hard-to-guess}

We make use of the following claim.

\begin{claim}\label{claim:X-star}
    Let $X \la \oo$ and let $X^*$ be a random variable over $\set{-1,1,\bot}$ (correlated with $X$) such that for every $b,b' \in \oo$:
    \begin{align*}
        \pr{X^* = b \mid X = b'} \leq e^{\eps}\cdot \pr{X^* = b \mid X = -b'} + \delta.
    \end{align*}
    Then
    \begin{align*}
        \pr{X^* = X} \leq e^{\eps}\cdot \pr{X^* = -X} + \delta.
    \end{align*}
\end{claim}
\begin{proof}
    Compute
    \begin{align*}
        \pr{X^* = X } 
        &= \frac12 \cdot \pr{X^* = -1 \mid X = -1} + \frac12 \cdot \pr{X^* = 1 \mid X = 1}\\
        &\leq  \frac12 \cdot \paren{e^{\eps}\cdot \pr{X^* = -1 \mid X = 1} + \delta} + \frac12 \cdot \paren{e^{\eps}\cdot \pr{X^* = 1 \mid X = -1} + \delta}\\
        &= e^{\eps} \cdot \paren{\frac12 \cdot \pr{X^* = -1 \mid X = 1} + \frac12 \cdot \pr{X^* = 1 \mid X = -1}} + \delta\\
        &= e^{\eps} \cdot \pr{X^*_i = -X_i} + \delta.
    \end{align*}
\end{proof}


We next prove \cref{prop:hard-to-guess-inf}, restated below.

\begin{proposition}[Restatement of \cref{prop:hard-to-guess-inf}]
    \propHardToGuessInf
\end{proposition}
\begin{proof}
    %Note that it is enough to prove the claim just for deterministic $g$'s because if the statement does not hold for a specific randomized $g$, then there exists a fixing of its randomness that it does not hold under this fixing. 
    
    Fix $b,b' \in \oo$ and $i \in [n]$. By \cref{claim:X-star}, it is sufficient to prove that
    \begin{align}\label{eq:X_i^*-goal}
        \pr{X_i^* = b \mid X_i = b'} \leq e^{\eps}\cdot \pr{X_i^* = b \mid X_i = -b'} + \delta.
    \end{align}
    For $x_{-i} = (x_1,\ldots,x_{i-1},x_{i+1},\ldots,x_n) \in \oo^{n-1}$, define the function
    \begin{align*}
        h_{x_{-i}}(y) = g(i,x_{-i},y).
    \end{align*}
    Since $f$ is $(\eps,\delta)$-\DP, for any $x_{-i} \in \oo^{n-1}$ it holds that
    \begin{align*}
        \pr{h_{x_{-i}}(f(x_1,\ldots,x_{i-1}, b', x_{i+1},\ldots,x_n)) = b} \leq e^{\eps}\cdot \pr{h_{x_{-i}}(f(x_1,\ldots,x_{i-1}, -b', x_{i+1},\ldots,x_n)) = b} + \delta.
    \end{align*}
    Thus,
    \begin{align*}
        \pr{X_i^* = b \mid X_i = b'}
        &= \eex{x_{-i} \la \oo^{n-1}}{\pr{g(i,x_{-i}, f(x_1,\ldots,x_{i-1}, b', x_{i+1},\ldots,x_n)) = b }}\\
        &= \eex{x_{-i} \la \oo^{n-1}}{\pr{h_{x_{-i}}(f(x_1,\ldots,x_{i-1}, b', x_{i+1},\ldots,x_n)) = b}}\\
        &\leq e^{\eps}\cdot \eex{x_{-i} \la \oo^{n-1}}{\pr{h_{x_{-i}}(f(x_1,\ldots,x_{i-1}, -b', x_{i+1},\ldots,x_n)) = b}} + \delta\\
        &= e^{\eps}\cdot\pr{X_i^* = b \mid X_i = -b'}+ \delta.
    \end{align*}
\end{proof}


We next prove \cref{prop:hard-to-guess-comp}, restated below.

\begin{proposition}[Restatement of \cref{prop:hard-to-guess-comp}]
    \propHardToGuessComp
\end{proposition}
\begin{proof}
    
    In the following, fix $b,b' \in \oo$, and for $\pk \in \bbN$, $i \in [n(\pk)]$ and $x_{-i} \in \oo^{n(\pk)-1}$, define 
    \begin{align*}
        h_{\pk}^{i,x_{-i}}(y) = b\cdot g_{\pk}(i,x_{-i},y).
    \end{align*}
    Note that for any ensemble $\set{(i,x_{-i})_{\pk} \in [n(\pk)]\times \oo^{n(\pk)-1}}_{\pk \in \bbN}$, the circuit family \\$\set{h_{\pk}^{(i, x_{-i})_{\pk}} = g_{\pk}((i,x_{-i})_{\pk},\cdot)}_{\pk \in \bbN}$ has poly-size. 
    Since $f = \set{f_{\pk}}_{\pk \in \bbN}$ is $(\eps,\delta)$-\CDP, then for large enough $\pk$, the following holds for every $i \in [n(\pk)]$ and $x_{-i} \in \oo^{n(\pk)-1}$:
    \begin{align*}
        \lefteqn{\pr{h_{\pk}^{i, x_{-i}}(f_{\pk}(x_1,\ldots,x_{i-1}, b', x_{i+1},\ldots,x_n)) = 1}}\\
        &\leq e^{\eps(\pk)}\cdot \pr{h_{\pk}^{i,x_{-i}}(f(x_1,\ldots,x_{i-1}, -b', x_{i+1},\ldots,x_n)) = 1} + \delta(\pk),
    \end{align*}
    as otherwise, there would exist an ensemble $\set{(i,x_{-i})_{\pk} \in [n(\pk)]\times \oo^{n(\pk)-1}}_{\pk \in \cS}$ for an infinite set $\cS \subseteq \bbN$ such that the circuit family 
    $\set{h_{\pk}^{(i, x_{-i})_{\pk}}}_{\pk \in \cS}$ violates the $(\eps,\delta)$-\CDP property of $f$. 

    In the following, fix such large enough $\pk$ and $i \in [n]$ for $n = n(\kappa)$, let $X = (X_1,\ldots,X_{n}) \la \oo^{n}$ and $X_i^* = g_{\pk}(i,X_{-i},f_{\pk}(X_1,\ldots,X_n))$, and compute
    \begin{align*}
        \lefteqn{\pr{X_i^* = b \mid X_i = b'}}\\
        &= \eex{x_{-i} \la \oo^{n-1}}{\pr{g_{\pk}(i,x_{-i}, f_{\pk}(x_1,\ldots,x_{i-1}, b', x_{i+1},\ldots,x_n)) = b }}\\
        &= \eex{x_{-i} \la \oo^{n-1}}{\pr{h_{\pk}^{i,x_{-i}}(f(x_1,\ldots,x_{i-1}, b', x_{i+1},\ldots,x_n)) = 1}}\\
        &\leq e^{\eps(\pk)}\cdot \eex{x_{-i} \la \oo^{n-1}}{\pr{h_{\pk}^{i,x_{-i}}(f(x_1,\ldots,x_{i-1}, -b', x_{i+1},\ldots,x_n)) = 1}} + \delta(\pk)\\
        &= e^{\eps(\pk)}\cdot \eex{x_{-i} \la \oo^{n-1}}{\pr{g_{\pk}(i,x_{-i}, f_{\pk}(x_1,\ldots,x_{i-1}, -b', x_{i+1},\ldots,x_n)) = b}} + \delta(\pk)\\
        &= e^{\eps(\pk)}\cdot\pr{X_i^* = b \mid X_i = -b'}+\delta(\pk).
    \end{align*}
    Since the above holds for any $b, b' \in \oo$, we conclude by \cref{claim:X-star} that 
    \begin{align*}
        \pr{X_i^* = X_i} \leq e^{\eps(\pk)}\cdot \pr{X_i^* = -X_i} + \delta(\pk),
    \end{align*}
    as required.
\end{proof}

\subsection{Proving \cref{lemma:property1:prediction}}\label{sec:prediction-lemma}

To prove \cref{lemma:property1:prediction}, we use the following lemma that measures the distance between two uniform stings conditioned on a random index $i$ either being fixed to $0$ or to $1$.

\def\distanceILemma{
    Let $R \la \zo^n$. For any (randomized) function $F:\{0,1\}^n\rightarrow \{0,1\}$ and $\alpha > 0$, it holds that
    \begin{align}\label{eq:f-alpha}
        \ppr{i \la [n]}{\size{\:\ex{F(R) \mid R_i = 0}-\ex{F(R) \mid R_i = 1}\:}\geq \alpha} \leq \frac{2}{n \alpha^2},
    \end{align}
    where the expectations are taken over $R$ and the randomness of $F$.
}

\begin{lemma}\label{lem:distance-I}
    \distanceILemma
\end{lemma}

The proof of \cref{lem:distance-I} uses the following fact.

\begin{fact}[Proposition 3.28 in \cite{HaitnerMST22}]\label{fact:I}
	Let $R$ be uniform random variable over $\{0,1\}^n$, and let $I$ be uniform random variable over $\mathcal{I}\subseteq[n]$, independent of $R$, then $SD(R|_{R_I=0},R|_{R_I=1})\leq1/\sqrt{\size{\cI}}$.
\end{fact}

We first prove \cref{lem:distance-I} using \cref{fact:I}.

\begin{proof}[Proof of \cref{lem:distance-I}]
    Assume towards a contradiction that there exist $F$ and $\alpha$ such that \cref{eq:f-alpha} does not hold.
    Namely, for $m = 1/\alpha^2$, there exist more than $2m$ indices $i \in [n]$ with $$\size{\:\ex{F(R) \mid R_i = 0}-\ex{F(R) \mid R_i = 1}\:}\geq 1/\sqrt{m}.$$
    This implies that there exist $b \in \oo$ and more than $m$ indices $i \in [n]$ with $$\ex{F(R) \mid R_i = b}-\ex{F(R) \mid R_i = 1-b} \geq 1/\sqrt{m}$$ (denote this set by $\cI$). 
    Thus, we deduce for $I \la \cI$ that
    \begin{align}\label{eq:big-R_I-gap}
        \size{\ex{F(R) \mid R_I = 0}-\ex{F(R) \mid R_I = 1}} 
        &\geq \ex{F(R) \mid R_I = b}-\ex{F(R) \mid R_I = 1-b}\\
        &\geq 1/\sqrt{m}.\nonumber
    \end{align}
    On the other hand, note that
    \begin{align*}
        SD(F(R)|_{R_I=0},F(R)|_{R_I=1})
        \leq SD(R|_{R_I=0},R|_{R_I=1})
        \leq 1/\sqrt{\size{\cI}}
        < 1/\sqrt{m},
    \end{align*}
    where the second inequality holds by \cref{fact:I}.
    Thus, we conclude that
    \begin{align*}
        \size{\ex{F(R) \mid R_I = 0}-\ex{F(R) \mid R_I = 1}}
        &\leq SD(F(R)|_{R_I=0},F(R)|_{R_I=1}) \cdot \sup_{r \in \zo^n, s \in \zo^*}(\size{F_s(r)})\\
        &< 1/\sqrt{m},
    \end{align*}
    where $F_s(r)$ denotes the function $F$ when fixing its random coins to $s$. 
    The above inequality contradicts \cref{eq:big-R_I-gap}, concluding the proof of the lemma.
\end{proof}

Using \cref{lem:distance-I}, we now prove \cref{lemma:property1:prediction}, restated below.

\begin{lemma}[Restatement of \cref{lemma:property1:prediction}]
    \PredictionLemma
\end{lemma}
\begin{proof}
	
	Let $\gamma \in (0,1)$ and $n \in \bbN$ and consider the following algorithm $\Gc = \Gc_{\gamma}$:
	\begin{algorithm}
		\item Inputs: $i \in [n]$,  $z_{-i} = (z_1,\ldots,z_{i-1},z_{i+1},\ldots,z_n) \in \oo^{n-1}$ and $w \in \zo^*$.
		\item Parameter: $\gamma \in (0,1)$.
		\item Oracle: $F \colon \zo^n \times \oo^{\leq n} \times \zo^* \rightarrow \zo$.
		\item Operation:~
		\begin{enumerate}
			\item For $b \in \oo$:
			\begin{enumerate}
				\item Let $z^b = (z_1,\ldots,z_{i-1},b,z_{i+1},\ldots,z_n)$.
				\item Estimate $\mu^b \eqdef \eex{r \la \zo^n \mid r_i = 1, \: F}{F(r,z^b_{r}, w)}$ as follows:
				\begin{itemize}
					\item Sample $r_1,\ldots,r_{s} \la \set{r \in \zo^n \colon r_i = 1}$, for $s = \ceil{\frac{128 \log (12n)}{\gamma^2}}$, and then sample $\tilde{\mu}^b \la \frac1{s} \sum_{j=1}^s F(r_j,z^b_{r_j}, w)$ (using $s$ oracle calls to $F$).
				\end{itemize}
			\end{enumerate}
			\item Estimate $\mu^* \eqdef \eex{r \la \zo^n \mid r_i = 0, \: f}{f(r,z^1_{r}, w)}$ as follows:
			\begin{itemize}
				\item Sample $r_1,\ldots,r_{s}  \la \set{r \in \zo^n \colon r_i = 0}$, for $s = \ceil{\frac{128 \log (12n)}{\gamma^2}}$, and then sample $\tilde{\mu}^* \la \frac1{s} \sum_{j=1}^s F(r_j,z^1_{r_j}, w)$ (using $s$ oracle calls to $F$).
			\end{itemize}
			\item If exists $b \in \oo$ s.t. $\size{\tilde{\mu}^b - \tilde{\mu}^*} < \gamma/4$ and $\size{\tilde{\mu}^{-b} - \tilde{\mu}^*} > \gamma/4$, output $b$.
			\item Otherwise, output $\bot$.
		\end{enumerate}
	\end{algorithm}
	In the following, fix a pair of (jointly distributed) random variables $(Z,W) \in \oo^n \times \zo^*$ and a randomized function  $F \colon \zo^n \times \oo^{\leq n} \times \zo^* \rightarrow \oo$ that satisfy 
	\begin{align*}
		\size{\ex{F(R,Z_{R},W) - F(R,Z^{(I)}_{R},W)}} \geq \gamma,
	\end{align*}
	for $R \la \zo^n$ and $I \la [n]$ that are sampled independently. 
	Our goal is to prove that 
	
	\begin{align}\label{eq:predic-lemma:UB}
		\pr{\Gc^F(I,Z_{-I}, W) = -Z_I} \leq O\paren{\frac{1}{\gamma^2 n}},
	\end{align}
	and 
	\begin{align}\label{eq:predic-lemma:LB}
		\pr{\Gc^F(I,Z_{-I}, W) = Z_I} \geq \Omega(\gamma) - O\paren{\frac{1}{\gamma^2 n}}.
	\end{align}

	Note that
	\begin{align}\label{eq:D}
		\text{For every random variable }D \in [-1,1]\text{ with }\size{\ex{D}} \geq \gamma >0: \quad \pr{\size{D} > \gamma/2} > \gamma/2,
	\end{align}
	as otherwise, $\size{\ex{D}} \leq \ex{\size{D}} \leq 1\cdot \frac{\gamma}{2} + \frac{\gamma}{2}(1-\frac{\gamma}{2}) < \gamma$.
	
	
	By applying \cref{eq:D} with $D = \eex{r \la \zo^n, F}{F(r,Z_{r},W) - f(r,Z^{(I)}_{r},W)}$, it holds that
	\begin{align}\label{eq:main-lemma:good}
		\ppr{(i,z,w) \la (I,Z,W)}{\size{\eex{r \la \zo^n, F}{F(r,z_{r},w) - F(r,z^{(i)}_{r},w)}} > \gamma/2} > \gamma/2.
	\end{align}
	On the other hand, for every fixing of $(z,w) \in \Supp(Z,W)$, we can apply \cref{lem:distance-I} with the function $F_{z,w}(r) = F(r,z_{r},w)$ and with $\alpha =\frac{\gamma}{16}$ to obtain that
	\begin{align*}
		\ppr{i\la I}{\: \size{\eex{r \la \zo^n \mid r_i = 0, \: F}{F(r,z_{r},w)} - \eex{r \la \zo^n \mid r_i = 1, \: F}{F(r,z_{r},w)} \:} \geq \frac{\gamma}{16}} \leq \frac{512}{n \gamma^2}.
	\end{align*}
	But since the above holds for every fixing of $(z,w)$, then in particular it holds that
	\begin{align}\label{eq:main-lemma:bad}
		\ppr{(i,z,w) \la (I,Z,W)}{\: \size{\eex{r \la \zo^n \mid r_i = 0, \: F}{F(r,z_{r},w)} - \eex{r \la \zo^n \mid r_i = 1, \: F}{F(r,z_{r},w)} \:} \geq \frac{\gamma}{16}} \leq \frac{512}{n \gamma^2}.
	\end{align}
	
	We next prove \cref{eq:predic-lemma:UB,eq:predic-lemma:LB} using \cref{eq:main-lemma:good,eq:main-lemma:bad}.
	
	In the following, for a triplet $t = (i,z,w) \in \Supp(I,Z,W)$, consider a random execution of $\Gc^F(i,z_{-i},w)$. For $x \in \set{-1,1,*}$, let $\mu^x_t$ be the value of $\mu^x$ in the execution, and let $M^x_t$ be the (random variable of the) value of $\tilde{\mu}^x$ in the execution. Note that by definition, it holds that $\mu_{i,z,w}^{z_i} = \eex{r \la \zo^n, F}{F(r,z_{r},w)}$ and $\mu_{i,z,w}^{-z_i} = \eex{r \la \zo^n, F}{F(r,z^{(i)}_{r},w)}$. Therefore, \cref{eq:main-lemma:good} is equivalent to 
	\begin{align}\label{eq:main-lemma:good2}
		\ppr{t=(i,z,w) \la (I,Z,W)}{\size{\mu_t^{z_i} - \mu_t^{-z_i}} > \gamma/2} > \gamma/2.
	\end{align}
	Furthermore, note that $\mu_{i,z,w}^* \eqdef \eex{r \la \zo^n \mid r_i = 0, \: F}{F(r,z^1_{r}, w)} = \eex{r \la \zo^n \mid r_i = 0, \: F}{F(r,z_{r}, w)}$ and that $\mu_{i,z,w}^{z_i} = \eex{r \la \zo^n \mid r_i = 1, \: F}{F(r,z^{z_i}_{r}, w)}$. Therefore, \cref{eq:main-lemma:bad} is equivalent to 
	\begin{align}\label{eq:main-lemma:bad2}
		\ppr{t = (i,z,w) \la (I,Z,W)}{\: \size{\mu_t^* - \mu_t^{z_i}}\geq \frac{\gamma}{16}}  \leq \frac{512}{n \gamma^2}.
	\end{align}
	
	We next prove the lemma using \cref{eq:main-lemma:good2,eq:main-lemma:bad2}.
	
	Note that by Hoeffding's inequality, for every $t = (i,z,w) \in \Supp(I,Z,W)$ and  $x \in \set{-1,1,*}$ it holds that $\pr{\size{M_t^x - \mu_t^x} \geq \frac{\gamma}{16}} \leq 2\cdot e^{-2 s \paren{\frac{\gamma}{16}}^2} \leq \frac1{6n}$, which yields that for every fixing of $t = (i,z,w) \in \Supp(I,Z,W)$, w.p.\ at least $1-\frac1{2n}$ we have for all $x \in \set{-1,1,*}$ that $\size{M_t^x - \mu_t^x} < \frac{\gamma}{16}$ (denote this event by $E_t$).
	
	The proof of \cref{eq:predic-lemma:UB} holds by the following calculation:
	\begin{align*}
		\lefteqn{\pr{\Gc^F(I,Z_{-I},W) = -Z_I}}\\
		&= \eex{(i,z,w) \la (I,Z,W)}{\pr{\Gc^F(i,z_{-i},w) = -z_i}}\\
		&= \eex{t = (i,z,w) \la (I,Z,W)}{\pr{\set{\size{M_t^* - M_t^{z_i}} > \gamma/4} \land \set{\size{M_t^* - M_t^{-z_i}} < \gamma/4}}}\\
		&\leq \eex{t =(i,z,w) \la (I,Z,W)}{\pr{\size{M_t^* - M_t^{z_i}} > \gamma/4}}\\
		&\leq \eex{t = (i,z,w) \la (I,Z,W)}{\pr{\size{M_t^* - M_t^{z_i}} > \gamma/4 \mid E_t}} + \frac{1}{2n}\\
		&\leq \ppr{t = (i,z,w) \la (I,Z,W)}{\size{\mu_t^* -  \mu_t^{z_i}} \geq \frac{\gamma}{16}} + \frac{1}{2n}\\
		&\leq \frac{512}{n \gamma^2} + \frac1{2n},
	\end{align*}
	The second inequality holds since $\pr{\neg E_t} \leq \frac1{2n}$ for every $t$. The penultimate inequality holds since conditioned on $E_t$, it holds that $\size{M_t^* - \mu_t^*} \leq \frac{\gamma}{16}$ and $\size{M_t^{z_i} - \mu_t^{z_i}} \leq \frac{\gamma}{16}$, which implies that $\size{M_t^* - M_t^{z_i}} > \gamma/4 \: \implies \: \size{\mu_t^* -  \mu_t^{z_i}} \geq \frac{\gamma}{4} - 2\cdot \frac{\gamma}{16} > \frac{\gamma}{16}$. The last inequality holds by \cref{eq:main-lemma:bad2}.
	
	It is left to prove \cref{eq:predic-lemma:LB}. 
	Observe that \cref{eq:main-lemma:good2,eq:main-lemma:bad2} imply with probability at least $\gamma/2 - \frac{512}{n \gamma^2}$ over $t = (i,z,w)\in (I,Z,W)$, we have $\size{\mu_t^* -  \mu_t^{z_i}} \leq  \frac{\gamma}{16}$ and $\size{\mu_t^{z_i}- \mu_t^{-z_i}} \geq\frac{\gamma}{2}$. Therefore, conditioned on event $E_t$ (i.e., $\forall x \in \set{-1,1,*}: \: \size{M_t^{x} - \mu_t^{x}} \leq \frac{\gamma}{16}$), we have for such triplets $t = (i,z,w)$ that 
	\begin{align*}
		\size{M_t^* - M_t^{-z_i}} 
		& \geq \size{\mu_t^{z_i} - \mu_t^{-z_i}} - \size{\mu_t^{z_i} - \mu_t^*} - \size{M_t^*  - \mu_t^*} -  \size{M_t^{-z_i} - \mu_t^{-z_i}}\\
		&\geq \frac{\gamma}{2} - 3\cdot \frac{\gamma}{16}\\
		&> \gamma/4,
	\end{align*}
	and 
	\begin{align*}
		\size{M_t^* - M_t^{z_i}}
		&\leq  \size{M_t^* - \mu_t^*} + \size{\mu_t^* - \mu_t^{z_i}} + \size{\mu_t^{z_i} - M_t^{z_i}} \\
		&\leq 3\cdot \frac{\gamma}{16}\\
		&< \gamma/4.
	\end{align*}
	
	Thus, we conclude that
	\begin{align*}
		\lefteqn{\pr{\Gc^F(I,Z_{-I},W) = Z_I}}\\
		&= \eex{(i,z,w) \la (I,Z,W)}{\pr{\Gc^F(i,z_{-i},w) = z_i}}\\
		&= \eex{t = (i,z,w) \la (I,Z,W)}{\pr{\set{\size{M_t^* - M_t^{-z_i}} > \gamma/4} \land \set{\size{M_t^* - M_t^{z_i}} < \gamma/4}}}\\
		&\geq \paren{1 - \frac1{2n}}\cdot \eex{t = (i,z,w) \la (I,Z,W)}{\pr{\set{\size{M_t^* - M_t^{-z_i}} > \gamma/4} \land \set{\size{M_t^* - M_t^{z_i}} < \gamma/4} \mid E_t}}\\
		&\geq \paren{1 - \frac1{2n}}\cdot \paren{\gamma/2 - \frac{512}{n \gamma^2}}\\
		&\geq \gamma/4 - \frac{512}{n \gamma^2},
	\end{align*}
	which proves \cref{eq:predic-lemma:LB}. The first inequality holds since $\pr{E_t} \geq 1-\frac1{2n}$ for every $t = (i,z,w)$, and the second one holds by the observation above.
	
\end{proof}


\end{document}
