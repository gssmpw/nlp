Following~\cite{FeigenbaumSY17}, we also investigate other than Euclidean norms $\lqnorm$ in dimensions $d>1$. Thus our main question is
\begin{quote}
What are the approximation ratios of median mechanism in $\lqnorm(\reals^d)$ for any $d\in\nats$ and $q\in\reals_{\ge 1}\cup\{\infty\}$?  
\end{quote}
As it turns out, the common belief of $\sqrt{d}$ approximation is wrong for any value of $q$. Specifically, we obtain the following results:    
\begin{enumerate}
        \item We show that coordinate-wise median has approximation ratio of at most $3$ in $\lqnorm(\reals^d)$ for any values of $d$ and $q$. This improves 
        previous upper bound of $\sqrt{d}$ for the Euclidean distance, and also gives new constant upper bounds for all $\lqnorm$ norms.  
        \item We formulate the question about approximation guarantee as a certain mathematical program and then consider appropriate relaxations. In particular, it is 
        convenient to make $d\to\infty$. After carefully examining various local optimum conditions we reduce the relaxed problem to a simple system of differential equations. 
        For each value of $q\in[1,\infty)$ we find solutions to this system and derive our upper bound $\UB(q)$ on the approximation ratio. 
        \item Our analysis is tight, as we demonstrate with a series of examples that yield lower bounds $\LB(q,d)$ approaching our upper bounds $\UB(q)$ at the rate 
        $O(\frac{1}{d})$ for every $q$. For the most well studied Euclidean $\lqnorm[2]$ norm, the difference between our $\UB(2)=\sqrt{6\sqrt{3}-8}<1.55$ and the existing 
        lower bound of 
        $\sqrt{2}>1.41$ is already quite small (less than $10\%$).
        \item By improving $\sqrt{d}$-approximation to a small constant, our work implies better guarantees in the facility location setting with MAC advice of Barak, Gupta, and Talgam-Cohen~\cite{barak2024mac}. We also apply our optimization framework to the generalized median mechanism of Agrawal et al.~\cite{AgrawalBGTX22} in 
        the setting of facility location with predictions. We derive slightly worse (by a similar factor of $\approx 10\%$ than for $d=2$) approximation guarantees for both consistency and robustness of the mechanism that holds in arbitrary dimension $d$.
    \end{enumerate}


