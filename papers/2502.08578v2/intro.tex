%Strategical facility location -- canonical setting
Strategic facility location is the canonical problem in the literature on mechanism design without money~\cite{procaccia2013approximate}. 
The goal in this problem is to find a good location for the new facility given preferences of $n$ strategic agents. 
This problem has been the focus of a long 
line of work in algorithmic mechanism design literature~\cite{procaccia2013approximate, FotakisT14, FotakisT16, SerafinoV16, walsh2020strategy, GkatzelisKST22}  as well as earlier body of work in social choice~\cite{moulin1980strategy,border1983straightforward,kim1984nonmanipulability,peters1993range,barbera1993generalized,ching1997strategy,peremans1997strategy,barbera1998strategy,schummer2002strategy}. In the former case the setting often served as an important domain for testing new concepts in approximate mechanism design, such as truthful mechanisms without monetary transfers~\cite{procaccia2013approximate} as well as mechanism design with predictions~\cite{GkatzelisKST22,XuL22,barak2024mac}, while in the later case it has been the primary domain with structured preferences that allowed one to escape strong impossibility results such as the Gibbard-Satterthwaite Theorem~\cite{Gibbard77,Satterthwaite75}.

% \nick{Let's change Procaccia 2013 reference to conference version, which first have appeared in 2009.}
%
%%% Median and generalized median mechanism is the most well studied solution.
The social choice literature very thoroughly studied strategy-proof voting scenarios with single-picked preferences~\cite{moulin1980strategy} in $\reals^1$ and its extensions~\cite{barbera1993generalized,barbera1998strategy,schummer2002strategy,ching1997strategy} usually  to $d$-dimensional Euclidean~\cite{kim1984nonmanipulability,peters1993range,border1983straightforward,peremans1997strategy} spaces $\reals^d$.
All this work crucially relies on the simple \emph{median} selection rule initially proposed by Black~\cite{Black48}. The median rule, which puts facility at the median of the agents' reported peaks, is strategy-proof. Its generalized\footnote{The mechanism simply adds a set of fixed points to the reports of all agents.} variants are the only deterministic strategy-proof mechanisms for agents residing on continuous or discrete lines~\cite{DokowFMN12,schummer2002strategy}, and it is the only strategy-proof mechanism with optimal utilitarian social cost, when agents have single-peaked preferences~\cite{moulin1980strategy}. Furthermore, the median rule naturally extends to strategy-proof \emph{coordinate-wise median} mechanism in higher dimensions $\reals^d$: facility's coordinates are separately computed as medians in each of the $d$ dimensions.
In Euclidean space $\lqnorm[2](\reals^2)$, the coordinate-wise median is the only deterministic strategy-proof mechanism that is Pareto optimal and anonymous~\cite{peters1993range}. 

%%% More recent papers on Mechanism Design -- focus on approximation guarantees.
The computer science community, on the other hand, has been mostly concerned with establishing approximate efficiency guarantees and extending original strategic facility location setting in multiple directions (see~\cite{chan2021mechanism} for a recent survey on the topic). The most notable extensions include randomized mechanisms, which  first appeared in~\cite{AlonFPT10} and are commonly used in general metric spaces~\cite{LuSWZ10,FotakisT10,FeldmanW13}; and 
opening of $k\ge 2$ facilities~\cite{procaccia2013approximate,lu2009,LuSWZ10,FotakisT10,EscoffierGTPS11,FotakisT16,walsh2020strategy}, which among other things showed impossibility of constant approximation deterministic mechanisms for installing $k>2$ facilities in $\reals^1$ and $k=2$ facilities in general metric spaces.
However, despite active research in the last fifteen years since the seminal work of Procaccia and Tennenholtz~\cite{procaccia2013approximate}, we do not have tight efficiency guarantees of the celebrated median mechanism in $\reals^d$. This is the question we address in our work. Before listing our results, let us first do a quick overview of what is known about approximation guarantees of the  coordinate-wise median.

%%% Approximation of coordinate-wise 
\begin{description}
\item[$d=1$.] The median mechanism is deterministic and achieves $1$-approximation in $\reals^d$ for $d=1$ and utilitarian social cost~\cite{procaccia2013approximate}. 
Feigenbaum et al.~\cite{FeigenbaumSY17} showed that its approximation guarantee in $\lqnorm[p](\reals^1)$ space is $2^{1-1/p}$, i.e., when the social cost is measured as the $\lqnorm[p]$ norm of agents' utilities for $p\in[1,+\infty]$. 
\item[$d=2$.] 
The coordinate-wise median is $\sqrt{2}$-approximation to the optimum in the Euclidean space $\lqnorm[2](\reals^2)$~\cite{meir2019strategyproof}. This guarantee is tight~\cite{GoelHC23} for $d=2$. Goel and Hann-Caruthers~\cite{GoelHC23} also considered other $\lqnorm(\reals^2)$ spaces. They showed that median rule has the lowest worst-case approximation ratio among deterministic, anonymous, and strategyproof mechanisms and that its approximation ratio is between $[2^{1-\frac{1}{q}},2^{\frac{3}{2}-\frac{2}{q}}]$ for $\lqnorm(\reals^2)$ spaces.
\item[$d>2$.]  Meir~\cite{meir2019strategyproof} also showed that the $\sqrt{2}$ bound extends to $\sqrt{d}$-approximation guarantee in $\lqnorm[2](\reals^d)$ spaces for any $d\ge 1$. This result\footnote{While it could have been known before, we are only aware of one source, where it was explicitly given -- the appendix of Meir's paper~\cite{meir2019strategyproof}.} for $\lqnorm[2](\reals^d)$ easily follows from two very simple observations: coordinate-wise median is optimal in $\lqnorm[1](\reals^d)$, and the canonical embedding of $\lqnorm[1](\reals^d)$ space into $\lqnorm[2](\reals^d)$ space has distortion\footnote{We refer here to the standard mathematical term ``distortion of metric embedding'', which is calculated as the ratio between maximal and minimal multiplicative discrepancies between two distance functions.} of $\sqrt{d}$. This bound of $\sqrt{d}$ is the best known up to date (see, e.g.,~\cite{barak2024mac}), and, rather surprisingly, no matching lower bound of order $\sqrt{d}$ are known for higher dimensions.
\item[Other Metrics.] There has been a lot of interest in extending it beyond $\reals^d$ 
to a broader set of metric spaces, e.g., given by shortest path distances on a network. However, for a simple cycle metric, any onto and strategy-proof mechanism must be dictatorial~\cite{schummer2002strategy}, and the worst-case approximation ratio of any deterministic strategy-proof mechanism is of order $\Omega(n)$~\cite{DokowFMN12}. This essentially limits the scope of possible application scenarios to trees~\cite{DokowFMN12,FeldmanW13}, or to the instances with $n=O(1)$ a constant number of agents~\cite{meir2019strategyproof}. Furthermore, the appropriately defined median mechanism is strategy-proof and achieves optimal social cost on trees~\cite{FilimonovM22}.
%\nick{What are approximation ratios for trees?}
\end{description}
In fact, a very recent line of work on mechanism design with predictions heavily relies on the $\sqrt{d}$-approximation guarantee of the median rule, either by using Meir's result in a black box manner~\cite{barak2024mac}, or significantly limiting their attention only to $\lqnorm[2](\reals^2)$ space~\cite{AgrawalBGTX22,BalkanskiGS24,ChristodoulouSV24}. It also appears that the consensus in the community about 
approximation ratio of the coordinate-wise median mechanism is on $\sqrt{d}$ side, and that we are simply missing examples with matching lower bounds. 

%
%
%
%
%
%
%
%
%
%\nick{Plan}
%\begin{enumerate}
%    \item Strategic facility location as canonical setting of mechanism design without money
%    \begin{itemize}
%        \item Intersection of Social choice theory, mechanism design
%        \item Central setting to circumvent Arrow and Gibbard–Satterthwaite impossibility results
%    \end{itemize}
%    \item Mendian and generalized median mechanism is the most well studied solution.
%    \item Known results, from the survey on facility location in the more recent approximate mechanism design literature
%    \begin{itemize}
%        \item
%        \item 
%        \item A large gap in the literature: approximation guarantee of the median mechanism in $d$-dimensional spaces,  where it has the most straightforward implementation. Note that other metric spaces, as simple as a circle, do not have any deterministic strategy-proof mechanisms with a constant. approximation guarantees. 
%    \end{itemize}
%    \item Our question: what is the approximation ratio of median mechanism in $\lqnorm(\reals^d)$ for any dimension $d\in\nats$ and any $q\in\reals_{\ge 1}\cup\{\infty\}$.  Our results:
%    \begin{itemize}
%        \item We obtain constant upper bound for any $d$ and $q$. Improves previous upper bound of $\sqrt{d}$ for the Euclidean distance, and also gives constant upper bounds for all $\lqnorm$ norms.
%        \item Our analysis is tight, as we demonstrate with a series of examples that give $\LB(q,d)$, which approach our $\UB(q)$ at the rate $O(\frac{1}{d})$. For the case of $d=2$, the difference in approximation ratio is $\UB(2)=1.55$ for arbitrary dimension $d$ versus $\sqrt{2}=1.41$ for $\lqnorm(\reals^2)$ is already quite small.
%        \item Immediate Application to the recent line of work on mechanism design with predictions. List recent papers that rely on approximation results of the median mechanism in $d=2$. We also extend our analysis to generalized median mechanism of EC22 paper to any dimension $d$.
%    \end{itemize}
%\end{enumerate}
%
%% %Strategical facility location -- canonical setting
%% Strategic facility location is the canonical problem in the literature on mechanism design without money~\cite{procaccia2018approximation} with a range of applications including incentive compatible learning~\cite{DekelFP10,MeirPR12}. 
%% The goal in this problem is to find a good location for the new facility given the preferences of $n$ strategic agents. 
%% This problem has been the focus of a long 
%% line of work in mechanism design literature~\cite{procaccia2013approximate, FotakisT14, FotakisT16, SerafinoV16, walsh2020strategy, GkatzelisKST22} as well as a even earlier body of work in social choice~\cite{moulin1980strategy,border1983straightforward,kim1984nonmanipulability,peters1993range,barbera1993generalized,ching1997strategy,peremans1997strategy,barbera1998strategy,schummer2002strategy}. In the former case the setting often served as an important domain for testing new concepts in approximate mechanism design, such as truthful mechanisms without monetary transfers~\cite{procaccia2013approximate,procaccia2018approximation} as well as mechanism design with predictions~\cite{GkatzelisKST22,XuL22,barak2024mac}, while in the later case it has been the primary domain with structured preferences that allowed one to escape strong impossibility results such as the Gibbard-Satterthwaite Theorem~\cite{Gibbard77,Satterthwaite75}.
%%
%%
%%
%%\nick{OLD INTRO, to be rewritten:}
%%
%%%%% Median mechanism is central for strategic facility location 
%%The coordinate-wise median (often called median-point or simply median) mechanism is the principal approach for handling strategic behavior in facility location problem. The mechanism solicits reported locations of $n$ agents residing in $d$-dimensional space $\reals^d$ and opens a facility at the point $(m_1,\ldots,m_d)\in\reals^d$, where each $m_j$ is the median of the $j$-th coordinates for all $n$ reports. 
%%This mechanism dates back to as early as mid-$20$th century~\cite{Black48}. 
%%The (generalized) median is the only deterministic strategy-proof mechanism for agents residing on a continuous or discrete line~\cite{DokowFMN12,schummer2002strategy}, and it is optimal in terms of utilitarian social cost objective for agents with single-peaked preferences~\cite{moulin1980strategy, procaccia2013approximate}. In fact, it is group strategy-proof in one-dimensional setting \cite{chan2021mechanism}, and in $d=2$ dimensions, the coordinate-wise median is the only deterministic strategy-proof mechanism that is Pareto optimal and anonymous \cite{peters1993range}. 
%%
%%%%%%% Goal: approximation guarantees (overview relevant literature)
%%%% Studies are mostly limited to lower dimensions. %%
%%Facility location has become a primary domain of study in \emph{approximate} mechanism design, since the seminal work~\cite{procaccia2013approximate}, which coined the term of \emph{approximate} mechanism design without money
%%and promoted facility location as a test bed for exploring new ideas and techniques in mechanism design and non-cooperative multi-agent systems~\cite{CaragiannisKM10,Feldman16,MeirPR12}. 
%%In dimension $d=1$, \cite{FeigenbaumSY17} showed that the approximation guarantee (for the most commonly used utilitarian social cost objective) of the median mechanism 
%%is exactly $2^{1-1/q}$ for each $L_q(\reals^d)$ norm with $q\in [1,+\infty]$.
%%In dimension $d=2$, the tight approximation ratio of the median-point mechanism is $\sqrt{2}$ for the Euclidean norm $L_2(\reals^2)$, which extends to the upper bound of $\sqrt{d}$ -- the metric distortion of embedding $L_2(\reals^d)$ into $L_1(\reals^d)$~\cite{meir2019strategyproof}.
%%This approximation guarantee in $L_2(\reals^2)$ was made more precise by~\cite{DBLP:journals/scw/GoelH23} with the tight dependency on the number of agents $n$.
%%Rather surprisingly, no matching lower bounds of $\Omega(\sqrt{d})$ are known for dimensions higher than $d>2$. Furthermore, to the best of our knowledge, all recent studies on approximate mechanism design that relied on coordinate-wise median mechanism have been largely limited to low dimensional spaces $L_2(\reals^d)$ with $d\in\{1,2\}$. E.g., in a recent line of work on strategic facility location with ML predictions~\cite{christodoulou2024mechanismdesignaugmentedoutput,fotakis2021learning} only consider the case $d=2$, and \cite{barak2024mac} use Meir's $\sqrt{d}$ upper bound as the worst-case guarantee.
%%%
%%%
%%%%% Our results
%%In this note, we study the approximation guarantees of the coordinate-wise median mechanism in arbitrary $d$-dimensional space and for all possible $\lqnorm(\reals^d)$ norms.
%%We provide a series of upper bounds $\UB(q)$ on the approximation factor independent of the dimension $d$ and number of agents $n$, which is an increasing sequence of numbers (e.g., $\UB(2)=\sqrt{6\sqrt{3}-8}\approx 1.55$, $\UB(3)\approx 1.84$ and $\UB(10)\approx 2.48$) approaching $\UB(q)\to 3$ as $q\to +\infty$. 
%%We supplement these upper bounds with almost matching (our upper bounds are tight for $d\to +\infty$) lower bounds $\LB(q,d)=\UB(q)\cdot(1-O(1/d))$. 
%
