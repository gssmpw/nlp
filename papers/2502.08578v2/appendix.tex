\subsection{Proof of Lemma~\ref{lm:expression for local optimum}}
\begin{proof}
For $S(\loc)\neq\emptyset$, according to Lemma~\ref{lm:local optimal condition}, we get $p_j=f_j/(1-c)$ for $j\in S$. Hence, $\qnorm{\loc-\facility}$ and $\qnorm{\loc}$ can be expressed as functions of $c$:
\begin{equation*}
\begin{split}
&\qnorm{\loc-\facility}=\left(\sum_{j\in S}(p_j-f_j)^q+\sum_{j\in \overline{S}}f_j^q\right)^{1/q}=\left(\left(\frac{c}{1-c}\right)^q\energy+\energyc\right)^{1/q}\\
&\qnorm{\loc}=\left(\sum_{j\in S}p_j^q\right)^{1/q}=\frac{1}{1-c}\cdot\energy^{1/q}.
\end{split}
\end{equation*}
Recall that $c=\lambda^{1/(q-1)}\cdot\qnorm{\loc-\facility}/\qnorm{\loc}$, we now derive the following equations that allow us to eliminate dependency on $c$ of $g(\loc)$:
\begin{align*}
&c=\lambda^{1/(q-1)}\cdot\frac{\left((\frac{c}{1-c})^q\cdot\energy+\energyc\right)^{1/q}}{\frac{1}{1-c}\cdot\esq}=\lambda^{1/(q-1)}\cdot\left(c^q+(1-c)^q\cdot\frac{\energyc}{\energy}\right)^{1/q}\\
&\Leftrightarrow c^q=\lambda^{q/(q-1)}\cdot\left(c^q+(1-c)^q\cdot\frac{\energyc}{\energy}\right)\\
&\Leftrightarrow c^q\cdot\left(1-\lambda^{q/(q-1)}\right)=\lambda^{q/(q-1)}\cdot(1-c)^q\cdot\frac{\energyc}{\energy}\\
&\Leftrightarrow\frac{c}{1-c}=\left(\frac{\energyc}{\energy}\right)^{1/q}\cdot\left(\frac{\lambda^{q/(q-1)}}{1-\lambda^{q/(q-1)}}\right)^{1/q}\\
&\Leftrightarrow\frac{1}{1-c}=\left(\frac{\energyc}{\energy}\right)^{1/q}\cdot\left(\frac{\lambda^{q/(q-1)}}{1-\lambda^{q/(q-1)}}\right)^{1/q}+1.
\end{align*}
Therefore,
\begin{align*}
g(\loc)&=\qnorm{\facility-\loc}-\lambda\cdot\qnorm{\loc}\\
&=\left(\left(\frac{c}{1-c}\right)^q\cdot\energy+\energyc\right)^{1/q}-\frac{1}{1-c}\cdot\energy^{1/q}\\
&=\left(\frac{\lambda^{q/(q-1)}}{1-\lambda^{q/(q-1)}}\cdot\energyc+\energyc\right)^{1/q}-\lambda\cdot\left(\left(\frac{\energyc}{\energy}\right)^{1/q}\cdot\left(\frac{\lambda^{q/(q-1)}}{1-\lambda^{q/(q-1)}}\right)^{1/q}+1\right)\cdot \energy^{1/q}\\
&=\left(\frac{1}{1-\lambda^{q/(q-1)}}\cdot\energyc\right)^{1/q}-\lambda^{q/(q-1)}\cdot \energyc^{1/q}\cdot\left(\frac{1}{1-\lambda^{q/(q-1)}}\right)^{1/q}-\lambda\cdot\energy^{1/q}\\
&=\left(1-\lambda^{q/(q-1)}\right)^{(q-1)/q}\cdot\ecsq-\lambda\cdot\esq.
\end{align*}
For $S(\loc)=\emptyset$, we first consider the situation when $T=\emptyset$. In this case, $\loc=\zerovec$, which is not local optimum. When $T\neq\emptyset$, for a fixed $T$, according to Lemma~\ref{lm:local optimal condition}, $(f_j-p_j)/(-p_j)=c$ for $j\in T$, which is equivalent to $-p_j=f_j/(c-1)$. Hence, $\qnorm{\facility-\loc}$ and $\qnorm{\loc}$ can be expressed as functions of c:
\begin{align*}
&\qnorm{\facility-\loc}=\left(\sum_{j\in T}(f_j-p_j)^q+\sum_{j\in \overline{T}}f_j^q\right)^{1/q}=\left(\left(\frac{c}{c-1}\right)^q\energyT+\energyTc\right)^{1/q}\\
&\qnorm{\loc}=\left(\sum_{j\in T}(-p_j)^q\right)^{1/q}=\frac{1}{c-1}\cdot\energyT^{1/q}.
\end{align*}
Recall that $c=\lambda^{1/(q-1)}\cdot\qnorm{\loc-\facility}/\qnorm{\loc}$, we now derive the following equations that allow us to eliminate dependency on $c$ of $g(\loc)$:
\begin{align*}
&c=\lambda^{1/(q-1)}\cdot\frac{\left((\frac{c}{c-1})^q\cdot\energyT+\energyTc\right)^{1/q}}{\frac{1}{c-1}\cdot\energyT^{1/q}}=\lambda^{1/(q-1)}\cdot\left(c^q+(c-1)^q\cdot\frac{\energyTc}{\energyT}\right)^{1/q}\\
&\Leftrightarrow c^q=\lambda^{q/(q-1)}\cdot\left(c^q+(c-1)^q\cdot\frac{\energyTc}{\energyT}\right)\\
&\Leftrightarrow c^q\cdot\left(1-\lambda^{q/(q-1)}\right)=\lambda^{q/(q-1)}\cdot(c-1)^q\cdot\frac{\energyTc}{\energyT}\\
&\Leftrightarrow\frac{c}{c-1}=\left(\frac{\energyTc}{\energyT}\right)^{1/q}\cdot\left(\frac{\lambda^{q/(q-1)}}{1-\lambda^{q/(q-1)}}\right)^{1/q}\\
&\Leftrightarrow\frac{1}{c-1}=\left(\frac{\energyTc}{\energyT}\right)^{1/q}\cdot\left(\frac{\lambda^{q/{q-1}}}{1-\lambda^{q/(q-1)}}\right)^{1/q}-1.
\end{align*}
Therefore,
\begin{align*}
g(\loc)&=\qnorm{\facility-\loc}-\lambda\cdot\qnorm{\loc}\\
&=\left(\left(\frac{c}{1-c}\right)^q\cdot\energyT+\energyTc\right)^{1/q}-\frac{1}{1-c}\cdot\energyT^{1/q}\\
&=\left(\frac{\lambda^{q/(q-1)}}{1-\lambda^{q/(q-1)}}\cdot\energyTc+\energyTc\right)^{1/q}-\lambda\cdot\left(\left(\frac{\energyTc}{\energyT}\right)^{1/q}\cdot\left(\frac{\lambda^{q/(q-1)}}{1-\lambda^{q/(q-1)}}\right)^{1/q}-1\right)\cdot \energyT^{1/q}\\
&=\left(\frac{1}{1-\lambda^{q/(q-1)}}\cdot\energyTc\right)^{1/q}-\lambda^{q/(q-1)}\cdot \energyTc^{1/q}\cdot\left(\frac{1}{1-\lambda^{q/(q-1)}}\right)^{1/q}+\lambda\cdot\energyT^{1/q}\\
&=\left(1-\lambda^{q/(q-1)}\right)^{(q-1)/q}\cdot\ectq+\lambda\cdot\etq.
\end{align*}
Then, we minimize over all possible $T$ to solve for $g(\loc)$ under $T\neq\emptyset$:
\begin{align*}
g(\loc)=\min\limits_{\substack{T\subseteq[d]\\T\neq\emptyset}}\left[\left(1-\lambda^{q/(q-1)}\right)^{(q-1)/q}\cdot\ectq+\lambda\cdot\etq\right].
\end{align*}
We view $\left(1-\lambda^{q/(q-1)}\right)^{(q-1)/q}\cdot\ectq+\lambda\cdot\etq$ as a function of $\energyT$ and, to simplify notation denote $x\eqdef\energyT$. First, we figure out the domain of $x$. Since $c>1$ from Lemma~\ref{lm:local optimal condition}, we have:
\begin{align*}
    \left(\frac{\energyTc}{\energyT}\right)^{1/q}\cdot\left(\frac{\lambda^{q/(q-1)}}{1-\lambda^{q/(q-1)}}\right)^{1/q}=\frac{c}{c-1}>1
\Leftrightarrow\quad \frac{\energyTc}{\energyT}\cdot\frac{\lambda^{q/(q-1)}}{1-\lambda^{q/(q-1)}}>1
\quad\Leftrightarrow\quad \energyT<\lambda^{q/q-1}.
\end{align*}
Together with $\energyT>0$, we have $x=\energyT\in(0,\lambda^{q/q-1})$. Then let's consider function
\begin{align*}
    t(x)=\left(1-\lambda^{q/(q-1)}\right)^{(q-1)/q}\cdot(1-x)^{1/q}+\lambda\cdot x^{1/q}.
\end{align*}
Since both $(1-x)^{1/q}$ and $x^{1/q}$ are concave in $x$, $t(x)$ (as a non-negative weighted sum of $(1-x)^{1/q}$ and $x^{1/q}$) is also a concave function of $x$. Hence, its minimum value is achieved either when $x$ goes to $0$, or when $x$ goes to $\lambda^{q/q-1}$. As
\begin{align*}
    t(x)\left\vert_{x\rightarrow 0}=1-\lambda^{q/(q-1)}<t(x)\right\vert_{x\rightarrow \lambda^{q/q-1}}=1, 
\end{align*}
we conclude that $g(\loc)\geq1-\lambda^{q/(q-1)}$ when $T\neq\emptyset$.
\end{proof}

\subsection{Proof of Lemma~\ref{lm:optimal solution for a}}
\begin{proof}
The optimal condition for $u(a)$ is that either $a^*\in\{0,z\}$ or $u'(a)=0$. Specifically,
\begin{align*}
   u'(a)=\frac{1}{q}\cdot\left(-\delta\cdot(1-a)^{(1-q)/q}-a^{(1-q)/q}\right)+2.
\end{align*}
First, notice that $u'(a)\rightarrow -\infty$ when $a\rightarrow 0$, saying $a=0$ is not optimal.
Then, we observe that $u'(a)=0\Leftrightarrow w(a)\eqdef\frac{1}{q}\cdot\left(-\delta\cdot(1-a)^{(1-q)/q}-a^{(1-q)/q}\right)+2=0$. We prove that there exists at most one solution to $u'(a)=0$ by showing that $w(a)$ is strictly increasing in $a$ on $a\in(0,z)$:
\begin{align*}
  w'(a)&=\frac{1-q}{q^2}\cdot(1-a)\cdot\left(\delta\cdot(1-a)^{(1-2q)/q}-a^{(1-2q)/q}\right) \\
  &= \frac{q-1}{q^2}\cdot(1-a)\cdot\left(-\delta\cdot(1-a)^{(1-2q)/q}+a^{(1-2q)/q}\right)> 0.
\end{align*}
The last inequality follows from the fact that $-\delta\cdot(1-x)^{(1-2q)/q}+x^{(1-2q)/q}$ is convex in $x$ when $x\in(0,z)$.
On the other hand, we prove that there exists one unique solution to $u'(a)=0$ on $a\in(0,z)$ by contradiction. We assume that there is no solution for $u'(a)=0$ on $a\in(0,z)$. I.e., $u'(z)<0$ and therefore $a^*=z$. Now, since we have $u(a^*)=0$, we get the following system of inequalities on $\delta$ and $z$.
\begin{equation}
\label{eq:z}
\left\{
    \begin{aligned}
    &\delta\cdot(1-z)^{1/q}-{z}^{1/q}-1+2\cdot z=0\\
    &\frac{1}{q}\cdot\left(-\delta\cdot(1-z)^{(1-q)/q}-z^{(1-q)/q}\right)+2<0.
    \end{aligned}
    \right.
\end{equation}
We multiply $(1-z)$ to both side of the second inequality of \eqref{eq:z}, which gives us
\begin{equation}
\label{eq:inequality of z}
    \frac{1}{q}\cdot\left(-\delta\cdot(1-z)^{1/q}+z^{1/q}-z^{(1-q)/q}\right)+2-2z<0.
\end{equation}
By adding the first equation in \eqref{eq:z} to both side of \eqref{eq:inequality of z}, we get
\begin{equation}
\label{eq:inequality of z(2)}
    \left(1-\frac{1}{q}\right)\cdot\left(\delta\cdot(1-z)^{1/q}-z^{1/q}\right)-\frac{1}{q}\cdot z^{(1-q)/q}+1<0.
\end{equation}
However,
\begin{align*}
   &\left(1-\frac{1}{q}\right)\cdot\left(\delta\cdot(1-z)^{1/q}-z^{1/q}\right)-\frac{1}{q}\cdot z^{(1-q)/q}+1\\
   &=\left(1-\frac{1}{q}\right)\cdot\left(z^{(1-2q)/q}\cdot(1-z)^2-z^{1/q}\right)-\frac{1}{q}\cdot z^{(1-q)/q}+1\\
   &=z^{(1-2q)/q}\cdot\left(\left(1-\frac{1}{q}\right)\cdot\left((1-z)^2-z^2\right)-\frac{1}{q}\cdot z\right)+1\\
   &=z^{(1-2q)/q}\cdot\left(1-2z-(1-z)\cdot\frac{1}{q}\right)+1\\
   % &\geq \frac{n}{2(1-z)^2}\cdot\lambda\cdot\left(z^{(1-2q)/q}\cdot\frac{1}{2}\left(1-3z\right)+1\right)\\
   % &\geq \frac{n}{2(1-z)^2}\cdot\lambda\cdot\left(z^{(1-2q)/q}\cdot\frac{1}{2}\left(-z\right)+1\right)\\
    &\ge -z^{(1-q)/q}\cdot\frac{1}{2}+1\\
   &\ge -\frac{1}{2z}+1\\
   &\geq 0.
\end{align*}
The equation in the second line follows from the fact that $\delta\cdot(1-z)^{(1-2q)/q}=z^{(1-2q)/q}$. The inequalities in the 5-th, 6-th, and 7-th lines hold, since $q\ge 1$ and $z\leq1/2$. Hence, there exists exactly one point with $u'(a)=0$ when $a\in [0,z]$. 
\end{proof}

\subsection{Proof of Lemma~\ref{lm:consistency optimal solution for a_1}}
\begin{proof}
The optimal condition for $u_1(a)$ is that $$a_1\in\{0,z,
\frac{1-c}{2}\}\ or~ u_1'(a_1)=0.$$ Specifically,
\begin{align*}
   u_1'(a)=\frac{1}{2}\cdot(1+c)\cdot\left(-\delta\cdot(1-a)^{-1/2}-a^{-1/2}\right)+2.
\end{align*}
First, notice that $u_1'(a)\rightarrow -\infty$ when $a\rightarrow 0$, showing that $a=0$ is not optimal.

Then, we observe that $u_1'(a)=0\Leftrightarrow w_1(a)\eqdef\frac{1}{2}\cdot(1+c)\cdot\left(-\delta\cdot(1-a)^{-1/2}-a^{-1/2}\right)+2=0$. We prove that there exists at most one solution to $u_1'(a)=0$ by showing that $w_1(a)$ is strictly increasing in $a$ on $a\in(0,z)$:
\begin{align*}
  w_1'(a)&=-\frac{1}{4}\cdot(1+c)\cdot(1-a)\cdot\left(\delta\cdot(1-a)^{-3/2}-a^{-3/2}\right) \\
  &= \frac{1}{4}\cdot(1+c)\cdot(1-a)\cdot\left(-\delta\cdot(1-a)^{-3/2}+a^{-3/2}\right)> 0.
\end{align*}
The last inequality follows from the fact that $-\delta\cdot(1-x)^{-3/2}+x^{-3/2}$ is convex in $x$ when $x\in(0,z)$.
On the other hand, we prove that the optimal solution can never be $a_1=z$ by contradiction. 

Assume that $a_1=z$, which implies that $u_1'(a)<0$ on $a\in[0,z)$. Since we have $u_1(a_1)=0$, we can get the following system of inequalities on $\delta$ and $z$
\begin{equation}
\label{eq:z for consistency}
\left\{
    \begin{aligned}
    &(1+c)\cdot(\delta\cdot(1-z)^{1/2}-{z}^{1/q})-1+2\cdot z+c=0\\
    &\frac{1}{2}\cdot(1+c)\cdot\left(-\delta\cdot(1-z)^{-1/2}-z^{-1/2}\right)+2<0.
    \end{aligned}
    \right.
\end{equation}
We multiply $(1-z)$ to both side of the second inequality of \eqref{eq:z for consistency}, which gives us
\begin{equation}
\label{eq:consistency inequality of z}
    \frac{1}{2}\cdot(1+c)\cdot\left(-\delta\cdot(1-z)^{1/2}+z^{1/2}-z^{-1/2}\right)+2-2z<0.
\end{equation}
By adding the first equation in \eqref{eq:z for consistency} to both side of \eqref{eq:consistency inequality of z}, we get
\begin{equation}
\label{eq:consistency inequality of z(2)}
    (1+c)\cdot\left(\frac{1}{2}\cdot\left(\delta\cdot(1-z)^{1/2}-z^{1/2}\right)-\frac{1}{2}\cdot z^{-1/2}+1\right)<0.
\end{equation}
However,
\begin{align*}
   &\frac{1}{2}\cdot\left(\delta\cdot(1-z)^{1/2}-z^{1/2}\right)-\frac{1}{2}\cdot z^{-1/2}+1\\
   =&\frac{1}{2}\cdot\left(z^{-3/2}\cdot(1-z)^2-z^{1/2}\right)-\frac{1}{2}\cdot z^{-1/2}+1\\
   =&z^{-3/2}\cdot\left(\frac{1}{2}\cdot\left((1-z)^2-z^2\right)-\frac{1}{2}\cdot z\right)+1\\
   =&z^{-3/2}\cdot\left(1-2z-(1-z)\cdot\frac{1}{2}\right)+1\\
   % &\geq \frac{n}{2(1-z)^2}\cdot\lambda\cdot\left(z^{(1-2q)/q}\cdot\frac{1}{2}\left(1-3z\right)+1\right)\\
   % &\geq \frac{n}{2(1-z)^2}\cdot\lambda\cdot\left(z^{(1-2q)/q}\cdot\frac{1}{2}\left(-z\right)+1\right)\\
   \ge& -z^{-1/2}\cdot\frac{1}{2}+1\\
   \ge& -\frac{1}{2z}+1\\
   \geq& 0.
\end{align*}
The equation in the second line follows from the fact that $\delta\cdot(1-z)^{-3/2}=z^{-3/2}$. The inequalities in the 5-th, 6-th, and 7-th lines hold, since $q\ge 1$ and $z\leq1/2$. Hence, we rule out the possibility of $a_1\in\{0,z\}$. I.e., either there is a unique $a_1$ on $a\in\left[0,min\{z,\frac{1-c}{2}\}\right]$ such that $u_1'(a)=0$ when $a=a_1$, or $a_1=\frac{1-c}{2}$.
\end{proof}

\subsection{Missing Proof of Theorem~\ref{thm:consistency_guarantee}}
\begin{proof}
First, we compute for $a'_1$ such that $u_1'(a'_1)=0$ and $u_1(a'_1)=0$. The corresponding $a'_1$ is the optimal solution (i.e. $a_1=a'_1$) if $a'_1\leq \frac{1-c}{2}$, or otherwise, $a_1=\frac{1-c}{2}$. The equations for $u_1'(a'_1)=0$ and $u_1(a'_1)=0$ are
\begin{equation}
\label{eq:equations for a'_1}
\left\{
    \begin{aligned}
    &\left(1+c\right)\cdot\left(\delta_1\cdot(1-a'_1)^{1/2}-{(a'_1)}^{1/2}\right)-1+2\cdot a'_1+c=0\\
    &\frac{1}{2}\cdot\left(1+c\right)\cdot\left(\delta_1\cdot(1-a'_1)^{-1/2}-(a'_1)^{-1/2}\right)+2=0.
    \end{aligned}
    \right.
\end{equation}
We multiply $2(a'_1-1)$ on both side of the second equation in~\eqref{eq:equations for a'_1} and get
\begin{equation}
\label{eq:second equation for a'_1}
    \left(1+c\right)\cdot\left(\delta_1\cdot(1-a'_1)^{1/2}-(a'_1)^{1/2}+(a'_1)^{-1/2}\right)+4(a'_1-1)=0.
\end{equation}
The first equation gives us $(1+c)\cdot(\delta_1\cdot(1-a'_1)^{1/2}-(a'_1)^{1/2})=1-2\cdot a'_1+c$. By plugging it in equation~\eqref{eq:second equation for a'_1}, we get
\begin{equation}
\label{eq:third equation for a'_1}
    2\cdot a'_1+(1+c)\cdot (a'_1)^{-1/2}-3-c=0.
\end{equation}
Let $t_1=(a'_1)^{1/2}$, we can derive
\begin{align*}
&2t_1^2+(1+c)/ t_1-3-c=0\\
\Leftrightarrow& 2t_1^3-(3+c)\cdot t_1 +(1+c)=0\\
\Leftrightarrow& (t_1-1)\cdot\left(2t_1^2+2t_1-(1+c)\right)=0\\
\Leftrightarrow& t_1 = \frac{-1+\sqrt{3+2c}}{2}.
\end{align*}
Therefore, $a'_1=t_1^2=(2+c-\sqrt{3+2c})/2$. We compare $a'_1$ with $\frac{1-c}{2}$, which gives us $a_1=a'_1=t_1^2=(2+c-\sqrt{3+2c})/2$ when $c\in[0,1/2)$ and $a_1=(1-c)/2$ when $c\in[1/2,1)$. Next, by plugging the value of $a_1$ into $u_1(a_1)=0$, we can obtain the value of $\delta_1$ and $\lambda_1$. 

When $c\in[0,1/2)$, we have
\begin{align*}
  \delta_1^2&=\left(\frac{1-2a_1-c}{1+c}+a_1^{1/2}\right)^2/(1-a_1)\\
  &=\frac{1}{(1+c)^2}\cdot\left(1-2a_1-c+a_1^{1/2}\cdot(1+c)\right)^2/(1-a_1) \\
  &=\frac{1}{(1+c)^2}\cdot\left(\frac{-3-5c+(3+c)\cdot\sqrt{3+2c}}{2}\right)^2/\left(\frac{-c+\sqrt{3+2c}}{2}\right) \\
  &=\frac{2c^3+40c^2+66c+36-(18+36c+10c^2)\cdot\sqrt{3+2c}}{-2c\cdot(1+c)^2+2(1+c)^2\cdot\sqrt{3+2c}} \\
  &=\frac{c^3+20c^2+33c+18-(9+18c+5c^2)\cdot\sqrt{3+2c}}{-c\cdot(1+c)^2+(1+c)^2\cdot\sqrt{3+2c}}.
\end{align*}
Therefore,
\begin{align*}
  \lambda_1&=(1+\delta_1^2)^{-1/2}\\
  &=\left(1+\frac{c^3+20c^2+33c+18-(9+18c+5c^2)\cdot\sqrt{3+2c}}{-c\cdot(1+c)^2+(1+c)^2\cdot\sqrt{3+2c}}\right)^{-1/2}\\
  &=\left(\frac{18c^2+32c+18-(4c^2+16c+8)\cdot\sqrt{3+2c}}{-c\cdot(c+1)^2+(1+c)^2\cdot\sqrt{3+2c}}\right)^{-1/2}\\
  &=(c+1)\cdot\left(4\sqrt{2c+3}\cdot c+6\sqrt{2c+3}-10c-8\right)^{-1/2}.
\end{align*}

When $c\in[1/2,1)$, we have
\begin{align*}
  \delta_1^2&=\left(\frac{1-2a_1-c}{1+c}+a_1^{1/2}\right)^2/(1-a_1)
  =\frac{a_1}{1-a_1}
  =\frac{1-c}{1+c},
\end{align*}
and
\begin{align*}
  \lambda_1=\left(1+\delta_1^2\right)^{-1/2}=\sqrt{\frac{c+1}{2}}.
\end{align*}
\end{proof}

\subsection{Proof of Theorem~\ref{thm:robustness_guarantee}}
\begin{proof}
%In the robustness scenario, we first need to figure out $\pred$ that achieves worst approximation ratio for our mechanism. For a given $\pred$, to locate $\cmp$ at $\mathbf{0}$ is equivalent to having the constraint that $\sum_{i\in[n]}\sigma(\loc_i)=-cn\cdot\sigma(\pred)$. Since only the signature of $\pred$ affects the outcome of our mechanism, the optimization problem for robustness guarantee can thus be stated as follows.


We follow the same reasoning as in the proof of Theorem~\ref{thm:consistency_guarantee}. First, we apply Lemma~\ref{lm:expression for local optimum} to express $g(\loci)$ as a function of $x_i$ (defined in~\eqref{eq:relaxed main}) for each $\loci\in\locs$. 
Then, we modify the constraint of~\eqref{eq:main} into
\begin{align}
    \label{eq: robustness relaxed_median_constraint}
    \sum\limits_{i\in[n]}x_i=\sum\limits_{i\in[n]}\energy[S(\loci)]=
    \sum_{i\in[n]}\sum_{j:p_{i,j}\ge 0^+}f_j^2
    =\sum_{j\in[d]}f_j^2\cdot\sum_{i:p_{i,j}\ge 0^+}1
    =\frac{n}{2}\cdot\sum_{j\in[d]}f_j^2=\frac{1+c}{2}\cdot n.
\end{align}
Thus, we get the following analog of~\eqref{eq:relaxed main} for q=2. 
\begin{align}
 \label{eq:robustness relaxed main}
 \min\limits_{\vx}& \sum_{i\in[n]} h(x_i)
 &\text{where }h(x_i)
 \eqdef \lambda\cdot\left(\delta\cdot(1-x_i)^{1/2}- x_i^{1/2}\right)\nonumber\\
 \text{s.t.}&  \sum_{i\in [n]}x_i = \frac{1+c}{2}\cdot n,&
\quad\quad \forall i\in[n]~x_i\in[0,1]
\end{align}
Again, we let $z\eqdef \delta^{-2/3}/(\delta^{-2/3}+1)<1/2$ for $q=2$ and by Lemma~\ref{optimal value of x_i} and the same argument for the median mechanism, the optimal value of $x_i$ lies in $\{ a,1\}$, where $a\in[0,z]$ and the space of optimal solution can be narrowed down to 2 parameters $a$ and $|\{i: x_i=a\}|$.

Using the constraint $\sum_{i=1}^n x_i = (1+c)\cdot n/2$, the number $|\{i: x_i=a\}|$ and $|\{i: x_i=1\}|$ must be, respectively, $\frac{1-c}{2(1-a)}\cdot n$ and $\frac{1-2a+c}{2(1-a)}\cdot n$. The number $|\{i: x_i=1\}|\geq 0$ implies that $1-2a+c\geq 0$. Therefore, we need to find $\lambda$ such that
\begin{multline}
    \label{eq: robustness optimization over a}   \min_{a\in\left[0,\min\{z,\frac{1+c}{2}\}\right]}\left[\frac{1-c}{2(1-a)}\cdot n\cdot h(a) + \frac{1-2a+c}{2(1-a)}\cdot n\cdot h(1)\right]\\
    =\min_{a\in\left[0,\min\{z,\frac{1+c}{2}\}\right]}\left[\frac{n\cdot\lambda}{2(1-a)}\cdot\left((1-c)\cdot\left(\delta\cdot(1-a)^{1/q}-a^{1/q}\right)-1+2\cdot a-c\right)\right]\ge 0.
\end{multline}
Thus, we need to minimize $$u_2(a)\eqdef(1-c)\cdot\left(\delta\cdot(1-a)^{1/q}-a^{1/q}\right)-1+2\cdot a-c.$$
Let $a_2\eqdef \argmin\limits_{a\in\left[0,\min\{z,\frac{1+c}{2}\}\right]} u_2(a)$. We observe that
\begin{lemma}
    \label{lm:robustness optimal solution for a_2}
    If $u_2(a_2)=0$, then $a_2$ is optimal $\Rightarrow u_2'(a_2)=0\ or\ a_2=\frac{1+c}{2}$.
\end{lemma}
The proof of Lemma~\ref{lm:robustness optimal solution for a_2} is essentially the same as the proof of Lemma~\ref{lm:consistency optimal solution for a_1} (we simply replace $c$ with $-c$ in the proof of Lemma~\ref{lm:consistency optimal solution for a_1}).

Lastly, we need to find $\lambda_2$ (analog of $\lambda^*$ from the proof of Theorem~\ref{thm:UB}) such that the minimum of \eqref{eq:robustness relaxed main} and respectively \eqref{eq: robustness optimization over a} is equal to $0$, which ensures us an $1/\lambda_2$-robustness guarantee. By Lemma~\ref{lm:robustness optimal solution for a_2}, the optimal $\lambda_2$ and $a_2$ must satisfy (i) $u_2(a_2)=0$ and (ii) $u_2'(a_2)=0$ or $a_2=(1+c)/2$. Then by denoting $\delta_2=\delta(\lambda_2)$, we can solve for $a_2$, $\delta_2$, $\lambda_2$ in the same way as before.

First, we compute for $a'_2$ such that $u_2'(a'_2)=0$ and $u_2(a'_2)=0$. The corresponding $a'_2$ is the optimal solution (i.e. $a_2=a'_2$) if $a'_2\leq \frac{1+c}{2}$, or otherwise, $a_2=\frac{1+c}{2}$. The equations for $u_2'(a'_2)=0$ and $u_2(a'_2)=0$ are
\begin{equation}
\label{eq:equations for a_2}
\left\{
    \begin{aligned}
    &\left(1-c\right)\cdot\left(\delta_2\cdot(1-a'_2)^{1/2}-{(a'_2)}^{1/2}\right)-1+2\cdot a'_2-c=0\\
    &\frac{1}{2}\cdot\left(1-c\right)\cdot\left(\delta_2\cdot(1-a'_2)^{-1/2}-(a'_2)^{-1/2}\right)+2=0.
    \end{aligned}
    \right.
\end{equation}
We multiply $2(a'_2-1)$ on both side of the second equation in~\eqref{eq:equations for a_2} and get
\begin{equation}
\label{eq:second equation for a_2}
    \left(1-c\right)\cdot\left(\delta_2\cdot(1-a'_2)^{1/2}-(a'_2)^{1/2}+(a'_2)^{-1/2}\right)+4(a'_2-1)=0.
\end{equation}
The first equation gives us $(1-c)\cdot(\delta_2\cdot(1-a'_2)^{1/2}-(a'_2)^{1/2})=1-2\cdot a'_2+c$. By plugging it in equation~\eqref{eq:second equation for a_2}, we get
\begin{equation}
\label{eq:third equation for a_2}
    2\cdot a'_2+(1-c)\cdot (a'_2)^{-1/2}-3+c=0
\end{equation}
Let $t_2=(a'_2)^{1/2}$, we get
\begin{align*}
&2t_2^2+(1-c)/t_2-3+c=0\\
\Leftrightarrow& 2t_2^3-(3-c)\cdot t_2 +(1-c)=0\\
\Leftrightarrow& (t_2-1)\cdot\left(2t_2^2+2t_2-(1-c)\right)=0\\
\Leftrightarrow& t_2 = \frac{-1+\sqrt{3-2c}}{2}.
\end{align*}
Therefore, $a'_2=t_2^2=(2-c-\sqrt{3-2c})/2$. Since $a'_2\leq\frac{1+c}{2}$ on $c\in[0,1)$, we get $a_2=a_2'$.

Next, by plugging the value of $a_2$ into $u_2(a_2)=0$, we can obtain the value of $\delta_2$ and $\lambda_2$. 

\begin{align*}
  \delta_2^2&=\left(\frac{1-2a_2+c}{1-c}+a_2^{1/2}\right)^2/(1-a_2)\\
  &=\frac{1}{(1-c)^2}\cdot\left(1-2a_2+c+a_2^{1/2}\cdot(1-c)\right)^2/(1-a_2) \\
  &=\frac{1}{(1-c)^2}\cdot\left(\frac{-3+5c+(3-c)\cdot\sqrt{3-2c}}{2}\right)^2/\left(\frac{c+\sqrt{3-2c}}{2}\right) \\
  &=\frac{-2c^3+40c^2-66c+36-(18-36c+10c^2)\cdot\sqrt{3-2c}}{2c\cdot(1-c)^2+2(1-c)^2\cdot\sqrt{3-2c}} \\
  &=\frac{-c^3+20c^2-33c+18-(9-18c+5c^2)\cdot\sqrt{3-2c}}{c\cdot(1-c)^2+(1-c)^2\cdot\sqrt{3-2c}}.
\end{align*}
Therefore,
\begin{align*}
  \lambda_2&=(1+\delta_2^2)^{-1/2}\\
  &=\left(1+\frac{-c^3+20c^2-33c+18-(9-18c+5c^2)\cdot\sqrt{3-2c}}{c\cdot(1-c)^2+(1-c)^2\cdot\sqrt{3-2c}}\right)^{-1/2}\\
  &=\left(\frac{18c^2-32c+18-(4c^2-16c+8)\cdot\sqrt{3-2c}}{c\cdot(c-1)^2+(1-c)^2\cdot\sqrt{3-2c}}\right)^{-1/2}\\
  &=(1-c)\cdot\left(-4\sqrt{3-2c}\cdot c+6\sqrt{3-2c}+10c-8\right)^{-1/2}.
\end{align*}
\end{proof}

