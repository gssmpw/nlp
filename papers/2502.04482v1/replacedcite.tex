\section{Related Work}
\label{Related_Work}
{Recent efforts} surrounding the investigation and improvement of gig work conditions coalesced around 1) the potential of worker data for making evident the conditions imparted by algorithmic platform practices and 2) the importance of concrete policy and regulation that ensure strong worker protections. 
Below, we describe related works that center our vision {and design} of a worker-centered data-sharing platform that meets {the needs of workers (across platforms)} for self-tracking, mutual aid, and policy advancements that improve current work conditions.


\subsection{{Demands of Gig Work: Shared and Divergent Challenges Across Platforms}}
%US groups them all together 
\label{Related_Work_Challenges}
In the U.S., \final{gig platforms and their workers} are often \final{referred to and regulated collectively}, where gig workers \final{(regardless of platform type)} are typically classified as independent contractors. This has resulted in limited policy or regulatory protections over work conditions, \final{burdening} workers \final{across platforms with confusing and overwhelming logistical} obligations related to self-employment{: navigating tax requirements \cite{taxing, tax, tax_lives, returns} through self-}tracking \final{of} earnings and expenses \cite{accountable, taming}, \final{conducting unpaid labor to find, procure and scope gigs \del{in times of precarity} \cite{youth, apouey2020gig, freelancecontrol}, assuming costs of work-induced injuries in lieu of workers' compensation and health insurance \cite{nilsenhealth, healthdrive, deliverysafe, hsieh2023co}, managing psychological costs to working alone \cite{atom, alienated, commodified}, and so on.} 

%Research has also studied experiences and surfaced harms specific to work context and/or matching algorithm type.
\final{Current research though has focused on highlighting issues specific to work context and platform worker-client matching methodology. For example, to the former,} studies surfaced safety hazards in ride-hailing and food delivery \cite{nilsenhealth, stressfulride, healthdrive, deliverysafe}, wage theft in care work contexts \cite{ming2024wage, cole2024wage, jerseycare}, or irregular schedules in online freelancing \cite{sousveillance, personal, freelancecontrol}. \final{To the latter, researchers have also reported how different platform mechanisms for matching workers to clients/projects impact worker autonomy/precarity.} \final{For example,} on-demand work \final{such as ridesharing and food delivery automatically matches workers and clients, limiting both groups' control in the process \cite{classification} and offloading} logistical overheads to workers \cite{own}. \final{And online marketplace platforms, common in freelance and caregiving contexts, employ rank-based methods and/or bidding mechanisms which skew client control over workers.} Ranking-based methods \final{allow clients to rank and sort candidates by historical performance metrics and relevance to projects \cite{context}}, pressuring workers to maintain \final{meticulous} portfolios \cite{making}. \final{Then, the bidding process where workers apply to postings \cite{personal}} requires additional unpaid labor from workers \final{to write proposals and scope projects without the promise of work} \cite{beyond, personal}.

%why explore multi platforms?
Most of these studies examined issues with respect to a specific platform, thereby limiting insights into whether uncovered stressors generalize to other contexts \final{for strengthened implications of potential policy or regulation or more opportunities for worker community building}. However, in the few cases where multi-platform analyses were conducted, studies revealed how platforms \textit{do} present common higher-level risks (e.g., privacy, financial, psychological, gender biases) \cite{privacy, toward, brush}. \final{Not only that, but} the underlying causes of work challenges are often similar: the aforementioned lack of labor/safety standards and regulation gives way to unbridled worker exploitation through algorithmic management \cite{dubal2023algorithmic, machines, excessive}, gamification tactics and information asymmetries \cite{algorithmic, locus, zhang2022algorithmic}, and an absent collective worker voice that stifles public awareness of harms \cite{ming2024wage, cole2024wage, lastmile}. \final{These similarities in overarching causes of challenges to gig work suggest an opportunity for technology interventions to build solidarity between the currently fragmented and scattered worker communities.} Interventions for unifying gig workers \final{are a compelling exploration} since they often do not self-identify as gig workers, and instead use platform-specific terms (e.g., Uber driver) to self-describe \cite{supporting}, \final{weakening the potential for a collective worker voice to raise public awareness of individual \textit{and} shared harms} \cite{ming2024wage, cole2024wage, lastmile}. 


\subsection{{Work Tracking: Individual Logging $\rightarrow$ Collective Support \& Decision-making}}
In the absence of peer support and higher power actors who assume or share the structural risks and challenges inherent to gigs, workers are left to their own devices to manage {various} accountabilities {and obligations} ____. 
Studies documented two main ways that workers understand and manage work: on their own through self-tracking, or with peers via online groups/forums. 
Recent work at the intersection of HCI and Personal Informatics revealed how gig workers currently (or might in the future) self-track to (1) protect themselves from the platform ____ or customers ____ (2) comply with tax obligations ____ (3) understand how algorithms operate ____ and (4) comprehend and improve their own earning patterns ____ using tools such as data probes {in addition to apps designed for tracking fuel, time, tax, mileage \footnote{Tracking apps include Fuelio and GasBuddy (for Fuel), Traqq (for time), Stride (for tax), as well as MileIQ, Everlance and Triplog (for Mileage)} and generalized gig work assistance ____. For instance, Mystro a commercial tool affording rideshare drivers the agency to auto-decline work across platforms that do not match their expressed preferences (e.g., earning rates, duration of gigs, work locations). 
% Para offer similar functionality for DoorDash that automatically declined low-wage work, but was rendered unfunctional after a DoorDash app update. 
% Third party apps cater to workers seeking targeted strategies to pursue their work preferences---for example, 
Gridwise and Farepilot provide workers data-driven insights about in-demand locations, while Stride assists with tax filing. } 

While such tools act as resource providers and \del{(sometimes automated) }assistants, they {fall short in} provid{ing} workers with social support or strategies in times of need. T\del{hus, t}o overcome \replace{social isolation 
}{the atomized nature gigs ____} and find a sense of   ``community'',  workers also {}leverage online forums (both {pages and groups on general-purpose sites like Reddit/Facebook} and platform-specific sites like uberpeople.net) to share strategies ____ and information ____ {so they can hypothesize and collective make sense of underlying platforms' algorithmic mechanisms}, solicit advice and social connections ____, as well as rant and commiserate ____.  
Online video tutorials (e.g., vlogging) are also emerging as more effortless ways for workers to learn about existing strategies and work conditions ____.
However, the loosely-organized structure of general purpose forums (and video sharing platforms) {makes them in}effective for sensemaking ____, while platform-specific sites limit {worker's abilities to discern unifying challenges shared across domains from characteristics that uniquely afflict workers of a single work context/platform}.
{Furthermore, ``Online forums are built to aid workers with a sense of immediacy, not to quantifiably or qualitatively monitor
request patterns or worker grievances over time'' ____, making them ill-suited for purposes of collective bargaining or identify-building.}
\vspace{-1mm}
\subsection{Gaps in Current Approaches to Gig Worker-Centered Data-Sharing for Policy}\label{Related_Work_Using_Data}
\final{
Recognizing \del{Increasingly, researchers and worker groups turn 
towards }the potential of work data to support workers' sensemaking and auditing of platform algorithms, researchers and worker groups increasingly turn to worker data exchange tools as a means to empower gig workers. Taking a first step towards gig worker-centered data collectives, \citet{stein2023you} deliberated on variants of data collection institutions with drivers, exploring \textit{data leverage} (also covered in \cite{levers}), \textit{governance structures}, and \textit{access control}.
Through lens of care ethics, \citet{sousveillance} used mock-ups to uncover freelancers' needs to relieve emotional strain, find legitimate gigs, and manage invisible labor -- emphasizing in particular how surveillance tools should not generate additional invisible labor for workers. 
To align worker needs with feasible policy changes, Hsieh and Zhang et al.
\cite{supporting} interviewed policy experts and co-designed with workers to understand their (mis)aligned policy priorities, in addition to the necessary supporting data needs. 
Although these studies surfaced key design requirements for worker data-sharing, the lack of a prototype working under realistic working conditions limits the degree to which they can identify and confirm the concrete challenges and practical needs of workers when integrating such hypothesized tools into their daily workflows.}

\final{Stepping beyond codesign, \citet{zhang2023stakeholder} created data probes to help Uber drivers explore and contextualize surfaced patterns with their positionality, well-being, and lived experiences. 
\citet{calacci2022bargaining} partnered with a worker organization to build the Shipt Calculator that audits algorithmic wage determination. Related organizations (e.g., Worker Info Exchange\footnote{\href{https://www.workerinfoexchange.org/}{workerinfoexchange.org/}} and Worker's Algorithm Observatory\footnote{\href{https://wao.cs.princeton.edu/}{wao.cs.princeton.edu/}}) formed to help platform workers collect data and investigate algorithmic decisions. 
\citet{imteyaz2024gigsense} used LLMs and online data (e.g., subreddits, app reviews) to help workers share and collaboratively identify issues, and subsequently generate solutions. 
While these tools leveraged data to support collectivism, few embedded into everyday gig workflows, and yet fewer considered broader, cross-platform gig work experiences.
% the Shipt Calculator solicited workers' pay data via an SMS bot \cite{calacci2022bargaining} while Turkopticon solicited worker-contributed ratings of crowdwork requesters to engage in sousveillance \cite{turkopticon}; 
% follow-up work (Dynamo \cite{dynamo}) directly solicited ideas for action around issues surrounding labor in Mechanical Turk, and subsequently supported workers to form publics and mobilize towards action. 
Working with only quantitative wage data, the Shift Calculator limited understanding of worker struggles to one primary data type. While the seminal work of Turkopticon and Dynamo surfaced diverse labor issues related to policy changes
and helped workers form publics to mobilize/unify towards action, they focused narrowly on online crowd work, whereas challenges afflicting gig workers (especially those offline) diverge significantly. 
}

\final{
Finally, previous work explored how to leverage worker data to advance driver-centered policies. \citet{parrott2018earnings} performed a formative economic analysis of Uber/Lyft app data to investigate working conditions and wages of drivers in New York City, subsequently proposing a minimum wage standard that was adopted by the city. This data-driven approach to assessing driver minimum wage needs was replicated in Seattle \cite{reich2020minimum} and Massachusetts \cite{jacobs2021massachusetts}. Non-profits and other researchers followed suit on smaller scales with worker surveys, due to access restrictions to app data  \cite{Leverage_Dalal_2022, McCullough_Dolber_Scoggins_Muna_Treuhaft_2022, McCullough_Dolber_2021, washington2019delivering}. 
% In follow up work to \cite{zhang2023stakeholder}, 
\citet{zhang2024data} explored how workers' data can support policymakers and policy informers, surfacing its potential to 1) inform policy creation, 2) support lobbying efforts, 3) support worker organization's (member) growth 4) aid regulatory efforts \footnote{One example is a nascent regulatory effort around \href{https://www.ftc.gov/business-guidance/blog/2024/03/price-fixing-algorithm-still-price-fixing}{algorithmic pricing investigations}}.
These efforts consisted of tools, systems or reports centered on quantitative data primarily pertaining to rideshare/delivery driving domains, but the policy changes proposed in such works focused narrowly on wages, missing insights and context on critical issues such as safety and discrimination. Lastly, such tools missed the opportunity to facilitate information exchange and collaboration between worker communities and supporting stakeholder groups, limiting their potential to align workers' community needs \cite[p. 4]{sousveillance} with policy advancements.
}
% Potentially limited by the numerical nature of existing datasets,
\vspace{-1mm}