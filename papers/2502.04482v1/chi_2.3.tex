\subsection{Gaps in Current Approaches to Gig Worker-Centered Data-Sharing for Policy}\label{Related_Work_Using_Data}
\final{
Recognizing \del{Increasingly, researchers and worker groups turn 
towards }the potential of work data to support workers' sensemaking and auditing of platform algorithms, researchers and worker groups increasingly turn to worker data exchange tools as a means to empower gig workers. Taking a first step towards gig worker-centered data collectives, \citet{stein2023you} deliberated on variants of data collection institutions with drivers, exploring \textit{data leverage} (also covered in \cite{levers}), \textit{governance structures}, and \textit{access control}.
Through lens of care ethics, \citet{sousveillance} used mock-ups to uncover freelancers' needs to relieve emotional strain, find legitimate gigs, and manage invisible labor -- emphasizing in particular how surveillance tools should not generate additional invisible labor for workers. 
To align worker needs with feasible policy changes, Hsieh and Zhang et al.
\cite{supporting} interviewed policy experts and co-designed with workers to understand their (mis)aligned policy priorities, in addition to the necessary supporting data needs. 
Although these studies surfaced key design requirements for worker data-sharing, the lack of a prototype working under realistic working conditions limits the degree to which they can identify and confirm the concrete challenges and practical needs of workers when integrating such hypothesized tools into their daily workflows.}

\final{Stepping beyond codesign, \citet{zhang2023stakeholder} created data probes to help Uber drivers explore and contextualize surfaced patterns with their positionality, well-being, and lived experiences. 
\citet{calacci2022bargaining} partnered with a worker organization to build the Shipt Calculator that audits algorithmic wage determination. Related organizations (e.g., Worker Info Exchange\footnote{\href{https://www.workerinfoexchange.org/}{workerinfoexchange.org/}} and Worker's Algorithm Observatory\footnote{\href{https://wao.cs.princeton.edu/}{wao.cs.princeton.edu/}}) formed to help platform workers collect data and investigate algorithmic decisions. 
\citet{imteyaz2024gigsense} used LLMs and online data (e.g., subreddits, app reviews) to help workers share and collaboratively identify issues, and subsequently generate solutions. 
While these tools leveraged data to support collectivism, few embedded into everyday gig workflows, and yet fewer considered broader, cross-platform gig work experiences.
% the Shipt Calculator solicited workers' pay data via an SMS bot \cite{calacci2022bargaining} while Turkopticon solicited worker-contributed ratings of crowdwork requesters to engage in sousveillance \cite{turkopticon}; 
% follow-up work (Dynamo \cite{dynamo}) directly solicited ideas for action around issues surrounding labor in Mechanical Turk, and subsequently supported workers to form publics and mobilize towards action. 
Working with only quantitative wage data, the Shift Calculator limited understanding of worker struggles to one primary data type. While the seminal work of Turkopticon and Dynamo surfaced diverse labor issues related to policy changes
and helped workers form publics to mobilize/unify towards action, they focused narrowly on online crowd work, whereas challenges afflicting gig workers (especially those offline) diverge significantly. 
}

\final{
Finally, previous work explored how to leverage worker data to advance driver-centered policies. \citet{parrott2018earnings} performed a formative economic analysis of Uber/Lyft app data to investigate working conditions and wages of drivers in New York City, subsequently proposing a minimum wage standard that was adopted by the city. This data-driven approach to assessing driver minimum wage needs was replicated in Seattle \cite{reich2020minimum} and Massachusetts \cite{jacobs2021massachusetts}. Non-profits and other researchers followed suit on smaller scales with worker surveys, due to access restrictions to app data  \cite{Leverage_Dalal_2022, McCullough_Dolber_Scoggins_Muna_Treuhaft_2022, McCullough_Dolber_2021, washington2019delivering}. 
% In follow up work to \cite{zhang2023stakeholder}, 
\citet{zhang2024data} explored how workers' data can support policymakers and policy informers, surfacing its potential to 1) inform policy creation, 2) support lobbying efforts, 3) support worker organization's (member) growth 4) aid regulatory efforts \footnote{One example is a nascent regulatory effort around \href{https://www.ftc.gov/business-guidance/blog/2024/03/price-fixing-algorithm-still-price-fixing}{algorithmic pricing investigations}}.
These efforts consisted of tools, systems or reports centered on quantitative data primarily pertaining to rideshare/delivery driving domains, but the policy changes proposed in such works focused narrowly on wages, missing insights and context on critical issues such as safety and discrimination. Lastly, such tools missed the opportunity to facilitate information exchange and collaboration between worker communities and supporting stakeholder groups, limiting their potential to align workers' community needs \cite[p. 4]{sousveillance} with policy advancements.
}
% Potentially limited by the numerical nature of existing datasets,