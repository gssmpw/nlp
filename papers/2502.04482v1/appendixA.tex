\appendix
\section{Appendix A: Mock ups, Protocols \& Features }
\begin{figure*}[h!]  % 'H' forces the figure to be placed here
\subsubsection{Mock-up examples}\label{A.1.1}
The screenshots below display examples of mock-up screens used to gather iterative feedback during pilot testing: Home (a), Story Sharing (b), Work Planner (c) and Collective Insights (d)
    \centering
    \includegraphics[width=\linewidth]{Screenshots/mock-ups.png}
    % \caption{}
    \label{mock-ups}
\end{figure*}
% \FloatBarrier
\subsection{Iterative Pilot Testing}
As overviewed in Section \ref{iterative}, participant feedback from the think-aloud sessions informed iterative refinements of the Gig2Gether prototype. These sessions were designed to evaluate the usability and functionality of essential features, including registration, data uploading, work planning, data review, qualitative input, and data-sharing preferences. Participants were prompted to articulate their thoughts throughout the tasks, enabling the research team to uncover usability issues, user preferences, and areas for improvement. Below is the list of questions and Figma mock-ups used during these sessions.
\subsubsection{Think-aloud questions}\label{thinkaloud}
\begin{itemize}
    \item Do you prefer signing up with a phone number or an email? Why?
    \item How comfortable were you with providing demographic information?
    \item Did any questions cause discomfort?
    \item Is there additional information you would need to feel comfortable completing the process?
    \item Do you think the registration process was too long?

    \item How easy or difficult was it to upload your data?
    \item Which version (mobile or web) did you prefer for this task? Why?
    \item Do you currently track data for your gigs? How do you typically do it?
    \item How does this process fit into your existing data collection habits?
    \item Would you use this app to contribute data in the future? Why or why not?

    \item How easy or difficult was it to use the 'trend' function?
    \item Which version (mobile or web) do you prefer for analyzing data? Why?
    \item Did you find the 'my trend' feature useful?
    \item What additional insights or trends would be helpful for your work?

    \item How easy or difficult was it to plan your work?
    \item Was the summary of earnings and hours useful?
    \item Did the planner feel intuitive to use? Why or why not?
    \item What additional features would help you better plan your work?

    \item How easy or difficult was it to use the 'share' function?
    \item What did you think of the available tags? Are there other tags you would like to see?
    \item Would you use this feature to share your story? Why or why not?
    \item What would you find most helpful to see in other workers' stories?

    \item How easy was it to set your sharing preferences?
    \item What data expiration settings do you prefer?
    \item Do you have any suggestions for improving location granularity settings?
    \item What other controls would you like over your data?
\end{itemize}

\subsection{Field Evaluation}
\subsubsection{Exit Interview questions}\label{exits}
\begin{itemize}
    \item How useful do you find the system? 
    \item Can you describe specific features or aspects that you find particularly helpful or unhelpful?
    \item How likely would you recommend this tool to other friends who are gig workers, if you have any who are workers?
    \item How likely are you to use this app to contribute data in the future? What factors would influence your decision?
    \item Given your experiences contributing so far, how comfortable would you share this information with policymakers and advocates?
    \item Can you reflect on your experiences of uploading your data? What parts of the process were straightforward, and where did you encounter difficulties?
    \item What times of the day did you usually upload? How did this fit into your work schedule?
    \item How does the feed compare to other forums? Do you feel more or less willing to share?
\end{itemize}

\subsection{Details of Gig2Gether Features}
\label{details}
\paragraph{Share Story} \label{share_story}
Each story must 1) be shared as a strategy or issue, 2) be associated with at least one tag, 3) contain story content via a title or textual description, and 4) have a selection of desired viewing audience -- this can include other worker users of the system, policymakers, advocates, or be entirely private (i.e. visible only to themselves). Optionally, workers can include an image or video to provide additional context. See the share story page on the left of Figure \ref{overview}(a).

\paragraph{Story Feed} The story feed provides a place for workers to exchange stories with peers on Gig2Gether. 
At registration time, users are advised to choose a username that will be viewable to other users of the system, and each post is associated with the user only through the username. Posts can be filtered by the story type (Issue or Strategy), as well as by work platform (currently Uber, Rover or Upwork). Gig2Gether allows for cross-{platform} user interaction -- users can currently view and ``like'' posts via thumbs-up buttons. Commenting is currently unsupported, in the absence of an established moderation structure.
The feed is chronologically ordered -- most recent posts appear first; an example can be found via the right side of Figure \ref{overview}(a).

\paragraph{Income} 
For Rover and Upwork users, Gig2Gether currently only supports manual data entry. 
In the income form, a worker can upload information pertaining to time spent, earnings (including the platform cut and tips), as well as information specific to job types, such as time spent travelling to house sits (Rover) and experience levels (Upwork). 
An example for the Rover manual upload is shown at \ref{overview}(d).

Uber users can manually upload data about Trips or upload CSVs that contain platform-collected data about their trips.
The \textit{Trip entry form} gathers information on the time spent, income, distance travelled, Uber fees, surge fees, as well as other specific items detailed in a Trip receipt. 
Finally, Uber allows drivers to download CSV files containing information on lifetime trips, payments and app analytics. Workers have a space to keep track of such information with Gig2Gether, offering a more expedited way of seeing personal work trends.

\paragraph{Expenses} Workers can manually input details about
incurred expenses related to gigs. To add an entry, users must enter the date and cost amount, while fields such as expense type, description and a photo uploads are optional for their own notetaking. Refer to \ref{overview}(d) for the expense upload page for Rover workers.

\paragraph{Personal Trends} 
To stay informed about earning patterns and work hours, workers can overview earnings, expense, hourly earning rates and hours worked in the ``My Trends'' page. Based on income and expense entries that users uploaded (process described in Section \ref{upload}), workers can view hourly and weekly earning trends, daily earnings by month, as well as summary statistics such as hourly pay and worked hours. The design of the hourly and calendar data visualizations in ``My Trends'' were informed by the personal data probes (in particular the Hourly and Calendar probes) from \citet{zhang2023stakeholder}. The Personal Trends page is displayed in \ref{overview}(b).

\paragraph{Collective Trends} In addition to personal metrics, workers can also view aggregate information about other Gig2Gether users via the ``Collective Insights'' page. At the time the study was conducted, this page is populated only with mock data rather than real data that workers inputted to protect the privacy of our small pool of test users. However, the page does include charts and options for dimensions of input (hourly income rate, tipping rate, and ratings) as well as demographic information to breakdown each dimension by (age, gender, ethnicity, income, education, tenure, and part/full-time work). Users can additionally compare their own data point against any breakdowns displayed. Refer to \ref{overview}(e) to view the Collective Insights page.

