%US groups them all together 
\label{Related_Work_Challenges}
In the U.S., \final{gig platforms and their workers} are often \final{referred to and regulated collectively}, where gig workers \final{(regardless of platform type)} are typically classified as independent contractors. This has resulted in limited policy or regulatory protections over work conditions, \final{burdening} workers \final{across platforms with confusing and overwhelming logistical} obligations related to self-employment{: navigating tax requirements \cite{taxing, tax, tax_lives, returns} through self-}tracking \final{of} earnings and expenses \cite{accountable, taming}, \final{conducting unpaid labor to find, procure and scope gigs \del{in times of precarity} \cite{youth, apouey2020gig, freelancecontrol}, assuming costs of work-induced injuries in lieu of workers' compensation and health insurance \cite{nilsenhealth, healthdrive, deliverysafe, hsieh2023co}, managing psychological costs to working alone \cite{atom, alienated, commodified}, and so on.} 

%Research has also studied experiences and surfaced harms specific to work context and/or matching algorithm type.
\final{Current research though has focused on highlighting issues specific to work context and platform worker-client matching methodology. For example, to the former,} studies surfaced safety hazards in ride-hailing and food delivery \cite{nilsenhealth, stressfulride, healthdrive, deliverysafe}, wage theft in care work contexts \cite{ming2024wage, cole2024wage, jerseycare}, or irregular schedules in online freelancing \cite{sousveillance, personal, freelancecontrol}. \final{To the latter, researchers have also reported how different platform mechanisms for matching workers to clients/projects impact worker autonomy/precarity.} \final{For example,} on-demand work \final{such as ridesharing and food delivery automatically matches workers and clients, limiting both groups' control in the process \cite{classification} and offloading} logistical overheads to workers \cite{own}. \final{And online marketplace platforms, common in freelance and caregiving contexts, employ rank-based methods and/or bidding mechanisms which skew client control over workers.} Ranking-based methods \final{allow clients to rank and sort candidates by historical performance metrics and relevance to projects \cite{context}}, pressuring workers to maintain \final{meticulous} portfolios \cite{making}. \final{Then, the bidding process where workers apply to postings \cite{personal}} requires additional unpaid labor from workers \final{to write proposals and scope projects without the promise of work} \cite{beyond, personal}.

%why explore multi platforms?
Most of these studies examined issues with respect to a specific platform, thereby limiting insights into whether uncovered stressors generalize to other contexts \final{for strengthened implications of potential policy or regulation or more opportunities for worker community building}. However, in the few cases where multi-platform analyses were conducted, studies revealed how platforms \textit{do} present common higher-level risks (e.g., privacy, financial, psychological, gender biases) \cite{privacy, toward, brush}. \final{Not only that, but} the underlying causes of work challenges are often similar: the aforementioned lack of labor/safety standards and regulation gives way to unbridled worker exploitation through algorithmic management \cite{dubal2023algorithmic, machines, excessive}, gamification tactics and information asymmetries \cite{algorithmic, locus, zhang2022algorithmic}, and an absent collective worker voice that stifles public awareness of harms \cite{ming2024wage, cole2024wage, lastmile}. \final{These similarities in overarching causes of challenges to gig work suggest an opportunity for technology interventions to build solidarity between the currently fragmented and scattered worker communities.} Interventions for unifying gig workers \final{are a compelling exploration} since they often do not self-identify as gig workers, and instead use platform-specific terms (e.g., Uber driver) to self-describe \cite{supporting}, \final{weakening the potential for a collective worker voice to raise public awareness of individual \textit{and} shared harms} \cite{ming2024wage, cole2024wage, lastmile}. 
