% Template for ISBI paper; to be used with:
%          spconf.sty  - ICASSP/ICIP LaTeX style file, and
%          IEEEbib.bst - IEEE bibliography style file.
% --------------------------------------------------------------------------
\documentclass{article}
\usepackage{spconf,amsmath,graphicx}

% It's fine to compress itemized lists if you used them in the
% manuscript
\usepackage{enumitem}
\usepackage{url}
\usepackage{graphicx}
\usepackage{float}
\usepackage{amsfonts}
\usepackage{multirow}
\usepackage{color}
\setlist{nosep, leftmargin=14pt}

% \usepackage{mwe} % to get dummy images

% Example definitions.
% --------------------
% \def\x{{\mathbf x}}
% \def\L{{\cal L}}

% Title.
% ------
\title{When are Diffusion Priors Helpful in Sparse Reconstruction? \\ A study with sparse-view CT}
%
% Single address.
% ---------------
\name{
Matt Y. Cheung$^{1,3,\dagger}$ \qquad Sophia Zorek$^{2,3,\dagger}$ \qquad \textit{Tucker J. Netherton}$^3$ \\ \textit{Laurence E. Court}$^3$ \qquad \textit{Sadeer Al-Kindi}$^4$ \qquad \textit{Ashok Veeraraghavan}$^1$ \qquad \textit{Guha Balakrishnan}$^1$ 
% Matt Y. Cheung$^{1,3,\dagger}$ \quad Sophia Zorek$^{2,3,\dagger}$ \quad Tucker J. Netherton$^3$ \quad Laurence E. Court$^3$ \quad Sadeer Al-Kindi$^4$
\thanks{$\dagger$: Equal contribution. M.C. would like to acknowledge support from a fellowship from the Gulf Coast Consortia on the NLM Training Program in Biomedical Informatics and Data Science T15LM007093. S.Z. would like to acknowledge that this material is based upon work supported by the U.S. Department of Energy, Office of Science, Office of Advanced Scientific Computing Research under Award Number DE-SC0025528. T.N. would like to acknowledge the support of the NIH LRP award.}
}
\address{$^1$ Department of Electrical $\&$ Computer Engineering, Rice University, Houston TX \\
        $^2$ Department of Statistics, Rice University, Houston TX \\
        $^3$ Department of Radiation Physics, The University of Texas MD Anderson Cancer Center, Houston TX \\
        $^4$ Department of Cardiology, DeBakey Heart and Vascular Center, Houston TX}


\begin{document}
% \ninept
\maketitle
%
\begin{abstract}
Diffusion models demonstrate state-of-the-art performance on image generation, and are gaining traction for sparse medical image reconstruction tasks. However, compared to classical reconstruction algorithms relying on simple analytical priors, diffusion models have the dangerous property of producing realistic looking results \emph{even when incorrect}, particularly with few observations. We investigate the utility of diffusion models as priors for image reconstruction by varying the number of observations and comparing their performance to classical priors (sparse and Tikhonov regularization) using pixel-based, structural, and downstream metrics.
We make comparisons on low-dose chest wall computed tomography (CT) for fat mass quantification.
First, we find that classical priors are superior to diffusion priors when the number of projections is ``sufficient''.
Second, we find that diffusion priors can capture a large amount of detail with very few observations, significantly outperforming classical priors. 
However, they fall short of capturing all details, even with many observations.
Finally, we find that the performance of diffusion priors plateau after extremely few ($\approx$10-15) projections.
Ultimately, our work highlights potential issues with diffusion-based sparse reconstruction and underscores the importance of further investigation, particularly in high-stakes clinical settings.
\end{abstract}
%
\begin{keywords}
Reconstruction, Diffusion, Priors
\end{keywords}

\section{Introduction}


\begin{figure}[t]
\centering
\includegraphics[width=0.6\columnwidth]{figures/evaluation_desiderata_V5.pdf}
\vspace{-0.5cm}
\caption{\systemName is a platform for conducting realistic evaluations of code LLMs, collecting human preferences of coding models with real users, real tasks, and in realistic environments, aimed at addressing the limitations of existing evaluations.
}
\label{fig:motivation}
\end{figure}

\begin{figure*}[t]
\centering
\includegraphics[width=\textwidth]{figures/system_design_v2.png}
\caption{We introduce \systemName, a VSCode extension to collect human preferences of code directly in a developer's IDE. \systemName enables developers to use code completions from various models. The system comprises a) the interface in the user's IDE which presents paired completions to users (left), b) a sampling strategy that picks model pairs to reduce latency (right, top), and c) a prompting scheme that allows diverse LLMs to perform code completions with high fidelity.
Users can select between the top completion (green box) using \texttt{tab} or the bottom completion (blue box) using \texttt{shift+tab}.}
\label{fig:overview}
\end{figure*}

As model capabilities improve, large language models (LLMs) are increasingly integrated into user environments and workflows.
For example, software developers code with AI in integrated developer environments (IDEs)~\citep{peng2023impact}, doctors rely on notes generated through ambient listening~\citep{oberst2024science}, and lawyers consider case evidence identified by electronic discovery systems~\citep{yang2024beyond}.
Increasing deployment of models in productivity tools demands evaluation that more closely reflects real-world circumstances~\citep{hutchinson2022evaluation, saxon2024benchmarks, kapoor2024ai}.
While newer benchmarks and live platforms incorporate human feedback to capture real-world usage, they almost exclusively focus on evaluating LLMs in chat conversations~\citep{zheng2023judging,dubois2023alpacafarm,chiang2024chatbot, kirk2024the}.
Model evaluation must move beyond chat-based interactions and into specialized user environments.



 

In this work, we focus on evaluating LLM-based coding assistants. 
Despite the popularity of these tools---millions of developers use Github Copilot~\citep{Copilot}---existing
evaluations of the coding capabilities of new models exhibit multiple limitations (Figure~\ref{fig:motivation}, bottom).
Traditional ML benchmarks evaluate LLM capabilities by measuring how well a model can complete static, interview-style coding tasks~\citep{chen2021evaluating,austin2021program,jain2024livecodebench, white2024livebench} and lack \emph{real users}. 
User studies recruit real users to evaluate the effectiveness of LLMs as coding assistants, but are often limited to simple programming tasks as opposed to \emph{real tasks}~\citep{vaithilingam2022expectation,ross2023programmer, mozannar2024realhumaneval}.
Recent efforts to collect human feedback such as Chatbot Arena~\citep{chiang2024chatbot} are still removed from a \emph{realistic environment}, resulting in users and data that deviate from typical software development processes.
We introduce \systemName to address these limitations (Figure~\ref{fig:motivation}, top), and we describe our three main contributions below.


\textbf{We deploy \systemName in-the-wild to collect human preferences on code.} 
\systemName is a Visual Studio Code extension, collecting preferences directly in a developer's IDE within their actual workflow (Figure~\ref{fig:overview}).
\systemName provides developers with code completions, akin to the type of support provided by Github Copilot~\citep{Copilot}. 
Over the past 3 months, \systemName has served over~\completions suggestions from 10 state-of-the-art LLMs, 
gathering \sampleCount~votes from \userCount~users.
To collect user preferences,
\systemName presents a novel interface that shows users paired code completions from two different LLMs, which are determined based on a sampling strategy that aims to 
mitigate latency while preserving coverage across model comparisons.
Additionally, we devise a prompting scheme that allows a diverse set of models to perform code completions with high fidelity.
See Section~\ref{sec:system} and Section~\ref{sec:deployment} for details about system design and deployment respectively.



\textbf{We construct a leaderboard of user preferences and find notable differences from existing static benchmarks and human preference leaderboards.}
In general, we observe that smaller models seem to overperform in static benchmarks compared to our leaderboard, while performance among larger models is mixed (Section~\ref{sec:leaderboard_calculation}).
We attribute these differences to the fact that \systemName is exposed to users and tasks that differ drastically from code evaluations in the past. 
Our data spans 103 programming languages and 24 natural languages as well as a variety of real-world applications and code structures, while static benchmarks tend to focus on a specific programming and natural language and task (e.g. coding competition problems).
Additionally, while all of \systemName interactions contain code contexts and the majority involve infilling tasks, a much smaller fraction of Chatbot Arena's coding tasks contain code context, with infilling tasks appearing even more rarely. 
We analyze our data in depth in Section~\ref{subsec:comparison}.



\textbf{We derive new insights into user preferences of code by analyzing \systemName's diverse and distinct data distribution.}
We compare user preferences across different stratifications of input data (e.g., common versus rare languages) and observe which affect observed preferences most (Section~\ref{sec:analysis}).
For example, while user preferences stay relatively consistent across various programming languages, they differ drastically between different task categories (e.g. frontend/backend versus algorithm design).
We also observe variations in user preference due to different features related to code structure 
(e.g., context length and completion patterns).
We open-source \systemName and release a curated subset of code contexts.
Altogether, our results highlight the necessity of model evaluation in realistic and domain-specific settings.





\newcommand{\tabincell}[2]{\begin{tabular}{@{}#1@{}}#2\end{tabular}}
\newcommand{\rowstyle}[1]{\gdef\currentrowstyle{#1}%
	#1\ignorespaces
}

\newcommand{\className}[1]{\textbf{\sf #1}}
\newcommand{\functionName}[1]{\textbf{\sf #1}}
\newcommand{\objectName}[1]{\textbf{\sf #1}}
\newcommand{\annotation}[1]{\textbf{#1}}
\newcommand{\todo}[1]{\textcolor{blue}{\textbf{[[TODO: #1]]}}}
\newcommand{\change}[1]{\textcolor{blue}{#1}}
\newcommand{\fetch}[1]{\textbf{\em #1}}
\newcommand{\phead}[1]{\vspace{1mm} \noindent {\bf #1}}
\newcommand{\wei}[1]{\textcolor{blue}{{\it [Wei says: #1]}}}
\newcommand{\peter}[1]{\textcolor{red}{{\it [Peter says: #1]}}}
\newcommand{\zhenhao}[1]{\textcolor{dkblue}{{\it [Zhenhao says: #1]}}}
\newcommand{\feng}[1]{\textcolor{magenta}{{\it [Feng says: #1]}}}
\newcommand{\jinqiu}[1]{\textcolor{red}{{\it [Jinqiu says: #1]}}}
\newcommand{\shouvick}[1]{\textcolor{violet(ryb)}{{\it [Shouvick says: #1]}}}
\newcommand{\pattern}[1]{\emph{#1}}
%\newcommand{\tool}{{{DectGUILag}}\xspace}
\newcommand{\tool}{{{GUIWatcher}}\xspace}


\newcommand{\guo}[1]{\textcolor{yellow}{{\it [Linqiang says: #1]}}}

\newcommand{\rqbox}[1]{\begin{tcolorbox}[left=4pt,right=4pt,top=4pt,bottom=4pt,colback=gray!5,colframe=gray!40!black,before skip=2pt,after skip=2pt]#1\end{tcolorbox}}

\section{Experiments}
\label{sec:exps}

In this section, we present comprehensive experiment results to evaluate the effectiveness of \texttt{ProDistill} across various settings. Code is available at \url{https://github.com/JingXuTHU/Scalable_Model_Merging_with_Progressive_Layerwise_Distillation}.

\begin{table*}[t]
\setlength{\tabcolsep}{4pt}
\centering
\caption{\textbf{Performance of merging ViT-B-32 models across eight downstream vision tasks.} \texttt{ProDistill} consistently outperforms the baselines under different data availability. The results for Localize-and-Stich are directly taken from~\citet{he2024localize}.}
\label{tab:vitb32}   
\begin{tabular}{l|cccccccc|cc}
\toprule
\textbf{Method} &\textbf{SUN397}& \textbf{Cars}& \textbf{RESISC45}& \textbf{EuroSAT}& \textbf{SVHN}& \textbf{GTSRB}& \textbf{MNIST}& \textbf{DTD} &\textbf{Avg}  \\
\midrule
{Individual}  & 75.34 & 77.73 & 95.98 & 99.89 & 97.46 & 98.73 & 99.69 & 79.36 & 90.52 \\
Task Arithmetic & 55.32 & 54.98 & 66.68 & 78.89 & 80.21 & 69.68 & 97.34 & 50.37 & 69.18 \\
\midrule
RegMean& 67.47 & 66.63 & 81.75 & 93.33 & 86.68 & 79.92 & 97.30 & 60.16 & 79.15 \\
Fisher merging & 63.95 & 63.84 & 66.86 & 83.48 & 79.54 & 60.11 & 91.27 & 49.36 & 69.80 \\
Localize-and-Stich  & 67.20 & 68.30 & 81.80 & 89.40 & 87.90 & 86.60 & 94.80 & 62.90 & 79.90 \\
AdaMerging& 63.69 & 65.74 & 77.65 & 91.00 & 82.48 & 93.12 & 98.27 & 62.29 & 79.28 \\ 
\rowcolor{lightyellow}
\texttt{ProDistill}~(Ours)& \textbf{68.90} & \textbf{71.21} & \textbf{89.89} & \textbf{99.37} & \textbf{96.13} & \textbf{95.29} & \textbf{99.46} & \textbf{68.03} & \textbf{86.04} \\
\bottomrule
\end{tabular}
\end{table*}

\begin{figure*}
    \centering
    \includegraphics[width=1.0\linewidth]{figure/svhn_tsne11.jpg}
    \caption{\textbf{The t-SNE visualization of ViT-B-32 model trained by different merging algorithms, on the SVHN dataset.} The features given by \texttt{ProDistill}  are the most separated, resembling those of fine-tuned models.}
    \label{fig:tsne_svhn}
\end{figure*}

\subsection{Setup}
\label{sec:setup}
We consider three main experimental setups: 
(1) Merging Vision Transformers~\citep{dosovitskiy2020image} on image classification tasks; 
(2) Merging BERT~\citep{devlin2018bert} and RoBERTa~\citep{liu2019roberta} models on natural language understanding~(NLU) tasks; 
(3) Merging LLAMA2~\citep{touvron2023llama2} model on natural language generation~(NLG) tasks. 

\paragraph{Tasks and Models:}
For image classification tasks, we follow the setting in~\citet{ilharco2022editing} and use Vision Transformer~(ViT) models pre-trained on the ImageNet dataset and subsequently fine-tuned on 8 downstream datasets. 
For NLU and NLG tasks, we merge the BERT-base and RoBERTa-base models fine-tuned on 8 NLU tasks from the GLUE~\citep{wang2018glue} benchmark, and perform pairwise merging of WizardLM-13B, WizardMath-13B and llama-2-13b-code-alpaca models, following the setting in~\citep{yu2024language}. 
Detailed information on the models and datasets can be found in Appendix~\ref{apx:dataset}.

\paragraph{Baselines:}
For vision and NLU tasks, we compare \texttt{ProDistill} with a wide range of baselines, including 
Task Arithmetic~\citep{ilharco2022editing}, 
Fisher merging~\citep{matena2022merging},
RegMean~\citep{jin2022dataless}, 
AdaMerging~\citep{yang2023adamerging} and Localize-and-Stich~\citep{he2024localize}. 
All methods, except Task Arithmetic, require a few-shot unlabeled validation dataset, which is randomly sampled from the training set, with validation shot set to 64 per task.
For NLG tasks, we compare \texttt{ProDistill} with Task Arithmetic~\citep{ilharco2022editing}, TIES-Merging~\citep{yadav2024ties} and WIDEN~\citep{yu2024extend}, due to scale constraints.
A detailed discussion of the baselines and their implementations is provided in Appendix~\ref{apx:baselines} and~\ref{apx:impl}.



\subsection{Results on Merging ViT models}
Table~\ref{tab:vitb32} presents the performance of merging ViT-B-32 models across eight downstream vision tasks.
The results for ViT-B-16 and ViT-L-14 are provided in Appendix~\ref{apx: more results}.

Our method consistently outperforms all baselines, yielding significant improvements in average performance.
Specifically, \texttt{ProDistill} achieves an average performance of 86.04\%, surpassing the baselines by 6.14\%. Notably, it is only 4\% below the average performance of the individual fine-tuned models.

We also visualize the final-layer activations of the merged model using t-SNE~\citep{van2008visualizing}. The results are given in Figure~\ref{fig:tsne_svhn} and Appendix~\ref{apx: tsne}. The visualization shows that the features given by \texttt{ProDistill} are more separated compared to the baselines, closely resembling those of the fine-tuned models.




\subsection{Results on Merging Encoder-based Language Models}
Table~\ref{tab:roberta} summarizes the results of merging RoBERTa models fine-tuned on the NLU tasks.
The results of BERT models are deferred to Appendix~\ref{apx: more results}.
Similar to the vision tasks, \texttt{ProDistill} achieves significant performance improvements of 6.61\% on the NLU tasks, outperforming all baselines across nearly all tasks. 

Unlike vision tasks, the NLU tasks in the GLUE benchmark have small class numbers. For example, SUN387 dataset consists of 397 classes, while CoLA only has 2 classes.  This class size disparity limits the performance of methods that operate directly on the model output logits, such as AdaMerging and Fisher merging.
Our method, along with RegMean, performs particularly well, emphasizing the importance of leveraging internal feature embeddings for effective model merging.

\begin{table*}[t]
\centering
\setlength{\tabcolsep}{5pt}
\caption{\textbf{Performance of merging RoBERTa models on the NLU tasks.} \texttt{ProDistill} achieves superior performance across almost all tasks.}
\label{tab:roberta} 
\begin{tabular}{l|cccccccc|cc}
\toprule
\textbf{Method} & \textbf{CoLA} & \textbf{SST-2} & \textbf{MRPC} & \textbf{STS-B} & \textbf{QQP} & \textbf{MNLI} & \textbf{QNLI} & \textbf{RTE} & \textbf{Avg} \\

\midrule
Individual & 0.5458 & 0.9450 & 0.8858 & 0.9030 & 0.8999 & 0.8710 & 0.9244 & 0.7292 & 0.8380\\
Task Arithmetic  & 0.0804 & 0.8475 & 0.7865 & 0.4890 & 0.8133 & 0.7063 & 0.7558 & 0.6534 & 0.6415 \\
\midrule
RegMean & 0.3022 & 0.9255 & 0.8183 & 0.5152 & \textbf{0.8176} & 0.7089 & 0.8503 & 0.6462 & 0.6980 \\
Fisher merging& 0.1633 & 0.7064 & 0.7264 & 0.1274 & 0.6962 & 0.4968 & 0.5599 & 0.5776 & 0.5068 \\
Localize-and-Stich& 0.0464 & 0.8922 & 0.7916 & \textbf{0.7232} & 0.7821 & 0.5709 & 0.7703 & 0.5632 & 0.6425 \\
AdaMerging& 0.000 & 0.8532 & 0.7875 & 0.5483 & 0.8086 & 0.7039 & 0.7247 & 0.6390 & 0.6332 \\
\rowcolor{lightyellow}
\rowcolor{lightyellow}
\rowcolor{lightyellow}
\texttt{ProDistill}~(Ours)& \textbf{0.4442} & \textbf{0.9312} & \textbf{0.8464} & 0.6942 & 0.8134 & \textbf{0.7857} & \textbf{0.8900} & \textbf{0.7076} & \textbf{0.7641} \\
\bottomrule
\end{tabular}
\end{table*}


\subsection{Results on Merging Large Language Models}
We present the results of merging the WizardMath-13B and Llama-2-13B-Code-Alpaca models in~Table~\ref{tab: code_math}, with additional results provided in Appendix~\ref{apx: more results} and generation examples provided in Appendix~\ref{apx: llm example}. These findings demonstrate that our method effectively scales up to models with over 10B parameters, and achieves superior performance compared to baselines.

\begin{table*}[t]
\centering
\caption{\textbf{Performance of merging LLM models on Code and Math tasks.} Our method demonstrates an improved performance and a strong scalability. The results of TIES-Merging and WIDEN are directly taken from~\citet{yu2024extend}.}
\begin{tabular}{l|cccc|cc}
\toprule
\textbf{Method} & \textbf{GSM8K} & \textbf{MATH} & \textbf{HumanEval} & \textbf{MBPP} & \textbf{Avg} & \textbf{Norm Avg} \\

\midrule
WizardMath-13B & 0.6361 & 0.1456 & 0.0671 & 0.0800 & 0.2322 & 0.6430 \\
Llama-2-13b-code-alpaca & 0.000 & 0.000 & 0.2378 & 0.2760 & 0.1285 & 0.5000 \\
\midrule
Task Arithmetic & \textbf{0.6467} & \textbf{0.1462} & 0.0854 & 0.0840 & 0.2406 & 0.6711\\
TIES-Merging & 0.6323 & 0.1356 & 0.0976 & 0.2240 & 0.2723 & 0.7868\\
WIDEN & 0.6422 & 0.1358 & 0.0976 & 0.0980 & 0.2434 & 0.6769\\
\rowcolor{lightyellow}
\texttt{ProDistill} (Ours) & 0.6279 & 0.1424 & \textbf{0.1280} & \textbf{0.2239} & \textbf{0.2806} & \textbf{0.8288}\\
\bottomrule
\end{tabular}
\label{tab: code_math}
\end{table*}
\section{Conclusion}


\sdeni{}{We introduced convex-concave generative-adversarial characterization of inverse Nash equilibria for a large class of games, including normal-form, finite state and action Markov games, and a number of continuous state and action Markov games. This novel formulation then allowed us to obtain polynomial-time computation guarantees for inverse equilibria in these games, a rather surprising result since the computation of a Nash equilibrium is in general PPAD-complete. Our result can be thus seen as a positive computation result for game theory. We then extended our characterization to a multiagent apprenticeship learning setting, where we souught to not only rationalize the observed behavior as an inverse Nash equilibrium but also make predictions based off the inverse Nash equilibrium, and have shown in experiments on prices in Spanish electricity markets that our approach to solving multiagent apprenticship learning can be effective at predicting behavior in multiagent systems. The approach to inverse game theory that we provided in this paper is a highly flexible one and thus can be used to solve inverse equilibrium beyond inverse Nash equilibria and future work could explore ways to extend our approach to other game-theoretic settings and equilibrium concepts.}


\amy{discuss extenstion to other eqm concepts? CE, CCE, etc.}

\amy{and other idea from yesterday. check text thread?}



\section{Compliance with Ethical Standards}
This research was conducted retrospectively using human data under an approved IRB protocol.


\bibliographystyle{IEEEbib}
\bibliography{refs}

\end{document}
