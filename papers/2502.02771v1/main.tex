% Template for ISBI paper; to be used with:
%          spconf.sty  - ICASSP/ICIP LaTeX style file, and
%          IEEEbib.bst - IEEE bibliography style file.
% --------------------------------------------------------------------------
\documentclass{article}
\usepackage{spconf,amsmath,graphicx}

% It's fine to compress itemized lists if you used them in the
% manuscript
\usepackage{enumitem}
\usepackage{url}
\usepackage{graphicx}
\usepackage{float}
\usepackage{amsfonts}
\usepackage{multirow}
\usepackage{color}
\setlist{nosep, leftmargin=14pt}

% \usepackage{mwe} % to get dummy images

% Example definitions.
% --------------------
% \def\x{{\mathbf x}}
% \def\L{{\cal L}}

% Title.
% ------
\title{When are Diffusion Priors Helpful in Sparse Reconstruction? \\ A study with sparse-view CT}
%
% Single address.
% ---------------
\name{
Matt Y. Cheung$^{1,3,\dagger}$ \qquad Sophia Zorek$^{2,3,\dagger}$ \qquad \textit{Tucker J. Netherton}$^3$ \\ \textit{Laurence E. Court}$^3$ \qquad \textit{Sadeer Al-Kindi}$^4$ \qquad \textit{Ashok Veeraraghavan}$^1$ \qquad \textit{Guha Balakrishnan}$^1$ 
% Matt Y. Cheung$^{1,3,\dagger}$ \quad Sophia Zorek$^{2,3,\dagger}$ \quad Tucker J. Netherton$^3$ \quad Laurence E. Court$^3$ \quad Sadeer Al-Kindi$^4$
\thanks{$\dagger$: Equal contribution. M.C. would like to acknowledge support from a fellowship from the Gulf Coast Consortia on the NLM Training Program in Biomedical Informatics and Data Science T15LM007093. S.Z. would like to acknowledge that this material is based upon work supported by the U.S. Department of Energy, Office of Science, Office of Advanced Scientific Computing Research under Award Number DE-SC0025528. T.N. would like to acknowledge the support of the NIH LRP award.}
}
\address{$^1$ Department of Electrical $\&$ Computer Engineering, Rice University, Houston TX \\
        $^2$ Department of Statistics, Rice University, Houston TX \\
        $^3$ Department of Radiation Physics, The University of Texas MD Anderson Cancer Center, Houston TX \\
        $^4$ Department of Cardiology, DeBakey Heart and Vascular Center, Houston TX}


\begin{document}
% \ninept
\maketitle
%
\begin{abstract}
Diffusion models demonstrate state-of-the-art performance on image generation, and are gaining traction for sparse medical image reconstruction tasks. However, compared to classical reconstruction algorithms relying on simple analytical priors, diffusion models have the dangerous property of producing realistic looking results \emph{even when incorrect}, particularly with few observations. We investigate the utility of diffusion models as priors for image reconstruction by varying the number of observations and comparing their performance to classical priors (sparse and Tikhonov regularization) using pixel-based, structural, and downstream metrics.
We make comparisons on low-dose chest wall computed tomography (CT) for fat mass quantification.
First, we find that classical priors are superior to diffusion priors when the number of projections is ``sufficient''.
Second, we find that diffusion priors can capture a large amount of detail with very few observations, significantly outperforming classical priors. 
However, they fall short of capturing all details, even with many observations.
Finally, we find that the performance of diffusion priors plateau after extremely few ($\approx$10-15) projections.
Ultimately, our work highlights potential issues with diffusion-based sparse reconstruction and underscores the importance of further investigation, particularly in high-stakes clinical settings.
\end{abstract}
%
\begin{keywords}
Reconstruction, Diffusion, Priors
\end{keywords}

\section{Introduction}
\label{sec:intro}

\begin{figure*}[tb]
    \centering
    \includegraphics[width=0.848\linewidth]{figs/circuitnn.pdf} 
    \caption{Illustration of differentiable CircuitNN. CircuitNN is designed based on differentiable NAND gates. After DAS is guided by PI and PO pairs of the truth table, CircuitNN can get the precise circuit architecture logic equivalent to the truth table.}
    \label{fig:circuitnn}
\end{figure*}

% 1. Describe the importance of logic synthesis
% 2. Existing Problems
% (a) Neural Architecture Search: Unstable, Predefined Setting, etc.
% (b) Circuit Generation: Probabilistic Model, Logic Equivalence

With the rapid advancement of technology, the scale of integrated circuits (ICs) has expanded exponentially. 
This expansion has introduced significant challenges in chip manufacturing, particularly concerning power and area metrics.
A primary objective in IC design is achieving the same circuit function with fewer transistors, thereby reducing power usage and area occupancy.

Logic synthesis~\cite{hachtel2005logicsynth}, a critical step in electronic design automation (EDA), transforms behavioral-level circuit designs into optimized gate-level circuits, ultimately yielding the final IC layout. 
The primary goal of logic synthesis is to identify the physical implementation with the fewest gates for a given circuit function. 
This task constitutes a challenging NP-hard combinatorial optimization problem. 
Current logic synthesis tools~\cite{brayton2010abc, wolf2013yosys} rely on human-designed heuristics, often leading to sub-optimal outcomes.

Differentiable architecture search (DAS) techniques~\cite{liu2018darts, chu2020darts} offer novel perspectives on addressing challenges in this problem.
Circuit functions can be represented through truth tables, which map binary inputs to their corresponding outputs. 
Truth tables provide a precise representation of input-output relationships, ensuring the design of functionally equivalent circuits.
Inspired by this, researchers~\cite{deepmind2024ai4sys, wang2024tnet} have begun exploring the application of DAS to synthesize circuits directly from truth tables.
Specifically, \citet{deepmind2024ai4sys} proposed CircuitNN, a framework that learns differentiable connection structures with logic gates, enabling the automatic generation of logic circuits from truth tables.
This approach significantly reduces the complexity of traditional circuit generation. 
Building on this, \citet{wang2024tnet} introduced T-Net, a triangle-shaped variant of CircuitNN, incorporating regularization techniques to enhance the efficiency of DAS.

Despite these advancements, several challenges remain. 
The computational complexity of DAS grows quadratically with the number of gates, posing scalability issues.
Although triangle-shaped architecture~\cite{wang2024tnet} partially mitigates this problem, redundancy persists. 
%Additionally, DAS is susceptible to converging to local optima, limiting the ability to search architectures that satisfy the given truth tables~\cite{liu2018darts}. 
%Furthermore, hyperparameters (network depth and layer width) require extensive searches, introducing complexity and prolonging the synthesis process. 
Additionally, DAS is susceptible to converging to local optima~\cite{liu2018darts} and hyperparameters (network depth and layer width) require extensive searches. 
The challenges arise from the vast search space in DAS. 
% Even with predefined settings for CircuitNN, finding a configuration that meets the truth table requires extensive trial and error during the DAS process. 
Intuitively, limiting the search space through predefined parameters (network depth, gates per layer, and connection probabilities) can significantly reduce the complexity.

Recent advances~\cite{openai2023gpt4, abramson2024alphafold3, esser2024sd3, li2024mar} in conditional generative models have demonstrated remarkable performance across language, vision, and graph generation tasks. 
Motivated by these developments, we propose a novel approach to circuit generation that generates preliminary circuit structures to guide DAS in generating refined circuits matching specified truth tables. 
Firstly, we introduce CircuitVQ, a tokenizer with a discrete codebook for circuit tokenization. 
Built upon our Circuit AutoEncoder framework~\cite{hou2022graphmae,li2023maskgae,wu2025mgvga}, CircuitVQ is trained through a circuit reconstruction task. 
Specifically, the CircuitVQ encoder encodes input circuits into discrete tokens using a learnable codebook, while the decoder reconstructs the circuit adjacency matrix based on these tokens.
Subsequently, the CircuitVQ encoder serves as a circuit tokenizer for CircuitAR pretraining, which employs a masked autoregressive modeling paradigm~\cite{chang2022maskgit, li2023mage}. 
In this process, the discrete codes function as supervision signals. 
After training, CircuitAR can generate discrete tokens progressively, which can be decoded into initial circuit structures by the decoder of the CircuitVQ. 
These prior insights can guide DAS in producing refined circuits that match the target truth tables precisely.

Our key contributions can be summarized as follows:
\begin{itemize}
\item We introduce CircuitVQ, a circuit tokenizer that facilitates graph autoregressive modeling for circuit generation, based on our Circuit AutoEncoder framework;
\item Develop CircuitAR, a model trained using masked autoregressive modeling, which generates initial circuit structures conditioned on given truth tables;
\item Propose a refinement framework that integrates differentiable architecture search to produce functionally equivalent circuits guided by target truth tables;
\item Comprehensive experiments demonstrating the scalability and capability emergence of our CircuitAR and the superior performance of the proposed circuit generation approach.
\end{itemize}

% Motivation
% (a) Diffusion (Vision, Graph), Autoregressive (Language, Vision)
% (b) Circuit Generation for Predefined Setting
% (c) Neural Architecture Search for Strict Logic Equivalence

% Contribution
% (a) Circuit Tokenizer (new transformer arch, training strategy)
% (b) CircuitAR (train and gen strategies, post-ar strategy)
% (c) Extensive Evaluation including BitD (Bit Distance) for Scalability

\iffalse
\begin{table*}[htbp]
\tiny
\begin{center}
\begin{tabular}{lccccccccccccc}\toprule
Model, ft setting & mem & \#param & ARC-c & ARC-e & BoolQ & HS & OBQA & PIQA & rte & SIQA & WG & Avg
%\\\cmidrule(lr){2-3}\cmidrule(lr){4-5} \cmidrule(lr){6-7} \cmidrule(lr){8-9}\cmidrule(lr){10-11} \cmidrule(lr){12-13} \cmidrule(lr){14-15} \cmidrule(lr){16-17} 
\\\cmidrule(lr){1-13}
Llama2(7B), LoRA, $r=64$ & 23.46GB & 159.9M(2.37\%) & \textbf{44.97} & 77.02 & 77.43 & \textbf{57.75} & 32.0 & \textbf{78.45} & 62.09 & \textbf{47.75} & 68.75 & 60.69\\
Llama2(7B), SPruFT, $r=128$ & \textbf{17.62GB} & 145.8M(2.16\%) & 43.60 & \textbf{77.26} & \textbf{77.77} & 57.47 & \textbf{32.6} & 78.07 & \textbf{64.98} & 46.67 & \textbf{69.30} & \textbf{60.86} \\\cmidrule(lr){2-13}
Llama2(7B), FA-LoRA, $r=64$ & 17.25GB & 92.8M(1.38\%) & 43.77 & \textbf{77.57} & 77.74 & \textbf{57.45} & 31.0 & 77.86 & \textbf{66.06} & \textbf{47.13} & 69.06 & 60.85\\
Llama2(7B), FA-SPruFT, $r=128$ & \textbf{15.21GB} & 78.6M(1.17\%) & \textbf{43.94} & 77.22 & \textbf{77.83} & 57.11 & \textbf{32.0} & \textbf{78.18} & 65.70 & 46.47 & \textbf{69.38} & \textbf{60.87}\\\midrule
Llama3(8B), LoRA, $r=64$ & 30.37GB & 167.8M(2.09\%) & \textbf{53.07} & \textbf{81.40} & \textbf{82.32} & \textbf{60.67} & 34.2 & \textbf{79.98} & 69.68 & \textbf{48.52} & \textbf{73.56} & \textbf{64.82}\\
Llama3(8B), SPruFT, $r=128$ & \textbf{24.49GB} & 159.4M(1.98\%) & 52.47 & 81.10 & 81.28 & 60.29 & \textbf{34.6} & 79.76 & \textbf{70.04} & 47.75 & 73.24 & 64.50 \\\cmidrule(lr){2-13}
Llama3(8B), FA-LoRA, $r=64$ & 24.55GB & 113.2M(1.41\%) & \textbf{52.47} & \textbf{81.36} & \textbf{82.23} & 60.17 & \textbf{35.0} & \textbf{79.76} & \textbf{70.04} & \textbf{48.31} & \textbf{73.56} & \textbf{64.77}\\
Llama3(8B), FA-SPruFT, $r=128$ & \textbf{22.41GB} & 92.3M(1.15\%) & 52.22 & 81.19 & 81.35 & \textbf{60.20} & 34.2 & 79.71 & 69.31 & 47.13 & 73.01 & 64.26 \\\bottomrule
\end{tabular}
%\vspace{-0.2cm}
\caption{Fine-tuning Llama on Alpaca dataset for 5 epochs and evaluating on 9 tasks from EleutherAI LM Harness. "mem" represents the memory usage, with further details provided in Appendix~\ref{apdx:measure}. \#param is the number of trainable parameters, where the difference of \#param between the two approaches depends on the architecture of Llama, as some layers have $d_{in} \neq d_{out}$. Note that 10 million trainable parameters only account for less than 0.15GB of memory requirement. FA indicates that we freeze attention layers, but not including MLP layers followed by attention blocks. HS, OBQA, and WG represent HellaSwag, OpenBookQA, and WinoGrande datasets. More details of datasets can be found in Appendix~\ref{apdx:data}. The ablation study for different $r$ and the comparison with other LoRA variants can be found in Appendix~\ref{apdx:ablation}. All reported results are accuracies on the corresponding tasks. \textbf{Bold} indicates the best results of two approaches on the same task.} \label{tab:llm} 
\end{center}
\end{table*}
\fi

\begin{table*}[htbp]
\tiny
\begin{center}
\begin{tabular}{lccccccccccccc}\toprule
Model, ft setting & mem & \#param & ARC-c & ARC-e & BoolQ & HS & OBQA & PIQA & rte & SIQA & WG & Avg
\\\cmidrule(lr){1-13}
Llama2(7B)\\ \cmidrule(lr){1-1} 
LoRA, $r=64$ & 23.46GB & 159.9M(2.37\%) & \textbf{44.97} & 77.02 & 77.43 & 57.75 & 32.0 & \textbf{78.45} & 62.09 & 47.75 & 68.75 & 60.69\\
VeRA, $r=64$ & 22.97GB & 1.374M(0.02\%) & 43.26 & 76.43 & 77.40 & 57.26 & 31.6 & 78.02 & 62.09 & 45.85 & 68.75 & 60.07\\
DoRA, $r=64$ & 44.85GB & 161.3M(2.39\%) & 44.71 & 77.02 & 77.55 & \textbf{57.79} & 32.4 & 78.29 & 61.73 & \textbf{47.90} & 68.98 & 60.71\\
RoSA, $r=32, d=1.2\%$ & 44.69GB & 157.7M(2.34\%) & 43.86 & \textbf{77.48} & \textbf{77.86} & 57.42 & 32.2 & 77.97 & 63.90 &  47.29 & 69.06 & 60.78\\
SPruFT, $r=128$ & \textbf{17.62GB} & 145.8M(2.16\%) & 43.60 & 77.26 & 77.77 & 57.47 & \textbf{32.6} & 78.07 & \textbf{64.98} & 46.67 & \textbf{69.30} & \textbf{60.86} %\\\cmidrule(lr){2-13}
%FA-LoRA, $r=64$ & 17.25GB & 92.8M(1.38\%) & 43.77 & \textbf{77.57} & 77.74 & \textbf{57.45} & 31.0 & 77.86 & 66.06 & \textbf{47.13} & 69.06 & 60.85\\
%FA-DoRA, $r=64$ & 30.61GB & 93.6M(1.39\%) & 43.94 & 77.44 & 77.49 & 57.44 & 31.0 & 77.86 & \textbf{66.43} & 46.98 & 69.14 & 60.86\\
%FA-RoSA, $r=32, d=1.2\%$ & 38.34GB & 98.3M(1.46\%) & \textbf{44.28} & 77.02 & 77.68 & 57.22 & 31.0 & 77.97 & 64.26 & 46.32 & 69.22 & 60.55\\
%FA-SPruFT, $r=128$ & \textbf{15.21GB} & 78.6M(1.17\%) & 43.94 & 77.22 & \textbf{77.83} & 57.11 & \textbf{32.0} & \textbf{78.18} & 65.70 & 46.47 & \textbf{69.38} & \textbf{60.87}
\\\midrule
Llama3(8B)\\ \cmidrule(lr){1-1} 
LoRA, $r=64$ & 30.37GB & 167.8M(2.09\%) & 53.07 & 81.40 & 82.32 & 60.67 & 34.2 & 79.98 & 69.68 & 48.52 & 73.56 & 64.82\\
VeRA, $r=64$ & 29.49GB & 1.391M(0.02\%) & 50.26 & 80.30 & 81.41 & 60.16 & 34.4 & 79.60 & 69.31 & 46.93 & 72.77 & 63.90\\
DoRA, $r=64$ & 51.45GB & 169.1M(2.11\%) & \textbf{53.33} & \textbf{81.57} & \textbf{82.45} & \textbf{60.71} & 34.2 & \textbf{80.09} & 69.31 & \textbf{48.67} & \textbf{73.64} & \textbf{64.88}\\
RoSA, $r=32, d=1.2\%$ & 48.40GB & 167.6M(2.09\%) & 51.28 & 81.27 & 81.80 & 60.18 & 34.4 & 79.87 & 69.31 & 47.95 & 73.16 & 64.36\\
SPruFT, $r=128$ & \textbf{24.49GB} & 159.4M(1.98\%) & 52.47 & 81.10 & 81.28 & 60.29 & \textbf{34.6} & 79.76 & \textbf{70.04} & 47.75 & 73.24 & 64.50 %\\\cmidrule(lr){2-13}
%FA-LoRA, $r=64$ & 24.55GB & 113.2M(1.41\%) & 52.47 & 81.36 & 82.23 & 60.17 & \textbf{35.0} & 79.76 & 70.04 & 48.31 & \textbf{73.56} & 64.77\\
%FA-DoRA, $r=64$ & 40.62GB & 114.3M(1.42\%) & \textbf{52.56} & \textbf{81.69} & \textbf{82.26} & \textbf{60.20} & 34.4 & \textbf{79.82} & \textbf{70.40} & \textbf{48.46} & 73.40 & \textbf{64.80}\\
%FA-RoSA, $r=32, d=1.2\%$ & 42.31GB & 124.3M(1.55\%) & 52.22 & 81.19 & 82.05 & 60.11 & 34.4 & 79.76 & 69.31 & 47.70 & 73.16 & 64.43\\
%FA-SPruFT, $r=128$ & \textbf{22.41GB} & 92.3M(1.15\%) & 52.22 & 81.19 & 81.35 & \textbf{60.20} & 34.2 & 79.71 & 69.31 & 47.13 & 73.01 & 64.26 
\\\bottomrule
\end{tabular}
%\vspace{-0.2cm}
\caption{Fine-tuning Llama on Alpaca dataset for 5 epochs and evaluating on 9 tasks from EleutherAI LM Harness. ``mem" represents the memory usage, with further details provided in Appendix~\ref{apdx:measure}. \#param is the number of trainable parameters, where the difference of \#param between the two approaches depends on the architecture of Llama, as some layers have $d_{in} \neq d_{out}$. %FA indicates that we freeze attention layers, but not including MLP layers followed by attention blocks. 
HS, OBQA, and WG represent HellaSwag, OpenBookQA, and WinoGrande datasets. %More details of datasets can be found in Appendix~\ref{apdx:data}. 
The ablation study for different $r$ can be found in Appendix~\ref{apdx:ranks}. All reported results are accuracies on the corresponding tasks. \textbf{Bold} indicates the best result on the same task. } \label{tab:llm} 
\end{center}
\end{table*}

\section{Experimental Setup}\label{sec:setup}

%(0.5 page)
%Why the chosen framework?
%Some prior approaches

%- parameter settings
%- uniform across layers vs greedy ... 
%- potential transformer-specific details

%Equations about what these methods do.. 

%(0.5 page)
%Which NN architectures are used, why?
%Number of parameters, layers, ...

%(Potential prior work on compression -- )

\subsection{Datasets} \label{subsec:dataset}
We use multiple datasets for different tasks. For image classification, we fine-tune models on the training split and evaluate it on the validation split of Tiny-ImageNet~\citep{tavanaei2020embedded}, CIFAR100~\citep{alex2009learning}, and Caltech101~\citep{li_andreeto_ranzato_perona_2022}. For text generation, we fine-tune LLMs on 256 samples from Stanford-Alpaca~\citep{alpaca} and assess zero-shot performance on nine EleutherAI LM Harness tasks~\citep{gao2021framework}. See Appendix~\ref{apdx:data} for details.

\subsection{Models and Baselines} \label{subsec:models}

We fine-tune full-precision Llama-2-7B and Llama-3-8B (float32) using our SPruFT, LoRA~\citep{hulora}, VeRA~\citep{kopiczko2024vera}, DoRA~\citep{liu2024dora}, and RoSA~\citep{nikdan2024rosa}. RoSA is chosen as the representative SFT method and is the only SFT due to the high memory demands of other SFT approaches, while full fine-tuning is excluded for the same reason. We freeze Llama’s classification layers and fine-tune only the linear layers in attention and MLP blocks.

Next, we evaluate importance metrics by fine-tuning Llamas and image models, including DeiT~\citep{touvron2021training}, ViT~\citep{dosovitskiy2020image}, ResNet101~\citep{he2016deep}, and ResNeXt101~\citep{xie2017aggregated} on CIFAR100, Caltech101, and Tiny-ImageNet. For image tasks, we set the fine-tuning ratio at 5\%, meaning the trainable parameters are a total of 5\% of the backbone plus classification layers.

\subsection{Training Details} \label{subsec:training}
Our fine-tuning framework is built on torch-pruning\footnote{Torch-pruning is not required, all their implementations are based on PyTorch.}~\citep{fang2023depgraph}, PyTorch~\citep{paszke2019pytorch}, PyTorch-Image-Models~\citep{rw2019timm}, and HuggingFace Transformers~\citep{wolf2020transformers}. Most experiments run on a single A100-80GB GPU, while DoRA and RoSA use an H100-96GB GPU. We use the Adam optimizer~\citep{KingBa15} and fine-tune all models for a fixed number of epochs without validation-based model selection.

%Structured pruning often considers parameter dependencies in importance evaluation~\citep{liu2021group, fang2023depgraph, ma2023llmpruner}. This becomes the following process in our work: first, searching for dependencies by tracing the computation graph of gradient; next, evaluating the importance of parameter groups; and finally, fine-tuning the parameters within those important groups collectively. For instance, if $\W^{a}_{\cdot j}$ and $\W^{b}_{i\cdot}$ are dependent, where $\W^{a}_{\cdot j}$ is the $j$-th column in parameter matrix (or the $j$-th input channels/features) of layer $a$ and $\W^{b}_{i\cdot}$ is the $i$-th row in parameter matrix (or the $i$-th output channels/features) of layer $b$, then $\W^{a}_{\cdot j}$ and $\W^{b}_{i\cdot}$ will be fine-tuned simultaneously while the corresponding $\M^{a}_{dep}$ for $\W^{a}_{\cdot j}$ becomes column selection matrix and $\W^a_s$ becomes $\W^a_{f,dep}\M^a_{dep}$. Consequently, fine-tuning $2.5\%$ output channels for layer $b$ will result in fine-tuning additional $2.5\%$ input channels in each dependent layer. Therefore, for the $5\%$ of desired fine-tuning ratio, the fine-tuning ratio with considering dependencies is set to $2.5\%$\footnote{In some complex models, considering dependencies results in slightly more than twice the number of trainable parameters. However, in most cases, the factor is 2.} for the approach that includes dependencies. More details for dependencies of NN can be found in Appendix~\ref{apdx:dep}. 

\textbf{Image models}: The learning rate is set to $10^{-4}$ with cosine annealing decay~\citep{loshchilov2017sgdr}, where the minimum learning rate is $10^{-9}$. All image models used in this study are pre-trained on ImageNet. 

\textbf{Llama}: For LoRA and DoRA, we set $\alpha = 16$, a dropout rate of $0.1$, and a learning rate of $10^{-4}$  with linear decay (
$0.01$ decay rate). For SPruFT, we control trainable parameters using rank instead of fine-tuning ratio for direct comparison. The learning rate is $2 \cdot 10^{-5}$ with the same decay settings. Linear decay is applied after a warmup over the first $3$\% of training steps. The maximum sequence length is $2048$, with truncation for longer inputs and padding for shorter ones.


\section{Experiments}
\label{sec:exps}

\subsection{Experimental Setup}
\label{sec:exps-setup}

We evaluate our method using the Gemini 1.5 Flash model \citep{gemini2024} as the base VLM. 
Gemini 1.5 Flash is a powerful, instruction-tuned VLM that can take as input interleaved text and images and it provides a strong base model.
We use standard supervised fine-tuning procedure (see Appendix~\ref{sec:training-details}).
We limit the game length to a maximum of three question-answer turns.
For conciseness, we refer to the self-improvement method of the fine-tuning on synthetically collected dialogs as "VLM Dialog Games".

\subsection{Experiments with General Images in Dialog Games}
\label{sec:exps-docci}

This section details our experiments using the DOCCI~\citep{OnoeDocci2024} and the OpenImages datasets \citep{kuznetsova2020openimages} to evaluate the effectiveness of our self-improvement method for image understanding through VQA tasks.

\subsubsection{Dataset and Game Configuration}
\label{sec:game-config}

\paragraph{DOCCI} dataset contains clusters of images grouped by their category.
We randomly sample \num{1000} image groups, each containing $N = 4$ images from one of \num{149} categories.
Figure~\ref{fig:docci_example} provides an example of a dialog game generated by prompted Gemini using this setup.

\begin{figure}[t]
    \centering
    \includegraphics[width=\columnwidth]{assets/docci-game.pdf}
    \caption{\textbf{An example dialog game using images from the DOCCI dataset}, grouped by clusters.
    The figure shows the Guesser's questions, the Describer's answers, and the Guesser's internal dialog summary.  The Guesser correctly identifies the target image (4) at the end of the dialog.}
    \vspace{-3mm}
    \label{fig:docci_example}
\end{figure}

\paragraph{OpenImages}
We select a subset of \num{1000} random images, forming them into games with $N=4$ images.
As the dataset does not contain clusters, we select the most similar images \citep{jia2021align} as distractors.
An example of a dialog game produced in this scenario is demonstrated in Figure~\ref{fig:open_image_example_dialog}.

\subsubsection{Evaluations Tasks}
\paragraph{Dialog success rate}

Following prior work using dialog games to assess VLM capabilities \citep{hakimov2024usinggameplayinvestigate}, we use the dialog game success rate as one of measures of the model's improvement.
We report the percentage of games where the Guesser correctly identifies the target image across all $N$ tested permutations (as described in Section~\ref{method-filtering}).  

\begin{figure}[t]
    \centering
    \includegraphics[width=\columnwidth]{assets/dialog4.pdf}
    \caption{\textbf{An example of a dialog game with OpenImages} grouped by the image similarity.
    The figure shows the Guesser's questions, the Describer's answers, and the Guesser's internal dialog summary.  The Guesser correctly identifies the target image (1) at the end of the dialog.}
    \vspace{-3mm}
    \label{fig:open_image_example_dialog}
\end{figure}



\paragraph{Visual question answering (VQA)}

To assess the broader impact of our self-improvement method on general visual understanding, we evaluate the fine-tuned model on a subset of the VQAv2 dataset~\citep{goyal2017making}.
We focus on two specific question types:

\begin{itemize}
    \item \textbf{Binary (yes/no) questions}: Semantically equivalent phrasings (e.g., "No" and "There is no cat") are treated as correct. We report the model accuracy.
    \item \textbf{Object counting questions}: All answers and ground truth labels are converted to numerical form (e.g., "one" becomes "1", "none" becomes "0"). We report a strict exact-match accuracy.
\end{itemize}

\subsubsection{Results}

Table~\ref{tab:docci_captioning} compares the performance of the base Gemini 1.5 Flash model with VLM Dialog Games method.
Fist, results demonstrate that the VLM Dialog Games method with either the DOCCI or OpenImages datasets improves performance within the game with both training and unseen images (e.g., games played on DOCCI by a model trained with OpenImages).
More importantly we also achieve better performance on broader visual understanding tasks as measured by VQA accuracy.
Note that evaluation images for it are drawn from a distinct dataset (VQAv2), demonstrating the generalization of our method.
Specifically, for DOCCI dialog games, the accuracy on the VQAv2 yes/no and counting subsets increased by \num{6.8}\% and \num{2.3}\%, respectively.
For OpenImages dialog games, yes/no question accuracy increases by \num{10.4}\% and remains unchanged for counting questions. 
We hypothesis that different image sources may be better suited for improving specific tasks.
For example, \citet{OnoeDocci2024} note that many DOCCI images contain references to counts, suggesting that this dataset is well-suited for self-improvement on counting task.

\begin{table*}[h]
    \centering
     \caption{\textbf{Comparison of VLM Dialog Games and the initial Gemini 1.5 Flash.} Fine-tuning on dialog game data improves both game success rate and VQA performance (yes/no and counting subsets).  Results demonstrate generalization across training and evaluation datasets.}
    \vspace{5mm}
    \begin{tabular}{l|r|r|r|r}
      \multicolumn{1}{c}{Model} & \multicolumn{2}{c}{game success} & VQA  & VQA \\
       & ~~DOCCI~~ & OpenImages & yes/no & counting \\
      \midrule
      Gemini 1.5 Flash & 20.3\%  & 18.4\% & 73.0\% & 56\% \\
      VLM Dialog Games (DOCCI) & 24.4\% & 21.9\% & 79.8\% (+6.8) & 58.3\% (+2.3) \\
      VLM Dialog Games (OpenImages) & 25.6\% & 23.6\% & 83.4\% (+10.4) & 56\% (+0.0)\\
    \end{tabular}
    \label{tab:docci_captioning}
\end{table*}

\subsection{Ablation Studies}
\label{sec:exps-openimages-ablations}

Next, we investigate the impact of key design choices: the number of images per game and the method of image grouping.
We test the different options on OpenImages dialog games and VQA yes/no question accuracy.

\paragraph{Impact of the number of images per game}
We study the effect of $N$ on the game complexity by varying $N$ from \num{2} to \num{8} (see Appendix~\ref{game-examples-n} for dialog examples). Table~\ref{tab:openimage_n_images} presents the game success rate, the number of question-answer pairs from successful dialogs, and the VQAv2 yes/no accuracy for each $N$.
While fine-tuning with data from any $N$ improves VQAv2 performance compared to the base Gemini 1.5 Flash model, the best result is achieved with $N = 4$ in this study.
With $N = 2$, the game is relatively simple, leading to a high success rate but potentially less informative data, and a higher probability of erroneous data due to the correct guesses by chance.
Conversely, with $N = 8$, the game becomes too difficult, resulting in few successful dialogs for fine-tuning.
These results confirm that balancing game difficulty and the quantity of training data is crucial for generating an optimal dataset for fine-tuning.

\begin{table}[t]
    \centering
    \caption{\textbf{Impact of varying the number of images $N$ per game}: We report the number of successful dialog games (out of \num{1000}), the total number of question-answer pairs extracted, and the VQAv2 yes/no accuracy after fine-tuning. The optimal $N$ in this case appears to be \num{4}, balancing game difficulty and data quantity.}
    \vspace{5mm}
    \begin{tabular}{c|r|r|r}
      $N$ & game  & QA & VQA \\
       & success & pairs & yes/no \\
      \midrule
      2  & 83.7\% & 879 & 81.3\%  (~~+8.3\%) \\
      4  & 18.4\% & 275 & 83.4\% (+10.4\%) \\
      8  & 0.24\% & 34 & 77.1\%  (~~+4.1\%) \\
      \midrule
      \multicolumn{3}{l}{Gemini 1.5 Flash} & \multicolumn{1}{|l}{73\%} \\
    \end{tabular}
    \label{tab:openimage_n_images}
\end{table}

\paragraph{Impact of Image Grouping Strategy}

We investigate how image grouping affects model performance by comparing two strategies: 1) similarity-based grouping (Section~\ref{sec:game-config}), which uses visually and conceptually related distractors to elicit more targeted Guesser questions, and 2) random distractor selection.
Table~\ref{tab:openimage_vqav2_grouping} compares models using these strategies. 
Both strategies improve over the initial Gemini 1.5 Flash checkpoint ($73.0$\%) significantly, therefore, the VLM Dialog Game can be effectively implemented even with random image groupings. 
However, using similar images yields slightly higher accuracy ($83.4$\% vs. $82.6$\%).
While random images produce a larger quantity of successful dialogs ($24.7$\% vs. $18.4$\%), the increased challenge of similar images in a game likely leads to more informative training data.
Thus, we believe that for the best results in fine-tuning, we need to find a right trade off between game difficulty and training data quantity.

\begin{table}[t]
    \centering
    \caption{\textbf{Impact of image grouping strategy:} Both random and semantically similar image groupings lead to significant performance gains compared to the baseline. Although using semantically similar images demonstrates slightly better results, the difference is small, highlighting the robustness of the VLM Dialog Game approach even with random image selection.}
    \vspace{5mm}
    \begin{tabular}{l|r|r}
      Image grouping & game & VQA \\
      strategy & success & yes/no \\
      \midrule
      None (initial) & N/A & 73.0\% \\
      Similar images & 18.4\% & 83.4\% (+10.4\%) \\
      Random images & 24.7\% & 82.6\% (~~+9.6\%) \\
    \end{tabular}
    \label{tab:openimage_vqav2_grouping}
\end{table}

\subsection{Robotics Dialog Games}
\label{sec:exps-robotics}

\begin{figure}[t]
    \centering
    \includegraphics[width=\columnwidth]{assets/robotics_dialog_example_success.pdf}
    \caption{
    \textbf{An example of a dialog game in the robotics domain.} The figure shows the Guesser's questions, the Describer's answers, and the Guesser's internal dialog summary.  The Guesser correctly identifies the target image (1) at the end of the dialog.}
    \vspace{-3mm}
    \label{fig:game-example-robotics-success}
\end{figure}

High-quality interleaved data is scarce in specialized domains, potentially limiting base model performance in applications.  
This section describes our experiments using the VLM Dialog Games on video frames from a robotics manipulation domain where we test VLM success detection in object manipulation tasks.

\subsubsection{Dataset and Game Configuration}
\label{sec:exps-robotics-setup}

We use image frames from videos recorded in the ALOHA setup (A Low-cost Open-source Hardware System for Bimanual Teleoperation)~\citep{zhao2023learning}.
The images feature bimanual robotic arms performing \num{10} object manipulation tasks (e.g., putting objects in containers).
% 1) fold the dress, 2) put the bowl into the drying rack, 3) unbuckle the belt, 4) open the drawer, 5) put the legos into the lego bag, 6) put the cheese in the basket, 7) remove the gears from the board, 8) put banana into the drying rack, 9) close the green trash bin lid, 10) put the giraffe in the rack.
We use images captured from an overhead camera perspective. 
Our dataset comprises \num{20} episodes (both successful and unsuccessful) for each of the \num{10} tasks, totaling \num{200} episodes. 
We limit the game to only two images randomly sampled from the \textit{same task} execution as the success rate drops significantly with more images.
We generate \num{1000} games for each of the \num{10} tasks by sampling different frame combinations.
Figure \ref{fig:game-example-robotics-success} shows a dialog game example.

\subsubsection{Evaluation Task: Success Detection in Robotics}

To evaluate the impact of our method on robotic task understanding, we measure the model's ability to perform success detection. 
Accurate success detection is critical for various robotics applications, including policy training, evaluation, and data curating.
We evaluate success detection on the final frame of video episodes, treating it as a zero-shot VQA task~\citep{du23successvqa}. 
The model is presented with the final frame image and a textual description of the intended task (e.g., "open the drawer") and it is prompted with a question on task completion (e.g., "Is the drawer open?").
We report the accuracy of the model's yes/no responses.

\subsubsection{Baselines}
\label{sec:exps-robotics-baselines}

To isolate the specific contribution of the VLM Dialog Games, we compare our method against the original Gemini 1.5 Flash model and several other baselines.

\paragraph{Description Supervised Fine-Tuning (SFT-Description)}
Since our dialog games design utilizes task descriptions for each robotic episode, we include a baseline fine-tuned directly on image-description pairs.
This baseline "SFT-Description" helps determine if simply exposing the model to paired image and task descriptions from the target domain is sufficient for improvement.

\paragraph{Self-Improving Question Answering (Self-QA)}
This baseline explores an alternative self-improvement approach based on question answering similar to the approach of~\citet{luu2024questioning} (without the image captioning).
The model performs two tasks:
\begin{enumerate}
    \item \textbf{Question generation:} Given an image from the ALOHA dataset, the model generates a question about the scene.
    \item \textbf{Answer generation:} Given an image and a generated question, the model provides an answer.
\end{enumerate}

The prompts used for these tasks are detailed in Appendix~\ref{qa-prompts}.
This baseline tests whether a simpler self-improvement loop without the goal-oriented dialog structure can achieve similar results.

\paragraph{VLM Dialog Games (Answers Only)}
Our fine-tuning data includes both Describer and Guesser perspectives. 
Since the final success detection task closely resembles the Describer's role of answering questions, we include a baseline fine-tuned only on the datapoints from the Describer.
This isolates the contribution of the Guesser's questions to the overall improvement.

\begin{table*}[t]
    \centering
    \caption{\textbf{Success detection accuracy on the ALOHA dataset}, averaged across \num{10} tasks.  Fine-tuning on dialog game data outperforms the initial checkpoint and the other baselines. Iterative refinement further improves performance.}
    \vspace{5mm}
    \begin{tabular}{l|r|r}
      Model   & Game Success & Success Detection Accuracy \\
      \midrule
      Gemini 1.5 Flash & 14.39\% & 56.5\% \\
      VLM Dialog Games (round 1) & 40.15\% (+25.76\%) & 69.5\% (+13.0\%) \\
      VLM Dialog Games (round 2) & 53.74\% (+39.35\%) & 73.0\% (+16.5\%) \\
      \midrule
      SFT-Description & N/A & 65.0\% (~~+8.5\%) \\
      Self-QA & N/A & 67.0\% (+10.5\%) \\
      VLM Dialog Games (answers only) & 17.92\% (+3.53\%) & 68\% (+12.5)\% \\
    \end{tabular}
    \label{tab:robotics_result}
\end{table*}

\paragraph{Multiple Rounds of Self-Improvement}
We expect fine-tuning to improve the model's performance in subsequent games.
Thus, we use the improved model to generates a new, higher-quality dataset of synthetic dialogs. 
These dialogs are filtered and used to fine-tune the next iteration of the model, a process we refer to as "round 1" and "round 2".

In all cases we generate datasets with a size equivalent to the corresponding dialog game dataset and use it to fine-tune the Gemini 1.5 Flash model with the same settings.


\subsubsection{Results}
\label{sec:exps-robotics-results}

Table~\ref{tab:robotics_result} presents the success detection accuracy and game success rates averaged across the $10$ robotic tasks.
The initial Gemini 1.5 Flash model achieves a success detection accuracy of $56.5$\% on this highly specialised domain, only slightly above chance. 
Both the SFT-Description and Self-QA baselines improve upon this, demonstrating the benefit of domain-specific fine-tuning ($65.0$\% and $67.0$\% accuracy, respectively).

However, fine-tuning on a single round of dialog game data (VLM Dialog Games (round 1)) yields a larger improvement, achieving a success detection accuracy of $69.5$\% surpassing the baseline Self-QA by $2.5$\%.
Interestingly, although the VLM received no explicit instructions for success detection, the need to distinguish between frames from the \textit{same} task type lead it to focus on the task progression.
In contrast, the Self-QA method primarily generated object-related questions (see Appendix~\ref{sec:qa-examples} for examples).

Importantly, this initial round of dialog game fine-tuning also substantially increases the game success rate, from $14.39$\% to $40.15$\%, thus enabling further improvement.
We performed a second round of fine-tuning (VLM Dialog Games (round 2)), using data generated by the round 1 model.
This further boosted both the game success rate (to $53.74$\%) and the success detection accuracy (to $73.0$\%), a $16.5$\% absolute improvement over the original base model.

The VLM Dialog Games (answers only) baseline, which uses only the Describer's answers from the dialog games, achieves a success detection accuracy comparable to VLM Dialog Games (round 1). 
However, its game success rate remains comparatively low ($17.92$\%) and does not enable further iterative improvement.
This suggests that while the Describer's answers are sufficient for improving success detection, the Guesser's questions play a crucial role in improving the model's ability to play the dialog game effectively, which is necessary for continued self-improvement.

To conclude, our dialog game framework enables significant adaptation to specialized tasks like robotic success detection, where standard VLM pre-training may be less effective due to the lack of the domain-specific data.
Crucially, this self-improvement is achieved with minimal task-specific supervision, requiring only video episodes to guide the dialog generation.

%!TEX root =  main.tex
\section{Conclusions}

We explored the offline-to-online learning problem within the multi-armed bandit framework. This problem involves starting with historical, offline data and then improving performance through online interactions. We proposed that a natural way to evaluate algorithm performance in this setting was to compare against the logging policy in short-horizon scenarios, where there was limited opportunity for effective exploration, and against the optimal arm in long-horizon settings, where accumulated data allowed for more informed decision-making. These two objectives are inherently competing, and the distinction between what constituted a short or long horizon depended on the specific instance, which is the central challenge we addressed.

To address this, we introduced a novel algorithm, \algoname, designed to dynamically balance the benefits of the Lower Confidence Bound (\alglcb) and Upper Confidence Bound (\algucb) algorithms. \algoname was shown to adapt seamlessly across different conditions without prior knowledge of whether to prioritize exploration or exploitation, maintaining robust performance across a range of scenarios. %Our theoretical analysis demonstrated that \algoname achieved near-optimal performance relative to the better of \alglcb and \algucb across varying offline-to-online transitions, underscoring its ability to adapt effectively across different offline-to-online learning scenarios.

Our experimental results further supported these findings. Through evaluations on both synthetic and real-world datasets, \algoname consistently demonstrated strong performance across different horizon lengths and problem instances. The experiments highlighted how \algoname effectively interpolated between the strengths of \alglcb and \algucb, confirming its robustness and adaptability in practice.

Overall, our work bridges a critical gap in offline-to-online learning and offers a robust, adaptive approach that we hope will inspire continued exploration in this evolving field. In particular, we believe that the ideas underlying \algoname can extend naturally to more complex settings, such as contextual bandits, reinforcement learning, and nonstationary environments, which reflect more practical scenarios closer to real-world applications.
\section{Compliance with Ethical Standards}
This research was conducted retrospectively using human data under an approved IRB protocol.


\bibliographystyle{IEEEbib}
\bibliography{refs}

\end{document}
