\section{Conclusion and Discussion}
In this paper, we answer the question ``when are diffusion priors helpful for sparse reconstruction?'' by comparing classical priors used in practice with state-of-the-art diffusion priors for different number of projections and metrics.
First, we find that classical priors are superior to diffusion priors when the number of projections is ``sufficient''.
Second, we find that diffusion priors can capture a large amount of detail with very few observations, significantly outperforming classical priors. 
However, they fall short of capturing all details, even with many observations.
Finally, we find that the performance of downstream metrics plateau after few ($\approx$10-15) projections for diffusion priors.

One possible explanation for the performance saturation of diffusion priors is guided diffusion with naive gradient descent may not converge to the optimal solution.
While the unconditional model steers the gradient toward the data space, the gradient of the external loss (i.e., the mean square error between predicted projections and ground truth projections) may point away from the data space~\cite{guo2024gradient}.
The non-optimal convergence may explain why, even with more projections, $IQR_{5,95}$ remains constant because the guidance process eventually becomes ``stuck'' balancing the steering of the diffusion prior and minimizing the external loss.
Therefore, when the number of projections is high, they still fail to capture the low-level features like vasculature.
A possible solution could be adaptive fine-tuning \cite{guo2024gradient}.
% We suspect this is due to the more complex geometry of the diffusion prior, which contrasts with the simpler ``diamond-shaped'' $L_1$ and circular'' $L_2$ prior constraint sets (Fig. ~\ref{fig:prior}). 
% The irregular structure of the diffusion prior may make linear projections suboptimal, potentially leading to multiple intersections with the prior's boundary.
% This could result in inefficiencies in the optimization process and the inability to reconstruct volumes perfectly.
% This may be the reason for the wide and constant interquantile range for diffusion priors at higher number of projections.
% Further investigation into the specific geometric properties of the diffusion prior and their impact on the projections is warranted.
% \begin{figure}[t]
%     \centering
%     \label{fig:prior}
%     \includegraphics[width=\linewidth]{figs/fig3_priors.png}
%     \caption{\textbf{Irregular structure of the diffusion prior ball may make linear projections suboptimal.} We show classical and diffusion prior ``balls''. Diffusion priors are complexly shaped, while classical priors have a more consistent structure.  {\color{red} Not sure if people know what a ``prior ball'' even means. What is the takeaway when comparing L1/L2 vs diffusion? Is it good or bad?}}
% \end{figure}

\textbf{Clinical Implications. }While diffusion priors may not capture low-level details well, we demonstrate the potential of diffusion priors for accurate assessment of thoracic fat quantity.
% , distribution, and quality (as measured by attenuation distribution). 
Using extremely few projections significantly reduces the radiation exposure required for imaging, lowering the dose by up to 97\% compared with full projection and by up to 75\% compared to classical priors. 
% This work contributes to the evolving landscape of AI-enhanced medical imaging, showing potential for improving clinical practice through enhanced image reconstruction with minimal input data.
The reduction in radiation is particularly advantageous for individuals who require repeated imaging or those with obesity, where standard imaging protocols typically involve higher radiation doses.
The enhanced performance of low-dose CT scalability makes it a viable solution for large-scale studies and clinical applications where minimizing radiation is critical. 
Fixed planar X-ray systems could be employed in lieu of expensive CT machines, making diffusion priors especially valuable in low-resource settings where access to advanced imaging technologies is limited.
% While our results demonstrate clear advantages in radiation reduction, future work should explore the potential for further optimization in 3-D reconstructions and validation across broader populations. 
% By integrating such approaches into routine clinical practice, we can improve patient safety while maintaining the accuracy of important cardiometabolic assessments.