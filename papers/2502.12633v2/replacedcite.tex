\section{Related Work}
\subsection{Intelligent Tutoring Systems}
Research has demonstrated the effectiveness of Intelligent Tutoring Systems (ITS) in improving student engagement and learning efficacy. Notably, ____ emphasizes that ITS provide personalized instruction can lead to substantial improvements in student performance. This finding is supported by contemporary studies, such as ____, which confirm that ITS can significantly enhance learning outcomes when compared to traditional classroom settings. A meta-analysis by ____ further highlights the effectiveness of ITS in education, showing considerable gains in students' problem-solving skills and knowledge retention.

Existing works design rule-based systems with human-crafted domain knowledge____, or data-driven approaches____ that perform supervised learning on a certain amount of human annotation, enabling naturalistic exchanges between students and the system. This boosts student motivation and self-regulated learning, contributing to a more inclusive and engaging educational environment____.

% References
% Bloom, B. S. (1984). "The 2 Sigma Problem: The Search for Methods of Group Instruction as Effective as One-on-One Tutoring." Educational Researcher, 13(6), 4-16.

% Stern, D., Doney, M. A., & Pritchard, R. (2021). "The Impact of Intelligent Tutoring on Student Learning: A Comprehensive Study." International Journal of Educational Technology in Higher Education, 18(1), 12. DOI: 10.1186/s41239-021-00259-y.

% Zhuang, Z., Zhang, S., & Lee, L. (2022). "A Meta-Analysis of Intelligent Tutoring Systems in STEM Education." Educational Technology Research and Development, 70(5), 1671-1707. DOI: 10.1007/s11423-021-10023-8.

% Recent advancements in machine learning and natural language processing have also improved the interaction quality within ITS, enabling more naturalistic exchanges between students and the system____. This not only aids personalization but also boosts student motivation and self-regulated learning, contributing to a more inclusive and engaging educational environment____.

% Miao, M., Sinha, M., & Woolf, B. P. (2023). "Natural Language Processing in Intelligent Tutoring Systems: Past, Present, and Future." User Modeling and User-Adapted Interaction, 33(1), 1-24. DOI: 10.1007/s11257-022-09381-9.

% Chen, X., Zhang, Y., & Koedinger, K. R. (2021). "Understanding Student Behavior in Intelligent Tutoring Systems: A Meta-Analysis." Computers & Education, 174, 104275. DOI: 10.1016/j.compedu.2021.104275.

\subsection{AI in Education}
Artificial Intelligence (AI) is transforming the educational landscape by enhancing learning experiences and personalizing educational journeys. AI technologies, such as machine learning, natural language processing, and data analytics, enable the development of adaptive learning systems, intelligent tutoring, and automated administrative tasks____. These innovations not only aid educators in identifying student needs but also create tailored learning paths that can improve engagement and outcomes.

Recent large language models show strong potential to build dialogue tutors with less data supervision and higher confidence____. ITS can be further improved by integrating LLMs with pedagogical and learning science principles____. Additionally, recent works____ demonstrate the potential of LLMs for individual student modeling.