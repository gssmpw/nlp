\documentclass[journal]{IEEEtran}
% \ifCLASSINFOpdf
% \else
% \fi
\usepackage{arydshln}
\usepackage{amsmath}
\usepackage{amssymb}
\usepackage{orcidlink}

\usepackage{booktabs}
\usepackage{multirow}
\usepackage{arydshln}
\usepackage{makecell}
\usepackage{nicematrix}
\usepackage{graphicx}
\usepackage{subfig}
\usepackage{float}
\usepackage{algorithm,algorithmic}
\usepackage{xcolor}
\usepackage{colortbl}
\definecolor{graybackground}{gray}{0.9}

\DeclareMathOperator*{\argmax}{arg\,max}
\hyphenation{op-tical net-works semi-conduc-tor}


\begin{document}
% \title{Network-Guided Bayesian Inference Framework for Joint ASR-Effective Speech Dereverberation and Blind RIR Estimation}
\title{VINP: Variational Bayesian Inference with Neural Speech Prior for Joint ASR-Effective Speech Dereverberation and Blind RIR Identification}

\author{Pengyu~Wang\orcidlink{0000-0001-5768-0658}\thanks{Pengyu Wang is with Zhejiang University and also with Westlake University, Hangzhou, China (e-mail: wangpengyu@westlake.edu.cn).},
Ying Fang\orcidlink{0009-0003-8767-1172}\thanks{Ying Fang is with Zhejiang University and also with Westlake University, Hangzhou, China (e-mail: fangying@westlake.edu.cn).},
and
Xiaofei~Li\orcidlink{0000-0003-0393-9905}
\thanks{Xiaofei Li is with the School of Engineering, Westlake University, and the Institute of Advanced Technology, Westlake Institute for Advanced Study, Hangzhou, China (e-mail: lixiaofei@westlake.edu.cn). }
\thanks{Xiaofei Li: corresponding author.}}
% \thanks{Manuscript received April 19, 2005; revised August 26, 2015.}}



% The paper headers
% \markboth{Journal of \LaTeX\ Class Files,~Vol.~14, No.~8, August~2015}%
% {Shell \MakeLowercase{\textit{et al.}}: Bare Demo of IEEEtran.cls for IEEE Journals}

\maketitle


\begin{abstract}
Reverberant speech, denoting the speech signal degraded by the process of reverberation, contains crucial knowledge of both anechoic source speech and room impulse response (RIR).
This work proposes a variational Bayesian inference (VBI) framework with neural speech prior (VINP) for joint speech dereverberation and blind RIR identification.
In VINP, a probabilistic signal model is constructed in the time-frequency (T-F) domain based on convolution transfer function (CTF) approximation. 
For the first time, we propose using an arbitrary discriminative dereverberation deep neural network (DNN) to predict the prior distribution of anechoic speech within a probabilistic model. 
By integrating both reverberant speech and the anechoic speech prior, VINP yields the maximum a posteriori (MAP) and maximum likelihood (ML) estimations of the anechoic speech spectrum and CTF filter, respectively.
After simple transformations, the waveforms of anechoic speech and RIR are estimated.
Moreover, VINP is effective for automatic speech recognition (ASR) systems, which sets it apart from most deep learning (DL)-based single-channel dereverberation approaches.
Experiments on single-channel speech dereverberation demonstrate that VINP reaches an advanced level in most metrics related to human perception and displays unquestionable state-of-the-art (SOTA) performance in ASR-related metrics.
For blind RIR identification, experiments indicate that VINP attains the SOTA level in blind estimation of reverberation time at 60~dB (RT60) and direct-to-reverberation ratio (DRR).
Codes and audio samples are available online\footnote{\url{https://github.com/Audio-WestlakeU/VINP}}.

\end{abstract}

% Note that keywords are not normally used for peerreview papers.
\begin{IEEEkeywords}
Speech dereverberation, reverberation impulse response identification, variational Bayesian inference, convolutive transfer function approximation, deep learning.
\end{IEEEkeywords}

\IEEEpeerreviewmaketitle



\section{Introduction}\label{sec:intro}

Augmented and Virtual Reality (AR/VR) has made significant advancements in recent years in terms of quality and affordability \cite{meta_motiv1, meta_motiv2}, through the use of machine learning algorithms.
One crucial algorithm is depth estimation from stereo sensors, which plays a vital role in spatial computing, hand tracking \cite{depth_gesture_recog}, and %spatial and
passthrough rendering.
Conventional DNN based stereo depth algorithms use expensive hierarchical processing \cite{stereonet, mobilestereonet, hitnet}, 
% resulting in high storage and processing energy
\je{which are challenging to accelerate on power constrained platforms}.
Increased resolutions and frame rates in newer systems further increase these costs \cite{near_sensor_distributed}.
Consequently, accelerating these networks on AR/VR devices while meeting real-time latency requirements and operating within the energy budget of limited battery devices presents a challenge.

In this work, we propose \textit{\projname{}}, an AR/VR stereo depth system comprising a flexible architecture for processing dynamic Regions of Interest (ROIs), and a comprehensive mapping methodology to optimize ROI processing for energy efficiency while maintaining real-time performance. Our contributions are as follows:

\begin{itemize}[leftmargin=*]
    % \item Analysis of stereo depth compute across variable ROI sizes;
    \item \je{The \textit{SteROI-D Algorithm}, which leverages Region-of-Interest (ROI) Sparsity to reduce per-frame depth extraction cost, and interleaved object detection and tracking to reduce ROI detection cost;}
    % \item A flexible ROI-based stereo depth processing system and a comprehensive mapping methodology to achieve near-ideal average inference energy by balancing resource availability for the largest ROIs while maintaining high efficiency for average ROIs;
    % \item Special Compute Unit (SCU) and multipacket routing to enable real time, high framerate operation; and
    \item \je{Special Compute Units (SCUs) and NoC Multipackets, to address compute and communication challenges in accelerating stereo depth networks;}
    % \item An evaluation of this end-to-end system design across a range of algorithm components and candidate datasets.
    \item \je{\textit{Binned Mapping}, a method for split online-offline algorithm mapping to enable efficient processing for a continuous range of ROI sizes; and}
    \item \je{A design space exploration framework for jointly optimizing an accelerator's SRAM allocation with it's Binned Mapping.}
\end{itemize}

To our knowledge, this is the first study to achieve ROI-based stereo depth processing.
This is also the first work to address variable ROI processing through a mapping-system co-design approach via an efficient design space exploration.
While prior work has exploited ROIs for eye tracking on AR devices, it was limited to a static architecture \cite{eyecod}.
Furthermore, although prior work has also proposed lightweight stereo depth systems for AR devices, they have not leveraged ROI sparsity \cite{tiefenrausch}.
% \section{Related Works}
\label{sec:related_works}
% Our 





% Introduce several speech enhancement and dereverberation methods based on DNN and Bayesian inference.

% VAE-NMF~\cite{baby2021speech}

% DVAE-VEM~\cite{bie2022unsupervised}

RVAE-EM~\cite{wang2024rvae}
The cubical increase in computation with input length addresses a major limitation observed in our previous publication RVAE-EM~\cite{wang2024rvae}.
% StoRM and its reference

% Introduce RIR and T60 estimation 

% BUDDy

% \cite{mateljan2003comparison}

% Fourier analyzer

% MLS-based system~\cite{rife1989transfer}

% swept sine system~\cite{farina2000simultaneous}


\section{Signal Model and Task Description}
\label{sec:problem_formulation}
In this section, we will introduce the signal model and define two tasks we aim to address: speech dereverberation and blind RIR identification.

\subsection{Signal Model}
% This signal model depicts the relationship between the anechoic source speech signal and the reverberant recording signal.

Considering the scenario of a single static speaker and stationary noise, the reverberant speech signal (observation) received by a distant microphone can be modeled in the time domain as
\begin{equation}
    \label{eq:signal_model_time}
    x(n)=h(n)*s(n)+w(n),
\end{equation}
where $*$ is the convolution operator, $n$ is the index of sampling points, $x(n)$ is the observation signal, $s(n)$ is the anechoic source speech signal, $h(n)$ is the RIR which describes a time-invariant linear filter, and $w(n)$ is the background additive noise.
Without loss of generality, we assume that the RIR begins with the impulse response of the direct-path propagation, followed with reflections and reverberation.

Analyzing and processing speech signals in the time domain poses significant challenges.
After performing short-time Fourier transform (STFT), 
according to the CTF approximation~\cite{talmon2009relative}, the observation model in the T-F domain becomes
\begin{equation}
    \label{eq:signal_model_stft}
    \begin{aligned}
        X(f,t)&\approx\sum_{l=0}^{L-1}H_l(f)S(f,t-l)+W(f,t) \\
        &=\mathbf{H}(f)\mathbf{S}(f,t)+W(f,t),\\
    \end{aligned}
\end{equation}
where $f$ and $t$ are the indices of frequency band and STFT frame, respectively; $L$ is the length of CTF filter; $X(f,t)$, $S(f,t)$, $W(f,t)$, and $H_l(f)$ are the complex-valued observation signal, source speech signal, noise signal, and CTF coefficient, respectively; 
$\mathbf{H}(f)=\left[H_{L-1}(f),\cdots,H_0(f)\right]\in\mathbb{C}^{1 \times L}$, $\mathbf{S}(f,t)=\left[S(f,t-L+1),\cdots,S(f,t)\right]^T\in \mathbb{C}^{L \times 1}$.
% Since there is no cross-band filter in the CTF approximation, the probabilistic graph model is independently built in each frequency band.
% In the following VBI procedure, these bands are also processed separately.
% Therefore, we will omit the frequency index $f$ in the signal model and VBI procedure.


% \subsection{Probabilistic Model}

Furthermore, the observation $X(f,t)$, anechoic source $S(f,t)$, and noise $W(f,t)$ are modeled as random signals.
We have the following assumptions regarding their distributions.

\begin{itemize}
\item Assumption 1: The anechoic source speech signal $S(f,t)$ follows a time-variant zero-mean complex-valued Gaussian distribution, while the noise signal $W(f,t)$ follows a time-invariant zero-mean complex-valued Gaussian distribution. 
Therefore, we have their prior distributions as
\begin{equation}
\left\{
\begin{aligned}
    &S(f,t)\sim\mathcal{CN}\left(0,\alpha^{-1}(f,t)\right)\\
    &W(f,t)\sim\mathcal{CN}\left(0,\delta^{-1}(f)\right),\\
\end{aligned}
\right.
\end{equation}
where $\alpha(f,t)$ and $\delta(f)$ are the precisions of the Gaussian distributions.
% The assumption of zero mean is based on the randomness of the phase.
% Assumption 1 also indicates that the noise signal is generally stationary.

\item Assumption~2: 
The anechoic source speech signal $S(f,t)$ and noise signal $W(f,t)$ are respectively independent for all T-F bins.
Defining $\mathbf{S}=[S(1,1),\cdots,S(1,T),\cdots,S(F,T)]^T\in \mathbb{C}^{1\times FT}$ and $\mathbf{W} = [W(1,1),\cdots,W(1,T),\cdots,W(F,T)]^T\in \mathbb{C}^{1\times FT}$, we have
\begin{equation}
\left\{
\begin{aligned}
    &\mathbf{S} \sim \prod_{f=1}^F\prod_{t=1}^T p\left(S(f,t)\right)\\
    &\mathbf{W}\sim\prod_{f=1}^F\prod_{t=1}^T p\left(W(f,t)\right).\\
\end{aligned}
\right.
\end{equation}

\item Assumption~3:
The anechoic source speech signal $\mathbf{S}$ and noise signal $\mathbf{W}$ are independent of each other, which means
\begin{equation}
    \mathbf{S},\mathbf{W} \sim p(\mathbf{S})p(\mathbf{W}).
\end{equation}

\end{itemize}

With all the aforementioned assumptions, the conditional probability of the observation signal can be expressed as follows.
\begin{equation}
\left\{
\begin{aligned}
    &X(f,t)|\mathbf{S}(f,t)\sim\mathcal{CN}\left(\mathbf{H}(f)\mathbf{S}(f,t),\delta^{-1}(f)\right) \\
    &\mathbf{X}|\mathbf{S}\sim\prod_{f=1}^F\prod_{t=1}^T p\left(X(f,t)|\mathbf{S}(f,t)\right).\\
\end{aligned}
\right.
\end{equation}
The probabilistic graphical model is summarized in Fig. \ref{fig:prob_model}.
\begin{figure}[t]
    \centering
    \includegraphics[width=0.32\textwidth]{figs/prob_model.pdf}
    \caption{Probabilistic graphical generation model of reverberant microphone recording (observation).}
    \label{fig:prob_model}
\end{figure}



% \subsection{Signal Model of RIR}

% According to the time-domain model proposed by Polack et al.~\cite{polack1988transmission,jot1997analysis}, the RIR $h(n)$ is described as a realization of a non-stationary stochastic process
% \begin{equation}
%     h(n)=b(n)e^{- \beta n}, n \geq 0,
% \end{equation}
% where $b(n)$ is a centered stationary Gaussian noise with the variance of $\sigma^2_b$, $\beta$ is related to RT60 (marked as $T_{60}$) by
% \begin{equation}
%     20\log_{10}\left(e^{- \beta T_{60}}\right)=-60.
% \end{equation}
% Furthermore, the ensemble average energy envelope of the RIR in the time domain can be expressed as
% \begin{equation}
%     \left<h^2(n)\right>\approx \sigma_{b}^2 e^{-2\beta n}.
% \end{equation}
% 
% 
% In practice, given a measured RIR $\hat h(n)$, 

% Moreover, Schroeder’s integrated energy decay curve (EDC) is traditionally used to show how the energy level decreases after the sound source stops emitting sound.
% EDC is defined as the remaining energy in the RIR as~\cite{schroeder1965new}
% \begin{equation}
% \label{eq:edc}
%     \mathrm{EDC}(n)=\sum_{m=n}^{\infty}h^2(m).
% \end{equation}
% % RT60 is usually calculated by the slope of the logarithmic EDC curve through linear fitting.

\subsection{Task Description}

In this work, we aim to jointly complete speech dereverberation and blind RIR identification by solving all hidden variables and parameters in Fig. \ref{fig:prob_model} through VBI.

For speech dereverberation, we care about the anechoic spectrum, which is a hidden variable in our probabilistic graphical model.
We aim to get its maximum a posteriori (MAP) estimation given the reverberant microphone recording, written as
\begin{equation}
    \hat{\mathbf{S}}=\argmax\limits_{\mathbf{S}}p(\mathbf{S}|\mathbf{X}),
\end{equation}
where the posterior distribution of anechoic source signal can be represented according to the Bayes rule as 
\begin{equation}
    \label{eq:posterior}
    p(\mathbf{S}|\mathbf{X})=\frac{p(\mathbf{S})p(\mathbf{X}|\mathbf{S})}{\int p(\mathbf{S})p(\mathbf{X}|\mathbf{S}) \mathrm{d}\mathbf{S}}.
\end{equation}

% The CTF filter is a representation of RIR in the T-F domain.
For blind RIR identification, we care about the CTF filter, which is a model parameter in our probabilistic graphical model.
Defining the CTF filter for all frequency bands as $\mathbf{H}=\left[\mathbf{H}(1),\cdots,\mathbf{H}(F)\right]$, we aim to get its maximum likelihood (ML) estimation, written as
\begin{equation}
    \hat{\mathbf{H}}=\argmax\limits_{\mathbf{H}}p(\mathbf{S},\mathbf{X}).
\end{equation}
The CTF filter is a representation of RIR in the T-F domain.
We transform the CTF filter into the RIR waveform through a pseudo measurement process, a process that will be elaborated upon later.


% the RIR waveform $\hat h(n)$ can be recovered according to the CTF filter $\mathbf{H}$.

\section{Proposed Method}
\label{sec:prop}
In this work, we propose a novel framework named VINP for joint speech dereverberation and blind RIR identification.
We propose using an arbitrary discriminative dereverberation DNN to predict the prior distribution of anechoic speech from reverberant microphone recording and then applying VBI to analytically estimate the anechoic spectrum and the CTF filter.
After that, we use a pseudo measurement process to transform the CTF filter into the RIR waveform.
The overview of VINP is shown in Fig. \ref{fig:overview}.
\begin{figure}[t]
    \centering
    \includegraphics[width=0.65\linewidth]{figs/overview.pdf}
    \caption{Overview of VINP.}
    \label{fig:overview}
\end{figure}

\subsection{Prediction of Anechoic Speech Prior}

The direct-path speech signal, which is a scaled and delayed version of the anechoic source speech signal, is free from noise and reverberation as well.
To avoid estimating the arbitrary direct-path propagation delay and attenuation, instead of the actual source speech, we setup the direct-path speech as the source speech and our target signal, still denoted as $s(n)$ or $\mathbf{S}$.  Correspondingly, the RIR begins with the impulse response of the direct-path propagation.
% with normalized amplitude of 1.
% By doing this, the source speech signal is the same as the direct-path speech signal. 
% Consequently, the same identifier $\mathbf{S}$ is employed to represent both of them in the following statement.


In VINP, we consider the power spectrum of the direct-path speech signal as the variance of the anechoic speech prior~$p(\mathbf{S})$.
Then, in each T-F bin, the oracle estimation of precision $\alpha(f,t)$ is
\begin{equation}
\label{eq:alpha_hat}
    \alpha(f,t) = 1/{|S(f,t)|^{2}}.
\end{equation}
However, the oracle estimation is unavailable in practice.
% Given the existence of many advanced discriminative dereverberation DNNs, 
We propose integrating an arbitrary discriminative dereverberation DNN as an estimator of anechoic speech prior. Note that, in the proposed framework, we need to redesign the training loss function of the discriminative dereverberation DNNs.

Specifically, the discriminative DNN constructs a mapping from reverberant magnitude spectrum $|\mathbf{X}|$ to the anechoic magnitude spectrum $| \hat{\mathbf{S}}|$ as 
\begin{equation}
\label{eq:fdnn}
    |\hat{\mathbf{S}}|=f_{\mathrm{DNN}}\left(\left|\mathbf X\right|\right).
\end{equation}
Then Eq. (\ref{eq:alpha_hat}) becomes
\begin{equation}
\label{eq:alpha_hat2}
    \alpha(f,t) = 1/{|\hat S(f,t)|^{2}}.
\end{equation}

Regarding the loss function, we employ the average Kullback-Leibler (KL) divergence~\cite{kullback1951information} to measure the distance of estimated prior distribution $p(\hat{\mathbf{S}})$ and oracle prior distribution $p(\mathbf{S})$ as
\begin{equation}
\begin{aligned}
\mathcal{L} &= \mathrm{E}_{\mathrm{data}}\left[\frac{\mathrm{KL}\left(p(\hat{\mathbf{S}})||p(\mathbf{S})\right)}{FT}\right]\\
% &= \mathrm{E}_{\mathrm{data}}\left[\frac{\sum_{f=1}^F\sum_{t=1}^T\left[\ln\left({\frac{\alpha(f,t)}{\hat{\alpha}(f,t)}}\right)+{\frac{\hat{\alpha}(f,t)}{\alpha(f,t)}}-1\right]}{FT}\right]\\
&= \mathrm{E}_{\mathrm{data}}\left[\frac{\sum_{f=1}^F\sum_{t=1}^T\left[\ln\left({\frac{|S(f,t)|^2}{|\hat{S}(f,t)|^2}}\right)+{\frac{|\hat{S}(f,t)|^2}{|S(f,t)|^2}}-1\right]}{FT}\right].\\
\end{aligned}
\end{equation}
In practice, we use 
\begin{equation}
\begin{aligned}
        &\mathcal{L}\\ 
        &=\mathrm{E}_{\mathrm{data}}\left[\frac{\sum_{f=1}^F\sum_{t=1}^T\left[\ln\left({\frac{|S(f,t)|^2+\epsilon}{|\hat{S}(f,t)|^2+\epsilon}}\right)+{\frac{|\hat{S}(f,t)|^2+\epsilon}{|S(f,t)|^2+\epsilon}}-1\right]}{FT}\right]\\
\end{aligned}
\end{equation}
instead to avoid numerical instabilities, where $\epsilon$ is a small constant.
Such a loss function is different from that in regular discriminative dereverberation DNNs which directly predict the anechoic spectrum.



% Since VINP does not care about the complex and unpredictable phase of the estimated anechoic speech, 

% The output of DNN is real-valued, which means that it does not need to predict the complex and unpredictable phase of the anechoic speech.
% Therefore, the task of the DNN has been reduced, making the training easier.
% Moreover, 
Research indicates that the complex nonlinear operations of DNNs often lead to outputs with unpredictable artificial errors. 
While these errors do not significantly affect speech perceptual quality, their impact on back-end speech recognition applications remains uncertain. 
As a result, DL-based approaches may not improve ASR performance~\cite{iwamoto2024does,menne2019investigation,iwamoto22_interspeech}.
However, in our method, by regarding the DNN output as a prior distribution of anechoic speech and utilizing the subsequent CTF-based VBI stage to refine it, this problem can be alleviated, leading to better ASR performance.
Experiments in Section~\ref{sec:expset} will provide evidence to support this conclusion.

\subsection{Variational Bayesian Inference }
Given the estimated prior distribution of anechoic speech and the observed recording, the estimation of every hidden variable and parameter is carried out through variational Bayesian inference.


The posterior distribution $p(\mathbf{S}|\mathbf{X})$ is intractable due to the integral term in Eq.~(\ref{eq:posterior}).
Therefore, we turn to VBI, which is a powerful tool for resolving hierarchical probabilistic models~\cite{wang2022off,bianco2019machine}.
More specifically, we employ the variational expectation-maximization (VEM) algorithm, which provides a way for approximating the complex posterior distribution $p(\mathbf{S}|\mathbf{X})$ with a factored distribution $q(\mathbf{S})$ according to the mean-field theory (MFT) as
\begin{equation}
    p(\mathbf{S}|\mathbf{X}) \approx q(\mathbf{S})=\prod_{f=1}^{F}\prod_{t=1}^{T} q\left(S(f,t)\right).
\end{equation}
After factorization, VEM can estimate the posterior $q(\mathbf{S})$ and model parameters $\boldsymbol{\theta}=\left\{\delta(f)|_{f=1}^F,H(f,t)|_{f=1,t=1}^{F,T}\right\}$ by E-step and M-step respectively and iteratively as
\begin{equation}
    \label{eq:estposterior}
    \ln{q\left(S(f,t)\right)}=\left<\ln{p(\mathbf{S},\mathbf{X})}\right>_{\mathbf{S}\backslash S(f,t)}
\end{equation}
and
\begin{equation}
    \hat{\boldsymbol{\theta}}=\argmax\limits_{\boldsymbol{\theta}}\ln{p(\mathbf{S},\mathbf{X})},
\end{equation}
where $\backslash$ denotes the set subtraction and $\left<\cdot\right>$ denotes expectation.
Because the solution of such a probabilistic graphical model is an underdetermined problem,
we do not update the precision parameter $\alpha(f,t)$ during the VEM iterations to prevent a deterioration in prior quality.

The specific update formulae are as follows:

\subsubsection{E-step}
In this step, we update the posterior distribution of the anechoic spectrum given the observation and estimated model parameters.
Substitute the probabilistic model into Eq.~(\ref{eq:estposterior}), we have
\begin{equation}
    \begin{aligned}
        &\ln{q\left({S}(f,t)\right)} \\
        &=\left<\ln{p({\mathbf{S}},\mathbf{X})}\right>_{{\mathbf{S}}\backslash {S}(f,t)}\\
        &=
        \left<\ln{p\left({S}(f,t)\right)}\right>_{{\mathbf{S}}\backslash {S}(f,t)}\\
        &\quad+\left<\sum_{l=0}^{L-1}\ln{p\left(X(f,t+l)|{\mathbf{S}}(f,t+l)\right)}\right>_{{\mathbf{S}}\backslash {S}(f,t)}+c,
    \end{aligned}
\end{equation}
% \begin{equation}
%     \begin{aligned}
%         &\ln{q\left(\tilde{S}(t)\right)} \\
%         &=\left<\ln{p(\tilde{\mathbf{S}},\mathbf{X})}\right>_{\tilde{\mathbf{S}}\backslash \tilde{S}(t)}\\
%         & \overset{\tilde{S}(t)}{=}
%         \left<\ln{p\left(\tilde{S}(t)\right)}
%         +\sum_{l=0}^{L-1}\ln{p\left(\mathbf{X}(t+l)|\tilde{\mathbf{S}}(t+l)\right)}\right>_{\tilde{\mathbf{S}}\backslash \tilde{S}(t)}+C,
%     \end{aligned}
% \end{equation}
where $c$ is a constant term that is independent of ${S}(f,t)$, and
\begin{equation}
\left\{
\begin{aligned}
    &\ln{p\left({S}(f,t)\right)}=\ln{\alpha(f,t)}-\alpha(f,t)\left|{S}(f,t)\right|^2+c \\
    &\ln{p\left(X(f,t+l)|{\mathbf{S}}(f,t+l)\right)} \\
    &\quad = \ln{\delta(f)}-\delta(f)\left|X(f,t+l)-\mathbf{H}(f){\mathbf{S}}(f,t+l)\right|^2+c. \\ 
\end{aligned}
\right.
\end{equation}
% \begin{equation}
% \left\{
% \begin{aligned}
%     &\ln{p\left(\tilde{S}(t)\right)}=\ln{\alpha(t)}-\alpha(t)\left|\tilde{S}(t)\right|^2/c^2+C, \\
%     &\ln{p\left(X(t+l)|\tilde{\mathbf{S}}(t+l)\right)} \\
%     &\qquad = \ln{\delta}-\delta\left|X(t+l)-\mathbf{H}\tilde{\mathbf{S}}(t+l)/c\right|^2+C \\ 
% \end{aligned}
% \right.
% \end{equation}

According to the property of Gaussian distribution, the posterior distribution is also Gaussian, written as 
\begin{equation}
    q\left({S}(f,t)\right)=\mathcal{CN}\left({\mu}(f,t),{\gamma}^{-1}(f,t)\right),
\end{equation}
% \begin{equation}
%     q\left(\tilde{S}(t)\right)=\mathcal{CN}\left(\tilde{\mu}(t),\tilde{\gamma}^{-1}(t)\right),
% \end{equation}
whose precision and mean have close-form solutions as
\begin{equation}
\label{eq:mu_var}
\left\{
\begin{aligned}
    &{\gamma}(f,t)=\alpha(f,t)+\delta(f) ||\mathbf{H}(f)||_2^2 \\
    &{\mu}(f,t)={\gamma}^{-1}(f,t)\delta(f)\\
    &\quad\times\left[\sum_{l=0}^{L-1}H_l(f)^*\left[X(f,t+l)-\mathbf{H}_{\backslash l}(f)\hat{\boldsymbol{\mu}}(f,t+l)\right]\right],
\end{aligned}
\right.
\end{equation}
where $\mathbf{H}_{\backslash l}(f)$ is same as $\mathbf{H}(f)$ except that $H_l(f)$ is set to $0$, 
and $\hat{\boldsymbol{\mu}}(f,t+l)=\left[\hat\mu(f,t+l-L+1),\cdots,\hat\mu(f,t+l)\right]^T$ contains the estimates of means from the previous VEM iteration.

In order to make VEM converge more smoothly, we further apply an exponential moving average (EMA) to the estimates in Eq.~(\ref{eq:mu_var}) as
\begin{equation}
\label{eq:mu_var_sm}
\left\{
\begin{aligned}
    &\hat{\gamma}^{-1}(f,t)=\lambda\hat\gamma^{-1}_{\mathrm{pre}}(f,t)+(1-\lambda)\gamma^{-1}(f,t)\\
    &\hat{\mu}(f,t)=\lambda\hat{\mu}_{\mathrm{pre}}(f,t)+(1-\lambda){\mu}(f,t),
\end{aligned}
\right.
\end{equation}
where $\lambda$ is a smoothing factor, $\hat{\mu}_{\mathrm{pre}}(f,t)$ and $\hat\gamma^{-1}_{\mathrm{pre}}(f,t)$ are the estimates from the previous VEM iteration. 


The prior of anechoic speech narrows down the infinite number of possible solutions when decoupling the anechoic speech and the RIR, which is the key to achieving dereverberation~\cite{10638210}.
The mean of the posterior distribution is the MAP estimate of anechoic spectrum $\hat{\mathbf{S}}$.
It is also worth noticing that the E-step can be implemented in parallel across all T-F bins.

\subsubsection{M-step}
In this step, VEM updates the noise precision and CTF filter by maximizing the logarithmic joint probability of the anechoic speech and observation, which is
\begin{equation}
    \begin{aligned}
        &\ln{p\left({\mathbf{S}},\mathbf{X}\right)}\\
        &=
        \ln{p\left(\mathbf{X}|{\mathbf{S}}\right)} +c\\
        &= T\ln{\delta(f)}-\delta(f)\sum_{t=1}^{T}\left|X(f,t)-\mathbf{H}(f){\mathbf{S}}(f,t)\right|^2 +c, \\
    \end{aligned}
\end{equation}
where $c$ is a constant term that is independent of $\delta(f)$ and $\mathbf{H}(f)$.
Setting the first derivative with respect to the parameters to zero, the noise precision is updated as
\begin{equation}
\label{eq:est_delta}
\begin{aligned}
    &\hat{\delta}(f)\\
    &= T\bigg/\sum\limits_{t=1}^T\left[\left|X(f,t)\right|^2-2\mathrm{Re}\left\{X^*(f,t)\mathbf{H}(f)\left<{\mathbf{S}}(f,t)\right>\right\}\right.\\
    &\quad +\left.{\mathbf{H}(f)\left<{\mathbf{S}}(f,t){\mathbf{S}}^H(f,t)\right>\mathbf{H}^H(f)}\right],
    % \\&\frac{T}{\sum\limits_{t=1}^T\left[\left|X(t)\right|^2-2\mathrm{Re}\left\{X^*(t)\mathbf{H}\left<{\mathbf{S}}(t)\right>\right\}+{\mathbf{H}\left<{\mathbf{S}}(t){\mathbf{S}}^H(t)\right>\mathbf{H}^H}\right]}, \\
\end{aligned}
\end{equation}
and the CTF filter is updated as
\begin{equation}
\label{eq:est_H}
    \begin{aligned}
        &\hat{\mathbf{H}}(f) \\
        &= \left[\sum_{t=1}^{T}
        X(f,t)\left<{\mathbf{S}}^H(f,t)\right>\right]
        \left[\sum_{t=1}^{T}\left<{\mathbf{S}}(f,t){\mathbf{S}}^H(f,t)\right>\right]^{-1},
    \end{aligned}
\end{equation}
where
\begin{equation}
    \left<\mathbf{S}(f,t)\right>=\boldsymbol{\mu}(f,t)=\left[\mu(f,t-L+1),\cdots,\mu(f,t)\right]^T,
\end{equation}
and
\begin{equation}
\begin{aligned}
    &\left<\mathbf{S}(f,t)\mathbf{S}^H(f,t)\right>\\
    &=\boldsymbol{\mu}(f,t)\boldsymbol{\mu}^H(f,t)\\
    &\quad +\mathrm{diag}([\alpha^{-1}(f,t-L+1),\cdots,\alpha^{-1}(f,t)]),
\end{aligned}
\end{equation}
$\mathrm{diag}(\cdot)$ denotes the operation of constructing a diagonal matrix.
Just like the E-step, the M-step can also be implemented in parallel across all T-F bins.



\subsubsection{Initialization of VEM Parameters}



The initialization of parameters plays a crucial role in the convergence of VEM.
Before iteration, the mean and variance of the anechoic speech posterior $p(\mathbf{S})$ are set to zero and the power spectrum of reverberant recording, respectively. 
This means we have
\begin{equation}
\label{eq:init1}
\left\{
\begin{aligned}
    &\mu(f,t)=0\\
    &\alpha(f,t)=1/\left|X(f,t)\right|^{2}.\\
\end{aligned}
\right.
\end{equation}
The CTF coefficients in each frequency band are set to 0 except that the first coefficient is set to 1, which means
\begin{equation}
\label{eq:init2}
\left\{
\begin{aligned}
    &H_{0}(f)=1\\
    &H_{l\neq 0}(f)=0.\\
\end{aligned}
\right.
\end{equation}
Because even during speech activity, the short-term power spectral density of observation often decays to values that are representative of the noise power level~\cite{martin2001noise}, the initial variance of the additional noise is set to the minimum power in each frequency band, which means
\begin{equation}
\label{eq:init3}
    \delta(f) = \left[\min\limits_t\left(\left|X(f,t)\right|^2\right)\right]^{-1}.
\end{equation}


The VBI procedure is summarized in Algorithm~\ref{algo:VBI}.
The final outputs of VBI procedure are the complex-valued anechoic spectrum and CTF filter estimates.
Notice that in the proposed framework, the prior distribution of anechoic speech is not updated during iterations.
Therefore, the DNN only infers once.


\begin{algorithm}[H]
    \renewcommand{\algorithmicrequire}{\textbf{Input:}}
	\renewcommand{\algorithmicensure}{\textbf{Output:}}
	\caption{VBI procedure.}
    \label{algo:VBI}
    \begin{algorithmic}[1] % 控制是否有序号
        \REQUIRE Reverberant microphone recording $\mathbf{X}$; % input 的内容
	    \ENSURE Anechoic speech spectrum estimate $\hat{\mathbf{S}}$, CTF filter estimate $\hat{\mathbf{H}}$; % output 的内容

        % \STATE Setting the hyperparameters 
        \STATE Estimate prior distribution of anechoic speech using Eq. (\ref{eq:fdnn}) and Eq. (\ref{eq:alpha_hat2});
        \STATE Initialize VEM parameters using Eq. (\ref{eq:init1}), Eq. (\ref{eq:init2}) and Eq. (\ref{eq:init3});
        \REPEAT
            \STATE E-step: update the posterior distribution of anechoic speech using Eq. (\ref{eq:mu_var}) and Eq. (\ref{eq:mu_var_sm});
            \STATE M-step: update the parameter estimates of the signal model using Eq. (\ref{eq:est_delta}) and Eq. (\ref{eq:est_H});
        \UNTIL{Converge or reach the maximum number of iterations.}
    \end{algorithmic}
\end{algorithm}


\subsection{Transformation to Waveforms}
After the VBI procedure, both the anechoic spectrum and the CTF filter are estimated.
We need to further transform these T-F representations into waveforms.
By applying inverse STFT to the anechoic spectrum estimate, we can easily get the anechoic speech waveform.
For the estimation of RIR, we design a pseudo intrusive measurement process as follows.

A common method for intrusive measuring the RIR of an acoustical system is to apply a known excitation signal and measure the microphone recording~\cite{stan2002comparison}.
Playing an excitation signal $e(n)$ with a loudspeaker, the noiseless microphone recording $y(n)$ can be written as
\begin{equation}
    y(n) = h(n)*e(n),
\end{equation}
where $h(n)$ has the same meaning as in Eq.~(\ref{eq:signal_model_time}).
A commonly used logarithmic sine sweep excitation signal can be expressed as~\cite{stan2002comparison,farina2000simultaneous}
\begin{equation}
\label{eq:excitation}
    e(n)=\sin \left[\frac{N\omega_1}{\ln \left({\omega_2}/{\omega_1}\right)}\left(e^{n\ln \left({\omega_2}/{\omega_1}\right)/N}-1\right)\right],
\end{equation}
where $\omega_1$ is the initial radian frequency and $\omega_2$ is the final radian frequency of the sweep with duration $N$.
Through an ideal inverse filter $v(n)$, the excitation signal can be transformed into a Dirac's delta function $\delta(n)$, as
\begin{equation}
    e(n)*v(n)=\delta(n).
\end{equation}
For the logarithmic sine sweep excitation, the inverse filter $v(n)$ is an amplitude-modulated and time-reversed version of itself~\cite{stan2002comparison,farina2000simultaneous}.
The RIR can be estimated by convolving the measurement $y(n)$ with the inverse filter $v(n)$ as
\begin{equation}
    h(n)=y(n)*v(n).
\end{equation}


% For blind RIR estimation, the exact excitation signal $e(n)$ and the corresponding measurement $y(n)$ are not available.
% Conversely, a reverberant speech recording $x(n)$ is given as an alternative.
% In our approach, the recording $x(n)$ is firstly enhanced by VBI to estimate the CTF filter $\hat{\mathbf{H}}$ in the T-F domain.
In our approach, the excitation signal is convoluted by the CTF estimates (along the time dimension) to get a pseudo measurement $\tilde{Y}(f,t)$ in the T-F domain as
\begin{equation}
\label{eq:pseudo_measure}
\tilde{Y}(f,t)=\hat{\mathbf{H}}(f)\mathbf{E}(f,t),
\end{equation}
where $\mathbf{E}(f,t)=\left[E(f,t-L+1),\cdots,E(f,t)\right]^T\in \mathbb{C}^{L \times 1}$, $\tilde{Y}(f,t)$ and $E(f,t)$ are the STFT coefficient of $\tilde{y}(n)$ and $e(n)$, respectively.
Finally, after applying inverse STFT to $\tilde{Y}(f,t)$, we use the pseudo measurement $\tilde{y}(n)$ to estimate the RIR waveform by
\begin{equation}
\label{eq:inverse_filtering}
    \hat{h}(n)=\tilde{y}(n)*v(n).
\end{equation}
The transformation from the CTF filter to RIR waveform is summarized in Algorithm~\ref{algo:RIR}.
\begin{algorithm}[H]
    \renewcommand{\algorithmicrequire}{\textbf{Input:}}
	\renewcommand{\algorithmicensure}{\textbf{Output:}}
	\caption{Transformation from CTF to RIR.}
    \label{algo:RIR}
    \begin{algorithmic}[1] % 控制是否有序号
        \REQUIRE CTF filter estimate $\hat{\mathbf{H}}$; % input 的内容
	    \ENSURE RIR waveform estimate $\hat{h}(n)$; % output 的内容
        % \STATE Estimate the CTF filter $\hat{\mathbf{H}}$ using Algorithm \ref{algo:VBI};
        \STATE Pick a pair of excitation signal $e(n)$ and its inverse filter $v(n)$;
        \STATE Build a pseudo measurement signal $\tilde{y}(n)$ using Eq. (\ref{eq:pseudo_measure});
        \STATE Estimate RIR $\hat{h}(n)$ by inverse filtering as Eq. (\ref{eq:inverse_filtering}).

    \end{algorithmic}
\end{algorithm}


% Owing to the length limitation of the CTF filter, the extremely late portion of the tail of the measured RIR is modeled as additive noise.
% However, the energy of the clipped part is negligible and there is essentially no influence on the estimation of acoustic parameters such as RT60 and DRR.
% In fact, the clipping of RIR is widely used for noise reduction in RT60 estimation~\cite{lundeby1995uncertainties,faiget1998optimization,guski2014comparison}.

% 由于CTF长度有限,非常晚期的混响是建模在噪声中的,所估计的RIR在时间上是被截断的。根据RIR的模型,在EDC曲线上加了一个常数。对T60的估计没有影响。此外在存在测量噪声的情况下估计RT60时本来就会在计算EDC时不考虑低于给定阈值样本的操作\cite{lundeby1995uncertainties,faiget1998optimization,guski2014comparison},因此我们使用重建的RIR进行T60估计相当于设置了一个较高的噪声门限。
\section{Experiments}
\label{sec:expset}


\subsection{Datasets}
VINP is designed for both speech dereverberation and blind RIR identification.
These two tasks share the same DNN for speech prior distribution prediction.
Therefore, we use a single training set and two different test sets.
% \subsubsection{{Training and validation sets}}
% Both these two sets are generated by convolving clean speech signals with simulated RIRs and then adding stationary noise.
\subsubsection{{Training Set}}
The training set is composed of 200 hours of high-quality English speech utterances from the corpora of EARS~\cite{richter2024ears}, DNS Challenge~\cite{dubey2023icassp}, and VCTK~\cite{valentini2016investigating}. 
All speech utterances are downsampled to 16 kHz if necessary. 
We simulate 100,000 pairs of reverberant and direct-path RIRs using the gpuRIR toolbox~\cite{diaz2021gpurir}.
The simulated speaker and microphone are randomly placed in rooms with dimensions randomly selected within a range of 3 m to 15 m (for length and width) and 2.5 m to 6 m (for height).
The minimum distance between the speaker/microphone and the wall is 1 m. 
Reverberant RIRs have RT60 values uniformly selected within the range of 0.2 s to 1.5 s. 
Direct-path RIRs are generated using the same geometric parameters as the reverberant ones but with an absorption coefficient of 0.99.
Noise recordings from the Noise92 corpus and the training set from the REVERB Challenge~\cite{kinoshita2013reverb} are employed.
The signal-to-noise ratio (SNR) is randomly selected within the range of 5 dB to 20 dB.


% In the validation set, we use 1.48 hours of clean speech from the 'si\_et\_05' subset in WSJ0 corpus~\cite{paul1992design}, 5,000 generated RIRs and the stationary noise from the test set of the REVERB challenge~\cite{kinoshita2013reverb}. 
% The RIR simulation settings for the validation set are the same as those for the training set.



\subsubsection{{Test Set for Speech Dereverberation}}
For speech dereverberation, we utilize the official single-channel test set from the REVERB Challenge~\cite{kinoshita2013reverb}, which includes both simulated data (marked as 'SimData') and real recordings (marked as 'RealData').

In SimData, there exist six distinct reverberation conditions: three room volumes (small, medium, and large), and two distances between the speaker and the microphone (50 cm and 200 cm). The RT60 values are approximately 0.25 s, 0.5 s, and 0.7 s. 
The noise is stationary background noise, mainly generated by air conditioning systems. 
SimData has a SNR of 20 dB.

RealData consists of utterances spoken by human speakers in a noisy and reverberant meeting room. It includes two reverberation conditions: one room and two distances between the speaker and the microphone array (approximately 100 cm and 250 cm). The RT60 is about 0.7 s.

\subsubsection{{Test Set for Blind RIR Identification}}
A test set named 'SimACE' is constructed to evaluate blind RIR estimation.
In SimACE, microphone signals are simulated by convolving the clean speech from the 'si\_et\_05' subset in WSJ0 corpus~\cite{paul1992design} with the downsampled recorded RIRs from the 'Single' subset in ACE Challenge~\cite{eaton2016estimation}, and adding noise from the test set in REVERB Challenge~\cite{kinoshita2013reverb}.
The minimum and maximum RT60s are 0.332~s and 1.22 s, respectively.
More details about the RIRs can be found in~\cite{eaton2016estimation}.
We create the SimACE test set because the RT60 labels of the RIRs in SimACE are more accurate than those in REVERB Challenge.
SimACE has a SNR of 20 dB.

\subsection{Implementation of VINP}
\subsubsection{Data Representation}
% VINP focuses on speech recordings sampled at 16kHz. 
Before feeding the speech into VINP, the reverberant waveform is normalized by its maximum absolute value. 
Subsequently, the utterance is transformed using STFT with a Hann window of 512 samples (32~ms) and a hop length of 128 samples (8~ms). 

\begin{figure}[H]
    \centering
    \subfloat[VINP-TCN+SA+S]{
        \centering
        {\includegraphics[width=0.45\linewidth]{figs/TCNSAS.pdf}}
    }
    \subfloat[VINP-oSpatialNet]{
        \centering
        {\includegraphics[width=0.45\linewidth]{figs/mOSPN.pdf}}
    }\hfill
    \caption{DNN architectures in VINP.}
    \label{fig:DNNarch}
\end{figure}

\subsubsection{DNN Architecture}
VINP is able to employ any discriminative dereverberation DNNs to predict the prior distribution of anechoic speech.
In this paper, we evaluate with the network proposed in TCN+SA+S~\cite{zhao2020monaural} and a modified network of the Mamba version of oSpatialNet~\cite{10570301}, resulting in two different versions marked as \textbf{'VINP-TCN+SA+S'} and \textbf{'VINP-oSpatialNet'}, respectively.
For both versions, the input and output of DNNs are the 10-based logarithmic magnitude spectra.
To avoid numerical issues, a small constant $10^{-8}$ is added to the magnitude spectra before taking the logarithm.

The DNN architecture in VINP-TCN+SA+S is composed of temporal convolutional networks (TCNs) and self-attention layers, as shown in Fig. \ref{fig:DNNarch}.
We use the same settings as the original TCN+SA+S~\cite {zhao2020monaural}, except that there is no activation function after the output layer and we do not use dropout.

The DNN architecture in VINP-oSpatialNet is derived from oSpatialNet~\cite{10570301}, as depicted in Fig. \ref{fig:DNNarch}.
A two-dimensional convolution layer with a kernel size of $3 \times 3$  is employed to expand the input to 96 dimensions.
% The stacked cross-band block and narrow-band block 
The remaining modules are the same as those in the original oSpatialNet. 
Except that, for offline processing, we replace the second forward Mamba layer in the narrow-band block with a backward Mamba layer by simply reversing the input and output along the time dimension.
% Subsequently, a linear layer is utilized to
% reduce the dimension to match the output.


% \begin{figure}[H]
%     \centering
%     \includegraphics[width=0.6\linewidth]{figs/mOSPN.pdf}
%     \caption{DNN architecture of NeGI-mOSPN.}
%     \label{fig:mOSPN}
% \end{figure}

\subsubsection{Training Configuration}
For DNN training, we set the hyperparameter in the loss function to $\epsilon=0.0001$.
The speech utterances are segmented into 3 s.
The batch size of VINP+TCN+SA+S and VINP-oSpatialNet are set to 16 and 4, respectively.
The AdamW optimizer~\cite{loshchilov2017decoupled} with an initial learning rate of 0.001 is employed.
The learning rate exponentially decays with $\mathrm{lr} \leftarrow{} 0.001\times0.97^{\mathrm{epoch}}$ and $\mathrm{lr} \leftarrow{} 0.001\times0.9^{\mathrm{epoch}}$ in VINP-TCN+SA+S and VINP-oSpatialNet, respectively.
Gradient clipping is applied with a L2-norm threshold of 10.
The training is carried out for 800,000 steps in total.
We average the model weights of the best three epochs as the final model.

\subsubsection{VBI Settings}
The length of the CTF filter is set to $L=30$. The fixed smoothing factor $\lambda$ in Eq. (\ref{eq:mu_var_sm}) is set to 0.7 to obtain a stable result.
In rare cases, if the likelihood of the complete data decreases, the VBI iteration will be stopped prematurely.
Since the fundamental frequency of human speech is always higher than 85 Hz~\cite{howard2007voice}, we ignore the three lowest frequency bands and set their coefficients to zero to avoid the effect of extremely low SNR in these bands. 
Therefore, VBI processes a total of 254 frequency bands.


\subsubsection{Pseudo Excitation Signal}
We use a logarithmic sine sweep signal with a frequency range of 100 Hz to 8000 Hz and a duration of 5 s as the pseudo excitation signal.
The formulae for the excitation signal and the corresponding inverse filter can be found in Eq. (\ref{eq:excitation}) and \cite{stan2002comparison,farina2000simultaneous}.

\subsubsection{RT60 and DRR Estimation}
RT60 and DRR are key acoustic parameters that feature the characteristic of RIR, which are thus used in this work for evaluating the accuracy of RIR identification. RT60 and DRR can be directly calculated from RIR.
% Here, we introduce our implementation.

RT60 is the time required for the sound energy in an enclosure to decay by 60 dB after the sound source stops.
% In practice, waiting for the sound energy to drop by 60dB completely may not be a good choice.
% Since the fact that sound energy decays exponentially over time, a relatively smaller range of energy decay is often measured, and then RT60 is estimated through a certain proportional relationship.
Given the RIR waveform, Schroeder’s integrated energy decay curve (EDC) is calculated as~\cite{schroeder1965new}
\begin{equation}
\label{eq:edc}
    \mathrm{EDC}(n)=\sum_{m=n}^{\infty}h^2(m).
\end{equation}
Since the fact that sound energy decays exponentially over time, a linear fitting is applied to a portion of logarithmic EDC, and the slope is utilized to calculate RT60.
During this process, the key to linear fitting lies in the heuristic strategy of selecting the fitting range.
% This strategy is usually heuristic.
% In this work, we conduct a traverse linear fitting operation on EDC.
% Specifically, 
In this work, the starting point of linear fitting is selected within the range of a time delay from 20 ms to 50 ms at the direct-path sample, which represents the sample corresponding to the maximum value of the RIR. 
Meanwhile, the ending point is established at a 5 dB attenuation relative to the starting point.
We fit all intervals that meet the conditions and apply the result with the maximum absolute Pearson correlation coefficient to the estimation process of RT60 as
% Subsequently, the fitting result with the maximum absolute Pearson correlation coefficient is applied to the estimation process of RT60, as
\begin{equation}
    T_{60}=-60/k,
\end{equation}
where $k$ is the slope of the fitted line.

DRR refers to the ratio between the energy of the direct-path sound and the reverberant sound.
In this work, the DRR is defined as
\begin{equation}
    \mathrm{DRR}=10\log_{10}\frac{\sum_{n=n_d-\Delta n_d}^{n_d+\Delta n_d}h^2(n)}{\sum_{n=0}^{n_d-\Delta n_d}h^2(n)+\sum_{n=n_d+\Delta n_d}^{\infty}h^2(n)},
\end{equation}
where the direct-path signal arrives at the $n_d$th sample, and $\Delta n_d$ is the additional sample spread for the direct-path response which typically corresponds to 2.5 ms~\cite{eaton2016estimation}.








\subsection{Comparison Methods}
\subsubsection{{Speech Dereverberation}}
We compare VINP with various advanced dereverberation approaches including GWPE~\cite{yoshioka2012generalization},
SkipConvNet~\cite{kothapally2020skipconvnet}\footnote{\url{https://github.com/zehuachenImperial/SkipConvNet}}, 
CMGAN~\cite{abdulatif2024cmgan}\footnote{\url{https://github.com/ruizhecao96/CMGAN}},
and StoRM~\cite{lemercier2023storm}\footnote{\url{https://github.com/sp-uhh/storm}}.
In addition, to demonstrate the effect of VINP, the DNNs in VINP-TCN+SA+S and VINP-oSpatialNet are also compared, i.e. TCN+SA+S~\cite{zhao2020monaural} and oSpatialNet~\cite{10570301}.
In TCN+SA+S, we use the recommended Griffin-Lim's iterative algorithm~\cite{griffin1984signal} to restore the phase spectrum. 
In oSpatialNet, for a fair comparison, the backward Mamba is also applied to oSpatialNet.
Different from VINP-oSpatialNet, the oSpatialNet employed for comparison processes the complex-valued speech spectrum as the original paper~\cite{10570301}.
And this version is marked as 'oSpatialNet*'. 





% In addition, we also compare with the speech dereverberation networks that used in the proposed framework, i.e. TCN+SA+S~\cite{zhao2020monaural} and oSpatialNet~\cite{10570301}, following the configurations of their original papers. Except that, in TCN+SA+S, we use Griffin-Lim's iterative algorithm~\cite{griffin1984signal} to restore the phase spectrum. 
% For fair comparison, the backward Mamba is also applied for oSpatialNet, and this version is marked as 'oSpatialNet*'. 


For all comparison methods, we use their recommended configuration and official codes (if available).
All DL-based approaches are trained from scratch on the same training set.
GWPE is implemented using the NaraWPE python package~\cite{Drude2018NaraWPE}\footnote{\url{https://github.com/fgnt/nara_wpe}}.

% Additionally, the DNN in VINP-oSpatialNet without VBI is also compared, in which we expand the number of channels in the input and output layers to 2. 
% Both the input and output are set to be the concatenated real and imaginary parts of the spectrum.
% And the negative scale-invariant signal-to-distortion ratio (SISDR) of speech waveform is adopted as the loss function.
% The DNN architecture is marked as 'oSpatialNet-mamba*'.

The number of parameters and the multiply-accumulate operations per second (MACs, G/s) of the DL-based approaches are shown in Table \ref{tab:para_mac}.
Additionally, under the same settings of STFT and CTF length, a comparison of MACs per second and per iteration between the VEM algorithm in VINP and the EM algorithm in our previous work RVAE-EM~\cite{wang2024rvae} with regard to speech length is presented in Fig. \ref{fig:macs}.
The asymptotic complexity of the VBI procedure in VINP is $O(FTL^2)$, indicating a linear growth with respect to the speech length, and its MACs is approximately a constant value of 0.27 G/s. 
In contrast, the asymptotic complexity of the EM algorithm in RVAE-EM is $O(FT^3)$ and there is always $T \gg L$.
\begin{table}[H]
\centering
\renewcommand\arraystretch{1.2}
\caption{Number of parameters and MACs per second\\ for DL-based methods}
\label{tab:para_mac}
\begin{tabular}{c|c|c}
    \Xhline{1pt}
    Method&Params~(M)&MACs~(G/s)\\
    \Xhline{0.4pt}
    SkipConvNet~\cite{kothapally2020skipconvnet}&64.3&11\\
    CMGAN~\cite{abdulatif2024cmgan} & 1.8 & 31\\
    StoRM~\cite{lemercier2023storm} & 27.3+27.8=55.1 & 2300\\
    \cdashline{1-3}
    TCN+SA+S~\cite{zhao2020monaural}& 4.7 & 0.7\\
    oSpatialNet*~\cite{10570301}&1.7&36.6\\
    \Xhline{0.4pt}
    VINP-TCN+SA+S& 4.7 & 0.7+0.27$\times$iterations\\
    VINP-oSpatialNet&1.7&36.6+0.27$\times$iterations\\
    \Xhline{1pt}
\end{tabular}
\end{table}
\begin{figure}
    \centering
    \includegraphics[width=0.46\linewidth]{figs/MACs.pdf}
    \caption{MACs per second and per iteration versus speech length.}
    \label{fig:macs}
\end{figure}

\subsubsection{{Blind RIR Identification}}
We measure RT60 and DRR through the RIR estimates to show the effectiveness of RIR identification.
The comparison methods include several classical blind RT60 and DRR estimation methods, such as Ratnam's approach~\cite{ratnam2003blind}\footnote{\url{https://github.com/nuniz/blind_rt60}} and Jeub's approach~\cite{jeub2011blind}\footnote{\url{https://ww2.mathworks.cn/matlabcentral/fileexchange/32752-blind-direct-to-reverberant-energy-ratio-drr-estimation}}.
As a DL-based blind RIR identification method, BUDDy~\cite{lemercier2024unsupervised}\footnote{\url{https://github.com/sp-uhh/buddy}} is also compared.
All these methods are implemented through the open-sourced official codes. 
Specifically, because the output of BUDDy is the RIR waveform, we employ the same implementation as in VINP to estimate the RT60 and DRR.
Notice that the acoustic conditions of our training set and test set are mismatched. 
BUDDy, as an unsupervised approach, is designed to bridge the performance gap between scenarios with matched and mismatched acoustic conditions. 
Therefore, we directly employ the model parameters provided by the authors without retraining.


\begin{table*}[htbp]
\centering
\renewcommand\arraystretch{1.2}
\caption{Dereverberation results on REVERB~(1-ch)}
\label{tab:results_SD}
\begin{tabular}{c|c|c|c|c|c|c|c|c|c|c|c|c}
    \Xhline{1pt}
    \multirow{3}{*}{Method}&\multicolumn{7}{c|}{SimData}&\multicolumn{5}{c}{RealData} \\
    \Xcline{2-13}{0.4pt}
    &\multirow{2}{*}{PESQ}&\multirow{2}{*}{ESTOI}&\multicolumn{2}{c|}{DNSMOS}&\multicolumn{3}{c|}{WER~(\%)}&\multicolumn{2}{c|}{DNSMOS}&\multicolumn{3}{c}{WER~(\%)}\\
    \Xcline{4-13}{0.4pt}
    &&&P.835&P.808&tiny&small&medium&P.835&P.808&tiny&small&medium\\
    \Xhline{0.4pt}
    Unprocessed&1.48&0.70&2.37&3.20&13.1&5.6&4.6&1.31&2.82&24.1&7.9&5.7\\
    \textcolor{gray}{Oracle}&\textcolor{gray}{-}&\textcolor{gray}{-}&\textcolor{gray}{3.76}&\textcolor{gray}{3.90}&\textcolor{gray}{7.6}&\textcolor{gray}{4.5}&\textcolor{gray}{4.0}&\textcolor{gray}{-}&\textcolor{gray}{-}&\textcolor{gray}{-}&\textcolor{gray}{-}&\textcolor{gray}{-}\\
    \Xhline{0.4pt}
    GWPE~\cite{yoshioka2012generalization}&1.55&0.72&2.41&3.22&12.1&5.5&4.6&1.42&2.83&21.4&6.8&5.7\\
    SkipConvNet~\cite{kothapally2020skipconvnet}&2.12&0.81&3.20&3.60&\cellcolor{graybackground}13.3&\cellcolor{graybackground}6.3&\cellcolor{graybackground}5.2&2.84&3.32&\cellcolor{graybackground}24.5&\cellcolor{graybackground}9.3&\cellcolor{graybackground}7.3\\
    
    CMGAN~\cite{abdulatif2024cmgan} &2.85&0.90&\textbf{3.82}&3.81&9.5&5.1&4.4&\textbf{3.87}&4.00&12.9&5.9&5.0\\
    StoRM~\cite{lemercier2023storm} &2.34&0.86&3.73&\textbf{3.96}&11.4&\cellcolor{graybackground}6.2&\cellcolor{graybackground}5.1&3.72&\textbf{4.01}&17.5&\cellcolor{graybackground}10.2&\cellcolor{graybackground}8.0\\
    % mFSN&&&&&&&&&&\\
    \cdashline{1-13}
    TCN+SA+S~\cite{zhao2020monaural}&2.60&0.86&3.50&3.73&12.1&\cellcolor{graybackground}6.5&\cellcolor{graybackground}5.5&3.37&3.73&\cellcolor{graybackground}27.3&\cellcolor{graybackground}12.6&\cellcolor{graybackground}10.0 \\
    oSpatialNet*~\cite{10570301}&\textbf{2.87}&\textbf{0.92}&3.57&3.88&8.9&4.9&4.3&3.48&3.87&10.5&5.4&4.5\\
    \Xhline{0.4pt}
    % VINP-TCN+SA+S (50 iterations)&2.46&0.87&3.47&\textbf{3.89}&8.9&5.1&4.5&3.19&3.80&11.9&6.2&5.1\\
    VINP-TCN+SA+S (prop.) &2.52&0.87&3.47&3.88&8.9&5.1&4.3&3.18&3.77&11.6&6.1&5.3\\
    % VINP-oSpatialNet (50 iterations)&2.74&\textbf{0.90}&3.49&3.86&\textbf{8.2}&\textbf{4.8}&\textbf{4.3}&3.33&3.81&\textbf{8.9}&\textbf{5.1}&\textbf{4.3}\\
    VINP-oSpatialNet (prop.)&2.82&0.90&3.49&3.86&\textbf{8.3}&\textbf{4.8}&\textbf{4.1}&3.33&3.80&\textbf{8.9}&\textbf{5.0}&\textbf{4.3}\\
    \Xhline{1pt}

\end{tabular}
\end{table*}



% \begin{table*}[htbp]
% \centering
% \renewcommand\arraystretch{1.2}
% \caption{Dereverberation results on REVERB~(1-ch)}
% \label{tab:results_SD}
% \begin{tabular}{c|c|c|c|c|c|c|c|c|c|c|c|c}
%     \Xhline{1pt}
%     \multirow{3}{*}{Method}&\multicolumn{7}{c|}{SimData}&\multicolumn{5}{c}{RealData} \\
%     \Xcline{2-13}{0.4pt}
%     &\multirow{2}{*}{PESQ}&\multirow{2}{*}{ESTOI}&\multicolumn{2}{c|}{DNSMOS}&\multicolumn{3}{c|}{WER~(\%)}&\multicolumn{2}{c|}{DNSMOS}&\multicolumn{3}{c}{WER~(\%)}\\
%     \Xcline{4-13}{0.4pt}
%     &&&P.835&P.808&base&small&medium&P.835&P.808&base&small&medium\\
%     \Xhline{0.4pt}
%     Unprocessed&1.48&0.70&2.37&3.20&10.1&5.6&4.7&1.31&2.82&14.6&7.7&5.6\\
%     \textcolor{gray}{Oracle}&-&-&3.76&3.90&7.9&4.5&4.1&-&-&-&-&-\\
%     \Xhline{0.4pt}
%     GWPE~\cite{yoshioka2012generalization}&1.57&0.72&2.41&3.22&9.8&5.4&4.7&1.43&2.83&12.3&7.0&5.5\\
%     SkipConvNet~\cite{kothapally2020skipconvnet}&2.12&0.81&3.20&3.60&10.4&6.4&5.3&2.74&3.32&15.5&9.3&7.4\\
    
%     CMGAN~\cite{abdulatif2024cmgan} &\textbf{2.85}&\textbf{0.90}&\textbf{3.82}&3.81&8.4&5.1&4.5&\textbf{3.86}&\textbf{4.00}&10.6&5.8&5.2\\
%     StoRM~\cite{lemercier2023storm} &2.34&0.86&\textbf{3.73}&\textbf{3.96}&10.4&6.0&5.1&\textbf{3.72}&\textbf{4.01}&18.1&9.9&9.7\\
%     % mFSN&&&&&&&&&&\\
%     \cdashline{1-13}
%     TCN+SA+S~\cite{zhao2020monaural}&2.59&0.86&3.50&3.73&10.3&6.5&5.0&3.37&3.73&18.8&12.9&6.7 \\
%     mOSPN&\textbf{2.87}&\textbf{0.92}&3.54&3.88&\textbf{8.0}&4.8&\textbf{4.3}&3.39&3.86&10.4&5.3&\textbf{4.5}\\
%     % \Xhline{0.4pt}
%     % VINP-TCN+SA+S (40 iterations)&2.43&0.87&3.47&\textbf{3.89}&9.1&&4.5&3.20&3.81&9.2&6.1&\\
%     % VINP-TCN+SA+S (80 iterations)&2.50&0.87&3.47&3.88&9.0&&4.4&&&&6.1&\\
%     % VINP-mOSPN (40 iterations)&2.70&\textbf{0.90}&3.49&3.86&\textbf{7.8}&&\textbf{4.3}&3.34&3.81&\textbf{7.2}&\textbf{5.1}&\\
%     % VINP-mOSPN (80 iterations)&2.80&\textbf{0.90}&3.48&3.86&8.3&&\textbf{4.3}&3.32&3.80&&\textbf{5.1}&\textbf{4.3}\\
%     \Xhline{0.4pt}
%     NeGI-TCN+SA+S-10&1.92&0.84&3.34&3.74&10.0&&4.7&3.16&3.76&&6.7&\\
%     NeGI-TCN+SA+S-20&2.20&0.86&3.45&3.85&8.7&&4.6&3.21&3.82&&6.3&\\
%     NeGI-TCN+SA+S-30&2.36&0.87&3.47&3.88&9.2&&4.5&3.21&3.82&&6.3&\\
%     NeGI-TCN+SA+S-40&2.43&0.87&3.47&3.89&9.1&&4.5&3.20&3.81&9.2&6.1&\\
%     NeGI-TCN+SA+S-50&2.46&0.87&3.47&3.89&9.1&5.1&4.5&3.19&3.80&9.2&6.2&5.1\\
%     NeGI-TCN+SA+S-60&2.48&0.87&3.47&3.88&9.1&&4.5&3.19&3.79&&6.1&\\
%     NeGI-TCN+SA+S-70&2.49&0.87&3.47&3.88&9.1&&4.4&3.18&3.79&&6.1&\\
%     NeGI-TCN+SA+S-80&2.50&0.87&3.47&3.88&9.0&&4.4\\
%     NeGI-TCN+SA+S-90&2.51&0.87&3.47&3.88&&&\\
%     NeGI-TCN+SA+S-100&2.51&0.87&3.47&3.88&9.1&5.1&4.4&3.17&3.77&&6.2&5.1\\
%     \Xhline{0.4pt}
%     NeGI-mOSPN-10&2.17&0.87&3.34&3.67&8.6&&4.4&3.24&3.71&&5.0&\\
%     NeGI-mOSPN-20&2.45&0.89&3.46&3.80&8.2&&4.3&3.33&3.79&&5.0&\\
%     NeGI-mOSPN-30&2.62&0.90&3.48&3.84&7.8&&4.3&3.34&3.81&&5.1&\\
%     NeGI-mOSPN-40&2.70&0.90&3.49&3.86&7.8&&4.3&3.34&3.81&7.2&5.1&\\
%     NeGI-mOSPN-50&2.74&0.90&3.49&3.86&8.0&4.8&4.3&3.33&3.81&7.1&5.1&4.3\\
%     NeGI-mOSPN-60&2.77&0.90&3.49&3.86&8.3&&4.3&3.33&3.81&&5.1&\\
%     NeGI-mOSPN-70&2.79&0.90&3.48&3.86&8.3&&4.3&3.32&3.80&&5.1&\\
%     NeGI-mOSPN-80&2.80&0.90&3.48&3.86&8.3&&4.3&3.32&3.80&7.1&5.1&\\
%     NeGI-mOSPN-90&2.81&0.90&3.48&3.86&&&&3.32&3.80&&5.0&\\
%     NeGI-mOSPN-100&2.82&0.90&3.48&3.86&8.5&4.8&&3.32&3.80&7.0&5.0&4.3\\
%     % EMA 0.8\\
%     % NeGI-TCN+SA+S (50 iterations)&2.40&0.87&3.47&3.88&9.1&4.5&3.21&3.82&8.9&5.3\\
%     % NeGI-TCN+SA+S (100 iterations)&2.51&0.87&3.48&3.88&9.1&4.5&3.19&3.79&8.8&5.1\\
%     % NeGI-mOSPN (50 iterations)&2.67&0.90&3.49&3.85&8.4&&3.33&3.81&\textbf{7.1}&\textbf{4.3}\\
%     % NeGI-mOSPN (100 iterations)&2.81&0.90&3.49&3.86&8.2&4.4&&&&\\
%     % \Xhline{0.4pt}
%     % EMA 0.8 0.7\\
%     % NeGI-TCN+SA+S (50 iterations)&&&&&&&3.19&3.82&9.2&\\
%     % \Xhline{0.4pt}
%     % EMA 0.7 0.8\\
%     % NeGI-TCN+SA+S (50 iterations)&&&&&&&3.21&3.81&8.9&\\

%     % % NeGI-TCN+SA+S-e111&2.45&0.87&3.46&3.88&9.1&&3.19&3.80&9.2&\\
%     % % \textbf{NeGI-TCN+SA+S-e119-121}&&&&&&&&&&\\
%     % % NeGI-mOSPN-ori-e21&\textbf{2.83}&\textbf{0.90}&3.47&\textbf{3.86}&&\textbf{4.3}&3.34&3.82&&\textbf{4.5}\\
%     % % \textbf{NeGI-mOSPN-e13}&2.77&0.90&3.49&3.87&8.3&4.3&3.33&3.82&9.1&4.4\\
%     % % \textbf{NeGI-mOSPN-e19}&2.75&0.90&3.48&3.86&8.3&&3.29&3.80&7.2&\\
%     % % \Xhline{0.4pt}
%     % % % \textcolor{gray}{mOSPN~(cMSE)}&2.84&0.91&3.58&3.86&9.5&4.3&3.38&3.80&10.8&4.7  \\
%     % % \textcolor{gray}{mOSPN-e10}&2.87&0.92&3.54&3.88&8.0&4.3&3.39&3.86&10.4&4.5\\
%     % % \textcolor{gray}{mOSPN~(KL)}&&&&&&&&&&\\
%     \Xhline{1pt}

% \end{tabular}
% \end{table*}


% \begin{table}[H]
% \centering
% \renewcommand\arraystretch{1.2}
% \caption{Dereverberation results on REVERB SimData}
% \label{tab:results_sim}
% \begin{tabular}{c|c|c|c|c|c|c|c}
%     \Xhline{1pt}
%     \multirow{2}{*}{Method}
%     &\multirow{2}{*}{PESQ}&\multirow{2}{*}{ESTOI}&\multicolumn{2}{c|}{DNSMOS}&\multicolumn{3}{c|}{WER~(\%)}\\
%     \Xcline{4-5}{0.4pt}
%     &&&P.835&P.808&m&s&b\\
%     \Xhline{0.4pt}
%     Unprocessed  &  1.48  & 0.70  & 2.37 & 3.20 & 4.7   \\
%     Oracle &-&-&3.76&3.90&4.1\\
%     \Xhline{0.4pt}
%     GWPE~\cite{yoshioka2012generalization}&1.57&0.72&2.41&3.22&4.7\\
%     SkipConvNet~\cite{kothapally2020skipconvnet} &2.12&0.81&3.20&3.60&5.3\\
%     TCN+SA+S~\cite{zhao2020monaural}&2.59&0.86&3.50&3.73&5.0 \\
%     CMGAN~\cite{abdulatif2024cmgan} &\textbf{2.85}&\textbf{0.90}&\textbf{3.82}&3.81&\textbf{4.5}\\
%     StoRM~\cite{lemercier2023storm} &2.34&0.86&\textbf{3.73}&\textbf{3.96}&5.1\\
%     \Xhline{0.4pt}
%     NeGI-mOSPN&\textbf{2.83}&\textbf{0.90}&3.47&\textbf{3.86}&\textbf{4.3}\\
%     NeGI (TCN+SA+S) \\
%     \Xhline{0.4pt}
%     NeGI-mOSPN-e1&2.53&0.88&3.40&3.88&\\
%     NeGI-mOSPN-e2&2.68&0.89&3.48&3.88&\\
%     NeGI-mOSPN-e3&2.64&0.88&3.41&3.86&\\
%     NeGI-mOSPN-e4&2.63&0.89&3.46&3.88&\\
%     NeGI-mOSPN-e5&2.66&0.89&3.48&3.88&\\
%     NeGI-mOSPN-e7&2.69&0.89&3.47&3.87&\\
%     NeGI-mOSPN-e13&2.77&0.90&3.49&3.87&4.3&&8.3\\
%     NeGI-mOSPN-e9&&&&&\\
%     \Xhline{0.4pt}
%     ablation\\
%     mOSPN-cMSE-e10&2.84&0.91&3.58&3.86&4.3\\
%     \Xhline{1pt}
% \end{tabular}
% \end{table}
% \begin{table}[H]
% \centering
% \renewcommand\arraystretch{1.2}
% \caption{Dereverberation results on REVERB RealData}
% \label{tab:results_real}
% \begin{tabular}{c|c|c|c}
%     \Xhline{1pt}
%     \multirow{2}{*}{Method}
%     &\multicolumn{2}{c|}{DNSMOS}&\multirow{2}{*}{WER~(\%)}\\
%     \Xcline{2-3}{0.4pt}
%     &P.835&P.808&\\
%     \Xhline{0.4pt}
%     Unprocessed  &1.31 & 2.82 &5.6 \\
%     \Xhline{0.4pt}
%     GWPE~\cite{yoshioka2012generalization}&1.43&2.83&5.5\\
%     SkipConvNet~\cite{kothapally2020skipconvnet} &2.74&3.32&7.4\\
%     TCN+SA+S~\cite{zhao2020monaural}&3.37&3.73&6.7 \\
%     CMGAN~\cite{abdulatif2024cmgan} &\textbf{3.86}&\textbf{4.00}&\textbf{5.2}\\
%     StoRM~\cite{lemercier2023storm} &\textbf{3.72}&\textbf{4.01}&9.7\\
%     \Xhline{0.4pt}
%     NeGI-mOSPN&3.34&3.82&\textbf{4.5}\\
%     \Xhline{0.4pt}
%     NeGI-mOSPN-e13&3.33&3.82&4.4\textbar5.1\textbar9.1\\
%     \Xhline{0.4pt}
%     ablation\\
%     mOSPN-cMSE-e10&3.38&3.80&4.7\textbar5.5\textbar10.8\\
%     \Xhline{1pt}
% \end{tabular}
% \end{table}



\subsection{Evaluation Metrics}
\subsubsection{{Speech Dereverberation}}

Speech dereverberation performance is evaluated in terms of both perception quality and ASR accuracy. 
% Apart from large-scale manual evaluation, there is no gold standard for perception quality assessment.
% Therefore,
We use the commonly used speech quality metrics including perceptual evaluation of speech quality (PESQ)~\cite{rix2001perceptual}, extended short-time objective intelligibility (ESTOI)~\cite{jensen2016algorithm}, and
% speech to reverberation modulation energy ratio (SRMR)~\cite{falk2010non,6953337}, 
deep noise suppression mean opinion score (DNSMOS)~\cite{reddy2021dnsmos,reddy2022dnsmos}.
% Notice that for SRMR, we use the normalized version proposed in \cite{6953337}.
Speech utterances are normalized by their maximum absolute value before evaluation. 
Higher scores indicate better speech quality and intelligibility.

To demonstrate the effectiveness of our method on ASR systems, we utilize the pre-trained Whisper~\cite{radford2023whisper} ‘tiny' model (with 39 M parameters), 'small' model (with 244 M parameters), and ‘medium’ model (with 769 M parameters) for ASR evaluation. 
No additional dataset-specific finetuning or retraining is applied before ASR inference. 
The word error rate (WER) is used as the evaluation metric.
Lower WER indicates better ASR performance.

\subsubsection{{Blind RIR Identification}}
We use the accuracy of RT60 and DRR estimation to evaluate the performance of blind RIR identification.
We present the mean absolute error (MAE) and the root mean square error (RMSE) of RT60 and DRR between their estimates and ground-truth values over the SimACE test set.
Lower MAE and RMSE indicate better results.




\begin{figure}
    \centering
    \subfloat[Reverberant speech]{
        \includegraphics[width=0.9\linewidth]{figs/demo/reverb.pdf}
        }\hfill
    \subfloat[Output of TCN+SA+S]{
        \includegraphics[width=0.9\linewidth]{figs/demo/TCNSAS.pdf}
        }\hfill
    \subfloat[Output of VINP-TCN+SA+S]{
        \includegraphics[width=0.9\linewidth]{figs/demo/NeGI-TCNSAS.pdf}
        }\hfill
    \subfloat[Oracle speech]{
        \includegraphics[width=0.9\linewidth]{figs/demo/oracle.pdf}
        }\hfill
    \caption{Examples of magnitude spectra (RT60=0.7 s).}
    \label{fig:demo}
\end{figure}
\begin{figure*}[!ht]
    \centering
    \subfloat[Logarithmic likelihood]{
        \includegraphics[width=0.185\linewidth]{figs/LIKELIvsSTEP.pdf}
        }
    \subfloat[PESQ]{
        \includegraphics[width=0.185\linewidth]{figs/PESQvsSTEP.pdf}
        }
    \subfloat[ESTOI]{
        \includegraphics[width=0.185\linewidth]{figs/ESTOIvsSTEP.pdf}
        }
    \subfloat[DNSMOS]{
        \includegraphics[width=0.185\linewidth]{figs/DNS808vsSTEP.pdf}
        }
    \subfloat[WER with 'tiny' model]{
        \includegraphics[width=0.185\linewidth]{figs/WERvsSTEP.pdf}
        }
    % \subfloat[Logarithmic likelihood on RealData]{
    %     \includegraphics[width=0.23\linewidth]{figs/LIKELIvsSTEP-Real.pdf}
    %     }
    % \subfloat[DNSMOS on RealData]{
    %     \includegraphics[width=0.23\linewidth]{figs/DNS808vsSTEP-Real.pdf}
    %     }
    % \subfloat[WER with tiny model on RealData]{
    %     \includegraphics[width=0.23\linewidth]{figs/WERvsSTEP-Real.pdf}
    %     }
    
    \caption{Logarithmic likelihood and dereverberation metrics as a function of VEM iterations on REVERB SimData.}
    \label{fig:CurveDereverb}
\end{figure*}
% \section{Experimental Results and Discussion}
% \label{sec:exp}
\subsection{Results and Analysis}
\subsubsection{Speech Dereverberation}

Setting the number of VEM iterations to 100, the dereverberation results are presented in Table \ref{tab:results_SD}, where the bold font denotes the best results.
The results with negative dereverberation effects are marked with a darkened background.

As the metrics of unprocessed speech and oracle speech indicate, a larger ASR system implies a stronger recognition ability for oracle speech and stronger robustness to reverberation.
Compared with using a large ASR system alone, introducing an ASR-effective front-end speech dereverberation system can significantly improve the performance of ASR with fewer parameters and lower computational costs.
Among the comparison methods, only a few approaches are proven to be effective for ASR, such as GWPE, CMGAN, oSpatialNet*, and the proposed VINP-TCN+SA+S and VINP-oSpatialNet.
The classical GWPE yields a marginal improvement in ASR performance.
% , which can be attributed to the fact that the whisper model employed herein exhibits a certain level of robustness against speech distortion.
Among the DL-based methods, SkipConvNet, StoRM and TCN+SA+S encounter failures in ASR.
TCN+SA+S and StoRM demonstrate marginal effectiveness solely when the 'tiny' ASR model is employed.
The reason for such ASR performance lies in the artifact induced by DNN, which is a common phenomenon in similar DL-based single-channel systems~\cite{iwamoto22_interspeech,iwamoto2024does}.
Different from the comparison methods, VINP employs a linear CTF filtering process to model the reverberation effect and leverages the DNN output as a prior distribution of anechoic source speech.
During the VBI procedure, VINP takes into account both this prior information and the observed data.
By doing this, VINP circumvents the direct utilization of DNN output and thereby reduces the artifacts.
Meanwhile, VINP still utilizes the powerful non-linear modeling ability of DNN.
Consequently, VINP exhibits a remarkable superiority over the other approaches in ASR performance.
When comparing VINP-TCN+SA+S with its backbone network TCN+SA+S, a gap in WER can be discerned, indicating that VINP can make an ASR-ineffective DNN effective. 
Both oSpatialNet* and VINP-oSpatialNet achieve ASR performance close to that of oracle speech on REVERB SimData when using the small and medium ASR systems.
In other situations, a gap still exists, indicating that VINP can make an ASR-effective DNN even more effective.
Because VINP-oSpatialNet provides an anechoic speech prior with higher quality, its ASR performance is superior to VINP-TCN+SA+S.
Notably, VINP-oSpatialNet achieves SOTA performance in ASR.



Regarding speech quality, the ESTOI of VINP is roughly the same as that of the backbone network, while the PESQ shows a slight decrease compared to the backbone networks.
This indicates that VINP is restricted by the capabilities of the backbone DNN. 
It is also worth noting that the PESQ increases with the growth of the number of VEM iterations, which we will demonstrate in detail in the experiments presented later.
In subjective evaluation, the DNSMOS metrics of VINP-TCN+SA+S and VINP-oSpatialNet reach an advanced level.
However, it does not rank the highest among all methods, especially when considering DNSMOS P.835.
We hold the view that DNSMOS exhibits a certain preference for enhanced speech.
As evidence, it can be seen that the DNSMOS of CMGAN and StoRM is even higher than that of oracle speech, indicating a preference for generative DNNs.
In addition, VINP-TCN+SA+S and VINP-oSpatialNet exhibit worse and comparable performance to TCN+SA+S in DNSMOS P.835.
However, in our auditory perception, the output speech of VINP exhibits a higher degree of naturalness.
To illustrate the characteristics of different methods, examples of the reverberant speech, the output of TCN+SA+S, the output of VINP-TCN+SA+S, and the oracle speech are shown in Fig.~\ref{fig:demo}.
% In addition, some examples of magnitude spectra are also illustrated in Fig. \ref{fig:demo}.
% Because the random components in the spectrum are difficult for DNNs to learn,
Due to the insufficient modeling capabilities of DNN, the output of TCN+SA+S has some overly smoothed artifacts.
On the contrary, VINP-TCN+SA+S not only refers to the priors provided by DNNs, but also directly utilizes the observation during the VBI procedure, resulting in a better estimation of the anechoic speech spectrum.
Moreover, VINP-TCN+SA+S demonstrates a better performance of background noise control.
A series of enhanced audio examples are available online\footnote{\url{https://github.com/Audio-WestlakeU/VINP}}.
% CMGAN tends to enhance the features of spectrograms by increasing the power of resonance peaks, whereas VINP tends to reduce more background noise.
% \be


To illustrate the influence of VBI iterations on the performance of VINP, we show the average logarithmic likelihood of the complete data $\ln p(\mathbf{S},\mathbf{X})$ (where the constant is omitted) and some metrics on REVERB SimData as the iteration count rises in Fig. \ref{fig:CurveDereverb}.
As the iteration count increases, the logarithmic likelihood and all metrics shift in a positive direction and tend to converge.
Specifically, ESTOI, DNSMOS, and WER exhibit rapid convergence. 
In contrast, PESQ shows a continuous improvement that closely resembles the behavior of the logarithmic likelihood.
This phenomenon indicates that better dereverberation performance can be achieved at the cost of computational complexity.
Additionally, the capability of the backbone network determines the upper bound of performance of PESQ, ESTOI and DNSMOS.
The logarithmic likelihood of VINP-TCN+SA+S and VINP-oSpatialNet converge to nearly the same values, yet their metrics are different, indicating that when the prior quality varies, the logarithmic likelihood cannot be directly used as an indicator to measure the enhancement performance.




\subsubsection{Blind RIR Identification}

Also setting the number of VEM iterations to 100, the RT60 and DRR estimation results are illustrated in Table \ref{tab:RIR_estimation}, where the bold font represents the best results.

VINP surpasses all comparison methods, reaching the SOTA level for all metrics.
Theoretically, one limitation of our method is that a CTF filter with a finite length can only model a RIR with a finite length.
Nevertheless, it remains sufficient for RT60 and DRR estimation even under large-reverberation conditions, because the signal energy at the tail of the RIR is relatively negligible.
From the perspective of spatial acoustic parameter estimation, VINP can estimate RIR well.
\begin{table}
\centering
\caption{Blind RT60 and DRR estimation results on SimACE}
\label{tab:RIR_estimation}
\renewcommand\arraystretch{1.2}
\begin{tabular}{c|c|c|c|c}
    \Xhline{1pt}
    \multirow{2}{*}{Method}& \multicolumn{2}{c|}{RT60 (s)}& \multicolumn{2}{c}{DRR (dB)}\\
    \Xcline{2-5}{0.4pt}
    &MAE&RMSE&MAE&RMSE\\
    \Xhline{0.4pt}
    Ratnam's~\cite{ratnam2003blind}&0.151&0.182&-&-\\
    Jeub's~\cite{jeub2011blind}&-&-&7.14&8.69\\
    BUDDy~\cite{lemercier2024unsupervised}&0.089&0.132&3.93&4.57\\
    \Xhline{0.4pt}
    VINP-TCN+SA+S (prop.)&\textbf{0.079}&\textbf{0.094}&\textbf{3.83}&\textbf{4.27}\\
    VINP-oSpatialNet (prop.)&\textbf{0.079}&0.098&3.87&4.32\\
    % \Xhline{0.4pt}
    % VINP-TCN+SA+S\\
    % 10step&0.113&0.127&4.67&5.03\\
    % 20step&0.097&0.113&4.38&4.76\\
    % 30step&0.090&0.107&4.23&4.63\\
    % 40step&0.087&0.103&4.12&4.53\\
    % 50step&0.084&0.101&4.04&4.46\\
    % 60step&0.082&0.098&3.98&4.40\\
    % 70step&0.081&0.097&3.93&4.36\\
    % 80step&0.080&0.096&3.89&4.32\\
    % 90step&0.079&0.095&3.85&4.30\\
    % 100step&0.079&0.094&3.83&4.27\\
    % \Xhline{0.4pt}
    % VINP-mOSPN\\
    % 10step&0.098&0.114&3.84&4.14\\
    % 20step&0.089&0.109&3.63&3.95\\
    % 30step&0.086&0.106&3.92&4.26\\
    % 40step&0.084&0.104&3.99&4.38\\
    % 50step&0.083&0.103&3.99&4.40\\
    % 60step&0.082&0.102&3.96&4.39\\
    % 70step&0.081&0.101&3.94&4.37\\
    % 80step&0.080&0.100&3.91&4.36\\
    % 90step&0.080&0.098&3.89&4.34\\
    % 100step&0.079&0.098&3.87&4.32\\
    \Xhline{1pt}
\end{tabular}
\end{table}
\begin{figure}
    \centering
    \subfloat[RT60 estimation]{
        \includegraphics[width=0.46\linewidth]{figs/T60MAEvsSTEP-RIR.pdf}
        }
    \subfloat[DRR estimation]{
        \includegraphics[width=0.46\linewidth]{figs/DRRMAEvsSTEP-RIR.pdf}
        }
    \caption{MAE of RT60 and DRR estimation as a function of VEM iterations on SimACE.}
    \label{fig:CurveRIR}
\end{figure}

We also show the MAE of RT60 and DRR estimation as the iteration count rises in Fig. \ref{fig:CurveRIR}.
As can be observed, as the iteration count increases, the errors of RT60 and DRR estimation decrease, which indicates an improved quality of RIR estimation.
After 30 iterations, VINP-TCN+SA+S and VINP-oSpatialNet start to outperform BUDDy.
When the iteration count is less than 70, VINP-oSpatialNet outperforms VINP-TCN+SA+S. 
Conversely, when the iteration count is larger than 70, VINP-TCN+SA+S is slightly superior to VINP-oSpatialNet.
This phenomenon may be related to the DNN structure. 
Owing to the complexity of the entire system, we are currently unable to draw a conclusion about the preference for a specific DNN architecture for RIR identification.
Unlike the experimental results of dereverberation, the identification of RIR does not seem to be sensitive to the capability of DNN when there are enough iterations.


\section{Conclusions}\label{s:Conclusion}
In this paper, we demonstrate that frequent ReRAM cell updates needed for DNN inference significantly shorten the lifespan of ReRAM-based accelerators due to the limited endurance cycles of ReRAM cells. To address this challenge, we introduce \textit{Hamun}, an approximate computation method designed to extend the lifespan of ReRAM-based accelerators through multiple optimizations. Hamun features a novel fault-handling scheme that identifies worn-out cells and retire them to prevent their impact on DNN accuracy. Additionally, Hamun employs wear-leveling and batch execution techniques to further increase longevity. To reduce the overhead of retiring cells, Hamun also incorporates an approximation method, thereby extending the accelerator’s lifespan with minimal degradation in DNN accuracy. Across a set of popular DNNs, Hamun achieves a $13.2\times$ lifespan improvement over the baseline on average, highlighting its potential in making ReRAM-based accelerators more viable for long-term use.

% \section*{Acknowledgment}
% Pengyu Wang wishes to express gratitude to Ying Fang from Westlake University for her assistance in ASR.


% needed in second column of first page if using \IEEEpubid
%\IEEEpubidadjcol




% An example of a floating figure using the graphicx package.
% Note that \label must occur AFTER (or within) \caption.
% For figures, \caption should occur after the \includegraphics.
% Note that IEEEtran v1.7 and later has special internal code that
% is designed to preserve the operation of \label within \caption
% even when the captionsoff option is in effect. However, because
% of issues like this, it may be the safest practice to put all your
% \label just after \caption rather than within \caption{}.
%
% Reminder: the "draftcls" or "draftclsnofoot", not "draft", class
% option should be used if it is desired that the figures are to be
% displayed while in draft mode.
%
%\begin{figure}[!t]
%\centering
%\includegraphics[width=2.5in]{myfigure}
% where an .eps filename suffix will be assumed under latex, 
% and a .pdf suffix will be assumed for pdflatex; or what has been declared
% via \DeclareGraphicsExtensions.
%\caption{Simulation results for the network.}
%\label{fig_sim}
%\end{figure}

% Note that the IEEE typically puts floats only at the top, even when this
% results in a large percentage of a column being occupied by floats.


% An example of a double column floating figure using two subfigures.
% (The subfig.sty package must be loaded for this to work.)
% The subfigure \label commands are set within each subfloat command,
% and the \label for the overall figure must come after \caption.
% \hfil is used as a separator to get equal spacing.
% Watch out that the combined width of all the subfigures on a 
% line do not exceed the text width or a line break will occur.
%
%\begin{figure*}[!t]
%\centering
%\subfloat[Case I]{\includegraphics[width=2.5in]{box}%
%\label{fig_first_case}}
%\hfil
%\subfloat[Case II]{\includegraphics[width=2.5in]{box}%
%\label{fig_second_case}}
%\caption{Simulation results for the network.}
%\label{fig_sim}
%\end{figure*}
%
% Note that often IEEE papers with subfigures do not employ subfigure
% captions (using the optional argument to \subfloat[]), but instead will
% reference/describe all of them (a), (b), etc., within the main caption.
% Be aware that for subfig.sty to generate the (a), (b), etc., subfigure
% labels, the optional argument to \subfloat must be present. If a
% subcaption is not desired, just leave its contents blank,
% e.g., \subfloat[].


% An example of a floating table. Note that, for IEEE style tables, the
% \caption command should come BEFORE the table and, given that table
% captions serve much like titles, are usually capitalized except for words
% such as a, an, and, as, at, but, by, for, in, nor, of, on, or, the, to
% and up, which are usually not capitalized unless they are the first or
% last word of the caption. Table text will default to \footnotesize as
% the IEEE normally uses this smaller font for tables.
% The \label must come after \caption as always.
%
%\begin{table}[!t]
%% increase table row spacing, adjust to taste
%\renewcommand{\arraystretch}{1.3}
% if using array.sty, it might be a good idea to tweak the value of
% \extrarowheight as needed to properly center the text within the cells
% \caption{An Example of a Table}
% \label{table_example}
% \centering
% % Some packages, such as MDW tools, offer better commands for making tables
% % than the plain LaTeX2e tabular which is used here.
% \begin{tabular}{|c||c|}
% \hline
% One & Two\\
% \hline
% Three & Four\\
% \hline
% \end{tabular}
% \end{table}


% Note that the IEEE does not put floats in the very first column
% - or typically anywhere on the first page for that matter. Also,
% in-text middle ("here") positioning is typically not used, but it
% is allowed and encouraged for Computer Society conferences (but
% not Computer Society journals). Most IEEE journals/conferences use
% top floats exclusively. 
% Note that, LaTeX2e, unlike IEEE journals/conferences, places
% footnotes above bottom floats. This can be corrected via the
% \fnbelowfloat command of the stfloats package.









% use section* for acknowledgment



% Can use something like this to put references on a page
% by themselves when using endfloat and the captionsoff option.
\ifCLASSOPTIONcaptionsoff
  \newpage
\fi



% trigger a \newpage just before the given reference
% number - used to balance the columns on the last page
% adjust value as needed - may need to be readjusted if
% the document is modified later
%\IEEEtriggeratref{8}
% The "triggered" command can be changed if desired:
%\IEEEtriggercmd{\enlargethispage{-5in}}

% references section
\bibliographystyle{IEEEtran}
\bibliography{bibtex/bib/IEEEexample}

% \begin{IEEEbiography}{Michael Shell}
% Biography text here.
% \end{IEEEbiography}

% if you will not have a photo at all:
% \begin{IEEEbiographynophoto}{Pengyu Wang}
% received the B.E. degree in electronic information engineering from the Xidian University, Xi'an, China, in 2019, and the M.S. degree in information and communication engineering from the University of Science and Technology of China, Hefei, China, in 2022.
% He is currently a Ph.D. candidate in computer science and technology at Zhejiang University, Hangzhou, China and Westlake University, Hangzhou, China.
% His research interests include speech enhancement, speech dereverberation, array signal processing, and deep learning.
% \end{IEEEbiographynophoto}
% \begin{IEEEbiographynophoto}{Ying Fang}
% received the B.E. degree in control science and engineering from Zhejiang University, Hangzhou, China, in 2022. She is currently pursuing her Ph.D. in a joint graduate program between Westlake University, Hangzhou, China, and Zhejiang University. Her research interests include speech and language processing.
% \end{IEEEbiographynophoto}
% \begin{IEEEbiographynophoto}{Xiaofei Li}
% received the Ph.D. degree from Peking University, Beijing, China, in 2013. From 2014 to 2016, he was with INRIA Grenoble Rhône-Alpes, Montbonnot-Saint-Martin, France, as a Postdoctoral Researcher, and Starting Research Scientist from 2016 to 2019. He is currently an Assistant Professor with Westlake University, Hangzhou, China. His research interests include the field of acoustic, audio and speech signal processing, including the topics of speech denoising, dereverberation, separation and localization, sound/speech semisupervised learning
% and unsupervised pre-training, and sound field reproduction and personal sound zone.
% \end{IEEEbiographynophoto}

% You can push biographies down or up by placing
% a \vfill before or after them. The appropriate
% use of \vfill depends on what kind of text is
% on the last page and whether or not the columns
% are being equalized.

%\vfill

% Can be used to pull up biographies so that the bottom of the last one
% is flush with the other column.
%\enlargethispage{-5in}



% that's all folks
\end{document}


