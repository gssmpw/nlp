\section{Related Work}
% \cite{BOHUS2009332} managing chat through schema or dialog flows

% \todo{evaluation}

% Knowledge mining from unstructured texts has traditionally focused on identifying named entities, events, and relationships and constructing factual knowledge graphs \cite{10.1145/1089815.1089817,10.5555/1214993,singhal2012introducing,wan2023gptre,xu2023large,Khraisha2023CanLL,qi2023mastering}. 
% Existing methods for extracting procedural knowledge predominantly rely on well-structured textual sources, such as manuals and documentation \cite{agarwal2020extracting,dunn2022structured,dannenfelser2024into}. In contrast, the task of extracting task-relevant steps from natural language interactions remains an under-explored challenge.
% % Procedural knowledge, which involves necessary steps in the correct order to accomplish some tasks, is another important tenet of knowledge with many realistic applications, such as agent planning \cite{Huang2022LanguageMA}, customer support automation \cite{min2023workflowguided}, workflow automation \cite{gallanti1985}, and synthetic customer-agent chat generation \cite{du2024dflowdiversedialogueflow}. 
% % Despite its broad applications, existing work on procedural knowledge extraction relies on 
% % well-written documents \cite{agarwal2020extracting, dunn2022structured, dannenfelser2024into}, and the task of extracting task-relevant steps from natural language interactions (e.g., conversations) remains under-explored.

%Prior work has primarily relied on human-authored dialog workflows \citep{mosig2020star,chen-etal-2021-action}. While dialog workflows are a specialized form of procedural knowledge, most 
Dialog workflows are a specialized form of procedural knowledge. While workflow extraction has received little attention, automatic procedural extraction has been widely studied, primarily focusing on ``how-to'' documents \citep{10.1145/2187980.2188194,maeta-etal-2015-framework,Chu2017DistillingTK,Park2018LearningPF} and instructional videos \citep{ushiku-etal-2017-procedural,10.5555/3504035.3504965,xu-etal-2020-benchmark}.
These works typically model linear sequences of explicitly stated actions, aiming to either predict procedural steps or generate summaries of task execution \citep{Koupaee2018WikiHowAL}.
% , which typically feature a linear sequence of actions clearly stated within the text. These studies primarily aimed at either predicting the sequence of actions or providing a summary of the procedural steps \citep{Koupaee2018WikiHowAL}.
In contrast, our work tackles dialog workflows, where actions are often implicit and depend on previous steps, user inputs, and system responses. This introduces decision-dependent variability, making extraction significantly more challenging than predicting fixed procedural sequences.

Specific to dialog systems, there has been extensive research on studying structures in task-oriented dialogs~\cite{Jurafsky1997SwitchboardSS,Chotimongkol2008LearningTS,shi2019unsuperviseddialogstructurelearning,xu2020discoveringdialogstructuregraph,chen2021dsbertunsuperviseddialoguestructurelearning,Nath2021TSCAND,wang2021modellinghierarchicalstructuredialogue,rony2022dialokgknowledgestructureawaretaskoriented,lu2022unsupervisedlearninghierarchicalconversation,qiu2022structureextractiontaskorienteddialogues,Yin_2023,pryor2024usingdomainknowledgeguide,burdisso2024dialog2flow}, focusing on how dialogs evolve using dialog acts, intent-slot pairs, or turn-level dependencies. Our work parallels workflow discovery \cite{hattami2023workflow, raimondo2023improving,min2023workflowguided}, which aims to predict the optimal next dialog action from the conversation's current state and all available actions. However, unlike this, we focus on extracting global workflows applicable across conversations for a specific intent. This increases the complexity of the task, as the model must filter out noisy actions, and consolidate multiple potential actions sequences from different conversations. Additionally, we also propose a new QA-CoT prompting method for workflow extraction and introduce a robust  end-to-end evaluation framework to assess the accuracy of the extracted workflows.
 
%