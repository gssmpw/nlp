\section{Related Work}
\label{sec:related_work}
	This section presents applications that use causal models and the Halpern-Pearl definition of actual causation____. We focus on robotics and automatic systems. The reviewed works show that practical applications of the concepts of actual causation have been focused on generating explanations, leaving the potential to guide decision-making processes aside. To the best of our knowledge, the probabilistic actual causation definition of ____ has not been used in any practical application. 
	
	____ use the actual causation framework by Halpern and Pearl in a human-robot interaction setting where a robot interacts with children with ASD (Autism Spectrum Disorder). The robot plays different interactive games with the children, aiming to improve the children's ability to see the world from the robot's point-of-view. The authors present a tool that uses a causal model of the interactive games and the actual causation framework. This is applied to explain events during the game's course. For example, if the robot cannot see an object involved in the interaction, it explains to the child why it cannot see it (e.g., ``I cannot see it because it is too high")____. The actual cause of an event is analyzed using a rule-based system, as opposed to a search over the possible counterfactuals____. The usefulness of the explanations generated by the system was evaluated by asking a group of observers to watch videos of the robot providing explanations in different situations and then rate each explanation. The rating was based on qualitative criteria, e.g., whether the explanation was understandable, sufficiently detailed, or informative about the interaction, among other aspects. 
	
	____ use the Halpen and Pearl actual causation framework to retrieve explanations of failure events in unmanned aerial vehicles (UAVs). In this context, an actual causation analysis aims to provide actionable explanations, that is, an explanation that indicates which corrective actions can be taken to prevent future failures. The analysis of actual failure causes was applied to different UAV failure scenarios (loss of control, events during take-off and cruise, and equipment problems). The causal models of failure events were constructed using flight logs recorded at run-time containing abstracted events and raw sensor data. In order to diagnose instances of a particular type of failure (e.g., instances of crash events in the logs), the causal graph of the failure and the actual values of the monitored event and sensor data were analyzed using a tool for the automatic checking of the Halpern and Pearl actual causation conditions. The correctness of the diagnoses was evaluated using a manually labeled ground-truth dataset.
	
	
	____ designed an algorithm to explain the output of neural network-based image classifiers in cases where parts of the classified object are occluded. The algorithm applies concepts of the Halpern and Pearl definition of actual causation to generate explanations. The explanation consists of a subset of image pixels, which is the minimal or approximately minimal subset that allows the neural network to classify the image. It can be tested if a pixel is a cause of the classification by considering a subset of pixels in the image that does not include that pixel. In this case, applying a masking color to any combination of pixels from this subset does not alter the classification output. However, if we apply a masking color to the entire subset along with the individual pixel, this will lead to a change in the classification result. In this way, the algorithm ranks pixels according to their importance for the classification. The authors evaluate the explanations of their system by comparing their results with the outputs of other explanation tools. They compare the size of the explanation (smaller is better) and the size of the intersection of the explanation with the occluding object (smaller is better), getting favorable results for their approach. Additional work on explaining image classifiers was done by ____ and has been applied to extend the work of ____, who used a simplified version of the actual causation framework. The authors show that extending the previous work to the full version of the framework is possible and can benefit future work in explaining image classifiers. However, they suggest that using the full definition might make computing the explanations more complex.