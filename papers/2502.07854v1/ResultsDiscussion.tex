\section{Results and Discussion}

The proposed framework is evaluated quantitatively and qualitatively against the baseline models and the results are depicted in Figure \ref{Fig:metrics_plots}. From the quantitative perspective, the method incorporating wavelet outperforms the LSTM model as expected. To quantify the impact of using decomposed components of the feature, the model $F^{'}$ was trained with and without the decomposed components as part of its feature set. It is evident that the model $F^{'}$ benefits with decomposed components as part of the input feature. Between $F$ and $F^{'}$, both models boosts superior forecasting abilities, $F^{'}$ consistently outperforms the other. $F^{'}$ also exhibits a lower degree of variance in comparison to the other models highlighting consistency with forecasting performance.

The qualitative results for \textit{DMA A} and \textit{B}, shows the ability of the model $F^{'}$  to follow daily and weekly patterns in demand closely than the other models. The baseline models  underforecasts the demand during the weekdays for \textit{DMA A}, while the proposed model captures the actual demand accurately throughout the week across both the DMAs. The model also show robustness to the perturbation in demand seen on the fifth day of \textit{DMA A}, by providing an accurate forecast for the sixth day. 

% Quantitatively, all three models based on wavelet scalograms perform well, with minor differences. While the accuracy, measured by MAPE, reaches around $95\%$, the model incorporating decomposed features outperforms the baseline wavelet method consistently.

These results demonstrate the effectiveness of our proposed model in accurately and consistently forecasting heat demand by learning inter-and-intra feature dependencies weighted through a cross-attention block. Additionally, the proposed methodology reduces the number of trainable parameters of the $F$ by $97\%$ from approximately 155 million to approximately 5.7 million parameters, without losing forecasting performance. The model parameters are listed in Appendix \ref{mp}.



\begin{figure}[htbp] 
    \centering
    \begin{subfigure}[t]{\linewidth} 
        \includegraphics[width=\linewidth]{images/mae_mape.pdf} 
        %\caption{8 days in Summer 2018} 
        \label{Fig:mape}
    \end{subfigure}


    \begin{subfigure}[t]{0.48\linewidth} 
       % \centering
        \includegraphics[width=\linewidth]{images/nord.pdf} 
        %\caption{MAPE - prediction over different zones} 
        %\label{Fig:nord}
    \end{subfigure}
    % \hfill
    \begin{subfigure}[t]{0.48\linewidth} 
        %\centering
        \includegraphics[width=\linewidth]{images/dcentg.pdf} 

    \end{subfigure}
    \caption{Quantitative evaluation for the year 2019 - MAE (top left), and MAPE (top right) across DMAs; Qualitative plots of forecasts over a week in December 2019 for \textit{DMA A} and \textit{DMA B}.}

    \label{Fig:metrics_plots}
\end{figure} 
