\section{Conclusion}

In this study, we introduced a robust framework for forecasting heat demand using a convolution-based model that integrates an attention mechanism and decomposed time series components. 
Our experiments showed that incorporating cross-attention between endogenous and exogenous features significantly improves the model's ability to learn feature-specific patterns and dynamically integrate contextual information. Quantitative evaluations confirmed that, especially with decomposed features, our model consistently outperforms traditional methods like LSTM and baseline wavelet-based models, achieving higher accuracy, reduced forecast variance, and a $97\%$ reduction in trainable parameters.
The architecture facilitates the integration of prior knowledge about demand-influencing factors into feature-specific branches and merges critical information through attention. Additionally, the forecast can be subjected to seasonal decomposition, allowing for intuitive comparisons between the forecasted and actual components, such as trends, seasonality, and residuals, to better understand the model's limitations and to address them with techniques such as regularisation. We believe that this approach, which emphasizes the identification and utilization of meaningful features, not only enhances model performance but also bridges the gap between research and practical application in the field of sustainable solutions.

