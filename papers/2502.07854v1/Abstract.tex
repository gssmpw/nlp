\begin{abstract}

% Global leaders and policy makers are in unison in their unequivocal push towards decarbonisation to facilitate the NetZero agreements due to the significant impact of climate change on the economy, security and life. Utility industries, on one hand, contribute to carbon emission as heating remain largely catered through fossil fuels, natural gas, wood, chips, and on the other hand face the burnt of climate change due its influence on potable water. 

% % Situation ----------------------------------------------
% Global leaders and policymakers are unified in their unequivocal commitment to decarbonization efforts in support of Net-Zero agreements.  % , driven by the profound effects of climate change on the economy, security, and human well-being. 
% % Complication ----------------------------------------------
% The utility sector, while contributing to carbon emissions due to the continued reliance on fossil fuels, natural gas, and wood-chips, for heating, is also vulnerable to climate change, particularly with the constraints to the availability of potable water. 
% As demographic demands grow and renewables become central strategy in decarbonising the heating sector, the need for accurate demand forecasting has intensified.
% %to boost operational and economic efficiency.
% % Solution ------------------------------------------------------
% Advances in metering technologies have paved the way for machine learning-based solutions to become the industry standard in time series forecasting. With a multitude of approaches to choose from, it is vital to evaluate how effectively algorithms manage diverse time series data, characterized by the specific water and heat demand patterns of various districts.
% This study assesses the forecasting performance of a baseline convolutional network that utilizes continuous wavelet transform to represent time series data as scalograms, both with and without the incorporation of an attention mechanism. 
% % Results and Observation - state the impact by quantifying the results
% We also compare the performance of the model with traditional statistical models such as ARIMA and DL models such as LSTM. 





% Situation ----------------------------------------------
Global leaders and policymakers are unified in their unequivocal commitment to decarbonization efforts in support of Net-Zero agreements.  % , driven by the effects of climate change on the economy, security, and human well-being. 
% Complication ----------------------------------------------
\acrfull{dhs}, while contributing to carbon emissions due to the continued reliance on fossil fuels for heat production, are embracing more sustainable practices albeit with some sense of vulnerability as it  could constrain their ability to adapt to dynamic demand and production scenarios.
As demographic demands grow and renewables become the central strategy in decarbonizing the heating sector, the need for accurate demand forecasting has intensified.
%to boost operational and economic efficiency.
% Solution ------------------------------------------------------
Advances in digitization have paved the way for \acrfull{ml} based solutions to become the industry standard for modeling complex time series patterns. In this paper, we focus on building a \acrfull{dl} model that uses deconstructed components of independent and dependent variables that affect heat demand as features to perform multi-step ahead forecasting of head demand. The model represents the input features in a time-frequency space and uses an attention mechanism to generate accurate forecasts.
% Results and Observation - state the impact by quantifying the results
The proposed method is evaluated on a real-world dataset and the forecasting performance is assessed against LSTM and CNN-based forecasting models. Across different supply zones, the attention-based models outperforms the baselines quantitatively and qualitatively, with an \acrfull{mae} of $0.105 \pm 0.06 kWh$ and a \acrfull{mape} of $5.4\% \pm 2.8 \%, $ in comparison the second best model with a \acrshort{mae} of $0.10 \pm 0.06 kWh$ and a \acrshort{mape} of $5.6\% \pm 3\%$. 








  % The abstract paragraph should be indented \nicefrac{1}{2}~inch (3~picas) on
  % both the left- and right-hand margins. Use 10~point type, with a vertical
  % spacing (leading) of 11~points.  The word \textbf{Abstract} must be centered,
  % bold, and in point size 12. Two line spaces precede the abstract. The abstract
  % must be limited to one paragraph.
\end{abstract}