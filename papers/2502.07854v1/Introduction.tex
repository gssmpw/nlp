\section{Introduction}


As the effects of climate change become increasingly pronounced, global policymakers are intensifying their efforts to combat these issues through climate-friendly policies \cite{european2012energy}. A major focus of these initiatives is the pursuit of carbon neutrality by decarbonizing energy systems within sustainable energy frameworks. Despite progress in renewable energy technologies, fossil fuels still provide 50\% of Europe’s primary energy, with 63\% of the residential sector's final energy consumption used for heating \cite{paardekooper2018heat}. District heating networks have significant potential for decarbonization through the integration of renewable energy sources \cite{steiner2015district}, underscoring the importance of precise heat demand forecasting methods. Such accuracy is crucial for efficient resource management, reducing energy waste, and facilitating the successful incorporation of renewables into existing energy systems.


With the growing proliferation of smart devices, time series modeling has greatly advanced, benefiting from the richer data streams and the progress in Machine Learning (ML) and Deep Learning (DL). While these advancements have led to sophisticated models, such as attention mechanisms that enable a model to prioritize and focus on the most critical parts of the input data, traditional statistical methods like ARIMA and SARIMAX remain popular \cite{DOTZAUER2002277}, \cite{FANG2016544}, \cite{chatterjee2021prediction}. These methods are valued for their intuitive additive and multiplicative approaches, offering a level of interpretability that complements the complexity of newer techniques. As attention-based models \cite{vaswani2023attentionneed}, including transformer models for time series, Large Language Models (LLMs) with embedded context, and foundation models for time series forecasting \cite{das2024decoderonlyfoundationmodeltimeseries}, become more prevalent, they still rely on well-crafted inputs to learn contextual representations effectively. Particularly with heat demand data, which exhibits strong daily and weekly seasonality along with yearly trends dictated by weather conditions, combining intuitive feature aggregation with these advanced models enhances their ability to capture the residual fluctuations that characterize time series data. This synergy between traditional and modern approaches is crucial for developing industrial applications aimed at reducing carbon emissions.

Given the widespread use of statistical modeling for its interpretability and the complex representations learned by DL models, we propose a DL-based forecasting approach that combines that employs an attention mechanism and incorporates individual components of decomposed time series as input features, allowing the model to capture the nuanced factors that influence heat consumption. By elucidating these individual components, we identify the key drivers of next-day heat demand. The effectiveness of the proposed approach is thoroughly evaluated using real-world heat demand data over an entire year, providing insights into its performance across different phases of the annual trend.

