




% \usepackage{placeins}
% \usepackage{tcolorbox}
% \usepackage{listings}
% \lstset{language=Python, basicstyle=\Large\ttfamily, keywordstyle=\color{blue}}
% \newtcolorbox{mybox}{
%     colback=gray!20, %
%     colframe=black, %
%     arc=1mm, %
%     boxrule=1pt, %
%     left=1mm, %
%     right=1mm, %
%     top=1mm, %
%     bottom=1mm %
% }


% \usepackage{tcolorbox}
% \colorlet{shadecolor}{gray!20}
% \definecolor{babyblueeyes}{rgb}{0.63, 0.79, 0.95}
% \definecolor{babyblue}{rgb}{0.54, 0.81, 0.94}
% \definecolor{bluegray}{rgb}{0.4, 0.6, 0.8}
% \definecolor{cadmiumgreen}{rgb}{0.0, 0.42, 0.24}
% \definecolor{camouflagegreen}{rgb}{0.47, 0.53, 0.42}
% \definecolor{darkseagreen}{rgb}{0.56, 0.74, 0.56}
% \definecolor{lightgray}{RGB}{239,240,241}

 
% \usepackage{tikz}
% \usetikzlibrary{fit,calc}
% \newcommand*{\tikzmk}[1]{\tikz[remember picture,overlay,] \node (#1) {};\ignorespaces}
% \newcommand{\boxit}[1]{\tikz[remember picture,overlay]
% {\node[xshift=-0pt,yshift=-0pt,fill=#1,opacity=.15,fit={(A)($(B)+(0.9\linewidth,.8\baselineskip)$)}] {};}\ignorespaces}      

% \colorlet{mypink}{red!30}
% \colorlet{myblue}{cyan!50}
% \colorlet{mygray}{gray!60}

\newcommand{\highlightbox}[1]{\colorbox[RGB]{239,240,241}{#1}}

\DeclareMathOperator*{\argmax}{arg\,max} % Custom operator definition
\DeclareMathOperator*{\argmin}{arg\,min}




% \definecolor{c0}{cmyk}{1,0.3968,0,0.2588} 
% \definecolor{c1}{cmyk}{0,0.6175,0.8848,0.1490} 
% \definecolor{c2}{cmyk}{0.1127,0.6690,0,0.4431} 
% \definecolor{c3}{cmyk}{0.3081,0,0.7209,0.3255} 
% \newtcbox{\hlprimary}{on line,colback=c0!10,colframe=white,size=fbox,arc=3pt, box align=base,before upper=\strut, top=-2pt, bottom=-4pt, left=-1pt, right=-1pt, boxrule=0pt}
% \newtcbox{\hlprimarytab}{on line, box align=base, colback=c0!10,colframe=white,size=fbox,arc=3pt, before upper=\strut, top=-2pt, bottom=-4pt, left=-2pt, right=-2pt, boxrule=0pt}
% \newtcbox{\hlsecondary}{on line,colback=c1!10,colframe=white,size=fbox,arc=3pt, box align=base,before upper=\strut, top=-2pt, bottom=-4pt, left=-1pt, right=-1pt, boxrule=0pt}
% \newtcbox{\hlsecondarytab}{on line, box align=base, colback=c1!20,colframe=white,size=fbox,arc=3pt, before upper=\strut, top=-2pt, bottom=-4pt, left=-2pt, right=-2pt, boxrule=0pt}
% \newtcolorbox{hlmultiline}{on line,colback=decentgrey!75,colframe=white,size=fbox,arc=3pt, box align=base, top=0pt, bottom=2pt, boxrule=0pt, before=\adjustbox{valign=c}\bgroup, after=\egroup, before upper=\strut}

% \newcolumntype{Y}{>{\centering\arraybackslash}X}
% \newcolumntype{Z}{>{\raggedleft\arraybackslash}X}


% \newcommand\mask{\_\_}
% \newcommand\given{\,{\mid}\,}
% \newcommand{\bt}{\fontseries{b}\selectfont}
% \newcommand{\dashifted}{{\tiny$\downarrow$}}
% \newcommand{\da}[1]{{\scriptsize\hlprimarytab{\dashifted{#1}}}}
% \newcommand{\dan}[1]{{\scriptsize\hlprimarytab{\,\dashifted\,{#1}}}}
% \newcommand{\uashifted}{{\tiny$\uparrow$}}
% \newcommand{\ua}[1]{{\scriptsize\hlsecondarytab{\uashifted{#1}}}}
% \newcommand{\uab}[1]{{\scriptsize\hlprimarytab{\uashifted{#1}\%}}}
% \newcommand{\negphantom}[1]{\settowidth{\dimen0}{#1}\hspace*{-\dimen0}}

% \definecolor{c4}{cmyk}{0.6765,0.2017,0,0.0667} 
% \definecolor{c5}{cmyk}{0,0.8765,0.7099,0.3647} 

% \definecolor{darkgrey}{RGB}{149,149,149}
% \definecolor{decentgrey}{RGB}{242,242,242}



% \usepackage{amsthm,amsmath}
% \theoremstyle{plain}
% \newtheorem{theorem}{Theorem}
% \newtheorem{lemma}{Lemma}
% \newtheorem{proposition}{Proposition}
% \newtheorem{claim}{Claim}
% \newtheorem{example}{Example}
% % \newtheorem{corollary}{Corollary}
% \newtheorem{remark}{Remark}
% \newtheorem{definition}{Definition}
% \newtheorem{assumption}{Assumption}
% \newtheorem{question}{Question}
% \newtheorem{observation}{Observation}
% \newtheorem{conjecture}{Conjecture}


\crefname{section}{\S}{\S}
\crefname{table}{Tb.}{Tbs.}
\crefname{appendix}{App.}{Apps.}
\Crefname{theorem}{Thm.}{Thms.}
\Crefname{proposition}{Prop.}{Props.}
\crefname{algorithm}{Alg.}{Algs.}
\Crefname{assumption}{Asm.}{Asms.}
\crefname{mechanism}{Mech.}{Mechs.}
\Crefname{definition}{Def.}{Def.}


\crefformat{footnote}{#2\footnotemark[#1]#3}
\crefmultiformat{footnote}{#2\footnotemark[#1]#3}%
{\textsuperscript{,}#2\footnotemark[#1]#3}{\textsuperscript{,}#2\footnotemark[#1]#3}{\textsuperscript{,}#2\footnotemark[#1]#3}

\newcommand{\red}[1]{\textcolor{red}{#1}}
\newcommand{\green}[1]{\textcolor{OliveGreen}{#1}}
\newcommand{\orange}[1]{\textcolor{orange}{#1}}
\newcommand{\blue}[1]{\textcolor{cyan}{#1}}
\definecolor{mygray}{gray}{0.6}
\newcommand{\gray}[1]{\textcolor{mygray}{#1}}
\newcommand{\authnote}[2]{{\bf \textcolor{blue}{#1}: \em \textcolor{red}{#2}}}


\newcommand{\chulin}[1]{\textcolor{black}{#1}}
\newcommand{\mynote}[1]{{\color{blue}#1}}
\newcommand{\zinan}[1]{\authnote{Zinan}{#1}}
\newcommand{\jana}[1]{\authnote{Jana}{#1}}
\newcommand{\sergey}[1]{\authnote{Sergey}{#1}}
\newcommand{\gopi}[1]{\authnote{Gopi}{#1}}
\newcommand{\harsha}[1]{\authnote{Harsha}{#1}}
\newcommand{\sepideh}[1]{\authnote{Sepideh}{#1}}



\newcommand{\myparatight}[1]{\noindent{\bf {#1}:}~}

\newcommand{\myparatightest}[1]{\noindent\textbf{{#1:}}~}
\newcommand{\myparatightestn}[1]{ \noindent\textbf{{#1}}}

\newcommand{\myparaemphtightestn}[1]{\noindent\emph{{#1}}}

% \newcounter{packednmbr}
% \newenvironment{packedenumerate}{\begin{list}{\thepackednmbr.}{\usecounter{packednmbr}\setlength{\itemsep}{0.5pt}\addtolength{\labelwidth}{-4pt}\setlength{\leftmargin}{\labelwidth}\setlength{\listparindent}{\parindent}\setlength{\parsep}{1pt}\setlength{\topsep}{0pt}}}{\end{list}}
% \newenvironment{packeditemize}{\begin{list}{$\bullet$}{\setlength{\itemsep}{0.5pt}\addtolength{\labelwidth}{-4pt}\setlength{\leftmargin}{\labelwidth}\addtolength{\leftmargin}{-2pt}\setlength{\listparindent}{\parindent}\setlength{\parsep}{1pt}\setlength{\topsep}{0pt}}}{\end{list}}
% \newenvironment{packedtrivlist}{\begin{list}{\setlength{\itemsep}{0.2pt}\addtolength{\labelwidth}{-4pt}\setlength{\leftmargin}{\labelwidth}\setlength{\listparindent}{\parindent}\setlength{\parsep}{1pt}\setlength{\topsep}{0pt}}}{\end{list}}


% \NewDocumentCommand{\codeword}{v}{%
% \texttt{\textcolor{blue}{#1}}%
% }


% \newcommand{\size}[2]{{\fontsize{#1}{0}\selectfont#2}}
% \newenvironment{sizepar}[2]
%  {\par\fontsize{#1}{#2}\selectfont}
%  {\par}
