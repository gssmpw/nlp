\section{Related Work}
\subsection{Retrieval Augmented Generation}
Retrieval-Augmented Generation (RAG) is a hybrid approach that integrates retrieval systems and generative models to enhance factual accuracy and contextual relevance in natural language generation ____. Unlike conventional language models that rely solely on parametric memory, RAG dynamically retrieves relevant external knowledge before generating a response.
One of the foundational works in RAG is %the model proposed by 
____, where a retrieval module fetches relevant passages from a large-scale knowledge corpus (e.g., Wikipedia), which are then fused with the question context to generate a more informed response. This technique has proven particularly effective in open-domain question answering (QA), fact verification, and context-aware text generation.
RAG systems have expanded beyond text and document retrieval to incorporate %various data formats. 
a wide variety of data types ____ — tables, graphs, charts, and diagrams. %, and has been proposed to integrate.
While RAG has been widely explored, its application in spatial reasoning question answering remains an unexplored research area. Existing studies have primarily focused on knowledge-grounded dialogues ____ but often struggle with integrating spatial computation into the question-answering process effectively.
\subsection{Spatial Questions}
Spatial questions in domain-specific applications can generally be categorized into two distinct types:
\textbf{1) Textual Knowledge-based Spatial Questions} These are spatial questions that can be answered by traditional QA methods without the need for spatial computation and reasoning ____.
% where both the question and its corresponding answers reside primarily in textual formats. 
For example, the question \textit{"What is the population of Los Angeles city?"} falls under this category. Despite their spatial context, these questions are essentially text-based and, hence, can be effectively addressed using traditional Retrieval-Augmented Generation (RAG) methods ____.
\textbf{2) Spatial Reasoning Questions} This category encapsulates spatial questions that demand a model's capability to comprehend and reason with spatial data and spatial relationships. A common example is a model being presented with textual information describing the spatial relationships among multiple objects ____. An example question could be, \textit{"What is the position of object A relative to object B?"}, where objects \textit{A} and \textit{B} are locations or entities specified on the map. Resolving such queries requires a profound understanding of spatial concepts and robust reasoning skills, which largely depend on the model’s training to handle spatial data.
Several studies  ____ have investigated the capacity of LLMs to understand spatial concepts, yet these models often struggle with accurate reasoning even after fine-tuning.   Other research ____ has attempted to enhance this ability by converting geolocation coordinates into addresses to enrich the semantic context.   However, these improvements tend to be marginal and are mostly limited to straightforward reasoning tasks like describing positions. Moreover, many existing methods rely on predefined sets of actions tailored to specific tasks.