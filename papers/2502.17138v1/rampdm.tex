% \section{RAMP-DM: RAMP Disaggregated Memory}
% \section{Applying \ramp to Disaggregated Memory}
\section{\rampdm: Resilient Disaggregated Memory}

\begin{figure}[!t]
\centering
\includegraphics[width=2.5in]{fig/rampdm-architecture.pdf}
\caption{\rampdm system architecture.}
\label{fig:rampdm-architecture}
\end{figure}

We design \rampdm, a resilient disaggregated memory system following the \ramp model.
\rampdm builds on Hydra~\cite{lee:hydra:fast:2022}, a state-of-the-art resilience mechanism for disaggregated memory systems that employs rack-level memory replication to tolerate memory-node failures. 
\rampdm extends Hydra with cross-layer memory resilience mechanisms, enabling it to tune memory protection strength, storage overhead, and performance. 


\subsection{System Overview}

Figure~\ref{fig:rampdm-architecture} shows \rampdm's system architecture. 
\rampdm has three main components to support cross-layer resilience: 
(i) \emph{Resilience Manager} coordinates resilience operations during remote read/write, 
(ii) \emph{Resource Monitor} handles memory management in a memory node, and 
(iii) \emph{FlexDIMM} offers a memory module with configurable memory error protection. 
Resilience Manager and Resource Monitor are software-level components inherited from Hydra~\cite{lee:hydra:fast:2022}, which together provide the memory-replication tier.
FlexDIMM is a new hardware-level component, which provides the memory-protection tier.
Resilience Manager and Resource Monitor are extended to interact with FlexDIMM via data-plane and control-plane operations, working together to provide efficient memory resilience.
Data-plane operations for accessing data in remote memory use one-sided remote DMA (RDMA) operations (RDMA READ/WRITE), which enable fast accesses to remote memory without involving the remote node processor.
Control-plane operations for configuring remote memory use two-sided RDMA operations (RDMA SEND/RECV). 
In this architecture, each machine can function as a compute node, a memory node, or both. 

\subsection{Configurable Memory Protection}

FlexDIMM provides configurable memory error protection by reusing the memory chip failure protection bits to detect and opportunistically correct bit errors at high performance.
FlexDIMM targets random bit cell errors, as we expect these to represent the majority of memory errors because of the high RBER of high-density memory technology~\cite{zhang:pm-chipkill:micro:2018, patil:dve:isca:2021}.

We design a FlexDIMM that provides a configurable chipkill-correct protection scheme, leveraging a recent chipkill design for high-density non-volatile memory~\cite{zhang:pm-chipkill:micro:2018}. 
The module employs two protection codes: 
(i) a fixed-protection code reuses the chip failure protection bits to opportunistically correct bit errors at high performance, and 
(ii) a configurable-protection code uses long ECC codewords to correct at a configurable storage cost bit errors that are detected but uncorrected by the fixed-protection code. The configurable code 
%uses a BCH(2312,2048,22) code for each ECC codeword
uses a BCH(n,k,t) code for each ECC codeword of length $n=k+t(\left \lceil{log_2 (k)} \right \rceil+1)$ to correct $t$ bad bits when protecting $k$ bits of data~\cite{zhang:pm-chipkill:micro:2018}.

The FlexDIMM provides a cache-line failure probability due to DUE that is the product of two terms:
the probability that the fixed-protection code fails to correct a bit error (whose value equals to 0.018 as is taken from the original design~\cite{zhang:pm-chipkill:micro:2018}) and 
the probability that the configurable-protection code fails to correct multiple bit errors in the BCH codeword (which happens when there are at least $t$ bit errors):

\begin{equation*}
\pcdue= 0.018 \times \sum_{i=t+1}^{n}\binom{n}{i}{RBER}^i\cdot{(1-RBER)}^{n-i}
\end{equation*}

The cache-line failure probability due to DUE can be used as input to the \ramp analytical model to determine and tune the length of the BCH codeword in combination with rack-scale memory replication. 
% First, we compute the base cache-line failure probability of the configurable tier due to DUE as the probability that the configurable tier fails to correct multiple bit errors in the BCH codeword (which happens when there are at least $t$ bit errors):
% 
% \begin{equation*}
% \pcdue= \sum_{i=t+1}^{n}\binom{n}{i}{RBER}^i\cdot{RBER}^{n-i}
% \end{equation*}
The model can estimate the combined DUE rate resulting from using available replicas to correct DUEs. 
Although not shown, the cache-line failure probability due to NDE can be also calculated and used as input to the model, following the analysis of Kim and Lee~\cite{kim:undetected-error-bch:ieee-tc:1996}.

The FlexDIMM leverages existing hardware error reporting mechanisms, such as Intel Machine Check Architecture (MCA)~\cite{intel:mce:book:2024}, to raise a machine check exception (MCE) in response to uncorrectable memory errors.
For error reporting mechanisms that do not provide a mechanism for detecting write failures, like Intel MCA, the module may issue an additional read after a write to check the success of the write.
While the module, in theory, offer only error detection without correction, this would leave error correction entirely to the upper memory-replication tier. 
However, the ability to correct errors opportunistically is crucial to avoid a flood of interrupts, which could severely slowdown the system~\cite{meza:dramfailures:dsn:2015}, as corroborated in the evaluation.

The default response to an MCE in the kernel is a kernel panic. However, recent Linux kernels allow returning an error to the caller instead of crashing in response to memory-error induced MCEs~\cite{xu:nova-fortis:sosp:2017}.
After the exception, MCA registers hold information that allows the OS to identify the memory address responsible for the exception, allowing the OS to report the error to the memory-replication tier for further handling.
With this approach, after error reporting, a memory node remains operational and continues to serve memory accesses to non-failed memory regions.
Thus, in contrast to previous work where uncorrectable memory errors may crash a memory node~\cite{shan:legoos:osdi:2018} making all data stored on that node unavailable, our fine-grain failure model enables the memory node to remain operational and serve future requests, improving availability.

Propagating MCEs from the FlexDIMM, which provides the memory-protection tier, to the Resilience Manager, which provides the memory-replication tier, is challenging due to the one-sided semantics of RDMA. 
One approach would have the remote OS handle exceptions by disregarding the error, completing pending DMA requests that triggered the Memory Corrective Error (MCE), and making a callback remote procedure call (RPC) to the Resilience Manager to report the error. 
However, since one-sided operations issued by the Resilience Manager directly access memory and do not involve the remote memory node processor, this process may introduce a race condition between when the Resilience Manager receives the callback RPC and the completion of the one-sided RDMA operation. 
To address this, the FlexDIMM requires extending the RDMA NIC hardware, which handles the RDMA request, to support propagating memory errors to the Resilience Manager by piggybacking on the existing error reporting mechanism of RDMA. 
The OS kernel informs the NIC of the MCE, and the NIC reports the memory error back to the Resilience Manager as part of the response to the RDMA operation, indicating an ECC failure that requires further handling. 


\subsection{Rack-scale Memory Replication}

The Resilience Manager works together with the Resource Monitor to provide the memory-replication tier. This tier uses rack-scale replication to correct memory errors that are detected but uncorrected by the memory-protection tier. 
The Resilience Manager provides a remote memory abstraction to client applications, exposing remote memory via a paging system integrated with virtual memory. 
This system transparently moves memory pages between local and remote memory using one-sided RDMA operations, enabling applications to access remote memory without any changes to the application code. 
% When an application attempts to access a virtual memory address that is not backed by a physical page in local memory, a page fault occurs. 
% This triggers remote paging to retrieve the corresponding page from remote memory over the network using one-sided remote DMA (RDMA) read operations. 
% Subsequently, any page that has been evicted from local memory is transferred to remote memory using one-sided RDMA write operations. 
The Resilience Manager runs in kernel mode and runs on every client (compute node) that needs access to remote memory.  
The Resource Monitor manages memory on a memory node and makes it available to the remote Resilience Manager. 
The Resource Monitor runs in user space and only participates in control-plane activities, responding to requests from Resilience Manager for allocating physical pages and configuring page protection settings.

The Resilience Manager handles all aspects of redundancy. 
For each virtual page, it ensures that the page is stored on multiple physical-page replicas, distributed across different nodes within the rack, to enhance availability. 
The Resilience Manager allocates physical pages through requests made to the Resource Monitor and 
and communicates with the Resource Monitor to configure the memory protection scheme and strength of each physical page to meet a target UBER and SDC rate, according to application requirements.
The Resilience Manager tracks and blacklists failed memory regions to avoid mapping replicas to regions with known errors. When the Resilience Manager encounters a data access error (DUE), it attempts to correct the memory error using another replica.





\subsection{Putting it all together}

Applications define their required level of resilience, which is communicated to the Resilience Manager. The Resilience Manager, in turn, ensures the appropriate redundancy is in place to meet these resilience requirements.
Applications access remote memory through virtual memory. 
When an application attempts to access a virtual memory address that is not backed by a physical page in local memory, a page fault occurs. This triggers remote paging to retrieve the corresponding page from remote memory over the network using one-sided RDMA operations. 
The memory controller of the remote memory node processing the RDMA request uses the page protection information to identify which protection technique to use for any given memory access.
If a memory access triggers a DUE, the memory node propagates it back to the Resilience Manager as an RDMA error, which is then handled by the Resilience Manager using other remote replicas.