\subsection{Rack-scale Memory Replication}

The Resilience Manager works together with the Resource Monitor to provide the memory-replication tier. This tier uses rack-scale replication to correct memory errors that are detected but uncorrected by the memory-protection tier. 
The Resilience Manager provides a remote memory abstraction to client applications, exposing remote memory via a paging system integrated with virtual memory. 
This system transparently moves memory pages between local and remote memory using one-sided RDMA operations, enabling applications to access remote memory without any changes to the application code. 
% When an application attempts to access a virtual memory address that is not backed by a physical page in local memory, a page fault occurs. 
% This triggers remote paging to retrieve the corresponding page from remote memory over the network using one-sided remote DMA (RDMA) read operations. 
% Subsequently, any page that has been evicted from local memory is transferred to remote memory using one-sided RDMA write operations. 
The Resilience Manager runs in kernel mode and runs on every client (compute node) that needs access to remote memory.  
The Resource Monitor manages memory on a memory node and makes it available to the remote Resilience Manager. 
The Resource Monitor runs in user space and only participates in control-plane activities, responding to requests from Resilience Manager for allocating physical pages and configuring page protection settings.

The Resilience Manager handles all aspects of redundancy. 
For each virtual page, it ensures that the page is stored on multiple physical-page replicas, distributed across different nodes within the rack, to enhance availability. 
The Resilience Manager allocates physical pages through requests made to the Resource Monitor and 
and communicates with the Resource Monitor to configure the memory protection scheme and strength of each physical page to meet a target UBER and SDC rate, according to application requirements.
The Resilience Manager tracks and blacklists failed memory regions to avoid mapping replicas to regions with known errors. When the Resilience Manager encounters a data access error (DUE), it attempts to correct the memory error using another replica.


