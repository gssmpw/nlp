\section{Conclusion}
We introduced \ramp, a framework for designing and tuning computing systems combining memory replication with memory protection to tolerate memory errors efficiently.
We demonstrate the utility of the \ramp framework by applying it to a state-of-the-art resilient disaggregated memory design that utilizes memory replication for availability, enabling a reduction in storage overhead for memory protection without compromising overall resilience.

A current limiting factor in aggressively relaxing memory protection for higher storage savings is the risk of increasing NDEs, where silently corrupted data might escape correction by the memory-replication tier. Future work could explore application-level checksumming to detect silent data corruption in the memory-replication tier. 
% These checksumming techniques could potentially be implemented in a programmable NIC to minimize latency overhead.

% Overall, our results are promising, demonstrating that relaxing memory protection strength can yield significant storage savings without compromising overall resilience, all while maintaining minimal performance overhead.


Overall, our results are promising, demonstrating that \ramp can facilitate the cost-optimization of two-tier memory resilience schemes. Our case study shows that relaxing memory protection strength can yield significant storage savings without compromising overall resilience, all while maintaining minimal performance overhead.


