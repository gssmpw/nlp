\subsection{System Overview}

Figure~\ref{fig:rampdm-architecture} shows \rampdm's system architecture. 
\rampdm has three main components to support cross-layer resilience: 
(i) \emph{Resilience Manager} coordinates resilience operations during remote read/write, 
(ii) \emph{Resource Monitor} handles memory management in a memory node, and 
(iii) \emph{FlexDIMM} offers a memory module with configurable memory error protection. 
Resilience Manager and Resource Monitor are software-level components inherited from Hydra~\cite{lee:hydra:fast:2022}, which together provide the memory-replication tier.
FlexDIMM is a new hardware-level component, which provides the memory-protection tier.
Resilience Manager and Resource Monitor are extended to interact with FlexDIMM via data-plane and control-plane operations, working together to provide efficient memory resilience.
Data-plane operations for accessing data in remote memory use one-sided remote DMA (RDMA) operations (RDMA READ/WRITE), which enable fast accesses to remote memory without involving the remote node processor.
Control-plane operations for configuring remote memory use two-sided RDMA operations (RDMA SEND/RECV). 
In this architecture, each machine can function as a compute node, a memory node, or both. 
