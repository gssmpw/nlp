\section{Replication-Aware Memory-error Protection}
\label{sec:shepherd}

We propose \emph{\textbf{R}eplication-\textbf{A}ware \textbf{M}emory-error \textbf{P}rotection} (\ramp), a framework for designing and tuning computing systems employing memory replication with memory protection to tolerate memory errors.
% to efficiently tolerate memory errors due to random bit cell errors 
% in disaggregated memory.
% We focus on random bit cell errors, as we expect these to represent the majority of memory errors because of the high RBER of high-density NVM.

% replication => gives choices to repair
% checksum => improves detection strength/ability
% so we can weaken the hardware protection or vice versa.

\begin{figure}[!t]
\centering
\includegraphics[width=2.5in]{fig/ramp-system-model.pdf}
\caption{Replication-Aware Memory-error Protection.}
\label{fig:ramp-architecture}
\end{figure}

\subsection{System Model}
% \ramp targets a memory system that combines memory replication with memory protection (ECC) to tolerate memory errors, as shown in 
Figure~\ref{fig:ramp-architecture} presents an abstract system model of computing systems that employ memory replication together with memory protection.
This model captures a variety of systems, ranging from conventional NUMA systems~\cite{patil:dve:isca:2021} to emerging disaggregated memory architectures~\cite{lee:hydra:fast:2022, zhou:carbink:osdi:2022, tsai:dpm:atc:2020, shan:legoos:osdi:2018}.
In this model, compute nodes provide processing power, while memory nodes offer memory capacity and implement memory protection (e.g., ECC) for their memory. 
Although memory nodes are shown as separate from compute nodes in this abstract model, the actual implementation may integrate them, as seen in NUMA systems, where a NUMA node consists of both processors and direct-attached memory~\cite{patil:dve:isca:2021}. Compute nodes interact with memory nodes through a replication engine (RE), which implements the replication logic and coordinates data replication across multiple memory nodes.


\subsection{Tolerating Memory Errors}
\ramp assumes a computing system tolerates memory errors through a two-tier protection scheme. 
The first tier, or memory-protection tier, is a performance-optimized tier that reuses the memory chip failure protection bits (\eg, ECC) to detect and opportunistically correct memory errors at high performance, while ensuring that no miscorrection occurs.
The second tier, or memory-replication tier, is a storage-optimized tier that employs memory replication across multiple memory nodes to correct memory errors that are uncorrected by the memory-protection tier, including errors due to cell failures, chip failures, channel failures, and complete memory node failures

The memory-protection tier provides configurable protection against memory errors.
It requires that the memory controller be able to compute and decode different codes, as well as a mechanism to determine which ECC technique to use for each memory access. 
For example, the memory-protection tier could support page-granularity protection by augmenting the virtual memory page table and TLB to include protection information for each page~\cite{yoon:virtualized-ecc:asplos:2010}. 
The configuration capability can range from allowing upper tiers select a memory protection scheme from a fixed set of protection schemes (\eg, SEC-DEC, chipkill) to providing complete flexibility in choosing the protection method and strength~\cite{yoon:virtualized-ecc:asplos:2010}.

Memory nodes that fail correction at the memory-protection tier report DUEs to the replication engine for further handling. 
Memory nodes can report DUEs either by piggybacking on coherence requests~\cite{patil:dve:isca:2021} or by leveraging hardware error reporting mechanisms, such as Intel Machine Check Architecture (MCA)~\cite{intel:mce:book:2024, xu:nova-fortis:sosp:2017}, to raise a machine check exception (MCE) in response to uncorrectable memory errors.
After reporting, a memory node remains operational and continues to serve memory accesses to non-failed
memory regions, thus improving availability. 
Depending on the overhead of the error reporting mechanism, the ability to opportunistically correct errors in the memory-protection tier, in addition to detection, becomes crucial for preventing severe performance slowdowns due to error reporting~\cite{meza:dramfailures:dsn:2015, barroso:wsc:book:2018}.

For each replicated data item, the memory-replication tier maintains multiple replicas across memory nodes. 
The memory-replication tier maps each replica to a memory node and memory region, and configures the hardware protection strength of each replica to meet a target UBER and SDC rate, following application requirements. 
% The memory-replication tier configures memory protection method (\eg, SEC-DEC, chipkill) and code strength at the granularity of individual pages. 
The memory-replication tier may track and blacklist failed memory regions to avoid mapping replicas to regions with known errors. 
When the memory-replication tier accessing a data item faces a DUE, it attempts to correct the memory error using another replica.

The memory-replication tier can implement any block-level static homogeneous replication method, including primary-backup replication, chain replication, quorum-based replication, and erasure coding. 
Static requires a fixed number of replicas whose protection strength does not change dynamically, thus relieving the replication engine from having to support frequent protection changes.
Homogeneous requires all replicas to have the same protection strength, thus simplifying replica strength reasoning.

\ignore{
    A lightweight service processor on the memory node handles the exception and returns an error as a response to the RDMA request by piggybacking on the existing error reporting mechanism of RDMA.
}
\ignore{
    To serve control plane operations and support fine-grain error reporting, the memory nodes include a lightweight service processor.
}

\ignore{
On read failure, it redirects the request to another replica.
On write failure, if the memory node remains operational, then it may attempt to write to another memory region within the same memory node. Otherwise it the memory node fails completely, software issues the writes to another memory node, and also remaps/migrates (asynchronously) all other replicas of the failed memory node. 
}

\ignore{
operation:
-compute nodes access memory using RDMA; rdma offers robust failure semantics compared to ld/st; piggyback on existing error reporting mechanism
-when an RDMA causes a memory side error, memory nodes do not crash, report operation failure, compute nodes recover by trying another replica, memory nodes remain functional
-memory nodes do not crash; instead poison affected region and continue servicing other requests (rely on an extended from of intel machine check architecture); current mce raises exception on read errors only. report both reads and non-posted-writes. we expect replication protocols to use non-posted writes to ensure writes reach nvm (cite snia)
}

\ignore{
Application software running on compute nodes may tolerate memory errors using application-level redundancy in the form of replication and checksumming.
\ramp does not dictate or implement a specific redundancy scheme, leaving the implementation to the application for maximum flexibility. 
Because targeted applications already employ redundancy for performance and availability (\cref{sec:ramp:idea}), we do not expect this requirement to significantly burden applications. 

Applications may use replication to tolerate DUEs. 
For each replicated data item, an application maintains multiple replicas across memory nodes.
Applications map each replica to a memory node and memory region, and configure the hardware protection strength of each replica to meet a target UBER and SDC rate.
Applications may also track and blacklist failed memory regions to avoid mapping replicas to regions with known errors.
When an application trying to access a data item faces a DUE, it attempts to correct the memory error using another replica. 

Applications may use checksumming to tolerate NDEs, including non-detectable bit cell errors and scribbles, that would otherwise silently corrupt data.
With checksumming, an application maintains a checksum for each data item.
When the application writes a data item, it computes and stores a corresponding checksum. When the application later reads the data, it may recompute the checksum and verify that the computed checksum matches the stored checksum.
}
\ignore{Application-level checksumming increases CPU utilization, but provides end-to-end protection against silent data corruption.}

\ignore{
Software running on a compute node accesses disaggregated memory using one-sided remote DMA (RDMA) reads and writes. 
When the network interface card (NIC) at a memory node receives an RDMA request, it performs a local DMA request to the node's memory controller, which in turn issues memory accesses to memory media.
The controller uses hardware ECC to detect and correct memory errors, and leverages existing hardware error reporting mechanisms, such as Intel Machine Check Architecture (MCA), to report DUEs. 
The controller transparently corrects correctable errors, and silently returns invalid data for undetectable errors.
For DUEs, the controller raises a hardware exception in response to uncorrectable memory errors. 
A lightweight service processor on the memory node handles the exception and returns an error as a response to the RDMA request by piggybacking on the existing error reporting mechanism of RDMA. 
After reporting the error, the memory node continues normal operation by servicing other pending RDMA requests. 
For error reporting mechanisms, that do not provide a mechanism for detecting store failures, like Intel Machine Check Architecture (MCA), the NIC issues an additional read after a write to check success of the write. 
}

Overall, \ramp enables co-designing memory replication together with memory protection to trade-off protection strength, storage efficiency, and performance.
\ignore{Maintaining multiple replicas across memory nodes enables the power of many choices. Instead of trying hard to eliminate uncorrectable memory errors within a single memory node using strong but expensive codes, \ramp accepts the possibility of uncorrectable errors.}
Computing systems that maintain multiple replicas across memory nodes can employ weaker but lower-storage-overhead ECC within individual replicas. 
While weaker ECC increases failure rate of individual replicas, a memory system can rely on the multiple choices offered by available replicas in other memory nodes to correct a memory error. 
% Similarly, memory systems can rely on checksums to detect silent corruptions that may evade the weaker ECC.
For example, in Figure~\ref{fig:ramp-architecture}, applications A and B have two replicas per page, so we can use weaker ECC, relying on the collective protection of multiple replicas to tolerate the increased per-replica error rate.
% Application C uses no replication so we deploy stronger ECC, as we rely exclusively on ECC to tolerate memory errors.



% For example, Figure~\ref{fig:ramp-architecture} shows three applications A, B, and C with different degrees of replication and levels of protection.
% Application A and B maintain two replicas per data item so they can use weaker ECC, relying on the collective protection of multiple replicas to tolerate the increased per-replica error rate.
% Application C uses no replication so it deploys stronger ECC as it relies exclusively on ECC to tolerate memory errors.

\subsection{Choosing Replica Protection Strength}

A key challenge in applying \ramp is choosing the right hardware protection strength of individual replicas. 
Weakening hardware-level protection of individual replicas lowers storage cost but makes DUEs and NDEs more frequent, increasing UBER and SDC rates respectively.
We can recoup the lost UBER by correcting a DUE using available replicas, at the expense of a performance overhead to access and process additional replicas.
However, we cannot always rely on replicas to recoup the lost SDC rate. 
This is because a NDE that silently corrupts data may not trigger a correction by the storage-optimizer tier, unless the application can detect the error through other means, such as checksumming~\cite{zhang:pangolin:atc:2019}.
Hence, SDC rate may limit how much we can weaken individual replica strength.

To help application and system designers choose protection strength, we develop an analytical model that estimates the expected reliability and expected performance overhead when using available replicas to correct a DUE. 
For the reliability, we estimate the combined DUE resulting from using replicas to correct a memory error. 
Because NDE does not benefit from replication, we do not compute a combined NDE. 
However, we do estimate the NDE of each individual replica as a lower reliability bound. 
For the performance overhead, we compute the average number of additional replicas that are read to correct a memory error.
We assume that reading and processing each replica contributes fixed network bandwidth and CPU overhead per replica.

Our analytical model targets block-level replication, which is a common replication approach. The model differentiates between logical and physical blocks. The logical block is the unit of recovery, that is the smallest unit of data that can be recovered by the replication protocol. To enable recovery, a replication protocol maps a logical block to multiple physical block replicas stored across multiple memory nodes. Reading a logical block may involve reading one or more physical blocks.

\ignore{
Since we focus on protection techniques against random bit cell errors, our model focuses on read failures caused by uncorrectable memory errors due to random bit cell errors. 
Our model ignores other correlated failures that affect multiple bits and blocks, such as chip or channel failures due to logic circuit errors.
}

\ignore{
The model uses the following parameters: CPU cache-line size: $c$:, Physical-block size: $b$, Cache-line failure probability due to DUE: $p_c$, Physical-block failure probability due to DUE: $p_b$.
}

\begin{table}
\caption{Analytical model symbol notation}
\label{tab:model}
\centering
\begin{tabular}{lp{6.5cm}}
\textbf{Symbol(s)} & \textbf{Description} \\
$c$, $b$        & \scriptsize{Cache-line size and physical-block size}\\
$\pcdue$, $\pcnde$  & \scriptsize{Cache-line failure probability due to DUE and NDE}\\
$\pbdue$, $\pbnde$  & \scriptsize{Physical-block failure probability due to DUE and NDE}\\
%$\pldue$, $\plnde$  & \scriptsize{Logical-block failure probability due to DUE and NDE}\\
$\pldue$  & \scriptsize{Logical-block failure probability due to DUE}\\
\end{tabular}
\end{table}

The model uses the symbol notation shown in Table \ref{tab:model}.
%
Cache-line failure probability is the probability to fail when reading a cache-line worth of data from the memory system.
This probability depends on the memory protection scheme and the RBER of the underlying memory technology.
Although the RBER may vary over the lifetime of a memory technology, for the sake of simplicity, we adopt a single worst-case value based on the RBER at the end of a specified time period. 
For example, in NVM where the RBER may increase with the amount of time since the last write or refresh~\cite{zhang:pm-chipkill:micro:2018}\ignore{, which can range from a week to a year}, we can use the RBER at the end of the refresh period as the worst-case value.

Physical-block failure probability is the probability to fail when reading a physical-block worth of data from the memory system.
Successfully reading a physical block entails successfully reading all the cache lines that comprise the block. 
We can derive the physical-block failure probability as follows:
\ignore{from the cache-line failure probability:}

% \begin{equation}
% p_b&= 1 - (1-p_c)^{b/c} \quad \mathrm{and} \quad  p_b&= 1 - (1-p_c)^{b/c} 
% \end{equation}

\[
\pbdue= 1 - (1-\pcdue)^{b/c}
\]
\[
\pbnde= 1 - (1-\pcnde)^{b/c}
\]

We next provide analytical models for replication and erasure coding.

\mypar{Replication}
For each logical block, the replication protocol maintains a primary physical block and a sequence of N-1 backup physical replica blocks, with the physical block size equal to the logical block size. 
% For example, for a 4 KB virtual page, each physical block size is also 4KB.

When a compute node needs to read a logical block, it first reads the primary physical block. If the read fails because of a DUE, then it tries the next backup physical block in the sequence, continuing this
process until it successfully reads a block. 
When the compute node exhausts trying all available physical blocks without successfully reading one, the logical-block read fails with a DUE.

Hence, the probability to have a DUE when reading a logical block is the joint probability of all physical-block reads to fail:

\[
\pldue = \pbdue^{N}
\]

The probability to have a NDE when reading a logical block is the union probability of any physical-block reads to fail due to NDE:

\[
\plnde = \sum_{i=0}^{N-1} (1-\pbdue)^i \pbnde
\]

The average number of additional physical blocks that are read after failing to read the first physical block is:

\[
\begin{split}
a_r&= -1 + \sum_{i=0}^{N-1} \pbdue^i(1-\pbdue)(i+1)
\end{split}
\]

% \[
% \begin{split}
% a_r&= -1 + Np^{N-1}+\sum_{i=0}^{N-2} p_b^i(1-p_b)(i+1)
% \end{split}
% \]

\mypar{Erasure Coding}
Erasure coding provides redundancy without the overhead of complete replication.
Erasure coding divides a logical block into K physical blocks,
and then encodes these blocks using Reed-Solomon code RS(K, N)~\cite{reed:code:journal-applied-math:1960} to generate R parity blocks, for a total of N=K+R blocks.
It finally writes each of these N blocks to a different remote memory node. 
Physical block size is equal to $\frac{\text{logical block size}}{K}$.
% For example, for a 4 KB virtual page, each physical block size is $\frac{4}{K}$KB.

When a compute node needs to read a logical block, it can perform the read using any K physical blocks of the N physical blocks. If the compute node fails to read any of the physical blocks (due to a DUE), then it tries another physical block. 
When the compute nodes exhausts trying all available physical blocks without successfully reading K blocks, the logical-block read fails with a DUE.

Hence, the probability to have a DUE when reading a logical block is the probability to have at least N-K+1 physical blocks fail due to a DUE:

% \begin{equation*}
% \pldue = \sum_{i=N-K+1}^{N}\binom{N}{i}\pbdue^{i}(1-\pbdue)^{N-i}
% \end{equation*}

\begin{equation*}
\pldue = \sum_{i=N-K+1}^{N}f(i,N,\pbdue)
\end{equation*}

\begin{equation*}
\mathrm{where}\quad
f(k,n,p) = \binom{n}{k}p^{k}(1-p)^{n-k}
\end{equation*}

The probability to have a NDE is the number of ways in which we can read K physical blocks plus extra physical blocks due to DUE multiplied by the probability to have physical blocks fail due to DUE multiplied by the probability to have at least one physical block fail due to a NDE: 

\begin{multline*}
\plnde = \sum_{i=0}^{N-K}\binom{N}{K+i}f(i,K+i-1,\pbdue) \\ 
\times \sum_{j=1}^{K-1}f(j,K-1,\pbnde)
\end{multline*}

The average number of additional physical blocks that are read after failing to read any of the first K physical blocks is:

% \begin{equation*}
% a_r= -K + \sum_{i=0}^{N-K} \binom{N}{K+i} \binom{K+i-1}{i}\pbdue^{i}(1-\pbdue)^{K-1}(K+i)
% \end{equation*}

\begin{equation*}
a_r= -K + \sum_{i=0}^{N-K} \binom{N}{K+i} f(i, K+i-1, \pbdue) (K+i)
\end{equation*}

\noindent where each sum term is the number of ways in which we can read physical blocks multiplied by the probability to have physical blocks fail due to DUE multiplied by the number of physical blocks read.
