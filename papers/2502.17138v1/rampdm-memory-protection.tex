\subsection{Configurable Memory Protection}

FlexDIMM provides configurable memory error protection by reusing the memory chip failure protection bits to detect and opportunistically correct bit errors at high performance.
FlexDIMM targets random bit cell errors, as we expect these to represent the majority of memory errors because of the high RBER of high-density memory technology~\cite{zhang:pm-chipkill:micro:2018, patil:dve:isca:2021}.

We design a FlexDIMM that provides a configurable chipkill-correct protection scheme, leveraging a recent chipkill design for high-density non-volatile memory~\cite{zhang:pm-chipkill:micro:2018}. 
The module employs two protection codes: 
(i) a fixed-protection code reuses the chip failure protection bits to opportunistically correct bit errors at high performance, and 
(ii) a configurable-protection code uses long ECC codewords to correct at a configurable storage cost bit errors that are detected but uncorrected by the fixed-protection code. The configurable code 
%uses a BCH(2312,2048,22) code for each ECC codeword
uses a BCH(n,k,t) code for each ECC codeword of length $n=k+t(\left \lceil{log_2 (k)} \right \rceil+1)$ to correct $t$ bad bits when protecting $k$ bits of data~\cite{zhang:pm-chipkill:micro:2018}.

The FlexDIMM provides a cache-line failure probability due to DUE that is the product of two terms:
the probability that the fixed-protection code fails to correct a bit error (whose value equals to 0.018 as is taken from the original design~\cite{zhang:pm-chipkill:micro:2018}) and 
the probability that the configurable-protection code fails to correct multiple bit errors in the BCH codeword (which happens when there are at least $t$ bit errors):

\begin{equation*}
\pcdue= 0.018 \times \sum_{i=t+1}^{n}\binom{n}{i}{RBER}^i\cdot{(1-RBER)}^{n-i}
\end{equation*}

The cache-line failure probability due to DUE can be used as input to the \ramp analytical model to determine and tune the length of the BCH codeword in combination with rack-scale memory replication. 
% First, we compute the base cache-line failure probability of the configurable tier due to DUE as the probability that the configurable tier fails to correct multiple bit errors in the BCH codeword (which happens when there are at least $t$ bit errors):
% 
% \begin{equation*}
% \pcdue= \sum_{i=t+1}^{n}\binom{n}{i}{RBER}^i\cdot{RBER}^{n-i}
% \end{equation*}
The model can estimate the combined DUE rate resulting from using available replicas to correct DUEs. 
Although not shown, the cache-line failure probability due to NDE can be also calculated and used as input to the model, following the analysis of Kim and Lee~\cite{kim:undetected-error-bch:ieee-tc:1996}.

The FlexDIMM leverages existing hardware error reporting mechanisms, such as Intel Machine Check Architecture (MCA)~\cite{intel:mce:book:2024}, to raise a machine check exception (MCE) in response to uncorrectable memory errors.
For error reporting mechanisms that do not provide a mechanism for detecting write failures, like Intel MCA, the module may issue an additional read after a write to check the success of the write.
While the module, in theory, offer only error detection without correction, this would leave error correction entirely to the upper memory-replication tier. 
However, the ability to correct errors opportunistically is crucial to avoid a flood of interrupts, which could severely slowdown the system~\cite{meza:dramfailures:dsn:2015}, as corroborated in the evaluation.

The default response to an MCE in the kernel is a kernel panic. However, recent Linux kernels allow returning an error to the caller instead of crashing in response to memory-error induced MCEs~\cite{xu:nova-fortis:sosp:2017}.
After the exception, MCA registers hold information that allows the OS to identify the memory address responsible for the exception, allowing the OS to report the error to the memory-replication tier for further handling.
With this approach, after error reporting, a memory node remains operational and continues to serve memory accesses to non-failed memory regions.
Thus, in contrast to previous work where uncorrectable memory errors may crash a memory node~\cite{shan:legoos:osdi:2018} making all data stored on that node unavailable, our fine-grain failure model enables the memory node to remain operational and serve future requests, improving availability.

Propagating MCEs from the FlexDIMM, which provides the memory-protection tier, to the Resilience Manager, which provides the memory-replication tier, is challenging due to the one-sided semantics of RDMA. 
One approach would have the remote OS handle exceptions by disregarding the error, completing pending DMA requests that triggered the Memory Corrective Error (MCE), and making a callback remote procedure call (RPC) to the Resilience Manager to report the error. 
However, since one-sided operations issued by the Resilience Manager directly access memory and do not involve the remote memory node processor, this process may introduce a race condition between when the Resilience Manager receives the callback RPC and the completion of the one-sided RDMA operation. 
To address this, the FlexDIMM requires extending the RDMA NIC hardware, which handles the RDMA request, to support propagating memory errors to the Resilience Manager by piggybacking on the existing error reporting mechanism of RDMA. 
The OS kernel informs the NIC of the MCE, and the NIC reports the memory error back to the Resilience Manager as part of the response to the RDMA operation, indicating an ECC failure that requires further handling. 

