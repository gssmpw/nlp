\section{Related Work}
\mypar{Tolerating Memory Errors}
Several approaches have been proposed to protect data from memory errors in NVM. One method optimizes chipkill-correct to address bit errors in high-density NVM with high random raw bit error rates (RBER)____. Pangolin____ and NOVA-Fortis____ use checksums and parity to protect application and file system data, respectively, from media errors in NVM. TVARAK____ offloads the management of checksums and parity to a hardware controller co-located with the last-level cache, reducing software overhead while protecting against memory errors. In contrast to our work, all of these solutions focus on error correction within a single memory node and do not ensure high availability in cases where a memory node becomes completely unavailable.

\mypar{Fault-tolerant Disaggregated Memory}
With thousands of interconnected compute and memory devices, failures become commonplace, thus making fault tolerance a critical property of any practical approach for disaggregated memory. Failures, if not addressed properly, may result in data loss and force tasks to stop and restart. Although most prior frameworks for disaggregated memory lack fault tolerance, recent proposals can tolerate memory node failures through replication or erasure coding____. Our work complements these approaches by providing an efficient cross-layer resilience method for tolerating memory errors in high-density NVM.

\mypar{Cross-layer Resilience}
DIRECT____ is an application-level cross-layer resilience technique that leverages replication in distributed storage systems, such as key-value stores, to efficiently handle uncorrectable errors in flash storage. 
In contrast, our work operates at the memory interface and utilizes both distributed erasure coding and replication to provide enhanced redundancy for improved memory error resilience.