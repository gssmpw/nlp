We used Python to execute implementations on the CPU and GPU. The study leveraged NVIDIA's GeForce RTX 3060 GPU and Intel's Core i9 12900H CPU for efficient execution. The LIFA model begins with astrocytes receiving external stimuli, such as optogenetic Ca\(^{2+}\) uncaging and electrical stimulation of synaptic afferents, which are crucial for initiating astrocytic responses. Astrocytes integrate these stimuli, leading to Ca\(^{2+}\) signals essential for gliotransmitter release. This integration phase is influenced by the number, duration, and strength of stimuli. 

\begin{figure}[htbp!]
    \centering
    \includegraphics[width=9cm]{dac2022/images/fig3.pdf}
    \caption{Schematic representation of the LIFA model.}
    \label{fig:astrocyte1}
\end{figure}

\autoref{fig:astrocyte1} illustrates modeling, clustering, error tolerance analysis, fault-tolerant hardware implementation, and fault simulation. Astrocytic mechanisms are incorporated into the LIFA model to improve network resilience and error recovery. The baseline design includes training with data, clustering for pattern recognition, error tolerance analysis, and fault injection testing (ARES).

\subsection{Memory Measurement and Management} 

This section describes the methodology to assess and optimize memory dynamics within SNNs, based on astrocytic activity's influence on synaptic plasticity and Hopfield network principles. Astrocytic mechanisms introduced by the LIFA model alter synaptic behavior fundamentally, influencing synaptic plasticity, memory formation, and retrieval. Astrocytic modulation evaluates the memory capacity and efficiency of SNNs, inspired by Hopfield networks known for robust pattern recognition capabilities. Our methodology optimizes the number of synaptic connections, using fewer connections to maintain or enhance memory functionality. We analyze memory design capacitance within the SNN, calculating updates due to astrocyte-mediated synaptic modulation. This analysis includes SRAM, DRAM, and memristor-based synaptic arrays, determining the capacitance for storing and updating a bit or synaptic weight. The LIFA model's integration into SNNs represents an innovative approach to memory measurement and management, leveraging astrocytic functions for enhanced efficiency and capacity in neuromorphic computing.

\subsection{Innovative Routing and Fault Tolerance}

\begin{figure}[h!]
	\centering
	\centerline{\includegraphics[width=0.70\columnwidth]{dac2022/images/fig4.pdf}}
	\caption{Astrocyte Integrated Clustering.}
	\label{fig:astrocyte2}
\end{figure}

\autoref{fig:astrocyte2} shows how clustered neural networks integrate astrocyte-like structures. Astrocytic modulation pathways or the flow of information is evident in each cluster, representing neurons or nodes in a neural network. This figure illustrates astrocyte integration in neural networks, enhancing computational abilities or resilience. Clusters with similar functions are interconnected, mirroring biological neural networks. Algorithm~\ref{alg:astrocyte} outlines the integration of \astro{s} into an inference model \ineq{G_M}, structured into clusters 0 to 6. For each cluster, the algorithm applies the \texttt{ARES} framework to introduce \ineq{N_r} random errors, evaluating the model's accuracy after each error. If the accuracy \ineq{a_{min}} falls below the threshold \ineq{a_{th}}, an \astro{} is added to the cluster. The process iterates across all clusters, balancing neuron distribution across multiple \astro{s}. Parameters \ineq{N_r} and \ineq{a_{th}} are user-defined, with typical settings of 10,000 and baseline model accuracy \ineq{a_{o}}. This ensures model robustness against accuracy degradation due to errors.

\begin{footnotesize}
\begin{align*}
G_M(C,E) =&~\text{Inference model with } C \text{ clusters and } E \text{ edges}\\
G_A(C_A,E) =&~\text{Astrocyte-enabled model with } C_A \text{ clusters and } E \text{ edges}\\
L =&~\text{Layers of a core. } L = \{L_x,L_y\} \text{ and } L = \{L_x,L_y,L_z\}
\end{align*}
\end{footnotesize}
\normalsize

We set an accuracy constraint, \ineq{a_{th} = a_0}, indicating a hardware failure induces a model error if the accuracy drops from the baseline. This guides the co-design approach from Algorithm~\ref{alg:astrocyte}. For each astrocyte-enabled cluster \ineq{C_j \in C_A} of \ineq{G_A}, we design a crossbar/\mubrain{} core tailored to the astrocyte-augmented layers. We fix the number of astrocytes per core, leveraging their frequency reconstruction property. Aiming for an average spike frequency of 2.17 Hz and a maximum reconstruction error of 10\%, our configuration requires 4452 neurons per astrocyte. Post-implementation on the FTN cores, unused astrocytes are disabled to optimize fault tolerance rate.

\begin{algorithm}[h]
    \scriptsize{
        \KwIn{\ineq{G_{M}= (\textbf{C},\textbf{E})}}
        \KwOut{\ineq{G_{A}= (\textbf{C}_A,\textbf{E})}}
        \For(\tcc*[f]{For each cluster in $C$}){$C_k\in C$}{
            Arrange $C_k$ into layers based on clustering shown in Fig.~\ref{fig:astrocyte2} \tcc*[r]{E.g., $C_k = \{C_k^0, C_k^1, \ldots\}$ for each cluster.}
            \For(\tcc*[f]{For each layer in $C_k$}){$C_k^i\in C_k$}{
                \While(\tcc*[f]{Run until all neurons of the layer are protected against errors}){(true)}{
                    Insert $N_r$ random errors using \texttt{ARES} and evaluate the minimum accuracy $a_{min}$\;
                    \uIf(\tcc*[f]{Min accuracy is less than threshold.}){$a_\text{min}<a_{th}$}{
                        $C_k^i = C_k^i\cup$ \texttt{A}\tcc*[r]{Add an \astro{} to the layer.}
                    }
                    \Else{
                        exit\;
                    }
                }
            }
            \tcc{Astrocytes are mapped based on the clustering to optimize fault tolerance.}
        }
    }
    \caption{Algorithm for integrating astrocytes into a clustered SNN model.}
    \label{alg:astrocyte}
\end{algorithm}

\subsection{Alternative Reduce Regime for Energy Consumption}

This section describes optimizing energy consumption in neuromorphic systems using the LIFA model, inspired by astrocytes' energy-efficient neurotransmitter release and calcium signaling processes. Astrocytes in the brain conserve energy while maintaining optimal neural activity. LIFA aims to replicate this efficiency in SNNs, reducing network energy consumption through single neuron component switching. This approach controls neuron activity precisely, enhancing overall SNN efficiency and contributing to energy savings. Incorporating LIFA into SNNs underscores our commitment to efficient and sustainable computing paradigms. Analyses and tests determine the effectiveness of the alternative reduce regime, quantifying energy savings using metrics like energy consumption per task and spike event. We compare traditional computational models with LIFA to demonstrate its ability to reduce SNN energy consumption through improved efficiency.






