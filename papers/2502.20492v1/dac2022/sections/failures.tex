We discuss logic and memory failures and formulate the fault model.
%in a \nm{} device and formulate the fault model that is assumed in this work.

\subsection{Logic Failures in Neuromorphic Devices}
%A neuromorphic system is often designed as a many-core architecture, where each core consists of silicon circuitry mimicking the behavior of biological neurons and synapses~\cite{loihi}.
%Technology scaling has lead to many undesirable effects, such as silicon aging and soft-errors~\cite{austin2008reliable}. These effects challenge the integrity of those silicon circuits, resulting in incorrect and often unpredictable results.

In a \nm{} device, neurons are implemented using silicon \ckts{}.
%We take the example of aging, which is manifested as the drift of circuit parameters from their nominal values upon prolonged use at elevated voltage and temperature.
%For scaled technology nodes, this aging happens even under nominal conditions and from the very start of using a silicon circuit, leading to breakdown.
We take Bias Temperature Instability (BTI) induced silicon aging as an example. BTI is manifested as an increase of the threshold voltage of a transistor~\cite{kraak2019parametric,ncrtm,song2020improving}.
If neuron \ckts{} are used continuously, BTI aging cannot
%Strongly depending on the workload, BTI is highly variable and it is partially reversible under nominal conditions on the removal of stress voltage.
%Figure~\ref{fig:bti_degradation} shows the stress and recovery of the threshold voltage of a transistor.
%However, if a circuit is used continuously, its parameter drifts cannot 
be reversed in these \ckts{}, resulting in 
%a permanent degradation of performance and 
hardware faults.
Aggressive device scaling increases power density and temperature, which accelerates BTI aging.
%, challenging the reliable operation of neuromorphic devices.

The mean-time-to-failure (MTTF) of a circuit due to BTI is
\begin{equation}
    \label{eq:bti_mttf}
    \footnotesize \text{MTTF} = \frac{A}{V^\gamma}e^{\frac{E_a}{KT}},
\end{equation}
where \ineq{A} and \ineq{\gamma} are material-related constants, \ineq{E_a} is the activation energy, \ineq{K} is the Boltzmann constant, \ineq{T} is the temperature, and \ineq{V} is the voltage applied to the circuit.
The {failure rate} is given by~\cite{huard2015bti}
\begin{equation}
    \label{eq:bti_failure}
    \footnotesize \lambda = \frac{1}{\text{MTTF}}
\end{equation}

\subsection{Memory Failures in Neuromorphic Devices}
Non-Volatile Memory (NVM) that are used in \nm{} devices
%(see for instance, Fig.~\ref{fig:crossbar}). 
%Unfortunately, NVMs 
suffer from limited endurance and read disturb~\cite{hamdioui2017test,liu2021fault,kannan2014detection,ren2020exploring,song2020case}.
%, which challenge the correctness of operations mapped to a neuromorphic device~\cite{swami2017reliable}.

\textbf{Limited Endurance:} This is a type of NVM failure, where an NVM cell breakdown (stuck-at-SET or stuck-at-RESET) upon repeated accesses. Recent system-level studies show that the MTTF due to limited endurance can be computed as~\cite{espine}
\begin{equation}
    \label{eq:mttf_endurance}
    \footnotesize \text{MTTF} = \text{exp}\left(\frac{\gamma}{T_{SH}}\right),
\end{equation}
where \ineq{\gamma} is a fitting parameter and \ineq{T_{SH}} is the self-heating temperature of the cell and is given
\begin{equation}
        \label{eq:tsh}
        \footnotesize T_{SH}=\frac{I_{cell}^2\cdot R_{cell}\cdot l^2}{kV}-\left[1-\text{exp}\left(-\frac{kt}{l^2C}\right)\right]+T_{amb},
\end{equation}
where \ineq{I_{cell}} is the current through the NVM cell, \ineq{R_{cell}} is the resistance of the cell, \ineq{l} is the thickness of the core NVM material (i.e., without the packaging), \ineq{V} is its volume, \ineq{C} is its heat capacity, \ineq{k} is the fraction of the core volume responsible for a specific NVM state, \ineq{t} is the time, and \ineq{T_{amb}} is the ambient temperature.% (see Sec.~\ref{sec:evaluation} for these parameters).

\textbf{Read Disturb:} This is a type of NVM failure, where the resistance state of an NVM cell drifts upon repeated accesses. MTTF due to read disturb is formulated as~\cite{shim2020impact,pauldt,song2021improving}
\begin{equation}
    \label{eq:mttf_disturbance}
    \footnotesize \text{MTTF} = 10^{-14.7\cdot V + 6.7},
\end{equation}
where \ineq{V} is the applied voltage.
The fault rate due to limited endurance and read disturb are also computed using Eq.~\ref{eq:bti_failure}.

% \begin{figure}[h!]%
%     \centering
%     \vspace{-20pt}
%     \subfloat[Threshold voltage degradation due to BTI breakdown mechanism.\label{fig:bti_degradation}]{{\includegraphics[width=5.0cm]{images/nbti_demo.pdf} }}%
%     \hfill
%     \subfloat[NVMs used as synaptic storage in a neuromorphic device.\label{fig:crossbar}]{{\includegraphics[width=3.2cm]{images/3d_view_new.pdf} }}%
%     %\subfloat[Scheduling sub-networks to \mubrain{} pipelines.\label{fig:lenet_mubrain_mapping}]{{\includegraphics[width=4.2cm]{images/lenet_platform_mapping.png} }}%
%     \vspace{-5pt}
%     \caption{(a) BTI degradation of the threshold voltage of a transistor and (b) 3D view of a crossbar.}%
%     \label{fig:mubrain_subnet}%
%     \vspace{-10pt}
% \end{figure}

\subsection{Fault Model}
Irrespective of the exact mechanism of logic and memory failures in a neuromorphic device, such failure mechanisms can be combined to generate an overall failure rate using the Sum-of-Failure-Rates (SOFR) model, which is used extensively in the industry~\cite{amari1997closed}.
SOFR assumes an exponential lifetime distribution for each failure mechanism.
Therefore, the overall failure rate is computed as
\begin{equation}
    \label{eq:sofr}
    \footnotesize \lambda_\text{overall} = \frac{1}{\text{MTTF}_\text{overall}} = \lambda_\text{Aging} + \lambda_\text{Endurance} +  \lambda_\text{Disturb}.
\end{equation}
Equation~\ref{eq:sofr} can be extended to consider other failure mechanisms~\cite{gebregirogis2020approximate,yadav2020analyzing,wu2020fault}. For neuromorphic devices, the MTTF due to aging is reported to be around 2 years~\cite{balaji2019framework}, while MTTF due to endurance can range
from \ineq{10^5} cycles (for Flash) to \ineq{10^{9-10}} (for OxRRAM), with PCM somewhere in between
(\ineq{\approx 10^{6-7}})~\cite{espine}. Read disturb errors can occur more frequently, e.g., one error after inferring about 1000 images~\cite{song2021improving}.
A commonly used model for random, mutually-independent failures happening in a time interval \ineq{T} is the Poisson process. The probability \ineq{P_n} of \ineq{n} failures in time interval \ineq{T} is
\begin{equation}
    \label{eq:probability}
    \footnotesize P_n = \frac{(\lambda_{overall}\cdot T)^n}{n!}\text{exp}\{-\lambda_{overall}\cdot T\}
\end{equation}
%For the selected time interval \ineq{T} and the failure rates, \ineq{\lambda_{overall}\cdot T < 1}. 
%Therefore, the probability of more than one failure in the selected time interval \ineq{T} is negligible. %We consider both single and double faults in our evaluations. %The proposed design methodology can be extended with minimal efforts to consider multiple faults.