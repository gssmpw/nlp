To understand the astrocytic dynamics in neuromorphic computing, we integrate the LIFA model into our system. The LIFA model is motivated by the principle that the astrocytic Calcium Ions (Ca\(^{2+}\)) elevate beyond a threshold; triggering the release of neuroactive molecules such as glutamate and ATP. The released molecules promote postsynaptic neural activity and known as gliotransmitters. Biophysical arguments support that subthreshold Ca\(^{2+}\) dynamics set the rate of gliotransmitter-mediated postsynaptic depolarizations. The relevant time constants are $\tau_N$ for neuronal activity ($v_N$), $\tau_G$ for astrocytic Ca\(^{2+}\) activity ($v_G$), and $\tau_p$ for gliotransmitter dynamics. The ODEs (Ordinary Differential Equations) are:
\begin{align}
    \tau_N \frac{dv_N}{dt} &= -v_N + I_N(t)\\
    \tau_G \frac{dv_G}{dt} &= -v_G + I_G(t)\\
    \tau_p \frac{dg}{dt} &= -g + G(1-g)r_G(t) 
\end{align}
where $I_N(t)$ and $I_G(t)$ are the synaptic inputs to neurons and glia, and $r_G(t)$ is the gliotransmitter release. Synaptic weights ($w$) are proportional to postsynaptic activation ($q$), i.e. $w=uq$, and astrocytes contribute to postsynaptic activation by $Qg$, so $w=u(q_0+Qg)$, where $q_0$ is the baseline postsynaptic activation.

Incorporating these aspects of the LIFA model into our framework aims to achieve a more biologically accurate and efficient simulation of astrocyte-neuronal interactions. This enhances the realism of our model and provides a robust foundation for advanced computational strategies.

\begin{figure}[htbp!]
    \centering
    \vspace{-10pt}
    \subfloat[Original network.\label{fig:original_connection}]{{\includegraphics[width=0.5\columnwidth]{dac2022/images/fig1.pdf}}}%
    \vspace{10pt}

    \subfloat[LIFA-modulated network.\label{fig:astrocyte_modulation}]{{\includegraphics[width=0.5\columnwidth]{dac2022/images/fig6.pdf}}}%
    \vspace{10pt}

    \subfloat[Operation of LIFA.\label{fig:astrocyte}]{{\includegraphics[width=0.5\columnwidth]{dac2022/images/fig5.pdf}}}
    
    \caption{Inserting LIFA in a neural network.}%
    \label{fig:astrocyte_neural_network}%
\end{figure}

\begin{figure}[h!]
	\centering
	\centerline{\includegraphics[width=0.99\columnwidth]{dac2022/images/fig2.pdf}}
	\caption{Self-repair mechanism of an \astro{}.}
	\label{fig:error_recovery}
\end{figure}

Figure \autoref{fig:original_connection} shows a neural network before astrocyte modulation. Figure \autoref{fig:astrocyte_modulation} shows the network after integrating astrocyte modulation, demonstrating structural and functional changes due to astrocyte integration. Figure \autoref{fig:astrocyte} shows LIFA's operation between synaptic site and post neuron activities. \autoref{fig:error_recovery} illustrates the network's stages: normal operation, stress, recovery, and potential failure. The black traces show the network's threshold voltage \( V_{th} \) under normal conditions, while the red traces show the threshold voltage under stress. Stress refers to conditions that push the network beyond typical operational parameters. Recovery denotes the network's ability to return to normal operation after stress, and $v_{\text{spk}}$ refers to the spike voltage that triggers neuron firing, while $v_{\text{idle}}$ denotes the idle voltage. The x-axis represents time, and the y-axis, labeled \(\Delta V_{th}\), represents the change in threshold voltage, showing the network's stress response and recovery capability over time.
