The field of neuromorphic computing is undergoing a transformative phase, driven by the integration of biologically-inspired components. This paper introduces a groundbreaking approach that integrates the Leaky Integrate-and-Fire Astrocyte (LIFA) model into Spiking Neural Networks (SNNs). SNNs are inspired by brain dynamics known for their capabilities in energy-efficient procesing and biologically plausible learning algorithms \cite{huynh2022implementing, putra2022softsnn, perera2024wet, gao2025sg}. However, SNNs remain vulnerable to faults that can impair their efficiency. Astrocytes play a critical role in regulating neuronal activity and synaptic transmission; contributing to the resilience of the biological networks \cite{ben2022fault, yerima2023fault, li2025neurove, tsybina2024adding}. This integration of astrocytic mechanisms into SNNs offers dynamic adjustments for fault tolerance \cite{isik2022design, haghiri2020digital, johnson2016fpga}.

Despite performance improvements with SNNs, technology scaling exhibits challenges such as increased power densities along with faults in neuron and synapse circuits. These challenges impact the SNN model's performance capabilities and overall robustness \cite{isik2023astrocyte, kumar2023implementation, lorenzo2024spiking}. The LIFA model is rooted in dynamic interplay between neurons and astrocytes. As a result, the LIFA model enhances neural processing by bolstering computational strength and efficiency \cite{de2022multiple}. Additionally, LIFA model integration introduces novel computational capabilities for neuromorphic computing \cite{kozachkov2023neuron, linne2022neuron, pan2022neuron, isik2024advancing}.

In order to achieve energy efficiency, memory measurement, efficient routing, and fault tolerance, we designed a methodology around four key pillars. LIFA enables accurate emulation of brain functions and marks a breakthrough in neuromorphic computing by shifting beyond neuron-centric approaches.

We propose four components of a fault-tolerant neuromorphic computing system:

\begin{itemize}
\item \textbf{Alternative Reduce Regime:} Optimizes energy consumption through single neuron component switching, demonstrating a commitment to sustainable and efficient computing.
\item \textbf{Memory Measurement and Management:} Explores synaptic dynamics inspired by Hopfield networks to optimize memory storage and retrieval, achieving superior memory efficiency with fewer connections.
\item \textbf{Innovative Routing and Fault Tolerance:} Implements robust routing mechanisms with fault-tolerant strategies, ensuring network integrity and resilience with minimal component usage.
\item \textbf{LIFA Model Implementation:} Involves adapting astrocytic and neuronal interactions from a theoretical model to a fully functional computational model, offering computational advantages and insights.
\end{itemize}

We set new benchmarks in energy efficiency, memory management, routing strategies, and fault tolerance. Our methodology, evaluated using various deep learning models, demonstrates efficacy in area and power efficiency while providing robust fault tolerance.

