In the rapidly advancing field of neuromorphic computing, the integration of biologically-inspired models like the Leaky Integrate-and-Fire Astrocyte (LIFA) into spiking neural networks (SNNs) opens new horizons for enhancing system robustness and performance. This paper introduces a groundbreaking approach that incorporates the LIFA model into SNNs, addressing critical areas such as energy efficiency, memory management, routing mechanisms, and fault tolerance.  Our methodology involves a core architecture comprising neurons, synapses, and astrocyte circuits, with each astrocyte encapsulating multiple neurons for self-repair in the event of neuron failure. This unique integration of astrocytes into a clustered model markedly improves the system's fault tolerance and operational efficiency, especially under adverse conditions. We have developed a specialized routing methodology to map the LIFA model effectively onto a fault-tolerant, many-core design, optimizing network functionality and efficiency. Through rigorous evaluation, our design methodology has demonstrated area and power efficiency while achieving superior fault tolerance compared to existing approaches. Our model features a remarkable fault tolerance rate of 81.10\% and a resilience improvement rate of 18.90\%, significantly surpassing other state-of-the-art implementations. The results validate the effectiveness of our approach in memory management, further underscoring its potential as a robust solution for advanced neuromorphic computing applications. The integration of astrocytes in neuromorphic systems marks a significant advancement, setting the stage for the development of more resilient and adaptable neuromorphic systems.
