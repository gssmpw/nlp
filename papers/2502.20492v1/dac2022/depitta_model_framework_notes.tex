%% Uncomment this line and comment the other to get rid of comments and
%% build the full manuscript with figures at the end.
\documentclass[a4paper,11pt,final]{article}
%\documentclass[a4paper,11pt,draft]{article}
\usepackage[hang,flushmargin]{footmisc} % left indendt of footnote
\usepackage[titletoc,toc,title]{appendix}
\usepackage{fullpage} % changes the margin
\usepackage{xcolor}
\usepackage{titling}  % Provide double title

%% Footnote symbols starting from \dagger (Needed for Eshel's authorship)
\makeatletter
	\renewcommand*{\@fnsymbol}[1]{\ensuremath{\ifcase#1\or\dagger\else\@ctrerr\fi}}
\makeatother

% Math pacakges and fonts
\usepackage{amsmath}
\usepackage{amssymb,wasysym,bm}
\usepackage{dsfont}
\usepackage{upgreek}
\usepackage{mathtools}
% Fancy lists
\usepackage{enumitem}

% Package for physical units
\usepackage{siunitx}
\DeclareSIUnit\Molar{\textsc{m}}

% Chemical reactions
\usepackage[version=4,arrows=pgf-filled]{mhchem}
%\mhchemoptions{arrows=pgf-filled} %% Arrow adjustment

% Figures
\usepackage[outdir=./Figures/]{epstopdf}
\usepackage{graphicx}
\graphicspath{{./},{./Figures/}}
% Captions will be centered (captions are provided in separate section due to their length)
\usepackage[aboveskip=1pt,
            labelfont=bf,
            labelsep=period,
            justification=centering,
            singlelinecheck=off]{caption}
% Package for subfigures
\usepackage[position=top,labelformat=simple]{subcaption}
\renewcommand*{\thesubfigure}{\Alph{subfigure}} % Upper case letters for subfigures
% Subfigure captions
\usepackage[export]{adjustbox}

%\usepackage{cite}
%\usepackage[sort&compress,numbers]{natbib}
\usepackage{natbib}
\allowdisplaybreaks

% Coloring
\usepackage{color}
\usepackage{xcolor}

% Listings
\usepackage{listings}
\usepackage{hyperref}
\hypersetup{colorlinks,
	        linkcolor={red!50!black},
    	    citecolor={blue!50!black},
    		urlcolor={blue!80!black}}

% Physical units
\usepackage{siunitx}
\DeclareSIUnit\Molar{\textsc{m}}

%----------------------------------------------------------------------------------------------------------------
% Tables
%----------------------------------------------------------------------------------------------------------------
%\usepackage{tabularx}
%\usepackage{longtable}
\usepackage{ltablex}
\newcommand{\rowgroup}[1]{\hspace{1em}#1}
%----------------------------------------------------------------------------------------------------------------
% \hline renew
%----------------------------------------------------------------------------------------------------------------
\let\oldhline\hline
\renewcommand{\hline}{\oldhline\noalign{\vskip .5ex}}
\newcommand\T{\rule{0pt}{2.4ex}}       % Top strut
\newcommand\B{\rule[-1.2ex]{0pt}{0pt}} % Bottom strut
%----------------------------------------------------------------------------------------------------------------
% Table caption on top
\usepackage{floatrow}
\floatsetup[table]{capposition=top}

%%% Add line numbers for reviewers
\usepackage{lineno}
%\linenumbers

%%% TODOs / Comments
\usepackage[textsize=tiny,obeyDraft]{todonotes}
%% To make comments: add \todo{...} at the point under scrutiny and recompile with 'draft' option
\newcommand{\HB}[1]{{\color{blue} #1}}

%----------------------------------------------------------------------------------------------------------------
% User-defined math commands
%----------------------------------------------------------------------------------------------------------------
\newcommand{\der}[2]{\frac{\mathrm{d}#1}{\mathrm{d}#2}}
\newcommand{\rpar}[1]{\left(#1\right)}
\newcommand{\Hill}[3]{\ensuremath{\mathcal{H}_{#1}\left(#2,#3\right)}}
\newcommand{\odm}[1]{\ensuremath{O\left(#1\right)}}
\newcommand{\dd}[1]{\ensuremath{\mathrm{d}#1}}
\renewcommand{\vec}[1]{\ensuremath{\mathbf{#1}}}
%%------------------------------------
%% Mathematical symbols
%%------------------------------------
\def\gs{\ensuremath{\mathbf{\upgamma}}}
\def\xn{\ensuremath{\mathbf{x}}}

%----------------------------------------------------------------------------------------------------------------
% User-defined appreviations and symbols
%----------------------------------------------------------------------------------------------------------------
\def\ip3{\ce{IP3}}
\def\ca{\ce{Ca^2+}}

%----------------------------------------------------------------------------------------------------------------
% CROSS-REFERENCES
%----------------------------------------------------------------------------------------------------------------
\newcommand*{\tabref}[1]{\tablename~\ref{#1}}
\newcommand*{\figref}[1]{\figurename~\ref{#1}}
\newcommand*{\secref}[1]{Section~\ref{#1}}
\newcommand*{\appref}[1]{Appendix~\ref{#1}}
\renewcommand*{\eqref}[1]{eq.~\ref{#1}}
\newcommand*{\reactref}[1]{reaction~\ref{#1}}
\newcommand*{\eref}[1]{(\ref{#1})}
\newcommand*{\see}[1]{\textbf{ref.} \textsc{#1}}

%----------------------------------------------------------------------------------------------------------------
%% Blank footnote for abbreviations
%----------------------------------------------------------------------------------------------------------------
\newcommand\blfootnote[1]{%
  \begingroup
  \renewcommand\thefootnote{}\footnote{#1}%
  \addtocounter{footnote}{-1}%
  \endgroup
}

%----------------------------------------------------------------------------------------------------------------
%% Title and authors
%----------------------------------------------------------------------------------------------------------------
\title{Spiking Neuron-Glial Networks (SNGNs): General formalism}
\author{
        Maurizio De Pitt\`a\\
        Krembil Research Institute, University Health Network\\
        maurizio.depitta@uhn.ca
        }
\date{\today}

%----------------------------------------------------------------------------------------------------------------
%% Bibliography settings
%----------------------------------------------------------------------------------------------------------------
%\bibliographystyle{apalike}

\begin{document}
%%
%% Set listing style
\lstset{language=Python, basicstyle=\small\ttfamily,
              showstringspaces=false,
              columns=fixed}

%% Title
\maketitle

% \tableofcontents

\section{General framework for synapse-astrocyte interactions}
In the most general scenario, a synaptic connection from neuron $j$ to neuron $i$ under the influence of astrocyte activity $g$ is a function $s_{ij}(x_i,x_j,g)$. Note the order of the subscripts in $s_{ij}$: the outer subscript is the presynaptic neuron, whereas the inner subscript is the postsynaptic (i.e., target) neuron. Let us assume that the function $s_{ij}$ is continuous and satisfies all the requirements for Taylor expansion. Then, Taylor-expanding $s_{ij}$ for $x_i=x_j=a=0$ up to the second order results in:  
\begin{align}
s_{ij} &\approx c_2^\mathrm{corr}x_i x_j + c_2^\mathrm{pre}x_i^2 + c_2^\mathrm{post}x_j^2\nonumber\\
             &\phantom{\approx}+c_2^\mathrm{g-pre}x_i\,g + c_2^\mathrm{g-post}x_j\,g + c_2^\mathrm{glia}g^2 \nonumber\\
             &\phantom{\approx}+c_1^\mathrm{pre}x_i + c_1^\mathrm{post}x_j + c_1^\mathrm{glia}\,g + c_0(s_{ij}) + \mathrm{O}(x_i,x_j,g) \label{eq:synpase-glia-interaction}
\end{align} 
\noindent
In the above, the coefficients' subscripts reflect the order of the derivatives from the Taylor expansion. Depending on the assumptions of the model (reflected by the values of such coefficients, you can retrieve multiple models. For example, the current model for fault tolerance is (e.g., equation~5 in Murat's paper):
\begin{align}\label{eq:syn-glia-sic}
s_{ij} &= c_1^\mathrm{glia}\,g + c_0(s_{ij}) 
\end{align}
If we focus on the biophysical components of synaptic efficacy, then we can separate between presynaptic release $u$, and postsynaptic receptor activation $q$, such that $s_{ij}=u_{ij}q_{ij}$, and thus
\begin{align}
\dot{s}_{ij} &= \dot{u}_{ij}q_{ij} + u_{ij}\dot{q}_{ij}
\end{align}
In our framework, $\dot{u}=0$, since $u=u_0$ so that $\dot{s}_{ij}=u_0\dot{q}_{ij}$. Moreover, the contribution of pre- and post-synaptic activities to $q$ is constant (it would change if we were considering any Hebbian learning), so that the only contribution to the time-dependent component of $q$ is by the astrocytic activity $g$, i.e.   
\begin{align}
\dot{s}_{ij} &= u_0\cdot c_{1\,ij}^\mathrm{glia}\,g
\end{align}
In this framework, it is helpful to consider the biophysical interpretation of the effect of astrocytes on synaptic efficacy in terms of gliotransmission. In this way, gliotransmission activates a fraction $\gamma$ of extrasynaptically located postsynaptic receptors. A simple first-order saturating kinetics for the binding of those receptors by gliotransmitter molecules is described by (see my previous work):  \begin{align}
\tau_p \dot{\gamma} &= -\gamma + G^\mathrm{post}(1-\gamma)g(t)\tau_p
\end{align}
where we multiply the r.h.s. by $\tau_p$ to express that $g(t)$ must be in units of \si{1/second}, that is, it is a rate of some sort. It thus follows that
\begin{align}
\tau_p\dot{s}_{ij} &= -u_0c_{1\,ij}\gamma + u_0c_{1\,ij}G^\mathrm{post}(1-\gamma)g(t)\tau_p
\end{align}
In conclusion, we can write
\begin{align}\label{eq:synapse-sic}
s_{ij} &\approx u_0q_0 + u_0c_{1\,ij}\dot{s}_{ij} = J_0 + \frac{G_{ij}}{\tau_p} \dot{\gamma}
\end{align}
where:
\begin{itemize}
    \item $J$ denotes the synaptic weight;
    \item $J_0$ denotes the direct pathway component of the synaptic weight;
    \item $G$ denotes the indirect pathway component of the synaptic weight that is scaled by $g(t)$ through, generally, a non-linear function of the latter ($\gamma$ in our case). 
\end{itemize}

\section{General SNGNs framework}
\subsection*{Neuronal nodes (a.k.a. neurons)}
\begin{itemize}
    \item $x$ is the activity readout of a neuron (e.g., membrane potential);
    \item $b$ is the external input to a neuron; 
    \item $\phi_N$ is the activity-to-firing rate transfer characteristics of a neuron;
    \item $\tau_N$ is the neuron (leak) time scale;
    \item $J$ are the synaptic connection weights.  
\end{itemize}
The equation for the activity of the general neuron will be:
\begin{align}
\tau_N \dot{x_i} &= -x_i + \sum_j  J_{ij} \phi_N(x_j) \tau_N + b_i
\end{align}
where $J_{ij}$ coincides with \eqref{eq:synapse-sic}.

\subsection{Astrocytic nodes (a.k.a. astrocyte domains)}
\begin{itemize}
    \item $y$ is the activity readout of an astrocyte domain (e.g., intracellular \ca);
    \item $d$ is the external input to an astrocyte domain; 
    \item $\phi_G$ is the activity-to-firing rate transfer characteristics of the domain;
    \item $\tau_G$ is the astrocyte (leak) time scale;
    \item $W$ are the synapse$\rightarrow$astrocyte connection weights;
    \item $T$ are the astrocyte$\rightarrow$astrocyte connection weights.
\end{itemize}
In the most general scenario, $W$ and $T$ are tensors. Accordingly, the general equation for the activity of the astrocytic domain $k$ is:
\begin{align}
\tau_G \dot{y_k} &= -y_k + \sum_{jk}  W_{ijk} \phi(x_j) \tau_G + \sum_l T_{lk}\phi_G(y_k)  + d_k
\end{align}
In our framework, for the moment, we do not consider interconnections between astrocyte nodes nor external inputs to the astrocytes, so the above equation simplifies dramatically:
\begin{align}
\tau_G \dot{y_k} &= -y_k + \sum_{jk}  W_{ijk} \phi(x_j) \tau_G
\end{align}
Note that $W_{ijk}$ are scalars when assuming feed forward gliotransmission as in our current framework.

\section{Routing considerations}
For the purpose of analysis, we can express our network model in terms of generalized vector equations (note: I am neglecting transpose superscripts here for brevity, but the correct equations should take that into account):
\begin{align}
\tau_N \dot{\mathbf{x}} &= -\mathbf{x} + \mathbf{J}(\bm{\upgamma})\phi_N(\mathbf{x}) \tau_N + \mathbf{b}\\
\tau_G \dot{\mathbf{y}} &= -\mathbf{y} + \mathbf{W}\phi_N(\mathbf{x}) \tau_G + \mathbf{T}\phi_G(\mathbf{y}) \tau_G + \mathbf{d}\\
\tau_p \dot{\bm{\upgamma}} &= -\bm{\upgamma} + \mathbf{G}\bm{\upgamma}g(\mathbf{y})\tau_p
\end{align}
where we conveniently express the gliotransmission rate as a function of astrocyte activity $y$, being $g = \mathbf{R}_G \phi(\mathbf{y})$ where $\mathbf{A}_G$ denotes the kernel accounting for the gliotransmitter release probability. The routing problem can be addressed by considering specific configurations for the connectivity tensors $J,W,T,G$. In particular, each of these tensors can be written in the generic form:
\begin{equation}
    \bm{M} = \bm{A} \odot \bm{R} \odot \hat{\bm{M}}
\end{equation}
where $\bm A$ is the adjacency tensor (binary), $\bm R$ is the stochastic kernel accounting, for example, for the probability of synaptic or glutamate release, or the probability or astrocyte-to-astrocyte transfer, etc., and $\hat{\bm{M}}$ is a dense matrix with all connection entries ($\odot$ denotes the element-wise (Hadamard) product).

%%%%%%%%%%%%%%%%%%%%%%%%%%%%%%%%%%%%%%%%%%%%%%%%%%%%%%%%%%%%%%%%%%%%%%%%%%%%%
%% Bibliography
%%%%%%%%%%%%%%%%%%%%%%%%%%%%%%%%%%%%%%%%%%%%%%%%%%%%%%%%%%%%%%%%%%%%%%%%%%%%%
\newpage
% \bibliography{./depitta_ecns.bib}

\end{document}
