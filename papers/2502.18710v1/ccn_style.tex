\documentclass[10pt,letterpaper]{article}

\usepackage{ccn}
\usepackage{apacite}
\usepackage[bookmarks=false]{hyperref} 
\usepackage{natbib}
\usepackage{bm}
\usepackage{xcolor}
\usepackage{array}
\usepackage{graphicx}
\usepackage{multicol}
\usepackage{booktabs}
\usepackage{amsmath, amsfonts}
\usepackage{dblfloatfix}

\definecolor{mplblue}{rgb}{0.0, 0.447, 0.741}
\definecolor{mplgreen}{rgb}{0.0, 0.621, 0.451}
\definecolor{mplorange}{rgb}{0.850, 0.325, 0.098}





\title{Bridging Critical Gaps in Convergent Learning: How Representational Alignment Evolves Across Layers, Training, and Distribution Shifts}
 
\author{{\large \bf Chaitanya Kapoor, Sudhanshu Srivastava, Meenakshi Khosla} \\~\\
Department of Cognitive Science, UC San Diego, \texttt{\{chkapoor, sus021, mkhosla\}@ucsd.edu}}
%   A Department, 1234 Example Street\\
% A City, State 12345 A country
%   \AND {\large \bf Another Person (AnotherPerson@this.planet.edu)} \\
%   A Department, 1234 Example Street\\
% A City, State 12345 A country}


\begin{document}
\maketitle


\section{Abstract}
{
\bf
Understanding convergent learning---the extent to which artificial and biological neural networks develop similar representations---is crucial for neuroscience and AI, as it reveals shared learning principles and guides brain-like model design. While several studies have noted convergence in early and late layers of vision networks, key gaps remain. First, much existing work relies on a limited set of metrics, overlooking transformation invariances required for proper alignment. We compare three metrics that ignore specific irrelevant transformations: linear regression (ignoring affine transformations), Procrustes (ignoring rotations and reflections), and permutation/soft-matching (ignoring unit order). Notably, orthogonal transformations align representations nearly as effectively as more flexible linear ones, and although permutation scores are lower, they significantly exceed chance, indicating a robust representational basis. A second critical gap lies in understanding when alignment emerges during training. Contrary to expectations that convergence builds gradually with task-specific learning, our findings reveal that nearly all convergence occurs within the first epoch---long before networks achieve optimal performance. This suggests that shared input statistics, architectural biases, or early training dynamics drive convergence rather than the final task solution. Finally, prior studies have not systematically examined how changes in input statistics affect alignment. Our work shows that out-of-distribution (OOD) inputs consistently amplify differences in later layers, while early layers remain aligned for both in-distribution and OOD inputs, suggesting that this alignment is driven by generalizable features stable across distribution shifts. These findings fill critical gaps in our understanding of representational convergence, with implications for neuroscience and AI.
%Understanding convergent learning---the extent to which different artificial and biological neural networks learn similar representations---is crucial for both neuroscience and AI, as it can reveal shared principles of learning and guide the design of brain-like models. While several studies have noted convergence in early and late layers of vision networks, important gaps remain. First, most existing work has relied on a limited set of metrics, often overlooking the role of specific transformation invariances needed to properly align representations across layers. We address this by comparing three metrics, each designed to ignore specific types of irrelevant transformations: linear regression (ignoring affine transformations), Procrustes (ignoring rotations and reflections), and permutation or soft-matching (ignoring the order of units). Strikingly, orthogonal transformations align representations almost as effectively as more flexible linear transformations, and permutation scores, though lower than the other metrics, significantly exceed chance levels, indicating that the representational basis is meaningfully distinguished. A second, critical gap in the literature is the lack of understanding of when representational alignment emerges during training. Contrary to the expectation that convergence builds gradually with task-specific learning, our findings reveal that nearly all convergence occurs within the first epoch---long before networks reach optimal performance. This surprising result suggests that shared input statistics, common architectural biases, or the early dynamics of training drive convergence instead of the final task solution. Finally, prior studies have not systematically examined how changes in input statistics affect representational alignment. Our work shows that out-of-distribution (OOD) inputs consistently amplify differences in later layers, reflecting sensitivity to input statistics, while early layers maintain alignment for both in-distribution and OOD inputs, suggesting that this alignment is driven by generalizable features stable across distribution shifts. These findings fill critical gaps in our understanding of representational convergence with significant implications for both neuroscience and AI. 	
}
\begin{quote}
\small
\textbf{Keywords: Artificial Intelligence, Machine Learning, Representational Convergence} 
\end{quote}

\section{Introduction}
\section{Introduction}


\begin{figure}[t]
\centering
\includegraphics[width=0.6\columnwidth]{figures/evaluation_desiderata_V5.pdf}
\vspace{-0.5cm}
\caption{\systemName is a platform for conducting realistic evaluations of code LLMs, collecting human preferences of coding models with real users, real tasks, and in realistic environments, aimed at addressing the limitations of existing evaluations.
}
\label{fig:motivation}
\end{figure}

\begin{figure*}[t]
\centering
\includegraphics[width=\textwidth]{figures/system_design_v2.png}
\caption{We introduce \systemName, a VSCode extension to collect human preferences of code directly in a developer's IDE. \systemName enables developers to use code completions from various models. The system comprises a) the interface in the user's IDE which presents paired completions to users (left), b) a sampling strategy that picks model pairs to reduce latency (right, top), and c) a prompting scheme that allows diverse LLMs to perform code completions with high fidelity.
Users can select between the top completion (green box) using \texttt{tab} or the bottom completion (blue box) using \texttt{shift+tab}.}
\label{fig:overview}
\end{figure*}

As model capabilities improve, large language models (LLMs) are increasingly integrated into user environments and workflows.
For example, software developers code with AI in integrated developer environments (IDEs)~\citep{peng2023impact}, doctors rely on notes generated through ambient listening~\citep{oberst2024science}, and lawyers consider case evidence identified by electronic discovery systems~\citep{yang2024beyond}.
Increasing deployment of models in productivity tools demands evaluation that more closely reflects real-world circumstances~\citep{hutchinson2022evaluation, saxon2024benchmarks, kapoor2024ai}.
While newer benchmarks and live platforms incorporate human feedback to capture real-world usage, they almost exclusively focus on evaluating LLMs in chat conversations~\citep{zheng2023judging,dubois2023alpacafarm,chiang2024chatbot, kirk2024the}.
Model evaluation must move beyond chat-based interactions and into specialized user environments.



 

In this work, we focus on evaluating LLM-based coding assistants. 
Despite the popularity of these tools---millions of developers use Github Copilot~\citep{Copilot}---existing
evaluations of the coding capabilities of new models exhibit multiple limitations (Figure~\ref{fig:motivation}, bottom).
Traditional ML benchmarks evaluate LLM capabilities by measuring how well a model can complete static, interview-style coding tasks~\citep{chen2021evaluating,austin2021program,jain2024livecodebench, white2024livebench} and lack \emph{real users}. 
User studies recruit real users to evaluate the effectiveness of LLMs as coding assistants, but are often limited to simple programming tasks as opposed to \emph{real tasks}~\citep{vaithilingam2022expectation,ross2023programmer, mozannar2024realhumaneval}.
Recent efforts to collect human feedback such as Chatbot Arena~\citep{chiang2024chatbot} are still removed from a \emph{realistic environment}, resulting in users and data that deviate from typical software development processes.
We introduce \systemName to address these limitations (Figure~\ref{fig:motivation}, top), and we describe our three main contributions below.


\textbf{We deploy \systemName in-the-wild to collect human preferences on code.} 
\systemName is a Visual Studio Code extension, collecting preferences directly in a developer's IDE within their actual workflow (Figure~\ref{fig:overview}).
\systemName provides developers with code completions, akin to the type of support provided by Github Copilot~\citep{Copilot}. 
Over the past 3 months, \systemName has served over~\completions suggestions from 10 state-of-the-art LLMs, 
gathering \sampleCount~votes from \userCount~users.
To collect user preferences,
\systemName presents a novel interface that shows users paired code completions from two different LLMs, which are determined based on a sampling strategy that aims to 
mitigate latency while preserving coverage across model comparisons.
Additionally, we devise a prompting scheme that allows a diverse set of models to perform code completions with high fidelity.
See Section~\ref{sec:system} and Section~\ref{sec:deployment} for details about system design and deployment respectively.



\textbf{We construct a leaderboard of user preferences and find notable differences from existing static benchmarks and human preference leaderboards.}
In general, we observe that smaller models seem to overperform in static benchmarks compared to our leaderboard, while performance among larger models is mixed (Section~\ref{sec:leaderboard_calculation}).
We attribute these differences to the fact that \systemName is exposed to users and tasks that differ drastically from code evaluations in the past. 
Our data spans 103 programming languages and 24 natural languages as well as a variety of real-world applications and code structures, while static benchmarks tend to focus on a specific programming and natural language and task (e.g. coding competition problems).
Additionally, while all of \systemName interactions contain code contexts and the majority involve infilling tasks, a much smaller fraction of Chatbot Arena's coding tasks contain code context, with infilling tasks appearing even more rarely. 
We analyze our data in depth in Section~\ref{subsec:comparison}.



\textbf{We derive new insights into user preferences of code by analyzing \systemName's diverse and distinct data distribution.}
We compare user preferences across different stratifications of input data (e.g., common versus rare languages) and observe which affect observed preferences most (Section~\ref{sec:analysis}).
For example, while user preferences stay relatively consistent across various programming languages, they differ drastically between different task categories (e.g. frontend/backend versus algorithm design).
We also observe variations in user preference due to different features related to code structure 
(e.g., context length and completion patterns).
We open-source \systemName and release a curated subset of code contexts.
Altogether, our results highlight the necessity of model evaluation in realistic and domain-specific settings.





\begin{figure*}[htbp!]
    \centering
    % cifar100
    \includegraphics[width=.25\textwidth]{./figures/network-hierarchy/r18_cifar100.png}\hfill
    \includegraphics[width=.25\textwidth]{./figures/network-hierarchy/r50_cifar100.png}\hfill
    \includegraphics[width=.25\textwidth]{./figures/network-hierarchy/vgg16_cifar100.png}\hfill
    \includegraphics[width=.25\textwidth]{./figures/network-hierarchy/vgg19_cifar100.png}\\
    


    % imagenet
    \includegraphics[width=.25\textwidth]{./figures/network-hierarchy/r18_imagenet.png}\hfill
    \includegraphics[width=.25\textwidth]{./figures/network-hierarchy/r50_imagenet.png}\hfill
    \includegraphics[width=.25\textwidth]{./figures/network-hierarchy/vgg16_imagenet.png}\hfill
    \includegraphics[width=.25\textwidth]{./figures/network-hierarchy/vgg19_imagenet.png}
    
    \caption{\textbf{Representational Convergence Across a Network Hierarchy.} We plot the evolution of alignment scores (computed between different seeds of the same network architecture) across the network hierarchy for four vision network architectures trained on CIFAR100 (\textbf{Top}) and ImageNet \textbf{(Bottom)}. A consistent downward trend across layers indicates decreasing representational convergence as networks deepen. Alignment consistently follows the order: \textcolor{mplblue}{Linear} $>$ \textcolor{mplorange}{Procrustes} $>$ \textcolor{mplgreen}{Permutation}, reflecting the progressively stricter nature of the metrics. Notably, \textcolor{mplorange}{Procrustes} transformations align representations nearly as well as \textcolor{mplblue}{Linear} transformations, suggesting that most variability is due to rotations rather than more complex transformations. Even \textcolor{mplgreen}{Permutation} scores---despite their strictness---achieve substantial alignment, indicating a strong one-to-one correspondence between neurons across seeds, which points to stable, convergent neuron-level representations. Error bars represent the standard deviation computed across $5$-fold cross-validation.}
    \label{fig:network-hierarchy}
\end{figure*}


\section{Problem Statement}
We consider two representations, 
\[
\bm{X}_i\in \mathbb{R}^{M\times N_x} \quad \text{and} \quad \bm{X}_j\in \mathbb{R}^{M\times N_y},
\]
obtained from different models over \(M\) unique stimuli, where \(N_x\) and \(N_y\) denote the number of neurons (or units) in each representation, respectively. To systematically identify the minimal transformations needed for alignment, we use three metrics that quantify similarities between networks while ignoring nuisance transformations. These metrics are ordered to reflect progressively more permissive mapping functions (from strict to flexible):

\begin{enumerate}
    \item \textbf{Permutation Score (and its extension, the Soft-Matching Score):}  
    This metric treats the order of units as arbitrary, reflecting similarity in representational form---i.e., how the representations are assigned at the level of individual neurons. In this case, we seek a mapping matrix \(\bm{M}\) that minimizes
    \[
    \min_{\bm{M}} \|\bm{X}_j - \bm{M}\bm{X}_i\|_2^2.
    \]
    For the permutation score, when \(N_x = N_y\), \(\bm{M}\) is constrained to be a permutation matrix:
    \[
    \bm{M} \in \mathcal{P}(N).
    \]
    This is computed using the linear sum assignment algorithm~\citep{burkard2012assignment}. When \(N_x \neq N_y\), we use a generalized version called the soft-matching score~\citep{khosla2024soft}. Here, the mapping matrix \(\bm{M} \in \mathbb{R}^{N_x \times N_y}\) is constrained such that its entries are nonnegative and satisfy
    \[
    \sum_{j=1}^{N_y} M_{ij} = \frac{1}{N_x} \quad \text{for all } i = 1, \dots, N_x,
    \]
    \[
    \sum_{i=1}^{N_x} M_{ij} = \frac{1}{N_y} \quad \text{for all } j = 1, \dots, N_y.
    \]
    These constraints place \(\bm{M}\) in the transportation polytope \(\mathcal{T}(N_x, N_y)\). The optimal soft-matching mapping is obtained by solving
    \[
    \min_{\bm{M} \in \mathcal{T}(N_x, N_y)} \|\bm{X}_j - \bm{M}\bm{X}_i\|_2^2,
    \]
    using the network simplex algorithm. In the special case when \(N_x = N_y\), the soft-matching score reduces to the permutation score.

    \item \textbf{Procrustes Score:}  
    The Procrustes metric discounts rotations and reflections, capturing similarity in the overall geometric shape of the representation. When \(N_x \neq N_y\), we first zero-pad the lower-dimensional representation so that both representations become square. Then, we seek an orthogonal mapping matrix \(\bm{M}\) by solving
    \[
    \min_{\bm{M} \in \mathcal{O}(N)} \|\bm{X}_j - \bm{M}\bm{X}_i\|_2^2,
    \]
    where
    \[
    \mathcal{O}(N) = \{\bm{M} \in \mathbb{R}^{N \times N} \mid \bm{M}^\top \bm{M} = \bm{I}\}.
    \]

    \item \textbf{Linear Regression Score:}  
    The linear regression approach imposes no constraints on the mapping matrix, thereby discounting only affine transformations. Here, \(\bm{M}\) is allowed to be any matrix in \(\mathbb{R}^{N_y \times N_x}\), and we solve
    \[
    \min_{\bm{M} \in \mathbb{R}^{N_y \times N_x}} \|\bm{X}_j - \bm{M}\bm{X}_i\|_2^2.
    \]
\end{enumerate}

Once the optimal mapping matrix \(\bm{M}\) is computed for each metric, we report the alignment using the pairwise correlation:
\[
\texttt{Alignment} = \texttt{corr}\left(\bm{X}_j,\ \bm{M}\bm{X}_i\right)
\]
Since the metric is asymmetric, we report the average alignment score computed in both directions: \texttt{corr}$\left(\bm{X}_j,\  \bm{M}_1\bm{X}_{i}\right)$ and \texttt{corr}$\left(\bm{X}_i,\ \bm{M}_2\bm{X}_{j}\right)$, where $\bm{M}_1$ and $\bm{M}_2$ are the respective transformation matrices.

\noindent
By leveraging these three metrics---which progressively relax the mapping constraints from strict (soft-matching / permutation) to flexible (linear regression)---we can dissect the nature of representational alignment across networks, distinguishing between similarity in representational form (captured by the soft-matching score, which reduces to the permutation score when \(N_x = N_y\)), overall geometric shape (captured by the Procrustes score), and information content (captured by the linear regression score).

\section{Method}
Below, we outline the general framework for evaluating alignment between different vision models (ResNet18, ResNet50~\citep{he2016deep}, VGG16, VGG19~\citep{simonyan2014very}).
\subsection{Network Training}
Consider a model type denoted by $M$. We train a pair of models, $\{M_1, M_2\}$ initialized with two different random seeds. We initialize the models using a uniform Xavier distribution~\citep{glorot2010understanding}. This controlled setup ensures that the two models are identical in architecture and achieve comparable task performance, allowing us to isolate the effects of stochastic variations in the SGD process (such as initialization differences and input order). By comparing the representations from these models, we can quantify the minimal set of transformations required to align them. All models are trained from scratch on CIFAR100 and ImageNet for $100$ and $80$ epochs respectively. We save model weights at every epoch and additionally store the best-performing weights based on test-set performance for each dataset.

\subsection{Comparing Convolutional Layers}
For a given convolutional layer, let the activations be represented by $\bm{X} \in \mathbb{R}^{m \times h \times w \times c}$, where \(m\) is the number of stimuli, \(h\) and \(w\) denote the spatial height and width, and \(c\) is the number of channels (i.e., convolutional filters). In theory, each convolutional layer produces a feature map whose spatial dimensions are equivariant to translations. That is, a circular shift along the spatial dimensions yields an equivalent representation (up to a shift). As a result, one could compare the full spatial activation patterns between networks by considering an equivalence relation that allows for spatial shifts. However, evaluating an alignment that optimizes over all possible shifts together with another alignment (\emph{e.g.}, Procrustes) is computationally costly.

Previous work has demonstrated that in many convolutional layers the optimal spatial shift parameters tend to be close to zero~\citep{williams2021generalized}. This observation motivates our simpler approach: rather than collapsing the spatial dimensions by flattening the entire feature map (which would yield an activation matrix of size \(m \times (h \cdot w \cdot c)\), we instead extract a single representative value from each channel. In our experiments, we choose the value at the center pixel of each channel. This reduces the activation tensor \(\bm{X}\) to a two-dimensional matrix $\bm{X}^{\prime} \in \mathbb{R}^{m \times c}$,
where each row corresponds to a stimulus and each column to a channel. This strategy drastically reduces the computational complexity of computing the optimal mapping. For instance, aligning the full spatially flattened representations would incur a runtime of \(\mathcal{O}(m h^2w^2c + h^3w^3c^3)\), whereas our center-pixel approach reduces the problem to aligning an \(m \times c\) matrix, resulting in a more tractable complexity.

%By using this simplified comparison strategy, we exploit the translation equivariance of convolutional layers while avoiding the heavy computational burden associated with exhaustively searching over spatial shifts.


\subsection{Computing Alignment}
Alignment scores for all networks trained on CIFAR100 and ImageNet are evaluated on their respective test sets. We compute alignment between corresponding layers of networks with the same architecture (Sec.~\nameref{sec: convergence-evolution}) and between all pairs of layers for networks with different architectures (Sec.~\nameref{par: inter-network}). All scores are reported as the mean across $k$-fold cross-validation, with $k = 5$ in our case. This helps prevent overfitting when computing the optimal transformation. 

\section{Results}

\subsection{Evolution of Convergence Across the Network Hierarchy}
\label{sec: convergence-evolution}



\paragraph{How convergence varies with network depth.}
When comparing representational convergence across the network hierarchy for different seeds of the same architecture, we observe that convergence is strongest in the earliest layers and gradually diminishes in deeper layers (Fig.~\ref{fig:network-hierarchy}). This pattern is consistent across all three metrics and across networks trained on different datasets (CIFAR100 and ImageNet). The high alignment in early layers likely arises because they capture fundamental, low-frequency features (\emph{e.g.,} edges, corners, contrast) that are universal across representations~\citep{rahaman2019spectral, bau2017network, zeiler2014visualizing}. In contrast, deeper layers, while still showing significant alignment ($>$ 0.5), exhibit greater variability due to their sensitivity to specific training conditions and noise. This trend also holds when comparing networks with different architectures, underscoring the robustness of hierarchical convergence across diverse models.

\begin{figure*}[htbp!]
    \centering
    \includegraphics[scale=0.5]{./figures/training-evo/cifar100_training_evo.png}
    \includegraphics[scale=0.5]{./figures/training-evo/imagenet_training_evo.png}
    \caption{\textbf{Representational Alignment through Training Evolution.} We visualize the evolution of Procrustes alignment between network pairs during task optimization on CIFAR100 \textbf{(Top)} and ImageNet \textbf{(Bottom)}. Lighter shades indicate earlier epochs, progressively darkening with later epochs. The plots span from epoch $0$ (untrained) to epoch $10$, with task performance improving over time. Epoch progression can be inferred from the increasing task performance along the $x$-axis.}
    \label{fig:training-evo}
\end{figure*}

\paragraph{Minimal transformations needed to align representations.}
Across all layers, we find that alignment scores increase as the mapping functions become more permissive (Permutation $\rightarrow$ Procrustes $\rightarrow$ Linear), as expected. However, linear mappings provide only modest improvements over Procrustes correlations, indicating that rotational transformations are sufficient to capture the majority of alignment information. This suggests that the added flexibility of linear mappings---such as scaling and shearing---does not substantially enhance alignment beyond what is achieved with Procrustes transformations. Importantly, because Procrustes is symmetric, this result highlights that alignment reflects a deeper similarity in the geometric structure of representations, rather than merely the ability of one representation to predict another.



\paragraph{Simple permutations achieve significant alignment.}  
\label{par: privileged-axes}
Despite the strict constraints imposed by permutation-based alignment, Permutation scores achieve surprisingly high alignment levels, indicating a strong one-to-one correspondence between individual neurons across network instances. This suggests that convergent learning extends down to the level of single neurons, even without allowing for more flexible transformations.

%\begin{table}[h!]
%    \centering
%    \resizebox{0.5\textwidth}{!}{
%    \begin{tabular}{cccc}
%    \toprule
%    Model (Dataset) & Native Basis & Rotated Basis & Mean Difference $(\%)$\\
%    \midrule
%        ResNet18 (CIFAR100) & $0.43$ & $0.37$ & $13.23\%$ \\
%        ResNet50 (CIFAR100) & $0.41$ & $0.39$ & $5.48\%$ \\
%        VGG16 (CIFAR100) & $0.44$ & $0.36$ & $16.36\%$ \\
%        VGG19 (CIFAR100) & $0.44$ & $0.37$ & $14.59\%$ \\
%        ResNet18 (ImageNet) & $0.44$ & $0.30$ & $29.72\%$ \\
%        ResNet50 (ImageNet) & $0.44$ & $0.34$ & $24.17\%$ \\
%        VGG16 (ImageNet) & $0.49$ & $0.32$ & $36.22\%$ \\
%        VGG19 (ImageNet) & $0.47$ & $0.31$ & $34.17\%$ \\
%    \bottomrule
%    \end{tabular}
%    }
%    \caption{\textbf{Sensitivity to Individual Neuron Tuning.} We rotate the native axes for each of the networks by a random rotation matrix, as described in Section~\nameref{sec:privileged-axes} and recompute alignment scores. We average the alignment scores across all layers, and observe a drop in these scores, pointing towards the fact that single neuron tuning affects how networks represent information.}
%    \label{tab: perm-rotated}
%\end{table}

\begin{table*}[!b]
    \centering
    %\resizebox{0.5\textwidth}{!}{
    \begin{tabular}{cccc}
    \toprule
    Model (Dataset) & Native (Min / Max) & Rotated (Min / Max) & Difference ($\%$) (Min / Max)\\
    \midrule
        ResNet18 (CIFAR100) & $0.247 \ / \ 0.752$ & $0.215 \ / \ 0.689$ & $6.40\% \ / \ 51.26\%$ \\
        ResNet50 (CIFAR100) & $0.254 \ / \ 0.828$ & $0.242 \ / \ 0.828$ & $-3.38\% \ / \ 35.29\%$ \\
        VGG16 (CIFAR100) & $0.277 \ / \ 0.769$ & $0.239 \ / \ 0.661$ & $5.66\% \ / \ 63.97\%$ \\
        VGG19 (CIFAR100) & $0.273 \ / \ 0.758$ & $0.231 \ / \ 0.684$ & $2.15\% \ / \ 35.36\%$ \\
        ResNet18 (ImageNet) & $0.202 \ / \ 0.721$ & $0.154 \ / \ 0.581$ & $10.47\% \ / \ 201.84\%$ \\
        ResNet50 (ImageNet) & $0.225 \ / \ 0.791$ & $0.143 \ / \ 0.739$ & $2.61\% \ / \ 180.75\%$ \\
        VGG16 (ImageNet) & $0.288 \ / \ 0.746$ & $0.181 \ / \ 0.599$ & $21.72\% \ / \ 81.12\%$ \\
        VGG19 (ImageNet) & $0.292 \ / \ 0.799$ & $0.171 \ / \ 0.562$ & $41.95\% \ / \ 89.68\%$ \\
    \bottomrule
    \end{tabular}
    %}
    \caption{\textbf{Sensitivity of permutation scores to representational axes.} We rotate the native axes for each of the networks by a random rotation matrix and recompute alignment scores. We report the min / max alignment scores across all layers for the alignment observed in the native and rotated bases and observe that rotations almost always reduce the alignment score, highlighting the distinguished nature of the representational axes for all layers in networks.}
    \label{tab:perm-rotated}
\end{table*}

To further probe this result and assess the depth of convergent learning, we tested the sensitivity of permutation alignment to changes in the representational basis. Specifically, we applied a random rotation matrix $\bm{Q} \in \mathbb{R}^{n \times n}$ to the converged basis of a neural representation, where $n$ is the number of neurons in a given layer. The rotation matrix was sampled from a Haar distribution via a $QR$ decomposition, ensuring that all orthogonal matrices were equally likely. We then recomputed the Permutation score after applying this rotation.


We conducted this analysis by taking response matrices from two identical DCNNs (initialized with different random seeds) at a given convolutional layer, $\{\bm{X}_1, \bm{X}_2\} \in \mathbb{R}^{m \times n}$, where $m$ represents the number of stimuli. We applied the random rotation $\bm{Q}$ to one network’s responses and computed the resulting permutation-based correlation score, $s_{\text{perm}}(\bm{X}_1\bm{Q}, \bm{X}_2)$. This process was repeated across all convolutional layers, with the alignment differences summarized in Table~\ref{tab:perm-rotated}.

These rotations consistently reduced alignment, with a drop between $\sim6-51\%$ on CIFAR-100 and $\sim 10-202\%$ on ImageNet for ResNet18 across all layers. This significant decrease highlights that the learned representations are not rotationally invariant and that the specific basis in which features are encoded is meaningfully preserved across networks. In other words, convergent learning aligns not just the overall representational structure but also the specific axes along which features are encoded. This observation echoes recent findings by~\citep{khosla2024privileged}, who report the existence of privileged axes in biological systems as well as the penultimate layer representations of trained artificial networks.  



\paragraph{Hierarchical correspondence holds across metrics.}
\label{par: inter-network}
\begin{figure*}[b!]
    \centering
    % cifar100
    \includegraphics[width=.16\textwidth]{./figures/inter-model-analysis-results/resnet18_resnet50_orthogonal_procrustes_cifar100_comparison.png}\hfill
    \includegraphics[width=.16\textwidth]{./figures/inter-model-analysis-results/resnet18_vgg16_orthogonal_procrustes_cifar100_comparison.png}\hfill
    \includegraphics[width=.16\textwidth]{./figures/inter-model-analysis-results/resnet18_vgg19_orthogonal_procrustes_cifar100_comparison.png}\hfill 
    \includegraphics[width=.16\textwidth]{./figures/inter-model-analysis-results/vgg16_resnet50_orthogonal_procrustes_cifar100_comparison.png}\hfill
    \includegraphics[width=.16\textwidth]{./figures/inter-model-analysis-results/vgg19_resnet50_orthogonal_procrustes_cifar100_comparison.png}\hfill 
    \includegraphics[width=.16\textwidth]{./figures/inter-model-analysis-results/vgg16_vgg19_orthogonal_procrustes_cifar100_comparison.png}\\
    
    % imagenet
    \includegraphics[width=.16\textwidth]{./figures/inter-model-analysis-results/resnet18_resnet50_orthogonal_procrustes_imagenet_comparison.png}\hfill
    \includegraphics[width=.16\textwidth]{./figures/inter-model-analysis-results/resnet18_vgg16_orthogonal_procrustes_imagenet_comparison.png}\hfill
    \includegraphics[width=.16\textwidth]{./figures/inter-model-analysis-results/resnet18_vgg19_orthogonal_procrustes_imagenet_comparison.png}\hfill 
    \includegraphics[width=.16\textwidth]{./figures/inter-model-analysis-results/resnet50_vgg16_orthogonal_procrustes_imagenet_comparison.png}\hfill
    \includegraphics[width=.16\textwidth]{./figures/inter-model-analysis-results/resnet50_vgg19_orthogonal_procrustes_imagenet_comparison.png}\hfill 
    \includegraphics[width=.16\textwidth]{./figures/inter-model-analysis-results/vgg16_vgg19_orthogonal_procrustes_imagenet_comparison.png}
    
    \caption{\textbf{Inter-Model Orthogonal Procrustes.} We consider all pairs of vision models, and for each pair, compute the alignment scores between every pair of layers using the orthogonal Procrustes metric trained on CIFAR100 \textbf{(Top)} and ImageNet \textbf{(Bottom)}. Gray line plots denote the \textbf{maximum} alignment value for each network over rows (right line) and columns (top line). A common trend that is observed here is the consistent relationships between layers of CNNs trained with different architectures.}
    \label{tab:procrustes_inter_model}
\end{figure*}
\begin{figure*}
    \centering
    % cifar100
    \includegraphics[width=.16\textwidth]{./figures/inter-model-analysis-results/resnet18_resnet50_soft_match_cifar100_comparison.png}\hfill
    \includegraphics[width=.16\textwidth]{./figures/inter-model-analysis-results/resnet18_vgg16_soft_match_cifar100_comparison.png}\hfill
    \includegraphics[width=.16\textwidth]{./figures/inter-model-analysis-results/resnet18_vgg19_soft_match_cifar100_comparison.png}\hfill 
    \includegraphics[width=.16\textwidth]{./figures/inter-model-analysis-results/vgg16_resnet50_soft_match_cifar100_comparison.png}\hfill
    \includegraphics[width=.16\textwidth]{./figures/inter-model-analysis-results/vgg19_resnet50_soft_match_cifar100_comparison.png}\hfill 
    \includegraphics[width=.16\textwidth]{./figures/inter-model-analysis-results/vgg16_vgg19_soft_match_cifar100_comparison.png}\\
    
    % imagenet
    \includegraphics[width=.16\textwidth]{./figures/inter-model-analysis-results/resnet18_resnet50_soft_match_imagenet_comparison.png}\hfill
    \includegraphics[width=.16\textwidth]{./figures/inter-model-analysis-results/resnet18_vgg16_soft_match_imagenet_comparison.png}\hfill
    \includegraphics[width=.16\textwidth]{./figures/inter-model-analysis-results/resnet18_vgg19_soft_match_imagenet_comparison.png}\hfill 
    \includegraphics[width=.16\textwidth]{./figures/inter-model-analysis-results/resnet50_vgg16_soft_match_imagenet_comparison.png}\hfill
    \includegraphics[width=.16\textwidth]{./figures/inter-model-analysis-results/resnet50_vgg19_soft_match_imagenet_comparison.png}\hfill 
    \includegraphics[width=.16\textwidth]{./figures/inter-model-analysis-results/vgg16_vgg19_soft_match_imagenet_comparison.png}
    
    \caption{\textbf{Inter-Model Soft-Match.} Identical to the preceding plot, we compute the layer-wise soft-matching score for all pairs of visual networks trained on CIFAR100 \textbf{(Top)} and ImageNet \textbf{(Bottom)}. This metric also reveals consistent relationships between layers of CNNs trained with different architectures.}
    \label{tab:sm_inter_model} 
\end{figure*}
Previous studies have shown that, for architecturally identical networks trained from different initializations, the most similar layer in one network to a given layer in another is the corresponding architectural layer~\citep{kornblith2019similarity}. However, this finding has primarily been supported using metrics invariant to affine transformations (\emph{e.g.,} CCA, SVCCA). Here, we extend this result by showing that stricter metrics---such as Procrustes and soft-matching scores---also reveal the same hierarchical correspondence (Figs.~\ref{tab:procrustes_inter_model}, ~\ref{tab:sm_inter_model}), even when comparing networks with different architectures. This suggests that the hierarchical alignment of representations is a fundamental property of neural networks, robust to both architectural differences and the choice of alignment metric. Moreover, our results show that both representational shape (captured by Procrustes) and neuron-level tuning (captured by soft-matching) follow similar alignment patterns, reinforcing the consistency of this hierarchical organization across different levels of representational analysis.


\subsection{Evolution of convergence over training}
We next explore how representational convergence evolves during the training process. Specifically, we compute Procrustes alignment scores between pairs of networks over the first $10$ epochs of training for both CIFAR100 and ImageNet.

% ood amplification
%\begin{figure}
%    \centering
%    \includegraphics[scale=0.33]{figures/ood-divergence/ood_amplification.png}
%    \caption{\textbf{Amplifying Model Differences.}}
%    \label{fig:enter-label}
%\end{figure}

\begin{figure*}[t!]
    \centering
    \includegraphics[scale=0.7]{./figures/ood-divergence/ood_amplification.png}
    \caption{\textbf{Procrustes Score-based alignment between networks sharing the same architecture but trained with different random seeds, plotted as a function of layer depth.} Alignment is measured using within-distribution (WD) stimuli (ImageNet test set) and out-of-distribution (OOD) stimuli, with OOD values averaged across $17$ datasets. Error bars represent the standard error computed across the $(n = 17)$ OOD datasets.}
    \label{fig:ood-amplification}
\end{figure*}


As shown in Fig.~\ref{fig:training-evo}, a striking pattern emerges: the majority of representational convergence happens within the first epoch---long before networks approach optimal task performance. This rapid early convergence suggests that factors independent of task optimization drive much of the representational alignment. Shared input statistics, architectural biases, and early training dynamics seem to play a dominant role in shaping the learned representations, overshadowing the influence of the final task-specific solution.

This observation challenges prevailing hypotheses that attribute convergence to constraints imposed by the task, such as the \emph{contravariance principle}~\citep{cao2021explanatory} or the \emph{convergence via task generality} hypothesis~\citep{huh2024platonic}. These theories propose that networks converge because they are steered toward a limited subspace of solutions capable of achieving high task performance. However, the early emergence of representational alignment---well before networks achieve such performance---implies that convergence is not primarily driven by task-related pressures but is instead rooted in the network’s inductive biases, statistical properties of the input data and early training dynamics.

Interestingly, this result aligns with findings from another study on the early phase of training~\citep{frankle2020early}, which demonstrated that substantial changes in network representations occur within the first few hundred iterations, even before meaningful task learning begins. Their work showed that perturbations to network weights in this early phase---such as re-initialization or weight shuffling---can significantly degrade performance, suggesting that early representations are already highly structured and data-dependent. Moreover, they found that pre-training using only the input distribution \emph{p(x)} (\emph{e.g.,} through self-supervised tasks) could approximate the changes seen in early supervised training, albeit requiring substantially longer pre-training. This also lends credence to the idea that shared input statistics, rather than task-specific labels, play a critical role in early convergence.


For the earliest convolutional layers, we find that training has minimal impact on representational similarity and can even reduce it in some cases. This phenomenon arises because early layers compute a largely linear function of the input, even in untrained networks, allowing them to align well with simple linear transformations. As training progresses, these layers may undergo minor adjustments that slightly disrupt this initial alignment, though they remain highly similar overall. This result highlights that the convergence of early-layer representations is not driven by the specifics of the training process.

%\begin{figure}
%    \centering
%    \includegraphics[scale=0.5]{figures/ood-divergence/pearson_ood_all_layers.png}
%    \caption{\textbf{Pearson Correlation Across Network Hierarchy.}}
%    \label{fig: corr-ood-network-hierarchy}
%\end{figure}

\begin{figure*}[htbp!]
    \begin{tabular}{cc}
      \includegraphics[scale=0.4]{./figures/ood-divergence/resnet50_ood_divergence.png} & \includegraphics[scale=0.5]{./figures/ood-divergence/pearson_ood_all_layers.png}\\
    \textbf{(a)} & \textbf{(b)} 
    \end{tabular}
    \caption{\textbf{Convergence on OOD Inputs.} \textbf{(a)} Procrustes alignment vs. task performance of ResNet50 on each of the $17$ datasets for the first convolutional layer \textbf{(left)} and the penultimate \textbf{(right)} layer. \textbf{(b)} Correlation between these variables as a function of network depth (normalized by each models depth). Here, the error bars represent the standard deviation of $r$ after excluding one dataset at a time (i.e., using $n=16$ remaining datasets).}
    \label{fig: ood-convergence}
\end{figure*}    


\subsection{Convergence across distribution shifts}

\label{sec:ood-results}
In the previous sections, we examined representational alignment under the same input distributions used for training. However, a critical question remains: does representational convergence persist under distribution shifts? To explore this, we analyzed the internal representations of ImageNet-trained DCNNs when exposed to out-of-distribution (OOD) stimuli. We used $17$ OOD datasets from~\citep{geirhos2018imagenet}, all sharing the same $16$ coarse labels as ImageNet~\citep{deng2009imagenet}, allowing for a controlled comparison of representational alignment under varying distributional shifts. The specifics of each OOD dataset has been described in Sec.~\nameref{sec: ood-dataset-details}.


We computed representational alignment across these datasets using the Procrustes metric and observed a consistent pattern: OOD inputs amplify differences in the later layers of the networks, while early layers maintain comparable alignment levels between in-distribution and OOD stimuli (Fig.~\ref{fig:ood-amplification}). We hypothesize that this pattern arises as a result of early layers capturing basic, universal features (\emph{e.g.,} edges, corners, textures) that remain nearly identical across distributions, whereas later layers encode more task-specific features that are more sensitive to distributional shifts, thus amplifying the divergence between models.

Moreover, we found a strong correlation between representational alignment in later layers and the networks' classification accuracy on the OOD datasets. Datasets where models maintained higher accuracy showed stronger alignment, whereas datasets with lower accuracy exhibited weaker alignment. This correlation was notably weaker in early layers but increased progressively with network depth across all architectures (Fig.~\ref{fig: ood-convergence}). We extend these analyses to other vision networks in Fig.~\ref{fig:ood-convergence-appendix}.

These results have several important implications. First, the stability of early-layer alignment across distributions suggests that these layers encode generalizable features that are consistent across both network initializations and input distributions. This highlights their role as a shared foundation for higher-level processing, which becomes more specialized and sensitive to distribution shifts in later layers. Second, these findings inform model-brain comparisons. Prior studies have shown that diverse architectures and learning objectives can yield similar brain predictivity~\citep{conwell2024large}. However, the observed amplification of representational divergence in later layers under OOD conditions suggests that using OOD stimuli could be an effective strategy for distinguishing between models and identifying more brain-like models. 

An important potential consequence of this result is that if the alignment between models in early layers is largely constant across data distributions, it might be possible to fine-tune models only in later layers to improve their OOD generalization capabilities.



\section{Discussion}
This study fills critical gaps in our understanding of convergent learning, offering a comprehensive analysis of how representational alignment between independently trained networks varies across network depth, training, and distribution shifts. We systematically explored how different alignment metrics—with varying levels of transformation invariance---capture representational similarities, providing a more nuanced view of convergent learning than previous work.

Despite these insights, the study has important limitations. While we show that alignment emerges early during training, we do not quantify precisely when within the first epoch this convergence occurs. Further, different networks may learn at varying rates---so epoch one for one network may not reflect an analogous state for another network. Comparing networks epoch-wise may overlook this difference in learning speeds. Nonetheless, since alignment at the end of the first epoch already matches the alignment seen at convergence, the broader conclusions about early representational alignment remain robust.

Another limitation stems from the alignment metrics themselves. Although our metrics reveal that alignment stabilizes quickly and does not improve significantly over training, this could reflect the limitations of the metrics rather than the absence of representational changes. Prior work~\citep{bo2024evaluating} has demonstrated that certain alignment metrics may fail to capture subtle shifts in representations that are critical for task performance. It remains possible that more sensitive or alternative metrics could reveal gradual representational alignments over training that are invisible to the methods employed here.

Finally, our method for computing alignment focuses on the center pixel of each feature map, enforcing a strict spatial correspondence between representations.  This approach may underestimate alignment, especially in cases where two filters detect the same feature at slightly shifted spatial locations. While more flexible approaches---such as incorporating spatial shifts into alignment computations---could offer a more accurate view, they are computationally intensive and were beyond the scope of this work. However, prior research has shown that optimal spatial shifts in many convolutional layers are typically close to zero~\citep{williams2021generalized}, supporting the validity of this approximation. Developing scalable methods that account for such spatial variability remains an important direction for future research.

\bibliographystyle{ccn_style}
\bibliography{ccn_style}

\appendix
\renewcommand{\thefigure}{A\arabic{figure}}
\renewcommand{\thetable}{A\arabic{table}}
\setcounter{figure}{0}  
\setcounter{table}{0}
\newpage
\section{Appendix}

\subsection{Out-Of-Distribution Dataset}
\label{sec: ood-dataset-details}
All OOD datasets were directly taken from~\citep{geirhos2018imagenet}, which share the same $16$ coarse labels as ImageNet. Concretely, this set consists of the following classes: Airplane, Bear, Bicycle, Bird, Boat, Bottle, Car, Cat, Chair, Clock, Dog, Elephant, Keyboard, Knife, Oven, Truck.\\

\noindent
Each of the $17$ stylized datasets are described below:
\begin{itemize}
    \item \textbf{Color:} Half of the images are randomly converted to grayscale, and the rest kept in their original colormap. 
    \item \textbf{Stylized:} Textures from one class are transferred to the shapes of another, ensuring that object shapes remain preserved.
    \item \textbf{Sketch:} Cartoon-style sketches of objects representing each class.  
    \item \textbf{Edges:} Generated from the original ImageNet dataset using the Canny edge detector to produce edge-based representations.
    \item \textbf{Silhouette:} Black objects on a white background generated from the original dataset.
    \item \textbf{Cue Conflict:} Images with textures that conflict with shape categories, generated using iterative style transfer~\citep{gatys2015neural}, where \textbf{Texture} dataset images serve as the style and \textbf{Original} dataset images as the content.
    \item \textbf{Contrast:} Image variants modified to different contrast levels.
    \item \textbf{High-Pass / Low-Pass:} Images processed with Gaussian filters to emphasize either high-frequency or low-frequency components.
    \item \textbf{Phase-Scrambling:} Images with phase noise added to frequency components, introducing varying levels of distortion from $0^\circ$ to $180^\circ$.
    \item \textbf{Power-Equalization:} The images were processed to normalize the power spectra across the dataset by adjusting all amplitude spectra to match the mean value.
    \item \textbf{False-Color:} The colors of the images were inverted to their opponent colors while maintaining constant luminance, using the DKL color space.
    \item \textbf{Rotation:} Rotated images ($0^\circ$, $90^\circ$, $180^\circ$, or $270^\circ$) to test rotational invariance.
    \item \textbf{Eidolon I, II, III:} The images were distorted using the Eidolon toolbox, with variations in the coherence and reach parameters to manipulate both local and global image structures for each intensity level.
    \item \textbf{Uniform Noise:} White uniform noise was added to the images in a varying range to assess robustness, with pixel values exceeding the bounds clipped to the range $[0, 255]$.
\end{itemize}

\subsection{Additional Results}
In this section, we compute Procrustes alignment for the remainder of vision networks at the first convolutional and penultimate layer to assess whether a similar phenomenon holds as described in Sec.~\nameref{sec:ood-results}. Indeed, in Fig.~\ref{fig:ood-convergence-appendix}, we observe a similar trend that was observed earlier, i.e., alignment mirrors task performance at higher network depths. 

%\begin{figure*}[h!]
%    \centering
%    \begin{tabular}{cc}
%        \includegraphics[scale=0.4]{figures/ood-divergence/resnet18_ood_divergence.png} &      
%        \includegraphics[scale=0.4]{figures/ood-divergence/vgg16_ood_divergence.png} \\
%        \textbf{(a)} ResNet18 & \textbf{(b)} VGG16 \\
%        \multicolumn{2}{c}{\includegraphics[scale=0.4]{8pager/figures/ood-divergence/vgg19_ood_divergence.png}} \\
%        \multicolumn{2}{c}{\textbf{(c)} VGG19}
%    \end{tabular}
%    \caption{\textbf{Procrustes alignment vs. task
%performance} We compute the Procrustes alignment of different network architectures on each of the $17$ datasets for the first convolutional layer \textbf{(left)} and the penultimate \textbf{(right)} layer from \textbf{(a)} - \textbf{(c)}.}
%    \label{fig:ood-convergence-appendix}
%\end{figure*}
\begin{figure}[h!]
    \centering
    \begin{tabular}{c}
        \includegraphics[scale=0.4]{./figures/ood-divergence/resnet18_ood_divergence.png}\\   
        \textbf{(a)} ResNet18\\
        \includegraphics[scale=0.4]{./figures/ood-divergence/vgg16_ood_divergence.png} \\
        \textbf{(b)} VGG16 \\
        \includegraphics[scale=0.4]{./figures/ood-divergence/vgg19_ood_divergence.png} \\
        \textbf{(c)} VGG19
    \end{tabular}
    \caption{\textbf{Procrustes alignment vs. task
performance} We compute the Procrustes alignment of different network architectures on each of the $17$ datasets for the first convolutional layer \textbf{(left)} and the penultimate \textbf{(right)} layer from \textbf{(a)} - \textbf{(c)}.}
    \label{fig:ood-convergence-appendix}
\end{figure}



\end{document}
