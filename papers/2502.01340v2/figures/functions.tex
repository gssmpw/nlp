\begin{figure}[h]
    \centering
    \begin{minipage}[b]{0.48\textwidth}
        \centering
        \includegraphics[width=\textwidth]{figures/reaction_probability_function.png}
        \subcaption{Reaction probability}
        \label{fig:reaction-probability}
    \end{minipage}
    \hfill
    \begin{minipage}[b]{0.48\textwidth}
        \centering
        \includegraphics[width=\textwidth]{figures/opinion_shift_function.png}
        \subcaption{Opinion shift}
        \label{fig:opinion-shift}
    \end{minipage}
    \caption{Reaction probability and opinion shift functions for different message opinion values ($o_m$). Left: Corresponding reaction probability functions show how interaction likelihood varies with user opinions, exhibiting peaks at ideologically aligned positions. Right: Opinion shift curves demonstrate how user opinions ($o_u$) change after message exposure, with positive/negative values indicating shifts toward/away from the message opinion. The crossing points at zero represent opinion stability thresholds. Both functions reveal stronger effects for extreme message values ($o_m = 1$), with the asymmetric shapes capturing homophilic preferences and cross-ideological dynamics in opinion formation and interaction patterns.}
    \label{fig:opinion-reaction-functions}
\end{figure}