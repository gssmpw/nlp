\begin{table*}[h!]
\centering
\small
\caption{Key Findings from User Study}
\label{tab:user-study-findings}
\begin{tabularx}{\textwidth}{>{\centering\arraybackslash}p{1.5cm}>{\raggedright\arraybackslash}X>{\raggedright\arraybackslash}X}
\toprule
\multicolumn{1}{c}{\textbf{Analysis}} & \multicolumn{1}{c}{\textbf{Key Finding}} & \multicolumn{1}{c}{\textbf{Theoretical Implication}} \\
\midrule
\multirow{3}{*}{\rotatebox[origin=c]{90}{\textbf{Perception}}} 
    & Users readily detect emotional content and group-based language in polarized discussions 
    & Social and emotional signals serve as primary markers for detecting polarization \\
\cmidrule{2-3}
    & The effect of perceived emotionality on perceived polarization is mediated through perceived group salience 
    & Group-based processing, not emotional content itself, is the primary pathway to polarization perception \\
\cmidrule{2-3}
    & Polarized discussions show marked decrease in expressed uncertainty 
    & Polarization creates perception of epistemic closure, limiting space for nuanced dialogue \\
\midrule
\multirow{3}{*}{\rotatebox[origin=c]{90}{\textbf{Engagement}}} 
    & Different interaction types serve distinct social functions, with minimal correlation between comments and reposts 
    & Platform affordances shape distinct modes of user participation \\
\cmidrule{2-3}
    & Users comment significantly more in unpolarized discussions compared to polarized environments
    & Moderate environments facilitate substantive textual engagement among non-polarized populations \\
\cmidrule{2-3}
    & Highest total engagement observed in polarized contra-bias condition, coinciding with strongest negative opinion shifts
    & Arguments against systemic change gain greater salience in polarized environments, activating status quo bias \\
\midrule
\multirow{2}{*}{\rotatebox[origin=c]{90}{\textbf{Opinion}}} 
    & While directional opinion change showed no significant effects, magnitude of opinion change revealed significant impacts
    & Polarized environments increase opinion volatility rather than pushing opinions in consistent directions \\
\cmidrule{2-3}
    & Experimental conditions directly affected opinion change magnitude, while perceptual variables showed no significant mediation
    & Suggests unconscious processing mechanisms responding to structural features without awareness \\
\bottomrule
\end{tabularx}
\end{table*}