\begin{table*}[h!]
\centering
\small
\caption{Key Findings from Offline Evaluation}
\label{tab:offline-evaluation-findings}
\begin{tabularx}{\textwidth}{>{\centering\arraybackslash}p{1.5cm}>{\raggedright\arraybackslash}X>{\raggedright\arraybackslash}X}
\toprule
\multicolumn{1}{c}{\textbf{Analysis}} & \multicolumn{1}{c}{\textbf{Key Finding}} & \multicolumn{1}{c}{\textbf{Theoretical Implication}} \\
\midrule
\multirow{2}{*}{\rotatebox[origin=c]{90}{\textbf{Polarization}}} 
    & Moderate cross-ideological exposure leads to higher polarization than complete echo chambers 
    & Supports theories that political identities strengthen through opposition rather than isolation; polarization intensifies through awareness of opposing views \\
\cmidrule{2-3}
    & Polarization emerges primarily through the presence of influential accounts at extreme positions, not through natural opinion drift 
    & Confirms role of opinion leaders in transforming neutral topics into partisan issues; suggests polarization requires active curation rather than emerging spontaneously \\
\midrule
\multirow{2}{*}{\rotatebox[origin=c]{90}{\textbf{Interaction}}} 
    & Different interaction types serve distinct social functions: likes show strict in-group preference while comments flourish across ideological lines 
    & Reflects how platform affordances shape identity expression; suggests comments often serve as vehicles for performative disagreement rather than dialogue \\
\cmidrule{2-3}
    & Influencer content becomes focal point for engagement in polarized conditions, creating self-reinforcing cycles 
    & Shows how influential accounts serve as lightning rods for cross-ideological conflict; engagement patterns amplify rather than reduce polarization \\
\midrule
\multirow{2}{*}{\rotatebox[origin=c]{90}{\textbf{Content}}} 
    & Polarized environments produce consistent changes in communication: increased group identity emphasis, higher emotional content, and reduced expressions of uncertainty 
    & Demonstrates how polarization fundamentally transforms communication style independent of specific topics; suggests activation of group identities changes how people express themselves \\
\cmidrule{2-3}
    & These changes in communication style occur independently of content exposure patterns 
    & Indicates that the social-psychological dynamics of polarized environments, rather than mere exposure to different views, drive changes in how people communicate \\
\bottomrule
\end{tabularx}
\end{table*}