%% 
%% Copyright 2007-2025 Elsevier Ltd
%% 
%% This file is part of the 'Elsarticle Bundle'.
%% ---------------------------------------------
%% 
%% It may be distributed under the conditions of the LaTeX Project Public
%% License, either version 1.3 of this license or (at your option) any
%% later version.  The latest version of this license is in
%%    http://www.latex-project.org/lppl.txt
%% and version 1.3 or later is part of all distributions of LaTeX
%% version 1999/12/01 or later.
%% 
%% The list of all files belonging to the 'Elsarticle Bundle' is
%% given in the file `manifest.txt'.
%% 
%% Template article for Elsevier's document class `elsarticle'
%% with numbered style bibliographic references
%% SP 2008/03/01
%% $Id: elsarticle-template-num.tex 272 2025-01-09 17:36:26Z rishi $
%%
\documentclass[preprint,12pt,authoryear]{elsarticle}

%% Use the option review to obtain double line spacing
%% \documentclass[authoryear,preprint,review,12pt]{elsarticle}

%% Use the options 1p,twocolumn; 3p; 3p,twocolumn; 5p; or 5p,twocolumn
%% for a journal layout:
%% \documentclass[final,1p,times]{elsarticle}
%% \documentclass[final,1p,times,twocolumn]{elsarticle}
%% \documentclass[final,3p,times]{elsarticle}
%% \documentclass[final,3p,times,twocolumn]{elsarticle}
%% \documentclass[final,5p,times]{elsarticle}
%% \documentclass[final,5p,times,twocolumn]{elsarticle}

%% For including figures, graphicx.sty has been loaded in
%% elsarticle.cls. If you prefer to use the old commands
%% please give \usepackage{epsfig}

\usepackage{natbib}
\usepackage{subcaption}
\usepackage[linesnumbered, ruled, vlined]{algorithm2e}
\usepackage{multirow}
\usepackage{amsmath}
\usepackage{booktabs}
\usepackage{graphicx}
\usepackage[T1]{fontenc}
\usepackage{tabularx}
\usepackage[margin=2.5cm]{geometry}% by courtesy of Mico

%% The lineno packages adds line numbers. Start line numbering with
%% \begin{linenumbers}, end it with \end{linenumbers}. Or switch it on
%% for the whole article with \linenumbers.
%% \usepackage{lineno}

\journal{arXiv.org}

\begin{document}

\begin{frontmatter}

%% Title, authors and addresses

%% use the tnoteref command within \title for footnotes;
%% use the tnotetext command for theassociated footnote;
%% use the fnref command within \author or \affiliation for footnotes;
%% use the fntext command for theassociated footnote;
%% use the corref command within \author for corresponding author footnotes;
%% use the cortext command for theassociated footnote;
%% use the ead command for the email address,
%% and the form \ead[url] for the home page:
%% \title{Title\tnoteref{label1}}
%% \tnotetext[label1]{}
%% \author{Name\corref{cor1}\fnref{label2}}
%% \ead{email address}
%% \ead[url]{home page}
%% \fntext[label2]{}
%% \cortext[cor1]{}
%% \affiliation{organization={},
%%             addressline={},
%%             city={},
%%             postcode={},
%%             state={},
%%             country={}}
%% \fntext[label3]{}

\title{Human-Agent Interaction in Synthetic Social Networks: A Framework for Studying Online Polarization}

%% use optional labels to link authors explicitly to addresses:
%% \author[label1,label2]{}
%% \affiliation[label1]{organization={},
%%             addressline={},
%%             city={},
%%             postcode={},
%%             state={},
%%             country={}}
%%
%% \affiliation[label2]{organization={},
%%             addressline={},
%%             city={},
%%             postcode={},
%%             state={},
%%             country={}}

\author[1]{Tim Donkers} %% Author name

\author[1]{J{\"u}rgen Ziegler} %% Author name

%% Author affiliation
\affiliation[1]{organization={University of Duisburg-Essen},%Department and Organization
            addressline={Forsthausweg 2}, 
            city={Duisburg},
            postcode={47058}, 
            state={North Rhine-Westphalia},
            country={Germany}}

%% Abstract
\begin{abstract}
Online social networks have dramatically altered the landscape of public discourse, creating both opportunities for enhanced civic participation and risks of deepening social divisions. Prevalent approaches to studying online polarization have been limited by a methodological disconnect: mathematical models excel at formal analysis but lack linguistic realism, while language model-based simulations capture natural discourse but often sacrifice analytical precision. This paper introduces an innovative computational framework that synthesizes these approaches by embedding formal opinion dynamics principles within LLM-based artificial agents, enabling both rigorous mathematical analysis and naturalistic social interactions. We validate our framework through comprehensive offline testing and experimental evaluation with 122 human participants engaging in a controlled social network environment. The results demonstrate our ability to systematically investigate polarization mechanisms while preserving ecological validity. Our findings reveal how polarized environments shape user perceptions and behavior: participants exposed to polarized discussions showed markedly increased sensitivity to emotional content and group affiliations, while perceiving reduced uncertainty in the agents' positions. By combining mathematical precision with natural language capabilities, our framework opens new avenues for investigating social media phenomena through controlled experimentation. This methodological advancement allows researchers to bridge the gap between theoretical models and empirical observations, offering unprecedented opportunities to study the causal mechanisms underlying online opinion dynamics.
\end{abstract}

%%Graphical abstract
%\begin{graphicalabstract}
%\includegraphics{grabs}
%\end{graphicalabstract}

%% Keywords
\begin{keyword}
Online Polarization \sep Opinion Dynamics \sep Human-Agent Interaction \sep Large Language Models \sep Experimental User Studies \sep Agent-based Simulation
%% keywords here, in the form: keyword \sep keyword

%% PACS codes here, in the form: \PACS code \sep code

%% MSC codes here, in the form: \MSC code \sep code
%% or \MSC[2008] code \sep code (2000 is the default)

\end{keyword}

\end{frontmatter}

\documentclass[../main.tex]{subfiles}
\graphicspath{{../images/}}
\makeatletter
\def\input@path{{../images/}}
\makeatother
\begin{document}
\section{Introduction}
\begin{figure}
\centering
\begin{tikzpicture}
\node[inner sep=0pt] (ws) at (0, 0) {
\includegraphics[height=.4\textwidth, trim={10cm 0 10cm 0},clip]{world_space.png}};
\node[inner sep=0pt] (cs) at (6,0) {\includegraphics[height=.4\textwidth, trim={10cm 1cm 10cm 4cm},clip]{conf_space.png}};
\end{tikzpicture}
\vspace{-5pt}
\label{fig:pbrm_intro}
\caption{\textbf{Left}: Shows world space obstacles as grey spheres. Robots start and goal configuration is colored red and green, respectively. Configurations along the computed path are colored transparent blue. \textbf{Right:} Mapped world space scenario to configuration space. Obstacle region is the grey mesh. Red spheres are collision-free regions computed by the neural SCDF. The optimized shortest path in the convex corridor is the blue curve.}
\vspace{-25pt}
\end{figure}
Motion planning is the problem of finding a collision-free trajectory that connects a given start and goal configuration. The planning takes place in the configuration space of the robot. For single body robots, like mobile robots or drones, the configuration space and the world space are usually the same. This simplifies the planning, since explicit obstacle representations are available which enables geometrical tools like separating hyperplanes, smallest distance to obstacles etc., to be used when designing motion planning algorithms. For multi-body robots like manipulators, the situation is completely different. The world space obstacles are usually mapped to non-convex regions, and to make the problem even harder, the mapping is usually not known. Forming explicit representations of the obstacle region in the configuration space is usually too expensive or intractable. Despite all of this, sampling based planners are used with great success, which mainly is due to their use of implicit representations of the obstacle region. The basic idea is to construct a graph in the configuration space that covers and connects the collision-free region. From this graph, a path can be extracted that connects a given start and goal configuration. The approach is computationally expensive, since the graph is constructed with the smallest geometrical building block available, points, which represents a collision-check. Furthermore, the extracted paths from the graph are non-smooth and jagged due to the stochastic nature of the approach. This adds an additional post-processing step to the process, where the paths are shortcutted and smoothened, before the path can be used for tracking. Clearly a lot of time is invested to form this graph and produce smooth paths. Thus, if the obstacles start to move, then all of this work is done in no use, since all points that make up this graph need to be re-verified, which is simply too time consuming to be done in real time.
\\\\
In this work, we want to address the existing drawbacks of the sampling based planners. Our main contribution is an improved motion planner where each vertex in the graph covers a collision-free region in the form of a sphere instead of a point and where the edges are formed with neighboring intersecting spheres. This representation has the advantage of instead of returning piecewise linear paths, returning a sequence of overlapping spheres, i.e. a convex corridor, that connects a given start and goal configuration, illustrated in Figure \ref{fig:pbrm_intro}. This convex corridor allows us to use convex optimization to produce smooth trajectories, instead of computationally expensive post-processing methods. The representation further allows us to estimate the coverage of the collision-free space, which gives us awareness and feedback in the offline roadmap construction phase. Finally, our representation is simple to adapt to moving obstacles, simply requery for the new radii and recheck for intersections. 
\\\\
The spherical collision-free regions are formed using a signed distance function (SDF), which is a function that returns the smallest distance from an arbitrary point to the boundary of an obstacle. As the name implies, the distance is signed, thus if the point is inside the obstacle it is negative otherwise positive. If the distance is positive, a sphere with radius equal to the distance is guaranteed to cover a collision-free region. Using an SDF in motion planning is not new, but what is novel about our approach is that we express the distance in the configuration space instead of the world space and by doing so allows us to form these convex collision-free regions. We refer to the resulting SDF as a signed configuration distance function (SCDF). Computing an SCDF analytically is non-trivial, our approach is therefore to parameterize the SCDF with a deep neural network and learn the mapping by supervised learning. Our resulting neural SCDF can compute distances for different parameter values of obstacle shapes and we also show how multiple distances can be combined, thus making our approach flexible.
\section{Related work}
Motion planning algorithms can roughly be divided into three families, grid-based, sampling based and optimization based methods. Grid-based methods (GBM) discretize the planning space from which a graph is then compiled. A standard search method is A$^\star$ \citep{a_star}, which is classified as an \textit{informed} search method, since it employs a heuristic function to speed up the search. A$^\star$ guarantees to return an optimal path at the level of discretization used. GBMs usually discretize the planning space by a regular lattice and this limits the GBMs to problems with low dimensionality due to the curse of dimensionality. Thus, GBMs are usually limited to single-body robots where the degrees of freedom (DOF) are low. To overcome the inherent scaling problem with the GBMs, stochastic methods are usually used for multi-body robots. These methods are termed as sampling-based methods (SBM) and core members within this family are the rapidly-exploring random trees (RRT) \citep{rrt} and the probabilistic roadmap (PRM) \citep{prm}. RRT grows a tree from the start configuration and explores the collision-free region in a rapid way until it is able to connect to the goal region. RRT is usually improved by bi-directional planning \citep{rrt_connect}, i.e. an additional tree is grown from the goal configuration and the trees are tested for connection after any tree has been expanded. RRT is a single-query method, thus it searches for a path from scratch each time it is queried. Contrary to this, PRM is a multi-query method, which solves for multiple queries without starting from scratch. PRM does this by creating a roadmap (graph) that covers the collision-free space as an offline step. The graph is then used to solve for multiple queries. PRMs are used in cases where the environment does not change since the extra offline step is too computationally costly and needs to be re-done if the environment is changed. In our work, we address this inherent issue by using a different roadmap representation. Our vertices in the graph cover a collision-free region in the form of spheres and we form the edges by checking for intersecting spheres. If something in the environment changes, we recompute the spheres radii and recheck the intersections, without relying on collision detection. We use a trained neural network to compute the sphere radius, therefore querying for the radius can be done fast, hence our representation enables the PRM for dynamic environments.
\\\\
In the recent decades, optimization based methods (OBM) \citep{chomp, schulman, itomp, stomp} have been introduced as an alternative to SBM for multi-body robots. Like the SBM, the OBMs scale well to higher dimensional problems and produce smoother motion. It is common to use a SDF in the optimization since it is a smooth function, thus enabling gradient-based methods. However, the standard way of expressing the SDF is in world space. The distance therefore needs to be mapped to the configuration space by the forward kinematics. This mapping makes the optimization problem a non-linear program (NLP), which is computationally expensive to solve. Recently, a different approach has been proposed. In \cite{mp_gcs} motion planning is formulated as a convex optimization problem by using the graph of convex sets framework \citep{gcs}. The underlying idea is to decompose the collision-free space into intersecting convex sets from which a convex optimization problem is formulated. In cases where an explicit representation of the obstacles in the configuration space exists, like for single-body robots, creating collision-free convex regions can be done fast \citep{iris}. For multi-body robots, this is non-trivial. Existing work does this successfully \citep{iris_nlp, iris_c} by an optimization based approach, but the methods are still too time consuming to be used in the presence of moving obstacles. Our approach is instead to use deep learning to learn an SDF expressed in the configuration space. With this, we can query for shortest distances to the collision boundary, which allows us to expand spherical regions which are collision-free. Our approach is fast and therefore enables our suggested roadmap planner to be used in dynamic environments.
\\\\
Recent research has focused on learning collision detection \citep{fk_kernel_distance, diffco, graphdistnet} by predicting the signed distance between the robot links and the surrounding obstacles in the world space. The learned SDF is used in trajectory optimization but since the distance is expressed in the world space, the problem becomes an NLP and therefore takes a long time to solve. We take a novel approach and suggest to instead express the signed distance in the configuration space. This allows us to improve the PRM at the same time as it enables convex optimization for trajectory optimization, which runs faster and is more reliable than NLP solvers. In \cite{cspf} a learned signed distance function in the configuration space is proposed similar to our approach. However, their approach is restricted to point cloud representations, while we propose to represent the obstacles as parameterized geometric shapes, e.g. spheres. Furthermore, we also show how to use our learned SCDF to improve an existing roadmap planner.
\section{Problem formulation}
A robot is located in the world space, $\W \subset \R^3 $. The unique location of the robot is given by its configuration $\q \in \C$, where $\C$ is the configuration space. The set of points covered by the robots bodies at a certain configuration is expressed as $\B(\q) \subset \W$. The robot is surrounded by $\NrObst$ obstacles $\O = \bigcup_{i=1}^{\NrObst} \O_i$, where  $\O_i \subset \W$. The representation of the obstacle in the configuration space is the set $\C\O_i = \{\q \in \C \: |\: \B(\q) \cap \O_i \neq \emptyset \}$. The obstacle space is formed as $\Co = \bigcup_{i=1}^{\NrObst} \C \O_i$. The complement is referred to as the free space, $\Cf = \C \setminus \Co$. The path planning problem is a tuple, ($\Cf$, $\qStart$, $\qGoal$), where we want to connect a query pair, consisting of a start, $\qStart$, and goal configuration, $\qGoal$, with a geometric path, $\q(s): [0, 1] \mapsto \Cf$, such that $\q(0)=\qStart$ and $\q(1)=\qGoal$, or report correctly when such a path does not exist.
\end{document}

\section{Related Work}
% \subsection{Vision Language Model}
% 시각장애인에서 상황을 설명할 DB가 없으니 만들었다. 그리고 이를 VLM에 튜닝했다.
\subsection{Technical approaches for assisting the visually-impaired}


\subsection{Datasets for visual instruction tuning}

% \begin{figure}
%     \centering
%     \includegraphics[width=0.5\linewidth]{Move_teaser.pdf}
%     \caption{Comparison of different dynamic compute approaches. length of arrow indicates residual transformation per token while width indicates velocity of transformation.}
%     \label{fig:enter-label}
% \end{figure}

\section{Method}
\label{sec:method}
Residual connections play a crucial role in shaping token representations, yet their dynamics remain underexplored in the context of efficient decoding. In this work, we delve deeper into transformer residual dynamics and investigate how modulating residual transformation velocity can improve inference efficiency in token-level processing, optimizing both dense and sparse MoE transformers.


\subsection{Residual Dynamics and Motivation for Multi-rate Residuals} \label{sec:motivation}

To analyze how hidden representations evolve across different layers of a transformer architecture, it's crucial to consider the effect of residual connections. Each transformer decoder layer typically has residual connections across attention and MLP submodules. As the residual stream $h_i$ traverses from interval $E_j$ to $E_{j+1}$, it undergoes a residual transformation given by:  
% \begin{equation}
% \label{eq:slow_residual_transformation}
% H_{E_{j+1}} = H_{E_j} \prod_{i=E_j}^{E_{j+1}} \left( I + \mathcal{A}_i \right) \left( I + \mathcal{M}_i \right) \quad \text{where} \quad \mathcal{A}_i = f(c_i, h_{i}), \mathcal{M}_i = g(h_i)
% \end{equation}

\begin{equation} \label{eq:slow_residual_transformation}
h_{E_{j+1}} = h_{E_j} + \sum_{i=E_j}^{E_{j+1}-1} \left( \mathcal{A}_i(h_i) + \mathcal{M}_i(h_i + \mathcal{A}_i(h_i)) \right) \quad \text{where} \quad \mathcal{A}_i = f(c_i, h_{i}), \mathcal{M}_i = g(h_i). 
\end{equation}

Here, \( \mathcal{A}_i \) denotes the non-linear transformation introduced by the multi-head attention mechanism at layer \( i \), while \( \mathcal{M}_i \) corresponds to the non-linear transformation of the MLP block at the same layer. These transformations depend on the input residual stream \( h_i \) and, in the case of \( \mathcal{A}_i \), the previous contextual representation \( c_i \).\footnote{Normalization layers are typically applied in practice but are omitted here for simplicity of the argument.}


% For easy tokens, the magnitude and direction of this delta transformation become progressively smaller with each successive layer as shown in \cref{fig:delta_transformation}. Consequently, it is feasible to predict these tokens after only a few residual connections, whereas harder tokens necessitate more extensive processing through additional layers.

\begin{figure}[ht]
    \centering
    \begin{subfigure}{0.48\textwidth}
        \centering
        \includegraphics[width=\textwidth]{sections/figures/residual_change.pdf}
        \caption{}
        \label{fig:residual_change}
    \end{subfigure}%
    \hfill
    \begin{subfigure}{0.48\textwidth}
        \centering
        \includegraphics[width=\textwidth]{sections/figures/alignment_wrt_dedicated_model.pdf}
        \caption{}
    \label{fig:alignment_wrt_dedicated_model}
    \end{subfigure}
    \caption{(a) As residual streams propagate through the model, the directional shifts in the residuals become progressively smaller. (b) A dedicated model with $k$ layers achieves a faster rate of change in residual streams and higher alignment than base model leveraging early exit mechanisms at layer $k$.}
    \label{fig}
\end{figure}


To examine whether residual transformations can be accelerated across layers, we conducted experiments using a diverse set of prompts on a pre-trained Phi3 model~\cite{phi3_report}. As illustrated in \cref{fig:residual_change}, we measured the directional shift in residual states as \( 1 - \mathcal{C}(h_{i-1}, h_i) \), where \(\mathcal{C}\) denotes normalized cosine similarity. This shift is notably higher in the initial layers, gradually decreasing in subsequent layers. This behavior allows traditional early exit approaches to effectively accelerate decoding by enabling earlier exits for simpler tokens. However, these approaches typically rely on a distance-based approximation, where the full residual transformation of the model is approximated by the residual transformations of the initial layers. To gain deeper insights into the distance versus velocity aspects of residual transformation, we conducted a comparative study. Specifically, we trained an early exit head at layer $k$ of the Phi3 model, which consists of 32 layers, restricting the distance traveled by each token. To accelerate the residual transformation relative to number of layers, we trained a smaller model consisting of only $k$ layers, while keeping all other hyperparameters consistent. We then compared the next-token prediction accuracy of the early exit head of the base model with that of the smaller model. To ensure an equal number of trainable parameters, we inserted low-rank adapters into the smaller model and trained only these adapters, whereas, in the distance-based approach, we trained solely the early exit head. In addition, to accelerate the residual transformation in smaller model, we distilled the residual streams from the larger model by incorporating a distillation loss ~\cite{sanh2019distilbert} between the residual state at layer \(i\) of the smaller model and the residual state at layer \(4 \times i\) of the larger model. As shown in ~\cref{fig:alignment_wrt_dedicated_model} the smaller model demonstrates a significantly faster rate of change in residual streams, leading to higher next token prediction accuracy after $k$ layers compared to the base model that employs traditional early exit mechanisms after $k$ layers \cite{schuster2022confident, chen2023eellm, varshney-etal-2024-investigating}. This experimental setup, which modifies only the rate of change in residual streams while keeping other factors constant, suggests that dense transformers, trained with a fixed number of layers, may inherently possess a slow residual transformation bias.

This observation raises an intriguing question: if the rate of change in residual streams could be accelerated relative to the number of layers, is it possible to facilitate earlier alignment for a greater proportion of tokens? Earlier alignment would be beneficial to not only facilitate dynamic computation but also for generating speculative tokens efficiently with high acceptance rates in speculative decoding setups ~\cite{leviathan2023fast, chen2023accelerating}. 

%thereby enhancing the efficiency of early exiting? 
 % This bias likely constrains the effectiveness of early exiting, particularly for easier tokens. By addressing this limitation through accelerated residual transformations, we hypothesize that it is possible to substantially improve the efficiency and accuracy of early exit strategies in transformer models.

\subsection{Multi-Rate Residual Transformation} \label{m2r2_method}

To address the slow residual transformation bias described in ~\cref{sec:motivation}, we introduce \textit{accelerated residual streams} that operate at rate $R$ relative to original slow residual stream. We pair slow residual stream, $h$ with an accelerated residual stream, $p$, which has an intrinsic bias towards earlier alignment. Relative to ~\cref{eq:slow_residual_transformation}, accelerated residual transformation from interval $E_j$ to $E_{j+1}$ can be represented as: 

% \begin{equation}
% \label{eq:fast_residual_transformation}
% P_{E_{j+1}} = P_{E_j} \prod_{i=E_j}^{E_{j+1}} \left( I + \hat{\mathcal{A}_i} \right) \left( I + \hat{\mathcal{M}_i} \right) \quad \text{where} \quad \hat{\mathcal{A}_i} = \hat{f}(c_i, P_{i}), \hat{\mathcal{M}_i} = \hat{g}(P_{i})
% \end{equation}


\begin{equation} \label{eq:fast_residual_transformation}
p_{E_{j+1}} = p_{E_j} + \sum_{i=E_j}^{E_{j+1}-1} \left( \hat{\mathcal{A}_i}(p_i) + \hat{\mathcal{M}_i}(p_i + \hat{\mathcal{A}_i}(p_i)) \right) \quad \text{where} \quad \hat{\mathcal{A}_i} = \hat{f}(c_i, p_{i}), \hat{\mathcal{M}_i} = \hat{g}(h_i), 
\end{equation}



where $\hat{\mathcal{A}_i}$ and $\hat{\mathcal{M}_i}$ denote non-linear transformation added by layer $i$ to previous accelerated residual $p_{i}$. Similar to $\mathcal{A}_i$, non-linear transformation $\hat{\mathcal{A}_i}$ attends to same context $c_i$ but uses a different transformation $\hat{f}$ for accelerating $p_{E_j}$ relative to $h_{E_j}$. 

We integrate accelerated residual transformation directly into the base network using parallel accelerator adapters such that rank of accelerator adapters $R_p << d$ where $d$ denotes base model hidden dimension. This setup allows the slow residual stream $h_{E_j}$ to pass through the base model layers while the accelerated residual stream $p_{E_j}$ utilizes these parallel adapters as shown in ~\cref{fig:m2r2_main}. Both slow and accelerated residuals are processed in same forward pass via attention masking and incur negligible additional inference latency in memory bound decoding setups, while in compute bound decoding setups where FLOPs optimization is essential, accelerated residual stream utilizes a fraction of attention heads that of slow residual (see ~\cref{sec:flops_optimization}). Additionally, to maximize the utility of accelerated residual transformations without introducing dedicated KV caches, we propose a shared caching mechanism between the slow and accelerated streams which minimally impact alignment benefits of our approach while offering substantial memory savings (see ~\cref{fig:koala_alignment}). Specifically, the attention operation on the slow residuals \( \text{MHA}(h_t, h_{\leq t}, h_{\leq t}) \) is redefined for accelerated residuals as 
\[
\hat{\mathcal{A}} = MHA(p_t, h_{<t} \oplus p_t, h_{<t} \oplus p_t),
\]
where the accelerated residual at time-step $t$, \( p_t \) attends to the slow residual’s KV cache, facilitating the reuse of contextual information across both residual streams without incurring additional caching costs. Here, \(MHA(q, k, v) \) represents multi-head attention between query \( q \), key \( k \), and value \( v \).

\begin{figure}
    \centering
    \includegraphics[width=0.8\linewidth]{sections//figures/m2r2_main2.pdf}
    \caption{Multi-rate Residuals Framework: Slow residual stream of base model is accompanied by a faster stream that operates at a $2-(J+1)\times$ rate relative to the slow stream, undergoing transformations via accelerator adapters as detailed in \cref{m2r2_method}, where J denotes number of early exit intervals. Colors within the slow and fast residual streams indicate similarity, with matching colors representing the most closely aligned residual states. At the beginning of the forward pass and at each exit point, the accelerated residual state is initialized from the corresponding slow residual state to avoid gradient conflict during training (see ~\cref{sec:grad_conflict}). Early exiting decisions are informed by the Accelerated Residual Latent Attention (ARLA) mechanism, described in \cref{method_arla}, which evaluates residual dynamics across consecutive exit gates.}
    \label{fig:m2r2_main}
\end{figure}

% Furthermore. to maximize the benefits of fast residual transformations without using dedicated KV caches, we propose sharing the fast network’s cache with the slow network. Formally speaking, We modify attention operation on slow residuals $MHA(H_t, H_{<=t}, H_{<=t})$ as $MHA(P_{t}, H_{<t} \oplus P_t, H_{<t}  \oplus P_t)$ such that accelerated residuals attend to previous slow context KV cache, where $MHA(q,k,v)$ denotes multi head attention between query, $q$, key $k$ and value $v$.


\subsection{Enhanced Early Residual Alignment}
Early residual alignment is instrumental in optimizing early exiting, speculative decoding, and Mixture-of-Experts (MoE) inference mechanisms. In this section, we provide a detailed analysis of how accelerated residuals enhance these inference setups.

% By aligning the residual states of intermediate layers with the final output representations, the model can maintain high prediction accuracy even when computations are truncated at earlier layers. This enables more reliable early exiting, reducing the overall computational cost while preserving performance. Additionally, in speculative decoding, early residual alignment allows the model to make confident predictions using faster, partial computations, thereby accelerating inference without sacrificing output quality.


\subsubsection{Early Exiting} \label{method_early_exiting}

A prevalent strategy for enabling early exiting at an intermediate layer $E_{j}$ involves approximating the residual transformation between $E_{j}$ and the final layer $N-1$ using a linear, context independent mapping, $\mathcal{T}$, such that $H_{N-1} \approx \mathcal{T}(H_{E_{j}})$. This approximation has been extensively employed in conventional approaches ~\cite{schuster2022confident, chen2023eellm, varshney-etal-2024-investigating}, providing a computationally efficient means to project the output of deeper layers from intermediate states. Specifically, residual state of layer $N-1$ with this approximation can be expressed as:


% \begin{equation}
% \label{eq: vanila_ea_assumption}
% \Phi(H_{E_{j}}) \sim H_{E_{j}} \prod_{i=E_{j}}^{N}\left( I + \mathcal{A}_i \right) \left( I + \mathcal{M}_i \right) \quad \text{where} \quad \Phi \perp C
% \end{equation}

\begin{equation} \label{eq:early_exiting}
h_{E_j} + \sum_{i=E_j}^{N-1} \left( \mathcal{A}_i(h_i) + \mathcal{M}_i(h_i + \mathcal{A}_i(h_i)) \right) \sim \mathcal{T}(h_{E_{j}})  \quad \text{where} \quad \mathcal{T} \perp c. 
\end{equation}


Here, $\mathcal{A}_i$ and $\mathcal{M}_i$ represent the residual contributions of the multi-head attention and MLP layers, respectively, while $\mathcal{T}$ remains independent of $c$, the preceding context.

This approach is inherently limited by two major factors: first, the assumption of linearity between $h_{E_{j}}$ and $h_{N-1}$ may not hold uniformly for all tokens, particularly when $E_j \ll N$. Second, the linear transformation $\mathcal{T}$ disregards the influence of the context $c$ and fails to account for the latent representations of previous contextual states. In contrast, M2R2 accelerated residual states mitigate both of these challenges by approximating the slow residual transformation of all layers via a faster residual transformation of fewer layers as:
% \begin{equation}
% H_{E_j} \prod_{i=E_j}^{N}\left( I + \mathcal{A}_i \right) \left( I + \mathcal{M}_i \right) \sim P_{E_j} \prod_{i=E_j}^{E_j+1}\left( I + \hat{\mathcal{A}_i} \right) \left( I + \hat{\mathcal{M}_i} \right)
% \end{equation}


\begin{equation} \label{eq:m2r2_approximating_ea}
h_{E_j} + \sum_{i=E_j}^{N-1} \left( \mathcal{A}_i(h_i) + \mathcal{M}_i(h_i + \mathcal{A}_i(h_i)) \right) \sim p_{E_j} + \sum_{i=E_j}^{E_{j+1}-1} \left( \hat{\mathcal{A}_i}(p_i) + \hat{\mathcal{M}_i}(p_i + \hat{\mathcal{A}_i}(p_i)) \right), 
\end{equation}

% \begin{equation} \label{eq:fast_residual_transformation}
% p_{E_{j+1}} = p_{E_j} + \sum_{i=E_j}^{E_{j+1}-1} \left( \hat{\mathcal{A}_i}(p_i) + \hat{\mathcal{M}_i}(p_i + \hat{\mathcal{A}_i}(p_i)) \right) \quad \text{where} \quad \hat{\mathcal{A}_i} = \hat{f}(c_i, p_{i}), \hat{\mathcal{M}_i} = \hat{g}(h_i) 
% \end{equation}






where $p_{E_j}$ is initialized from the slow residual state $h_{E_j}$ at each early exit interval $E_j$ using an identity transformation (see ~\cref{fig:m2r2_main}). As shown in ~\cref{fig:m2r2_residual_sim}, accelerated residuals offer a smoother, more consistent shift in residual direction across layers, in contrast to the abrupt changes typically seen at early exit points in standard early exit methods. Moreover, the normalized cosine similarity between accelerated states at early exit intervals and final residual states is substantially higher compared to traditional early exit techniques, highlighting improved alignment with final layer representations. Traditional adaptive compute methods are constrained by two principal factors: the number of tokens eligible for early exit at intermediate layers and the precision of early exit decision. If residual streams fail to saturate early, the majority of tokens remain ineligible for exit, thereby diminishing potential speedups. Additionally, imprecise delineations between tokens suitable for early exit can lead to underthinking (premature exits that adversely affect accuracy) or overthinking (unnecessary processing that compromises efficiency) ~\cite{zhou2020self, dai2020dynamic}. Enhanced early alignment using ~\cref{eq:m2r2_approximating_ea} helps to address  first issue. To address the second issue we introduce Accelerated Residual Latent Attention, which dynamically assesses the saturation of the residual stream, allowing for a more precise differentiation between tokens that can exit early and those requiring further processing.

% This results in uniform change in residual direction    
% % We keep $\mathcal{A} = \hat{\mathcal{A}}$, while $\hat{\mathcal{M}}$ is accelerated by a factor of $2 - (N_{E}+1)X$ relative to the slower residual transformation $\mathcal{M}$, where $N_E$ represents number of early exiting intervals.
% Figure~\cref{fig:rate_change_comparison} illustrates the comparative rate of change between these transformation streams.



% fig:rate_change_comparison
% - grid plot x axis -> layer id (0, 8) , y axis -> layer id -> dark color cell for max similarity , lighter for lower 
% 
-------------------------------------------------------
Let's consider residual stream $h_i$ traverses through interval $E_j$ to $E_{j+1}$ and undergoes residual transformation given by 
\begin{equation}
h_{E_{j+1}} = h_{E_j} \prod_{i=E_j}^{E_{j+1}} \left( 1 + \delta_i \right)    
\end{equation}

where $\delta_i$ denotes non-linear transformation added by layer $i$. Each non-linear transformation of layer $i$ is a function of previous contextual representation, $c_i$ and input residual stream $h_i-1$ as
$\delta_i = f(c_i, h_{i-1})$ 

One way to exit early at exit $E_j+1$ is to assume that residual transformation from $E_j+1$ to final layer $N-1$ can be approximated by a linear function $\phi$ as $h_{N-1} \sim \Phi(h_{E_j+1})$ and most conventional approaches such as \todo{cite EA papers} use this approach. In other words, 

\begin{equation}
\Phi(h_{E_j+1} \sim h_{E_j+1} \prod_{i=E_j+1}^{N} \left( 1 + \delta_i \right)   
\end{equation}

This approach suffers from two primary issues, linearity assumption from $h_E_j+1$ to $H_N-1$ if often incorrect, particularly when $E_j << N$. More importantly, linear transformation $\Phi$ doesn't consider effect of context $C_i$. M2R2  effectively addresses these issues as accelerated residual stream at interval $E_j+1$ can be represented as 

\begin{equation}
r_{E_{j+1}} = r_{E_j} \prod_{i=E_j}^{E_{j+1}} \left( 1 + \gamma_i \right)    
\end{equation}

where $\gamma_i$ denotes non-linear transformation added by layer $i$ to previous accelerated residual $r_i-1$. Similar to $\delta_i$, non-linear transformation $\gamma_i$ considers context $C_i$ as 
$\gamma_i = g(c_i, r_{i-1})$. So in summary, slow residual transformation is approximated by accelerated residual as: 

\begin{equation}
h_{E_j} \prod_{i=E_j}^{N} \left( 1 + \delta_i \right) \sim h_{E_j} \prod_{i=E_j}^{E_j+1} \left( 1 + \gamma_i \right)
\end{equation}

It's worth noting that accelerated residual $r_i$ and slow residual $h_i$ are processed concurrently at layer $i$ by constructing proper attention mask such as attention of slow residual is represented as 

$MHA(H_it, H_{i<=t}, H_{i<=t}$ while attention of fast residual is computed as 

$MHA(r_it, H_{i<=t}, H_{i<=t}$ where $MHA(q,k,v$ denotes multi head attention between query, $q$, key $k$ and value $v$.


------------------------------------------------------------------

Vertical latent attention on accelerated residual is computed as 
$MHA(S_mt, S(Ej<=i<=m)t, S(Ej<=i<=m)t)$ where $Smt$ denotes query/key/value projection in latent domain at layer $m$ at time $t$. 
------------------------------------------------------------------

Gradient conflict Avoidance: 

Let's consider $w_j$ is a trainable parameter that belongs to a layer between $E_j$ and $E_j+1$. Consider early exit loss at gate $E_j+1$, $L_j+1$, gradient propagation of $w_j$ at another trainable parameter $w_j-n$ can be gives as 

$\sum_{k=E_j-n}^{E_j} \beta_k \frac{\partial L_{E_k}}{\partial w_k}$

where $\beta_j$ denotes backward transformation coefficient for weight $w_j$ to reach gate $E_j$. 
 
On the other hand, gradient propagation in proposed approach can be represented as 

\[
\frac{\partial L_{E_j}}{\partial w_j} = 
\begin{cases} 
\beta_j \frac{\partial L_{E_j}}{\partial w_j} & \text{if } E_j \leq w_j \leq E_{j+1} \\
0 & \text{otherwise}
\end{cases}
\]







% \begin{figure}[ht]
%     \centering
%     \includegraphics[width=0.8\textwidth, height=5cm]{rate_change_comparison.png}
%     \caption{Rate of change comparison between fast and slow residual streams.}
%     \label{fig:rate_change_comparison}
% \end{figure}

%vary k and and plot EA accuracy for larger and smaller models. 

% \begin{figure}[ht]
%     \centering
%     \includegraphics[width=0.5\textwidth,height=5cm]{sections/figures/alignment_comparison_dialogsum.pdf}
%     \caption{Alignment of exited tokens for different early exit layers using traditional early exiting heads, dedicated faster networks, and faster residuals.}
%     \label{fig:small_model_early_exiting}
% \end{figure}


\textbf{Accelerated Residual Latent Attention} \label{method_arla}

In the context of residual streams, we observe that the decision to exit at a given layer can be more effectively informed by analyzing the dynamics of residual stream transformations, instead of solely relying on a classification head applied at the early exit interval $E_j$. To capture the subtle dynamics of residual acceleration, we propose a \textit{Accelerated Residual Latent Attention} (ARLA) mechanism. This approach involves making the exit decision at gate $E_j$ by attending to the residuals spanning from gate $E_{j-1}$ to $E_j$, rather than considering only the residual at gate $E_j$. To minimize the computational overhead associated with exit decision-making, the attention mechanism operates within the latent domain as depicted in ~\cref{fig:arla_arch}. Formally, for each interval $[E_j, E_{j+1}]$, the accelerated residuals are projected into Query ($Q^s_{E_j}, \ldots, Q^s_{E_{j+1}}$), Key ($K^s_{E_j}, \ldots, K^s_{E_{j+1}}$), and Value ($V^s_{E_j}, \ldots, V^s_{E_{j+1}}$) vectors, with latent dimension $d^s$ for $Q^s$, $K^s$, and $V^s$ being significantly smaller than hidden dimension of $p$.\footnote{We use $d^s = 64$ for experiments described in ~\cref{sec:experiments}.} Notably, when the router is allowed to make exit decisions at gate $E_j$ based on residual change dynamics, we observe that the attention is not confined to the residual state at $E_j$ but is distributed across residual states from $E_{j-1}$ to $E_j$, %as illustrated in Figure~\ref{fig:vertical_latent_attention_dynamics}. 
This broader focus on residual dynamics significantly reduces decision ambiguity in early exits, as demonstrated in Figure~\ref{fig:roc_arla}, which contrasts routers based on the last hidden state, and the proposed ARLA router.

%show R -> S transformation. 
%show parameter and flop overhead as compared to adapter on last hidden state.

% \begin{figure}[ht]
%     \centering
%     \includegraphics[width=0.5\textwidth,height=5cm]{sections/figures/roc_arla.pdf}
%     \caption{ROC curves of early exit decision strategies: confidence-based methods (CALM/LITE), routers based on the accelerated hidden state, and latent attention routers.}
%     \label{fig:decision_making_comparison}
% \end{figure}

% \begin{figure}[ht]
%     \centering
%     \includegraphics[width=0.5\textwidth,height=5cm]{vertical_latent_attention.png}
%     \caption{Vertical latent attention mechanism for optimizing early exit decisions by considering residuals from gate \(M\) through \(M-1\).}
%     \label{fig:vertical_latent_attention}
% \end{figure}

\begin{figure}[ht]
    \centering
    \begin{subfigure}{0.52\textwidth}
        \centering
        \includegraphics[width=\textwidth, height = 4cm]{sections/figures/arla_arch.pdf}
        \caption{Accelerated Residual Latent Attention (ARLA): Accelerated residuals between early exit gates are projected into latent domain and attention over residual states within the interval is computed to capture residual dynamics and exit decision is made based on residual saturation.}
        \label{fig:arla_arch}
    \end{subfigure}%
    \hfill
    \begin{subfigure}{0.45\textwidth}
        \centering
        \includegraphics[width=\textwidth, height = 4.5cm]{sections/figures/vla_roc.pdf}
        \caption{ROC classification curves of early exit decision strategies using a linear router used on last residual state ~\cite{schuster2022confident, varshney-etal-2024-investigating, chen2023eellm}  and using ARLA approach that considers residual dynamics. }
        \label{fig:roc_arla}
    \end{subfigure}
    \caption{Effectiveness of ARLA in capturing residual dynamics for early exiting decisions.}


\end{figure}



% \begin{figure}[ht]
%     \centering
%     \includegraphics[width=1\textwidth,height=5cm]{sections/figures/arla.pdf}
%     \caption{fig that plots 32 rows 2 cols heatmap showing attention at each gate}
%     \label{fig:vertical_latent_attention_dynamics}
% \end{figure}

\subsubsection{Self Speculative Decoding} \label{method_self_speculative_decoding}

An alternative means to exploit the early alignment properties of our approach is through the use of accelerated residual states for speculative token sampling to accelerate autoregressive decoding. Speculative decoding aims to speed up memory-bound transformer inference by employing a lightweight draft model to predict candidate tokens, while verifying speculated tokens in parallel and advancing token generation by more than one token per full model invocation \cite{leviathan2023fast, chen2023accelerating, xia2023speculative, miao2023specinfer}. Despite its effectiveness in accelerating large language models (LLMs), speculative decoding introduces substantial complexity in both deployment and training. A separate draft model must be specifically trained and aligned with the target model for each application, which increases the training load and operational complexity ~\cite{chen2023accelerating}. Additionally, this approach is resource-inefficient, as it requires both the draft and target models to be simultaneously maintained in memory during inference \cite{leviathan2023fast, chen2023accelerating}. 

One strategy to address this inefficiency is to leverage the initial layers of the target model itself to generate speculative candidates, as depicted in ~\cite{Tang2024}. While this method reduces the autoregressive overhead associated with speculation, it suffers from suboptimal acceptance rates. This occurs because the linear transformation employed for translating hidden states from layer $k$ to the final layer $N$ is typically a poor approximation, as discussed in ~\cref{sec:motivation} and ~\cref{method_early_exiting}. Our approach resolves this limitation by utilizing accelerated residuals, which demonstrate higher fidelity to their slower counterparts. By utilizing accelerated residuals operating at a rate of $N/k$, where $k$ denotes the number of layers used for candidate speculation, we are able to efficiently generate speculative tokens for decoding.\footnote{We typically set $k = 4$ to balance the trade-off between autoregressive drafting overhead and acceptance rate, as discussed in~\cref{sec:experiments}.}
 This technique not only obviates the need for multiple models during inference but also improves the overall efficiency and effectiveness of speculative decoding.

\begin{figure}
    \centering    \includegraphics[width=1\linewidth]{sections/figures/m2r2_aot_loading.pdf}
    \caption{Ahead-of-Time Expert Loading: M2R2 accelerated residual stream predicts experts required for future layers, reducing reliance on on-demand lazy loading. Speculative pre-loading is efficiently overlapped with computation of multi-head attention (MHA) and MLP transformations. Only incorrectly speculated experts are loaded lazily, resulting in faster inference steps and improved computational efficiency. Here, H indicates LBM Host while D indicates HBM Device.}
    \label{fig:moe_expert_aot_loading}
\end{figure}


\subsubsection{Ahead of Time Expert Loading:} \label{method_aot_expert_loading}

Recent advancements in sparse Mixture-of-Experts (MoE) architectures ~\cite{shazeer2017outrageously, fedus2022switch, artetxe2019massively, lepikhin2020gshard, zoph2022designing} have introduced a paradigm shift in token generation by dynamically activating only a subset of experts per input, achieving superior efficiency in comparison to dense models, particularly under memory-bound constraints of autoregressive decoding \cite{fedus2022switch, zoph2022designing}. This sparse activation approach enables MoE-based language models to generate tokens more swiftly, leveraging the efficiency of selective expert usage and avoiding the overhead of full dense layer invocation. In dense transformer models, pre-loading layers is a common strategy to enhance throughput, as computations of current layer can be overlapped with pre-loading of next layer parameters ~\cite{narayanan2021efficient, shoeybi2020megatron}. However, MoE models face a unique challenge: expert selection occurs dynamically based on previous layer’s output, making it infeasible to preload next layer’s experts in parallel. This limitation results in inherent latency, as expert loading becomes a sequential, on-demand process ~\cite{lepikhin2020gshard, fedus2022switch}.

To address this inefficiency, our method introduces a mechanism with \textit{accelerated residuals}, which not only captures key characteristics of base slower residual states but also exhibit high cosine similarity with their final counterparts (as illustrated in \cref{fig:m2r2_residual_sim}). By employing accelerated residual streams, we can effectively predict the necessary experts for future layers well in advance of their actual invocation. Specifically, using a $2\times$ accelerated residual, the experts needed for layers $2i+2$ and $2i+3$ can be identified while still computing in layer $i$, thus overcoming the bottleneck of sequential, on-demand expert selection and mitigating latency in the decoding pipeline, as shown in \cref{fig:moe_expert_aot_loading}. Note that, we use fixed set of accelerator adapters for transforming accelerated residuals (as discussed in ~\cref{m2r2_method}) while slow residual is transformed via expert routing mechanism. 

Furthermore, our approach integrates a Least Recently Used (LRU) caching strategy, which enhances memory efficiency by replacing the least recently used experts with speculated experts that are anticipated to be needed in upcoming layers. This hybrid approach of preemptive expert loading with LRU caching yields substantial improvements over traditional on-demand loading or standalone caching strategies. By minimizing cache misses and efficiently managing memory, this approach addresses both compute and memory bottlenecks, leading to faster, more resource-efficient token generation in MoE architectures. A comprehensive evaluation of this strategy, in relation to state-of-the-art methods, is provided in \cref{experiments_aot}, and the compute and memory traces on an A100 GPU are detailed in \cref{fig:moe_aot_cuda_trace}.



% Recent advancements in sparse Mixture-of-Experts (MoE) architectures have introduced the concept of utilizing distinct computational paths for different tokens \cite{shazeer2017outrageously}. This approach, wherein only a subset of experts are activated per input, enables MoE-based language models to generate tokens more swiftly compared to their dense counterparts due to memory-bound nature of auto-regressive decoding. In dense models, pre-loading layers in advance is a common strategy to enhance computational efficiency. However, this technique is not applicable to MoE models, where expert selection occurs dynamically based on the outputs of previous layers, preventing parallel pre-fetching of experts.

% Our proposed method addresses this inefficiency. Accelerated residuals, which are highly similar to their slower counterparts (see \cref{fig:similarity}), can reliably predict the necessary experts ahead of time. For instance, by utilizing $2X$ accelerated residual stream, we can predict the experts needed for the layer $2i+1$ and $2i+3$ while carrying out computation in layer $i$. This enables us to commence expert loading significantly earlier, as illustrated in \cref{expert_loading}, effectively mitigating the delays observed with the naive on-demand expert loading. Additionally, our method benefits from incorporating a Least Recently Used (LRU) strategy, where speculated experts replace those that are least recently utilized, resulting in improved performance compared to using either strategy alone. For a comprehensive evaluation, refer to \cref{moe_trace}, which provides a CUDA compute and memory trace of our approach executed on <>.



% A naive solution involves using the residual state of the previous layer along with the gating function of the next layer to predict which experts need to be loaded, and initiating the expert loading process in parallel with the attention computation of the next layer. Yet, as shown in \cref{fig:MOE_attn_vs_loading_time}, the attention computation for medium to long contexts is considerably faster than the expert loading time, making this approach inefficient.




\subsection{Training} \label{method_training}
% This approach is feasible due to the absence of gradient conflicts, as discussed in \cref{sec:grad_conflict}.

To accelerate residual streams, we employ parallel accelerator adapters as described in \cref{m2r2_method}.  For the early exiting use-case outlined in \cref{method_early_exiting}, we define the training objective for these adapters using the following loss function, which combines cross-entropy loss at each exit $E_j$ with distillation loss at each layer $i$. Loss weights coefficients $\alpha_0$ and $\alpha_1$ are employed to balance contribution of corresponding losses.

\begin{align} \label{eq:mr_loss}
L_{\text{m2r2}} = \underbrace{-\alpha_0 \sum_{j=1}^{J} \sum_{t=1}^{T} \log p_{\theta} \left( \hat{y}_t^{E_j} \mid y_{<t}, x \right)}_{\text{cross-entropy loss}} 
+ \underbrace{\alpha_1\sum_{i=1}^{E_{J-1}} \sum_{t=1}^{T} \| \mathbf{p}_{t}^{i} - \mathbf{h}_{t}^{((i - E_{j(i)}) \cdot R_i) + E_{j(i)})} \|^2}_{\text{distillation loss}}.
\end{align}

where $\hat{y}_t^{E_j}$ denotes the predictions from the accelerated residual stream at layer $E_j$ and time step $t$, $y_t$ represents the corresponding ground truth tokens, and $x$ indicates previous context tokens. The distillation loss at each layer $i$ is computed by comparing accelerated residuals at layer $i$ with slow residuals at layer $(i - E_{j(i)}) \cdot R_i + E_{j(i)}$, where $R_i$ denotes the rate of accelerated residuals at layer $i$ while $E_{j(i)}$ represents the most recent gate layer index such that $E_{j(i)} <= i$. \( J \) represents the total number of early exit gates, N denotes number of hidden layers and $E_j$ denotes layer index corresponding to gate index $j$ and \( T \) denotes the sequence length. 

In dynamic compute settings, after training of accelerator adapters, we optimize the query, key, and value parameters governing the ARLA routers (see ~\cref{method_arla}) across all exits in parallel on binary cross entropy loss between predicted decision and ground truth exiting decision. The ground truth labels for the router are determined based on whether the application of the final logit head on $\hat{y}_t^{E_j}$ yields the correct next-token prediction. 


% The objective for this optimization is defined by the following loss function:


%TODO are equations required ? 
% \begin{equation} \label{eq:arla_loss_combined}\small
%     L_{\text{arla}} = -\frac{1}{N} \sum_{t=1}^{T} \left( \sum_{j=1}^{E_n} \left[ O_t^{E_j} \log(\hat{O}_t^{E_j}) + (1 - O_t^{E_j}) \log(1 - \hat{O}_t^{E_j}) \right] \right), \quad \text{where} \quad 
%     O_t^{E_j} = \begin{cases} 
%     1, & \text{if } L(\hat{y}_t^{E_j}) = y_t^{E_j} \\
%     0, & \text{otherwise}
%     \end{cases}
% \end{equation}

% where $\hat{O}_t^{E_j}$ represents the binary predicted logits produced by the vertical latent attention router, as described in \cref{sec:arla}, at gate $E_j$ and time step $t$, and $O_t^{E_j}$ denotes the corresponding ground truth labels. The ground truth labels for the router are determined based on whether the application of the logit head on $\hat{y}_t^{E_j}$ yields the correct next-token prediction. The parameters controlling vertical latent attention are trained concurrently to ensure consistency and efficient use of computational resources.

For self-speculative decoding, as described in \cref{method_self_speculative_decoding}, the training objective remains the same as \cref{eq:mr_loss}, but with the number of intervals set to $J = 1$ and the rate of residual transformation set to $R_n = N/k$, where the first $k$ layers generate speculative candidate tokens. In the context of Ahead-of-Time Expert Loading for Mixture-of-Experts (MoE) models (see \cref{method_aot_expert_loading}), setting the rate of residual transformation to $R_n = 2$ typically offers a good trade-off between the accuracy of expert speculation and AoT pre-loading of experts. 

% Thus, we set $J = 1$ and $E_1 = 16$.


~\subsection{FLOPs Optimization} \label{sec:flops_optimization}

Naively implemented, M2R2 incurs higher FLOP overhead compared to traditional speculative decoding and early exiting approaches such as ~\cite{medusa, schuster2022confident, Tang2024}. However, modern accelerators demonstrate compute bandwidth that exceeds memory access bandwidth by an order of magnitude or more~\cite{databricksLLMInference2023, jouppi2021ten}, meaning increased FLOPs do not necessarily translate to increased decoding latency. Nevertheless, to ensure fair comparison and efficiency in compute bound scenarios, we introduce targeted optimizations.

~\textbf{Attention FLOPs Optimization} For medium-to-long context lengths, attention computation dominates FLOPs in the self-attention layer, surpassing the contribution from MLP layers. Specifically, matrix multiplications involving queries, cached keys, and cached values scale with $l_{kv} * l_{q}$ where $l_{kv}$ denotes previous context length and $l_q$ denotes current query length. Since M2R2 pairs accelerated residuals with slow residuals, a naive implementation results in twice the FLOPs consumption compared to a standard attention layer. To address this, we limit the attention of accelerated residual stream to selectively attend to the top-k most relevant tokens, identified by the slow residual stream based on top attention coefficients\footnote{We set to k = 64 and attend to top 64 tokens as identified by the slow residual stream.}. This is possible since slow and accelerated residual streams are processed in same forward pass and accelerated streams have access to attention coefficients of slow stream. Note that, the faster residual stream still retains the flexibility to assign distinct attention coefficients to these tokens. Furthermore, we design the faster residual stream to employ only 8 attention heads, compared to the 32 heads used in the slow residual stream of the Phi-3 model, reducing query, key, value, and output projection FLOPs by a factor of 1/4. ~\cref{fig:m2r2_num_heads_ablation} indicates effect of using a slicker stream on alignment. As depicted, using $\hat{n}_h = 8$ offers a good trade-off between alignment and FLOPs overhead. 

~\textbf{MLP FLOPs Optimization} The accelerator adapters operating on the accelerated residual stream are intentionally designed with lower rank than their counterparts in the base model. This reduces FLOP overhead by a factor proportional to $hiddenSize / rank$. Additionally, since the faster residual stream uses only 8 attention heads (compared to 32 in the slow residual stream of Phi-3), the subsequent MLP layers process a smaller set of activations, further reducing FLOPs by another factor of 1/4.

These optimizations significantly reduce the FLOP overhead per speculative draft generation, as illustrated in ~\cref{fig:flops_optmization}. Notably, while traditional early-exiting speculative approaches such as DEED require propagating the full slow residual state through the initial layers, incurring substantial computational costs, M2R2 achieves efficient token generation via slimmer, low-rank faster residual streams. In contrast, Medusa introduces considerable FLOP overhead due to per-head computations scaling with $d^2+dv$\footnote{Here $d$ denotes hidden state dimension while $v$ denotes vocab size.}, whereas M2R2 employs low-rank layers for both MLP and language modeling heads, maintaining computational efficiency. All experiments involving the M2R2 approach, as detailed in ~\cref{sec:experiments}, are conducted using these FLOPs optimizations.









% \[
% O_t^{E_j} = 
% \begin{cases} 
% 1, & \text{if } L(\hat{y}_t^{E_j}) = y_t^{E_j} \\
% 0, & \text{otherwise}
% \end{cases}
% \]




%add distillation
% We train accelerator adapters described in \cref{m2r2_method} to accelerate residual streams on next token prediction all in parallel since there are no gradient conflict issues as described in \cref{sec:grad_conflict}.

% \begin{align} \label{eq:mr_loss}
% L_{mr} =  & -\sum_{j = 1}^{E_n} (\sum_{t=1}^{T}\log p_{\theta} (\hat{y}_t^{E_j} | \hat{y}_{<t}, x)) \nonumber
% \end{align}

% where $\hat{y_t^{E_j}}$ denotes predicted logits obtained from accelerated residual stream at gate $E_j$ and time-step $t$ while $y_t^{E_j}$ denotes corresponding truth tokens. 

% Upon training of adapters responsible for accelerating residual streams, we train query, key, value parameters responsible for vertical latent attention of all gates in parallel as

% \begin{equation} \label{eq:arla_loss}
%     L_{arla} = -\frac{1}{N} (\sum_{t=1}^{T}(1\sum_{j=1}^{E_n} \left[ O_t^{E_j} \log(\hat{O}_t^{E_j}) + (1 - o_t^{E_j}) \log(1 - \hat{o_t}_{E_j}) \right]))
% \end{equation}

% where $\hat{O_t^{E_j}}$ denotes binary predicted logits obtained from vertical latent attention router described in \cref{sec:arla} at gate $E_j$ and timestep $t$ while $O_t^{E_j}$ denotes corresponding truth label. Truth labels for router are obtained by computing whether logit head application on $\hat{y}_t^j$ results in true next token prediction. Formally speaking, 

% $O_t^{E_j} = 1 if L(\hat{y_t^{E_j}}) == y_t^{E_j} , 0 otherwise$. 

% Parameters responsible for vertical latent attention are also trained in parallel as well. 

%todo: training slow and fast residuals together and distillation can be two training mdoes. 
%Distillation can be an ablation. 




% Although transformer decoding is memory bound on most mainstream accelerators, there could be scenarios where flop savings are crucial. For instance, on on-device settings power consumption is directly correlated with flops per decoding step and reducing flops does help with overall energy consumption. Vanilla early exiting methods help with flop reduction but suffer from mismatch between training and inference due to early exited tokens. If token at decoding step $t$, $T_t$ exited at layer $E_i$, while token $T_{t+k}$ exits at layer $E_j$ such that $E_i < E_j$, hidden state $H_{t+k}l$ does not have corresponding hidden state $H_tl$ to attend to where $E_i < l <= E_j$. One solution that's often used in literature is to rely on last hidden state available, $H_t{E_j}$, however it tends to be sub-optimal and does affect generation quality \cite{ref}.  To alleviate this mismatch while reducing flops, we train router such that attention mask between token $T_{t+k}$ and token $T_{<t+k}$ is given by: 

% \begin{equation}
%     a_{T_{{t+k}{T_{<t+k}}} = 1 if  E_{T_{<t+k}} >= E{T_{t+k}}
%     else 0
% \end{equation}

% This attention mask enables router to account for exited tokens and get trained accordingly. Since attention mechanism during decoding remains exactly same as that during training, impact on generation quality tends to be minimal as noted in \cref{fig:gen_auality_with_and_without_recompute_attention_show_flops}.  Although MoD does not suffer from training and inference mismatch, we observe that it suffers from discountinuity between pre-training and super-vised fine-tuning resulting in sub-optimal perplexity. On the other hand, our method doesn't not require pre-training , doesn't suffer from discountinuity, and achieves much better perplexity in super-vised fine-tuning and instruction tuning setups as shown in \cref{fig:Mod_vs_m2r2_loss_curves}.






% Our techniques are directly applicable in such scenarios.    




%expert loading with cuda streams in experiments
\input{content/offline_evaluation}
\section{User Study}

Having verified the basic ability of our model to produce polarizing debates in the previous section, we now present an exploratory investigation into how discussion polarization and algorithmic content curation in social media environments affect human perception of debates and their engagement behavior. Rather than testing specific hypotheses, this study aims to examine whether our framework can capture and reproduce fundamental mechanisms of online polarization identified in previous social science research. Through a controlled experiment using a simulated social media platform, we examine how different levels of discussion polarization (polarized vs. moderate) and recommendation bias (pro, balanced, contra) shape opinion formation and interaction patterns. This exploratory approach allows us to assess the framework's potential as a research tool for investigating human behavior in polarized online spaces, while providing initial insights into the complex interplay between platform design, user perception, and engagement patterns. The key findings of our user study are summarized in Table~\ref{tab:user-study-findings}.

\subsection{Experimental Design}

We employed a $2 \times 3$ between-subjects factorial design to investigate the dynamics of opinion formation and perception of polarization in online discussions. The experimental design manipulated two key dimensions: the \emph{Polarization Degree} in the artificial agent population and a systematic \emph{Recommendation Bias} while maintaining Universal Basic Income (UBI) (cf. Section \ref{sec:offline-evaluation}) as the consistent discussion topic.

The first experimental dimension contrasted highly polarized discussions with moderate ones through the manipulation of artificial agent behavior. In the polarized condition, artificial agents expressed extreme viewpoints and employed confrontational discourse patterns, characterized by emotional language, strong assertions, and minimal acknowledgment of opposing viewpoints. The moderate condition featured more nuanced discussions and cooperative interaction styles, with agents expressing uncertainty, acknowledging limitations in their knowledge, and engaging constructively with opposing views.

\begin{table*}[h!]
\centering
\small
\caption{Key Findings from User Study}
\label{tab:user-study-findings}
\begin{tabularx}{\textwidth}{>{\centering\arraybackslash}p{1.5cm}>{\raggedright\arraybackslash}X>{\raggedright\arraybackslash}X}
\toprule
\multicolumn{1}{c}{\textbf{Analysis}} & \multicolumn{1}{c}{\textbf{Key Finding}} & \multicolumn{1}{c}{\textbf{Theoretical Implication}} \\
\midrule
\multirow{3}{*}{\rotatebox[origin=c]{90}{\textbf{Perception}}} 
    & Users readily detect emotional content and group-based language in polarized discussions 
    & Social and emotional signals serve as primary markers for detecting polarization \\
\cmidrule{2-3}
    & The effect of perceived emotionality on perceived polarization is mediated through perceived group salience 
    & Group-based processing, not emotional content itself, is the primary pathway to polarization perception \\
\cmidrule{2-3}
    & Polarized discussions show marked decrease in expressed uncertainty 
    & Polarization creates perception of epistemic closure, limiting space for nuanced dialogue \\
\midrule
\multirow{3}{*}{\rotatebox[origin=c]{90}{\textbf{Engagement}}} 
    & Different interaction types serve distinct social functions, with minimal correlation between comments and reposts 
    & Platform affordances shape distinct modes of user participation \\
\cmidrule{2-3}
    & Users comment significantly more in unpolarized discussions compared to polarized environments
    & Moderate environments facilitate substantive textual engagement among non-polarized populations \\
\cmidrule{2-3}
    & Highest total engagement observed in polarized contra-bias condition, coinciding with strongest negative opinion shifts
    & Arguments against systemic change gain greater salience in polarized environments, activating status quo bias \\
\midrule
\multirow{2}{*}{\rotatebox[origin=c]{90}{\textbf{Opinion}}} 
    & While directional opinion change showed no significant effects, magnitude of opinion change revealed significant impacts
    & Polarized environments increase opinion volatility rather than pushing opinions in consistent directions \\
\cmidrule{2-3}
    & Experimental conditions directly affected opinion change magnitude, while perceptual variables showed no significant mediation
    & Suggests unconscious processing mechanisms responding to structural features without awareness \\
\bottomrule
\end{tabularx}
\end{table*}


\begin{figure}[htbp]
    \centering
    \includegraphics[width=\textwidth]{figures/prototype_screenshot.png}
    \caption{The screenshot depicts the simulated social media platform interface. The Newsfeed is displayed with a single post and one comment, including reaction handles for liking, reposting, and commenting. The interface emulates common social media design patterns, including a field for posting new messages, discovering new users, and inspecting the user profile.}
    \label{fig:prototype-screenshot}
\end{figure}


The second dimension introduced systematic bias in the recommendation system, implemented across three levels: neutral ($50\%$ pro-UBI, $50\%$ contra-UBI content), pro-bias ($70\%$ pro-UBI, $30\%$ contra-UBI content), and contra-bias ($30\%$ pro-UBI, $70\%$ contra-UBI content). This manipulation aimed to investigate how algorithmic content curation influences opinion formation and perception of debate polarization.

\subsection{System Prototype}

The prototype implementation consists of a web application that simulates a social media platform, reminiscent of X (formerly Twitter), to study social polarization dynamics. The interface, as depicted in Figure~\ref{fig:prototype-screenshot}, adheres to a familiar social media layout, facilitating user engagement and interaction.

\subsubsection{User Interface}

The application's main interface is divided into three primary sections: a navigation sidebar on the left, a central Newsfeed, and a recommendation panel on the right. The navigation sidebar provides quick access to essential functionalities such as the user's profile, a general user overview, and a logout option. The central Newsfeed serves as the primary interaction space, where users can view and engage with posts from other users. At the top of the Newsfeed, a text input area invites users to share their thoughts, mimicking the spontaneous nature of social media communication.

The Newsfeed displays a series of posts, each accompanied by user avatars, usernames, timestamps, and interaction metrics such as \emph{likes}, \emph{comments}, and \emph{reposts}. This design encourages user engagement and provides visual cues about the popularity and impact of each post. The recommendation panel on the right side of the interface suggests other users to follow, potentially influencing the user's network expansion and exposure to diverse viewpoints.

User profiles are dynamically generated, displaying the user's posts, follower relationships, and other relevant metadata like a user's handle and biography (see Section~\ref{subsec:agent-model} and Section~\ref{subsec:social-network-model} for details). It is also possible to follow and unfollow artificial users.

\subsubsection{Newsfeed Recommendations}

The web application implements an adaptive recommendation system for content presentation that evolves with user engagement. This system employs two distinct algorithmic approaches: a default variant for initial users and a collaborative variant that activates once users establish an interaction history.

The default variant implements a popularity-based scoring mechanism that considers multiple forms of engagement to determine content visibility. For a given message $m$, the system calculates a composite popularity score:

\begin{align}
    S_p(m) = l_m + 2c_m + 2r_m
\end{align}

where $l_m$, $c_m$, and $r_m$ represent the number of the message's \emph{likes}, \emph{comments}, and \emph{reposts} respectively. The weighted coefficients reflect the relative importance assigned to different forms of engagement, with more active forms of interaction carrying greater weight.

As users begin to interact with the platform, the system transitions to a collaborative variant that incorporates popularity metrics, ideological proximity, and a stochastic element to ensure recommendation diversity. The enhanced scoring function combines these elements into a composite score:

\begin{align}
    S_c(m) = \omega_p \cdot \frac{S_p(m)}{S_{max}} + \omega_i \cdot \frac{2 - |o_u - o_a|}{2} + \omega_r \cdot \epsilon
\end{align}

where $S_p(m)$ represents the popularity score normalized by the maximum observed score $S_{max}$, $o_u$ and $o_a$ denote the opinion scores of the active user (determined as the average of the opinion scores of the artificial users interacted with) and the (artificial) message author respectively, $\epsilon$ represents a uniform random variable in the interval $[0,1]$, and $\omega_p$, $\omega_i$, and $\omega_r$ are weighting parameters that sum to unity ($\omega_p + \omega_i + \omega_r = 1$). In the current implementation, these weights are set to $\omega_p = 0.6$, $\omega_i = 0.2$, and $\omega_r = 0.2$, balancing the influence of popularity, ideological similarity, and randomization.

Both variants maintain temporal relevance by presenting the user's most recent content contributions at the beginning of their feed when accessing the first page. This approach ensures users maintain awareness of their own contributions while experiencing the broader content landscape through the scoring-based recommendations.

This dual-variant approach enables the system to provide meaningful content recommendations even in the absence of user interaction data while transitioning smoothly to more personalized recommendations as users engage with the platform. The incorporation of popularity metrics, (mild) ideological factors, and controlled randomization creates diverse recommendations that is designed to give the impression of a dynamic network environment, while still falling under the conditional \emph{Recommendation Bias} regime. The stochastic element particularly aids in preventing recommendation stagnation and ensures dynamic content delivery.

\subsection{Procedure}

The experiment consisted of three phases: pre-interaction, interaction, and post-interaction.

\paragraph{Pre-interaction Phase} Participants first completed a comprehensive questionnaire assessing various baseline measures. These included demographic information, social media usage patterns, and initial attitudes towards UBI. 

\paragraph{Interaction Phase} Afterwards, participants were introduced to our simulated social media platform. They were instructed to engage with the platform naturally, as they would in their regular social media use. Unbeknownst to the participants, all other users on the platform were artificial agents programmed to behave according to the assigned experimental conditions. Participants were given the following instructions:

\begin{quote}
"You will now interact with a social media platform discussing \emph{Universal Basic Income}. Please use the platform as you normally would use social media. You can read posts, like them, comment on them, or create your own posts. Your goal is to form an opinion on the topic. You will have 10 minutes for this task."
\end{quote}

During this phase, participants' interactions, including likes, comments, reposts, and follows were recorded for later analysis.

\paragraph{Post-interaction Phase} After the interaction period, participants completed a post-test questionnaire. This included measures of their perception of the key constructs listed below. Additionally, participants evaluated the realism and effectiveness of the simulated platform.

\subsection{Measures}

Key constructs measured in this study included:

\begin{itemize}
    \item \emph{Attitude Change}: Measured shifts in participants' opinions about UBI between pre- and post-interaction phases using a seven-item scale (1 = Strongly Disagree to 5 = Strongly Agree), capturing changes in both the direction and magnitude of support.
    
    \item \emph{Perceived Polarization}: Assessed participants' perception of opinion extremity and ideological division within the observed discussions using a four-item scale (1 = Strongly Disagree to 5 = Strongly Agree), focusing on the perceived distance between opposing viewpoints.
    
    \item \emph{Perceived Group Salience}: Evaluated the extent to which participants perceived the discussion as being driven by group identities rather than individual perspectives, using a four-item scale (1 = Strongly Disagree to 5 = Strongly Agree).
    
    \item \emph{Perceived Emotionality}: Measured participants' assessment of the emotional intensity and affective tone in discussions using a four-item scale (1 = Strongly Agree to 5 = Strongly Disagree), capturing the perceived level of emotional versus rational discourse.
    
    \item \emph{Perceived Uncertainty}: Captured the degree to which participants observed expressions of doubt and acknowledgment of knowledge limitations in the discussions using a four-item scale (1 = Strongly Disagree to 5 = Strongly Agree).
    
    \item \emph{Perceived Bias}: Evaluated participants' assessment of viewpoint balance and fair representation of different perspectives in the discussion using a four-item scale (1 = Strongly Agree to 5 = Strongly Disagree).
\end{itemize}

\subsection{Participants}

We recruited $122$ participants through the Prolific platform. The sample exhibited a gender distribution favoring male participants ($63.6\%$) over female participants ($35.7\%$), with a single participant preferring not to disclose their gender. The age distribution revealed a predominantly young to middle-aged sample, with approximately $61.5\%$ of participants falling between $20$ and $39$ years old. The modal age group was $25$-$29$ years (18.6\%), followed by $35$-$39$ years ($15.0\%$).

Regarding educational background, nearly half of the participants ($49.3\%$) held university degrees, indicating a relatively high level of formal education in the sample. The remaining participants were distributed across various educational qualifications, with A-levels/IB, GCSE, and vocational certifications each representing approximately $11\%$ of the sample.

The majority of participants were professionally active, with $58.6\%$ being employees and $10.7\%$ self-employed. The sample also included a notable proportion of students ($15.7\%$ combined university and school students), reflecting diverse occupational backgrounds.

\begin{table}[htbp]
\centering
\footnotesize
\caption{Factor loadings and scale reliability for key measures. Note: [R] indicates reverse-coded items. Factor loadings are displayed for all items retained after cleaning (loading threshold $|.40|$).}
\begin{tabularx}{\textwidth}{>{\raggedright\arraybackslash}X>{\centering\arraybackslash}p{2cm}}
\toprule
\multicolumn{2}{l}{\textbf{Universal Basic Income (Pre)} ($\alpha = .893$)} \\
\midrule
A universal basic income would benefit society as a whole & .828 \\
Providing everyone with a basic income would do more harm than good [R] & .728 \\
Universal basic income is a fair way to ensure everyone's basic needs are met & .701 \\
Giving everyone a fixed monthly payment would reduce people's motivation to work [R] & .671 \\
Universal basic income would lead to a more stable and secure society & .867 \\
People should earn their income through work rather than receiving it unconditionally from the government [R] & .649 \\
A universal basic income would give people more freedom to make choices about their lives & .724 \\
\midrule
\multicolumn{2}{l}{\textbf{Universal Basic Income (Post)} ($\alpha = .890$)} \\
\midrule
A universal basic income would benefit society as a whole & .847 \\
Providing everyone with a basic income would do more harm than good [R] & .769 \\
Universal basic income is a fair way to ensure everyone's basic needs are met & .782 \\
Giving everyone a fixed monthly payment would reduce people's motivation to work [R] & .610 \\
Universal basic income would lead to a more stable and secure society & .921 \\
People should earn their income through work rather than receiving it unconditionally from the government [R] & .522 \\
A universal basic income would give people more freedom to make choices about their lives & .698 \\
\midrule
\multicolumn{2}{l}{\textbf{Perceived Polarization} ($\alpha = .776$)} \\
\midrule
The discussions on the platform were highly polarized & .674 \\
Users on the platform expressed extreme views & .755 \\
Users appeared to be firmly entrenched in their positions & .644 \\
There were frequent hostile interactions between users with differing views & .644 \\
\midrule
\multicolumn{2}{l}{\textbf{Perceived Emotionality} ($\alpha = .842$)} \\
\midrule
The discussions were highly charged with emotional content & .864 \\
Users frequently expressed strong feelings in their messages & .691 \\
The debate maintained a predominantly calm and neutral tone [R] & .692 \\
Participants typically communicated in an unemotional manner [R] & .779 \\
\midrule
\multicolumn{2}{l}{\textbf{Perceived Group Salience} ($\alpha = .639$)} \\
\midrule
Messages frequently emphasized "us versus them" distinctions & .585 \\
Users often referred to their group membership when making arguments & .806 \\
\midrule
\multicolumn{2}{l}{\textbf{Perceived Uncertainty} ($\alpha = .715$)} \\
\midrule
The agents frequently acknowledged limitations in their knowledge & .541 \\
Users often expressed doubt about their own positions & .772 \\
Messages typically contained absolute statements without room for doubt [R] & .566 \\
The agents seemed very certain about their claims and positions [R] & .671 \\
\midrule
\multicolumn{2}{l}{\textbf{Perceived Bias} ($\alpha = .830$)} \\
\midrule
The discussion seemed to favor one particular viewpoint & .782 \\
Certain perspectives received more attention than others in the debate & .738 \\
The platform provided a balanced representation of different viewpoints [R] & .821 \\
Different perspectives were given equal consideration in the discussion [R] & .646 \\
\bottomrule
\end{tabularx}
\label{tab:factor-loadings}
\end{table}

Participants demonstrated high engagement with social media platforms, with $80\%$ reporting daily or near-constant usage. The majority ($75\%$) spent between one and four hours daily on social media platforms. YouTube ($25.0\%$) and Facebook ($24.3\%$) emerged as the most frequently used platforms, followed by Instagram ($17.9\%$) and X, formerly Twitter ($10.7\%$). This usage pattern suggests participants were well-acquainted with social media interfaces and interaction patterns, making them suitable subjects for the study's simulated social media environment.

\subsection{Preliminary Analysis}

We evaluated the ecological validity of our experimental platform. Participants rated various aspects on $7$-point scales ($1$ = Not at all, $7$ = Extremely), with higher scores indicating more positive evaluations. The platform received favorable ratings across multiple dimensions, consistently scoring above the scale midpoint of $4$. Particularly noteworthy was the interface usability ($M = 5.52$, $SD = 1.15$), which participants rated as highly satisfactory. The platform's similarity to real social media platforms ($M = 4.69$, $SD = 1.65$) and its ability to facilitate meaningful discussions ($M = 4.59$, $SD = 1.41$) were also rated positively. The overall platform realism received satisfactory ratings ($M = 4.47$, $SD = 1.61$), suggesting that participants found the experimental environment sufficiently realistic and engaging for the purposes of this study. The attitudes of participants toward Universal Basic Income exhibited a slight decline from the pre-interaction phase ($M = 3.12$, $SD = 0.90$) to the post-interaction phase ($M = 2.99$, $SD = 0.99$). However, these attitudes remained relatively close to the scale midpoint, indicating that participants held moderate views on the subject matter under discussion.

\begin{table}[ht]
\footnotesize
\centering
\caption{Effects of Polarization and Recommendation Bias on Key Dependent Variables}
\label{tab:perception-anova}
\begin{tabularx}{\textwidth}{>{\raggedright\arraybackslash}p{2.5cm}>{\raggedright\arraybackslash}p{1.5cm}*{3}{>{\centering\arraybackslash}X}>{\centering\arraybackslash}p{1.2cm}>{\centering\arraybackslash}p{1.2cm}>{\centering\arraybackslash}p{1.2cm}}
\toprule
\multirow{2}{*}{\textbf{Metric}} & \multirow{2}{*}{\textbf{Condition}} & \multicolumn{3}{c}{\textbf{Recommendation Bias}} & \multicolumn{3}{c}{\textbf{ANOVA $\boldsymbol{F}$($\boldsymbol{p}$)}} \\
\cmidrule(lr){3-5} \cmidrule(lr){6-8}
& & Contra & Balanced & Pro & Pol & Rec & Pol×Rec \\
\midrule
\multirow{2}{*}{Opinion Change} 
   & Polarized & \textbf{-0.41} (0.60) & -0.08 (0.55) & -0.10 (0.33) & 3.48 & 1.74 & 1.90 \\
   & Unpolarized & -0.06 (0.38) & -0.09 (0.43) & -0.02 (0.26) & (.065) & (.180) & (.154) \\
\midrule
\multirow{2}{*}{|Opinion Change|} 
   & Polarized & \textbf{0.56} (0.45) & 0.35 (0.42) & 0.22 (0.26) & \textbf{5.21*} & \textbf{3.61*} & 2.54 \\
   & Unpolarized & 0.24 (0.30) & 0.32 (0.28) & 0.18 (0.18) & (.024) & (.030) & (.083) \\
\midrule
\multirow{2}{*}{Polarization} 
   & Polarized & \textbf{4.47} (0.82) & \textbf{4.61} (0.97) & \textbf{4.29} (0.75) & \textbf{56.48***} & 0.02 & 1.50 \\
   & Unpolarized & 3.30 (0.69) & 3.23 (0.83) & 3.54 (0.84) & (.000) & (.979) & (.228) \\
\midrule
\multirow{2}{*}{Emotionality} 
   & Polarized & \textbf{3.80} (0.48) & \textbf{3.59} (0.99) & \textbf{3.36} (0.80) & \textbf{73.54***} & 1.42 & 2.44 \\
   & Unpolarized & 2.50 (0.74) & 2.17 (0.60) & 2.65 (0.83) & (.000) & (.245) & (.092) \\
\midrule
\multirow{2}{*}{Group Salience} 
   & Polarized & \textbf{2.94} (0.60) & \textbf{2.86} (0.51) & \textbf{3.21} (0.42) & \textbf{17.18***} & \textbf{5.63**} & 0.07 \\
   & Unpolarized & 2.51 (0.59) & 2.42 (0.65) & 2.86 (0.40) & (.000) & (.005) & (.934) \\
\midrule
\multirow{2}{*}{Bias} 
   & Polarized & \textbf{3.50} (0.87) & 3.20 (0.76) & \textbf{3.83} (0.59) & \textbf{15.08***} & 0.44 & \textbf{3.73*} \\
   & Unpolarized & 3.05 (0.82) & 3.02 (0.87) & 2.68 (0.87) & (.000) & (.648) & (.027) \\
\midrule
\multirow{2}{*}{Uncertainty} 
   & Polarized & 2.29 (0.60) & 2.03 (0.60) & 1.90 (0.75) & \textbf{48.86***} & 0.97 & 1.06 \\
   & Unpolarized & \textbf{2.92} (0.71) & \textbf{2.83} (0.69) & \textbf{2.96} (0.48) & (.000) & (.383) & (.350) \\
\bottomrule
\multicolumn{8}{p{.95\textwidth}}{\small \textbf{Note:} Values show means with standard deviations in parentheses. Bold values indicate significantly higher means between polarized and unpolarized conditions. F-statistics in bold are significant. Pol = Polarization main effect, Rec = Recommendation main effect, Pol×Rec = Interaction effect. *$p$ < .05, **$p$ < .01, ***$p$ < .001} \\
\end{tabularx}
\end{table}


Furthermore, we examined the psychometric properties of our key measures (see Table~\ref{tab:factor-loadings}). Principal component analyses were conducted for each scale, with items loading on their intended factors. Most scales showed good reliability ($\alpha$ ranging from $.715$ to $.893$) and satisfactory factor loadings ($|.40|$ or greater). The \emph{Perceived Group Salience} scale required modification from its original four-item structure. Two items (\emph{"The debate focused on ideas rather than group affiliations"} and \emph{"Individual perspectives were more prominent than group identities in the discussions"}) were dropped due to poor factor loadings ($.049$ and $.051$ respectively). The remaining two items showed modest to acceptable loadings ($.585$ and $.806$), though below optimal thresholds. Given the theoretical importance of group salience in our research design, we retained this measure for further analyses while acknowledging its psychometric limitations.


\subsection{Analysis of Debate Perception}
\label{subsec:debate-perception}

Our first analysis examines how users perceive and process discussions under varying conditions of \emph{Polarization Degree} and \emph{Recommendation Bias}. We specifically investigate whether participants recognize polarized discourse patterns, how they process emotional and group-based content, and how \emph{Recommendation Bias} might moderate these perceptions. Through this analysis, we aim to understand the psychological mechanisms through which discussion climate and content curation shape users' experience of online debates.

\subsubsection{Variance Analysis}

\begin{figure}[h!]
    \centering
    \includegraphics[width=\textwidth]{figures/interaction_plots_perception.png}
    \caption{Interaction plots showing the effects of polarization and recommendation type on key dependent variables. Red lines represent the polarized condition, green lines represent the unpolarized condition. Error bars represent $95\%$ confidence intervals.}
    \label{fig:perception-interaction-plots}
\end{figure}



The effects of discussion \emph{Polarization Degree} and \emph{Recommendation Bias} were analyzed using a series of $2 \times 3$ analyses of variance. As shown in Table~\ref{tab:perception-anova}, the results revealed consistent main effects of \emph{Polarization Degree} across multiple dependent variables, while \emph{Recommendation Bias} showed more limited impact. Figure~\ref{fig:perception-interaction-plots} visualizes the interaction effects for our key constructs.


Regarding \emph{Opinion Change}, although the effects did not reach statistical significance, the data suggested some noteworthy patterns. Neither \emph{Polarization Degree} ($F(1, 117) = 3.48$, $p = .065$, $\eta_p^2 = 0.029$) nor \emph{Recommendation Bias} ($F(2, 117) = 1.74$, $p = .180$, $\eta_p^2 = 0.029$) significantly influenced opinion change. The descriptive statistics revealed a tendency toward stronger opinion changes in the polarized condition with contra-\emph{Recommendation Bias} ($M = -0.408$, $SD = 0.598$) compared to other conditions, where changes were minimal (means ranging from $-0.087$ to $-0.015$).

Given that directional opinion change might mask important dynamics by allowing positive and negative changes to cancel each other out, we conducted an additional analysis focusing on the magnitude of opinion change (i.e., absolute values). This analysis revealed significant main effects for both \emph{Polarization Degree} ($F(1, 117) = 5.21$, $p = .024$, $\eta_p^2 = 0.043$) and \emph{Recommendation Bias} ($F(2, 117) = 3.61$, $p = .030$, $\eta_p^2 = 0.058$). Participants in the polarized condition showed stronger opinion changes ($M = 0.378$, $SD = 0.378$) compared to the unpolarized condition ($M = 0.248$, $SD = 0.254$, Hedges' $g = 0.390$). Post-hoc analyses for \emph{Recommendation Bias} revealed that participants in the contra-bias condition showed significantly stronger opinion changes compared to the pro-bias condition (Hedges' $g = 0.566$, $p = .009$). The neutral condition showed significantly stronger opinion changes compared to the pro-bias condition (Hedges' $g = 0.456$, $p = .039$), but did not significantly differ from the contra-bias condition (Hedges' $g = 0.147$, $p = .493$)

The \emph{Polarization Degree} manipulation strongly influenced participants' perception of the discussion climate. \emph{Perceived Polarization} showed a substantial main effect ($F(1, 116) = 56.48$, $p < .001$, $\eta_p^2 = 0.327$), with participants in the polarized condition reporting significantly higher levels ($M = 4.46$, $SD = 0.85$) compared to the unpolarized condition ($M = 3.35$, $SD = 0.79$). Post-hoc tests confirmed this significant difference ($t(116.44) = 7.52$, $p < .001$, Hedges' $g = 1.36$).

\emph{Perceived Emotionality} emerged as the strongest effect in the study ($F(1, 116) = 73.54$, $p < .001$, $\eta_p^2 = 0.388$). Participants in the polarized condition perceived substantially higher emotional content ($M = 3.58$, $SD = 0.76$) than those in the unpolarized condition ($M = 2.44$, $SD = 0.72$). Post-hoc analysis revealed a large effect size (Hedges' $g = 1.53$, $t(116.99) = 8.44$, $p < .001$). While \emph{Recommendation Bias} did not show a significant main effect ($F(2, 116) = 1.42$, $p = .245$, $\eta_p^2 = 0.024$), there was a marginal interaction effect ($F(2, 116) = 2.44$, $p = .092$, $\eta_p^2 = 0.040$).

The analysis of \emph{Perceived Uncertainty} revealed another strong main effect of \emph{Polarization Degree} ($F(1, 116) = 48.86$, $p < .001$, $\eta_p^2 = 0.296$). Participants exposed to polarized discussions reported significantly lower levels of expressed uncertainty ($M = 2.07$, $SD = 0.65$) compared to those in the unpolarized condition ($M = 2.90$, $SD = 0.63$). Post-hoc tests indicated a strong effect (Hedges' $g = -1.26$, $t(117.89) = -6.97$, $p < .001$).

\emph{Perceived Group Salience} showed significant main effects for both \emph{Polarization Degree} ($F(1, 116) = 17.18$, $p < .001$, $\eta_p^2 = 0.129$) and \emph{Recommendation Bias} ($F(2, 116) = 5.63$, $p = .005$, $\eta_p^2 = 0.089$). The polarized condition elicited higher reports of group-based discourse ($M = 3.00$, $SD = 0.51$) compared to the unpolarized condition ($M = 2.59$, $SD = 0.55$). Post-hoc analysis revealed a medium to large effect size (Hedges' $g = 0.73$, $t(120.00) = 4.10$, $p < .001$). Post-hoc tests for \emph{Recommendation Bias} showed significant differences between pro-bias and both contra-bias (Hedges' $g = -0.56$, $t(76.97) = -2.61$, $p = .011$) and neutral conditions (Hedges' $g = -0.74$, $t(72.39) = -3.34$, $p = .001$).

\emph{Perceived Bias} demonstrated both a significant main effect of \emph{Polarization Degree} ($F(1, 116) = 15.08$, $p < .001$, $\eta_p^2 = 0.115$) and a significant interaction effect with \emph{Recommendation Bias} ($F(2, 116) = 3.73$, $p = .027$, $\eta_p^2 = 0.060$). In the polarized condition, participants reported higher \emph{Perveived Bias} ($M = 3.51$, $SD = 0.74$) compared to the unpolarized condition ($M = 2.92$, $SD = 0.85$). Post-hoc tests confirmed this difference with a medium to large effect size (Hedges' $g = 0.69$, $t(119.97) = 3.85$, $p < .001$). The significant interaction suggests that the effect of \emph{Polarization Degree} on bias perception varied across \emph{Recommendation Bias} conditions, with particularly pronounced differences in the pro-bias condition (polarized: $M = 3.83$, $SD = 0.59$; unpolarized: $M = 2.68$, $SD = 0.87$).

To examine whether these effects might vary based on participants' prior attitudes toward Universal Basic Income, we conducted additional analyses grouping participants according to their initial position (strongly pro/contra, moderately pro/contra). However, these analyses did not yield any significant results, suggesting that the observed effects of \emph{Polarization Degree} and \emph{Recommendation Bias} on opinion change magnitude operate similarly across different initial attitude positions.

\subsubsection{Path Analysis}

To understand the mechanisms through which our experimental conditions influence both perceptual dimensions and opinion change, we employed structural equation modeling (SEM). This approach allows us to simultaneously estimate multiple interdependent relationships while accounting for measurement error and covariation between constructs.

\begin{figure}[htbp]
    \centering
    \includegraphics[width=\textwidth]{figures/sem.png}
    \caption{Structural equation model showing the effects of political information polarization and recommendation system exposure on perceived polarization and opinion change. Path coefficients represent standardized regression weights. Solid lines indicate direct effects, dashed lines represent covariances. $^*p < .05$, $^{**}p < .01$, $^{***}p < .001$.}
    \label{fig:path-model}
\end{figure}


We developed a theoretical model examining how \emph{Polarization Degree} and \emph{Recommendation Bias} affect \emph{Opinion Change Magnitude} and \emph{Perceived Polarization} both directly and through various perceptual pathways (see Figure~\ref{fig:path-model}). The model was estimated using maximum likelihood estimation with standardized variables. The results demonstrated excellent fit to the data ($\chi^2(6) = 2.335$, $p = .886$, CFI $= 1.000$, TLI $= 1.109$, RMSEA $= .000$ [90\% CI: .000, .068], SRMR $= .023$), explaining substantial variance in key outcome variables (e.g., \emph{Perceived Emotionality}: 53.3\%, \emph{Perceived Polarization}: 38.0\%, \emph{Perceived Uncertainty}: 30.5\%).

The SEM analysis revealed several significant pathways. First, both experimental conditions showed significant direct effects on \emph{Opinion Change Magnitude}. Higher \emph{Polarization Degree} led to increased \emph{Opinion Change Magnitude} ($\beta = .275$, $p = .007$), while pro-UBI \emph{Recommendation Bias} decreased \emph{Opinion Change Magnitude} ($\beta = -.280$, $p = .006$). Notably, these effects emerged despite none of the perceptual variables showing significant direct effects on \emph{Opinion Change Magnitude}.

Regarding the perceptual pathways, \emph{Polarization Degree} showed strong direct effects on multiple dimensions: \emph{Perceived Emotionality} ($\beta = .472$, $p < .001$), \emph{Perceived Uncertainty} ($\beta = -.541$, $p < .001$), \emph{Perceived Bias} ($\beta = .433$, $p < .001$), and \emph{Perceived Group Salience} ($\beta = .251$, $p = .019$). The analysis further revealed how \emph{Polarization Degree} shapes \emph{Perceived Polarization} through multiple pathways. Beyond its direct effect ($\beta = .247$, $p = .022$), \emph{Polarization Degree} also operated through two indirect paths: via \emph{Perceived Group Salience} ($IE = .105$, $p = .044$) and via \emph{Perceived Emotionality} ($IE = .050$, $p = .406$). The combination of these direct and indirect effects resulted in a substantial total effect of \emph{Polarization Degree} on \emph{Perceived Polarization} ($\beta = .402$, $p < .001$).

The model also captured interesting relationships between mediating variables. \emph{Perceived Group Salience} showed significant positive effects on both \emph{Perceived Emotionality} ($\beta = .447$, $p < .001$) and \emph{Perceived Polarization} ($\beta = .417$, $p < .001$). Additionally, we found a significant negative covariance between \emph{Perceived Uncertainty} and \emph{Perceived Bias} ($\beta = -.291$, $p = .012$), suggesting these perceptions tend to operate in opposition to each other.

These findings reveal a complex pattern where experimental conditions shape both behavioral outcomes (\emph{Opinion Change Magnitude}) and perceptual experiences, though these paths appear to operate independently rather than sequentially. The substantial total effect of \emph{Polarization Degree} on \emph{Perceived Polarization}, decomposed into direct and indirect pathways, highlights how environmental features can influence user perceptions through multiple complementary mechanisms.

\subsubsection{Discussion}

Our findings reveal significant insights into how users perceive, process, and respond to polarized discussions in social media environments. Through complementary analytical approaches, we demonstrate that the manipulation of \emph{Polarization Degree} had profound effects across multiple perceptual dimensions and behavioral outcomes, while \emph{Recommendation Bias} played a more nuanced role in shaping user experiences.

A key finding emerged in our analysis of opinion change. While directional opinion change showed no significant effects, the magnitude of opinion change revealed significant impacts of both experimental conditions. This discrepancy suggests that focusing solely on directional change may mask important dynamics by allowing opposing changes to cancel each other out. The significant effects on magnitude indicate that polarized environments and recommendation patterns do influence opinion formation, but not in a uniformly directional way. This aligns with theoretical perspectives suggesting that polarized environments might increase opinion volatility without necessarily pushing opinions in a consistent direction.

Particularly intriguing is our finding that while experimental conditions significantly predicted \emph{Opinion Change Magnitude}, none of the measured perceptual variables showed significant effects. This creates an interesting puzzle: why would objective environmental conditions affect opinion change while subjective perceptions of these conditions do not? Several theoretical explanations merit consideration. First, this could reflect unconscious processing mechanisms, where participants respond to structural features of the environment without consciously processing them, aligning with dual-process theories of attitude change \citep{petty_elaboration_1986, chaiken_the_2014}. Second, our post-hoc perception measures might not capture the dynamic nature of how perceptions evolved during the interaction period. Third, unmeasured mediating variables like information processing depth or emotional arousal might better explain the mechanism of influence. This pattern aligns with Kurt Lewin's concept of "environmental press," suggesting that behavior might be influenced by objective environmental properties independent of their subjective interpretation \citep{lewin_principles_2013}.

The strong effect of \emph{Polarization Degree} on \emph{Perceived Emotionality} ($\eta_p^2 = 0.388$) suggests that participants were highly attuned to the emotional tenor of discussions. However, our structural equation model revealed an unexpected pattern: the direct effect of \emph{Perceived Emotionality} on \emph{Perceived Polarization} became non-significant when controlling for \emph{Perceived Group Salience}. This finding has profound theoretical implications. It suggests that the relationship between emotional content and polarization perception is primarily mediated through group-based processing, aligning with theories of affective polarization and social identity. Drawing on Carl Schmitt's friend-enemy distinction \citep{schmitt_concept_2008} and its extensions by Laclau and Mouffe \citep{laclau_hegemony_2014}, this might indicate that emotional content primarily influences polarization perception by activating group-based antagonisms rather than through direct affective pathways. The strong mediating role of \emph{Perceived Group Salience} ($\beta = .417$) supports this interpretation, suggesting that emotional content primarily serves to highlight group boundaries and distinctions.

The absence of significant effects from \emph{Perceived Uncertainty} and \emph{Perceived Bias} on polarization perception, despite strong condition effects on these variables, suggests that the path to \emph{Perceived Polarization} is more specific than previously theorized. While polarized environments clearly influence these perceptual dimensions (uncertainty: $\beta = -.541$; bias: $\beta = .433$), their lack of predictive power for polarization perception suggests they might operate through different psychological mechanisms or serve different functions in processing polarized discourse.

Nevertheless, the inverse relationship between \emph{Polarization Degree} and \emph{Perceived Uncertainty} ($\eta_p^2 = 0.296$) aligns with theoretical frameworks suggesting that polarized discourse often manifests through increased assertiveness and reduced acknowledgment of epistemic limitations. This pattern may help explain the self-reinforcing nature of polarized discussions: as uncertainty expressions diminish, the space for nuanced dialogue potentially contracts.

These results carry important implications for platform design and intervention strategies. The clear user sensitivity to emotional content and its indirect effect through group salience suggests that interventions might need to target both emotional expression and group dynamics simultaneously. Moreover, the finding that conscious perceptions do not mediate opinion change suggests that traditional perception-focused interventions might have limited effectiveness. Instead, interventions might need to address structural features of the environment that influence behavior more directly.

Our findings particularly highlight the central role of group processes in polarization dynamics. The strong mediating role of \emph{Perceived Group Salience} in translating emotional content into polarization perceptions suggests that effective interventions might need to focus on reducing the salience of group boundaries rather than just moderating emotional content. This insight, supported by both our variance and path analyses, suggests a more nuanced approach to platform design that considers how features might inadvertently enhance group distinctions even when attempting to reduce emotional intensity.

\subsection{Analysis of User Engagement}

Our second analysis examines how polarization and recommendation bias shape user engagement behaviors on the platform. We investigate both the quantity and quality of interactions, analyzing how different experimental conditions affect users' preferences for specific types of engagement (\emph{likes}, \emph{comments}, \emph{reposts}, and \emph{follows}). This analysis aims to understand whether polarized environments and algorithmic bias influence not just how much users engage, but also how they choose to participate in discussions.

\subsubsection{Descriptive and Variance Analysis}

\begin{table}[ht]
\centering
\small
\caption{Main Effects of Experimental Conditions on User Engagement}
\label{tab:interactions-anova}
\begin{tabularx}{\textwidth}{>{\raggedright\arraybackslash}p{2cm}>{\raggedright\arraybackslash}p{1.8cm}*{3}{>{\centering\arraybackslash}X}>{\centering\arraybackslash}p{1.2cm}>{\centering\arraybackslash}p{1cm}}
\toprule
\multirow{2}{*}{\textbf{Metric}} & \multirow{2}{*}{\textbf{Condition}} & \multicolumn{3}{c}{\textbf{Recommendation Bias}} & \multicolumn{2}{c}{\textbf{ANOVA}} \\
\cmidrule(lr){3-5} \cmidrule(lr){6-7}
& & Contra & Balanced & Pro & $F$ & $p$ \\
\midrule
\multirow{2}{*}{Interactions} 
   & Polarized & \textbf{10.30} (12.43) & 4.82 (4.49) & 4.80 (5.29) & \multirow{2}{*}{\textbf{3.25}$^a$} & \multirow{2}{*}{.042*} \\
   & Unpolarized & 7.92 (7.62) & 4.71 (3.26) & \textbf{9.25} (12.85) & & \\
\midrule
\multirow{2}{*}{Likes} 
   & Polarized & \textbf{5.65} (9.53) & 2.55 (2.65) & 2.35 (2.54) & \multirow{2}{*}{2.25$^a$} & \multirow{2}{*}{.110} \\
   & Unpolarized & 4.38 (4.03) & 2.64 (2.44) & \textbf{4.95} (8.22) & & \\
\midrule
\multirow{2}{*}{Reposts} 
   & Polarized & 0.65 (1.03) & 0.45 (0.67) & 0.40 (0.82) & \multirow{2}{*}{1.89$^a$} & \multirow{2}{*}{.156} \\
   & Unpolarized & \textbf{1.08} (2.30) & 0.29 (0.46) & \textbf{1.10} (1.86) & & \\
\midrule
\multirow{2}{*}{Comments} 
   & Polarized & 0.35 (0.57) & 0.23 (0.43) & 0.25 (0.55) & \multirow{2}{*}{\textbf{6.51}$^b$} & \multirow{2}{*}{.012*} \\
   & Unpolarized & 0.50 (1.18) & 0.71 (1.12) & \textbf{1.10} (1.94) & & \\
\midrule
\multirow{2}{*}{Follows} 
   & Polarized & \textbf{3.30} (4.12) & 1.45 (1.84) & 1.50 (2.82) & \multirow{2}{*}{2.60$^a$} & \multirow{2}{*}{.078} \\
   & Unpolarized & 1.88 (3.38) & 1.04 (1.57) & 1.95 (3.33) & & \\
\bottomrule
\multicolumn{7}{p{.95\textwidth}}{\small \textbf{Note:} Values show means with standard deviations in parentheses. Bold values indicate highest means within recommendation bias conditions. $^a$F-statistic for main effect of Recommendation (df = 2, 134). $^b$F-statistic for main effect of Polarization (df = 1, 135). *$p$ < .05} \\
\end{tabularx}
\end{table}


Our analysis revealed distinct patterns of user engagement across the experimental conditions. Of the total sample, participants generated $946$ interactions throughout the study period, with individual engagement levels varying substantially ($M = 6.91$, $SD = 8.50$, $Mdn = 5.00$, Range: $0-56$). The majority of participants ($81.02\%$) engaged at least once with the content. Among active users, engagement levels were distributed across three categories: low engagement ($1$-$5$ interactions; $36.5\%$ of participants), moderate engagement ($6$-$10$ interactions; $24.82\%$), and high engagement ($>10$ interactions; $19.71\%$).


Figure~\ref{fig:stacked-interaction-distribution} illustrates the distribution of interaction types across conditions. The visualization reveals a clear hierarchy in users' preferred forms of engagement, with \emph{likes} consistently representing the dominant form of interaction, accounting for $54.02\%$ of all engagements ($M = 3.73$, $SD = 5.63$). This was followed by \emph{follows} ($26.53\%$; $M = 1.83$, $SD = 2.98$), \emph{reposts} ($9.41\%$; $M = 0.65$, $SD = 1.36$), and \emph{comments} ($7.61\%$; $M = 0.53$, $SD = 1.11$). This pattern suggests a preference for low-effort engagement forms over more demanding interactions like commenting or reposting.

Correlation analysis revealed strong associations between certain interaction types. \emph{Total interactions} showed strong positive correlations with \emph{likes} ($r = .93$, $p < .001$) and \emph{follows} ($r = .73$, $p < .001$), moderate correlations with \emph{comments} ($r = .45$, $p < .001$), and weaker correlations with \emph{reposts} ($r = .37$, $p < .001$). Notably, \emph{comments} and \emph{reposts} showed minimal correlation with each other ($r = .01$, $p = .899$), suggesting these forms of engagement might serve distinct purposes for users.

\begin{figure}[h!]
    \centering
    \includegraphics[width=\textwidth]{figures/interaction_distribution_stacked.png}
    \caption{Distribution of interaction types across experimental conditions. The stacked bars show the relative proportion of different interaction types (\emph{likes}, \emph{reposts}, \emph{comments}, and \emph{follows}) for each combination of polarization level and recommendation bias. Total interaction counts are displayed above each bar.}
    \label{fig:stacked-interaction-distribution}
\end{figure}


\begin{figure}[h!]
    \centering
    \includegraphics[width=\textwidth]{figures/interaction_plots_activity.png}
    \caption{Interaction plots showing the effects of polarization and recommendation type on different forms of user engagement. Red lines represent the polarized condition, green lines represent the unpolarized condition. Error bars represent $95\%$ confidence intervals.}
    \label{fig:interaction-plot-recommendation-bias}
\end{figure}


Analysis of variance (see Table~\ref{tab:interactions-anova} and Figure~\ref{fig:interaction-plot-recommendation-bias}) revealed several significant effects of our experimental manipulations. For \emph{total interactions}, we found a significant main effect of \emph{Recommendation Bias} ($F(2,134) = 3.25$, $p = .042$, $\eta^2_p  = .046$). Post-hoc analyses indicated in the contra conditions ($M = 9.11$, $SD = 10.02$) significantly more interactions were generated than in the balanced ($M = 4.77$, $SD = 3.88$, $p = .008$, Hedges' $g = 0.56$). The comparison between contra and and pro-bias ($M = 7.03$, $SD = 9.07$) did not reach significance ($p = .345$), nor did the comparison between balanced and pro-bias ($p = .180$).

The analysis of specific interaction types revealed distinct patterns. Most notably, we found a significant main effect of \emph{Polarization Degree} on \emph{commenting} behavior ($F(1,135) = 6.51$, $p = .012$, $\eta^2_p = .046$). Participants in unpolarized conditions exhibited higher commenting rates ($M = 0.77$, $SD = 1.41$) compared to those in polarized conditions ($M = 0.28$, $SD = 0.52$), suggesting that a less polarized environment might facilitate more substantive engagement through \emph{comments}.

While other interaction types did not show significant main effects, several trending patterns emerged. The analysis of following behavior revealed a marginal effect of \emph{Recommendation Bias} ($F(2,134) = 2.60$, $p = .078$, $\eta^2_p  = .037$), with participants showing a tendency to follow more users when exposed to opposing views ($M = 2.59$, $SD = 3.75$) compared to balanced content ($M = 1.25$, $SD = 1.71$). Similarly, \emph{likes} showed a pattern consistent with \emph{total interactions}, though the effect did not reach statistical significance ($F(2,134) = 2.25$, $p = .110$, $\eta^2_p  = .032$).

Examining the combined effects of \emph{Polarization Degree} and \emph{Recommendation Bias} revealed interesting patterns in user behavior. In polarized conditions, participants showed the highest level of engagement in the contra condition ($M = 10.30$, $SD = 12.43$), while engagement with neutral ($M = 4.82$, $SD = 4.49$) and pro ($M = 4.80$, $SD = 5.29$) was notably lower. In unpolarized conditions, engagement was more evenly distributed across recommendation types, though still elevated for contra ($M = 7.92$, $SD = 7.62$) and pro-bias ($M = 9.25$, $SD = 12.85$) compared to balanced ($M = 4.71$, $SD = 3.26$).

\subsubsection{Discussion}

The patterns of user activity reveal nuanced insights into how \emph{Polarization Degree} and \emph{Recommendation Bias} shape engagement in online discussions. Our analyses demonstrate that these factors influence not merely the quantity of interactions but also alter how users choose to participate in debates.

The clear hierarchy in engagement forms---with \emph{likes} ($54.02\%$) dominating over \emph{follows} ($26.53\%$), \emph{reposts} ($9.41\%$), and \emph{comments} ($7.61\%$)---reflects fundamental patterns in social media behavior. Notably, the minimal correlation between \emph{comments} and \emph{reposts} ($r = .01$) challenges the intuition that all forms of active engagement serve similar functions. Instead, these behaviors may represent distinct modes of participation, with commenting indicating dialectical engagement while reposting signals content amplification.

The relationship between \emph{Polarization Degree} and commenting behavior ($\eta^2_p = .046$) warrants careful interpretation. While higher commenting rates in unpolarized conditions appear to suggest that moderate discourse environments foster more substantive engagement \citep{koudenburg_polarized_2022, yousafzai_political_2022}, this interpretation requires qualification. Our sample exhibited predominantly moderate attitudes toward UBI, making them potentially unrepresentative of users who actively engage in polarized discussions \citep{simchon_troll_2022}. The reduced commenting in polarized conditions might thus reflect a mismatch between discussion climate and user predispositions rather than an inherent effect of polarization.

The significant effect of \emph{Recommendation Bias} on \emph{total interactions} ($\eta^2_p = .046$) is particularly noteworthy in the context of opinion formation. The higher engagement in the polarized contra-bias condition ($M = 10.30$) coincided with the strongest observed opinion shifts ($M = -0.408$). Given that UBI represents a proposal for systemic economic change, arguments against it may have resonated more strongly with users' status quo bias and loss aversion. In a polarized environment, these contra-UBI arguments might have appeared particularly salient and consequential, leading to both increased engagement and stronger opinion shifts. This suggests that the combination of topic-specific factors---namely the potentially threatening nature of economic system changes---with polarized discourse might amplify user engagement, particularly when arguments align with psychological tendencies toward preserving existing systems.

While these findings provide initial insights into the relationship between platform design, user engagement, and opinion dynamics, several limitations suggest the need for more extensive research. The short-term nature of our study and its focus on a single topic limit generalizability. Future research should pursue longitudinal studies comparing topics with varying degrees of polarization and personal involvement to distinguish between topic-specific effects and general patterns of online discourse dynamics. Our findings thus represent a starting point for understanding the complex interplay between platform design, user behavior, and opinion dynamics in online discussions.

\section{Overall Discussion}

This study presents a novel methodological framework for investigating online polarization through controlled experimental manipulation of social media environments using LLM-based artificial agents. Our findings demonstrate that this approach can successfully reproduce key characteristics of polarized online discourse while enabling precise control over environmental factors that shape user perceptions and behaviors. The integration of sophisticated language models with traditional opinion dynamics frameworks represents a significant advancement in our ability to study the microfoundations of polarization processes.

Our experimental framework advances beyond observational and theoretical approaches by providing systematic empirical evidence for how online environments influence user perceptions and behaviors. Our results demonstrate that group identity processes are fundamental to polarization dynamics in online environments, even when interactions occur with artificial agents. This finding extends social identity theory in important ways: where \citep{tajfel_integrative_1979} established how group identities could emerge from minimal categorical distinctions, and \citep{huddy_social_2001} emphasized the need to examine real-world complexity of identity formation, our work reveals how these processes manifest in digital spaces.

The amplification of group-based polarization we observe aligns with fundamental theories of political identity formation. \citep{laclau_hegemony_2014}'s concept of the constitutive outside posits that political identities fundamentally emerge and strengthen through the recognition of an opposing force---an "other" against which the group defines itself. This process is closely related to what \citep{schmitt_concept_2008} terms the friend-enemy distinction, where political identities crystallize around the identification of a fundamental antagonist. Our offline evaluation provides empirical support for these theoretical perspectives: agents in moderate homophily conditions showed significantly higher polarization than those in high homophily conditions. This finding aligns with \citep{bateson_naven_1958}'s notion of complementary schismogenesis---a process where interaction between groups leads to a progressive differentiation of their behaviors and identities, with each group's actions eliciting more extreme counter-actions from the other. This challenges the predominant ``echo chamber'' framework \citep{bakshy_exposure_2015} for understanding online polarization, suggesting that exposure to opposing views, rather than isolation, can intensify group identities.

Our user study reinforces this understanding of polarization as an active process of group differentiation, showing that participants perceived significantly higher emotional content and group identity salience in polarized conditions. This finding substantiates \citep{bliuc_online_2021}'s theoretical framework, which emphasizes how conflicting collective narratives---rather than mere isolation---drive polarization. Our observation that polarized conditions led to reduced uncertainty expression reveals a key mechanism in this process: the replacement of epistemic humility with group-based certainty. This pattern aligns with \citep{mason_cross-cutting_2016}'s findings on emotional reactivity in political messaging, while extending them by demonstrating these effects in controlled, artificial environments. Similarly, it complements \citep{fischer_emotion_2023}'s analysis of anger amplification in social media by revealing how increased certainty accompanies heightened emotional expression in polarized discourse.

The relationship between polarization and emotional discourse that we observe extends contemporary models of affective polarization in significant ways. While \citep{iyengar_fear_2015} demonstrated how partisan identities drive emotional responses in political contexts, our findings reveal a more complex dynamic centered on group-based processing. Our path analysis shows that polarized conditions significantly reduce uncertainty expression while increasing both emotional content and group salience. However, the relationship between these factors is more nuanced than initially apparent: while polarization directly affects all these dimensions, our structural equation model reveals that group salience, rather than emotional content or uncertainty, plays the key mediating role in shaping polarization perceptions. This builds upon \citep{albertson_dog-whistle_2015}'s work on group-based political messaging by showing how these identity-based dynamics operate in digital environments where group boundaries become particularly salient.

These findings collectively suggest that online polarization emerges primarily from the fundamental role of group opposition in political identity formation, rather than from mere information exposure patterns or purely affective mechanisms. This helps explain why simple exposure to opposing views often fails to reduce polarization \citep{bail_exposure_2018} and suggests that effective interventions must address the deeper dynamics of group identity formation in digital spaces.

The methodological innovation of our approach is particularly strengthened by the convergence between offline LLM-based assessments and human participants' perceptions (cf. Sections~\ref{subsec:offline-message-analysis} and~\ref{subsec:debate-perception}). Both artificial agents and human participants demonstrated consistent patterns in evaluating emotional intensity, uncertainty expression, and group identity salience across conditions. This dual-validation approach combines the advantages of systematic large-scale analysis with crucial insights into human perception and behavior, while demonstrating that LLM-based content analysis effectively captures psychologically relevant aspects of online discourse.

By establishing the feasibility of using LLM-based agents to create controlled yet ecologically valid social media environments, we provide researchers with a powerful new tool for studying online social dynamics. This approach enables precise manipulation of discourse characteristics, allows for systematic variation in network structure and content exposure, and facilitates the collection of fine-grained behavioral data that would be difficult to obtain in naturalistic settings. Moreover, our demonstration that artificial agents can create environments that elicit authentic human responses challenges assumptions about the necessary conditions for studying social media effects, suggesting that users respond to perceived rather than actual social presence.

These methodological advances have important implications for ongoing debates about social media's role in democratic discourse. While existing theories often frame platform effects as either uniformly negative (promoting polarization) or positive (enabling diverse discourse), our results suggest a more complex theoretical framework is needed. The observed interaction between recommendation algorithms, discourse characteristics, and user behavior indicates that platform effects are highly context-dependent and mediated by multiple psychological and social processes. This suggests the need for more sophisticated theoretical models that can account for the dynamic interplay between technological affordances, social psychological processes, and content characteristics.

%The systematic variation we observed in engagement patterns across different polarization conditions provides new theoretical insights into how platform design choices shape user behavior, moving beyond simple deterministic models of technology effects. These findings suggest that certain aspects of polarized discourse may be more structurally determined than previously thought, pointing to the need for theories that better integrate technological and social explanations of polarization.

However, several important limitations must be acknowledged. The short-term nature of our experimental exposure means we cannot make definitive claims about long-term polarization dynamics. While our results demonstrate that the simulation framework can reproduce key features of polarized discourse and elicit expected user responses, questions remain about how these effects might evolve over extended periods of interaction. Additionally, the artificial nature of the experimental environment, despite its ecological validity, may not capture all relevant aspects of real-world social media use. 

Specifically, our methodological approach necessitated making several assumptions about parameter values and interaction dynamics in the absence of comprehensive empirical data. The framework's sophisticated non-linear functions, while capturing key phenomena like confirmation bias and backfire effects, introduce numerous parameters that currently lack empirical validation. Although this was a deliberate choice to create a detailed proxy for polarized discourse, the high number of free parameters raises valid concerns about potential overfitting and the robustness of our findings. While we designed the framework with future empirical calibration in mind, the current parameter values remain largely theoretical and may not accurately reflect real-world behavioral patterns. This limitation is particularly important when interpreting our results, as different parameter configurations could potentially lead to significantly different outcomes. The framework's detailed parameterization, though offering flexibility for future refinement, also increases model complexity and makes it more challenging to identify which specific mechanisms drive observed effects.

Another significant limitation concerns our focus on Universal Basic Income as the sole discussion topic. While UBI's characteristics - moderate pre-existing polarization and potential for opinion formation - made it suitable for our initial investigation, this single-topic approach limits generalizability. Different topics may generate distinct polarization dynamics: established political issues might show stronger ideological entrenchment, while technical discussions could exhibit different group formation patterns. The observed effects of emotional content and group identity salience might vary across topics with different emotional resonance or existing group affiliations.

The sample characteristics and recruitment strategy present additional methodological concerns. The participant pool, recruited through Prolific, skewed toward younger, highly educated, and moderate individuals, potentially limiting our ability to understand how more diverse populations or those with stronger initial positions might interact with polarized environments. Moreover, while our experimental platform successfully reproduced key social media features, the simplified interface lacks many nuanced interaction affordances present in real platforms that might influence user behavior in important ways. Our current implementation also assumes English-language discourse patterns, which may not generalize to other cultural and linguistic contexts where polarization dynamics could manifest differently.

The role of individual differences and contextual factors also remains incompletely explored. The study's design, while allowing for controlled manipulation of environmental features, could not fully account for how personal characteristics such as digital literacy, prior platform experience, or general political engagement might moderate the observed effects. Additionally, the fixed nature of agent behaviors, though methodologically necessary, may not capture the dynamic adaptations that characterize human discourse patterns, particularly in heated debates where rhetorical strategies often evolve in response to opponent reactions.

Future research should address these limitations by conducting longitudinal studies that track user behavior and attitude change over extended periods. The framework could be extended to incorporate more sophisticated network dynamics, explore the role of different content recommendation algorithms, and investigate potential intervention strategies. Additionally, cross-platform studies could examine how different interface designs and interaction affordances influence polarization processes. Studies examining multiple topics simultaneously could help distinguish universal polarization mechanisms from topic-specific effects, while varying topic characteristics (e.g., emotional salience, technical complexity, moral loading) could reveal how content domain influences polarization dynamics.

Particularly promising directions for future work include investigating the role of influencer dynamics in polarization processes, exploring how different moderation strategies affect discourse quality, and examining how varying levels of algorithmic content curation influence user behavior and perception. The framework could also be adapted to study other aspects of online social behavior, such as information diffusion patterns or the emergence of new social norms.
\section*{Conclusion}
This paper aims to enhance our understanding of the computational complexity of computing various Shapley value variants. We found that for various ML models --- including decision trees, regression tree ensembles, weighted automata, and linear regression --- both local and global interventional and baseline SHAP can be computed in polynomial time under HMM modeled distributions. This extends popular algorithms, such as TreeSHAP, beyond their empirical distributional scope. We also establish strict complexity gaps between the various SHAP variants (baseline, interventional, and conditional) and prove the intractability of computing SHAP for tree ensembles and neural networks in simplified scenarios. Overall, we present SHAP as a versatile framework whose complexity depends on four key factors: \begin{inparaenum}[(i)] \item model type, \item SHAP variant, \item distribution modeling approach, \item and local vs. global explanations\end{inparaenum}. We believe this perspective provides deeper insight into the computational complexity of SHAP, paving the way for future work.




%We believe that our framework provides a more intricate understanding of SHAP computation complexity across different models, distributions, and variants, paving the way for further research.

Our work opens promising directions for future research. First, expanding our computational analysis to other SHAP-related metrics, such as asymmetric SHAP~\citep{frye20} and SAGE~\citep{covert2020understanding}, would be valuable. Additionally, we aim to explore more expressive distribution classes and relaxed assumptions beyond those in Section \ref{sec:tractable} while maintaining tractable SHAP computation. Finally, when exact computation is intractable (Section \ref{sec:intractable}), investigating the approximability of SHAP metrics through approximation and parameterized complexity theory~\citep{downey2012parameterized} is an important direction.

%Our work opens several promising avenues for future research on the computational properties of explainable AI methods, with a particular focus on SHAP. First, it would be interesting to broaden the computational analysis conducted in this work to include other popular SHAP-related metrics in the literature, such as asymmetric SHAP \cite{frye20} and SAGE \cite{covert2020understanding}. Also, in the future, we aim to explore more expressive distribution classes and relaxed distributional assumptions—extending beyond those examined in Section \ref{sec:tractable} —that still yield tractable SHAP computation. Finally, when exact computation proves intractable (Section \ref{sec:intractable}), it is worthwhile to theoretically investigate the question of the approximability of computing the SHAP metrics across various configurations, through the lens of approximation and parametrized complexity theory \cite{arora2009computational}.

%This paper aims to deepen our understanding of the computational complexity involved in obtaining different Shapley value variants. We found that for a variety of ML models, including decision trees, tree ensembles for regression, weighted automata, and linear regression models — computing both local and global interventional and baseline SHAP can be done in polynomial time when distributions are modeled by HMMs. This extends the distributional scope of popular algorithms like TreeSHAP, which is limited to empirical distributions. Additionally, we demonstrate a strict complexity gap between SHAP variants, showing that interventional and baseline SHAP can be strictly easier to compute than conditional SHAP. Despite these positive results, we uncovered intractability for various SHAP variants in neural networks and tree ensembles. Finally, we provided generalized complexity relations across SHAP variants. We believe that our framework offers a deeper understanding of the complexity involved in computing SHAP across various variants, models, distributions, as well as in both local and global computations, laying the groundwork for future research.

%% following line to enable line numbers
%% \linenumbers

\bibliographystyle{elsarticle-harv.bst}  
\bibliography{bib}

\end{document}

\endinput
%%
%% End of file `elsarticle-template-num.tex'.
