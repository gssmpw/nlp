%% 
%% Copyright 2007-2025 Elsevier Ltd
%% 
%% This file is part of the 'Elsarticle Bundle'.
%% ---------------------------------------------
%% 
%% It may be distributed under the conditions of the LaTeX Project Public
%% License, either version 1.3 of this license or (at your option) any
%% later version.  The latest version of this license is in
%%    http://www.latex-project.org/lppl.txt
%% and version 1.3 or later is part of all distributions of LaTeX
%% version 1999/12/01 or later.
%% 
%% The list of all files belonging to the 'Elsarticle Bundle' is
%% given in the file `manifest.txt'.
%% 
%% Template article for Elsevier's document class `elsarticle'
%% with numbered style bibliographic references
%% SP 2008/03/01
%% $Id: elsarticle-template-num.tex 272 2025-01-09 17:36:26Z rishi $
%%
\documentclass[preprint,12pt,authoryear]{elsarticle}

%% Use the option review to obtain double line spacing
%% \documentclass[authoryear,preprint,review,12pt]{elsarticle}

%% Use the options 1p,twocolumn; 3p; 3p,twocolumn; 5p; or 5p,twocolumn
%% for a journal layout:
%% \documentclass[final,1p,times]{elsarticle}
%% \documentclass[final,1p,times,twocolumn]{elsarticle}
%% \documentclass[final,3p,times]{elsarticle}
%% \documentclass[final,3p,times,twocolumn]{elsarticle}
%% \documentclass[final,5p,times]{elsarticle}
%% \documentclass[final,5p,times,twocolumn]{elsarticle}

%% For including figures, graphicx.sty has been loaded in
%% elsarticle.cls. If you prefer to use the old commands
%% please give \usepackage{epsfig}

\usepackage{natbib}
\usepackage{subcaption}
\usepackage[linesnumbered, ruled, vlined]{algorithm2e}
\usepackage{multirow}
\usepackage{amsmath}
\usepackage{booktabs}
\usepackage{graphicx}
\usepackage[T1]{fontenc}
\usepackage{tabularx}
\usepackage[margin=2.5cm]{geometry}% by courtesy of Mico

%% The lineno packages adds line numbers. Start line numbering with
%% \begin{linenumbers}, end it with \end{linenumbers}. Or switch it on
%% for the whole article with \linenumbers.
%% \usepackage{lineno}

\journal{arXiv.org}

\begin{document}

\begin{frontmatter}

%% Title, authors and addresses

%% use the tnoteref command within \title for footnotes;
%% use the tnotetext command for theassociated footnote;
%% use the fnref command within \author or \affiliation for footnotes;
%% use the fntext command for theassociated footnote;
%% use the corref command within \author for corresponding author footnotes;
%% use the cortext command for theassociated footnote;
%% use the ead command for the email address,
%% and the form \ead[url] for the home page:
%% \title{Title\tnoteref{label1}}
%% \tnotetext[label1]{}
%% \author{Name\corref{cor1}\fnref{label2}}
%% \ead{email address}
%% \ead[url]{home page}
%% \fntext[label2]{}
%% \cortext[cor1]{}
%% \affiliation{organization={},
%%             addressline={},
%%             city={},
%%             postcode={},
%%             state={},
%%             country={}}
%% \fntext[label3]{}

\title{Human-Agent Interaction in Synthetic Social Networks: A Framework for Studying Online Polarization}

%% use optional labels to link authors explicitly to addresses:
%% \author[label1,label2]{}
%% \affiliation[label1]{organization={},
%%             addressline={},
%%             city={},
%%             postcode={},
%%             state={},
%%             country={}}
%%
%% \affiliation[label2]{organization={},
%%             addressline={},
%%             city={},
%%             postcode={},
%%             state={},
%%             country={}}

\author[1]{Tim Donkers} %% Author name

\author[1]{J{\"u}rgen Ziegler} %% Author name

%% Author affiliation
\affiliation[1]{organization={University of Duisburg-Essen},%Department and Organization
            addressline={Forsthausweg 2}, 
            city={Duisburg},
            postcode={47058}, 
            state={North Rhine-Westphalia},
            country={Germany}}

%% Abstract
\begin{abstract}
Online social networks have dramatically altered the landscape of public discourse, creating both opportunities for enhanced civic participation and risks of deepening social divisions. Prevalent approaches to studying online polarization have been limited by a methodological disconnect: mathematical models excel at formal analysis but lack linguistic realism, while language model-based simulations capture natural discourse but often sacrifice analytical precision. This paper introduces an innovative computational framework that synthesizes these approaches by embedding formal opinion dynamics principles within LLM-based artificial agents, enabling both rigorous mathematical analysis and naturalistic social interactions. We validate our framework through comprehensive offline testing and experimental evaluation with 122 human participants engaging in a controlled social network environment. The results demonstrate our ability to systematically investigate polarization mechanisms while preserving ecological validity. Our findings reveal how polarized environments shape user perceptions and behavior: participants exposed to polarized discussions showed markedly increased sensitivity to emotional content and group affiliations, while perceiving reduced uncertainty in the agents' positions. By combining mathematical precision with natural language capabilities, our framework opens new avenues for investigating social media phenomena through controlled experimentation. This methodological advancement allows researchers to bridge the gap between theoretical models and empirical observations, offering unprecedented opportunities to study the causal mechanisms underlying online opinion dynamics.
\end{abstract}

%%Graphical abstract
%\begin{graphicalabstract}
%\includegraphics{grabs}
%\end{graphicalabstract}

%% Keywords
\begin{keyword}
Online Polarization \sep Opinion Dynamics \sep Human-Agent Interaction \sep Large Language Models \sep Experimental User Studies \sep Agent-based Simulation
%% keywords here, in the form: keyword \sep keyword

%% PACS codes here, in the form: \PACS code \sep code

%% MSC codes here, in the form: \MSC code \sep code
%% or \MSC[2008] code \sep code (2000 is the default)

\end{keyword}

\end{frontmatter}

\section{Introduction}
\label{sec:introduction}
The business processes of organizations are experiencing ever-increasing complexity due to the large amount of data, high number of users, and high-tech devices involved \cite{martin2021pmopportunitieschallenges, beerepoot2023biggestbpmproblems}. This complexity may cause business processes to deviate from normal control flow due to unforeseen and disruptive anomalies \cite{adams2023proceddsriftdetection}. These control-flow anomalies manifest as unknown, skipped, and wrongly-ordered activities in the traces of event logs monitored from the execution of business processes \cite{ko2023adsystematicreview}. For the sake of clarity, let us consider an illustrative example of such anomalies. Figure \ref{FP_ANOMALIES} shows a so-called event log footprint, which captures the control flow relations of four activities of a hypothetical event log. In particular, this footprint captures the control-flow relations between activities \texttt{a}, \texttt{b}, \texttt{c} and \texttt{d}. These are the causal ($\rightarrow$) relation, concurrent ($\parallel$) relation, and other ($\#$) relations such as exclusivity or non-local dependency \cite{aalst2022pmhandbook}. In addition, on the right are six traces, of which five exhibit skipped, wrongly-ordered and unknown control-flow anomalies. For example, $\langle$\texttt{a b d}$\rangle$ has a skipped activity, which is \texttt{c}. Because of this skipped activity, the control-flow relation \texttt{b}$\,\#\,$\texttt{d} is violated, since \texttt{d} directly follows \texttt{b} in the anomalous trace.
\begin{figure}[!t]
\centering
\includegraphics[width=0.9\columnwidth]{images/FP_ANOMALIES.png}
\caption{An example event log footprint with six traces, of which five exhibit control-flow anomalies.}
\label{FP_ANOMALIES}
\end{figure}

\subsection{Control-flow anomaly detection}
Control-flow anomaly detection techniques aim to characterize the normal control flow from event logs and verify whether these deviations occur in new event logs \cite{ko2023adsystematicreview}. To develop control-flow anomaly detection techniques, \revision{process mining} has seen widespread adoption owing to process discovery and \revision{conformance checking}. On the one hand, process discovery is a set of algorithms that encode control-flow relations as a set of model elements and constraints according to a given modeling formalism \cite{aalst2022pmhandbook}; hereafter, we refer to the Petri net, a widespread modeling formalism. On the other hand, \revision{conformance checking} is an explainable set of algorithms that allows linking any deviations with the reference Petri net and providing the fitness measure, namely a measure of how much the Petri net fits the new event log \cite{aalst2022pmhandbook}. Many control-flow anomaly detection techniques based on \revision{conformance checking} (hereafter, \revision{conformance checking}-based techniques) use the fitness measure to determine whether an event log is anomalous \cite{bezerra2009pmad, bezerra2013adlogspais, myers2018icsadpm, pecchia2020applicationfailuresanalysispm}. 

The scientific literature also includes many \revision{conformance checking}-independent techniques for control-flow anomaly detection that combine specific types of trace encodings with machine/deep learning \cite{ko2023adsystematicreview, tavares2023pmtraceencoding}. Whereas these techniques are very effective, their explainability is challenging due to both the type of trace encoding employed and the machine/deep learning model used \cite{rawal2022trustworthyaiadvances,li2023explainablead}. Hence, in the following, we focus on the shortcomings of \revision{conformance checking}-based techniques to investigate whether it is possible to support the development of competitive control-flow anomaly detection techniques while maintaining the explainable nature of \revision{conformance checking}.
\begin{figure}[!t]
\centering
\includegraphics[width=\columnwidth]{images/HIGH_LEVEL_VIEW.png}
\caption{A high-level view of the proposed framework for combining \revision{process mining}-based feature extraction with dimensionality reduction for control-flow anomaly detection.}
\label{HIGH_LEVEL_VIEW}
\end{figure}

\subsection{Shortcomings of \revision{conformance checking}-based techniques}
Unfortunately, the detection effectiveness of \revision{conformance checking}-based techniques is affected by noisy data and low-quality Petri nets, which may be due to human errors in the modeling process or representational bias of process discovery algorithms \cite{bezerra2013adlogspais, pecchia2020applicationfailuresanalysispm, aalst2016pm}. Specifically, on the one hand, noisy data may introduce infrequent and deceptive control-flow relations that may result in inconsistent fitness measures, whereas, on the other hand, checking event logs against a low-quality Petri net could lead to an unreliable distribution of fitness measures. Nonetheless, such Petri nets can still be used as references to obtain insightful information for \revision{process mining}-based feature extraction, supporting the development of competitive and explainable \revision{conformance checking}-based techniques for control-flow anomaly detection despite the problems above. For example, a few works outline that token-based \revision{conformance checking} can be used for \revision{process mining}-based feature extraction to build tabular data and develop effective \revision{conformance checking}-based techniques for control-flow anomaly detection \cite{singh2022lapmsh, debenedictis2023dtadiiot}. However, to the best of our knowledge, the scientific literature lacks a structured proposal for \revision{process mining}-based feature extraction using the state-of-the-art \revision{conformance checking} variant, namely alignment-based \revision{conformance checking}.

\subsection{Contributions}
We propose a novel \revision{process mining}-based feature extraction approach with alignment-based \revision{conformance checking}. This variant aligns the deviating control flow with a reference Petri net; the resulting alignment can be inspected to extract additional statistics such as the number of times a given activity caused mismatches \cite{aalst2022pmhandbook}. We integrate this approach into a flexible and explainable framework for developing techniques for control-flow anomaly detection. The framework combines \revision{process mining}-based feature extraction and dimensionality reduction to handle high-dimensional feature sets, achieve detection effectiveness, and support explainability. Notably, in addition to our proposed \revision{process mining}-based feature extraction approach, the framework allows employing other approaches, enabling a fair comparison of multiple \revision{conformance checking}-based and \revision{conformance checking}-independent techniques for control-flow anomaly detection. Figure \ref{HIGH_LEVEL_VIEW} shows a high-level view of the framework. Business processes are monitored, and event logs obtained from the database of information systems. Subsequently, \revision{process mining}-based feature extraction is applied to these event logs and tabular data input to dimensionality reduction to identify control-flow anomalies. We apply several \revision{conformance checking}-based and \revision{conformance checking}-independent framework techniques to publicly available datasets, simulated data of a case study from railways, and real-world data of a case study from healthcare. We show that the framework techniques implementing our approach outperform the baseline \revision{conformance checking}-based techniques while maintaining the explainable nature of \revision{conformance checking}.

In summary, the contributions of this paper are as follows.
\begin{itemize}
    \item{
        A novel \revision{process mining}-based feature extraction approach to support the development of competitive and explainable \revision{conformance checking}-based techniques for control-flow anomaly detection.
    }
    \item{
        A flexible and explainable framework for developing techniques for control-flow anomaly detection using \revision{process mining}-based feature extraction and dimensionality reduction.
    }
    \item{
        Application to synthetic and real-world datasets of several \revision{conformance checking}-based and \revision{conformance checking}-independent framework techniques, evaluating their detection effectiveness and explainability.
    }
\end{itemize}

The rest of the paper is organized as follows.
\begin{itemize}
    \item Section \ref{sec:related_work} reviews the existing techniques for control-flow anomaly detection, categorizing them into \revision{conformance checking}-based and \revision{conformance checking}-independent techniques.
    \item Section \ref{sec:abccfe} provides the preliminaries of \revision{process mining} to establish the notation used throughout the paper, and delves into the details of the proposed \revision{process mining}-based feature extraction approach with alignment-based \revision{conformance checking}.
    \item Section \ref{sec:framework} describes the framework for developing \revision{conformance checking}-based and \revision{conformance checking}-independent techniques for control-flow anomaly detection that combine \revision{process mining}-based feature extraction and dimensionality reduction.
    \item Section \ref{sec:evaluation} presents the experiments conducted with multiple framework and baseline techniques using data from publicly available datasets and case studies.
    \item Section \ref{sec:conclusions} draws the conclusions and presents future work.
\end{itemize}
\section{RELATED WORK}
\label{sec:relatedwork}
In this section, we describe the previous works related to our proposal, which are divided into two parts. In Section~\ref{sec:relatedwork_exoplanet}, we present a review of approaches based on machine learning techniques for the detection of planetary transit signals. Section~\ref{sec:relatedwork_attention} provides an account of the approaches based on attention mechanisms applied in Astronomy.\par

\subsection{Exoplanet detection}
\label{sec:relatedwork_exoplanet}
Machine learning methods have achieved great performance for the automatic selection of exoplanet transit signals. One of the earliest applications of machine learning is a model named Autovetter \citep{MCcauliff}, which is a random forest (RF) model based on characteristics derived from Kepler pipeline statistics to classify exoplanet and false positive signals. Then, other studies emerged that also used supervised learning. \cite{mislis2016sidra} also used a RF, but unlike the work by \citet{MCcauliff}, they used simulated light curves and a box least square \citep[BLS;][]{kovacs2002box}-based periodogram to search for transiting exoplanets. \citet{thompson2015machine} proposed a k-nearest neighbors model for Kepler data to determine if a given signal has similarity to known transits. Unsupervised learning techniques were also applied, such as self-organizing maps (SOM), proposed \citet{armstrong2016transit}; which implements an architecture to segment similar light curves. In the same way, \citet{armstrong2018automatic} developed a combination of supervised and unsupervised learning, including RF and SOM models. In general, these approaches require a previous phase of feature engineering for each light curve. \par

%DL is a modern data-driven technology that automatically extracts characteristics, and that has been successful in classification problems from a variety of application domains. The architecture relies on several layers of NNs of simple interconnected units and uses layers to build increasingly complex and useful features by means of linear and non-linear transformation. This family of models is capable of generating increasingly high-level representations \citep{lecun2015deep}.

The application of DL for exoplanetary signal detection has evolved rapidly in recent years and has become very popular in planetary science.  \citet{pearson2018} and \citet{zucker2018shallow} developed CNN-based algorithms that learn from synthetic data to search for exoplanets. Perhaps one of the most successful applications of the DL models in transit detection was that of \citet{Shallue_2018}; who, in collaboration with Google, proposed a CNN named AstroNet that recognizes exoplanet signals in real data from Kepler. AstroNet uses the training set of labelled TCEs from the Autovetter planet candidate catalog of Q1–Q17 data release 24 (DR24) of the Kepler mission \citep{catanzarite2015autovetter}. AstroNet analyses the data in two views: a ``global view'', and ``local view'' \citep{Shallue_2018}. \par


% The global view shows the characteristics of the light curve over an orbital period, and a local view shows the moment at occurring the transit in detail

%different = space-based

Based on AstroNet, researchers have modified the original AstroNet model to rank candidates from different surveys, specifically for Kepler and TESS missions. \citet{ansdell2018scientific} developed a CNN trained on Kepler data, and included for the first time the information on the centroids, showing that the model improves performance considerably. Then, \citet{osborn2020rapid} and \citet{yu2019identifying} also included the centroids information, but in addition, \citet{osborn2020rapid} included information of the stellar and transit parameters. Finally, \citet{rao2021nigraha} proposed a pipeline that includes a new ``half-phase'' view of the transit signal. This half-phase view represents a transit view with a different time and phase. The purpose of this view is to recover any possible secondary eclipse (the object hiding behind the disk of the primary star).


%last pipeline applies a procedure after the prediction of the model to obtain new candidates, this process is carried out through a series of steps that include the evaluation with Discovery and Validation of Exoplanets (DAVE) \citet{kostov2019discovery} that was adapted for the TESS telescope.\par
%



\subsection{Attention mechanisms in astronomy}
\label{sec:relatedwork_attention}
Despite the remarkable success of attention mechanisms in sequential data, few papers have exploited their advantages in astronomy. In particular, there are no models based on attention mechanisms for detecting planets. Below we present a summary of the main applications of this modeling approach to astronomy, based on two points of view; performance and interpretability of the model.\par
%Attention mechanisms have not yet been explored in all sub-areas of astronomy. However, recent works show a successful application of the mechanism.
%performance

The application of attention mechanisms has shown improvements in the performance of some regression and classification tasks compared to previous approaches. One of the first implementations of the attention mechanism was to find gravitational lenses proposed by \citet{thuruthipilly2021finding}. They designed 21 self-attention-based encoder models, where each model was trained separately with 18,000 simulated images, demonstrating that the model based on the Transformer has a better performance and uses fewer trainable parameters compared to CNN. A novel application was proposed by \citet{lin2021galaxy} for the morphological classification of galaxies, who used an architecture derived from the Transformer, named Vision Transformer (VIT) \citep{dosovitskiy2020image}. \citet{lin2021galaxy} demonstrated competitive results compared to CNNs. Another application with successful results was proposed by \citet{zerveas2021transformer}; which first proposed a transformer-based framework for learning unsupervised representations of multivariate time series. Their methodology takes advantage of unlabeled data to train an encoder and extract dense vector representations of time series. Subsequently, they evaluate the model for regression and classification tasks, demonstrating better performance than other state-of-the-art supervised methods, even with data sets with limited samples.

%interpretation
Regarding the interpretability of the model, a recent contribution that analyses the attention maps was presented by \citet{bowles20212}, which explored the use of group-equivariant self-attention for radio astronomy classification. Compared to other approaches, this model analysed the attention maps of the predictions and showed that the mechanism extracts the brightest spots and jets of the radio source more clearly. This indicates that attention maps for prediction interpretation could help experts see patterns that the human eye often misses. \par

In the field of variable stars, \citet{allam2021paying} employed the mechanism for classifying multivariate time series in variable stars. And additionally, \citet{allam2021paying} showed that the activation weights are accommodated according to the variation in brightness of the star, achieving a more interpretable model. And finally, related to the TESS telescope, \citet{morvan2022don} proposed a model that removes the noise from the light curves through the distribution of attention weights. \citet{morvan2022don} showed that the use of the attention mechanism is excellent for removing noise and outliers in time series datasets compared with other approaches. In addition, the use of attention maps allowed them to show the representations learned from the model. \par

Recent attention mechanism approaches in astronomy demonstrate comparable results with earlier approaches, such as CNNs. At the same time, they offer interpretability of their results, which allows a post-prediction analysis. \par


\section{Method}\label{sec:method}
\begin{figure}
    \centering
    \includegraphics[width=0.85\textwidth]{imgs/heatmap_acc.pdf}
    \caption{\textbf{Visualization of the proposed periodic Bayesian flow with mean parameter $\mu$ and accumulated accuracy parameter $c$ which corresponds to the entropy/uncertainty}. For $x = 0.3, \beta(1) = 1000$ and $\alpha_i$ defined in \cref{appd:bfn_cir}, this figure plots three colored stochastic parameter trajectories for receiver mean parameter $m$ and accumulated accuracy parameter $c$, superimposed on a log-scale heatmap of the Bayesian flow distribution $p_F(m|x,\senderacc)$ and $p_F(c|x,\senderacc)$. Note the \emph{non-monotonicity} and \emph{non-additive} property of $c$ which could inform the network the entropy of the mean parameter $m$ as a condition and the \emph{periodicity} of $m$. %\jj{Shrink the figures to save space}\hanlin{Do we need to make this figure one-column?}
    }
    \label{fig:vmbf_vis}
    \vskip -0.1in
\end{figure}
% \begin{wrapfigure}{r}{0.5\textwidth}
%     \centering
%     \includegraphics[width=0.49\textwidth]{imgs/heatmap_acc.pdf}
%     \caption{\textbf{Visualization of hyper-torus Bayesian flow based on von Mises Distribution}. For $x = 0.3, \beta(1) = 1000$ and $\alpha_i$ defined in \cref{appd:bfn_cir}, this figure plots three colored stochastic parameter trajectories for receiver mean parameter $m$ and accumulated accuracy parameter $c$, superimposed on a log-scale heatmap of the Bayesian flow distribution $p_F(m|x,\senderacc)$ and $p_F(c|x,\senderacc)$. Note the \emph{non-monotonicity} and \emph{non-additive} property of $c$. \jj{Shrink the figures to save space}}
%     \label{fig:vmbf_vis}
%     \vspace{-30pt}
% \end{wrapfigure}


In this section, we explain the detailed design of CrysBFN tackling theoretical and practical challenges. First, we describe how to derive our new formulation of Bayesian Flow Networks over hyper-torus $\mathbb{T}^{D}$ from scratch. Next, we illustrate the two key differences between \modelname and the original form of BFN: $1)$ a meticulously designed novel base distribution with different Bayesian update rules; and $2)$ different properties over the accuracy scheduling resulted from the periodicity and the new Bayesian update rules. Then, we present in detail the overall framework of \modelname over each manifold of the crystal space (\textit{i.e.} fractional coordinates, lattice vectors, atom types) respecting \textit{periodic E(3) invariance}. 

% In this section, we first demonstrate how to build Bayesian flow on hyper-torus $\mathbb{T}^{D}$ by overcoming theoretical and practical problems to provide a low-noise parameter-space approach to fractional atom coordinate generation. Next, we present how \modelname models each manifold of crystal space respecting \textit{periodic E(3) invariance}. 

\subsection{Periodic Bayesian Flow on Hyper-torus \texorpdfstring{$\mathbb{T}^{D}$}{}} 
For generative modeling of fractional coordinates in crystal, we first construct a periodic Bayesian flow on \texorpdfstring{$\mathbb{T}^{D}$}{} by designing every component of the totally new Bayesian update process which we demonstrate to be distinct from the original Bayesian flow (please see \cref{fig:non_add}). 
 %:) 
 
 The fractional atom coordinate system \citep{jiao2023crystal} inherently distributes over a hyper-torus support $\mathbb{T}^{3\times N}$. Hence, the normal distribution support on $\R$ used in the original \citep{bfn} is not suitable for this scenario. 
% The key problem of generative modeling for crystal is the periodicity of Cartesian atom coordinates $\vX$ requiring:
% \begin{equation}\label{eq:periodcity}
% p(\vA,\vL,\vX)=p(\vA,\vL,\vX+\vec{LK}),\text{where}~\vec{K}=\vec{k}\vec{1}_{1\times N},\forall\vec{k}\in\mathbb{Z}^{3\times1}
% \end{equation}
% However, there does not exist such a distribution supporting on $\R$ to model such property because the integration of such distribution over $\R$ will not be finite and equal to 1. Therefore, the normal distribution used in \citet{bfn} can not meet this condition.

To tackle this problem, the circular distribution~\citep{mardia2009directional} over the finite interval $[-\pi,\pi)$ is a natural choice as the base distribution for deriving the BFN on $\mathbb{T}^D$. 
% one natural choice is to 
% we would like to consider the circular distribution over the finite interval as the base 
% we find that circular distributions \citep{mardia2009directional} defined on a finite interval with lengths of $2\pi$ can be used as the instantiation of input distribution for the BFN on $\mathbb{T}^D$.
Specifically, circular distributions enjoy desirable periodic properties: $1)$ the integration over any interval length of $2\pi$ equals 1; $2)$ the probability distribution function is periodic with period $2\pi$.  Sharing the same intrinsic with fractional coordinates, such periodic property of circular distribution makes it suitable for the instantiation of BFN's input distribution, in parameterizing the belief towards ground truth $\x$ on $\mathbb{T}^D$. 
% \yuxuan{this is very complicated from my perspective.} \hanlin{But this property is exactly beautiful and perfectly fit into the BFN.}

\textbf{von Mises Distribution and its Bayesian Update} We choose von Mises distribution \citep{mardia2009directional} from various circular distributions as the form of input distribution, based on the appealing conjugacy property required in the derivation of the BFN framework.
% to leverage the Bayesian conjugacy property of von Mises distribution which is required by the BFN framework. 
That is, the posterior of a von Mises distribution parameterized likelihood is still in the family of von Mises distributions. The probability density function of von Mises distribution with mean direction parameter $m$ and concentration parameter $c$ (describing the entropy/uncertainty of $m$) is defined as: 
\begin{equation}
f(x|m,c)=vM(x|m,c)=\frac{\exp(c\cos(x-m))}{2\pi I_0(c)}
\end{equation}
where $I_0(c)$ is zeroth order modified Bessel function of the first kind as the normalizing constant. Given the last univariate belief parameterized by von Mises distribution with parameter $\theta_{i-1}=\{m_{i-1},\ c_{i-1}\}$ and the sample $y$ from sender distribution with unknown data sample $x$ and known accuracy $\alpha$ describing the entropy/uncertainty of $y$,  Bayesian update for the receiver is deducted as:
\begin{equation}
 h(\{m_{i-1},c_{i-1}\},y,\alpha)=\{m_i,c_i \}, \text{where}
\end{equation}
\begin{equation}\label{eq:h_m}
m_i=\text{atan2}(\alpha\sin y+c_{i-1}\sin m_{i-1}, {\alpha\cos y+c_{i-1}\cos m_{i-1}})
\end{equation}
\begin{equation}\label{eq:h_c}
c_i =\sqrt{\alpha^2+c_{i-1}^2+2\alpha c_{i-1}\cos(y-m_{i-1})}
\end{equation}
The proof of the above equations can be found in \cref{apdx:bayesian_update_function}. The atan2 function refers to  2-argument arctangent. Independently conducting  Bayesian update for each dimension, we can obtain the Bayesian update distribution by marginalizing $\y$:
\begin{equation}
p_U(\vtheta'|\vtheta,\bold{x};\alpha)=\mathbb{E}_{p_S(\bold{y}|\bold{x};\alpha)}\delta(\vtheta'-h(\vtheta,\bold{y},\alpha))=\mathbb{E}_{vM(\bold{y}|\bold{x},\alpha)}\delta(\vtheta'-h(\vtheta,\bold{y},\alpha))
\end{equation} 
\begin{figure}
    \centering
    \vskip -0.15in
    \includegraphics[width=0.95\linewidth]{imgs/non_add.pdf}
    \caption{An intuitive illustration of non-additive accuracy Bayesian update on the torus. The lengths of arrows represent the uncertainty/entropy of the belief (\emph{e.g.}~$1/\sigma^2$ for Gaussian and $c$ for von Mises). The directions of the arrows represent the believed location (\emph{e.g.}~ $\mu$ for Gaussian and $m$ for von Mises).}
    \label{fig:non_add}
    \vskip -0.15in
\end{figure}
\textbf{Non-additive Accuracy} 
The additive accuracy is a nice property held with the Gaussian-formed sender distribution of the original BFN expressed as:
\begin{align}
\label{eq:standard_id}
    \update(\parsn{}'' \mid \parsn{}, \x; \alpha_a+\alpha_b) = \E_{\update(\parsn{}' \mid \parsn{}, \x; \alpha_a)} \update(\parsn{}'' \mid \parsn{}', \x; \alpha_b)
\end{align}
Such property is mainly derived based on the standard identity of Gaussian variable:
\begin{equation}
X \sim \mathcal{N}\left(\mu_X, \sigma_X^2\right), Y \sim \mathcal{N}\left(\mu_Y, \sigma_Y^2\right) \Longrightarrow X+Y \sim \mathcal{N}\left(\mu_X+\mu_Y, \sigma_X^2+\sigma_Y^2\right)
\end{equation}
The additive accuracy property makes it feasible to derive the Bayesian flow distribution $
p_F(\boldsymbol{\theta} \mid \mathbf{x} ; i)=p_U\left(\boldsymbol{\theta} \mid \boldsymbol{\theta}_0, \mathbf{x}, \sum_{k=1}^{i} \alpha_i \right)
$ for the simulation-free training of \cref{eq:loss_n}.
It should be noted that the standard identity in \cref{eq:standard_id} does not hold in the von Mises distribution. Hence there exists an important difference between the original Bayesian flow defined on Euclidean space and the Bayesian flow of circular data on $\mathbb{T}^D$ based on von Mises distribution. With prior $\btheta = \{\bold{0},\bold{0}\}$, we could formally represent the non-additive accuracy issue as:
% The additive accuracy property implies the fact that the "confidence" for the data sample after observing a series of the noisy samples with accuracy ${\alpha_1, \cdots, \alpha_i}$ could be  as the accuracy sum  which could be  
% Here we 
% Here we emphasize the specific property of BFN based on von Mises distribution.
% Note that 
% \begin{equation}
% \update(\parsn'' \mid \parsn, \x; \alpha_a+\alpha_b) \ne \E_{\update(\parsn' \mid \parsn, \x; \alpha_a)} \update(\parsn'' \mid \parsn', \x; \alpha_b)
% \end{equation}
% \oyyw{please check whether the below equation is better}
% \yuxuan{I fill somehow confusing on what is the update distribution with $\alpha$. }
% \begin{equation}
% \update(\parsn{}'' \mid \parsn{}, \x; \alpha_a+\alpha_b) \ne \E_{\update(\parsn{}' \mid \parsn{}, \x; \alpha_a)} \update(\parsn{}'' \mid \parsn{}', \x; \alpha_b)
% \end{equation}
% We give an intuitive visualization of such difference in \cref{fig:non_add}. The untenability of this property can materialize by considering the following case: with prior $\btheta = \{\bold{0},\bold{0}\}$, check the two-step Bayesian update distribution with $\alpha_a,\alpha_b$ and one-step Bayesian update with $\alpha=\alpha_a+\alpha_b$:
\begin{align}
\label{eq:nonadd}
     &\update(c'' \mid \parsn, \x; \alpha_a+\alpha_b)  = \delta(c-\alpha_a-\alpha_b)
     \ne  \mathbb{E}_{p_U(\parsn' \mid \parsn, \x; \alpha_a)}\update(c'' \mid \parsn', \x; \alpha_b) \nonumber \\&= \mathbb{E}_{vM(\bold{y}_b|\bold{x},\alpha_a)}\mathbb{E}_{vM(\bold{y}_a|\bold{x},\alpha_b)}\delta(c-||[\alpha_a \cos\y_a+\alpha_b\cos \y_b,\alpha_a \sin\y_a+\alpha_b\sin \y_b]^T||_2)
\end{align}
A more intuitive visualization could be found in \cref{fig:non_add}. This fundamental difference between periodic Bayesian flow and that of \citet{bfn} presents both theoretical and practical challenges, which we will explain and address in the following contents.

% This makes constructing Bayesian flow based on von Mises distribution intrinsically different from previous Bayesian flows (\citet{bfn}).

% Thus, we must reformulate the framework of Bayesian flow networks  accordingly. % and do necessary reformulations of BFN. 

% \yuxuan{overall I feel this part is complicated by using the language of update distribution. I would like to suggest simply use bayesian update, to provide intuitive explantion.}\hanlin{See the illustration in \cref{fig:non_add}}

% That introduces a cascade of problems, and we investigate the following issues: $(1)$ Accuracies between sender and receiver are not synchronized and need to be differentiated. $(2)$ There is no tractable Bayesian flow distribution for a one-step sample conditioned on a given time step $i$, and naively simulating the Bayesian flow results in computational overhead. $(3)$ It is difficult to control the entropy of the Bayesian flow. $(4)$ Accuracy is no longer a function of $t$ and becomes a distribution conditioned on $t$, which can be different across dimensions.
%\jj{Edited till here}

\textbf{Entropy Conditioning} As a common practice in generative models~\citep{ddpm,flowmatching,bfn}, timestep $t$ is widely used to distinguish among generation states by feeding the timestep information into the networks. However, this paper shows that for periodic Bayesian flow, the accumulated accuracy $\vc_i$ is more effective than time-based conditioning by informing the network about the entropy and certainty of the states $\parsnt{i}$. This stems from the intrinsic non-additive accuracy which makes the receiver's accumulated accuracy $c$ not bijective function of $t$, but a distribution conditioned on accumulated accuracies $\vc_i$ instead. Therefore, the entropy parameter $\vc$ is taken logarithm and fed into the network to describe the entropy of the input corrupted structure. We verify this consideration in \cref{sec:exp_ablation}. 
% \yuxuan{implement variant. traditionally, the timestep is widely used to distinguish the different states by putting the timestep embedding into the networks. citation of FM, diffusion, BFN. However, we find that conditioned on time in periodic flow could not provide extra benefits. To further boost the performance, we introduce a simple yet effective modification term entropy conditional. This is based on that the accumulated accuracy which represents the current uncertainty or entropy could be a better indicator to distinguish different states. + Describe how you do this. }



\textbf{Reformulations of BFN}. Recall the original update function with Gaussian sender distribution, after receiving noisy samples $\y_1,\y_2,\dots,\y_i$ with accuracies $\senderacc$, the accumulated accuracies of the receiver side could be analytically obtained by the additive property and it is consistent with the sender side.
% Since observing sample $\y$ with $\alpha_i$ can not result in exact accuracy increment $\alpha_i$ for receiver, the accuracies between sender and receiver are not synchronized which need to be differentiated. 
However, as previously mentioned, this does not apply to periodic Bayesian flow, and some of the notations in original BFN~\citep{bfn} need to be adjusted accordingly. We maintain the notations of sender side's one-step accuracy $\alpha$ and added accuracy $\beta$, and alter the notation of receiver's accuracy parameter as $c$, which is needed to be simulated by cascade of Bayesian updates. We emphasize that the receiver's accumulated accuracy $c$ is no longer a function of $t$ (differently from the Gaussian case), and it becomes a distribution conditioned on received accuracies $\senderacc$ from the sender. Therefore, we represent the Bayesian flow distribution of von Mises distribution as $p_F(\btheta|\x;\alpha_1,\alpha_2,\dots,\alpha_i)$. And the original simulation-free training with Bayesian flow distribution is no longer applicable in this scenario.
% Different from previous BFNs where the accumulated accuracy $\rho$ is not explicitly modeled, the accumulated accuracy parameter $c$ (visualized in \cref{fig:vmbf_vis}) needs to be explicitly modeled by feeding it to the network to avoid information loss.
% the randomaccuracy parameter $c$ (visualized in \cref{fig:vmbf_vis}) implies that there exists information in $c$ from the sender just like $m$, meaning that $c$ also should be fed into the network to avoid information loss. 
% We ablate this consideration in  \cref{sec:exp_ablation}. 

\textbf{Fast Sampling from Equivalent Bayesian Flow Distribution} Based on the above reformulations, the Bayesian flow distribution of von Mises distribution is reframed as: 
\begin{equation}\label{eq:flow_frac}
p_F(\btheta_i|\x;\alpha_1,\alpha_2,\dots,\alpha_i)=\E_{\update(\parsnt{1} \mid \parsnt{0}, \x ; \alphat{1})}\dots\E_{\update(\parsn_{i-1} \mid \parsnt{i-2}, \x; \alphat{i-1})} \update(\parsnt{i} | \parsnt{i-1},\x;\alphat{i} )
\end{equation}
Naively sampling from \cref{eq:flow_frac} requires slow auto-regressive iterated simulation, making training unaffordable. Noticing the mathematical properties of \cref{eq:h_m,eq:h_c}, we  transform \cref{eq:flow_frac} to the equivalent form:
\begin{equation}\label{eq:cirflow_equiv}
p_F(\vec{m}_i|\x;\alpha_1,\alpha_2,\dots,\alpha_i)=\E_{vM(\y_1|\x,\alpha_1)\dots vM(\y_i|\x,\alpha_i)} \delta(\vec{m}_i-\text{atan2}(\sum_{j=1}^i \alpha_j \cos \y_j,\sum_{j=1}^i \alpha_j \sin \y_j))
\end{equation}
\begin{equation}\label{eq:cirflow_equiv2}
p_F(\vec{c}_i|\x;\alpha_1,\alpha_2,\dots,\alpha_i)=\E_{vM(\y_1|\x,\alpha_1)\dots vM(\y_i|\x,\alpha_i)}  \delta(\vec{c}_i-||[\sum_{j=1}^i \alpha_j \cos \y_j,\sum_{j=1}^i \alpha_j \sin \y_j]^T||_2)
\end{equation}
which bypasses the computation of intermediate variables and allows pure tensor operations, with negligible computational overhead.
\begin{restatable}{proposition}{cirflowequiv}
The probability density function of Bayesian flow distribution defined by \cref{eq:cirflow_equiv,eq:cirflow_equiv2} is equivalent to the original definition in \cref{eq:flow_frac}. 
\end{restatable}
\textbf{Numerical Determination of Linear Entropy Sender Accuracy Schedule} ~Original BFN designs the accuracy schedule $\beta(t)$ to make the entropy of input distribution linearly decrease. As for crystal generation task, to ensure information coherence between modalities, we choose a sender accuracy schedule $\senderacc$ that makes the receiver's belief entropy $H(t_i)=H(p_I(\cdot|\vtheta_i))=H(p_I(\cdot|\vc_i))$ linearly decrease \emph{w.r.t.} time $t_i$, given the initial and final accuracy parameter $c(0)$ and $c(1)$. Due to the intractability of \cref{eq:vm_entropy}, we first use numerical binary search in $[0,c(1)]$ to determine the receiver's $c(t_i)$ for $i=1,\dots, n$ by solving the equation $H(c(t_i))=(1-t_i)H(c(0))+tH(c(1))$. Next, with $c(t_i)$, we conduct numerical binary search for each $\alpha_i$ in $[0,c(1)]$ by solving the equations $\E_{y\sim vM(x,\alpha_i)}[\sqrt{\alpha_i^2+c_{i-1}^2+2\alpha_i c_{i-1}\cos(y-m_{i-1})}]=c(t_i)$ from $i=1$ to $i=n$ for arbitrarily selected $x\in[-\pi,\pi)$.

After tackling all those issues, we have now arrived at a new BFN architecture for effectively modeling crystals. Such BFN can also be adapted to other type of data located in hyper-torus $\mathbb{T}^{D}$.

\subsection{Equivariant Bayesian Flow for Crystal}
With the above Bayesian flow designed for generative modeling of fractional coordinate $\vF$, we are able to build equivariant Bayesian flow for each modality of crystal. In this section, we first give an overview of the general training and sampling algorithm of \modelname (visualized in \cref{fig:framework}). Then, we describe the details of the Bayesian flow of every modality. The training and sampling algorithm can be found in \cref{alg:train} and \cref{alg:sampling}.

\textbf{Overview} Operating in the parameter space $\bthetaM=\{\bthetaA,\bthetaL,\bthetaF\}$, \modelname generates high-fidelity crystals through a joint BFN sampling process on the parameter of  atom type $\bthetaA$, lattice parameter $\vec{\theta}^L=\{\bmuL,\brhoL\}$, and the parameter of fractional coordinate matrix $\bthetaF=\{\bmF,\bcF\}$. We index the $n$-steps of the generation process in a discrete manner $i$, and denote the corresponding continuous notation $t_i=i/n$ from prior parameter $\thetaM_0$ to a considerably low variance parameter $\thetaM_n$ (\emph{i.e.} large $\vrho^L,\bmF$, and centered $\bthetaA$).

At training time, \modelname samples time $i\sim U\{1,n\}$ and $\bthetaM_{i-1}$ from the Bayesian flow distribution of each modality, serving as the input to the network. The network $\net$ outputs $\net(\parsnt{i-1}^\mathcal{M},t_{i-1})=\net(\parsnt{i-1}^A,\parsnt{i-1}^F,\parsnt{i-1}^L,t_{i-1})$ and conducts gradient descents on loss function \cref{eq:loss_n} for each modality. After proper training, the sender distribution $p_S$ can be approximated by the receiver distribution $p_R$. 

At inference time, from predefined $\thetaM_0$, we conduct transitions from $\thetaM_{i-1}$ to $\thetaM_{i}$ by: $(1)$ sampling $\y_i\sim p_R(\bold{y}|\thetaM_{i-1};t_i,\alpha_i)$ according to network prediction $\predM{i-1}$; and $(2)$ performing Bayesian update $h(\thetaM_{i-1},\y^\calM_{i-1},\alpha_i)$ for each dimension. 

% Alternatively, we complete this transition using the flow-back technique by sampling 
% $\thetaM_{i}$ from Bayesian flow distribution $\flow(\btheta^M_{i}|\predM{i-1};t_{i-1})$. 

% The training objective of $\net$ is to minimize the KL divergence between sender distribution and receiver distribution for every modality as defined in \cref{eq:loss_n} which is equivalent to optimizing the negative variational lower bound $\calL^{VLB}$ as discussed in \cref{sec:preliminaries}. 

%In the following part, we will present the Bayesian flow of each modality in detail.

\textbf{Bayesian Flow of Fractional Coordinate $\vF$}~The distribution of the prior parameter $\bthetaF_0$ is defined as:
\begin{equation}\label{eq:prior_frac}
    p(\bthetaF_0) \defeq \{vM(\vm_0^F|\vec{0}_{3\times N},\vec{0}_{3\times N}),\delta(\vc_0^F-\vec{0}_{3\times N})\} = \{U(\vec{0},\vec{1}),\delta(\vc_0^F-\vec{0}_{3\times N})\}
\end{equation}
Note that this prior distribution of $\vm_0^F$ is uniform over $[\vec{0},\vec{1})$, ensuring the periodic translation invariance property in \cref{De:pi}. The training objective is minimizing the KL divergence between sender and receiver distribution (deduction can be found in \cref{appd:cir_loss}): 
%\oyyw{replace $\vF$ with $\x$?} \hanlin{notations follow Preliminary?}
\begin{align}\label{loss_frac}
\calL_F = n \E_{i \sim \ui{n}, \flow(\parsn{}^F \mid \vF ; \senderacc)} \alpha_i\frac{I_1(\alpha_i)}{I_0(\alpha_i)}(1-\cos(\vF-\predF{i-1}))
\end{align}
where $I_0(x)$ and $I_1(x)$ are the zeroth and the first order of modified Bessel functions. The transition from $\bthetaF_{i-1}$ to $\bthetaF_{i}$ is the Bayesian update distribution based on network prediction:
\begin{equation}\label{eq:transi_frac}
    p(\btheta^F_{i}|\parsnt{i-1}^\calM)=\mathbb{E}_{vM(\bold{y}|\predF{i-1},\alpha_i)}\delta(\btheta^F_{i}-h(\btheta^F_{i-1},\bold{y},\alpha_i))
\end{equation}
\begin{restatable}{proposition}{fracinv}
With $\net_{F}$ as a periodic translation equivariant function namely $\net_F(\parsnt{}^A,w(\parsnt{}^F+\vt),\parsnt{}^L,t)=w(\net_F(\parsnt{}^A,\parsnt{}^F,\parsnt{}^L,t)+\vt), \forall\vt\in\R^3$, the marginal distribution of $p(\vF_n)$ defined by \cref{eq:prior_frac,eq:transi_frac} is periodic translation invariant. 
\end{restatable}
\textbf{Bayesian Flow of Lattice Parameter \texorpdfstring{$\boldsymbol{L}$}{}}   
Noting the lattice parameter $\bm{L}$ located in Euclidean space, we set prior as the parameter of a isotropic multivariate normal distribution $\btheta^L_0\defeq\{\vmu_0^L,\vrho_0^L\}=\{\bm{0}_{3\times3},\bm{1}_{3\times3}\}$
% \begin{equation}\label{eq:lattice_prior}
% \btheta^L_0\defeq\{\vmu_0^L,\vrho_0^L\}=\{\bm{0}_{3\times3},\bm{1}_{3\times3}\}
% \end{equation}
such that the prior distribution of the Markov process on $\vmu^L$ is the Dirac distribution $\delta(\vec{\mu_0}-\vec{0})$ and $\delta(\vec{\rho_0}-\vec{1})$, 
% \begin{equation}
%     p_I^L(\boldsymbol{L}|\btheta_0^L)=\mathcal{N}(\bm{L}|\bm{0},\bm{I})
% \end{equation}
which ensures O(3)-invariance of prior distribution of $\vL$. By Eq. 77 from \citet{bfn}, the Bayesian flow distribution of the lattice parameter $\bm{L}$ is: 
\begin{align}% =p_U(\bmuL|\btheta_0^L,\bm{L},\beta(t))
p_F^L(\bmuL|\bm{L};t) &=\mathcal{N}(\bmuL|\gamma(t)\bm{L},\gamma(t)(1-\gamma(t))\bm{I}) 
\end{align}
where $\gamma(t) = 1 - \sigma_1^{2t}$ and $\sigma_1$ is the predefined hyper-parameter controlling the variance of input distribution at $t=1$ under linear entropy accuracy schedule. The variance parameter $\vrho$ does not need to be modeled and fed to the network, since it is deterministic given the accuracy schedule. After sampling $\bmuL_i$ from $p_F^L$, the training objective is defined as minimizing KL divergence between sender and receiver distribution (based on Eq. 96 in \citet{bfn}):
\begin{align}
\mathcal{L}_{L} = \frac{n}{2}\left(1-\sigma_1^{2/n}\right)\E_{i \sim \ui{n}}\E_{\flow(\bmuL_{i-1} |\vL ; t_{i-1})}  \frac{\left\|\vL -\predL{i-1}\right\|^2}{\sigma_1^{2i/n}},\label{eq:lattice_loss}
\end{align}
where the prediction term $\predL{i-1}$ is the lattice parameter part of network output. After training, the generation process is defined as the Bayesian update distribution given network prediction:
\begin{equation}\label{eq:lattice_sampling}
    p(\bmuL_{i}|\parsnt{i-1}^\calM)=\update^L(\bmuL_{i}|\predL{i-1},\bmuL_{i-1};t_{i-1})
\end{equation}
    

% The final prediction of the lattice parameter is given by $\bmuL_n = \predL{n-1}$.
% \begin{equation}\label{eq:final_lattice}
%     \bmuL_n = \predL{n-1}
% \end{equation}

\begin{restatable}{proposition}{latticeinv}\label{prop:latticeinv}
With $\net_{L}$ as  O(3)-equivariant function namely $\net_L(\parsnt{}^A,\parsnt{}^F,\vQ\parsnt{}^L,t)=\vQ\net_L(\parsnt{}^A,\parsnt{}^F,\parsnt{}^L,t),\forall\vQ^T\vQ=\vI$, the marginal distribution of $p(\bmuL_n)$ defined by \cref{eq:lattice_sampling} is O(3)-invariant. 
\end{restatable}


\textbf{Bayesian Flow of Atom Types \texorpdfstring{$\boldsymbol{A}$}{}} 
Given that atom types are discrete random variables located in a simplex $\calS^K$, the prior parameter of $\boldsymbol{A}$ is the discrete uniform distribution over the vocabulary $\parsnt{0}^A \defeq \frac{1}{K}\vec{1}_{1\times N}$. 
% \begin{align}\label{eq:disc_input_prior}
% \parsnt{0}^A \defeq \frac{1}{K}\vec{1}_{1\times N}
% \end{align}
% \begin{align}
%     (\oh{j}{K})_k \defeq \delta_{j k}, \text{where }\oh{j}{K}\in \R^{K},\oh{\vA}{KD} \defeq \left(\oh{a_1}{K},\dots,\oh{a_N}{K}\right) \in \R^{K\times N}
% \end{align}
With the notation of the projection from the class index $j$ to the length $K$ one-hot vector $ (\oh{j}{K})_k \defeq \delta_{j k}, \text{where }\oh{j}{K}\in \R^{K},\oh{\vA}{KD} \defeq \left(\oh{a_1}{K},\dots,\oh{a_N}{K}\right) \in \R^{K\times N}$, the Bayesian flow distribution of atom types $\vA$ is derived in \citet{bfn}:
\begin{align}
\flow^{A}(\parsn^A \mid \vA; t) &= \E_{\N{\y \mid \beta^A(t)\left(K \oh{\vA}{K\times N} - \vec{1}_{K\times N}\right)}{\beta^A(t) K \vec{I}_{K\times N \times N}}} \delta\left(\parsn^A - \frac{e^{\y}\parsnt{0}^A}{\sum_{k=1}^K e^{\y_k}(\parsnt{0})_{k}^A}\right).
\end{align}
where $\beta^A(t)$ is the predefined accuracy schedule for atom types. Sampling $\btheta_i^A$ from $p_F^A$ as the training signal, the training objective is the $n$-step discrete-time loss for discrete variable \citep{bfn}: 
% \oyyw{can we simplify the next equation? Such as remove $K \times N, K \times N \times N$}
% \begin{align}
% &\calL_A = n\E_{i \sim U\{1,n\},\flow^A(\parsn^A \mid \vA ; t_{i-1}),\N{\y \mid \alphat{i}\left(K \oh{\vA}{KD} - \vec{1}_{K\times N}\right)}{\alphat{i} K \vec{I}_{K\times N \times N}}} \ln \N{\y \mid \alphat{i}\left(K \oh{\vA}{K\times N} - \vec{1}_{K\times N}\right)}{\alphat{i} K \vec{I}_{K\times N \times N}}\nonumber\\
% &\qquad\qquad\qquad-\sum_{d=1}^N \ln \left(\sum_{k=1}^K \out^{(d)}(k \mid \parsn^A; t_{i-1}) \N{\ydd{d} \mid \alphat{i}\left(K\oh{k}{K}- \vec{1}_{K\times N}\right)}{\alphat{i} K \vec{I}_{K\times N \times N}}\right)\label{discdisc_t_loss_exp}
% \end{align}
\begin{align}
&\calL_A = n\E_{i \sim U\{1,n\},\flow^A(\parsn^A \mid \vA ; t_{i-1}),\N{\y \mid \alphat{i}\left(K \oh{\vA}{KD} - \vec{1}\right)}{\alphat{i} K \vec{I}}} \ln \N{\y \mid \alphat{i}\left(K \oh{\vA}{K\times N} - \vec{1}\right)}{\alphat{i} K \vec{I}}\nonumber\\
&\qquad\qquad\qquad-\sum_{d=1}^N \ln \left(\sum_{k=1}^K \out^{(d)}(k \mid \parsn^A; t_{i-1}) \N{\ydd{d} \mid \alphat{i}\left(K\oh{k}{K}- \vec{1}\right)}{\alphat{i} K \vec{I}}\right)\label{discdisc_t_loss_exp}
\end{align}
where $\vec{I}\in \R^{K\times N \times N}$ and $\vec{1}\in\R^{K\times D}$. When sampling, the transition from $\bthetaA_{i-1}$ to $\bthetaA_{i}$ is derived as:
\begin{equation}
    p(\btheta^A_{i}|\parsnt{i-1}^\calM)=\update^A(\btheta^A_{i}|\btheta^A_{i-1},\predA{i-1};t_{i-1})
\end{equation}

The detailed training and sampling algorithm could be found in \cref{alg:train} and \cref{alg:sampling}.




\section{Offline Evaluation}
\label{sec:offline-evaluation}

This section presents a comprehensive analysis of our simulation model to understand the dynamics of opinion polarization and user behavior in online social networks. Our evaluation serves as a foundation for the subsequent experimental user study and aims to validate that our model captures realistic aspects of online discourse while reflecting theoretically grounded mechanisms of opinion formation and polarization.

Table~\ref{tab:offline-evaluation-findings} provides an overview of our key findings across three main analyses. First, we examine polarization dynamics under varying conditions, investigating how different recommendation algorithms---through varied same-stance ratios $h_r$ (cf. Section~\ref{subsec:social-network-model})---and initial opinion distributions affect the evolution of opinions across the network. This analysis focuses particularly on the role of cross-ideological exposure in opinion formation. Second, we analyze the evolution of user engagement patterns in different network configurations. This investigation explores how reaction probabilities and interaction patterns develop over time, helping us understand the relationship between opinion polarization and user behavior in our simulated environment. Finally, we perform a content-based analysis of messages generated within our simulated networks. This examination focuses on message characteristics such as emotionality, uncertainty, and group identity salience, providing insights into how content features vary under different network conditions. Through these three analyses, we aim to validate our model's capacity to reproduce empirically observed patterns of online polarization while offering a detailed understanding of the underlying mechanics of our simulation framework.

\begin{table*}[h!]
\centering
\small
\caption{Key Findings from Offline Evaluation}
\label{tab:offline-evaluation-findings}
\begin{tabularx}{\textwidth}{>{\centering\arraybackslash}p{1.5cm}>{\raggedright\arraybackslash}X>{\raggedright\arraybackslash}X}
\toprule
\multicolumn{1}{c}{\textbf{Analysis}} & \multicolumn{1}{c}{\textbf{Key Finding}} & \multicolumn{1}{c}{\textbf{Theoretical Implication}} \\
\midrule
\multirow{2}{*}{\rotatebox[origin=c]{90}{\textbf{Polarization}}} 
    & Moderate cross-ideological exposure leads to higher polarization than complete echo chambers 
    & Supports theories that political identities strengthen through opposition rather than isolation; polarization intensifies through awareness of opposing views \\
\cmidrule{2-3}
    & Polarization emerges primarily through the presence of influential accounts at extreme positions, not through natural opinion drift 
    & Confirms role of opinion leaders in transforming neutral topics into partisan issues; suggests polarization requires active curation rather than emerging spontaneously \\
\midrule
\multirow{2}{*}{\rotatebox[origin=c]{90}{\textbf{Interaction}}} 
    & Different interaction types serve distinct social functions: likes show strict in-group preference while comments flourish across ideological lines 
    & Reflects how platform affordances shape identity expression; suggests comments often serve as vehicles for performative disagreement rather than dialogue \\
\cmidrule{2-3}
    & Influencer content becomes focal point for engagement in polarized conditions, creating self-reinforcing cycles 
    & Shows how influential accounts serve as lightning rods for cross-ideological conflict; engagement patterns amplify rather than reduce polarization \\
\midrule
\multirow{2}{*}{\rotatebox[origin=c]{90}{\textbf{Content}}} 
    & Polarized environments produce consistent changes in communication: increased group identity emphasis, higher emotional content, and reduced expressions of uncertainty 
    & Demonstrates how polarization fundamentally transforms communication style independent of specific topics; suggests activation of group identities changes how people express themselves \\
\cmidrule{2-3}
    & These changes in communication style occur independently of content exposure patterns 
    & Indicates that the social-psychological dynamics of polarized environments, rather than mere exposure to different views, drive changes in how people communicate \\
\bottomrule
\end{tabularx}
\end{table*}

For our simulations, we selected Universal Basic Income (UBI) as the focal topic of discussion. This choice was driven by several strategic considerations. Unlike heavily polarized topics where individuals often hold entrenched positions, UBI represents an emerging policy proposal where public opinion remains relatively malleable, making it ideal for studying opinion formation and polarization dynamics. While UBI evokes fewer preset opinions, it remains sufficiently concrete and consequential to generate meaningful discourse, with its complexity spanning economic, social, and technological dimensions. Recent polling data supports UBI's suitability, showing a balanced distribution of opinions with approximately $51.2\%$ of Europeans \citep{vlandas_politics_2019} and $48\%$ of Americans \citep{hamilton_people_2022} expressing support, while significant portions remain undecided or hold moderate views. Additionally, UBI's limited real-world implementation means participants' opinions are more likely to be based on theoretical arguments rather than direct experience or partisan allegiances, allowing us to examine how social media dynamics can influence opinion formation before entrenched polarization takes hold.

\subsection{Polarization Analysis}

\subsubsection{Experimental Design}

This study employs a factorial design to investigate the effects of \emph{Recommendation Homophily} and \emph{Initial Opinion Distribution} on social polarization in online networks. The experiment aims to elucidate the dynamics of opinion formation and the emergence of polarization under various conditions.

We manipulate two primary independent variables. For \emph{Recommendation Homophily}, characterized by the homophily parameter $h_r$, we examine two levels: \emph{High Homophily} ($h_r = 1.0$) and \emph{Moderate Homophily} ($h_r = 0.7$). For \emph{Initial Opinion Distribution}, we investigate three variants: a \emph{Unimodal Distribution} ($\mu = 0$, $\sigma = 0.03$), a \emph{Symmetric Trimodal Distribution} with extreme influencers at both ends of the opinion spectrum ($\mu_1 = 0.8$, $\mu_3 = -0.8$) and regular users initialized around $\mu_2 = 0$, and an \emph{Asymmetric Trimodal Distribution} with one-sided extreme influencers ($\mu_1 = 0.8$, $\mu_3 = -0.1$).

The primary dependent variables are the Esteban-Ray polarization index \citep{esteban_on_1994}, which measures the overall polarization within the network, and the evolution of individual opinions $o_i(t)$, which tracks individual agent opinion trajectories over time.

The simulation is configured with $n = 20$ agents over $15$ iterations, with $k = 8$ recommendations per iteration. Regular users have a posting probability $p_{reg} = 0.2$, while influencers post with probability $p_{inf} = 0.6$. The network includes $4$ influencers. These parameters were selected to balance computational feasibility with the need for sufficient interactions to observe meaningful dynamics. The opinion update function follows the parameterization described in Section~\ref{subsec:agent-model}, with a base learning rate $\lambda = 0.2$, selected based on preliminary experiments.

\begin{figure}[htbp]
    \centering
    \includegraphics[width=\textwidth]{figures/comparative_polarization_plot.png}
    \caption{Evolution of opinion polarization across different user groups and recommendation settings. The figure shows polarization dynamics for (left) the entire user population, (middle) regular users, and (right) influencers over multiple iterations. Lines represent different combinations of recommendation bias (SSR: Same Stance Ratio) and initial opinion distributions. Notably, moderate cross-ideological exposure (70\% SSR, solid lines) leads to higher polarization than complete echo chambers (100\% SSR, dashed lines), particularly in trimodal distributions. This counterintuitive finding challenges the echo chamber hypothesis, suggesting that awareness of opposing views can intensify polarization through antagonistic group dynamics. The effect varies across different initial opinion distributions: Unimodal, Trimodal (three distinct opinion clusters), and Asymmetric Trimodal (three unequally weighted clusters).}
    \label{fig:polarization-index}
\end{figure}


\subsubsection{Results}

The polarization index across different experimental conditions reveals distinct patterns of opinion dynamics and group polarization (see Figure~\ref{fig:polarization-index}). Our analysis combines visual inspection of polarization trends with statistical comparisons using t-tests and Area Under the Curve (AUC) measurements (see Table~\ref{tab:polarization-comparison}).

In scenarios with a \emph{Unimodal Distribution}, we observe relatively low levels of polarization. The condition with \emph{Moderate Homophily} ($h_r = 0.7$) shows minimal polarization, with the polarization index fluctuating between $0.02$ and $0.04$ throughout the $15$ iterations. Even with \emph{High Homophily} ($h_r = 1.0$), the index remains comparatively low, stabilizing around $0.13$-$0.14$ after an initial increase. This observation is supported by the low AUC values for \emph{Unimodal Distribution} ($0.850$ and $1.760$ for $h_r = 0.7$ and $h_r = 1.0$, respectively) compared to the \emph{Trimodal Distribution} conditions.

Conversely, the \emph{Symmetric Trimodal Distribution} leads to substantially higher polarization. In the case of \emph{High Homophily} and \emph{Symmetric Trimodal Distribution} ($h_r = 1.0$), we see a steady increase from $0.096$ to $0.816$ over the $15$ iterations. The effect of initial distribution is statistically significant at both $h_r$ levels. For regular users at $h_r = 0.7$, the difference is substantial ($t = 7.987$, $p < .001$), with AUC values of $14.629$ for \emph{Symmetric Trimodal Distribution} versus $0.850$ for \emph{Unimodal Distribution}.

\begin{figure}[htbp]
    \centering
    \begin{minipage}[b]{0.3\textwidth}
        \centering
        \includegraphics[width=\textwidth]{figures/opinion_plots/individual_opinions_ssr_0.7_normal_2.png}
        \subcaption{$h_r = 0.7$, unimodal distribution}
        \label{fig:subfig1}
    \end{minipage}
    \hfill
    \begin{minipage}[b]{0.3\textwidth}
        \centering
        \includegraphics[width=\textwidth]{figures/opinion_plots/individual_opinions_ssr_0.7_trimodal.png}
        \subcaption{$h_r = 0.7$, trimodal}
        \label{fig:subfig2}
    \end{minipage}
    \hfill
    \begin{minipage}[b]{0.3\textwidth}
        \centering
        \includegraphics[width=\textwidth]{figures/opinion_plots/individual_opinions_ssr_0.7_trimodal_one.png}
        \subcaption{$h_r = 0.7$, asym. trimodal }
        \label{fig:subfig3}
    \end{minipage}
    
    \vspace{1em}
    
    \begin{minipage}[b]{0.3\textwidth}
        \centering
        \includegraphics[width=\textwidth]{figures/opinion_plots/individual_opinions_ssr_1.0_normal.png}
        \subcaption{$h_r = 1.0$, unimodal}
        \label{fig:subfig4}
    \end{minipage}
    \hfill
    \begin{minipage}[b]{0.3\textwidth}
        \centering
        \includegraphics[width=\textwidth]{figures/opinion_plots/individual_opinions_ssr_1.0_trimodal.png}
        \subcaption{$h_r = 1.0$, trimodal}
        \label{fig:subfig5}
    \end{minipage}
    \hfill
    \begin{minipage}[b]{0.3\textwidth}
        \centering
        \includegraphics[width=\textwidth]{figures/opinion_plots/individual_opinions_ssr_1.0_trimodal_one.png}
        \subcaption{$h_r = 1.0$, asym. trimodal}
        \label{fig:subfig6}
    \end{minipage}
    
    \caption{Opinion evolution for each individual user over $T = 15$ iterations subject to different same-stance connectivity probabilities ($h_r$) and initial opinion distributions (unimodal vs. trimodal. vs. asymmetric trimodal).}
    \label{fig:individual-polarization}
\end{figure}

The same-stance probability $h_r$ shows a substantial effect on polarization dynamics, particularly in \emph{Trimodal Distribution} scenarios. Contrary to the echo chamber hypothesis, \emph{Moderate Homophily} ($h_r = 0.7$) results in significantly higher polarization ($1.492$) compared to \emph{High Homophily} ($h_r = 1.0$, $0.816$). This is reflected in the AUC values for regular users ($14.629$ vs $7.239$, $t = 3.809$, $p = .001$). This finding suggests that, in our model, exposure to opposing views, rather than pure echo chambers, actually intensifies polarization through heightened awareness of ideological differences.

\begin{table}[h!]
\centering
\caption{Effect of Initial Distribution, Homophily, and Symmetry on Polarization}
\label{tab:polarization-comparison}
\footnotesize
\begin{tabularx}{\textwidth}{>{\raggedright\arraybackslash}p{2.8cm}>{\centering\arraybackslash}p{1.2cm}*{6}{>{\centering\arraybackslash}X}}
\toprule
\multirow{2}{*}{\textbf{Exp. Factor}} & \multirow{2}{*}{\textbf{Users}} & \multicolumn{2}{c}{\textbf{AUC}} & \multicolumn{2}{c}{\textbf{Slope}} & \multicolumn{2}{c}{\textbf{Statistics}} \\
\cmidrule(lr){3-4} \cmidrule(lr){5-6} \cmidrule(lr){7-8}
 &  & Cond. 1 & Cond. 2 & Cond. 1 & Cond. 2 & $t$ & $p$ \\
\midrule
\textbf{Initial Distribution} & & \multicolumn{6}{l}{\textit{Trimodal vs Unimodal, $h_r = 0.7$}} \\
    & Overall & \textbf{16.628} & 0.845 & \textbf{0.071} & 0.002 & \textbf{13.057} & <.001*** \\
    & Regular & \textbf{14.629} & 0.850 & \textbf{0.101} & 0.003 & \textbf{7.987} & <.001*** \\
    & Influencers & \textbf{21.213} & 0.627 & \textbf{0.001} & -0.001 & \textbf{475.880} & <.001*** \\
\addlinespace[0.5em]
& & \multicolumn{6}{l}{\textit{Trimodal vs Unimodal, $h_r = 1.0$}} \\
    & Overall & \textbf{11.343} & 1.760 & \textbf{0.042} & 0.003 & \textbf{13.531} & <.001*** \\
    & Regular & \textbf{7.239} & 1.760 & \textbf{0.053} & 0.004 & \textbf{6.241} & <.001*** \\
    & Influencers & \textbf{20.627} & 1.357 & \textbf{0.017} & 0.001 & \textbf{44.437} & <.001*** \\
\midrule
\textbf{Homophily Level} & & \multicolumn{6}{l}{\textit{Moderate vs High ($h_r = 0.7$ vs $1.0$), Symmetric}} \\
    & Overall & \textbf{16.628} & 11.343 & \textbf{0.071} & 0.042 & \textbf{3.771} & .001** \\
    & Regular & \textbf{14.629} & 7.239 & \textbf{0.101} & 0.053 & \textbf{3.809} & .001** \\
    & Influencers & \textbf{21.213} & 20.627 & 0.001 & \textbf{0.017} & 1.612 & .118 \\
\addlinespace[0.5em]
& & \multicolumn{6}{l}{\textit{Moderate vs High ($h_r = 0.7$ vs $1.0$), Asymmetric}} \\
    & Overall & \textbf{8.402} & 4.938 & \textbf{0.039} & -0.010 & \textbf{5.236} & <.001*** \\
    & Regular & \textbf{5.079} & 1.176 & \textbf{0.052} & 0.007 & \textbf{4.620} & <.001*** \\
    & Influencers & \textbf{14.457} & 10.752 & \textbf{0.013} & -0.052 & \textbf{3.974} & <.001*** \\
\midrule
\textbf{Symmetry} & & \multicolumn{6}{l}{\textit{Symmetric vs Asymmetric, $h_r = 0.7$}} \\
    & Overall & \textbf{16.628} & 4.938 & \textbf{0.071} & -0.010 & \textbf{9.548} & <.001*** \\
    & Regular & \textbf{14.629} & 1.176 & \textbf{0.101} & 0.007 & \textbf{7.782} & <.001*** \\
    & Influencers & \textbf{21.213} & 10.752 & \textbf{0.001} & -0.052 & \textbf{12.407} & <.001*** \\
\addlinespace[0.5em]
& & \multicolumn{6}{l}{\textit{Symmetric vs Asymmetric, $h_r = 1.0$}} \\
    & Overall & \textbf{11.343} & 8.402 & \textbf{0.042} & 0.039 & \textbf{3.042} & .005** \\
    & Regular & \textbf{7.239} & 5.079 & \textbf{0.053} & 0.052 & 1.703 & .100 \\
    & Influencers & \textbf{20.627} & 14.457 & \textbf{0.017} & 0.013 & \textbf{10.960} & <.001*** \\
\bottomrule
\multicolumn{8}{p{.95\linewidth}}{\small \textbf{Note:} AUC = Area Under the Curve. "Cond. 1" and "Cond. 2" columns represent the respective conditions being compared. Bold values indicate larger values between conditions for AUC and Slope, and significant t-statistics ($p$ < .05). Positive t-statistics indicate higher polarization in the first condition. *$p$ < .05, **$p$ < .01, ***$p$ < .001} \\
\end{tabularx}
\end{table}

The condition with \emph{Asymmetric Trimodal Distribution} ($\mu_1 = 0.8$, $\mu_3 = -0.1$) demonstrates that even one-sided extreme opinions can drive significant polarization under \emph{High Homophily}, with the polarization index increasing from $0.024$ to $0.740$. However, reducing homophily ($h_r = 0.7$) effectively mitigates this polarization, with polarization peaking at only $0.148$. This mitigation is confirmed by significantly different AUC values ($5.079$ vs $1.176$ for regular users, $t = -4.620$, $p < .001$).

The evolution of individual opinions reveals distinct patterns across experimental conditions (see Figure~\ref{fig:individual-polarization}). In scenarios with \emph{Unimodal Distribution}, opinions remain relatively moderate regardless of homophily level. Under \emph{Moderate Homophily} ($h_r = 0.7$), opinions converge toward neutral values ($o_i \approx 0$), while \emph{High Homophily} ($h_r = 1.0$) induces slight opinion segregation ($o_i \rightarrow \pm0.2$) while maintaining overall moderate positions.

\emph{Symmetric Trimodal Distribution} generates more pronounced opinion dynamics. Under \emph{Moderate Homophily}, regular users migrate toward extreme positions ($o_i = \pm1$), particularly those starting near neutral positions. Interestingly, \emph{High Homophily} conditions show less extreme polarization, with opinions settling at moderate-high values ($|o_i| \approx 0.8$), suggesting that extreme homophily may paradoxically moderate opinion extremity.

The \emph{Asymmetric Trimodal Distribution} with one-sided extreme influencers produces distinct dynamics based on homophily levels. Under \emph{Moderate Homophily}, opinions cluster around moderate-high positive values ($o_i \approx 0.65$). \emph{High Homophily} amplifies this effect on the side with extreme influencers ($o_i \rightarrow 0.8$) while maintaining moderate opinions ($o_i \approx -0.1$) on the opposite side.

These results demonstrate the complex interplay between \emph{Initial Opinion Distribution} and \emph{Recommendation Homophily} in shaping polarization dynamics. While initial distributions largely determine polarization potential, homophily can either amplify or moderate these effects depending on the network configuration. The presence of extreme influencers significantly impacts opinion evolution, but their influence can be moderated through careful recommendation system design. Notably, \emph{Moderate Homophily} alone proves insufficient to prevent polarization when extreme opinions are present.

\subsubsection{Discussion}

The results of our polarization analysis reveal complex dynamics that both challenge and refine current understanding of opinion formation in social networks. Our findings particularly illuminate three key aspects of online polarization: the role of cross-ideological exposure, the impact of influential actors, and the mechanisms of opinion drift.

Our most crucial finding—that \emph{Moderate Homophily} ($h_r = 0.7$) leads to higher polarization than \emph{High Homophily} ($h_r = 1.0$)—aligns with fundamental theories of political identity formation. This result supports Laclau and Mouffe's concept of the constitutive outside, suggesting that political identities are strengthened through recognition of opposing views rather than isolation \citep{laclau_hegemony_2014}. The higher polarization observed under \emph{Moderate Homophily} demonstrates how awareness of opposing viewpoints can intensify group identities through what Bateson terms "complementary schismogenesis"—a process where interaction between groups reinforces their differences rather than diminishing them \citep{bateson_naven_1958}. This finding challenges the conventional narrative about echo chambers while aligning with recent research on affective polarization, which suggests that mere exposure to opposing views can exacerbate rather than reduce political animosity \citep{bail_exposure_2018,suhay_polarizing_2018}.

The crucial role of extreme influencers in driving polarization, evidenced by the stark difference between \emph{Trimodal Distribution} (polarization $> 0.8$) and \emph{Unimodal Distribution} (polarization $\approx 0.03$), supports the two-step flow theory of communication, suggesting that mass opinion formation is mediated by opinion leaders who serve as focal points for attitude formation \citep{katz_personal_2017}. Historical examples support this pattern: immigration policy in the United States remained largely non-partisan until the 1990s, when influential political actors began framing it as a partisan issue \citep{smeltz_are_2022}. Similarly, climate change was once a bipartisan concern, with both Republican and Democratic leaders supporting environmental protection measures in the 1970s \citep{capstick_international_2015}. Even public health measures, historically viewed through a primarily scientific lens, became deeply polarized during the COVID-19 pandemic through the active efforts of political and media influencers. These cases demonstrate how previously neutral topics can become focal points for political division through strategic positioning by influential actors.

In contrast, the dynamics we observe in the \emph{Unimodal Distribution} scenarios are particularly informative for understanding natural opinion drift. Without external triggers or strong personalities championing fringe viewpoints, opinions tend to remain clustered around moderate positions ($\approx 0.03$). This supports coordination game theory's prediction that groups tend to converge on central focal points unless pulled toward extremes by external forces \citep{schelling_the_1980}. The relative stability of moderate opinions in these scenarios suggests that polarization often requires active curation rather than emerging spontaneously from network dynamics alone.

The \emph{Asymmetric Trimodal Distribution} ($\mu_1 = 0.8$, $\mu_3 = -0.1$) further demonstrates the relationship between network homophily and opinion leadership. Under \emph{Moderate Homophily} ($h_r = 0.7$), the absence of counter-balancing extreme influences produces a clear pattern: all initially moderate users from the unrepresented side gradually shift toward the opposing extreme position. This simulation result illustrates a theoretical scenario where moderate cross-ideological exposure combined with unilateral influential voices leads to systematic opinion shifts. In contrast, under \emph{High Homophily} ($h_r = 1.0$), the model shows distinct segregation: users exposed to extreme influences adopt more radical positions, while the isolated opposing group maintains moderate positions due to lack of exposure - a pattern more closely aligned with traditional echo chamber dynamics. While these conditions are necessarily simplified, they help isolate potential mechanisms of opinion formation under different network structures. The contrast between these two conditions demonstrates how network homophily can fundamentally alter the pattern of opinion evolution in our simulated environment. However, note that these simplified conditions serve primarily to demonstrate specific theoretical mechanisms within our model, acknowledging both the limitations of simulation-based findings and the greater complexity of real-world opinion dynamics.

Summarized, this analysis suggests that online polarization emerges from a complex interplay of social identity, opinion leadership, and network structure, rather than from simple exposure patterns. Understanding and addressing polarization requires attention to these multiple, interacting mechanisms rather than focusing on single factors like echo chambers or algorithmic curation alone. The path to reducing polarization may lie not in simply increasing exposure to diverse viewpoints, but in fundamentally rethinking how social platforms structure interaction patterns and manage the flow of influence through their networks.

\subsection{Interaction Analysis}

\subsubsection{Experimental Design}
The evaluation of interaction patterns analyzes how varying \emph{Initial Opinion Distributions} affect reaction patterns in our model. We simulated a network of $30$ agents ($24$ regular users, $3$ influencers for each stance) over $10$ iterations, with $8$ message recommendations per agent per iteration. Regular users had a posting probability of $0.2$, while influencers posted with probability $0.6$.

The comparison contrasts \emph{Polarized} and \emph{Unpolarized} conditions. The \emph{Polarized Condition} used a bimodal opinion distribution (centered at $\pm0.8$, $\sigma = 0.1$), representing a divided community, while the \emph{Unpolarized Condition} employed a unimodal distribution (centered at $0$, $\sigma = 0.1$) for a more cohesive starting point. The agents maintained their initially assigned opinion values throughout the simulation.

The reaction mechanism implements distinct hyperparameters for different engagement types: likes ($p_b = 0.7$, $c = 0.0$), reposts ($p_b = 0.3$, $c = 0.1$), and comments ($p_b = 0.3$, $c = 0.5$). All reaction types share an opinion strength importance of $w = 0.8$. Throughout the simulation, we tracked reaction frequencies, their distribution across opinion differences, and cross-stance interaction patterns.

\subsubsection{Results}

Mixed effects modeling revealed substantial enhancement of engagement in \emph{Polarized Conditions} across reaction types (see Table~\ref{tab:reaction-analysis}). For reactions submitted, \emph{Polarization} demonstrated significant positive effects across most interaction types, with particularly strong effects in influencers' commenting behavior ($\beta = 5.900$, $p = .018$). Temporal effects indicated consistent positive growth across all reaction types, with influencers showing the strongest growth rates in both likes ($\beta = 0.819$, $p < .001$) and comments ($\beta = 0.906$, $p < .001$).

\begin{figure}[h]
    \centering
    \includegraphics[width=\textwidth]{figures/comparative_reactions_plot.png}
    \caption{The reactions submitted and received (likes, comments, and reposts) are presented for both the polarized and non-polarized conditions. A distinction is made between influencers and regular users.}
    \label{fig:influencer-regular-reactions}
\end{figure}


The analysis of reactions received demonstrated pronounced \emph{Polarization Effects}, especially for influencers compared to regular users, as illustrated in Figure~\ref{fig:influencer-regular-reactions}. Influencer-generated content elicited substantially more engagement in \emph{Polarized Conditions}, particularly for comments ($\beta = 8.633$, $p = .026$), with robust temporal growth ($\beta = 1.620$, $p < .001$). Regular users showed more modest but significant positive effects of \emph{Polarization} across all reaction types.

\begin{figure}[htbp]
    \centering
    \includegraphics[width=\textwidth]{figures/comparative_stance_reactions_plot.png}
    \caption{The same-stance and opposite-stance reactions (likes, comments, and reposts) are presented for both the polarized and non-polarized conditions. A distinction is made between influencers and regular users.}
    \label{fig:stance-reactions}
\end{figure}


Stance-based analysis revealed distinct patterns for cross-stance engagement (Figure~\ref{fig:stance-reactions}). Same-stance likes showed strong positive effects under \emph{Polarization} ($\beta = 3.127$, $p = .014$), while opposing-stance likes were very rare ($\beta = -1.663$, $p = .019$). Notably, opposing-stance likes were entirely absent in the \emph{Polarized Condition}, while same-stance engagement flourished. Comments exhibited positive \emph{Polarization Effects} for both same-stance ($\beta = 0.807$, $p = .010$) and opposing-stance interactions ($\beta = 1.680$, $p = .022$).

A particularly notable pattern emerges in the analysis of cross-stance interactions. While positive endorsements (likes) show a clear in-group bias ($\beta = 3.127$, $p = .014$ for same-stance vs. $\beta = -1.663$, $p = .019$ for opposing-stance), more substantive forms of engagement exhibit different patterns. Comments show positive \emph{Polarization Effects} for both same-stance ($\beta = 0.807$, $p = .010$) and opposing-stance interactions ($\beta = 1.680$, $p = .022$), with opposing-stance comments actually showing a stronger effect. Reposts occupy a middle ground, with significant positive effects for same-stance sharing ($\beta = 0.773$, $p = .017$) but no significant effect for opposing-stance sharing ($\beta = -0.083$, $p = .343$). These patterns suggest a qualitative difference in how users engage with opposing viewpoints across different interaction mechanisms.

These findings indicate that \emph{Polarization} fundamentally increases interaction dynamics, with particularly pronounced effects on influencer engagement and stance-based reactions, as evident in Table~\ref{tab:reaction-analysis}. The temporal effects demonstrate consistent growth patterns across conditions, with \emph{Polarization} amplifying the magnitude of engagement while maintaining stable growth trajectories.

\subsubsection{Discussion}

The simulation results demonstrate interaction patterns that align with several key phenomena observed in real social media environments while also revealing some notable divergences. The simulated amplification of engagement under \emph{Polarized Conditions} mirrors documented patterns on online platforms, where polarizing content typically generates higher engagement rates \citep{simchon_troll_2022, horwitz_facebook_2020}. Particularly notable is our model's reproduction of the distinctive role of influencers in \emph{Polarized Discourse}, reflecting empirical observations of opinion leader effects in online political discussions \citep{soares_influencers_2018, dubois_multiple_2014}.

\begin{table}[h!]
\centering
\footnotesize
\caption{Mixed Effects Analysis of User Reactions}
\label{tab:reaction-analysis}
\begin{tabularx}{\textwidth}{>{\raggedright\arraybackslash}p{2cm}>{\raggedright\arraybackslash}p{2cm}*{6}{>{\centering\arraybackslash}X}}
\toprule
\multirow{2}{*}{\textbf{Reaction}} & \multirow{2}{*}{\textbf{User Type}} & \multicolumn{2}{c}{\textbf{Polarization Effect}} & \multicolumn{2}{c}{\textbf{Temporal Effect}} & \multicolumn{2}{c}{\textbf{AUC}} \\
\cmidrule(lr){3-4} \cmidrule(lr){5-6} \cmidrule(lr){7-8}
 &  & $\beta$ & $p$ & $\beta$ & $p$ & Pol. & Unpol. \\
\midrule
\multicolumn{8}{l}{\textit{\textbf{Reactions Submitted}}} \\
\midrule
\multirow{3}{*}{Likes} 
    & Overall & \textbf{1.463} & 0.012* & \textbf{0.792} & <.001*** & \textbf{45.600} & 32.183 \\
    & Influencers & \textbf{3.317} & 0.001** & \textbf{0.819} & <.001*** & \textbf{55.917} & 25.250 \\
    & Regular & \textbf{1.000} & 0.044* & \textbf{0.785} & <.001*** & \textbf{43.021} & 33.917 \\
\midrule
\multirow{3}{*}{Comments} 
    & Overall & \textbf{2.487} & 0.017* & \textbf{0.392} & <.001*** & \textbf{30.400} & 7.950 \\
    & Influencers & \textbf{5.900} & 0.018* & \textbf{0.906} & <.001*** & \textbf{69.417} & 16.417 \\
    & Regular & \textbf{1.633} & 0.017* & \textbf{0.264} & <.001*** & \textbf{20.646} & 5.833 \\
\midrule
\multirow{3}{*}{Reposts} 
    & Overall & 0.690 & 0.066 & \textbf{0.237} & <.001*** & \textbf{14.250} & 8.150 \\
    & Influencers & 0.667 & 0.333 & \textbf{0.511} & <.001*** & \textbf{23.750} & 17.917 \\
    & Regular & \textbf{0.696} & 0.022* & \textbf{0.168} & <.001*** & \textbf{11.875} & 5.708 \\
\midrule
\multicolumn{8}{l}{\textit{\textbf{Reactions Received}}} \\
\midrule
\multirow{3}{*}{Likes} 
    & Overall & \textbf{1.463} & 0.012* & \textbf{0.792} & <.001*** & \textbf{45.600} & 32.183 \\
    & Influencers & 0.900 & 0.258 & \textbf{3.330} & <.001*** & \textbf{150.417} & 142.333 \\
    & Regular & \textbf{1.604} & 0.025* & \textbf{0.157} & <.001*** & \textbf{19.396} & 4.646 \\
\midrule
\multirow{3}{*}{Comments} 
    & Overall & \textbf{2.487} & 0.017* & \textbf{0.392} & <.001*** & \textbf{30.400} & 7.950 \\
    & Influencers & \textbf{8.633} & 0.026* & \textbf{1.620} & <.001*** & \textbf{110.917} & 33.750 \\
    & Regular & \textbf{0.950} & 0.012* & \textbf{0.085} & <.001*** & \textbf{10.271} & 1.500 \\
\midrule
\multirow{3}{*}{Reposts} 
    & Overall & 0.690 & 0.066 & \textbf{0.237} & <.001*** & \textbf{14.250} & 8.150 \\
    & Influencers & 1.967 & 0.191 & \textbf{1.055} & <.001*** & \textbf{53.583} & 36.667 \\
    & Regular & \textbf{0.371} & 0.012* & \textbf{0.032} & <.001*** & \textbf{4.417} & 1.021 \\
\midrule
\multicolumn{8}{l}{\textit{\textbf{Stance-based Reactions}}} \\
\midrule
\multirow{2}{*}{Likes} 
    & Same Stance & \textbf{3.127} & 0.014* & \textbf{0.639} & <.001*** & \textbf{45.600} & 17.233 \\
    & Opposing & \textbf{-1.663} & 0.019* & \textbf{0.153} & <.001*** & 0.000 & \textbf{14.950} \\
\midrule
\multirow{2}{*}{Comments} 
    & Same Stance & \textbf{0.807} & 0.010* & \textbf{0.158} & <.001*** & \textbf{11.133} & 3.817 \\
    & Opposing & \textbf{1.680} & 0.022* & \textbf{0.234} & <.001*** & \textbf{19.267} & 4.133 \\
\midrule
\multirow{2}{*}{Reposts} 
    & Same Stance & \textbf{0.773} & 0.017* & \textbf{0.146} & <.001*** & \textbf{11.117} & 4.183 \\
    & Opposing & -0.083 & 0.343 & \textbf{0.090} & <.001*** & 3.133 & 3.967 \\
\bottomrule
\multicolumn{8}{p{.95\linewidth}}{\small \textbf{Note:} Polarization Effect represents the increase in engagement in polarized compared to unpolarized condition. Temporal Effect represents the growth rate over time. AUC = Area Under the Curve. The larger value between Polarized and Unpolarized conditions is shown in bold. *$p$ < .05, **$p$ < .01, ***$p$ < .001} \\
\end{tabularx}
\end{table}

The emergent patterns in cross-ideological interactions present a particularly interesting parallel to real-world observations. Our simulation captures a fundamental asymmetry in how users engage with opposing viewpoints across different interaction types. The complete cessation of opposing-stance likes while maintaining---and even amplifying---opposing-stance comments ($\beta = 1.680$) aligns with observed social media behavior where users often engage in comments with opposing viewpoints while withholding positive endorsements \citep{bond_political_2022, an_political_2019}. We interpret this pattern as evidence of what we term "performative disagreement" - users engaging with opposing content not primarily to understand or consider alternative viewpoints, but to signal disagreement and reinforce their own position. This interpretation is supported by both the stronger \emph{Polarization Effect} for opposing-stance comments compared to same-stance comments ($\beta = 1.680$ vs. $\beta = 0.807$) and the differentiated pattern across reaction types. Likes, being public endorsements, serve primarily as identity markers and thus show strong homophily, while comments enable both agreement and disagreement, becoming vehicles for identity performance through opposition. Reposts occupy a middle ground between these two patterns, reflecting the distinct social identity signaling functions of different platform affordances.

The amplified engagement of influencers in \emph{Polarized Conditions}, particularly in receiving comments ($\beta = 8.633$), suggests they serve as focal points for cross-ideological conflict. The findings from \citep{soares_influencers_2018} on influencer engagement patterns in polarized discussions support this interpretation. The substantial temporal effects for influencer-received comments ($\beta = 1.620$) further suggests that these confrontational dynamics intensify over time, potentially creating self-reinforcing cycles of \emph{Polarization}.

While the model captures these key dynamics, it also reveals some limitations. The stark binary nature of the like-avoidance effect may oversimplify the more nuanced patterns of cross-ideological interaction documented in empirical studies. Additionally, the model's prediction of stable temporal effects across conditions may not adequately capture the documented volatility of real social media engagement patterns, particularly in response to external events or platform changes.

Summarized, this analysis suggests that \emph{Polarization} fundamentally shapes interaction dynamics in social networks, with influencers playing a central role in amplifying engagement and cross-ideological conflict. The findings highlight the importance of considering both structural and behavioral factors in understanding and addressing online polarization.

\subsection{Message Analysis}
\label{subsec:offline-message-analysis}

\subsubsection{Experimental Setup}

Using a $2\times3$ factorial design, we investigated how \emph{Polarization Level} and \emph{Recommendation Bias} influence content characteristics in social network discussions. We maintained similar network parameters as in the interaction analysis, but with a distinct approach to message recommendations.

For each condition (\emph{Polarized} and \emph{Unpolarized}), we implemented three \emph{Recommendation Bias} configurations: \emph{Pro-Biased} ($70$-$30$ ratio of pro to contra messages), \emph{Contra-Biased} ($30$-$70$ ratio), and \emph{Balanced} (equal proportions). Unlike the homophily-based recommendations in the polarization analysis, where recommendations were tailored to match each individual's stance, these biases were applied uniformly across all users regardless of their personal positions.

Our LLM-based content analysis measured four message dimensions: opinion (scale from $-1$ to $1$), group identity salience ($0$ to $1$), emotionality ($0$ to $1$), and uncertainty ($0$ to $1$). For each agent, the LLM evaluated all recommended messages across iterations, assessing the fundamental stance, group emphasis, emotional-rational balance, and expressed doubt in the discourse.

\subsubsection{Results}

Our analysis revealed significant effects of both \emph{Polarization Level} and \emph{Recommendation Bias} across all measured content dimensions (see Table~\ref{tab:message-analysis-anova} and Figure~\ref{fig:message-analysis-violin-plots}). The opinion analysis demonstrated strong main effects for \emph{Polarization Level} ($F(1, 14095) = 49.36$, $p < .001$) and \emph{Recommendation Homophily} ($F(2, 14095) = 1100.14$, $p < .001$), as well as a significant interaction between these factors ($F(2, 14095) = 901.86$, $p < .001$). In \emph{Polarized Conditions}, messages exhibited marginally negative average opinions ($M = -0.01$, $SD = 0.83$) compared to slightly more negative opinions in \emph{Unpolarized Conditions} ($M = -0.07$, $SD = 0.20$). The interaction manifested particularly strongly in opinion expression, where \emph{Polarized Conditions} showed marked differences between \emph{Pro-Biased} ($M = 0.52$, $SD = 0.66$), \emph{Contra-Biased} ($M = -0.43$, $SD = 0.71$), and \emph{Balanced} ($M = -0.13$, $SD = 0.81$) positions, while \emph{Unpolarized Conditions} exhibited substantially smaller variations between these positions (\emph{Pro-Biased}: $M = -0.04$, $SD = 0.18$; \emph{Contra-Biased}: $M = -0.06$, $SD = 0.21$; \emph{Balanced}: $M = -0.12$, $SD = 0.20$).

\emph{Group Identity Salience} showed a particularly pronounced main effect of \emph{Polarization Level} ($F(1, 14066) = 122007.50$, $p < .001$), with \emph{Polarized Conditions} eliciting substantially higher group identity expression ($M = 0.72$, $SD = 0.09$) compared to \emph{Unpolarized Conditions} ($M = 0.15$, $SD = 0.11$). While position effects were statistically significant ($F(2, 14066) = 52.47$, $p < .001$), the practical differences between positions were minimal, suggesting that polarization, rather than recommendation patterns, primarily drives group identity expression.

\begin{figure}[h]
    \centering
    \begin{minipage}[b]{0.49\textwidth}
        \centering
        \includegraphics[width=\textwidth]{figures/violin_plots/violin_plots_opinion.png}
        \subcaption{Message Opinion}
        \label{fig:subfig1}
    \end{minipage}
    \hfill
    \begin{minipage}[b]{0.49\textwidth}
        \centering
        \includegraphics[width=\textwidth]{figures/violin_plots/violin_plots_group_identity_salience.png}
        \subcaption{Group Identity Salience}
        \label{fig:subfig2}
    \end{minipage}
    
    \vspace{1em}
    
    \begin{minipage}[b]{0.49\textwidth}
        \centering
        \includegraphics[width=\textwidth]{figures/violin_plots/violin_plots_emotionality.png}
        \subcaption{Emotionality}
        \label{fig:subfig4}
    \end{minipage}
    \hfill
    \begin{minipage}[b]{0.49\textwidth}
        \centering
        \includegraphics[width=\textwidth]{figures/violin_plots/violin_plots_uncertainty.png}
        \subcaption{Uncertainty}
        \label{fig:subfig5}
    \end{minipage}

    \caption{The violin plots illustrate the distribution of values obtained from the LLM with respect to varying content dimensions. A differentiation is made between polarized and non-polarized populations, and the bias in the distribution of messages (pro, contra, balanced) is also considered.}
    \label{fig:message-analysis-violin-plots}
\end{figure}


\begin{table}[ht]
\centering
\small
\caption{Main Effects and Interactions Across Message Content Dimensions}
\label{tab:message-analysis-anova}
\begin{tabularx}{\textwidth}{>{\raggedright\arraybackslash}p{2.2cm}>{\raggedright\arraybackslash}p{1.8cm}*{3}{>{\centering\arraybackslash}X}>{\centering\arraybackslash}p{1.2cm}>{\centering\arraybackslash}p{1cm}}
\toprule
\multirow{2}{*}{\textbf{Metric}} & \multirow{2}{*}{\textbf{Condition}} & \multicolumn{3}{c}{\textbf{Position}} & \multicolumn{2}{c}{\textbf{ANOVA}} \\
\cmidrule(lr){3-5} \cmidrule(lr){6-7}
& & Contra & Balanced & Pro & $F$ & $p$ \\
\midrule
\multirow{2}{*}{Opinion} 
    & Polarized & \textbf{-0.43} (0.71) & -0.13 (0.81) & \textbf{0.52} (0.66) & \multirow{2}{*}{\textbf{901.86}$^a$} & \multirow{2}{*}{<.001***} \\
    & Unpolarized & -0.06 (0.21) & -0.12 (0.20) & -0.04 (0.18) & & \\
\midrule
\multirow{2}{*}{Group Identity} 
    & Polarized & \textbf{0.70} (0.10) & \textbf{0.74} (0.08) & \textbf{0.72} (0.09) & \multirow{2}{*}{\textbf{74.69}$^a$} & \multirow{2}{*}{<.001***} \\
    & Unpolarized & 0.15 (0.11) & 0.15 (0.11) & 0.14 (0.11) & & \\
\midrule
\multirow{2}{*}{Emotionality} 
    & Polarized & \textbf{0.71} (0.10) & \textbf{0.76} (0.08) & \textbf{0.76} (0.10) & \multirow{2}{*}{\textbf{156.85}$^a$} & \multirow{2}{*}{<.001***} \\
    & Unpolarized & 0.43 (0.09) & 0.42 (0.07) & 0.42 (0.07) & & \\
\midrule
\multirow{2}{*}{Uncertainty} 
    & Polarized & 0.19 (0.12) & 0.20 (0.08) & 0.21 (0.08) & \multirow{2}{*}{\textbf{17.58}$^a$} & \multirow{2}{*}{<.001***} \\
    & Unpolarized & \textbf{0.55} (0.07) & \textbf{0.56} (0.06) & \textbf{0.56} (0.07) & & \\
\bottomrule
\multicolumn{7}{p{.95\textwidth}}{\small \textbf{Note:} Values show means with standard deviations in parentheses. Bold values indicate significantly higher means between polarized and unpolarized conditions for each position. All F-statistics are significant at $p$ < .001.} \\
\multicolumn{7}{p{.95\textwidth}}{\small $^a$ F-statistic for Polarization × Position interaction (df = 2, 14095). ***$p$ < .001} \\
\end{tabularx}
\end{table}

The \emph{Emotionality} analysis revealed strong effects of \emph{Polarization Level} ($F(1, 14066) = 50758.88$, $p < .001$), with messages in \emph{Polarized Conditions} showing markedly higher emotional content ($M = 0.75$, $SD = 0.10$) compared to \emph{Unpolarized Conditions} ($M = 0.42$, $SD = 0.07$). Though \emph{Recommendation Homophily} effects were significant ($F(2, 14066) = 104.63$, $p < .001$), the differences were relatively small in practical terms.

\emph{Uncertainty} levels displayed an inverse relationship with \emph{Polarization Level} ($F(1, 14066) = 66628.45$, $p < .001$), with \emph{Unpolarized Conditions} generating substantially higher uncertainty expression ($M = 0.56$, $SD = 0.06$) compared to \emph{Polarized Conditions} ($M = 0.20$, $SD = 0.10$). \emph{Recommendation Homophily} effects, while statistically significant ($F(2, 14066) = 41.64$, $p < .001$), showed only minor differences.

These findings collectively suggest that \emph{Polarization Level} plays a crucial role in shaping message content across all measured dimensions, with \emph{Polarized Conditions} generally amplifying opinion differences, increasing group identity salience and emotionality, while reducing uncertainty. \emph{Recommendation Homophily} effects, while significant, showed varying practical importance across different content dimensions, with the strongest impact observed in opinion expression.

\subsubsection{Discussion}

Our simulation of message content dynamics provides valuable insights into how different initial conditions might shape online discourse patterns. The model's behavior reveals several interesting parallels with empirical observations while also highlighting potential limitations in capturing real-world complexity.

The simulated patterns of opinion expression, particularly the amplification effect in \emph{Polarized Conditions}, align with documented phenomena in social media studies. The model successfully reproduces the tendency toward more extreme language and position-taking in polarized environments, a pattern frequently observed in empirical research \citep{wahlstrom_dynamics_2021, simchon_troll_2022}. However, the clean separation between pro and contra positions in our simulation may oversimplify the more nuanced opinion distributions typically found in real online discussions.

A particularly interesting aspect of our simulation is the emergence of strong group identity expressions under \emph{Polarized Conditions}. This aligns with social identity theory \citep{tajfel_integrative_1979, huddy_social_2001} and empirical observations of group-based language in polarized debates \citep{albertson_dog-whistle_2015,ruiz-sanchez_us_2019,bliuc_online_2021,iyengar_fear_2015}, though the strength of this effect in our model ($M = 0.72$ vs $M = 0.15$) may be more pronounced than typically observed in real-world settings. The simulation thus captures the fundamental mechanism of group identity activation while potentially overemphasizing its magnitude.

The simulated relationship between polarization and emotional content mirrors documented patterns in social media discourse, where polarized discussions often exhibit higher emotional intensity \citep{asker_thinking_2019,fischer_emotion_2023}. Similarly, the inverse relationship between polarization and uncertainty expression in our model reflects observed patterns of increased certainty in situations where inter-group differences are salient \citep{holtz_intergroup_2001, holtz_relative_2008, winter_toward_2019}. However, the linear nature of these relationships in our simulation may not fully capture the complex interplay between emotions, uncertainty, and polarization observed in real social media environments.

The results of our simulation demonstrate that the framework is not only capable of generating polarized language, but also exhibits reliable classification of corresponding markers. However, while the model successfully reproduces several key patterns observed in empirical studies, the clarity and strength of these effects suggest that our simulation may not fully capture the noise and complexity inherent in real social media interactions. The strong main effects of polarization across all content dimensions, while theoretically informative, likely represent an idealized version of the more complex and nuanced patterns found in actual online discourse.

\section{User Study}

Having verified the basic ability of our model to produce polarizing debates in the previous section, we now present an exploratory investigation into how discussion polarization and algorithmic content curation in social media environments affect human perception of debates and their engagement behavior. Rather than testing specific hypotheses, this study aims to examine whether our framework can capture and reproduce fundamental mechanisms of online polarization identified in previous social science research. Through a controlled experiment using a simulated social media platform, we examine how different levels of discussion polarization (polarized vs. moderate) and recommendation bias (pro, balanced, contra) shape opinion formation and interaction patterns. This exploratory approach allows us to assess the framework's potential as a research tool for investigating human behavior in polarized online spaces, while providing initial insights into the complex interplay between platform design, user perception, and engagement patterns. The key findings of our user study are summarized in Table~\ref{tab:user-study-findings}.

\subsection{Experimental Design}

We employed a $2 \times 3$ between-subjects factorial design to investigate the dynamics of opinion formation and perception of polarization in online discussions. The experimental design manipulated two key dimensions: the \emph{Polarization Degree} in the artificial agent population and a systematic \emph{Recommendation Bias} while maintaining Universal Basic Income (UBI) (cf. Section \ref{sec:offline-evaluation}) as the consistent discussion topic.

The first experimental dimension contrasted highly polarized discussions with moderate ones through the manipulation of artificial agent behavior. In the polarized condition, artificial agents expressed extreme viewpoints and employed confrontational discourse patterns, characterized by emotional language, strong assertions, and minimal acknowledgment of opposing viewpoints. The moderate condition featured more nuanced discussions and cooperative interaction styles, with agents expressing uncertainty, acknowledging limitations in their knowledge, and engaging constructively with opposing views.

\begin{table*}[h!]
\centering
\small
\caption{Key Findings from User Study}
\label{tab:user-study-findings}
\begin{tabularx}{\textwidth}{>{\centering\arraybackslash}p{1.5cm}>{\raggedright\arraybackslash}X>{\raggedright\arraybackslash}X}
\toprule
\multicolumn{1}{c}{\textbf{Analysis}} & \multicolumn{1}{c}{\textbf{Key Finding}} & \multicolumn{1}{c}{\textbf{Theoretical Implication}} \\
\midrule
\multirow{3}{*}{\rotatebox[origin=c]{90}{\textbf{Perception}}} 
    & Users readily detect emotional content and group-based language in polarized discussions 
    & Social and emotional signals serve as primary markers for detecting polarization \\
\cmidrule{2-3}
    & The effect of perceived emotionality on perceived polarization is mediated through perceived group salience 
    & Group-based processing, not emotional content itself, is the primary pathway to polarization perception \\
\cmidrule{2-3}
    & Polarized discussions show marked decrease in expressed uncertainty 
    & Polarization creates perception of epistemic closure, limiting space for nuanced dialogue \\
\midrule
\multirow{3}{*}{\rotatebox[origin=c]{90}{\textbf{Engagement}}} 
    & Different interaction types serve distinct social functions, with minimal correlation between comments and reposts 
    & Platform affordances shape distinct modes of user participation \\
\cmidrule{2-3}
    & Users comment significantly more in unpolarized discussions compared to polarized environments
    & Moderate environments facilitate substantive textual engagement among non-polarized populations \\
\cmidrule{2-3}
    & Highest total engagement observed in polarized contra-bias condition, coinciding with strongest negative opinion shifts
    & Arguments against systemic change gain greater salience in polarized environments, activating status quo bias \\
\midrule
\multirow{2}{*}{\rotatebox[origin=c]{90}{\textbf{Opinion}}} 
    & While directional opinion change showed no significant effects, magnitude of opinion change revealed significant impacts
    & Polarized environments increase opinion volatility rather than pushing opinions in consistent directions \\
\cmidrule{2-3}
    & Experimental conditions directly affected opinion change magnitude, while perceptual variables showed no significant mediation
    & Suggests unconscious processing mechanisms responding to structural features without awareness \\
\bottomrule
\end{tabularx}
\end{table*}


\begin{figure}[htbp]
    \centering
    \includegraphics[width=\textwidth]{figures/prototype_screenshot.png}
    \caption{The screenshot depicts the simulated social media platform interface. The Newsfeed is displayed with a single post and one comment, including reaction handles for liking, reposting, and commenting. The interface emulates common social media design patterns, including a field for posting new messages, discovering new users, and inspecting the user profile.}
    \label{fig:prototype-screenshot}
\end{figure}


The second dimension introduced systematic bias in the recommendation system, implemented across three levels: neutral ($50\%$ pro-UBI, $50\%$ contra-UBI content), pro-bias ($70\%$ pro-UBI, $30\%$ contra-UBI content), and contra-bias ($30\%$ pro-UBI, $70\%$ contra-UBI content). This manipulation aimed to investigate how algorithmic content curation influences opinion formation and perception of debate polarization.

\subsection{System Prototype}

The prototype implementation consists of a web application that simulates a social media platform, reminiscent of X (formerly Twitter), to study social polarization dynamics. The interface, as depicted in Figure~\ref{fig:prototype-screenshot}, adheres to a familiar social media layout, facilitating user engagement and interaction.

\subsubsection{User Interface}

The application's main interface is divided into three primary sections: a navigation sidebar on the left, a central Newsfeed, and a recommendation panel on the right. The navigation sidebar provides quick access to essential functionalities such as the user's profile, a general user overview, and a logout option. The central Newsfeed serves as the primary interaction space, where users can view and engage with posts from other users. At the top of the Newsfeed, a text input area invites users to share their thoughts, mimicking the spontaneous nature of social media communication.

The Newsfeed displays a series of posts, each accompanied by user avatars, usernames, timestamps, and interaction metrics such as \emph{likes}, \emph{comments}, and \emph{reposts}. This design encourages user engagement and provides visual cues about the popularity and impact of each post. The recommendation panel on the right side of the interface suggests other users to follow, potentially influencing the user's network expansion and exposure to diverse viewpoints.

User profiles are dynamically generated, displaying the user's posts, follower relationships, and other relevant metadata like a user's handle and biography (see Section~\ref{subsec:agent-model} and Section~\ref{subsec:social-network-model} for details). It is also possible to follow and unfollow artificial users.

\subsubsection{Newsfeed Recommendations}

The web application implements an adaptive recommendation system for content presentation that evolves with user engagement. This system employs two distinct algorithmic approaches: a default variant for initial users and a collaborative variant that activates once users establish an interaction history.

The default variant implements a popularity-based scoring mechanism that considers multiple forms of engagement to determine content visibility. For a given message $m$, the system calculates a composite popularity score:

\begin{align}
    S_p(m) = l_m + 2c_m + 2r_m
\end{align}

where $l_m$, $c_m$, and $r_m$ represent the number of the message's \emph{likes}, \emph{comments}, and \emph{reposts} respectively. The weighted coefficients reflect the relative importance assigned to different forms of engagement, with more active forms of interaction carrying greater weight.

As users begin to interact with the platform, the system transitions to a collaborative variant that incorporates popularity metrics, ideological proximity, and a stochastic element to ensure recommendation diversity. The enhanced scoring function combines these elements into a composite score:

\begin{align}
    S_c(m) = \omega_p \cdot \frac{S_p(m)}{S_{max}} + \omega_i \cdot \frac{2 - |o_u - o_a|}{2} + \omega_r \cdot \epsilon
\end{align}

where $S_p(m)$ represents the popularity score normalized by the maximum observed score $S_{max}$, $o_u$ and $o_a$ denote the opinion scores of the active user (determined as the average of the opinion scores of the artificial users interacted with) and the (artificial) message author respectively, $\epsilon$ represents a uniform random variable in the interval $[0,1]$, and $\omega_p$, $\omega_i$, and $\omega_r$ are weighting parameters that sum to unity ($\omega_p + \omega_i + \omega_r = 1$). In the current implementation, these weights are set to $\omega_p = 0.6$, $\omega_i = 0.2$, and $\omega_r = 0.2$, balancing the influence of popularity, ideological similarity, and randomization.

Both variants maintain temporal relevance by presenting the user's most recent content contributions at the beginning of their feed when accessing the first page. This approach ensures users maintain awareness of their own contributions while experiencing the broader content landscape through the scoring-based recommendations.

This dual-variant approach enables the system to provide meaningful content recommendations even in the absence of user interaction data while transitioning smoothly to more personalized recommendations as users engage with the platform. The incorporation of popularity metrics, (mild) ideological factors, and controlled randomization creates diverse recommendations that is designed to give the impression of a dynamic network environment, while still falling under the conditional \emph{Recommendation Bias} regime. The stochastic element particularly aids in preventing recommendation stagnation and ensures dynamic content delivery.

\subsection{Procedure}

The experiment consisted of three phases: pre-interaction, interaction, and post-interaction.

\paragraph{Pre-interaction Phase} Participants first completed a comprehensive questionnaire assessing various baseline measures. These included demographic information, social media usage patterns, and initial attitudes towards UBI. 

\paragraph{Interaction Phase} Afterwards, participants were introduced to our simulated social media platform. They were instructed to engage with the platform naturally, as they would in their regular social media use. Unbeknownst to the participants, all other users on the platform were artificial agents programmed to behave according to the assigned experimental conditions. Participants were given the following instructions:

\begin{quote}
"You will now interact with a social media platform discussing \emph{Universal Basic Income}. Please use the platform as you normally would use social media. You can read posts, like them, comment on them, or create your own posts. Your goal is to form an opinion on the topic. You will have 10 minutes for this task."
\end{quote}

During this phase, participants' interactions, including likes, comments, reposts, and follows were recorded for later analysis.

\paragraph{Post-interaction Phase} After the interaction period, participants completed a post-test questionnaire. This included measures of their perception of the key constructs listed below. Additionally, participants evaluated the realism and effectiveness of the simulated platform.

\subsection{Measures}

Key constructs measured in this study included:

\begin{itemize}
    \item \emph{Attitude Change}: Measured shifts in participants' opinions about UBI between pre- and post-interaction phases using a seven-item scale (1 = Strongly Disagree to 5 = Strongly Agree), capturing changes in both the direction and magnitude of support.
    
    \item \emph{Perceived Polarization}: Assessed participants' perception of opinion extremity and ideological division within the observed discussions using a four-item scale (1 = Strongly Disagree to 5 = Strongly Agree), focusing on the perceived distance between opposing viewpoints.
    
    \item \emph{Perceived Group Salience}: Evaluated the extent to which participants perceived the discussion as being driven by group identities rather than individual perspectives, using a four-item scale (1 = Strongly Disagree to 5 = Strongly Agree).
    
    \item \emph{Perceived Emotionality}: Measured participants' assessment of the emotional intensity and affective tone in discussions using a four-item scale (1 = Strongly Agree to 5 = Strongly Disagree), capturing the perceived level of emotional versus rational discourse.
    
    \item \emph{Perceived Uncertainty}: Captured the degree to which participants observed expressions of doubt and acknowledgment of knowledge limitations in the discussions using a four-item scale (1 = Strongly Disagree to 5 = Strongly Agree).
    
    \item \emph{Perceived Bias}: Evaluated participants' assessment of viewpoint balance and fair representation of different perspectives in the discussion using a four-item scale (1 = Strongly Agree to 5 = Strongly Disagree).
\end{itemize}

\subsection{Participants}

We recruited $122$ participants through the Prolific platform. The sample exhibited a gender distribution favoring male participants ($63.6\%$) over female participants ($35.7\%$), with a single participant preferring not to disclose their gender. The age distribution revealed a predominantly young to middle-aged sample, with approximately $61.5\%$ of participants falling between $20$ and $39$ years old. The modal age group was $25$-$29$ years (18.6\%), followed by $35$-$39$ years ($15.0\%$).

Regarding educational background, nearly half of the participants ($49.3\%$) held university degrees, indicating a relatively high level of formal education in the sample. The remaining participants were distributed across various educational qualifications, with A-levels/IB, GCSE, and vocational certifications each representing approximately $11\%$ of the sample.

The majority of participants were professionally active, with $58.6\%$ being employees and $10.7\%$ self-employed. The sample also included a notable proportion of students ($15.7\%$ combined university and school students), reflecting diverse occupational backgrounds.

\begin{table}[htbp]
\centering
\footnotesize
\caption{Factor loadings and scale reliability for key measures. Note: [R] indicates reverse-coded items. Factor loadings are displayed for all items retained after cleaning (loading threshold $|.40|$).}
\begin{tabularx}{\textwidth}{>{\raggedright\arraybackslash}X>{\centering\arraybackslash}p{2cm}}
\toprule
\multicolumn{2}{l}{\textbf{Universal Basic Income (Pre)} ($\alpha = .893$)} \\
\midrule
A universal basic income would benefit society as a whole & .828 \\
Providing everyone with a basic income would do more harm than good [R] & .728 \\
Universal basic income is a fair way to ensure everyone's basic needs are met & .701 \\
Giving everyone a fixed monthly payment would reduce people's motivation to work [R] & .671 \\
Universal basic income would lead to a more stable and secure society & .867 \\
People should earn their income through work rather than receiving it unconditionally from the government [R] & .649 \\
A universal basic income would give people more freedom to make choices about their lives & .724 \\
\midrule
\multicolumn{2}{l}{\textbf{Universal Basic Income (Post)} ($\alpha = .890$)} \\
\midrule
A universal basic income would benefit society as a whole & .847 \\
Providing everyone with a basic income would do more harm than good [R] & .769 \\
Universal basic income is a fair way to ensure everyone's basic needs are met & .782 \\
Giving everyone a fixed monthly payment would reduce people's motivation to work [R] & .610 \\
Universal basic income would lead to a more stable and secure society & .921 \\
People should earn their income through work rather than receiving it unconditionally from the government [R] & .522 \\
A universal basic income would give people more freedom to make choices about their lives & .698 \\
\midrule
\multicolumn{2}{l}{\textbf{Perceived Polarization} ($\alpha = .776$)} \\
\midrule
The discussions on the platform were highly polarized & .674 \\
Users on the platform expressed extreme views & .755 \\
Users appeared to be firmly entrenched in their positions & .644 \\
There were frequent hostile interactions between users with differing views & .644 \\
\midrule
\multicolumn{2}{l}{\textbf{Perceived Emotionality} ($\alpha = .842$)} \\
\midrule
The discussions were highly charged with emotional content & .864 \\
Users frequently expressed strong feelings in their messages & .691 \\
The debate maintained a predominantly calm and neutral tone [R] & .692 \\
Participants typically communicated in an unemotional manner [R] & .779 \\
\midrule
\multicolumn{2}{l}{\textbf{Perceived Group Salience} ($\alpha = .639$)} \\
\midrule
Messages frequently emphasized "us versus them" distinctions & .585 \\
Users often referred to their group membership when making arguments & .806 \\
\midrule
\multicolumn{2}{l}{\textbf{Perceived Uncertainty} ($\alpha = .715$)} \\
\midrule
The agents frequently acknowledged limitations in their knowledge & .541 \\
Users often expressed doubt about their own positions & .772 \\
Messages typically contained absolute statements without room for doubt [R] & .566 \\
The agents seemed very certain about their claims and positions [R] & .671 \\
\midrule
\multicolumn{2}{l}{\textbf{Perceived Bias} ($\alpha = .830$)} \\
\midrule
The discussion seemed to favor one particular viewpoint & .782 \\
Certain perspectives received more attention than others in the debate & .738 \\
The platform provided a balanced representation of different viewpoints [R] & .821 \\
Different perspectives were given equal consideration in the discussion [R] & .646 \\
\bottomrule
\end{tabularx}
\label{tab:factor-loadings}
\end{table}

Participants demonstrated high engagement with social media platforms, with $80\%$ reporting daily or near-constant usage. The majority ($75\%$) spent between one and four hours daily on social media platforms. YouTube ($25.0\%$) and Facebook ($24.3\%$) emerged as the most frequently used platforms, followed by Instagram ($17.9\%$) and X, formerly Twitter ($10.7\%$). This usage pattern suggests participants were well-acquainted with social media interfaces and interaction patterns, making them suitable subjects for the study's simulated social media environment.

\subsection{Preliminary Analysis}

We evaluated the ecological validity of our experimental platform. Participants rated various aspects on $7$-point scales ($1$ = Not at all, $7$ = Extremely), with higher scores indicating more positive evaluations. The platform received favorable ratings across multiple dimensions, consistently scoring above the scale midpoint of $4$. Particularly noteworthy was the interface usability ($M = 5.52$, $SD = 1.15$), which participants rated as highly satisfactory. The platform's similarity to real social media platforms ($M = 4.69$, $SD = 1.65$) and its ability to facilitate meaningful discussions ($M = 4.59$, $SD = 1.41$) were also rated positively. The overall platform realism received satisfactory ratings ($M = 4.47$, $SD = 1.61$), suggesting that participants found the experimental environment sufficiently realistic and engaging for the purposes of this study. The attitudes of participants toward Universal Basic Income exhibited a slight decline from the pre-interaction phase ($M = 3.12$, $SD = 0.90$) to the post-interaction phase ($M = 2.99$, $SD = 0.99$). However, these attitudes remained relatively close to the scale midpoint, indicating that participants held moderate views on the subject matter under discussion.

\begin{table}[ht]
\footnotesize
\centering
\caption{Effects of Polarization and Recommendation Bias on Key Dependent Variables}
\label{tab:perception-anova}
\begin{tabularx}{\textwidth}{>{\raggedright\arraybackslash}p{2.5cm}>{\raggedright\arraybackslash}p{1.5cm}*{3}{>{\centering\arraybackslash}X}>{\centering\arraybackslash}p{1.2cm}>{\centering\arraybackslash}p{1.2cm}>{\centering\arraybackslash}p{1.2cm}}
\toprule
\multirow{2}{*}{\textbf{Metric}} & \multirow{2}{*}{\textbf{Condition}} & \multicolumn{3}{c}{\textbf{Recommendation Bias}} & \multicolumn{3}{c}{\textbf{ANOVA $\boldsymbol{F}$($\boldsymbol{p}$)}} \\
\cmidrule(lr){3-5} \cmidrule(lr){6-8}
& & Contra & Balanced & Pro & Pol & Rec & Pol×Rec \\
\midrule
\multirow{2}{*}{Opinion Change} 
   & Polarized & \textbf{-0.41} (0.60) & -0.08 (0.55) & -0.10 (0.33) & 3.48 & 1.74 & 1.90 \\
   & Unpolarized & -0.06 (0.38) & -0.09 (0.43) & -0.02 (0.26) & (.065) & (.180) & (.154) \\
\midrule
\multirow{2}{*}{|Opinion Change|} 
   & Polarized & \textbf{0.56} (0.45) & 0.35 (0.42) & 0.22 (0.26) & \textbf{5.21*} & \textbf{3.61*} & 2.54 \\
   & Unpolarized & 0.24 (0.30) & 0.32 (0.28) & 0.18 (0.18) & (.024) & (.030) & (.083) \\
\midrule
\multirow{2}{*}{Polarization} 
   & Polarized & \textbf{4.47} (0.82) & \textbf{4.61} (0.97) & \textbf{4.29} (0.75) & \textbf{56.48***} & 0.02 & 1.50 \\
   & Unpolarized & 3.30 (0.69) & 3.23 (0.83) & 3.54 (0.84) & (.000) & (.979) & (.228) \\
\midrule
\multirow{2}{*}{Emotionality} 
   & Polarized & \textbf{3.80} (0.48) & \textbf{3.59} (0.99) & \textbf{3.36} (0.80) & \textbf{73.54***} & 1.42 & 2.44 \\
   & Unpolarized & 2.50 (0.74) & 2.17 (0.60) & 2.65 (0.83) & (.000) & (.245) & (.092) \\
\midrule
\multirow{2}{*}{Group Salience} 
   & Polarized & \textbf{2.94} (0.60) & \textbf{2.86} (0.51) & \textbf{3.21} (0.42) & \textbf{17.18***} & \textbf{5.63**} & 0.07 \\
   & Unpolarized & 2.51 (0.59) & 2.42 (0.65) & 2.86 (0.40) & (.000) & (.005) & (.934) \\
\midrule
\multirow{2}{*}{Bias} 
   & Polarized & \textbf{3.50} (0.87) & 3.20 (0.76) & \textbf{3.83} (0.59) & \textbf{15.08***} & 0.44 & \textbf{3.73*} \\
   & Unpolarized & 3.05 (0.82) & 3.02 (0.87) & 2.68 (0.87) & (.000) & (.648) & (.027) \\
\midrule
\multirow{2}{*}{Uncertainty} 
   & Polarized & 2.29 (0.60) & 2.03 (0.60) & 1.90 (0.75) & \textbf{48.86***} & 0.97 & 1.06 \\
   & Unpolarized & \textbf{2.92} (0.71) & \textbf{2.83} (0.69) & \textbf{2.96} (0.48) & (.000) & (.383) & (.350) \\
\bottomrule
\multicolumn{8}{p{.95\textwidth}}{\small \textbf{Note:} Values show means with standard deviations in parentheses. Bold values indicate significantly higher means between polarized and unpolarized conditions. F-statistics in bold are significant. Pol = Polarization main effect, Rec = Recommendation main effect, Pol×Rec = Interaction effect. *$p$ < .05, **$p$ < .01, ***$p$ < .001} \\
\end{tabularx}
\end{table}


Furthermore, we examined the psychometric properties of our key measures (see Table~\ref{tab:factor-loadings}). Principal component analyses were conducted for each scale, with items loading on their intended factors. Most scales showed good reliability ($\alpha$ ranging from $.715$ to $.893$) and satisfactory factor loadings ($|.40|$ or greater). The \emph{Perceived Group Salience} scale required modification from its original four-item structure. Two items (\emph{"The debate focused on ideas rather than group affiliations"} and \emph{"Individual perspectives were more prominent than group identities in the discussions"}) were dropped due to poor factor loadings ($.049$ and $.051$ respectively). The remaining two items showed modest to acceptable loadings ($.585$ and $.806$), though below optimal thresholds. Given the theoretical importance of group salience in our research design, we retained this measure for further analyses while acknowledging its psychometric limitations.


\subsection{Analysis of Debate Perception}
\label{subsec:debate-perception}

Our first analysis examines how users perceive and process discussions under varying conditions of \emph{Polarization Degree} and \emph{Recommendation Bias}. We specifically investigate whether participants recognize polarized discourse patterns, how they process emotional and group-based content, and how \emph{Recommendation Bias} might moderate these perceptions. Through this analysis, we aim to understand the psychological mechanisms through which discussion climate and content curation shape users' experience of online debates.

\subsubsection{Variance Analysis}

\begin{figure}[h!]
    \centering
    \includegraphics[width=\textwidth]{figures/interaction_plots_perception.png}
    \caption{Interaction plots showing the effects of polarization and recommendation type on key dependent variables. Red lines represent the polarized condition, green lines represent the unpolarized condition. Error bars represent $95\%$ confidence intervals.}
    \label{fig:perception-interaction-plots}
\end{figure}



The effects of discussion \emph{Polarization Degree} and \emph{Recommendation Bias} were analyzed using a series of $2 \times 3$ analyses of variance. As shown in Table~\ref{tab:perception-anova}, the results revealed consistent main effects of \emph{Polarization Degree} across multiple dependent variables, while \emph{Recommendation Bias} showed more limited impact. Figure~\ref{fig:perception-interaction-plots} visualizes the interaction effects for our key constructs.


Regarding \emph{Opinion Change}, although the effects did not reach statistical significance, the data suggested some noteworthy patterns. Neither \emph{Polarization Degree} ($F(1, 117) = 3.48$, $p = .065$, $\eta_p^2 = 0.029$) nor \emph{Recommendation Bias} ($F(2, 117) = 1.74$, $p = .180$, $\eta_p^2 = 0.029$) significantly influenced opinion change. The descriptive statistics revealed a tendency toward stronger opinion changes in the polarized condition with contra-\emph{Recommendation Bias} ($M = -0.408$, $SD = 0.598$) compared to other conditions, where changes were minimal (means ranging from $-0.087$ to $-0.015$).

Given that directional opinion change might mask important dynamics by allowing positive and negative changes to cancel each other out, we conducted an additional analysis focusing on the magnitude of opinion change (i.e., absolute values). This analysis revealed significant main effects for both \emph{Polarization Degree} ($F(1, 117) = 5.21$, $p = .024$, $\eta_p^2 = 0.043$) and \emph{Recommendation Bias} ($F(2, 117) = 3.61$, $p = .030$, $\eta_p^2 = 0.058$). Participants in the polarized condition showed stronger opinion changes ($M = 0.378$, $SD = 0.378$) compared to the unpolarized condition ($M = 0.248$, $SD = 0.254$, Hedges' $g = 0.390$). Post-hoc analyses for \emph{Recommendation Bias} revealed that participants in the contra-bias condition showed significantly stronger opinion changes compared to the pro-bias condition (Hedges' $g = 0.566$, $p = .009$). The neutral condition showed significantly stronger opinion changes compared to the pro-bias condition (Hedges' $g = 0.456$, $p = .039$), but did not significantly differ from the contra-bias condition (Hedges' $g = 0.147$, $p = .493$)

The \emph{Polarization Degree} manipulation strongly influenced participants' perception of the discussion climate. \emph{Perceived Polarization} showed a substantial main effect ($F(1, 116) = 56.48$, $p < .001$, $\eta_p^2 = 0.327$), with participants in the polarized condition reporting significantly higher levels ($M = 4.46$, $SD = 0.85$) compared to the unpolarized condition ($M = 3.35$, $SD = 0.79$). Post-hoc tests confirmed this significant difference ($t(116.44) = 7.52$, $p < .001$, Hedges' $g = 1.36$).

\emph{Perceived Emotionality} emerged as the strongest effect in the study ($F(1, 116) = 73.54$, $p < .001$, $\eta_p^2 = 0.388$). Participants in the polarized condition perceived substantially higher emotional content ($M = 3.58$, $SD = 0.76$) than those in the unpolarized condition ($M = 2.44$, $SD = 0.72$). Post-hoc analysis revealed a large effect size (Hedges' $g = 1.53$, $t(116.99) = 8.44$, $p < .001$). While \emph{Recommendation Bias} did not show a significant main effect ($F(2, 116) = 1.42$, $p = .245$, $\eta_p^2 = 0.024$), there was a marginal interaction effect ($F(2, 116) = 2.44$, $p = .092$, $\eta_p^2 = 0.040$).

The analysis of \emph{Perceived Uncertainty} revealed another strong main effect of \emph{Polarization Degree} ($F(1, 116) = 48.86$, $p < .001$, $\eta_p^2 = 0.296$). Participants exposed to polarized discussions reported significantly lower levels of expressed uncertainty ($M = 2.07$, $SD = 0.65$) compared to those in the unpolarized condition ($M = 2.90$, $SD = 0.63$). Post-hoc tests indicated a strong effect (Hedges' $g = -1.26$, $t(117.89) = -6.97$, $p < .001$).

\emph{Perceived Group Salience} showed significant main effects for both \emph{Polarization Degree} ($F(1, 116) = 17.18$, $p < .001$, $\eta_p^2 = 0.129$) and \emph{Recommendation Bias} ($F(2, 116) = 5.63$, $p = .005$, $\eta_p^2 = 0.089$). The polarized condition elicited higher reports of group-based discourse ($M = 3.00$, $SD = 0.51$) compared to the unpolarized condition ($M = 2.59$, $SD = 0.55$). Post-hoc analysis revealed a medium to large effect size (Hedges' $g = 0.73$, $t(120.00) = 4.10$, $p < .001$). Post-hoc tests for \emph{Recommendation Bias} showed significant differences between pro-bias and both contra-bias (Hedges' $g = -0.56$, $t(76.97) = -2.61$, $p = .011$) and neutral conditions (Hedges' $g = -0.74$, $t(72.39) = -3.34$, $p = .001$).

\emph{Perceived Bias} demonstrated both a significant main effect of \emph{Polarization Degree} ($F(1, 116) = 15.08$, $p < .001$, $\eta_p^2 = 0.115$) and a significant interaction effect with \emph{Recommendation Bias} ($F(2, 116) = 3.73$, $p = .027$, $\eta_p^2 = 0.060$). In the polarized condition, participants reported higher \emph{Perveived Bias} ($M = 3.51$, $SD = 0.74$) compared to the unpolarized condition ($M = 2.92$, $SD = 0.85$). Post-hoc tests confirmed this difference with a medium to large effect size (Hedges' $g = 0.69$, $t(119.97) = 3.85$, $p < .001$). The significant interaction suggests that the effect of \emph{Polarization Degree} on bias perception varied across \emph{Recommendation Bias} conditions, with particularly pronounced differences in the pro-bias condition (polarized: $M = 3.83$, $SD = 0.59$; unpolarized: $M = 2.68$, $SD = 0.87$).

To examine whether these effects might vary based on participants' prior attitudes toward Universal Basic Income, we conducted additional analyses grouping participants according to their initial position (strongly pro/contra, moderately pro/contra). However, these analyses did not yield any significant results, suggesting that the observed effects of \emph{Polarization Degree} and \emph{Recommendation Bias} on opinion change magnitude operate similarly across different initial attitude positions.

\subsubsection{Path Analysis}

To understand the mechanisms through which our experimental conditions influence both perceptual dimensions and opinion change, we employed structural equation modeling (SEM). This approach allows us to simultaneously estimate multiple interdependent relationships while accounting for measurement error and covariation between constructs.

\begin{figure}[htbp]
    \centering
    \includegraphics[width=\textwidth]{figures/sem.png}
    \caption{Structural equation model showing the effects of political information polarization and recommendation system exposure on perceived polarization and opinion change. Path coefficients represent standardized regression weights. Solid lines indicate direct effects, dashed lines represent covariances. $^*p < .05$, $^{**}p < .01$, $^{***}p < .001$.}
    \label{fig:path-model}
\end{figure}


We developed a theoretical model examining how \emph{Polarization Degree} and \emph{Recommendation Bias} affect \emph{Opinion Change Magnitude} and \emph{Perceived Polarization} both directly and through various perceptual pathways (see Figure~\ref{fig:path-model}). The model was estimated using maximum likelihood estimation with standardized variables. The results demonstrated excellent fit to the data ($\chi^2(6) = 2.335$, $p = .886$, CFI $= 1.000$, TLI $= 1.109$, RMSEA $= .000$ [90\% CI: .000, .068], SRMR $= .023$), explaining substantial variance in key outcome variables (e.g., \emph{Perceived Emotionality}: 53.3\%, \emph{Perceived Polarization}: 38.0\%, \emph{Perceived Uncertainty}: 30.5\%).

The SEM analysis revealed several significant pathways. First, both experimental conditions showed significant direct effects on \emph{Opinion Change Magnitude}. Higher \emph{Polarization Degree} led to increased \emph{Opinion Change Magnitude} ($\beta = .275$, $p = .007$), while pro-UBI \emph{Recommendation Bias} decreased \emph{Opinion Change Magnitude} ($\beta = -.280$, $p = .006$). Notably, these effects emerged despite none of the perceptual variables showing significant direct effects on \emph{Opinion Change Magnitude}.

Regarding the perceptual pathways, \emph{Polarization Degree} showed strong direct effects on multiple dimensions: \emph{Perceived Emotionality} ($\beta = .472$, $p < .001$), \emph{Perceived Uncertainty} ($\beta = -.541$, $p < .001$), \emph{Perceived Bias} ($\beta = .433$, $p < .001$), and \emph{Perceived Group Salience} ($\beta = .251$, $p = .019$). The analysis further revealed how \emph{Polarization Degree} shapes \emph{Perceived Polarization} through multiple pathways. Beyond its direct effect ($\beta = .247$, $p = .022$), \emph{Polarization Degree} also operated through two indirect paths: via \emph{Perceived Group Salience} ($IE = .105$, $p = .044$) and via \emph{Perceived Emotionality} ($IE = .050$, $p = .406$). The combination of these direct and indirect effects resulted in a substantial total effect of \emph{Polarization Degree} on \emph{Perceived Polarization} ($\beta = .402$, $p < .001$).

The model also captured interesting relationships between mediating variables. \emph{Perceived Group Salience} showed significant positive effects on both \emph{Perceived Emotionality} ($\beta = .447$, $p < .001$) and \emph{Perceived Polarization} ($\beta = .417$, $p < .001$). Additionally, we found a significant negative covariance between \emph{Perceived Uncertainty} and \emph{Perceived Bias} ($\beta = -.291$, $p = .012$), suggesting these perceptions tend to operate in opposition to each other.

These findings reveal a complex pattern where experimental conditions shape both behavioral outcomes (\emph{Opinion Change Magnitude}) and perceptual experiences, though these paths appear to operate independently rather than sequentially. The substantial total effect of \emph{Polarization Degree} on \emph{Perceived Polarization}, decomposed into direct and indirect pathways, highlights how environmental features can influence user perceptions through multiple complementary mechanisms.

\subsubsection{Discussion}

Our findings reveal significant insights into how users perceive, process, and respond to polarized discussions in social media environments. Through complementary analytical approaches, we demonstrate that the manipulation of \emph{Polarization Degree} had profound effects across multiple perceptual dimensions and behavioral outcomes, while \emph{Recommendation Bias} played a more nuanced role in shaping user experiences.

A key finding emerged in our analysis of opinion change. While directional opinion change showed no significant effects, the magnitude of opinion change revealed significant impacts of both experimental conditions. This discrepancy suggests that focusing solely on directional change may mask important dynamics by allowing opposing changes to cancel each other out. The significant effects on magnitude indicate that polarized environments and recommendation patterns do influence opinion formation, but not in a uniformly directional way. This aligns with theoretical perspectives suggesting that polarized environments might increase opinion volatility without necessarily pushing opinions in a consistent direction.

Particularly intriguing is our finding that while experimental conditions significantly predicted \emph{Opinion Change Magnitude}, none of the measured perceptual variables showed significant effects. This creates an interesting puzzle: why would objective environmental conditions affect opinion change while subjective perceptions of these conditions do not? Several theoretical explanations merit consideration. First, this could reflect unconscious processing mechanisms, where participants respond to structural features of the environment without consciously processing them, aligning with dual-process theories of attitude change \citep{petty_elaboration_1986, chaiken_the_2014}. Second, our post-hoc perception measures might not capture the dynamic nature of how perceptions evolved during the interaction period. Third, unmeasured mediating variables like information processing depth or emotional arousal might better explain the mechanism of influence. This pattern aligns with Kurt Lewin's concept of "environmental press," suggesting that behavior might be influenced by objective environmental properties independent of their subjective interpretation \citep{lewin_principles_2013}.

The strong effect of \emph{Polarization Degree} on \emph{Perceived Emotionality} ($\eta_p^2 = 0.388$) suggests that participants were highly attuned to the emotional tenor of discussions. However, our structural equation model revealed an unexpected pattern: the direct effect of \emph{Perceived Emotionality} on \emph{Perceived Polarization} became non-significant when controlling for \emph{Perceived Group Salience}. This finding has profound theoretical implications. It suggests that the relationship between emotional content and polarization perception is primarily mediated through group-based processing, aligning with theories of affective polarization and social identity. Drawing on Carl Schmitt's friend-enemy distinction \citep{schmitt_concept_2008} and its extensions by Laclau and Mouffe \citep{laclau_hegemony_2014}, this might indicate that emotional content primarily influences polarization perception by activating group-based antagonisms rather than through direct affective pathways. The strong mediating role of \emph{Perceived Group Salience} ($\beta = .417$) supports this interpretation, suggesting that emotional content primarily serves to highlight group boundaries and distinctions.

The absence of significant effects from \emph{Perceived Uncertainty} and \emph{Perceived Bias} on polarization perception, despite strong condition effects on these variables, suggests that the path to \emph{Perceived Polarization} is more specific than previously theorized. While polarized environments clearly influence these perceptual dimensions (uncertainty: $\beta = -.541$; bias: $\beta = .433$), their lack of predictive power for polarization perception suggests they might operate through different psychological mechanisms or serve different functions in processing polarized discourse.

Nevertheless, the inverse relationship between \emph{Polarization Degree} and \emph{Perceived Uncertainty} ($\eta_p^2 = 0.296$) aligns with theoretical frameworks suggesting that polarized discourse often manifests through increased assertiveness and reduced acknowledgment of epistemic limitations. This pattern may help explain the self-reinforcing nature of polarized discussions: as uncertainty expressions diminish, the space for nuanced dialogue potentially contracts.

These results carry important implications for platform design and intervention strategies. The clear user sensitivity to emotional content and its indirect effect through group salience suggests that interventions might need to target both emotional expression and group dynamics simultaneously. Moreover, the finding that conscious perceptions do not mediate opinion change suggests that traditional perception-focused interventions might have limited effectiveness. Instead, interventions might need to address structural features of the environment that influence behavior more directly.

Our findings particularly highlight the central role of group processes in polarization dynamics. The strong mediating role of \emph{Perceived Group Salience} in translating emotional content into polarization perceptions suggests that effective interventions might need to focus on reducing the salience of group boundaries rather than just moderating emotional content. This insight, supported by both our variance and path analyses, suggests a more nuanced approach to platform design that considers how features might inadvertently enhance group distinctions even when attempting to reduce emotional intensity.

\subsection{Analysis of User Engagement}

Our second analysis examines how polarization and recommendation bias shape user engagement behaviors on the platform. We investigate both the quantity and quality of interactions, analyzing how different experimental conditions affect users' preferences for specific types of engagement (\emph{likes}, \emph{comments}, \emph{reposts}, and \emph{follows}). This analysis aims to understand whether polarized environments and algorithmic bias influence not just how much users engage, but also how they choose to participate in discussions.

\subsubsection{Descriptive and Variance Analysis}

\begin{table}[ht]
\centering
\small
\caption{Main Effects of Experimental Conditions on User Engagement}
\label{tab:interactions-anova}
\begin{tabularx}{\textwidth}{>{\raggedright\arraybackslash}p{2cm}>{\raggedright\arraybackslash}p{1.8cm}*{3}{>{\centering\arraybackslash}X}>{\centering\arraybackslash}p{1.2cm}>{\centering\arraybackslash}p{1cm}}
\toprule
\multirow{2}{*}{\textbf{Metric}} & \multirow{2}{*}{\textbf{Condition}} & \multicolumn{3}{c}{\textbf{Recommendation Bias}} & \multicolumn{2}{c}{\textbf{ANOVA}} \\
\cmidrule(lr){3-5} \cmidrule(lr){6-7}
& & Contra & Balanced & Pro & $F$ & $p$ \\
\midrule
\multirow{2}{*}{Interactions} 
   & Polarized & \textbf{10.30} (12.43) & 4.82 (4.49) & 4.80 (5.29) & \multirow{2}{*}{\textbf{3.25}$^a$} & \multirow{2}{*}{.042*} \\
   & Unpolarized & 7.92 (7.62) & 4.71 (3.26) & \textbf{9.25} (12.85) & & \\
\midrule
\multirow{2}{*}{Likes} 
   & Polarized & \textbf{5.65} (9.53) & 2.55 (2.65) & 2.35 (2.54) & \multirow{2}{*}{2.25$^a$} & \multirow{2}{*}{.110} \\
   & Unpolarized & 4.38 (4.03) & 2.64 (2.44) & \textbf{4.95} (8.22) & & \\
\midrule
\multirow{2}{*}{Reposts} 
   & Polarized & 0.65 (1.03) & 0.45 (0.67) & 0.40 (0.82) & \multirow{2}{*}{1.89$^a$} & \multirow{2}{*}{.156} \\
   & Unpolarized & \textbf{1.08} (2.30) & 0.29 (0.46) & \textbf{1.10} (1.86) & & \\
\midrule
\multirow{2}{*}{Comments} 
   & Polarized & 0.35 (0.57) & 0.23 (0.43) & 0.25 (0.55) & \multirow{2}{*}{\textbf{6.51}$^b$} & \multirow{2}{*}{.012*} \\
   & Unpolarized & 0.50 (1.18) & 0.71 (1.12) & \textbf{1.10} (1.94) & & \\
\midrule
\multirow{2}{*}{Follows} 
   & Polarized & \textbf{3.30} (4.12) & 1.45 (1.84) & 1.50 (2.82) & \multirow{2}{*}{2.60$^a$} & \multirow{2}{*}{.078} \\
   & Unpolarized & 1.88 (3.38) & 1.04 (1.57) & 1.95 (3.33) & & \\
\bottomrule
\multicolumn{7}{p{.95\textwidth}}{\small \textbf{Note:} Values show means with standard deviations in parentheses. Bold values indicate highest means within recommendation bias conditions. $^a$F-statistic for main effect of Recommendation (df = 2, 134). $^b$F-statistic for main effect of Polarization (df = 1, 135). *$p$ < .05} \\
\end{tabularx}
\end{table}


Our analysis revealed distinct patterns of user engagement across the experimental conditions. Of the total sample, participants generated $946$ interactions throughout the study period, with individual engagement levels varying substantially ($M = 6.91$, $SD = 8.50$, $Mdn = 5.00$, Range: $0-56$). The majority of participants ($81.02\%$) engaged at least once with the content. Among active users, engagement levels were distributed across three categories: low engagement ($1$-$5$ interactions; $36.5\%$ of participants), moderate engagement ($6$-$10$ interactions; $24.82\%$), and high engagement ($>10$ interactions; $19.71\%$).


Figure~\ref{fig:stacked-interaction-distribution} illustrates the distribution of interaction types across conditions. The visualization reveals a clear hierarchy in users' preferred forms of engagement, with \emph{likes} consistently representing the dominant form of interaction, accounting for $54.02\%$ of all engagements ($M = 3.73$, $SD = 5.63$). This was followed by \emph{follows} ($26.53\%$; $M = 1.83$, $SD = 2.98$), \emph{reposts} ($9.41\%$; $M = 0.65$, $SD = 1.36$), and \emph{comments} ($7.61\%$; $M = 0.53$, $SD = 1.11$). This pattern suggests a preference for low-effort engagement forms over more demanding interactions like commenting or reposting.

Correlation analysis revealed strong associations between certain interaction types. \emph{Total interactions} showed strong positive correlations with \emph{likes} ($r = .93$, $p < .001$) and \emph{follows} ($r = .73$, $p < .001$), moderate correlations with \emph{comments} ($r = .45$, $p < .001$), and weaker correlations with \emph{reposts} ($r = .37$, $p < .001$). Notably, \emph{comments} and \emph{reposts} showed minimal correlation with each other ($r = .01$, $p = .899$), suggesting these forms of engagement might serve distinct purposes for users.

\begin{figure}[h!]
    \centering
    \includegraphics[width=\textwidth]{figures/interaction_distribution_stacked.png}
    \caption{Distribution of interaction types across experimental conditions. The stacked bars show the relative proportion of different interaction types (\emph{likes}, \emph{reposts}, \emph{comments}, and \emph{follows}) for each combination of polarization level and recommendation bias. Total interaction counts are displayed above each bar.}
    \label{fig:stacked-interaction-distribution}
\end{figure}


\begin{figure}[h!]
    \centering
    \includegraphics[width=\textwidth]{figures/interaction_plots_activity.png}
    \caption{Interaction plots showing the effects of polarization and recommendation type on different forms of user engagement. Red lines represent the polarized condition, green lines represent the unpolarized condition. Error bars represent $95\%$ confidence intervals.}
    \label{fig:interaction-plot-recommendation-bias}
\end{figure}


Analysis of variance (see Table~\ref{tab:interactions-anova} and Figure~\ref{fig:interaction-plot-recommendation-bias}) revealed several significant effects of our experimental manipulations. For \emph{total interactions}, we found a significant main effect of \emph{Recommendation Bias} ($F(2,134) = 3.25$, $p = .042$, $\eta^2_p  = .046$). Post-hoc analyses indicated in the contra conditions ($M = 9.11$, $SD = 10.02$) significantly more interactions were generated than in the balanced ($M = 4.77$, $SD = 3.88$, $p = .008$, Hedges' $g = 0.56$). The comparison between contra and and pro-bias ($M = 7.03$, $SD = 9.07$) did not reach significance ($p = .345$), nor did the comparison between balanced and pro-bias ($p = .180$).

The analysis of specific interaction types revealed distinct patterns. Most notably, we found a significant main effect of \emph{Polarization Degree} on \emph{commenting} behavior ($F(1,135) = 6.51$, $p = .012$, $\eta^2_p = .046$). Participants in unpolarized conditions exhibited higher commenting rates ($M = 0.77$, $SD = 1.41$) compared to those in polarized conditions ($M = 0.28$, $SD = 0.52$), suggesting that a less polarized environment might facilitate more substantive engagement through \emph{comments}.

While other interaction types did not show significant main effects, several trending patterns emerged. The analysis of following behavior revealed a marginal effect of \emph{Recommendation Bias} ($F(2,134) = 2.60$, $p = .078$, $\eta^2_p  = .037$), with participants showing a tendency to follow more users when exposed to opposing views ($M = 2.59$, $SD = 3.75$) compared to balanced content ($M = 1.25$, $SD = 1.71$). Similarly, \emph{likes} showed a pattern consistent with \emph{total interactions}, though the effect did not reach statistical significance ($F(2,134) = 2.25$, $p = .110$, $\eta^2_p  = .032$).

Examining the combined effects of \emph{Polarization Degree} and \emph{Recommendation Bias} revealed interesting patterns in user behavior. In polarized conditions, participants showed the highest level of engagement in the contra condition ($M = 10.30$, $SD = 12.43$), while engagement with neutral ($M = 4.82$, $SD = 4.49$) and pro ($M = 4.80$, $SD = 5.29$) was notably lower. In unpolarized conditions, engagement was more evenly distributed across recommendation types, though still elevated for contra ($M = 7.92$, $SD = 7.62$) and pro-bias ($M = 9.25$, $SD = 12.85$) compared to balanced ($M = 4.71$, $SD = 3.26$).

\subsubsection{Discussion}

The patterns of user activity reveal nuanced insights into how \emph{Polarization Degree} and \emph{Recommendation Bias} shape engagement in online discussions. Our analyses demonstrate that these factors influence not merely the quantity of interactions but also alter how users choose to participate in debates.

The clear hierarchy in engagement forms---with \emph{likes} ($54.02\%$) dominating over \emph{follows} ($26.53\%$), \emph{reposts} ($9.41\%$), and \emph{comments} ($7.61\%$)---reflects fundamental patterns in social media behavior. Notably, the minimal correlation between \emph{comments} and \emph{reposts} ($r = .01$) challenges the intuition that all forms of active engagement serve similar functions. Instead, these behaviors may represent distinct modes of participation, with commenting indicating dialectical engagement while reposting signals content amplification.

The relationship between \emph{Polarization Degree} and commenting behavior ($\eta^2_p = .046$) warrants careful interpretation. While higher commenting rates in unpolarized conditions appear to suggest that moderate discourse environments foster more substantive engagement \citep{koudenburg_polarized_2022, yousafzai_political_2022}, this interpretation requires qualification. Our sample exhibited predominantly moderate attitudes toward UBI, making them potentially unrepresentative of users who actively engage in polarized discussions \citep{simchon_troll_2022}. The reduced commenting in polarized conditions might thus reflect a mismatch between discussion climate and user predispositions rather than an inherent effect of polarization.

The significant effect of \emph{Recommendation Bias} on \emph{total interactions} ($\eta^2_p = .046$) is particularly noteworthy in the context of opinion formation. The higher engagement in the polarized contra-bias condition ($M = 10.30$) coincided with the strongest observed opinion shifts ($M = -0.408$). Given that UBI represents a proposal for systemic economic change, arguments against it may have resonated more strongly with users' status quo bias and loss aversion. In a polarized environment, these contra-UBI arguments might have appeared particularly salient and consequential, leading to both increased engagement and stronger opinion shifts. This suggests that the combination of topic-specific factors---namely the potentially threatening nature of economic system changes---with polarized discourse might amplify user engagement, particularly when arguments align with psychological tendencies toward preserving existing systems.

While these findings provide initial insights into the relationship between platform design, user engagement, and opinion dynamics, several limitations suggest the need for more extensive research. The short-term nature of our study and its focus on a single topic limit generalizability. Future research should pursue longitudinal studies comparing topics with varying degrees of polarization and personal involvement to distinguish between topic-specific effects and general patterns of online discourse dynamics. Our findings thus represent a starting point for understanding the complex interplay between platform design, user behavior, and opinion dynamics in online discussions.

\section{Overall Discussion}

This study presents a novel methodological framework for investigating online polarization through controlled experimental manipulation of social media environments using LLM-based artificial agents. Our findings demonstrate that this approach can successfully reproduce key characteristics of polarized online discourse while enabling precise control over environmental factors that shape user perceptions and behaviors. The integration of sophisticated language models with traditional opinion dynamics frameworks represents a significant advancement in our ability to study the microfoundations of polarization processes.

Our experimental framework advances beyond observational and theoretical approaches by providing systematic empirical evidence for how online environments influence user perceptions and behaviors. Our results demonstrate that group identity processes are fundamental to polarization dynamics in online environments, even when interactions occur with artificial agents. This finding extends social identity theory in important ways: where \citep{tajfel_integrative_1979} established how group identities could emerge from minimal categorical distinctions, and \citep{huddy_social_2001} emphasized the need to examine real-world complexity of identity formation, our work reveals how these processes manifest in digital spaces.

The amplification of group-based polarization we observe aligns with fundamental theories of political identity formation. \citep{laclau_hegemony_2014}'s concept of the constitutive outside posits that political identities fundamentally emerge and strengthen through the recognition of an opposing force---an "other" against which the group defines itself. This process is closely related to what \citep{schmitt_concept_2008} terms the friend-enemy distinction, where political identities crystallize around the identification of a fundamental antagonist. Our offline evaluation provides empirical support for these theoretical perspectives: agents in moderate homophily conditions showed significantly higher polarization than those in high homophily conditions. This finding aligns with \citep{bateson_naven_1958}'s notion of complementary schismogenesis---a process where interaction between groups leads to a progressive differentiation of their behaviors and identities, with each group's actions eliciting more extreme counter-actions from the other. This challenges the predominant ``echo chamber'' framework \citep{bakshy_exposure_2015} for understanding online polarization, suggesting that exposure to opposing views, rather than isolation, can intensify group identities.

Our user study reinforces this understanding of polarization as an active process of group differentiation, showing that participants perceived significantly higher emotional content and group identity salience in polarized conditions. This finding substantiates \citep{bliuc_online_2021}'s theoretical framework, which emphasizes how conflicting collective narratives---rather than mere isolation---drive polarization. Our observation that polarized conditions led to reduced uncertainty expression reveals a key mechanism in this process: the replacement of epistemic humility with group-based certainty. This pattern aligns with \citep{mason_cross-cutting_2016}'s findings on emotional reactivity in political messaging, while extending them by demonstrating these effects in controlled, artificial environments. Similarly, it complements \citep{fischer_emotion_2023}'s analysis of anger amplification in social media by revealing how increased certainty accompanies heightened emotional expression in polarized discourse.

The relationship between polarization and emotional discourse that we observe extends contemporary models of affective polarization in significant ways. While \citep{iyengar_fear_2015} demonstrated how partisan identities drive emotional responses in political contexts, our findings reveal a more complex dynamic centered on group-based processing. Our path analysis shows that polarized conditions significantly reduce uncertainty expression while increasing both emotional content and group salience. However, the relationship between these factors is more nuanced than initially apparent: while polarization directly affects all these dimensions, our structural equation model reveals that group salience, rather than emotional content or uncertainty, plays the key mediating role in shaping polarization perceptions. This builds upon \citep{albertson_dog-whistle_2015}'s work on group-based political messaging by showing how these identity-based dynamics operate in digital environments where group boundaries become particularly salient.

These findings collectively suggest that online polarization emerges primarily from the fundamental role of group opposition in political identity formation, rather than from mere information exposure patterns or purely affective mechanisms. This helps explain why simple exposure to opposing views often fails to reduce polarization \citep{bail_exposure_2018} and suggests that effective interventions must address the deeper dynamics of group identity formation in digital spaces.

The methodological innovation of our approach is particularly strengthened by the convergence between offline LLM-based assessments and human participants' perceptions (cf. Sections~\ref{subsec:offline-message-analysis} and~\ref{subsec:debate-perception}). Both artificial agents and human participants demonstrated consistent patterns in evaluating emotional intensity, uncertainty expression, and group identity salience across conditions. This dual-validation approach combines the advantages of systematic large-scale analysis with crucial insights into human perception and behavior, while demonstrating that LLM-based content analysis effectively captures psychologically relevant aspects of online discourse.

By establishing the feasibility of using LLM-based agents to create controlled yet ecologically valid social media environments, we provide researchers with a powerful new tool for studying online social dynamics. This approach enables precise manipulation of discourse characteristics, allows for systematic variation in network structure and content exposure, and facilitates the collection of fine-grained behavioral data that would be difficult to obtain in naturalistic settings. Moreover, our demonstration that artificial agents can create environments that elicit authentic human responses challenges assumptions about the necessary conditions for studying social media effects, suggesting that users respond to perceived rather than actual social presence.

These methodological advances have important implications for ongoing debates about social media's role in democratic discourse. While existing theories often frame platform effects as either uniformly negative (promoting polarization) or positive (enabling diverse discourse), our results suggest a more complex theoretical framework is needed. The observed interaction between recommendation algorithms, discourse characteristics, and user behavior indicates that platform effects are highly context-dependent and mediated by multiple psychological and social processes. This suggests the need for more sophisticated theoretical models that can account for the dynamic interplay between technological affordances, social psychological processes, and content characteristics.

%The systematic variation we observed in engagement patterns across different polarization conditions provides new theoretical insights into how platform design choices shape user behavior, moving beyond simple deterministic models of technology effects. These findings suggest that certain aspects of polarized discourse may be more structurally determined than previously thought, pointing to the need for theories that better integrate technological and social explanations of polarization.

However, several important limitations must be acknowledged. The short-term nature of our experimental exposure means we cannot make definitive claims about long-term polarization dynamics. While our results demonstrate that the simulation framework can reproduce key features of polarized discourse and elicit expected user responses, questions remain about how these effects might evolve over extended periods of interaction. Additionally, the artificial nature of the experimental environment, despite its ecological validity, may not capture all relevant aspects of real-world social media use. 

Specifically, our methodological approach necessitated making several assumptions about parameter values and interaction dynamics in the absence of comprehensive empirical data. The framework's sophisticated non-linear functions, while capturing key phenomena like confirmation bias and backfire effects, introduce numerous parameters that currently lack empirical validation. Although this was a deliberate choice to create a detailed proxy for polarized discourse, the high number of free parameters raises valid concerns about potential overfitting and the robustness of our findings. While we designed the framework with future empirical calibration in mind, the current parameter values remain largely theoretical and may not accurately reflect real-world behavioral patterns. This limitation is particularly important when interpreting our results, as different parameter configurations could potentially lead to significantly different outcomes. The framework's detailed parameterization, though offering flexibility for future refinement, also increases model complexity and makes it more challenging to identify which specific mechanisms drive observed effects.

Another significant limitation concerns our focus on Universal Basic Income as the sole discussion topic. While UBI's characteristics - moderate pre-existing polarization and potential for opinion formation - made it suitable for our initial investigation, this single-topic approach limits generalizability. Different topics may generate distinct polarization dynamics: established political issues might show stronger ideological entrenchment, while technical discussions could exhibit different group formation patterns. The observed effects of emotional content and group identity salience might vary across topics with different emotional resonance or existing group affiliations.

The sample characteristics and recruitment strategy present additional methodological concerns. The participant pool, recruited through Prolific, skewed toward younger, highly educated, and moderate individuals, potentially limiting our ability to understand how more diverse populations or those with stronger initial positions might interact with polarized environments. Moreover, while our experimental platform successfully reproduced key social media features, the simplified interface lacks many nuanced interaction affordances present in real platforms that might influence user behavior in important ways. Our current implementation also assumes English-language discourse patterns, which may not generalize to other cultural and linguistic contexts where polarization dynamics could manifest differently.

The role of individual differences and contextual factors also remains incompletely explored. The study's design, while allowing for controlled manipulation of environmental features, could not fully account for how personal characteristics such as digital literacy, prior platform experience, or general political engagement might moderate the observed effects. Additionally, the fixed nature of agent behaviors, though methodologically necessary, may not capture the dynamic adaptations that characterize human discourse patterns, particularly in heated debates where rhetorical strategies often evolve in response to opponent reactions.

Future research should address these limitations by conducting longitudinal studies that track user behavior and attitude change over extended periods. The framework could be extended to incorporate more sophisticated network dynamics, explore the role of different content recommendation algorithms, and investigate potential intervention strategies. Additionally, cross-platform studies could examine how different interface designs and interaction affordances influence polarization processes. Studies examining multiple topics simultaneously could help distinguish universal polarization mechanisms from topic-specific effects, while varying topic characteristics (e.g., emotional salience, technical complexity, moral loading) could reveal how content domain influences polarization dynamics.

Particularly promising directions for future work include investigating the role of influencer dynamics in polarization processes, exploring how different moderation strategies affect discourse quality, and examining how varying levels of algorithmic content curation influence user behavior and perception. The framework could also be adapted to study other aspects of online social behavior, such as information diffusion patterns or the emergence of new social norms.
\section{Conclusion}
In this work, we propose a simple yet effective approach, called SMILE, for graph few-shot learning with fewer tasks. Specifically, we introduce a novel dual-level mixup strategy, including within-task and across-task mixup, for enriching the diversity of nodes within each task and the diversity of tasks. Also, we incorporate the degree-based prior information to learn expressive node embeddings. Theoretically, we prove that SMILE effectively enhances the model's generalization performance. Empirically, we conduct extensive experiments on multiple benchmarks and the results suggest that SMILE significantly outperforms other baselines, including both in-domain and cross-domain few-shot settings.

%% following line to enable line numbers
%% \linenumbers

\bibliographystyle{elsarticle-harv.bst}  
\bibliography{bib}

\end{document}

\endinput
%%
%% End of file `elsarticle-template-num.tex'.
