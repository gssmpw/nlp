\section{Introduction}
\label{sec:intro}

The digital era has transformed how individuals engage with public discourse, granting unprecedented access to platforms for exchanging information and debating topical issues. While these online networks facilitate large-scale interactions and democratic participation, they also pose significant challenges to societal cohesion. A prominent concern is the intensification of social polarization---the emergence of ideologically opposed groups that often struggle to find common ground \cite{grover_dilemma_2022, kubin_role_2021, bail_exposure_2018}.

Current research addressing online polarization largely divides into two methodological streams. On one hand, observational analyses leverage large-scale data from social media platforms, applying sophisticated techniques such as sentiment analysis \cite{karjus_evolving_2024, alsinet_measuring_2021, buder_does_2021}, network clustering \cite{treuillier_gaining_2024, bond_political_2022, al_amin_unveiling_2017}, and topic modeling \cite{kim_polarized_2019, chen_modeling_2021}. These studies offer insight into real-world polarization patterns but rely on passively obtained data that cannot be manipulated experimentally, thus limiting causal inferences. On the other hand, theoretical and simulation-based research uses formal opinion dynamics models to explore the underlying mechanisms of polarization \cite{hegselmann_opinion_2002,degroot_reaching_1974,sasahara_social_2021,del_vicario_modeling_2017}. Such models allow for controlled variable manipulation but often simplify social interactions in ways that may underrepresent the complexities of actual communication behaviors.

Bridging these two traditions requires empirical investigations that place human participants into experimentally controlled but contextually realistic environments. Advances in large language models (LLMs) offer new opportunities to develop more nuanced simulations of online behavior and discourse \cite{chuang_simulating_2024,breum_persuasive_2024,ohagi_polarization_2024}. However, the literature still lacks comprehensive user studies that integrate LLM-based artificial agents into a systematic framework for studying polarization. Empirical research with human participants interacting alongside artificial agents has the potential to illuminate the causal pathways of opinion formation and polarization in a manner neither observational data nor simplified simulations can fully capture.

In this article, we present a novel framework that marries mathematical opinion dynamics principles with LLM-based artificial agents in a synthetic social network platform. Building on an underlying agent-based model, we implement a robust offline validation process to verify how well our LLM-driven agents reproduce realistic communication patterns and polarization dynamics. Subsequently, we conduct a user study to assess how real participants engage with the resulting debate space, measuring changes in their opinions, perceptions of the platform, and interaction behaviors before and after exposure to polarized content.

Our study makes three overarching contributions:
\begin{enumerate}
    \item \textbf{Integration of Theory and Advanced Simulation:} We introduce a platform that unites formal opinion dynamics with LLM-based agents, thereby enabling more sophisticated modeling and ecological validity than existing approaches.
    \item \textbf{Comprehensive Offline and Online Validation:} We demonstrate, through extensive offline tests, that our LLM-driven agents can replicate key features of polarized discourse. We then validate these findings in a live user study to illuminate how actual participants respond to such environments.
    \item \textbf{Framework for Future Research:} We provide a reproducible experimental pipeline that other researchers can adopt or extend to study areas like echo-chamber formation, information diffusion, and intervention strategies aimed at mitigating harmful polarization.
\end{enumerate}

Our results suggest that a polarized environment influenced by agent-generated content can intensify perceived emotional stakes and group identity markers among human participants, while recommendation systems further shape patterns of engagement in polarized contexts. Taken together, these findings underscore the methodological value of combining offline simulation techniques with empirical user studies, paving the way for deeper insights into how polarized discourses arise, evolve, and might be counteracted.

The remainder of this paper is organized as follows. Section~\ref{sec:related_work} situates our work within the broader literature on online polarization and LLM-based simulations, underscoring the need for a novel experimental bridge between theoretical and user-focused research. Section~\ref{sec:simulation_model} details our synthetic social network and the integration of LLM-based agents. Section~\ref{sec:offline_evaluation} presents the offline validation of our agents’ polarization behaviors, while Section~\ref{sec:user_study} describes the user study design and key findings. In Section~\ref{sec:discussion}, we reflect on the broader implications of our results and future directions, and we conclude in Section~\ref{sec:conclusion}.
