\section{Conclusion}

This study introduces a novel methodological framework for investigating online polarization through the integration of LLM-based artificial agents and controlled social network simulation. Our findings demonstrate both the technical feasibility and empirical utility of this approach, providing researchers with a powerful new tool for studying social media dynamics. The successful reproduction of key polarization characteristics, validated through both offline evaluation and user testing, suggests that synthetic social networks can serve as valuable experimental environments for investigating online social phenomena.

The convergence between computational predictions and human responses not only validates our methodological approach but also provides empirical support for theoretical models of opinion dynamics and polarization. The observed effects of polarized environments on user perceptions and behaviors highlight the complex interplay between platform design, content characteristics, and social interaction patterns in shaping online discourse.

Looking forward, this framework opens new avenues for investigating various aspects of online social behavior, from testing intervention strategies to examining the impact of different platform features on user engagement. While our current study focused on short-term effects, the methodology we have developed provides a foundation for longer-term investigations of polarization dynamics and other social media phenomena. As online platforms continue to evolve and shape public discourse, such controlled experimental approaches will become increasingly valuable for understanding and potentially mitigating harmful polarization effects while promoting more constructive online dialogue.