\section{System Prototype}

The prototype implementation consists of a web application that simulates a social media platform, reminiscent of Twitter, to study social polarization dynamics. The interface, as depicted in the provided screenshot, adheres to a familiar social media layout, facilitating user engagement and interaction.

\subsection{User Interface}

The application's main interface is divided into three primary sections: a navigation sidebar on the left, a central Newsfeed, and a recommendation panel on the right. The navigation sidebar provides quick access to essential functionalities such as the user's profile, a general user overview, and a logout option. The central Newsfeed serves as the primary interaction space, where users can view and engage with posts from other users. At the top of the Newsfeed, a text input area invites users to share their thoughts, mimicking the spontaneous nature of social media communication.

The Newsfeed displays a series of posts, each accompanied by user avatars, usernames, timestamps, and interaction metrics such as comments, reposts, and likes. This design encourages user engagement and provides visual cues about the popularity and impact of each post. The recommendation panel on the right side of the interface suggests other users to follow, potentially influencing the user's network expansion and exposure to diverse viewpoints.

User profiles are dynamically generated, displaying the user's posts, follower relationships, and other relevant metadata like a user's biography. It is also possible to follow and unfollow artificial users.

\subsection{Newsfeed Recommendations}

The web application implements an adaptive recommendation system for content presentation that evolves with user engagement. This system employs two distinct algorithmic approaches: a default variant for initial users and a collaborative variant that activates once users establish an interaction history.

The default variant implements a popularity-based scoring mechanism that considers multiple forms of engagement to determine content visibility. For a given message $m$, the system calculates a composite popularity score:

\begin{align}
    S_p(m) = l_m + 2c_m + 3r_m
\end{align}

where $l_m$, $c_m$, and $r_m$ represent the number of the message's likes, comments, and reposts respectively. The weighted coefficients reflect the relative importance assigned to different forms of engagement, with more active forms of interaction carrying greater weight.

As users begin to interact with the platform, the system transitions to a collaborative variant that incorporates both popularity metrics and ideological proximity. The enhanced scoring function combines these elements into a composite score:

\begin{align}
    S_c(m) = \omega_p \cdot \frac{S_p(m)}{S_{max}} + \omega_i \cdot (2 - |o_u - o_a|)
\end{align}

where $S_p(m)$ represents the popularity score normalized by the maximum observed score $S_{max}$, $o_u$ and $o_a$ denote the opinion scores of the active user and message author respectively, and $\omega_p$ and $\omega_o$ are weighting parameters that balance the influence of popularity and opinion similarity.

Both variants maintain temporal relevance by presenting the user's most recent content contributions at the beginning of their feed when accessing the first page. This approach ensures users maintain awareness of their own contributions while experiencing the broader content landscape through the scoring-based recommendations.

The system implements this recommendation logic through a paginated query structure that enables efficient content delivery while maintaining scoring consistency. For each page request, the system computes scores for eligible messages—those from artificial agents that the user hasn't previously interacted with—and returns the highest-scoring subset based on the specified page size.

This dual-variant approach enables the system to provide meaningful content recommendations even in the absence of user interaction data while transitioning smoothly to more personalized recommendations as users engage with the platform. The incorporation of both popularity metrics and ideological factors creates a balanced recommendation environment that promotes engaging content while maintaining exposure to diverse perspectives.