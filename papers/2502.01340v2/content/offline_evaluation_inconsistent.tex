\section{Offline Evaluation}
\label{sec:offline-evaluation}

This section presents a comprehensive analysis of our simulation model to understand the dynamics of opinion polarization and user behavior in online social networks. Our evaluation serves as a foundation for the subsequent experimental user study and aims to validate that our model captures realistic aspects of online discourse while reflecting theoretically grounded mechanisms of opinion formation and polarization.

\begin{table*}[h!]
\centering
\small
\caption{Key Findings from Offline Evaluation}
\label{tab:offline-evaluation-findings}
\begin{tabularx}{\textwidth}{>{\centering\arraybackslash}p{1.5cm}>{\raggedright\arraybackslash}X>{\raggedright\arraybackslash}X}
\toprule
\multicolumn{1}{c}{\textbf{Analysis}} & \multicolumn{1}{c}{\textbf{Key Finding}} & \multicolumn{1}{c}{\textbf{Theoretical Implication}} \\
\midrule
\multirow{2}{*}{\rotatebox[origin=c]{90}{\textbf{Polarization}}} 
    & Moderate cross-ideological exposure leads to higher polarization than complete echo chambers 
    & Supports theories that political identities strengthen through opposition rather than isolation; polarization intensifies through awareness of opposing views \\
\cmidrule{2-3}
    & Polarization emerges primarily through the presence of influential accounts at extreme positions, not through natural opinion drift 
    & Confirms role of opinion leaders in transforming neutral topics into partisan issues; suggests polarization requires active curation rather than emerging spontaneously \\
\midrule
\multirow{2}{*}{\rotatebox[origin=c]{90}{\textbf{Interaction}}} 
    & Different interaction types serve distinct social functions: likes show strict in-group preference while comments flourish across ideological lines 
    & Reflects how platform affordances shape identity expression; suggests comments often serve as vehicles for performative disagreement rather than dialogue \\
\cmidrule{2-3}
    & Influencer content becomes focal point for engagement in polarized conditions, creating self-reinforcing cycles 
    & Shows how influential accounts serve as lightning rods for cross-ideological conflict; engagement patterns amplify rather than reduce polarization \\
\midrule
\multirow{2}{*}{\rotatebox[origin=c]{90}{\textbf{Content}}} 
    & Polarized environments produce consistent changes in communication: increased group identity emphasis, higher emotional content, and reduced expressions of uncertainty 
    & Demonstrates how polarization fundamentally transforms communication style independent of specific topics; suggests activation of group identities changes how people express themselves \\
\cmidrule{2-3}
    & These changes in communication style occur independently of content exposure patterns 
    & Indicates that the social-psychological dynamics of polarized environments, rather than mere exposure to different views, drive changes in how people communicate \\
\bottomrule
\end{tabularx}
\end{table*}

Table~\ref{tab:offline-evaluation-findings} provides an overview of our key findings across three main analyses. First, we examine polarization dynamics under varying conditions, investigating how different recommendation algorithms (through varied same-stance ratios) and initial opinion distributions affect the evolution of opinions across the network. This analysis focuses particularly on the role of cross-ideological exposure in opinion formation. Second, we analyze the evolution of user engagement patterns in different network configurations. This investigation explores how reaction probabilities and interaction patterns develop over time, helping us understand the relationship between opinion polarization and user behavior in our simulated environment. Finally, we perform a content-based analysis of messages generated within our simulated networks. This examination focuses on message characteristics such as emotionality, uncertainty, and group identity salience, providing insights into how content features vary under different network conditions. Through these three analyses, we aim to validate our model's capacity to reproduce empirically observed patterns of online polarization while offering a detailed understanding of the underlying mechanics of our simulation framework.

For our simulations, we selected Universal Basic Income (UBI) as the focal topic of discussion. This choice was driven by several strategic considerations. Unlike heavily polarized topics where individuals often hold entrenched positions, UBI represents an emerging policy proposal where public opinion remains relatively malleable, making it ideal for studying opinion formation and polarization dynamics. While UBI evokes fewer preset opinions, it remains sufficiently concrete and consequential to generate meaningful discourse, with its complexity spanning economic, social, and technological dimensions. Recent polling data supports UBI's suitability, showing a balanced distribution of opinions with approximately $51.2\%$ of Europeans \cite{vlandas_politics_2019} and $48\%$ of Americans \cite{hamilton_people_2022} expressing support, while significant portions remain undecided or hold moderate views. Additionally, UBI's limited real-world implementation means participants' opinions are more likely to be based on theoretical arguments rather than direct experience or partisan allegiances, allowing us to examine how social media dynamics can influence opinion formation before entrenched polarization takes hold.

\subsection{Polarization Analysis}

\subsubsection{Experimental Design}
This study employs a factorial design to investigate the effects of recommendation algorithms and initial opinion distributions on social polarization in online networks. The experiment aims to elucidate the dynamics of opinion formation and the emergence of polarization under various conditions.

Two primary independent variables are manipulated in this study. The first is the recommendation algorithm, characterized by the same-stance probability $p_s$. We examine two levels: high homophily ($p_s = 1.0$) and moderate homophily ($p_s = 0.7$). The second independent variable is the initial opinion distribution. We investigate three distinct distributions: a normal distribution ($\mu = 0$, $\sigma = 0.03$), a trimodal distribution with extreme influencers at both ends of the opinion spectrum ($\mu_1 = 0.8$, $\mu_2 = -0.8$) and regular users being initialized around $\mu_3 = 0$, and an asymmetric trimodal distribution with one-sided extreme influencers ($\mu_1 = 0.8$, $\mu_2 = -0.1$).

The primary dependent variables are the Esteban-Ray polarization index \citep{esteban_on_1994}, which measures the overall polarization within the network, and the evolution of individual opinions $o_i(t)$, which tracks individual agent opinion trajectories over time.

The simulation is configured with $n = 20$ agents over $15$ iterations, with $k = 8$ recommendations per iteration. Regular users have a posting probability $p_{reg} = 0.2$, while influencers post with probability $p_{inf} = 0.6$. The network includes $4$ influencers. These parameters were selected to balance computational feasibility with the need for sufficient interactions to observe meaningful dynamics.

The opinion update function follows the parameterization described in Section~\ref{subsec:agent-model}, with a base learning rate $\lambda = 0.2$ and alignment steepness $h = 1$. These parameters were selected based on preliminary experiments and theoretical considerations to capture realistic opinion dynamics while allowing for observable changes within the constrained simulation timeframe.

%For each combination of recommendation algorithm and initial opinion distribution, multiple simulation runs are conducted to account for stochastic variations. The polarization index and individual opinion trajectories are recorded at each iteration.

\begin{figure}[htbp]
    \centering
    \includegraphics[width=\textwidth]{figures/comparative_polarization_plot.png}
    \caption{Evolution of opinion polarization across different user groups and recommendation settings. The figure shows polarization dynamics for (left) the entire user population, (middle) regular users, and (right) influencers over multiple iterations. Lines represent different combinations of recommendation bias (SSR: Same Stance Ratio) and initial opinion distributions. Notably, moderate cross-ideological exposure (70\% SSR, solid lines) leads to higher polarization than complete echo chambers (100\% SSR, dashed lines), particularly in trimodal distributions. This counterintuitive finding challenges the echo chamber hypothesis, suggesting that awareness of opposing views can intensify polarization through antagonistic group dynamics. The effect varies across different initial opinion distributions: Unimodal, Trimodal (three distinct opinion clusters), and Asymmetric Trimodal (three unequally weighted clusters).}
    \label{fig:polarization-index}
\end{figure}


\subsubsection{Results}

The polarization index across different experimental conditions reveals distinct patterns of opinion dynamics and group polarization (see Figure~\ref{fig:polarization-index}). Our analysis combines visual inspection of polarization trends with statistical comparisons using t-tests and Area Under the Curve (AUC) measurements (see Table~\ref{tab:polarization-comparison}).

In scenarios with a normal initial distribution, we observe relatively low levels of polarization. The condition with moderate homophily ($p_s = 0.7$) shows minimal polarization, with $\rho$ fluctuating between 0.02 and 0.04 throughout the 15 iterations. Even with high homophily ($p_s = 1.0$), the polarization index remains comparatively low, stabilizing around 0.13-0.14 after an initial increase. This observation is supported by the low AUC values for normal distributions (0.850 and 1.760 for $p_s = 0.7$ and $p_s = 1.0$, respectively) compared to trimodal distributions.

Conversely, the trimodal initial distributions lead to substantially higher polarization. In the case of high homophily and symmetric trimodal distribution ($p_s = 1.0$), we see a steady increase in $\rho$ from 0.096 to 0.816 over the $15$ iterations. The effect of initial distribution is statistically significant at both $p_s$ levels. For regular users at $p_s = 0.7$, the difference is substantial ($t = 7.987$, $p < .001$), with AUC values of 14.629 for trimodal versus 0.850 for normal distributions.

The same-stance probability $p_s$ shows a substantial effect on polarization dynamics, particularly in trimodal scenarios. Contrary to the echo chamber hypothesis, moderate cross-ideological exposure ($p_s = 0.7$) results in significantly higher polarization ($\rho = 1.492$) compared to complete homophily ($p_s = 1.0$, $\rho = 0.816$). This is reflected in the AUC values for regular users (14.629 vs 7.239, $t = 3.809$, $p = .001$). This finding suggests that exposure to opposing views, rather than pure echo chambers, may actually intensify polarization through heightened awareness of ideological differences

The condition with asymmetric extreme influencers ($\mu_1 = 0.8$, $\mu_2 = -0.1$) demonstrates that even one-sided extreme opinions can drive significant polarization under high homophily, with $\rho$ increasing from 0.024 to 0.740. However, reducing recommendation homophily ($p_s = 0.7$) effectively mitigates this polarization, with $\rho$ peaking at only 0.148. This mitigation is confirmed by significantly different AUC values (5.079 vs 1.176 for regular users, $t = -4.620$, $p < .001$).

\begin{figure}[htbp]
    \centering
    \begin{minipage}[b]{0.3\textwidth}
        \centering
        \includegraphics[width=\textwidth]{figures/opinion_plots/individual_opinions_ssr_0.7_normal_2.png}
        \subcaption{$h_r = 0.7$, unimodal distribution}
        \label{fig:subfig1}
    \end{minipage}
    \hfill
    \begin{minipage}[b]{0.3\textwidth}
        \centering
        \includegraphics[width=\textwidth]{figures/opinion_plots/individual_opinions_ssr_0.7_trimodal.png}
        \subcaption{$h_r = 0.7$, trimodal}
        \label{fig:subfig2}
    \end{minipage}
    \hfill
    \begin{minipage}[b]{0.3\textwidth}
        \centering
        \includegraphics[width=\textwidth]{figures/opinion_plots/individual_opinions_ssr_0.7_trimodal_one.png}
        \subcaption{$h_r = 0.7$, asym. trimodal }
        \label{fig:subfig3}
    \end{minipage}
    
    \vspace{1em}
    
    \begin{minipage}[b]{0.3\textwidth}
        \centering
        \includegraphics[width=\textwidth]{figures/opinion_plots/individual_opinions_ssr_1.0_normal.png}
        \subcaption{$h_r = 1.0$, unimodal}
        \label{fig:subfig4}
    \end{minipage}
    \hfill
    \begin{minipage}[b]{0.3\textwidth}
        \centering
        \includegraphics[width=\textwidth]{figures/opinion_plots/individual_opinions_ssr_1.0_trimodal.png}
        \subcaption{$h_r = 1.0$, trimodal}
        \label{fig:subfig5}
    \end{minipage}
    \hfill
    \begin{minipage}[b]{0.3\textwidth}
        \centering
        \includegraphics[width=\textwidth]{figures/opinion_plots/individual_opinions_ssr_1.0_trimodal_one.png}
        \subcaption{$h_r = 1.0$, asym. trimodal}
        \label{fig:subfig6}
    \end{minipage}
    
    \caption{Opinion evolution for each individual user over $T = 15$ iterations subject to different same-stance connectivity probabilities ($h_r$) and initial opinion distributions (unimodal vs. trimodal. vs. asymmetric trimodal).}
    \label{fig:individual-polarization}
\end{figure}

\begin{table}[h!]
\centering
\caption{Effect of Initial Distribution, Homophily, and Symmetry on Polarization}
\label{tab:polarization-comparison}
\footnotesize
\begin{tabularx}{\textwidth}{>{\raggedright\arraybackslash}p{2.8cm}>{\centering\arraybackslash}p{1.2cm}*{6}{>{\centering\arraybackslash}X}}
\toprule
\multirow{2}{*}{\textbf{Exp. Factor}} & \multirow{2}{*}{\textbf{Users}} & \multicolumn{2}{c}{\textbf{AUC}} & \multicolumn{2}{c}{\textbf{Slope}} & \multicolumn{2}{c}{\textbf{Statistics}} \\
\cmidrule(lr){3-4} \cmidrule(lr){5-6} \cmidrule(lr){7-8}
 &  & Cond. 1 & Cond. 2 & Cond. 1 & Cond. 2 & $t$ & $p$ \\
\midrule
\textbf{Initial Distribution} & & \multicolumn{6}{l}{\textit{Trimodal vs Unimodal, $h_r = 0.7$}} \\
    & Overall & \textbf{16.628} & 0.845 & \textbf{0.071} & 0.002 & \textbf{13.057} & <.001*** \\
    & Regular & \textbf{14.629} & 0.850 & \textbf{0.101} & 0.003 & \textbf{7.987} & <.001*** \\
    & Influencers & \textbf{21.213} & 0.627 & \textbf{0.001} & -0.001 & \textbf{475.880} & <.001*** \\
\addlinespace[0.5em]
& & \multicolumn{6}{l}{\textit{Trimodal vs Unimodal, $h_r = 1.0$}} \\
    & Overall & \textbf{11.343} & 1.760 & \textbf{0.042} & 0.003 & \textbf{13.531} & <.001*** \\
    & Regular & \textbf{7.239} & 1.760 & \textbf{0.053} & 0.004 & \textbf{6.241} & <.001*** \\
    & Influencers & \textbf{20.627} & 1.357 & \textbf{0.017} & 0.001 & \textbf{44.437} & <.001*** \\
\midrule
\textbf{Homophily Level} & & \multicolumn{6}{l}{\textit{Moderate vs High ($h_r = 0.7$ vs $1.0$), Symmetric}} \\
    & Overall & \textbf{16.628} & 11.343 & \textbf{0.071} & 0.042 & \textbf{3.771} & .001** \\
    & Regular & \textbf{14.629} & 7.239 & \textbf{0.101} & 0.053 & \textbf{3.809} & .001** \\
    & Influencers & \textbf{21.213} & 20.627 & 0.001 & \textbf{0.017} & 1.612 & .118 \\
\addlinespace[0.5em]
& & \multicolumn{6}{l}{\textit{Moderate vs High ($h_r = 0.7$ vs $1.0$), Asymmetric}} \\
    & Overall & \textbf{8.402} & 4.938 & \textbf{0.039} & -0.010 & \textbf{5.236} & <.001*** \\
    & Regular & \textbf{5.079} & 1.176 & \textbf{0.052} & 0.007 & \textbf{4.620} & <.001*** \\
    & Influencers & \textbf{14.457} & 10.752 & \textbf{0.013} & -0.052 & \textbf{3.974} & <.001*** \\
\midrule
\textbf{Symmetry} & & \multicolumn{6}{l}{\textit{Symmetric vs Asymmetric, $h_r = 0.7$}} \\
    & Overall & \textbf{16.628} & 4.938 & \textbf{0.071} & -0.010 & \textbf{9.548} & <.001*** \\
    & Regular & \textbf{14.629} & 1.176 & \textbf{0.101} & 0.007 & \textbf{7.782} & <.001*** \\
    & Influencers & \textbf{21.213} & 10.752 & \textbf{0.001} & -0.052 & \textbf{12.407} & <.001*** \\
\addlinespace[0.5em]
& & \multicolumn{6}{l}{\textit{Symmetric vs Asymmetric, $h_r = 1.0$}} \\
    & Overall & \textbf{11.343} & 8.402 & \textbf{0.042} & 0.039 & \textbf{3.042} & .005** \\
    & Regular & \textbf{7.239} & 5.079 & \textbf{0.053} & 0.052 & 1.703 & .100 \\
    & Influencers & \textbf{20.627} & 14.457 & \textbf{0.017} & 0.013 & \textbf{10.960} & <.001*** \\
\bottomrule
\multicolumn{8}{p{.95\linewidth}}{\small \textbf{Note:} AUC = Area Under the Curve. "Cond. 1" and "Cond. 2" columns represent the respective conditions being compared. Bold values indicate larger values between conditions for AUC and Slope, and significant t-statistics ($p$ < .05). Positive t-statistics indicate higher polarization in the first condition. *$p$ < .05, **$p$ < .01, ***$p$ < .001} \\
\end{tabularx}
\end{table}

The evolution of individual opinions reveals distinct patterns across experimental conditions (see Figure~\ref{fig:individual-polarization}). In scenarios with normal initial distributions, opinions remain relatively moderate regardless of homophily level. Under moderate homophily ($p_s = 0.7$), opinions converge toward neutral values ($o_i \approx 0$), while high homophily ($p_s = 1.0$) induces slight opinion segregation ($o_i \rightarrow \pm0.2$) while maintaining overall moderate positions.

Symmetric trimodal distributions generate more pronounced opinion dynamics. Under moderate homophily, regular users migrate toward extreme positions ($o_i = \pm1$), particularly those starting near neutral positions. Interestingly, high homophily conditions show less extreme polarization, with opinions settling at moderate-high values ($|o_i| \approx 0.8$), suggesting that extreme homophily may paradoxically moderate opinion extremity.

The asymmetric condition with one-sided extreme influencers produces distinct dynamics based on homophily levels. Under moderate homophily, opinions cluster around moderate-high positive values ($o_i \approx 0.65$). High homophily amplifies this effect on the side with extreme influencers ($o_i \rightarrow 0.8$) while maintaining moderate opinions ($o_i \approx -0.1$) on the opposite side.

These results demonstrate the complex interplay between initial opinion distributions and recommendation algorithms in shaping polarization dynamics. While initial distributions largely determine polarization potential, recommendation homophily can either amplify or moderate these effects depending on the network configuration. The presence of extreme influencers significantly impacts opinion evolution, but their influence can be moderated through careful recommendation system design. Notably, moderate recommendation diversity alone proves insufficient to prevent polarization when extreme opinions are present, suggesting the need for more comprehensive approaches to managing opinion dynamics in social networks.

\subsubsection{Discussion}

The results of our polarization analysis reveal complex dynamics that both challenge and refine current understanding of opinion formation in social networks. Our findings particularly illuminate three key aspects of online polarization: the role of cross-ideological exposure, the impact of influential actors, and the mechanisms of opinion drift.

Our most crucial finding---that moderate cross-ideological exposure ($p_s = 0.7$) leads to higher polarization than complete echo chambers ($p_s = 1.0$)---aligns with fundamental theories of political identity formation. This result supports Mouffe and Laclau's concept of the constitutive outside, suggesting that political identities are strengthened through recognition of opposing views rather than isolation. The higher polarization observed under moderate exposure conditions demonstrates how awareness of opposing viewpoints can intensify group identities through what Bateson terms ``complementary schismogenesis''---a process where interaction between groups reinforces their differences rather than diminishing them. This finding challenges the conventional narrative about echo chambers while aligning with recent research on affective polarization, which suggests that mere exposure to opposing views can exacerbate rather than reduce political animosity.

The crucial role of extreme influencers in driving polarization, evidenced by the stark difference between trimodal ($\rho > 0.8$) and normal ($\rho \approx 0.03$) distributions, supports the two-step flow theory of communication. This theory, as developed by Katz and Lazarsfeld, suggests that mass opinion formation is mediated by opinion leaders who serve as focal points for attitude formation. Historical examples support this pattern: immigration policy in the United States remained largely non-partisan until the 1990s, when influential political actors began framing it as a partisan issue. Similarly, climate change was once a bipartisan concern, with both Republican and Democratic leaders supporting environmental protection measures in the 1970s. Even public health measures, historically viewed through a primarily scientific lens, became deeply polarized during the COVID-19 pandemic through the active efforts of political and media influencers. These cases demonstrate how previously neutral topics can become focal points for political division through strategic positioning by influential actors.

The asymmetric polarization condition ($\mu_1 = 0.8$, $\mu_2 = -0.1$) particularly illuminates how network effects interact with opinion leadership. Even with one-sided extreme influences, polarization emerges through what network theorists describe as asymmetric information flow patterns. This aligns with bounded confidence models in social simulations, which demonstrate how asymmetries in information flow can create opinion clusters even without opposing extreme influences. Recent examples from social media platforms demonstrate this effect: controversial statements by prominent figures can rapidly split previously moderate audiences into opposing camps, even when strong counter-narratives are initially absent.

The dynamics we observe in the normal initial distribution scenarios are particularly informative for understanding natural opinion drift. Without external triggers or strong personalities championing fringe viewpoints, opinions tend to remain clustered around moderate positions ($\rho \approx 0.03$). This supports coordination game theory's prediction that groups tend to converge on central focal points unless pulled toward extremes by external forces. The relative stability of moderate opinions in these scenarios suggests that polarization often requires active curation rather than emerging spontaneously from network dynamics alone.

%These findings have significant implications for platform design and intervention strategies. Rather than focusing solely on breaking down echo chambers, effective interventions might need to consider how to manage the influence of opinion leaders and structure cross-ideological interactions. The strong effects of influential actors suggest that platform policies might need to focus more on how information flows through network hierarchies rather than just on connection patterns between ideological groups. Current approaches to content moderation and recommendation systems might need to be reconsidered in light of these findings.

%Future research could extend these insights in several directions. First, investigating how different forms of cross-ideological exposure might affect polarization dynamics could help identify more constructive modes of engagement. Second, examining the role of temporal factors and adaptive recommendation algorithms could reveal opportunities for dynamic intervention strategies. Additionally, incorporating more sophisticated models of individual behavior could help understand how psychological factors like motivated reasoning interact with these structural dynamics.

Summarized, this analysis suggests that online polarization emerges from a complex interplay of social identity, opinion leadership, and network structure, rather than from simple exposure patterns. Understanding and addressing polarization requires attention to these multiple, interacting mechanisms rather than focusing on single factors like echo chambers or algorithmic curation alone. The path to reducing polarization may lie not in simply increasing exposure to diverse viewpoints, but in fundamentally rethinking how social platforms structure interaction patterns and manage the flow of influence through their networks.

\begin{figure}[h]
    \centering
    \includegraphics[width=\textwidth]{figures/comparative_reactions_plot.png}
    \caption{The reactions submitted and received (likes, comments, and reposts) are presented for both the polarized and non-polarized conditions. A distinction is made between influencers and regular users.}
    \label{fig:influencer-regular-reactions}
\end{figure}


\subsection{Interaction Analysis}

\subsubsection{Experimental Design}
The evaluation of interaction patterns analyzes how varying initial opinion distributions affect reaction patterns in our model. We simulated a network of $30$ agents ($24$ regular users, $3$ influencers for each stance) over $10$ iterations, with $8$ message recommendations per agent per iteration. Regular users had a posting probability of $0.2$, while influencers posted with probability $0.6$.

The comparison contrasts polarized and unpolarized conditions. The polarized condition used a bimodal opinion distribution (centered at $\pm0.8$, $\sigma = 0.1$), representing a divided community, while the unpolarized condition employed a single normal distribution (centered at $0$, $\sigma = 0.1$) for a more cohesive starting point. The agent's maintained their initially assigned opinion values throughout the simulation.

The reaction mechanism implements distinct hyperparameters for different engagement types: likes ($p_b = 0.7$, $c = 0.0$), reposts ($p_b = 0.3$, $c = 0.1$), and comments ($p_b = 0.3$, $c = 0.5$). All reaction types share an opinion strength importance of $w = 0.8$. Throughout the simulation, we tracked reaction frequencies, their distribution across opinion differences, and cross-stance interaction patterns.

\subsubsection{Results}

\begin{figure}[htbp]
    \centering
    \includegraphics[width=\textwidth]{figures/comparative_stance_reactions_plot.png}
    \caption{The same-stance and opposite-stance reactions (likes, comments, and reposts) are presented for both the polarized and non-polarized conditions. A distinction is made between influencers and regular users.}
    \label{fig:stance-reactions}
\end{figure}


Mixed effects modeling revealed substantial enhancement of engagement in polarized conditions across reaction types (see Table~\ref{tab:reaction-analysis}). For reactions submitted, polarization demonstrated significant positive effects across most interaction types, with particularly strong effects in influencers' commenting behavior ($\beta = 5.900$, $p = .018$). Temporal effects indicated consistent positive growth across all reaction types, with influencers showing the strongest growth rates in both likes ($\beta = 0.819$, $p < .001$) and comments ($\beta = 0.906$, $p < .001$).

The analysis of reactions received demonstrated pronounced polarization effects, especially for influencers compared to regular users, as illustrated in Figure~\ref{fig:influencer-regular-reactions}. Influencer-generated content elicited substantially more engagement in polarized conditions, particularly for comments ($\beta = 8.633$, $p = .026$), with robust temporal growth ($\beta = 1.620$, $p < .001$). Regular users showed more modest but significant positive effects of polarization across all reaction types.

Stance-based analysis revealed distinct patterns for cross-stance engagement (Figure~\ref{fig:stance-reactions}). Same-stance likes showed strong positive effects under polarization ($\beta = 3.127$, $p = .014$), while opposing-stance likes were very rare ($\beta = -1.663$, $p = .019$). Notably, opposing-stance likes were entirely absent in the polarized condition, while same-stance engagement flourished. Comments exhibited positive polarization effects for both same-stance ($\beta = 0.807$, $p = .010$) and opposing-stance interactions ($\beta = 1.680$, $p = .022$). 

A particularly notable pattern emerges in the analysis of cross-stance interactions. While positive endorsements (likes) show a clear in-group bias ($\beta = 3.127$, $p = .014$ for same-stance vs. $\beta = -1.663$, $p = .019$ for opposing-stance), more substantive forms of engagement exhibit different patterns. Comments show positive polarization effects for both same-stance ($\beta = 0.807$, $p = .010$) and opposing-stance interactions ($\beta = 1.680$, $p = .022$), with opposing-stance comments actually showing a stronger effect. Reposts occupy a middle ground, with significant positive effects for same-stance sharing ($\beta = 0.773$, $p = .017$) but no significant effect for opposing-stance sharing ($\beta = -0.083$, $p = .343$). These patterns suggest a qualitative difference in how users engage with opposing viewpoints across different interaction mechanisms.

These findings indicate that polarization fundamentally increases interaction dynamics, with particularly pronounced effects on influencer engagement and stance-based reactions, as evident in Table~\ref{tab:reaction-analysis}. The temporal effects demonstrate consistent growth patterns across conditions, with polarization amplifying the magnitude of engagement while maintaining stable growth trajectories.

\subsubsection{Discussion}

The simulation results demonstrate interaction patterns that align with several key phenomena observed in real social media environments while also revealing some notable divergences. The simulated amplification of engagement under polarized conditions mirrors documented patterns on online platforms, where polarizing content typically generates higher engagement rates \citep{simchon_troll_2022, horwitz_facebook_2020}. Particularly notable is our model's reproduction of the distinctive role of influencers in polarized discourse, reflecting empirical observations of opinion leader effects in online political discussions \citep{soares_influencers_2018, dubois_multiple_2014}.

\begin{table}[h!]
\centering
\footnotesize
\caption{Mixed Effects Analysis of User Reactions}
\label{tab:reaction-analysis}
\begin{tabularx}{\textwidth}{>{\raggedright\arraybackslash}p{2cm}>{\raggedright\arraybackslash}p{2cm}*{6}{>{\centering\arraybackslash}X}}
\toprule
\multirow{2}{*}{\textbf{Reaction}} & \multirow{2}{*}{\textbf{User Type}} & \multicolumn{2}{c}{\textbf{Polarization Effect}} & \multicolumn{2}{c}{\textbf{Temporal Effect}} & \multicolumn{2}{c}{\textbf{AUC}} \\
\cmidrule(lr){3-4} \cmidrule(lr){5-6} \cmidrule(lr){7-8}
 &  & $\beta$ & $p$ & $\beta$ & $p$ & Pol. & Unpol. \\
\midrule
\multicolumn{8}{l}{\textit{\textbf{Reactions Submitted}}} \\
\midrule
\multirow{3}{*}{Likes} 
    & Overall & \textbf{1.463} & 0.012* & \textbf{0.792} & <.001*** & \textbf{45.600} & 32.183 \\
    & Influencers & \textbf{3.317} & 0.001** & \textbf{0.819} & <.001*** & \textbf{55.917} & 25.250 \\
    & Regular & \textbf{1.000} & 0.044* & \textbf{0.785} & <.001*** & \textbf{43.021} & 33.917 \\
\midrule
\multirow{3}{*}{Comments} 
    & Overall & \textbf{2.487} & 0.017* & \textbf{0.392} & <.001*** & \textbf{30.400} & 7.950 \\
    & Influencers & \textbf{5.900} & 0.018* & \textbf{0.906} & <.001*** & \textbf{69.417} & 16.417 \\
    & Regular & \textbf{1.633} & 0.017* & \textbf{0.264} & <.001*** & \textbf{20.646} & 5.833 \\
\midrule
\multirow{3}{*}{Reposts} 
    & Overall & 0.690 & 0.066 & \textbf{0.237} & <.001*** & \textbf{14.250} & 8.150 \\
    & Influencers & 0.667 & 0.333 & \textbf{0.511} & <.001*** & \textbf{23.750} & 17.917 \\
    & Regular & \textbf{0.696} & 0.022* & \textbf{0.168} & <.001*** & \textbf{11.875} & 5.708 \\
\midrule
\multicolumn{8}{l}{\textit{\textbf{Reactions Received}}} \\
\midrule
\multirow{3}{*}{Likes} 
    & Overall & \textbf{1.463} & 0.012* & \textbf{0.792} & <.001*** & \textbf{45.600} & 32.183 \\
    & Influencers & 0.900 & 0.258 & \textbf{3.330} & <.001*** & \textbf{150.417} & 142.333 \\
    & Regular & \textbf{1.604} & 0.025* & \textbf{0.157} & <.001*** & \textbf{19.396} & 4.646 \\
\midrule
\multirow{3}{*}{Comments} 
    & Overall & \textbf{2.487} & 0.017* & \textbf{0.392} & <.001*** & \textbf{30.400} & 7.950 \\
    & Influencers & \textbf{8.633} & 0.026* & \textbf{1.620} & <.001*** & \textbf{110.917} & 33.750 \\
    & Regular & \textbf{0.950} & 0.012* & \textbf{0.085} & <.001*** & \textbf{10.271} & 1.500 \\
\midrule
\multirow{3}{*}{Reposts} 
    & Overall & 0.690 & 0.066 & \textbf{0.237} & <.001*** & \textbf{14.250} & 8.150 \\
    & Influencers & 1.967 & 0.191 & \textbf{1.055} & <.001*** & \textbf{53.583} & 36.667 \\
    & Regular & \textbf{0.371} & 0.012* & \textbf{0.032} & <.001*** & \textbf{4.417} & 1.021 \\
\midrule
\multicolumn{8}{l}{\textit{\textbf{Stance-based Reactions}}} \\
\midrule
\multirow{2}{*}{Likes} 
    & Same Stance & \textbf{3.127} & 0.014* & \textbf{0.639} & <.001*** & \textbf{45.600} & 17.233 \\
    & Opposing & \textbf{-1.663} & 0.019* & \textbf{0.153} & <.001*** & 0.000 & \textbf{14.950} \\
\midrule
\multirow{2}{*}{Comments} 
    & Same Stance & \textbf{0.807} & 0.010* & \textbf{0.158} & <.001*** & \textbf{11.133} & 3.817 \\
    & Opposing & \textbf{1.680} & 0.022* & \textbf{0.234} & <.001*** & \textbf{19.267} & 4.133 \\
\midrule
\multirow{2}{*}{Reposts} 
    & Same Stance & \textbf{0.773} & 0.017* & \textbf{0.146} & <.001*** & \textbf{11.117} & 4.183 \\
    & Opposing & -0.083 & 0.343 & \textbf{0.090} & <.001*** & 3.133 & 3.967 \\
\bottomrule
\multicolumn{8}{p{.95\linewidth}}{\small \textbf{Note:} Polarization Effect represents the increase in engagement in polarized compared to unpolarized condition. Temporal Effect represents the growth rate over time. AUC = Area Under the Curve. The larger value between Polarized and Unpolarized conditions is shown in bold. *$p$ < .05, **$p$ < .01, ***$p$ < .001} \\
\end{tabularx}
\end{table}

The emergent patterns in cross-ideological interactions present a particularly interesting parallel to real-world observations. Our simulation captures a fundamental asymmetry in how users engage with opposing viewpoints across different interaction types. The complete cessation of opposing-stance likes while maintaining---and even amplifying---opposing-stance comments ($\beta = 1.680$) aligns with theoretical frameworks of motivated interaction in political discourse. This pattern supports the ``performative disagreement'' hypothesis \citep{bond_political_2022}, which suggests that users engage with opposing content not primarily to understand or consider alternative viewpoints, but to signal disagreement and reinforce their own position. The stronger polarization effect for opposing-stance comments compared to same-stance comments ($\beta = 1.680$ vs. $\beta = 0.807$) particularly supports this interpretation.

The differentiated pattern across reaction types---with likes showing strong in-group bias, comments showing cross-cutting engagement, and reposts occupying a middle ground---reflects what \citet{bail_breaking_2018} terms the ``social identity signaling'' function of different platform affordances. Likes, being public endorsements, serve primarily as identity markers and thus show strong homophily. Comments, enabling both agreement and disagreement, become vehicles for identity performance through opposition. This aligns with research on ``political theater'' in social media \citep{an_political_2019}, where comment sections often become stages for ideological confrontation rather than dialogue.

The amplified engagement of influencers in polarized conditions, particularly in receiving comments ($\beta = 8.633$), suggests they serve as focal points for cross-ideological conflict. This matches empirical observations of how influential accounts often become ``lightning rods'' for opposing viewpoints \citep{soares_influencers_2018}. The substantial temporal effects for influencer-received comments ($\beta = 1.620$) further suggests that these confrontational dynamics intensify over time, potentially creating self-reinforcing cycles of polarization.

While the model captures these key dynamics, it also reveals some limitations. The stark binary nature of the like-avoidance effect may oversimplify the more nuanced patterns of cross-ideological interaction documented in empirical studies. Additionally, the model's prediction of stable temporal effects across conditions may not adequately capture the documented volatility of real social media engagement patterns, particularly in response to external events or platform changes.

%These findings have important implications for platform design and intervention strategies. The strong cross-cutting engagement through comments, while potentially driving platform activity, may actually reinforce rather than reduce polarization through performative disagreement. This suggests that simply promoting cross-ideological exposure or interaction may be insufficient or even counterproductive. Instead, platforms might need to consider how different interaction affordances shape the nature and consequences of cross-ideological engagement.

%Future research could explore how to foster more constructive forms of cross-ideological interaction, perhaps by designing new interaction mechanisms that encourage genuine dialogue rather than performative opposition. Additionally, investigating the role of platform affordances in shaping interaction patterns could reveal opportunities for promoting more constructive engagement across ideological lines.

\subsection{Message Analysis}

\subsubsection{Experimental Setup}

Using a $2\times3$ factorial design, we investigated how initial opinion distributions, i.e. \emph{Polarization Degree}, and \emph{Recommendation Bias} influence content characteristics in social network discussions. We maintained the same initialization procedure and network parameters as in the interaction analysis.

For each distribution condition (polarized and unpolarized), we implemented three \emph{Recommendation Bias} configurations: pro-biased ($70$-$30$ ratio of pro to contra messages), contra-biased ($30$-$70$ ratio), and balanced (equal proportions).
Our LLM-based content analysis measured four message dimensions: opinion (scale from $-1$ to $1$), group identity salience ($0$ to $1$), emotionality ($0$ to $1$), and uncertainty ($0$ to $1$). For each agent, the LLM evaluated all recommended messages across iterations, assessing the fundamental stance, group emphasis, emotional-rational balance, and expressed doubt in the discourse.

\subsubsection{Results}

Our analysis revealed significant effects of both \emph{Polarization Degree} and \emph{Recommendation Bias} conditions across all measured content dimensions (see Table~\ref{tab:message-analysis-anova} and Figure~\ref{fig:message-analysis-violin-plots}). The \emph{Opinion} analysis demonstrated strong main effects for \emph{Polarization Degree} ($F(1, 14095) = 49.36$, $p < .001$) and \emph{Recommendation Bias} ($F(2, 14095) = 1100.14$, $p < .001$), as well as a significant interaction between these factors ($F(2, 14095) = 901.86$, $p < .001$). In polarized conditions, messages exhibited marginally negative average opinions ($M = -0.01$, $SD = 0.83$) compared to slightly more negative opinions in unpolarized conditions ($M = -0.07$, $SD = 0.20$). The interaction manifested particularly strongly in opinion expression, where polarized conditions showed marked differences between pro ($M = 0.52$, $SD = 0.66$), contra ($M = -0.43$, $SD = 0.71$), and balanced ($M = -0.13$, $SD = 0.81$) positions, while unpolarized conditions exhibited substantially smaller variations between these positions (pro: $M = -0.04$, $SD = 0.18$; contra: $M = -0.06$, $SD = 0.21$; balanced: $M = -0.12$, $SD = 0.20$).

\begin{figure}[h]
    \centering
    \begin{minipage}[b]{0.49\textwidth}
        \centering
        \includegraphics[width=\textwidth]{figures/violin_plots/violin_plots_opinion.png}
        \subcaption{Message Opinion}
        \label{fig:subfig1}
    \end{minipage}
    \hfill
    \begin{minipage}[b]{0.49\textwidth}
        \centering
        \includegraphics[width=\textwidth]{figures/violin_plots/violin_plots_group_identity_salience.png}
        \subcaption{Group Identity Salience}
        \label{fig:subfig2}
    \end{minipage}
    
    \vspace{1em}
    
    \begin{minipage}[b]{0.49\textwidth}
        \centering
        \includegraphics[width=\textwidth]{figures/violin_plots/violin_plots_emotionality.png}
        \subcaption{Emotionality}
        \label{fig:subfig4}
    \end{minipage}
    \hfill
    \begin{minipage}[b]{0.49\textwidth}
        \centering
        \includegraphics[width=\textwidth]{figures/violin_plots/violin_plots_uncertainty.png}
        \subcaption{Uncertainty}
        \label{fig:subfig5}
    \end{minipage}

    \caption{The violin plots illustrate the distribution of values obtained from the LLM with respect to varying content dimensions. A differentiation is made between polarized and non-polarized populations, and the bias in the distribution of messages (pro, contra, balanced) is also considered.}
    \label{fig:message-analysis-violin-plots}
\end{figure}

\emph{Group Identity Salience} showed a particularly pronounced main effect of \emph{Polarization Degree} ($F(1, 14066) = 122007.50$, $p < .001$), with polarized conditions eliciting substantially higher group identity expression ($M = 0.72$, $SD = 0.09$) compared to unpolarized conditions ($M = 0.15$, $SD = 0.11$). While position effects were statistically significant ($F(2, 14066) = 52.47$, $p < .001$), the practical differences between positions were minimal, suggesting that initial polarization, rather than message bias, primarily drives group identity expression.


The \emph{Emotionality} analysis revealed strong effects of \emph{Polarization Degree} ($F(1, 14066) = 50758.88$, $p < .001$), with messages in polarized conditions showing markedly higher emotional content ($M = 0.75$, $SD = 0.10$) compared to unpolarized conditions ($M = 0.42$, $SD = 0.07$). Though \emph{Recommendation Bias} effects were significant ($F(2, 14066) = 104.63$, $p < .001$), the differences were relatively small in practical terms.

\emph{Uncertainty} levels displayed an inverse relationship with \emph{Polarization Degree} ($F(1, 14066) = 66628.45$, $p < .001$), with unpolarized conditions generating substantially higher uncertainty expression ($M = 0.56$, $SD = 0.06$) compared to polarized conditions ($M = 0.20$, $SD = 0.10$). \emph{Recommendation Bias} effects, while statistically significant ($F(2, 14066) = 41.64$, $p < .001$), showed only minor differences.

\begin{table}[ht]
\centering
\small
\caption{Main Effects and Interactions Across Message Content Dimensions}
\label{tab:message-analysis-anova}
\begin{tabularx}{\textwidth}{>{\raggedright\arraybackslash}p{2.2cm}>{\raggedright\arraybackslash}p{1.8cm}*{3}{>{\centering\arraybackslash}X}>{\centering\arraybackslash}p{1.2cm}>{\centering\arraybackslash}p{1cm}}
\toprule
\multirow{2}{*}{\textbf{Metric}} & \multirow{2}{*}{\textbf{Condition}} & \multicolumn{3}{c}{\textbf{Position}} & \multicolumn{2}{c}{\textbf{ANOVA}} \\
\cmidrule(lr){3-5} \cmidrule(lr){6-7}
& & Contra & Balanced & Pro & $F$ & $p$ \\
\midrule
\multirow{2}{*}{Opinion} 
    & Polarized & \textbf{-0.43} (0.71) & -0.13 (0.81) & \textbf{0.52} (0.66) & \multirow{2}{*}{\textbf{901.86}$^a$} & \multirow{2}{*}{<.001***} \\
    & Unpolarized & -0.06 (0.21) & -0.12 (0.20) & -0.04 (0.18) & & \\
\midrule
\multirow{2}{*}{Group Identity} 
    & Polarized & \textbf{0.70} (0.10) & \textbf{0.74} (0.08) & \textbf{0.72} (0.09) & \multirow{2}{*}{\textbf{74.69}$^a$} & \multirow{2}{*}{<.001***} \\
    & Unpolarized & 0.15 (0.11) & 0.15 (0.11) & 0.14 (0.11) & & \\
\midrule
\multirow{2}{*}{Emotionality} 
    & Polarized & \textbf{0.71} (0.10) & \textbf{0.76} (0.08) & \textbf{0.76} (0.10) & \multirow{2}{*}{\textbf{156.85}$^a$} & \multirow{2}{*}{<.001***} \\
    & Unpolarized & 0.43 (0.09) & 0.42 (0.07) & 0.42 (0.07) & & \\
\midrule
\multirow{2}{*}{Uncertainty} 
    & Polarized & 0.19 (0.12) & 0.20 (0.08) & 0.21 (0.08) & \multirow{2}{*}{\textbf{17.58}$^a$} & \multirow{2}{*}{<.001***} \\
    & Unpolarized & \textbf{0.55} (0.07) & \textbf{0.56} (0.06) & \textbf{0.56} (0.07) & & \\
\bottomrule
\multicolumn{7}{p{.95\textwidth}}{\small \textbf{Note:} Values show means with standard deviations in parentheses. Bold values indicate significantly higher means between polarized and unpolarized conditions for each position. All F-statistics are significant at $p$ < .001.} \\
\multicolumn{7}{p{.95\textwidth}}{\small $^a$ F-statistic for Polarization × Position interaction (df = 2, 14095). ***$p$ < .001} \\
\end{tabularx}
\end{table}

These findings collectively suggest that \emph{Polarization Degree} plays a crucial role in shaping message content across all measured dimensions, with polarized conditions generally amplifying opinion differences, increasing group identity salience and emotionality, while reducing uncertainty. \emph{Recommendation Bias} effects, while significant, showed varying practical importance across different content dimensions, with the strongest impact observed in opinion expression.

\subsubsection{Discussion}

Our simulation of message content dynamics provides valuable insights into how different initial conditions might shape online discourse patterns. The model's behavior reveals several interesting parallels with empirical observations while also highlighting potential limitations in capturing real-world complexity.

The simulated patterns of opinion expression, particularly the amplification effect in polarized conditions, align with documented phenomena in social media studies. The model successfully reproduces the tendency toward more extreme language and position-taking in polarized environments, a pattern frequently observed in empirical research \citep{wahlstrom_dynamics_2021, simchon_troll_2022}. However, the clean separation between pro and contra positions in our simulation may oversimplify the more nuanced opinion distributions typically found in real online discussions.

A particularly interesting aspect of our simulation is the emergence of strong group identity expressions under polarized conditions. This aligns with social identity theory \citep{tajfel_integrative_1979, huddy_social_2001} and empirical observations of group-based language in polarized debates \citep{albertson_dog-whistle_2015,ruiz-sanchez_us_2019,bliuc_online_2021,iyengar_fear_2015}, though the strength of this effect in our model ($M = 0.72$ vs $M = 0.15$) may be more pronounced than typically observed in real-world settings. The simulation thus captures the fundamental mechanism of group identity activation while potentially overemphasizing its magnitude.

The simulated relationship between polarization and emotional content mirrors documented patterns in social media discourse, where polarized discussions often exhibit higher emotional intensity \citep{asker_thinking_2019,fischer_emotion_2023}. Similarly, the inverse relationship between polarization and uncertainty expression in our model reflects observed patterns of increased certainty in situations where inter-group differences are salient \citep{holtz_intergroup_2001, holtz_relative_2008, winter_toward_2019}. However, the linear nature of these relationships in our simulation may not fully capture the complex interplay between emotions, uncertainty, and polarization observed in real social media environments.

The results of our simulation demonstrate that the framework is not only capable of generating polarized language, but also exhibits reliable classification of corresponding markers. However, while the model successfully reproduces several key patterns observed in empirical studies, the clarity and strength of these effects suggest that our simulation may not fully capture the noise and complexity inherent in real social media interactions. The strong main effects of polarization across all content dimensions, while theoretically informative, likely represent an idealized version of the more complex and nuanced patterns found in actual online discourse.

The relationship between opinion distribution and message characteristics in our simulation provides a valuable starting point for understanding causality in online discourse dynamics. However, the relative simplicity of these relationships compared to real-world observations suggests opportunities for model refinement, particularly in incorporating more complex interaction effects and environmental influences.

