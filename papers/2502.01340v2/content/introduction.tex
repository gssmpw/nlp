\section{Introduction}

The advent of online social networks has fundamentally transformed public discourse, offering unprecedented opportunities for democratic engagement while simultaneously introducing new challenges to societal cohesion. These platforms have become central arenas for information exchange and opinion formation, yet they have also been implicated in the intensification of social polarization—a phenomenon characterized by the formation of distinct, often opposing groups with minimal constructive cross-communication \citep{grover_dilemma_2022, kubin_role_2021, bail_exposure_2018}.

Research on online polarization has primarily followed two different methodological approaches. Observational studies have leveraged large-scale data from social media platforms, employing sophisticated analytical techniques such as sentiment analysis \citep{karjus_evolving_2024, alsinet_measuring_2021, buder_does_2021}, network clustering \citep{treuillier_gaining_2024, bond_political_2022, al_amin_unveiling_2017}, and topic modeling \citep{kim_polarized_2019, chen_modeling_2021} to identify polarization patterns. While these studies have provided valuable insights into the macro-level dynamics of polarization, they are constrained by their reliance on passive data collection and the inability to manipulate variables experimentally. Complementing these empirical investigations, theoretical research has developed mathematical models of opinion dynamics and simulations to explore the mechanisms underlying polarization \citep{hegselmann_opinion_2002,degroot_reaching_1974,sasahara_social_2021,del_vicario_modeling_2017}. These models offer precise mathematical formulations and enable controlled offline experimentation, yet often rely on simplified interaction rules that fail to capture the nuanced complexity of real-world communication.

Recent developments in large language models have introduced a third approach to studying polarization, where LLM-based agents are used to generate realistic social media discourse and interactions \citep{chuang_simulating_2024,breum_persuasive_2024,ohagi_polarization_2024}. While these agents excel at producing naturalistic content that mimics human communication patterns, they typically operate independently from the mathematical frameworks used in traditional opinion dynamics models, which explicitly define how beliefs evolve through specific update functions. This creates a methodological divide between formal analytical approaches that can accurately track opinion trajectories and language-based simulations that capture the richness of human discourse but lack the mathematical rigor to model belief change appropriately. 

Moreover, despite these technological advances, experimental investigations involving human participants remain comparatively rare. This methodological gap is significant given that experimental studies could provide crucial insights into the causal mechanisms of opinion formation and polarization dynamics that neither observational data nor pure simulations can fully capture. The persistent lack of empirical research examining how human users interact with and are influenced by social environments---whether artificial or natural---represents a critical limitation in our understanding of polarization mechanisms. This gap is particularly significant given that insights into human behavior and opinion formation in polarized spaces are essential for developing effective interventions to mitigate the harmful effects of polarization.

To address these gaps, we present a novel experimental framework that bridges formal models of opinion dynamics and LLM-based approaches by combining precise mathematical representations of opinions and belief updates with naturalistic language generation. Our agents operate on continuous opinion values and follow rigorous probabilistic rules for interactions, while using LLMs to generate contextually appropriate content that reflects these underlying mathematical states. This integration of mathematical modeling with sophisticated language generation creates, for the first time, a controlled experimental environment in which human participants can meaningfully engage with artificial agents in polarized debates. Within this framework, artificial agents interact through message posting and various forms of engagement (likes, reposts, comments), generating communication data that enables highly realistic interactions between humans and simulated agents. Our experimental design includes comprehensive pre- and post-interaction data collection on participants' positions on the topic discussed, as well as their perceptions of the platform and the debate space, allowing us to empirically track how exposure to polarized content and interactions with artificial agents influence human opinion formation and evolution.

Our research makes several significant contributions to the field:

\begin{enumerate} 

\item \textbf{Methodological Innovation:} We develop a synthetic social network platform populated with LLM-based artificial agents that produce realistic communication data. This setup allows for controlled experimentation while maintaining the complexity of real-world social interactions.

\item \textbf{Empirical Insights:} Through an empirical user study, we provide novel insights into how interactions within a synthetic, polarized debate space influence human users' opinions and perceptions, contributing to the understanding of polarization dynamics.

\item \textbf{Framework for Future Research:} Our experimental approach offers a replicable framework that can be utilized in future studies to explore various aspects of online social behavior, including information diffusion, the formation of echo chambers, and the impact of intervention strategies.

\end{enumerate}

Our empirical investigation yields several key insights into the dynamics of online polarization. First, we demonstrate that LLM-based artificial agents can successfully reproduce characteristic features of polarized discourse, as validated through both computational analysis and human perception. Second, our user study reveals that polarized environments significantly influence how participants perceive and engage with online discussions, increasing emotional perception and group identity salience while reducing expressed uncertainty. Third, we find that recommendation bias interacts with polarization to shape user engagement patterns, with particularly pronounced effects in highly polarized conditions. These findings provide crucial empirical support for theoretical models of opinion dynamics while highlighting the potential of synthetic social networks as a methodological tool for studying online social behavior.

The remainder of this paper is structured as follows. Section~$2$ presents a comprehensive review of related work on polarization in online social networks, highlighting the limitations of existing approaches and establishing the need for our methodology. Section~$3$ details the design and implementation of our synthetic social network and its LLM-based agents. Section~$4$ evaluates our framework through systematic offline testing, analyzing agent behavior and discourse patterns. Section~$5$ presents our experimental user study investigating how humans perceive and interact with the simulated environment. Section~$6$ discusses the broader implications of our findings, methodological contributions, and future research directions. Finally, Section~$7$ concludes by summarizing our contributions and emphasizing the significance of our approach for studying online polarization.

