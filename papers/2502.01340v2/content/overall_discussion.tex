\section{Overall Discussion}

This study presents a novel methodological framework for investigating online polarization through controlled experimental manipulation of social media environments using LLM-based artificial agents. Our findings demonstrate that this approach can successfully reproduce key characteristics of polarized online discourse while enabling precise control over environmental factors that shape user perceptions and behaviors. The integration of sophisticated language models with traditional opinion dynamics frameworks represents a significant advancement in our ability to study the microfoundations of polarization processes.

Our experimental framework advances beyond observational and theoretical approaches by providing systematic empirical evidence for how online environments influence user perceptions and behaviors. Our results demonstrate that group identity processes are fundamental to polarization dynamics in online environments, even when interactions occur with artificial agents. This finding extends social identity theory in important ways: where \citep{tajfel_integrative_1979} established how group identities could emerge from minimal categorical distinctions, and \citep{huddy_social_2001} emphasized the need to examine real-world complexity of identity formation, our work reveals how these processes manifest in digital spaces.

The amplification of group-based polarization we observe aligns with fundamental theories of political identity formation. \citep{laclau_hegemony_2014}'s concept of the constitutive outside posits that political identities fundamentally emerge and strengthen through the recognition of an opposing force---an "other" against which the group defines itself. This process is closely related to what \citep{schmitt_concept_2008} terms the friend-enemy distinction, where political identities crystallize around the identification of a fundamental antagonist. Our offline evaluation provides empirical support for these theoretical perspectives: agents in moderate homophily conditions showed significantly higher polarization than those in high homophily conditions. This finding aligns with \citep{bateson_naven_1958}'s notion of complementary schismogenesis---a process where interaction between groups leads to a progressive differentiation of their behaviors and identities, with each group's actions eliciting more extreme counter-actions from the other. This challenges the predominant ``echo chamber'' framework \citep{bakshy_exposure_2015} for understanding online polarization, suggesting that exposure to opposing views, rather than isolation, can intensify group identities.

Our user study reinforces this understanding of polarization as an active process of group differentiation, showing that participants perceived significantly higher emotional content and group identity salience in polarized conditions. This finding substantiates \citep{bliuc_online_2021}'s theoretical framework, which emphasizes how conflicting collective narratives---rather than mere isolation---drive polarization. Our observation that polarized conditions led to reduced uncertainty expression reveals a key mechanism in this process: the replacement of epistemic humility with group-based certainty. This pattern aligns with \citep{mason_cross-cutting_2016}'s findings on emotional reactivity in political messaging, while extending them by demonstrating these effects in controlled, artificial environments. Similarly, it complements \citep{fischer_emotion_2023}'s analysis of anger amplification in social media by revealing how increased certainty accompanies heightened emotional expression in polarized discourse.

The relationship between polarization and emotional discourse that we observe extends contemporary models of affective polarization in significant ways. While \citep{iyengar_fear_2015} demonstrated how partisan identities drive emotional responses in political contexts, our findings reveal a more complex dynamic centered on group-based processing. Our path analysis shows that polarized conditions significantly reduce uncertainty expression while increasing both emotional content and group salience. However, the relationship between these factors is more nuanced than initially apparent: while polarization directly affects all these dimensions, our structural equation model reveals that group salience, rather than emotional content or uncertainty, plays the key mediating role in shaping polarization perceptions. This builds upon \citep{albertson_dog-whistle_2015}'s work on group-based political messaging by showing how these identity-based dynamics operate in digital environments where group boundaries become particularly salient.

These findings collectively suggest that online polarization emerges primarily from the fundamental role of group opposition in political identity formation, rather than from mere information exposure patterns or purely affective mechanisms. This helps explain why simple exposure to opposing views often fails to reduce polarization \citep{bail_exposure_2018} and suggests that effective interventions must address the deeper dynamics of group identity formation in digital spaces.

The methodological innovation of our approach is particularly strengthened by the convergence between offline LLM-based assessments and human participants' perceptions (cf. Sections~\ref{subsec:offline-message-analysis} and~\ref{subsec:debate-perception}). Both artificial agents and human participants demonstrated consistent patterns in evaluating emotional intensity, uncertainty expression, and group identity salience across conditions. This dual-validation approach combines the advantages of systematic large-scale analysis with crucial insights into human perception and behavior, while demonstrating that LLM-based content analysis effectively captures psychologically relevant aspects of online discourse.

By establishing the feasibility of using LLM-based agents to create controlled yet ecologically valid social media environments, we provide researchers with a powerful new tool for studying online social dynamics. This approach enables precise manipulation of discourse characteristics, allows for systematic variation in network structure and content exposure, and facilitates the collection of fine-grained behavioral data that would be difficult to obtain in naturalistic settings. Moreover, our demonstration that artificial agents can create environments that elicit authentic human responses challenges assumptions about the necessary conditions for studying social media effects, suggesting that users respond to perceived rather than actual social presence.

These methodological advances have important implications for ongoing debates about social media's role in democratic discourse. While existing theories often frame platform effects as either uniformly negative (promoting polarization) or positive (enabling diverse discourse), our results suggest a more complex theoretical framework is needed. The observed interaction between recommendation algorithms, discourse characteristics, and user behavior indicates that platform effects are highly context-dependent and mediated by multiple psychological and social processes. This suggests the need for more sophisticated theoretical models that can account for the dynamic interplay between technological affordances, social psychological processes, and content characteristics.

%The systematic variation we observed in engagement patterns across different polarization conditions provides new theoretical insights into how platform design choices shape user behavior, moving beyond simple deterministic models of technology effects. These findings suggest that certain aspects of polarized discourse may be more structurally determined than previously thought, pointing to the need for theories that better integrate technological and social explanations of polarization.

However, several important limitations must be acknowledged. The short-term nature of our experimental exposure means we cannot make definitive claims about long-term polarization dynamics. While our results demonstrate that the simulation framework can reproduce key features of polarized discourse and elicit expected user responses, questions remain about how these effects might evolve over extended periods of interaction. Additionally, the artificial nature of the experimental environment, despite its ecological validity, may not capture all relevant aspects of real-world social media use. 

Specifically, our methodological approach necessitated making several assumptions about parameter values and interaction dynamics in the absence of comprehensive empirical data. The framework's sophisticated non-linear functions, while capturing key phenomena like confirmation bias and backfire effects, introduce numerous parameters that currently lack empirical validation. Although this was a deliberate choice to create a detailed proxy for polarized discourse, the high number of free parameters raises valid concerns about potential overfitting and the robustness of our findings. While we designed the framework with future empirical calibration in mind, the current parameter values remain largely theoretical and may not accurately reflect real-world behavioral patterns. This limitation is particularly important when interpreting our results, as different parameter configurations could potentially lead to significantly different outcomes. The framework's detailed parameterization, though offering flexibility for future refinement, also increases model complexity and makes it more challenging to identify which specific mechanisms drive observed effects.

Another significant limitation concerns our focus on Universal Basic Income as the sole discussion topic. While UBI's characteristics - moderate pre-existing polarization and potential for opinion formation - made it suitable for our initial investigation, this single-topic approach limits generalizability. Different topics may generate distinct polarization dynamics: established political issues might show stronger ideological entrenchment, while technical discussions could exhibit different group formation patterns. The observed effects of emotional content and group identity salience might vary across topics with different emotional resonance or existing group affiliations.

The sample characteristics and recruitment strategy present additional methodological concerns. The participant pool, recruited through Prolific, skewed toward younger, highly educated, and moderate individuals, potentially limiting our ability to understand how more diverse populations or those with stronger initial positions might interact with polarized environments. Moreover, while our experimental platform successfully reproduced key social media features, the simplified interface lacks many nuanced interaction affordances present in real platforms that might influence user behavior in important ways. Our current implementation also assumes English-language discourse patterns, which may not generalize to other cultural and linguistic contexts where polarization dynamics could manifest differently.

The role of individual differences and contextual factors also remains incompletely explored. The study's design, while allowing for controlled manipulation of environmental features, could not fully account for how personal characteristics such as digital literacy, prior platform experience, or general political engagement might moderate the observed effects. Additionally, the fixed nature of agent behaviors, though methodologically necessary, may not capture the dynamic adaptations that characterize human discourse patterns, particularly in heated debates where rhetorical strategies often evolve in response to opponent reactions.

Future research should address these limitations by conducting longitudinal studies that track user behavior and attitude change over extended periods. The framework could be extended to incorporate more sophisticated network dynamics, explore the role of different content recommendation algorithms, and investigate potential intervention strategies. Additionally, cross-platform studies could examine how different interface designs and interaction affordances influence polarization processes. Studies examining multiple topics simultaneously could help distinguish universal polarization mechanisms from topic-specific effects, while varying topic characteristics (e.g., emotional salience, technical complexity, moral loading) could reveal how content domain influences polarization dynamics.

Particularly promising directions for future work include investigating the role of influencer dynamics in polarization processes, exploring how different moderation strategies affect discourse quality, and examining how varying levels of algorithmic content curation influence user behavior and perception. The framework could also be adapted to study other aspects of online social behavior, such as information diffusion patterns or the emergence of new social norms.