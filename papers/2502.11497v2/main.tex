%%
%% This is file `sample-sigconf-authordraft.tex',
%% generated with the docstrip utility.
%%
%% The original source files were:
%%
%% samples.dtx  (with options: `all,proceedings,bibtex,authordraft')
%% 
%% IMPORTANT NOTICE:
%% 
%% For the copyright see the source file.
%% 
%% Any modified versions of this file must be renamed
%% with new filenames distinct from sample-sigconf-authordraft.tex.
%% 
%% For distribution of the original source see the terms
%% for copying and modification in the file samples.dtx.
%% 
%% This generated file may be distributed as long as the
%% original source files, as listed above, are part of the
%% same distribution. (The sources need not necessarily be
%% in the same archive or directory.)
%%
%%
%% Commands for TeXCount
%TC:macro \cite [option:text,text]
%TC:macro \citep [option:text,text]
%TC:macro \citet [option:text,text]
%TC:envir table 0 1
%TC:envir table* 0 1
%TC:envir tabular [ignore] word
%TC:envir displaymath 0 word
%TC:envir math 0 word
%TC:envir comment 0 0
%%
%%
%% The first command in your LaTeX source must be the \documentclass
%% command.
%%
%% For submission and review of your manuscript please change the
%% command to \documentclass[manuscript, screen, review]{acmart}.
%%
%% When submitting camera ready or to TAPS, please change the command
\documentclass[sigconf]{acmart} 
\settopmatter{authorsperrow=4}
%%or whichever template is required
%% for your publication.
%%
%%
%\documentclass[manuscript,review,anonymous]{acmart}
\usepackage{subfig}


%%
%% \BibTeX command to typeset BibTeX logo in the docs
\AtBeginDocument{%
  \providecommand\BibTeX{{%
    Bib\TeX}}}

%% Rights management information.  This information is sent to you
%% when you complete the rights form.  These commands have SAMPLE
%% values in them; it is your responsibility as an author to replace
%% the commands and values with those provided to you when you
%% complete the rights form.
% \setcopyright{acmlicensed}
% \copyrightyear{2018}
% \acmYear{2018}
\acmDOI{XXXXXXX.XXXXXXX}


%% These commands are for a PROCEEDINGS abstract or paper.
% \acmConference[Conference acronym 'XX]{Make sure to enter the correct
%   conference title from your rights confirmation emai}{June 03--05,
%   2018}{Woodstock, NY}
%%
%%  Uncomment \acmBooktitle if the title of the proceedings is different
%%  from ``Proceedings of ...''!
%%
%%\acmBooktitle{Woodstock '18: ACM Symposium on Neural Gaze Detection,
%%  June 03--05, 2018, Woodstock, NY}
% \acmISBN{978-1-4503-XXXX-X/18/06}


%%
%% Submission ID.
%% Use this when submitting an article to a sponsored event. You'll
%% receive a unique submission ID from the organizers
%% of the event, and this ID should be used as the parameter to this command.
%%\acmSubmissionID{123-A56-BU3}

%%
%% For managing citations, it is recommended to use bibliography
%% files in BibTeX format.
%%
%% You can then either use BibTeX with the ACM-Reference-Format style,
%% or BibLaTeX with the acmnumeric or acmauthoryear sytles, that include
%% support for advanced citation of software artefact from the
%% biblatex-software package, also separately available on CTAN.
%%
%% Look at the sample-*-biblatex.tex files for templates showcasing
%% the biblatex styles.
%%

%%
%% The majority of ACM publications use numbered citations and
%% references.  The command \citestyle{authoryear} switches to the
%% "author year" style.
%%
%% If you are preparing content for an event
%% sponsored by ACM SIGGRAPH, you must use the "author year" style of
%% citations and references.
%% Uncommenting
%% the next command will enable that style.
%%\citestyle{acmauthoryear}


%%
%% end of the preamble, start of the body of the document source.
\begin{document}

%%
%% The "title" command has an optional parameter,
%% allowing the author to define a "short title" to be used in page headers.
% \title{Enhancing Passthrough Comfort with Reprojection}
% \title{Enhancing Passthrough Comfort with Geometry-Aware Reprojection}
% \title{Improving Visually-Induced Discomfort with Geometry-Aware Passthrough}
% \title{Mitigating Visually-Induced Discomfort with Geometry-Aware Passthrough}
% \title{Mind thSe GAP: Geometry-Aware Passthrough Improves Comfort}
% \title{Mind the GAP: Geometry-Aware Passthrough Mitigates Visually-Induced Discomfort}
\title{Geometry Aware Passthrough Mitigates Cybersickness}

%%
%% The "author" command and its associated commands are used to define
%% the authors and their affiliations.
%% Of note is the shared affiliation of the first two authors, and the
%% "authornote" and "authornotemark" commands
%% used to denote shared contribution to the research.
\author{Trishia El Chemaly}
%%\authornote{Both authors contributed equally to this research.}
\email{tchemaly@stanford.edu}
\orcid{0000-0002-4234-3082}
%%\author{G.K.M. Tobin}
%%\authornotemark[1]
%%\email{webmaster@marysville-ohio.com}
\affiliation{%
  \institution{Stanford University}
  \city{Stanford}
  \state{CA}
  \country{USA}
}

\author{Mohit Goyal}
\email{mohitgl@google.com}
\affiliation{%
  \institution{Google}
  \city{Mountain View}
  \state{CA}
  \country{USA}
}

\author{Tinglin Duan}
\email{tduan@google.com}
\affiliation{%
  \institution{Google}
  \city{Mountain View}
  \state{CA}
  \country{USA}
}

\author{Vrushank Phadnis}
\email{vrushank@google.com}
\affiliation{%
  \institution{Google}
  \city{Mountain View}
  \state{CA}
  \country{USA}
}
\author{Sakar Khattar}
\email{sakark@google.com}
\affiliation{%
  \institution{Google}
  \city{Mountain View}
  \state{CA}
  \country{USA}
}
\author{Bjorn Vlaskamp}
\email{bjornvlaskamp@google.com}
\affiliation{%
  \institution{Google}
  \city{Seattle}
  \state{WA}
  \country{USA}
}
\author{Achin Kulshrestha}
\email{kulac@google.com}
\affiliation{%
  \institution{Google}
  \city{Toronto}
  \state{ON}
  \country{Canada}
}
\author{Eric Lee Turner}
\email{elturner@google.com}
\affiliation{%
  \institution{Google}
  \city{Cambridge}
  \state{MA}
  \country{USA}
}
\author{Aveek Purohit}
\email{aveek@google.com}
\affiliation{%
  \institution{Google}
  \city{Mountain View}
  \state{CA}
  \country{USA}
}
\author{Gregory Neiswander}
\email{neiswander@google.com}
\affiliation{%
  \institution{Google}
  \city{Mountain View}
  \state{CA}
  \country{USA}
}
\author{Konstantine Tsotsos}
\email{ktsotsos@google.com}
\affiliation{%
  \institution{Google}
  \city{Toronto}
  \state{ON}
  \country{Canada}
}


%%
%% By default, the full list of authors will be used in the page
%% headers. Often, this list is too long, and will overlap
%% other information printed in the page headers. This command allows
%% the author to define a more concise list
%% of authors' names for this purpose.
\renewcommand{\shortauthors}{El Chemaly et al.}


\newcommand{\tbf}[1]{\textbf{#1}}
\newcommand{\directpassthrough}{direct }
\newcommand{\Directpassthrough}{Direct }
\newcommand{\DirectPassthrough}{Direct }

\newcommand{\depthpassthrough}{geometry aware }
\newcommand{\Depthpassthrough}{Geometry aware }
\newcommand{\DepthPassthrough}{Geometry Aware }
\newcommand{\DepthPassthroughAbb}{GAP}

\newcommand{\GAP}{GAP }
\newcommand{\DP}{DP }
\newcommand{\edits}[1]{\textcolor{black}{#1}}
%%
%% The abstract is a short summary of the work to be presented in the
%% article.
\begin{abstract}
%   Virtual Reality (VR) head-mounted displays (HMDs) provide immersive experiences while limiting the user’s awareness of their physical surroundings. Video see-through (VST) HMDs address this by using outward-facing cameras to reconstruct the user's environment, creating an Augmented Reality (AR) experience. However, relying on video capture from HMD cameras for perception raises concerns about visual discomfort and cybersickness. Since the cameras are positioned outwards and are not located at the eye position, VST HMDs rely on complex image reprojection techniques to create a natural view from the user’s perspective. We first show that direct passthrough (i.e displaying the raw camera feed) leads to exaggerated movements and inaccurate object distances due to inaccurate depth information. Instead, estimating geometry and performing depth-based reprojection can help address these issues but may introduce additional latency and warping artifacts. Using fundamental principles, we discuss a structured approach to designing depth-based passthrough algorithms and introduce metrics to capture warping and perceived geometrical errors. We also design and conduct a user study across 24 participants to compare direct passthrough and depth-based passthrough. Our results demonstrate reduced nausea and disorientation symptoms with depth-based passthrough and uncover several potential avenues to further mitigate visually-induced discomfort.
Virtual Reality headsets isolate users from the real-world by restricting their perception to the virtual-world. Video See-Through
(VST) headsets address this by utilizing world-facing cameras to create Augmented Reality experiences. However, directly displaying
camera feeds causes visual discomfort and cybersickness due to the inaccurate perception of scale and exaggerated motion parallax.
This paper demonstrates the potential of geometry aware passthrough systems in mitigating cybersickness through accurate depth
perception. We first present a methodology to benchmark and compare passthrough algorithms. Furthermore, we design a protocol to
quantitatively measure cybersickness experienced by users in VST headsets. Using this protocol, we conduct a user study to compare
direct passthrough and geometry aware passthrough systems. To the best of our knowledge, our study is the first one to reveal significantly reduced nausea, disorientation, and total scores of cybersickness with geometry aware passthrough. It also
uncovers several potential avenues to further mitigate visually-induced discomfort.
% Virtual Reality headsets isolate users from the real-world by restricting their perception to the virtual-world. Video See-Through (VST) headsets address this by utilizing world-facing cameras to create Augmented Reality experiences. 
% However, directly displaying camera feeds causes visual discomfort and cybersickness due to the inaccurate perception of scale and exaggerated motion parallax. 
% This paper demonstrates the potential of geometry aware passthrough systems in mitigating cybersickness through accurate depth perception. We first present a methodology to benchmark and compare passthrough algorithms. Furthermore, we design a protocol to quantitatively measure cybersickness experienced by users in VST headsets. Using this protocol, we conduct a user study to compare direct passthrough and geometry aware passthrough systems. 
% Our study revealed significantly reduced nausea, disorientation, and total scores of cybersickness with geometry aware passthrough ($p$<0.05). It also uncovers several potential avenues to further mitigate visually-induced discomfort.
\end{abstract}

%%
%% The code below is generated by the tool at http://dl.acm.org/ccs.cfm.
%% Please copy and paste the code instead of the example below.
%%
\begin{CCSXML}
<ccs2012>
   <concept>
       <concept_id>10003120.10003121.10003122</concept_id>
       <concept_desc>Human-centered computing~HCI design and evaluation methods</concept_desc>
       <concept_significance>500</concept_significance>
       </concept>
   <concept>
       <concept_id>10003120.10003121.10003122.10003334</concept_id>
       <concept_desc>Human-centered computing~User studies</concept_desc>
       <concept_significance>500</concept_significance>
       </concept>
   <concept>
       <concept_id>10003120.10003121.10003122.10010854</concept_id>
       <concept_desc>Human-centered computing~Usability testing</concept_desc>
       <concept_significance>500</concept_significance>
       </concept>
 </ccs2012>
\end{CCSXML}


\ccsdesc[500]{Human-centered computing~HCI design and evaluation methods}
\ccsdesc[500]{Human-centered computing~User studies}
\ccsdesc[500]{Human-centered computing~Usability testing}

%%
%% Keywords. The author(s) should pick words that accurately describe
%% the work being presented. Separate the keywords with commas.
\keywords{Video see-through, Cybersickness, Augmented Reality Headsets, Motion sickness, View synthesis}
%% A "teaser" image appears between the author and affiliation
%% information and the body of the document, and typically spans the
%% page.
\begin{teaserfigure}
\centering
  \includegraphics[width=0.88\textwidth]{images/PassthroughComfort.pdf}
%   \caption{A typical Video See-Through Headset utilizes high resolution cameras to reveal the physical world to the user. However, due to hardware constraints, these cameras don't capture the same viewpoint as the user's eyes would see without the headset and directly showing camera-feed (direct passthrough) requires solving for disocclusion (1a). Moreover, this difference in viewpoints can also change the perceived objects scale and exaggerate their perceived motion depending on the distance to the cameras (1b). Therefore, geometry aware passthrough algorithms are needed for performing accurate view synthesis that minimizes such errors, and this paper presents metrics to evaluate the perceived object location and any warping induced due to imperfect depth estimation (1c). While these algorithms can improve the accuracy of reprojection, this paper proposes a comprehensive user study based on real-life AR applications and shows the significant impact of this reprojection on cybersickness.}
\caption{Video see-through headsets typically employ high resolution cameras to display the user's physical environment. However, due to inherent hardware limitations, the camera's perspective deviates from the user's natural viewpoint. Hence, directly displaying camera feeds (direct passthrough) can result in visual artifacts such as disocclusion (1a), inaccurate perception of object positions, and exaggerated motion parallax (1b). In this work, we demonstrate that geometry aware passthrough algorithms can circumvent these artifacts, enabling precise view synthesis tailored to the user's eyes. We present metrics for evaluating errors in perceived object location and warping artifacts arising from imperfect depth estimation, fundamental to a seamless passthrough experience (1c). Furthermore, a comprehensive user study is presented to investigate the impact of reprojection algorithms on cybersickness in real-world augmented reality scenarios, highlighting the significance of geometry aware passthrough systems (1d).}
  \Description{Paper Overview: Sub-figure (a) shows a passthrough headset and demonstrates the difference between eye and camera visibility. The difference in viewpoints creates a region of disocclusion where objects are not visible to the camera but visible to the eye. Sub-figure (b) shows three head positions and the corresponding image of a laptop as seen through eyes versus direct passthrough. Passthrough exaggerates motion. Sub-figure (c) demonstrates how we evaluate passthrough algorithms with different metrics. It shows the input image, ground truth depth, estimated depth, and error maps. These are used to calculate object location errors. It also shows a reference and passthrough image. These images are used to calculate warping errors. Sub-figure (d) represents our user study for evaluating cybersickness in VST. It visualizes the three tasks of typing (a person wearing a headset types on a laptop), navigation (a person wearing a headset navigates waypoints and collects cones), and interaction (puzzle). }
  \label{fig:teaser}
\end{teaserfigure}

% \received{20 February 2007}
% \received[revised]{12 March 2009}
% \received[accepted]{5 June 2009}

%%
%% This command processes the author and affiliation and title
%% information and builds the first part of the formatted document.
\maketitle

% 
% 
The widespread integration of communication networks and smart devices in modern control systems has increased the vulnerability of industrial systems to online cyber-attacks, e.g., Industroyer, Blackenergy, etc \citep{osti_1505628}.
% Modern control systems have seen a large push to include communication networks and smart devices to increase performance, made possible by improvements in communication device cost and energy consumption. This trend has been coupled with the usage of open-standard communication protocols among industrial control systems, making them vulnerable to online cyber-attacks such as Industroyer, Blackenergy, etc \citep{osti_1505628}. 
To counter this, methods have been developed to improve security by achieving attack detection, mitigation, and monitoring, among others \citep{sandberg2022secure}. This paper focuses on active attack diagnosis to mitigate stealthy attacks. 
%
%\subsection{Literature review}

Active diagnosis techniques rely on the inclusion of additional moduli to control systems
% inclusion within the control system of additional moduli 
to alter the behavior of the system compared to information known by the attacker. 
For instance, the concept of additive watermarking was introduced in \cite{mo2015physical}, where noise signals of known mean and variance are added at the plant and compensated for it at the controller. 
This compensation, however, is not exact, causing some performance degradation. Thus, trade-offs between performance and detectability  are necessary \citep{zhu2023detection}.
% A later work \citep{zhu2023detection} designs the watermark signal by trading performance for detection. Thus, although additive watermarking serves as a good detection scheme, they endure performance losses even in the nominal case. 

In encrypted control \citep{darup2021encrypted}, the sensor data is encrypted, sent to the controller, and then operated on directly. Encrypted input signals are sent back to the plant for decryption. Although encryption is widespread in IT security, in control systems it presents some concerns, such as the introduction of time delays \citep{stabile2024verifiable}, while it may present inherent weaknesses \citep{alisic2023model}.
% they are not preferred as they introduce time delays \citep{stabile2024verifiable} which can cause instability, and some encryption schemes can be very weak  \citep{alisic2023model}. 

In moving target defense \citep{griffioen2020moving}, the plant is augmented with fictitious dynamics, known to the controller. The plant output is transmitted to the controller along with the fictitious states over a network under attack. 
The additional measurements then aide in the detection of attacks. 
This comes at the cost of higher communication bandwidth needs, which increases rapidly with the dimension of the augmented systems.
% Since the dynamics of the fictitious dynamics are exactly known to the controller, the attack is detected easily. However, when the scale of the system increases, the communication bandwidth used by moving the target defense approach increases rapidly. 

Other recently proposed works include two-way coding \citep{fang2019two}, a weak encryuption technique, and dynamic masking \citep{abdalmoaty2023privacy}, which enhances privacy as well as security, have been shown to be effective against zero-dynamics attacks.
% Two-way coding \citep{fang2019two} and dynamic masking \citep{abdalmoaty2023privacy} are other recently proposed approaches. Two-way coding is another form of weak encryption technique whilst dynamic masking proposes an architecture that enhances both privacy and security. These schemes are shown to be effective against zero dynamics attacks but remain to be studied for other classes of attacks. 
% Recent extensions include \citep{mukherjee2021secure,ramos2024privacy}.
% Some other works which are related are \citep{mukherjee2021secure}, an extension of \cite{fang2019two}. The work \citep{ramos2024privacy} is an extension of moving target defense for multi-agent systems. 
Furthermore, filtering techniques for attack detection are proposed by \cite{murguia2020security,hashemi2022codesign,escudero2023safety}, while not focusing on stealthy attacks.
% The works \citep{murguia2020security,hashemi2022codesign,escudero2023safety} develop filtering techniques to guarantee safety, without being focused on stealthy covert attacks.

Multiplicative watermarking (mWM) has been proposed by the authors as a diagnosis technique \citep{ferrari2020switching}. mWM consists of a pair of filters on each communication channel between the plant and its controller; the scheme is affine to weak encryption, whereby ``encoding'' and ``decoding'' are done by changing signals' dynamic characteristics through inverse pairs of filters. This enables original signals to be recovered exactly, and thus does not lead to performance degradation.
% A multiplicative watermark is an affine to a weak encryption technique, through which the signal is ``encoded'' by a filter, changing its dynamic behavior. The use of inverse pairs means that the original signal can be recovered, through ``decoding'' via an inverse filter. As such, differently to techniques based on additive watermarking, no performance is lost due to the injection of noise, and there are no bandwidth limitations.

%\subsection{Contributions}
One of the critical features of multiplicative watermarking is that to detect stealthy attacks, the mWM filter parameters must be switched over time. In this paper, an algorithm to optimally design the mWM parameters after a switching event is presented, enhancing detection performance, without changing the switching time.
% This is done without changing the switching time, which is taken as given.

\textcolor{black}{
To formalize the filter design problem, we suppose the defender is interested in optimal performance against adversaries injecting covert attacks with matched system parameters \citep{smith2015covert}, including the mWM parameters prior to the switch. This scenario represents a worst case where malicious agents can take full control of the system while remaining undetected.
Thus, the attack strategy is explicitly included within the formulation of the closed-loop system, and the mWM filters are chosen by solving an optimization problem minimizing the attack-energy-constrained output-to-output gain (AEC-OOG) \citep{anand2023risk}, a variation of the output-to-output gain proposed in  \cite{teixeira2015strategic}.
}
The main contributions of this paper are:
% We consider an adversary injecting a covert attack with matched system parameters \citep{smith2015covert}, i.e., an attacker with full knowledge of the control system parameters, including those of the mWM filters before the switch. This scenario is taken as a worst case, as it has been shown that this class of attacks can be made stealthy. To quantitatively define a cost, the output-to-output gain (OOG) \citep{teixeira2015strategic} is leveraged,
% a metric introduced to evaluate the impact of an additive attack in a control system. %Specifically, OOG evaluates the worst-case performance loss that an attacker injecting an undetectable attack can obtain. 
% Here, the maximum performance loss caused by a stealthy adversary with limited energy is taken, the attack-energy-constrained OOG (AEC-OOG) \citep{anand2023risk}. The main contributions of this paper are:
\begin{enumerate}
%[label=\alph*.]
\item The problem of optimally designing the switching mWM filters is formulated as an optimization problem, with the AEC-OOG is taken as the objective;%where the AEC-OOG is taken as the impact metric; 
\item The worst-case scenario of a covert attack with exact knowledge of plant and mWM filter parameters is embedded within the design problem;
% The optimization problem is defined to incorporate the worst-case scenario of a covert attack with exact knowledge of plant and mWM filter parameters;
\item The feasibility of the optimization problem is shown to be dependent only on stability conditions; 
\item A solution scheme is proposed to promote randomization of the mWM filter parameters such that an eavesdropping adversary cannot remain stealthy.
\end{enumerate} 

This builds on the results of \cite{ferrari2020switching}, where the focus was on the design of the switching protocols, rather than the parameters themselves.
Compared to previous work \citep{gallo2021design}, this paper introduces an optimization problem which is always feasible (thanks to the use of AEC-OOG in the objective), while also considering a more sophisticated class of covert attacks, where the presence of watermark is known to the adversary. 
Moreover, this paper poses a different objective than \citep{zhang2023hybrid}; indeed, while \citep{zhang2023hybrid} provided a design strategy to ensure certain privacy properties, in this paper we address the problem of optimal parameter design following a switching event.


%\subsection{Organization}
The rest of the paper is organized as follows. 
After formulating the problem in Section~\ref{sec:PF}, we propose our design algorithm in Section~\ref{sec:main}, and analyze its properties. It is then evaluated through a numerical example in Section~\ref{sec:NE}, and concluding remarks are given Section~\ref{sec:Con}.
% We provide the problem background in Section~\ref{sec:PF}. We formulate the design problem in Section~\ref{sec:main}, together with an analysis of its properties. The proposed algorithm is evaluated through a numerical example in Section \ref{sec:NE}. Concluding remarks are offered in Section \ref{sec:Con}.



\section{Related Work}
\subsection{Cybersickness and Comfort in VST}
Literature on VST discomfort builds upon the seminal work on simulator sickness, which informs our current understanding of the psycho-physical causes of cybersickness, such as sensory conflict and postural instability \cite{reason1975motion}. As VR HMDs have become more pervasive over the last few decades, research has validated and expanded our understanding of cybersickness by contextualizing the findings in VR applications \cite{li2022mixed}. LaViola et al.  \cite{laviola2000discussion} studied the visual-vestibular mismatch in VR environments associated with nausea and disorientation and identified vection, the perception of self-motion projected by visual stimuli despite the user being stationary. This effect is particularly pronounced with a wide field of view and rapid changes in the virtual scene, and it may be further accentuated in VST due to increased sensitivity to real-world cues \cite{suwa2022reducing}. Blum et al. investigated out-of-focus blur in VR and found it less problematic, noting lower levels of diplopia (double vision) and higher tolerance for blur \cite{blum2010effect}. They acknowledged that individual differences, such as ocular dominance and susceptibility to motion sickness, influence the accommodation of blur in VR. Additionally, hardware factors like display type, field of view, latency, and graphic realism can contribute to VR sickness \cite{chang2020virtual}. 

Compared to the literature on VR sickness, research on VST-specific discomfort is relatively sparse. Studies have explored various approaches to mitigate simulator sickness in VST systems, including the effects of visual displacement conditions \cite{kim2014effects} and the use of fisheye lenses to expand peripheral vision \cite{orlosky2014fisheye}. Freiwald et al. illustrate the complexities in developing VST technologies by using an offline computing method to create a system to reduce latency \cite{freiwald2018camera}. This approach provided better stabilization, reducing the disconnect introduced by the mismatch between camera and HMD refresh rates. More recently, Li et al \cite{li2022mixed} investigated the effects of mixed-reality tunneling methods on simulator sickness. In our work, we propose a comprehensive evaluation framework that includes both machine-readable metrics and user evaluations (subjective ratings) to address hardware and software aspects of VST. We apply this framework to investigate the effects of geometry aware passthrough on cybersickness, an area that has not yet been examined.
% Compared to the literature on VR sickness, research on VST-specific discomfort is relatively sparse. Hardware and real-time runtime limitations have hindered the full exploration of VST research topics. In a recent example, Freiwald et al. illustrates the complexities in developing VST technologies by using an offline computing method to create a system to reduce latency \cite{freiwald2018camera}. The authors intentionally injected delays in the tracking stream to match the camera system latency. This approach provided better stabilization, reducing the disconnect introduced by the mismatch between camera and HMD refresh rates. 

\subsection{ Novel-View Synthesis and Evaluation.}
    Geometry aware passthrough algorithms fall into the class of novel view synthesis techniques. In contrast to synthesizing views based on fixed camera and eye positions, novel view synthesis aims to solve for a more generalized case i.e. \emph{any} new camera viewpoint given images from a few known viewpoints of the same scene. Early work in this space utilized image-based rendering techniques \cite{SilhouetteIBR, li2023dynibar}], where multiple views were used to construct the scene geometry and then blended together to render novel views. Then, Mildenhall et al. \cite{mildenhall2020nerf} proposed learning a volumetric scene function to model the entire scene which could be queried from novel camera viewpoints. Tretschk et al. \cite{tretschk2020nonrigid} extended this approach to non-rigid scenes, allowing novel view reconstruction over time. Recently, Gaussian splatting \cite{kerbl3Dgaussians} was proposed which aims to model the world with 3D Gaussians and learn these Gaussians to minimize the rendering error on images from known viewpoints. Generally, to evaluate novel view synthesis, novel viewpoints are manually collected using cameras or synthetically rendered. These images then serve as a reference and can be directly compared with the estimated image using metrics like PSNR (Peak Signal-to-Noise Ratio), SSIM (Structural Similarity) and Perceptual Similarity \cite{zhang2018perceptual, kerbl3Dgaussians}. However, these metrics rely on the assumption that reference images are available for evaluation.  \edits{While collecting paired input images of the camera and the eye view is possible \cite{IBRreview}, it is hard to scale for VST headsets and doesn't directly allow assessment from a geometrical standpoint. Instead, we propose metrics that solely utilize depth to compute reprojection errors at the eye, focusing on the geometrical correctness of passthrough systems.}
    
    
    \subsection{Reprojection in VST HMDs} For VST HMDs, the passthrough cameras are placed in front of the user's eyes, typically a few centimeters away.
    The scene as viewed by the user's eyes needs to be reconstructed and displayed back to the user through VST displays. This process is generally referred to as \emph{reprojection}, where the camera images are reprojected to the user's eyes. 
    Past work on performing on-device reprojection for passthrough either utilizes classical depth estimation to synthesize eye-views \cite{chaurasia2020passthroughplus} or relies on dedicated GPUs to perform real-time accurate reprojection \cite{novelviewsynthdevice, IBRreview, Lei2022Neuralpassthrough}. In our work, we follow Chaurasia et al. \cite{chaurasia2020passthroughplus}  and implement our geometry aware passthrough pipeline using a low powered, on-device depth estimation algorithm. 
    \edits{While many of these studies discuss several technical aspects that are important for reprojection in headsets, their impact and the significance of reprojection itself on cybersickness remain underinvestigated. With the growing adoption of VST HMDs, we believe that further research into the relationship between reprojection and user comfort is crucial for driving fundamental advancements in VST technology, ultimately enhancing user experience and usability in AR applications.}
\section{Methodology}
In this section, we outline the key research questions driving this study, followed by a detailed description of the methodology used to design and conduct the survey.
\subsection{Research Questions}
\begin{enumerate}
    \item[\textbf{RQ1:}] How do developers allocate their time during a typical workweek, and how does this compare to their perception of an \textbf{ideal workweek?}
    \item[\textbf{RQ2:}] How are developer's satisfaction and productivity affected by \textbf{deviations} from their ideal workweek?
     \item[\textbf{RQ3:}] For which tasks do developers prefer using \textbf{AI tools}, and how does the frequency of AI tool usage \textbf{influence} their satisfaction and productivity?
\end{enumerate}

\subsection{Survey Design}
% Describe how the survey was conducted, survey structure, sample size, which activities were selected and how, incentives, etc. 

To gain insights into the types of activities developers engage in during a typical work week, we conducted a series of exploratory interviews with 12 randomly selected participants. These semi-structured interviews provided a qualitative foundation, allowing us to iteratively develop a comprehensive list of higher-level activities that reflect both ideal and actual workweek allocations. The findings from these interviews were instrumental in refining our survey questions and design.

% - When was it distributed
% - How many people were invited
% - how was the survey advertised
% - incentive provided to participants
% - how many responses received (with response rates)
% - Board of ethics description \& instruments
% - Describe the main questions asked in the survey

The survey was distributed in \textcolor{blue}{May 2024} to software engineers working in Microsoft teams across India and the United States. A total of 6000 developers were invited to participate via email. Framed as a study aimed at boosting developer productivity by understanding how they allocate their time in a workday, the survey received 510 complete responses (responses rate of 8.5\%). After finishing the survey, the participants could enter a sweepstake to win one out of ten \$50 Amazon.com Gift Cards.
\textcolor{blue}{description of ethics}.

The main questions in the survey were as follows:
\begin{enumerate}
    \item Their roles and years of experience in the industry/team
    \item The hours spent on various activities in their typical workweek
    \item Ideally, the percentage of time they would want to allocate to each activity in a workweek
    \item How productive and satisfied were they by their past workweek
    \item Activities they find most cognitively challenging
    \item How often do they use AI tools to assist in their daily activities
    \item Two open-ended questions about the activities they would want to automate using AI tools, and advice for new hires to boost their productivity and satisfaction levels 
\end{enumerate}



\subsection{Data Analysis \& Exploration}
% Here, we could start with discussing the survey group:
% - demographic observations
% - distribution of participants (based on the years experience in the industry/team), 

From the exploratory interviews, we identified sixteen key activities, which were subsequently used to quantify the developers' time allocation across their work week. 

\subsection{Limitations}
\section{User Study} \label{study}

As AR headsets are becoming popular with an increasing number of applications in medicine, education, and gaming, it becomes important to holistically evaluate the discomfort experienced by users in these passthrough systems. To address this, we introduce a protocol specifically tailored to quantify cybersickness in the context of VST HMD use cases. \edits{To our knowledge, no VST studies comprehensively combine user motion, interaction, and sickness metrics, making this protocol an early effort to define reasonable benchmarks.}

\edits{\subsection{Experiment Design Considerations}
{Our key considerations for the experimental setup were reproducibility, repeatability, and real-life relevance while the study design focused on reliably eliciting signals related to user comfort. To achieve this, we began with tasks identified in the literature, tested them in a pilot study, and iteratively refined the task nature and duration based on participant feedback. We detail the specific steps we took to ensure these principles were upheld:
\subsubsection{Standardization of Setup and Metrics}
{We standardized all aspects of the experimental setup to ensure reproducibility and consistency. This included the layout of the room, object placement, and controlled environmental factors such as lighting and headset parameters. We provide this information in our Supplementary material. For cybersickness and discomfort measurements, we employed established tools which are widely validated in prior literature.}
\subsubsection{Pilot Testing and Iterative Refinement}
Pilot studies played a crucial role in refining task design and duration to reliably induce motion sickness while maintaining ecological validity. Tasks were initially drawn from existing research that we summarize in our Supplementary material and adjusted based on observed results and participant feedback. Specific triggers of motion sickness such as head motion, depth perception tasks, and visual-motor coordination were emphasized in the task design. For example, we replaced straight-line walking tasks with multi-directional navigation requiring turns and object interactions to better simulate realistic scenarios that elicit depth perception challenges and frequent head motion.
\subsubsection{Real-Life Relevance}
The protocol was designed to mirror real-world VST use cases, ensuring relevance to everyday applications. Tasks included typing on a physical keyboard for productivity, navigating complex physical environments, and interacting with tangible objects, such as completing assembly tasks. This mix of stationary (near-field interaction) and dynamic (locomotion-based) activities ensured a comprehensive evaluation of user discomfort across different contexts.
}}

Below, we outline the design of our protocol, including task selection and quantification methods. We then apply our protocol to understand the effects of the reprojection algorithms on discomfort and cybersickness when using a VST HMD. Given the subjective nature of cybersickness and discomfort, we employ a within-subject design in our protocol where participants complete three tasks involving head motion, hand-eye coordination, and untethered locomotion. We collect both objective task performance metrics and quantitative and qualitative subjective feedback under three conditions: Natural Vision (NV), \Directpassthrough Passthrough (DP), and \DepthPassthrough Passthrough (GAP).

\begin{figure*}[ht]
    \centering
    \includegraphics[width=\textwidth,trim={0.2cm 10cm 0.2cm 0.2cm},clip]{images/Tasks.pdf}
    \caption{\textbf{User Study Setup.} Pictures of the lab setup for the three tasks completed by the participants while wearing the VST HMD.}
    \Description{The figure shows the user study setup in the lab space. The left sub-figure shows the setup for the typing task: a laptop and Dvorak keyboard. Their location is marked with tape for consistency between participants. The middle sub-figure shows the setup for the navigation task. The cones are numbered and placed around pieces of furniture such as a plant, 2 tables, and 2 chairs. The right sub-figure shows the setup for the interaction task. Puzzle pieces are placed on shelves. The table is marked with tape to indicate the frame in which the completed puzzle should fit.}
    \label{fig:tasks}
\end{figure*}

\subsection{Participants}
For the user study, we recruited 25 participants who had normal or corrected-to-normal vision. The study included a diverse range of ages, genders, levels of VR usage, and job profiles. Table \ref{tab:demographics} gives a full overview of the demographics. Recruitment was done in accordance with the ethics board of our institution. We excluded data from one participant from our analysis since they spent an abnormally long time to complete the protocol. \edits{Since the rest of the participants exhibited no irregular behavior, all their data were retained for analysis including participants who reported higher than average sickness scores.}

\begin{table}[!ht]
  \caption{Participant demographics for the user study, showing diversity across gender, age, normal versus corrected vision, and VR usage.}
  \label{tab:demographics}
  \begin{tabular}{l|l|l}
    \toprule
     \textbf{Variable} & \textbf{Categories} & \textbf{\#Participants}\\
    \midrule
    \textbf{Gender} &Men &14\\
    & Women &11\\
    \midrule
    \textbf{Age Group} &18-24 &1\\
    &25-34 &12\\
    &35-44 &9\\
    &45-54 &3\\
    \midrule
    \textbf{Vision} &Normal &19\\
    &Contact Lenses &6\\
    \midrule
    \textbf{VR Usage} &Never &1\\
    &Once &12\\
    &Once a week &3\\
    &Once a month &8\\
    &At least once a day &1\\
    \bottomrule
  \end{tabular}
\end{table}

% (14 males, 11 females). 1 participant was 18-24 years old, 12 were 25-34 years old, 9 were 35-44 years old, and 3 were 45-54 years old. None of the participants were over 55 years old. 
% Participants had normal or corrected-to-normal vision (19 participants had normal vision and 6 participants wore contact lenses). 1 participant never used VR before, 12 participants used VR once, 3 participants use VR once a week, 8 participants use VR once a month, and 1 participant uses VR at least once a day. 
\subsection{Tasks}\label{subsec:tasks} 

We devised our protocol focusing \emph{exclusively} on passthrough-based real-world interactions and ensured no virtual elements \cite{de2024visual} were visible to participants. Figure \ref{fig:tasks} shows pictures of the lab setup for the three tasks completed by the participants. The tasks were inspired from fundamental real-world AR applications such as working with laptops for productivity, navigation in the physical world, and interaction with real-objects. They emphasized on the user head motion while  necessitating inspection and spatial awareness of the physical world.
% head motion, realistic interaction, and spatial awareness within a physical environment. 
Our design ensured a cumulative duration to be roughly 15 minutes for all the tasks.

\subsubsection{Typing} Typing is a familiar activity that effectively engages both visual and motor components, making it relevant for evaluating fine-selecting physical objects and digital screen usage. \edits{This task was specifically chosen to reflect emerging applications in productivity such as using VST HMDs like the Apple Vision Pro as extensions of traditional work setups where users interact with physical keyboards and screens. It is representative of fine motor interactions requiring frequent gaze shifts between the keyboard and the laptop screen, a common trigger for visual discomfort. } A physical Dvorak keyboard \cite{dvorak1936typewriting} was used to minimize reliance on muscle memory and encourage frequent visual engagement with the keys. A typing assessment was conducted to measure speed, accuracy, and overall proficiency. Participants typed for 6 minutes as quickly and accurately as possible, with the typing text randomized for each session to ensure variability. \edits{To maintain repeatability and reproducibility, the placement of the laptop and keyboard was standardized across all participants, and screen brightness was kept consistent throughout the study.}
\subsubsection{Navigation} The realistic and holistic use of a VST HMD involves navigating physical spaces, avoiding obstacles, and interacting with real-world objects \cite{bailenson2024seeing,de2018augmented,erickson2019cold}. Previous research has also explored the impact of spatial navigation, such as waypoint-following, on sickness in VR  \cite{al2019effect}. We designed a navigation task where participants collected and dropped off 10 numbered cones, one at a time, into a designated drop zone. \edits{To emphasize geometry perception, the task included narrow passages and required multi-directional movements, including turning to locate cones and interacting with objects at varying heights.} Further, the cones were placed at different heights, requiring a range of movements to pick and place. Participants were given 2-3 minutes to familiarize themselves with the cone locations before beginning the trial. During the task, participants were instructed to move naturally but carefully to avoid colliding with any objects in the room.
\subsubsection{Interaction} Li et al. \cite{li2022mixed} introduced a Tangram puzzle task as a representative activity that simulates common assembly tasks requiring both motor and cognitive skills. Inspired by their approach, we adapted this task by using jigsaw puzzles consisting of 24 pieces and measuring 2 x 3 feet. This larger puzzle size was selected to accommodate head motion which is often associated with motion sickness. We chose a 24-piece puzzle to mitigate prolonged VST exposure and limited the time of each condition to a maximum of 20 minutes.

To maintain a consistent level of difficulty across conditions, we utilized three different jigsaw puzzles from the same series by the same manufacturer. To standardize the puzzle completion strategy, we divided the puzzle pieces into 8 batches placed on the same shelves for all trials. Participants were restricted to retrieving and working on only one batch at a time within a rectangular frame marked on the table. This setup allowed us to incorporate locomotion into the task as participants moved between batches. The puzzles were assigned to the conditions randomly but were not repeated.

\subsection{Measures}
\subsubsection{Cybersickness} 
Although multiple VR sickness measurement schemas such as ARSQ, VRSQ, and CSQ-VR have been proposed, the Simulator Sickness Questionnaire (SSQ) remains the most prevalent in the literature and is still widely regarded as the standard for measuring VR sickness \cite{vinkers2024visual, kourtesis2023cybersickness, kim2018virtual, hussain2023augmented, hirzle2021critical}. SSQ is a self-reported checklist consisting of 16 symptoms categorized under four subscales: Nausea, Disorientation, Oculomotor, and Total Severity. 
%To assess the feasibility of including alternatives to the SSQ, we included the digital eye strain (DES) measure to augment the SSQ in our pilot studies \cite{}. However, our initial findings did not find the DES to be as effective as the SSQ. As a result, in our work, we primarily rely on the SSQ.

Participants rated the severity of these symptoms on a 4-point Likert scale (0 = none, 1 = slight, 2 = moderate, 3 = severe). SSQ scores were collected right before and after each condition. This allowed us to isolate the SSQ per condition by evaluating the difference between post and pre task SSQ scores, as evidenced in prior literature \cite{li2022mixed}. 

We also collected discomfort scores by asking participants to rate their discomfort immediately after completing each task while still wearing the headset. At the beginning of the study, participants were informed that discomfort referred to any sensation that would make them want to leave the setup \cite{fernandes2016combating}, including nausea, disorientation, and other symptoms captured by the SSQ. They answered the following question: “On a scale of 0 to 10, where 0 represents how you felt before starting and 10 means you want to stop, how do you feel now?” This approach, adapted from Fernandes and Feiner \cite{fernandes2016combating}, has been employed in previous VR and VST studies to facilitate real-time monitoring of discomfort and sickness throughout the trial \cite{adhanom2020effect,freiwald2018camera}. The discomfort score collected after the interaction task was considered as the ending discomfort score, and the average discomfort score was calculated using all the collected scores.

\subsubsection{Task Performance} In addition to the self-reported scores, we collected objective performance metrics for each task outlined in Section \ref{subsec:tasks}. For the typing task, we measured speed in characters per minute (CPM), calculated as the total number of correctly typed characters (including spaces) normalized to 60 seconds, and the error rate (ER). In the navigation task, we recorded navigation time in seconds and the number of cones dropped outside the designated drop zone (ER). For the interaction task, we tracked completion time and the number of correctly placed puzzle pieces. This allowed us to calculate the interaction performance metric of correctly placed puzzle pieces per minute (PPM).
\subsubsection{Qualitative Feedback} At the end of each trial, participants were asked to select their preferred VST condition and provide detailed feedback on their choice. Data from open-ended survey questions and observations were collected during and after participant experiences with the VST conditions. 

\subsection{Hypotheses}
Considering the previously described measures, we formulated the following hypotheses:\\
 \textbf{\hypertarget{hypo:H1}{H1}:} \Depthpassthrough passthrough (GAP) reduces cybersickness and subjective discomfort over \Directpassthrough passthrough (DP) \\
 \textbf{\hypertarget{hypo:H2}{H2}:} \Depthpassthrough passthrough (GAP) is preferred by users over \Directpassthrough passthrough (DP) \\
 \textbf{\hypertarget{hypo:H3}{H3}:} \Depthpassthrough passthrough (GAP) results in higher task performance compared to \Directpassthrough passthrough (DP) \\
 \textbf{\hypertarget{hypo:H4}{H4}:} DP and GAP induce a common set of symptoms which are not experienced under natural vision (NV)

\subsection{Procedure}
Before starting the trial, participants signed a consent form and completed a demographics questionnaire. Their IPD was measured with an optical digital pupilometer, and the HMD was adjusted to match their IPD. The experimenter then provided instructions and demonstrated the procedures for the typing, navigation, and interaction tasks. Participants were randomly assigned to one of six different condition orders, \edits{which stemmed from varying the sequence of NV, DP, and GAP,} determined using a balanced Latin square design. \edits{In each condition, participants completed the typing, navigation, and interaction tasks in a fixed order to focus on overall cybersickness rather than task-specific effects. Randomizing the task order could have introduced variability from headset wear time, conflating task-specific cybersickness with overall exposure effects.}

Participants wore the headset continuously until they finished all tasks for a given condition. Throughout each condition, the experimenter collected discomfort scores and noted down key observations. Participants filled out the SSQ before and after each condition. To allow cybersickness symptoms to subside, participants were required to take a break of at least 15 minutes between conditions. During these breaks, they had access to water and a space with windows.

After completing all conditions, participants answered open-ended survey questions to provide insights into their experiences. The interview began with a discussion of preferences between the VST conditions, followed by probing reasons for those preferences. The entire session, including the two 15-minute breaks, took approximately 90 minutes to complete.

% \subsection{Data Analysis} Descriptive statistics were used to potentially accentuate the differences in experienced cybersickness between conditions. Two researchers were tasked to analyse the qualitative data using a reflexive thematic analysis approach to uncover subtle and layered experiences and perceptions \cite{braun2006using,braun2019reflecting}. The choice of this methodology was motivated by previous studies that focused on understanding user experiences with VR technology \cite{chen2024d,tan2022understanding,knibbe2018dream}. The coders followed the six steps suggested by Braun and Clarke \cite{braun2006using}. The analysis followed an extensive review of the dataset to gain a deep understanding of the data and its context. Inter-rater reliability was computed using Cohen’s kappa \cite{cohen1960coefficient}. After this, the coders conducted a reflection session to identify insightful quotes and create a final classification on which they both agreed. This step allowed us to ensure that individual perspectives and biases did not greatly affect the analysis.

% \begin{figure*}[htpb!]
% \label{}
% \centering

%     {{\label{ROCIowaCedar} \includegraphics[width=\textwidth/3]{figures/IowaCedar_roc.png}}}%
%     \qquad
%     {{\label{ROCIowaDesMoines} \includegraphics[width=\textwidth/3]{figures/IowaDesMoines_roc.png} }%
%   \captionsetup{justification=centering}
%   \caption{\Acf{ROC} curves for \acf{RW} Iowa (CR) and  \acf{RW} Iowa (DM) dataset. Dummy model here represents a model whose output is solely a ``no Flood'' for all pixels.}
%   \label{fig:RW_ROC_Curves}%
% \end{figure*}



\section{Results and Discussions}
\label{sec:Results}

In this section, we aim to answer three main questions. First, we want to validate our hypothesis that \ac{SYN} data is a viable proxy for \ac{RW} data when training ML models for downscaling. Secondly, we seek to assess how much more skillful ML-based downscaling is compared to classical, non-data-driven techniques, such as our baseline methods, \textit{i.e.}, thresholded bicubic and Lanczos interpolation. Finally, we would like to appraise the extent to which data-driven models like ours are transferable (in terms of usefulness) to other regions without major performance degradations.  
To assess the quality of the models, we conduct a multiple comparison test --namely the Holm-Bonferroni procedure \cite{HolmBonferroni1979} -- that is designed to control the \ac{FWER}. We notice that, with a \ac{FWER} of $10^{-3}$, all the differences in model performance are significant. The only exception to this trend was observed in \ac{RW}-GH for whom the pairwise differences between \ac{RCAN} and \ac{ESRT}, Lanczos and Bicubic were not significant with the aforementioned \ac{FWER}. 

%Finally, we aim to find out the factors influencing the transferability of our models from one region to another.

\subsection{Potential of using SYN Data for RW downscaling}

In order to evaluate the utility of synthetic data for training, we compare performances of our candidate models on both \ac{SYN} and \ac{RW} Iowa data whose results are presented in Table \ref{tab:IowaResults}. We notice that 
\textbf{(i)} For the Iowa datasets, there is a drop in performance of all the models when going from \ac{SYN} to \ac{RW} datasets, 
\textbf{(ii)} for the \ac{RW}-IA (CR) as well as \ac{RW}-IA (DM) datasets, both bicubic and Lanczos interpolation have accuracies and MCC up to 70.89\% and 0.42 respectively while the deep learning models have accuracies and MCC up to 73.34\% and 0.46 respectively, 
\textbf{(iii)} There is a roughly 6\% accuracy improvement for the \ac{SYN} data for the deep learning models compared to the bicubic and lanczos models and this improvement drops to about 3\% for \ac{RW} data,  
\textbf{(iv)} the performance of all the models remain consistent across both \ac{RW}-IA datasets and \textbf{(v)} in \figref{fig:RW_ROC_Curves}, we observe that there is a high degree of overlap among the \ac{ROC} curves for the data-driven models.

From (i) and (iv) we can conclude that \ac{SYN} data is more intricate than \ac{RW} data. This implies that the benefits yielded by training with \ac{SYN} dataset, while significant, is not as prominent in the \ac{RW} Iowa datasets. 
% This may be due to sensor noise prevalent in the \ac{RW} Landsat-8 data that can be harder to reproduce in the synthetically generated examples. 
(i), (iii) and (v) implies that while \ac{SYN} data is not an exact replacement for \ac{RW} data, it provides a rather significant edge, which is all the more important when there is insufficient \ac{RW} for training. From (ii) we can conclude that the three proposed data driven models outperform classical super-resolution techniques such as bicubic and lanczos, conclusion supported by the \ac{ROC} curves in Figure \ref{fig:RW_ROC_Curves} for whom the data-driven models, in general, lie above the non-data-driven alternatives. Observation (iv) shows that  for the climatically similar \ac{RW}-Iowa(CR) and \ac{RW}-Iowa(DM) regions, training on \ac{SYN} Iowa data does indeed provide an edge. 

% have similar climate. 

\begin{figure*}[t!]
    \centering
    \begin{subfigure}[t]{0.5\textwidth}
        \centering
        \includegraphics[width=\textwidth/2]{figures/IowaCedar_roc.png}
        \caption{}
    \end{subfigure}%
    ~ 
    \begin{subfigure}[t]{0.5\textwidth}
        \centering
        \includegraphics[width=\textwidth/2]{figures/IowaDesMoines_roc.png}
        \caption{}
    \end{subfigure}
    \vspace*{0.5cm}
    \caption{    \label{fig:RW_ROC_Curves} \Acf{ROC} curves for (a) RW-IA (CR) and (b) RW-IA (DM) dataset. Na\"ive model here represents a model whose output is solely a ``no Flood'' for all pixels. Star here represents the pixel-wise classifier with a threshold of 0.5.}
\end{figure*}


\subsection{Effectiveness of data-driven approaches}

In order to evaluate the effectiveness of ML models in the downscaling task, we compare performances of our candidate models to Lanczos and bicubic interpolation methods by looking at figures of some sample predictions from Iowa (Figure \ref{fig:RWIowaDesMoines}), performance comparison in the region of Iowa in Table \ref{tab:IowaResults} and the ROC curves in Figure \ref{fig:RW_ROC_Curves} for \ac{RW} data. We notice that 
\textbf{(vi)} For RW-IA (DM) samples, the deep learning models maintain a higher degree of spatial continuity in the predicted \ac{FIM}, 
\textbf{(vii)} We observe that  bicubic and Lanczos interpolation produces over-smoothed \ac{FIM} reconstructions, while the plain \ac{RDN}, \ac{RCAN} and \ac{ESRT} models are more detail-inclusive. Similar conclusions can be drawn upon inspecting the \ac{ROC} curves in Figure \ref{fig:RW_ROC_Curves} and 
\textbf{(viii)} For RW-IA (CR), the ML models show a performance improvement of 3.06\% when comparing the best ML model and non-data-driven method and, while for RW-IA (DM) there is a performance improvement of 2.45\%.


Figures \ref{fig:EUSamples} and \ref{fig:RWIowaDesMoines} show the spatial disparity among the models whose details are often obscured in aggregated metrics such as accuracy. (vi) This implies that these data-driven models are better are recognizing an underlying stream network geometry than the classical methods. However, when it comes to narrow river streams, all the models struggle capturing the nuances of the \ac{FIM} resultant from localized high elevation features such as small islands within rivers or man-made structures. (vii) shows a clear advantage of our data-driven approaches over the non-data-driven alternatives. (viii) indicates the benefits of the data-driven models when evaluated over Iowa. 



\subsection{Applicability of our models to external regions}

To evaluate how transferable our models are, we draw conclusions from figures of the sample predictions from Western Europe (Figure \ref{fig:EUSamples}) and Ghana (Figure \ref{fig:GhanaSamples}) as well as the performance comparison in Table \ref{tab:ExternalResults}. We notice that 
\textbf{(ix)} for Ghana all of the models fail to adequately inundate the pixels over separated areas on account of several disconnected regions of inundation in the chosen area,
\textbf{(x)} the ML models outperform non-data driven methods for RW-EU, 
\textbf{(xi)} for the RW-EU dataset, there is an improvement of 4.89\% when comparing the accuracy of the best data- and non-data-driven methods, 
\textbf{(xii)} For RW-RR and RW-GH, there is marginal improvement (up to 0.77\% in accuracy) of the ML methods over the non-data driven methods and 
\textbf{(xiii)} For RW-EU, we notice that the ML models produce more connected streams over the non-data-driven models. 

(x) and (xi) implies that the models are transferable when considering hydroclimaticalogically similar regions since Iowa and the Meuse river in Europe lie within mid temperate zones. Similar to the observation (vi) for RW-IA (DM), (xiii) implies that the benefits of the ML model in identifying underlying network streams is also transferable to hydroclimatologically similar regions. In contrast, (xii) and (ix) both imply that the trained ML models struggle to generalize to RW-RR \& RW-GH. We speculate that this may be due to the significant differences in geography and climate when compared to Iowa. 

% More specifically, we notice that Ghana has a lot of disconnected regions when compared to Iowa and Western Europe, possibly indicating a geomorphological dissimilarity. Additionally, in the case of Red River and Ghana, we also speculate that they include drivers to flood inundation that are different from Iowa and Western Europe, which lie within mild temperate zones. Ghana on the other hand has a tropical (dry and hot) climate.

Our study directly implies that good quality synthetic data can be useful surrogates for downscaling low-resolution \acp{WFM} to high-resolution \acp{FIM} in regions, where such data are hard to come by, even when downscaling by a factor of 10. We noticed that such models were readily transferable to climatically similar regions as the region of training. However, Such derived ML models did not feature significantly different transferability when evaluated over hydroclimatologically dissimilar regions, which we attribute to different flood inundation characteristics, primarily at finer scales. A possible avenue to circumvent such issues is to explore additional training approaches that fall under the general area of domain adaptation. Nevertheless, data-driven models are still advantageous (and, hence, preferable) over non-data-driven alternatives in transfer scenarios like the one we considered here. 


%%%%%%%%%%%%%%%%%%%%%%%%%%%%%%% unused text %%%%%%%%%%%%%%%%%%%%%%%%%%%%%%%%%%%%%%%



% \tabref{tab:AccuracyResults} depicts test accuracies obtained by our models on both \ac{SYN} and \ac{RW} data. For Iowan floods, a comparison of \ac{SYN} and \ac{RW} results shows \textbf{(i)} bicubic and Lanczos interpolations remarkably gaining about $3\%$ in accuracy, as well as \textbf{(ii)} \ac{RDN} and \ac{RCAN} remaining relatively stable, while \textbf{(iii)} topography-aware models loosing $2.7\%$ in performance. From (i) one can conclude that \ac{SYN} data are morphologically slightly more intricate than \ac{RW} data. Also, (i) and (ii) likely imply that \ac{SYN} data, excluding topography, can serve as satisfactory surrogates of \ac{RW} data. However, as implied by (iii), our topography-dependent models seems to be particularly sensitive to distributional shifts of their combined inputs (\acp{WFM} and topographic features). More specifically, the topography-informed models' performance edge, while still statistically significant, is extremely marginal, even when compared to our non-data-driven approaches. Next, when comparing results between the cases of Iowan and Ghanaian \ac{RW} data, one observes that \textbf{(iv)} the accuracy of bicubic and Lanczos interpolations drops by almost $5\%$ due to over-smoothing. This may imply that Ghanaian \acp{FIM} bare a more complex morphology, when compared to Iowan \acp{FIM}. Also, \textbf{(v)} our topography-agnostic, data-driven models' performance degrades more gracefully (by about $2\%$), while \textbf{(vi)} our topography-aware models perform, virtually, as bad as our non-data-driven approaches. Hence, the differences in the data populations of the two regions we considered are significant enough to render our topography-dependent models noncompetitive. 



\section{Discussion and Future Work}\label{sec:discussion}
This paper pioneers the novel approach of selective response, showing that withholding responses can be a powerful tool for GenAI systems. By opting not to answer every query as accurately as it can---particularly when new or complex topics emerge---GenAI can encourage user participation on community-driven platforms and thereby generate more high-quality data for future training. This mechanism ultimately enhances GenAI's long-term performance and revenue. From a welfare perspective, our results indicate that such selective engagement can also benefit users, leading to better solutions and increased overall satisfaction. Since this work is the first to address selective response strategies for GenAI, numerous promising directions remain for future research; we highlight some of them below. 

First, from a technical standpoint, all of the results in this paper rely on Assumption~\ref{assumption: data lip}, involving the lipshitz condition of the accuracy function and the sensitivity parameter $\beta$. Future work could seek to relax this assumption. Furthermore, our constrained optimization approach in Subsection~\ref{sec: welfare constrained revenue maximization} could be extended to approximate the optimal (continuous) strategy instead of the optimal discrete strategy.

Second, our stylized model adopts the simplifying---though unrealistic---assumption that only a single GenAI platform exists. Admittedly, this makes it easier to focus on the idea of selective responses, and indeed, this assumption is pivotal in keeping our analysis tractable. Future research could explore scenarios with multiple GenAI platforms and human-centered forums. In such settings, one platform's selective response might redirect users not only to forums but also to competing GenAI platforms, leading to the tragedy of the commons \cite{hardin1968tragedy}: Although all GenAI platforms benefit from fresh data generation, none may choose to respond selectively if it means losing users to competitors. 

Third, we assumed Forum behaves non-strategically. In reality, human-centered platforms often monetize their data by selling it to GenAI platforms, adding a further layer of strategic interaction for GenAI. Moreover, data transfer between the platforms can form the basis for collaboration: GenAI could employ selective response to bolster Forum content creation, and Forum could, in turn, attribute that content to GenAI for subsequent use in retraining.


%Third, we make the (again) simplifying assumption that Forum is non-strategic. However, in practice, human-centered platforms can sell their data to GenAI platforms. This adds additional considerations for GenAI. Furthermore, data transmission between the platforms can also become the basis for collaboration: GenAI can use selective response to ensure enough content is generated in Forum, and Forum could provide the data attributed to this mechanism back to GenAI. 


%Second, this paper makes the simplifying yet unrealistic assumption of the existence of one GenAI platform. Indeed, this simplifies many aspects and allows us to analyze selective responses. Future work could address the data generation process with more than one GenAI platform and possibly several human-centered forums. In such a case, selective response of one GenAI platform can either drive users to forums or to other GenAI platforms; thus, we might face a tragedy of the commons situation~\ref{hardin1968tragedy}, where all GenAI platforms are interested in fresh data generation but none volunteer to selectively respond and lose users. 

%This paper examines the competition between a generative AI platform and human-based platforms, challenging the assumption that always providing answers is optimal. We analyzed the impact of withholding answers on GenAI's revenue and developed an efficient approximately optimal algorithm for this purpose. We further explored how withholding affects users, showing that it can lead to better outcomes compared to always answering. Specifically, we demonstrated that withholding can Pareto-dominate this strategy and derived the necessary and sufficient conditions for that. Finally, we proposed a second approximately optimal algorithm that maximizes GenAI's revenue while ensuring users are better off than when GenAI answers all queries.

%On a more conceptual level, our model assumes that GenAI’s data comes solely from the competing platform (Forum). Future research could explore a scenario where GenAI can purchase additional data from a third party. This extension could provide valuable insights into the interplay between withholding answers and data purchasing, and whether these two strategies can complement each other or must be traded off.
Software development is increasingly conceived as a collaboration activity between developers and AIs. Indeed, IDEs already implement features to enable interactive development, with AI suggesting implementations that are reused by developers.

Although multiple studies show this interaction can be successful, there is still limited understanding of how the models must be configured and used in the context of code generation tasks. This study addresses this gap, systematically investigating the impact of several key parameters, including the repeated submission of a prompt to accommodate for the non-deterministic nature of the models.

Our study reveals several key findings about the usage of ChatGPT. In particular, we discovered how creativity, although up to a limited extent, is useful to increase the range of methods whose code can be generated correctly. A major role is played by parameter top-p, which is commonly underrated, and instead has a major impact on the correctness of the results, with lower values producing better results. Finally, prompts should be submitted multiple times, with $5$ repetitions combined with a temperature of $1.2$ resulting in an effective configuration in our experiments.  

Future work concerns two main research directions. One is about replicating this experiment with other AI assistants, to validate our findings in multiple contexts. The second research direction concerns finding strategies to deal with the need to submit the same prompt multiple times to obtain a useful result, and thus developing approaches able to select or merge multiple responses automatically. 

\begin{acks}
To Abhishek Kar for his guidance during ideation of this work.
\end{acks}

%%
%% The acknowledgments section is defined using the "acks" environment
%% (and NOT an unnumbered section). This ensures the proper
%% identification of the section in the article metadata, and the
%% consistent spelling of the heading.
% \begin{acks}
% To Robert, for the bagels and explaining CMYK and color spaces.
% \end{acks}

%%
%% The next two lines define the bibliography style to be used, and
%% the bibliography file.
\bibliographystyle{ACM-Reference-Format}
\bibliography{sample-base}


%%
%% If your work has an appendix, this is the place to put it.
\appendix


\end{document}
\endinput
%%
%% End of file `sample-sigconf-authordraft.tex'.
