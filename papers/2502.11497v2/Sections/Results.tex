\section{Results}

\subsection{Quantitative Measures}

\subsubsection{Cybersickness} 
We report four sub-scores: Total Severity, Oculomotor, Nausea, and Disorientation calculated from the SSQ responses using the methodology established by Kennedy et al. \cite{kennedy1993simulator}. For our analysis, we used the differences between post-condition and pre-condition measurements as the dependent variable. The results of statistical analysis are summarized in Table \ref{tab:ssq}.

The Shapiro-Wilk test indicated that the SSQ scores were not normally distributed ($p$ < 0.001). Consequently, Friedman test was conducted for each SSQ sub-scores, which showed significant differences between conditions for Nausea ($\chi^2$ = 31.89, $p$ < 0.001), Oculomotor ($\chi^2$ = 25.97, $p$ < 0.001), Disorientation ($\chi^2$ = 23.89, $p$ < 0.001), and Total Severity ($\chi^2$ = 33.82, $p$ < 0.001). To identify specific condition differences, we performed pairwise Wilcoxon signed-rank tests with Holm-Bonferroni adjustment. NV resulted in significantly lower Nausea, Oculomotor, Disorientation, and Total Severity scores compared to both DP and GAP ($p$ < 0.05). \GAP led to significantly lower Nausea ($p$ = 0.016), Disorientation ($p$ = 0.029), and Total Severity ($p$ = 0.011) scores than \DP. Although \GAP had lower mean Oculomotor scores compared to \DP, the difference was not statistically significant.
Figure \ref{fig:ssq} displays the distribution of simulator sickness scores for all three conditions across sub-scores. Table \ref{tab:ssq} summarizes the \edits{mean, standard deviation, median, and interquartile range} of these sub-scores. We observed that disorientation disturbances contributed to simulator sickness the most, followed by oculomotor and nausea.

Inspired by Vovk et al. \cite{vovk2018simulator}, we examined individual SSQ symptoms in isolation to obtain a fine-grained understanding of VST induced cybersickness. As shown in Table \ref{tab:freq}, we observed that eyestrain and sweating were a major factor contributing to simulator sickness in VST in comparison to NV. The most reported symptoms for both \DP and \GAP were sweating, eyestrain, general discomfort, and headache. Wilcoxon signed-rank tests indicated that symptoms of general discomfort, sweating, eyestrain, headache, and blurred vision were significantly lower ($p<0.05$) for NV compared to both DP and GAP.

\begin{figure*}[!h]
  \centering
  \includegraphics[width=\textwidth]{images/ssq.png}
  \caption{\textbf{SSQ} Box plots of Nausea, Oculomotor, Disorientation, and Total subscores of simulator sickness comparing all conditions. NV led to significantly lower N, O, D, and total sickness scores than both \DP and \GAP (p < 0.05). \GAP led to significantly lower N (p = 0.016), D (p = 0.029), and total (p = 0.011) sickness scores than \DP.}
  \Description{Box plots of the SSQ scores: Compared to NV and GAP, DP has a broader range and higher values across all scores: nausea, oculomotor, disorientation, and total.}
  \label{fig:ssq}
\end{figure*}

\begin{table*}[!h]
  \caption{\edits{Mean, standard deviation, median, and interquartile range} of simulator sickness for all conditions}
  \label{tab:ssq}
  \resizebox{\textwidth}{!}{
  \begin{tabular}{r|cccc|cccc|cccc|cccc}
    \toprule
    &\multicolumn{4}{c}{Nausea (N)}&\multicolumn{4}{c}{Oculomotor (O)}& \multicolumn{4}{c}{Disorientation (D)}& \multicolumn{4}{c}{Total Score}\\
    \midrule
     \edits{Condition} &Mean &SD &Median &IQR &Mean &SD &Median &IQR &Mean &SD &Median &IQR &Mean &SD &Median &IQR\\
    \midrule
    \textit{\edits{NV}} & \textit{-1.19} &\textit{5.72}
    &\edits{\textit{0}}
    &\edits{\textit{0}}
    &\textit{-0.63} &\textit{4.85}
    &\edits{\textit{0}}
    &\edits{\textit{0}}
    &\textit{-2.32} &\textit{8.68}
    &\edits{\textit{0}}
    &\edits{\textit{0}}
    &\textit{-1.40} &\textit{5.99}
    &\edits{\textit{0}}
    &\edits{\textit{0}}\\
    \DP &25.44 &22.82 &\edits{19.08} &\edits{28.62} &25.27 &33.09 &\edits{15.16} &\edits{45.48} &37.70 &47.96 &\edits{13.92} &\edits{59.16} &32.57 &36.79 &\edits{18.70} &\edits{59.84}\\
    \GAP &\textbf{14.71} &20.05 &\edits{\textbf{9.54}} &\edits{19.08} &\textbf{17.06} &27.79 &\edits{\textbf{11.37}} &\edits{17.05} &\textbf{18.56} &34.72 &\edits{\textbf{6.96}} &\edits{17.40} &\textbf{19.17} &29.97 &\edits{\textbf{11.22}} &\edits{19.64}\\
  \bottomrule
\end{tabular}
}
\end{table*}

\begin{table*}[!h]
  \caption{Mean, standard deviation, and percentage of participants (\%) with SSQ symptoms for all conditions. (N – Nausea, O – Oculomotor, D – Disorientation) }
  \label{tab:freq}
  \resizebox{\textwidth}{!}{%
  \begin{tabular}{r|ccc|ccc|ccc|ccc|ccc}
    \toprule
                         & \multicolumn{3}{c|}{NV}         & \multicolumn{3}{c|}{DP} & \multicolumn{3}{c|}{GAP} & \multicolumn{3}{c|}{DP $\cap$ GAP} &   &   &   \\ \hline
SSQ                      & Mean  & SD   & \% & Mean    & SD      & \%    & Mean     & SD      & \%    & Mean     & SD      & \%    & N & O & D \\ \hline
General discomfort       & -0.04 & 0.36 & 4\%             & 0.67    & 0.80    & 50\%               & 0.50     & 0.82    & 42\%               & 0.58     & 0.81    & 25\%               & x & x &   \\
Fatigue                  & 0.04  & 0.36 & 8\%             & 0.38    & 0.70    & 29\%               & 0.21     & 0.71    & 17\%               & 0.29     & 0.71    & 17\%               &   & x &   \\
Headache                 & -0.04 & 0.20 & 0\%             & 0.50    & 0.82    & 33\%               & 0.25     & 0.43    & 25\%               & 0.38     & 0.67    & 25\%               &   & x &   \\
Eyestrain                & 0.00  & 0.29 & 4\%             & 0.50    & 0.71    & 42\%               & 0.54     & 0.82    & 46\%               & 0.52     & 0.76    & 33\%               &   & x &   \\
Difficulty focusing      & 0.00  & 0.00 & 0\%             & 0.42    & 0.70    & 29\%               & 0.25     & 0.52    & 21\%               & 0.33     & 0.62    & 17\%               &   & x & x \\
Increased Salivation     & 0.00  & 0.00 & 0\%             & 0.04    & 0.20    & 4\%                & 0.04     & 0.20    & 4\%                & 0.04     & 0.20    & 0\%                & x &   &   \\
Sweating                 & 0.08  & 0.28 & 8\%             & 0.92    & 0.76    & 71\%               & 0.58     & 0.64    & 50\%               & 0.75     & 0.72    & 50\%               & x &   &   \\
Nausea                   & -0.08 & 0.28 & 0\%             & 0.50    & 0.65    & 42\%               & 0.25     & 0.53    & 21\%               & 0.38     & 0.60    & 17\%               & x &   & x \\
Difficulty concentrating & 0.00  & 0.00 & 0\%             & 0.33    & 0.55    & 29\%               & 0.17     & 0.47    & 13\%               & 0.25     & 0.52    & 13\%               & x & x &   \\
Fullness of the head     & 0.00  & 0.29 & 4\%             & 0.33    & 0.62    & 25\%               & 0.25     & 0.66    & 18\%               & 0.29     & 0.64    & 17\%               &   &   & x \\
Blurred vision           & -0.04 & 0.20 & 0\%             & 0.54    & 0.82    & 38\%               & 0.33     & 0.69    & 25\%               & 0.44     & 0.76    & 21\%               &   & x & x \\
Dizziness (eyes open)    & -0.04 & 0.20 & 0\%             & 0.50    & 0.71    & 42\%               & 0.08     & 0.28    & 8\%                & 0.29     & 0.58    & 8\%                &   &   & x \\
Dizziness (eyes closed)  & 0.00  & 0.00 & 0\%             & 0.30    & 0.61    & 21\%               & 0.04     & 0.35    & 8\%                & 0.17     & 0.51    & 8\%                &   &   & x \\
Vertigo                  & 0.00  & 0.00 & 0\%             & 0.13    & 0.33    & 13\%               & 0.13     & 0.33    & 13\%               & 0.13     & 0.33    & 8\%                &   &   & x \\
Stomach awareness        & -0.08 & 0.28 & 0\%             & 0.13    & 0.44    & 17\%               & 0.00     & 0.50    & 4\%                & 0.063    & 0.47    & 4\%                & x &   &   \\
Burping                  & 0.00  & 0.00 & 0\%             & 0.08    & 0.28    & 8\%                & 0.00     & 0.00    & 0\%                & 0.04     & 0.20    & 0\%                & x &   &   \\ 
  \bottomrule
\end{tabular}%
}
\end{table*}

% \begin{table}[!h]
%   \caption{Mean, standard deviation, and percentage of participants with SSQ symptoms for VST conditions. (N – nausea, O – oculomotor, D – disorientation) }
%   \label{tab:freq}
%   \begin{tabular}{rcccccccccccc}
%     \toprule
%     &\multicolumn{3}{c}{\DP}&\multicolumn{3}{c}{\GAP}&\multicolumn{3}{c}{Both VST conditions} & & &\\
%     \midrule
%     SSQ &Mean &SD &>0 &Mean &SD &>0 &Mean &SD &>0 &N &O &D\\
%     \midrule
%     General discomfort &0.67 &0.80 &50\% &0.50 &0.82 &42\% &0.58 &0.81 &25\% &x &x &\\
%     Fatigue &0.38 &0.70 &29\% &0.21 &0.71 &17\% &0.29 &0.71	&17\% & &x &\\
%     Headache &0.50 &0.82 &33\% &0.25 &0.43 &25\% &0.38 &0.67 &25\% & &x &\\
%     Eyestrain &0.50 &0.71 &42\% &0.54 &0.82 &46\% &0.52 &0.76 &33\% & &x &\\
%     Difficulty focusing &0.42 &0.70 &29\% &0.25 &0.52 &21\% &0.33 &0.62 &17\% & &x &x\\
%     Increased Salivation &0.04 &0.20 &4\% &0.04 &0.20 &4\% &0.04 &0.20 &0\% &x & &\\
%     Sweating &0.92 &0.76 &71\% &0.58 &0.64 &50\% &0.75 &0.72 &50\% &x & &\\
%     Nausea &0.50 &0.65 &42\% &0.25 &0.53 &21\% &0.38 &0.60 &17\% &x & &x\\
%     Difficulty concentrating &0.33 &0.55 &29\% &0.17 &0.47 &13\% &0.25 &0.52 &13\% &x &x &\\
%     Fullness of the head &0.33 &0.62 &25\% &0.25 &0.66 &18\% &0.29 &0.64 &17\% & & &x\\
%     Blurred vision &0.54 &0.82 &38\% &0.33 &0.69 &25\% &0.44 &0.76 &21\% & &x &x\\
%     Dizziness (eyes open) &0.50 &0.71 &42\% &0.08 &0.28 &8\% &0.29 &0.58 &8\% & & &x\\
%     Dizziness (eyes closed) &0.30 &0.61 &21\% &0.04 &0.35 &8\% &0.17 &0.51 &8\% & & &x\\
%     Vertigo &0.13 &0.33 &13\% &0.13 &0.33 &13\% &0.13 &0.33 &8\% & & &x\\
%     Stomach awareness &0.13 &0.44 &17\% &0.00 &0.50 &4\% &0.063 &0.47 &4\% &x & &\\
%     Burping &0.08 &0.28 &8\% &0.00 &0.00 &0\% &0.04 &0.20 &0\% &x & &\\
%   \bottomrule
% \end{tabular}
% \end{table}

% \begin{figure}[!ht]
%   \centering
%   \includegraphics[width=\linewidth]{images/post-pre-ssq.png}
%   \caption{Simulator sickness scores comparing all conditions.}
%   \label{fig:post-pre-ssq}
%   \Description{A woman and a girl in white dresses sit in an open car.}
% \end{figure}

Subjective discomfort scores across all tasks and conditions were also tested for normality using the Shapiro-Wilk test, which showed non-normal distribution ($p$ < 0.001). Friedman tests revealed significant differences in discomfort scores between conditions for typing ($\chi^2$ = 26.60, $p$ < 0.001), navigation ($\chi^2$ = 28.00, $p$ < 0.001), interaction ($\chi^2$ = 33.34, $p$ < 0.001), and average discomfort ($\chi^2$ = 33.23, $p$ < 0.001). Pairwise Wilcoxon signed-rank tests with Holm-Bonferroni adjustment demonstrated that NV led to significantly lower discomfort scores across all tasks compared to both \DP and \GAP ($p$ < 0.05). NV also resulted in significantly lower average discomfort scores compared to both \DP and \GAP ($p$ < 0.05). GAP led to significantly lower discomfort scores across all tasks compared to DP, including typing ($p$ = 0.046), navigation ($p$ = 0.041), and interaction ($p$ = 0.022), as well as significantly lower average discomfort scores ($p$=0.016). Figure \ref{fig:disc-all} illustrates the distribution of discomfort scores for all three conditions across tasks, with navigation showing the highest mean discomfort, followed by interaction. Means and standard deviations of the discomfort scores are summarized in Table \ref{tab:subj-disc}.

\begin{figure*}[!ht]
  \centering
  \includegraphics[width=\textwidth]{images/disc_all.png}
  \caption{\textbf{Subjective Discomfort} Box plots of discomfort scores and preference for all conditions across the typing, navigation, and interaction tasks.  NV led to significantly lower discomfort scores than both \DP and \GAP (p < 0.05) across all tasks.  Compared to \DP, \GAP led to significantly lower discomfort scores across all tasks, i.e., typing (p = 0.046), navigation(p = 0.041), interaction (p = 0.022).}
  \label{fig:disc-all}
  \Description{Box plots of the subjective discomfort scores across the typing, navigation, and interaction tasks: Compared to NV and GAP, DP has higher values. However, GAP has a broader range. A stacked bar plot of user preference shows that GAP ranked first for all tasks and the overall experience.}
\end{figure*}

\begin{table}[!ht]
  \caption{Mean and standard deviation of subjective discomfort across tasks for all conditions}
  \label{tab:subj-disc}
  \resizebox{\columnwidth}{!}{
  \begin{tabular}{r|cc|cc|cc|cc}
    \toprule
    &\multicolumn{2}{c}{Typing}&\multicolumn{2}{c}{Navigation}& \multicolumn{2}{c}{Interaction}&\multicolumn{2}{c}{Average}\\
    \midrule
     \edits{Condition} &Mean &SD &Mean &SD &Mean &SD &Mean &SD\\
    \midrule
    \textit{\edits{NV}} & \textit{0.63} &\textit{1.11} &\textit{0.79} &\textit{1.12} &\textit{0.46} &\textit{0.96} &\textit{0.63}	&\textit{0.99}\\
    \DP &3.21 &2.47 &4.00 &2.57 &3.79 &2.63 &3.67 &2.38\\
    \GAP &\textbf{2.71} &2.57 &\textbf{3.25} &2.73 &\textbf{3.00} &2.65 &\textbf{2.99} &2.52\\
  \bottomrule
\end{tabular}
}
\end{table}

% \begin{figure}[h]
%   \centering
%   \includegraphics[width=\linewidth]{images/subj-disc.png}
%   \caption{Subjective discomfort scores comparing all conditions across the typing, navigation, and interaction tasks.}
%   \label{fig:subj-disc}
%   \Description{A woman and a girl in white dresses sit in an open car.}
% \end{figure}

% \begin{table}[!ht]
%   \caption{Mean and standard deviation of simulator sickness for all conditions (Post)}
%   \label{tab:freq}
%   \begin{tabular}{r|cc|cc|cc|cc}
%     \toprule
%     &\multicolumn{2}{c}{Nausea (N)}&\multicolumn{2}{c}{Oculomotor (O)}& \multicolumn{2}{c}{Disorientation (D)}& \multicolumn{2}{c}{Total Score}\\
%     \midrule
%      &Mean &SD &Mean &SD &Mean &SD &Mean &SD\\
%     \midrule
%     \textit{Natural vision} & \textit{3.18} &\textit{5.95} &\textit{4.42} &\textit{5.75} &\textit{0.58} &\textit{2.78} &\textit{3.58} &\textit{4.25}\\
%     Planar projection VST &28.22 &24.24 &28.74 &32.23 &38.28 &47.76 &35.53 &36.61\\
%     Depth reprojection VST &\textbf{20.27} &22.25 &\textbf{24.32} &30.55 &\textbf{24.94} &43.46 &\textbf{26.65} &34.42\\
%   \bottomrule
% \end{tabular}
% \end{table}

\begin{table*}[!ht]
  \caption{The results of statistical analysis for the SSQ and subjective discomfort scores ($^{\ast} p < 0.001$; $^{\ast\ast} p < 0.05$)}
  \label{tab:stats}
  \resizebox{0.7\textwidth}{!}{
  \begin{tabular}{ll|l|l}
    \toprule
     \multicolumn{2}{l|}{\textbf{Dependent Variables}} & \textbf{Friedman} & \textbf{Wilcoxon signed-rank} \\
    \midrule
    \textbf{SSQ} & Nausea & $\chi^2 = 31.89^{\ast}$ & \DP > NV$^{\ast\ast}$, \GAP > NV$^{\ast\ast}$, \DP > \GAP$^{\ast\ast}$ \\
    & Oculomotor & $\chi^2 = 25.97^{\ast}$ & \DP > NV$^{\ast\ast}$, \GAP > NV$^{\ast\ast}$ \\
    & Disorientation & $\chi^2 = 23.89^{\ast}$ & \DP > NV$^{\ast\ast}$, \GAP > NV$^{\ast\ast}$, \DP > \GAP$^{\ast\ast}$ \\
    & Total & $\chi^2 = 33.82^{\ast}$ & \DP > NV$^{\ast\ast}$, \GAP > NV$^{\ast\ast}$, \DP > \GAP$^{\ast\ast}$ \\
    \midrule
    \textbf{Discomfort} & Typing & $\chi^2 = 26.60^{\ast}$ & \DP > NV$^{\ast\ast}$, \GAP > NV$^{\ast\ast}$, \DP > \GAP$^{\ast\ast}$ \\
    & Navigation & $\chi^2 = 28.00^{\ast}$ & \DP > NV$^{\ast\ast}$, \GAP > NV$^{\ast\ast}$, \DP > \GAP$^{\ast\ast}$ \\
    & Interaction & $\chi^2 = 33.34^{\ast}$ & \DP > NV$^{\ast\ast}$, \GAP > NV$^{\ast\ast}$, \DP > \GAP$^{\ast\ast}$ \\
    & Average & $\chi^2 = 33.23^{\ast}$ & \DP > NV$^{\ast\ast}$, \GAP > NV$^{\ast\ast}$, \DP > \GAP$^{\ast\ast}$ \\
    \bottomrule
  \end{tabular}
  }
\end{table*}

\subsubsection{Task Performance} Shapiro-Wilk tests indicated that the data for CPM, typing error rate, and navigation error rate were not normally distributed ($p$ < 0.001). Friedman tests revealed significant differences in CPM ($\chi^2$ = 20.33, $p$ < 0.001) and typing error rate ($\chi^2$ = 6.10, $p$ < 0.001) between conditions. However, no significant differences were found in navigation error rate. Pairwise Wilcoxon tests with Holm-Bonferroni adjustment showed that both DP and GAP had significantly decreased CPM compared to NV ($p$ < 0.001), but no significant difference was found between DP and GAP. Typing error rate did not differ significantly between conditions.

Data for navigation time and PPM were normally distributed for NV, DP, and GAP ($p$ > 0.001). A repeated-measures ANOVA revealed significant differences in both navigation time (\( F(2, 46) = 27.373 \), \( p < 0.001 \), \( \eta^2 = 0.543 \)) and PPM (\( F(2, 46) = 8.88 \), \( p < 0.001 \), \(\eta^2 = 0.279 \)). Post-hoc paired t-tests with Holm-Bonferroni adjustment showed that navigation time and PPM were significantly different for NV compared to both DP and GAP (\( p < 0.05 \)). No significant differences were found between DP and GAP for navigation time (\( p = 0.779 \)) or PPM (\( p = 0.880 \)).

Although differences between DP and GAP were not statistically significant, GAP demonstrated improved mean typing CPM, typing error rate, navigation time, and navigation error rate compared to DP. Means and standard deviations of the task performance scores are summarized in Table \ref{tab:perf}.

\begin{figure*}[!ht]
  \centering
  \includegraphics[width=\textwidth]{images/task-perf.png}
  \label{fig:task-perf}
  \caption{\textbf{Task performance results:} (a) Typing speed (characters per minute), (b) Typing error rate (\%), (c) Navigation time (min), (c) Navigation error rate (\%), and (d) Puzzle speed (pieces per minute).}
  \Description{Bar plots of the task performance scores. The trends in the means are as follows. CPM: NV>GAP>DP, Typing ER (\%): NV<GAP<DP, Navigation Time (min): NV<GAP<DP, Navigation Error (\%): NV<GAP<DP, PPM: NV>DP>GAP. DP mean values are very slightly different than GAP mean values.}
\end{figure*}

\begin{table*}[!ht]
  \caption{Mean and standard deviation of task performance scores (CPM - characters per minute, ER - error rate in typing and navigation tasks; PPM - pieces per minute in interaction task) for all conditions. The best task performance is highlighted in bold.}
  \label{tab:perf}
  \resizebox{0.7\textwidth}{!}{
  \begin{tabular}{r|cc|cc|cc|cc|cc}
    \toprule
    &\multicolumn{4}{c}{Typing}|&\multicolumn{4}{c}{Navigation}|& \multicolumn{2}{c}{Interaction}\\
    \midrule
    &\multicolumn{2}{c}{CPM}&\multicolumn{2}{c}{ER (\%)}|& \multicolumn{2}{c}{Time (s)}&\multicolumn{2}{c}{ER (\%)}|&\multicolumn{2}{c}{PPM}\\
    \midrule
     \edits{Condition} &Mean &SD &Mean &SD &Mean &SD &Mean &SD &Mean &SD\\
    \midrule
    \textit{\edits{NV}} & \textit{60.83} &\textit{17.92} &\textit{1.75} &\textit{1.51} &\textit{120.13} &\textit{16.67} &\textit{1.25}	&\textit{3.38} &5.97 &1.95\\
    \DP &44.13 &13.04 &3.29 &3.06 &137.76 &21.22 &3.75 &5.76 &\textbf{4.65} &1.27\\
    \GAP &\textbf{46.21} &19.20 &\textbf{3.04} &2.65 &\textbf{136.21} &16.50 &\textbf{2.08} &4.15 &4.57 &1.44\\
  \bottomrule
\end{tabular}
}
\end{table*}

\subsection{Qualitative Feedback}
\label{subsec:qualfeedback}
% In this section, we present the results corresponding to the qualitative data collected using the open-ended survey. We first present our findings related to participant preference. We then discuss our thematic analysis to capture participants' experiences with the two VST conditions. The analysis provides detailed insights into discomfort, cybersickness, and perceived visual artifacts in each condition. We explore how the two VST modes compare in terms of their impact on user experience. 
% Table \ref{tab:thematic} displays the five identified themes and codes as well as the number of related text segments.  The kappa value $\kappa$ was 0.92, which shows an almost perfect level of agreement.

\subsubsection{\textbf{Users Prefer GAP}} Based on overall experience, participants generally preferred \GAP over \DP as show in Figure \ref{fig:disc-all}. For typing, 46\% participants preferred \GAP while 33\% preferred DP. Similarly, participants preferred \GAP for both the tasks navigation (GAP: 46\% vs. DP: 17\%) and interaction (GAP: 46\% vs. DP: 38\%).

\subsubsection{\textbf{GAP Reduces Cybersickness and Discomfort Compared to DP}}
User feedback indicated that DP led to more frequent cybersickness symptoms and discomfort , such as dizziness, eyestrain, and headache compared to GAP. This is consistent with the quantitative SSQ results, which showed significantly higher total cybersickness scores for DP. For instance, P4 commented, \textit{"The passthrough with [DP] was more straining on my eyes,"} and noted, \textit{"I would get tired of using this mode over a long period."}
% User feedback indicated that DP led to more frequent cybersickness symptoms and discomfort (27/38), such as dizziness (8/11), eyestrain (4/5), and headache (3/3) compared to GAP. This is consistent with the quantitative SSQ results, which showed significantly higher total cybersickness scores for DP. For instance, P4 commented, \textit{"The passthrough with [DP] was more straining on my eyes,"} and noted, \textit{"I would get tired of using this mode over a long period."}

% Although some symptoms were reported in open-ended survey questions, additional symptoms emerged from the SSQ (Table \ref{tab:freq}), suggesting that the SSQ captures a more comprehensive range of symptoms.  

\subsubsection{\textbf{Sensory Conflict in DP Causes Discomfort}}
Several participants reported a mismatch between vision and motion with DP, aligning with the sensory-conflict theory, the most accepted explanation for motion sickness. This was brought up more frequently in DP than in GAP by the participants, supporting our hypothesis that DP exaggerates motion effects. P17 described their experience: \textit{"What I see does not match what is happening...Do you trust what you see or how you move? These don't overlap, and I don't know which one to trust."} P12 also noted, \textit{"[DP] is worse when there is head motion, especially when completing the puzzle, because everything feels like it is moving."} 

\subsubsection{\textbf{Geometry Enhances Spatial Awareness}}
Participants reported impaired spatial awareness and depth perception with DP compared to GAP. Observations from the researcher confirmed that participants had to move closer to cones during the navigation task with DP due to inaccurate depth cues. User feedback supported these observations. For example, P17 said, \textit{"Depth perception was worse with [DP], and I tested that with the cone experiment. With [GAP], I could toss the cone in and not miss but with [DP] I had to be more careful about where to put the cone."} Additionally, unstable gait and more frequent collisions with furniture were noted in DP: 
\textit{"I almost knocked things over. I had to be more cognizant of my steps around the furniture and not hurt myself with the furniture. With [GAP], I did not have that problem."}

\subsubsection{\textbf{Artifacts in VST}}
Artifacts such as motion blur with DP and warping with GAP were recurrent themes. P17 described the experience with DP: 
\textit{"It is similar to car sickness because what I see does not match what is happening...The mismatch between vision and motion is more noticeable with [DP] because of the blur. It is another layer of having your brain decode the blurry image to understand the lock between perception and motion"}. While most participants preferred GAP for all tasks, some favored DP for typing due to warping issues on the keyboard: \textit{"I prefer \GAP for movements but \DP for typing because of the distortions on the keys"} (P1). A few participants adapted to warping over time:\textit{"I noticed distortions before starting to walk but it became better during the experience"} (P5). Warping was particularly noticeable at the edges, with feedback such as,
\textit{"[GAP] is less dizzying but feels like swimming in water. It feels good as long as I am looking in the center"} (P12). \edits{Furthermore, we also asked participants to report the reasoning for preferring a passthrough mode (\DP vs \GAP), wherein out of 25 participants recruited for our study, only 18 provided reasons for disliking a particular mode. 10 of these participants specifically indicated that image distortions (warping on objects and edges) were bothersome. This finding further emphasizes on the importance of measuring and minimizing warping artifacts for designing better passthrough systems.} 


\subsubsection{\textbf{Avenues for Future Research}}
Participant feedback also helped identify several areas for future research into enhancing comfort in VST. Four issues emerged: frame drops, overexposed images, latency, and blurry vision.  Several participants reported that slight delays when moving their heads caused nausea and discomfort. For instance, P1 noted, \textit{"The slight lagging when moving my head causes nausea"}. Blurry vision was frequently cited as a significant issue, particularly with DP. Out of 93 comments, 18 specifically mentioned blurry vision, with a higher incidence observed in DP. Participants described how blurry visuals affected their ability to perform tasks comfortably. P19 observed, 
\textit{"Overall, my vision was less clear with [DP], especially with the typing challenge...I had to squint longer and go closer to the screen to be able to see the words."} 

% \begin{table}[!ht]
%   \caption{Themes and code frequencies of the participant open-ended survey data.}
%   \label{tab:thematic}
%   \begin{tabular}{lllll}
%     \toprule
%     Themes & Codes & \DP & \GAP & Total \\
%     \midrule
%     Cybersickness and Discomfort &Nausea &3 &3 &6 \\
%     \cline{2-5} 
%      & Dizziness &8 &3 &11 \\
%     \cline{2-5} 
%      & Disorientation &1  &2  &3  \\
%      \cline{2-5} 
%      & Eyestrain &4  &1  &5  \\
%      \cline{2-5} 
%      & Headache &3  &0  &3  \\
%      \cline{2-5} 
%      & Fatigue &0  &1  &1  \\
%      \cline{2-5} 
%      & Motion Sickness &4  &0  &4  \\
%      \cline{2-5} 
%      & Sensory Conflict &4  &1  &5  \\
%     \midrule
%     Artifacts &Motion Blur &3 &0 &3 \\
%     \cline{2-5} 
%      & Warping &0 &5 &5 \\
%      \cline{2-5} 
%      & Latency &3  &1  &4  \\
%      \cline{2-5} 
%      & Frame Drops &1  &2  &3  \\
%      \cline{2-5} 
%      & Overexposed Images &3  &2  &5  \\
%     \midrule
%     Visual Clarity &Blurry Vision &13 &5 &18 \\
%     \cline{2-5} 
%      & Difficulty Reading &3 &3 &6 \\
%      \midrule
%     Depth Perception and Focus &Impaired Depth Perception &6 &5 &11 \\
%     \cline{2-5} 
%      & Focus Issues &1 &4 &5 \\
%     \cline{2-5} 
%      & Spatial Awareness &0  &2  &2  \\
%     \midrule
%     Usability & Comfort for Extended Use &0  &1  &1  \\
%     \bottomrule
%   \end{tabular}
% \end{table}

% \begin{figure}[!ht]
%   \centering
%   \includegraphics[width=\linewidth]{images/post-ssq.png}
%   \caption{Simulator sickness scores comparing all conditions.}
%   \Description{A woman and a girl in white dresses sit in an open car.}
% \end{figure}