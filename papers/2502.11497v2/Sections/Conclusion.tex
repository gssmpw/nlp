\section{Conclusion}
In this work, we investigated the impact of reprojection algorithms on user discomfort and cybersickness in VST HMDs. We first proposed metrics to evaluate geometry correctness and warping artifacts generated by \depthpassthrough reprojection. We show how these metrics allow comparison between different passthrough algorithms with a focus on the percevied scene geometry and object shapes. 

We presented a comprehensive protocol aimed at evaluating visually-induced discomfort and cybersickness in VST HMDs through key use cases. Our results indicate that GAP significantly reduces nausea, disorientation, and total scores of cybersickness as well as subjective discomfort scores as compared to DP. Notably, our findings revealed specific symptoms unique to VST systems. Following sweating, eyestrain, a symptom in the oculomotor group, emerged as the main factor contributing to simulator sickness in VST. This suggests that cybersickness in VST systems is a complex phenomenon, distinct from both VR cybersickness and traditional motion sickness. Although the levels of cybersickness observed were generally low to moderate, these symptoms can influence user acceptance and overall experience. The feedback collected from users revealed new avenues for further research and improvements in VST HMD comfort. We hope that our comprehensive protocol sets a foundation for future studies aimed at refining these systems and enhancing user comfort in VST technologies.