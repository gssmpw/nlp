\section{User Study} \label{study}

As AR headsets are becoming popular with an increasing number of applications in medicine, education, and gaming, it becomes important to holistically evaluate the discomfort experienced by users in these passthrough systems. To address this, we introduce a protocol specifically tailored to quantify cybersickness in the context of VST HMD use cases. \edits{To our knowledge, no VST studies comprehensively combine user motion, interaction, and sickness metrics, making this protocol an early effort to define reasonable benchmarks.}

\edits{\subsection{Experiment Design Considerations}
{Our key considerations for the experimental setup were reproducibility, repeatability, and real-life relevance while the study design focused on reliably eliciting signals related to user comfort. To achieve this, we began with tasks identified in the literature, tested them in a pilot study, and iteratively refined the task nature and duration based on participant feedback. We detail the specific steps we took to ensure these principles were upheld:
\subsubsection{Standardization of Setup and Metrics}
{We standardized all aspects of the experimental setup to ensure reproducibility and consistency. This included the layout of the room, object placement, and controlled environmental factors such as lighting and headset parameters. We provide this information in our Supplementary material. For cybersickness and discomfort measurements, we employed established tools which are widely validated in prior literature.}
\subsubsection{Pilot Testing and Iterative Refinement}
Pilot studies played a crucial role in refining task design and duration to reliably induce motion sickness while maintaining ecological validity. Tasks were initially drawn from existing research that we summarize in our Supplementary material and adjusted based on observed results and participant feedback. Specific triggers of motion sickness such as head motion, depth perception tasks, and visual-motor coordination were emphasized in the task design. For example, we replaced straight-line walking tasks with multi-directional navigation requiring turns and object interactions to better simulate realistic scenarios that elicit depth perception challenges and frequent head motion.
\subsubsection{Real-Life Relevance}
The protocol was designed to mirror real-world VST use cases, ensuring relevance to everyday applications. Tasks included typing on a physical keyboard for productivity, navigating complex physical environments, and interacting with tangible objects, such as completing assembly tasks. This mix of stationary (near-field interaction) and dynamic (locomotion-based) activities ensured a comprehensive evaluation of user discomfort across different contexts.
}}

Below, we outline the design of our protocol, including task selection and quantification methods. We then apply our protocol to understand the effects of the reprojection algorithms on discomfort and cybersickness when using a VST HMD. Given the subjective nature of cybersickness and discomfort, we employ a within-subject design in our protocol where participants complete three tasks involving head motion, hand-eye coordination, and untethered locomotion. We collect both objective task performance metrics and quantitative and qualitative subjective feedback under three conditions: Natural Vision (NV), \Directpassthrough Passthrough (DP), and \DepthPassthrough Passthrough (GAP).

\begin{figure*}[ht]
    \centering
    \includegraphics[width=\textwidth,trim={0.2cm 10cm 0.2cm 0.2cm},clip]{images/Tasks.pdf}
    \caption{\textbf{User Study Setup.} Pictures of the lab setup for the three tasks completed by the participants while wearing the VST HMD.}
    \Description{The figure shows the user study setup in the lab space. The left sub-figure shows the setup for the typing task: a laptop and Dvorak keyboard. Their location is marked with tape for consistency between participants. The middle sub-figure shows the setup for the navigation task. The cones are numbered and placed around pieces of furniture such as a plant, 2 tables, and 2 chairs. The right sub-figure shows the setup for the interaction task. Puzzle pieces are placed on shelves. The table is marked with tape to indicate the frame in which the completed puzzle should fit.}
    \label{fig:tasks}
\end{figure*}

\subsection{Participants}
For the user study, we recruited 25 participants who had normal or corrected-to-normal vision. The study included a diverse range of ages, genders, levels of VR usage, and job profiles. Table \ref{tab:demographics} gives a full overview of the demographics. Recruitment was done in accordance with the ethics board of our institution. We excluded data from one participant from our analysis since they spent an abnormally long time to complete the protocol. \edits{Since the rest of the participants exhibited no irregular behavior, all their data were retained for analysis including participants who reported higher than average sickness scores.}

\begin{table}[!ht]
  \caption{Participant demographics for the user study, showing diversity across gender, age, normal versus corrected vision, and VR usage.}
  \label{tab:demographics}
  \begin{tabular}{l|l|l}
    \toprule
     \textbf{Variable} & \textbf{Categories} & \textbf{\#Participants}\\
    \midrule
    \textbf{Gender} &Men &14\\
    & Women &11\\
    \midrule
    \textbf{Age Group} &18-24 &1\\
    &25-34 &12\\
    &35-44 &9\\
    &45-54 &3\\
    \midrule
    \textbf{Vision} &Normal &19\\
    &Contact Lenses &6\\
    \midrule
    \textbf{VR Usage} &Never &1\\
    &Once &12\\
    &Once a week &3\\
    &Once a month &8\\
    &At least once a day &1\\
    \bottomrule
  \end{tabular}
\end{table}

% (14 males, 11 females). 1 participant was 18-24 years old, 12 were 25-34 years old, 9 were 35-44 years old, and 3 were 45-54 years old. None of the participants were over 55 years old. 
% Participants had normal or corrected-to-normal vision (19 participants had normal vision and 6 participants wore contact lenses). 1 participant never used VR before, 12 participants used VR once, 3 participants use VR once a week, 8 participants use VR once a month, and 1 participant uses VR at least once a day. 
\subsection{Tasks}\label{subsec:tasks} 

We devised our protocol focusing \emph{exclusively} on passthrough-based real-world interactions and ensured no virtual elements \cite{de2024visual} were visible to participants. Figure \ref{fig:tasks} shows pictures of the lab setup for the three tasks completed by the participants. The tasks were inspired from fundamental real-world AR applications such as working with laptops for productivity, navigation in the physical world, and interaction with real-objects. They emphasized on the user head motion while  necessitating inspection and spatial awareness of the physical world.
% head motion, realistic interaction, and spatial awareness within a physical environment. 
Our design ensured a cumulative duration to be roughly 15 minutes for all the tasks.

\subsubsection{Typing} Typing is a familiar activity that effectively engages both visual and motor components, making it relevant for evaluating fine-selecting physical objects and digital screen usage. \edits{This task was specifically chosen to reflect emerging applications in productivity such as using VST HMDs like the Apple Vision Pro as extensions of traditional work setups where users interact with physical keyboards and screens. It is representative of fine motor interactions requiring frequent gaze shifts between the keyboard and the laptop screen, a common trigger for visual discomfort. } A physical Dvorak keyboard \cite{dvorak1936typewriting} was used to minimize reliance on muscle memory and encourage frequent visual engagement with the keys. A typing assessment was conducted to measure speed, accuracy, and overall proficiency. Participants typed for 6 minutes as quickly and accurately as possible, with the typing text randomized for each session to ensure variability. \edits{To maintain repeatability and reproducibility, the placement of the laptop and keyboard was standardized across all participants, and screen brightness was kept consistent throughout the study.}
\subsubsection{Navigation} The realistic and holistic use of a VST HMD involves navigating physical spaces, avoiding obstacles, and interacting with real-world objects \cite{bailenson2024seeing,de2018augmented,erickson2019cold}. Previous research has also explored the impact of spatial navigation, such as waypoint-following, on sickness in VR  \cite{al2019effect}. We designed a navigation task where participants collected and dropped off 10 numbered cones, one at a time, into a designated drop zone. \edits{To emphasize geometry perception, the task included narrow passages and required multi-directional movements, including turning to locate cones and interacting with objects at varying heights.} Further, the cones were placed at different heights, requiring a range of movements to pick and place. Participants were given 2-3 minutes to familiarize themselves with the cone locations before beginning the trial. During the task, participants were instructed to move naturally but carefully to avoid colliding with any objects in the room.
\subsubsection{Interaction} Li et al. \cite{li2022mixed} introduced a Tangram puzzle task as a representative activity that simulates common assembly tasks requiring both motor and cognitive skills. Inspired by their approach, we adapted this task by using jigsaw puzzles consisting of 24 pieces and measuring 2 x 3 feet. This larger puzzle size was selected to accommodate head motion which is often associated with motion sickness. We chose a 24-piece puzzle to mitigate prolonged VST exposure and limited the time of each condition to a maximum of 20 minutes.

To maintain a consistent level of difficulty across conditions, we utilized three different jigsaw puzzles from the same series by the same manufacturer. To standardize the puzzle completion strategy, we divided the puzzle pieces into 8 batches placed on the same shelves for all trials. Participants were restricted to retrieving and working on only one batch at a time within a rectangular frame marked on the table. This setup allowed us to incorporate locomotion into the task as participants moved between batches. The puzzles were assigned to the conditions randomly but were not repeated.

\subsection{Measures}
\subsubsection{Cybersickness} 
Although multiple VR sickness measurement schemas such as ARSQ, VRSQ, and CSQ-VR have been proposed, the Simulator Sickness Questionnaire (SSQ) remains the most prevalent in the literature and is still widely regarded as the standard for measuring VR sickness \cite{vinkers2024visual, kourtesis2023cybersickness, kim2018virtual, hussain2023augmented, hirzle2021critical}. SSQ is a self-reported checklist consisting of 16 symptoms categorized under four subscales: Nausea, Disorientation, Oculomotor, and Total Severity. 
%To assess the feasibility of including alternatives to the SSQ, we included the digital eye strain (DES) measure to augment the SSQ in our pilot studies \cite{}. However, our initial findings did not find the DES to be as effective as the SSQ. As a result, in our work, we primarily rely on the SSQ.

Participants rated the severity of these symptoms on a 4-point Likert scale (0 = none, 1 = slight, 2 = moderate, 3 = severe). SSQ scores were collected right before and after each condition. This allowed us to isolate the SSQ per condition by evaluating the difference between post and pre task SSQ scores, as evidenced in prior literature \cite{li2022mixed}. 

We also collected discomfort scores by asking participants to rate their discomfort immediately after completing each task while still wearing the headset. At the beginning of the study, participants were informed that discomfort referred to any sensation that would make them want to leave the setup \cite{fernandes2016combating}, including nausea, disorientation, and other symptoms captured by the SSQ. They answered the following question: “On a scale of 0 to 10, where 0 represents how you felt before starting and 10 means you want to stop, how do you feel now?” This approach, adapted from Fernandes and Feiner \cite{fernandes2016combating}, has been employed in previous VR and VST studies to facilitate real-time monitoring of discomfort and sickness throughout the trial \cite{adhanom2020effect,freiwald2018camera}. The discomfort score collected after the interaction task was considered as the ending discomfort score, and the average discomfort score was calculated using all the collected scores.

\subsubsection{Task Performance} In addition to the self-reported scores, we collected objective performance metrics for each task outlined in Section \ref{subsec:tasks}. For the typing task, we measured speed in characters per minute (CPM), calculated as the total number of correctly typed characters (including spaces) normalized to 60 seconds, and the error rate (ER). In the navigation task, we recorded navigation time in seconds and the number of cones dropped outside the designated drop zone (ER). For the interaction task, we tracked completion time and the number of correctly placed puzzle pieces. This allowed us to calculate the interaction performance metric of correctly placed puzzle pieces per minute (PPM).
\subsubsection{Qualitative Feedback} At the end of each trial, participants were asked to select their preferred VST condition and provide detailed feedback on their choice. Data from open-ended survey questions and observations were collected during and after participant experiences with the VST conditions. 

\subsection{Hypotheses}
Considering the previously described measures, we formulated the following hypotheses:\\
 \textbf{\hypertarget{hypo:H1}{H1}:} \Depthpassthrough passthrough (GAP) reduces cybersickness and subjective discomfort over \Directpassthrough passthrough (DP) \\
 \textbf{\hypertarget{hypo:H2}{H2}:} \Depthpassthrough passthrough (GAP) is preferred by users over \Directpassthrough passthrough (DP) \\
 \textbf{\hypertarget{hypo:H3}{H3}:} \Depthpassthrough passthrough (GAP) results in higher task performance compared to \Directpassthrough passthrough (DP) \\
 \textbf{\hypertarget{hypo:H4}{H4}:} DP and GAP induce a common set of symptoms which are not experienced under natural vision (NV)

\subsection{Procedure}
Before starting the trial, participants signed a consent form and completed a demographics questionnaire. Their IPD was measured with an optical digital pupilometer, and the HMD was adjusted to match their IPD. The experimenter then provided instructions and demonstrated the procedures for the typing, navigation, and interaction tasks. Participants were randomly assigned to one of six different condition orders, \edits{which stemmed from varying the sequence of NV, DP, and GAP,} determined using a balanced Latin square design. \edits{In each condition, participants completed the typing, navigation, and interaction tasks in a fixed order to focus on overall cybersickness rather than task-specific effects. Randomizing the task order could have introduced variability from headset wear time, conflating task-specific cybersickness with overall exposure effects.}

Participants wore the headset continuously until they finished all tasks for a given condition. Throughout each condition, the experimenter collected discomfort scores and noted down key observations. Participants filled out the SSQ before and after each condition. To allow cybersickness symptoms to subside, participants were required to take a break of at least 15 minutes between conditions. During these breaks, they had access to water and a space with windows.

After completing all conditions, participants answered open-ended survey questions to provide insights into their experiences. The interview began with a discussion of preferences between the VST conditions, followed by probing reasons for those preferences. The entire session, including the two 15-minute breaks, took approximately 90 minutes to complete.

% \subsection{Data Analysis} Descriptive statistics were used to potentially accentuate the differences in experienced cybersickness between conditions. Two researchers were tasked to analyse the qualitative data using a reflexive thematic analysis approach to uncover subtle and layered experiences and perceptions \cite{braun2006using,braun2019reflecting}. The choice of this methodology was motivated by previous studies that focused on understanding user experiences with VR technology \cite{chen2024d,tan2022understanding,knibbe2018dream}. The coders followed the six steps suggested by Braun and Clarke \cite{braun2006using}. The analysis followed an extensive review of the dataset to gain a deep understanding of the data and its context. Inter-rater reliability was computed using Cohen’s kappa \cite{cohen1960coefficient}. After this, the coders conducted a reflection session to identify insightful quotes and create a final classification on which they both agreed. This step allowed us to ensure that individual perspectives and biases did not greatly affect the analysis.