\section{Discussion}
Our work primarily focuses on the impact of reprojection in passthrough on cybersickness and discomfort when using a VST HMD. Below, we first discuss the importance and impact of the proposed metrics in designing passthrough systems. We then discuss the main results of our user study and provide implications for improving VST systems. Lastly, we provide directions for future work while discussing the limitations of our work.

\subsection{Benchmarking and Technical Metrics}

One of our main findings is the improvement in cybersickness experienced by the users with GAP over DP. However, it is very expensive and time consuming to run user studies for evaluating comfort for every aspect of reprojection such as depth estimation network, device calibration, and with different tasks. Moreover, we expect small changes in GAP to only have small changes in the SSQ measurements. In this light, we proposed two metrics which directly aim to evaluate the two fundamental aspects of passthrough i.e. (i) perceived location of objects on the HMD display and (ii) the warping artifacts like stretching and bending of objects. We utilize these metrics to select the right geometry estimation method. 
Our spatial reprojection error metric both evaluates and elucidates perceived geometrical errors. Further, the warping error metric quantifies and thus enables minimization of added distortions that could alter the shape of rigid objects in the reprojected imagery. Together, these metrics can fundamentally enhance GAP systems, with user studies employed judiciously.
% We note that our spatial reprojection error metric not only evaluates but explains perceived geometrical errors. 
% In addition, the warping error metric allows quantifying and therefore minimizing added distortions that could change shape of rigid objects in the reprojected imagery. 
% In conjunction, these two metrics can therefore be utilized to fundamentally improve \depthpassthrough passthrough systems and relying on user studies judiciously.

\subsection{GAP Mitigates Cybersickness}
The results of our user study showed a signifcant reduction in  nausea, disorientation, and total scores of simulator sickness with GAP compared to DP. GAP also led to significantly lower subjective discomfort scores across all tasks: typing, navigation, and interaction as well as a significantly lower average discomfort score. These two findings confirm (\hyperlink{hypo:H1}{H1}). We identified that \GAP exposed participants to low levels of cybersickness while \DP exposed participants to moderate levels of cybersickness \cite{hale2014handbook,stanney2020identifying}. 
Participants who experienced greater discomfort with DP showed a clear preference for GAP, which was ranked first in all tasks and for the overall experience, leading to
acceptance of (\hyperlink{hypo:H2}{H2}). Qualitative feedback provided additional insights into the sources of discomfort associated with DP. Participants experienced varying levels of cybersickness depending on the task. Specifically, many preferred GAP for navigation tasks due to a noticeable mismatch between visual and inertial cues, which was evident in their unstable gait and impaired spatial awareness. In contrast, some participants were more affected by head motion during the interaction task as they assembled the puzzle. Despite these subjective preferences, no significant differences were found in the five objective measures of task performance, leading to the rejection of (\hyperlink{hypo:H3}{H3}).
% We considered the differences between the post-condition and pre-condition measurements as the dependent variable. 
\subsection{Distinctive Features of Cybersickness in VST}  
As expected, natural vision outperformed both VST conditions in cybersickness and discomfort as well as task performance. In line with previous work on VST \cite{de2024visual}, our results revealed a VST symptom profile for cybersickness that is distinguished from other types of motion sickness. Based on the results of the SSQ, cybersickness in VST showed a  Disorientation > Oculomotor > Nausea profile. In contrast, previous studies have reported different symptom profiles for other types of motion sickness: VR cybersickness typically follows a Disorientation > Nausea > Oculomotor pattern, simulator sickness follows an Oculomotor > Nausea > Disorientation pattern, and sea sickness follows a Nausea > Oculomotor > Disorientation pattern \cite{de2024visual,gallagher2018cybersickness,somrak2019estimating}. Symptoms of general discomfort, sweating, eye strain, headache, and blurred vision were significantly lower for NV compared to both DP and GAP. This finding supports (\hyperlink{hypo:H4}{H4}). 
While symptoms of nausea, difficulty concentrating, difficulty focusing, and dizziness (eyes open) are significantly lower with NV compared to DP, they show no significant differences between NV and GAP. 
This suggests that GAP effectively reduces some aspects of cybersickness but does not fully bridge the gap to natural vision. 
The most frequent symptoms experienced with both VST conditions are sweating followed by eyestrain, headache, and general discomfort. 
Interestingly, these symptoms are also the most frequent symptoms identified by Vovk et al. in their simulator sickness evaluation with the Microsoft HoloLens, an AR OST HMD \cite{vovk2018simulator}. 
The least frequent ones are burping and increased salivation.
\subsection{Implications for VST HMD Design}
Our study demonstrates that GAP significantly enhances user comfort compared to DP. This implies that VST HMD design should incorporate GAP to improve user comfort and facilitate the adoption of VST technology. However, we must be aware that GAP introduces additional computational demands compared to DP. Future design efforts should strive for a balance that maintains the benefits of GAP while minimizing the impact on system performance and latency. 
% For instance, efficient machine learning models for estimating depth or geometry can be utilized to further reduce the computation needs for GAP based systems. 
Lastly, we perform depth estimation to facilitate the reprojection process, but many recent view synthesis approaches \cite{kerbl3Dgaussians, mildenhall2020nerf, li2023dynibar} have come up which can be optimized for real-time on-device usage to devise geometrically accurate and comfortable passthrough solutions.
% \edits{\subsection{Impact of Reprojection Error on Comfort}We also conducted a small pilot experiment with 6 new participants demonstrating the impact of our metrics on comfort. We considered two passthrough systems with varying depth quality (GAP - HighQ, GAP - LowQ) and ran these on our protocol. GAP - HighQ is same as GAP we employed for our user study in previous sections, and GAP - LowQ uses a degraded version of the depth than GAP - HighQ such that it has about 8 times higher reprojection error and 28\% smaller warping error. We observed that our protocol could clearly distinguish improved comfort between GAP - HighQ and GAP - LowQ (sickness scores increase by a factor of 2 on average with degradation of depth). This indicates that increased spatial reprojection error has raised user discomfort even with lesser warping artifacts. While precisely calculating correlation or causality is out of scope for our work, we hope that similar studies with larger sample sizes can become great avenues for future research.}
\subsection{Limitations and Future Work}
Apart from reprojection and geometry perception, there are a wide range of other factors in VST HMDs which affect the user comfort such as the latency and temporal consistency of scene updates, field of view, physical fit of the device, image quality factors such as noise and sharpness levels, accuracy of display colors, camera optics of the device among others. 
This is reflected in the qualitative feedback where latency, frame drops, overexposed images, and blurry vision are recurrently mentioned. Particularly, blurry vision is the most frequently cited issue with a higher incidence in DP. Essentially, all the discrepancies between human natural vision and VST HMD need to be minimized for a comfortable VST experience. %Modern camera and imaging pipelines are typically created for digital photography use case but not designed from first principles for VST HMD comfort. 
% Perceived scene brightness and dynamic range are dictated by a combination of VST camera’s hardware and software image processing pipelines as well as the choice of digital display technology such as OLED vs LCD. Either too high or too low perceived brightness outside of an optimal range may cause eye strain in prolonged usage sessions [REF]. What may work well for purely virtual content in a VR HMD device may not work as well for VST in an AR or XR device. Accuracy of display colors, image noise, and sharpness also affects user comfort. Most software image processing pipelines are created and tuned for photography use cases for modern smartphones and digital cameras. For HMD VST, low latency and perceptual temporal consistency are far more important. Designing camera hardware and software image processing specifically for the HMD VST use case and treating user comfort as first principle has received much less attention in the literature [REF].\newline
This suggests the need for more research to understand the impact of these factors on mitigating user discomfort and cybersickness. 

\edits{Our findings emphasize that motion sickness mitigation should not be taken for granted in VST systems. While GAP significantly reduces motion sickness compared to DP, it does so at a computational cost, requiring additional power and pipeline complexity. However, the value that GAP adds to user comfort and sickness mitigation justifies this cost. We encourage future work to explore the balance between system performance, energy efficiency, and user comfort in XR systems. GAP pipelines that dynamically scale computational demands based on user motion or environmental factors could further optimize this balance, reducing unnecessary energy expenditure while maintaining the benefits of sickness mitigation.}

\edits{One limitation of our study was the within-subjects design with shorter washout periods between conditions. We acknowledge that assuming participants can fully recover from cybersickness symptoms within the same session may be risky as recovery can take up to 24 hours for some individuals. However, we adopted this approach to directly capture user preferences across conditions and to balance the practical constraints of participant recruitment and retention. To mitigate carryover effects, we implemented proper counterbalancing and provided participants with a 15-minute break between conditions, informing them they could extend this break if needed. None of the participants reported requiring additional time. Furthermore, we analyzed the differences between post and pre task SSQ scores for each condition to isolate the relative cybersickness impact.}

\edits{Assessing passthrough systems is challenging due to their complexity and the interdependency between various factors (framerate, power consumption, depth quality, rendering pipeline, Motion2Photon latency), making it hard and potentially redundant to evaluate or correlate a single component’s impact on comfort.} While in this work, we solely focused on isolating the impact of GAP, we hope that our proposed protocol would encourage future work in understanding the impact of other factors on visually-induced sickness in VST HMDs. \edits{Given the growing prevalence of VST technology, we believe it is crucial to further investigate comfort and usability, which remain underexplored.} Finally, we expect future work to correlate our proposed metrics to the user comfort. This could bring new insights and potentially new measures for assessing passthrough comfort.

% Lastly, our study did not include overlaid virtual information as we
% aimed at focusing on the actual VST methods. Future work can look at cybersickness induced by virtual content in VST such as world-locked content.