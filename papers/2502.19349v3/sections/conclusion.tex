\section{Conclusion}
% In this paper, we present a novel framework for predicting next-day 
% cryptocurrency closing prices. To our knowledge, this is the first model 
% to successfully integrate three crucial factors influencing 
% cryptocurrency dynamics: macro environment fluctuations, 
% technical indicators, individual 
% cryptocurrency price changes, and market sentiment. 
\begin{comment}  CIKM 
We present a novel framework for predicting next-day cryptocurrency closing prices by integrating three crucial factors: macro environment fluctuations, individual cryptocurrency price changes, and market sentiment.
We employ a dual-prediction mechanism that separately forecasts prices using cross-correlated data from the top $n$ cryptocurrencies and the price dynamics of the target cryptocurrency. These predictions are then scaled and fused based on market sentiment derived from LLM-based cryptocurrency news analysis. Extensive experiments on a real-world dataset consistently show performance improvements over eight comparison methods, demonstrating the effectiveness of our proposed design.
\end{comment}
In this paper, we present ``CryptoPulse'', a new approach to predicting the next-day closing prices of cryptocurrencies. This model integrates three primary factors: fluctuations in the macro environment, changes in individual cryptocurrency prices and technical indicators, and the current crypto market mood. By leveraging a dual prediction mechanism, the model captures both the macro market environment and the specific price and technical indicator dynamics of the target cryptocurrency. Moreover, a fusion component based on the market sentiment information integrates these predictions to improve the results. The experimental evaluation shows that our model achieves higher accuracy in predicting cryptocurrency fluctuations compared to ten different methods, making it suitable for application in the highly unpredictable cryptocurrency market.
\section{Acknowledgment}
This work is supported in part by the National Science Foundation via grants NSF CNS-2431176 and NSF ITE-2431845. The US Government is authorized to reproduce and distribute reprints of this work for Governmental purposes notwithstanding any copyright annotation thereon. Disclaimer: The views and conclusions contained herein are those of the authors and should not be interpreted as necessarily representing the official policies or endorsements, either expressed or implied, of NSF, or the U.S. Government.