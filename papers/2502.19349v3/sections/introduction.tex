\section{Introduction}\label{sec:intro}

Cryptocurrencies have recently become a topic of conversation due to their great 
impact on the financial world.
This heightened attention is fueled by several factors including the sudden drops and 
shocks in cryptocurrency markets~\cite{10.1016/j.irfa.2016.02.008}, which offer 
opportunities for substantial returns, and the innovative technologies underpinning 
these assets, such as Blockchain~\cite{10.2139/ssrn.2692487,10.1108/mf-09-2018-0451}.
Unlike traditional financial markets such as bonds and stocks, the cryptocurrency 
market is characterized by a comparatively smaller market capitalization and 
pronounced volatility in short-term fluctuations~\cite{cheah2015speculative}, creating 
a unique and challenging investment landscape. 
This volatility stems from a complex interplay of factors that perpetuate a self-
fulfilling cycle.
On one hand, a large proportion of cryptocurrency investors seek short-term 
investments to exploit opportunities for rapid and substantial 
returns~\cite{fang2022cryptocurrency}, thereby intensifying market volatility. 
On the other hand, given this context, these investors tend to be highly sensitive to 
market-influencing events reported in news~\cite{10.22541/au.167285886.66422340/v1}, 
such as regulatory actions and fraud events, with their often exaggerated reactions 
further fueling market fluctuations. 
Regardless, cryptocurrency is increasingly recognized as a viable alternative 
investment avenue by those with higher risk tolerances or an interest in short-term, 
high-yield opportunities~\cite{trabelsi2018there}. 
Therefore, the ability to accurately predict short-term cryptocurrency prices not only 
holds significant practical importance but also contributes integrally to 
understanding the dynamics of the financial markets as a whole.
 
Many studies have employed machine learning techniques such as
SVM~\cite{akyildirim2021prediction} and Random Forests~\cite{akyildirim2021prediction}
%AutoRegressive Integrated Moving Average (ARIMA)~\cite{icsil2018bitcoin,8566476,si2022using} 
to forecast the returns of major cryptocurrencies based on historical price data. 
However, these methods often suffer from varied and unstable performance across 
different timescales and cryptocurrencies~\cite{akyildirim2021prediction}, due to 
their inability to capture complex and rapidly changing market dynamics. 
To address this, recent research has focused on using deep learning models like LSTM, 
bi-LSTM, GRU~\cite{zhengyang2019prediction,seabe2023forecasting,hamayel2021novel} and 
CNN-LSTM~\cite{li2020bitcoin} to forecast the prices of major cryptocurrencies.
However, this study group is confined  only to the top few cryptocurrencies by market 
capitalization, ignoring those with different behaviors and lower liquidity. 
Furthermore, these studies primarily relied on historical price data and did not 
incorporate technical indicators and sentiment analysis, potentially overlooking the 
influence of overbought or oversold market conditions, market sentiment shifts, and 
external news events on price volatility.

More recently, researchers have integrated market sentiment by analyzing news data and 
integrating it with historical price data to predict cryptocurrency prices, 
specifically focusing on Bitcoin and 
Ethereum~\cite{lamon2017cryptocurrency,pang2019cryptocurrency}. 
NLP approaches are employed to categorize news sentiment, which is then fed into deep 
learning models like LSTM, along with the historical price data, to predict future 
prices~\cite{vo2019sentiment}. 
However, such studies are rare and typically limited to specific cryptocurrencies 
because they rely on manually labeling sentiment data, which is labor-intensive and 
doesn't scale well for real-time predictions across multiple 
cryptocurrencies~\cite{vo2019sentiment}, and using investors' expectations caused by 
news alone as a trading strategy has been found to be inadequate, as concluded by 
Brown and Cliff~\cite{brown2004investor}.

To overcome the above-mentioned challenges, this paper introduces ``CryptoPulse,'' a 
novel framework designed for forecasting next-day closing prices by leveraging three 
primary factors: 1) broad market sentiment as reflected in real-time news, 2) complex 
dynamics of price changes embedded in the historical data and technical indicators of 
the target cryptocurrency, and 3) macro investing environment indicated by the 
fluctuations of major cryptocurrencies. In particular, the key contributions and 
highlights of this paper are summarized as follows:
\begin{itemize}%[topsep=5pt, partopsep=0pt, leftmargin=1em]
    \item Formulated a novel framework for next-day cryptocurrency 
    forecasting, leveraging short-term observations of key market 
    indicators including market sentiment, macro investing environment,
    technical indicators, and inherent pricing dynamics.
    \item Designed a novel prompting strategy using few-shot learning and 
    consistency-based calibration for effective LLM-based market sentiment 
    analysis of cryptocurrency news.
    \item Developed a dual-prediction mechanism that separately forecasts 
    prices based on macro conditions and cryptocurrency dynamics, then 
    fuses them using a market sentiment-driven strategy for enhanced 
    accuracy.
    \item Conducted extensive evaluations on a newly curated, large-scale 
    real-world dataset to demonstrate our model's effectiveness in next-day 
    price prediction against ten comparison methods. This dataset, sourced 
    from Yahoo Finance\footnote{https://finance.yahoo.com/crypto/} and 
    Cointelegraph\footnote{https://cointelegraph.com/}, along with the 
    source code, will be publicly available for download upon acceptance.
    %This dataset and the source code are available for download~\footnote{Removed to conform with double-blind submission requirements.}.
\end{itemize}

% In particular, we propose a novel prompting strategy that combines few-shot learning with a consistency-based calibration method to leverage large language models (LLMs) for cryptocurrency news article market sentiment analysis without manually labeling news datasets for each cryptocurrency. 
% Furthermore, we introduce a dual-prediction mechanism that forecasts future prices using the macro market environment and the target cryptocurrency's price dynamics separately. 
% These dual predictions are then integrated through a market sentiment-driven fusion component to produce the final prediction.
% The contributions and highlights of this paper include:
% \begin{itemize}[topsep=5pt, partopsep=0pt, leftmargin=1em]
%     \item Designed a novel prompting strategy using few-shot learning and consistency-based calibration for effective LLM-based market sentiment analysis of cryptocurrency news.
%     \item Developed a dual-prediction mechanism that separately forecasts prices based on macro conditions and cryptocurrency dynamics, then fuses them using a market sentiment-driven strategy for enhanced accuracy.
%     \item Conducted extensive evaluations on a new large-scale real-world dataset to demonstrate our model's effectiveness in next-day price prediction against 8 comparison methods. This dataset and the source code are available for download~\footnote{Removed to conform with double-blind submission requirements.}.
% \end{itemize}

    % Developed a dual-prediction mechanism that separately forecasts prices based on macro market conditions and cryptocurrency price dynamics, which are then scaled and fused using a market sentiment-driven strategy for enhanced forecasting accuracy.
    
% Using a 7-day rolling window, our approach aims to capture both short-term trends and sentiment shifts to predict next-day closing prices of cryptocurrencies effectively. 

    %This dataset, sourced from Yahoo Finance\footnote{https://finance.yahoo.com/crypto/} and Cointelegraph\footnote{https://cointelegraph.com/}, along with the source code, are publicly available for download\footnote{Removed to conform with double-blind submission requirements.}.

% \textbullet\ Designed a novel prompting strategy using few-shot learning and consistency-based calibration for effective LLM-based market sentiment analysis of cryptocurrency news. \\
% \textbullet\ Developed a dual-prediction mechanism that separately forecasts prices based on macro market conditions and cryptocurrency price dynamics, which are then scaled and fused using a market sentiment-driven strategy for enhanced forecasting accuracy.\\
% % Developed a dual-prediction mechanism that forecasts future prices based on macro market conditions and target cryptocurrency price dynamics, and proposed a market sentiment-driven fusion strategy to effectively combine these predictions for improved accuracy.\\
% \textbullet\ Conducted extensive evaluations on a new large-scale real-world dataset to demonstrate our model's effectiveness in next-day price prediction against 8 comparison methods. 
% This dataset, sourced from Yahoo Finance\footnote{https://finance.yahoo.com/crypto/} and Cointelegraph\footnote{https://cointelegraph.com/}, along with the source code, are publicly available for download\footnote{Removed to conform with double-blind submission requirements.}.

% Conducted extensive empirical evaluations on a newly curated, large-scale real-world dataset to demonstrate our model's effectiveness in modulating cryptocurrency volatility and correlating macro market sentiment with price fluctuations. This dataset, sourced from Yahoo Finance\footnote{https://finance.yahoo.com/crypto/} and Cointelegraph\footnote{https://cointelegraph.com/}, along with the source code, are publicly available for download\footnote{Removed to conform with double-blind submission requirements.}.


% \begin{itemize}
%     % \item Formulated a novel framework for next-day cryptocurrency forecasting, leveraging short-term observations of key market indicators including market sentiment, macro investing environment, and inherent pricing dynamics.
%     \item Designed  novel prompting strategy using few-shot learning and consistency-based calibration for effective LLM-based market sentiment analysis of cryptocurrency news.
%     \item Developed a cross-correlation dual-prediction mechanism that forecasts future prices based on macro market conditions and target cryptocurrency price dynamics, and proposed a market sentiment-driven fusion strategy to effectively combine these predictions for improved accuracy.    
%     % Developed a cross-correlation dual-prediction mechanism that separately forecasts future prices based on macro market conditions and target cryptocurrency price dynamics, which are then integrated through a market sentiment-driven fusion component to enhance the accuracy of final price forecasts.
%     \item Conducted extensive empirical evaluations on a newly curated, large-scale real-world dataset to demonstrate our model's effectiveness in modulating cryptocurrency volatility and correlating macro market sentiment with price fluctuations. This dataset, sourced from Yahoo Finance\footnote{https://finance.yahoo.com/crypto/} and Cointelegraph\footnote{https://cointelegraph.com/}, along with the source code, are publicly available for download\footnote{Removed to conform with double-blind submission requirements.}.
% \end{itemize}

% \textbullet\ Formulated a novel framework for next-day cryptocurrency forecasting utilizing short-term observations of key market indicators, including market sentiment, the macro investing environment, and inherent pricing dynamics.

% \textbullet\ Designed a novel prompting strategy that utilizes few-shot learning and consistency-based calibration for LLM-based effective market sentiment analysis of cryptocurrency news.

% \textbullet\ Developed a cross-correlation dual-prediction mechanism that separately forecasts future prices based on macro market conditions and target cryptocurrency price dynamics, which are then integrated through a market sentiment-driven fusion component to enhance the accuracy of final price forecasts

% \textbullet\ Conducted an extensive empirical evaluation using a newly curated, large-scale real-world dataset, demonstrating our model’s ability to capture the dynamics of cryptocurrency volatility and the correlation between macro market sentiment and price fluctuations. The dataset includes historical price data sourced from Yahoo Finance and news data from Cointelegraph, which will be made available to the research community for further exploration.