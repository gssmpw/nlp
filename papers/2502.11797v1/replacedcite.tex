\section{Related Work}
Much of the progress in participatory budgeting has built on prior work in multiwinner voting, i.e.,
a special case of participatory budgeting with unit costs ____.
The Extended Justified Representation (EJR) axiom was first introduced in this context by ____.

%\paragraph{Proportional Multiwinner Voting Methods}{
Besides MES, Phragm\'en's method ____ and Proportional Approval Voting (PAV) ____ are well-established proportional rules in the multiwinner voting setting; ____ provides
an excellent overview.
PAV satisfies EJR ____ and has optimal proportionality degree, but is NP-hard to compute. Its threshold-based local search variant 
is polynomial-time computable  ____, but may be hard to explain to voters. 
Phragm\'en's sequential method is also market-based.
Unlike MES, it is exhaustive, but it does not satisfy EJR. %It inspired the Method of Equal Shares ____, which is considered the first ``natural" polynomial-time rule to satisfy EJR in the multiwinner voting context.  
%However, ____ shows that Phragm\'en and MES have the same proportionality degree making Phragm\'en preferable in the MWV setting.}
Moreover, while ____ extended
MES to participatory budgeting with general additive utilities, neither PAV nor Phragm\'en have been adapted to this general setting.

Exact Equal Shares for cardinal utilities was implicitly studied by ____ and ____.
____ introduce a stability notion for participatory budgeting that is satisfied by the outcome of Exact Equal Shares. ____ propose an adaptive version of EES for cardinal utilities, which uses the outcome of EES for a smaller budget to compute the outcome of EES for a larger budget more efficiently, an alternative approach that complements our work.
They also consider the problem of finding the minimum budget increment that changes the election outcome, and propose an $O(n^2m)$ algorithm for this problem in the context of cardinal utilities. However, in practice, both in cities and in DAOs, the number of voters ($n$) is usually large, while the number of projects ($m$) is relatively small. As EES itself has a linear dependency on $n$, a completion method with quadratic dependency on $n$ is undesirable.


%They also give a naive $O(mn^2 \log n)$\footnote{Note that in participatory budgeting usually the number of voters $n$ is large while the number of projects $m$ is small, so a linear dependence on $n$ is desirable for such a method to be useful in practice.} algorithm to compute the next outcome for which the projects and/or voters paying for them change.
%Improving upon this result, our work provides a "natural" $O(mn)$ algorithm for cardinal utilities and introduces an $O(m^2n)$ time algorithm for the case of general identical utilities.