% \section{Interviews of Educational Researchers Who identified programming plans}

%\section{Interviewing Instructors to Unveil The Design Process For Programming Plans}


\section{Challenges in the Process of Programming Plan Identification}
\label{sec:interview_results}
To understand the current process of programming plan identification by educators and its challenges, we conducted semi-structured interviews with ten computing instructors (see Table~\ref{tab:formative-participants}) who have identified lists of plans for use in instruction or understanding student skill development. 

\subsection{Study Design}
Mirroring the approach of Fowler et al.~\cite{craig_max_methoda}, we recruited instructors who authored a computing education research publication where plan identification was a part of their methodology or results. This allowed us to discuss their concrete programming plan identification experiences, rather than have participants speculate on the process in general. Computing education researchers are an ideal source to learn instructional best practices because this population consists largely of instructor-scholars: those who not only teach computing but are also informed by research and innovative practices in computing education. 



% adding sentences to complete the overview

% complete the study overview -- its kinda incomplete
% why those participants and why these methods

%Our literature review showed that authors rarely provide detail about the process they undertake to identify programming plans. To more deeply understand the process of programming plan identification, we interviewed the authors of the publications from the literature review.


% \subsection{Participant selection and recruitment}
% \edit{\subsubsection{Participants}



% \edit{\subsubsection{Researcher–Participant Relationship}}
% we were avoiding this

% \subsection{Participant Recruitment}
% \subsubsection{Literature Review of plan identification techniques}

% \subsection{Participant Recruitment}
% To locate individuals who had identified programming plans for educational purposes, we reviewed the computing education and HCI literature for publications where the authors identified programming plans as a part of their methodology or results. We contacted the authors of these papers to request an interview. By targeting the authors of such publications, we were able to ask participants about a specific and concrete programming plan identification progress, rather than have participants speculate on the process in general.



%\edit{Specifically, we tried to identify computing educators who identified novel programming plans in their prior work. While several researchers have designed systems to identify common code sequences , most of such publications just focus on identifying code plans. In this paper, we try to unveil the process and challenges associated with the process of plan identification with a specific focus on their use in pedagogy to meet learner's goals. }


% We searched the ACM digital library within SIGCSE and SIGCHI sponsored venues, as well as the Computer Science Education journal for papers containing any of the following terms:

% \begin{quote}
%    "programming plan*" OR "program plan*" OR "program plan*" OR "programming template*" OR "pattern-oriented instruction" OR "plan-like" OR "unplan-like" 
% \end{quote}

% After eliminating posters and any non-publications, our search resulted in 61 results from SIGCSE sponsored venues, 55 results from SIGCHI sponsored venues, and 5 results from Computer Science Education. 
% The second author reviewed these publications to identify those where the authors specified that they had generated some number of programming plans, and where the programming plans in question were used to understand learners or to design for learners' needs. Papers where the authors explicitly adopted their programming plans from another source were not included. In cases where the second author was unsure about whether the paper met the inclusion criteria, both authors discussed to reach a final conclusion. 
% As the second author reviewed the papers, they used a ``backwards snowball'' process \cite{Claes_SnowballingTechniques_EASE-2014} where works cited by papers in the original search were also included if they met the inclusion criteria. 
% The final set of papers that met the inclusion criteria included 40 publications, dating from 1979-2023. 

% \subsubsection{\edit{Summary of the literature review}}
% \edit{Here we briefly summarize the work of the papers we identified in the literature review.}

% \subsubsection{Selection criteria}
% We contacted the authors of the papers identified in our literature review to request an interview. 
% We began with the first author and moved through the author list until we interviewed an author or all authors were emailed. 
%\edit{We contacted 3-4 educators every week until we reached the end of our duration that we set aside for conducting the interviews and until we reached the end of the shortlisted papers.  We interviewed all educators who showed willingness to answer questions except for any responses we received after the timeline for conducting interviews ended.} 
% We did not contact authors who are no longer active in CSEd research, either because (a) they hadn't published in the last 15 years, or (b) they were a student author who after graduation no longer worked in academia nor published.
%\edit{At the end of each interview, asked participants if they recommended any other potential interviewees.}



% describe why our population is good/why we chose our population

% \subsection{Study Participants}

% Researcher Description
% \edit{\subsubsection{Interviewers:} The first author conducted this study under the active supervision of the second author.  
% % Theyhave been working with the first author since early this year and have a thorough and extensive understanding of the research and its goals. 
% The second author has performed identification of novel programming plans in order to inform the design of curricula.}
% %\edit{In that work, programming plans are influenced by the propositions presented by Soloway.}

% Our ten participants hailed from North America, Europe, and Australasia, with diverse backgrounds that span not only instruction and research, but also industry (see Table~\ref{tab:participants}). Our interviewees included both those early in their teaching careers (1-5 years of instruction experience) as well as those highly senior in their field (20+ years of experience). Most, but not all, taught introductory programming (CS1). Most of our participants had been performing programming plans research for 1-6 years, while a couple had over 20 years of experience. All of our participants used programming plans in their instructional practice.


% better phrase and connect to RQs
% \subsection{Data Collection}
% \subsection{Interviews}
% Our interviews were performed over Zoom, and lasted thirty minutes to one hour. One interviewee requested written communication rather than a Zoom interview. 

% Re-write based on the style of the other paper to include more detail
% \subsection{Study Setup}
We conducted online semi-structured interviews that ranged between 30 minutes and one hour. 
%We noted the extent of their experience teaching and researching with programming plans, as well as their understanding of what a programming plan was. 
%For example, we explicitly asked the interviewees questions related to the time they have spent teaching computer science and researching in computer science education with a particular focus on programming plans research. 
%We also asked questions about their understanding of programming plans in general.
We asked participants to describe the specific activities they undertook to identify programming plans in the paper they authored. This included questions about what they were looking for when they identified programming plans, the resources that were involved in this process, and the procedures they undertook.
We also posed questions to gain insights into the difficulties the interviewees faced when they were identifying programming plans.
%and their ideas of solutions to solve these challenges to identify potential areas where technology can help.}
Lastly, we prompted a discussion about how they would attempt to identify programming plans in a new topic area they were unfamiliar with.
%to move towards automated plan identification in new domains.}

%\edit{During this process, after every section of the questions, we asked follow-up questions for clarifications if necessary.}


% During the interviews, we asked questions about \edit{\textbf{(a) participants' background} to understand the extent of their experience researching with programming plans and its use in instruction, \textbf{(b) their understanding of programming plans generally} to ensure consistency with the definition of a programming plan, \textbf{(c) the specific activities they undertook to identify programming plans in their paper} to deeply understand their process of plan identification, \textbf{(d) challenges they encountered} to identify areas where technology can help, and \textbf{(e) how they would identify programming plans in a new topic area they were unfamiliar with} to move towards automated plan identification in new domains}. 


% \subsection{Qualitative Analysis}
% \subsection{Data Analysis}
We used a transcription service to transcribe the video recordings and used Dedoose, a qualitative analysis software, for analyzing the transcripts with an inductive, reflexive coding process, influenced by thematic analysis approaches~\cite{clarke2021thematic}. During coding, we highlighted the connections between the beliefs of our participants and their choices in the pattern identification process. 
Note that while we use the term ``programming plan'' in this paper to clarify our focus on relatively small coding chunks rather than design patterns or architectural patterns, many of the educators we spoke with used ``plan'' and ``pattern'' interchangeably, or even preferred the term ``pattern.''
% Inspired by Causation Coding~\cite{Saldana_CodingManual_2021}, we coded the links between conditions or variables and outcomes.
%, we primarily used Causation Coding~\cite{Saldana_CodingManual_2021}. In this method, codes are designed to highlight the causal beliefs in qualitative data, and include a multi-part structure of outcome, mediating variable, and antecedent condition. 
% For example, one of our codes was ``plans mean different things to different people OR people have different notions of patterns > uncertainty''. 
%This code communicates that the interviewee suggested that people interpret plans differently because of the absence of a definition of a plan. This uncertainty was a part of the challenges they faced while performing plan identification in their study. 
%The phrase before the ``>'' conveys the causal belief or the mediating variable and the phrase after is the outcome.

%refers to inferring ``causal beliefs'' from qualitative data . In Vivo Coding refers to capturing verbatim from the qualitative data to ``honor the participant's voice.'' The preferred coding method was kept as Causation Coding to get to the root of the reasoning behind our participants’ actions to move towards design implications 
% In cases where causation wasn't clear in the transcript, or the participants' own words captured causal information, we applied In Vivo Coding~\cite{Saldana_CodingManual_2021}.
% The first author performed all coding, and both authors met 1-3 times a week over the course of a month to discuss themes among the data.

% \subsection{Reflexivity Statement}

% As our qualitative coding methods are reflexive, we share a statement about the ways in which our experiences may influence our interpretations of the interview data. Similar to our interviewees, we have a history in programming plans research. At the time of the interviews, the first author has performed research with programming plans for eight months, and the second author has performed research about programming plans for five years. The second author also performed a plan identification process as part of their past research, described in [anonymous]. We believe that this personal experience supported us to better understand the processes of the instructors we interviewed. At the same time, we came to this study with our own perspectives about the nature of programming plans, which are most strongly influenced by the work of Soloway and his colleagues~\cite{Soloway1984EmpiricalSO, Soloway_PlansMechanismsExplanations_CACM-1986, soloway_programcomprehensionreview_1988}. 

%% We come from an eliot soloway perspective, so we may be more likely to adpt his ideas. Considered all aspects of transcript. 



% Some participants believed...
% In contrast, others thought ....
% Expressed 
% Valued
% Indicated that
% Repeatedly described





\begin{table}[h]
\caption{Demographics of the Instructor-Scholars Interviewed in our Formative Study.}
    \centering
    \footnotesize
    \label{tab:formative-participants}
    \begin{tabular}{l|cccccc}
    \toprule
            & \shortstack{Years \\ in CSEd \\ Research} & \shortstack{Years in \\ Plans \\ Research} & \shortstack{Years \\ in CS \\ Instruction} & \shortstack{Teaches \\ CS1?} & \shortstack{Uses \\ Plans in \\ Instruction?}
            % & \thead{Industry \\ Experience?}
    \\\midrule
        P1 & 10-20 & 1-3 & 10-20 & Yes  & Yes \\
        P2 & 5-10 & 4-6 & 10-20 & Yes  & Yes \\ 
        P3 & 5-10 & 4-6 & 10-20 & Yes  & Yes \\ 
        P4 & 1-5 & 1-3 & 5-10 & No  & Yes \\ 
        P5 & 1-5 & 1-3 & 1-5 & Yes  & Yes \\ 
        P6 & 10-20 & 1-3 & 20+ & Yes  & Yes \\ 
        P7 & 5-10 & 4-6 & 20+ & No  & Yes \\ 
        P8 & 1-5 & 1-3 & 5-10 & Yes  & Yes \\ 
        P9 & 20+ & 20+ & 20+ & Yes  & Yes \\ 
        P10 & 20+ & 20+ & 20+ & Yes  & Yes \\ 
    %\\\bottomrule
    \end{tabular}%
\end{table}


\subsection{Findings}
\label{sec:challenges}

Here, we describe the three most salient challenges relevant to the identification of domain-specific programming plans by instructors. These include difficulties these computing educators experienced while identifying programming plans for use in instruction, as well as a potential challenge for instructors new to the plan identification process. As we describe the challenges, we also include information about the plan identification process and approaches educators currently use to address these challenges.

%Identifying the predicaments in the plan identification process will help us get closer to uncovering design implications. Thus, this section highlights the challenges that our participants faced while identifying patterns.



\subsubsection{The Challenge of Finding Plans in the Practice (C1)} 
\label{sec:challenges_practice}
Our interviews confirmed that state-of-the-art programming plan identification is an entirely manual process. Echoing the belief that \textit{``patterns are `mined' from the practice''} (P10), a large part of that process included reviewing existing code, problem statements, or instructional material. This included GitHub repositories (P7), programs written by industry professionals (sometimes including their own code) (P7, P8, P10), textbooks (P1, P2, P5), and other instructional material including lecture notes, assignments, student programs, and testing material (P1, P2, P4, P6, P9).

While our participants had ready access to these resources, they still found it onerous to translate this practice into plans. As P7 described, \textit{``the challenge was trying to infer general characteristics from a large collection of specific examples.''} Another participant described this part of the process as \textit{``tedious''}, stating: 

\begin{quote}
``There was a lot of just paging through textbooks, either physical or digital versions and just looking at code, just trying to see if there's something we haven't seen before.'' (P1)
\end{quote}

Understandably, determining which aspects of a given example are widely applicable enough to ``count'' as a programming plan is a difficult task, as the possibilities for writing programs are so extensive. To manage this challenge, plan identifiers leaned on a combination of their own expertise (\textit{``you know it when you see it''} said P7) and collaboration with others.
%Collaboration was highlighted to be helpful for the plan identification process. \textit{``The teamwork was very helpful,''}  said P9. 
This collaboration could involve discussions with co-instructors (P8, P9), TAs (P4, P8), study participants (P4), other researchers (P7, P9), and developers in the industry (P8). Recall that we interviewed authors of computing education publications, and it turns out that our interviewees frequently discussed plan identification with their co-authors (P1, P2, P3, P7, P9, P10) and even paper reviewers (P5).
 %One of the interviewees reasoned that individual perspectives alone cannot be considered representative of common practice because they thought that it is \textit{``very likely their experience is different from another person's experience.''} (P8) 
These discussions occurred across different stages of the plan identification process, from when candidate plans are first presented, as well as when they are refined into their final form or removed from consideration. 
% Below we summarize the types of discussions that occurred over time, 

\subsubsection{The Challenge of Refining Plans for Student Needs (C2)} 
\label{sec:challenges_abstraction}
Once initial plans are chosen, the challenge only begins. Instructors judged the quality of the plans they identified based on multiple metrics, which could conflict. 
Not surprisingly, the instructors we interviewed generally agreed that a good programming plan should be used frequently in practice, whether that practice was classroom assignments (P1, P4, P8) or professional programming (P5, P8). A good plan \textit{``needs to be useful in many other contexts and ideally have some degree of flexibility to be adapted to similar situations"} said P3. 

At the same time, instructors believed that a good programming plan should be readily usable by learners, spanning the gulf between a problem statement and a code solution. P1 described plans as an \textit{``important step between understanding the syntax of a language and understanding how to do problem solving.''} P9 agreed, saying \textit{``knowing the pattern helps getting to a good solution, an expert solution.''} In the same way that interface designers narrow the ``gulf of execution'' between a user's goal and how to achieve that goal~\cite{Norman_UserCenteredSystemDesign_1986}, instructors believed that programming plans should facilitate learners' ability to design and implement coding solutions. \textit{``Understanding how to tune it should not be the issue,''} said P2. P10 agreed, stating \textit{``The solution has to be easy to put into practice (with practice).''}
 
Balancing the goals of a plan being \textbf{common in practice} while also being easily \textbf{usable by novice learners} was a major challenge for instructors who identified programming plans. 
P2 cautioned that \textit{``If it is too specific, it's not recurrent enough.''} On the other hand, P2 believed it is essential to ensure the plan is not \textit{``too abstract''} so that students \textit{``have difficulties in grasping the idea and using [the plans].''} Instructors described the challenge of balancing these factors as finding the right level of \textit{``abstraction''} (P2, P10), \textit{``generality''} (P2), or \textit{``granularity''} (P3). Ideally, instructors wanted to refine plans in a way that ensured they could be applied across multiple contexts, while also providing sufficient concrete detail to support learners to implement the plans correctly. 

Compounding this difficulty was the need to choose that level of abstraction appropriately for the learner audience. Most participants thought it is essential to \textit{``teach novices and experts patterns very differently''} (P1). This is because \textit{``advanced students can handle more abstract explanations''} (P10). 
To address the challenge of refining a plan to the correct level of abstraction, instructors employed an iterative process, involving discussions with colleagues (\textit{``A group of us would meet once or twice a year for several days and kick ideas about.''} said P10.). 
% On an even longer time frame, they may also refine their plans after seeing how they work in the classroom (who?).

\subsubsection{Challenge of Ensuring Robust and Shareable Plan Definitions (C3)}
\label{sec:challenges_robust_shareable}
Our formative study revealed a fair bit of variety in the components of a programming plan that instructors looked for during the plan identification process. The instructors we interviewed mentioned one or more of the following components as crucial parts of a plan: (a) the \textit{name}, which is a short description of the plan (P1, P5, P6); (b) the \textit{goal}, which is a natural language phrase for what the plan achieves (P2, P3, P5, P6); (c) the \textit{solution}, which is how the plan is implemented, whether in code (P3), pseudocode (P1), or a description in natural language (P6); and (d) the \textit{changeable areas}, which show the places in the solution should be modified based on the specific context (P1, P6). These discrepancies present a challenge: plans identified by one instructor might not meet the needs or expectations of another. 

Our interviewees were highly knowledgeable about programming plans and plan-based pedagogies, but this is certainly not true of all computing instructors. Lack of knowledge about how to identify and use programming plans may be even more common among instructors who focus on application-specific programming areas and other topics beyond introductory programming, since the computing education research community as a whole tends to focus on introductory programming more than other programming topics~\cite{Noa_SIGCSE-focus-topics_2019}. Our sample of instructors was heavily biased towards CS1 instructors (see Figure~\ref{tab:formative-participants}), mirroring that trend. To meet the goal of supporting more instructors with more diverse content knowledge to design domain-specific programming plans, it may be necessary to constrain their experience and provide structure so they achieve similar outcomes to those of the plan identification experts we interviewed.

% Make the implicit explicit? Seems more like a solution than a challenge




% \subsection{Opportunities}
% \label{sec:opportunities}

% Here, we take inspiration from our interview findings to describe four opportunities for increasing the ability of instructors to identify domain-specific programming plans. We include details from the interviews where appropriate.

% \subsubsection{Opportunity to make exploration of the practice more efficient (O1)}

% Instructors found the process of reviewing resources

% \subsubsection{Opportunity to support plan refinement  (O2)}






% \subsubsection{Opportunity to support instructors with limited domain knowledge (O4)}

% The instructors we interviewed had significant expertise in both instruction as well as particular programming topics. The interview study made it clear that they relied on both types of knowledge as they identified and refined plans. However, if more


% Several participants suggested collaborating with an expert from the relevant domain to understand the \textit{``typical things [experts] need to do and how [those are] typically done} (P2). This discussion was necessary to \textit{``discover the intent behind code''} (P1), which is challenging to discover from looking at code alone. Some of our educators mentioned that using emerging AI tools like GitHub Co-Pilot (P6) or ChatGPT (P7) to generate solutions may also help with plan identification when plan identifiers lacked the relevant expertise.


% The key hurdle our participants anticipated in identifying plans from new domains was not finding common code, but understanding if that code implemented a meaningful goal. The task of identifying plans for educational purposes is multifaceted: it involves understanding \textit{``what goals [students] need to achieve in this particular context''} and \textit{``how much do they know at this point in their instruction''} (P3). While educators might understand what students are ready to learn, they cannot understand the important subgoals and tasks of a programming domain without the relevant expertise.



%``why they're doing what they're doing'' (P1), Need to ask a domain expert ``does this actually mean something to you?'' (P4)

% more quotes for this section

% P1
% somebody to sit down with you and walk you through, pick up a representative piece of software for doing that, explaining why they're doing what they're doing
% I think the intent is one of the things that's often hardest to discover in large pieces of code and that people may be much better at explaining what the code is doing rather than why it's doing. 
% And I think a virtue of pattern is that it should hopefully represent best practice or at least good practice.

% P2
% I'd find an expert and ask them what are typical things you need to do and show me how this is typically done. 

% P3
% Well, then I would first have to understand what are the constituents of this context, so what students need to perform and what goals they need to achieve in this particular context. 
% hen I would look into how much do they know at this point in their instruction and match those and say, "Oh, this is enough."
% What kind of actions do they have available? What are the rule sets? What are the goals they need to achieve?

% P5
% I don't know if there are textbooks on web scraping, but looking at some kind of rigorous sources on web scraping, seeing what do people do? How do they describe these things? Are there threads of commonality between different resources?

% P4
% And so that there's sort of the backwards and forwards being able to do that program analysis and then go to a domain expert and be like, does this actually mean something to you? And then going to the domain expert and be like, okay, what means something to you? What are the patterns that you're commonly using? And then can you show me? 

% Some of our educators mentioned that using emerging AI tools like GitHub Co-Pilot (P6) or ChatGPT (P7) to generate solutions could help in plan identification.

% P8
% I would probably look at what others who are working or doing stuff in this domain, how would they do things and how would they solve common problems? What are common problems, what are common solutions to that problems?
% And then they started explaining what features they like to have in there and it helped me a lot to see how they argue about things, because they started arguing about things on a very different level than I expect because they have different ways of thinking about things.

% % addressing the third challenge
%\edit{While conducting the study, we presented our interviewees with a hypothetical scenario in which they had to identify plans in a domain that they were not knowledgeable about. We asked them about how the process of plan identification would be similar or different in that scenario. While our interviewees did not perform plan identification in domains unfamiliar to them, they suggested that collaboration with experts from those particular domains could yield insights into the code's intent (Section 4.4.2). With these conversations, then these instructors would be able to use their educational expertise to create plans from those domains.}




% \subsection{Opportunities}

% Just spitballing in prep for editing the next section: 

% Instructors cannot be replaced. Their insight about students' abilities cannot be replicated by  technology. Instead, we should attempt to improve the current process with a human-in-the-loop process. 

% Any design solution will need to not only support the identification of common code patterns, but also meta information like goals, plan names, and changeable areas of plan solutions.

% The plan identification process is essentially a problem of (a) search (b) abstraction and (c) description. How can we support abstraction? We know that LLMs can generate content. However, they aren't particularly good at abstraction. But, just content generation is not enough. That content should be organized in a way that can support the action of abstraction and description. 























