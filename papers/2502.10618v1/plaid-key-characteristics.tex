
\subsection{Key Characteristics of PLAID}

PLAID supports the process of plan identification in data processing with Pandas, machine learning with Pytorch, web development using Django, and web scraping using BeautifulSoup. Advancing on the features outlined in the design artifact (Section~\ref{sec:design-artifact}) and the actions instructors performed and suggested in the design workshop, PLAID enables the following key interactions assisting instructors in refining candidates to design plans for their instruction. 


\subsubsection{Viewing Programs and Problems.}

Instructors valued the availability of code examples in condition A from our design workshop (Section~\ref{sec:design-workshop-findings}). We saw instructors scanning examples, selecting desired code pieces, and copying them over into their plan templates in all conditions in the study. Motivated to provide instructors with reference programs, PLAID provides the ``Programs (Organized by Use Case)'' tab equipping instructors with full-code programs organized in a list of short natural language descriptions of common use cases in their domain of expertise. Enabling the interactions executed by instructors in condition A, the ``Create Plan from Selection'' and ``Create Plan from Program'' buttons allow 
% < Highlight code in full code and code pane in tab1 and make a plan (D1)
% < Add a button to add full program as a plan too (D1)
users to select a part of any given code piece to create a plan \textbf{(D1)}. This interaction copies over the selected code and its respective use case into the solution and name fields, respectively. The ``Programs (Full Text)'' tab also enables viewing a list of complete code examples, allowing instructors to look at materials they would typically search for when designing plans.
% < Code explanation plugin for strange syntax (GPT) (D2)
While looking for new plans, occasionally, instructors came across syntax unfamiliar to them (e.g., due to the deprecation of old methods and the introduction of new ones like append and concat in Pandas). In this case, participants hesitated to use the suggested syntax in their plans because its functionality was unclear to them. PLAID supports a button named ``View Explanation'' where the user can select a method, function, or line of code that is unclear and click on it to understand its working \textbf{(D2)}. 
Participants also suggested being able to run code within the interface to examine the code behavior and thus mitigate the challenge of being faced with unfamiliar syntax. Thus, using PLAID, instructors are able to run complete programs to view their output \textbf{(D2)}.
% < Search in the use cases (and full progs) (D3)
Frequently, instructors relied on their expertise and experience to formulate ideas about goals for which they wanted to create plans. While interacting with condition C in the design workshop, interviewees suggested including a mechanism to search for specific keywords within code and  its natural language description. To facilitate the instructor-LLM collaboration, allowing users to find examples implementing their ideas, PLAID includes a search bar that helps users navigate the given use cases, complete programs, and effectively find specific examples they may be looking for \textbf{(D3)}.

\subsubsection{Creating Initial Candidates.}

Participants indicated difficulty mining plans from code examples (Section~\ref{sec:challenges_practice}). To alleviate these challenges, our pipeline clustered similar code examples together into plan-ful examples. The ``suggest plan'' button within PLAID presents the instructors with these plan-ful examples for refinement.
% < Keyword search/embedding filter for potential values (D1)
On clicking on any plan component fields, PLAID suggests potential values after searching its code corpus for similar examples using a keyword search \textbf{(D1)}.
% < Show use case button in solution (add highlighting) (D2)
In the design workshop, few instructors emphasized the importance of presenting worked and contextualized examples to students. To allow instructors to view suggested plans as part of a complete example, PLAID has the ‘go to a use case’ button that redirects the user to the tab with full code programs and highlights the plan as part of a complete example \textbf{(D2)}.

\subsubsection{Finalizing Plan Design.}

% From Section~\ref{sec:process_intro_plan_design}, instructors indicated drawing on their experience in the application-specific domain and instructional expertise to think about how to best solve a problem. 
In our design workshop, participants created copies of their plans to display alternative solutions to achieve the same goal, emphasizing that multiple possible solutions in code could accomplish the same goal.
% < Duplicating plans (D3)
To accelerate this process of teaching a variety of possible solutions, PLAID allows users to ``duplicate'' plans on the canvas and further edit them to present alternative solutions for the same plan \textbf{(D3)}.
% Highlight text from solution to change it to changeable areas (highlighting code itself) (D4)
In conditions A and B, instructors highlighted the changeable areas in the code itself. To allow participants to emphasize the changeable areas in code in PLAID, we implemented the ``add to changeable areas'' button. After selecting the changeable piece of code, clicking on this button highlights the text in a different color and adds it to the box of changeable areas to complete the templated plan design (\textbf{D4}).
% Grouping plans into categories (D4)
% < Multiple selection of the boxes (D4)
% < Naming groups of boxes (D4)
A handful of users postulated each plan as an example question that can be used on assessments. They intended to create multiple variants of the same question for students. They suggested that being able to visualize the different categories would be helpful. Using PLAID, users can select multiple patterns together, add them to a group, and name the group \textbf{(D4)}. 
