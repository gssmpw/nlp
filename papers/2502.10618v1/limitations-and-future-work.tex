
% future work
% how can instructors take what they create to make instructional content
% time
% participant

\section{Limitations and Future Work}

While our study establishes foundational steps for plan identification, a few limitations exist. Most importantly, we evaluated our system in sessions limited to 15 minutes. These sessions may not have comprehensively captured all the ways in which instructors would interact with the interface because instructor behavior could change as they become more familiar with the system. Indeed, some participants expressed that they could not explore the features enabled in the system due to time constraints and primarily used interactions that were explicitly demonstrated before the tasks. Capturing how instructors' perceptions and behaviors change over time as they interact with the system could be valuable.

In addition, our study design required the recruitment of a population with expertise in instruction and an application-focused domain. Consequently, the sample size of our participants was comparatively small ($N=12$). This sample size may not be enough to observe statistical significance in some of our tests, including the mixed effects model and the Wilcoxon signed rank tests. However, along with our qualitative observations, we find the results to still be informative and encouraging for future research. 

% We selected three diverse domains
% In particular, predict limitations for niche domains.
\edit{The quality of generated content is an important consideration for any system that utilizes large language models. An inherent limitation of LLMs in educational technology is the potential generation of inconsistent, inaccurate, or incorrect content that might be harmful to learners.  PLAID mitigates these problems with its instructor-in-the-loop design where LLM-generated content is reviewed and refined by instructors before being presented to students. By presenting many code examples and alternative suggestions for all components of a plan, PLAID encourages instructors to carefully compare content. Our user study also showed this behavior as instructors often edited parts of plans they created from examples. Moreover, our initial exploration in Section \ref{sec:llm-pipeline} showed the generated content to be syntactically accurate in most cases, suggesting that the generated content is meaningful and useful for instructors to review. The usefulness of PLAID may still be impacted by the quality of its LLM-generated content: poorer quality of the initial content may lead to lower efficiency in instructors' ability to generate final content. Future work could explore other automatic approaches to identify inaccurate content before it is presented to instructors to reduce the effort they would need to put in to refine the content to a greater extent.}

\edit{Future work should explore the generalizability of PLAID to new programming domains. In our study, we evaluated PLAID in three distinct programming domains demonstrating its usefulness beyond introductory programming. Even in complex domains like Django and PyTorch, we did not observe any statistically significant differences between instructors' performance while identifying plans. However, newer programming languages or domains may be underrepresented in the LLM training data, affecting the quality of the generated code. Examining how instructors use PLAID in these niche domains can provide important insights into the utility of PLAID and its design considerations for larger computing education research. Moreover, designing programming plans in complex domains like app development might benefit from more interactions, such as support to organize code in multiple files.}

% <Our system supports instructors in considering key concepts, but it does not create content that is immediately ready for the classroom. ...Design opportunities for student-facing tools>
An open question for future work is how to design novel systems that present programming plans to students and automate additional aspects of plan-based pedagogies.
% student-facing systems for students using programming plans. 
Prior research on plans has shown that explicit plan-based pedagogical instruction may motivate students and support better knowledge acquisition~\cite{Cunningham_PurposeFirstProgramming_CHI-2021}. 
%In our studies, we followed this convention and prompted instructors to identify plans that could be used for teaching common concepts in lectures.
While the plans instructors generated in PLAID may be appropriate for use in lectures, there are many more opportunities to use these identified plans to support instruction. Our instructors proposed many interesting applications, including \textit{plans as cheatsheets}, where each plan explains a common task that students usually struggle with; \textit{plans as question generators}, where a plan with changeable areas is treated as a multiple-choice question; and \textit{plans as example generators}, creating on-demand worked examples~\cite{Atkinson_WorkedExamples_2000}.
% including slides and workbooks, similar to having worked examples that illustrate concepts and best practices. 
%However, 
% Our instructors had versatile ideas for tools that could be facilitated by having a set of programming plans.
% maybe connected to case-based learning stuff?
% For example, some instructors defined 
% MCQ generation?
% Other instructors found our system useful for utilizing 
% if our system supports abstraction, why not have students benefit from it as well? 
% Many instructors proposed having a ``student view'' in the system to allow students to mimic this plan identification process to use \textit{plans as reflection questions}.
% These applications were not explored in prior research on plan-based pedagogies, suggesting that empowering instructors in identifying domain-specific programming plans can also produce new ways of interacting with programming plans for all educators. 

% Designing and evaluating programming plans in diverse contexts is important for supporting plan-focused pedagogies in more domains, and presents novel design opportunities for supporting students. % These novel approaches for presenting programming plans to students can also address some concerns instructors raised for adopting programming plans, such as a reluctance to re-design

% Future work can explore enabling student-facing views for instructors to complete the pedagogical design process.

% Future work can further develop interfaces that support instructors in creating assessments and in-class examples using plans.

% Future work can then examine the effectiveness of instructional content created using plans in a classroom setting.