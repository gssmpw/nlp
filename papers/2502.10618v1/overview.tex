\begin{figure*}[h]
    \centering
    \Description{Three main parts of the study explained in the section, with research questions at top, methods at middle, and results at bottom levels.}
    \includegraphics[width=0.92\textwidth]{img/overview.pdf}
    \caption{An overview of the three-step process to motivate, develop, and evaluate PLAID.}
    \label{fig:prelim-overview}
\end{figure*}

\section{Overview of Methodology}

% PLAID's design was motivated by instructor practices informed by our formative study, iteratively refined by observations from instructor behaviors from our design workshop, and evaluated 

We followed a three-stage design pipeline centering the needs of instructors throughout the  process
% to understand what challenges prevent plan-focused pedagogies from wider adoption and how to support instructors in using programming plans in application-focused domains
(see ~\cref{fig:prelim-overview}). First, to identify state-of-the-art techniques in plan identification and the key challenges educators face when using these strategies to design plan-focused pedagogies, we interviewed computing educators experienced in designing instructional material with programming plans.
We build a deep understanding of this previously undocumented process and uncover the key challenges for plan identification in \cref{sec:interview_results}.

Informed by these insights, we propose three key design characteristics that address these primary barriers. We evaluate these characteristics through a series of design workshops with instructors teaching application-specific domains of programming (e.g., data analysis with Pandas and web development with Django), validating their usefulness and revealing additional design considerations in \cref{sec:design-workshop}.
% Then, we prevent plan-focused pedagogies from reaching a larger audience. 
% We invited seven instructors teaching application-specific domains of programming (e.g. data analysis with Pandas, web development with Django) to try our design artifacts to observe additional design goals that might emerge in these domains. 
% We report on these workshops in .

Enacting the identified design goals, 
% Following the main challenges and these additional design goals, 
we designed PLAID, a system that supports instructors in identifying plans in application-focused domains. We evaluated PLAID in a user study with twelve instructors and teaching assistants with varying teaching experience, finding that participants faced lower cognitive demands and were more productive compared to the baseline condition. We describe PLAID in \cref{sec:system-design} and report our user study in \cref{sec:user-study}.