\section{Prompts Used in System Design}
\label{sec:appendix-pipeline}

% \subsection{Queries for Content Generation using LLMs}


% \subsection{Query ChatGPT for 100 use cases in a domain} \\
\subsection{Get 100 Use Cases in Each Domain} 
\label{sec:use_case_prompt}

% We used the following query to get 100 use cases in a given domain: 

\begin{quote}
    \tt ``Give me 100 use cases of $\{DOMAIN\_NAME\}$. A use case describes a task you can achieve with the given library. For example, for the math library in Python, calculating the area of a circle would be an appropriately specific use case, but doing calculations would be too general. List these use cases without any comments before or after. Basically, just give me a list of use cases and no other text. In addition, these use cases should be appropriate for instruction for novices.''
\end{quote}

% \subsection{Query ChatGPT for code for all use cases} \\
\subsection{Extract Code for all Generated Use Cases}
\label{sec:code_prompt}

% To get code for the generated use cases in the previous step, we queried ChatGPT with each use case in the following prompt:

\begin{quote}
    \tt ``Write code to use $\{DOMAIN\_NAME\}$ to achieve the task I give to you. Return the code block without any text before or after. Basically, just give me the code block and no other text. 
    
    Write code to do the following: $\{USE\_CASE\}$.''
\end{quote}

% \subsection{Query ChatGPT to comment each piece of code to yield subgoals} \\
\subsection{Annotate Subgoals in Full Programs}
\label{sec:subgoals_prompt}

% To annotate subgoals in each full program, we used the following query:

\begin{quote}
    \tt ``For this piece of code, instead of putting a comment for every line, could you combine comments to add subgoals? Subgoals should describe small chunks of code that achieve a task that can be explained in natural language, rather than describing the code on a single line. Each subgoal should be put as a comment before the code starts. 
    
    Here is the code: $\{FULL\_PROGRAM\}$.''
\end{quote}

% \subsection{Prompt ChatGPT to identify changeable areas for each code piece} \\ 
\subsection{Identify Changeable Areas in Code Snippets} 
\label{sec:ca_prompt}

% After creating code snippets by breaking code by subgoals generated in the last step, we prompted ChatGPT with each code snippet and to annotate the changeable areas using:

\begin{quote}
    \tt ``We define a domain-specific programming plan as a piece of code common in programs from a particular application area (e.g., web parsing) that achieves a specific goal. I am providing you with a piece of code. Based on this definition, can you highlight the changeable areas?  Changeable areas are the parts of the idiom that would change when it is used in different scenarios. Could you give me the exact block of code from the line that would change. Don't give me the whole line. Just give me the part of the line that would change. For example, if just the URL changes in a line, give me just the URL. Give me each of these in a code block. List these code blocks without any comments before or after. Basically, just give me a list of code blocks of code parts that would change and no other text. 
    
    Here is the code: $\{CODE\_SNIPPET\}$''
\end{quote}

\subsection{Render Names for Clusters of Code}
\label{sec:name_prompt}

% After generating the plan components in all code snippets, we clustered similar code pieces together to create candidate plans. We prompted ChatGPT to create names for the plans with the following query:

\begin{quote}
    \tt ``I am giving you a cluster with comments which are the goals of some pieces of code along with the code. Come up with a name for this cluster of plans. Programming plans are pieces of code common in programs from a particular application area that are used to achieve a given goal. A name is reflective of what that plan achieves. So produce a name that would help me understand what goal the code  will be achieving. To reiterate, be very specific, and consider what each subgoal is doing. Do not consider the context. Just consider what the code is doing. Return the result to me in the form of the following string, "Name: ". 
    
    Here is the cluster: $\{PROGRAMS\_IN\_CLUSTER\}$''
\end{quote}