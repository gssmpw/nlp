% \section{Interviews of Educational Researchers Who identified programming plans}

%\section{Interviewing Instructors to Unveil The Design Process For Programming Plans}
\section{Understanding the Programming Plan Design Process}
\label{sec:interview_results}
To understand the current process of programming plan identification by educators, we conducted semi-structured interviews with ten computing instructors (see Table~\ref{tab:participants}) who have curated lists of plans for instruction. Mirroring the approach of Fowler et al.~\cite{craig_max_methoda}, we recruited instructors who authored a computing education research publication where plan identification was a part of their methodology or results. Computing education researchers are an ideal source to learn instructional best practices because this population consists largely of instructor-scholars: those who not only teach computing but are also informed by research and recent practices in computing education. This allowed us to discuss their concrete programming plan identification experiences, rather than have participants speculate on the process in general.



% adding sentences to complete the overview

% complete the study overview -- its kinda incomplete
% why those participants and why these methods

%Our literature review showed that authors rarely provide detail about the process they undertake to identify programming plans. To more deeply understand the process of programming plan identification, we interviewed the authors of the publications from the literature review.


% \subsection{Participant selection and recruitment}
% \edit{\subsubsection{Participants}

\begin{table*}
\caption{Participant Demographics.}
    \centering
    \footnotesize
    \label{tab:participants}
    \begin{tabular}{l|ccccccc}
    \toprule
            & \thead{Years in CSEd \\ Research} & \thead{Years in Plans \\ Research} & \thead{Years in CS \\ Instruction} & \thead{Teaches \\ CS1?} & \thead{Uses Plans \\ in Instruction?} & \thead{Industry \\ Experience?}
    \\\midrule
        P1 & 10-20 & 1-3 & 10-20 & Yes  & Yes & Yes \\
        P2 & 5-10 & 4-6 & 10-20 & Yes  & Yes & No \\ 
        P3 & 5-10 & 4-6 & 10-20 & Yes  & Yes & Yes \\ 
        P4 & 1-5 & 1-3 & 5-10 & No  & Yes & Yes \\ 
        P5 & 1-5 & 1-3 & 1-5 & Yes  & Yes & No \\ 
        P6 & 10-20 & 1-3 & 20+ & Yes  & Yes & Yes \\ 
        P7 & 5-10 & 4-6 & 20+ & No  & Yes & Yes \\ 
        P8 & 1-5 & 1-3 & 5-10 & Yes  & Yes & Yes \\ 
        P9 & 20+ & 20+ & 20+ & Yes  & Yes & No \\ 
        P10 & 20+ & 20+ & 20+ & Yes  & Yes & Yes \\ 
    %\\\bottomrule
    \end{tabular}%
\end{table*}

% \edit{\subsubsection{Researcher–Participant Relationship}}
% we were avoiding this

% \subsection{Participant Recruitment}
% \subsubsection{Literature Review of plan identification techniques}

% \subsection{Participant Recruitment}
% To locate individuals who had identified programming plans for educational purposes, we reviewed the computing education and HCI literature for publications where the authors identified programming plans as a part of their methodology or results. We contacted the authors of these papers to request an interview. By targeting the authors of such publications, we were able to ask participants about a specific and concrete programming plan identification progress, rather than have participants speculate on the process in general.



%\edit{Specifically, we tried to identify computing educators who identified novel programming plans in their prior work. While several researchers have designed systems to identify common code sequences , most of such publications just focus on identifying code plans. In this paper, we try to unveil the process and challenges associated with the process of plan identification with a specific focus on their use in pedagogy to meet learner's goals. }


% We searched the ACM digital library within SIGCSE and SIGCHI sponsored venues, as well as the Computer Science Education journal for papers containing any of the following terms:

% \begin{quote}
%    "programming plan*" OR "program plan*" OR "program plan*" OR "programming template*" OR "pattern-oriented instruction" OR "plan-like" OR "unplan-like" 
% \end{quote}

% After eliminating posters and any non-publications, our search resulted in 61 results from SIGCSE sponsored venues, 55 results from SIGCHI sponsored venues, and 5 results from Computer Science Education. 
% The second author reviewed these publications to identify those where the authors specified that they had generated some number of programming plans, and where the programming plans in question were used to understand learners or to design for learners' needs. Papers where the authors explicitly adopted their programming plans from another source were not included. In cases where the second author was unsure about whether the paper met the inclusion criteria, both authors discussed to reach a final conclusion. 
% As the second author reviewed the papers, they used a ``backwards snowball'' process \cite{Claes_SnowballingTechniques_EASE-2014} where works cited by papers in the original search were also included if they met the inclusion criteria. 
% The final set of papers that met the inclusion criteria included 40 publications, dating from 1979-2023. 

% \subsubsection{\edit{Summary of the literature review}}
% \edit{Here we briefly summarize the work of the papers we identified in the literature review.}

% \subsubsection{Selection criteria}
% We contacted the authors of the papers identified in our literature review to request an interview. 
% We began with the first author and moved through the author list until we interviewed an author or all authors were emailed. 
%\edit{We contacted 3-4 educators every week until we reached the end of our duration that we set aside for conducting the interviews and until we reached the end of the shortlisted papers.  We interviewed all educators who showed willingness to answer questions except for any responses we received after the timeline for conducting interviews ended.} 
% We did not contact authors who are no longer active in CSEd research, either because (a) they hadn't published in the last 15 years, or (b) they were a student author who after graduation no longer worked in academia nor published.
%\edit{At the end of each interview, asked participants if they recommended any other potential interviewees.}



% describe why our population is good/why we chose our population

% \subsection{Study Participants}

% Researcher Description
% \edit{\subsubsection{Interviewers:} The first author conducted this study under the active supervision of the second author.  
% % Theyhave been working with the first author since early this year and have a thorough and extensive understanding of the research and its goals. 
% The second author has performed identification of novel programming plans in order to inform the design of curricula.}
% %\edit{In that work, programming plans are influenced by the propositions presented by Soloway.}

% Our ten participants hailed from North America, Europe, and Australasia, with diverse backgrounds that span not only instruction and research, but also industry (see Table~\ref{tab:participants}). Our interviewees included both those early in their teaching careers (1-5 years of instruction experience) as well as those highly senior in their field (20+ years of experience). Most, but not all, taught introductory programming (CS1). Most of our participants had been performing programming plans research for 1-6 years, while a couple had over 20 years of experience. All of our participants used programming plans in their instructional practice.


% better phrase and connect to RQs
% \subsection{Data Collection}
% \subsection{Interviews}
% Our interviews were performed over Zoom, and lasted thirty minutes to one hour. One interviewee requested written communication rather than a Zoom interview. 

% Re-write based on the style of the other paper to include more detail
% \subsection{Study Setup}
We conducted online semi-structured interviews that ranged between 30 minutes and one hour. 
%We noted the extent of their experience teaching and researching with programming plans, as well as their understanding of what a programming plan was. 
%For example, we explicitly asked the interviewees questions related to the time they have spent teaching computer science and researching in computer science education with a particular focus on programming plans research. 
%We also asked questions about their understanding of programming plans in general.
We asked participants to describe the specific activities they undertook to identify programming plans in the paper they authored. This included questions about what they were looking for when they identified programming plans, the resources that were involved in this process, and the procedures they undertook.
We also posed questions to gain insights into the difficulties the interviewees faced when they were identifying programming plans.
%and their ideas of solutions to solve these challenges to identify potential areas where technology can help.}
Lastly, we prompted a discussion about how they would attempt to identify programming plans in a new topic area they were unfamiliar with.
%to move towards automated plan identification in new domains.}

%\edit{During this process, after every section of the questions, we asked follow-up questions for clarifications if necessary.}


% During the interviews, we asked questions about \edit{\textbf{(a) participants' background} to understand the extent of their experience researching with programming plans and its use in instruction, \textbf{(b) their understanding of programming plans generally} to ensure consistency with the definition of a programming plan, \textbf{(c) the specific activities they undertook to identify programming plans in their paper} to deeply understand their process of plan identification, \textbf{(d) challenges they encountered} to identify areas where technology can help, and \textbf{(e) how they would identify programming plans in a new topic area they were unfamiliar with} to move towards automated plan identification in new domains}. 


% \subsection{Qualitative Analysis}
% \subsection{Data Analysis}
We used a transcription service to transcribe the video recordings and used Dedoose, a qualitative analysis software, for coding the transcripts with an inductive, reflexive thematic analysis process \cite{clarke2021thematic}. During coding, we highlighted the connections between the beliefs of our participants and their choices in the pattern identification process. 
Note that while we use the term ``programming plan'' in this paper to clarify our focus on relatively small coding chunks rather than design patterns or architectural patterns, many of the educators we spoke with used ``plan'' and ``pattern'' interchangeably, or even preferred the term ``pattern.''
% Inspired by Causation Coding~\cite{Saldana_CodingManual_2021}, we coded the links between conditions or variables and outcomes.
%, we primarily used Causation Coding~\cite{Saldana_CodingManual_2021}. In this method, codes are designed to highlight the causal beliefs in qualitative data, and include a multi-part structure of outcome, mediating variable, and antecedent condition. 
% For example, one of our codes was ``plans mean different things to different people OR people have different notions of patterns > uncertainty''. 
%This code communicates that the interviewee suggested that people interpret plans differently because of the absence of a definition of a plan. This uncertainty was a part of the challenges they faced while performing plan identification in their study. 
%The phrase before the ``>'' conveys the causal belief or the mediating variable and the phrase after is the outcome.

%refers to inferring ``causal beliefs'' from qualitative data . In Vivo Coding refers to capturing verbatim from the qualitative data to ``honor the participant's voice.'' The preferred coding method was kept as Causation Coding to get to the root of the reasoning behind our participants’ actions to move towards design implications 
% In cases where causation wasn't clear in the transcript, or the participants' own words captured causal information, we applied In Vivo Coding~\cite{Saldana_CodingManual_2021}.
% The first author performed all coding, and both authors met 1-3 times a week over the course of a month to discuss themes among the data.

% \subsection{Reflexivity Statement}

% As our qualitative coding methods are reflexive, we share a statement about the ways in which our experiences may influence our interpretations of the interview data. Similar to our interviewees, we have a history in programming plans research. At the time of the interviews, the first author has performed research with programming plans for eight months, and the second author has performed research about programming plans for five years. The second author also performed a plan identification process as part of their past research, described in [anonymous]. We believe that this personal experience supported us to better understand the processes of the instructors we interviewed. At the same time, we came to this study with our own perspectives about the nature of programming plans, which are most strongly influenced by the work of Soloway and his colleagues~\cite{Soloway1984EmpiricalSO, Soloway_PlansMechanismsExplanations_CACM-1986, soloway_programcomprehensionreview_1988}. 

%% We come from an eliot soloway perspective, so we may be more likely to adpt his ideas. Considered all aspects of transcript. 

% \definecolor{editCol}{RGB}{255,140,0}
% \newcommand{\edit}[1]{{\textcolor{editCol}{#1}}}
%\subsection{Findings from Instructor Interviews}


% Some participants believed...
% In contrast, others thought ....
% Expressed 
% Valued
% Indicated that
% Repeatedly described

%The educators we interviewed described a variety of details about the approaches they utilized and the objectives they hoped to reach during the plan identification process. They also shared challenges and suggestions for potential improvements.



\subsection{Components of Programming Plans} 
\label{sec:components}

\begin{figure}
\centering
\includegraphics[width=0.3\textwidth]{img/new_plan_comp4.pdf}
 \caption{Components of a programming plan mentioned by our interviewees, illustrated with a common introductory plan.}
\label{fig:plan-components}
  % \Description{This diagram shows the different components of a plan. It has the name of the components along with an example plan with all these components. It has three gray boxes on the top that each depict a part of the plan, including the name of a plan, 
  % this case, Search. The second gray box has 
  % the goal that the plan accomplishes. 
  % For our example, this is to check if the list contains a specific value. The last gray box has 
  % and the rationale behind the plan.
  % % ; in this case, to use the plan to find only one value in the list. 
  % Then, we have three boxes underneath. In our text, we mention that a solution can be presented in the form of a natural language description, pseudo-code or code. Each of these bottom boxes depicts one of these possibilities. The first box has the solution to our example plan in English, the second box to the right has the pseudo code solution, and the last box has the Python solution to our example with the changeable areas highlighted in green.}
\end{figure}

%Our participants had a range of ideas about what a programming plan meant to them.  
% Participants differed in the way they defined a programming plan, as well as which aspects of a programming plan they believed were important to identify.
Our participants looked for a variety of items when they identified programming plans
%In this section, we discuss all plan components mentioned by two or more of the instructors we interviewed 
(see Figure~\ref{fig:plan-components} for a summary). %We include only those components that were mentioned by more than two interviewees.
% More or less specificity about how plans are implemented

\subsubsection{Name}
\label{sec:name}
%For some instructors, programming plans must have a name. 
As P5 put it, \textit{``[plans are] useful common problems that people name.''} The plan name is often reflective of what the plan accomplishes. P6 believed that plans could have \textit{``generic names''} or \textit{``problem-specific names''}, but that the problem-specific names better supported students during problem-solving: \textit{``it was also important for [students] to reflect, `Okay, if I need a counter [plan], what exactly am I counting here?' ''}
The name connects the plan to the types of problems it can solve.

\subsubsection{Goal} 
\label{sec:goal}
%Some of our participants indicated that plans should have goals. This was
A goal, written in natural language, was
a key part of the plan for instructors like P3, who believed \textit{``a plan has to accomplish a specific goal. That was already the starting point.''} 
%However, educators had different perspectives on what a goal meant to them. For some, the goal of the plan is as simple as \textit{``to produce the output that is the solution to the programming problem.''} (P4) %% Commented because KC doesn't understand
For some, a goal emphasizes the deep, underlying structure of certain problem types, even if implementation details differ in practice. This led one instructor to solely focus on goals during plan identification, because
%\begin{quote}
\textit{``according to the programming language you have, the plan can vary. There might be big differences [in syntax], but the goal is the same.''} (P2)
%\end{quote}
%i'm not sure if heart conveys what i'm trying to say, none of the participants actually mentioned this explicitly, for example, we have a stock analyst and they have a list of stocks and they need to find the greatest selling stock, now 'the heart' of the problem in this case would be calculating the max.
% the underlying something?

% subgoal of a problem statement
% underlying structure of the problem
% deep, underlying structure of certain problem types
% Shared outcome of this type of problem
% \subsubsection{Rationale}
% \label{sec:rationale}
% For one of our participants, a programming plan came with a description of the reasons why the plan works and also where it might not. They reasoned that students \textit{``need to look for the reasons behind practice, not just the solutions. This brings insight.''}  (P10)  %\textit{``What are main mistakes that could happen with this one?''} (10)

\subsubsection{Solution}
\label{sec:solution}
For nearly all our participants, a programming plan included some type of solution to the problem it typically solved.
These solutions could be in a variety of forms. Two of our participants thought that implementation in a specific programming language is a key part of a plan. For example, P3 said, \textit{``The code cannot be detached from the way the plan is built.''} P1 agreed, stating \textit{``In order to code in that language in the best way, it's important to understand some of those language-specific patterns.''} 

In contrast, some participants used pseudocode or natural language to describe solutions, in the interest of communicating ideas that would potentially cross languages. In addition to language-specific plans, P1 argued that \textit{``you should be able to explain them abstractly, independent of the syntax of the language''} 
%like, what are you trying to accomplish?''} 
and thus \textit{``in pseudocode you should be able to explain what's going on and talk about what it does and why it gets used.''} However, for P6, writing plans in pseudocode got in the way, and they re-wrote plans in natural language: \textit{``It's really not about the notation...so, for the patterns I went really for just plain English.''}

\subsubsection{Changeable Areas}
% \textbf{Changeable Areas:}
\label{sec:changeable}
Two of our participants used a template-based approach where specific changeable areas in the plan are highlighted. P1 described their approach: \textit{``we use pseudocode, so we with dot, dot, dot on all the [changeable] parts just to highlight, here's the pattern part''}. Similarly, P6 believed that it is crucial to have \textit{``clear instantiation points''} in the solution for the learners while designing plans. %They also felt that the language they used in the plan showed learners where they needed to input specific information:   \textit{``for the English "if", the variable satisfies the conditions, so you know you need to put there a condition.''}. %% Tricky to describe this, commenting out for now

%\textbf{Examples:} One of our participants uses example situations to illustrate the use of the pattern. They do not touch on this aspect in the interview, however, our review of their papers helped us identify this characteristic. 

% \subsubsection{Changeable Areas}


\subsection{Characteristics Used to Judge a Good Programming Plan}
\label{sec:judging}

While different educators had different ideas about what they value in a ``good plan'', we found four themes: commonality, usability, right level of abstraction, and appropriateness for learning goals.
%While some focused on using the patterns effectively giving rise to the ``commonality'' and ``usability'' metrics, others concentrated on pattern design i.e., ``finding the right level of abstraction.'' Thus, we corroborated the following themes about what makes a pattern ``good.''


\subsubsection{Plans Should Be Common}
\label{sec:commonality}
%A spectrum of responses from our interviewees gave us insight into whether it is necessary for a plan to be commonly used in well-written programs.
All but one of the educators we spoke with considered the frequency with which a programming plan is used in practice to be a measure of its worth. As P10 said, \textit{``For it to be a  ``pattern'' it has to be seen in the world. [Plan identification] isn't about creativity, but about exploration and recognition.''}
%These respondents believed that the definition of a programming plan made it \textit{``inevitable''} (P4) that a plan should be frequently employed in standard practice. 
%While our interviewees were generally in agreement that it was important for plans to be used recurrently, 
%However, there were differences in how they judged commonality and how much value they gave to this measure.
% Seeing the programing plan in practice

% Put somehere! Patterns can be based on techniques used in common practice or can be representative of the best practice, two of which may or may not coincide.




%This idea seems trivial for plan identification in an introductory programming context \textit{``Since elementary patterns aren't very  "deep" they are seen in use everywhere.''} (P10) without conscious effort. 

%% This is a great quote but I think it's repetitive
%Yet, sometimes it is also just for patterns to be representative of general implementation approaches used in practice as demonstrated by one of our interviewees, \textit{``But, patterns aren't about "creation" or ["]new ideas". They are about established practice and why it works.''} (P10). 

%P6 suggested a concrete strategy for understanding whether something is common enough to be a programming plan: \textit{``you need to look for at least like three instances in real code to make sure that this is a solution needing a pattern.''}.

% There are different types of common.

% Common to instructional practice
% Common to the world of programmers
% ((((Common enough to solve multiple problems))) Not a real theme??

% But, they had different ideas about the type of practice that was important (instructional practice, and industry practice)

%% Commonality is important, but it's nuanced. There are different types of commonality.

While the instructors we interviewed generally agreed that a good plan should be used frequently in practice, they had different ideas about the \textit{type} of practice that should be used for this measurement. %(instructional practice, and industry practice)
%Throughout the spectrum, we had responses suggesting that commonality is a metric that should be considered during pattern identification, but is not the primary or the deciding factor. 
% One of our interviewees urged caution in what types of programs were used to judge commonality:
% \begin{quote}
%      It's not something where you necessarily want to have a popular vote on things. Just because more people eat at McDonald's than a fancy restaurant doesn't mean that McDonald's is better food. (P1)
% \end{quote}
P1 urged caution in only using general practice to judge commonality:
\textit{``It's not something where you necessarily want to have a popular vote on things. Just because more people eat at McDonald's than a fancy restaurant doesn't mean that McDonald's is better food.''} 

% add the part where the problem itself is usable

Instructors found commonality in instructional resources to be a valid measure of commonality. P1 described two types of commonality, one that draws only from the classroom, and one that draws from professional and other programming practice: \textit{``One is, do we see it a lot in the textbooks and or assignments that we looked at? There's another kind of common, which is when I think about all the programming that happens in the world, is it important in that context also?''} 
%P1 \textit{``And so I think we included things that were common in at least one of those domains.''.} %% Move to process possibly
P6 chose plans based on their frequency in class assignments: \textit{``I was not going to go for patterns if I can't immediately think of at least three or four exercises or questions for the assignments that would need this pattern.''}

%%%% Talk about "usefulness"
% Able to be applied to many contexts? Able to solve many types of problems.
% It must have a clear goal, and I would say that it needs to be useful in many other contexts and ideally have some degree of flexibility to be adapted to similar situations, but it's not always the same, accomplishes a certain goal, but it's useful in many different problems, but flexible enough to be adapted in those different contexts. -P3

% ``frequency of use had merit, but wasn't the overriding concern.'' (p10)
%Only one of our interviewees was not aligned on the value of commonality. They believed that commonality was not crucial, as long as the plan was representative of how a learner thinks (P3). %% Cut for space

% In light of this, our participants also brought to the fore a few resources to validate if plans are common. These include looking at course materials and evaluating the recurrence of patterns, discussing with other people including students, TAs, colleagues, and, using self-introspection to intuit commonality. %% This is more of a process

\subsubsection{Plans Should Be Usable}
\label{sec:usability}
%%% Missing % change as a measure of usability...
Most of our interviewees believed a critical quality of a programming plan is that it can be easily applied to a variety of situations that share similar goals. 
%As P6 described, learners should \textit{``....Easily know how to apply the pattern.''}
P10 said, \textit{``The solution has to be easy to put into practice (with practice).''}
This requires that the plan be adaptable in a way that the learner understands. % while neither needing to be changed entirely nor not needing to be changed at all. 
%its flexibility. To elaborate, it is pivotal for a pattern to be adaptable 
\textit{``Understanding how to tune it should not be the issue''} (P2).

%
% \begin{quote}
%     %It [the pattern] can't be ready to be used for a situation. 
%     You need to adapt [the plan], but you need not to restructure it completely to be used. It should be the right idea to start from. Then you have to fix it for your purpose, to tune it...
%     %, but not just to understand ... 
%     Understanding how to tune it should not be the issue. %(P2) %% Maybe just emphasize this part
% \end{quote}

%The importance instructors placed on usability stemmed from their reasons for teaching with programming plans in the first place. They believed that programming learners often lack the ability to interpret a problem statement and to adapt a similar solution without guidance. As P7 said, \textit{``The bridge between there [the problem] and code is too large.''} %P6 agreed: \textit{``students, of course, always struggled to go from the English description to the code, and so I thought, "Okay, what could we put in between to help them?"} (P6)

A good programming plan supports students to span the gulf between a problem statement and a code solution. P1 described plans as an \textit{``important step between understanding the syntax of a language and understanding how to do problem solving.''} P9 agreed, saying \textit{``knowing the pattern helps getting to a good solution, an expert solution.''} In the same way that interface designers narrow the ``gulf of execution'' between a user's goal and how to achieve that goal~\cite{Norman_UserCenteredSystemDesign_1986}, instructors believed that programming plans should facilitate learners' ability to design and implement solutions.

% Participants motivated the correct level of abstraction by what would help students solve problems. 

%P6 \textit{``I can assure the student has a conceptual understanding before they dive into the weeds''}

%This points us in the direction of finding a bridge between problem solving techniques and their syntactic implementation.
%, which also lays the groundwork behind discussions about whether a pattern should be syntax dependent or language agnostic. 


%A vast majority of our participants believe that the purpose behind a pattern is to push towards conceptual understanding
% another quote if needed
% % So I don't think there's anything magic about patterns. I think it's just, any time that I can assure the student has a conceptual understanding before they dive into the weeds, and then patterns are one way to do that, I guess. 
% and providing \textit{``scaffolding for thinking''} (P6) before diving into syntax implementation. 
% \begin{quote}
%     Before you write a single line of code, let's step back and understand, what's the problem we're trying to solve? And what aspects of that problem are new for you, that you haven't done before? (P7)
% \end{quote}

%% 
% P7 furthered their belief by explaining their process of helping the students learn about the problem conceptually and then start writing code. They thought that they will first present \textit{``examples of the pattern in action''} so that the students don't have to focus on the \textit{``syntax because the scaffolding is there.''} Then, they will \textit{``start pulling the scaffolding away''} so that the students \textit{``begin to learn the syntax.''}

%A usable plan should be easily adaptable by the learner. It should support the learner in translating a problem statement into code.

\subsubsection{Plans Should Have the Right Level of Abstraction} 
\label{sec:abstraction}

% \begin{figure}
% \centering
% \includegraphics[width=0.5\textwidth]{img/one_col_abstraction_plan.png}
% \caption{An example of more or less abstract versions of the same programming plan.}
% \label{fig:abstraction}
%   % \Description{Three code snippets are shown, with increasingly complex portions of the code snippets marked as changeable areas.}
% \end{figure}

% \begin{figure}
% \centering
% \includegraphics[width=0.5\textwidth]{img/new-balance.png}
% \caption{Relationships between the characteristics educators value when judging quality of programming plans for instruction.}
% \label{fig:metrics}
%   % \Description{A circle in the center for representing the abstraction level. A circle on the top-left for representing commonality. An arrow from the abstraction level to commonality with a plus sign to say that as the abstraction level of a plan increases, it becomes more common and hence establish a positive relationship between these metrics. A circle at the bottom-left expressing usability. An arrow from the abstraction level circle to usability with a negative sign indicating that as abstraction level increases, the plans become less usable. A two-way arrow with a plus sign between commonality and usability to indicate that as usability increases, the plans become more common and vice-versa. Another circle on the center-right for determining the appropriateness of the plan for beginners. An arrow from abstraction level to appropriateness for beginners with a negative sign to say that as plans become more abstract, plans become less appropriate for beginners.}
% \end{figure}

 %Finding just the right level was key to making the plan useful for students.

%Almost all of our participants thought that it is necessary to find the right level of abstraction for designing patterns. 
%Finding the right balance between concrete and general was key for most of our interviewees when identifying programming plans. 
Pinpointing the right level of abstraction or \textit{``granularity''} (P3) in a plan should ensure that the plan can be applied across multiple contexts, while also providing sufficient concrete detail to support learners to implement the plan in practice. 
% Figure~\ref{fig:abstraction} shows how changing the parts of a plan can lead to more or less abstraction.

% \begin{quote}
%     If you list too many [plans], you lose the idea of the schema, the schemata to it. If it is too abstract, you cannot use it in practice. Students have difficulties in grasping the idea and using them. (P2)
% \end{quote} 

% This is about experience level.
% Furthermore, P6 underlined,
% \begin{quote}
%     ...  
%     %  experiencing in program, that's
%     when we are more experienced in programming, that's also the way we think. ... we are looking at the problem already at the more abstract level." So it was also to convey and to make explicit the kinds of things that probably go on in our minds, but often they are not verbalized or declared. (P6)
% \end{quote}

% What does abstraction mean in terms of programming plans? Instructors defined this concept in terms of how much change students had to implent to put a plan into action.

% Not unchanged, but not totally changable. 
P2 cautioned that plans must be applicable in multiple contexts, saying \textit{``If it is too specific, it's not recurrent enough.''} and students might see the plan \textit{``as a recipe expected to be ready to be used.''} On the other hand, P2 believed it is essential to ensure the plan is not \textit{``too abstract''} so that students \textit{``have difficulties in grasping the idea and using [the plans].''}

% This could be because the instructors want learners to gain expertise, which they see as having genreral schemas. More expertise == more abstract schemas! \cite{}

The educators we interviewed believed that the right level of abstraction was impacted by the intended audience for the programming plan. 
%One of the other researchers also highlighted the importance of \textit{``teaching by diminishing deception''} (P1). They 
Most participants thought it is essential to \textit{``teach novices and experts patterns very differently''} (P1). This is because \textit{``advanced students can handle more abstract explanations''} (P10). 

%P1 believed that as students gather more expertise, it could be useful to then introduce \textit{``more advanced defined patterns''} and emphasize \textit{``the wrong ways to do things and why this solution is better''}. By contrast, a good plan for beginners is relatively \textit{``concrete''}, and focuses on \textit{``how you solve this problem''}. P3 shared that \textit{``it doesn't make sense to talk about plans in general''} since for \textit{``high school students, even declaring a variable is a plan and later they move on to complex plans''}.

%This goal and presentation disparity in patterns for advanced students and novices may or may not overlap with each other. For example, P3 said that \textit{``with high school students''} or novices, they \textit{``don't show them the inner parts''} and focus on the \textit{``goals''} over \textit{``details''}. They start introducing these goals in \textit{``undergraduate courses''}. They also highlighted that 


% Some participants described good plans for beginners as different than plans that were best for experts. These researchers believed that a good plan for novices is relatively ``concrete'' (PX). PX shared, 
% \textit{``Yeah, again, so the abstract thing is talking about... A lot of times patterns are in the design patterns book for example, they're introduced using UML diagrams, whereas }


% PX shared, ``we use pseudocode and actual pieces of code...it is just a much more concrete representation versus something like UML %end here?

% , which is...
% %just like, we blow these kids away and it's [UML is] 
% useful for more sophisticated things than we're introducing.'' 
% 3

% PX believed that as students gather more expertise, it could be useful to then introduce the ``properties of the solution.''

% And then we don't sort of say... I think in more advanced defined patterns, you might talk about thing the wrong ways to do things and why this solution is better, sort of the properties of the solution. Whereas we would just say, "This is how you solve this problem.''




% difference in goal
% difference in content
% difference in presentation influenced by prior knowledge
% i think that's all the kinds of differences i saw

% Following these claims, it becomes necessary to understand whether the design of patterns should be different for learners with different levels of expertise. This raises the question if patterns different 
% % trying to write something like 'in terms of content' -- like idk maybe for loops for summing for beginners and connecting to an api for more advanced students
% % unsure of how to phrase it correctly
% % in terms of content?? -- contentually
% conceptually and content-ually for people with different levels of expertise or is the difference only at the presentation level. 





 

% \subsubsection{Plans Should Meet Learning Goals}
\subsubsection{Plans Should Be Appropriate for Learning Goals}% of Students}
\label{sec:learning-goals}
%% The content varies by novice vs expert, and the level of abstraction varies between novice vs expert.

One of our interviewees highlighted that from a \textit{``teaching perspective''} (P8), a quality plan should be appropriate to the learners' goals. They elaborated, \textit{``if you're in the beginner level, you definitely want to understand what is a programming language, how do I do stuff?''} For those learners, better plans illustrated the operation of language features, to \textit{``help [them] build an idea of [their] notional machine''}~\cite{sorva_notional_2013}. However, for more advanced students, \textit{``you probably want to prepare for industry,''} so ideal plans would be representative of the \textit{``most recent programming paradigms''} (P8). 

%P5 exemplified that \textit{``there are patterns for system architecture and if we're lecture one of CS101 and said, "All right here are some system architecture patterns," I think that would maybe not help too much.''} %% Not my fave quote KC

%Most of our participants believed that due to the differences in prior knowledge and familiarity with different types of content, the \textit{``content that they [learners] would learn would be different for more novice versus more advanced learners''} (P8).

% \begin{quote}
%     So you would "probably, a program plan that is more based on those paradigms would be much more useful for me. Still, the other one could be good from a teaching perspective because it tells me so many things about how program language works, how the computer works, it helps me build an idea of my notional machine.(P8)
% \end{quote}

%if you're in the beginner level, you definitely want to understand what is a programming language, how do I do stuff? And later, you probably want to prepare for industry, so you want to definitely know about those things so this is definitely a difference in goal. (P8)

% difference in presentation



% For example, sometimes with the high school students, I just say like, "Yep, the goal is to be able to repeat something." They don't need to know the details. The plan that I'm presenting to them is a repetition of X steps. I don't show them the inner parts, the inner sub plans that are there. That's something that I usually actually do on the undergraduate courses. 

% It doesn't make sense to talk about plans in general. For my high school students, beginners, plans for them, declaring of creating a variable, it's a plan for them. Maybe the beginning, later they move on to more complex plans and so on. (P3)


%% include in final?
% Thus, for these participants, an ideal plan is independent of programming languages and helps a student focus on the structure of the solution before getting into its mechanics. 
% Some other reasons supporting this argument include flexibility and freedom of using language independent plans and the need for being able to solve problems across languages using the same patterns. 

% However, two (P1, P3) of our participants believed that a pattern should in fact be language dependent as highlighted in the previous section.




\subsection{The Process of Plan Identification}
\label{sec:process_intro_plan_design}
Our interviewees took diverse approaches to identify programming plans. While some educators focused on techniques that helped them gather insight from common practice,
%(such as ``looking at sample programs'', ``having discussions'',  or``doing a literature review'') 
others relied on their personal expertise. Most of the educators we interviewed leveraged a combination of both these strategies. 

% The following subsections detail each activity educators undertook to identify programming plans.

\subsubsection{Viewing Programs and Problems}
\label{sec:viewing-programs}
Echoing the belief that \textit{``patterns are "mined" from the practice''} (P10), many educators looked at some sort of code, problem statement, or instructional material during their plan identification process. More specifically, they looked at GitHub repositories (P7), programs written by industry professionals which may or may not include their own code (P7, P8, P10), textbooks (P1, P2, P5), and other instructional material including lecture notes, assignments, student programs, and testing material (P1, P2, P4, P6, P9).

\begin{quote}
``There was a lot of just paging through textbooks, either physical or digital versions and just looking at code, just trying to see if there's something we haven't seen before.'' (P1)
\end{quote}

% During this approach, we could say that the patterns that are thus created are largely influenced by others' designs including \textit{``language designers, pattern related content publishers''} (P5), instructors, and so forth. 
% One of our interviewees emphasized this idea by claiming that,
% \begin{quote}
%     Patterns are discovered in practice, not invented. Patterns are "mined" from the practice. (P10)
% \end{quote}

As the educators reviewed this content, their own expertise and experience played a role in finding the programming plans. \textit{``You know it when you see it,''}  said P7. P10 highlighted that gathering plans from common practice is not independent of the identifier's own perspective: \textit{``there's a personal selection path that always happens.''} 
%During the review of the sample programs, problems, and instructional content, the educators used their past experiences as instructors, programmers, or both. Some researchers stressed on this fact by claiming that  Another approach is to identify recurrent pieces of code and then use personal expertise to determine if they are worthy of being "patterns." 

% In addition, another one of our participants highlighted that even when the researchers use intuition to come up with patterns, it is another instance of identifying patterns based on others' designs since their experience is also influenced by what is common practice. 
% \begin{quote}
%     ...we're already impacted, affected by what we've seen. (P2)
% \end{quote}  



\subsubsection{Creating Initial Candidates}
\label{sec:process-candidates}
% \subsubsection{Discussions}
% \label{sec:discussions}

We observed that most of our interviewees followed an iterative approach, starting with initial candidate plans and refining them through various approaches. 

Collaboration was highlighted to be helpful for the plan identification process. \textit{``The teamwork was very helpful,''}  said P9. Collaboration could involve discussions with co-authors (P1, P2, P3, P7, P9, P10), co-instructors (P8, P9), TAs (P4, P8), paper reviewers (P5), study participants (P4), other researchers (P7, P9), and developers in the industry (P8). 
 %One of the interviewees reasoned that individual perspectives alone cannot be considered representative of common practice because they thought that it is \textit{``very likely their experience is different from another person's experience.''} (P8) 
% the following line can maybe be shortened and summarized
%They added that discussions with \textit{``people from different levels, different cultures, different student beginners levels, advanced student, people in industry, people in different countries. People who maybe came from a totally different background, they didn't want to learn programming to be a programmer, but maybe to just analyze data, became data scientists.''} (P8) to consider different perspectives is crucial in the pattern identification process.
These discussions occurred across different stages of the plan identification process including from when candidate plans are first presented, to when they are refined into their final form or removed from consideration. 
% Below we summarize the types of discussions that occurred over time, 


% \textbf{Identification Discussions}: 
While identifying patterns, P9 highlighted that they talked about most of the components of the plan including \textit{``what to name a pattern''}, 
%\textit{``choosing problems''}, 
\textit{``how to write this pseudocode so it can be generalized to different languages''}, and the difference between language constructs and plans, i.e., \textit{``maybe this is not a plan, it's just an idea how to do it right.''}
P1 highlighted that they deliberated about which plans to include in their final inventory: \textit{``we just talked about them, be like, ``Yeah, that's really kind of arcane. Maybe that's not a general purpose pattern.''} or {\textit{``Oh yeah, that makes a lot of sense.''}} 

% P3 endorsed this assertion by detailing the role of discussions in their plan identification process by saying, \textit{``Yes, a lot with [co-author] and we discussed it a lot, particularly the level of granularity that we should aim for. We didn't know if we should add the very small stuff, like memory, location and so on.''}

% \textbf{Refinement Discussions}: 
% Some of the researchers created an inventory of patterns and then used discussions to refine their patterns. For example, one of our participants created a set of candidate plans based on their personal preferences and then conducted discussions with other contributors to deliberate on the plan properties and whether the plan was good enough to include in their final list. 

% \begin{quote}
%     This was done, I should say iteratively. We're starting with something, we discussed, we found something that could be reasonable, and we start analyzing some real materials, real instructors material and check if something was left out, wasn't covered or wasn't recurrent enough. (P2)
% \end{quote}

% the following para can maybe be put across some other way
% One of our participants decided to take an indirect approach in this refinement stage where they did not discuss the constituents of the plan directly but asked their co-authors and TAs about \textit{``How complex does [the plan] feel to [them]?''} (P8) and then they redefined their patterns according to the responses to the question. This is because they believe that \textit{``complexity is a proxy for how much do I know it?''} To elaborate, they suppose that \textit{``If I know something, no matter how complex it is, I think it's simple. If I don't know it, even if it's a little bit simple, I would think it's still complex.''}

% \textbf{Decision Discussions: } 
%P1 highlighted that they deliberated about which plans to include in their final inventory: \textit{``we just talked about them, be like, ``Yeah, that's really kind of arcane. Maybe that's not a general purpose pattern.''} or {``Oh yeah, that makes a lot of sense.''} 
%P10 explained that their plan identification process included \textit{``discussions about the ``why" and ``why not".''} %% Maybe this is about rationale? KC

% \subsubsection{Literature Review}
% \label{sec:lit-review}
As another approach for creating candidates, almost all of our participants conducted a literature review to take inspiration from the existing patterns and tailor it to their vision. \textit{``Having a bunch of literature of course helps a lot,'}' shared P8. \textit{``We also took those [plans from literature] because they were broadly researched, people talked about it, so we had some benchmarks.''} But finding a plan in the literature wasn't always the end of the search. P8 shared that they don't take plans from the literature as the final word.
\begin{quote}
Someone telling you this is a good candidate for a programming plan, you still have to think, is it still a programming plan? Or maybe not. Is it only for this particular context? (P8)
\end{quote}

\subsubsection{Finalizing Plan Design}
\label{sec:individual-design}

The final step in this process was refining the candidate plans into programming plans appropriate for teaching.
All of our interviewees have teaching experience, and most of them have had long careers as instructors (see Table \ref{tab:participants}). Instructional expertise is helpful for transmitting programming plans effectively to an audience of learners. As \textit{``plans are useful only if they can be communicated between humans''} (P1), it's important for plans to be appropriate for the knowledge level of the audience. One of our interviewees also highlighted that a plan identifier needs to have relevant programming background or find it in someone else: \textit{``...you need some experience in that practice or the ability to have discussions with people who have that experience''} (P10).
These instructors believed that only those with expertise in a programming topic could shed light on the crucial issue of code's intent. As P4 described such knowledge: \textit{``
%Is this a pattern or is this just a construct of your language? 
Is this an unfortunate oddity of syntax, or is this actually something that means something to you in your domain?''}

Thus, one of the techniques employed by our interviewees was to draw on their experience in order to come up with plans on their own. By considering important problems and exploring \textit{``how you would imagine best solving the problem''} (P9), the instructors generated first draft programming plans they could refine and use in instruction. Envisioning assignments at the appropriate level and inferring relevant plans was a similar strategy for some: \textit{``just also thinking about what's an interesting ten line program that you can assign students to write.''} (P1) \textit{``Now what other simple problem do I want?...Okay, let's try to encapsulate that as a pattern''} (P6)

% Just putting this other quote here in case can be used
% "Now what other simple problem do I want? A simple type of problem, do I want them to be able to solve?" And so then, "Okay, let's try to encapsulate that as a pattern."

With this strategy, the instructors used their own expertise to complement searching through the literature or example programs: \textit{``I knew that's the way you solved the problem so, in a way, I didn't need any kind of evidence.''} (P6)

%% Commented out as too distant from the theme
% People who identified patterns in a CS1 context, argued that sometimes their patterns might seem quite similar to others' even after self-identification since there aren't too many new ways to achieve the same goal. For example, 

% \begin{quote}
%     ... but in essence [the plan] seemed to be the same. Because, as I said, there are not many ways of iterating over a list or counting or summing things up. (P6)
% \end{quote}

% Some other tools that researchers used to identify patterns included looking at the \textit{``higher order functions''} (P5) existing in high-level languages since they are representative of common goals in programming. Sometimes, the way of identifying patterns for some researchers was also influenced by the flow of content in their instruction as illustrated by P6, \textit{``and the problems were chosen based on prior assignments''}.

% In light of this, our participants also brought to the fore that these resources can also be used to validate if plans are common. 
% These include looking at course materials and evaluating the recurrence of patterns, discussing with other people including students, TAs, colleagues, and, using self-introspection to intuit commonality.
% All of these approaches help researchers create an inventory of candidate plans.

% \subsection{How Expertise Influences the Plan Identification Process}
% \label{sec:expertise}
% Many of our interviewees could reflect on long careers that included both instruction and industry experience (see Table~\ref{tab:participants}). As P10 put it, their team of plan identifiers had \textit{``long experience both with programming and with teaching it.''} From our interviews, it was clear that our participants drew on two distinct types of expertise as they identified programming plans: their  \textit{instructional expertise} and their \textit{programming expertise}. We found that these types of expertise are distinct but complementary. 

% \subsubsection{Instructor expertise supports judgment of a plan's appropriateness for learners}
% \label{sec:instructor-expertise}
% Instructional expertise helps the plan identifier with aspects of plan identification related to transmitting programming plans effectively to an audience of learners. As \textit{``plans are useful only if they can be communicated between humans''} (P1), it's important for plans to be appropriate for the knowledge level of the audience. Fortunately, instructional expertise made our plan identifiers ready for the job: \textit{``This is one of the things that academics do, we reinterpret things in terms that are easier to teach to people''}~(P7).

% The instructors believed that their teaching experience gave them insight into \textit{``....the mental state of the person that's going to receive this information and how do I communicate that effectively?''} (P1). During the plan identification process, our educators put themselves in the minds of their students. P3 described this process: \textit{``we imagine, this is what students would have to learn from scratch. This is day one.''} This understanding helps them in refining plans so they are an appropriate bridge for problem-solving. P3 described this as a  \textit{``process that takes a lot of time and a very deep understanding on what you're trying to accomplish and who is your audience that you are trying to create those plans.''} 


% EXPERTISE: Instructors seem to have a sense for what the right usability level is for their students.

% About applying the plan model
%``We always have these very lengthy discussions on what is a plan for you? What is a plan for your students and what is your notional machine? Yeah, that's a ''

% The process is lengthly, and requires a large amount of instructor expertise about what your learning goals are and what your audience understands.

%In another scenario, while having discussion with programmers, this teaching experience can help get rid of \textit{``expert blind spot.''} (P8)

%Both experiences give the pattern identifiers the skill set for self-examination of patterns which helps create a set of candidate plans. 
% Area is important. Where do I mention this?

% \subsubsection{Programming expertise supports judgment of a plan's frequency in practice} 
% \label{sec:programmer-expertise}
% Programming experience also played a role in understanding the value of candidate programming plans. % \subsubsection{Opportunity to Foster Collaboration between Educators and Developers}
% As P7 described, \textit{``In my case, I had a large internal vocabulary of examples already because of my own mileage in code writing.''} P8 agreed that programming experience \textit{``gives you a huge pool of examples.''}

% Not only does programming experience give a plan identifier a set of sample plans, it also equips them to reflect on their own code writing process to inform the plan identification process. P7 highlighted that one can \textit{``think back''} to consider what part of their code writing process was \textit{``easy or hard''}. They also added that during this process, it is possible to find \textit{``sets of features that separate different categories''} of plans.

% One of our participants mentioned that an advantage of having this experience was how it supported them to make judgments about candidate plans:
% \textit{``If you don't know it [the candidate plan] and you already saw so much, I think it's very unlikely that someone else knows.''} (P8)
% Without this expertise, a plan identifier would have to have more discussions with people with this expertise which is a time-consuming process.

% One of our interviewees also highlighted that a plan identifier needs to have relevant programming background or find it in someone else: \textit{``...you need some experience in that practice or the ability to have discussions with people who have that experience''} (P10).
% These instructors believed that only those with expertise in a programming topic could shed light on the crucial issue of code's intent. As P4 described such knowledge: \textit{``Is this a pattern or is this just a construct of your language? Is this an unfortunate oddity of syntax, or is this actually something that means something to you in your domain?''}

%In addition, one also gets access to a large dataset of code for identifying patterns in the form of code written in the past of lecture notes. P7 endorsed this claim by saying, 
% to think about the challenges that they faced while writing code. They could also think about the times where they had to achieve the same goal in different contexts and then evaluate the similarities and differences in the code that they wrote.
% quote -- So if I sort of start to think back and say, all right, what made these test cases easy, hard, whatever? Is there a set of features that separate this category from that category of tests?

% This could help identify patterns since finding the "pattern" is at the core of the pattern identification process.

%``I went for a...software development internship. And also there, I saw people using those plans and also talking about code using those plans. So they were not saying, and here's a for loop and here's a variable...they said to us, "And here we compute a maximum, here we have minimum, here we have the sum, here we have the average".'' (P8)


% look at what others who are working or doing stuff in this domain

% \subsubsection{Something}
% % hmm it doesnt color the subsubsection headings, maybe we don't even need one
% but without it, it seems to be a part of programming
% \label{sec:something}

% \edit{Some participants suggested ways in which educators who understand the concepts of programming plans could work with programmers who understand their programming domain:
% \begin{quote}
% %I think one thing that we didn't do, which I think could be interesting, again, 
% ...if you're trying to find novel patterns...
% %, would be to 
% sit down with a developer at a real company and just have them walk you through the code and explain what's going on. And that may be a good way to simultaneously have the coding expertise, but not be susceptible to the expert blind spot. 
% % %Because if you have somebody who doesn't understand the code and the person's explaining it to you, you sort of get both of those things that they have to be explicit about why they're doing the things that they're doing 
% (P1)
% \end{quote}}

% %One of our participants said that one could work with a \textit{``forwards''} (P4) approach to consider expert opinions while identifying plans where one asks the expert about things that \textit{``mean something''} to them. Or, one could take a \textit{``backwards''} approach and create a set of candidate plans and then ask the expert if those patterns are important. 


\subsection{Challenges}
\label{sec:challenges}

While our participants had all successfully identified programming plans to be used in an educational context, they did name some challenges they experienced in their process, or imagined challenges for future plan identification. % These cOur participants expressed that they faced challenges while identifying plans from practice and while choosing the suitable level of abstraction of these plans for their audience.}

%Identifying the predicaments in the plan identification process will help us get closer to uncovering design implications. Thus, this section highlights the challenges that our participants faced while identifying patterns.

\subsubsection{The Challenge of Choosing Plans from Practice (C1)} 
\label{sec:challenges_practice}
While our participants had ready access to resources that could be used to identify plans, like example programs and problems, they still found it onerous to translate this practice into plans. As P7 described, \textit{``the challenge was trying to infer general characteristics from a large collection of specific examples.''} Another participant described this part of the process as \textit{``tedious''} (P1). 
% \edit{One of our participants expressed that they faced uncertainty while \textit{``defining the plan''} because they had \textit{``ambiguous definitions (P2).''}}

This challenge is particularly relevant when educators ``mine'' plans from existing materials, a technique that the vast majority of our participants employed during plan identification (see Section \ref{sec:viewing-programs}). Understandably, determining which aspects of a given example are widely applicable enough to ``count'' as a programming plan is a difficult task, as the possibilities for writing programs are so extensive. Currently, plan identifiers lean on a combination of their own expertise and a manual search of examples as they tackle this challenge.
% Our plan identifiers this search space can be quite large, leading to a lengthy exploration for something that qualifies as a common pattern.  
%We found that it is difficult for educators to identify programming plans from practice because the search space can be quite large. %When instructors lack the programming expertise to evaluate the meaningfulness of potential plans, it also becomes highly challenging to identify plans.

\subsubsection{The Challenge of Finding the Right Abstraction Level for a Plan (C2)} 
\label{sec:challenges_abstraction}
Once plans are chosen, the challenge only begins. As mentioned previously, finding the right amount of abstraction is key in identifying good plans, according to computing educators. However, finding the appropriate balance of specific and general was a hurdle in our interviewees' plan identification process. 

\begin{quote}
    The difficult part is probably defin[ing] the level. What's the plan you are looking for, how general it is. Level of generality, abstraction to define it. (P2)
\end{quote}

P7 illustrated their challenge by describing the conflicting goals of this process: a good plan should be \textit{far enough above the level of code that it really does insulate students from syntax and mechanics, but not so far that it becomes useless.} Currently, educators make use of discussions with colleagues as one approach to manage this difficulty. On an even longer time frame, they may also refine their plans after seeing how they work in the classroom.

% \begin{quote}
%     But trying to infer what the pattern would be that they had in common was the hard part. (P7)
% \end{quote}

%% This challenge doesn't come across as convincing to me
% \subsubsection{Designing Patterns}
% Even though some of our interviewees believed that defining patterns is not as hard as identifying patterns, \textit{``I don't think we have problems defining them, maybe choosing them.''} (P9), there were some considerable challenges while designing patterns that came across in the transcripts. Most of these challenges were associated with putting the identified patterns in the form of a structure that meets the definition of a "pattern and
% that best fit their audience like \textit{``Maybe I should rephrase this a little bit so that it becomes more consistent across the patterns.''} (P6)

%% Not super convincing and the talking with students is maybe not the best example. But maybe I'll change it back.
% \subsubsection{Communication and Collaboration Barriers}
% One of our participants encountered impediments in the form of language barriers while conducting their plan identification study because it involved oral exchanges. P4 highlighted that gathering intent from students about their planning process, although better than just interpreting code, still comes with its hardships. For example, \textit{``"they just didn't know what they were doing and throwing whatever at the wall, see what sticks"''} and \textit{``"some students English wasn't as clear as others, whether it's because they were trying to do it shorthand, whether it's because we have students who English isn't their first language."''}

% Moreover, some of our participants emphasized the importance of collaboration in the pattern identification process, for example, \textit{``Most of the work in programming patterns was highly collaborative. A group of us would meet once or twice a year for several days and kick ideas about.''} (P10). One of our participants, thus highlighted distance as a struggle however, we believe that the new technological advancements would help reduce such hardships. 


\subsubsection{The Challenge of Identifying Plans in an Unfamiliar Domain (C3)}
Identifying programming plans from more programming topic areas beyond introductory programming could enable better support of learners with diverse interests. However, when we asked our participants about how they would identify programming plans in a new and unfamiliar domain, nearly every one said that it would be quite challenging. Our educators were quick to emphasize the importance of having programming expertise in the relevant topic when performing plan identification.  \textit{``Looking for material and try to recognize the schema without having expertise in the area, I think that would be impossible,''} said P2.

The key hurdle our participants anticipated in identifying plans from new domains was not finding common code, but understanding if that code implemented a meaningful goal. The task of identifying plans for educational purposes is multifaceted: it involves understanding \textit{``what goals [students] need to achieve in this particular context''} and \textit{``how much do they know at this point in their instruction''} (P3). While educators might understand what students are ready to learn, they cannot understand the important subgoals and tasks of a programming domain without the relevant expertise.

Several participants suggested collaborating with an expert from the relevant domain to understand the \textit{``typical things [experts] need to do and how [those are] typically done} (P2). This discussion was necessary to \textit{``discover the intent behind code''} (P1), which is challenging to discover from looking at code alone. Some of our educators mentioned that using emerging AI tools like GitHub Co-Pilot (P6) or ChatGPT (P7) to generate solutions may also help with plan identification when plan identifiers lacked the relevant expertise.

%``why they're doing what they're doing'' (P1), Need to ask a domain expert ``does this actually mean something to you?'' (P4)

% more quotes for this section

% P1
% somebody to sit down with you and walk you through, pick up a representative piece of software for doing that, explaining why they're doing what they're doing
% I think the intent is one of the things that's often hardest to discover in large pieces of code and that people may be much better at explaining what the code is doing rather than why it's doing. 
% And I think a virtue of pattern is that it should hopefully represent best practice or at least good practice.

% P2
% I'd find an expert and ask them what are typical things you need to do and show me how this is typically done. 

% P3
% Well, then I would first have to understand what are the constituents of this context, so what students need to perform and what goals they need to achieve in this particular context. 
% hen I would look into how much do they know at this point in their instruction and match those and say, "Oh, this is enough."
% What kind of actions do they have available? What are the rule sets? What are the goals they need to achieve?

% P5
% I don't know if there are textbooks on web scraping, but looking at some kind of rigorous sources on web scraping, seeing what do people do? How do they describe these things? Are there threads of commonality between different resources?

% P4
% And so that there's sort of the backwards and forwards being able to do that program analysis and then go to a domain expert and be like, does this actually mean something to you? And then going to the domain expert and be like, okay, what means something to you? What are the patterns that you're commonly using? And then can you show me? 

% Some of our educators mentioned that using emerging AI tools like GitHub Co-Pilot (P6) or ChatGPT (P7) to generate solutions could help in plan identification.

% P8
% I would probably look at what others who are working or doing stuff in this domain, how would they do things and how would they solve common problems? What are common problems, what are common solutions to that problems?
% And then they started explaining what features they like to have in there and it helped me a lot to see how they argue about things, because they started arguing about things on a very different level than I expect because they have different ways of thinking about things.

% % addressing the third challenge
%\edit{While conducting the study, we presented our interviewees with a hypothetical scenario in which they had to identify plans in a domain that they were not knowledgeable about. We asked them about how the process of plan identification would be similar or different in that scenario. While our interviewees did not perform plan identification in domains unfamiliar to them, they suggested that collaboration with experts from those particular domains could yield insights into the code's intent (Section 4.4.2). With these conversations, then these instructors would be able to use their educational expertise to create plans from those domains.}