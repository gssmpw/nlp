\subsection{Solutions-first vs problems-first}
% Our participants varied in the way they defined a programming plan. Most participants took what we term a ``solution-first'' approach. This definition emphasizes ..the specific implementation of a programming pattern...

For most of our participants, a programming plan is at its core a solution. P2 defined a programming plan as \textit{``a stereotypical solution to a recurrent problem in programming.''} P3 defined programming plans as being even more specific: \textit{``A stereotypical solution to a given problem translated in a programming language.''} Whether educators believed that a programming plans should be language agnostic or not, they had a ``solution-first'' perspective where their primary focus was on the implementation of the plan.

%However, our interviewees had a variety of ways they presented this ``solution'' to learners. Approaches included code (P1, P3), pseudocode (P1), and description in natural language (P5, P6, P9). %% This shows up later

% While some participants belived that the solutions should be language agnostic, others believed that patterns were specific to the language they were implemented in.


In contrast, 
some pattern identifiers took a ``problem-first'' approach towards pattern identification. 
% Ideally we would have another sentence here explaining this perspective with more specifics. (Helping students see the underlying concepts. Helping students understand which problems should be solved in similar ways)
For these participants, a plan is \textit{``...anything that's trying to get at making these useful problem types explicit in a way that we can teach them.''} (P5). So, in this case, the focus is primarily on \textit{``analyzing what type[s] of problem[s]''} (P9) exist and then designing solutions to each problem type, which all together is referred to as a pattern.



%% Maybe we can fit in the different perspective, but right now we're skipping -KC
% However, for one of our participants, a programming pattern was a term used to describe a programming plan that was repeatedly used to achieve the same specific goal in that context. Thus, a programming plan was just a solution to a specific problem in discussion.

% \begin{quote}
%     The plan is sort of a roadmap to solve a particular problem, whereas patterns would be similarities across those plans that have been created. (P4)
% \end{quote}


% Some people include both in their definition. 

%P6 believes that one should \textit{``....Easily know how to apply the pattern.''} and thus this approach provides scaffolding for that. %Moved to usability section

%Another one of our participants also revealed that a pattern is \textit{``very individual[istic].''} (P10) %% Unsure how this fits the theme

% They added that a pattern is 
% \begin{quote}
%     .... some piece of code they are very, very used to. They know how it's structured, 
%     % so 
%     [how] it's organized, how it works. What are main mistakes that could happen with this one? (10) %% Moved to rationale section
% \end{quote}

%They also underlined that one cannot learn a pattern immediately, rather, this learning process follows a \textit{``scale from you don't know it at all and you really know this plan.''} %% Unsure how this fits the theme



%%%% Literature review

%Finally, the second author reviewed the methodology sections of all the papers in the literature search to find descriptions of the plan identification process.

%\subsection{How publications describe plan identification}

% \subsubsection{Early cognitive era (1970-1990)}

% \subsubsection{Instructional focus era (1990-2015)}

% \subsubsection{Recent era (2015-present)}



% For plan identification in new domains, a couple of our participants highlighted that 
% % talking with professionals in the domains 
% \textit{``look[ing] at what others [are doing] who are working or doing stuff in this domain''} (P8) helps give insight into \textit{``patterns that you[they] commonly think about in your[their] code''} (P4) and \textit{``what features they like to have in there''} (P8). P1 re-iterated that that it helps if \textit{``you can convince somebody to sit down with you and walk you through, pick up a representative piece of software for doing that, explaining why they're doing what they're doing''}.



% P7 encapsulated the process of familiarizing oneself with the intricacies of the domain by saying, 
% \begin{quote}
%     Doing a lot of internet searches, reading the results, and trying to distill them into a couple of paragraphs and seeing what the main ideas are. (P7)
% \end{quote}

% For self-learning, there are different places where one can look for patterns.

% \subsubsection{Example Programs}
% It is possible to look at example programs as described in the plan identification section to identify plans in a new domain. The researcher could look for repeating pieces of code and then make judgement calls to consider them as patterns or not.

% One of our interviewees also suggested, \textit{``look at APIs, libraries, and to see, because in the tutorials often they say, "Okay, this is how you should use a API or give an example.''} (P6)

% Other approaches to identify plans in a new domain using example programs includes a "problem-first" approach as described in the plan identification section where one can look at common problems existing in the domain, then for the solutions commonly used to solve that and thus create a set of candidate plans.
 

%One of our researchers highlighted that it might be useful to look at work on pattern identification in the past in that domain and hope that \textit{``someone before me had the sense to realize that there was a pattern-ish way of teaching it, and maybe then look for how did they do that?''} (P7) %% I think this is about the literature?