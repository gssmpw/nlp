
\edit{\section{Opportunities for Design to Support Programming Plan Identification}}

\begin{figure*}
\centering
\includegraphics[width=\textwidth]{img/design.png}
\caption{A proposed sociotechnical plan identification process that mirrors educators' current plan identification procedures while leveraging both human expertise and automated processing for improved efficiency.}
\label{fig:design}
  \Description{This diagram summarizes the proposed sociotechnical plan identification process. The first box represent the candidate plan identification stage where plans are created based on common code sequences, from literature, or from scratch. There is an arrow that originates from this box to the next box that represents the plan refinement stage showing the successive steps. In the plan refinement stage, we modify the abstraction level of plans and receive automatic and crowd-sourced feedback on metrics. This box has a self-loop arrow representing that it is an iterative process until goal is reached. Then, there is an arrow pointing to a plan library symbolizing that a collection of candidate plans has been created. The last arrow from the plan refinement box is to the last box representing the plan elimination stage. At this stage, if the candidate plan does not reach acceptable metrics or does not meet learning goals, it is discarded.}
\end{figure*}

With a more precise understanding of how plan identification is currently performed by educators, \edit{as well as the barriers in the current process of plan identification,
% as well as an understanding of several opportunities for improving the efficiency of the plan identification process, 
we are able to propose several opportunities for improving educators'} process for plan identification. These new processes mirror the stages of educators' current plan identification procedures, while leveraging both distributed human expertise and automated processing. We describe each \edit{of these avenues} 
% stage of this process 
below and present a summary in Figure~\ref{fig:design}.

% \subsection{Some Plan Structures Better Support Automated Metrics}

% % take a stance
% % We need code in plans to better find the level of abstraction. In plans with just natural language description, it would be difficult to measure the level of abstraction

% \edit{To allow educators and developers to work together, the system must use a consistent plan definition which includes choosing the necessary plan components that best meet the learning goals of the intended audience. We believe that choosing a programming plan definition where plans are defined in code and where plans have changeable areas will best facilitate automated processes that draw on pools of code examples.}

% \subsection{Choice of Plan components and Perspective}

% To allow educators and developers to working together, the system must use a consistent plan definition. Designers of plan identification systems should choose from the list of components (Section~\ref{sec:components}). We believe that choosing a programming plan definition where plans are defined in code and where plans have changeable areas will best facilitate automated processes that draw on pools of code examples.

% Implications for the choice of plan components


%Based on , we believe that a plan definition that uses a solution in code rather than pseudocode or natural language (Section ~\ref{}) presents more opportunities to partially automate the plan identification process because it allows systems to take advantage of the data in existing code repositories, like GitHub or StackOverflow. Further, we believe that the plan representation in a plan identification system should include specific changeable areas (Section~\ref{}), as it enables automatic calculation of the abstraction level . This idea is further explained below in Section~\ref{}.
 
\subsection{\edit{Opportunities to support candidate plan identification}} % Creation stage

% \subsubsection{Opportunity to Involve AI or Data Mining Tools}
% \textit{``Since it learns from so many, it probably has already kind of done the synthesis itself,''} reasoned P6. P6 also suggested that 

\edit{The time-consuming nature of creating a representative, general-purpose programming plan from many examples slows the process of plan identification for educators.}
Reflecting instructors' experiences with plan identification, there are multiple avenues to introduce candidate plans into a system in a way that accelerates the process of plan identification: those that automate the process and those that draw on past work or an educator or programmer's expertise. 

% Some of our educators mentioned that using emerging AI tools like GitHub Co-Pilot (P6) or ChatGPT (P7) to generate solutions could help in plan identification.

First, in order to make the challenging process of plan choice more efficient, a plan identification system can generate candidate plans by searching relevant pools of code for common pieces of code~\cite{haggis_code_similarity}, or by prompting a large language model for examples from a particular domain. Candidate goals and names may also be generated by a large language model. The user can specify the topics they are interested in identifying programming plans for, and the system could then suggest relevant code corpuses drawn from online code repositories (i.e., Github, Stackoverflow) or instructional materials. As educators vary in their beliefs about the type of practice that plans should represent, users should have the ability to limit the code pools used for candidate generation to those by certain types of authors or another measure of quality.

%\edit{Our participants suggested using emerging AI tools to ask for solutions to the same problems in different contexts and then evaluate code similarity to search for plans.}

Another method of candidate plan generation could draw on the instructor's current insight. As some instructors prefer to design their own plans, plan identification systems should include the possibility for users to create a candidate plan from their own intuition. %This process of course depends on a user's programming expertise.

Finally, as some of the educators we interviewed found plans from the literature important in their own plan identification process, systems could allow users to search the existing plan literature as well as programming plans created by other users.

\subsection{\edit{Opportunities to support plan refinement}} %Refinement stage

% \subsubsection{Opportunity for Automated Metrics for Abstraction, Commonality, and Usability}
\edit{We found that plan refinement is an crucial stage in educators' plan identification process, but also an area where educators face challenges as they struggle to find the right balance between a plan's frequency in practice (commonality), ease of use (usability), and appropriateness for their learner population. The process of plan identification could become more efficient if educators could receive quicker and clearer feedback about how their candidate plans measure up in each of these characteristics. }
%evaluating the metrics of plan quality that instructors value could be automated or crowd-sourced.} 

\edit{With automated metrics of commonality,  usability, and level of abstraction, the feedback loop that educators undergo as they search for the right version of their programming plan could be shortened. Such automated metrics may be possible, if the right plan components are available. If programming plans are represented in a format that includes their programming language code, a system could search relevant code repositories for similar code in order to provide a measure of a plan's \textit{commonality}. If the system represents programming plans not only in code but also with changeable areas (see Section~\ref{sec:changeable}), then the system can infer the level of abstraction based on what percentage of the plan consists of changeable areas, as well as the complexity of typical code in the changeable areas.}

\edit{To measure usability, which is the ease with which learners can modify and apply the plan, code comparison with a relevant corpus may not yield much insight. However, it may be possible to get learners involved through a learnersourcing~\cite{weir2015learnersourcing} activity. The system could connect with novice programmers at the appropriate level and give them an automated task to implement the candidate plan. Based on the correctness of this activity, it could provide a metric for usability.}

%\edit{If educators could quickly understand how changes to the programming plan they are refining affect the metrics that they care about, it may increase the efficiency of the plan identification process.}

During the refinement stage of a plan identification system, instructors can edit the candidate plan with the goal of balancing the metrics of commonality, usability, and level of abstraction to values within their desired thresholds. As some instructors value some metrics more than others, users should have the ability to modify the acceptable ranges for each of these metrics. 

As instructors add or remove changeable areas to their plan, they receive feedback from the automated metric values. Instructors' expertise is particularly relevant in the refinement stage, as they not only rely on the system for information about the value of their candidate plan, but also judge its appropriateness themselves. 

\edit{Automated metrics may not completely replace the benefits educators gain from conversations with their peers. In our interviews, we learned that the educators we worked with had extensive conversations about plan identification during their process (see Section~\ref{sec:discussions}). Conversations between educators can also be supported in a plan identification system, perhaps through chats, a commenting system, or suggestions for plan changes.}
%as the instructors we interviewed were clear that these conversations led to improved plan identification outcomes. These reflections on the quality of the plans and conversations can occur through chats and even spontaneous video calls enabled by the system. }

\subsection{\edit{Opportunities to support plan identification in domains unfamiliar to educators}} %Refinement stage

Enabling collaboration between instructors and developers could allow for easier identification of programming plans in specific application areas, as the professional collaborator can suggest plans from their personal expertise or during a guided think-aloud, and the instructor collaborator can refine those plans so they are appropriate for learners. \edit{This approach could make it possible to greatly expand the number of topics where the plan identification process can be applied, as it would no longer depend on a single individual with both types of expertise.}

\edit{Based on our findings, it's important that those with programming expertise and those with instructional expertise contribute in ways that make use of their particular strengths (see Section~\ref{sec:expertise}). For experts in the programming domain, a system could encourage their participation in the candidate plan stage in particular. This may include sharing information about a plan's commonality in a particular domain, as well as information about the intent of a candidate plans. Then, educators may become involved in the plan refinement stage, particularly as they are better positioned to judge how understandable plan components are for the relevant learner level. }

% \subsubsection{Opportunity for Facilitating Discussions Between Educators}


%\subsection{Elimination stage}

%Finally, not all candidate plans will make it into the final plan library. If instructors cannot balance the metrics for a particular plan to acceptable values, or, if the instructors believe that the plan doesn't meet their learning goals, they can remove the plan from consideration. 
