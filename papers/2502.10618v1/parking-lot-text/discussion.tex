\section{Discussion}

% Next we summarize the challenges and opportunities in plan identification based on our interview findings, and 


In the prior section, we identified the \textit{components, characteristics, and challenges relevant for the plan identification process}. 
This section focuses on the implications of those results.
% Our findings shed light on the ways that instructors who identify programming plans for educational purposes address inherent difficulties in the plan identification process. 
% In this section, we reflect on the nature of the challenges in plan identification

%%%%
% 1. Translating Values to Metrics
% only use the term metrics for when values can be observed through measurable changes in the code


% ## concrete concepts
%% > Definitions
% ## how to measure the concepts based on code features
%% > Comprehensive look at the balance between plan value
%% > How to actually calculate plan metrics


% ## Plan structures


% Plan identification requires instructor and programing expertise.



% The plan identification procress is iterative and collaborative
%5 In order to balance the tradeoffs
% - Discussion
% - Back and forth between creation and discussion


%%%%%

% \subsection{Educators Must Balance a Variety of Interconnected Values During Plan Identification}

% % \begin{figure}
% % \centering
% % \includegraphics[width=0.5\textwidth]{img/metrics-balance.png}
% % \caption{Relationships between the characteristics educators value when judging quality of programming plans for instruction}.
% % \label{fig:metrics}
% %   % \Description{A circle in the center for representing the abstraction level. A circle on the top-left for representing commonality. An arrow from the abstraction level to commonality with a plus sign to say that as the abstraction level of a plan increases, it becomes more common and hence establish a positive relationship between these metrics. A circle at the bottom-left expressing usability. An arrow from the abstraction level circle to usability with a negative sign indicating that as abstraction level increases, the plans become less usable. A two-way arrow with a plus sign between commonality and usability to indicate that as usability increases, the plans become more common and vice-versa. Another circle on the center-right for determining the appropriateness of the plan for beginners. An arrow from abstraction level to appropriateness for beginners with a negative sign to say that as plans become more abstract, plans become less appropriate for beginners.}
% % \end{figure}

% % describe the triangle of abstraction level in detail
% % tie it back into the previous section
% Based the findings presented in Section \ref{sec:abstraction}, the right level of abstraction was a key characteristic that helped educators determine the quality of a plan. In Section \ref{sec:challenges_abstraction}, we also acknowledge that creating a suitable level of abstraction was a challenge that most of our participants encountered. Educators identifying plans aren't the only ones to face this challenge. In their study of a system for capturing developers' programming strategy knowledge, Arab et al. found that developers had difficulty ``finding the right scope for their strategy'' to share with others~\cite{Arab_StartegySharingSystem_CHI-2022}.

% \begin{figure}
% \centering
% \includegraphics[width=0.5\textwidth]{img/new-balance.png}
% \caption{Relationships between the characteristics educators value when judging quality of programming plans for instruction.}
% \label{fig:metrics}
%   % \Description{A circle in the center for representing the abstraction level. A circle on the top-left for representing commonality. An arrow from the abstraction level to commonality with a plus sign to say that as the abstraction level of a plan increases, it becomes more common and hence establish a positive relationship between these metrics. A circle at the bottom-left expressing usability. An arrow from the abstraction level circle to usability with a negative sign indicating that as abstraction level increases, the plans become less usable. A two-way arrow with a plus sign between commonality and usability to indicate that as usability increases, the plans become more common and vice-versa. Another circle on the center-right for determining the appropriateness of the plan for beginners. An arrow from abstraction level to appropriateness for beginners with a negative sign to say that as plans become more abstract, plans become less appropriate for beginners.}
% \end{figure}

% Synthesizing our interviewees' responses yields insight into why finding that correct level of abstraction is so difficult: A plan's level of abstraction directly impacts three other desired characteristics, some positively and some negatively (see Figure~\ref{fig:metrics}).  In essence, plan identifiers are performing a balancing act between a plan's commonality, usability, and appropriateness for their students.

% Abstraction level has a direct connection with the \textit{commonality} of a plan. When a plan is more general, it will be seen in practice more often. However, abstraction level has an inverse relationship with how \textit{usable} a plan is. The less specifically a plan is defined, the more interpretation a programmer needs to use to put the plan into use. For instance, with a more abstract plan%(e.g., the bottom plan in Figure~\ref{fig:abstraction})
% , programmers have to make more implementation decisions on their own when they put the plan into practice.

% % this probably needs more unpacking, it makes sense for me but a new reader might not understand this
% %% This is kinda confusing, so I commented out for now
% %\edit{Commonality and usability also have a positive relationship. If a plan is more usable, then it is seen in practice more often and thus it can be termed common. Similarly, the more common a plan is, it can be used in more contexts and thus is more usable.}

% The level of abstraction of a plan has a negative relationship with how appropriate it is for beginners, an important concern for educators specifically. While learners can benefit from instruction about abstract principles~\cite{doi:10.3102/0013189X025004005}, novices have more difficulty applying more abstract representations~\cite{kaminski2006children}. Novices have less familiarity with programming, and therefore programs that an expert sees as similar may be seen as distinct by the novice.
% % a higher level of abstraction makes the programming plan less appropriate for beginners 
% %\edit{due to the prior conceptual background needed to understand them. To illustrate with an example, from Figure 3, if a novice is presented with the third plan (goal: make a selection), it will be challenging for them to understand the structure and purpose of the plan. However, if the same novice is presented with the first plan (goal: validate for equality), it would prove to be less intense because of the amount of scaffolding that comes with the plan.}





% % to determine the desired level of each metric based on their target audience. A more experienced audience may be comfortable with more abstract plans and the educators can focus more on finding a common ground with respect to the other metrics, however, when creating plans for apprentices, focusing on learning goals might be more important than others.

% % i don't think this fits here anymore
% % disbursing it to other sections
%  % The process of plan identification could become more efficient if evaluating the metrics of plan quality that instructors value could be automated or crowdsourced. The instructors we interviewed noted that many of their discussions were about whether plan candidates met their measures of plan quality, particularly the correct level of abstraction. If educators could quickly understand how changes to the programming plan they are refining affect the metrics that they care about, it may increase the efficiency of the plan identification process.
 
% % \subsection{The Plan Identification Process is Iterative and Benefits From Feedback}
% % Iterative — it is, but why. Collaborative and discussion to balance metrics and address challenges

% % introducing the section
% % From the interview study, we find that having discussions with colleagues and other researchers was a key component of our participants' plan identification process. Based on the findings in Section \ref{sec:discussions}, the discussions were often regarding identifying plans, refining them, or making decisions to eliminate them, if necessary.

% % % maybe rephrase later
% % Our interviewees collaboratively used their expertise and resources to look at the large search space for code and identify candidate plans. This collaborative effort helped them address the challenge of choosing plans from practice since each of them identified their own plans and then deliberated upon how to translate code into a programming plan for learners with the necessary components.

% % The instructors also noted that many of their discussions were about plan properties and whether plan candidates met their measures of plan quality, particularly the correct level of abstraction. We believe that these discussions helped them address the challenges they faced to identify the suitable abstraction level that fit the learning goals of their audience. Our participants used their experience to inform their judgment of the characteristics described in the previous section. With their pooled expertise, they were able to balance those values and make decisions about the candidate plans.
% % \edit{In the refinement discussions, our interviewees highlighted that they deliberated about plan properties and plan quality. We believe that such discussions helped them address the challenge of finding the suitable level of abstraction that fit the learning goals of their audience.}




\subsection{Plan Identification for Education Requires More Than Finding Common Pieces of Code}

% findings new programming plans may require both these expertise
% we have evidence from the new domain ques from the interview

% overview
As we note in Section \ref{sec:individual-design}, experience with the relevant programming domain as well as having instructional expertise is necessary in the process of plan identification to accurately identify useful plans suitable for learners. These types of expertise appear to be complementary yet distinct. Attempts to glean programming plans from domain experts alone seem unlikely to be successful without instructional expertise.
%Instructional expertise may be crucial in the attempt to glean programming plans from developers. 
When Arab et al. studied developers' ability to share strategic knowledge, about half of the developers they studied found it difficult to write strategies in a way that novice developers could understand~\cite{Arab_StartegySharingSystem_CHI-2022}.

Simply knowing that a piece of code is common in practice does not solve the problem of plan identification. While this may be a good starting place, instructional and programming expertise is necessary to create a quality programming plan. Those with programming expertise can answer questions about whether the pattern is a meaningful problem-solving step in the domain, and what the goal and intent of the code is. Those with instructional expertise can answer questions about whether the current form of the plan is understandable and usable by students with a particular background. Human expertise is a valuable, and possibly indispensable, part of the plan identification process.

% why instructional
 

% \edit{While }

% % combine both
% \edit{Some participants suggested ways in which educators who understand the concepts of programming plans could work with programmers who understand their programming domain.} 

%\edit{This strategy mimics having the coding expertise to understand the code structure but also prevents being influenced by expert blind spot to keep the plans in favor of the learner's goals.}
% rephrase, this is the language from a quote
% \edit{And that may be a good way to simultaneously have the coding expertise, but not be susceptible to the expert blind spot. }


% this i feel like belongs more in the opportunities section since it deals with the automation
% Such automated metrics may be possible, if the right plan components are available. If programming plans are represented in a format that includes their programming language code, a system could search relevant code repositories for similar code in order to provide a measure of \textit{commonality}. If the system represents programming plans not only in code but also with changeable areas (see Section~\ref{sec:changeable}), then the system can infer how high the level of abstraction based on what percentage of the plan consists of changeable areas, as well as the complexity of typical code in the changeable areas.

% To measure usability, which is the ease with which learners can modify and apply the plan, code comparison with a relevant corpus may not yield much insight. However, it may be possible to get learners involved through a learnersourcing~\cite{weir2015learnersourcing} activity. The system could connect with novice programmers at the appropriate level and give them an automated task to implement the candidate plan. Based on the correctness of this activity, it could provide a metric for usability.





% \subsection{Plan identification in new domains}
% To date, most of the patterns identified have been restricted to a CS1 context. From our interviews, we know about some efforts that are being taken to identify plans in domains other than CS1 including CS0 and advanced level concepts like software testing and data structures. To move towards identifying more patterns in new domains, we investigate about the process that researchers could employ to identify these plans.

% All of our participants believed that to identify patterns in a domain unfamiliar to oneself, one has to first understand the typical structures %find words for this
% or the typical ways of doing things in that domain.
% One of our researchers argued that ,
% \begin{quote}
%     You do things for 'reasons' that you haven't thoroughly examined and that may have negative consequences. (P10)
% \end{quote}
% Thus familiarizing oneself with the domain is a meta requirement. 

%  TEXT THAT NEEDS TO BE MOVED


% \textbf{Ambiguous Plan Definitions: } More than half of our interviewees faced challenges while identifying patterns because of ambiguous plan definitions which lead to uncertainty. 

% P5 revealed that \textit{``There's not really any universal agreement''} in the definition of patterns. This also leads to misunderstanding in the form of different interpretations of any published material. This perplexity has lead to an absence of a \textit{``pattern of finding patterns.''} (P9)

% One of our researchers also confirmed that knowing about the concept of "programming patterns" consciously helped them incorporate this style of instruction for more advanced concepts. \textit{``Now that I sort of know patterns are a thing, I will probably be more aggressively on the lookout for, I'm teaching topics X, Y, and Z. Are any of them amenable to this approach?''} (P7)