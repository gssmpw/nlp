%%
%% This is file `sample-manuscript.tex',
%% generated with the docstrip utility.
%%
%% The original source files were:
%%
%% samples.dtx  (with options: `manuscript')
%% 
%% IMPORTANT NOTICE:
%% 
%% For the copyright see the source file.
%% 
%% Any modified versions of this file must be renamed
%% with new filenames distinct from sample-manuscript.tex.
%% 
%% For distribution of the original source see the terms
%% for copying and modification in the file samples.dtx.
%% 
%% This generated file may be distributed as long as the
%% original source files, as listed above, are part of the
%% same distribution. (The sources need not necessarily be
%% in the same archive or directory.)
%%
%% Commands for TeXCount
%TC:macro~\cite [option:text,text]
%TC:macro~\citep [option:text,text]
%TC:macro~\citet [option:text,text]
%TC:envir table 0 1
%TC:envir table* 0 1
%TC:envir tabular [ignore] word
%TC:envir displaymath 0 word
%TC:envir math 0 word
%TC:envir comment 0 0
%%
%%
%% The first command in your LaTeX source must be the \documentclass command.
%%%% Small single column format, used for CIE, CSUR, DTRAP, JACM, JDIQ, JEA, JERIC, JETC, PACMCGIT, TAAS, TACCESS, TACO, TALG, TALLIP (formerly TALIP), TCPS, TDSCI, TEAC, TECS, TELO, THRI, TIIS, TIOT, TISSEC, TIST, TKDD, TMIS, TOCE, TOCHI, TOCL, TOCS, TOCT, TODAES, TODS, TOIS, TOIT, TOMACS, TOMM (formerly TOMCCAP), TOMPECS, TOMS, TOPC, TOPLAS, TOPS, TOS, TOSEM, TOSN, TQC, TRETS, TSAS, TSC, TSLP, TWEB.
% \documentclass[acmsmall]{acmart}

%%%% Large single column format, used for IMWUT, JOCCH, PACMPL, POMACS, TAP, PACMHCI
% \documentclass[acmlarge,screen]{acmart}

%%%% Large double column format, used for TOG
% \documentclass[acmtog, authorversion]{acmart}

%%%% Generic manuscript mode, required for submission
%%%% and peer review
% \documentclass[manuscript,review,anonymous]{acmart}
% \documentclass[authorversion]{acmart}
\documentclass[sigconf, screen]{acmart}
\setcopyright{none}

\usepackage{xcolor}  % Include this line in the preamble
% \usepackage{subfig}
\usepackage{listings}

\lstset{
  breaklines=true,            % Enable breaking long lines
  breakindent=0pt,            % Ensure no indentation after line
  basicstyle=\ttfamily,       % Set font for listings
  frame=single,               % Add a frame around the listings
  breakautoindent=false,      % Disable automatic indentation after
  prebreak={}                 % No special character or space before
}

% \newcommand{\edit}[1]{\textcolor{orange}{#1}}


% Extra packages
\usepackage{multirow}
\usepackage{makecell}
\usepackage{cleveref}
\usepackage{float}
\crefname{figure}{Figure}{}
\crefname{section}{Section}{}
\usepackage{soul} %for better underlining
\usepackage{graphicx}
\usepackage{subcaption}


%% Fonts used in the template cannot be substituted; margin 
%% adjustments are not allowed.
%%
%% \BibTeX command to typeset BibTeX logo in the docs
\AtBeginDocument{%
  \providecommand\BibTeX{{%
    \normalfont B\kern-0.5em{\scshape i\kern-0.25em b}\kern-0.8em\TeX}}}

%% Rights management information.  This information is sent to you
%% when you complete the rights form.  These commands have SAMPLE
%% values in them; it is your responsibility as an author to replace
%% the commands and values with those provided to you when you
%% complete the rights form.
% \setcopyright{acmcopyright}
% \copyrightyear{2025}
% \acmYear{2025}
% \acmDOI{XXXXXXX.XXXXXXX}

% %% These commands are for a PROCEEDINGS abstract or paper.
% \acmConference[CHI '25]{CHI Conference on Human Factors in Computing Systems}{April 26--May 1, 2025}{Yokohama, Japan}
% %
% %  Uncomment \acmBooktitle if th title of the proceedings is different
% %  from ``Proceedings of ...''!
% %
% \acmBooktitle{CHI '25: CHI Conference on Human Factors in Computing Systems,
%  April 26--May 1, 2025, Yokohama, Japan} 
% \acmPrice{15.00}
% \acmISBN{978-1-4503-XXXX-X/18/06}

% \copyrightyear{2025}
% \acmYear{2025}
% \setcopyright{cc}
% \setcopyright{none}
% \setcctype{by}
\acmConference[CHI '25]{CHI Conference on Human Factors in Computing 
Systems}{April 26-May 1, 2025}{Yokohama, Japan}
% \acmBooktitle{CHI Conference on Human Factors in Computing Systems (CHI 
% '25), April 26-May 1, 2025, Yokohama, 
% Japan}
\acmDOI{10.1145/3706598.3713832}
\acmISBN{979-8-4007-1394-1/25/04}



%%
%% Submission ID.
%% Use this when submitting an article to a sponsored event. You'll
%% receive a unique submission ID from the organizers
%% of the event, and this ID should be used as the parameter to this command.
%%\acmSubmissionID{123-A56-BU3}

%%
%% For managing citations, it is recommended to use bibliography
%% files in BibTeX format.
%%
%% You can then either use BibTeX with the ACM-Reference-Format style,
%% or BibLaTeX with the acmnumeric or acmauthoryear sytles, that include
%% support for advanced citation of software artefact from the
%% biblatex-software package, also separately available on CTAN.
%%
%% Look at the sample-*-biblatex.tex files for templates showcasing
%% the biblatex styles.
%%

%%
%% The majority of ACM publications use numbered citations and
%% references.  The command~\citestyle{authoryear} switches to the
%% "author year" style.
%%
%% If you are preparing content for an event
%% sponsored by ACM SIGGRAPH, you must use the "author year" style of
%% citations and references.
%% Uncommenting
%% the next command will enable that style.
%%\citestyle{acmauthoryear}

%%
%% end of the preamble, start of the body of the document source.
\begin{document}

%%
%% The "title" command has an optional parameter,
%% allowing the author to define a "short title" to be used in page headers.
\title{PLAID: Supporting Computing Instructors to Identify Domain-Specific Programming Plans at Scale} %: Needs, Challenges, and Solutions}
%\title{The Process of Programming Plan Identification and Its Challenges}

\author{Yoshee Jain}
\authornote{Indicates that both authors contributed equally to the paper.}
\orcid{0009-0005-6291-5438}
\affiliation{%
  % \department{Siebel School of Computing \\ and Data Science}
  \institution{University of Illinois Urbana-Champaign}
  % \streetaddress{201 N. Goodwin Ave}
  \city{Urbana}
  \state{Illinois}
  \country{USA}
}
\email{yosheej2@illinois.edu}

\author{Mehmet Arif Demirta\c{s}}
\authornotemark[1]
\orcid{0000-0001-5674-5878}
\affiliation{%
  %\department{Siebel School of \\ Computing and Data Science}
  \institution{University of Illinois Urbana-Champaign}
  % \streetaddress{201 N. Goodwin Ave}
  \city{Urbana}
  \state{Illinois}
  \country{USA}
}
\email{mad16@illinois.edu}

\author{Kathryn Cunningham}
\orcid{0000-0002-9702-2796}
\affiliation{%
  % \department{Siebel School of Computing and \\ Data Science}
  \institution{University of Illinois Urbana-Champaign}
  % \streetaddress{201 N. Goodwin Ave}
  \city{Urbana}
  \state{Illinois}
  \country{USA}
}
\email{katcun@illinois.edu}

%%
%% The "author" command and its associated commands are used to define
%% the authors and their affiliations.
%% Of note is the shared affiliation of the first two authors, and the
%% "authornote" and "authornotemark" commands
%% used to denote shared contribution to the research.

%%
%% By default, the full list of authors will be used in the page
%% headers. Often, this list is too long, and will overlap
%% other information printed in the page headers. This command allows
%% the author to define a more concise list
%% of authors' names for this purpose.
\renewcommand{\shortauthors}{Yoshee Jain, Mehmet Arif Demirta\c{s}, \& Kathryn Cunningham}

%%%portion for highlighting edits
% \definecolor{editCol}{RGB}{255,140,0}
\definecolor{editCol}{RGB}{0,0,0}
% \definecolor{editCol}{RGB}{0,0,0}
\newcommand{\edit}[1]{{\textcolor{editCol}{#1}}}

%%
%% The abstract is a short summary of the work to be presented in the
%% article.
\begin{abstract}
Pedagogical approaches focusing on stereotypical code solutions, known as programming plans, can increase problem-solving ability and motivate diverse learners. However, plan-focused pedagogies are rarely used beyond introductory programming. Our formative study (N=10 educators) showed that identifying plans is a tedious process. To advance plan-focused pedagogies in application-focused domains, we created an LLM-powered pipeline that automates the effortful parts of educators' plan identification process by providing use-case-driven program examples and candidate plans. In design workshops (N=7 educators), we identified design goals to maximize instructors' efficiency in plan identification by optimizing interaction with this LLM-generated content. Our resulting tool, PLAID, enables instructors to access a corpus of relevant programs to inspire plan identification, compare code snippets to assist plan refinement, and facilitates them in structuring code snippets into plans. We evaluated PLAID in a within-subjects user study (N=12 educators) and found that PLAID led to lower cognitive demand and increased productivity compared to the state-of-the-art. Educators found PLAID beneficial for generating instructional material. Thus, our findings suggest that human-in-the-loop approaches hold promise for supporting plan-focused pedagogies at scale.



% % Programming plans are important for programming ability
% Knowledge of programming plans, which are stereotypical code patterns that achieve a goal, is key to programmers' ability to write programs.
% % Programming ability may be improved with explicit instruction
% When computing educators are armed with a set of programming plans, they can take advantage of instructional techniques that may accelerate their students' learning.
% % Explicit instruction requires sets of plans
% However, plan identification is an effortful process that is not well documented in prior work. 
% % 
% Moreover, existing sets of plans are primarily drawn from introductory programming content, so plan-based instructional techniques are not yet possible for many programming application domains, like web scraping, data analysis, web development, or machine learning model creation.
% % Through interviews with ten computing educators who have identified novel plans
% With our formative study, we contribute a comprehensive understanding of the current plan identification process, including the plan components instructors value, the characteristics by which plans are judged, and critical challenges.
% Using a design workshop with seven instructors, we present key design guidelines for developing systems that support instructors' plan identification process. 
% Applying these design guidelines, we designed <should we name the system?> that presents instructors with complete contextualized programs as reference material.
% <name of the system> supports various interactions, enabling educators to use reference material effectively, understand domain-specific content, gather inspiration from given material, and organize information into plans.
% Our within-subjects evaluation study demonstrates...


% an LLM-supported plan identification pipeline and evaluate it in comparison to a set of expert-identified plans in the web scraping domain. 
% Our work provides important implications for automating plan identification, including where and how processes may be automated, and how these processes can be augmented with an instructor in the loop.

% When instructors know the common patterns programmers use to write code in their topic of expertise, they can design their teaching in more fruitful ways. However, these programming patterns or programming plans have mostly been identified in the context of CS1 problems, so our community does not yet understand code patterns in most application-focused or advanced programming domains. In order for instructors to develop pattern-focused instruction across more topics in computer science, more programming patterns need to be identified across a wider number of topic areas. 
% The ability of CS Educators to describe programming plans is hampered by the fact that the process of programming pattern identification is not well-understood. While several researchers have cataloged programming patterns, they rarely share their methodology. Better understanding of this process is not only useful for others to repeat programming plan identification by hand, but also to inform the design of sociotechnical systems that support instructors to identify programming plans more efficiently, effectively, and in new domains. 
% We interviewed 10 CS Education researchers who identified programming plans in order to describe the process of plan identification in specific detail. We list typical components of programming plans, metrics by which plans are judged, procedures used to identify plans, and the types of expertise these pattern identifiers brought to bear. We also provide implications for the design of systems for improved plan identification, including where and how processes may be automated, and which procedures still require an instructor ``in the loop''.

%To move towards an effective process for domain-specific plan identification, we propose metrics to describe the quality of domain-specific plans and detail %three 
%techniques for discovering them. 
%We hope this poster encourages discussion around approaches for identifying domain-specific programming plans, as well as how to measure success in such efforts. 


%%%%% Alternative abstract proposal
% 
%\\\\
% % Programming plans are important for programming ability
% Knowledge of programming plans, which are stereotypical code patterns that achieve a goal, is key to programmers' ability to write programs.
% % Programming ability may be improved with plan focused instruction
% When computing educators are provided with a set of programming plans for teaching, they can take advantage of instructional techniques that may accelerate their students' learning.
% % Plan-focused instruction cannot be applied beyond introductory
% However, programming plans beyond introductory programming have not been curated in prior research efforts.
% % Our effort carries this beyond introduction
% Our work unveils the process of identifying programming plans and scaffolds the experience of instructors in this process through a novel design artifact.
% % Through interviews with ten computing educators who have identified novel plans
% We conduct interviews with 10 instructors who have curated introductory plans in prior work, uncovering their process through a set of values and challenges.
% Informed by our interviews, we create a set of design considerations through workshops with 7 participants from 2 application-specific domains to imagine how the plan design process should be structured beyond introductory programming.
% Our tool, PLAID, combines insights from interviews with design considerations from workshops to empower instructors with domain knowledge to design instructional content for novice learners. 
% PLAID allows instructors to access many relevant programs at once, compare program pieces easily, and modify programs to structure them into plans.
% We evaluate PLAID in a within-subjects user study with 12 participants from 3 application-specific domains (machine learning with Pytorch, data analysis with Pandas, and web development with Django). 
% We show that participants experienced lower cognitive demand and were more productive using PLAID. Moreover, instructors found PLAID to be beneficial for generating instructional material.

% \\\\
% Knowledge of programming plans, which are stereotypical code patterns that achieve a goal, is key to programmers' ability to write programs. When computing educators are equipped with a set of programming plans for teaching, they can take advantage of instructional techniques that may accelerate their students' learning. However, programming plans have been primary identified in introductory programming. The process of plan identification is not well documented in prior work yielding to the lack of plans beyond introductory programming. Our work unveils the process of identifying programming plans and scaffolds the experience of instructors in this process through a novel design artifact.
\end{abstract}

%%
%% The code below is generated by the tool at http://dl.acm.org/ccs.cfm.
%% Please copy and paste the code instead of the example below.
%%
\begin{CCSXML}
<ccs2012>
   <concept>
       <concept_id>10003456.10003457.10003527</concept_id>
       <concept_desc>Social and professional topics~Computing education</concept_desc>
       <concept_significance>500</concept_significance>
       </concept>
   <concept>
       <concept_id>10003456.10003457.10003527.10003531.10003533</concept_id>
       <concept_desc>Social and professional topics~Computer science education</concept_desc>
       <concept_significance>500</concept_significance>
       </concept>
 </ccs2012>
\end{CCSXML}

\ccsdesc[500]{Social and professional topics~Computing education}
\ccsdesc[500]{Social and professional topics~Computer science education}

%%
%% Keywords. The author(s) should pick words that accurately describe
%% the work being presented. Separate the keywords with commas.
\keywords{programming plan, programming pattern, pattern identification, instructor support}

%% A "teaser" image appears between the author and affiliation
%% information and the body of the document, and typically spans the
%% page.
\begin{teaserfigure}
\centering
\Description{A three-stage figure showing how PLAID can be utilized. On the left, an instructor is looking at a set of LLM-generated programs and editing parts of the program to identify similarities. An arrow saying ``Identify programming parts in application focused-domains'' connects this to middle part, where multiple plans are shown as short code snippets with annotations. An arrow saying ``Support plan-focused pedagogies in that domain'' connects this to the right part, which shows different use cases for plans. These include `Organize course content', `Create examples', and `Build assessments'.}
\includegraphics[width=0.84\textwidth]{img/teaser-new.png}
\caption{PLAID supports instructors to more efficiently identify programming plans (i.e., common code patterns and information about their use) in application-focused programming domains by supporting their ability to explore and refine AI-generated content. Plan identification is a crucial step in the development of promising plan-based pedagogies. 
%With PLAID, instructors explore example programs generated and annotated with AI, refine suggested content into programming plans, and organize plans in preparation for supporting students.
}
\label{fig:teaser-why-plaid-matters}
\end{teaserfigure}
% \begin{teaserfigure}
%   \centering
%   \includegraphics[width=\textwidth]{img/example_plans_teaser.png}
%   \centering
%   \caption{Programming plans that educators have already identified from a variety of programming applications, including (from left to right) web scraping with Beautiful Soup~\cite{Cunningham_PurposeFirstProgramming_CHI-2021}, introductory object-oriented programming~\cite{Iyer_PatternCensus_SIGCSE-2021}, introductory procedural programming~\cite{Wermelinger_problems-to-programs}, and testing with RSpec~\cite{LojoFox_TestingPlan_ITiCSE-2022}.}
%   %, code patterns from a domain area are identified and annotated with information that highlights their purpose. These patterns are used to develop a curriculum where learners study plans, write code, debug code, and understand code.}
%   \Description{This diagram shows 4 examples of different plans that we observed in our literature review. It includes the web scraping plan that helps get a soup from multiple web-pages, the software engineering plan that expresses encapsulation in object-oriented programming, the introductory programming plan that finds a value of the list that satisfies a certain condition and the software testing plan that symbolizes arrange-act-assert testing.}
%   \label{fig:teaser}
% \end{teaserfigure}

%%
%% This command processes the author and affiliation and title
%% information and builds the first part of the formatted document.
\maketitle

\section{Introduction}


\begin{figure}[t]
\centering
\includegraphics[width=0.6\columnwidth]{figures/evaluation_desiderata_V5.pdf}
\vspace{-0.5cm}
\caption{\systemName is a platform for conducting realistic evaluations of code LLMs, collecting human preferences of coding models with real users, real tasks, and in realistic environments, aimed at addressing the limitations of existing evaluations.
}
\label{fig:motivation}
\end{figure}

\begin{figure*}[t]
\centering
\includegraphics[width=\textwidth]{figures/system_design_v2.png}
\caption{We introduce \systemName, a VSCode extension to collect human preferences of code directly in a developer's IDE. \systemName enables developers to use code completions from various models. The system comprises a) the interface in the user's IDE which presents paired completions to users (left), b) a sampling strategy that picks model pairs to reduce latency (right, top), and c) a prompting scheme that allows diverse LLMs to perform code completions with high fidelity.
Users can select between the top completion (green box) using \texttt{tab} or the bottom completion (blue box) using \texttt{shift+tab}.}
\label{fig:overview}
\end{figure*}

As model capabilities improve, large language models (LLMs) are increasingly integrated into user environments and workflows.
For example, software developers code with AI in integrated developer environments (IDEs)~\citep{peng2023impact}, doctors rely on notes generated through ambient listening~\citep{oberst2024science}, and lawyers consider case evidence identified by electronic discovery systems~\citep{yang2024beyond}.
Increasing deployment of models in productivity tools demands evaluation that more closely reflects real-world circumstances~\citep{hutchinson2022evaluation, saxon2024benchmarks, kapoor2024ai}.
While newer benchmarks and live platforms incorporate human feedback to capture real-world usage, they almost exclusively focus on evaluating LLMs in chat conversations~\citep{zheng2023judging,dubois2023alpacafarm,chiang2024chatbot, kirk2024the}.
Model evaluation must move beyond chat-based interactions and into specialized user environments.



 

In this work, we focus on evaluating LLM-based coding assistants. 
Despite the popularity of these tools---millions of developers use Github Copilot~\citep{Copilot}---existing
evaluations of the coding capabilities of new models exhibit multiple limitations (Figure~\ref{fig:motivation}, bottom).
Traditional ML benchmarks evaluate LLM capabilities by measuring how well a model can complete static, interview-style coding tasks~\citep{chen2021evaluating,austin2021program,jain2024livecodebench, white2024livebench} and lack \emph{real users}. 
User studies recruit real users to evaluate the effectiveness of LLMs as coding assistants, but are often limited to simple programming tasks as opposed to \emph{real tasks}~\citep{vaithilingam2022expectation,ross2023programmer, mozannar2024realhumaneval}.
Recent efforts to collect human feedback such as Chatbot Arena~\citep{chiang2024chatbot} are still removed from a \emph{realistic environment}, resulting in users and data that deviate from typical software development processes.
We introduce \systemName to address these limitations (Figure~\ref{fig:motivation}, top), and we describe our three main contributions below.


\textbf{We deploy \systemName in-the-wild to collect human preferences on code.} 
\systemName is a Visual Studio Code extension, collecting preferences directly in a developer's IDE within their actual workflow (Figure~\ref{fig:overview}).
\systemName provides developers with code completions, akin to the type of support provided by Github Copilot~\citep{Copilot}. 
Over the past 3 months, \systemName has served over~\completions suggestions from 10 state-of-the-art LLMs, 
gathering \sampleCount~votes from \userCount~users.
To collect user preferences,
\systemName presents a novel interface that shows users paired code completions from two different LLMs, which are determined based on a sampling strategy that aims to 
mitigate latency while preserving coverage across model comparisons.
Additionally, we devise a prompting scheme that allows a diverse set of models to perform code completions with high fidelity.
See Section~\ref{sec:system} and Section~\ref{sec:deployment} for details about system design and deployment respectively.



\textbf{We construct a leaderboard of user preferences and find notable differences from existing static benchmarks and human preference leaderboards.}
In general, we observe that smaller models seem to overperform in static benchmarks compared to our leaderboard, while performance among larger models is mixed (Section~\ref{sec:leaderboard_calculation}).
We attribute these differences to the fact that \systemName is exposed to users and tasks that differ drastically from code evaluations in the past. 
Our data spans 103 programming languages and 24 natural languages as well as a variety of real-world applications and code structures, while static benchmarks tend to focus on a specific programming and natural language and task (e.g. coding competition problems).
Additionally, while all of \systemName interactions contain code contexts and the majority involve infilling tasks, a much smaller fraction of Chatbot Arena's coding tasks contain code context, with infilling tasks appearing even more rarely. 
We analyze our data in depth in Section~\ref{subsec:comparison}.



\textbf{We derive new insights into user preferences of code by analyzing \systemName's diverse and distinct data distribution.}
We compare user preferences across different stratifications of input data (e.g., common versus rare languages) and observe which affect observed preferences most (Section~\ref{sec:analysis}).
For example, while user preferences stay relatively consistent across various programming languages, they differ drastically between different task categories (e.g. frontend/backend versus algorithm design).
We also observe variations in user preference due to different features related to code structure 
(e.g., context length and completion patterns).
We open-source \systemName and release a curated subset of code contexts.
Altogether, our results highlight the necessity of model evaluation in realistic and domain-specific settings.






\section{Background}\label{sec:backgrnd}

\subsection{Cold Start Latency and Mitigation Techniques}

Traditional FaaS platforms mitigate cold starts through snapshotting, lightweight virtualization, and warm-state management. Snapshot-based methods like \textbf{REAP} and \textbf{Catalyzer} reduce initialization time by preloading or restoring container states but require significant memory and I/O resources, limiting scalability~\cite{dong_catalyzer_2020, ustiugov_benchmarking_2021}. Lightweight virtualization solutions, such as \textbf{Firecracker} microVMs, achieve fast startup times with strong isolation but depend on robust infrastructure, making them less adaptable to fluctuating workloads~\cite{agache_firecracker_2020}. Warm-state management techniques like \textbf{Faa\$T}~\cite{romero_faa_2021} and \textbf{Kraken}~\cite{vivek_kraken_2021} keep frequently invoked containers ready, balancing readiness and cost efficiency under predictable workloads but incurring overhead when demand is erratic~\cite{romero_faa_2021, vivek_kraken_2021}. While these methods perform well in resource-rich cloud environments, their resource intensity challenges applicability in edge settings.

\subsubsection{Edge FaaS Perspective}

In edge environments, cold start mitigation emphasizes lightweight designs, resource sharing, and hybrid task distribution. Lightweight execution environments like unikernels~\cite{edward_sock_2018} and \textbf{Firecracker}~\cite{agache_firecracker_2020}, as used by \textbf{TinyFaaS}~\cite{pfandzelter_tinyfaas_2020}, minimize resource usage and initialization delays but require careful orchestration to avoid resource contention. Function co-location, demonstrated by \textbf{Photons}~\cite{v_dukic_photons_2020}, reduces redundant initializations by sharing runtime resources among related functions, though this complicates isolation in multi-tenant setups~\cite{v_dukic_photons_2020}. Hybrid offloading frameworks like \textbf{GeoFaaS}~\cite{malekabbasi_geofaas_2024} balance edge-cloud workloads by offloading latency-tolerant tasks to the cloud and reserving edge resources for real-time operations, requiring reliable connectivity and efficient task management. These edge-specific strategies address cold starts effectively but introduce challenges in scalability and orchestration.

\subsection{Predictive Scaling and Caching Techniques}

Efficient resource allocation is vital for maintaining low latency and high availability in serverless platforms. Predictive scaling and caching techniques dynamically provision resources and reduce cold start latency by leveraging workload prediction and state retention.
Traditional FaaS platforms use predictive scaling and caching to optimize resources, employing techniques (OFC, FaasCache) to reduce cold starts. However, these methods rely on centralized orchestration and workload predictability, limiting their effectiveness in dynamic, resource-constrained edge environments.



\subsubsection{Edge FaaS Perspective}

Edge FaaS platforms adapt predictive scaling and caching techniques to constrain resources and heterogeneous environments. \textbf{EDGE-Cache}~\cite{kim_delay-aware_2022} uses traffic profiling to selectively retain high-priority functions, reducing memory overhead while maintaining readiness for frequent requests. Hybrid frameworks like \textbf{GeoFaaS}~\cite{malekabbasi_geofaas_2024} implement distributed caching to balance resources between edge and cloud nodes, enabling low-latency processing for critical tasks while offloading less critical workloads. Machine learning methods, such as clustering-based workload predictors~\cite{gao_machine_2020} and GRU-based models~\cite{guo_applying_2018}, enhance resource provisioning in edge systems by efficiently forecasting workload spikes. These innovations effectively address cold start challenges in edge environments, though their dependency on accurate predictions and robust orchestration poses scalability challenges.

\subsection{Decentralized Orchestration, Function Placement, and Scheduling}

Efficient orchestration in serverless platforms involves workload distribution, resource optimization, and performance assurance. While traditional FaaS platforms rely on centralized control, edge environments require decentralized and adaptive strategies to address unique challenges such as resource constraints and heterogeneous hardware.



\subsubsection{Edge FaaS Perspective}

Edge FaaS platforms adopt decentralized and adaptive orchestration frameworks to meet the demands of resource-constrained environments. Systems like \textbf{Wukong} distribute scheduling across edge nodes, enhancing data locality and scalability while reducing network latency. Lightweight frameworks such as \textbf{OpenWhisk Lite}~\cite{kravchenko_kpavelopenwhisk-light_2024} optimize resource allocation by decentralizing scheduling policies, minimizing cold starts and latency in edge setups~\cite{benjamin_wukong_2020}. Hybrid solutions like \textbf{OpenFaaS}~\cite{noauthor_openfaasfaas_2024} and \textbf{EdgeMatrix}~\cite{shen_edgematrix_2023} combine edge-cloud orchestration to balance resource utilization, retaining latency-sensitive functions at the edge while offloading non-critical workloads to the cloud. While these approaches improve flexibility, they face challenges in maintaining coordination and ensuring consistent performance across distributed nodes.



\section{Overview}

\revision{In this section, we first explain the foundational concept of Hausdorff distance-based penetration depth algorithms, which are essential for understanding our method (Sec.~\ref{sec:preliminary}).
We then provide a brief overview of our proposed RT-based penetration depth algorithm (Sec.~\ref{subsec:algo_overview}).}



\section{Preliminaries }
\label{sec:Preliminaries}

% Before we introduce our method, we first overview the important basics of 3D dynamic human modeling with Gaussian splatting. Then, we discuss the diffusion-based 3d generation techniques, and how they can be applied to human modeling.
% \ZY{I stopp here. TBC.}
% \subsection{Dynamic human modeling with Gaussian splatting}
\subsection{3D Gaussian Splatting}
3D Gaussian splatting~\cite{kerbl3Dgaussians} is an explicit scene representation that allows high-quality real-time rendering. The given scene is represented by a set of static 3D Gaussians, which are parameterized as follows: Gaussian center $x\in {\mathbb{R}^3}$, color $c\in {\mathbb{R}^3}$, opacity $\alpha\in {\mathbb{R}}$, spatial rotation in the form of quaternion $q\in {\mathbb{R}^4}$, and scaling factor $s\in {\mathbb{R}^3}$. Given these properties, the rendering process is represented as:
\begin{equation}
  I = Splatting(x, c, s, \alpha, q, r),
  \label{eq:splattingGA}
\end{equation}
where $I$ is the rendered image, $r$ is a set of query rays crossing the scene, and $Splatting(\cdot)$ is a differentiable rendering process. We refer readers to Kerbl et al.'s paper~\cite{kerbl3Dgaussians} for the details of Gaussian splatting. 



% \ZY{I would suggest move this part to the method part.}
% GaissianAvatar is a dynamic human generation model based on Gaussian splitting. Given a sequence of RGB images, this method utilizes fitted SMPLs and sampled points on its surface to obtain a pose-dependent feature map by a pose encoder. The pose-dependent features and a geometry feature are fed in a Gaussian decoder, which is employed to establish a functional mapping from the underlying geometry of the human form to diverse attributes of 3D Gaussians on the canonical surfaces. The parameter prediction process is articulated as follows:
% \begin{equation}
%   (\Delta x,c,s)=G_{\theta}(S+P),
%   \label{eq:gaussiandecoder}
% \end{equation}
%  where $G_{\theta}$ represents the Gaussian decoder, and $(S+P)$ is the multiplication of geometry feature S and pose feature P. Instead of optimizing all attributes of Gaussian, this decoder predicts 3D positional offset $\Delta{x} \in {\mathbb{R}^3}$, color $c\in\mathbb{R}^3$, and 3D scaling factor $ s\in\mathbb{R}^3$. To enhance geometry reconstruction accuracy, the opacity $\alpha$ and 3D rotation $q$ are set to fixed values of $1$ and $(1,0,0,0)$ respectively.
 
%  To render the canonical avatar in observation space, we seamlessly combine the Linear Blend Skinning function with the Gaussian Splatting~\cite{kerbl3Dgaussians} rendering process: 
% \begin{equation}
%   I_{\theta}=Splatting(x_o,Q,d),
%   \label{eq:splatting}
% \end{equation}
% \begin{equation}
%   x_o = T_{lbs}(x_c,p,w),
%   \label{eq:LBS}
% \end{equation}
% where $I_{\theta}$ represents the final rendered image, and the canonical Gaussian position $x_c$ is the sum of the initial position $x$ and the predicted offset $\Delta x$. The LBS function $T_{lbs}$ applies the SMPL skeleton pose $p$ and blending weights $w$ to deform $x_c$ into observation space as $x_o$. $Q$ denotes the remaining attributes of the Gaussians. With the rendering process, they can now reposition these canonical 3D Gaussians into the observation space.



\subsection{Score Distillation Sampling}
Score Distillation Sampling (SDS)~\cite{poole2022dreamfusion} builds a bridge between diffusion models and 3D representations. In SDS, the noised input is denoised in one time-step, and the difference between added noise and predicted noise is considered SDS loss, expressed as:

% \begin{equation}
%   \mathcal{L}_{SDS}(I_{\Phi}) \triangleq E_{t,\epsilon}[w(t)(\epsilon_{\phi}(z_t,y,t)-\epsilon)\frac{\partial I_{\Phi}}{\partial\Phi}],
%   \label{eq:SDSObserv}
% \end{equation}
\begin{equation}
    \mathcal{L}_{\text{SDS}}(I_{\Phi}) \triangleq \mathbb{E}_{t,\epsilon} \left[ w(t) \left( \epsilon_{\phi}(z_t, y, t) - \epsilon \right) \frac{\partial I_{\Phi}}{\partial \Phi} \right],
  \label{eq:SDSObservGA}
\end{equation}
where the input $I_{\Phi}$ represents a rendered image from a 3D representation, such as 3D Gaussians, with optimizable parameters $\Phi$. $\epsilon_{\phi}$ corresponds to the predicted noise of diffusion networks, which is produced by incorporating the noise image $z_t$ as input and conditioning it with a text or image $y$ at timestep $t$. The noise image $z_t$ is derived by introducing noise $\epsilon$ into $I_{\Phi}$ at timestep $t$. The loss is weighted by the diffusion scheduler $w(t)$. 
% \vspace{-3mm}

\subsection{Overview of the RTPD Algorithm}\label{subsec:algo_overview}
Fig.~\ref{fig:Overview} presents an overview of our RTPD algorithm.
It is grounded in the Hausdorff distance-based penetration depth calculation method (Sec.~\ref{sec:preliminary}).
%, similar to that of Tang et al.~\shortcite{SIG09HIST}.
The process consists of two primary phases: penetration surface extraction and Hausdorff distance calculation.
We leverage the RTX platform's capabilities to accelerate both of these steps.

\begin{figure*}[t]
    \centering
    \includegraphics[width=0.8\textwidth]{Image/overview.pdf}
    \caption{The overview of RT-based penetration depth calculation algorithm overview}
    \label{fig:Overview}
\end{figure*}

The penetration surface extraction phase focuses on identifying the overlapped region between two objects.
\revision{The penetration surface is defined as a set of polygons from one object, where at least one of its vertices lies within the other object. 
Note that in our work, we focus on triangles rather than general polygons, as they are processed most efficiently on the RTX platform.}
To facilitate this extraction, we introduce a ray-tracing-based \revision{Point-in-Polyhedron} test (RT-PIP), significantly accelerated through the use of RT cores (Sec.~\ref{sec:RT-PIP}).
This test capitalizes on the ray-surface intersection capabilities of the RTX platform.
%
Initially, a Geometry Acceleration Structure (GAS) is generated for each object, as required by the RTX platform.
The RT-PIP module takes the GAS of one object (e.g., $GAS_{A}$) and the point set of the other object (e.g., $P_{B}$).
It outputs a set of points (e.g., $P_{\partial B}$) representing the penetration region, indicating their location inside the opposing object.
Subsequently, a penetration surface (e.g., $\partial B$) is constructed using this point set (e.g., $P_{\partial B}$) (Sec.~\ref{subsec:surfaceGen}).
%
The generated penetration surfaces (e.g., $\partial A$ and $\partial B$) are then forwarded to the next step. 

The Hausdorff distance calculation phase utilizes the ray-surface intersection test of the RTX platform (Sec.~\ref{sec:RT-Hausdorff}) to compute the Hausdorff distance between two objects.
We introduce a novel Ray-Tracing-based Hausdorff DISTance algorithm, RT-HDIST.
It begins by generating GAS for the two penetration surfaces, $P_{\partial A}$ and $P_{\partial B}$, derived from the preceding step.
RT-HDIST processes the GAS of a penetration surface (e.g., $GAS_{\partial A}$) alongside the point set of the other penetration surface (e.g., $P_{\partial B}$) to compute the penetration depth between them.
The algorithm operates bidirectionally, considering both directions ($\partial A \to \partial B$ and $\partial B \to \partial A$).
The final penetration depth between the two objects, A and B, is determined by selecting the larger value from these two directional computations.

%In the Hausdorff distance calculation step, we compute the Hausdorff distance between given two objects using a ray-surface-intersection test. (Sec.~\ref{sec:RT-Hausdorff}) Initially, we construct the GAS for both $\partial A$ and $\partial B$ to utilize the RT-core effectively. The RT-based Hausdorff distance algorithms then determine the Hausdorff distance by processing the GAS of one object (e.g. $GAS_{\partial A}$) and set of the vertices of the other (e.g. $P_{\partial B}$). Following the Hausdorff distance definition (Eq.~\ref{equation:hausdorff_definition}), we compute the Hausdorff distance to both directions ($\partial A \to \partial B$) and ($\partial B \to \partial A$). As a result, the bigger one is the final Hausdorff distance, and also it is the penetration depth between input object $A$ and $B$.


%the proposed RT-based penetration depth calculation pipeline.
%Our proposed methods adopt Tang's Hausdorff-based penetration depth methods~\cite{SIG09HIST}. The pipeline is divided into the penetration surface extraction step and the Hausdorff distance calculation between the penetration surface steps. However, since Tang's approach is not suitable for the RT platform in detail, we modified and applied it with appropriate methods.

%The penetration surface extraction step is extracting overlapped surfaces on other objects. To utilize the RT core, we use the ray-intersection-based PIP(Point-In-Polygon) algorithms instead of collision detection between two objects which Tang et al.~\cite{SIG09HIST} used. (Sec.~\ref{sec:RT-PIP})
%RT core-based PIP test uses a ray-surface intersection test. For purpose this, we generate the GAS(Geometry Acceleration Structure) for each object. RT core-based PIP test takes the GAS of one object (e.g. $GAS_{A}$) and a set of vertex of another one (e.g. $P_{B}$). Then this computes the penetrated vertex set of another one (e.g. $P_{\partial B}$). To calculate the Hausdorff distance, these vertex sets change to objects constructed by penetrated surface (e.g. $\partial B$). Finally, the two generated overlapped surface objects $\partial A$ and $\partial B$ are used in the Hausdorff distance calculation step.

\section{Problem Statement} \label{sec: statement}


\subsection{Deploying GNN locally Causes Vulnerabilities} \label{ps: gnn inference}
Deploying GNNs on local devices requires access to graph data in addition to the trained model, which introduces unique security and privacy challenges. 
Similar to DNN deployment, the IP of the well-performed local model, including its trained weights and biases, is valuable asset that must be protected against model extraction attacks.
Beyond the model IP, local GNN inference raises additional privacy concerns due to the nature of GNN architecture. 
Specifically, during the message-passing phase of GNN inference, target nodes aggregate information from neighboring nodes to update their embeddings. 
This process involves accessing sensitive edge data, such as user-product interactions in recommender systems.
In our work, we will address the GNN IP infringement and edge data breach vulnerability during GNN deployment.

\subsection{Edge Privacy is Valuable} \label{motivation: edge importance}

\begin{figure}[t]
    \centering
    \includegraphics[width=0.95\linewidth]{imgs/scenario.pdf}
    \caption{\textbf{Motivation Example:} Alice (victim) builds a graph of products and trains a GNN RS. She deploys both edge data and RS on local devices. Bob (attacker) accesses this device and steals the edge data and model parameters.}
    \label{fig: problem-statement}
\end{figure}

Membership inference attack is the most common data privacy threat to machine learning models~\cite{shokri2017membership}, where the goal is to determine whether a given data point belongs to the training set. 
However, in the context of GNNs, edge data raises additional privacy concerns. 
Link stealing attacks~\cite{he2021stealing, ding2023vertexserum} aim to infer the connectivity between any pair of given nodes. 
In this work, we focus on the adjacency information (edges), while considering the node features as public.
A real-world example is illustrated in Fig.~\ref{fig: problem-statement}, where Alice (victim) deploys a recommender system (RS) on local edge devices. 
In such a product graph, the node features are public attributes of the products—such as price, user reviews, or categories—that are available to any user. 
However, the internal relationships between products require intensive learning from user behavior data, which is valuable IP for the model vendor. 
Therefore, safeguarding the node connectivity information during GNN local inference is of great importance.


\subsection{TEE Has Memory Restrictions} \label{ps: TEE}
The introduction of TEE greatly enhances data security and privacy with secure compartments. 
However, TEE platforms face significant memory limitations, a critical constraint for secure computation. 
For instance, for Intel SGX trusted enclaves, the physical reserved memory (PRM) is limited to 128MB, with 96 MB of it allocated to the Enclave Page Cache (EPC)~\cite{intel2017sgx}. 
Excessive memory allocation will lead to frequent page swapping between the unprotected main memory and the protected enclave, which can cause high overhead and additional encryption/decryption to ensure data integrity~\cite{costan2016intel}.
This memory constraint poses a significant challenge for deploying GNN models and the entire graph (including node features and adjacency information) within the secure enclave, which often far exceed the PRM limitation of enclaves.



% \section{Discussion of Assumptions}\label{sec:discussion}
In this paper, we have made several assumptions for the sake of clarity and simplicity. In this section, we discuss the rationale behind these assumptions, the extent to which these assumptions hold in practice, and the consequences for our protocol when these assumptions hold.

\subsection{Assumptions on the Demand}

There are two simplifying assumptions we make about the demand. First, we assume the demand at any time is relatively small compared to the channel capacities. Second, we take the demand to be constant over time. We elaborate upon both these points below.

\paragraph{Small demands} The assumption that demands are small relative to channel capacities is made precise in \eqref{eq:large_capacity_assumption}. This assumption simplifies two major aspects of our protocol. First, it largely removes congestion from consideration. In \eqref{eq:primal_problem}, there is no constraint ensuring that total flow in both directions stays below capacity--this is always met. Consequently, there is no Lagrange multiplier for congestion and no congestion pricing; only imbalance penalties apply. In contrast, protocols in \cite{sivaraman2020high, varma2021throughput, wang2024fence} include congestion fees due to explicit congestion constraints. Second, the bound \eqref{eq:large_capacity_assumption} ensures that as long as channels remain balanced, the network can always meet demand, no matter how the demand is routed. Since channels can rebalance when necessary, they never drop transactions. This allows prices and flows to adjust as per the equations in \eqref{eq:algorithm}, which makes it easier to prove the protocol's convergence guarantees. This also preserves the key property that a channel's price remains proportional to net money flow through it.

In practice, payment channel networks are used most often for micro-payments, for which on-chain transactions are prohibitively expensive; large transactions typically take place directly on the blockchain. For example, according to \cite{river2023lightning}, the average channel capacity is roughly $0.1$ BTC ($5,000$ BTC distributed over $50,000$ channels), while the average transaction amount is less than $0.0004$ BTC ($44.7k$ satoshis). Thus, the small demand assumption is not too unrealistic. Additionally, the occasional large transaction can be treated as a sequence of smaller transactions by breaking it into packets and executing each packet serially (as done by \cite{sivaraman2020high}).
Lastly, a good path discovery process that favors large capacity channels over small capacity ones can help ensure that the bound in \eqref{eq:large_capacity_assumption} holds.

\paragraph{Constant demands} 
In this work, we assume that any transacting pair of nodes have a steady transaction demand between them (see Section \ref{sec:transaction_requests}). Making this assumption is necessary to obtain the kind of guarantees that we have presented in this paper. Unless the demand is steady, it is unreasonable to expect that the flows converge to a steady value. Weaker assumptions on the demand lead to weaker guarantees. For example, with the more general setting of stochastic, but i.i.d. demand between any two nodes, \cite{varma2021throughput} shows that the channel queue lengths are bounded in expectation. If the demand can be arbitrary, then it is very hard to get any meaningful performance guarantees; \cite{wang2024fence} shows that even for a single bidirectional channel, the competitive ratio is infinite. Indeed, because a PCN is a decentralized system and decisions must be made based on local information alone, it is difficult for the network to find the optimal detailed balance flow at every time step with a time-varying demand.  With a steady demand, the network can discover the optimal flows in a reasonably short time, as our work shows.

We view the constant demand assumption as an approximation for a more general demand process that could be piece-wise constant, stochastic, or both (see simulations in Figure \ref{fig:five_nodes_variable_demand}).
We believe it should be possible to merge ideas from our work and \cite{varma2021throughput} to provide guarantees in a setting with random demands with arbitrary means. We leave this for future work. In addition, our work suggests that a reasonable method of handling stochastic demands is to queue the transaction requests \textit{at the source node} itself. This queuing action should be viewed in conjunction with flow-control. Indeed, a temporarily high unidirectional demand would raise prices for the sender, incentivizing the sender to stop sending the transactions. If the sender queues the transactions, they can send them later when prices drop. This form of queuing does not require any overhaul of the basic PCN infrastructure and is therefore simpler to implement than per-channel queues as suggested by \cite{sivaraman2020high} and \cite{varma2021throughput}.

\subsection{The Incentive of Channels}
The actions of the channels as prescribed by the DEBT control protocol can be summarized as follows. Channels adjust their prices in proportion to the net flow through them. They rebalance themselves whenever necessary and execute any transaction request that has been made of them. We discuss both these aspects below.

\paragraph{On Prices}
In this work, the exclusive role of channel prices is to ensure that the flows through each channel remains balanced. In practice, it would be important to include other components in a channel's price/fee as well: a congestion price  and an incentive price. The congestion price, as suggested by \cite{varma2021throughput}, would depend on the total flow of transactions through the channel, and would incentivize nodes to balance the load over different paths. The incentive price, which is commonly used in practice \cite{river2023lightning}, is necessary to provide channels with an incentive to serve as an intermediary for different channels. In practice, we expect both these components to be smaller than the imbalance price. Consequently, we expect the behavior of our protocol to be similar to our theoretical results even with these additional prices.

A key aspect of our protocol is that channel fees are allowed to be negative. Although the original Lightning network whitepaper \cite{poon2016bitcoin} suggests that negative channel prices may be a good solution to promote rebalancing, the idea of negative prices in not very popular in the literature. To our knowledge, the only prior work with this feature is \cite{varma2021throughput}. Indeed, in papers such as \cite{van2021merchant} and \cite{wang2024fence}, the price function is explicitly modified such that the channel price is never negative. The results of our paper show the benefits of negative prices. For one, in steady state, equal flows in both directions ensure that a channel doesn't loose any money (the other price components mentioned above ensure that the channel will only gain money). More importantly, negative prices are important to ensure that the protocol selectively stifles acyclic flows while allowing circulations to flow. Indeed, in the example of Section \ref{sec:flow_control_example}, the flows between nodes $A$ and $C$ are left on only because the large positive price over one channel is canceled by the corresponding negative price over the other channel, leading to a net zero price.

Lastly, observe that in the DEBT control protocol, the price charged by a channel does not depend on its capacity. This is a natural consequence of the price being the Lagrange multiplier for the net-zero flow constraint, which also does not depend on the channel capacity. In contrast, in many other works, the imbalance price is normalized by the channel capacity \cite{ren2018optimal, lin2020funds, wang2024fence}; this is shown to work well in practice. The rationale for such a price structure is explained well in \cite{wang2024fence}, where this fee is derived with the aim of always maintaining some balance (liquidity) at each end of every channel. This is a reasonable aim if a channel is to never rebalance itself; the experiments of the aforementioned papers are conducted in such a regime. In this work, however, we allow the channels to rebalance themselves a few times in order to settle on a detailed balance flow. This is because our focus is on the long-term steady state performance of the protocol. This difference in perspective also shows up in how the price depends on the channel imbalance. \cite{lin2020funds} and \cite{wang2024fence} advocate for strictly convex prices whereas this work and \cite{varma2021throughput} propose linear prices.

\paragraph{On Rebalancing} 
Recall that the DEBT control protocol ensures that the flows in the network converge to a detailed balance flow, which can be sustained perpetually without any rebalancing. However, during the transient phase (before convergence), channels may have to perform on-chain rebalancing a few times. Since rebalancing is an expensive operation, it is worthwhile discussing methods by which channels can reduce the extent of rebalancing. One option for the channels to reduce the extent of rebalancing is to increase their capacity; however, this comes at the cost of locking in more capital. Each channel can decide for itself the optimum amount of capital to lock in. Another option, which we discuss in Section \ref{sec:five_node}, is for channels to increase the rate $\gamma$ at which they adjust prices. 

Ultimately, whether or not it is beneficial for a channel to rebalance depends on the time-horizon under consideration. Our protocol is based on the assumption that the demand remains steady for a long period of time. If this is indeed the case, it would be worthwhile for a channel to rebalance itself as it can make up this cost through the incentive fees gained from the flow of transactions through it in steady state. If a channel chooses not to rebalance itself, however, there is a risk of being trapped in a deadlock, which is suboptimal for not only the nodes but also the channel.

\section{Conclusion}
This work presents DEBT control: a protocol for payment channel networks that uses source routing and flow control based on channel prices. The protocol is derived by posing a network utility maximization problem and analyzing its dual minimization. It is shown that under steady demands, the protocol guides the network to an optimal, sustainable point. Simulations show its robustness to demand variations. The work demonstrates that simple protocols with strong theoretical guarantees are possible for PCNs and we hope it inspires further theoretical research in this direction.

%\subsection{LLMs May Provide Opportunities to Support Programming Plan Identification}
\label{sec:opportunities}
% \begin{figure*}
% \centering
% \includegraphics[width=\textwidth]{img/design.png}
% \caption{A proposed sociotechnical plan identification process that mirrors educators' current plan identification procedures while leveraging both human expertise and automated processing for improved efficiency.}
% \label{fig:design}
%   % \Description{This diagram summarizes the proposed sociotechnical plan identification process. The first box represent the candidate plan identification stage where plans are created based on common code sequences, from literature, or from scratch. There is an arrow that originates from this box to the next box that represents the plan refinement stage showing the successive steps. In the plan refinement stage, we modify the abstraction level of plans and receive automatic and crowd-sourced feedback on metrics. This box has a self-loop arrow representing that it is an iterative process until goal is reached. Then, there is an arrow pointing to a plan library symbolizing that a collection of candidate plans has been created. The last arrow from the plan refinement box is to the last box representing the plan elimination stage. At this stage, if the candidate plan does not reach acceptable metrics or does not meet learning goals, it is discarded.}
% \end{figure*}

By providing a more precise understanding of how plan identification is currently performed by educators, as well as the barriers in the current process of plan identification,
% as well as an understanding of several opportunities for improving the efficiency of the plan identification process, 
our interviews highlighted opportunities to leverage generative AI tools like ChatGPT for improving educators' process for plan identification. 
%% KC other thoughts
%
% Our formative study established that a review of practice is a key part of the plan identification. However, reviewing this practice can be time-consuming and tedious. In addition, common code patterns aren't the only important part of programming plans. Educators look for other pieces of information that help students translate the programming plan into practice.
% They use literature. They like a place to start from.
% By getting examples ready and organized, we may be able to improve and speed up the process of programming plan identification. 
% To investigate the feasibility of this plan, we attempt to generate 
% 

% This new processes mirror the stages of educators' current plan identification procedures, while leveraging both distributed human expertise and automated processing. We describe each of these avenues
% % stage of this process 
% below and present a summary in Figure~\ref{fig:design}.

% \subsection{Some Plan Structures Better Support Automated Metrics}

% % take a stance
% % We need code in plans to better find the level of abstraction. In plans with just natural language description, it would be difficult to measure the level of abstraction

% \edit{To allow educators and developers to work together, the system must use a consistent plan definition which includes choosing the necessary plan components that best meet the learning goals of the intended audience. We believe that choosing a programming plan definition where plans are defined in code and where plans have changeable areas will best facilitate automated processes that draw on pools of code examples.}

% \subsection{Choice of Plan components and Perspective}

% To allow educators and developers to working together, the system must use a consistent plan definition. Designers of plan identification systems should choose from the list of components (Section~\ref{sec:components}). We believe that choosing a programming plan definition where plans are defined in code and where plans have changeable areas will best facilitate automated processes that draw on pools of code examples.

% Implications for the choice of plan components


%Based on , we believe that a plan definition that uses a solution in code rather than pseudocode or natural language (Section ~\ref{}) presents more opportunities to partially automate the plan identification process because it allows systems to take advantage of the data in existing code repositories, like GitHub or StackOverflow. Further, we believe that the plan representation in a plan identification system should include specific changeable areas (Section~\ref{}), as it enables automatic calculation of the abstraction level . This idea is further explained below in Section~\ref{}.
 
% \subsubsection{Opportunities to generate candidate plans} % Creation stage

% \subsubsection{Opportunity to Involve AI or Data Mining Tools}
% \textit{``Since it learns from so many, it probably has already kind of done the synthesis itself,''} reasoned P6. P6 also suggested that 

A key challenge mentioned by our participants was how plan identification requires a lengthy and tedious process where the instructor needs to get familiar with many example programs for the domain they are working in. The time-consuming nature of creating a representative, general-purpose programming plan from many examples slows the process of plan identification for educators.
% Reflecting instructors' experiences with plan identification, there are multiple avenues to introduce candidate plans into a system in a way that accelerates the process of plan identification: those that automate the process and those that draw on past work or an educator or programmer's expertise. 
% Some of our educators mentioned that using emerging AI tools like GitHub Co-Pilot (P6) or ChatGPT (P7) to generate solutions could help in plan identification.
In order to support the challenging process of identifying common examples, a plan identification system can \textit{generate candidate plans} by searching relevant pools of code for common pieces of code.
% ~\cite{haggis_code_similarity}, or by prompting a large language model for examples from a particular domain. 
Large language models, which are trained on large corpora of code and include data from many example programs, have been shown to perform reasonably well on code generation and interpretation tasks~\cite{juryEvaluatingLLMgeneratedWorked2024a, finnie-ansleyRobotsAreComing2022}. Thus, LLMs may be able to provide a starting point for instructors by presenting them with potential plan candidates. %with all components that meet the success metrics. %by generating initial plan candidates, including solution code and explanations.

% The user can specify the topics they are interested in identifying programming plans for, and the system could then suggest relevant code corpuses drawn from online code repositories (i.e., Github, Stackoverflow) or instructional materials. As educators vary in their beliefs about the type of practice that plans should represent, users should have the ability to limit the code pools used for candidate generation to those by certain types of authors or another measure of quality.

%\edit{Our participants suggested using emerging AI tools to ask for solutions to the same problems in different contexts and then evaluate code similarity to search for plans.}

% Another method of candidate plan generation could draw on the instructor's current insight. As some instructors prefer to design their own plans, plan identification systems should include the possibility for users to create a candidate plan from their own intuition. %This process of course depends on a user's programming expertise.

% Finally, as some of the educators we interviewed found plans from the literature important in their own plan identification process, systems could allow users to search the existing plan literature as well as programming plans created by other users.

% \subsection{Opportunities to support plan refinement} %Refinement stage

% \subsubsection{Opportunity for Automated Metrics for Abstraction, Commonality, and Usability}
% We found that plan refinement is an crucial stage in educators' plan identification process, but also an area where educators face challenges as they struggle to find the right balance between a plan's frequency in practice (commonality), ease of use (usability), and appropriateness for their learner population. The process of plan identification could become more efficient if educators could receive quicker and clearer feedback about how their candidate plans measure up in each of these characteristics. 
%evaluating the metrics of plan quality that instructors value could be automated or crowd-sourced.} 

% With automated metrics of commonality,  usability, and level of abstraction, the feedback loop that educators undergo as they search for the right version of their programming plan could be shortened. Such automated metrics may be possible, if the right plan components are available. If programming plans are represented in a format that includes their programming language code, a system could search relevant code repositories for similar code in order to provide a measure of a plan's \textit{commonality}. If the system represents programming plans not only in code but also with changeable areas (see Section~\ref{sec:changeable}), then the system can infer the level of abstraction based on what percentage of the plan consists of changeable areas, as well as the complexity of typical code in the changeable areas.

% To measure usability, which is the ease with which learners can modify and apply the plan, code comparison with a relevant corpus may not yield much insight. However, it may be possible to get learners involved through a learnersourcing~\cite{weir2015learnersourcing} activity. The system could connect with novice programmers at the appropriate level and give them an automated task to implement the candidate plan. Based on the correctness of this activity, it could provide a metric for usability.

%\edit{If educators could quickly understand how changes to the programming plan they are refining affect the metrics that they care about, it may increase the efficiency of the plan identification process.}

% During the refinement stage of a plan identification system, instructors can edit the candidate plan with the goal of balancing the metrics of commonality, usability, and level of abstraction to values within their desired thresholds. As some instructors value some metrics more than others, users should have the ability to modify the acceptable ranges for each of these metrics. 

% As instructors add or remove changeable areas to their plan, they receive feedback from the automated metric values. Instructors' expertise is particularly relevant in the refinement stage, as they not only rely on the system for information about the value of their candidate plan, but also judge its appropriateness themselves. 

% Automated metrics may not completely replace the benefits educators gain from conversations with their peers. In our interviews, we learned that the educators we worked with had extensive conversations about plan identification during their process (see Section~\ref{sec:discussions}). Conversations between educators can also be supported in a plan identification system, perhaps through chats, a commenting system, or suggestions for plan changes.
%as the instructors we interviewed were clear that these conversations led to improved plan identification outcomes. These reflections on the quality of the plans and conversations can occur through chats and even spontaneous video calls enabled by the system. }

% \subsubsection{Opportunities to support plan identification in domains unfamiliar to educators} %Refinement stage

Moreover, LLMs may be able to allow instructors to identify plans in unfamiliar domains by presenting them with plan candidates from those domains. 
% Enabling direct collaboration between instructors and large language models could allow for easier identification of programming plans in specific application areas, as the LLM can suggest plans from the large amount of data it was trained on,
% personal expertise or during a guided think-aloud
A system that acts as a collaborator for the instructor might expedite the domain-specific plan identification process significantly. Through such a system, an instructor can obtain initial plan candidates in any domain and refine those plans such that they are appropriate for learners. This approach could make it possible to greatly expand the number of domains where the plans can be used for instruction, as it would no longer require a single person to have both instruction expertise and in-depth domain knowledge. 

% Based on our findings, it's important to combine programming expertise with instructional expertise 
% contribute in ways that make use of their particular strengths 
% (see Section~\ref{sec:individual-design}). 
% For experts in the programming domain, a system could encourage their participation in the candidate plan stage in particular. This may include sharing information about a plan's commonality in a particular domain, as well as information about the intent of a candidate plans. 
% The LLM could contribute to the candidate plan stage in particular. Then, educators may become involved in the plan refinement stage, particularly as they are better positioned to judge how understandable plan components are for the relevant learner audience.

% \subsubsection{Opportunity for Facilitating Discussions Between Educators}


%\subsection{Elimination stage}

%Finally, not all candidate plans will make it into the final plan library. If instructors cannot balance the metrics for a particular plan to acceptable values, or, if the instructors believe that the plan doesn't meet their learning goals, they can remove the plan from consideration. 


% % \begin{figure}
% \centering
% % \includegraphics[width=0.5\textwidth]{img/pipeline-new.png}
% \includegraphics[width=\textwidth]{img/new-plan-pipeline.png}
% \caption{The three stage process for generating example programs, segmenting them with plan components, and clustering these plan-ful examples.
% %collecting and processing responses from ChatGPT into plan-ful examples}
% %\caption{The pipeline for LLM plan generation.}
% }
% \label{fig:llm-methods}
% \end{figure}

\subsubsection{Generating In-Domain Programs}
% Informed by the insights identified in our interview study, we generated programming plans relevant to an application-focused domain: web scraping via BeautifulSoup. We utilized an LLM-based approach to generate these plans with the GPT-4 model from OpenAI using its publicly available API in an iterative workflow. 

Our participants examined example programs and conducted literature reviews (Section \ref{sec:viewing-programs}) as key parts of their plan identification process. Inspired by this, we used Open AI's GPT-4, a state-of-the-art large language model for code generation that is trained on a large corpus of computer programs~\cite{liu2023isyourcode}, to generate candidate programs along with its respective plan components in the programs. First, we prompted the model to generate \texttt{100 use cases of using BeautifulSoup}. Subsequently, we asked the model to \texttt{write pieces of code that use BeautifulSoup to achieve <use case>}. 
% This collection of example programs (which we refer to as 
%dataset 
% $\mathcal{D}$) was used as our primary dataset for further analysis.


\subsubsection{Segmenting Programs Into Plan-ful Examples}
% We then proceed to compile these examples with each of the plan components generated using ChatGPT. We construct a new dataset with these components, Dataset \((\mathcal{D}^{\textit{Comp}})\).
We used the generated programs from
% \mathcal{D}$
the prior step as the input in a set of prompts (see Stage 2 in Figure~\ref{fig:llm-methods}), where each prompt was used to generate one of the plan components identified in our interview study (see Section \ref{sec:components}). 
%We fragmented these generated programs into smaller code pieces by generating \textit{subgoals} in the program. Then, each goal (Section \ref{sec:goal}) and the accompanying code solution (Section \ref{sec:solution}) were added as a single unit of data in our plan-ful example dataset of components, \(\mathcal{D}^{\textit{Plan-ful}}\). For each of these datapoints, we prompted the model to identify \textit{changeable areas} (Section \ref{sec:changeable}). %The name (Section \ref{sec:name}) was determined later in the pipeline (Stage 2 in Figure \ref{fig:llm-methods}).


% From the results of our qualitative study, we now know about the parts of a programming plan. In order to extract these plans automatically, we used ChatGPT. We accessed it using its publicly available API and we used the GPT-4 model. We selected 3 domains that are interesting for non-majors. This included . 

% For each of these domains, we first asked the LLM to generate 100 use cases. We then re-prompted it with the use cases it generated and asked it to generate code that would be written to accomplish that use case.
% potential for another table?
% add code metrics from stackoverflow github work for chatgpt
% With all these code pieces collected, we then asked ChatGPT to generate each of the plan parts one-by-one.

\subsubsection*{Extracting Goals and Solutions}Generated programs 
% in \(\mathcal{D}\) 
typically included a comment before each line, which described that line's functionality. However, these comments did not capture the high-level purpose of the code, as required by a plan goal. To generate more abstract goals for a piece of code, we defined subgoals as \texttt{short descriptions of small pieces of code that do something meaningful} in a prompt and asked the LLM to \texttt{highlight subgoals as comments in the code.} %In our query, we also added the way we define subgoals to provide the relevant context to the model. Specifically, we wrote that 
The output from this prompt was a modified version of each program
% from \(\mathcal{D}\), 
where blocks of code are preceded by a comment describing the goal of that block. % of code. % instead of restating functionality. 

We split each complete program into multiple segments based on these new comments. Thus, the subgoal comments from each complete program I
% n the modified \(\mathcal{D}\) 
became a plan goal, and the code following that comment became the associated solution. %, collected in \(\mathcal{D}^{\textit{Plan-ful}}\). % After it returned the annotated code piece, we extracted the comment and the following lines of code before the next comment. This pair acted as a subgoal-code piece. We collected all such pairs across all use cases from \(\mathcal{D}\) and added them to \(\mathcal{D}^{\textit{Plan-ful}}\).
Each goal 
%(Section \ref{sec:goal}) 
and solution pair
%(Section \ref{sec:solution}) 
was added as a single unit of data in our plan-ful example dataset.
% , \(\mathcal{D}^{\textit{Plan-ful}}\).

\subsubsection*{Extracting Changeable Areas}To annotate the changeable areas for a plan, we defined changeable areas as \texttt{parts of the plan that would change when it is used in a different context} in our prompt and asked the model to \texttt{return the exact part of the code from the line that would change} for all code pieces from the dataset with plan-ful examples.
% from \(\mathcal{D}^{\textit{Plan-ful}}\). 
% This data was added to \(\mathcal{D}^{\textit{Plan-ful}}\).

% to-do
\subsubsection{Clustering Plan-ful Examples into Plans}
\label{sec:clustering}
% We perform k-means clustering on the plans \(\mathcal{D}^{\textit{Plan-ful}}\) to identify clusters of similar code pieces and thus, programming plans.

We used a clustering algorithm to group similar plan-ful examples together as a programming plan. For clustering the code pieces, we used the CodeBERT model from Microsoft \cite{codebert} to obtain embeddings for each code piece in our dataset of plan-ful examples
% in \(\mathcal{D}^{\textit{Plan-ful}}\) 
and applied Principal Component Analysis (PCA) \cite{PCAanalysis} to reduce the dimensionality of the embedding vectors while preserving 90\% of the variance. These embeddings were clustered using the K-means algorithm~\cite{kmeansclustering}. The optimal number of clusters \(\mathcal{K}\) was determined by assessing all possible \(\mathcal{K}\) values 
% (where \(\mathcal{K} \in [2, \texttt{length}(\mathcal{D}^{\textit{Plan-ful}})]\))
using the mean silhouette coefficient \cite{silhouettecoeff}. We assigned each example 
% in \(\mathcal{D}^{\textit{Plan-ful}}\) 
to a cluster of similar code pieces. 

\subsubsection*{Extracting Names}
To generate names for the plan-ful examples, we first defined the properties for a name in the prompt by expressing that \texttt{a name reflects the code's purpose} and it should focus \texttt{what the code is achieving and not the context.} Then, all code snippets from each cluster of examples were provided as input to the LLM along with a prompt asking it to \texttt{devise a name for that cluster of plans}.

% \section{LLM Plan Generation Evaluation}

To evaluate the plan-ful examples created by generative AI tools, we performed a mixed methods evaluation 
%using quantitative metrics and qualitative analysis based on 
guided by
the characteristics instructors use to judge good programming plans (see Section ~\ref{sec:judging}).
% commonality, usability, and appropriateness for learners.
%to determine the feasibility of using LLMs in programming plan generation.

% Cunningham et al. \cite{Cunningham_PurposeFirstProgramming_CHI-2021} devised a set of programming plans for BeautifulSoup, a Python library, to assist non-majors learn web scraping 

We assessed our programming plans in reference to a set of programming plans identified and used by Cunningham et al. \cite{Cunningham_PurposeFirstProgramming_CHI-2021} to teach web scraping to undergraduate conversational programmers. These plans were designed by researchers with programming plan expertise as well as instructional experience in the domain, and were also validated with web scraping experts~\cite{Cunningham_PurposeFirstProgramming_CHI-2021}. 
To obtain a control set, we extracted the same set of plan components (name, goal, solution, and changeable areas) from the publicly available curriculum\footnote{runestone.academy/ns/books/published/PurposeFirstWebScraping/index.html} and created clusters of plan-ful examples (denoted as \(\mathcal{D}^{\textit{Control}}\)). %in future. 

We also subsampled the generated plan-ful examples in \(\mathcal{D}^{\textit{Plan-ful}}\) to have an equivalent number of examples as the control set. To achieve this, we chose the 10 largest clusters with the most data points and calculated the centroid of each cluster using the embeddings. We then selected the four closest plans to this centroid as the most representative examples in each cluster (compiled together as \(\mathcal{D}^{\textit{Plan-ful*}}\)). 
% Thus, both \(\mathcal{D}^{\textit{Control}}\) and \(\mathcal{D}^{\textit{Plan-ful*}}\) had the same number of examples.

% Our quantitative analysis focuses on analysis of the plan-ful example solutions, in order to understand their validity across the entire datasets and to compare them with non-LLM generated code. Our qualitative analysis was necessary to gain insight about the non-code components of the plan-ful examples, particularly the qualities of goals and plan names.

\subsection{Quantitative Analysis}

\subsubsection{Syntactic Validity}
\label{sec:quant_accuracy}

Before comparing \(\mathcal{D}^{\textit{Plan-ful*}}\) to \(\mathcal{D}^{\textit{Control}}\), we tested the syntactic validity of the generated programs from our original dataset \(\mathcal{D}\).
%, which included 100 complete Python programs. 
We note that from our set of 100 complete Python programs, all but one were syntactically valid. That program included a syntax error and could not be parsed nor executed. Thus, we conclude that the raw code generated by the LLM is mostly accurate and reliable, at least in our target domain. 

% Summary statistics of these errors and their numbers are presented in ~\ref{table:errors}. On closer examination, we find that most of the compilation errors can be accounted to parsing and data cleaning. ChatGPT does not follow a uniform structure while generating its responses. Thus, it was unlikely to ...

%     \begin{table}
%         \caption{Errors in Non-Compiling ChatGPT Code}
%         \centering
%         \label{tab:errors}
%         \begin{tabular}{|p{4.5cm}|p{3cm}|}
%         \toprule
%                 Error Type & Number of Occurences
%         \\\midrule
%             Unterminated String Literal & 8 \\
%             Invalid Syntax & 9 \\
%             Unterminated f-string Literal & 5 \\
%             Unindent Does Not Match Any Outer Indentation Level & 1 \\
            
%         \end{tabular}
%     \end{table}

% This evidence suggests that while there may be some inaccuracies in data generated using generative AI tools, a lot of them can be controlled using a more systematic data cleaning approach.

\subsubsection{Appropriateness for Learners}
\label{sec:quant_learners_appropriateness}

Our instructors emphasized the importance of plans being suitable for their learner audience (Section~\ref{sec:judging}). Thus, we compared \(\mathcal{D}^{\textit{Plan-ful}}\) to \(\mathcal{D}^{\textit{Control}}\) with standard code complexity metrics
%examined the examples in \(\mathcal{D}^{\textit{Plan-ful}}\) in contrast to \(\mathcal{D}^{\textit{Control}}\) using quantitative metrics on four metrics: 
%average lines of code, cyclomatic complexity~\cite{cyclomatic_complexity_mccabe}, Halstead's volume~\cite{halstead_metrics}, and cognitive complexity~\cite{cognitive_complexity_Campbell} 
to determine their suitability for novices: non-comment lines of code, cyclomatic complexity~\cite{cyclomatic_complexity_mccabe}, Halstead volume~\cite{halstead_metrics}, and cognitive complexity~\cite{cognitive_complexity_Campbell}.


%Cyclomatic complexity~\cite{cyclomatic_complexity_mccabe} and cognitive complexity~\cite{cognitive_complexity_Campbell} are measures of the understandability and modification scope of code, whereas
%,  wherein a lower score indicates that the code is easy to understand and modify. We estimate the cyclomatic complexity of each plan and report the mean score in table~\ref{tab:metrics_appropriateness} across each dataset \(\mathcal{D}^{\textit{Plan-ful}}\) and \(\mathcal{D}^{\textit{Control}}\). 
%Halstead volume~\cite{halstead_metrics} is representative of the size of a program in terms of its operands and operators and hence also reflective of the code's complexity.
%Analogous to cyclomatic complexity, a lower score is indicative of a more straightforward code snippet. Similar to cyclomatic complexity, the mean Halstead volume score for each dataset in reported in table~\ref{tab:metrics_appropriateness}.
%Cognitive complexity is another metric that estimates the understandability of code proposed by Campbell~\cite{cognitive_complexity_Campbell}.

Table \ref{tab:metrics_appropriateness} shows the mean value for all metrics across the datasets. For all metrics, a lower value indicates a simpler program that is more appropriate for beginning learners. We also conducted a two-sided non-parametric Mann-Whitney U-test for each complexity metric.
%with rank biserial correlation as the effect size. 

While \(\mathcal{D}^{\textit{Control}}\) is marginally less complex compared to generated code according to the metrics, we did not find any statistically significant trends (p $>$ 0.05 for all comparisons). Thus, it is reasonable to claim that the examples generated using ChatGPT can be used in instruction for novices.

% 


\begin{table}
\caption{Mean Code Complexity Metrics}
    \centering
    \label{tab:metrics_appropriateness}
    \begin{tabular}{cccc}
    \toprule
        Metric & \thead{\(\mathcal{D}^{\textit{Plan-ful}}\) \\ (n = 781)} & \thead{\(\mathcal{D}^{\textit{Plan-ful*}}\) \\ (n = 40)} & \thead{\(\mathcal{D}^{\textit{Control}}\) \\ (n = 43)}
    \\\midrule
        % Number of Plan-ful Examples & 781 & 40 & 43 \\
        Lines of Code & 2.30 & 3.10 &  2.72 \\
        Cyclomatic Complexity & 2.43 & 2.21 & 2.40 \\
        Halstead Volume & 173.69 & 178.91 & 114.02 \\
        Cognitive Complexity & 0.217 & 0.375 & 0.233 \\
        % Starsinic Readability & -0.362 & \\
    \end{tabular}
\end{table}

% \[
% \begin{matrix}
%   & \(\mathcal{D}^{\textit{Plan-ful*}}\ & \(\mathcal{D}^{\textit{Plan-ful}}\ \\
%   \(\mathcal{D}^{\textit{Control}}\ & & d 
% \end{matrix}
% \]

\subsubsection{Usability}
\label{sec:quant_usability}

Aside from generating code that is accurate and appropriate for learners, it is also important that programming plans be representative of key functionalities in the domain.
%to provide its users with the flexibility and scaffolding to perform a variety of tasks. 
To this end, we compared the number of distinct method calls used in \(\mathcal{D}^{\textit{Plan-ful}}\) and \(\mathcal{D}^{\textit{Control}}\). Having examples on more distinct methods may indicate a set of examples that can be employed to solve a larger number of problems.% (See % (See Figure~\ref{fig:methods-vs-clusters}).

%We found that 
\(\mathcal{D}^{\textit{Control}}\) included four distinct method calls (\texttt{append, find, find\_all, get}), which were also included in \(\mathcal{D}^{\textit{Plan-ful*}}\), the plan-ful examples from the 10 largest clusters generated by the LLM, such as \texttt{select} and \texttt{select\_one}. Moreover, these largest clusters also included five additional methods not included in \(\mathcal{D}^{\textit{Control}}\).
%showing that our clustering algorithm did not cause a loss in method variance. 
This shows that our pipeline generates plans with similar functionality to those designed by an instructor. 



% To examine this behavior more closely, we investigated the number of distinct method calls introduced by each cluster of examples, starting from the largest cluster.
% to illustrate the relationship between the number of distinct methods being used in the clusters as the number of clusters rises. 
% As evident from Figure~\ref{fig:methods-vs-clusters}, 
% By selecting the 10 largest clusters, a majority of distinct methods in the generated set is captured. Moreover, the variance of methods in generated code increases steeply compared to the previously proposed plans.
% almost more of all unique methods present in the set of all generated examples \(\mathcal{D}^{\textit{Plan-ful}}\) are also present in the subset \(\mathcal{D}^{\textit{Plan-ful*}}\). Moreover, 
% follows an upward trend suggesting that more clusters form a larger coverage over the domain's features.

% \begin{figure}
%     \centering
%     \includegraphics[width=0.4\textwidth]{img/graph-new-pfp-and-gpt.png}
%     \caption{Comparative Analysis of Methods Used As the Number of Clusters Grow in the Plan-ful Examples from ChatGPT}
%     \label{fig:methods-vs-clusters}
% \end{figure}

% We also drew a comparison between the distinct types of methods used in \(\mathcal{D}^{\textit{Control}}\) with \(\mathcal{D}^{\textit{Plan-ful}}\). We find that \(\mathcal{D}^{\textit{Control}}\) encompasses 4 distinct methods including \texttt{['append', 'find', 'find\_all', 'get']} in 6 clusters. The code plans in \(\mathcal{D}^{\textit{Plan-ful}}\) need 7 clusters for addressing these 4 distinct methods. This demonstrates that the ChatGPT generated examples are as varied as the researcher created plans and thus it is unlikely that there will be a loss of variability if examples are generated using AI tools.

% the most common in the D plan-ful have the same methods as D control


\subsubsection{Commonality}
\label{sec:quant_commonality}

A significant motive for using programming plans in instruction is to equip novices with the necessary technical skills to contribute to real-world code problems. Thus, it is essential that plans used in instruction are representative of actual practice.
%to provide learners with the appropriate scaffolding. 
To obtain an estimate of how the LLM-generated plan-ful examples compare to actual practice, we compared \(\mathcal{D}^{\textit{Plan-ful}}\) and \(\mathcal{D}^{\textit{Control}}\) to web scraping files from GitHub. We created a new dataset \(\mathcal{D}^{\textit{GitHub}}\) by collecting Python files from public repositories via GitHub's API that met the following criteria: contained a BeautifulSoup import statement, included the BeautifulSoup contructor, and was not a test file. 
%that were written in Python and included the string \textit{BeautifulSoup}
%in any files in any public repository using the publicly available GitHub REST API. This resulted in a collection of 1016 files. We filtered this dataset to include files that had an import statement (\textit{import BeautifulSoup, import bs4, from BeautifulSoup import BeautifulSoup} or \textit{from bs4 import BeautifulSoup} and a constructor initialization (\textit{= BeautifulSoup(}). Finally, we removed test files by filtering any of the following statements: \textit{import unittest, (unittest.TestCase), from bs4.testing import}. 
This resulted in the final dataset with 733 files. Then, we generated the embeddings for these programs using CodeBERT in a similar manner to Section~\ref{sec:clustering} to compare the sets \(\mathcal{D}^{\textit{GitHub}}\) and \(\mathcal{D}^{\textit{Control}}\) as well as \(\mathcal{D}^{\textit{GitHub}}\) and \(\mathcal{D}^{\textit{Plan-ful*}}\).

% To compare the commonality of generated plans \(\mathcal{D}^{\textit{Plan-ful*}}\) to \(\mathcal{D}^{\textit{Control}}\), we compute the distance between embeddings of the code in each dataset to \(\mathcal{D}^{\textit{GitHub}}\). The embeddings for the new datasets are computed with CodeBERT as similar to the prior Section~\ref{sec:clustering}. %, denoted as \(\mathcal{D}^{\textit{GitHub}}\).

%We now compare \(\mathcal{D}^{\textit{Plan-ful}}\), \(\mathcal{D}^{\textit{Control}}\), and \(\mathcal{D}^{\textit{GitHub}}\) where \(\mathcal{D}^{\textit{GitHub}}\) is our baseline, \(\mathcal{D}^{\textit{Plan-ful}}\) is our test set, \(\mathcal{D}^{\textit{Control}}\) is our control set. 
% Using a number of quantitative metrics, we perform analysis that yields information about how the plans compare to real-world code. We first convert the code pieces in \(\mathcal{D}^{\textit{Control}}\) and \(\mathcal{D}^{\textit{GitHub}}\) to embeddings using CodeBERT like before. For \(\mathcal{D}^{\textit{Plan-ful}}\), we use the previously collected embeddings from \(\mathcal{D}^{\textit{Emb}}\). 

% For our first experiment, we compute the average embedding per dataset using a mean of values across each dimension. This yields one multi-dimensional vector per dataset representative of the average embedding of that dataset. Then we perform a cosine distance analysis using functions from the \texttt{sklearn} library in Python on these pairs: \(\mathcal{D}^{\textit{Control}}\) and \(\mathcal{D}^{\textit{GitHub}}\) and \(\mathcal{D}^{\textit{Plan-ful}}\) and \(\mathcal{D}^{\textit{GitHub}}\) to gain insight into which dataset is more similar to the baseline. Cosine distance is measured on a scale of 0 to 2 where a value closer to 0 indicates high similarity and a value nearer to 2 represents high dissimilarity. We observe a score of 0.188 between \(\mathcal{D}^{\textit{Plan-ful}}\) and \(\mathcal{D}^{\textit{GitHub}}\) and 0.213 between \(\mathcal{D}^{\textit{Control}}\) and \(\mathcal{D}^{\textit{GitHub}}\). Accordingly we infer that the example plans generated using ChatGPT are closer to real-world practice than the purpose-first programming plans. This can be accounted by the fact that LLMs like ChatGPT are trained on public data sources \cite{} whereas the purpose-first programming plans were created by instructors and researchers crafted for novices. We also note the similarity between \(\mathcal{D}^{\textit{Plan-ful}}\) and \(\mathcal{D}^{\textit{Control}}\) which is 0.012 which indicates that the LLM generated plans and the purpose-first programming plans are highly similar.

To evaluate the similarity between sets, we computed Hausdorff distance~\cite{tahaEfficientAlgorithmCalculating2015} and Wasserstein distance~\cite{ramdasWassersteinTwoSample2015}, which are common metrics for comparing generated content to reference sets~\cite{pmlr-v70-arjovsky17a, weilihausdorff}. For both Hausdorff (\(\mathcal{D}^{\textit{Plan-ful}}=13.66\), \(\mathcal{D}^{\textit{Control}}=14.92\)) and Wasserstein (\(\mathcal{D}^{\textit{Plan-ful}}=12.97\), \(\mathcal{D}^{\textit{Control}}=13.97\)) distances, the set of generated examples had smaller distance to code from GitHub in comparison to the control set of previously proposed plans, conveying that the ChatGPT can generate plan-ful code that is more representative of real-world examples compared to instructor code.


% We compute Hausdorff distance between sets of embeddings, which is a measure of the maximum distance from a random point in a set to a reference set~\cite{tahaEfficientAlgorithmCalculating2015}. A smaller Hausdorff distance indicates that points for all points in the set have a close corresponding point in the reference set. We found that the distance between \(\mathcal{D}^{\textit{Plan-ful}}\) and \(\mathcal{D}^{\textit{GitHub}}\) as 13.66 and the distance between \(\mathcal{D}^{\textit{Control}}\) and \(\mathcal{D}^{\textit{GitHub}}\) as 14.92, suggesting that the ChatGPT generated code pieces are more similar to GitHub examples compared to the plan-ful examples from the purpose-first programming study.

% We also compute the Wasserstein Distance, which is a measure of the similarity of the probability distributions of two sets~\cite{ramdasWassersteinTwoSample2015}. % cost of an optimal transport plan to convert one distribution into the other \cite{}, using in-built functions in the \texttt{scipy} library in Python, between the pairs of the datasets. Analogous to the prior result, 
% Analogous to the previous result, the distance between \(\mathcal{D}^{\textit{Plan-ful}}\) and \(\mathcal{D}^{\textit{GitHub}}\) (12.97) is lesser than the distance between \(\mathcal{D}^{\textit{Control}}\) and \(\mathcal{D}^{\textit{GitHub}}\) (13.97), 
%We also measure the Hausdorff Distance, measure of the discrepancy between two sets of points \cite{}, using the same Python library, among the datasets which also yielded similar results with the score between 

%These findings suggest that examples generated using AI tools are more likely to be used in common code practice, over examples generated by instructors from scratch.

\subsection{Qualitative Evaluation}

To obtain a richer picture of the strengths and weaknesses of plan generation with LLMs, we conducted a qualitative evaluation of the generated plan-ful examples, inspired by thematic analysis approaches in prior work on code generation~\cite{kazemitabaarHowNovicesUse2023}.

We started our analysis with a free-form discussion on both generated plans and previously proposed plans from \(\mathcal{D}^{\textit{Control}}\) to familiarize ourselves with the data. One member of the research team prepared an initial codebook, with codes organized under two main dimensions reflecting the \textit{components} and \textit{characteristics} highlighted in Section \ref{sec:interview_results}.
Two researchers coded a subset of examples (10\% of the data) and obtained inter-rater reliability of 0.76 using percentage agreement\cite{miles1994qualitative}. The codebook was refined through discussion, and two researchers achieved an IRR of 0.89 after the second subset. One member of the team coded the rest of the data according to the refined codebook\footnote{The codebook is available online: https://tinyurl.com/fk6pzat8}.

\subsubsection{Components}
\label{sec:qual_components}

The generated plan-ful examples were `mostly accurate' (90\%, n=36). Only four examples in \(\mathcal{D}^{\textit{Plan-ful*}}\) had `mostly inaccurate' code, indicating that LLMs can generate the solution component of a plan reliably.

Changeable areas of the examples were also somewhat successfully generated: there was only a single case where an unalterable part of the code was annotated as a changeable area. Yet, 22.5\% of examples were missing changeable areas (n=9), and another 22.5\% had changeable areas that were considered `improbable' (n=9). For example, some default arguments of the commonly used functions were annotated as changeable. While technically correct, these areas are not likely to be modified in simpler examples and were not included in previously proposed plans from \(\mathcal{D}^{\textit{Control}}\).

The generation of goals and names was less satisfactory. On the example level, more than half of the generated goals were `descriptive' (55\%, n=22), but 17.5\% of examples were missing a goal label (n=7), and 12.5\% of examples had an `insufficient' or `too general' goal (n=5). On the cluster level, only 40\% of generated names were `descriptive' (n=4), with other names either being `insufficient' to understand when to use a plan (n=2) or `overstating' what the plan actually does (n=4). For example, a cluster that accesses multiple attributes of an object was named ``Data Extraction and Database Management'', even though it does not have any database interaction.

\subsubsection{Characteristics}
\label{sec:qual_characteristics}

The most consistent characteristic in generated examples was commonality: 80\% of examples had `common syntax' with plans placed in the same cluster (n=32) and 67.5\% of them had `common goals' with the plans in the cluster (n=27). Another 12.5\% of examples shared `vague commonalities' (n=5), where it was hard to find the overall goal of the cluster due to great differences in syntax and structure. Moreover, some code statements were repeated in multiple plans (30\%, n=12), and the code for shorter plans, such as importing libraries or calling the BeautifulSoup constructor, was also included within some of the larger plans.

From a usability perspective, most plans were `cohesive' examples of a given use case (67.5\%, n=27), and they were `generalizable' to new contexts (57.5\%, n=23). Moreover, some of the shorter plans did not require customization but could still be useful to students, e.g. ``Importing Libraries''.
% However, some of the other plans were so simple that they would not be beneficial to solve a new problem.
% For example, one of the plans was ``Importing Libraries'' and it just included a set of import statements.

Finally, the appropriateness of the generated content for beginners was questionable: while there were similarities to the ones defined in \(\mathcal{D}^{\textit{Control}}\), 42.5\% of plans used `technical jargon' in the name and goals (n=17). These included revealing some web technologies that were abstracted away in the previously proposed plans, such as GET requests and HTML structure, as seen in Figure \ref{fig:contrasting-cases}. Furthermore, some plans included `advanced concepts' in Python (15\%, n=6) such as list comprehensions or exception handling. 

\begin{figure}
    \centering
    \includegraphics[width=0.4\textwidth]{img/contrasting-cases.pdf}
    \caption{Two LLM-generated plan-ful examples from the same cluster, with an example almost identical to an instructor-generated plan from prior work (top), and an example that includes technical jargon and improbable changeable areas, making it potentially confusing for novices (bottom).}
    \label{fig:contrasting-cases}
\end{figure}

% Abstractness


%%%Accuracy

% 36/40 mostly accurate

%%%Quality of Generated Components

%% names
% 4 descriptive
% 2 insufficient
% 4 overstated


%% subgoals
%22 of 40 descriptive
%7 missing
%4 insufficient or 1 too general 

%% changeable areas
%9 missing
%9 improbable
%1 inaccurate

%%%characteristics
% 32 common syntax
% 27 common goal
% 5 vague commonality
% 12 common across

% 27 cohesive and 23 generalizable

% 17 technical jargon
% 6 advanced, 7 confusing


%Appropriateness to Beginners


% \subsection{Results}


%  Accuracy

% A. Are the clusters cohesive?
% B. Are the changeable areas, goal, name, solution accurate?


% Qual


% A. Look at each of the parts
% 	- what do we see



% Quant 

% A. Basic metrics comparison (usability for beginners)
% - lines of code
% - cyclomatic complexity

% B. Embedding distance comparison (commonality)
% - 1. Comparison to GitHub (using prior method)
%     - Which average is closer
%     - How many clusters are closer
%   - # of anomalites

%\section{Probing Into Plan Design Process through Design Workshops}
\section{Exploring Opportunities to Support Plan Identification in a Design Workshop}
\label{sec:design-workshop}

% Isn't this about finding out how humans can interact with LLM-generated materials to design plans?

%% KC: feels like something is missing here that connects the findings of the interview study with the key characteristic of the tool: the LLM-generated candidate content. 

The challenges we identified in instructors' current plan identification process (Section~\ref{sec:challenges}) suggest that there are multiple opportunities to increase instructors' efficiency and ability to identify domain-specific plans. In this section, we describe arguments for three design characteristics that may improve the plan identification process.
%, as well as reasons why large language models (LLMs) can enable those characteristics. 
Then, we report on the findings of design workshops in which instructors used and provided feedback about design artifacts with these characteristics. To ensure that our findings were applicable to domain-specific plan identification, we worked with instructors who teach application-focused programming domains and tailored their experience to include plan identification in those areas.
%: web development using Django, web scraping via BeautifulSoup, and data processing with Pandas. 
Finally, we report a set of design goals for systems that support domain-specific programming plan identification. %that can support instructors to search, abstract, and describe programming plans with LLM-based support.


% To understand how our insights about challenges in the plan identification process (Section~\ref{sec:challenges}) can inform the design of tools for plan identification, we conducted design workshops with instructors who teach application-specific programming topics (web development using Django, web scraping via BeautifulSoup, and data processing with Pandas). This workshop involved a preliminary design artifact that utilizes the capabilities of large language models to assist instructors in designing plans.
% The key characteristics of this artifact were influenced by instructor practices in introductory programming.
% In the design workshops, we explored additional interactions that can support instructors refine candidate plans and improve the instructor-LLM collaboration.

% To understand how our insights generalize to application-specific domains, we conducted design workshops with instructors from such domains using a design artifact informed by instructor practices in introductory programming. 

\begin{figure*}[ht]
    \centering
    \Description{Illustration of three main characteristics of design artifacts. On left, a window annotated A shows a list of programs with short explanations on top. On top right, one of three clusters is shown to be selected, named `Summary Statistics'. On bottom right, an `Item Details' pane for the candidate plan obtained from cluster, annotated C, and the suggested potential values for plan components given on the right, annotated B.} 
    \includegraphics[width=0.8\linewidth]{img/characteristics-new.pdf}
    \caption{Illustrations of the three characteristics of the design artifacts, 
    proposed to support instructors' plan identification process by addressing the challenges from our formative study. 
    %as informed by the practices observed in Section \ref{sec:interview_results}. 
    (A) Instructors can view a vast library of generated programs that achieve a diverse set of tasks in the relevant programming application area to inspire their plan identification; (B) Instructors can compare similar code snippets and other plan components to assist their refinement of plans; and (C) Instructors can follow suggested fields that endorse the structure their final plans should follow.}
    \label{fig:baseline-prototype}
\end{figure*}

\subsection{Characteristics of the Design Artifacts}
\label{sec:design-artifact}

The three key characteristics (see Figure~\ref{fig:baseline-prototype}) of our design artifacts are inspired by the three challenges identified in our formative study. These characteristics support a multi-stage workflow that mirrors educators' current plan identification process: initial identification of a plan candidates, refinement of the plan's details, and, finally, creating a complete description of the plan. 
%a multi-stage key practices adopted by instructors who identify common patterns in code and design programming plans to teach these patterns. The artifact reflects these three practices using the enabled interactions.

\subsubsection*{(A) Support \textbf{Initial Plan Identification} with \textbf{Quick Exploration of Many Authentic Programs and Problems}} % LLM generated content (A)
The participants in our formative study emphasized the importance of exploring content that captures a diverse set of authentic goals and implementations when identifying programming plans. However, they also found this search process to be tedious and time-consuming (Section \ref{sec:challenges_practice}). The instructors we spoke to indicated looking across multiple code files or examining textbooks to perform a survey of the topic in which they were identifying plans. \textit{If that reference material was readily available in a single interface, it may make the plan identification process quicker and easier.}
To address this challenge, we provide a large set of programs that address a diverse set of use cases in the target domain to reduce the tedium faced by instructors as they browse reference material and the burden of finding appropriate reference material for a certain programming domain.
These programs are generated by a large language model (LLM), described in full detail in Section~\ref{sec:llm-pipeline}. %By presenting instructors, we lessen the burden of searching many sources for appropriate examples.  

\subsubsection*{(B) Support \textbf{Plan Refinement} with \textbf{Comparisons of Similar Content}} % Clustering / which shows related programs (B)

Making decisions about exactly how to explain a particular concept in a programming plan was another common challenge highlighted by our formative study (see Section~\ref{sec:challenges_abstraction}). In refining their plan, instructors considered both how easily a potential plan could be used by their students (usability) and how common that plan was in practice (commonality). 
%A key process outlined by our participants to mitigate this challenge was comparing programs to recognize similarities to design initial candidates. Our interviewees employed different approaches to creating these candidates, including exploring reference material and discussions. % and literature reviews. % (Section \ref{sec:process-candidates}).
% More -- commonality / usability.
\textit{If instructors could view and compare multiple pieces of programs related to their potential plan, as well as a variety of potential plan goals, they may be able to more quickly evaluate how common or usable their plan is.}
We bootstrap this practice by clustering similar program snippets from the example programs, using a combination of heuristics and code embeddings (described in full detail in Section~\ref{sec:clustering}).
%This supports instructors by reducing the search space of relevant programs, potentially lowering the mental load of plan identification.

\subsubsection*{(C) Support \textbf{Robust and Shareable Plan Descriptions} with \textbf{Structured Fields}}
%{Finalizing Plan Design} % Plan components template (C)
% Support Complete Plan Descriptions with Structured Fields
The final practice we focus from the plan identification process is refining initial candidates into the structure of programming plans. 
We found that, surprisingly, instructors did not agree on their preferred structure for a programming plan, presenting the potential challenge that plans identified by one instructor may not be usable by another (see Section~\ref{sec:challenges_robust_shareable}). We observed that plan identification requires iteration and refinement, and instructors have varied opinions on the components that they incorporate in their plans.
% is an iterative process that is finalized through refinement of candidate plans (Section \ref{sec:individual-design}). We also identified a set of components used by instructors in this refinement phase (Section \ref{sec:components}). 
\textit{If instructors were given a structured template, they may iterate on their plans faster and achieve a convenient format for sharing with others.} By defining plans with all of the programming plan components mentioned by instructors in our formative study, they could ensure that plans they identify can be readily used by another instructor. Moreover, constraining instructors to work in such a structure could also support other educators who want to identify plans but are new to the pedagogical idea, potentially expediting their work by clarifying what exactly they should be looking for.
Thus, our interface enables examining plans with all their components for refinement rather than just code and comments in a general-purpose text editor or IDE.


\subsection{Workshop Methodology}

%We conducted a design workshop to explore features that address the needs of instructors in using the LLM-generated constructively.
To understand whether the design characteristics described accelerate domain-specific programming plan identification, as well as to identify what interactions with a plan identification system are valued by instructors, we hosted a series of design workshops.
%To evaluate how our design artifact reflects and improves instructor practices, 
We invited seven instructors with expertise in a variety of application-focused programming domains to participate (see Table~\ref{tab:participants-design-workshop}).
%in different computing applications
These areas included data analysis with Pandas (four instructors), web development with Django (two instructors), and web scraping with BeautifulSoup (one instructor). We did not require prior plan identification experience from participants. 



%Instructors participated in a video-recorded 90-minute design session with a \$75 compensation for creating plans in the aforementioned topics. 

%To understand how design practices for plan identification from introductory programming can be employed to design plans in application-focused computing domains, 


 

%\textbf{Protocol.}
\subsubsection{Protocol.}
Each instructor completed a screen- and audio-recorded 90 minute session involving three plan identification tasks and an interview. Participants were compensated with a \$75 Amazon gift card. % The study was determined as NHSR by our institutional review board.
Before the tasks, participants received a brief overview of the definition of programming plans and their components, based on the findings of our formative study (see Section~\ref{sec:challenges_robust_shareable}). %Once participants indicated familiarity,
We told participants that they should create plans appropriate for students with some Python familiarity but no experience in the application-focused domain (i.e., Pandas, Django, or Beautifulsoup).
Then, participants proceeded to perform plan identification in their domain of expertise across three conditions.
%that each provided different types of support. 
In each condition, participants were asked to create up to five programming plans during a 15 minute window in a thinkaloud setting. 

%Canvas conditions were realized using Miro boards\footnote{https://miro.com}, and the prototype interface was deployed as a web application.% through our institution.

%We employed a three-step process where 
%Conditions provided an increasing amount of support, so that we could
Our goal was to observe (a) how instructors interacted with the three characteristics in a low-constraint environment akin to paper prototyping~\cite{Sefelin_PaperPrototyping_CHIEA-2003} (Conditions 1 and 2), and (b) how instructors interacted with our more highly-constrained prototype (Condition 3). Across these conditions, we hoped that instructors' actions and feedback might suggest useful interactions to best make use of our proposed characteristics, as well as information about whether our prototype supported plan identification as expected. 
%Each task introduced an additional characteristic of the design artifact,
%with the third step being a prototype interface enabling all three characteristics. 
\begin{itemize}
    \item Condition 1: a Miro board\footnote{https://miro.com} populated with 30 example programs from the domain and associated one-line explanations in natural language. In addition, 70 more example programs could be found in a linked Google Sheet. Instructors were asked to create their plans in boxes labeled with each of the plan components. \textit{(Characteristics A and C)},
    \item Condition 2: a Miro board populated with common code snippets from the example programs, clustered into groups by similarity and ranked by frequency. Suggested goals, names, and changeable areas were also listed with the code snippets. Instructors were asked to create their plans in boxes labeled with each of the plan components. \textit{(Characteristics B and C)}, % Add link to 
    \item Condition 3: a web application that supports navigation through the example programs from Condition 1, viewing of "suggested plans" that contained clustered code snippets and associated goals, names, and changeable areas from Condition 2, and a plan creation area where plans can be edited within the suggested fields. \textit{(Characteristics A, B, and C)}
\end{itemize}
Participants were informed that they were not required to use provided content and could access external sources at any time, including web searches, library documentation, their own code, or their teaching material.
%with access to generated example programs \textit{(A)}. In the next stage, they identified plans in a second setting using the clustered set of candidate plans \textit{(A and B)}. Lastly, the participants interacted with the prototype that incorporated reference content enabling baseline interactions (\textit{(A, B, and C)}, see Figure~\ref{fig:baseline-prototype}).
%They were also reminded to think aloud while working on the tasks. 
The sessions ended with a short interview, during which the participants were asked to rank all three conditions based on how instructor-friendly they were, make requests for potential new features, and describe their likelihood of using a similar interface to generate programming plans for their courses. 
%Their insights also helped us formulate a comparative analysis between existing tools and the designed interface.


\begin{table}
\caption{Demographics of the Instructor Participants in our Design Workshop.}
    \centering
    \footnotesize
    \label{tab:participants-design-workshop}
    \begin{tabular}{l|ccccc}
    \toprule
            & Domain & \shortstack{Teaching \\ Experience \\ in CS} & \shortstack{Teaching \\ Experience in \\ Domain} & \shortstack{Used \\ Plans in \\ Instruction?}
    \\\midrule
        W1 & Pandas & 16-20 & 1-5 & No \\
        W2 & Django & 6-10 & 1-5 & No \\
        W3 & Pandas & 6-10 & 1-5 & Yes \\
        W4 & BeautifulSoup & 6-10 & <1 & Yes \\
        W5 & Pandas & 1-5 & 1-5 & No \\
        W6 & Django & 16-20 & 1-5 & No \\
        W7 & Pandas & 6-10 & <1 & Yes \\
    %\\\bottomrule
    \end{tabular}%
\end{table}

% \begin{table}
% \caption{Demographics of the Instructor Participants in our Design Workshop.}
%     \centering
%     \label{tab:participants-design-workshop}
%     \begin{tabular}{l|cccc}
%     \toprule
% \multirow{2}{*}{Workshop} & \multirow{2}{*}{Domain} & \multicolumn{2}{c}{Experience Teaching} & \multirow{2}{*}{Used Plans } \\
%                           &                         & CS   & Domain                     & in Instruction? \\
%     \\\midrule
%         W1 & Pandas & 16-20 & 1-5 & No \\
%         W2 & Django & 6-10 & 1-5 & No \\
%         W3 & Pandas & 6-10 & 1-5 & Yes \\
%         W4 & BeautifulSoup & 6-10 & <1 & Yes \\
%         W5 & Pandas & 1-5 & 1-5 & No \\
%         W6 & Django & 16-20 & 1-5 & No \\
%         W7 & Pandas & 6-10 & <1 & Yes \\
%     %\\\bottomrule
%     \end{tabular}%
% \end{table}

\subsection{Findings}
\label{sec:design-workshop-findings}

\subsubsection{Condition 1: Interactions with Example Programs from the Application-Specific Domain.}
% Before starting the plan design process, m
Most participants reviewed the example programs and use cases before designing any plans. Interviewees indicated that they found the provided examples to be meaningful and authentic (``\textit{this idea of merging datasets is really really valuable}'', said W1). Participants found value in both the code and the associated natural language descriptions. W7 added that reading ``\textit{titles are probably more useful than the code}.'' Instructors used these examples as inspiration for ideas about the different concepts that they can build plans around.
%Some participants edited the given code to ``\textit{give a theme}'' (W1) to motivate students by specifying their plans in the same context (e.g., a set of data analysis tasks for Taylor Swift album sales). 
%We noted that almost all participants browsed the given references to consolidate ideas about plans and then started their design process.

%While some instructors identified plans that were closely related to certain example programs, others used the examples as only a starting point. 
W2 and W7 explicitly communicated that designing one plan inspired them with ideas for new plans, rather than sticking to a list of ideas they came up with at the beginning of the session. % of writing plans also motivated the examples they presented subsequently.

However, participants did not always find what they were looking for. W1 and W4 had particular ideas for designing their plans, but they could not find the implementation of those ideas in the reference materials, leading to time wasted in a fruitless search. While most participants were positive about the provided examples, some preferred to gather inspiration from their own teaching experience or an external resource. Specifically, W5 preferred to design plans using their own prior work (a course website with modules for Pandas), and W7 consulted the Pandas documentation for clarifications and ideas. 

In this condition, participants created plans with different structures from one another.
%The structure of the designed plans were also different from one another. 
Most notably, participants annotated changeable areas in different ways: highlighting parts of code using a different color or changing the text color (W1, W6), drawing rectangles or ellipses around parts of changeable code (W4, W7), or writing a natural language description of changeable parts (W1, W2, W3, W4, W5).

Nonetheless, the process of plan identification was not straightforward. Sometimes, in the given reference material, participants saw syntax that they were unfamiliar with. In this case, instructors performed web searches to clarify their understanding of the structures. For instance, W4 redirected to StackOverflow to find the difference between \textit{``\texttt{.text} and \texttt{.content}''} and clarified that \textit{``there's ..., not a substantive difference''}. W5 asked questions about the ``\textit{specificity}'' of plans, and W6 requested clarifications about the intended audience. In addition, participants found it challenging to understand the difference between the plan names and goals. W2 articulated that names might be less meaningful to students, even if instructors do get it.
% Mirroring the challenges that instructors in our formative study indicated with finding the correct granularity of plans, W4 added that ``\textit{if [a plan] needs too much tinkering, it may not be a pattern}.'' 
% \begin{quote}
%     A plan should be something you can pull out of your back pocket. (W4)
% \end{quote}
Despite the hurdles, instructors reacted positively to 
%the concept of programming plans after working in 
this condition, praising the existence of reference materials. They ranked this condition second highest among the three.

\subsubsection{Condition 2: Interactions with Clusters of Code Snippets and Suggested Plan Components.}
\label{sec:workshop-findings-condition2}
In contrast to condition A, participants often struggled to create plans with the reference material provided in condition B and found the experience ``\textit{overwhelming}'' (W1) or tedious because they had to navigate a large canvas (W4).
%They suggested that a list of reference content would allow easier navigation.
More importantly, participants expressed dislike of the content itself, which consisted of clusters of similar small code snippets as well as suggested goals, names, and changeable areas. Many participants preferred to include complete worked examples rather than code snippets in their final plans, with the idea that it would help students to avoid any need for \textit{``implied knowledge ''} (W1). Similarly, some instructors were concerned about the size of the snippets, finding them ``\textit{too fine-grained''} (W4). 
%, i.e., assuming that students are familiar with the context. 
W4 felt the suggested names of plans were ``\textit{too generic}'' and that many of the code snippets presented the same ideas repeatedly.
%Yet instructors did find some characteristics that supported their plan identification process. 
While in a few cases, instructors were able to find inspiration among the clusters of code snippets, this  condition received the lowest ranking by all but one participant.

\subsubsection{Condition 3: Prototype with Improved Navigation of Content.}

Participants found the
%interactions and the plan template useful
interactions in the prototype to be useful, and this condition was ranked the highest by most interviewees.  
While identical reference information as in earlier conditions was presented, the features of the prototype made it easier for instructors to navigate and use the given content. W4 indicated that the interactions for browsing candidate plans reduced the effort in reviewing many clusters in previous conditions. W2 and W7 appreciated the two-column view of example programs, where they could scan the list of natural language descriptions in one column  and click to review code on the other column if they find that use case interesting. W3 felt that the features of the prototype improved their efficiency over the earlier conditions:
% (``\textit{having the examples is half the battle}'', expressed W6)
\begin{quote}
    The Miro board itself doesn't really bring any value....
    %And then that's where the canvas here gets a 2 because I think 
    I better understand the ways where things like the use case palette and the library [in the prototype] could be used to quickly do the compositing. %\footnote{Participants rated each condition on a scale of 1 to 5, with 1 representing strongly positive, and 5 representing strongly negative opinions.}. 
\end{quote}

At the same time, participants requested ways to further reduce their search space, such as searching for important keywords (W2).
%Interestingly, we noted that participants very rarely wrote code from scratch while using our prototype. Most participants used the code from reference vmaterials and refined it to design their plans.
While the prototype addressed some concerns of instructors in the other conditions, our interviewees did request that some features of Miro boards be incorporated (e.g., highlighting code to emphasize changeable areas (W1)). When asked about additional features, almost all interviewees requested syntax highlighting on code examples and snippets, similar to the format of code editors.

W2, a Django instructor, found it challenging to represent plans in a single template as many Django programs are spread over multiple files. In such domains, visualizing connections between multiple plans could be helpful. 


% Exactly the same content, but 


\subsection{Design Goals}
\label{sec:design-goals}

% another option
Reflecting on our findings from this workshop, we formulate four design goals to inform the development of PLAID (or other systems) that assist instructors in designing programming plans with LLM-generated content.
% In this section, we describe four design goals summarizing our findings to inform the development of systems for assisting instructors in designing programming plans with LLM-generated content.

% many - number
% diverse - concepts covered?
% authentic - represent real-world practice
Mirroring the process of reviewing code from Section~\ref{sec:challenges_practice}, we observed that instructors valued the availability of many complete example programs for inspiration and brainstorming.
% as it helped them brainstorm ideas for designing plans. 
% While all participants were requested to design plans using our suggested plan components, we noted that instructors had different opinions about how they defined each component.
However, the ways in which instructors interacted with these examples varied based on their personal values. %, which affected how they used the example programs to design plans. 
Some instructors emphasized the importance of having contextualized examples in their plans to motivate students. 
Others focused on incorporating best practices or conventional techniques that 
% often come up in large projects and 
are important to remember. 
So, PLAID should be able to present instructors with \textbf{authentic} code examples that capture context and key functionality at the same time.
Some instructors also described multiple approaches to instantiate the same plan, emphasizing the different ways of achieving the same goal. 
%To enable instructors to highlight different solutions or find the ideal implementation,
PLAID should provide \textbf{diverse} examples to instructors to help them capture these different approaches.
In addition to having access to a corpus of programs, the content should be \textbf{displayed in a compact view} to enable instructors to easily navigate and search for key concepts.
% find what they are looking for.
% Thus, PLAID should present \textbf{many} examples to address the wide-ranging needs of instructors.

\begin{quote}
    \textbf{DG1: PLAID should inspire instructors by presenting many diverse and authentic examples in a compact view.}
    % as sources of inspiration for instructors.}
\end{quote}

% Viewing many examples is key to helping instructors create plans that represent best practices and industry standards. 
While instructors have expertise in their target domains, the dynamic nature of domain-specific libraries (e.g. Pandas) makes it hard for experts to keep up with up-to-date conventions and practices.
% the deprecations or updates to methods and functions. 
In our design workshop, when participants saw strange syntax, they used external resources like library documentation and web searches to understand unfamiliar code. 
% to review concepts and syntax.
So, PLAID should incorporate mechanisms that can help instructors \textbf{understand} unfamiliar programming paradigms. 
Moreover, most computing instructors write programs using code editors where they can frequently run code to review their implementations. So, PLAID should allow instructors to \textbf{view the output} of their programs to validate and refine their material.
% who use code editors for.

\begin{quote}
    \textbf{DG2: PLAID should assist instructors in mitigating uncertainties with example code.}
\end{quote}

\begin{figure*}[h]
        \Description{Three annotated screenshots, showing the buttons and menus explained in the text.}
        \includegraphics[width=\textwidth]{img/jane-workflow-new.png}
        \caption{Jane's Workflow Diagram. Jane (1) asks the system to suggest a plan; (2) edits the name and goal components of her plan, (3) marks the changeable areas in the plan, (4) browses the list of use case to find code that meets her goals, (5) selects the relevant part of the code from the full program and creates a plan from the selection, (6) switches to the full programs tab to search for specific pieces of code, (7) uses the search bar to search for keywords, and (8) groups plans with similar goals.}
        \label{fig:jane-workflow}
\end{figure*}

A key challenge indicated by instructors was the tiresome nature of combining reference content from many different sources to design plans. Exploring multiple sources was distracting, and navigating various platforms reduced their efficiency. Moreover, we saw that some instructors preferred to use the given material as an initial draft and refine it rather than write code from scratch. 
So, PLAID should support efficient interactions for editing reference material to \textbf{speed up} plan refinement. Participants indicated that it is challenging to abstract high-level ideas from multiple similar programs, so PLAID should \textbf{support abstraction} by allowing instructors to quickly combine content from a variety of sources.
% \textbf{interactions that can allow instructors to structure reference content to design plans with great efficiency.}

\begin{quote}
    % \textbf{DG3: PLAID should accelerate instructors in structuring reference content by providing efficient interactions for designing plans. PLAID should also facilitate abstracting high-level ideas from multiple code snippets. .}
    % \textbf{PLAID should accelerate instructor workflows with efficient interactions for designing plans and producing abstract summaries.}


    \textbf{DG3: PLAID should accelerate instructor workflows with efficient interactions for abstracting concepts and designing plans.}


    % \textbf{PLAID should accelerate abstracting concepts and using reference content for designing plans.}

    % \textbf{PLAID should accelerate abstracting concepts using reference content for designing plans.}
\end{quote}




To adopt plan-based pedagogies in courses, instructors need to make decisions about how to organize and present plans to students. 
In our design workshop,
% participants expressed uncertainties about how they would adopt plan-based pedagogy in their existing curricula.
we observed that participants attempted to document how they would use plans during instruction by clustering plan boxes on the canvas by theme. They also ordered plans in the sequence that students should learn about them.
% Moreover, they used the plan components to add explanations 
% as variants of a question for creating assessments and ordering plans in a hierarchy arranged by difficulty for organization on lecture modules.
So, PLAID should present instructors with ways of organizing content, as they \textbf{navigate their pedagogical concerns} about how students should be learning about plans.
% by helping them move towards organizing plans as they would present them to students.

\begin{quote}
    \textbf{DG4: PLAID should support instructors in organizing content to address pedagogical concerns.
    % about ways of adopting plan-based pedagogy in their curricula.
    } %Anticipating 
\end{quote}


% Using reference material effectively
% < Highlight code in full code and code pane in tab1 and make a plan
% < Add a button to add full program as a plan too

% Making sense of the content: % Viewing metainformation or contextual information about the content % Exploring code functionality and context
% < Code explanation plugin for strange syntax (GPT)
% < Code execution feature

% Getting inspiration from content / being able to find what you need to find: % Viewing many examples
% < Search in the use cases (and full progs)
% < Duplicating plans
% < Keyword search/embedding filter for potential values
% < Show use case button in solution (add highlighting)

% Organizing plans effectively
% Grouping plans into categories
% < Multiple selection of the boxes
% < Naming groups of boxes



% \subsection{Practical Illustration}

To understand how instructors use can PLAID to more easily adopt plan-based pedagogies, we follow Jane, a computer science instructor using PLAID to design programming plans for her course (summarized in \cref{fig:jane-workflow}).

Jane is teaching a programming course for psychology majors and wants to introduce her students to data analysis using Pandas. As her students have limited prior programming experience and use programming for specific goals, she organizes her lecture material around programming plans to emphasize purpose over syntax. 
% that explain practical concepts to students and help them focus on the purpose behind the code they write.
% However, she realizes that all introductory computer science courses offered at her institution only teach basic programming constructs like data structures. After exploring Google Scholar for effective instructional methods to teach application-focused programming to non-computer science majors, she learned about plan-based pedagogies that help them focus on the purpose behind the code they write. In her literature review, she finds out about PLAID, a tool that can help her design domain-specific plans. She reviews the domains supported by the tool (Pandas, Pytorch, Beautifulsoup, and Django) and decides to use Pandas, a popular and powerful data analysis and manipulation library, to create her curriculum. 

She logs in to the PLAID web interface, % and takes time to explore the system's features. 
and asks PLAID to suggest a plan (\cref{fig:jane-workflow}, 1). The first plan recommended to her 
% she sees is a plan to help students learn about
is about reading CSV files. 
She thinks the topic is important and the solution code aligns with her experience; % the solution is promising and represents an important concept that students need to know about.
% She is satisfied with the given solution 
but she finds the generated name and goal to be too generic. She edits (\cref{fig:jane-workflow}, 2) these fields to provide more context that she feels is right for her students.
% She refines those fields and then 
To make this plan more abstract and appropriate for more use cases, %explain how this plan can be used for reading data from different file formats,
she marks the file path as a changeable area (\cref{fig:jane-workflow}, 3), generalizing the plan for reading data from different file formats.

Inspired by the first plan, she decides to create a plan for saving data to disk. She wants to teach the most conventional way of saving data, so she switches to the use case tab (\cref{fig:jane-workflow}, 4) and explores example programs that save data to get a sense of common practices.  %interact with the list of complete programs.
She finds a complete example where a DataFrame is created and and saved to a file. %performs cleaning tasks like deleting NaN values, and exports it.
% She realizes that something she hadn't thought of before: saving new data is almost always necessary after performing data manipulation operations!
She selects the part of the code that exports data to a file and creates a plan from that selection (\cref{fig:jane-workflow}, 5).


For the next plan, she reflects on her own experience with Pandas. She recalls that merging DataFrames was a key concept, but cannot remember the full syntax. 
% Jane reflects on her experience working with Pandas and recalls that merging DataFrames is a key operation when working with big data.
She switches to the full programs tab (\cref{fig:jane-workflow}, 6) that includes complete code examples and searches (\cref{fig:jane-workflow}, 7) for ``\texttt{.merge}'' to find different instances of merging operations. % and tries to use the search bar to find a relevant program that contains ``.merge''. 
After finding a comprehensive example, she selects the relevant section of the code and creates a plan from it.
% She again selects a part of the example, creates plan from the selection, and refines it. She engages with the system iteratively and designs twenty plans for her lecture. 

After designing a set of plans that capture the important topics, she organizes them into groups (\cref{fig:jane-workflow}, 8) 
% also grouped similar plans together
to emphasize sets of plans with similar goals but different implementations. For instance, she takes her plans about \texttt{.merge} and \texttt{.concat} and groups them together to form a category of plans that students can reference when they want to {combine data from different sources}.

% combining data using ``merge'' or ``concat''.

% the the she used plans isn't very good right now
% She exports these plans and starts preparing her lecture slides, using the plans as a way of presenting key concepts to students with minimal programming experience.
She exports these plans to support her students with minimal programming experience by preparing lecture slides that organize the sections around plan goals, using plan solutions as worked examples in class, and providing students with cheat sheets that include relevant plans for their laboratory sessions.
% using the plan goals as titles for different sections of her slides, and using the solutions as references for the examples she creates. Finally, she makes a PDF cheatsheet with all the plans for students to reference during the week's laboratory.
% The next day, she starts preparing her lecture slides and realizes that the names and goals she wrote for her plans represent key concepts in Pandas. She references the plans she created to design annotated examples that she includes on her lecture slides.

%% How does Jane actually use the plans? 
%% > Important to be careful to note that this isn't actually part of the system....
%% > She uses the generated plans to (a) as inspiration for worked examples in teh course, (b) as stems for questions that test how code should be completed
%% > She notices she now has a list of key concepts in the area



%PlanGREEN

%GEN-Plan

%G- generate
%R-refine
%E- edit

%% GREEN-plan

%% PURPLE
\begin{figure*}
    \centering
    \Description{PLAID's system architecture diagram. Top part shows the database (a), and bottom part shows the interface (b). The system starts from bottom right as an instructor is interested in a programming domain, then the pipeline described in the text produces reference materials at different levels of granularity, and these are presented in the interface.}
    \includegraphics[width=\textwidth]{img/system-architecture-subgoals.png}
    \caption{PLAID's reference content is generated through an LLM pipeline
    %inspired by the practices of instructors who have successfully identified programming plans. 
    that produces output on three levels.
    First, a wide variety of use cases are generated to create example programs that focus on code's applications. Next, using LLM's explanatory comments that represent subgoals within the code, the examples are segmented into meaningful code snippets. The LLM is queried to generate other plan components for each code snippet. Finally, the code snippets are clustered to identify the most common patterns, representing plan candidates. The full programs are presented in `Programs' views of PLAID interface, whereas snippets are presented in clusters in the `Plan Creation' view.}
    \label{fig:system-pipeline}
\end{figure*}
\section{PLAID: A System for Supporting Plan Identification}
\label{sec:system-design}

Following the design goals devised from the design workshop, we refined our early prototype into PLAID: a
%LLM-powered
tool to assist instructors in their plan identification process.
PLAID synthesizes the capabilities of LLMs in code generation with interactions enabling plan identification practices observed in our studies with instructors.
As we noted in the findings of our design workshop, the LLM-generated candidate plans are not ready to be used as is in instruction, but instructors can readily adapt and correct them (\cref{sec:workshop-findings-condition2}).
PLAID enables collaboration between instructors and LLMs, enhancing the plan identification process by automating its time-intensive information-gathering tasks and facilitating instructors' ability to refine LLM-generated candidate plans based on their knowledge about pedagogy and the programming domain. 



\subsection{Practical Illustration}

To understand how instructors use can PLAID to more easily adopt plan-based pedagogies, we follow Jane, a computer science instructor using PLAID to design programming plans for her course (summarized in \cref{fig:jane-workflow}).

Jane is teaching a programming course for psychology majors and wants to introduce her students to data analysis using Pandas. As her students have limited prior programming experience and use programming for specific goals, she organizes her lecture material around programming plans to emphasize purpose over syntax. 
% that explain practical concepts to students and help them focus on the purpose behind the code they write.
% However, she realizes that all introductory computer science courses offered at her institution only teach basic programming constructs like data structures. After exploring Google Scholar for effective instructional methods to teach application-focused programming to non-computer science majors, she learned about plan-based pedagogies that help them focus on the purpose behind the code they write. In her literature review, she finds out about PLAID, a tool that can help her design domain-specific plans. She reviews the domains supported by the tool (Pandas, Pytorch, Beautifulsoup, and Django) and decides to use Pandas, a popular and powerful data analysis and manipulation library, to create her curriculum. 

She logs in to the PLAID web interface, % and takes time to explore the system's features. 
and asks PLAID to suggest a plan (\cref{fig:jane-workflow}, 1). The first plan recommended to her 
% she sees is a plan to help students learn about
is about reading CSV files. 
She thinks the topic is important and the solution code aligns with her experience; % the solution is promising and represents an important concept that students need to know about.
% She is satisfied with the given solution 
but she finds the generated name and goal to be too generic. She edits (\cref{fig:jane-workflow}, 2) these fields to provide more context that she feels is right for her students.
% She refines those fields and then 
To make this plan more abstract and appropriate for more use cases, %explain how this plan can be used for reading data from different file formats,
she marks the file path as a changeable area (\cref{fig:jane-workflow}, 3), generalizing the plan for reading data from different file formats.

Inspired by the first plan, she decides to create a plan for saving data to disk. She wants to teach the most conventional way of saving data, so she switches to the use case tab (\cref{fig:jane-workflow}, 4) and explores example programs that save data to get a sense of common practices.  %interact with the list of complete programs.
She finds a complete example where a DataFrame is created and and saved to a file. %performs cleaning tasks like deleting NaN values, and exports it.
% She realizes that something she hadn't thought of before: saving new data is almost always necessary after performing data manipulation operations!
She selects the part of the code that exports data to a file and creates a plan from that selection (\cref{fig:jane-workflow}, 5).


For the next plan, she reflects on her own experience with Pandas. She recalls that merging DataFrames was a key concept, but cannot remember the full syntax. 
% Jane reflects on her experience working with Pandas and recalls that merging DataFrames is a key operation when working with big data.
She switches to the full programs tab (\cref{fig:jane-workflow}, 6) that includes complete code examples and searches (\cref{fig:jane-workflow}, 7) for ``\texttt{.merge}'' to find different instances of merging operations. % and tries to use the search bar to find a relevant program that contains ``.merge''. 
After finding a comprehensive example, she selects the relevant section of the code and creates a plan from it.
% She again selects a part of the example, creates plan from the selection, and refines it. She engages with the system iteratively and designs twenty plans for her lecture. 

After designing a set of plans that capture the important topics, she organizes them into groups (\cref{fig:jane-workflow}, 8) 
% also grouped similar plans together
to emphasize sets of plans with similar goals but different implementations. For instance, she takes her plans about \texttt{.merge} and \texttt{.concat} and groups them together to form a category of plans that students can reference when they want to {combine data from different sources}.

% combining data using ``merge'' or ``concat''.

% the the she used plans isn't very good right now
% She exports these plans and starts preparing her lecture slides, using the plans as a way of presenting key concepts to students with minimal programming experience.
She exports these plans to support her students with minimal programming experience by preparing lecture slides that organize the sections around plan goals, using plan solutions as worked examples in class, and providing students with cheat sheets that include relevant plans for their laboratory sessions.
% using the plan goals as titles for different sections of her slides, and using the solutions as references for the examples she creates. Finally, she makes a PDF cheatsheet with all the plans for students to reference during the week's laboratory.
% The next day, she starts preparing her lecture slides and realizes that the names and goals she wrote for her plans represent key concepts in Pandas. She references the plans she created to design annotated examples that she includes on her lecture slides.

%% How does Jane actually use the plans? 
%% > Important to be careful to note that this isn't actually part of the system....
%% > She uses the generated plans to (a) as inspiration for worked examples in teh course, (b) as stems for questions that test how code should be completed
%% > She notices she now has a list of key concepts in the area


\begin{figure*}[h]
        \Description{An annotated screenshot of PLAID's `Programs' view. On the left, a list of use cases such as `Renaming columns in a Frame' and `Plotting a histogram of a column' is shown, with a scrollable list and a search bar. The latter one is selected, and on the right, we see the contents of the program in a monospaced font, with four buttons explained in the caption.}
        \includegraphics[width=\textwidth]{img/system-diagram-1-fixed.png}
        \caption{Plan Identification using PLAID: (a) list of example programs for instructors organized by natural language descriptions, (b) list of full programs of code, (c) search bar enabling easy navigation of given content to find code for specific ideas, (d) button to create a plan using the selected part of the code, (e) button to create a plan using the complete example program, (f) button to view an explanation for a selected code snippet, and (g) button for executing the selected code.}
        \label{fig:system-diagram-1}
\end{figure*}

\subsection{System Design}

At a high level, PLAID\footnote{The code for PLAID can be found at: https://github.com/yosheejain/plaid-interface.} operates on two subsystems: (1) a database of LLM-generated reference materials created through a pipeline that uses \edit{OpenAI's GPT-4o\footnote{https://openai.com/index/hello-gpt-4o/}~\cite{achiam2023gpt}}, inspired by instructors' best practices for identifying programming plans (see ~\cref{fig:system-pipeline})
%LLM for identifying plans in application-focused domains 
and (2) an interface that allows instructors to browse reference materials for relevant code snippets 
% and other plan components to achieve a goal that meets their needs. Then, they refine the candidates to mine plans 
and refine suggested content into programming plans
(see Figures~\ref{fig:system-diagram-1} and~\ref{fig:system-diagram-2}).
% In this section, we describe the implementation of the pipeline generating the reference materials and the key interface features of PLAID.



\subsubsection{Database of Reference Materials for Application-Focused Domains}

PLAID extracts information from reference materials at three levels of granularity to support each instructor's unique workflow: complete programs that address a particular use case, annotated program snippets that include goals and changeable areas, and plan candidates that cluster relevant program snippets together.

\textbf{Generating complete example programs.}
The content at the lowest level of granularity in the PLAID database are the complete programs. 
%These candidate plans were generated using a pipeline to generate \textit{plan-ful examples}, which we define as examples of programming plans in use, with all plan components identified (see Section~\ref{sec:components}). This implementation had three stages: (1) generating in-domain programs, (2) segmenting programs into plan-ful examples, and (3) clustering plan-ful examples into plans. 
\label{sec:llm-pipeline}
% \begin{figure}
% \centering
% % \includegraphics[width=0.5\textwidth]{img/pipeline-new.png}
% \includegraphics[width=\textwidth]{img/new-plan-pipeline.png}
% \caption{The three stage process for generating example programs, segmenting them with plan components, and clustering these plan-ful examples.
% %collecting and processing responses from ChatGPT into plan-ful examples}
% %\caption{The pipeline for LLM plan generation.}
% }
% \label{fig:llm-methods}
% \end{figure}
% \subsubsection{Generating In-Domain Programs}
% Informed by the insights identified in our interview study, we generated programming plans relevant to an application-focused domain: web scraping via BeautifulSoup. We utilized an LLM-based approach to generate these plans with the GPT-4 model from OpenAI using its publicly available API in an iterative workflow. 
% Our participants examined example programs and conducted literature reviews (Section \ref{sec:viewing-programs}) as key parts of their plan identification process. 
As these examples should capture a variance of use cases in the real world, we utilized an LLM trained on a large corpus of computer programs and natural language descriptions~\cite{liu2023isyourcode}.
% Inspired by this, we used Open AI's GPT-4, a state-of-the-art large language model for code generation that is trained on a large corpus of computer programs~\cite{liu2023isyourcode},
% to generate candidate programs along with its respective plan components in the programs.
We prompted\footnote{Full prompts can be found in \cref{sec:appendix-pipeline}.} the model to generate \texttt{specific use cases of <application-focused library>}, defining use case as \texttt{a task you can achieve 
with the given library} (see \cref{sec:use_case_prompt}). Subsequently, we prompted the model to \texttt{write code to do the following: <use case>}, producing a set of 100 example programs with associated tasks (see \cref{sec:code_prompt}). By generating the use cases first and generating the solution later, we avoided the problems with context windows of LLMs where the earlier input might get `forgotten', resulting in the model producing the same output repeatedly. For practical purposes, we generated 100 programs per domain. \edit{To test for potential ``hallucinations'' where the LLM generates plausible yet incorrect code~\cite{Ji_2023_hallucination}, we checked the syntactic validity of the generated programs before developing the rest of our pipeline. No more than one out of 100 generated programs included syntax errors in each of our domains, i.e., Pandas, Django, and PyTorch. Thus, we concluded that hallucinations are not a major threat to the code generation aspect of PLAID.}
%while hallucinations in LLMs are a pressing concern for systems that utilize these models,
% This collection of example programs (which we refer to as 
%dataset 
% $\mathcal{D}$) was used as our primary dataset for further analysis.

\textbf{Generating annotated program snippets.}
% \subsubsection{Segmenting Programs Into Plan-ful Examples}
% We then proceed to compile these examples with each of the plan components generated using ChatGPT. We construct a new dataset with these components, Dataset \((\mathcal{D}^{\textit{Comp}})\).
The second level of granularity in PLAID consists of small program snippets and a goal, with changeable areas annotated. 
We used the generated programs from
% \mathcal{D}$
the prior step as the input to the LLM to add subgoal labels, where we prompted the LLM to annotate subgoals (see \cref{sec:subgoals_prompt}) as comments that describe \texttt{small chunks of code that achieve a task that can be explained in natural language}. These subgoal labels were used to break the full program into shorter snippets. Each snippet was fed back to the model to generate changeable areas (see \cref{sec:ca_prompt}), defined in the prompt as \texttt{parts of the idiom that would change when it is used in different scenarios}. The subgoal label that explained a code snippet corresponded to its goal in the plan view and the list of elements assigned as changeable was used for annotations.
% (see Stage 2 in Figure~\ref{fig:llm-methods}),
%We fragmented these generated programs into smaller code pieces by generating \textit{subgoals} in the program. Then, each goal (Section \ref{sec:goal}) and the accompanying code solution (Section \ref{sec:solution}) were added as a single unit of data in our plan-ful example dataset of components, \(\mathcal{D}^{\textit{Plan-ful}}\). For each of these datapoints, we prompted the model to identify \textit{changeable areas} (Section \ref{sec:changeable}). %The name (Section \ref{sec:name}) was determined later in the pipeline (Stage 2 in Figure \ref{fig:llm-methods}).


% From the results of our qualitative study, we now know about the parts of a programming plan. In order to extract these plans automatically, we used ChatGPT. We accessed it using its publicly available API and we used the GPT-4 model. We selected 3 domains that are interesting for non-majors. This included . 

% For each of these domains, we first asked the LLM to generate 100 use cases. We then re-prompted it with the use cases it generated and asked it to generate code that would be written to accomplish that use case.
% potential for another table?
% add code metrics from stackoverflow github work for chatgpt
% With all these code pieces collected, we then asked ChatGPT to generate each of the plan parts one-by-one.

% \subsubsection*{Extracting Goals and Solutions}Generated programs 
% in \(\mathcal{D}\) 
% typically included a comment before each line, which described that line's functionality. However, these comments did not capture the high-level purpose of the code, as required by a plan goal. To generate more abstract goals for a piece of code, we defined subgoals as \texttt{short descriptions of small pieces of code that do something meaningful} in a prompt and asked the LLM to \texttt{highlight subgoals as comments in the code.} %In our query, we also added the way we define subgoals to provide the relevant context to the model. Specifically, we wrote that 
% The output from this prompt was a modified version of each program
% from \(\mathcal{D}\), 
% where blocks of code are preceded by a comment describing the goal of that block. % of code. % instead of restating functionality. 

% We split each complete program into multiple segments based on these new comments. Thus, the subgoal comments from each complete program I
% n the modified \(\mathcal{D}\) 
% became a plan goal, and the code following that comment became the associated solution. %, collected in \(\mathcal{D}^{\textit{Plan-ful}}\). % After it returned the annotated code piece, we extracted the comment and the following lines of code before the next comment. This pair acted as a subgoal-code piece. We collected all such pairs across all use cases from \(\mathcal{D}\) and added them to \(\mathcal{D}^{\textit{Plan-ful}}\).
% Each goal 
% %(Section \ref{sec:goal}) 
% and solution pair
% %(Section \ref{sec:solution}) 
% was added as a single unit of data in our plan-ful example dataset.
% , \(\mathcal{D}^{\textit{Plan-ful}}\).

% \subsubsection*{Extracting Changeable Areas}To annotate the changeable areas for a plan, we defined changeable areas as \texttt{parts of the plan that would change when it is used in a different context} in our prompt and asked the model to \texttt{return the exact part of the code from the line that would change} for all code pieces from the dataset with plan-ful examples.
% from \(\mathcal{D}^{\textit{Plan-ful}}\). 
% This data was added to \(\mathcal{D}^{\textit{Plan-ful}}\).

% to-do
% \subsubsection{Clustering Plan-ful Examples into Plans}
\textbf{Generating clustered plan candidates.}
\label{sec:clustering}
% We perform k-means clustering on the plans \(\mathcal{D}^{\textit{Plan-ful}}\) to identify clusters of similar code pieces and thus, programming plans.
The highest level of granularity provided in PLAID
%presents users with 
are
plan candidates, in the form of clusters of annotated program snippets. To compare the similarity of program snippets, we used CodeBERT embeddings following prior work~\cite{codebert} and applied Principal Component Analysis (PCA) \cite{PCAanalysis} to reduce the dimensionality of the embedding while preserving 90\% of the variance. The snippets were clustered using the K-means algorithm~\cite{kmeansclustering}, using the mean silhouette coefficient for determining optimal K~\cite{silhouettecoeff}. Each cluster is treated as a plan candidate, with the goal, code, and changeable areas from each program snippet in the cluster presented as a suggested value for the plan attributes.
% We used a clustering algorithm to group similar program snippets 
% plan-ful examples together as a programming plan. For clustering the code pieces, we used the CodeBERT model from Microsoft \cite{codebert} to obtain embeddings for each code piece in our dataset of plan-ful examples
% % in \(\mathcal{D}^{\textit{Plan-ful}}\) 
% and applied Principal Component Analysis (PCA) \cite{PCAanalysis} to reduce the dimensionality of the embedding vectors while preserving 90\% of the variance. These embeddings were clustered using the K-means algorithm~\cite{kmeansclustering}. The optimal number of clusters \(\mathcal{K}\) was determined by assessing all possible \(\mathcal{K}\) values 
% % (where \(\mathcal{K} \in [2, \texttt{length}(\mathcal{D}^{\textit{Plan-ful}})]\))
% using the mean silhouette coefficient \cite{silhouettecoeff}. We assigned each example 
% % in \(\mathcal{D}^{\textit{Plan-ful}}\) 
% to a cluster of similar code pieces. 
% \subsubsection*{Extracting Names}
For each plan candidate, a name (see \cref{sec:name_prompt}) that summarizes all snippets in the cluster was generated by prompting an LLM with the contents of the snippets and stating that it should generate \texttt{a name that reflects the code's purpose} and it should focus on \texttt{what the code is achieving and not the context}. 
% Then, all code snippets from each cluster of examples were provided as input to the LLM along with a prompt asking it to \texttt{devise a name for that cluster of plans}.

% \subsection{Interface for Refining Candidate Plans}

% %nd the back-end server relied on routes written in Flask. The domain-specific candidate plans suggested to the user are queried from the database of candidate plans generated using the LLM. Each participant was required to log in to the web page using their unique credentials, which allowed us to record their activity for analysis. While the complete details of our implementation of the web-based application are out of scope for this paper, we describe its main features in Section~\ref{sec:implementation_of_webinterface}.

% \subsubsection{\edit{Preliminary Technical Evaluation of Generated Content}}

% \edit{syntactic validity and standard code complexity metrics to determine
% their suitability for novices}


\begin{figure*}[h]
    \Description{An annotated screenshot of PLAID's Plan Creation view with three panes, with plans shown as boxes on the left. A plan is highlighted, and we see its components on the middle pane. On the rightmost pane, we see suggested values for the selected component.}
    \includegraphics[width=\textwidth]{img/system-diagram-2-new.png}        
    \caption{Plan Identification using PLAID: (h) button that suggests a domain-specific candidate plan from the system database, (i) pane enabling viewing of similar values for the selected plan component, (j) button to view the solution code as part of a complete program, (k) pane with a structured template for plan design with editable fields to refine plan components, (l) button to copy a selected plan, (m) button to mark snippets of code from the plan solution as changeable areas, and (n) a button to group plans together into a category and add a name.}
    \label{fig:system-diagram-2}
\end{figure*}

% \subsubsection{Key Characteristics}
% PLAID supports the process of plan identification in data processing with Pandas, machine learning with Pytorch, web development using Django, and web scraping using BeautifulSoup. 

\subsubsection{Interface for Designing Programming Plans}
Building on the 
%characteristics addressed in the artifact (Section~\ref{sec:design-artifact}) and 
design goals identified in the design workshop (\cref{sec:design-goals}), PLAID enables a set of key interactions to assist instructors in refining candidates to design plans for their instruction. 



\textbf{Interactions for Initial Plan Identification.}
% Initial Plan Identification with Quick Exploration of Many Authentic Programs
While instructors valued the availability of code examples in the design workshop (Section~\ref{sec:design-workshop-findings}), we observed many opportunities for scaffolding their interaction with the reference material. To this end, PLAID presents example programs in two different views \textbf{(DG1)}. 
% We saw instructors scanning examples, selecting desired code pieces, and copying them over into their plan templates in all conditions in the study. 
The ``Programs (Organized by Use Case)'' (\cref{fig:system-diagram-1}a) tab includes a list of use cases where instructors can click on an item to expand the program for that use case.
The ``Programs (Full Text)''  tab (\cref{fig:system-diagram-1}b) lists all the programs and enables instructors to scroll or search through (\cref{fig:system-diagram-1}c) all the code at once.
% presents the contents of all the programs expanded viewing a list of complete code examples, allowing instructors to look at materials they would typically search for when designing plans.
% equipping instructors with full-code programs organized in a list of short natural language descriptions of common use cases in their domain of expertise. 
Both views support directly creating a plan from the whole example (\cref{fig:system-diagram-1}e), or a selected part of it (\cref{fig:system-diagram-1}d), by copying the solution and the goal of the program into an empty plan template
% < Highlight code in full code and code pane in tab1 and make a plan (D1)
% < Add a button to add full program as a plan too (D1)
further supporting efficient use of the material \textbf{(DG3)}.
% This interaction copies over the selected code and its respective use case into the solution and name fields, respectively. 
% < Code explanation plugin for strange syntax (GPT) (D2)

To facilitate understanding unfamiliar code and syntax, we implemented a ``View Explanation'' button (\textbf{DG2}) that generates a short description of the selected line(s) of code by prompting an LLM (\cref{fig:system-diagram-1}f). 
% In this case, participants hesitated to use the suggested syntax in their plans because its functionality was unclear to them. PLAID supports a button named ``View Explanation'' where the user can select a method, function, or line of code that is unclear and click on it to understand its working \textbf{(D2)}. 
Participants also looked for code execution to validate and understand a program. However, since the code snippets instructors work with are often incomplete in this task, we implemented a ``Run Code'' feature (\textbf{DG2}) that predicts the output of a selected code snippet by prompting an LLM to walk through the code \texttt{step by step}, using Chain-of-Thought prompting~\cite{wei2022chain} (\cref{fig:system-diagram-1}g). Only the predicted output for the code is presented, ignoring other output from the LLM.

% to examine the code behavior and thus mitigate the challenge of being faced with unfamiliar syntax. Thus, using PLAID, instructors are able to run complete programs to view their output \textbf{(D2)}.
% < Search in the use cases (and full progs) (D3)
% Frequently, instructors relied on their expertise and experience to formulate ideas about goals for which they wanted to create plans. While interacting with condition C in the design workshop, interviewees suggested including a mechanism to search for specific keywords within code and  its natural language description. To facilitate the instructor-LLM collaboration, allowing users to find examples implementing their ideas, PLAID includes a search bar that helps users navigate the given use cases, complete programs, and effectively find specific examples they may be looking for \textbf{(D3)}.

\textbf{Interactions for Plan Refinement.}
% Support Plan Refinement with Comparisons of content
% Participants indicated difficulty mining plans from code examples (Section~\ref{sec:challenges_practice}). 
To provide suggestions for code patterns common enough to be potential programming plans,
%To alleviate challenges in identifying content common enough for designing plans, 
we utilize the clustered program snippets from our database. In the ``Plan Creation'' view of PLAID, instructors can ask for suggestions (\cref{fig:system-diagram-2}h) to see a candidate plan to refine (\textbf{DG3}).  \edit{This functionality allows instructors to draw on their experience to recognize common code snippets and decide if they are valuable to teach students.}
% If instructors want to demonstrate their plan as part of a complete code example, they can review these examples reducing the effort that they would need to put in to recall syntax and construct a complete example. 
\edit{This promotes recognition over recall \cite{recognition_over_recall}, thus helping reduce the cognitive effort that instructors may have to put in while designing programming plans traditionally.}
To allow instructors to better understand the context of a plan under refinement, PLAID 
also includes a button for searching for the current solution within the entire set of full programs
%, showing the code snippet in context 
%as part of a complete example
(\textbf{DG3}, \cref{fig:system-diagram-2}j).
% < Keyword search/embedding filter for potential values (D1)

As instructors edit the components of a plan, they are shown similar values from the corresponding component in that cluster (\cref{fig:system-diagram-2}i). By clicking on any suggested value, instructors can replace a plan component with a suggestion that better captures that aspect of the plan \textbf{(DG1)}. \edit{By allowing instructors to view the plan they are working on along with other related code pieces in a split screen view, we promote instructor efficiency by reducing the split-attention effect \cite{tarmizi1988guidance}. In the current plan creation process, even when using LLMs from their chat interface, instructors would have to switch between windows with code examples and their text editor which may increase the load on the instructors' working memory \cite{clark2023learning}. In PLAID, instructors can edit their plans and view similar code pieces at the same time.}

% \edit{By enabling these interactions and thus organizing ``knowledge in the world'' effectively, PLAID reduces the need for instructors to store and retrieve the ``knowledge in their head'' \cite{Norman_DOET}. Thus, PLAID optimizes the plan creation process by allowing efficient search within the ``knowledge in the world'' and reducing the cognitive load while storing and retrieving ``knowledge in the head'', minimizing the total effort required \cite{} by instructors.}
% after searching its code corpus for similar examples using a keyword search \textbf{(DG1)}.
% < Show use case button in solution (add highlighting) (D2)
% To help instructors easily consider the context of a plan as they refine it, PLAID 
% In the design workshop, few instructors emphasized the importance of presenting worked and contextualized examples to students. 

% ‘go to a use case’ button that redirects the user to the tab with full code programs and highlights the plan as part of a complete example \textbf{(D2)}.

\textbf{Interactions for Building Robust and Shareable Plan Descriptions.}
% Support robust/sharable plan descriptions
% From Section~\ref{sec:process_intro_plan_design}, instructors indicated drawing on their experience in the application-specific domain and instructional expertise to think about how to best solve a problem. 
PLAID encourages instructors to design plans in a structured template (\cref{fig:system-diagram-2}k). Moreover, PLAID reinforces the plan template by providing a dedicated method for annotating changeable areas by highlighting any part of the code (\textbf{DG3}, \cref{fig:system-diagram-2}m). Instructors can further explain the changeable areas by adding comments as text.
% \edit{The structured template view of the plan encourages instructors to articulate their mental models of how the plan would generalize to other problems, allowing the transfer of ``knowledge in the head'' to ``knowledge in the world''.}

Our design workshop showed that participants would create a plan and copy it to emphasize alternatives or modifications to the underlying idea. To support this workflow,
% In our design workshop, participants created copies of their plans to display alternative solutions to achieve the same goal, emphasizing that multiple possible solutions in code could accomplish the same goal.
% < Duplicating plans (D3)
% To accelerate this process of teaching a variety of possible solutions, 
PLAID allows users to ``duplicate'' plans on the canvas and further edit them to present alternative solutions for the same plan \textbf{(DG3}, \cref{fig:system-diagram-2}l).
% Highlight text from solution to change it to changeable areas (highlighting code itself) (D4)

% In conditions A and B, instructors highlighted the changeable areas in the code itself.
% To allow participants to emphasize the changeable areas in code in PLAID, we implemented the ``add to changeable areas'' button. After selecting the changeable piece of code, clicking on this button highlights the text in a different color and adds it to the box of changeable areas to complete the templated plan design (\textbf{D4}).
% Grouping plans into categories (D4)
% < Multiple selection of the boxes (D4)
% < Naming groups of boxes (D4)
To encourage instructors to think about organizing plans in ways that they would present them to students, PLAID provides an open canvas view for instructors that allows them to arrange plans as they prefer. In addition, PLAID supports a ``grouping'' feature (\cref{fig:system-diagram-2}n), which allows instructors to combine plans with similar goals together into one category (\textbf{DG4}).

% A handful of users postulated each plan as an example question that can be used on assessments. They intended to create multiple variants of the same question for students. They suggested that being able to visualize the different categories would be helpful. Using PLAID, users can select multiple patterns together, add them to a group, and name the group \textbf{(D4)}.  % :(

\subsubsection{System Architecture}
The pipeline to create reference materials is implemented in Python, using the state-of-the-art large language model GPT-4o (Model Version: 2024-05-13). The interface for PLAID is implemented as a web application in Python as a Flask webserver, with an SQLite database. The user-facing interface is implemented using HTML, CSS, and JavaScript, with the canvas interactions realized with the library `\textit{interact.js}'. 




% \input{system-implementation}

\section{Evaluation of PLAID}
\label{sec:user-study}

To evaluate PLAID, we aimed to determine if computing instructors were able to use PLAID to identify plans in an application-specific programming domain more efficiently and with a more positive user experience than the current state-of-the-art. 
Specifically, we performed a within-subjects user study to gather insight into (1) instructors' productivity in the plan identification process, (2) the task load for using the system, and (3) the overall usability of PLAID. 
%With a within-subjects user study with 12 participants, we evaluate PLAID's performance in enabling instructors to create programming plans in application-focused domains. In this section, we detail our study design and report the key findings from all phases of the study.



\subsection{Study Design}
\subsubsection{Participants}
\edit{The target end-users for PLAID are computing instructors who intend to create instructional content to teach an application-specific computing course. So, we} recruited four instructors and eight graduate teaching assistants with at least a year of experience in teaching \edit{a programming course (see Table~\ref{tab:participants-evaluation}) at the undergraduate level whether in introductory or upper-level programming courses}. In addition to teaching experience, our inclusion criteria required participants to indicate expertise in at least one application-focused programming domain: data analysis with Pandas (six participants), machine learning using Pytorch (four participants), and web programming with Django (two participants). None of our participants had prior experience identifying plans for instruction. Each participant engaged in a 60-minute design session and was compensated with a \$50 Amazon gift card.

\subsubsection{Procedure}
We conducted a within-subjects study, where each instructor performed plan identification with a baseline condition representing the current state-of-the-art and with PLAID. The study was counterbalanced, with half of our participants seeing the baseline condition first and the other half seeing PLAID first.

Each session began with a description of what a programming plan is, using an example from introductory programming. Then, participants were given 15 minutes in each condition to identify programming plans in their application-focused domain. We prompted them to create these plans as if they will be used in lectures that teach important concepts to students with no experience in their domain. To encourage instructors to undertake a significant amount of plan identification, they were given a suggested target of four to five plans. They were encouraged to continue if they reached this goal before their time was over. %In each condition, participants were tasked with creating five plans or as many as possible within 15 minutes.
After each condition, participants completed the NASA-TLX questionnaire~\cite{cao2009nasa} to indicate their workload while performing the task. % across six dimensions.

\textit{Baseline condition.} Participants worked on an empty Google document. They were given one example programming plan from introductory programming, which they could refer to to understand the expected plan structure. Participants were allowed access to external resources, including web searches, ChatGPT or other AI tools, or their own code and content. 


\textit{PLAID condition.} Participants were given access to the PLAID web interface after a short demonstration of fundamental interactions supported by the system by the interviewer. Like the baseline condition, they were allowed to access any external resources besides the content suggested by PLAID.




\subsubsection{Post-task reflection.}
% In addition to collecting qualitative data in the think-aloud settings of both tasks, we utilized two surveys to quantitatively evaluate (1) the workload of participants under each condition and (2) the usability of PLAID.
 
The sessions ended with a short interview, asking participants for feedback on the system and their opinions on plan-based pedagogy.
Participants also evaluated PLAID with the PSSUQ Version 3 usability survey \cite{pssuq_usability, sauro2016quantifying}. 

\begin{table}
\caption{Demographics of the Participants in our User Study.}
    \centering
    \footnotesize
    \label{tab:participants-evaluation}
    \begin{tabular}{l|cccccc}
    \toprule
            & Domain & Academic Title & \shortstack{Teaching \\ Experience \\ in CS} & \shortstack{Teaching \\ Experience \\ in Domain} & \shortstack{Used \\ Plans in \\ Instruction?}
    \\\midrule
        E1 & Django & Instructor & 20+ & 1-5 & No \\
        E2 & Pandas & Instructor & 1-5 & 1-5 & No \\
        E3 & Django & Instructor & 11-15 & 6-10 & No \\
        E4 & Pytorch & Graduate TA & <1 & <1 & No \\
        E5 & Pandas & Instructor & 1-5 & 1-5 & No \\
        E6 & Pytorch & Graduate TA & 1-5 & <1 & No \\
        E7 & Pandas & Graduate TA & 1-5 & <1 & No \\
        E8 & Pytorch & Graduate TA & 1-5 & <1 & No \\
        E9 & Pandas & Graduate TA & 1-5 & <1 & No \\
        E10 & Pytorch & Graduate TA & 1-5 & <1 & No \\
        E11 & Pandas & Graduate TA & 1-5 & 1-5 & No \\
        E12 & Pandas &  Instructor & 20+ & 6-10 & No \\
    %\\\bottomrule
    \end{tabular}%
\end{table}

\begin{figure}[h]
    \centering
    \Description{A box plot with seven pairs of horizontal bars. Each bar corresponds to one of the measures on NASA TLX, with the top bar being the overall score. Median value for PLAID is better than baseline for all measures, and the difference is significant for overall score, physical demand, and mental demand.}
    \includegraphics[width=\linewidth]{img/cog-load-new.png}
    \caption{Participants' responses on the NASA Task Load Index survey administered after both the baseline condition and PLAID condition. For all items, lower values are preferred. The chart also indicates the results of the Wilcoxon signed rank test between the baseline and PLAID conditions. $**$, $*$, $ns$ indicate $p < 0.01$, $p < 0.05$, and $p > 0.5$ respectively.}
    \label{fig:cognitive-load}
\end{figure}

\begin{figure*}[h]
    \Description{Bar plots showing the distribution of responses for each item across system usability (SYSUSE), information quality (INFOQUAL), and interface quality (INTERQUAL) on the PSSUQ survey. For most items, The majority of items are rated above then the median option (Neither Agree or Disagree).}
    \includegraphics[width=\textwidth]{img/pssuq-chi-25.pdf}
    % \centering
    % \begin{subfigure}{0.48\textwidth}
    %     \centering
    % \includegraphics[width=\textwidth]{img/legend-usability.png}
    % \end{subfigure}
    % \newline
    % \newline
    % \begin{subfigure}{0.32\textwidth}
    %     \centering
    %     \includegraphics[width=\textwidth, trim=0 4 0 0, clip]{img/sysuse.png}
    %     \caption{SYSUSE}
    %     \label{fig:subfig-a}
    % \end{subfigure}
    % \hfill
    % \begin{subfigure}{0.32\textwidth}
    %     \centering
    %     \includegraphics[width=\textwidth, trim=0 4 0 0, clip]{img/infoqual.png}
    %     \caption{INFOQUAL}
    %     \label{fig:subfig-b}
    % \end{subfigure}
    % \hfill
    % \begin{subfigure}{0.32\textwidth}
    %     \centering
    %     \includegraphics[width=\textwidth, trim=0 4 0 0, clip]{img/interqual.png}
    %     \caption{INTERQUAL}
    %     \label{fig:subfig-c}
    % \end{subfigure}
    
    \caption{Self-reported reflections of participants on the usability of PLAID using the PSSUQ survey. The graph encapsulates their responses for each question across each category on a 7-point Likert Scale.}
    \label{fig:three-horizontal}
\end{figure*}

\subsection{Findings}

% \subsubsection{Does the system work?}

% \subsubsection{Task efficiency}
\subsubsection{PLAID enables instructors to identify plans more efficiently.}
Participants created more plans when using PLAID (4.75 plans on average) compared to the baseline condition (3.92 plans on average).
Seven participants were able to reach the target of identifying five plans in the PLAID condition, whereas only three of twelve participants were able to identify five plans in the baseline condition.
% Most participants created more plans using PLAID than the baseline condition (7 participants).
% Seven participants in the PLAID condition were able to identify five or more plans, whereas only three participants managed to achieve this goal in the baseline condition.
% Participants also spent less time identifying each plan: on average, they spent 8 minutes and 53 seconds per plan in PLAID, compared to the average of 9 minutes and 38 seconds per plan in the baseline. 


%Participants who started identifying plans in the baseline condition were found to identify programming plans the fastest. 

However, there was a wide variety in participants' ability to identify plans, which could be impacted by many factors, such as the individual instructor's content knowledge, the particular domain they are working in, the condition they are in, the condition they started with, and how much time they spent doing plan identification so far.

To understand how other experimental factors affect the time instructors take to identify plans, we built a linear mixed-effects model with the time spent per plan as the outcome variable. Fixed effects were the experimental condition (baseline or PLAID), session order (started with baseline or PLAID), how far into the task the instructor is (number of plans they have identified before this plan), and their domain (Pandas, Django, or PyTorch). The participants were modeled as random effects to control for differences in their expertise and other individual values. Even with a small sample size of 12 instructors and 104 identified plans, we observed a marginally significant coefficient for the experimental condition ($b = -57.2 (sec), t=-1.77, p = .079$) when controlling for these other factors, indicating that instructors were faster in identifying plans using PLAID by almost one minute per plan compared to the baseline. We also observed a statistically significant difference between the specific plans within a task and the time taken to design each plan ($b=129.2 (sec), t=12.85, p < .001$), indicating that instructors spent more time designing each plan for the later plans they suggested, potentially due to starting with easier concepts and moving to more complex ones.

% Although the difference observed was not statistically significant, it could be attributed to the small study sample.

% \subsubsection{Did the system make the process easier?}
% \subsubsection{NASA-TLX measures}

% \begin{figure*}[h]
%     \centering
%     \Description{Bar plot showing the distribution of responses for each item on the PSSUQ survey. For most items, The majority of items are rated above then the median option (Neither Agree or Disagree).}
%     \includegraphics[width=\textwidth]{img/usability-textsize-fixed.png}
%     \caption{Self-reported reflections of participants on the usability of PLAID using the PSSUQ survey. The graph encapsulates their responses for each question on a 7-point Likert Scale.}
%     \label{fig:usability}
% \end{figure*}

\subsubsection{PLAID decreases cognitive demands and overall task load during plan identification.}

% Participants reported a lower overall workload across all measures on NASA-TLX in the PLAID condition (\cref{fig:cognitive-load}). 
We find that the average task load for instructors\footnote{Computed across E2 to E12. E1 was excluded due to procedural error.} was significantly lower with PLAID (\cref{fig:cognitive-load}), indicated by Wilcoxon signed-rank test ~\cite{wilcoxon1992individual} (PLAID: $M = 2.83$ , $SD = 1.40$, Baseline: $M = 3.94$, $SD = 1.57$, $p = .04$). In addition, differences in two sub-measures were statistically significant: mental demand (PLAID: $M = 3.09$, $SD = 1.04$, Baseline: $M = 5.18$, $SD = 1.32$, $p = .008$), and physical demand (PLAID: $M = 1.63$, $SD = 0.92$, Baseline: $M = 2.54$, $SD = 1.75$, $p = .047$).

% \subsubsection{PSSUQ usability survey}
\subsubsection{PLAID provides instructors with a satisfactory experience.}
Participants responded positively to the PLAID user experience as indicated by responses to the PSSUQ survey\footnote{Option 1 indicated strong agreement and Option 7 indicated strong disagreement.} items (see Figure~\ref{fig:three-horizontal}). The responses aggregated into an overall mean of $M = 2.73$ ($SD= 1.49$), $M = 2.42$ for System Usefulness (SYSUSE, $SD= 1.46$), $M = 2.99$ for Information Quality (INFOQUAL, $SD = 1.57$), and $M = 2.81$ for Interface Quality (INTERQUAL, $SD = 1.44$).
%containing 16 items on a 7-point Likert scale .
% feels redundant to say it explicitly here because we report descriptions of each category in the results
% Responses to PSSUQ items are used to compute scores in three subscales, measuring system usability, information quality, and interface quality, in addition to the overall score.


% \subsubsection{How the system made the process easier?}
% \subsubsection{Participant interactions}

% Each participant's system interaction trace is visualized in \cref{fig:trace-diagram}. % We see that provided reference materials and the plan structure were highly utilized...

\begin{figure*}[t!]
    \centering
    \Description{A visualization of participant actions in the system. Each participant is represented as a horizontal sequence of square markers, and the color of the marker corresponds to one of these actions: Create Empty Plan, Create Candidate Plan, Delete Plan, Edit Plan, Annotate Changeable Area, Browse Reference Material, Browse External Material. We can see most instructors use browse reference materials, edit plans, and annotate changeable areas, but there are many individual differences between participants.}
    \includegraphics[width=\textwidth]{img/trace-diagram-vert.pdf}
    \caption{Trace diagram depicting participant interactions with PLAID. Each participant is shown as a horizontal line consisting of a series of actions.}
    \label{fig:trace-diagram}
\end{figure*}

\subsubsection{PLAID scaffolds instructors at multiple stages of the plan design process.}

Using our think-aloud data, post-task interviews, and trace diagrams (see \cref{fig:trace-diagram}), we noted instructors using PLAID to effectively and efficiently identify plans. 
% a variety of authentic instructor behaviors that were supported by PLAID. %In this section, we report the findings from our qualitative analysis and visualize the key interactions supported by PLAID using logged data in .

\textbf{PLAID accelerates plan identification for instructors by providing easily navigable reference material.}
Almost all participants appreciated the example programs included as part of PLAID.
While only some participants utilized the automated suggestions based on clusters of similar code snippets (orange in \cref{fig:trace-diagram}), all participants except E12 primarily interacted with the given reference content by browsing the example programs 
and reading their short descriptions 
%as part of the use cases 
% interaction with the reference material was browsing the example programs and use cases 
(pink in \cref{fig:trace-diagram}). 
 % utilized the program view, and most participants used it repeatedly. 
Participants indicated that developing initial ideas for designing plans was the most challenging stage of plan identification. E10 said it is easier to ``\textit{derive from an existing codebase...because the sample code is the key part}'', clarifying that they believed they were more efficient when using PLAID. 
E6 appreciated the inclusion of ``\textit{readily available code snippets}'' and E11 valued the ``\textit{condensed view}'', expressing that it felt like ``\textit{going through an email inbox}''. 
% Participants not only found the provided use cases advantageous, but also satisfying to explore.
% We also observed that the program view was not just useful, but also satisfying to explore.
Participants found it captivating to browse the list of use cases and search for key concepts. 
% mining fundamental patterns to create plans.
% Participants liked going through the list and finding important patterns to create plans from, even after they were done with the task. 
After they completed their timed task, E2 added ``\textit{I could keep going...I almost just want to read the list at this point.}'' 

While participants were allowed access to alternative reference content, they indicated that PLAID's technique of presenting examples was more suited to their needs. For example, multiple interviewees used ChatGPT to design plans in the baseline condition; however, they still found the process tedious. E9 communicated that the output was verbose and that it was ``\textit{quite an effort to ask even ChatGPT [for ideas]}''. E6 reported that ChatGPT split code into snippets at a different granularity than they would prefer. E5 prompted ChatGPT for ``\textit{things students struggle with when using Pandas}'' but did not find the output appropriate for beginners. ``\textit{I don't even know if I fully understand [this concept]}'', said E5.
While the queries that instructors used to prompt ChatGPT were not so different from the prompts employed as part of PLAID's pipeline,
PLAID prioritizes goal-focused information. More precisely, PLAID shows a brief natural language description of the relevant use case or candidate plan before displaying any code. Without this guidance in ChatGPT, reviewing output might be overwhelming.
% and code snippets are always shown with associated LLM-generated goals.
% the goal-oriented structure of the programs presented in PLAID improved instructors' perceptions about the generated content.
Arguably, a chat interface as the only mode of interaction is challenging, as important information is very spread out and interleaved with verbose explanatory text; participants like E6 spent a long time combining code to abstract high-level ideas from multiple responses given by ChatGPT to design plans.
% deriving high-level ideas from multiple answers generated by ChatGPT.
% <The prompts that instructors used to query ChatGPT in the baseline condition were actually rather similar to the early stages of our content generation pipeline. However, our additional processing of the candidate content seemed to yield more benefits for instructors. ChatGPT's default includes comments on nearly every line -- this doesn't helpful for encouraging abstraction....  >


Participants noted weaknesses in other external reference resources as well. 
% For example, web searches were utilized when both working in PLAID and the baseline condition. 
% However, it was not ideal: 
E10 found it tedious and challenging to compare inconsistent examples from multiple webpages and to find differences between these variable implementations. E8 stated, ``\textit{I know the material for this on the Internet isn't especially good}'' before they transitioned on to reference the code that they authored in the past. Even with their own code, we observed that participants needed to substantially modify their programs to meet the needs of their students. 
For example, E8 copied a snippet from code they wrote for a project and edited it, saying that ``\textit{This isn't necessarily optimal, but it's simple. That should be good for teaching material}''. E12 explained that they added structures they would otherwise not use in a complete program to help students understand (\textit{I'll do it one time, but I won't do it repeatedly}'', said E12). In contrast to other external resources, instructors referenced the documentation, often for reviewing the syntax they wanted to use in their plans.

 

% Participants appreciated the changeable areas, and found it useful for learning
\textbf{PLAID helps instructors create learner-friendly material by providing a structured template.}
Participants valued having a structure for designing plans. E10 found that stating explicit goals was useful for students to ``\textit{get more motivated that [they] know the purpose of learning}'' about the code. 
According to most instructors, the most advantageous part of the plan template were the changeable areas. Instructors perceived these annotations as a strategy of providing support to students. For example, E9 stated: \begin{quote}
``If I'm creating exercises in it, I'm specifying [to students] very clearly that `This is the overall intuition of the coding flow, and these are the areas that you can play with.' It kind of helps me direct the attention of the student towards the exact problem that we should be thinking about.''
\end{quote}
Most instructors used the annotation tool in PLAID to mark changeable areas in their solutions (yellow in \cref{fig:trace-diagram}). We also observed that participants who started with PLAID looked for a similar annotation mechanism in the baseline condition, pointing that there is no ``\textit{intuitive}'' (E9) way to achieve it.  

\textbf{PLAID supports the diverse iterative workflows of instructors.} We observed that instructors preferred to build high-level narratives with their plans, such as designing multiple plans that are all part of a complete program. PLAID's canvas, which shows all the in-progress plans at once, supports this behavior. E10 explained how this view was more helpful compared to the baseline: \begin{quote}
    ``The visual aspect of it [viewing boxes with only plan names], as opposed to seeing the whole thing [written-out plans in the baseline document], made it more modular, I like that abstraction. So I could focus on higher level takeaway of what I want the class to be about, instead of getting fixated of details of each [program].''
\end{quote}
Some participants imitated this process in the baseline by creating a list of initial ideas and then elaborating on each idea with other details. However, we observed that PLAID encouraged participants to keep refining and iterating at various granularities. For example, E8 designed one plan, started exploring the reference material for another, then found a concept that fit the previous plan better, and quickly went back to the previous plan to modify it as well. Similarly, E4 copied a code snippet from the reference material and made some changes, including adding a name and a goal. Then, instead of going back to the reference material or creating an empty plan, they copied the same plan and created another variation on it with small modifications.

\textbf{PLAID offers promise in introducing plan-based pedagogy to application-specific courses.}
Even though instructors did not have prior exposure to plan-focused instruction, instructors had overwhelmingly positive responses when asked about incorporating plan-based pedagogy in their instruction and using PLAID for designing plans. E9 described plan-based instruction as a ``\textit{step-by-step walkthrough of fundamental concepts}''. E11 indicated that learning about programming plans would help students retain common and important tasks that ``\textit{you can never remember the code for}''. Without any prompting, E2 and E3 even requested access to PLAID to design their upcoming courses.
%without any cues.
However, a few participants expressed concerns about using plan-based pedagogies for instruction. For E8, plans were useful for teaching ``\textit{many small individual things}'', but they were uncertain about their usefulness when it came to combining these smaller tasks into larger projects. E1 and E5 found programming plans valuable for conceptual understanding but were hesitant to design their existing course around these structures from scratch. E12 stated that plans could be useful for some learners, but also explained that they would prefer to include executable, full programs in lecture instead.


% For in the course, in a practice oriented course, doing a bunch of compound tasks that consist of.

% 00:47:10.000 --> 00:47:14.000
% Many small individual things that

% 00:47:14.000 --> 00:47:17.000
% You learn in the 1st month or so.

% 00:47:17.000 --> 00:47:18.000
% But

% 00:47:18.000 --> 00:47:26.000
% In increasingly challenging ways. So there are limitations to that structure. I think.

% E8 e


% Most participants were on board with the idea, but few opposed

% Breaking down into parts is hard, suggestions were useful

% Overall, participants found it satisfying.


% \input{final-evaluation}

% \section{Discussion}

In this section, we address RQ2: \textit{How can LLMs (e.g. ChatGPT) support the identification of domain-specific plans?} 
%With the results of our mixed methods study, we are able to identify the strengths of supporting the plan identification process with LLMs while noting its shortcomings.

% Component generation
We found that our ChatGPT pipeline can reliably generate common domain-specific code. 
%It performs well on code generation tasks, i.e. generating the solution component of a plan, as evident from the 
Our quantitative and qualitative evaluation showed that generated plan solutions were almost entirely syntactically valid (see Section~\ref{sec:quant_accuracy} and Section~\ref{sec:qual_components}). 
In addition, the generated code is representative of actual practice (Section~\ref{sec:quant_commonality}), as shown by similarity to a reference set of plans validated by experts, and comparable similarity to Github files from the same domain (Section~\ref{sec:commonality}). Moreover, accounting from the similar number of distinct methods covered in the two datasets, we infer that ChatGPT can generate plans that capture a variety of use cases in the domain (Section~\ref{sec:quant_usability}). Furthermore, the complexity of the generated code appears to be similar to those generated by instructors with domain expertise, indicating that the generated examples can be appropriate for novices (Section~\ref{sec:quant_learners_appropriateness}). 
Overall, using LLMs for early phases in plan identification by generating common examples and recognizing candidates is a promising avenue.

However, our approach is not consistently able to describe the code appropriately for learners.
It especially falls short on code interpretation, namely, generating other components of a plan such as names and goals. A number of its generated responses were either overly general or overstating what the code achieved (see Section~\ref{sec:qual_components}). This might reflect the existing challenges for LLMs on in-context learning tasks, observed by prior work~\cite{llms_hard_incontext_learning}.
In addition, ChatGPT sometimes generated technical jargon in the names and goals (see~\ref{sec:qual_characteristics}), which makes those plan components unsatisfactory for novice learners. However, despite these pitfalls, the generated plan components were somewhat accurate, implying that they may be appropriate starting points for instructors to refine.

%These findings suggest the role that LLMs can best play in domain-specific plan identification: they can generate candidate plans, however, they lack the capability to create plans that fully meet instructors' desired characteristics. 

We present suggestive evidence that using LLMs to generate candidate plans as a part of a plan generation pipeline could reduce the tedium in the identification process by eliminating the need for instructors to view programs or perform a literature review (Section~\ref{sec:viewing-programs} and Section~\ref{sec:lit-review}) prior to creating plans. Instead, LLMs could provide instructors with candidate plans that they would modify for their learner audience to ensure that the explanatory components are accurate and reasonable. A promising direction for the design of an automated plan identification system is to foster collaboration between an LLM and instructors in order to scale domain-specific plan identification. 
%that allows the instructor to refine the generated candidate plans to meet their learner's goals. 

% ...support instructors through simulating a collaborative process, where the LLM presents initial candidates and the instructor can modify and refine the candidates for their learning goals

% components/ accuracy
% name and goal
    % it seems to perform well on code generation (both qual and quant) but not as well on interpretating what the code does (qual).
%     Evidence on generation: quant similarity to existing plans
%     Related evidence: changeable areas not accurate, goals needed work (from qual)
% solution
    % accurate
% changeable areas
    % accurate yet not quality worthy

    % appropriateness to learning goals - not a metric that can be translated for llms
    
% bigger picture
% can generate quality candidate plans but not the final plans
% needs instructor feedback

% Instructors are still needed in the loop to ensure that explanatory components are correct, and to . A reasonable number were partially correct. 


% Which part of the plan identification process can be automated with LLMs?

% LLMS have the potential to replace the example search and literature review part. + supporting the instructor in domain expertise (by having more variety)

%This section can answer RQ2: How can tools for explaining and generating code
%(e.g. ChatGPT) support the identification of domain-specific plans?




%\subsection{Limitations}

\section{Discussion}
% Our findings in Section~\ref{sec:} illustrate that PLAID can support ...

%% Potential discussion nugget:
% LLMs have frequently been used to generate new student-facing instructional materials. However, LLMs are much less commonly used to support instructors. 
% LLMs have been applied to some instructor-facing tasks, such as labeling multiple choice questions with their key concepts. 
%Arguably, instructors are particularly well-positioned to make use of LLM-generated content for their needs, as they have sufficient content knowledge to not be sidetracked by hallucinactions, and appropraite pedagogical knowledge to recognize when LLM-generated content is too technical or verbose.


% 
Computing education research is a leading subject area for applications of large language models (LLMs) in education, as LLMs capable of generating code at scale were made publicly available much earlier than general-purpose models like ChatGPT. 
% There have been many student-facing tools that utilize LLM-generated content, but few instructor-facing systems.
There have been many studies that designed student-facing tools around this technology~\cite{ferdowsiValidatingAIGeneratedCode2024,jinTeachAIHow2024,kazemitabaarCodeAidEvaluatingClassroom2024,logachevaEvaluatingContextuallyPersonalized2024,yangDebuggingAITutor2024,yanIvieLightweightAnchored2024}. 
However, a main limitation explored in these works is the untrustworthy nature of LLMs, which could generate hallucinated, incorrect responses or content that is not appropriate for learners.
Arguably, instructors are particularly well-positioned to incorporate LLMs in educational workflows, as they have sufficient content knowledge to detect hallucinations and appropriate pedagogical knowledge to recognize when LLM-generated content is too technical or verbose.
However, tools that use LLMs for supporting instructors have been less common, even though there have been some successful examples (e.g.,~\cite{choiVIVIDHumanAICollaborative2024a}).

% While some approaches seem to eliminate the role of the instructor, we find that LLMs can be well-suited to *support* instructors, resulting in a better outcome when LLMs and instructors interact.
% Thus, HCI researchers working on computing education have taken on the task of developing tools through a critical lens, utilizing this emerging technology without following a reductionist approach toward the valuable pedagogical expertise of instructors.

Our evaluation of PLAID confirms that LLMs can be used in the design of tools that support instructors, specifically by automating the tedious parts of instructors' workflows without undermining opportunities for them to apply their domain-specific and pedagogical expertise. 
% content knowledge or pedagogical knowledge.  
One ubiquitous observation from our user study that validates this argument is the 
% that supports this claim is the 
highly positive response to the LLM-generated reference material. 
All instructors valued having access to diverse and authentic examples because this accelerated an initial content collection process that would otherwise have been performed manually. 
% preparation work they would have to do otherwise.
By delegating this process to a large language model, instructors are able to focus on refining and iterating on the process of designing plans. 
% Even though the generated content wasn't appropraite for use with students, instructors could easily modify that ``first draft'' to meet their needs

Notably, instructors who attempted to use ChatGPT as an external resource in the baseline condition did not benefit as much as instructors using PLAID did, even though ChatGPT was used to generate the reference content for PLAID. %report as high a level of satisfaction as when they used PLAID.
Moreover, they found the interactions with ChatGPT to be tedious and the outputs to be verbose, even though their prompts were not so different from our queries to generate content passed into PLAID.
This highlights the importance of presenting LLM-generated content with appropriate interactions that reflect existing instructor practices. 
Thus, our findings suggest that human-in-the-loop approaches that equip instructors with preliminary content and facilitate refinement of that content are promising for the design of educational technology. 
% <The prompts that instructors used to query ChatGPT in the baseline condition were actually rather similar to the early stages of our content generation pipeline. However, our additional processing of the candidate content seemed to yield more benefits for instructors. ChatGPT's default includes comments on nearly every line -- this doesn't helpful for encouraging abstraction....  >

PLAID presents encouraging results for utilizing LLM-powered tools to promote best practices and theory-informed approaches for education. Most instructors with no experience in plan-focused pedagogies were interested in using programming plans for instruction after a relatively short exposure to the concept. By streamlining the opaque and tedious process of designing a programming plan, PLAID successfully sparked interest among these instructors for adopting plan-focused pedagogies. While PLAID presently supports four application-focused domains (Pandas, Pytorch, Django, and BeautifulSoup), our versatile pipeline and design of the interface are able to support instructors in identifying plans in any domain of interest. In a sense, `robots are here'~\cite{pratherRobotsAreHere2023} for the boring and repetitive work of gathering content and organizing it into broad categories for the first draft. This delegation of work empowers instructors to focus on building overarching narratives and refining content for learners,
% The incorporation of LLMs allowed instructors to focus on how designing plans can help students learn, 
instead of going through a repetitive and unclear process of searching for programs that capture common patterns. Utilizing LLMs to automate repetitive information-gathering tasks, allowing instructors to use their expertise on problems with higher impact, could be an important goal for designers working on similar tools.
% <In addition to scaffolding how instructors create educational content, examples were provided. Even though they were bad examples, they still set expectations.>
% <Robots are doing the boring work of gathering relevant content, organizing it into broad categories, and suggesting a first draft, leaving instructors more free to focus on refining the content for learners according to their expertise. >


%%%% IMPORTANT: PLAID can be applied to any domain, prompted to create generate examples in new domains
While most instructors saw value in programming plans, we noted that there were some logistic concerns about adopting plan-based pedagogies. Instructors who have been teaching a 

well-structured course for a long time expressed reluctance to go through the effort of a major redesign. The most positive responses came from graduate teaching assistants or instructors in the process of designing a course from scratch, who had the chance to incorporate programming plans into their courses in the first place. 
Researchers working on educational technologies for instructors should consider this hesitancy to adopt new approaches, and identify additional design considerations relevant to such instructors.







% purpose oriented tasks, goal-oriented language
% system can support broadening of plan based pedgagogy
% however newer instructors are more open to that adoption due to logistics
% \subsection{Does PLAID Work?}

% \subsection{Does PLAID make Plan Identification Easier?}

% \subsection{How does PLAID facilitate Efficient Plan Identification?}

%% How do instructors interact with LLMs?
%% Does it help with abstraction through purpose-first language / goal-oriented language. 
%% How/where to use these plans in instructionn



% future work
% how can instructors take what they create to make instructional content
% time
% participant

\section{Limitations and Future Work}

While our study establishes foundational steps for plan identification, a few limitations exist. Most importantly, we evaluated our system in sessions limited to 15 minutes. These sessions may not have comprehensively captured all the ways in which instructors would interact with the interface because instructor behavior could change as they become more familiar with the system. Indeed, some participants expressed that they could not explore the features enabled in the system due to time constraints and primarily used interactions that were explicitly demonstrated before the tasks. Capturing how instructors' perceptions and behaviors change over time as they interact with the system could be valuable.

In addition, our study design required the recruitment of a population with expertise in instruction and an application-focused domain. Consequently, the sample size of our participants was comparatively small ($N=12$). This sample size may not be enough to observe statistical significance in some of our tests, including the mixed effects model and the Wilcoxon signed rank tests. However, along with our qualitative observations, we find the results to still be informative and encouraging for future research. 

% We selected three diverse domains
% In particular, predict limitations for niche domains.
\edit{The quality of generated content is an important consideration for any system that utilizes large language models. An inherent limitation of LLMs in educational technology is the potential generation of inconsistent, inaccurate, or incorrect content that might be harmful to learners.  PLAID mitigates these problems with its instructor-in-the-loop design where LLM-generated content is reviewed and refined by instructors before being presented to students. By presenting many code examples and alternative suggestions for all components of a plan, PLAID encourages instructors to carefully compare content. Our user study also showed this behavior as instructors often edited parts of plans they created from examples. Moreover, our initial exploration in Section \ref{sec:llm-pipeline} showed the generated content to be syntactically accurate in most cases, suggesting that the generated content is meaningful and useful for instructors to review. The usefulness of PLAID may still be impacted by the quality of its LLM-generated content: poorer quality of the initial content may lead to lower efficiency in instructors' ability to generate final content. Future work could explore other automatic approaches to identify inaccurate content before it is presented to instructors to reduce the effort they would need to put in to refine the content to a greater extent.}

\edit{Future work should explore the generalizability of PLAID to new programming domains. In our study, we evaluated PLAID in three distinct programming domains demonstrating its usefulness beyond introductory programming. Even in complex domains like Django and PyTorch, we did not observe any statistically significant differences between instructors' performance while identifying plans. However, newer programming languages or domains may be underrepresented in the LLM training data, affecting the quality of the generated code. Examining how instructors use PLAID in these niche domains can provide important insights into the utility of PLAID and its design considerations for larger computing education research. Moreover, designing programming plans in complex domains like app development might benefit from more interactions, such as support to organize code in multiple files.}

% <Our system supports instructors in considering key concepts, but it does not create content that is immediately ready for the classroom. ...Design opportunities for student-facing tools>
An open question for future work is how to design novel systems that present programming plans to students and automate additional aspects of plan-based pedagogies.
% student-facing systems for students using programming plans. 
Prior research on plans has shown that explicit plan-based pedagogical instruction may motivate students and support better knowledge acquisition~\cite{Cunningham_PurposeFirstProgramming_CHI-2021}. 
%In our studies, we followed this convention and prompted instructors to identify plans that could be used for teaching common concepts in lectures.
While the plans instructors generated in PLAID may be appropriate for use in lectures, there are many more opportunities to use these identified plans to support instruction. Our instructors proposed many interesting applications, including \textit{plans as cheatsheets}, where each plan explains a common task that students usually struggle with; \textit{plans as question generators}, where a plan with changeable areas is treated as a multiple-choice question; and \textit{plans as example generators}, creating on-demand worked examples~\cite{Atkinson_WorkedExamples_2000}.
% including slides and workbooks, similar to having worked examples that illustrate concepts and best practices. 
%However, 
% Our instructors had versatile ideas for tools that could be facilitated by having a set of programming plans.
% maybe connected to case-based learning stuff?
% For example, some instructors defined 
% MCQ generation?
% Other instructors found our system useful for utilizing 
% if our system supports abstraction, why not have students benefit from it as well? 
% Many instructors proposed having a ``student view'' in the system to allow students to mimic this plan identification process to use \textit{plans as reflection questions}.
% These applications were not explored in prior research on plan-based pedagogies, suggesting that empowering instructors in identifying domain-specific programming plans can also produce new ways of interacting with programming plans for all educators. 

% Designing and evaluating programming plans in diverse contexts is important for supporting plan-focused pedagogies in more domains, and presents novel design opportunities for supporting students. % These novel approaches for presenting programming plans to students can also address some concerns instructors raised for adopting programming plans, such as a reluctance to re-design

% Future work can explore enabling student-facing views for instructors to complete the pedagogical design process.

% Future work can further develop interfaces that support instructors in creating assessments and in-class examples using plans.

% Future work can then examine the effectiveness of instructional content created using plans in a classroom setting.

\section{Conclusion}
\edit{In this paper, we present PLAID, a tool that assists instructors in generating programming plans in application-focused programming domains, a crucial step towards the use of plan-based pedagogies.
Such pedagogies have shown promising outcomes for introductory programming learners, but have not been applied to application-focused programming domains such as data analysis or machine learning.
Through formative interviews (N=10 educators), we identified how creating programming plans that capture high-level patterns in these domains can be challenging and tedious, even with access to AI-generated content. 
Through design workshops (N=7 educators), we derived design goals detailing how AI-generated content should be presented to instructors to make this plan creation process more efficient.
Our findings from a mixed methods within-subjects user study (N=12 educators) show that during the plan identification process, instructors experience lower cognitive demands and overall task load with PLAID, find it more satisfying to use, and prefer it over traditional approaches.
Our work not only addresses the challenges in generating programming plans for application-focused programming domains but also contributes design considerations to guide the development of ``human-in-the-loop'' AI tools.
We find that instructors can leverage LLMs to effectively author instructional material when the LLM-generated content is presented in a format that reflects best practices and reduces distractions. }
% Future work should explore other interactions for presenting LLM-generated content 
We believe PLAID can be easily extended to support plan identification in many application-focused programming domains, potentially encouraging the adoption of plan-based pedagogies at scale. Moreover, our design goals and insights from instructor-LLM interactions can inform the design of tools that support instructors in creating content in their domain of expertise. 

% By providing access to relevant code examples generated by large language models compiled in one interface with interactions designed to ensure greater efficiency,  refining the content.

% Plan-based pedagogies have shown promising outcomes in enhancing knowledge acquisition for novices in introductory programming. 
% Introducing plan-focused pedagogies in application-focused domains, like data analysis or machine learning, can help propagate these positive outcomes to a broader audience. 
% However, our formative study revealed that the current state of the art in programming plan identification, a prerequisite for plan-based pedagogies, is a lengthy manual procedure. % good!
% Following a design workshop with instructors, we propose design goals to motivate the design of interfaces that help instructors interact with LLM-generated content effectively and efficiently.
% We developed PLAID, an interface that enables educators to access a corpus of relevant reference material that they can navigate to generate abstract, high-level programming plans.
% Due to its LLM pipeline, PLAID can be easily extended to support plan identification in many application-focused programming domains, potentially encouraging adoption of plan-based pedagogies at scale.
% By enabling effective interactions between LLMs and instructors, PLAID combines LLMs' content generation capabilities and instructors' pedagogical expertise to create instructional content that aligns with best practices. 
% \edit{We designed PLAID to improve instructor experience while creating programming plans to encourage more instructors to adopt plan-based pedagogies. However, designing programming plans is an intermediate step in creating instructional content that uses plan-based pedagogy and future work can explore techniques that effectively incorporate plans in instruction to meet the learning objectives of students.} 
% We believe our design goals and insights from instructor-LLM interactions can inform the design of tools that support instructors in creating content in their domain of expertise.





% It is motivating to explore if they can be introduced in the design of curricula in application-focused domains like data analysis or machine learning. 
% PLAID addresses the challenges for this broader adoption by generating candidate plans and programs, enabling efficient exploration of this content, and providing tools for effective refinement of plans.
% Importantly, as PLAID generates content using an automated pipeline, it could be easily applied to any domain.
% Our studies with instructors show that LLM-generated content can greatly improve instructor processes by scaffolding information-gathering tasks, and instructors become interested in applying plan-focused pedagogies in their domains.
% We believe our design goals and insights from instructor-LLM interactions can inform tools for supporting instructors to enhance their ability to create quality content for learners.






 %enhance their ability to create quality content for learners.

% create learner-centric content.

% Thus, PLAID can assist instructors from any domain to incorporate plan-based pedagogy into their curricula. 

% PLAID builds on the relative strengths of both LLMs and instructors, using an LLM pipeline based on instructors' desired characteristics to generate ``first draft'' content, and supporting instructors' ability to abstract general concepts from the LLM-generated examples.

% Due to its LLM pipeline, PLAID can be easily extended to support plan identification in a wide number of application-focused programming domains, potentially encouraging adoption of plan-based pedagogies at scale.


% Our evaluation study highlights that participants experience a lower overall task load and found PLAID beneficial for generating instructional material.
% Thus, PLAID may help instructors design plans for in-class modules, questions on assessments, or cheatsheets.

% Instructors can utilize LLM-generated content in instructional design, as long as certain design goals are met...
% Systems can support instructors to more quickly generate key abstractions that can support instructional design. By enabling instructors to efficiently view and manipulate a diverse set of use cases, their ability to identify abstract concepts was improved, and their experience was more positive. 
% Instructors can benefit from systems that support curriculum design by providing key abstractions %that support instructional design. 
% Having access to a diverse set of use cases that can be efficiently viewed and manipulated improved their experience in identifying abstract concepts. 
% We find that LLM-generated content can be beneficial to instructors by providing a diverse set of examples of programming use cases, but this benefit might only be present if the content is organized correctly. 






% llms can support instructors but not out of the box. instructor needs to make use of content but we cannot replace instructors



% The current state of the art in programming plan identification is a lengthy manual procedure. ...



% With a more efficient, technologically-supported plan identification process, educators may be able to use programming plans in their instruction for a wide range of subjects beyond introductory programming. In this study, we gathered concrete details about the current state of the art in plan identification processes to understand what plan components instructors seek, what metrics they use to judge success, and where they face challenges. 
%Given the rise of large language models and generative AI technologies, 
% Then, we explored the use of ChatGPT as an aid in the plan identification process 
%and described its strengths and shortcomings. 
% and found that it could generate candidate programming plans with significant similarity to instructor-designed plans, however, many of the plans' explanatory components were not well attuned to the needs of beginning learners. These results suggest that the way forward in plan identification across domains should involve collaboration between LLMs and instructors.
% The current state of the art in programming plan identification is a lengthy manual procedure. With the aid of a more efficient, technologically-supported plan identification process, educators may be able to use programming plans in their instruction for a wide range of subjects beyond introductory programming.  In this study, we gathered concrete details about the current state of the art in plan identification processes to understand what is important to instructors \edit{and where they face challenges, in order to} inform the design of a system that automates some of the plan identification process. As we lay a groundwork for the design of systems for plan identification, we note that it is unlikely that humans will no longer be a part of choosing programming plans for educational purposes. 
% %Plan identifiers' instructional experience and programming experience both play a role in choosing and refining plans. While it may be possible to spread the work of one educator by distributing it to different individuals, the human element will not disappear.
% As computing educators are tasked with teaching an increasingly diverse set of learners with widely varied interests, we believe that plan identification systems can support them to develop instructional material that not only based on the psychology of programming, but also is more tailored to their students' interests. 
%\section{Domain-specific Plan Quality}
%We propose three metrics for the quality of a %domain-specific programming 
%plan.
%\begin{itemize}
%     \item \textit{Frequency in Professional Use}: The plan should be commonly used in well-written programs.
%     \item \textit{Comprehensibility to Novices}: While understanding likely requires domain-specific content knowledge, plan information should not be too complex. 
%     %Since these plans are being identified for novice instruction, the understandability of the piece of code is essential.
%     \item \textit{Alignment with Domain Context}: The plan should achieve a goal specific to its domain. %rather than something generic.
% \end{itemize}


% \section{Identification Approaches}
% \subsection{Code Corpus Analysis}

% Repositories like StackOverflow and GitHub hold billions of lines of code \cite{billions_cite}.
% %that undoubtedly contain domain-specific patterns. 
% By scraping, one can collect corpuses of programs in particular topics and search them for patterns, by hand (as in \cite{Cunningham_PurposeFirstProgramming_CHI-2021}) or possibly with code similarity algorithms \cite{Allamanis_MiningCodeIdioms_FSM-2014}.
 
% By analyzing many programs on a platform used by % industry
% professionals, these plans can claim \textit{frequency in professional use}. However, the goal of a plan is challenging to intuit from code alone, although comments or metadata may yield insight. Ensuring \textit{alignment with domain context} likely requires domain-specific expertise. Similarly, this approach offers no inherent advantage in \textit{comprehensibility}.% will require additional work.% as well.

% \subsection{Expert Instructor Opinion}

% % Domain-specific programming 
% Plans may also be found in code written by instructors of %advanced or
% specialized courses. These patterns can be collected from assignments
% %or exams designed for university students, 
% or by interviewing instructors. %directly.

% Instructors 
% %may or may not have connections to industry practice, but they 
% likely have %special 
% insight into designing plans \textit{comprehensible to novices}. 
% As instructors are knowledgeable in their topic area, the plans they create are probably \textit{aligned with domain context}.  
% However, this approach can feasibly include only few instructors, who may or may not be familiar with industry standards. These plans' \textit{frequency in professional use} is unclear.

% \subsection{Guided Expert Pattern Elicitation}

% %Experts in a domain certainly know relevant programming patterns \cite{Soloway1984EmpiricalSO}.
% %, although they may need support to articulate them. 
% Plans may be gathered from domain experts by interviewing them about their %own 
% programs, with questions designed to elicit common patterns and goals.

% With enough interviews, this approach could identify plans \textit{frequent in professional use}. Well-designed prompts may support \textit{alignment with domain context}. Expert blind spot \cite{Nathan_ExpertBlindSpot_2001} presents a challenge in achieving \textit{comprehensibility to novices}, although questions inspired by cognitive task analysis \cite{Catrambone_TAPS_2011} may help.

%%
%% The acknowledgments section is defined using the "acks" environment
%% (and NOT an unnumbered section). This ensures the proper
%% identification of the section in the article metadata, and the
%% consistent spelling of the heading.
\begin{acks}
We thank the members of the TRAILS Lab for their insightful feedback. We also express our gratitude to the reviewers for their thoughtful and valuable suggestions. We extend our appreciation to the CS STARS program, the Siebel School of Computing and Data Science, the Campus Honors Program, and the Office of Undergraduate Research at the University of Illinois Urbana-Champaign for funding that contributed to this work.
\end{acks}
\newline
%%
%% The next two lines define the bibliography style to be used, and
%% the bibliography file.
\bibliographystyle{ACM-Reference-Format}
\bibliography{bib/bib_new
, bib/background, bib/sample-base, bib/paperpile, bib/mark, bib/cps, bib/SBF-review}

\appendix
\subsection{Lloyd-Max Algorithm}
\label{subsec:Lloyd-Max}
For a given quantization bitwidth $B$ and an operand $\bm{X}$, the Lloyd-Max algorithm finds $2^B$ quantization levels $\{\hat{x}_i\}_{i=1}^{2^B}$ such that quantizing $\bm{X}$ by rounding each scalar in $\bm{X}$ to the nearest quantization level minimizes the quantization MSE. 

The algorithm starts with an initial guess of quantization levels and then iteratively computes quantization thresholds $\{\tau_i\}_{i=1}^{2^B-1}$ and updates quantization levels $\{\hat{x}_i\}_{i=1}^{2^B}$. Specifically, at iteration $n$, thresholds are set to the midpoints of the previous iteration's levels:
\begin{align*}
    \tau_i^{(n)}=\frac{\hat{x}_i^{(n-1)}+\hat{x}_{i+1}^{(n-1)}}2 \text{ for } i=1\ldots 2^B-1
\end{align*}
Subsequently, the quantization levels are re-computed as conditional means of the data regions defined by the new thresholds:
\begin{align*}
    \hat{x}_i^{(n)}=\mathbb{E}\left[ \bm{X} \big| \bm{X}\in [\tau_{i-1}^{(n)},\tau_i^{(n)}] \right] \text{ for } i=1\ldots 2^B
\end{align*}
where to satisfy boundary conditions we have $\tau_0=-\infty$ and $\tau_{2^B}=\infty$. The algorithm iterates the above steps until convergence.

Figure \ref{fig:lm_quant} compares the quantization levels of a $7$-bit floating point (E3M3) quantizer (left) to a $7$-bit Lloyd-Max quantizer (right) when quantizing a layer of weights from the GPT3-126M model at a per-tensor granularity. As shown, the Lloyd-Max quantizer achieves substantially lower quantization MSE. Further, Table \ref{tab:FP7_vs_LM7} shows the superior perplexity achieved by Lloyd-Max quantizers for bitwidths of $7$, $6$ and $5$. The difference between the quantizers is clear at 5 bits, where per-tensor FP quantization incurs a drastic and unacceptable increase in perplexity, while Lloyd-Max quantization incurs a much smaller increase. Nevertheless, we note that even the optimal Lloyd-Max quantizer incurs a notable ($\sim 1.5$) increase in perplexity due to the coarse granularity of quantization. 

\begin{figure}[h]
  \centering
  \includegraphics[width=0.7\linewidth]{sections/figures/LM7_FP7.pdf}
  \caption{\small Quantization levels and the corresponding quantization MSE of Floating Point (left) vs Lloyd-Max (right) Quantizers for a layer of weights in the GPT3-126M model.}
  \label{fig:lm_quant}
\end{figure}

\begin{table}[h]\scriptsize
\begin{center}
\caption{\label{tab:FP7_vs_LM7} \small Comparing perplexity (lower is better) achieved by floating point quantizers and Lloyd-Max quantizers on a GPT3-126M model for the Wikitext-103 dataset.}
\begin{tabular}{c|cc|c}
\hline
 \multirow{2}{*}{\textbf{Bitwidth}} & \multicolumn{2}{|c|}{\textbf{Floating-Point Quantizer}} & \textbf{Lloyd-Max Quantizer} \\
 & Best Format & Wikitext-103 Perplexity & Wikitext-103 Perplexity \\
\hline
7 & E3M3 & 18.32 & 18.27 \\
6 & E3M2 & 19.07 & 18.51 \\
5 & E4M0 & 43.89 & 19.71 \\
\hline
\end{tabular}
\end{center}
\end{table}

\subsection{Proof of Local Optimality of LO-BCQ}
\label{subsec:lobcq_opt_proof}
For a given block $\bm{b}_j$, the quantization MSE during LO-BCQ can be empirically evaluated as $\frac{1}{L_b}\lVert \bm{b}_j- \bm{\hat{b}}_j\rVert^2_2$ where $\bm{\hat{b}}_j$ is computed from equation (\ref{eq:clustered_quantization_definition}) as $C_{f(\bm{b}_j)}(\bm{b}_j)$. Further, for a given block cluster $\mathcal{B}_i$, we compute the quantization MSE as $\frac{1}{|\mathcal{B}_{i}|}\sum_{\bm{b} \in \mathcal{B}_{i}} \frac{1}{L_b}\lVert \bm{b}- C_i^{(n)}(\bm{b})\rVert^2_2$. Therefore, at the end of iteration $n$, we evaluate the overall quantization MSE $J^{(n)}$ for a given operand $\bm{X}$ composed of $N_c$ block clusters as:
\begin{align*}
    \label{eq:mse_iter_n}
    J^{(n)} = \frac{1}{N_c} \sum_{i=1}^{N_c} \frac{1}{|\mathcal{B}_{i}^{(n)}|}\sum_{\bm{v} \in \mathcal{B}_{i}^{(n)}} \frac{1}{L_b}\lVert \bm{b}- B_i^{(n)}(\bm{b})\rVert^2_2
\end{align*}

At the end of iteration $n$, the codebooks are updated from $\mathcal{C}^{(n-1)}$ to $\mathcal{C}^{(n)}$. However, the mapping of a given vector $\bm{b}_j$ to quantizers $\mathcal{C}^{(n)}$ remains as  $f^{(n)}(\bm{b}_j)$. At the next iteration, during the vector clustering step, $f^{(n+1)}(\bm{b}_j)$ finds new mapping of $\bm{b}_j$ to updated codebooks $\mathcal{C}^{(n)}$ such that the quantization MSE over the candidate codebooks is minimized. Therefore, we obtain the following result for $\bm{b}_j$:
\begin{align*}
\frac{1}{L_b}\lVert \bm{b}_j - C_{f^{(n+1)}(\bm{b}_j)}^{(n)}(\bm{b}_j)\rVert^2_2 \le \frac{1}{L_b}\lVert \bm{b}_j - C_{f^{(n)}(\bm{b}_j)}^{(n)}(\bm{b}_j)\rVert^2_2
\end{align*}

That is, quantizing $\bm{b}_j$ at the end of the block clustering step of iteration $n+1$ results in lower quantization MSE compared to quantizing at the end of iteration $n$. Since this is true for all $\bm{b} \in \bm{X}$, we assert the following:
\begin{equation}
\begin{split}
\label{eq:mse_ineq_1}
    \tilde{J}^{(n+1)} &= \frac{1}{N_c} \sum_{i=1}^{N_c} \frac{1}{|\mathcal{B}_{i}^{(n+1)}|}\sum_{\bm{b} \in \mathcal{B}_{i}^{(n+1)}} \frac{1}{L_b}\lVert \bm{b} - C_i^{(n)}(b)\rVert^2_2 \le J^{(n)}
\end{split}
\end{equation}
where $\tilde{J}^{(n+1)}$ is the the quantization MSE after the vector clustering step at iteration $n+1$.

Next, during the codebook update step (\ref{eq:quantizers_update}) at iteration $n+1$, the per-cluster codebooks $\mathcal{C}^{(n)}$ are updated to $\mathcal{C}^{(n+1)}$ by invoking the Lloyd-Max algorithm \citep{Lloyd}. We know that for any given value distribution, the Lloyd-Max algorithm minimizes the quantization MSE. Therefore, for a given vector cluster $\mathcal{B}_i$ we obtain the following result:

\begin{equation}
    \frac{1}{|\mathcal{B}_{i}^{(n+1)}|}\sum_{\bm{b} \in \mathcal{B}_{i}^{(n+1)}} \frac{1}{L_b}\lVert \bm{b}- C_i^{(n+1)}(\bm{b})\rVert^2_2 \le \frac{1}{|\mathcal{B}_{i}^{(n+1)}|}\sum_{\bm{b} \in \mathcal{B}_{i}^{(n+1)}} \frac{1}{L_b}\lVert \bm{b}- C_i^{(n)}(\bm{b})\rVert^2_2
\end{equation}

The above equation states that quantizing the given block cluster $\mathcal{B}_i$ after updating the associated codebook from $C_i^{(n)}$ to $C_i^{(n+1)}$ results in lower quantization MSE. Since this is true for all the block clusters, we derive the following result: 
\begin{equation}
\begin{split}
\label{eq:mse_ineq_2}
     J^{(n+1)} &= \frac{1}{N_c} \sum_{i=1}^{N_c} \frac{1}{|\mathcal{B}_{i}^{(n+1)}|}\sum_{\bm{b} \in \mathcal{B}_{i}^{(n+1)}} \frac{1}{L_b}\lVert \bm{b}- C_i^{(n+1)}(\bm{b})\rVert^2_2  \le \tilde{J}^{(n+1)}   
\end{split}
\end{equation}

Following (\ref{eq:mse_ineq_1}) and (\ref{eq:mse_ineq_2}), we find that the quantization MSE is non-increasing for each iteration, that is, $J^{(1)} \ge J^{(2)} \ge J^{(3)} \ge \ldots \ge J^{(M)}$ where $M$ is the maximum number of iterations. 
%Therefore, we can say that if the algorithm converges, then it must be that it has converged to a local minimum. 
\hfill $\blacksquare$


\begin{figure}
    \begin{center}
    \includegraphics[width=0.5\textwidth]{sections//figures/mse_vs_iter.pdf}
    \end{center}
    \caption{\small NMSE vs iterations during LO-BCQ compared to other block quantization proposals}
    \label{fig:nmse_vs_iter}
\end{figure}

Figure \ref{fig:nmse_vs_iter} shows the empirical convergence of LO-BCQ across several block lengths and number of codebooks. Also, the MSE achieved by LO-BCQ is compared to baselines such as MXFP and VSQ. As shown, LO-BCQ converges to a lower MSE than the baselines. Further, we achieve better convergence for larger number of codebooks ($N_c$) and for a smaller block length ($L_b$), both of which increase the bitwidth of BCQ (see Eq \ref{eq:bitwidth_bcq}).


\subsection{Additional Accuracy Results}
%Table \ref{tab:lobcq_config} lists the various LOBCQ configurations and their corresponding bitwidths.
\begin{table}
\setlength{\tabcolsep}{4.75pt}
\begin{center}
\caption{\label{tab:lobcq_config} Various LO-BCQ configurations and their bitwidths.}
\begin{tabular}{|c||c|c|c|c||c|c||c|} 
\hline
 & \multicolumn{4}{|c||}{$L_b=8$} & \multicolumn{2}{|c||}{$L_b=4$} & $L_b=2$ \\
 \hline
 \backslashbox{$L_A$\kern-1em}{\kern-1em$N_c$} & 2 & 4 & 8 & 16 & 2 & 4 & 2 \\
 \hline
 64 & 4.25 & 4.375 & 4.5 & 4.625 & 4.375 & 4.625 & 4.625\\
 \hline
 32 & 4.375 & 4.5 & 4.625& 4.75 & 4.5 & 4.75 & 4.75 \\
 \hline
 16 & 4.625 & 4.75& 4.875 & 5 & 4.75 & 5 & 5 \\
 \hline
\end{tabular}
\end{center}
\end{table}

%\subsection{Perplexity achieved by various LO-BCQ configurations on Wikitext-103 dataset}

\begin{table} \centering
\begin{tabular}{|c||c|c|c|c||c|c||c|} 
\hline
 $L_b \rightarrow$& \multicolumn{4}{c||}{8} & \multicolumn{2}{c||}{4} & 2\\
 \hline
 \backslashbox{$L_A$\kern-1em}{\kern-1em$N_c$} & 2 & 4 & 8 & 16 & 2 & 4 & 2  \\
 %$N_c \rightarrow$ & 2 & 4 & 8 & 16 & 2 & 4 & 2 \\
 \hline
 \hline
 \multicolumn{8}{c}{GPT3-1.3B (FP32 PPL = 9.98)} \\ 
 \hline
 \hline
 64 & 10.40 & 10.23 & 10.17 & 10.15 &  10.28 & 10.18 & 10.19 \\
 \hline
 32 & 10.25 & 10.20 & 10.15 & 10.12 &  10.23 & 10.17 & 10.17 \\
 \hline
 16 & 10.22 & 10.16 & 10.10 & 10.09 &  10.21 & 10.14 & 10.16 \\
 \hline
  \hline
 \multicolumn{8}{c}{GPT3-8B (FP32 PPL = 7.38)} \\ 
 \hline
 \hline
 64 & 7.61 & 7.52 & 7.48 &  7.47 &  7.55 &  7.49 & 7.50 \\
 \hline
 32 & 7.52 & 7.50 & 7.46 &  7.45 &  7.52 &  7.48 & 7.48  \\
 \hline
 16 & 7.51 & 7.48 & 7.44 &  7.44 &  7.51 &  7.49 & 7.47  \\
 \hline
\end{tabular}
\caption{\label{tab:ppl_gpt3_abalation} Wikitext-103 perplexity across GPT3-1.3B and 8B models.}
\end{table}

\begin{table} \centering
\begin{tabular}{|c||c|c|c|c||} 
\hline
 $L_b \rightarrow$& \multicolumn{4}{c||}{8}\\
 \hline
 \backslashbox{$L_A$\kern-1em}{\kern-1em$N_c$} & 2 & 4 & 8 & 16 \\
 %$N_c \rightarrow$ & 2 & 4 & 8 & 16 & 2 & 4 & 2 \\
 \hline
 \hline
 \multicolumn{5}{|c|}{Llama2-7B (FP32 PPL = 5.06)} \\ 
 \hline
 \hline
 64 & 5.31 & 5.26 & 5.19 & 5.18  \\
 \hline
 32 & 5.23 & 5.25 & 5.18 & 5.15  \\
 \hline
 16 & 5.23 & 5.19 & 5.16 & 5.14  \\
 \hline
 \multicolumn{5}{|c|}{Nemotron4-15B (FP32 PPL = 5.87)} \\ 
 \hline
 \hline
 64  & 6.3 & 6.20 & 6.13 & 6.08  \\
 \hline
 32  & 6.24 & 6.12 & 6.07 & 6.03  \\
 \hline
 16  & 6.12 & 6.14 & 6.04 & 6.02  \\
 \hline
 \multicolumn{5}{|c|}{Nemotron4-340B (FP32 PPL = 3.48)} \\ 
 \hline
 \hline
 64 & 3.67 & 3.62 & 3.60 & 3.59 \\
 \hline
 32 & 3.63 & 3.61 & 3.59 & 3.56 \\
 \hline
 16 & 3.61 & 3.58 & 3.57 & 3.55 \\
 \hline
\end{tabular}
\caption{\label{tab:ppl_llama7B_nemo15B} Wikitext-103 perplexity compared to FP32 baseline in Llama2-7B and Nemotron4-15B, 340B models}
\end{table}

%\subsection{Perplexity achieved by various LO-BCQ configurations on MMLU dataset}


\begin{table} \centering
\begin{tabular}{|c||c|c|c|c||c|c|c|c|} 
\hline
 $L_b \rightarrow$& \multicolumn{4}{c||}{8} & \multicolumn{4}{c||}{8}\\
 \hline
 \backslashbox{$L_A$\kern-1em}{\kern-1em$N_c$} & 2 & 4 & 8 & 16 & 2 & 4 & 8 & 16  \\
 %$N_c \rightarrow$ & 2 & 4 & 8 & 16 & 2 & 4 & 2 \\
 \hline
 \hline
 \multicolumn{5}{|c|}{Llama2-7B (FP32 Accuracy = 45.8\%)} & \multicolumn{4}{|c|}{Llama2-70B (FP32 Accuracy = 69.12\%)} \\ 
 \hline
 \hline
 64 & 43.9 & 43.4 & 43.9 & 44.9 & 68.07 & 68.27 & 68.17 & 68.75 \\
 \hline
 32 & 44.5 & 43.8 & 44.9 & 44.5 & 68.37 & 68.51 & 68.35 & 68.27  \\
 \hline
 16 & 43.9 & 42.7 & 44.9 & 45 & 68.12 & 68.77 & 68.31 & 68.59  \\
 \hline
 \hline
 \multicolumn{5}{|c|}{GPT3-22B (FP32 Accuracy = 38.75\%)} & \multicolumn{4}{|c|}{Nemotron4-15B (FP32 Accuracy = 64.3\%)} \\ 
 \hline
 \hline
 64 & 36.71 & 38.85 & 38.13 & 38.92 & 63.17 & 62.36 & 63.72 & 64.09 \\
 \hline
 32 & 37.95 & 38.69 & 39.45 & 38.34 & 64.05 & 62.30 & 63.8 & 64.33  \\
 \hline
 16 & 38.88 & 38.80 & 38.31 & 38.92 & 63.22 & 63.51 & 63.93 & 64.43  \\
 \hline
\end{tabular}
\caption{\label{tab:mmlu_abalation} Accuracy on MMLU dataset across GPT3-22B, Llama2-7B, 70B and Nemotron4-15B models.}
\end{table}


%\subsection{Perplexity achieved by various LO-BCQ configurations on LM evaluation harness}

\begin{table} \centering
\begin{tabular}{|c||c|c|c|c||c|c|c|c|} 
\hline
 $L_b \rightarrow$& \multicolumn{4}{c||}{8} & \multicolumn{4}{c||}{8}\\
 \hline
 \backslashbox{$L_A$\kern-1em}{\kern-1em$N_c$} & 2 & 4 & 8 & 16 & 2 & 4 & 8 & 16  \\
 %$N_c \rightarrow$ & 2 & 4 & 8 & 16 & 2 & 4 & 2 \\
 \hline
 \hline
 \multicolumn{5}{|c|}{Race (FP32 Accuracy = 37.51\%)} & \multicolumn{4}{|c|}{Boolq (FP32 Accuracy = 64.62\%)} \\ 
 \hline
 \hline
 64 & 36.94 & 37.13 & 36.27 & 37.13 & 63.73 & 62.26 & 63.49 & 63.36 \\
 \hline
 32 & 37.03 & 36.36 & 36.08 & 37.03 & 62.54 & 63.51 & 63.49 & 63.55  \\
 \hline
 16 & 37.03 & 37.03 & 36.46 & 37.03 & 61.1 & 63.79 & 63.58 & 63.33  \\
 \hline
 \hline
 \multicolumn{5}{|c|}{Winogrande (FP32 Accuracy = 58.01\%)} & \multicolumn{4}{|c|}{Piqa (FP32 Accuracy = 74.21\%)} \\ 
 \hline
 \hline
 64 & 58.17 & 57.22 & 57.85 & 58.33 & 73.01 & 73.07 & 73.07 & 72.80 \\
 \hline
 32 & 59.12 & 58.09 & 57.85 & 58.41 & 73.01 & 73.94 & 72.74 & 73.18  \\
 \hline
 16 & 57.93 & 58.88 & 57.93 & 58.56 & 73.94 & 72.80 & 73.01 & 73.94  \\
 \hline
\end{tabular}
\caption{\label{tab:mmlu_abalation} Accuracy on LM evaluation harness tasks on GPT3-1.3B model.}
\end{table}

\begin{table} \centering
\begin{tabular}{|c||c|c|c|c||c|c|c|c|} 
\hline
 $L_b \rightarrow$& \multicolumn{4}{c||}{8} & \multicolumn{4}{c||}{8}\\
 \hline
 \backslashbox{$L_A$\kern-1em}{\kern-1em$N_c$} & 2 & 4 & 8 & 16 & 2 & 4 & 8 & 16  \\
 %$N_c \rightarrow$ & 2 & 4 & 8 & 16 & 2 & 4 & 2 \\
 \hline
 \hline
 \multicolumn{5}{|c|}{Race (FP32 Accuracy = 41.34\%)} & \multicolumn{4}{|c|}{Boolq (FP32 Accuracy = 68.32\%)} \\ 
 \hline
 \hline
 64 & 40.48 & 40.10 & 39.43 & 39.90 & 69.20 & 68.41 & 69.45 & 68.56 \\
 \hline
 32 & 39.52 & 39.52 & 40.77 & 39.62 & 68.32 & 67.43 & 68.17 & 69.30  \\
 \hline
 16 & 39.81 & 39.71 & 39.90 & 40.38 & 68.10 & 66.33 & 69.51 & 69.42  \\
 \hline
 \hline
 \multicolumn{5}{|c|}{Winogrande (FP32 Accuracy = 67.88\%)} & \multicolumn{4}{|c|}{Piqa (FP32 Accuracy = 78.78\%)} \\ 
 \hline
 \hline
 64 & 66.85 & 66.61 & 67.72 & 67.88 & 77.31 & 77.42 & 77.75 & 77.64 \\
 \hline
 32 & 67.25 & 67.72 & 67.72 & 67.00 & 77.31 & 77.04 & 77.80 & 77.37  \\
 \hline
 16 & 68.11 & 68.90 & 67.88 & 67.48 & 77.37 & 78.13 & 78.13 & 77.69  \\
 \hline
\end{tabular}
\caption{\label{tab:mmlu_abalation} Accuracy on LM evaluation harness tasks on GPT3-8B model.}
\end{table}

\begin{table} \centering
\begin{tabular}{|c||c|c|c|c||c|c|c|c|} 
\hline
 $L_b \rightarrow$& \multicolumn{4}{c||}{8} & \multicolumn{4}{c||}{8}\\
 \hline
 \backslashbox{$L_A$\kern-1em}{\kern-1em$N_c$} & 2 & 4 & 8 & 16 & 2 & 4 & 8 & 16  \\
 %$N_c \rightarrow$ & 2 & 4 & 8 & 16 & 2 & 4 & 2 \\
 \hline
 \hline
 \multicolumn{5}{|c|}{Race (FP32 Accuracy = 40.67\%)} & \multicolumn{4}{|c|}{Boolq (FP32 Accuracy = 76.54\%)} \\ 
 \hline
 \hline
 64 & 40.48 & 40.10 & 39.43 & 39.90 & 75.41 & 75.11 & 77.09 & 75.66 \\
 \hline
 32 & 39.52 & 39.52 & 40.77 & 39.62 & 76.02 & 76.02 & 75.96 & 75.35  \\
 \hline
 16 & 39.81 & 39.71 & 39.90 & 40.38 & 75.05 & 73.82 & 75.72 & 76.09  \\
 \hline
 \hline
 \multicolumn{5}{|c|}{Winogrande (FP32 Accuracy = 70.64\%)} & \multicolumn{4}{|c|}{Piqa (FP32 Accuracy = 79.16\%)} \\ 
 \hline
 \hline
 64 & 69.14 & 70.17 & 70.17 & 70.56 & 78.24 & 79.00 & 78.62 & 78.73 \\
 \hline
 32 & 70.96 & 69.69 & 71.27 & 69.30 & 78.56 & 79.49 & 79.16 & 78.89  \\
 \hline
 16 & 71.03 & 69.53 & 69.69 & 70.40 & 78.13 & 79.16 & 79.00 & 79.00  \\
 \hline
\end{tabular}
\caption{\label{tab:mmlu_abalation} Accuracy on LM evaluation harness tasks on GPT3-22B model.}
\end{table}

\begin{table} \centering
\begin{tabular}{|c||c|c|c|c||c|c|c|c|} 
\hline
 $L_b \rightarrow$& \multicolumn{4}{c||}{8} & \multicolumn{4}{c||}{8}\\
 \hline
 \backslashbox{$L_A$\kern-1em}{\kern-1em$N_c$} & 2 & 4 & 8 & 16 & 2 & 4 & 8 & 16  \\
 %$N_c \rightarrow$ & 2 & 4 & 8 & 16 & 2 & 4 & 2 \\
 \hline
 \hline
 \multicolumn{5}{|c|}{Race (FP32 Accuracy = 44.4\%)} & \multicolumn{4}{|c|}{Boolq (FP32 Accuracy = 79.29\%)} \\ 
 \hline
 \hline
 64 & 42.49 & 42.51 & 42.58 & 43.45 & 77.58 & 77.37 & 77.43 & 78.1 \\
 \hline
 32 & 43.35 & 42.49 & 43.64 & 43.73 & 77.86 & 75.32 & 77.28 & 77.86  \\
 \hline
 16 & 44.21 & 44.21 & 43.64 & 42.97 & 78.65 & 77 & 76.94 & 77.98  \\
 \hline
 \hline
 \multicolumn{5}{|c|}{Winogrande (FP32 Accuracy = 69.38\%)} & \multicolumn{4}{|c|}{Piqa (FP32 Accuracy = 78.07\%)} \\ 
 \hline
 \hline
 64 & 68.9 & 68.43 & 69.77 & 68.19 & 77.09 & 76.82 & 77.09 & 77.86 \\
 \hline
 32 & 69.38 & 68.51 & 68.82 & 68.90 & 78.07 & 76.71 & 78.07 & 77.86  \\
 \hline
 16 & 69.53 & 67.09 & 69.38 & 68.90 & 77.37 & 77.8 & 77.91 & 77.69  \\
 \hline
\end{tabular}
\caption{\label{tab:mmlu_abalation} Accuracy on LM evaluation harness tasks on Llama2-7B model.}
\end{table}

\begin{table} \centering
\begin{tabular}{|c||c|c|c|c||c|c|c|c|} 
\hline
 $L_b \rightarrow$& \multicolumn{4}{c||}{8} & \multicolumn{4}{c||}{8}\\
 \hline
 \backslashbox{$L_A$\kern-1em}{\kern-1em$N_c$} & 2 & 4 & 8 & 16 & 2 & 4 & 8 & 16  \\
 %$N_c \rightarrow$ & 2 & 4 & 8 & 16 & 2 & 4 & 2 \\
 \hline
 \hline
 \multicolumn{5}{|c|}{Race (FP32 Accuracy = 48.8\%)} & \multicolumn{4}{|c|}{Boolq (FP32 Accuracy = 85.23\%)} \\ 
 \hline
 \hline
 64 & 49.00 & 49.00 & 49.28 & 48.71 & 82.82 & 84.28 & 84.03 & 84.25 \\
 \hline
 32 & 49.57 & 48.52 & 48.33 & 49.28 & 83.85 & 84.46 & 84.31 & 84.93  \\
 \hline
 16 & 49.85 & 49.09 & 49.28 & 48.99 & 85.11 & 84.46 & 84.61 & 83.94  \\
 \hline
 \hline
 \multicolumn{5}{|c|}{Winogrande (FP32 Accuracy = 79.95\%)} & \multicolumn{4}{|c|}{Piqa (FP32 Accuracy = 81.56\%)} \\ 
 \hline
 \hline
 64 & 78.77 & 78.45 & 78.37 & 79.16 & 81.45 & 80.69 & 81.45 & 81.5 \\
 \hline
 32 & 78.45 & 79.01 & 78.69 & 80.66 & 81.56 & 80.58 & 81.18 & 81.34  \\
 \hline
 16 & 79.95 & 79.56 & 79.79 & 79.72 & 81.28 & 81.66 & 81.28 & 80.96  \\
 \hline
\end{tabular}
\caption{\label{tab:mmlu_abalation} Accuracy on LM evaluation harness tasks on Llama2-70B model.}
\end{table}

%\section{MSE Studies}
%\textcolor{red}{TODO}


\subsection{Number Formats and Quantization Method}
\label{subsec:numFormats_quantMethod}
\subsubsection{Integer Format}
An $n$-bit signed integer (INT) is typically represented with a 2s-complement format \citep{yao2022zeroquant,xiao2023smoothquant,dai2021vsq}, where the most significant bit denotes the sign.

\subsubsection{Floating Point Format}
An $n$-bit signed floating point (FP) number $x$ comprises of a 1-bit sign ($x_{\mathrm{sign}}$), $B_m$-bit mantissa ($x_{\mathrm{mant}}$) and $B_e$-bit exponent ($x_{\mathrm{exp}}$) such that $B_m+B_e=n-1$. The associated constant exponent bias ($E_{\mathrm{bias}}$) is computed as $(2^{{B_e}-1}-1)$. We denote this format as $E_{B_e}M_{B_m}$.  

\subsubsection{Quantization Scheme}
\label{subsec:quant_method}
A quantization scheme dictates how a given unquantized tensor is converted to its quantized representation. We consider FP formats for the purpose of illustration. Given an unquantized tensor $\bm{X}$ and an FP format $E_{B_e}M_{B_m}$, we first, we compute the quantization scale factor $s_X$ that maps the maximum absolute value of $\bm{X}$ to the maximum quantization level of the $E_{B_e}M_{B_m}$ format as follows:
\begin{align}
\label{eq:sf}
    s_X = \frac{\mathrm{max}(|\bm{X}|)}{\mathrm{max}(E_{B_e}M_{B_m})}
\end{align}
In the above equation, $|\cdot|$ denotes the absolute value function.

Next, we scale $\bm{X}$ by $s_X$ and quantize it to $\hat{\bm{X}}$ by rounding it to the nearest quantization level of $E_{B_e}M_{B_m}$ as:

\begin{align}
\label{eq:tensor_quant}
    \hat{\bm{X}} = \text{round-to-nearest}\left(\frac{\bm{X}}{s_X}, E_{B_e}M_{B_m}\right)
\end{align}

We perform dynamic max-scaled quantization \citep{wu2020integer}, where the scale factor $s$ for activations is dynamically computed during runtime.

\subsection{Vector Scaled Quantization}
\begin{wrapfigure}{r}{0.35\linewidth}
  \centering
  \includegraphics[width=\linewidth]{sections/figures/vsquant.jpg}
  \caption{\small Vectorwise decomposition for per-vector scaled quantization (VSQ \citep{dai2021vsq}).}
  \label{fig:vsquant}
\end{wrapfigure}
During VSQ \citep{dai2021vsq}, the operand tensors are decomposed into 1D vectors in a hardware friendly manner as shown in Figure \ref{fig:vsquant}. Since the decomposed tensors are used as operands in matrix multiplications during inference, it is beneficial to perform this decomposition along the reduction dimension of the multiplication. The vectorwise quantization is performed similar to tensorwise quantization described in Equations \ref{eq:sf} and \ref{eq:tensor_quant}, where a scale factor $s_v$ is required for each vector $\bm{v}$ that maps the maximum absolute value of that vector to the maximum quantization level. While smaller vector lengths can lead to larger accuracy gains, the associated memory and computational overheads due to the per-vector scale factors increases. To alleviate these overheads, VSQ \citep{dai2021vsq} proposed a second level quantization of the per-vector scale factors to unsigned integers, while MX \citep{rouhani2023shared} quantizes them to integer powers of 2 (denoted as $2^{INT}$).

\subsubsection{MX Format}
The MX format proposed in \citep{rouhani2023microscaling} introduces the concept of sub-block shifting. For every two scalar elements of $b$-bits each, there is a shared exponent bit. The value of this exponent bit is determined through an empirical analysis that targets minimizing quantization MSE. We note that the FP format $E_{1}M_{b}$ is strictly better than MX from an accuracy perspective since it allocates a dedicated exponent bit to each scalar as opposed to sharing it across two scalars. Therefore, we conservatively bound the accuracy of a $b+2$-bit signed MX format with that of a $E_{1}M_{b}$ format in our comparisons. For instance, we use E1M2 format as a proxy for MX4.

\begin{figure}
    \centering
    \includegraphics[width=1\linewidth]{sections//figures/BlockFormats.pdf}
    \caption{\small Comparing LO-BCQ to MX format.}
    \label{fig:block_formats}
\end{figure}

Figure \ref{fig:block_formats} compares our $4$-bit LO-BCQ block format to MX \citep{rouhani2023microscaling}. As shown, both LO-BCQ and MX decompose a given operand tensor into block arrays and each block array into blocks. Similar to MX, we find that per-block quantization ($L_b < L_A$) leads to better accuracy due to increased flexibility. While MX achieves this through per-block $1$-bit micro-scales, we associate a dedicated codebook to each block through a per-block codebook selector. Further, MX quantizes the per-block array scale-factor to E8M0 format without per-tensor scaling. In contrast during LO-BCQ, we find that per-tensor scaling combined with quantization of per-block array scale-factor to E4M3 format results in superior inference accuracy across models. 


\end{document}
\endinput
%%
%% End of file `sample-authordraft.tex'.
