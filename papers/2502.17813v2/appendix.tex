\documentclass{article}

\usepackage{bm}
\usepackage{svg}
\usepackage{times}
\usepackage{epsfig}
\usepackage{amsmath}
\usepackage{amssymb}
\usepackage{graphics}
\usepackage{multirow}
\usepackage{booktabs}
\usepackage{algorithm}
\usepackage{subcaption}
\usepackage[noend]{algpseudocode}

\let\labelindent\relax
\usepackage{enumitem}

\graphicspath{{../figures/}}
\DeclareMathOperator*{\argmin}{\arg\!\min}
\DeclareMathOperator*{\argmax}{\arg\!\max}
\DeclareMathOperator*{\maxdist}{d_{\text{max}}}
\DeclareMathOperator*{\maxcost}{c_{\text{max}}}

\captionsetup{belowskip=-5pt}
\captionsetup[algorithm]{format=hang,singlelinecheck=false}

\title{\LARGE \bf
Safe Multi-Agent Navigation guided by Goal-Conditioned Safe Reinforcement Learning
}

\author{Meng Feng$^{*1}$, Viraj Parimi$^{*1}$ and Brian Williams $^{1}$
\thanks{$^{1}$ Computer Science and Artificial Intelligence Laboratory, Massachusetts Institute of Technology, Cambridge, MA 01239. Corresponding at {\tt\footnotesize \{mfeng,vparimi,williams\}@mit.edu}. *These authors contributed equally to the paper.}
\thanks{This work was also supported by Defence Science and Technology Agency, Singapore}
}

\begin{document}
\section{Appendix}

\subsection{Hyperparameters}

\begin{table*}[ht]
  \centering
  \caption{Hyperparameters and Training Settings for Visual Navigation}
  \label{tab:hyperparameters}
  \begin{tabular}{ll}
    \toprule
    \textbf{Parameter} & \textbf{Value} \\
    \midrule
    Actor Learning Rate           & 1e-5 \\
    Actor Update Interval         & 1 \\
    Critic Learning Rate          & 1e-4 \\
    Cost Critic Learning Rate     & 1e-4 \\
    Distance Critic Bins          & 20 \\
    Cost Critic Bins              & 40 \\
    Targets Update Interval       & 5 \\
    Polyak Update Coefficient     & 0.05 \\
    Initial Lagrange Multiplier   & 0 \\
    Lagrange Learning Rate        & 0.035 \\
    Optimizer                     & Adam \\
    Visual Inputs Dimensions      & (4, 32, 32, 4) \\
    Replay Buffer Size            & 100000 \\
    Batch Size                    & 64 \\
    Initial Collect Steps         & 1000 \\
    Training Iterations           & 600000 \\
    Neural Network Architecture   & Conv(16, 8, 4) + Conv(32, 4, 4) + FC(256) \\
    Maximum Episode Steps         & 20 \\
    \bottomrule
  \end{tabular}
\end{table*}


\subsection{Limitations}

One key limitation of our approach is the absence of strict risk-bounded guarantees on the cumulative risk incurred by all agents. In real-world scenarios, an ideal system would accept user input regarding risk averseness and automatically adjust its behavior accordingly. Although there are existing methods for multi-objective optimization, to our knowledge, none guarantee bounded execution risk while also maximizing reward. Additionally, the constrained low-level policy was trained on a single agent in a fixed environment, so its applicability in dynamic settings has not been empirically validated—even though SoRB has demonstrated effectiveness in diverse environments. In fast-changing, dynamic settings, the safer behaviors provided by our approach may be less effective.

\subsection{Real-World Connections}

To validate the approach in real-world settings, it is essential to ensure that agents adhere to the timings of each waypoint determined by the high-level CBS search in order to avoid collisions with other agents. If deviations from the nominal trajectories occur, low-level agents should utilize individual collision avoidance strategies by communicating their positions with one another, ensuring that when agents come too close, one can yield until the other has passed. Moreover, for smoother deployment, the operating height of the agents must be considered, as it defines the obstacle boundaries relevant to the safety function. Incorporating the height (z-coordinate) could lead to more natural plans, such as enabling drones to avoid obstacles by flying over them.

\end{document}