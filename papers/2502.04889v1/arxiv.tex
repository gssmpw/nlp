\documentclass[11pt,a4paper]{article}

\usepackage[dvipdfmx]{graphicx}
\usepackage{amsmath}
\usepackage{amssymb}
\usepackage{authblk}
\usepackage{bm}
\usepackage{caption}
\usepackage{color}
\usepackage[T1]{fontenc}
\usepackage[colorlinks=true]{hyperref}
\usepackage{mathtools}
\usepackage[numbers]{natbib}
\usepackage{stmaryrd}
\usepackage{tgtermes}

\usepackage{amsthm}
\newcommand{\set}{}
\newtheorem{theorem}{Theorem}
\newtheorem{definition}[theorem]{Definition}
\newtheorem{lemma}[theorem]{Lemma}
\newtheorem{proposition}[theorem]{Proposition}
\newtheorem{corollary}[theorem]{Corollary}
\newtheorem{remark}{Remark}
\hypersetup{
  colorlinks=true,
  linkcolor={red!15!green!35!blue!95},
  citecolor={green!50!black},
  urlcolor={red!50!black}
}

%
% --- inline annotations
%
\newcommand{\red}[1]{{\color{red}#1}}
\newcommand{\todo}[1]{{\color{red}#1}}
\newcommand{\TODO}[1]{\textbf{\color{red}[TODO: #1]}}
% --- disable by uncommenting  
% \renewcommand{\TODO}[1]{}
% \renewcommand{\todo}[1]{#1}



\newcommand{\VLM}{LVLM\xspace} 
\newcommand{\ours}{PeKit\xspace}
\newcommand{\yollava}{Yo’LLaVA\xspace}

\newcommand{\thisismy}{This-Is-My-Img\xspace}
\newcommand{\myparagraph}[1]{\noindent\textbf{#1}}
\newcommand{\vdoro}[1]{{\color[rgb]{0.4, 0.18, 0.78} {[V] #1}}}
% --- disable by uncommenting  
% \renewcommand{\TODO}[1]{}
% \renewcommand{\todo}[1]{#1}
\usepackage{slashbox}
% Vectors
\newcommand{\bB}{\mathcal{B}}
\newcommand{\bw}{\mathbf{w}}
\newcommand{\bs}{\mathbf{s}}
\newcommand{\bo}{\mathbf{o}}
\newcommand{\bn}{\mathbf{n}}
\newcommand{\bc}{\mathbf{c}}
\newcommand{\bp}{\mathbf{p}}
\newcommand{\bS}{\mathbf{S}}
\newcommand{\bk}{\mathbf{k}}
\newcommand{\bmu}{\boldsymbol{\mu}}
\newcommand{\bx}{\mathbf{x}}
\newcommand{\bg}{\mathbf{g}}
\newcommand{\be}{\mathbf{e}}
\newcommand{\bX}{\mathbf{X}}
\newcommand{\by}{\mathbf{y}}
\newcommand{\bv}{\mathbf{v}}
\newcommand{\bz}{\mathbf{z}}
\newcommand{\bq}{\mathbf{q}}
\newcommand{\bff}{\mathbf{f}}
\newcommand{\bu}{\mathbf{u}}
\newcommand{\bh}{\mathbf{h}}
\newcommand{\bb}{\mathbf{b}}

\newcommand{\rone}{\textcolor{green}{R1}}
\newcommand{\rtwo}{\textcolor{orange}{R2}}
\newcommand{\rthree}{\textcolor{red}{R3}}
\usepackage{amsmath}
%\usepackage{arydshln}
\DeclareMathOperator{\similarity}{sim}
\DeclareMathOperator{\AvgPool}{AvgPool}

\newcommand{\argmax}{\mathop{\mathrm{argmax}}}     



\usepackage[top=30truemm,bottom=30truemm,left=25truemm,right=25truemm]{geometry}

\title{Any-stepsize Gradient Descent for Separable Data \\ under Fenchel--Young Losses}
\author[1,2]{Han Bao}
\author[3,4]{Shinsaku Sakaue}
\author[1,2]{Yuki Takezawa}
\affil[1]{Kyoto University}
\affil[2]{OIST}
\affil[3]{The University of Tokyo}
\affil[4]{RIKEN AIP}
\affil[ ]{Corresponding to: \texttt{bao@i.kyoto-u.ac.jp}}
\date{\today}

\allowdisplaybreaks[4]

\usepackage[capitalize]{cleveref}
\newlist{assumpenum}{enumerate}{1}
\setlist[assumpenum]{label={\Alph*.},ref={\theassumption\Alph*}}
\crefname{assumpenumi}{Assumption}{Assumptions}
\crefname{assumption}{Assumption}{Assumptions}
\crefname{corollary}{Corollary}{Corollaries}
\crefname{definition}{Definition}{Definitions}
\crefname{lemma}{Lemma}{Lemmas}
\crefname{proposition}{Proposition}{Propositions}
\crefname{figure}{Figure}{Figures}

\begin{document}

\maketitle

\begin{abstract}%
  The gradient descent (GD) has been one of the most common optimizer in machine learning.
  In particular, the loss landscape of a neural network is typically sharpened during the initial phase of training, making the training dynamics hover on the edge of stability.
  This is beyond our standard understanding of GD convergence in the stable regime where arbitrarily chosen stepsize is sufficiently smaller than the edge of stability.
  Recently, Wu~et~al.~(COLT2024) have showed that GD converges with arbitrary stepsize under linearly separable logistic regression.
  Although their analysis hinges on the self-bounding property of the logistic loss, which seems to be a cornerstone to establish a modified descent lemma, our pilot study shows that other loss functions without the self-bounding property can make GD converge with arbitrary stepsize.
  To further understand what property of a loss function matters in GD, we aim to show arbitrary-stepsize GD convergence for a general loss function based on the framework of \emph{Fenchel--Young losses}.
  We essentially leverage the classical perceptron argument to derive the convergence rate for achieving $\epsilon$-optimal loss, which is possible for a majority of Fenchel--Young losses.
  Among typical loss functions, the Tsallis entropy achieves the GD convergence rate $T=\Omega(\epsilon^{-1/2})$,
  and the R{\'e}nyi entropy achieves the far better rate $T=\Omega(\epsilon^{-1/3})$.
  We argue that these better rate is possible because of \emph{separation margin} of loss functions, instead of the self-bounding property.
\end{abstract}

\section{Introduction}
\label{section:introduction}
Gradient-based optimizers are prevalent in the modern machine learning community with deep learning thanks to its scalability and plasticity.
Among many variants, the gradient descent remains to be a standard choice.
GD with constant stepsize is written as follows:
\begin{equation}
  \label{equation:gd} \tag{GD}
  \wbf_{t+1} \defeq \wbf_t - \eta\nabla L(\wbf_t), \qquad \text{for $t=0,1,\dots,T-1$,}
\end{equation}
where $\wbf\in\Rbb^d$ is the optimization variables, $L(\cdot)$ is the loss function, and $\eta>0$ is stepsize fixed across all steps.
In convergence analysis of GD, a key ingredient is the \emph{descent lemma}~\citep[Section~1.2.3]{Nesterov2018}: for $\beta$-smooth objective $L$, the stepsize choice $\eta<2/\beta$ ensures that $L(\wbf_t)$ monotonically decreases.
Nonetheless, little optimization theory has been known beyond the threshold $\eta>2/\beta$; though modern neural network exhibits much smaller smoothness values than practically used stepsize values~\citep{Yao2018NeurIPS,Tsuzuku2020ICML}.
Moreover, recent studies have reported that GD trajectories of neural networks tend to inflate the sharpness of the loss landscape and hover on the \emph{edge of stability} (EoS) before convergence~\citep{Lewkowycz2020,Cohen2021ICLR,Ahn2022ICML}.

\begin{figure}
  \centering
  \includegraphics[width=0.4\textwidth]{figure/loss_tsallis_0.5} \hspace{10pt}
  \includegraphics[width=0.4\textwidth]{figure/loss_logistic} \\
  \includegraphics[width=0.4\textwidth]{figure/loss_tsallis_1.5} \hspace{10pt}
  \includegraphics[width=0.4\textwidth]{figure/loss_tsallis_2.0}
  \caption{
    Pilot studies of GD with the same toy dataset as \citet{Wu2024COLT}.
    The dataset consists of four points, $\xbf_1=[1,0.2]^\top$, $y_1=1$, $\xbf_2=[-2,0.2]^\top$, $y_2=1$, $\xbf_3=[-1,-0.2]^\top$, $y_3=-1$, $\xbf_4=[2,-0.2]^\top$, and $y_4=-1$.
    GD is run with initialization $\wbf_0=[0,0]^\top$.
    Note that the logistic loss corresponds to the Tsallis $1$-loss.
    The Tsallis $2$- and $q$-loss are also known as the modified Huber loss~\citep{Zhang2004ICML} and $q$-entmax loss~\citep{Peters2019ACL}, respectively.
  }
  \label{figure:pilot}
\end{figure}

Among several recent developments in the theory of arbitrary-stepsize GD (which we will review in \cref{section:related}), \citet{Wu2024COLT} is most closely relevant to ours.
They investigated the arbitrary-stepsize behavior of GD by using the binary logistic regression with a linearly separable data, a minimal synthetic setting.
They showed that GD initially oscillates with non-monotonic loss values (the EoS phase), which terminates in finite time (phase transition), and then the loss value decreases monotonically (the stable phase).
Beyond the logistic loss, these results have been extended to loss functions with the the \emph{self-bounding property}: for a continuously differentiable loss function $\ell\colon\Rbb\to\Rbb$ and its unsigned derivative $g(\cdot)\defeq|\ell'(\cdot)|$, $\ell$ satisfies
\begin{equation}
  \label{equation:self_bounding_property}
  \ell(z)\le\ell(x)+\ell'(x)(z-x)+C_\beta g(x)(z-x)^2
  \quad \text{$\forall z, x$ with $|z-x|<1$,} \quad \text{for some $C_\beta>0$.}
\end{equation}
The self-bounding property generalizes the polynomially-tailed loss introduced by \citet{Ji2020COLT} and \citet{Ji2021ALT}, and refines the standard smoothness property by allowing the smoothness modulus locally adaptive to the unsigned derivative, such that $C_\beta g(x)$.
Thus, large $\eta$ can be cancelled out with the vanishingly small loss gradient after the phase transition~\citep[Lemma~29]{Wu2024COLT}, and then the GD trajectory follows the descent direction.

In this paper, we study GD with arbitrary stepsize under a wider range of loss functions to identify a key factor to induce the convergent behavior.
This is motivated by our pilot study shown in \cref{figure:pilot}, where we found that GD with large stepsize such as $\eta=2^4$ remains to converge under the Tsallis $q$-loss (detailed in \cref{section:examples}), even if the stepsize has gone beyond the classical stable regime.
It is noteworthy therein that the Tsallis $q$-loss with $q>1$ does not enjoy the self-bounding property.
How much does the self-bounding property play a vital role in arbitrary-stepsize GD convergence?
We specifically consider \emph{Fenchel--Young losses}~\citep{Blondel2020JMLR}, a class of convex loss functions generated by a potential function $\phi$, as a template of loss functions.
Fenchel--Young losses have been used in applications such as structured prediction~\citep{Niculae2018ICML}, differentiable programming~\citep{Berthet2020NeurIPS}, and model selection~\citep{Bao2021AISTATS}, while being used as a theoretical tool for online learning~\citep{Sakaue2024COLT,Sakaue2025}.
We identify that Fenchel--Young losses with \emph{separation margin} (formally introduced in \cref{section:preliminary}), a relevant notion to the margin in support vector machines,
can benefit from better GD convergence rates.
Specifically, our main result is informally stated as follows.
\begin{theorem}[{Informal version of \cref{theorem:gd}}]
  \label{theorem:main}
  Consider a binary classification dataset that is linearly separable.
  We run \eqref{equation:gd} with arbitrary constant stepsize $\eta>0$ and initialization $\wbf_0=\zerobf$ under a Fenchel--Young loss (generated by twice continuously differentiable and convex potential~$\phi$) with separation margin.
  For $\epsilon>0$, after at most $T$ steps of \eqref{equation:gd}, where
  \[
    T=\Omega(\epsilon^{-\alpha}) \quad \text{and} \quad \alpha=\limsup_{\mu\downarrow0}\frac{\phi'(\mu)}{\mu\phi''(\mu)}\left[1-\frac{\phi(\mu)}{\mu\phi'(\mu)}\right],
  \]
  we have $L(\wbf_T)\le\epsilon$.%
  \footnote{
    Throughout this paper, we consider $\eta=\Theta(1)$ with respect to the error tolerance $\epsilon$ when we say arbitrary stepsize
    unless otherwise noted.
  }
\end{theorem}
Therefore, \eqref{equation:gd} converges regardless of the choice of stepsize $\eta$ under Fenchel--Young losses with separation margin.
The order of the convergence rate $T=\Omega(\epsilon^{-\alpha})$ differs across various potential $\phi$.
With a specific choice, the rate can be $T=\Omega(\epsilon^{-1/2})$ (with $\phi$ being the Tsallis $2$-entropy)
and $T=\Omega(\epsilon^{-1/3})$ (with $\phi$ being the R{\'e}nyi $2$-entropy, also known as the collision entropy~\citep{Bosyk2012}).
Remarkably, these convergence rates are better than the classical GD convergence rate $T=\Omega(\epsilon^{-1})$ under the stable regime,
and even better than the convergence rate of the logistic loss after undergoing the EoS and phase transition~\citep{Wu2024COLT}.
Both the Tsallis and R{\'e}nyi entropies above lack the self-bounding property but entail separation margin.
Therefore, we advocate the importance of separation margin for better GD convergence rates.
We compare different Fenchel--Young losses in \cref{section:examples} and contrast our convergence result with EoS and implicit bias in \cref{section:discussion}.

We present the formal statement of \cref{theorem:main} in \cref{section:main}.
Its proof significantly leverages the classical perceptron argument~\citep{Novikoff1962} without relying on the descent lemma.
Intuitively speaking, we track the growth of the parameter alignment $\inpr{\wbf_t}{\wbf_*}$ with the optimal separator $\wbf_*$.
When a loss entails separation margin, $\inpr{\wbf_t}{\wbf_*}$ cannot grow arbitrarily large (as we simulate in \cref{figure:norm} later)
while each step of \eqref{equation:gd} improves a lower bound of $\inpr{\wbf_t}{\wbf_*}$, leading to the convergence.
\Cref{section:proof_sketch} describes this proof overview in detail.


\subsection{Related work}
\label{section:related}
Gradient-based optimization with large stepsize has attracted significant attention recently.
Specifically, non-monotonic behaviors of loss functions~\citep{Xing2018} and the sharpness adaptivity to loss landscapes~\citep{Lewkowycz2020,Cohen2021ICLR} have been observed empirically.
\citet{Cohen2021ICLR} argued that the sharpness tends to initially increases until the classical stable regime breaks down, and hovers on this boundary, termed as the edge of stability.
This observation mainly sparks two questions: why the loss landscape hovers on EoS, and why converging.
Answering either question must go beyond the classical optimization theory under the stable regime.

On why the loss landscape hovers on EoS, let us make a brief review, though it is not a central focus of this paper: \citet{Ahn2022ICML} is a seminal work to empirically investigate the homogeneity of loss functions contributes to maintain EoS.
Later, \citet{Lyu2022NeurIPS} showed that normalized GD (represented by scale-invariant losses) adaptively leads their intrinsic stepsize toward sharpness reduction.
\citet{Ma2022} and \citet{Damian2023ICLR} attribute the sharpness fluctuation to the non-negligible third-order Taylor remainder of the loss landscape.

We rather focus on why GD converges with arbitrary stepsize.
In this line, previous studies show convergence based on specific models such as multi-scale loss function~\citep{Kong2020NeurIPS}, quadratic functions~\citep{Arora2022ICML}, matrix factorization~\citep{Wang2022ICLR,Chen2023ICML}, a scalar multiplicative model~\citep{Zhu2023ICLR,Kreisler2023ICML}, a sparse coding model~\citep{Ahn2023NeurIPS}, and linear logistic regression~\citep{Wu2023NeurIPS}.
Among them, we advocate the logistic regression setup proposed by \citet{Wu2023NeurIPS} because it is relevant to implicit bias of GD~\citep{Soudry2018,Ji2019COLT,Ravi2024NeurIPS}, and moreover, the follow-up work by \citet{Wu2024COLT} corroborates the benefit of large stepsize in GD convergence rate.
Our work is provoked by \citet{Wu2024COLT}, questioning what structure in a loss function yields GD convergence.
Indeed, we do observe in \cref{figure:pilot} that loss functions without the self-bounding property~\eqref{equation:self_bounding_property} can make GD converge, though the self-bounding property seems essential to calm EoS down to the stable phase~\citep{Wu2024COLT} as well as to establish the max-margin directional convergence~\citep{Ji2019COLT,Ravi2024NeurIPS}. 
A similar question to ours is raised by \citet{Tyurin2024}, who argues that the stable convergence of large-stepsize logistic regression might be an artifact due to the functional form of the logistic loss---%
eventually \citet{Tyurin2024} asserts that large-stepsize logistic regression behaves like the classical perceptron.
To this end, we show in \cref{theorem:gd} that arbitrary-stepsize GD can converge under a wide range of losses even without the self-bounding property \eqref{equation:self_bounding_property},
and moreover, occasionally yielding a better rate than the classical stable convergence rate.
We discuss it more in \cref{section:discussion}.
Lastly, we note that \citet{Meng2024} attempts to extend the separable logistic regression setup to the non-separable one; yet, we still do not have satisfactory results beyond the one-dimensional case.
Due to its intricateness, we follow the linearly separable case for now.

\subsection{Notation}
\label{section:notation}
Let $\Rbb_{\ge0}$ be the set of nonnegative reals.
Let $[n]\defeq\set{1,\dots,n}$ for $n\in\Nbb$.
Let $\onebf$ be the all-ones vector and $\ebf_i\in\Rbb^d$ be the $i$-th standard basis vector, i.e., all zeros except for the $i$-th entry being one.
For $\Scal\subseteq\Rbb^d$, $\interior(\Scal)$ denotes its (relative) interior, and $I_{\Scal}\colon\Rbb\to\set{0,\infty}$ its indicator function, which takes zero if $\mubf\in\Scal$ and $\infty$ otherwise.
For $\Omega\colon\Rbb^d\to\Rbb\cup\set{\infty}$, $\domain(\Omega)\defeq\setcomp{\mubf\in\Rbb^d}{\Omega(\mubf)<\infty}$ denotes its effective domain and $\Omega^*(\thetabf)\defeq\sup\setcomp{\inpr{\thetabf}{\mubf}-\Omega(\mubf)}{\mubf\in\Rbb^d}$ its convex conjugate.
Let $\triangle^d\defeq\setcomp{\mubf\in\Rbb_{\ge0}^d}{\inpr{\onebf}{\mubf}=1}$ be the probability simplex.
We introduce $\Ccal^k(\Ical)$ as the set of $k$-th continuously differentiable functions on the interval $\Ical\subseteq\Rbb$.

Let $\Psi\colon\Rbb^d\to\Rbb\cup\set{\infty}$ be a strictly convex function differentiable throughout $\interior(\domain\Psi)\ne\varnothing$.
We say $\Psi$ is of \emph{Legendre-type} if $\lim_{i\to\infty}\|\nabla\Psi(\xbf_i)\|_2=\infty$ whenever $\xbf_1, \xbf_2, \dots$ is a sequence in $\interior(\domain\Psi)$ converging to a boundary point of $\interior(\domain\Psi)$ (see \citet[Section 26]{Rockafeller1970}).


\section{Preliminary on Fenchel--Young losses}
\label{section:preliminary}
Fenchel--Young losses have been introduced by \citet{Blondel2020JMLR} as a general class of surrogate loss functions for structured prediction
and entail the calibration property~\citep{Wang2024JMLR}.
We choose Fenchel--Young losses because a vast majority of natural and common loss functions are included in this family.
Moreover, the separation margin property, one of the key features of Fenchel--Youngs losses, controls GD behaviors significantly.
\begin{definition}
  \label{definition:fy_loss}
  Let $\Omega\colon\Rbb^K\to\Rbb\cup\set{\infty}$ be a potential function.
  The induced \emph{Fenchel--Young loss} $L_\Omega\colon\domain(\Omega^*)\times\domain(\Omega)\to\Rbb_{\ge0}$ by $\Omega$ is defined as
  \[
    L_\Omega(\thetabf;\mubf)\defeq\Omega^*(\thetabf)+\Omega(\mubf)-\inpr{\thetabf}{\mubf}.
  \]
\end{definition}
In multiclass classification, $L_\Omega(\thetabf;\mubf)$ measures the proximity of a score vector $\thetabf$ and a target label $\mubf=\ebf_i$ (for a class $i\in[K]$).
By definition, $L_\Omega(\cdot,\mubf)$ is convex for any $\mubf\in\domain(\Omega)$.
Moreover, $L_\Omega(\thetabf;\mubf)\ge0$ from the Fenchel--Young inequality, and $L_\Omega(\thetabf;\mubf)=0$ holds if and only if $\mubf\in\partial\Omega^*(\thetabf)$.

We follow \citet[Section 4.4]{Blondel2020JMLR} to instantiate loss functions for binary classification, $K=2$.
For simplicity, the following set of assumptions is imposed on a potential function $\Omega$.
\begin{assumption}
  \label{assumption:regularizer}
  For a potential function $\Omega$, assume $\domain(\Omega)\subseteq\triangle^K$ and that $\Omega$ satisfies the zero-entropy condition $\Omega(\mubf)=0$ for $\mubf\in\set{\ebf_i}_{i\in[K]}$;
  convexity $\Omega((1-\alpha)\mubf+\alpha\mubf')\le(1-\alpha)\Omega(\mubf)+\alpha\Omega(\mubf')$ for $\mubf\ne\mubf'$ and $\alpha\in(0,1)$;
  symmetry $\Omega(\mubf)=\Omega(\Pbf\mubf)$ for any $K\times K$ permutation $\Pbf$.
\end{assumption}
Let us restrict ourselves to $K=2$ (binary classification) and write $\phi(\mu)\defeq\Omega([\mu,1-\mu]^\top)$.
If we choose $\thetabf=[s,-s]^\top\in\Rbb^2$ as a score vector, the Fenchel--Young loss can be written as
\[
  L_\Omega(\thetabf;\ebf_i)=\begin{cases}
    \phi^*(-s) & \text{if $i=1$,} \\
    \phi^*(s) & \text{if $i=2$,}
  \end{cases}
\]
and $\domain(\phi^*)=\Rbb$.
Hence, the Fenchel--Young loss is simplified as $\phi^*(-ys)$ if we relabel two classes $i=1$ and $i=2$ with $y=1$ and $y=-1$, respectively.
For the rest of this paper, we suppose a loss function takes the form $\ell(z)\defeq\phi^*(-z)$ with \cref{assumption:regularizer}, which is a symmetric margin-based loss with $z=ys$~\citep{Scott2012}.

\paragraph{Separation margin.}
For specific potential functions, Fenchel--Young losses entail a \emph{separation margin}~\citep[Section 5]{Blondel2020JMLR}, which is a generalized notion of classical margin in support vector machines.
\begin{definition}
  \label{definition:separation_margin}
  For a loss $\ell\colon\Rbb\to\Rbb_{\ge0}$, we say $\ell$ has the \emph{separation margin property} if there exists $m>0$ such that $z\ge m \implies \ell(z)=0$.
  We call the smallest possible $m$ the \emph{separation margin} of $\ell$.
\end{definition}
\citet[Proposition 6]{Blondel2020JMLR} shows that the existence of the separation margin property can be tested through the subgradient $\partial\phi$.
\begin{proposition}[{\citet{Blondel2020JMLR}}]
  \label{proposition:separation_margin}
  A binary Fenchel--Young loss $\ell(z)=\phi^*(-z)$ satisfying \cref{assumption:regularizer} has separation margin if and only if $\partial\phi(\mu)\ne\varnothing$ for any $\mu\in[0,1]$.
  When $\phi\in\Ccal^1((0,1))$ has separation margin $m$, we have
  \[
    m=-\lim_{\mu\downarrow0}\phi'(\mu).
  \]
\end{proposition}
For a differentiable $\phi$, the nonempty-subgradient condition requires that the derivative $\phi'(\mu)$ does not explode at the boundary points of the domain $\mu\in\domain(\phi)=\set{0,1}$.
In this case, $\phi$ is \emph{not} of Legendre-type~\citep{Rockafeller1970}.
As we will see later, the convergence behavior of GD hinges on the separation margin property of a loss function.
More detailed analysis of the separation margin property for binary classification can be found in \citet{Bao2021AISTATS}.

\paragraph{Examples.}
If we choose the binary Shannon negentropy $\phi(\mu)=\mu\ln\mu+(1-\mu)\ln(1-\mu)$, we recover the logistic loss $\phi^*(-z)=\ln(1+\exp(-z))$.
For this reason, (the negative of) a potential function $\phi$ is sometimes referred to as a generalized entropy function.
If we choose the negative of the Gini index $\phi(\mu)=\mu^2-\mu$, we can generate the modified Huber loss $\phi^*(-z)=\max\set{0,1-z}^2/4$ if $z\ge-1$ and $\phi^*(-z)=-z$ otherwise~\citep{Zhang2004ICML}, which is the binarized sparsemax loss~\citep{Martins2016ICML}.
If we choose $\phi(\mu)=\max\set{\mu,1-\mu}$, we recover the hinge loss $\phi^*(-z)=\max\set{0,1-z}$.
We discuss more examples in \cref{section:examples}.


\section{Convergence of large stepsize GD under Fenchel--Young losses}
\label{section:main}
We consistently assume the dataset is bounded and linearly separable.
\begin{assumption}
  \label{assumption:data}
  Assume the training data $(\xbf_i,y_i)_{i\in[n]}$ satisfies
  \begin{itemize}
    \item for every $i\in[n]$, $\|\xbf_i\|\le1$ and $y_i\in\set{\pm1}$;
    \item there is $\gamma>0$ and a unit vector $\wbf_*$ such that $\inpr{\wbf_*}{\zbf_i}\ge\gamma$ for every $i\in[n]$, where $\zbf_i\defeq y_i\xbf_i$.
  \end{itemize}
\end{assumption}
Instead of logistic regression, we choose a Fenchel--Young loss $\ell(z)=\phi^*(-z)$ associated with a binary potential function $\phi$, and minimize the following risk by \eqref{equation:gd} with fixed stepsize $\eta>0$ to learn a linear classifier $\wbf$:
\begin{equation}
  \label{equation:risk}
  L(\wbf)\defeq\frac1n\sum_{i\in[n]}\ell(\inpr{\wbf}{y_i\xbf_i})=\frac1n\sum_{i\in[n]}\ell(\inpr{\wbf}{\zbf_i}).
\end{equation}
We impose the following assumptions on our loss function.
\begin{assumption}
  \label{assumption:loss}
  Consider a loss $\ell\colon\Rbb\to\Rbb_{\ge0}$.
  \begin{assumpenum}
    \item \label{assumption:fy_loss} \textbf{Fenchel--Young loss.} Assume that $\ell(z)$ is a Fenchel--Young loss $\phi^*(-z)$ generated by a potential $\phi:\Rbb\to\Rbb\cup\set{\infty}$
    such that $\phi\in\Ccal^2((0,1))$ satisfies \cref{assumption:regularizer}, $\phi$ is strictly convex, and
    $\phi''>0$ on the interval $(0,1)$.

    \item \label{assumption:regular_loss} \textbf{Regularity.} Assume that $\rho(\lambda)\defeq\min_{z\in\Rbb}\lambda\ell(z)+z^2$ (for $\lambda\ge1$) is well-defined.

    \item \label{assumption:lipschitz_loss} \textbf{Lipschitz continuity.} For $g(\cdot)\defeq|\ell'(\cdot)|$, assume $g(\cdot)\le C_g$ for some $C_g>0$.
  \end{assumpenum}
\end{assumption}
We will later see that $\rho$ characterizes the growth rate of the parameter norm $\|\wbf_t\|$ during GD in~\eqref{equation:norm_ub}.
Now, we are ready to state our main result, the GD convergence rate for linearly separable data under Fenchel--Young losses.
Remarkably, we show convergence without the self-bounding property of a loss function, unlike \citet{Wu2024COLT}.
\begin{theorem}[Main result]
  \label{theorem:gd}
  Suppose \cref{assumption:data}
  and consider \eqref{equation:gd} with stepsize $\eta>0$ and $\wbf_0=\zerobf$ under a Fenchel--Young loss $\ell$ satisfying \cref{assumption:loss}.
  For any $\bar\epsilon\in(0,1)$, let
  \begin{equation}
    \label{equation:exponent}
    \alpha\defeq\sup_{\mu\in(0,\bar\epsilon]}\frac{\phi'(\mu)}{\mu\phi''(\mu)}\left[1-\frac{\phi(\mu)}{\mu\phi'(\mu)}\right]
    \quad \text{and} \quad
    C_\phi\defeq\frac{\bar\mu}{[\bar\mu\phi'(\bar\mu)-\phi(\bar\mu)]^\alpha},
  \end{equation}
  where $C_\phi>0$ depends on $\phi$ and $\bar\epsilon$ solely and $\bar\mu\defeq \min\set{g(\ell^{-1}(\bar\epsilon)),1}$.
  If $\alpha,C_\phi\in(0,\infty)$ and
  \[
    \text{for $\epsilon\in(0,n\bar\epsilon)$,} \quad
    T > \frac{n^{1+\alpha}}{C_\phi\gamma^2}\left(\frac{4\sqrt{\rho(\gamma^2\eta T)}}{\eta}+C_g\right)\epsilon^{-\alpha}
  \]
  holds, then we have $L(\wbf_T)\le\epsilon$.
\end{theorem}
This convergence guarantee even applies to non-smooth Fenchel--Young losses
as long as \cref{assumption:loss} is satisfied---note that $\phi$ must be strongly convex to ensure the smoothness of the associated Fenchel--Young loss~\citep[Proposition~2.4]{Blondel2020JMLR}.
As we see later in \cref{section:examples}, $\alpha$ and $C_\phi$ do neither diverge nor degenerate for arbitrarily small $\bar\epsilon$ under many interesting examples of $\phi$.
When $\ell$ has a separation margin, we can simplify the statement of \cref{theorem:gd} (by applying \cref{lemma:rho_separation_margin} in \cref{appendix:lemma}).
\begin{corollary}
  \label{corollary:gd_separation_margin}
  Under the same setup with \cref{theorem:gd}, we additionally assume that $\ell$ has a separation margin $m>0$.
  If $(\alpha,C_\phi)$ with \eqref{equation:exponent} satisfies $\alpha,C_\phi\in(0,\infty)$ and
  \[
    \text{for $\epsilon\in(0,n\bar\epsilon)$,} \quad
    T > \frac{n^{1+\alpha}}{C_\phi\gamma^2}\left(\frac{4m}{\eta}+C_g\right)\epsilon^{-\alpha}
  \]
  holds, then we have $L(\wbf_T)\le\epsilon$.
\end{corollary}
Therefore, \eqref{equation:gd} under many common Fenchel--Young losses converges regardless of the choice of $\eta$.
Note that the classical GD convergence analysis under convex smooth functions provides $T=\Omega(\epsilon^{-1})$.
As we see later in \cref{section:examples}, some loss functions entail better rates with $\alpha<1$, summarized in \cref{table:loss}.


\subsection{Proof outline}
\label{section:proof_sketch}
The proof of \cref{theorem:gd} essentially relies on the perceptron convergence analysis~\citep{Novikoff1962} and the asymptotical order evaluation of rate functions~\citep{Bao2023COLT}.
We sketch the proof in this section to highlight the structure of the GD convergence in our setup and complete the proof in \cref{proof:gd}.

When we show the convergence of perceptron, we leverage an inequality of the following type:
\begin{equation}
  \label{equation:perceptron_argument}
  \underbrace{C_\text{L}t \le \inpr{\wbf_t}{\wbf_*}}_{(\clubsuit)} \le \underbrace{\|\wbf_t\| \le C_\text{U}(t)}_{(\diamondsuit)}
  \quad \text{for $t\ge1$},
\end{equation}
where $C_\text{L}>0$ is a non-degenerate constant independent of $t$.
The inequality $(\clubsuit)$ holds only while perceptron misclassifies some examples.
Thus, perceptron correctly classifies all examples after at most $T$ iterations such that $C_\text{L}T>C_\text{U}(T)$.
Such $T$ exists as long as $C_\text{U}(t)$ is sublinear in~$T$.

When it comes to our setup, an inequality of $(\clubsuit)$-type is obtained by recursively expanding the update~\eqref{equation:gd}
\[
  \inpr{\wbf_t}{\wbf_*}\ge\inpr{\wbf_{t-1}}{\wbf_*}+\frac{\gamma\eta}{n}g(\inpr{\wbf_{t-1}}{\zbf_{i_{t-1}}})
  \ge \dots \ge \frac{\gamma\eta}{n}\sum_{k=0}^{t-1}g(\inpr{\wbf_k}{\zbf_{i_k}}),
\]
where $\zbf_{i_k}$ is a misclassified example by $\wbf_k$.
Perceptron enjoys an inequality of $(\clubsuit)$-type immediately because it optimizes the loss function $\ell_{\text{per}}(z)=\max\set{-z,0}$, which yields $g(z)=1$ if $z<0$ (i.e., if misclassified).
When considering a Fenchel--Young loss satisfying \cref{assumption:loss}, we do not have a non-degenerate lower bound for $g(z)$ because we can make $g(z)$ arbitrarily close to zero.
Instead, we lower-bound $g(z)$ by a (non-degenerate) error tolerance $\epsilon_1>0$, $g(z)\ge\epsilon_1$, before we attain the $\epsilon$-optimal loss.
\cref{lemma:w_lb} and (a part of) \cref{lemma:order_evaluation_gd} in \cref{proof:gd} are relevant to $(\clubsuit)$.

To obtain an inequality of $(\diamondsuit)$-type, the standard perceptron analysis directly expands the update \eqref{equation:gd} $\|\wbf_t\|^2=\|\wbf_{t-1}-\eta\nabla L(\wbf_{t-1})\|^2$ recursively and upper-bound this by noting that $\ell_{\text{per}}$ has separation margin, leading to $C_\text{U}(t)=\Ocal(\sqrt{t})$.
Though this is possible for a Fenchel--Young loss with separation margin, we can improve this bound by the \emph{split optimization technique}, introduced by \citet{Wu2024COLT}.
Eventually, we can upper-bound $\|\wbf_t\|$ as follows:
\begin{equation}
  \label{equation:norm_ub}
  \|\wbf_t\| \le \frac{4\sqrt{\rho(\gamma^2\eta t)} + \eta C_g}{\gamma}.
\end{equation}
In particular, we have $\rho(\lambda)=\Ocal(1)$ when a loss has separation margin (see \cref{lemma:rho_separation_margin}), and therein $C_\text{U}(t)=\Ocal(1)$.
This is where separation margin plays a crucial role.
We recap the split optimization technique in \cref{lemma:split_optimization}, based on which \cref{lemma:risk_eos} in \cref{proof:gd_self_bounding} shows this inequality of $(\diamondsuit)$-type.

The remaining piece is to assess the order of the convergence rate.
After solving the inequality~$(\clubsuit,\diamondsuit)$ with $t=T$ being the stopping time, we have $T$ as a function of the error tolerance $\epsilon$, $T=f(\epsilon)$, where $f$ is a nondecreasing rate function depending on $\phi$.
To characterize the asymptotic order at vanishing $\epsilon$, we attempt to evaluate in the form $f(\epsilon)\simeq\epsilon^{\alpha_0}$ for an order parameter $\alpha_0>0$, which can be estimated by
\[
  \frac{\epsilon f'(\epsilon)}{f(\epsilon)} \overset{\epsilon\downarrow0}{\longrightarrow} \alpha_0,
  \quad \text{if the limit exists.}
\]
Thus, the order parameter $\alpha_0$ is solely determined by the functional form of potential function $\phi$.
This technique has been initially developed in functional analysis to estimate moduli of Banach and Orlicz spaces~\citep{Simonenko1964,Hudzik1991,Borwein2009}, and recently introduced in convex analysis to approximate a convex function by power functions~\citep{Ishige2022} and estimate moduli of convexity~\citep{Bao2023COLT,Bao2024}.
The general statement of the order evaluation is given in \cref{lemma:order_evaluation} and instantiated for GD convergence in \cref{lemma:order_evaluation_gd} in \cref{proof:gd}.


\section{Examples of loss functions}
\label{section:examples}

Now, we instantiate \cref{theorem:gd} for several examples of Fenchel--Young losses to discuss the convergence rate.
Instead of specifying a loss function $\ell(z)=\phi^*(-z)$, we directly specify its potential function $\phi$ subsequently.
For each $\phi$, we compute $(\alpha,C_\phi)$ in \eqref{equation:exponent} to investigate the convergence rate given by \cref{theorem:gd}, by leading $\bar\epsilon$ (and thus $\bar\mu$) vanishingly small.
In addition, we can compute separation margin $m$ by \cref{proposition:separation_margin} if exists; otherwise, we need to compute $\rho$ for a loss (see \cref{lemma:rho_estimate}).
\cref{table:loss} summarizes different loss functions and their GD convergence rates.
All the detailed calculations are deferred to \cref{appendix:example},
where we have an additional example of $\phi$ (pseudo-spherical entropy) with non-converging $\alpha$.

\begin{table}
  \centering
  \caption{
    Comparison of Fenchel--Young losses generated by different potential function $\phi$.
    Here, $m=\infty$ and $\beta=\infty$ indicate the lack of separation margin and smoothness, respectively.
    Since we do not have closed-form $\beta$ for the R{\'e}nyi entropy with $q\in(1,2)$, we merely show its lower bound.
    The convergence rates ignore the dependency on $\set{m,n,\gamma,\eta}$, and hold for arbitrary stepsize $\eta$ regardless of $\eta<2/\beta$.
  }
  \label{table:loss}
  \begin{tabular}{cc|cccc}
    \toprule
    {Potential $\phi$} & {Parameter $q$} & {Sep. mgn. $m$} & {Smoothness $\beta$} & {Order $\alpha$} & {Conv. rate for $T$} \\
    \midrule
    {Shannon} & {---} & {$\infty$} & {$1/4$} & {$1$} & {$\tilde\Omega(\epsilon^{-1})$} \\ \midrule
    {Semi-circle} & {---} & {$\infty$} & {$1/4$} & {$2$} & {$\Omega(\epsilon^{-4})$} \\ \midrule
    \multirow{3}{*}{Tsallis} & {$(0,1)$} & {$\infty$} & \multirow{2}{*}{\footnotesize$\dfrac{2^{q-3}}{q}$} & \multirow{2}{*}{\footnotesize$\dfrac1q$} & {$\Omega(\epsilon^{-2/q})$} \\ \cmidrule(l){3-3}
                          {} & {$(1,2]$} & \multirow{2}{*}{\footnotesize$\dfrac{q}{q-1}$} & {} & {} & {$\Omega(\epsilon^{-1/q})$} \\ \cmidrule(l){4-5}
                          {} & {$(2,\infty)$} & {} & {$\infty$} & {$1/2$} & {$\Omega(\epsilon^{-1/2})$} \\ \midrule
    \multirow{3}{*}{R{\'e}nyi} & {$(0,1)$} & {$\infty$} & {$1/4q$} & \multirow{2}{*}{\footnotesize$\dfrac{1}{q}$} & {$\Omega(\epsilon^{-2/q})$} \\ \cmidrule(l){3-3}
                            {} & {$(1,2)$} & \multirow{2}{*}{\footnotesize$\dfrac{q}{q-1}$} & {\color{gray} $(\ge1/4q)$} & {} & {$\Omega(\epsilon^{-1/q})$} \\ \cmidrule(l){5-5}
                            {} & {$2$} & {} & {$\infty$} & {$1/3$} & {$\Omega(\epsilon^{-1/3})$} \\
    \bottomrule
  \end{tabular}
\end{table}

\paragraph{Shannon entropy.}
Consider the binary Shannon (neg)entropy $\phi(\mu)=\mu\ln\mu+(1-\mu)\ln(1-\mu)$.
The generated Fenchel--Young loss is the logistic loss $\ell(z)=\ln(1+\exp(-z))$, which enjoys the self-bounding property and hence does not have separation margin (see \cref{appendix:separation_margin}).
The loss parameters are $\alpha=1$ and $C_\phi=1$.
Moreover, we know $C_g=1$ and $\rho(\lambda)\le1+\ln^2(\lambda)$~\citep{Wu2024COLT}.
Plugging this back to \cref{theorem:gd}, we have the $\epsilon$-optimal risk at most after
\[
  T\gtrsim\left[\frac{4\sqrt2(\log_2(\gamma^2\eta)+1)}{\eta}+\frac{1}{\ln2}\right]\frac{n^2\epsilon^{-1}}{\gamma^2}
  \quad \text{iterations,}
\]
where logarithmic factors in $\epsilon^{-1}$ are ignored.
This indicates the rate $T=\tilde\Omega(\epsilon^{-1})$, recovering the standard GD convergence rate under the stable regime but with arbitrary stepsize $\eta$.
In \cref{section:discussion}, we compare this rate with \citet{Wu2024COLT} in more detail.

\paragraph{Semi-circle entropy.}
Consider $\phi(\mu)=-2\sqrt{\mu(1-\mu)}$.
The generated Fenchel--Young loss (we call the semi-circle loss)
$\ell(z)=(-z+\sqrt{z^2+4})/{2}$
enjoys the self-bounding property and does not have separation margin since $\phi'(\mu)\to-\infty$ as $\mu\downarrow0$ (see \cref{appendix:separation_margin}).
The semi-circle loss is relevant to the exponential/boosting loss $\ell_\text{exp}(z)=\exp(-z)$, which has the semi-circle entropy as the Bayes risk~\citep{Buja2005,Agarwal2014JMLR}.
The loss parameters are $\alpha=2$ and $C_\phi=1$.
Moreover, we have $C_g=1$ and $\rho(\lambda)\le5\lambda/(2\ln\lambda)$.
Plugging this back to \cref{theorem:gd}, we have the $\epsilon$-optimal risk at most after
\[
  T>\underbrace{\frac{40n^6}{\gamma^2\eta\ln(2\gamma^2\eta)}\epsilon^{-4}}_\text{extra price for lacking separation margin} + \frac{2n^3}{\gamma^2}\epsilon^{-2}
  \quad \text{iterations,}
\]
where the first term $\Omega(\epsilon^{-4})$ is an extra price due to the lack of separation margin of the semi-circle loss.
For arbitrary stepsize $\eta$, the convergence rate is $T=\Omega(\epsilon^{-4})$, and stepsize $\eta$ as large as $\eta=\Omega(\epsilon^{-2})$ improves the rate to be $T=\tilde\Omega(\epsilon^{-2})$ by cancelling the extra term out.

This convergence rate of the semi-circle loss is even worse than the GD convergence rate for general convex smooth functions, $T=\Omega(\epsilon^{-1})$.
This is because the perceptron argument is merely sufficient for GD convergence.
Nonetheless, the perceptron argument more informatively states that we have $\inpr{\wbf_t}{\wbf_*}/\|\wbf_t\|\gtrsim\epsilon^\alpha$ after minimizing the risk at the $\epsilon$-optimal level---%
by combining the inequalities $(\clubsuit,\diamondsuit)$ (in \cref{equation:perceptron_argument}).
This indicates that the loss function with larger $\alpha$ yields slower parameter alignment toward $\wbf_*$.

\paragraph{Tsallis entropy.}
For $q>0$ with $q\ne1$, consider the Tsallis $q$-(neg)entropy
\[
  \phi(\mu)=\frac{\mu^q+(1-\mu)^q-1}{q-1}
\]
generalizing the Shannon entropy for non-extensive systems~\citep{Tsallis1988}.
It recovers the Shannon entropy at the limit $q\to1$.
The generated Fenchel--Young loss is known as the $q$-entmax loss~\citep{Peters2019ACL}.
We divide the case depending on parameter $q$:
\begin{itemize}
  \item \underline{When $0<q<1$:}
  $(\alpha,C_\phi)=(1/q, 1)$, and $\phi^*$ does not have separation margin.

  \item \underline{When $1<q\le2$:}
  $(\alpha,C_\phi)=(1/q, 1)$, and $\phi^*$ has separation margin $m=q/(q-1)$.

  \item \underline{When $2<q$:}
  $(\alpha,C_\phi)=(1/2, \sqrt{2/q})$, and $\phi^*$ has separation margin $m=q/(q-1)$.
\end{itemize}
For all cases, $\alpha$ and $C_\phi$ stay in $(0,\infty)$.
The convergence rate is $T=\Omega(\epsilon^{-2/q})$ for $q\in(0,1)$ (by \cref{corollary:gd_no_separation_margin}); $T=\Omega(\epsilon^{-1/q})$ for $q\in(1,2)$; $T=\Omega(\epsilon^{-1/2})$ for $2\le q$.
This suggests that we have a better convergence rate over the Shannon case when $q>1$ and the best rate is $\Omega(\epsilon^{-1/2})$.

\paragraph{R{\'e}nyi entropy.}
For $q\in(0,2]\setminus\set{1}$, consider the R{\'e}nyi $q$-(neg)entropy
\[
  \phi(\mu)=\frac{1}{q-1}\ln\left[\mu^q+(1-\mu)^q\right]
\]
generalizing the Shannon entropy (with the limit $q\to1$) while preserving additivity for independent events~\citep{Renyi1961}.
The R{\'e}nyi entropy extended beyond $q>2$ becomes nonconvex, which we do not consider.
The R{\'e}nyi $2$-entropy is referred to as the collision entropy~\citep{Bosyk2012}.

We divide the case depending on parameter $q$:
\begin{itemize}
  \item \underline{When $0<q<1$:}
  $(\alpha,C_\phi)=(1/q,1)$, and $\phi^*$ does not have separation margin.

  \item \underline{When $1<q<2$:}
  $(\alpha,C_\phi)=(1/q,1)$, and $\phi^*$ has separation margin $m=q/(q-1)$.

  \item \underline{When $q=2$:}
  $(\alpha,C_\phi)=(1/3,\sqrt[3]{3/8})$, and $\phi^*$ has separation margin $m=2$.
\end{itemize}
For all cases, $\alpha$ and $C_\phi$ stay in $(0,\infty)$.
The convergence rate is $T=\Omega(\epsilon^{-2/q})$ for $q\in(0,1)$ (by \cref{corollary:gd_no_separation_margin});
$T=\Omega(\epsilon^{-1/q})$ for $q\in(1,2)$;
$T=\Omega(\epsilon^{-1/3})$ for $q=2$.
Surprisingly, we have a ``leap'' of the order from $\epsilon^{-1/q}$ to $\epsilon^{-1/3}$ as $q\uparrow2$, and the convergence rate $\Omega(\epsilon^{-1/3})$ is far better than the Shannon and Tsallis cases.
When $q=2$, \cref{corollary:gd_separation_margin} implies that we have the $\epsilon$-optimal risk at most after
\[
  T>\sqrt[3]{8/3}\frac{n^{4/3}}{\gamma^2}\left(\frac{8}{\eta}+1\right)\epsilon^{-1/3} \quad \text{iterations.}
\]


\section{Discussion and open problems}
\label{section:discussion}

\paragraph{Phase transition.}
We recap \citet{Wu2024COLT}, who shows the existence of the phase transition from the EoS to stable phases.
\begin{assumption}
  \label{assumption2:loss}
  Consider a loss $\ell\in\Ccal^1(\Rbb)$ that is convex, nonincreasing, and $\ell(+\infty)=0$.
  \begin{assumpenum}
    \item \label{assumption:self_bounding} \textbf{Self-bounding property.}
    For some $C_\beta>0$, $g(\cdot)\le C_\beta\ell(\cdot)$ and
    \[
      \ell(z)\le\ell(x)+\ell'(x)(z-x)+C_\beta g(x)(z-x)^2 \quad \text{for $z$ and $x$ such that $|z-x|<1$.}
    \]

    \item \label{assumption:exponential_tail} \textbf{Exponential tail.}
    There is a constant $C_e>0$ such that $\ell(z)\le C_eg(z)$ for $z\ge0$.
  \end{assumpenum}
\end{assumption}
\begin{theorem}[{\citet{Wu2024COLT}}]
  \label{theorem:gd_self_bounding}
  Consider \eqref{equation:gd} with stepsize $\eta>0$ and initialization $\wbf_0=\zerobf$ under a loss $\ell$ satisfying \cref{assumption:regular_loss,assumption:lipschitz_loss}, and~\ref{assumption:self_bounding}.
  Let $T$ be the maximum number of steps.
  Then, we have the following:
  \begin{itemize}
    \item \textbf{The EoS phase.} For every $t>0$, we have
    \[
      \frac1t\sum_{k=0}^{t-1}L(\wbf_k) \le \frac{[6\sqrt{\rho(\gamma^2\eta t)} + \eta C_g]^2}{8\gamma^2\eta t}.
    \]

    \item \textbf{The stable phase.} If $s<T$ is such that
    \begin{equation}
      \label{equation:enter_stable_phase}
      L(\wbf_s) \le \min\set{\frac{1}{4C_\beta^2\eta}, \frac{\ell(0)}{n}},
    \end{equation}
    then \eqref{equation:gd} is in the stable phase, that is, $(L(\wbf_t))_{t=s}^T$ decreases monotonically, and moreover,
    \[
      L(\wbf_t) \le 5\frac{\rho(\gamma^2\eta(t-s))}{\gamma^2\eta(t-s)}, \quad t\in(s,T].
    \]

    \item \textbf{Phase transition time.} There exists a constant $C_1>0$ that only depends on $C_g$, $C_\beta$, and $\ell(0)$ such that the following holds. Let
    \[
      \tau\defeq\frac{1}{\gamma^2}\max\set{\frac{\psi^{-1}(C_1(\eta+n))}{\eta}, C_1(\eta+n)\eta}, \quad \text{where} \quad \psi(\lambda)\defeq\frac{\lambda}{\rho(\lambda)}.
    \]
    If $\tau\le T$, \eqref{equation:enter_stable_phase} holds for some $s\le\tau$.
    Moreover, if $\ell$ additionally satisfies \cref{assumption:exponential_tail} and $\eta\ge1$, there exists $C_2>0$ that depends on $C_e$, $C_g$, $C_\beta$, $\ell(0)$, and $n$ such that $\tau$ is improved as follows:
    \[
      \tau\defeq\frac{C_2}{\gamma^2}\max\set{\eta, n}.
    \]
  \end{itemize}
\end{theorem}
The proof consists of \cref{lemma:risk_eos} (the EoS phase), \cref{lemma:stable_phase_convergence} (the stable phase), and \cref{lemma:phase_transition,lemma:phase_transition_exponential_tail} (phase transition time), respectively, in \cref{proof:gd_self_bounding}.
We restate these proofs for linear models.

To get insight from \cref{theorem:gd_self_bounding}, consider $\ell$ is the logistic loss.
Since the logistic loss satisfies both \cref{assumption:self_bounding,assumption:exponential_tail} with $\rho(\lambda)=\tilde\Theta(1)$~\citep[Proposition~5]{Wu2024COLT}, the EoS phase terminates within $\tau=\Ocal(\eta)$ steps at most and converges in the rate $L(\wbf_t)=\tilde\Ocal(1/(\eta t))$ under the stable phase.
That said, stepsize $\eta$ trades off the phase transition time for the stable convergence rate; if we know the maximum number of steps $T$ in advance, the choice $\eta=\Theta(T)$ balances them, achieving the acceleration to $L(\wbf_t)=\tilde\Ocal(1/T^2)$.
This is arguably a groundbreaking instance to demonstrate how GD benefits from large stepsize and why EoS is necessary therein.

Nevertheless, we would like to highlight two caveats.
First, we must ensure $L(\wbf_s)\le\ell(0)/n$ in \eqref{equation:enter_stable_phase} before entering the stable phase.
This means that \emph{our linear model has already classified all points correctly during the EoS phase} since any single point $\zbf_i$ incurs loss at most $\ell(\inpr{\wbf_s}{\zbf_i})\le\ell(0) \implies \inpr{\wbf_s}{\zbf_i}\ge0$ (cf. \cref{lemma:correct_classification} in \cref{proof:gd_self_bounding}).%
\footnote{
  \citet{Cai2024NeurIPS} extends \citet{Wu2023NeurIPS} for two-layer near-homogeneous NNs, where it is not explicit that the model correctly classifies all points after the EoS phase.
  Taking a closer look, we can see that their Lemma~A.7 leverages the well-controlled risk, which is an alternative expression to ``$L(\wbf_s)\le\ell(0)/n$'' under their setup.
}
GD keeps improving the logistic loss after the stable phase just because the logistic loss does not enjoy separation margin and never touches strict zero.
We refer interested readers to the relevant discussion in \citet{Tyurin2024}, who argues that the stable loss improvement might be due to the artifact of the logistic loss.

Second, \eqref{equation:enter_stable_phase} suggests that the risk must be once $\Ocal(1/T)$-optimal (with the optimally balancing choice $\eta=\Theta(T)$) before benefitting from the super-fast rate $\tilde\Ocal(1/T^2)$.
Yet, GD under some loss functions including the Tsallis $q$-loss ($q>1$) and the R{\'e}nyi $q$-loss ($q>1$) achieves better risk with the same GD steps.
If our goal is simply to classify all training points, these alternative losses might do better jobs in terms of optimization solely.

\begin{figure}
  \centering
  \includegraphics[width=0.3\textwidth]{figure/norm_tsallis_0.5} \hspace{5pt}
  \includegraphics[width=0.3\textwidth]{figure/norm_logistic} \hspace{5pt}
  \includegraphics[width=0.3\textwidth]{figure/norm_tsallis_2.0}
  \caption{
    Under the same setup as \cref{figure:pilot}, we show $\|\wbf_t\|$ along the number of steps $t$ with different loss functions.
  }
  \label{figure:norm}
\end{figure}

\paragraph{Self-bounding property and implicit bias.}
Having said that, the phase transition may play an important role in implicit bias.
\citet{Wu2023NeurIPS} shows under the linearly separable case that logistic regression optimized with large-stepsize GD enlarges the norm $\|\wbf_t\|$ toward the max-margin direction in rate $\Omega(\ln(t))$.
Thus, we may argue that $\wbf_t$ gradually comes to classify all data points correctly during the EoS phase and evolves toward the max-margin direction in the stable phase.
Such a max-margin implicit bias is ubiquitous for separable logistic regression~\citep{Soudry2018},
and is demonstrated even for a nonlinear model as well~\citep{Cai2024NeurIPS}.

We reported how $\|\wbf_t\|$ evolves under the pilot setup in \cref{figure:norm} with different loss functions.
As seen, the logistic loss inflates $\|\wbf_t\|$ endlessly as well as the Tsallis $0.5$-loss, which does not have separation margin and satisfies the self-bounding property.
In stark contrast, the Tsallis $2$-loss prevents $\|\wbf_t\|$ from growing endlessly just because of its separation margin---%
recall the norm upper bounds of the norm $\|\wbf_t\|$ in \eqref{lemma:w_lb} and the growth rate $\rho$ in \cref{lemma:rho_separation_margin}.
This raises two open questions:
(1) Do we have similar implicit bias aligning toward the max-margin direction under a loss function with separation margin?
(2) What are benefits and caveats of endless growing of $\|\wbf_t\|$?
The latter is interesting in particular because excessively large $\|\wbf_t\|$ may practically lead to overconfidence issue~\citep{Wei2022ICML} and worse generalization due to prohibitively large within-class variance~\citep{Hou2022NeurIPS}.

The study on implicit bias beyond the logistic loss (or those with the self-bounding property) has been very scarce.
To our knowledge, \citet{Lizama2020} crafted the complete hinge loss,
which behaves like the hinge loss before GD converges to the zero risk yet incurs an extra penalty to artificially align the parameter toward the max-margin direction---%
\citet[Figure 2.1]{Lizama2020} illustrates this idea clearly.
Together with the benefits and caveats of the max-margin implicit bias, we believe this is an interesting open topic.

\paragraph{Finite-time convergence.}
Last but not least, we may potentially have another benefit of loss functions with separation margin.
Take a look at \cref{figure:pilot} again.
Loss functions without separation margin, such as the Tsallis $1.5$- and $2$-losses, converge to exact zero within finite time when sufficiently large stepsize is used.
Such finite-time convergence under the linearly separable case can be shown without significant challenges if we use perceptron, or even the hinge loss, while becoming highly non-trivial in case of twice-differentiable loss functions.
This is because the perceptron argument requires a non-degenerate lower bound on $\inpr{\wbf_t}{\wbf_*}$ (see \eqref{equation:perceptron_argument}), which is not straightforward therein as the loss gradient can be arbitrarily small positive (due to the twice differentiability of the loss).
We conjecture that an additional data assumption is necessary because the loss gradient could be adversarially vanishing against GD convergence, and leave this as future work.



\section*{Acknowledgment}
HB is supported by JST PRESTO (Grant No. JPMJPR24K6).
SS is supported by JST ERATO (Grant No. JPMJER1903).
YT is supported by JSPS KAKENHI (Grant No. 23KJ1336).

\bibliographystyle{plainnat}
\bibliography{reference}

\newpage
\appendix


\section{Technical lemmas}
\label{appendix:lemma}

We introduce the gradient potential for a loss function in consideration as follows:
\begin{equation}
  \label{equation:gradient_potential}
  G(\wbf)\defeq\frac1n\sum_{i=1}^ng(\inpr{\wbf}{\zbf_i}).
\end{equation}

\begin{lemma}
  \label{lemma:loss_bound}
  Consider a loss $\ell$ satisfying \cref{assumption:regular_loss}.
  Then, we have
  \[
    \ell\left(\sqrt{\rho(\lambda)}\right) \le \frac{\rho(\lambda)}{\lambda}.
  \]
\end{lemma}

\begin{proof}
  See \citet[Lemma 20]{Wu2024COLT}.
\end{proof}

\begin{lemma}
  \label{lemma:loss_monotone}
  Consider a Fenchel--Young loss $\ell(z)=\phi^*(-z)$ satisfying \cref{assumption:fy_loss}.
  Then, $\ell$ and $g$ are nonincreasing.
  Moreover, $\ell$ is strictly decreasing on $(-\infty,m)(\subseteq\Rbb)$ if $\ell$ has separation margin $m>0$;
  otherwise, $\ell$ is strictly decreasing on $\Rbb$.
\end{lemma}

\begin{proof}
  We have by Danskin's theorem~\citep{Danskin1966} $(\phi^*)'=(\phi')^{-1}$, and then
  \[
    \ell'(z) = -(\phi^*)'(-z) = -\underbrace{(\phi')^{-1}(-z)}_{\in\domain(\phi)\subseteq[0,1]} \le 0,
  \]
  which implies that $\ell$ is nonincreasing.
  Since $\ell$ is convex and nonincreasing we have that $g(\cdot)=|\ell'(\cdot)|=-\ell'(\cdot)$ is nonincreasing.

  For the latter part, if nonincreasing $\ell$ has separation margin, $\ell(z)=0$ if and only if $z\ge m$.
  Then, we have $\ell\equiv0$ on the interval $[m,\infty)\subseteq\Rbb$ and $\ell>0$ on the interval $(-\infty,m)\subseteq\Rbb$.
  On the latter interval, $\ell$ must be strictly decreasing because of its convexity.
  We can prove similarly for $\ell$ lacking separation margin.
\end{proof}

\begin{lemma}
  \label{lemma:rho_separation_margin}
  Consider a loss $\ell$ satisfying \cref{assumption:regular_loss}.
  Suppose that $\ell$ has separation margin $m>0$.
  Then, we have $\rho(\lambda) \le m^2$ for any $\lambda\ge1$.
\end{lemma}

\begin{proof}
  When $\ell$ has separation margin $m$ (see \cref{definition:separation_margin}), we have
  \[
    \lambda\ell(z)+z^2 = z^2 \quad \text{for $z\ge m$.}
  \]
  By the definition of $\rho$, we have
  \[
    \begin{aligned}
      \rho(\lambda)
      = \min_{z\in\Rbb}\lambda\ell(z)+z^2
      \le \min_{z\ge m}z^2
      = m^2.
    \end{aligned}
  \]
\end{proof}

\begin{lemma}[{Split optimization~\citep{Wu2024COLT}}]
  \label{lemma:split_optimization}
  Suppose \cref{assumption:data}
  and consider a convex and nonincreasing loss $\ell$ satisfying \cref{assumption:lipschitz_loss} and let $\ubf\defeq\ubf_1+\ubf_2$ such that
  \[
    \ubf_1=\theta\wbf_*, \qquad \ubf_2=\frac{\eta C_g}{2\gamma}\wbf_*.
  \]
  For every $t\ge1$, we have
  \[
    \frac{\|\wbf_t-\ubf\|^2}{2\eta t} + \frac1t\sum_{k=0}^{t-1}L(\wbf_k)
    \le \ell(\gamma\theta) + \frac{1}{2\eta t}\left(\theta+\frac{\eta C_g}{2\gamma}\right)^2.
  \]
\end{lemma}

\begin{proof}
  For $k<t$, we have
  \[
    \begin{aligned}
      \|\wbf_{k-1}-\ubf\|^2
      &= \|\wbf_k-\ubf\|^2 + 2\eta\inpr{\nabla L(\wbf_k)}{\ubf-\wbf_k} + \eta^2\|\nabla L(\wbf_k)\|^2 \\
      &= \|\wbf_k-\ubf\|^2 + 2\eta\inpr{\nabla L(\wbf_k)}{\ubf_1-\wbf_k} + \eta(2\inpr{\nabla L(\wbf_k)}{\ubf_2} + \eta\|\nabla L(\wbf_k)\|^2).
    \end{aligned}
  \]
  For the second term, we have
  \begin{equation}
    \label{equation:proof:supp:2}
    \begin{aligned}
      \inpr{\nabla L(\wbf_k)}{\ubf_1-\wbf_k}
      &= \frac1n\sum_{i=1}^n\ell'(\inpr{\wbf_k}{\zbf_i})\inpr{\zbf_i}{\ubf_1-\wbf_k} \\
      &= \frac1n\sum_{i=1}^n\ell'(\inpr{\wbf_k}{\zbf_i})(\inpr{\ubf_1}{\zbf_i} - \inpr{\wbf_k}{\zbf_i}) \\
      &\le \frac1n\sum_{i=1}^n\left[\ell(\inpr{\ubf_1}{\zbf_i}) - \ell(\inpr{\wbf_k}{\zbf_i})\right]
        && \text{($\ell$: convex)} \\
      &\le \ell(\gamma\theta) - L(\wbf_k).
        && \text{($\ell$: nonincreasing)}
    \end{aligned}
  \end{equation}
  For the third term, we have
  \[
    \begin{aligned}
      2&\inpr{\nabla L(\wbf_k)}{\ubf_2} + \eta\|\nabla L(\wbf_k)\|^2 \\
      &= \frac2n\sum_{i=1}^n\ell'(\inpr{\wbf_k}{\zbf_i})\inpr{\zbf_i}{\ubf_2} + \eta\left\|\frac1n\sum_{i=1}^n\ell'(\inpr{\wbf_k}{\zbf_i})\zbf_i\right\|^2 \\
      &\le \frac2n\sum_{i=1}^n\ell'(\inpr{\wbf_k}{\zbf_i})\inpr{\zbf_i}{\ubf_2} + \eta\left(\frac1n\sum_{i=1}^n\ell'(\inpr{\wbf_k}{\zbf_i})\right)^2
        && \text{($\|\zbf_i\|\le1$)} \\
      &\le \frac{2\|\ubf_2\|}{n}\sum_{i=1}^n\ell'(\inpr{\wbf_k}{\zbf_i})\inpr{\zbf_i}{\wbf_*} + \eta C_g\cdot G(\wbf_k)
        && \text{(\cref{assumption:lipschitz_loss} and $G(\wbf_k)\ge0$)} \\
      &\le -2\gamma\|\ubf_2\|G(\wbf_k) + \eta C_g\cdot G(\wbf_k)
        && \text{(\cref{assumption:data} and $G(\wbf_k)\ge0$)} \\
      &= 0,
    \end{aligned}
  \]
  where the last equality is by the choice of $\ubf_2$ and $G(\wbf)$ is defined in \eqref{equation:gradient_potential}.

  By combining them altogether, we have for $k<t$,
  \[
    \|\wbf_{k+1}-\ubf\|^2 \le \|\wbf_k-\ubf\|^2 + 2\eta\left[\ell(\gamma\theta) - L(\wbf_k)\right].
  \]
  Telescoping the sum from $0$ to $t-1$ and rearranging, we get
  \[
    \frac{\|\wbf_t-\ubf\|^2}{2\eta t} + \frac1t\sum_{k=0}^{t-1}L(\wbf_k)
    \le \ell(\gamma\theta) + \frac{\|\wbf_0-\ubf\|^2}{2\eta t},
  \]
  which completes the proof.
\end{proof}


\section{Proof of Theorem~\ref{theorem:gd}}
\label{proof:gd}

\begin{lemma}
  \label{lemma:w_lb}
  Suppose \cref{assumption:data}
  and consider \eqref{equation:gd} with any stepsize $\eta>0$ under a Fenchel--Young loss $\ell$ that satisfies \cref{assumption:fy_loss}.
  For $t\ge1$, assume $G(\wbf_k)\ge G_{\min} > 0$ for $k\in[t]$, where $G(\wbf)$ is defined in \eqref{equation:gradient_potential}.
  Then, we have
  \[
    \gamma\eta G_{\min}t\le\inpr{\wbf_t}{\wbf_*} - \inpr{\wbf_0}{\wbf_*}.
  \]
\end{lemma}

\begin{proof}
  By the perceptron argument~\citep{Novikoff1962}, we have
  \[
    \begin{aligned}
      \inpr{\wbf_{k+1}}{\wbf_*}
      &= \inpr{\wbf_k}{\wbf_*} - \eta\inpr{\nabla L(\wbf_k)}{\wbf_*} \\
      &= \inpr{\wbf_k}{\wbf_*} - \frac\eta n\inpr{\sum_{i=1}^n\ell'(\inpr{\wbf_k}{\zbf_i})\zbf_i}{\wbf_*} \\
      &= \inpr{\wbf_k}{\wbf_*} + \frac\eta n\sum_{i=1}^ng(\inpr{\wbf_k}{\zbf_i})\inpr{\wbf_*}{\zbf_i}
        && \text{(use $g(\cdot)=-\ell'(\cdot)$ by \cref{lemma:loss_monotone})} \\
      &\ge \inpr{\wbf_k}{\wbf_*} + \gamma\eta G(\wbf_k)
        && \text{(note $g(\cdot)\ge0$ and \cref{assumption:data})} \\
      &\ge \inpr{\wbf_k}{\wbf_*} + \gamma\eta G_{\min}.
    \end{aligned}
  \]
  Telescoping the sum, we have the desired inequality.
\end{proof}

\begin{lemma}[Order evaluation]
  \label{lemma:order_evaluation}
  Let $\Ical\subseteq\Rbb_{\ge0}$ be an open interval containing zero as the left end.
  For $f\colon\Rbb_{\ge0}\to\Rbb_{\ge0}\cup\set{\infty}$ that is nondecreasing and differentiable on $\Ical$ and satisfies $f(0)=0$, let
  \begin{equation}
    \label{equation:exponent_def}
    \alpha\defeq\sup_{x\in(0,x_0]}\frac{xf'(x)}{f(x)} \quad \text{for some $x_0\in\Ical$.}
  \end{equation}
  Then, for any $x\in(0,x_0)$, we have
  \[
     f(x)\ge Cx^\alpha, \quad \text{where} \quad C\defeq\frac{f(x_0)}{x_0^\alpha}.
  \]
\end{lemma}

\begin{proof}
  By the definition of $\alpha$, we have
  \[
    \alpha \ge \frac{xf'(x)}{f(x)} \quad \text{for all $x\in(0,x_0]$.}
  \]
  Then, for any $x\in(0,x_0)$, we have
  \[
    \alpha\ln\frac{x_0}{x} = \alpha\int_{x}^{x_0}\frac{\rd{s}}{s}\ge\int_x^{x_0}\frac{f'(s)}{f(s)}\rd{s} = \ln\frac{f(x_0)}{f(x)}.
  \]
  By reorganizing this inequality, we can prove the original argument.
\end{proof}

\begin{lemma}
  \label{lemma:order_evaluation_gd}
  Consider a Fenchel--Young loss $\ell(z)=\phi^*(-z)$ satisfying \cref{assumption:fy_loss}.
  Then, for arbitrary $0<\bar\epsilon<1$ and $\alpha$ defined in \eqref{equation:exponent}, we have
  \[
    g(\ell^{-1}(\epsilon)) \ge C_\phi\epsilon^{\alpha} \quad \text{for $\epsilon\in(0,\bar\epsilon)$}, \quad
    \text{where} \quad C_\phi\defeq \frac{g(\ell^{-1}(\bar\epsilon))}{\bar\epsilon^{\alpha}}.
  \]
\end{lemma}

\begin{proof}
  Choose any $\epsilon_0>0$.
  For $\epsilon\in(0,\epsilon_0)$, we can invert to get $z\equiv\ell^{-1}(\epsilon)$ because $\ell$ is strictly decreasing on $\ell^{-1}((0,1))$ (by \cref{lemma:loss_monotone})
  and hence invertible.
  Note that
  \[
    \begin{cases}
      z\in(\ell^{-1}(\epsilon_0),m) & \text{if $\ell$ has separation margin $m>0$,} \\
      z\in(\ell^{-1}(\epsilon_0),\infty) & \text{otherwise,}
    \end{cases}
  \]
  because $\ell^{-1}(\cdot)$ is nonincreasing.
  We write this range as $\Ical$,
  then $z=\ell^{-1}(\epsilon)\in\Ical$ for $\epsilon\in(0,\epsilon_0)$.

  Let us verify $g(z)=-\ell'(z)$ is differentiable at $z\in\Ical$ first.
  By the definition of Fenchel--Young losses, we have
  \[
    \ell(z)
    =\phi^*(-z)
    =\sup_{x\in(0,1)}\left[x\cdot(-z)-\phi(x)\right]
    =-z\cdot(\phi')^{-1}(-z)-\phi\left((\phi')^{-1}(-z)\right),
  \]
  for $z\in\Ical$,
  where we used the first-order optimality $-z=\phi'(x)$ of the convex conjugate at the last identity.
  Since $\phi$ is twice continuously differentiable and $\phi''>0$ on the interval $(0,1)$ by \cref{assumption:fy_loss},
  we can apply the inverse function theorem to have
  \[
    \ell'(z)
    =-(\phi')^{-1}(-z) - \frac{z}{\phi''\left((\phi')^{-1}(-z)\right)} - \frac{\phi'\left((\phi')^{-1}(-z)\right)}{\phi''\left((\phi')^{-1}(-z)\right)}
    =-(\phi')^{-1}(-z),
  \]
  for $z\in\Ical$.
  Since $\phi$ is twice continuously differentiable, we can apply the inverse function theorem once again to get $\ell''(z)$ for $z\in\Ical$, and hence $g$ is differentiable on $\Ical$.

  In addition, $\ell$ is continuously differentiable with non-degenerate derivative at $z\in\Ical$
  because $\ell$ is strictly decreasing on $\Ical$.
  From these observations, we can see that $g(\ell^{-1}(\cdot)) \eqdef f(\cdot)$ is nondecreasing on $\Ical$ and differentiable on $\Ical$
  (because it is the composition of two nondecreasing and differentiable functions $g$ and $\ell^{-1}$).
  Now we can apply \cref{lemma:order_evaluation} to this $f$.
  Let us compute the exponent $\alpha_\epsilon$ defined in \eqref{equation:exponent_def}.
  By the differentiability and non-degenerate derivative of $\ell$ on $\Ical$, we can apply the inverse function theorem on $\ell$ to have
  \begin{align*}
    \frac{\epsilon f'(\epsilon)}{f(\epsilon)}
    &= \frac{\epsilon}{g(\ell^{-1}(\epsilon))} \cdot g'(\ell^{-1}(\epsilon)) \cdot \frac{1}{\ell'(\ell^{-1}(\epsilon))}
      && \text{(inverse function theorem)} \\
    &= \frac{\ell(z)g'(z)}{g(z)\ell'(z)}
      && \text{($\epsilon\equiv\ell(z)$)} \\
    &= \frac{\ell(z)\ell''(z)}{[\ell'(z)]^2}
      && \text{($g(\cdot)=-\ell'(\cdot)$)} \\
    &= \frac{\phi^*(\bar z)\cdot(\phi^*)''(\bar z)}{[(\phi^*)'(\bar z)]^2}
      && \text{($\bar z\defeq-z$)} \\
    &\overset{\text{(A)}}{=} \frac{[\mu\phi'(\mu)-\phi(\mu)] \cdot \frac{1}{\phi''(\mu)}}{\mu^2}
      && \text{($\mu\equiv(\phi^*)'(\bar z)$)} \\
    &= \frac{\phi'(\mu)}{\mu\phi''(\mu)}\left[1-\frac{\phi(\mu)}{\mu\phi'(\mu)}\right],
  \end{align*}
  where at (A) we introduce $\mu$ as the dual of $\bar z$ by the mirror map $\phi'$ such that
  \[
    \bar z=\phi'(\mu) \quad \text{and} \quad \mu=(\phi^*)'(\bar z),
  \]
  which implies $\phi^*(\bar z)=\mu\phi'(\mu)-\phi(\mu)$ together with the definition of the convex conjugate,
  and
  \[
    [\phi''(\mu)]\cdot[(\phi^*)''(\bar z)]=1
  \]
  with Danskin's theorem~\citep{Danskin1966} and the inverse function theorem.
  Note that this identity is often referred to as Crouzeix's identity~\citep{Crouzeix1977}.
  Here, we have
  \[
    \begin{cases}
      \bar z\in\left(-m,-\ell^{-1}(\epsilon_0)\right) \text{~~and~~} \mu\in\left(0,g(\ell^{-1}(\epsilon_0))\right) & \text{if $\ell$ has separation margin $m>0$,} \\
      \bar z\in\left(-\infty,-\ell^{-1}(\epsilon_0)\right) \text{~~and~~} \mu\in\left(0,g(\ell^{-1}(\epsilon_0))\right) & \text{otherwise,}
    \end{cases}
  \]
  by noting that $g(z)=-\ell'(z)=(\phi^*)'(-z)$ is nonincreasing.
  With this primal-dual relationship, we have
  \[
    \phi^*(\bar z) = \mu\bar z - \phi(\mu) \quad \text{and} \quad
    [\phi''(\mu)] \cdot [(\phi^*)''(\bar z)] = 1
  \]
  by the definition of the convex conjugate and Crouzeix's identity.
  Now we are ready to apply \cref{lemma:order_evaluation}, which yields
  \[
    g(\ell^{-1}(\epsilon)) \ge C_\phi\epsilon^{\alpha} \quad \text{for $\epsilon\in(0,\bar\epsilon)$},
  \]
  where
  \[
    \alpha \defeq \sup_{\mu\in(0,\bar\epsilon]} \frac{\phi'(\mu)}{\mu\phi''(\mu)}\left[1-\frac{\phi(\mu)}{\mu\phi'(\mu)}\right],
    \quad
    C_\phi \defeq \frac{g(\ell^{-1}(\bar\epsilon))}{\bar\epsilon^{\alpha}},
    \quad \text{and} \quad
    \bar\epsilon\defeq g(\ell^{-1}(\epsilon_0)).
  \]
  Since the choice of $\epsilon_0>0$ was arbitrary and $\image(g)=\image((\phi^*)')=\domain(\phi')\subseteq[0,1]$,
  we can choose such $\bar\epsilon\in(0,1)$.
\end{proof}


{\renewcommand{\proofname}{Proof of \cref{theorem:gd}.}
\begin{proof}
  For $k\in[T-1]$, if $L(\wbf_k)>\epsilon$, there exists $i\in[n]$ such that $\ell(\inpr{\wbf_k}{\zbf_i})>\epsilon/n$.
  Then, we have for this $i\in[n]$,
  \[
    \begin{aligned}
      & \inpr{\wbf_k}{\zbf_i} \le \ell^{-1}(\epsilon/n) && \text{($\ell$ is strictly decreasing when $\ell>0$ by \cref{lemma:loss_monotone})} \\
      \implies & g(\inpr{\wbf_k}{\zbf_i}) \ge g(\ell^{-1}(\epsilon/n)). && \text{($g$ is nonincreasing by \cref{lemma:loss_monotone})}
    \end{aligned}
  \]
  This implies that
  \[
    G(\wbf_k)=\frac1n\sum_{j\in[n]}g(\inpr{\wbf_k}{\zbf_j})\ge \frac1ng(\inpr{\wbf_k}{\zbf_i}) \ge \frac1ng\left(\ell^{-1}\left(\frac\epsilon n\right)\right)
  \]
  holds while $L$ is $\epsilon$-suboptimal, that is, $L(\wbf_k)>\epsilon$.

  Next, fix $\wbf_0=\zerobf$.
  By \cref{lemma:loss_monotone}, we can use \cref{lemma:risk_eos}.
  By \cref{lemma:w_lb,lemma:risk_eos}, we can take $G_{\min}=g(\ell^{-1}(\epsilon/n))/n$ and have
  \begin{align*}
    \frac{\gamma\eta g\left(\ell^{-1}\left(\frac\epsilon n\right)\right)T}{n}
    &\le \inpr{\wbf_T}{\wbf_*} - \inpr{\wbf_0}{\wbf_*}
      && \text{(\cref{lemma:w_lb})} \\
    &= \inpr{\wbf_T}{\wbf_*}
      && \text{(with the choice of $\wbf_0=\zerobf$)} \\
    &\le \|\wbf_T\|
      && \text{(the Cauchy-Schwarz inequality with $\|\wbf_*\|=1$)} \\
    &\le \frac{4\sqrt{\rho(\gamma^2\eta T)} + \eta C_g}{\gamma}.
      && \text{(\cref{lemma:risk_eos})}
  \end{align*}
  By reorganizing and applying \cref{lemma:order_evaluation_gd}, we leverage the primal-dual relationship to have
  \[
    T \le \frac{n}{\gamma^2}\left(\frac{4\sqrt{\rho(\gamma^2\eta T)}}{\eta}+C_g\right)\cdot\frac{1}{g\left(\ell^{-1}\left(\frac\epsilon n\right)\right)}
    \le \frac{n}{\gamma^2}\left(\frac{4\sqrt{\rho(\gamma^2\eta T)}}{\eta}+C_g\right)\cdot\frac{1}{C_\phi\left(\frac\epsilon n\right)^{\alpha}},
  \]
  where $C_\phi$ is defined in \cref{lemma:order_evaluation_gd}.
  Therefore, $L$ is $\epsilon$-suboptimal after at most
  \[
    \frac{n^{1+\alpha}}{C_\phi\gamma^2}\left(\frac{4\sqrt{\rho(\gamma^2\eta t)}}{\eta}+C_g\right)\epsilon^{-\alpha}
  \]
  iterations.
  Finally, we verify that $C_\phi$ defined in \cref{lemma:order_evaluation_gd} matches \eqref{equation:exponent}.
  By introducing $\bar z$ as the dual of $\bar\mu$ such that
  \[
    \bar z=\phi'(\bar\mu) \quad \text{and} \quad \bar\mu=(\phi^*)'(\bar z),
  \]
  we have
  \[
    \begin{aligned}
      \bar\epsilon
      &= \ell(g^{-1}(\bar\mu))
        && \text{($g$ is invertible when $0<g(\cdot)<1$)} \\
      &= \phi^*(\bar z)
        && \text{($\bar\mu=(\phi^*)'(\bar z)=g(-\bar z)$ implies $g^{-1}(\bar\mu)=-\bar z$)} \\
      &= \bar\mu\phi'(\bar\mu)-\phi(\bar\mu),
    \end{aligned}
  \]
  where the invertibility of $g$ can be verified through the differentiability of $g$ as in \cref{lemma:order_evaluation_gd} (by relying upon $\phi''>0$ in \cref{assumption:fy_loss}),
  and the last identity follows by the definition of the convex conjugate.
  Plugging this into $C_\phi$ defined in \cref{lemma:order_evaluation_gd}, we see that it matches $C_\phi$ in \eqref{equation:exponent}.
  Thus, we have proven all statements.
\end{proof}
}


\section{Proof of Theorem~\ref{theorem:gd_self_bounding}}
\label{proof:gd_self_bounding}
Most of the results in this section have already been provided in \citet{Wu2024COLT}.
We restate the statements and proofs here to make the paper self-contained, and moreover,
simplify the statements from the NTK setup to the linear-model case
to highlight the essential structures.

\begin{lemma}[EoS phase]
  \label{lemma:risk_eos}
  Suppose \cref{assumption:data}
  and consider a convex and nonincreasing loss $\ell$ satisfying \cref{assumption:regular_loss,assumption:lipschitz_loss}.
  For every $t\ge1$,
  \[
    \frac1t\sum_{k=0}^{t-1}L(\wbf_k) \le \frac{[6\sqrt{\rho(\gamma^2\eta t)} + \eta C_g]^2}{8\gamma^2\eta t},
  \]
  and
  \[
    \|\wbf_t\| \le \frac{4\sqrt{\rho(\gamma^2\eta t)} + \eta C_g}{\gamma}.
  \]
\end{lemma}

\begin{proof}
  By invoking \cref{lemma:split_optimization} with the choice of $\theta$
  \[
    \theta=\frac{\sqrt{\rho(\gamma^2\eta t)}}{\gamma},
  \]
  we have
  \[
    \begin{aligned}
      \frac{\|\wbf_t-\ubf\|^2}{2\eta t} + \frac1t\sum_{k=0}^{t-1}L(\wbf_k)
      &\le \ell(\gamma\theta) + \frac{1}{2\eta t}\left(\theta+\frac{\eta C_g}{2\gamma}\right)^2 \\
      &\le \frac{\rho(\gamma^2\eta t)}{\gamma^2\eta t} + \frac{1}{2\eta t}\left(\theta+\frac{\eta C_g}{2\gamma}\right)^2,
        && \text{(\cref{lemma:loss_bound})} \\
    \end{aligned}
  \]
  which implies that
  \[
    \begin{aligned}
      \|\wbf_t\|
      &\le \|\wbf_t - \ubf\| + \|\ubf\| \\
      &\le \sqrt{\frac{2\rho(\gamma^2\eta t)}{\gamma^2} + \left(\theta+\frac{\eta C_g}{2\gamma}\right)^2} + \left(\theta+\frac{\eta C_g}{2\gamma}\right) \\
      &\le \frac{\sqrt{2\rho(\gamma^2\eta t)}}{\gamma} + 2\left(\theta+\frac{\eta C_g}{2\gamma}\right)
        && \text{($\sqrt{a+b}\le\sqrt{a}+\sqrt{b}$)} \\
      &\le \frac{4\sqrt{\rho(\gamma^2\eta t)} + \eta C_g}{\gamma},
    \end{aligned}
  \]
  and
  \[
    \begin{aligned}
      \frac1t\sum_{k=0}^{t-1}L(\wbf_k)
      &\le \frac{\rho(\gamma^2\eta t)}{\gamma^2\eta t} + \frac{1}{2\eta t}\left(\theta+\frac{\eta C_g}{2\gamma}\right)^2 \\
      &= \frac{\rho(\gamma^2\eta t)}{\gamma^2\eta t} + \frac{(2\sqrt{\rho(\gamma^2\eta t)} + \eta C_g)^2}{8\gamma^2\eta t} \\
      &\le \frac{[2(1+\sqrt{2})\sqrt{\rho(\gamma^2\eta t)} + \eta C_g]^2}{8\gamma^2\eta t}
        && \text{($a^2+b^2\le(a+b)^2$ for $a,b\ge0$)} \\
      &\le \frac{[6\sqrt{\rho(\gamma^2\eta t)} + \eta C_g]^2}{8\gamma^2\eta t}.
    \end{aligned}
  \]
  Thus, the proof is completed.
\end{proof}

\begin{lemma}
  \label{lemma:gradient_potential_eos}
  Suppose \cref{assumption:data}
  and consider a convex and nonincreasing loss $\ell$ satisfying \cref{assumption:regular_loss,assumption:lipschitz_loss}.
  Then, we have
  \[
    \frac1t\sum_{k=0}^{t-1}G(\wbf_k) \le \frac{4\sqrt{\rho(\gamma^2\eta t)} + \eta C_g}{\gamma^2\eta t}, \quad t \le T,
  \]
  where $G(\wbf)$ is defined in \eqref{equation:gradient_potential}.
\end{lemma}

\begin{proof}
  By the perceptron argument~\citep{Novikoff1962}, we have
  \[
    \begin{aligned}
      \inpr{\wbf_{t+1}}{\wbf_*}
      &= \inpr{\wbf_t}{\wbf_*} - \eta\inpr{\nabla L(\wbf_t)}{\wbf_*} \\
      &= \inpr{\wbf_t}{\wbf_*} -\frac\eta n\sum_{i=1}^n\ell'(\inpr{\wbf_t}{\zbf_i})\inpr{\wbf_*}{\zbf_i} \\
      &\ge \inpr{\wbf_t}{\wbf_*} - \frac{\gamma\eta}{n}\sum_{i=1}^n\ell'(\inpr{\wbf_t}{\zbf_i})
        && \text{(\cref{assumption:data} and note $-\ell'(\cdot)\ge0$)} \\
      &= \inpr{\wbf_t}{\wbf_*} - \gamma\eta G(\wbf_t).
    \end{aligned}
  \]
  Telescoping the sum, we have
  \[
    \frac1t\sum_{k=0}^{t-1}G(\wbf_k) \le \frac{\inpr{\wbf_t}{\wbf_*} - \inpr{\wbf_0}{\wbf_*}}{\gamma\eta t}
    \le \frac{\|\wbf_t\|}{\gamma\eta t}
    \le \frac{4\sqrt{\rho(\gamma^2\eta t)} + \eta C_g}{\gamma^2\eta t},
  \]
  where the last inequality is due to the parameter bound in \cref{lemma:risk_eos}.
\end{proof}

\begin{lemma}[Modified descent lemma]
  \label{lemma:stable_phase}
  Consider a loss satisfying \ref{assumption:self_bounding}.
  Suppose there exists $s<T$ such that
  \[
    L(\wbf_s)\le\frac{1}{4C_\beta^2\eta},
  \]
  then for every $t\in[s,T]$ we have
  \begin{enumerate}
    \item $L(\wbf_t)\le1/(4C_\beta^2\eta)$ and $G(\wbf_t)\le1/(4C_\beta\eta)$,
    \item $L(\wbf_{t+1})\le L(\wbf_t)-\frac{3\eta}{4}\|\nabla L(\wbf_t)\|^2\le L(\wbf_t)$,
  \end{enumerate}
  where $G(\wbf)$ is defined in \eqref{equation:gradient_potential}.
\end{lemma}

\begin{proof}
  We first show that Claim~1 implies Claim~2.
  By \cref{assumption:self_bounding}, we have
  \[
    \begin{aligned}
      \ell(\inpr{\wbf_{t+1}}{\zbf_i})
      &\le \ell(\inpr{\wbf_t}{\zbf_i}) + \ell'(\inpr{\wbf_t}{\zbf_i})\inpr{\wbf_{t+1}-\wbf_t}{\zbf_i} + C_\beta g(\inpr{\wbf_t}{\zbf_i})\inpr{\wbf_{t+1}-\wbf_t}{\zbf_i}^2 \\
      &\le \ell(\inpr{\wbf_t}{\zbf_i}) + \ell'(\inpr{\wbf_t}{\zbf_i})\inpr{\wbf_{t+1}-\wbf_t}{\zbf_i} + C_\beta g(\inpr{\wbf_t}{\zbf_i})\|\wbf_{t+1}-\wbf_t\|^2.
    \end{aligned}
  \]
  Taking average over $i\in[n]$, we get
  \[
    \begin{aligned}
      L(\wbf_{t+1})
      &\le L(\wbf_t) + \inpr{\nabla L(\wbf_t)}{\wbf_{t+1}-\wbf_t} + C_\beta G(\wbf_t)\|\wbf_{t+1}-\wbf_t\|^2 \\
      &= L(\wbf_t) - \eta\|\nabla L(\wbf_t)\|^2 + C_\beta\eta^2 G(\wbf_t)\|\nabla L(\wbf_t)\|^2 \\
      &\le L(\wbf_t) - \eta\|\nabla L(\wbf_t)\|^2 + \frac{\eta}{4}\|\nabla L(\wbf_t)\|^2
        && \text{(Claim 1)} \\
      &= L(\wbf_t) - \frac{3\eta}{4}\|\nabla L(\wbf_t)\|^2,
    \end{aligned}
  \]
  which verifies Claim~2.

  Next, we prove Claim~1 by induction.
  The base case $t=s$ holds by \cref{assumption:self_bounding} as follows:
  \begin{equation}
    \label{equation:self_bounding_potential}
    G(\wbf_s) = \frac1n\sum_{i=1}^ng(\inpr{\wbf_s}{\zbf_i}) \le \frac1n\sum_{i=1}^nC_\beta\ell(\inpr{\wbf_s}{\zbf_i}) = C_\beta L(\wbf_s) \le \frac{1}{4C_\beta\eta}.
  \end{equation}
  To prove the step case, we suppose $L(\wbf_k)\le1/(4C_\beta^2\eta)$ and $G(\wbf_k)\le1/(4C_\beta\eta)$ for $k=s,s+1,\dots,t$ and prove them for $k=t+1$.
  Since Claim~1 implies Claim~2, we have
  \[
    L(\wbf_{t+1})\le L(\wbf_t) \le \dots \le L(\wbf_s) \le \frac{1}{4C_\beta^2\eta}.
  \]
  Since $G(\wbf_{t+1})\le C_\beta L(\wbf_{t+1})$ holds as in \eqref{equation:self_bounding_potential}, we have
  \[
    G(\wbf_{t+1}) \le C_\beta L(\wbf_{t+1}) \le \frac{1}{4C_\beta\eta}.
  \]
  Thus, the step case is shown, and all claims are proven.
\end{proof}

\begin{lemma}
  \label{lemma:correct_classification}
  Consider a nonincreasing and nonnegative loss $\ell$.
  For every $\wbf$ such that
  \[
    L(\wbf) \le \frac{\ell(0)}{n},
  \]
  we have $y_i\inpr{\wbf}{\xbf_i}\ge0$ for $i\in[n]$.
\end{lemma}

\begin{proof}
  See \citet[Lemma 31]{Wu2024COLT}.
\end{proof}

\begin{lemma}[Stable phase]
  \label{lemma:stable_phase_convergence}
  Consider a nonincreasing loss $\ell$ satisfying \cref{assumption:regular_loss,assumption:self_bounding}.
  Suppose there exists $s<T$ such that
  \[
    L(\wbf_s)\le\min\set{\frac{1}{4C_\beta^2\eta}, \frac{\ell(0)}{n}}.
  \]
  Then, for every $t\in[0,T-s]$, we have
  \[
    L(\wbf_{s+t})\le 5\frac{\rho(\gamma^2\eta t)}{\gamma^2\eta t}.
  \]
\end{lemma}

\begin{proof}
  By the lemma assumption, we can apply \cref{lemma:stable_phase} for $s$ onwards.
  Therefore, we have for $k\ge0$,
  \begin{equation}
    \label{equation:proof:supp:1}
    \eta\|\nabla L(\wbf_{s+k})\|^2 \le \frac43[L(\wbf_{s+k}) - L(\wbf_{s+k+1})] \le \frac43L(\wbf_{s+k}).
  \end{equation}
  Choose a comparator centered at $\wbf_s$,
  \[
    \ubf \defeq \wbf_s + \theta\wbf_*, \quad
    \theta \defeq \frac{\sqrt{\rho(\eta^2\gamma t)}}{\gamma}.
  \]
  For $k\le t-1$, we have
  \[
    \begin{aligned}
      \|\wbf_{s+k+1}-\ubf\|^2
      &= \|\wbf_{s+k}-\ubf\|^2 + 2\eta\inpr{\nabla L(\wbf_{s+k})}{\ubf-\wbf_{s+k}} + \eta^2\|\nabla L(\wbf_{s+k})\|^2 \\
      &\le \|\wbf_{s+k}-\ubf\|^2 + 2\eta\inpr{\nabla L(\wbf_{s+k})}{\ubf-\wbf_{s+k}} + \frac43\eta L(\wbf_{s+k}).
        && \text{(by \eqref{equation:proof:supp:1})}
    \end{aligned}
  \]
  Following the same derivation of \eqref{equation:proof:supp:2}, we can bound the second term as follows:
  \[
    \inpr{\nabla L(\wbf_{s+k})}{\ubf-\wbf_{s+k}}
    \le \frac1n\sum_{i=1}^n\ell(\theta\gamma + \inpr{\wbf_s}{\zbf_i}) - L(\wbf_{s+k}).
  \]
  The assumption $L(\wbf_s)\le\ell(0)/n$ allows us to apply \cref{lemma:correct_classification}, so $\inpr{\wbf_s}{\zbf_i}\ge0$ and thus
  \[
    \ell(\theta\gamma+\inpr{\wbf_s}{\zbf_i}) 
    \le \ell(\theta\gamma)
    = \ell(\sqrt{\rho(\gamma^2\eta t)}),
  \]
  where $\ell$ is nonincreasing due to the lemma assumption.
  Consequently, we can control the second term by
  \[
    \inpr{\nabla L(\wbf_{s+k})}{\ubf-\wbf_{s+k}} \le \ell(\sqrt{\rho(\gamma^2\eta t)}) - L(\wbf_{s+k}).
  \]
  Plugging this back, we get
  \[
    \begin{aligned}
      \|\wbf_{s+k+1}-\ubf\|^2
      &\le \|\wbf_{s+k}-\ubf\|^2 + 2\eta[\ell(\sqrt{\rho(\gamma^2\eta t)}) - L(\wbf_{s+k})] + \frac43\eta L(\wbf_{s+k}) \\
      &\le \|\wbf_{s+k}-\ubf\|^2 + 2\eta\ell(\sqrt{\rho(\gamma^2\eta t)}) - \frac23\eta L(\wbf_{s+k}).
    \end{aligned}
  \]
  Telescoping the sum from $0$ to $t-1$ and rearranging, we get
  \[
    \begin{aligned}
      \frac{3\|\wbf_{s+t}-\ubf\|^2}{2\eta t} + \frac1t\sum_{k=0}^{t-1}L(\wbf_{s+k})
      &\le 3\ell(\sqrt{\rho(\gamma^2\eta t)}) + \frac{3\|\wbf_s-\ubf\|^2}{2\eta t} \\
      &\le 3\frac{\rho(\gamma^2\eta t)}{\gamma^2\eta t} + \frac{3\|\wbf_s-\ubf\|^2}{2\eta t}.
        && \text{(\cref{lemma:loss_bound})}
    \end{aligned}
  \]
  Finally, we can show the claims as follows:
  \[
    \begin{aligned}
      \frac1t\sum_{k=0}^{t-1}L(\wbf_{s+k})
      &\le 3\frac{\rho(\gamma^2\eta t)}{\gamma^2\eta t} + \frac{3\|\wbf_s-\ubf\|^2}{2\eta t} \\
      &= \frac92\frac{\rho(\gamma^2\eta t)}{\gamma^2\eta t} \\
      &\le 5\frac{\rho(\gamma^2\eta t)}{\gamma^2\eta t}.
    \end{aligned}
  \]
  By \cref{lemma:stable_phase}, $L(\wbf_t)$ is nonincreasing for $t\ge s$, and thus we have
  \[
    L(\wbf_{s+t}) \le \frac1t\sum_{k=0}^{t-1}L(\wbf_{s+k})
    \le 5\frac{\rho(\gamma^2\eta t)}{\gamma^2\eta t}.
  \]
\end{proof}

\begin{lemma}[Phase transition]
  \label{lemma:phase_transition}
  Suppose \cref{assumption:data}
  and consider a convex and nonincreasing loss $\ell$ that satisfies \cref{assumption:regular_loss,assumption:lipschitz_loss}.
  Define
  \[
    \psi(\lambda)=\frac{\lambda}{\rho(\lambda)}, \quad \lambda>0.
  \]
  Then, there is $C>0$ depending on $C_g$, $C_\beta$, and $\ell(0)$ such that the following holds.
  Let
  \[
    \tau\defeq\frac{1}{\gamma^2}\max\set{\frac{\psi^{-1}(C(\eta+n))}{\eta}, C(\eta+n)\eta}.
  \]
  If $\tau<T$, then there exists $s\in[0,\tau]$ such that
  \[
    L(\wbf_s)\le\min\set{\frac{1}{4C_\beta^2\eta}, \frac{\ell(0)}{n}}.
  \]
\end{lemma}

\begin{proof}
  Applying \cref{lemma:risk_eos} with $t=\tau$, we have
  \[
    \frac1\tau\sum_{k=0}^{\tau-1}L(\wbf_k)
    \le \frac{[6\sqrt{\rho(\gamma^2\eta\tau)} + \eta C_g]^2}{8\gamma^2\eta\tau}
    = \left[\frac{3}{\sqrt2}\sqrt{\frac{\rho(\gamma^2\eta\tau)}{\gamma^2\eta\tau}} + \frac{\sqrt2}{4}\frac{\eta C_g}{\sqrt{\gamma^2\eta\tau}}\right]^2.
  \]
  Choose $\tau$ such that
  \[
    \gamma^2\eta\tau
    \ge \max\set{
      \psi^{-1}\left(18[4C_\beta^2\eta+n/\ell(0)]\right),
      \frac12(\eta C_g)^2(4C_\beta^2\eta+n/\ell(0))
    }.
  \]
  It is clear that
  \[
    \frac{1}{\psi(\lambda)}=\frac{\rho(\lambda)}{\lambda}=\min_z\ell(z)+\frac{z^2}{\lambda}
  \]
  is a decreasing function.
  Then, we have
  \[
    \frac{3}{\sqrt2}\sqrt{\frac{\rho(\gamma^2\eta\tau)}{\gamma^2\eta\tau}}
    = \frac{3}{\sqrt2}\sqrt{\frac{1}{\psi(\gamma^2\eta\tau)}}
    \le \frac{3}{\sqrt2}\sqrt{\frac{1}{18[4C_\beta^2\eta+n/\ell(0)]}}
    = \frac12\frac{1}{\sqrt{4C_\beta^2\eta+n/\ell(0)}}
  \]
  and
  \[
    \frac{\sqrt2}{4}\frac{\eta C_g}{\sqrt{\gamma^2\eta\tau}}
    \le \frac12\frac{1}{\sqrt{4C_\beta^2\eta+n/\ell(0)}}.
  \]
  These two inequalities together imply that
  \[
    \frac1\tau\sum_{k=0}^{\tau-1}L(\wbf_k)
    \le \frac{1}{4C_\beta^2\eta+n/\ell(0)}
    \le \min\set{\frac{1}{4C_\beta^2\eta}, \frac{\ell(0)}{n}},
  \]
  which implies that there exists $s\le\tau$ for $L(\wbf_s)$ satisfies the right-hand side bound.
\end{proof}

\begin{lemma}[Phase transition time under exponential tail]
  \label{lemma:phase_transition_exponential_tail}
  Suppose \cref{assumption:data}
  and consider a nonincreasing loss $\ell$ satisfying \cref{assumption:regular_loss,assumption:lipschitz_loss}, and~\ref{assumption:exponential_tail}.
  Furthermore, assume $\eta\ge1$.
  Then, there exists $C>0$ depending on $C_e$, $C_g$, $C_\beta$, $\ell(0)$, and $n$ such that the following holds.
  Let
  \[
    \tau\defeq\frac{C}{\gamma^2}\max\set{\eta, n\ln n}.
  \]
  If $\tau\le T$, then there exists $s\in[0,\tau]$ such that
  \[
    L(\wbf_s)\le\min\set{\frac{1}{4C_\beta^2\eta}, \frac{\ell(0)}{n}}.
  \]
\end{lemma}

\begin{proof}
  Under \cref{assumption:exponential_tail}, we have
  \[
    \ell(z)\le C_eg(z) = -C_e\ell'(z), \quad \text{for $z\ge 0$,}
  \]
  which implies
  \[
    \frac{\ell'(z)}{\ell(z)}\le -C_e^{-1}, \quad \text{for $z\ge0$.}
  \]
  Integrating both sides, we get
  \[
    \ln\ell(z)\le\ln\ell(0)+\int_0^z\frac{\ell'(\zeta)}{\ell(\zeta)}\rd{\zeta}
    \le\ln\ell(0)-C_e^{-1}z, \quad \text{for $z\ge0$,}
  \]
  which implies
  \[
    \ell(z)\le\ell(0)\exp(-C_e^{-1}z), \quad \text{for $z\ge0$.}
  \]
  Using the exponential tail property, we have
  \[
    \rho(\lambda)
    =\min_{z\in\Rbb}\lambda\ell(z)+z^2
    \le \lambda\ell(C_e\ln(\lambda)) + C_e^2\ln^2(\lambda)
    \le \ell(0) + C_e^2\ln^2(\lambda).
  \]
  Applying \cref{lemma:gradient_potential_eos}, we have
  \begin{align*}
    \frac1\tau\sum_{k=0}^{\tau-1}G(\wbf_k)
    &\le \frac{4\sqrt{\rho(\gamma^2\eta\tau)}+\eta C_g}{\gamma^2\eta\tau} \\
    &\le \frac{4\sqrt{\ell(0)+C_e^2\ln^2(\gamma^2\eta\tau)}+\eta C_g}{\gamma^2\eta\tau} \\
    &\le \frac{4\sqrt{\ell(0)}+4C_e\ln(\gamma^2\eta\tau)+\eta C_g}{\gamma^2\eta\tau}
      && (\sqrt{a+b}\le\sqrt{a}+\sqrt{b}) \\
    &\le \frac{4C_e}{\eta}\frac{\ln(\gamma^2\tau)}{\gamma^2\tau} + \frac{C_g+4C_e}{\gamma^2\tau} + \frac{4\sqrt{\ell(0)}}{\eta}\frac{1}{\gamma^2\tau}.
  \end{align*}
  Here, we take $C>0$ depending on $C_e$, $C_g$, $C_\beta$, $\ell(0)$, and additionally $n$ such that
  \[
    \gamma^2\tau\ge C\max\set{\eta, n}
  \]
  and
  \[
    \frac{\ln C}{C}\le \frac{\min\set{\frac{1}{4C_eC_\beta^2}, \frac{\ell(0)}{C_e}}}{4C_e(1+\ln n)+C_g+4C_e+4\sqrt{\ell(0)}}.
  \]
  This choice is possible with sufficiently large $C\ge e$ because $(\ln C)/C$ is strictly decreasing in $C\ge e$ toward zero.
  Such $C$ enables us to have
  \begin{align*}
    \frac1\tau&\sum_{k=0}^{\tau-1}G(\wbf_k) \\
    &\le \frac{1}{C\max\set{\eta, n}}\left[\frac{4C_e}{\eta}(\ln C+\ln\max\set{\eta,n}) + C_g+4C_e + \frac{4\sqrt{\ell(0)}}{\eta}\right] \\
    &\le \frac{4C_e(\ln C+\ln n) + C_g+4C_e + 4\sqrt{\ell(0)}}{C\max\set{\eta, n}}
      && (\eta\ge1) \\
    &\le \frac{\ln C}{C}\frac{4C_e(1+\ln n) + C_g+4C_e + 4\sqrt{\ell(0)}}{\max\set{\eta, n}}
      && (\ln C\ge1) \\
    &\le \frac{\min\set{\frac{1}{4C_eC_\beta^2}, \frac{\ell(0)}{C_e}}}{\max\set{\eta,n}} \\
    &\le \min\set{\frac{1}{4C_eC_\beta^2\eta}, \frac{\ell(0)}{C_en}}.
  \end{align*}
  From this we have some $s\le\tau$ such that
  \[
    G(\wbf_s)\le\min\set{\frac{1}{4C_eC_\beta^2\eta}, \frac{\ell(0)}{C_en}}.
  \]
  This ensures that for every $i\in[n]$,
  \[
    \frac1ng(\inpr{\wbf_s}{\zbf_i})\le G(\wbf_s)
    =\frac1n\sum_{i=1}^ng(\inpr{\wbf_s}{\zbf_i})
    \le\frac{\ell(0)}{C_en}
    \le\frac{g(0)}{n},
  \]
  where the last inequality is due to \cref{assumption:exponential_tail}.
  The above implies $\inpr{\wbf_s}{\zbf_i}\ge0$ since $g(\cdot)$ is nonincreasing.
  Thus, we can apply \cref{assumption:exponential_tail} for $\inpr{\wbf_s}{\zbf_i}$ and get
  \[
    \ell(\inpr{\wbf_s}{\zbf_i})\le C_eg(\inpr{\wbf_s}{\zbf_i}).
  \]
  Taking an average over $i\in[n]$, we have
  \[
    L(\wbf_s)\le C_eG(\wbf_s).
  \]
  We complete the proof by plugging in the upper bound on $G(\wbf_s)$.
\end{proof}

\section{Separation margin and self-bounding property}
\label{appendix:separation_margin}
In this section, we discuss the relationship between separation margin and the self-bounding property.
First, we show that a loss function does not have separation margin if it satisfies the self-bounding property.

\begin{proposition}
  \label{proposition:no_separation_margin}
  Consider a loss $\ell\colon\Rbb\to\Rbb$ that is continuously differentiable and nonincreasing, and satisfies $\ell(z_0)>0$ for some $z_0\in\Rbb$.
  If $\ell$ satisfies \cref{assumption:self_bounding}, then $\ell$ does not have separation margin.
\end{proposition}
\begin{proof}
  Choose any $\epsilon\in(0,1/C_\beta)$.
  The convexity of $\ell$ implies that
  \[
    g(z) = -\ell'(z) \ge \frac{\ell(z) - \ell(z+\epsilon)}{\epsilon}
    \quad \text{for any $z\in\Rbb$.}
  \]
  By the self-bounding property (\cref{assumption:self_bounding}), we further have
  \[
    \frac{\ell(z) - \ell(z+\epsilon)}{\epsilon} \le g(z) \le C_\beta\ell(z).
  \]
  Solving this, we have
  \[
    \ell(z+\epsilon) \ge (1-C_\beta\epsilon)\ell(z).
  \]
  Thus, if $\ell(z)>0$ holds, we additionally have $\ell(z+\epsilon)>0$ for $\epsilon\in(0,1/C_\beta)$,
  and we conclude that $\ell$ cannot have separation margin because $\ell>0$ holds on the entire $\Rbb$.
\end{proof}

Next, we argue that the converse of \cref{proposition:no_separation_margin} does not hold,
that is, even if a loss $\ell$ does not have separation margin, it does not always imply that $\ell$ satisfies the self-bounding property.
A counterexample is a Fenchel--Young loss generated by the following potential function:
\[
  \phi(\mu)=\int_0^\mu\Phi^{-1}(p)\rd{p},
  \quad \text{where $\Phi$ is the standard normal CDF}
  \;\; \Phi(x)\defeq\frac12\left[1+\mathrm{erf}\left(\frac{x}{\sqrt2}\right)\right]
\]
and $\mathrm{erf}$ is the error function.
The generated Fenchel--Young loss is relevant to the probit model~\citep{McCullagh1989} because $\phi'$ is nothing else but the probit link function prevailing in generalized linear models.
Hence, we call the generated Fenchel--Young loss the \emph{probit Fenchel--Young loss} for convenience.
We can have a concise form of the probit Fenchel-Young loss:
\begin{align*}
  \ell(z)
  &= \phi^*(-z) \\
  &= \int_{-\infty}^{-z}(\phi')^{-1}(\zeta)\rd{\zeta} \\
  &= \int_{-\infty}^{-z}\Phi(\zeta)\rd{\zeta} \\
  &= [\zeta\Phi(\zeta)+\Phi'(\zeta)]_{-\infty}^{-z} \\
  &= -z\Phi(-z)+\Phi'(-z).
\end{align*}
The probit Fenchel--Young loss does not have separation margin because $\phi'(\mu)=\Phi^{-1}(\mu)\to-\infty$ as $\mu\downarrow0$ (see \cref{proposition:separation_margin}).
However, it does not satisfy the self-bounding property.
To see this, we have
\begin{align*}
  \frac{g(z)}{\ell(z)}
  &= -\frac{\ell'(z)}{\ell(z)} \\
  &= -\frac{-\Phi(-z)}{-z\Phi(-z)+\Phi'(-z)} \\
  &= \frac{\Phi(\bar z)}{\bar z\Phi(\bar z)+\Phi'(\bar z)},
    && (\bar z\equiv -z)
\end{align*}
which implies
\begin{align*}
  \lim_{z\to\infty}\frac{g(z)}{\ell(z)}
  &= \lim_{\bar z\to-\infty}\frac{\Phi'(\bar z)}{\Phi(\bar z)+\bar z\Phi'(\bar z)+\Phi''(\bar z)}
    && \text{(L'H{\^o}pital's rule)} \\
  &= \lim_{\bar z\to-\infty}\frac{\Phi'(\bar z)}{\Phi(\bar z)} \\
  &= \lim_{\bar z\to-\infty}\frac{\Phi''(\bar z)}{\Phi'(\bar z)}
    && \text{(L'H{\^o}pital's rule)} \\
  &= \lim_{\bar z\to-\infty}\frac{-\bar z\Phi'(\bar z)}{\Phi'(\bar z)} \\
  &= \infty.
\end{align*}
Hence, $g(z)$ cannot always be bounded from above by $\ell(z)$, that is, the self-bounding property is not satisfied.


\section{Omitted calculation for examples}
\label{appendix:example}
Here, we compute for each $\phi$,
\[
  \lim_{\mu\downarrow0}\frac{\phi'(\mu)}{\mu\phi''(\mu)}\left[1-\frac{\phi(\mu)}{\mu\phi'(\mu)}\right]
\]
to estimate the power $\alpha$ of the convergence rate provided in \cref{theorem:gd}, by making the error parameter $\bar\epsilon>0$ in \eqref{equation:exponent} arbitrarily small.
Correspondingly, we compute
\[
  \lim_{\mu\downarrow0}\frac{\mu}{[\mu\phi'(\mu)-\phi(\mu)]^\alpha}
\]
to estimate the constant $C_\phi$ in the convergence rate, verifying that $C_\phi$ neither degenerates nor diverges for arbitrarily small error parameter $\bar\epsilon>0$.

Before proceeding with each example, we provide a rough estimate of $\rho$ for loss functions without separation margin.
\begin{lemma}
  \label{lemma:rho_estimate}
  Consider a loss $\ell$ satisfying \cref{assumption:fy_loss,assumption:regular_loss} that does not have separation margin.
  Then,
  \[
    \rho(\lambda) \le -\phi\left(\frac12\right)\lambda.
  \]
\end{lemma}

\begin{proof}
  First, we rewrite $\rho$ as a dual form.
  By introducing the dual variable $\mu$ of $z$ by
  \[
    z=\phi'(\mu) \quad \text{and} \quad \mu=(\phi^*)'(z),
  \]
  we have
  \begin{align*}
    \rho(\lambda) &= \min_{z\in\Rbb}\lambda\ell(z)+z^2 \\
    &= \min_{z\in\Rbb}\lambda\phi^*(z)+z^2 \\
    &= \min_{\mu\in[0,1]}\lambda[\mu\phi'(\mu)-\phi(\mu)]+[\phi'(\mu)]^2,
  \end{align*}
  where we use the definition of the convex conjugate $\phi^*(z)=\mu z-\phi(\mu)$ at the last identity.
  Now, we write the objective as $R(\mu)$:
  \[
    R(\mu) \defeq \lambda[\mu\phi'(\mu)-\phi(\mu)] + [\phi'(\mu)]^2.
  \]
  Differentiating $R$, we have
  \[
    R'(\mu_\star) = [\lambda\mu_\star+2\phi'(\mu_\star)]\phi''(\mu_\star) = 0
    \quad \overset{\phi''>0}{\implies} \quad
    \phi'(\mu_\star)=-\frac\lambda2\mu_\star
  \]
  at the minimizer $\mu_\star$ of $R$.
  Plugging this back to $R$, we have
  \[
    \rho(\lambda)
    = R(\mu_\star)
    = \lambda\left[\mu_\star\left(-\frac\lambda2\mu_\star\right)-\phi(\mu_\star)\right] + \left(-\frac\lambda2\mu_\star\right)^2
    = -\lambda\phi(\mu_\star)
    \le -\phi\left(\frac12\right)\lambda,
  \]
  where the last inequality owes to that a convex potential satisfying \cref{assumption:regularizer} is minimized at the uniform distribution $\mu_\star=1/2$~\citep[Proposition 4]{Blondel2020JMLR}.
\end{proof}

By using \cref{lemma:rho_estimate}, we can simplify the convergence rate of \eqref{equation:gd} given by \cref{theorem:gd} for a loss that does not have separation margin.
Note that the following convergence rate is not sufficiently tight due to overestimation of $\rho$ by \cref{lemma:rho_estimate};
nevertheless, the provided convergence rate is convenient when we do not have an access to $\rho$ analytically.
\begin{corollary}
  \label{corollary:gd_no_separation_margin}
  Under the same setup with \cref{theorem:gd}, we additionally assume that $\ell$ does not have a separation margin.
  If $(\alpha,C_\phi)$ with \eqref{equation:exponent} satisfies $\alpha,C_\phi\in(0,\infty)$ and
  \[
    T>\frac{2C_gn^{1+\alpha}}{C_\phi\gamma^2}\epsilon^{-\alpha} + \frac{16[-\phi(1/2)]n^{2+2\alpha}}{C_\phi^2\gamma^2\eta}\epsilon^{-2\alpha}
    \quad \text{for $\epsilon\in(0,\bar\epsilon)$,}
  \]
  then we have $L(\wbf_T)\le\epsilon$.
\end{corollary}

\begin{proof}
  Combining \cref{theorem:gd} and \cref{lemma:rho_estimate}, we have the following convergence rate:
  \[
    T > \frac{4n^{1+\alpha}\sqrt{-\phi(1/2)}\epsilon^{-\alpha}}{C_\phi\gamma\sqrt{\eta}}\sqrt{T} + \frac{C_gn^{1+\alpha}\epsilon^{-\alpha}}{C_\phi\gamma^2}.
  \]
  Defining
  \[
    a\defeq\frac{4n^{1+\alpha}\sqrt{-\phi(1/2)}\epsilon^{-\alpha}}{C_\phi\gamma\sqrt{\eta}}
    \quad \text{and} \quad
    b\defeq\frac{C_gn^{1+\alpha}\epsilon^{-\alpha}}{C_\phi\gamma^2},
  \]
  we have the following inequality in $T$:
  \[
    T^2 - (a^2+2b)T + b^2 > 0.
  \]
  This can be solved for $T\ge1$:
  \[
    T>\frac{a^2+2b}{2}\Biggl[1+\underbrace{\sqrt{1-\left(\frac{2b}{a^2+2b}\right)^2}}_{\le1}\,\Biggr],
  \]
  for which $T>a^2+2b$ is sufficient.
  Thus, we have shown the statement.
\end{proof}

Throughout this section, we repeatedly use L'H{\^o}pital's rule.
When it is used, we notate by $(\ddagger)$.

\subsection{Shannon entropy}
For the Shannon entropy $\phi(\mu)=\mu\ln\mu+(1-\mu)\ln(1-\mu)$, we have
\[
  \phi'(\mu) = \ln\mu - \ln(1-\mu) \quad \text{and} \quad \phi''(\mu) = \frac1\mu + \frac{1}{1-\mu},
\]
which imply
\begin{align*}
  \lim_{\mu\downarrow0}\frac{\phi'(\mu)}{\mu\phi''(\mu)}\left[1-\frac{\phi(\mu)}{\mu\phi'(\mu)}\right]
  &= \lim_{\mu\downarrow0}\frac{\ln\frac{\mu}{1-\mu}}{1-\frac{\mu}{1-\mu}}\left[1-\frac{\mu\ln\mu+(1-\mu)\ln(1-\mu)}{\mu\ln\mu-\mu\ln(1-\mu)}\right] \\
  &= \lim_{\mu\downarrow0}\ln\frac{\mu}{1-\mu}\cdot\frac{\mu\ln\mu-\mu\ln(1-\mu)-\mu\ln\mu-(1-\mu)\ln(1-\mu)}{\mu\ln\frac{\mu}{1-\mu}} \\
  &= \lim_{\mu\downarrow0}\frac{-\ln(1-\mu)}{\mu} \\
  &\overset{(\ddagger)}= \lim_{\mu\downarrow0}\frac{1}{1-\mu} \\
  &= 1,
\end{align*}
and
\begin{align*}
  \lim_{\mu\downarrow0}\frac{\mu}{\mu\phi'(\mu)-\phi(\mu)}
  &= \lim_{\mu\downarrow0}\frac{\mu}{\mu\ln\mu-\mu\ln(1-\mu)-\mu\ln\mu-(1-\mu)\ln(1-\mu)} \\
  &= \lim_{\mu\downarrow0}\frac{\mu}{-\ln(1-\mu)}\cdot\frac{1}{2\mu-1} \\
  &= \lim_{\mu\downarrow0}\frac{\mu}{\ln(1-\mu)} \\
  &\overset{(\ddagger)}= \lim_{\mu\downarrow0}(1-\mu) \\
  &= 1.
\end{align*}

Finally, we derive the convergence rate for the logistic loss.
Plugging $\alpha=1$, $C_\phi=1$, $C_g=1$, and $\rho(\lambda)\le1+\ln^2(\lambda)\le2\ln^2(\lambda)$ to \cref{theorem:gd}, we have
\[
  T > \frac{n^2}{\gamma^2}\left(\frac{4\sqrt2\ln(\gamma^2\eta T)}{\eta}+1\right)\epsilon^{-1}
  = \left[\frac{4\sqrt2\ln(\gamma^2\eta)}{\eta}+1+\frac{4\sqrt2}{\eta}\ln T\right]\frac{n^2\epsilon^{-1}}{\gamma^2}.
\]
Dividing both ends by $\ln T$, we have
\[
  \frac{T}{\ln T} > \left[\left(\frac{4\sqrt2\ln(\gamma^2\eta)}{\eta}+1\right)\frac{1}{\ln T}+\frac{4\sqrt2}{\eta}\right]\frac{n^2\epsilon^{-1}}{\gamma^2},
\]
for which the following is sufficient when $T\ge2$:
\begin{align*}
  \frac{T}{\ln T} &> \left[\left(\frac{4\sqrt2\ln(\gamma^2\eta)}{\eta}+1\right)\frac{1}{\ln2}+\frac{4\sqrt2}{\eta}\right]\frac{n^2\epsilon^{-1}}{\gamma^2} \\
  &= \left[\frac{4\sqrt2\log_2(\gamma^2\eta)}{\eta}+\frac{1}{\ln2}+\frac{4\sqrt2}{\eta}\right]\frac{n^2\epsilon^{-1}}{\gamma^2}.
\end{align*}
By ignoring the logarithmic factor, we have
\[
  T\gtrsim\left[\frac{4\sqrt2\log_2(\gamma^2\eta)}{\eta}+\frac{1}{\ln2}+\frac{4\sqrt2}{\eta}\right]\frac{n^2\epsilon^{-1}}{\gamma^2}.
\]

\subsection{Semi-circle entropy}
For the semi-circle entropy $\phi(\mu)=-2\sqrt{\mu(1-\mu)}$, we first derive the analytical form of the corresponding Fenchel--Young loss.
We have
\[
  \phi'(\mu) = \frac{2\mu-1}{\sqrt{\mu(1-\mu)}} \quad \text{and} \quad \phi''(\mu) = \frac{1}{2[\mu(1-\mu)]^{3/2}}.
\]
The dual transform $(\phi^*)'$ is given by
\[
  (\phi^*)'(z) = (\phi')^{-1}(z) = \frac12\left[\frac{\frac z2}{\sqrt{\left(\frac z2\right)^2+1}}+1\right],
\]
thanks to the Danskin's theorem~\citep{Danskin1966}.
Then, we can derive the Fenchel--Young loss by the definition of the convex conjugate:
\[
  \ell(z) = \phi^*(-z) = -z(\phi^*)'(z)-\phi\left((\phi^*)'(-z)\right) = \frac{-z+\sqrt{z^2+4}}{2}.
\]

Next, we compute the loss parameters $\alpha$ and $C_\phi$ respectively as follows:
\begin{align*}
  \lim_{\mu\downarrow0}\frac{\phi'(\mu)}{\mu\phi''(\mu)}\left[1-\frac{\phi(\mu)}{\mu\phi'(\mu)}\right]
  &= \lim_{\mu\downarrow0}\frac{\frac{2\mu-1}{\sqrt{\mu(1-\mu)}}}{\frac{\mu}{2[\mu(1-\mu)]^{3/2}}}\left[1+\frac{2\sqrt{\mu(1-\mu)}}{\frac{\mu(2\mu-1)}{\sqrt{\mu(1-\mu)}}}\right] \\
  &= \lim_{\mu\downarrow0}2(2\mu-1)(1-\mu)\left[1+\frac{2(1-\mu)}{2\mu-1}\right] \\
  &= 2,
\end{align*}
and
\begin{align*}
  \lim_{\mu\downarrow0}\frac{\mu}{[\mu\phi'(\mu)-\phi(\mu)]^2}
  &= \lim_{\mu\downarrow0}\frac{\mu}{\left[\mu\frac{2\mu-1}{\sqrt{\mu(1-\mu)}}+2\sqrt{\mu(1-\mu)}\right]^2} \\
  &= \lim_{\mu\downarrow0}(1-\mu) \\
  &= 1.
\end{align*}

To estimate $\rho(\lambda)$,
\begin{align*}
  \rho(\lambda) &= \min_{z\in\Rbb}\lambda\ell(z)+z^2 \\
  &\le \lambda\frac{-\ln\lambda+\sqrt{\ln^2\lambda+4}}{2}+\ln^2\lambda && (z=\ln\lambda) \\
  &= \frac{2\lambda}{\ln\lambda+\sqrt{\ln^2\lambda+4}}+\ln^2\lambda \\
  &\le \frac{5\lambda}{2\ln\lambda},
\end{align*}
where we used
\[
  \frac{\lambda}{\ln\lambda} \ge \frac{2\lambda}{\ln\lambda+\sqrt{\ln^2\lambda+4}} \ge \frac23\cdot\ln^2\lambda \quad \text{for $\lambda\ge1$.}
\]

Finally, we derive the convergence rate for the semi-circle loss.
Plugging $\alpha=2$, $C_\phi=1$, $C_g=1$, and $\rho(\lambda)\le5\lambda/(2\ln\lambda)$ to \cref{theorem:gd}, we have
\[
  T>\frac{n^3}{\gamma^2}\left(\frac{4\sqrt{\frac52\frac{\gamma^2\eta T}{\ln(\gamma^2\eta T)}}}{\eta}+1\right)\epsilon^{-2}
  = \left[\frac{2\sqrt{10}}{\eta}\sqrt{\frac{\gamma^2\eta T}{\ln(\gamma^2\eta T)}}+1\right]\frac{n^3\epsilon^{-2}}{\gamma^2},
\]
for which the following is sufficient when $T\ge2$:
\[
  T>\left[\frac{2\sqrt{10}}{\eta}\sqrt{\frac{\gamma^2\eta T}{\ln(2\gamma^2\eta)}}+1\right]\frac{n^3\epsilon^{-2}}{\gamma^2}.
\]
Subsequently, we follow the same flow as in the proof of \cref{corollary:gd_no_separation_margin}.
Defining
\[
  a\defeq\frac{2\sqrt{10}n^3\epsilon^{-2}}{\gamma^2\eta}\sqrt{\frac{\gamma^2\eta}{\ln(2\gamma^2\eta)}}
  \quad \text{and} \quad
  b\defeq\frac{n^3\epsilon^{-2}}{\gamma^2},
\]
we have the following inequality in $T$:
\[
  T^2 - (a^2+2b)T + b^2 > 0.
\]
This can be solved for $T\ge1$:
\[
  T>\frac{a^2+2b}{2}\Biggl[1+\underbrace{\sqrt{1-\left(\frac{2b}{a^2+2b}\right)^2}}_{\le1}\Biggr],
\]
for which $T>a^2+2b$ is sufficient, namely,
\[
  T>\frac{40n^6}{\gamma^2\eta\ln(2\gamma^2\eta)}\epsilon^{-4} + \frac{2n^3}{\gamma^2}\epsilon^{-2}
\]
is sufficient.
Thus, the convergence rate is $T=\Omega(\epsilon^{-4})$.

\subsection{Tsallis entropy}
For the Tsallis entropy
\[
  \phi(\mu)=\frac{\mu^q+(1-\mu)^q-1}{q-1},
\]
define
\begin{align*}
  \phi_0(\mu) &= \mu^q+(1-\mu)^q-1, \\
  \phi_1(\mu) &= \mu^{q-1}-(1-\mu)^{q-1}, \\
  \phi_2(\mu) &= \mu^{q-2}+(1-\mu)^{q-2}
\end{align*}
When $0<q<2$ ($q\ne 1$),
\begin{align*}
  \lim_{\mu\downarrow0}\frac{\phi'(\mu)}{\mu\phi''(\mu)}\left[1-\frac{\phi(\mu)}{\mu\phi'(\mu)}\right]
  &= \frac{1}{q(q-1)}\lim_{\mu\downarrow0}\frac{1}{\phi_2(\mu)} \cdot \frac{q\mu\phi_1(\mu)-\phi_0(\mu)}{\mu^2} \\
  &= \frac{1}{q(q-1)}\lim_{\mu\downarrow0}\frac{1}{1+\left(\frac{\mu}{1-\mu}\right)^{2-q}} \cdot \frac{q\mu\phi_1(\mu)-\phi_0(\mu)}{\mu^q} \\
  &= \frac{1}{q(q-1)}\lim_{\mu\downarrow0}\frac{q\mu\phi_1(\mu)-\phi_0(\mu)}{\mu^q} \\
  &\overset{(\ddagger)}= \frac{1}{q(q-1)}\lim_{\mu\downarrow0}\frac{q\phi_1(\mu)+q(q-1)\mu\phi_2(\mu)-q\phi_1(\mu)}{q\mu^{q-1}} \\
  &= \frac{1}{q}\lim_{\mu\downarrow0}\frac{\mu^{q-1}+(1-\mu)^{q-2}\mu}{\mu^{q-1}} \\
  &\overset{(\ddagger)}= \frac{1}{q}\lim_{\mu\downarrow0}\frac{(q-1)\mu^{q-2}+(1-\mu)^{q-2}-(q-2)(1-\mu)^{q-3}\mu}{(q-1)\mu^{q-2}} \\
  &= \frac{1}{q}\lim_{\mu\downarrow0}\left[1+\frac{1}{q-1}\left(\frac{\mu}{1-\mu}\right)^{2-q}-(q-2)\left(\frac{\mu}{1-\mu}\right)^{3-q}\right] \\
  &= \frac1q.
\end{align*}
In addition, we have
\begin{align*}
  &\lim_{\mu\downarrow0}\frac{\mu}{[\mu\phi'(\mu)-\phi(\mu)]^{1/q}} \\
  &= \lim_{\mu\downarrow0}\frac{(q-1)^{1/q}\mu}{[q\mu\phi_1(\mu)-\phi_0(\mu)]^{1/q}} \\
  &= (q-1)^{1/q}\left\{\lim_{\mu\downarrow0}\frac{q\mu\phi_1(\mu)-\phi_0(\mu)}{\mu^q}\right\}^{-1/q} \\
  &= (q-1)^{1/q}\left\{q-1-\lim_{\mu\downarrow0}\frac{q\mu(1-\mu)^{q-1}+(1-\mu)^q-1}{\mu^q}\right\}^{-1/q} \\
  &\overset{(\ddagger)}= (q-1)^{1/q}\left\{q-1-\lim_{\mu\downarrow0}\frac{q(1-\mu)^{q-1}-q(q-1)\mu(1-\mu)^{q-2}-q(1-\mu)^{q-1}}{q\mu^{q-1}}\right\}^{-1/q} \\
  &= (q-1)^{1/q}\left\{q-1-(q-1)\lim_{\mu\downarrow0}\left(\frac{\mu}{1-\mu}\right)^{2-q}\right\}^{-1/q} \\
  &= (q-1)^{1/q}\cdot(q-1+0)^{-1/q} \\
  &= 1.
\end{align*}

When $q\ge2$,
\begin{align*}
  &\lim_{\mu\downarrow0}\frac{\phi'(\mu)}{\mu\phi''(\mu)}\left[1-\frac{\phi(\mu)}{\mu\phi'(\mu)}\right] \\
  &= \frac{1}{q(q-1)}\lim_{\mu\downarrow0}\frac{1}{\phi_2(\mu)} \cdot \frac{q\mu\phi_1(\mu)-\phi_0(\mu)}{\mu^2} \\
  &= \frac{1}{q(q-1)}\lim_{\mu\downarrow0}\frac{q\mu\phi_1(\mu)-\phi_0(\mu)}{\mu^2} \\
  &= \frac{1}{q(q-1)}\lim_{\mu\downarrow0}\frac{q[\mu^q-(1-\mu)^{q-1}\mu]-\mu^q-(1-\mu)^q}{\mu^2} \\
  &\overset{(\ddagger)}= \frac{1}{q(q-1)}\lim_{\mu\downarrow0}\frac{q[q\mu^{q-1}-(1-\mu)^{q-1}+(q-1)(1-\mu)^{q-2}\mu]-q\mu^{q-1}+q(1-\mu)^{q-1}}{2\mu} \\
  &= \lim_{\mu\downarrow0}\frac{\mu^{q-2}+(1-\mu)^{q-2}}{2} \\
  &= \frac12.
\end{align*}
In addition, we have
\begin{align*}
  \lim_{\mu\downarrow0}\frac{\mu}{[\mu\phi'(\mu)-\phi(\mu)]^{1/2}}
  &= (q-1)^{1/2}\left\{\lim_{\mu\downarrow0}\frac{q\mu\phi_1(\mu)-\phi_0(\mu)}{\mu^2}\right\}^{-1/2} \\
  &\overset{(\ddagger)}= (q-1)^{1/2}\left\{\lim_{\mu\downarrow0}\frac{q\phi_1(\mu)+q(q-1)\mu\phi_2(\mu)-q\phi_1(\mu)}{2\mu}\right\}^{-1/2} \\
  &= (q-1)^{1/2}\left\{\frac{q(q-1)}{2}\lim_{\mu\downarrow0}[\mu^{q-2}+(1-\mu)^{q-2}]\right\}^{-1/2} \\
  &= (q-1)^{1/2}\cdot\left[\frac{q(q-1)}{2}\right]^{-1/2} \\
  &= \sqrt{\frac2q}.
\end{align*}

\subsection{R{\'e}nyi entropy}
For the R{\'e}nyi entropy
\[
  \phi(\mu)=\frac{1}{q-1}\ln\left[\mu^q+(1-\mu)^q\right],
\]
define
\begin{align*}
  \phi_0(\mu) &= \mu^q+(1-\mu)^q, \\
  \phi_1(\mu) &= \mu^{q-1}-(1-\mu)^{q-1}, \\
  \phi_2(\mu) &= \mu^{q-2}+(1-\mu)^{q-2}, \\
  \phi_3(\mu) &= \mu^{q-3}-(1-\mu)^{q-3}.
\end{align*}
When $0<q<2$ with $q\ne1$,
\begin{align*}
  &\lim_{\mu\downarrow0}\frac{\phi'(\mu)}{\mu\phi''(\mu)}\left[1-\frac{\phi(\mu)}{\mu\phi'(\mu)}\right] \\
  &= \lim_{\mu\downarrow0}\frac{\frac{\phi_1(\mu)}{\phi_0(\mu)}}{(q-1)\mu\frac{\phi_2(\mu)}{\phi_0(\mu)}-q\mu\frac{\phi_1(\mu)^2}{\phi_0(\mu)^2}}\left[1-\frac{\frac{1}{q-1}\ln\phi_0(\mu)}{\frac{q}{q-1}\mu\frac{\phi_1(\mu)}{\phi_0(\mu)}}\right] \\
  &= \lim_{\mu\downarrow0}\frac{1}{(q-1)\frac{\mu\phi_2(\mu)}{\phi_1(\mu)}-q\frac{\mu\phi_1(\mu)}{\phi_0(\mu)}} \cdot \left[1-\frac{\phi_0(\mu)\ln\phi_0(\mu)}{q\mu\phi_1(\mu)}\right] \\
  &= \frac{1}{(q-1)\lim_{\mu\downarrow0}\frac{\mu\phi_2(\mu)}{\phi_1(\mu)}-q\lim_{\mu\downarrow0}\frac{\mu\phi_1(\mu)}{\phi_0(\mu)}} \cdot \left[1-\frac{\lim_{\mu\downarrow0}\phi_0(\mu)}{q}\cdot\lim_{\mu\downarrow0}\frac{\ln\phi_0(\mu)}{\mu\phi_1(\mu)}\right] \\
  &= \frac{1}{(q-1)\cdot1-q\cdot0}\cdot\left[1-\frac{1}{q}\cdot1\right] \\
  &= \frac1q,
\end{align*}
where we use
\[
  \phi_0(\mu) \to 1, \quad
  \frac{\mu\phi_2(\mu)}{\phi_1(\mu)} = \frac{1+\left(\frac{\mu}{1-\mu}\right)^{2-q}}{1-\left(\frac{\mu}{1-\mu}\right)^{2-q}} \to 1, \quad
  \mu\phi_1(\mu) = \mu^q-\frac{\mu}{(1-\mu)^{1-q}} \to 0,
\]
and
\begin{align*}
  \frac{\ln\phi_0(\mu)}{\mu\phi_1(\mu)}
  &\overset{(\ddagger)}\to \frac{1}{\phi_0(\mu)}\cdot\frac{\phi_0'(\mu)}{\mu\phi_1'(\mu)+\phi_1(\mu)} \\
  &\to \frac{\phi_0'(\mu)}{\mu\phi_1'(\mu)+\phi_1(\mu)} \\
  &= \frac{q\phi_1(\mu)}{(q-1)\mu\phi_2(\mu)+\phi_1(\mu)} \\
  &= \frac{q}{(q-1)\frac{\mu\phi_2(\mu)}{\phi_1(\mu)}+1} \\
  &\to \frac{q}{(q-1)\cdot1+1} \\
  &= 1.
\end{align*}
In addition, we have
\begin{align*}
  &\lim_{\mu\downarrow0}\frac{\mu}{[\mu\phi'(\mu)-\phi(\mu)]^{1/q}} \\
  &= \left\{\lim_{\mu\downarrow0}\frac{\mu^q}{\mu\phi'(\mu)-\phi(\mu)}\right\}^{1/q} \\
  &= \left\{\lim_{\mu\downarrow0}\frac{(q-1)\mu^q\phi_0(\mu)}{q\mu\phi_1(\mu)-\phi_0(\mu)\ln\phi_0(\mu)}\right\}^{1/q} \\
  &= \left\{\lim_{\mu\downarrow0}\frac{(q-1)\mu^q}{q\mu\phi_1(\mu)-\phi_0(\mu)\ln\phi_0(\mu)}\right\}^{1/q}
    \qquad\qquad (\phi_0(\mu)\to1) \\
  &\overset{(\ddagger)}= \left\{(q-1)\lim_{\mu\downarrow0}\frac{q\mu^{q-1}}{q\phi_1(\mu)+q(q-1)\mu\phi_2(\mu)-q\phi_1(\mu)\ln\phi_0(\mu)-q\phi_1(\mu)}\right\}^{1/q} \\
  &= \left\{(q-1)\lim_{\mu\downarrow0}\frac{\mu^{q-1}}{(q-1)\mu\phi_2(\mu)-\phi_1(\mu)\ln\phi_0(\mu)}\right\}^{1/q} \\
  &\overset{(\ddagger)}= \left\{(q-1)\lim_{\mu\downarrow0}\frac{(q-1)\mu^{q-2}}{(q-1)\phi_2(\mu)+(q-1)(q-2)\mu\phi_3(\mu)-\frac{q\phi_1(\mu)^2}{\phi_0(\mu)}-(q-1)\phi_2(\mu)\ln\phi_0(\mu)}\right\}^{1/q} \\
  &= \lim_{\mu\downarrow0}\!\bigg\{\!\frac{\big[1+\big(\frac{\mu}{1-\mu}\big)^{2-q}\big] \!\! + \! (q-2)\big[1-\big(\frac{\mu}{1-\mu}\big)^{3-q}\big] \!\! - \! \frac{q\phi_1(\mu)^2}{(q-1)\mu^{q-2}\phi_0(\mu)} \! - \! \big[1+\big(\frac{\mu}{1-\mu}\big)^{2-q}\big]\ln\phi_0(\mu)}{q-1}\!\bigg\}^{\!\!-\frac1q} \\
  &\overset{\text{(A)}}= \left\{\frac{1+(q-2)\cdot1-0-1\cdot0}{q-1}\right\}^{-1/q} \\
  &= 1,
\end{align*}
where at (A) we used
\begin{align*}
  \frac{\phi_1(\mu)^2}{\mu^{q-2}\phi_0(\mu)}
  &\to \mu^{2-q}\phi_1(\mu)^2 \\
  &= \mu^q-2(1-\mu)^{q-1}\mu+(1-\mu)^{2q-2}\mu^{2-q} \\
  &\to -2(1-\mu)^{q-1}\mu+(1-\mu)^{2q-2}\mu^{2-q} \\
  &= \frac{(1-\mu)^{2q-2}-2(1-\mu)^{q-1}\mu^{q-1}}{\mu^{q-2}} \\
  &\overset{(\ddagger)}\to \frac{(2q-2)(1-\mu)^{2q-3}+2(q-1)(1-\mu)^{q-2}\mu^{q-1}-2(q-1)(1-\mu)^{q-1}\mu^{q-2}}{(q-2)\mu^{q-3}} \\
  &= \frac{2(q-1)}{q-2}\cdot\frac{1}{(1-\mu)^{2-q}}\cdot\frac{(1-\mu)^{q-1}+\mu^{q-1}-(1-\mu)\mu^{q-2}}{\mu^{q-3}} \\
  &\to \frac{2(q-1)}{q-2}\cdot1\cdot\frac{(1-\mu)^{q-1}+\mu^{q-1}-(1-\mu)\mu^{q-2}}{\mu^{q-3}} \\
  &= \frac{2(q-1)}{q-2}\cdot\left[\left(\frac{\mu}{1-\mu}\right)^{1-q}\mu^2+\mu^2-(1-\mu)\mu\right] \\
  &\to 0.
\end{align*}

When $q=2$, we leverage
\[
  \phi_0(\mu)=2\mu^2-2\mu+1, \quad \phi_1(\mu)=2\mu-1, \quad \phi_2(\mu)=2, \quad
  \phi_0'(\mu)=2\phi_1(\mu), \quad \phi_1'(\mu)=2
\]
to have
\begin{align*}
  &\lim_{\mu\downarrow0}\frac{\phi'(\mu)}{\mu\phi''(\mu)}\left[1-\frac{\phi(\mu)}{\mu\phi'(\mu)}\right] \\
  &= \lim_{\mu\downarrow0}\phi_0(\mu)\cdot\frac{2\mu\phi_1(\mu)-\phi_0(\mu)\ln\phi_0(\mu)}{4\mu^2[\phi_0(\mu)-\phi_1(\mu)^2]} \\
  &= \lim_{\mu\downarrow0}\frac{2\mu\phi_1(\mu)-\phi_0(\mu)\ln\phi_0(\mu)}{4\mu^2[\phi_0(\mu)-\phi_1(\mu)^2]}
    && (\phi_0(\mu)\to1) \\
  &\overset{(\ddagger)}= \lim_{\mu\downarrow0}\frac{2[\phi_1(\mu)+2\mu]-2\phi_1(\mu)\ln\phi_0(\mu)-2\phi_1(\mu)}{4\{2\mu[\phi_0(\mu)-\phi_1(\mu)^2]+\mu^2[2\phi_1(\mu)-4\phi_1(\mu)]\}} \\
  &= \lim_{\mu\downarrow0}\frac{2\mu-\phi_1(\mu)\ln\phi_0(\mu)}{4\mu[\phi_0(\mu)-\phi_1(\mu)^2-\mu\phi_1(\mu)]} \\
  &\overset{(\ddagger)}= \lim_{\mu\downarrow0}\frac{2-2\ln\phi_0(\mu)-\frac{2\phi_1(\mu)^2}{\phi_0(\mu)}}{4[\phi_0(\mu)-\phi_1(\mu)^2-\mu\phi_1(\mu)]+4\mu[2\phi_1(\mu)-4\phi_1(\mu)-\phi_1(\mu)-2\mu]} \\
  &= \lim_{\mu\downarrow0}\frac{1-\ln\phi_0(\mu)-\frac{\phi_1(\mu)^2}{\phi_0(\mu)}}{2[\phi_0(\mu)-\phi_1(\mu)^2-4\mu\phi_1(\mu)-2\mu^2]} \\
  &\overset{(\ddagger)}= \lim_{\mu\downarrow0}\frac{-\frac{2\phi_1(\mu)}{\phi_0(\mu)}-\frac{4\phi_0(\mu)\phi_1(\mu)-2\phi_1(\mu)^3}{\phi_0(\mu)^2}}{2[2\phi_1(\mu)-4\phi_1(\mu)-4\phi_1(\mu)-8\mu-4\mu]} \\
  &= \lim_{\mu\downarrow0}\frac{\phi_1(\mu)}{\phi_0(\mu)^2}\cdot\frac{3\phi_0(\mu)-\phi_1(\mu)^2}{6[\phi_1(\mu)+2\mu]} \\
  &=\frac13.
\end{align*}
In addition, defining
\[
  \zeta\defeq\frac{2\mu\phi_1(\mu)-\phi_0(\mu)\ln\phi_0(\mu)}{\phi_0(\mu)},
\]
we have
\begin{align*}
  (\xi&\defeq) \lim_{\mu\downarrow0}\frac{\mu}{[\mu\phi'(\mu)-\phi(\mu)]^{1/3}} \\
  &= \lim_{\mu\downarrow0}\frac{\mu}{\left[\frac{2\mu\phi_1(\mu)-\phi_0(\mu)\ln\phi_0(\mu)}{\phi_0(\mu)}\right]^{1/3}}
    \quad \left(\text{implies~~} \xi=\lim_{\mu\downarrow0}\mu\zeta^{-1/3} \text{;~we will use this below at (\$)}\right) \\
  &\overset{(\ddagger)}= \lim_{\mu\downarrow0}\frac{3\zeta^{2/3}}{\frac{2\phi_1(\mu)}{\phi_0(\mu)}+2\mu\frac{2\phi_0(\mu)-2\phi_1(\mu)^2}{\phi_0(\mu)^2}-\frac{2\phi_1(\mu)}{\phi_0(\mu)}} \\
  &= \lim_{\mu\downarrow0}\frac{3\phi_0(\mu)^2}{4}\frac{\zeta^{2/3}}{\mu[\phi_0(\mu)-\phi_1(\mu)^2]} \\
  &\overset{(\ddagger)}= \lim_{\mu\downarrow0}\frac{3}{4}\frac{\frac23\cdot2\mu\frac{2\phi_0(\mu)-2\phi_1(\mu)^2}{\phi_0(\mu)^2}}{\left\{\phi_0(\mu)-\phi_1(\mu)^2+\mu[2\phi_1(\mu)-4\phi_1(\mu)]\right\}\zeta^{1/3}} \\
  &= \lim_{\mu\downarrow0}\frac{2\mu[\phi_0(\mu)-\phi_1(\mu)^2]}{[\phi_0(\mu)-\phi_1(\mu)^2-2\mu\phi_1(\mu)]\zeta^{1/3}} \\
  &\overset{(\ddagger)}= \lim_{\mu\downarrow0}\frac{3\zeta^{2/3}}{2}\frac{1}{\mu\phi_0(\mu)[\phi_0(\mu)-\phi_1(\mu)^2]-3\frac{\phi_0(\mu)^3[\phi_1(\mu)+\mu]\zeta}{\phi_0(\mu)-\phi_1(\mu)^2-2\mu\phi_1(\mu)}} \\
  &= \lim_{\mu\downarrow0}\frac{3\zeta^{2/3}}{2}\frac{1}{\mu[\phi_0(\mu)-\phi_1(\mu)^2]+\frac{3\zeta}{\phi_0(\mu)-\phi_1(\mu)^2-2\mu\phi_1(\mu)}} \\
  &= \left\{\lim_{\mu\downarrow0}\frac23\frac{\phi_0(\mu)-\phi_1(\mu)^2}{\mu}\cdot(\mu\zeta^{-1/3})^2 + \lim_{\mu\downarrow0}\frac{2\mu}{\phi_0(\mu)-\phi_1(\mu)^2-2\mu\phi_1(\mu)}\frac{1}{\mu\zeta^{-1/3}}\right\}^{-1} \\
  &\overset{\text{(\$)}}= \left\{\frac23\xi^2\lim_{\mu\downarrow0}(2-2\mu) + \frac1\xi\lim_{\mu\downarrow0}\frac{1}{2-3\mu}\right\}^{-1} \\
  &= \left\{\frac43\xi^2+\frac{1}{2\xi}\right\}^{-1},
\end{align*}
which implies
\[
  \xi = \frac{1}{\frac43\xi^2+\frac{1}{2\xi}}.
\]
By solving this, we have
\[
  \lim_{\mu\downarrow0}\frac{\mu}{[\mu\phi'(\mu)-\phi(\mu)]^{1/3}} = \xi = \left(\frac38\right)^{1/3}.
\]

\subsection{Pseudo-spherical entropy}
Consider the $q$-pseudo-spherical entropy $\phi(\mu)=[\mu^q+(1-\mu)^q]^{1/q}-1$ for $q>1$~\citep{Gneiting2007}. It is also known as the $q$-norm (neg)entropy~\citep{Boekee1980}.
When $q=2$, it recovers the spherical entropy associated with the spherical loss~\citep{Agarwal2014JMLR}.
When $q\uparrow\infty$, it approaches $\phi_\infty(\mu)=\max\set{\mu,1-\mu}-1$, which is the Bayes risk of the hinge/0-1 losses~\citep{Buja2005}.
As seen in \cref{figure:norm_ent}, the limit $\alpha$ (in \eqref{equation:exponent}) does not exist, which indicates that we cannot guarantee the $\epsilon$-optimal risk for vanishingly small $\epsilon$.

\begin{figure}
  \centering
  \includegraphics[width=0.75\textwidth]{figure/order_norm_ent}
  \caption{
    For the pseudo-spherical entropy, $\alpha(\mu)=[\phi'(\mu)/\mu\phi''(\mu)] \cdot [1-\phi(\mu)/\mu\phi'(\mu)]$ is shown.
  }
  \label{figure:norm_ent}
\end{figure}




\end{document}
