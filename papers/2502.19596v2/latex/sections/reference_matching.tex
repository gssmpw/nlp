\begin{figure*}[t!]
\centering
%\small
\renewcommand{\arraystretch}{1.15}
\resizebox{\textwidth}{!}{%
\begin{tabular}{cclc}
\toprule
\multicolumn{4}{c}{\textbf{Question:} Has there been an example where improving the vehicle structure prevented the fracture of components?} \\ \midrule
\multicolumn{1}{c|}{\textbf{Segment 1}} & \textbf{Sentence 1} & \multicolumn{1}{l|}{Yes, there have been cases where improving the vehicle structure prevented the fracture of components.} & \textbf{Chunk 1,2,3} \\ \midrule
\multicolumn{1}{c|}{\multirow{2}{*}{\textbf{Segment 2}}} & \textbf{Sentence 2} & \multicolumn{1}{l|}{First, in the 'HDC EV' project, the design of the front bumper beam (FRT B/Beam) was changed to prevent fracture.} & \multirow{2}{*}{\textbf{Chunk 2}} \\ \cline{2-3}
\multicolumn{1}{c|}{} & \textbf{Sentence 3} & \multicolumn{1}{l|}{\begin{tabular}[c]{@{}l@{}}In the T-CAR model, fracture occurred in the FRT B/Beam, but in the Proto model, it was modified to a closed \\ cross-section shape, preventing fracture.\end{tabular}} & \\ \midrule
\multicolumn{1}{c|}{\multirow{3}{*}{\textbf{Segment 3}}} & \textbf{Sentence 4} & \multicolumn{1}{l|}{\begin{tabular}[c]{@{}l@{}}Additionally, in the 'HGC' project, body structure improvements were made to reduce the risk of deformation \\ and fracture of several components.\end{tabular}} & \multirow{3}{*}{\textbf{Chunk 1}} \\ \cline{2-3}
\multicolumn{1}{c|}{} & \textbf{Sentence 5} & \multicolumn{1}{l|}{For example, reinforcing the upper part of the A-pillar and the COWL BAR improved bending deformation.} & \\ \cline{2-3}
\multicolumn{1}{c|}{} & \textbf{Sentence 6} & \multicolumn{1}{l|}{Adding UPR HINGE B/H beads and other deformation-guiding methods reduced opening misalignment.} & \\ \midrule
\multicolumn{1}{c|}{\textbf{Segment 4}} & \textbf{Sentence 7} & \multicolumn{1}{l|}{\begin{tabular}[c]{@{}l@{}}Finally, in the 'QYZ' project, a plan was proposed to improve the tendency of fracture by adding reinforcements \\ to the S/MBR RR lower end.\end{tabular}} & \textbf{Chunk 3} \\ \midrule
\multicolumn{1}{c|}{\textbf{Segment 5}} & \textbf{Sentence 8} & \multicolumn{1}{l|}{All of these improvements aim to optimize the body structure and prevent component fracture during collisions.} & \textbf{Chunk 1,2,3} \\ \bottomrule
\end{tabular}%
}
\caption{An example where multiple chunks need to be referred to in order to answer the question, in contrast to Figure \ref{fig:reference_case_1} (\ref{sec:appendix_ref}), which requires a single chunk.}
\vspace{-1em}
\label{fig:reference_case_2}
\end{figure*}


\subsection{Evaluation}
To evaluate the reference matching task, we randomly sample 100 triplets ($q, \mathbf{D}_{\text{top}_n}, a'$) from the test set and perform human annotation to identify which chunk ($d_i \in \mathbf{D}_{\text{top}_n}$) each sentence in the generated answer ($a'$) referenced. Each sentence is annotated with one or more chunks, allowing for multiple references. The interface of the annotation tool is in Figure \ref{fig:ref_annotation_tool} (\ref{sec:appendix_humaneval}). We use sentence-level precision as the evaluation metric, defined as the proportion of sentences where the model prediction is included in the annotated reference set for an answer, averaged over the entire evaluation set. Since references are identified for concatenated sentences, selecting the best match for the grouped information is key. Recall would unfairly penalize valid predictions for not selecting every reference.
% To assess the performance with varying numbers of reference chunks, we created two versions of the evaluation set by adjusting the number of top-ranked chunks ($D_{\text{top}_k}, k = [5, 10]$) passed to the generation model along with the $(q, a)$ pairs.

\subsection{Result}
\label{subsec:reference_matching_result}
Table \ref{tab:result_ref} compares the reference matching performance of our algorithm (Section \ref{sec:reference_matching_algorithm}) without thresholding and the fine-tuned Qwen 72B, while Figure \ref{fig:ref_threshold} illustrates the score distribution and the increase in precision with score thresholding for our algorithm. Without thresholding, our algorithm achieves a sentence-level precision of 0.72, while the LLM reaches 0.81, showing a noticeable gap in performance. However, when thresholding is applied, the precision reaches 0.86 at a threshold of 0.5, with performance improving proportionally as the threshold is increased further. This demonstrates the reliability of our algorithm's scoring, allowing for controlled adjustment of reference matching quality through threshold selection. Our re-ranker used in the matching algorithm has only 0.5B parameters, compared to the LLM's 72B, demonstrating that our algorithm can achieve impressive performance with a smaller model and faster inference speed. The LLM prompt used for reference matching is in Figure \ref{fig:prompt_ref} (\ref{subsubsec:ref_prompt}). The answers are divided into 2-3 segments on average, and we identify two types of reference matching: one where all the necessary information to answer the question is contained within a single chunk, resulting in a single-segment answer, and another where information from multiple chunks is needed to answer the question, leading to a multi-segment answer. The first type often occurs in factual questions related to specific vehicle crash collision test, while the second type is more common in open-ended questions that require referencing multiple tests (Figure \ref{fig:reference_case_2}).

\begin{table}[h!]
\centering
\tiny
\centering
\resizebox{\columnwidth}{!}{%
\begin{tabular}{c|rrr}
\toprule
\textbf{Method}                                              & \multicolumn{1}{c}{\textbf{Segment}}                       & \multicolumn{1}{c}{\textbf{\begin{tabular}[c]{@{}c@{}}Sentence-level \\ Precision\end{tabular}}} & \multicolumn{1}{c}{\textbf{\begin{tabular}[c]{@{}c@{}}Inference\\ Time\end{tabular}}} \\ \midrule
\begin{tabular}[c]{@{}c@{}}Matching\\ Algorithm\end{tabular} & \begin{tabular}[c]{@{}r@{}}2.7 \\ ($\pm$ 1.5)\end{tabular} & \begin{tabular}[c]{@{}r@{}}0.72\\ ($\pm$ 0.27)\end{tabular}                               & 3.92                                                                                      \\ \midrule
LLM                                                          & \begin{tabular}[c]{@{}r@{}}1.9 \\ ($\pm$ 0.9)\end{tabular} & \begin{tabular}[c]{@{}r@{}}0.81\\ ($\pm$ 0.26)\end{tabular}                               & 13.54                                                                                      \\ \bottomrule
\end{tabular}%
}
\caption{Performance comparison of our reference matching algorithm (without thresholding) and the fine-tuned Qwen 72B. \textbf{Segment} is the average number of segments per answer, \textbf{Sentence-level Precision} is the average proportion of successfully matched sentences, and \textbf{Inference Time} is the average time (in seconds) to complete reference matching for one answer.}
\label{tab:result_ref}
\vspace{-1em}
\end{table}

\begin{figure}[h!]
    \centering
    \includegraphics[width=1.0\linewidth]{latex/figures/ref_threshold_subplots.png}
    \caption{Distribution of matching scores from our reference matching algorithm (left) and the relationship between precision and the score threshold (right). Increasing the threshold leads to a proportional improvement in matching precision.}
    \label{fig:ref_threshold}
\vspace{-1em}
\end{figure}

% (1) 필요한 정보가 chunk 하나에 다 있는 경우
% - NE 북미(미국) P2 40kph 경사 30˚(RH) 시험에서 사용된 백빔 재질은 무엇이었나요?

% (2) 여러 chunk에서 필요한 정보를 참조해야하는 경우
% - SG2 북미(미국) P2 62kph 측면 27˚(LH) USMDB 시험의 주요 목적은 무엇이었습니까?
% - 차체 구조 개선을 통해 부품의 파단을 방지한 사례가 있었나요?
% - 안전성 평가에서 목표를 초과 달성한 사례가 있었나요?
% - 후석 승객의 왼쪽 복합력(Left Combined force)이 가장 높게 나타난 시험은 어떤 것이었나요?

