\vspace{-0.1cm}
\section{Introduction}
\vspace{-0.1cm}
\label{sec:intro}
Recent works like SwinUNet\cite{cao2022swin}, MambaUNet \cite{wang2024mamba} and MONAI \cite{cardoso2022monai} develop medical-tailored foundation models on large-scale medical image datasets. 
Intense interest has emerged in adapting these foundation models for specific medical image analysis tasks. 
%allowing for the development of generalizable models with limited training samples.
%Intense interest in applying these foundation models in specific medical image analysis tasks is widespread.
However, the generalization capability of foundation models is limited by the large variability in training data, due to complex modalities, intricate anatomical structures, and wide-range object scales in medical images.
Therefore, we seek to answer this critical question: how to effectively adapt these foundation models to our desired medical image processing tasks? 
% \section{Introduction}


\begin{figure}[t]
\centering
\includegraphics[width=0.6\columnwidth]{figures/evaluation_desiderata_V5.pdf}
\vspace{-0.5cm}
\caption{\systemName is a platform for conducting realistic evaluations of code LLMs, collecting human preferences of coding models with real users, real tasks, and in realistic environments, aimed at addressing the limitations of existing evaluations.
}
\label{fig:motivation}
\end{figure}

\begin{figure*}[t]
\centering
\includegraphics[width=\textwidth]{figures/system_design_v2.png}
\caption{We introduce \systemName, a VSCode extension to collect human preferences of code directly in a developer's IDE. \systemName enables developers to use code completions from various models. The system comprises a) the interface in the user's IDE which presents paired completions to users (left), b) a sampling strategy that picks model pairs to reduce latency (right, top), and c) a prompting scheme that allows diverse LLMs to perform code completions with high fidelity.
Users can select between the top completion (green box) using \texttt{tab} or the bottom completion (blue box) using \texttt{shift+tab}.}
\label{fig:overview}
\end{figure*}

As model capabilities improve, large language models (LLMs) are increasingly integrated into user environments and workflows.
For example, software developers code with AI in integrated developer environments (IDEs)~\citep{peng2023impact}, doctors rely on notes generated through ambient listening~\citep{oberst2024science}, and lawyers consider case evidence identified by electronic discovery systems~\citep{yang2024beyond}.
Increasing deployment of models in productivity tools demands evaluation that more closely reflects real-world circumstances~\citep{hutchinson2022evaluation, saxon2024benchmarks, kapoor2024ai}.
While newer benchmarks and live platforms incorporate human feedback to capture real-world usage, they almost exclusively focus on evaluating LLMs in chat conversations~\citep{zheng2023judging,dubois2023alpacafarm,chiang2024chatbot, kirk2024the}.
Model evaluation must move beyond chat-based interactions and into specialized user environments.



 

In this work, we focus on evaluating LLM-based coding assistants. 
Despite the popularity of these tools---millions of developers use Github Copilot~\citep{Copilot}---existing
evaluations of the coding capabilities of new models exhibit multiple limitations (Figure~\ref{fig:motivation}, bottom).
Traditional ML benchmarks evaluate LLM capabilities by measuring how well a model can complete static, interview-style coding tasks~\citep{chen2021evaluating,austin2021program,jain2024livecodebench, white2024livebench} and lack \emph{real users}. 
User studies recruit real users to evaluate the effectiveness of LLMs as coding assistants, but are often limited to simple programming tasks as opposed to \emph{real tasks}~\citep{vaithilingam2022expectation,ross2023programmer, mozannar2024realhumaneval}.
Recent efforts to collect human feedback such as Chatbot Arena~\citep{chiang2024chatbot} are still removed from a \emph{realistic environment}, resulting in users and data that deviate from typical software development processes.
We introduce \systemName to address these limitations (Figure~\ref{fig:motivation}, top), and we describe our three main contributions below.


\textbf{We deploy \systemName in-the-wild to collect human preferences on code.} 
\systemName is a Visual Studio Code extension, collecting preferences directly in a developer's IDE within their actual workflow (Figure~\ref{fig:overview}).
\systemName provides developers with code completions, akin to the type of support provided by Github Copilot~\citep{Copilot}. 
Over the past 3 months, \systemName has served over~\completions suggestions from 10 state-of-the-art LLMs, 
gathering \sampleCount~votes from \userCount~users.
To collect user preferences,
\systemName presents a novel interface that shows users paired code completions from two different LLMs, which are determined based on a sampling strategy that aims to 
mitigate latency while preserving coverage across model comparisons.
Additionally, we devise a prompting scheme that allows a diverse set of models to perform code completions with high fidelity.
See Section~\ref{sec:system} and Section~\ref{sec:deployment} for details about system design and deployment respectively.



\textbf{We construct a leaderboard of user preferences and find notable differences from existing static benchmarks and human preference leaderboards.}
In general, we observe that smaller models seem to overperform in static benchmarks compared to our leaderboard, while performance among larger models is mixed (Section~\ref{sec:leaderboard_calculation}).
We attribute these differences to the fact that \systemName is exposed to users and tasks that differ drastically from code evaluations in the past. 
Our data spans 103 programming languages and 24 natural languages as well as a variety of real-world applications and code structures, while static benchmarks tend to focus on a specific programming and natural language and task (e.g. coding competition problems).
Additionally, while all of \systemName interactions contain code contexts and the majority involve infilling tasks, a much smaller fraction of Chatbot Arena's coding tasks contain code context, with infilling tasks appearing even more rarely. 
We analyze our data in depth in Section~\ref{subsec:comparison}.



\textbf{We derive new insights into user preferences of code by analyzing \systemName's diverse and distinct data distribution.}
We compare user preferences across different stratifications of input data (e.g., common versus rare languages) and observe which affect observed preferences most (Section~\ref{sec:analysis}).
For example, while user preferences stay relatively consistent across various programming languages, they differ drastically between different task categories (e.g. frontend/backend versus algorithm design).
We also observe variations in user preference due to different features related to code structure 
(e.g., context length and completion patterns).
We open-source \systemName and release a curated subset of code contexts.
Altogether, our results highlight the necessity of model evaluation in realistic and domain-specific settings.






Unlike natural image analysis with large-scale labeled datasets, in medical image analysis, another major challenge is the lack of labeled data, as 
annotating disease-specific medical images is not only time-consuming but also demands %costly, 
specialty-oriented skills, leading to the problem of few-shot domain adaptation (FSDA). 
Most solutions to conventional domain adaptation problems either assume access to source data \cite{bermudez2018domain}, which is not always feasible in real-world medical scenarios with various regulatory standards and ethical considerations,
or they require a substantial amount of unlabeled target data to reduce the distribution gap across domains, as seen in unsupervised domain adaptation (UDA) methods  \cite{wu2021unsupervised}.
FSDA, on the other hand, addresses the situation when only a limited number of target examples are available for training, whether labeled or unlabeled.
Previous FSDA methods \cite{gu2019progressive} propose to use intermediate/auxiliary domains to facilitate domain adaptation.
However, this multi-step domain adaptation strategy requires fine-tuning the model twice or more.
In this work, we propose to incorporate auxiliary datasets to solve the FSDA problem in a source-free manner through a single-round fine-tuning.

Training with auxiliary data introduces an inductive bias that helps models capture meaningful representations and reduces the risk of overfitting to spurious correlations \cite{navon2021auxiliary}.
Multi-task learning methods \cite{graham2023one} cannot extend to a large number of tasks because the complexity of the search space will be exponentially explosive \cite{albalak2024improving}.
Other strategies in auxiliary learning and transfer learning hand-pick which auxiliary data to use based on heuristics \cite{yang2021joint} or metrics \cite{yu2020gradient} prior to training, sometimes resulting in sub-optimal outcomes.
Recent dynamic auxiliary learning works \cite{navon2021auxiliary} propose to dynamically combine auxiliary objectives through task or data schedulers, but these methods involve complex and computationally demanding bi-level optimization steps.

To address the above issues, we propose an \textbf{A}ctive and \textbf{S}equential domain \textbf{A}da\textbf{P}tation (ASAP) framework for FSDA.
Using a novel dynamic dataset selection strategy,
%By designing dynamic and efficient dataset selection algorithms, 
the proposed framework prioritizes training on auxiliary datasets with similar solution spaces to the target task in a \textbf{single-round} computational complexity.
Specifically,
we formulate FSDA as a multi-armed bandit problem in active learning \cite{macready1998bandit} and relate the set of auxiliary datasets to the arms.
We introduce the classic trace upper confidence bound algorithm \cite{auer2002finite} to solve the multi-armed bandit problem.
By balancing the trade-off between the exploration of unobserved arms and the exploitation of high-reward arms, we actively and sequentially select the auxiliary dataset at each turn, maximizing their benefits.
The reward functions we design add minimal memory and computational overhead.

Extensive experiments on three public medical datasets validate the effectiveness of our proposed ASAP framework.
We efficiently adapt pre-trained 
UNet \cite{ronneberger2015u}, SwinUNet \cite{cao2022swin} and MambaUNet \cite{wang2024mamba} from Flemme
\cite{zhang2024flemme}, 
a flexible and modular learning medical platform, for various target medical image segmentation tasks.
% a freely available open-source medical resource platform, for various target medical image segmentation tasks. 
%Our algorithms outperform direct fine-tuning by 32.34\% on MRI and 12.82\% on CT datasets in terms of segmentation Dice score. 
%When incorporating the collection of auxiliary datasets,
Our method outperforms the FSDA auxiliary learning methods with lower computation costs.
Our main contributions are as follows:
\begin{itemize}
    \item \textbf{\textit{An active and sequential domain adaptation framework}}: we propose a novel framework that incorporates auxiliary datasets to effectively adapt foundation models in a single-round fine-tuning for various medical segmentation tasks, optimizing the use of public medical resources.
    \item \textbf{\textit{An exploration-exploitation balanced  FSDA algorithm}}: we design an efficient reward function and successfully apply the multi-armed bandit algorithm to dynamic auxiliary dataset selection through the ASAP framework.
\end{itemize}




