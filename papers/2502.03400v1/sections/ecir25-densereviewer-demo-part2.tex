\begin{figure*}[t!]
\centering
\includegraphics[width=\columnwidth]{ranking_mode_2docs_clicked_highlighted_no_footnote}
\caption{Ranking mode. \circled{1}{Red} contains the PICO query. \circled{2}{Blue} lists the studies; users can use keyboard or mouse controls to expand a study to read and judge. Assessed studies are highlighted in purple. \circled{3}{Green} contains page controls, with a pause button to save the review's progress and toggles to stop upon reaching the last page. \circled{4}{Navy} shows two pie charts that display the ratio of reviewed to unreviewed studies and the distribution of judgements and a line chart that displays the relevance discovery curve, indicating the saturation of relevant studies throughout the screening progresses. \circled{5}{Yellow} allows users to enter focus mode (see Figure~\ref{fig:ui_full_screen_mode}).}
\label{fig:ui_ranking_mode}
\end{figure*}

\begin{figure*}[t!]
\centering
\includegraphics[width=\textwidth]{fullscreen_mode_highlighted}
\caption{Focus mode. \circled{1}{Navy} displays the study's title, list of authors, and PubMed ID. \circled{2}{Red}~contains the full abstract. \circled{4}{Blue} includes the three assessment options, a ranking index of the current study on the page, and the page number, positioned centrally, to the left, and right, respectively. Buttons \circled{5}{Yellow} and \circled{6}{Yellow} allow users to navigate between studies.}
\label{fig:ui_full_screen_mode}
\end{figure*}


\section{Overview of DenseReviewer}
DenseReviewer offers two modes: ranking mode and focus mode.
Figure~\ref{fig:ui_ranking_mode} provides an overview of ranking mode, allowing users to browse and make assessments on studies from the perspective of a paginated, ranked list. Figure~\ref{fig:ui_full_screen_mode} provides an overview of focus mode, allowing users to review each study individually and efficiently. In this mode, keyboard controls can be used to quickly move between studies and make assessments. 

Users can upload their corpus, retrieved from PubMed in \texttt{nbib} format, and submit a structured PICO query~\cite{scells2017integrating}. Uploaded studies must include the pmid (PubMed identifier), title, abstract and a list of authors. A dense retriever ranks uploaded studies with respect to the PICO query. After assessing each page of studies, the remaining unjudged studies are re-ranked based on Rocchio's algorithm with positive (include) and negative (exclude) feedback~\cite{mao2024dense}. DenseReviewer is open source, and all its components, including the front-end, API back-end, and database, are packaged within Docker containers. The containers can be downloaded from a single git repository and self-hosted with ease. Aside from the GUI-based system for screening, we provide a Python library%
\footnote{\url{https://github.com/ielab/dense-screening-feedback}} 
to enable experimentation using existing datasets~\cite{wang2022little,scells2017test,kanoulas2017clef,kanoulas2018clef,kanoulas2019clef}, and training or loading dense retrievers with customised backbone models.