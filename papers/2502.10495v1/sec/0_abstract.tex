\begin{abstract}
In the rapidly evolving landscape of image generation, Latent Diffusion Models (LDMs) have emerged as powerful tools, enabling the creation of highly realistic images. However, this advancement raises significant concerns regarding copyright infringement and the potential misuse of generated content. Current watermarking techniques employed in LDMs often embed constant signals to the generated images that compromise their stealthiness, making them vulnerable to detection by malicious attackers. In this paper, we introduce SWA-LDM, a novel approach that enhances watermarking by randomizing the embedding process, effectively eliminating detectable patterns while preserving image quality and robustness. Our proposed watermark presence attack reveals the inherent vulnerabilities of existing latent-based watermarking methods, demonstrating how easily these can be exposed.
Through comprehensive experiments, we validate that SWA-LDM not only fortifies watermark stealthiness but also maintains competitive performance in watermark robustness and visual fidelity. This work represents a pivotal step towards securing LDM-generated images against unauthorized use, ensuring both copyright protection and content integrity in an era where digital image authenticity is paramount.
\end{abstract}
