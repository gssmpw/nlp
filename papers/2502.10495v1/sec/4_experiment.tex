%-------------------------------------------------------------------------------
\section{Experiment}
%-------------------------------------------------------------------------------

\subsection{Setup}
\label{sec:experiment_setup}
\noindent
\textbf{Latent Diffusion Models.} 
We employed three widely-used Stable Diffusion models as base models: Stable Diffusion v1-5 (SD v1-5), Stable Diffusion v2-1 (SD v2-1), and SD-XL 1.0-base (SDXL 1.0). 
For customized models, we downloaded 60 checkpoints from Hugging Face \cite{huggingface}, fine-tuned from three base models (SD v1-5, SD v2-1, and SDXL 1.0), with each base model comprising 20 different checkpoints. Detailed on these 60 checkpoints is provided in the Appendix.
Compared to previous work, our study covers the largest model set to date (60 models, \vs Tree-ring~\cite{wen2023tree} with 1, Gaussian Shading~\cite{yang2024gaussian} with 3, and DiffuseTrace~\cite{lei2024diffusetrace} with 2).

\noindent
\textbf{Image Generation Details.}  
To generate images, we use prompts from the Stable-Diffusion-Prompts dataset~\cite{Gustavosta}. The generated image resolution is 512×512 pixels, with latent noise dimensions set to 4×64×64 and a guidance scale of 7.5. We use DDIM sampling ~\cite{song2020denoising} with 50 timesteps. In practice, the original prompts of the generated images are often not shared. Hence, we use an empty prompt for diffusion inversion~\cite{dimm_inversion}. In this process, we set the guidance scale to 1 and perform 50 timesteps of DDIM inversion.

% \vspace{3pt}
\noindent
\textbf{Baselines.} 
We evaluate three representative latent-noise-based watermarking methods: Tree-ring~\cite{wen2023tree}, Gaussian Shading~\cite{yang2024gaussian}, and DiffuseTrace~\cite{lei2024diffusetrace}. 
For Gaussian Shading, we test both implementations, with and without the ChaCha20~\cite{bernstein2008chacha} secure stream cipher, which shuffles the watermark sequence. Detail of these methods are in the Appendix.

\noindent
\textbf{Evaluation Metrics.} To evaluate watermark presence attacks, we use the area under the ROC curve (AUC). Attack results on watermarking methods indicate stealthiness, calculated as (1 - \text{AUC of watermark presence attack}). We benchmark watermark effectiveness by reporting AUC and TPR at 1\% FPR (noted as TPR@1\%FPR) and bit accuracy for encoded information. For watermarked image quality, we use the CLIP score~\cite{radford2021learning} between generated images and prompts, measured using OpenCLIP-ViT/G~\cite{Cherti_2023_CVPR} and the Fréchet Inception Distance (FID)~\cite{NIPS2017_heuselgans}. FID, which evaluates feature similarity between generated and original images, is calculated from 5,000 images per base model generated using the MS-COCO-2017 dataset\cite{lin2014microsoft}.


\noindent
\textbf{Setup of Watermark Presence Attack.}  
The attacker generates 1,000 clean images using three base models (SD v1-5, SD v2-1, SDXL 1.0) and evaluates performance by averaging results across models. For the watermark feature extractor, we use a 12-layer CNN with convolutional and fully connected layers, ReLU activations, and layer normalization, outputting a 100-dimensional feature vector. Training uses SGD optimizer with a learning rate of 0.01, momentum of 0.9, and a scheduler with a 0.5 decay factor every 50 steps. Detailed architecture is in the Appendix.

\noindent
\textbf{Setup of \tool{}.} 
We integrate SWA-LDM with three baseline methods: \tool with Tree-Ring (\tool{}(T-R)), \tool with DiffuseTrace (\tool{}(D-T)), and \tool with Gaussian Shading (\tool{}(G-S)). Each method uses a key channel count of 1 to construct an 8-bit key with 64 redundant bits. The number of watermark channels is set to 1 for \tool{}(T-R) and 3 for both \tool{}(D-T) and \tool{}(G-S).

\begin{table*}[t]
\vspace{-0.3cm}
\centering
\caption{
Comparison of \tool and baselines. The watermark effectiveness is evaluated with AUC, TPR@1\%FPR, and bit accuracy. The quality of the generated images is assessed using FID and CLIP scores. The stealthiness represents the failure rate of the proposed watermark presence attacks. Left to right are LDMs fine-tuned from SD v1-5/SD v2-1/SDXL 1.0.}
% \vspace{-0.3cm}
\label{tab:base_performance}
\resizebox{\textwidth}{!}{%
\begin{tabular}{@{}cccccccc@{}}
\toprule
\multirow{2}{*}{\textbf{Methods}} & \multirow{2}{*}{\begin{tabular}[c]{@{}c@{}}\textbf{Nonce}\\\textbf{Management}\end{tabular}} & \multicolumn{6}{c}{\textbf{Metrics}} \\ \cmidrule(l){3-8} 
 &  & \textbf{AUC} & \textbf{TPR@1\%FPR} & \textbf{Bit Acc.} & \textbf{FID $\downarrow$} & \textbf{CLIP-Score $\uparrow$} & \textbf{Stealthiness $\uparrow$} \\ \midrule
No watermark & \XSolidBrush & - & - & - & 29.77/27.01/75.83 & 0.324/0.291/0.304 & - \\ \midrule
Tree-Ring
% ~\cite{wen2023tree}
& \XSolidBrush & 0.999/0.999/0.999 & 0.987/0.996/0.998 & - & 30.53/28.32/78.97 & 0.325/0.296/0.305 & 0.208/0.212/0.227 \\
DiffuseTrace
% ~\cite{lei2024diffusetrace}
& \XSolidBrush & 0.999/0.983/0.840 & 0.989/0.944/0.434 & 0.978/0.951/0.692 & 30.15/26.83/83.68 & 0.324/0.296/0.302 & 0.204/0.218/0.296 \\
Gaussian Shading
% ~\cite{yang2024gaussian} 
& \XSolidBrush & 1.000/1.000/1.000 & 1.000/1.000/1.000 & 0.999/0.999/0.999 & 31.58/29.82/70.39 & 0.325/0.297/0.305 & 0.005/0.019/0.084 \\
$\text{G-S}_{ChaCha20}$
% ~\cite{yang2024gaussian}
& \Checkmark & 1.000/1.000/1.000 & 1.000/1.000/1.000 & 0.999/0.999/0.999 & 29.69/27.21/75.83 & 0.324/0.297/0.304 & 0.427/0.505/0.478 \\ \midrule
\tool(T-R) & \XSolidBrush & 0.999/0.997/0.996 & 0.999/0.991/0.993 & - & 30.24/27.43/70.21 & 0.324/0.297/0.305 & 0.475/0.495/0.474 \\
\tool(D-T) & \XSolidBrush & 0.999/0.978/0.810 & 0.983/0.942/0.354 & 0.974/0.945/0.666 & 29.80/26.90/76.89 & 0.323/0.295/0.301 & 0.496/0.497/0.504 \\
\tool(G-S) & \XSolidBrush & 0.999/0.997/0.998 & 0.999/0.995/0.998 & 0.999/0.997/0.998 & 30.53/27.28/75.29 & 0.324/0.297/0.304 & 0.469/0.513/0.508 \\ \bottomrule
\end{tabular}
}
\vspace{-0.3cm}
\end{table*}
\subsection{Comparison to Baseline Methods}
\label{sec:comparison_baselines}

\noindent
\textbf{Stealthiness Comparison.}
% \label{sec:stealthiness_comparison}
We conduct watermark presence attack experiments across \tool and baselines. The attack performance, summarized in the "Stealthiness" column of \cref{tab:base_performance}, shows the average stealthiness achieved by each method against an attacker using different base models. 

The results show that watermark presence attacks effectively detect watermarks in baseline methods. \tool improves stealthiness and provides defense against these attacks. Among baseline methods, Gaussian Shading is the most detectable, with the lowest stealthiness, while DiffuseTrace and Tree-Ring offer slight improvements but remain vulnerable. Gaussian Shading with ChaCha20 increases stealthiness but requires costly per-image nonce management. In contrast, \tool achieves ChaCha20-level stealthiness without nonce dependency, integrating smoothly with DiffuseTrace, Tree-Ring, and Gaussian Shading. Further analysis on the base model’s impact on detection is in \cref{sec:ablation_studies}.


\noindent
\textbf{Watermarking Effectiveness Comparison.}
For the evaluation of watermark effectiveness, As detailed in \cref{sec:experiment_setup}, each base model (SD v1-5, SD v2-1, SDXL 1.0) is fine-tuned to produce 20 checkpoints, each generating 1,000 images, resulting in 60,000 watermarked and 60,000 clean images per method. As shown in \cref{tab:base_performance}, \tool maintains AUC, TPR@1\%FPR, and bit accuracy comparable to original methods, with slight metric decreases due to key construction from latent noise for enhanced stealthiness. \tool also has minimal impact on FID and CLIP scores, preserving LDM-generated image quality.


\section{Loss Robustness}
\label{sec:robustness}

% We extend the concept of label noise to the autoregressive language modeling domain, focusing on asymmetric or class-conditional noise. Specifically, at each step $t$, the label $\xbm_t$ in the training data of the black-box model is flipped to  $\tilde \xbm_t \in V$ with probability $p^*(\tilde \xbm_t|\xbm_t)$, while the feature vectors or preceding tokens $(\xbm_{t-1:1})$ remain unchanged. Consequently, the black-box model observes samples from a noisy distribution given by  $p^*(\tilde \xbm_t, \xbm_{t-1:1}) = \sum_{\xbm_t}p^*(\tilde \xbm_t | \xbm_t)p^*(\xbm_t|\xbm_{t-1:1})p^*(\xbm_{t-1:1})$.

% Denote by $T_t  \in [0, 1]^{|V|\times |V|}$, the noise transition matrix at step $t$ specifying the probability of one label being flipped to another, so that $\forall i, j \;\; T_{t_{ij}}=p^*(\tilde \xbm_t = \ebm^j | \xbm_t = \ebm^i)$. The matrix is row-stochastic and not necessarily symmetric across the classes. 

% To address asymmetric label noise, we modify the loss $\bm{\ell}$ to ensure robustness. Initially, assuming the noise transition matrix $T_t$ is known, we apply a loss correction inspired by prior work~\citep{patrini2017making, sukhbaatar2015training}. Subsequently, we relax this assumption and estimate $T_t$ directly, forming the foundation of our plugin model approach.

We extend label noise modeling to the autoregressive language setting, focusing on asymmetric or class-conditional noise. At each step $t$, the label $\xbm_t$ in the black-box model’s training data is flipped to $\tilde \xbm_t \in V$ with probability $p^*(\tilde \xbm_t|\xbm_t)$, while preceding tokens $(\xbm_{t-1:1})$ remain unchanged. As a result, the black-box model observes samples from a noisy distribution: $p^*(\tilde \xbm_t, \xbm_{t-1:1}) = \sum_{\xbm_t} p^*(\tilde \xbm_t | \xbm_t) p^*(\xbm_t|\xbm_{t-1:1}) p^*(\xbm_{t-1:1}).$

We define the noise transition matrix $T_t \in [0,1]^{|V|\times |V|}$ at step $t$, where each entry $T_{t_{ij}} = p^*(\tilde \xbm_t = \ebm^j | \xbm_t = \ebm^i)$ represents the probability of label flipping. This matrix is row-stochastic but not necessarily symmetric.

To handle asymmetric label noise, we modify the loss $\bm{\ell}$ for robustness. Initially, assuming a known $T_t$, we apply a loss correction inspired by~\citep{patrini2017making, sukhbaatar2015training}. We then relax this assumption by estimating $T_t$ directly, forming the basis of our \textit{Plugin} model approach.

We observe that a language model trained with no loss correction would result in a predictor for noisy labels $b(\tilde \xbm_t | \xbm_{t-1:1})$. We can make explicit the dependence on $T_t$. For example, with cross-entropy we have:

\begin{align*}
&\ell(\ebm^i, b(\tilde \xbm_t | \xbm_{t-1:1})) = -\log b(\tilde\xbm_t = \ebm^i | \xbm_{t-1:1}) \\
&= -\log \sum_{j=1}^{|V|} p^*(\tilde\xbm_t = \ebm^i | \xbm_t = \ebm^j) b(\xbm_t = \ebm^j | \xbm_{t-1:1}) \\ 
&= -\log \sum_{j=1}^{|V|} T_{t_{ji}} {b}(\xbm_t = \ebm^j | \xbm_{t-1:1}), \numberthis
\label{eq:fc}
\end{align*}
or in matrix form
\begin{equation}
    \label{eq:fc-mat}
    \bm{\ell}(b(\tilde \xbm_t|\xbm_{t-1:1})) = -\log T_t^\top b(\xbm_t|\xbm_{t-1:1}).
\end{equation}

% This loss compares the noisy label $\tilde \xbm_t$ to the noisy predictions averaged using the transition matrix $T_t$ at step $t$. Note that the cross-entropy loss is commonly employed for next-token prediction tasks. Cross-entropy is a \emph{proper composite loss} with the softmax function as its \emph{inverse link function}~\citep{patrini2017making}. Consequently, from Theorem 2 of~\citep{patrini2017making}, the minimizer of the \emph{forwardly-corrected} loss in Equation~\eqref{eq:fc-mat} for noisy data corresponds to the minimizer of the actual loss for clean data. Formally, this can be expressed as:

% \begin{align*}
%     \label{eq:loss-minimizers}
%     & \argmin_{w} E^*_{\tilde \xbm_t,\xbm_{t-1:1}}\Big[\bm{\ell}(\xbm_t, T_t^T b(\xbm_t|\xbm_{t-1:1})) \Big] \\ &= 
%     \argmin_{w} E^*_{\xbm_t,\xbm_{t-1:1}}\Big[\bm{\ell}(\xbm_t, b(\xbm_t|\xbm_{t-1:1})) \Big],
% \end{align*}
% where $w$ are the weights of the language model, and their dependence is implicitly embedded in the definition of the softmax output $b$ from the black-box language model. This result indicates that if the transition matrix $T_t$ were known, we could transform the softmax output $b(\bm{x}_t \mid \bm{x}_{t-1:1})$ using $T_t^T$, use the transformed predictions as the final outputs, and re-train the black-box model accordingly with the corrected loss. However, the transition matrix $T_t$ is not known a priori, and we do not have access to the training data. Thus, estimating $T_t$ from clean data becomes a crucial step in our approach.

This loss compares the noisy label $\tilde \xbm_t$ to the noisy predictions averaged via the transition matrix $T_t$ at step $t$. Cross-entropy loss, commonly used for next-token prediction, is a \emph{proper composite loss} with the softmax function as its \emph{inverse link function}~\citep{patrini2017making}. Consequently, from Theorem 2 of~\citet{patrini2017making}, the minimizer of the \emph{forwardly-corrected} loss in Equation~\eqref{eq:fc-mat} on noisy data aligns with the minimizer of the true loss on clean data, i.e., 
\begin{align*}
    \label{eq:loss-minimizers}
    & \argmin_{w} E^*_{\tilde \xbm_t,\xbm_{t-1:1}}\Big[\bm{\ell}(\xbm_t, T_t^\top b(\xbm_t|\xbm_{t-1:1})) \Big] \\ &= 
    \argmin_{w} E^*_{\xbm_t,\xbm_{t-1:1}}\Big[\bm{\ell}(\xbm_t, b(\xbm_t|\xbm_{t-1:1})) \Big],
\end{align*}
where $w$ are the language model’s weights, implicitly embedded in the softmax output $b$ from the black-box model. This result suggests that if $T_t$ were known, we could transform the softmax output $b(\xbm_t \mid \xbm_{t-1:1})$ using $T_t^T$, use the transformed predictions as final outputs, and retrain the model accordingly. However, since $T_t$ is unknown and training data is inaccessible, estimating $T_t$ from clean data is essential to our approach.


\subsection{Estimation of Transition Matrix}
\label{ssec:estimatingT}

% In our problem setup, we assume access to a small amount of clean language data for the task. Under the assumption that the black-box model is expressive enough to model $p^*(\tilde{\bm{x}}_t \mid \bm{x}_{t-1:1})$ (Assumption (2) in Theorem 3 of~\citep{patrini2017making}), the transition matrix $T_t$ can be estimated using this clean data. Considering the supervised classification problem at step $t$, let $\mathcal{X}_t^i$ denote all samples in the clean data where $\bm{x}_t = \bm{e}^i$ and the preceding tokens are $(\bm{x}_{t-1}, \dots, \bm{x}_1)$. A naive estimate of the transition matrix can be computed as follows:

We assume access to a small amount of target language data for the task. Given that the black-box model is expressive enough to approximate $p^*(\tilde{\xbm}_t \mid \xbm_{t-1:1})$ (Assumption (2) in Theorem 3 of~\citet{patrini2017making}), the transition matrix $T_t$ can be estimated from this target data. Considering the supervised classification setting at step $t$, let $\mathcal{X}_t^i$ represent all target data samples where $\xbm_t = \ebm^i$ and the preceding tokens are $(\xbm_{t-1:1})$. A naive estimate of the transition matrix is: $\hat T_{t_{ij}}=b(\tilde \xbm_t = \ebm^j|\xbm_t=\ebm^i)=\frac{1}{|\mathcal{X}_t^i|}\sum_{x\in\mathcal{X}_t^i}b(\tilde \xbm_t = \ebm^j|\xbm_{t-1:1})$.


While this setup works for a single step $t$, there are two key challenges in extending it across all steps in the token prediction task:

\begin{enumerate}[leftmargin=0.4cm]
    \item \textbf{Limited sample availability:} The number of samples where $\bm{x}_t = \bm{e}^i$ and the preceding tokens $(\bm{x}_{t-1}, \dots, \bm{x}_1)$ match exactly is limited in the clean data, especially with large vocabulary sizes (e.g., $|V| = O(100K)$ for LLaMA~\citep{dubey2024llama}). This necessitates modeling the transition matrix as a function of features derived from $\bm{x}_{t-1:1}$, akin to text-based autoregressive models.
    \item \textbf{Large parameter space:} With a vocabulary size of $|V| = O(100K)$, the transition matrix $T_t$ at step $t$ contains approximately 10 billion parameters. This scale may exceed the size of the closed-source LLM and cannot be effectively learned from limited target data. Therefore, structural restrictions must be imposed on $T_t$ to reduce its complexity.
\end{enumerate}

To address these challenges, we impose the restriction that the transition matrix $T_t$ is diagonal. While various constraints could be applied to simplify the problem, assuming $T_t$ is diagonal offers two key advantages. First, it allows the transition matrix—effectively a vector in this case—to be modeled using standard autoregressive language models, such as a \emph{GPT-2 model with $k$ transformer blocks}, a \emph{LLaMA model with $d$-dimensional embeddings}, or a fine-tuned \emph{GPT-2-small} model. These architectures can be adjusted based on the size of the target data. Second, a diagonal transition matrix corresponds to a symmetric or class-independent label noise setup, where $\xbm_t = \ebm^i$ flips to any other class with equal probability in the training data. This assumption, while simplifying, remains realistic within the framework of label noise models.

By enforcing this diagonal structure, we ensure efficient estimation of the transition matrix while maintaining practical applicability within our framework. In the next section, we outline our approach for adapting closed-source language models to target data.













\noindent
\textbf{Benchmarking Watermark Robustness}
\label{sec:robustness_exp}
To evaluate the robustness of \tool, we assess its performance under seven common image perturbations as potential attacks: JPEG compression, random crop, random drop, resize and restore(Resize), Gaussian blur (GauBlur), median filter (MedFilter), brightness adjustments. The parameter ranges are shown in the Appendix.
For each parameter setting of every perturbation, we used 2,000 images generated by the SD v1-5 to evaluate performance. The average verification AUC for each perturbation is reported in \cref{table:robustness}, which compares the robustness of various watermarking methods, both with and without the integration of \tool.
Results indicate that \tool maintains robust watermark verification under moderate image perturbations, demonstrating its robustness. However, incorporating \tool impacts the original robustness of these watermarking methods, especially under high-intensity distortions. This occurs because \tool requires complete recovery of each bit in the key to retrieve the watermark, which can reduce robustness. Nevertheless, unless the image undergoes quality-compromising levels of perturbation, watermark remains practical. 

\subsection{Ablation Studies}
\label{sec:ablation_studies}


\begin{table}[t]
\vspace{-0.1cm}
\caption{Impact of different clean SD models on the watermark detection attacks. Left to right are target LDMs fine-tuned from SD v1-5/SD v2-1/SDXL 1.0.}
% \vspace{-0.3cm}
\label{tab:ab_sd_version}
\resizebox{\columnwidth}{!}{%
\begin{tabular}{@{}cccc@{}}
\toprule
\multirow{2}{*}{\textbf{Methods}} & \multicolumn{3}{c}{\textbf{SD version used to generate the clean images}} \\ \cmidrule(l){2-4} 
 & SD v1-5 & SD v2-1 & SD-XL v1.0 \\ \midrule
Tree-ring & 0.240/0.255/0.275 & 0.163/0.158/0.255 & 0.223/0.223/0.153 \\
DiffuseTrace & 0.183/0.229/0.303 & 0.207/0.213/0.303 & 0.223/0.213/0.284 \\
Gaussian Shading & 0.010/0.015/0.100 & 0.000/0.030/0.085 & 0.005/0.012/0.068 \\
$\text{G-S}_{ChaCha20}$ & 0.445/0.546/0.500 & 0.383/0.400/0.435 & 0.453/0.570/0.500 \\ \midrule
\tool (T-R) & 0.481/0.518/0.485 & 0.478/0.498/0.468 & 0.465/0.470/0.470 \\
\tool (D-T) & 0.478/0.528/0.520 & 0.491/0.484/0.463 & 0.520/0.479/0.530 \\
\tool (G-S) & 0.438/0.475/0.528 & 0.500/0.515/0.495 & 0.468/0.548/0.503 \\ \bottomrule
\end{tabular}%
}
\vspace{-0.3cm}
\end{table}


\noindent
\textbf{Impact of the clean SD model on watermark presence attack.} 
We evaluated whether the effectiveness of the watermark presence attack is influenced by the base model used by the attacker to generate clean images. Results shown in \cref{tab:ab_sd_version} indicate that the choice of base model has minimal impact on attack performance, demonstrating that the watermark detection attack remains effective without requiring knowledge related to the target model.


\begin{table}[t]
\vspace{-0.5cm}
\caption{Impact of image quantities on the watermark presence attacks. Results are shown as stealthiness. 
% The target checkpoints are fine-tuned from SD v1-5/SD v2-1/SDXL 1.0.
}
% \vspace{-0.3cm}
\label{tab:ab_clean_img_num}
\resizebox{\columnwidth}{!}{%
\begin{tabular}{@{}ccccc@{}}
\toprule
\multirow{2}{*}{\textbf{Methods}} & \multicolumn{4}{c}{\textbf{Clean Image Quantity}} \\ \cmidrule(l){2-5} 
 & 500 & 1,000 & 1,500 & 2,000 \\ \midrule
Tree-ring & 0.256/0.331/0.194 & 0.208/0.212/0.227 & 0.212/0.172/0.214 & 0.224/0.172/0.186 \\
DiffuseTrace & 0.220/0.208/0.325 & 0.204/0.218/0.296 & 0.244/0.204/0.288 & 0.221/0.214/0.300 \\
Gaussian Shading & 0.039/0.014/0.081 & 0.005/0.019/0.084 & 0.011/0.004/0.071 & 0.049/0.008/0.076 \\
$\text{G-S}_{ChaCha20}$ & 0.423/0.468/0.438 & 0.427/0.505/0.478 & 0.438/0.486/0.533 & 0.423/0.478/0.546 \\ \midrule
\tool (T-R) & 0.440/0.459/0.521 & 0.475/0.495/0.474 & 0.413/0.480/0.515 & 0.509/0.483/0.454 \\
\tool (D-T) & 0.491/0.530/0.470 & 0.496/0.497/0.504 & 0.525/0.475/0.500 & 0.498/0.516/0.479 \\
\tool (G-S) & 0.428/0.480/0.527 & 0.469/0.513/0.508 & 0.410/0.485/0.520 & 0.500/0.456/0.528 \\ \bottomrule
\end{tabular}%
}
\vspace{-0.2cm}
\end{table}

\noindent
\textbf{Impact of image quantity on watermark presence attack.} 
Following the setup in \cref{sec:experiment_setup}, we varied the number of images generated by the watermark presence attacker to assess its effect on attack performance. Results shown in \cref{tab:ab_clean_img_num}, indicate that within our sampled range, the watermark presence attack's effectiveness remains stable regardless of image quantity. 

\begin{figure}[t]
    \centering
    \begin{minipage}{\linewidth}
        \centering
        \includegraphics[width=0.8\linewidth]{Figure/ablation_studies/legend_image.png} % 调整图例图片的宽度
        \vspace{-0.3cm} 
    \end{minipage}
    
    \resizebox{\columnwidth}{!}{%
    \subfloat[\tool{}(G-S).]{\label{fig:redundancy_G-S}\includegraphics[width=0.35\linewidth]{Figure/ablation_studies/key_redundancy_G-S.png}}\hspace{-0.015\linewidth}
     \subfloat[\tool{}(T-R).]{\label{fig:redundancy_T-R}\includegraphics[width=0.35\linewidth]{Figure/ablation_studies/key_redundancy_T-R.png}}\hspace{-0.015\linewidth}
     \subfloat[\tool{}(D-T).]{\label{fig:redundancy_D-T}\includegraphics[width=0.35\linewidth]{Figure/ablation_studies/key_redundancy_D-T.png}}
     }
     \vspace{-0.3cm}
    \caption{Performance of \tool with varying numbers of redundancies. The effectiveness is demonstrated through AUC and stealthiness metrics, where (a) compares \tool(G-S) with Gaussian Shading and (b) compares \tool(T-R) with Tree-Ring. (c) compares \tool(D-T) with DiffuseTrace.}
    \label{fig:redundancy}
    \vspace{-0.3cm}
\end{figure}

\noindent
\textbf{Impact of key redundancy on stealthiness and verification performance.} 
Following the setup in Section~\ref{sec:experiment_setup}, we evaluate how varying key redundancy levels affects watermark stealthiness and verification AUC. Results in \cref{fig:redundancy} show that with minimal redundancy (4 redundancies), \tool achieves a verification AUC around 0.8, compared to a near-perfect verification AUC of 1 for watermarking methods without \tool, indicating an 80\% key recovery success rate. As redundancy increases to 8, the recovery probability improves to 90\%, and with redundancy over 40, \tool achieves near-complete key recovery without compromising verification AUC.  Across all redundancy levels, \tool maintains consistently high stealthiness.
