
%\include{Part 1}
%\include{Intro_LG24}

\chapter{Frobenius Manifolds}
\begin{center}
\includegraphics[scale=0.4]{LOVE.PNG}
  \end{center}

\section{A short story of Frobenius manifolds}


A Frobenius manifold $\cM$ is a geometric realization of a certain class of partial nonlinear differential equations of degree $3$. These equations, commonly referred to as the WDVV equations—named after Witten, Dijkgraaf, Verlinde, and Verlinde—are also known in some mathematical contexts as the Associativity Equations.

\,

There is a subtle distinction between Frobenius manifolds and the WDVV equations. While the notion of Frobenius manifolds represents a geometric object, the WDVV equations belong to the realm of analysis. The terminology “Frobenius manifolds” and “Associativity equations” appears predominantly in mathematical literature, whereas “WDVV equations” is more frequently encountered in discussions closer to physics.

\,

These perspectives, however, are deeply interwoven. As shown by Yu. I. Manin \cite{Man99} and B. Dubrovin \cite{Du}, they ultimately coincide.

\,

In the context of physics, solutions of the WDVV equations encode the moduli spaces of topological conformal field theories. These solutions are pivotal in the formulation of mirror symmetry for Calabi--Yau $3$-folds. Notably, specific solutions to the WDVV equations with particular properties serve as generating functions for the Gromov--Witten invariants of Kähler or symplectic manifolds.

\,
\subsection{Homological Mirror Symmetry (and more!)}
In the 1990s, the concept of Frobenius manifolds gained traction among both algebraic and analytic geometers due to the discovery of three significant classes, all stemming from the profound interplay between mathematics and physics. This development led to the creation of advanced mathematical tools and generated an abundance of challenging and intriguing problems. Some of these arose naturally from attempts to provide a rigorous mathematical interpretation of mirror symmetry (e.g., Homological Mirror Symmetry). Others emerged from the ambition to axiomatize and deepen the mathematical understanding of topological quantum field theory.

\,

The first class of Frobenius manifolds originated in the study of singularities. Specifically, Saito spaces—predating the explicit definition of Frobenius manifolds—were associated with the unfolding spaces of singularities. A particularly straightforward example is the space of complex polynomials of a fixed degree $d$ with distinct roots, which can also be identified with the configuration space of $d$ marked points on the complex plane.

\,

Subsequent classes discovered in the 1990s have a formal nature. Examples include the (formal) moduli spaces of solutions to Maurer–Cartan equations modulo gauge equivalence and the formal completions of cohomology spaces of smooth projective (or compact symplectic) manifolds. The latter is commonly referred to as quantum cohomology.

\,

More recently, a new class of Frobenius manifolds has emerged, demonstrating the versatility of this structure. It has been shown that under certain conditions, the manifold of probability distributions can generate a Frobenius manifold (this is the Combe–Manin construction). This discovery inspires many open questions, particularly regarding the complete classification of Frobenius manifolds, the identification of novel classes, and the elucidation of the intricate relationships between the existing ones.

\,

In the first section of this chapter, we provide a concise survey of Frobenius manifolds. We introduce the definitions of Frobenius, pre-Frobenius, and potential pre-Frobenius manifolds and briefly outline the WDVV equations. We also touch upon the hidden algebraic structures intrinsic to Frobenius manifolds. This section concludes with a brief summary.

\,

In the second section, we delve more deeply into the theoretical framework of Frobenius manifolds.

\,

In the third section, we present a collection of exercises and open problems for the reader to explore.

\section{Preliminary Notions: Affine Structures}

An essential ingredient for the definition of Frobenius manifolds is the concept of affine flat structures. For simplicity, we shall often refer to these as {\it affine structures}.

\,

\subsection{}
Let $\cM$ be a smooth manifold of dimension $\cM$. Affine flat structures can be defined in multiple equivalent ways.

\,
\subsubsection{}
An affine structure on an $n$-dimensional manifold $\cM$ is defined via a collection of coordinate charts $\{(U_\alpha, \phi_\alpha)\}$, where $\{U_\alpha\}$ forms an open cover of $\cM$, and $\phi_\alpha: U_\alpha \to \mathbb{R}^n$ is a local coordinate system such that the transition functions $\phi_\beta \circ \phi_\alpha^{-1}$ are affine transformations on $\phi_\alpha(U_\alpha \cap U_\beta)$, mapping it to $\phi_\beta(U_\alpha \cap U_\beta)$.

\,

Recall that the group of affine transformations is given by:
\[
\left\{
\begin{pmatrix}
A & b \\ 
0 & 1
\end{pmatrix}
\; \bigg\vert \; A \in GL(n,\mathbb{R}), \; b \in \mathbb{R}^n
\right\}.
\]

\,

\begin{definition}
An affine manifold is a smooth manifold equipped with an affine structure.
\end{definition}

\,
\subsubsection{}
The existence of an affine flat structure on $\cM$ is equivalent to the presence of a specific class of connections on the tangent bundle of $\cM$. Namely, there is a bijective correspondence between affine flat structures and flat, torsion-free affine connections $\nabla$ on $\cM$.

\,

In the context of differential geometry, a manifold over $\mathbb{K}$ with an affine structure is characterized by a tangent bundle whose underlying $G$-structure corresponds to the group of affine transformations $Aff(n) = GL(n, \mathbb{K}) \rtimes \mathbb{K}^n$, where $GL(n, \mathbb{K})$ is the general linear group over the field $\mathbb{K}$.

\,
\subsubsection{}
Alternatively, an affine flat structure can be described by a subsheaf $\cT_\cM^f \subset \cT_\cM$ of linear spaces of pairwise commuting vector fields. Locally, one has the tensor product over the ground field:
\[
\cT_\cM = \mathcal{O}_M \otimes \cT_\cM^f.
\]
Sections of $\cT_\cM^f$ correspond to flat vector fields. Furthermore, the metric $g$ is compatible with the structure $\cT_\cM^f$ if $g(X, Y)$ is constant for all flat vector fields $X$ and $Y$ (Exercises $12.1$ and $12.2$).

\,
\subsection{Crystallographic groups and fundamental groups}
A group $\Lambda$ is called an $n$-crystallographic group if it contains a normal, torsion-free, maximal abelian subgroup of rank $n$ and finite index. Crystallographic groups fit into the short exact sequence:
\[
0 \to V \to \Lambda \to P \to 1,
\]
where $V$ is a complex vector space, and $P \leq GL(n, \mathbb{Z}) \cong \mathrm{Aut}(V)$ is a finite group acting faithfully on $V$.

\,

A complex crystallographic group is a discrete subgroup $\Lambda \subset \mathrm{Iso}(\mathbb{C}^n)$ such that $\mathbb{C}^n / \Lambda$ is compact, where $\mathrm{Iso}(\mathbb{C}^n)$ denotes the group of biholomorphisms preserving the standard Hermitian metric.

\,

We note the following property:
\begin{lemma}
The fundamental group of a compact, complete flat affine manifold is an affine crystallographic group.
\end{lemma}

\,

Among the crystallographic groups, one encounters the Bieberbach groups, which are torsion-free crystallographic groups.
\

\subsection{Pre-Lie structures}
This affine structure leads to interesting algebraic properties on the tangent bundle/tangent sheaf. Let $\Gamma(TM)$ denote the space of vector fields on a given manifold $\cM$.

\,

According to the previous chapters, a flat and torsionless connection satisfied the following. 
An affine connection $\nabla$ is {\it torsion-free} (or torsionless) if
\begin{equation}\label{E:1}
\nabla_X (Y) - \nabla_Y (X) - [X, Y] = 0.
\end{equation}
The connection is called {\it flat} if
\begin{equation}\label{E:2}
\nabla_X (Y) - \nabla_Y (X) - \nabla_{[X, Y]} = 0.
\end{equation}

Such a connection defines a covariant differentiation for vector fields $X, Y \in \Gamma(TM)$:
\[
\nabla_X: \Gamma(TM) \longrightarrow \Gamma(TM), \quad 
\nabla_X(Y) \longmapsto \nabla_X(Y).
\]

\,

\begin{ex}\label{Ex:preLie}
A pre-Lie algebra is an algebra satisfying the relation 
\[
a(bc) - b(ac) = (ab)c - (ba)c.
\]    
Show that one can recover the structure of a pre-Lie algebra on the space of vector fields on $\cM$, by putting  $X \circ Y := \nabla_X(Y)$ for an affine flat, torsionless connection $\nabla$.
\end{ex}


\,


\begin{example}
Let $V$ be a finite-dimensional real Euclidean space endowed with a real inner product, and let $C$ be a closed convex cone $C \subset V $ with its vertex at the origin. The polar of the cone is defined by
\[
C^* = \{ y \in V \; | \; \langle x, y \rangle > 0, \; \forall x \in C \}.
\]
A symmetric cone satisfies $C^* = C $.

Such symmetric cones carry an affine flat structure. The tangent sheaf of this cone is equipped with the structure of a pre-Lie algebra, defined by the operation $X \circ Y := \nabla_X(Y) $, where $X, Y $ are vector fields on $C$.
\end{example}

\subsection{2-Dimensional Affine Structures}

\,

Let us consider two well-known structures in topology: the torus and the Klein bottle. These topological objects are the only compact $2$-dimensional manifolds that admit Euclidean structures.

\,

Assume $\cM$ is a closed manifold different from the torus or Klein bottle. Then there exists {\it no} affine structure on it. This is a direct consequence of the following result:

\begin{theorem}[Benzecri, 1955]
A closed surface admits affine structures if and only if its Euler characteristic vanishes.
\end{theorem}

\subsection{$n$-dimensional affine structures, $n \geq 3$}

When dealing with manifolds of dimension greater than two, there is no definitive criterion for determining the existence of an affine structure.

\,

In particular, according to Smillie's theorem, a closed manifold does \textit{not admit} an affine structure if its fundamental group is built up out of finite groups by taking free products, direct products, and finite extensions. Specifically, a connected sum of closed manifolds with finite fundamental groups admits \textit{no affine structure}. This result provides a profound insight into the interplay between algebraic and geometric properties in the study of such manifolds.

\,

Certain Seifert fiber spaces also do not possess affine structures. This is clarified by the following statement:

\begin{proposition}[Y. Carri\`ere, F. Dal’bo, G. Meigniez]
Let $\cM$ be a Seifert fiber space with vanishing first Betti number. Then, $\cM$ does not admit any affine structure.
\end{proposition}

\subsection{Complex cases}
We discuss the complex case. Due to works of Kobayashi \cite{Ko1}, a classification has been established. 
\begin{theorem}
In the complex case, one has the following collection of compact complex manifolds that admit affine structures.  
\begin{enumerate}
    \item the complex tori,
    \item the hyperelliptic surfaces,
    \item the minimal elliptic surfaces with odd $b_1,p_g>0$ and $c_1^2=0$,
    \item the minimal surfaces with $b_1=1$, $b_2=0$ and $p_g=0$ with the exception of the Hopf surfaces which are covered by primary Hopf surfaces satisfying certain specific properties. 
\end{enumerate}
\end{theorem}

However, it is interesting to note that there is {\it no other} compact complex surface admitting even holomorphic affine connections.


\section{Pre-Frobenius manifolds}
The pre-Frobenius manifolds exhibit interesting relations to the Monge--Ampère domains. Before we outline such relations, we recall the definition of such manifolds, using the tools that have been previously introduced.  

\subsection{Affine structure and metric}
Let us consider an affine flat structure on a manifold $\cM$. To define such a structure, several ingredients are required:
\begin{itemize}
    \item An atlas with transition functions that are affine and linear (in the affine case).
    \item A metric $g$ that is compatible with the affine flat structure.
    \item A symmetric tensor of rank $3$, denoted as:
    \[
    A: S^3(\cT_\cM) \longrightarrow \mathbb{R}.
    \]
    
  
\end{itemize}


\subsection{Multiplication Operation } 
We define a multiplication operation $\circ$ on the tangent sheaf $\cT_\cM$.


 Define a bilinear symmetric multiplication $\circ = \circ_{A, g}$ on the tangent sheaf $\cT_\cM$ as follows:
\[
\cT_\cM \otimes \cT_\cM \longrightarrow S^2(\cT_\cM) \xrightarrow{\ \mathcal{A}' \ } {\cT}^* \xrightarrow{\ g' \ } \cT_\cM,
\]
such that:
\[
X \otimes Y \longrightarrow X \circ Y,
\]
where the prime denotes partial dualization. 


\subsection{Compatibility Relations}
A compatibility relation between the rank-$3$ tensor $A$, the rank-$2$ tensor $g$, and the multiplication operation $\circ$ is given by:
    \[
    A(X, Y, Z) = g(X \circ Y, Z) = g(X, Y \circ Z).
    \]

This invariance of the metric with respect to multiplication ensures that the structure is well-defined.

\, 

\begin{definition}
    A pre-Frobenius manifold is a manifold $\cM$ equipped with the above  properties.
\end{definition}

Certain additional requirements on the algebraic structure of the tangent sheaf $(\cT_\cM, \circ)$ lead to having a \emph{Frobenius manifold.} 
\, 

There are however two important axioms to keep under consideration and that we shall express below. 

\subsection{Potential pre-Frobenius manifolds}

An important axiom to have is the one of potentiality. Namely, this axiom requires the existence of a family of local potentials $\Phi$, such that:
\[
g(X \circ Y, Z) = g(X, Y \circ Z) = (XYZ) \Phi.
\]
This axiom is particularly important regarding the relations to the Monge--Ampère equations.


If $ \mathscr{D}$ is a strictly convex bounded subset of $\mathbb{R}^n$ then for any nonnegative function $f$ on $\mathscr{D}$ and continuous $\tilde{g}:\partial  \mathscr{D} \to \mathbb{R}^n$ there is a unique convex smooth function $\Phi\in C^{\infty}( \mathscr{D})$ such that 
\begin{equation}\label{E:EMA}
\det \mathrm{Hess}(\Phi)= f, 
\end{equation} in $D$ and $\Phi=\tilde{g}$ on $\partial \mathscr{D}$.

\, 

An elliptic Monge--Ampère equation domain refers to the geometric data generated by $(\mathscr{D}, \Phi)$, where \begin{itemize}
    \item $\mathscr{D}$ is a strictly convex domain 
    \item $\Phi$ a real convex smooth function (with arbitrary and smooth boundary values of $\Phi$)  
    \end{itemize}
    such that Eq.~\eqref{E:EMA} is satisfied.  


We state the following result: 
\begin{theorem}
  A potential pre-Frobenius manifold satisfies everywhere locally the Monge--Amp\`ere equation. In other words, a potential pre-Frobenius manifold can be identified with an (elliptic) Monge--Amp\`ere domain.
\end{theorem}

\begin{ex}\label{Ex:Ma}
    Make a proof of the statement above.
\end{ex}

\subsection{Associative pre-Frobenius manifolds}
An associative pre-Frobenius manifold is a pre-Frobenius manifold such that there exists an associativity property \[(X \circ Y)\circ Z=X\circ(Y\circ Z),\]
where $X,Y,Z$ are vector fields. We will discuss this axiom fully in the context of Frobenius manifolds, below.  A Frobenius manifold is a pre-Frobenius manifold where the axioms of potentiality and associativity both hold. 
\section{Frobenius Manifolds}
\subsection{Witten-Dijkgraaf-Verlinde-Verlinde equation}

To derive the Witten-Dijkgraaf-Verlinde-Verlinde (WDVV) equation, let us rewrite the associativity of the multiplication $\circ$:
\[
(\partial_a \circ \partial_b) \circ \partial_c = \partial_a \circ (\partial_b \circ \partial_c).
\]
This yields a non-linear system of associativity equations, which are partial differential equations for the potential $\Phi$. For all $a, b, c, d$, these equations are written as:
\[
\sum_{ef} \Phi_{abe} g^{ef} \Phi_{fcd} = \sum_{ef} \Phi_{bce} g^{ef} \Phi_{fad}.
\]

These equations are highly non-linear and of third order.


\subsection{Geometrization}


Let $\cM$ be a manifold. A Frobenius algebra $(\mathcal{A}, \circ)$ over a field $\mathbb{K}$ is a commutative, associative, and unital algebra with a multiplication operation $\circ$ equipped with a symmetric bilinear form $\langle -, - \rangle$ satisfying:
\[
\langle x \circ y, z \rangle = \langle x, y \circ z \rangle,
\]
for all $x, y, z \in \mathcal{A}$.

\begin{definition}
A Frobenius manifold is an associative potential pre-Frobenius manifold.
\end{definition}

\, 

A manifold $\cM$ admits the structure of a Frobenius manifold if:
\begin{itemize}
    \item At any point of $\cM$, the tangent space has the structure of a Frobenius algebra $\mathcal{A}$.
    \item The invariant inner product $\langle -, - \rangle$ defines a flat metric on $\cM$.
    \item The unity vector field satisfies $\nabla e = 0$.
    \item The tensor of rank 4, $(\nabla_W A)(X, Y, Z)$, is fully symmetric, where $X,Y,Z,W$ are vector fields.
    \item A vector field $E$, called the Euler field, exists on $\cM$ such that $\nabla(\nabla E) = 0$.
\end{itemize}

The Euler field $E$ belongs to the class of affine vector fields. Its existence is inherently tied to the affine structure on $\cM$. This can be observed through the equivalence of the following statements:
\begin{itemize}
    \item[(1)] $E$ is an affine vector field.
    \item[(2)] $\nabla(\nabla E) = 0$.
    \item[(3)] For all vector fields $Y, Z$ on $\cM$:
    \[
    \nabla_Y(\nabla_Z E) = \nabla_{\nabla_Y Z} E.
    \]
    \item[(4)] The coefficients of $E$ are affine functions. Writing $E = \sum_m E^m \partial_m$, we have:
    \[
    E^m = a^m_j x^j + b^m,
    \]
    where $a^m_j$ and $b^m$ are constants in $\mathbb{R}$.
\end{itemize}

Thus, we propose a more concise definition of Frobenius manifolds based on Frobenius bundles. This new approach offers a practical and geometrical perspective.



%The proof of this theorem can be found in Chapter $12$.

\begin{remark}
If we choose local flat coordinates $(x^a)$ and the corresponding local basis of tangent fields $\partial_a$, then:
\[
(\partial_a \circ \partial_b \circ \partial_c) \Phi = \partial_a \partial_b \partial_c \Phi,
\]
and the compatibility of $\Phi$ and $g$ implies:
\[
\partial_a \circ \partial_b = \sum \Phi_{ab}^c \partial_c,
\]
where
\[
\Phi_{ab}^c := \sum (\partial_a \partial_b \partial_c \Phi) g^{ec}.
\]
Here,
\[
g_{ab} := (g_{ab})^{-1},
\]
with $g_{ab}$ interpreted as the inverse metric tensor.
\end{remark}

\subsection{Structure Connections of Pre-Frobenius manifolds}
Consider a pre-Frobenius manifold given by a triple $(\cM, g, A)$. Define the following geometrical objects:
\begin{itemize}
    \item A connection:
    \[
    \nabla_0: \cT_\cM \longrightarrow \Omega_M^1 \otimes \cT_\cM,
    \]
    where $\nabla_0$ is determined by the condition that flat fields are $\nabla_0$-horizontal.
    \item A pencil of connections depending on a parameter $\lambda$:
    \[
    \nabla_{\lambda, X}(Y) := \nabla_{0, X}(Y) + \lambda (X \circ Y).
    \]
    This is called the \textit{structure connection} of $(\cM, g, A)$.
\end{itemize}


\begin{theorem}[Manin]\label{T:Manin}
Let $(M, g, A)$ be a pre-Frobenius manifold. 
Let $\nabla_{\lambda}$ be the structure connection of a pre-Frobenius manifold $(\cM, g, A)$. $(\cM, g, A)$ is a Frobenius manifold if and only if the pencil $\{\nabla_\lambda\}$ is flat.
\end{theorem}

\begin{ex}\label{Ex:ManinProof}
Prove the Theorem \ref{T:Manin}.
\end{ex}

\section{Emergence of Hidden Structures for Statistical Manifolds}

We give a glimpse of how in statistical manifolds, it is possible to unravel the structures of a Frobenius algebra on the tangent space. This is only a short explanation that will be further developed in the next chapters, and serves only as a guide giving and intuition behind the construction.

\, 

Let $\bar{T} = T \cdot g^{-1}$ denote the mixed $(1, 2)$ tensor of third rank (that is the 1 contravariant and 2 covariant tensor). In components, this is defined by:
\[
\bar{T}_{ij}^k = \sum_m g^{km} T_{ijm}.
\]
where $g$ is the metric tensor, compatible with the affine connection on the manifold under consideration. 

We illustrate the construction of the operation $\circ$, defined on $\cT_\sfS$, where $\sfS$ is a statistical manifold of exponential type and of finite dimension. 

Generally:
\[
\bar{T}_{ij}^k = \bar{T} \big|_{P_{\theta}} (\partial_i \ell_{\theta}, \partial_j \ell_{\theta}, a^k) = \mathbb{E}_{P_{\theta}} [\partial_i \ell_{\theta} \partial_j \ell_{\theta} \partial_k \ell_{\theta} a_{\theta}^k].
\]
where $\{a^i\}$ form a dual basis to $\{\partial_j\ell_{\theta}\}$.
The hidden multiplication structure is explained by the following theorem:

\begin{theorem}
The tensor $\bar{T}$ defines a multiplication $\circ$ on $\cT_{P_{\theta}} \sfS$, as follows:
\[
\bar{T}: \cT_{P_{\theta}}\sfS \times \cT_{P_{\theta}}\sfS \longrightarrow \cT_{P_{\theta}}\sfS,
\]
and for $u, v \in \cT_{P_{\theta}}\sfS$:
\[
u \circ v = \bar{T}(u, v).
\]
\end{theorem}

The following lemma aids in understanding this hidden multiplication structure:

\begin{lemma}
For $u, v, w \in \cT_{P_{\theta}}\sfS$:
\[
g(u \circ v, w) = g(u, v \circ w).
\]
\end{lemma}


\subsection{Summary of Key structures} 

We sumarize the key elements existing for pre-Frobenius manifolds, in the example of statistical manifolds. This is as follows:
\begin{itemize}
    \item An affine flat structure. This structure exists for statistical manifolds of exponential type (see Exercise $12.3$).
    \item A triple $(\sfS, g, T)$, where $\sfS$ is a finite-dimensional statistical manifold:
    \begin{itemize}
        \item $g$ is a Riemannian (Fisher) metric,
        \item $T$ is the rank-3 symmetric Amari–Chentsov tensor,
        \item A multiplication defined by:
        \[
        u \circ \nu = \bar{T}(u, \nu),
        \]
        where locally, 
        \[
        \bar{T}_{ij}^k = \sum g^{km} T_{ijm},
        \]
            \end{itemize}
        \item Metric invariance:
        \[
        g(u, \nu \circ w) = g(u \circ \nu, w), \text{for flat vector fields } u,v,w.
        \]
        \item The flatness condition, which requires that the affine space of connections $\nabla_{\lambda}$ is flat.

\end{itemize}


\section{Semisimple Frobenius Manifolds}
We take as our point of departure the classical notion of a semisimple Frobenius manifold, formulated within the established framework. Semisimple Frobenius manifolds are interesting in relation to configuration spaces and Saito spaces. 

\,

This preliminary exposition, serves as a scaffolding upon which a more intrinsic and geometrically transparent reformulation will be erected. The latter will emerge naturally from a careful reconsideration of the underlying structures, unveiling with clarity the interplay between the metric, the associative multiplication, and the coherence conditions that bind them into a unified whole.


\subsection{}
 
Let $(\cM, g, A)$ be a triple, where $\cM$ is an associative pre-Frobenius manifold of dimension $n$. We introduce the following definition, which will play a fundamental role in the structure theory of such manifolds:

\begin{definition}
The manifold $\cM$ is said to be \emph{semisimple} (or \emph{split semisimple}) if there exists an isomorphism of sheaves of $\mathcal{O}_M$-algebras
\[
(\cT_\cM, \circ) \xrightarrow{\sim} (\mathcal{O}_M^n, \cdot).
\]
Here, $\circ$ denotes the multiplication on $\cT_\cM$, while $\cdot$ represents componentwise multiplication in $\mathcal{O}_M^n$. The isomorphism is required to exist everywhere locally (or globally).
\end{definition}

Let $(e_1, e_2, \dots, e_n)$ be a local basis of $\cT_\cM$. In this local basis, the multiplication takes the form:
\[
\left( \sum f_i e_i \right) \circ \left( \sum g_j e_j \right) = \sum f_i g_i e_i.
\]
In the simplest case, this reduces to
\[
e_i \circ e_j = \delta_{ij} e_i.
\]
Thus, the basis $(e_i)$ provides a well-defined (up to renumbering) family of idempotents. If $\cM$ is semisimple, there exists an unramified covering of $\cM$ (of degree at most $n!$) on which the induced pre-Frobenius structure becomes a splitting structure.

\, 

\subsection{}
Let $(e_i)$ be a local coordinate basis and let $(\epsilon^i)$ be its dual basis i.e. $1$-forms. 
The structure tensor $A$, encoding the pre-Frobenius structure, is given by the condition:
\[
A(X, Y, Z) = g(X \circ Y, Z) = g(X, Y \circ Z),
\]
where $g$ is the flat metric and 
$\circ$ denotes the associative product on $\cT_\cM$.
Since the basis vectors satisfy $e_i \circ e_j = \delta_{ij} e_i$, it follows, by a direct application of the above identity, that
\[
g(e_i, e_j) = g(e_i \circ e_i, e_j) = g(e_i, e_i \circ e_j) = \delta_{ij} g(e_i, e_i).
\]
We introduced the notation $\eta_i$, for the diagonal components of the metric,
so that $\eta_i=g(e_i, e_i)$. 

With this notation, the three-tensor $A$ takes the form:
\[
A = \sum_{i=1}^n \eta_i (\epsilon^i)^3,
\]
exhibiting its  diagonalizability in a chosen coordinate system.

Finally, considering the identity element $e = \sum_{i=1}^n e_i$ in $(\cT_\cM, \circ)$, we obtain the corresponding co-identity:
\[
\mathcal{E} = \sum_{i=1}^n \eta_i \epsilon^i.
\]

As a consequence of the above discussion, we are able to reformulate our initial definition differently. 
\subsection{Definition: semisimple Frobenius structure}
\begin{definition}
A \emph{semisimple Frobenius structure} on a smooth manifold  $\cM$ consists of the following data:
\begin{itemize}
    \item[1.] A reduction of the structure group of the tangent sheaf $\cT_\cM$;
    \item[2.] A flat metric $g$, diagonal in a distinguished  basis $(e_i)$ and its dual $(\epsilon^i)$;
    \item[3.] A diagonal cubic tensor $A$, sharing the same coefficients as $g$.
\end{itemize}
\end{definition}
\begin{remark} It is interesting to note that while the conditions of potentiality and flatness imposed on 
$g$ are of a non-trivial nature,  the associativity of the product structure on $\cT_\cM$ follows automatically under these hypotheses.
\end{remark}

\subsection{}
We are now in a position to articulate a fundamental characterization of Frobenius structures within this formalism. This characterization, which encapsulates the essential interplay between the multiplication, the metric, and their compatibility conditions, will serve as a guiding principle in the subsequent development of the theory.

\begin{theorem}
\label{thm:3.3}
The semisimple pre-Frobenius structure on $\cM$ defines a Frobenius structure if and only if:
\begin{itemize}
    \item The vector fields $e_i = \frac{\partial}{\partial u^i}$, in a system of canonical coordinates $\{u^i\}$, satisfy $[e_i, e_j] = 0$. We have $\epsilon^i = du^i$ in this system of canonical coordinates $\{u^i\}$;
    \item The functions $\eta_i$ satisfy the relation
$\eta_i = e_i \eta$, for some local function $\eta$ uniquely determined up to the addition of a constant. Equivalently, the form $\mathcal{E}$ is closed.
\end{itemize}
\end{theorem}

The function $\eta$ is referred to as the \textit{potential metric} of the structure. This metric corresponds to a Hessian metric of the form:
\[
g_{ij} = \frac{\partial^2 \Phi}{\partial u^i \partial u^j},
\]
where $\Phi$ is the potential. Note that the canonical coordinates $\{u^i\}$ are defined up to renumbering and constant shifts.

\begin{proof}

Consider the structure connection of a pre-Frobenius manifold, $\nabla_\lambda$. According to the previous theorem (cf. Theorem\ref{T:Manin}), the manifold $\cM$ is Frobenius if and only if the curvature $\nabla_\lambda^2$ vanishes. This is equivalent to the satisfying the following expression:
\begin{equation}\label{E:*}
    [\nabla_{\lambda, e_i}, \nabla_{\lambda, e_j}](e_k) = \nabla_{\lambda, [e_i, e_j]}(e_k).
\end{equation}


Since $\cM$ is assumed to be associative and $g$ is flat, we only need to consider the $\lambda$-linear terms in the above equation \ref{E:*}. Let $\{e_i\}$ be a basis, and let $\Gamma_{ik}^j$ denote the coefficients of the Riemannian connection:
\begin{equation}\label{E:coeff}
    \nabla_{0, e_i}(e_k) = \sum_j \Gamma_{ik}^j e_j.
\end{equation}


Since the structure connections are given by $\nabla_{\lambda, X}(Y) = \nabla_{0, X}(Y) + \lambda X \circ Y$, the left-hand side of $\ref{E:*}$ produces the $\lambda$-term:
\begin{equation}\label{E:**}
    \big(\nabla_{0, e_i} + \lambda e_i \circ\big)\big(\nabla_{0, e_j} + \lambda e_j \circ\big)(e_k) - \{i \leftrightarrow j\}.
\end{equation}
Inputting Eq. \ref{E:coeff} in Formula \ref{E:**}, we get that the $\lambda$-term reduces to:
\begin{equation}\label{E:3.8}
    \lambda \sum_q \big(\delta_{iq} \Gamma_{jk}^q + \delta_{jk} \Gamma_{ik}^q - \delta_{jk} \Gamma_{ik}^q - \delta_{ik} \Gamma_{jk}^q\big)e_q + \dots
\end{equation}


Now consider the Lie bracket $[e_i, e_j] = \sum\limits_q f_{ij}^q e_q$. The $\lambda$-term in the right-hand side of \ref{E:*} becomes:

\[
\nabla_{\lambda, [e_i, e_j]}(e_k) = \lambda \sum_q f_{ij}^q (e_q \circ e_k) + \dots.
\]

The coefficients of $e_k$ vanish in Eq.\ref{E:3.8}. If $\cM$ carries a Frobenius structure then the equality in \ref{E:*} holds so that $f_{ij}^k=0$.

Thus, the basis elements $e_i$ pairwise commute, and local canonical coordinates $\{u^i\}$ exist.

For the Levi-Civita connection of the metric $g = \sum g_{ij} \, du^i \, du^j$, the connection coefficients are given by:
\[
\Gamma_{ij}^k = \sum_l \Gamma_{ijl} g^{lk},
\]
where:
\[
\Gamma_{ijk} = \frac{1}{2}\big(e_i g_{jk} - e_k g_{ij} + e_j g_{ki}\big).
\]

For the metric $g = \sum \eta_i (du^i)^2$, the non-vanishing coefficients are:
\[
\Gamma_{ii}^i = \frac{1}{2} \eta_i^{-1} e_i \eta_i, \quad \Gamma_{ij}^i = \Gamma_{ji}^i = \frac{1}{2} \eta_i^{-1} e_j \eta_i \quad (i \neq j).
\]


Therefore,
\[
\nabla_i(e_i)= \frac{1}{2}\eta_i^{-1}e_i \eta_i  \cdot  e_i - \Sigma_{i \neq j}\frac{1}{2} \eta_j^{-1} e_j \eta_i \cdot e_j,
\]
and 
\begin{equation}\label{E:****}
    \nabla_i(e_j)= \frac{1}{2}\eta_i^{-1}e_j \eta_i  \cdot  e_i + \frac{1}{2} \eta_j^{-1} e_i \eta_j \cdot e_j,
\end{equation}

Finally, the vanishing of the $\lambda$-terms implies the following fundamental identity, valid for all indices  $i, j, k$:
\begin{equation}\label{E:*6}
e_i \circ \nabla_j(e_k) + \nabla_i(e_j \circ e_k) = (i \leftrightarrow j).
\end{equation}

Applying directly Eq.\ref{E:****} one observes that that Equation~\ref{E:*6} is identically satisfied for $i=j$, as well as in the case $i \neq j \neq k \neq i$. However, when considering the particular case  $i \neq j = k$, one obtains the relation 

\[
e_i \eta_j = e_j \eta_k.
\]
By symmetry, the same condition must hold for $k = i \neq j$, leading to the conclusion that \[\eta_i = e_i \eta\] for some $\eta$, defined at least locally.
This verifies the required condition in all cases and thus establishes the result.
\end{proof}
\subsection{Structure Connection and Curvature} We now turn to an analysis of the geometric properties of the pencil of connections $\nabla_\lambda$, particularly in relation to the structures of a pre-Frobenius and Frobenius manifold.

\, 

Let $\nabla_\lambda$ denote the structure connection associated with the pre-Frobenius manifold 
$(\cM, g, A)$. The curvature of this connection satisfies a quadratic relation in the parameter $\lambda$, taking the form:
\[
\nabla_\lambda^2 = R_1 \lambda^2 + R_2 \lambda + R_3,
\]
where $R_1, R_2, R_3$ are curvature terms determined by the pre-Frobenius structure. A key observation is that the term 
$R_3 $ coincide with $\lambda_0^2$ which satisfies $\lambda_0^2 = 0$,  leading to the simplification:
\[
\nabla_\lambda^2 = R_2 \lambda^2 + R_1 \lambda.
\]

This relation gives rise to the following fundamental theorem, characterizing the Frobenius condition in terms of the vanishing of specific curvature terms.
\begin{theorem}
Let $\nabla_\lambda$ be the structure connection associated with the pre-Frobenius manifold $(\cM, g, A)$. Then:
\begin{itemize}
    \item  The vanishing of $R_1$, i.e. $R_1 = 0$ if and only if $(M, g, A)$ is equivalent to the potentiality of $(M, g, A)$;
    \item The vanishing of $R_2$, i.e. $R_2 = 0$ is equivalent to the associativity of $(\cM, g, A)$.
\end{itemize}
Thus, the manifold $(M, g, A)$ is Frobenius atisfies the full Frobenius condition if and only if the pencil of connections $\nabla_\lambda$ is flat.
\end{theorem}
\begin{ex}\label{Ex:Manin2}
    Write a proof. 
\end{ex}

\subsection{Semisimple Frobenius Manifolds and Webs}
\subsubsection{Webs}
In this subsection, we discuss the construction of flat $3$-webs via semisimple $3$-dimensional Frobenius manifolds and provide a geometric interpretation of the Chern connection associated with these webs. We show that these webs are biholomorphic to the characteristic webs on the solutions of the corresponding associativity equations. Furthermore, these webs are hexagonal and possess at least one infinitesimal symmetry at each singular point.

\, 

Consider local trivial fibrations of class $C^k$. This is given by a triple $(Y,X,\pi)=\lambda$, where $\pi:Y\to X$  is a projection of $Y$ onto $X$ and $Y$ and $X$ are differentiable (smooth) manifolds of respective dimensions $m$ and $n$ where $m > n$. 
\begin{enumerate}
    \item For each point $x \in X$ the set $\pi^{-1}(x)\subset Y$ is a submanifold of dimension $m-n$ which is diffeomorphic to a manifold $F$;
 \item For each point $x \in X$ there exists a neighbourhood $U_x\subset X$ such that $\pi^{-1}(U_x)$ is diffeomorphic to the product $U_x \times F$, and the diffeomorphism between $\pi^{-1}(U_x)$ and  $U_x \times F$  is compatible with $\pi$ and the projection $pr_{U_x} : U_x \times F \to U_x$.
\end{enumerate}

Given our interest  in the local structure of such manifolds we can often think of them as connected domains of a Euclidean space of the same dimension.

\, 

If $T_{p}(Y)$ is the tangent space of $Y$ at a point ${p}$. A fibre $F$ passing through ${p}$ determines in $T_{p}(Y)$ a subspace $T_{p}(F_x)$, of codimension $r$ say, tangent to $F_x$ at ${p}$. We provide $T_{p}(Y)$ with a local moving frame $\{e_i,\,  e_\alpha; i=1,...,r; \alpha =r+1,...,m\}$, where $dim F =m - r$ and $dim X =r$. It is natural then to obtain a co-frame $\{\omega^i, \omega^\alpha\}$ dual to $\{e_i, e_\alpha\}$, such that  $\omega^i(e_{\alpha})=0$. Fibers of the fibration are integral manifolds of the system of equations $$\omega^i=0.$$

Since there exists a unique fibre $F$ through a point ${p}\in Y$, the system above $\{\omega^i=0, i=1,\cdots, r\}$ is completely integrable. By the Frobenius theorem, the integrability condition is given by 
\[d\omega^i=\sum_{j=1}^r\omega^j\wedge \phi^i_j\]
where $\phi^i_j$ are differential forms.

\, 

Assume $Y$ is an $m$-dimensional manifold. A $k$-dimensional distribution $\theta$ on an $m$-dimensional manifold $Y$, $0\leq k\leq m$ is a smooth field of $k$-dimensional tangential directions.  To each point ${p} \in Y$ there is a function which assigns a linear $k$-dimensional subspace of the tangent space $T_{p}(Y)$ to ${p}$. These surfaces are called the leaves of the foliation. The numbers $k$ is the dimension of the foliation.

\, 

Let $Y=X$ be a differentiable manifold of dimension $nr$. We  say that a $d$-web $W(d,n,r)$ of codimension $r$ is given in an open domain $D\subset Y$  by a set of $d$ foliations of codimension $r$ which are in general position. 

\, 
\subsubsection{Some results}
We begin by stating the following two theorems.

\begin{theorem}\label{thm:flatweb1}
Let the implicit cubic ODE
\[
p^3 + a(x, y)p^2 + b(x, y)p + c(x, y) = 0
\]
have a flat web of solutions and satisfy the regularity condition at 
$m = (x_0, y_0, z_0) \in \mathcal{C} \subset \sfS$. Then there exists a local diffeomorphism around $\pi(m) = (x_0, y_0)$ that reduces the ODE to:
\begin{itemize}
    \item $p^3 + px - y = 0$, if $p_0$ is a triple root and $\mathcal{C}$ is Legendrian;
    \item $p^3 + 2xp + y = 0$, if $p_0$ is a triple root and $\mathcal{C}$ is not Legendrian.
\end{itemize}
\end{theorem}

\begin{theorem}\label{thm:flatweb2}
Let the implicit cubic ODE
\[
p^3 + a(x, y)p^2 + b(x, y)p + c(x, y) = 0
\]
have a flat web of solutions and satisfy the regularity condition at 
$m = (x_0, y_0, z_0) \in \mathcal{C} \subset \sfS$. Then there exists a local diffeomorphism around $\pi(m) = (x_0, y_0)$ that reduces the ODE to:
\begin{itemize}
    \item $p^3 + px - y = 0$, if $p_0$ is a triple root and $\mathcal{C}$ is Legendrian;
    \item $p^3 + 2px + y = 0$, if $p_0$ is a triple root and $\mathcal{C}$ is not Legendrian;
    \item $p^2 - y = 0$, if $p_0$ is a double root and $\mathcal{C}$ is Legendrian;
    \item $p^2 - x = 0$, if $p_0$ is a double root and $\mathcal{C}$ is not Legendrian.
\end{itemize}
\end{theorem}

\section{Flat Coordinates and Darboux--Egoroff Coordinates}
The concept of coordinates is foundational in geometry. In this section, we examine flat and canonical coordinates. Flat coordinates play a crucial role in the theory of Frobenius manifolds. 

The problem of finding flat coordinates is essential in studying the Gauss--Manin systems, which are deeply connected to the differential equations for the integrals of basic differential forms over vanishing cycles associated with a given singularity. These systems also relate closely to the theory of primitive forms. The Gauss--Manin connection can be understood as a way to differentiate cohomology classes with respect to parameters.

Let $(\mathcal{U}, x, g)$ be a triple, where $\mathcal{U}$ is a domain in $\mathbb{R}^n$, $x = (x^1, \cdots, x^n)$ are local coordinates, and $g$ is a metric. Let $y = (y^1, \cdots, y^n)$ be another system of coordinates. Since $g$ is a symmetric tensor of rank $2$, the metric can be written in both coordinate systems as:
\[
\tilde{g}(y) = \sum_{k=1}^n \sum_{l=1}^n g_{kl}(x) \frac{\partial x^k}{\partial y^i} \frac{\partial x^l}{\partial y^j}.
\]
We say that the coordinates are flat if the matrix associated with $g$ is constant.

Consider the metric $g$, and let $T_x \cM$ denote the tangent space of the manifold $\cM$ at $x$. Let $v \in T_x \cM$. Define the kernel of the metric $g$ as
\[
\ker(g) := \{ v \in T_x \cM \mid g(v, \cdot) = 0 \}.
\]
At any point $x$, there exist functions $\Gamma_{ijk}$ and $\Gamma_{ij}^k$, defined as:
\[
\Gamma_{ijk} := \frac{1}{2} \left( \frac{\partial g_{im}}{\partial x^j} + \frac{\partial g_{jm}}{\partial x^i} - \frac{\partial g_{ij}}{\partial x^m} \right),
\]
\[
\Gamma_{ij}^k := \frac{1}{2} g^{km} \left( \frac{\partial g_{im}}{\partial x^j} + \frac{\partial g_{jm}}{\partial x^i} - \frac{\partial g_{ij}}{\partial x^m} \right).
\]
Then, the following theorem holds.

\begin{theorem}\label{thm:flatmetric}
For any $ i, j, k $, the condition  
\begin{equation}\label{E:Gamma1}
\sum_{l=1}^n \left( \Gamma_{ij}^l g_{kl} + \Gamma_{kj}^l g_{il} \right) = \frac{\partial g_{ij}}{\partial x^k} 
\end{equation}
is equivalent to the condition:  
\begin{equation}\label{E:Gamma2}
\sum_{l=1}^n \Gamma_{ki}^l v^l = 0, 
\end{equation}
where $v \in \ker(g) $ and $\Gamma_{ki}^i = \Gamma_{ik}^i $.  

If the rank of $ g $ is constant and condition \eqref{E:Gamma2} holds, then there exists a smooth function $ \Gamma_{ki}^i(x) $ satisfying \eqref{E:Gamma1}.  
\end{theorem}

\begin{proof}
Fix a point $x = (x^1, \cdots, x^n)$. Consider the system of equations
\[
\sum_{l=1}^n (\Gamma_{ij}^l g_{kl} + \Gamma_{kj}^l g_{il}) = \frac{\partial g_{ij}}{\partial x^k}.
\]
This is a linear system in the unknowns $\Gamma_{ij}^l$, $\Gamma_{kj}^l$, with $g_{kl}$, $g_{il}$, and $\frac{\partial g_{ij}}{\partial x^k}$ as coefficients. Let $A$ be the coefficient matrix, $y$ the vector of unknowns, and $b$ the vector of constants. The system has a solution if and only if, for every vector $a$ such that $a^\top A = 0$, it holds that $a^\top b = 0$. This leads to:
\[
\sum_{l=1}^n g_{lk} \Gamma_{ij}^l = \Gamma_{ij, k}.
\]
From this, we deduce that:
\[
\sum_{l=1}^n \Gamma_{ki}^l v^l = 0.
\]
\end{proof}

Finally, we have the following theorem characterizing the existence of flat coordinates.

\begin{theorem}\label{thm:flatcoord}
Flat coordinates for the metric $g$ exist if and only if there exist smooth functions $\Gamma_{ij}^k(x)$, symmetric in $i$ and $j$, satisfying:
\begin{itemize}
    \item The compatibility condition:
    \[
    \sum_{l=1}^n (\Gamma_{jk}^l g_{il} + \Gamma_{ik}^l g_{jl}) = \frac{\partial g_{ij}}{\partial x^k};
    \]
    \item The vanishing of the Riemann curvature tensor:
    \[
    R_{ijkm} = 0,
    \]
    where
    \[
    R_{ijkm} := \sum_l g_{il} \left( \frac{\partial}{\partial x^k} \Gamma_{jm}^l - \frac{\partial}{\partial x^m} \Gamma_{jk}^l + 
    \sum_t (\Gamma_{kt}^l \Gamma_{mj}^t - \Gamma_{mt}^l \Gamma_{jk}^t) \right).
    \]
\end{itemize}
\end{theorem}

\section{Exercises}
Exercise \ref{Ex:preLie}. Using the relations in \eqref{E:1} and \eqref{E:2}, it is easy to see that one can obtain the structure of a pre-Lie algebra on the tangent sheaf.

\, 

Exercise \ref{Ex:Ma} is easily shown by using the paper of Calabi (1954)\cite{Ca54} and the definition of a potential pre-Frobenius manifold.

\,

Exercises \ref{Ex:ManinProof} and \ref{Ex:Manin2}. The proof proceeds in two stages.
\noindent\textbf{Part 1.}  

\, 

We calculate the coefficient of the $\lambda$ term, $R_1$, in the following expression:
\[
\big[\nabla_{0, \partial_a} + \lambda \partial_a \circ, \nabla_{0, \partial_b} + \lambda \partial_b \circ\big](\partial_c).
\]
It follows immediately that $R_1 = 0$ if and only if, for any $a, b, c, d$ and for a mixed $(1, 2)$ rank tensor $A_{bc}^e$,  
\[
\partial_a A_{bc}^e = (-1)^{ab} \partial_b A_{ac}^e,
\]
or equivalently, for a rank-3 covariant tensor:
\[
\partial_a A_{abc} = (-1)^{ab} \partial_b A_{aac}.
\]

When $A$ is a potential, the symmetry of the rank-3 tensor $A(X, Y, Z)$, written as $A(X, Y, Z) = (XYZ)\Phi$, ensures that the above condition holds.

Suppose that the relation $\partial_a A_{abc} = (-1)^{ab} \partial_b A_{aac}$ is true. Then, for all $c, d$, the form $\sigma_b \, dx^b \, A_{bcd}$ is closed. Locally, we can find functions $B_{cd}$ satisfying $B_{cd} = (-1)^{cd} B_{dc}$. Taking into account the symmetry of the rank-3 covariant tensor $A$, we obtain:
\[
A_{bcd} = \partial_b B_{cd} = (-1)^{bc} \partial_c B_{bd} = (-1)^{bc} A_{cbd}.
\]
Hence, for any $d$, the expression $\sigma_c \, dx^c \, B_{cd}$ is closed. Analogously, we find locally $B_{cd} = \partial_c C_d$. Since $C_d = \partial_d \Phi$, it follows that $A_{bcd} = \partial_b \partial_c \partial_d \Phi$.

This completes the first part of the proof.

\noindent\textbf{Part 2.}  
We compute the coefficient of the $\lambda^2$ term in $[\nabla_{\lambda, X}, \nabla_{\lambda, Y}](Z)$:
\[
R_{2, XY}(Z) = X \circ (Y \circ Z) - (-1)^{\overline{X}\, \overline{Y}} Y \circ (X \circ Z).
\]
If the multiplication $\circ$ is associative, then $R_2 = 0$ because $\circ$ is always commutative. 

Conversely, if $R_2 = 0$, then:
\[
X \circ (Y \circ Z) = (-1)^{\tilde{X} \tilde{Y}} Y \circ (X \circ Z) = (-1)^{\tilde{X} (\tilde{Y} + \tilde{Z})} (Y \circ X) \circ Z.
\]
Thus, associativity of $\circ$ follows. This completes the proof of the theorem.




\chapter{Unveiling the Hidden Geometry of Statistical Manifolds}
In this chapter, and the following one, we reveal the intricate (and often hidden!) geometric structures underlying statistical manifolds. In particular, we revisit certain geometric concepts that have long been overlooked, such as $m$-pairs, and demonstrate their relevance in this context. By bringing these ideas back into focus, we aim to provide a deeper understanding of the rich interplay between geometry and statistical structures.
\section{Projective geometry, $m$-pairs, Grassmannians}
\subsection{$m$-pairs}\label{S:4.3}
Let us consider an object $X_{d}$, a $d$-dimensional surface residing in an $n$-dimensional projective space $\bbP^{n}$, where the constraint $d \leq n$ holds. 

\, 


\begin{definition}
We say that the surface $X_{d}$ is \textit{normalized} if, for every point $p \in X_{d}$, we associate two distinct hyperplanes:
\begin{enumerate}
\item Normal of first type, $P_{I}$, is of dimension $(n-d)$ and intersects the tangent $d$-plane $T_{p}X_{d}$ at a unique point $p$.
\item Normal of the second type, $P_{II}$, is of dimension $(d-1)$, situated  within the $d$-plane $T_{p}X_d$, and does not pass through the point $p$.
\end{enumerate}
\end{definition}


\begin{example}
    Consider the case where $d = 2$ and  $n = 3 $. This means we have a 2-dimensional surface $ X_2$  within the 3-dimensional projective space $\bbP^{3}$.

\, 

At a point $p$ on $X_2 $:

\begin{itemize}
    \item The tangent plane  $T_{p}X_{2}$ is a 2-dimensional plane in the ambient $\bbP^{3}$ space.
    \item The normal of first type   $P_{I} $ would be a line that intersects the tangent plane   $T_{p}X_{2}$  at the point   $p$.
    \item The normal of second type  $P_{II}$ is another line that lies completely within the tangent plane  $ T_{p}X_{2}$ but does not contain the point  $p$.
    \end{itemize}
    \end{example}


This embodies a duality, intrinsic to projective geometry. Notably, in the  case where $d = n$, the hyperplane $P_{I}$ identifies to the point $p$, and $P_{II}$ becomes the $(n-1)$-dimensional surface devoid of the point $p$. This situation reflects the classical notion of duality in projective spaces, leading to the identification of $X_{n}$ with the projective space $\bbP^{n}$.

\subsection{$m$-pairs}
\begin{definition}\label{D:mpairs}
We define an \textit{$m$-pair} as a pair constituted of an $m$-plane and an $(n-m-1)$-plane.
\end{definition}

More precisely, an \textbf{m-pair} is a pair consisting of:
\begin{enumerate}
    \item An \textbf{m-plane}: A linear subspace of dimension $m $ in an $n $-dimensional space.
    \item An \textbf{(n-m-1)-plane}: A linear subspace of dimension $n-m-1$ in the same $n$-dimensional space.
\end{enumerate}

These two planes are typically considered in the context of projective geometry or linear algebra, where they may satisfy certain geometric or algebraic relationships.
\begin{ex}
Draw an $m$-pair in the low dimensional cases. 
\end{ex}


\subsection{Observations and Examples}
\subsection*{3D Space ($n = 3 $)}
\begin{itemize}
    \item \textbf{0-pair}: A point (0-plane) and a plane (2-plane). This can be identified with the projective space $\bbP^3$
    \item \textbf{1-pair}: A line (1-plane) and another line (1-plane). Two lines in 3D space can either intersect at a point, be parallel (not intersecting but lying on a common plane), or be skew (not intersecting and not parallel).
\end{itemize}

\subsection*{4D Space ($n = 4 $)}
\begin{itemize}
    \item \textbf{1-pair}: A line (1-plane) and a plane (2-plane). In 4D space, a line and a plane can intersect at a point, not intersect at all, or intersect along a line if the line lies entirely within the plane.
    \item \textbf{2-pair}: Two planes (2-planes). In 4D space, two planes can intersect at a point, along a line, or not intersect at all.
\end{itemize}

\subsection*{Projective Geometry}
In projective geometry, an $m $-pair can be used to describe configurations of points, lines, and planes at infinity. 

\subsection*{Grassmannians}
The concept of $m $-pairs is closely related to Grassmannians, which are spaces that parameterize all $m $-dimensional subspaces of an $n$-dimensional space. The study of $m $-pairs can be seen as a way to explore the relationships between different Grassmannians.

\begin{ex}
Determine the relation between Grassmanianns and $m$-pairs. 
\end{ex}
\subsection*{Applications in Computer Vision}
In computer vision, $m$-pairs can be used to model the relationship between different views of a scene. For example, in structure from motion, the relationship between 2D image planes (2-planes) and 3D space (3-planes) can be analyzed using the concept of $m $-pairs.

\begin{ex}
Generate an example using your favourite software $m$-pairs modelisin the relationship between different views of a scene.
\end{ex}

 The examples above illustrate how $m $-pairs can be applied in various contexts, from projective geometry to computer vision.\,


\subsection{Some properties}

Normalized surfaces associated with an $m$-pair space possesses the following properties:
\medskip
\begin{lemma}\label{L:pairs}
\
\begin{enumerate}
\item The collection of $m$-pairs forms a projective differentiable manifold.
\item For any integer $m \geq 0$, a manifold of $m-$pairs contains two flat, affine, and symmetric connections.
\end{enumerate}
\end{lemma}

\section{Paracomplex numbers and modules}
Paracomplex numbers are a generalization of complex numbers, where instead of the imaginary unit   i    satisfying  $ i^2 = -1 $  , the paracomplex unit   $\epsilon $   satisfies  $ \epsilon^2 = 1 $  . The algebra of paracomplex numbers is defined as follows:

\subsection{Definition}
A \textbf{paracomplex number} is an element of the form:
\[
z = x + \epsilon y,
\]
where  $ x, y \in \mathbb{R} $   are real numbers, and  $ \epsilon  $   is the paracomplex unit satisfying:
\[
\epsilon ^2 = 1.
\]
The set of all paracomplex numbers is denoted by   $\mathbb{P}$   .

\subsection{Algebraic Structure}
The algebra of paracomplex numbers   $\mathfrak{P} $  is a two-dimensional commutative algebra over the real numbers   $\mathbb{R}$   . It is isomorphic to the direct sum  $ \mathbb{R} \oplus \mathbb{R} $  , and its multiplication rule is given by:
\[
(x_1 + \epsilon y_1)(x_2 + \epsilon y_2) = (x_1x_2 + y_1y_2) + \epsilon (x_1y_2 + x_2y_1).
\]

\subsection{Properties}
\begin{itemize}
    \item \textbf{Idempotent Basis}: Paracomplex numbers can be expressed in terms of idempotent elements. Define:
    \[
    e_+ = \frac{1 + \epsilon }{2}, \quad e_- = \frac{1 - \epsilon }{2}.
    \]
    These elements satisfy  $ e_+^2 = e_+ $  ,   $e_-^2 = e_- $  , and   $e_+ e_- = 0 $ . Any paracomplex number   $z = x + \epsilon y  $  can be written as:
    \[
    z = (x + y)e_+ + (x - y)e_-.
    \]

    \item \textbf{Conjugation}: The paracomplex conjugate of  $ z = x + \epsilon y $   is defined as:
    \[
    \overline{z} = x - \epsilon y.
    \]

    \item \textbf{Norm}: The norm of a paracomplex number   $z = x + \epsilon y $   is given by:
    \[
    \|z\| = z \overline{z} = x^2 - y^2.
    \]
    Note that this norm is not positive definite, as it can take negative values.
\end{itemize}

\begin{ex}
Show that $(a+a\epsilon)\cdot(b-b\epsilon)=0$.
\end{ex}
\section{Modules over Paracomplex Algebras}
A module over a paracomplex algebra generalizes the concept of a vector space, where the scalars are paracomplex numbers instead of real or complex numbers.

\subsection{Definition}
Let  $ \mathfrak{P}$ be the algebra of paracomplex numbers. A \textbf{module over  $ \mathfrak{P}$} is an abelian group   M    together with a scalar multiplication:
\[
\cdot :  \mathfrak{P} \times M \to M,
\]
satisfying the following properties for all   $z, z_1, z_2 \in  \mathfrak{P}$     and   $m, m_1, m_2 \in M $  :
\begin{enumerate}
    \item   $z \cdot (m_1 + m_2) = z \cdot m_1 + z \cdot m_2   $,
    \item  $ (z_1 + z_2) \cdot m = z_1 \cdot m + z_2 \cdot m $  ,
    \item   $(z_1 z_2) \cdot m = z_1 \cdot (z_2 \cdot m) $  ,
    \item  $ 1 \cdot m = m $  .
\end{enumerate}

\subsection{Examples}
\begin{itemize}
    \item \textbf{Paracomplex Vector Space}: The simplest example of a module over   $ \mathfrak{P}$     is  $ \mathfrak{P}^n$   , the set of   n   -tuples of paracomplex numbers. Scalar multiplication is defined component-wise:
    \[
    z \cdot (z_1, z_2, \dots, z_n) = (z z_1, z z_2, \dots, z z_n).
    \]

    \item \textbf{Decomposition into Real Submodules}: Using the idempotent basis   $e_+ $   and  $ e_- $  , any module  $ M$    over   $\mathfrak{P} $   can be decomposed into two real submodules:
    \[
    M = e_+ M \oplus e_- M.
    \]
    Here,  $ e_+ M $   and  $ e_- M  $  are real vector spaces, and the action of  $\mathfrak{P}$  on  $ M $   is determined by the actions of  $ e_+  $  and $  e_- $  .
\end{itemize}

\subsection{Applications}
Modules over paracomplex algebras appear in various areas of mathematics and physics, including:
\begin{itemize}
    \item \textbf{Geometry}: Paracomplex structures are used in the study of para-Hermitian and para-Kähler manifolds.
    \item \textbf{Physics}: Paracomplex numbers and modules are used in the study of supersymmetry and integrable systems.
\end{itemize}


\section{Applications to statistical manifolds}

The algebra of paracomplex numbers has a remarkable incidence on the manifold of probability distributions. We discuss this in the following propositions and statements. 
This leads us to the following salient proposition:
\begin{proposition}\label{P:isome}
The space of $0$-pairs within the projective space $\bbP^{n}$ is isometric to the hermitian projective space over the algebra of paracomplex numbers.
\end{proposition}
\begin{proof}
Refer to section 4.4.5 of~\cite{Ro97} for detailed proof.
\end{proof}
\medskip
\begin{proposition}\label{P:zero}
Let $(X, \mathcal{F})$ be a finite measurable set with dimension $n+1$, where measures vanish exclusively on an ideal $\mathcal{I}$. Define $\mathcal{H}_{n}$ as the space of probability distributions on $(X, \mathcal{F})$. It follows that the space $\mathcal{H}_{n}$ embodies a manifold of $0$-pairs.
\end{proposition}
\begin{proof}
The $n$-dimensional surface $\mathcal{H}_{n}$ arises as the intersection of the hyperplane constrained by $\mu(X) = 1$ and the cone $\mathcal{C}_{n+1}$ of strictly positive measures within the affine space $\mathcal{W}_{n+1}$ of signed bounded measures. It is interpreted as an $n$-dimensional surface within the projective space $\bbP^{n}$. The geometrical structure of this surface is thus inherited from projective geometry. By invoking the remark from the initial paragraph of section 0.4.3 in\cite{Ro97} alongside definition\ref{D:mpairs} of $0$-pairs, we conclude the correspondence with a manifold of $0$-pairs.
\end{proof}
\medskip
\begin{theorem}\label{Th:main}
Consider $(X, \mathcal{F})$ as a finite measurable set with dimension $n+1$, where measures vanish solely on an ideal $\mathcal{I}$. The space $\mathcal{H}_{n}$ of probability distributions on $(X,\mathcal{F})$ is isomorphic to the hermitian projective space over the cone $M_{+}(2,\fC)$.
\end{theorem}





\chapter{Statistical Frobenius manifold and Learning}


\section{Statistical Gromov--Witten invariants and learning}

\vskip-.2cm



We introduce Gromov–Witten invariants for statistical manifolds (denoted GWS), extending an analog of classical Gromov–Witten invariants to the realm of information geometry. Originally, these invariants are rational numbers that enumerate (pseudo-)holomorphic curves satisfying specific conditions in a symplectic manifold. In our generalization, GWs encode fundamental geometric structures of statistical manifolds, reflecting the intersection theory of (para-)holomorphic curves within this framework.

\,

Furthermore, this perspective reveals an intrinsic connection between the geometry of statistical learning and the dynamics of the learning process. The presence or obstruction of certain pseudo-holomorphic structures, as captured by GWs, provides a criterion for determining whether a learning system successfully acquires information or encounters fundamental limitations, thereby offering a novel geometric approach to the theory of learning.

\,

\subsection{Brief Recollections of Gromov–Witten Invariants}

Gromov–Witten invariants are fundamental numerical invariants in symplectic geometry and algebraic geometry, capturing intersection properties of (pseudo-)holomorphic curves in a given space.

Given a compact symplectic manifold $(\cM,\varpi)$, the Gromov–Witten invariant counts the number of (pseudo-)holomorphic maps
\[u:(\mathscr{S},j)\to (\cM,\varpi),\]
from a compact Riemann surface $\mathscr{S}$ (with complex structure 
$j$) into $\cM$, satisfying certain constraints on their homology class and intersection conditions with given cycles.

\begin{itemize}
    \item This notion appears in {\bf Enumerative Geometry}:
Gromov–Witten invariants count the number of holomorphic curves passing through prescribed points or satisfying intersection constraints.
\item {\bf Quantum Cohomology:}
They define a deformation of classical cohomology, giving rise to a quantum product that encodes curve counts in a ring structure.
\item {\bf Holomorphic Curve Moduli Space}:
The counts arise from integration over the moduli space of stable maps. We refer to \cite{Man99} for more information on this topic. \end{itemize}

The point of view that we adopt, here, is inspired from quantum cohomology, which is a formal Frobenius manifold. 

\,  

Let us consider the (formal) Frobenius manifold $(H,g)$. We denote by $k$ a field of characteristic 0 (such as $\bbC$ or $\bbR$). Let $H$ be a $k$-module of finite rank and \[g:H\otimes H\to k,\]  an even symmetric pairing (which is non degenerate). We denote $H^*$ the dual to $H$. 

\,

An important part of the {\bf Frobenius manifold structure} is encoded in the existence of a {\bf potential function} 
\[{\bf \Phi} \in k[[H^*]],\]
which governs the multiplication structure on the manifold. In local coordinates, under suitable conditions, this function can be expressed as: 
\[{\bf \Phi}=\sum_{n\geq 3}\frac{1}{n!}Y_n,\] 
where $Y_n\in (H^*)^{\otimes n}$ is a symmetric multilinear map 
\[Y_n\in (H^*)^{\otimes n},\quad Y_n: H^{\otimes n} \to k.\] 
This system of multilinear forms defines a system of {\it abstract correlation functions} on the pair $(H,g)$, where $H$ is a vector space equipped with a non-degenerate pairing $g$. These functions are symmetric and (in the context of Gromov–Witten theory) correspond to intersection numbers on the moduli space of stable maps. The Gromov--Witten invariants are generated from those multi-linear maps. 

\,

In the context of Gromov–Witten invariants, the potential function 
$\Phi$ serves as the generating function for the intersection numbers of moduli spaces of holomorphic curves. The symmetric multilinear maps $Y_n$ correspond to the correlation functions computed via topological field theory techniques.
\, 

 The function $\Phi$ satisfies the WDVV equations (associativity conditions on quantum cohomology), which govern the Frobenius manifold structure.

\,

The maps $Y_n$ define the higher-order correlation functions, whose values give the Gromov–Witten invariants.

\, 

This formulation provides a bridge between Frobenius manifolds, quantum cohomology, and Gromov–Witten theory, showing how the potential function encodes geometric intersection theory in an algebraic and formal power series framework. The abstract correlation functions $Y_n$	
  serve as the structural foundation from which the Gromov–Witten invariants emerge, linking the geometry of moduli spaces with the algebraic structure of Frobenius manifolds.

  \subsection{Links to statistics}
In the framework of statistical geometry, we return to the study of statistical manifolds, emphasizing the discrete case of the exponential family. The fundamental relation governing this structure is given by the expansion:
\begin{equation}\label{E:2}\sum_{\omega\in \Omega} \exp\{-\sum \beta^jX_j(\omega)\}=
\sum_{\omega\in \Omega}\sum_{m\geq 1}\frac{1}{m!}\left\{ -\sum_{j} \beta^jX_j(\omega)\right\}^{\otimes m}, \end{equation}
where:

\begin{itemize}
    \item The parameter $\beta$, given by $\beta=(\beta_0,....,\beta_n)\in \mathbb{R}^{n+1}$, provides an   affine canonical parametrisation.
    \item The objects $X_j(\omega)$ are directional co-vectors, 
    forming a finite set of necessary and sufficient statistics, denoted by $\mathcal{X}_n$. 
    \item The co-vectors, $X_1(\omega),\dots ,X_{n}(\omega)$ are linearly independent co-vectors and we impose the normalization $X_0(\omega)\equiv 1$. %among which the {\it energy}  $
\end{itemize}
The family in (\ref{E:2}) defines an analytic $n$-dimensional hypersurface within the statistical manifold, which can be uniquely determined by $n+1$ points in general position. 

\subsection{Gromov--Witten Invariants for Statistical Manifolds}
We introduce the notion of Gromov–Witten invariants for statistical manifolds (GWS) as follows:
\begin{definition}
Let $k$ be the field of real numbers. Let $\sfS$ be the statistical manifold. The Gromov--Witten invariants for statistical manifolds (GWS) are defined from the family of multilinear maps:
 \[\tilde{Y}_n:\sfS^{\otimes n}\to k.\] 
\end{definition}
Equivalently, these invariants may be written in terms of the generating expansion:
\[\tilde{Y}_n\in \left(-\sum_{j}\beta^jX_j(\omega)\right)^{\otimes n}.\]

\subsection{Interpretation via the Relative Entropy Function}
These invariants naturally emerge as part of the potential function
$\tilde{\bf \Phi}$, which is identified with the Kullback--Liebler entropy function of the statistical system. The entropy function itself is expressed in the form: 

\begin{equation}\label{E:3}\tilde{\bf \Phi}= \ln \sum_{\omega\in\Omega} \exp{(-\sum_{j}\beta^jX_j(\omega))}.\end{equation}

This formulation suggests a deeper geometric and categorical interpretation of statistical learning, where the intersection theory of statistical structures plays a fundamental role. Within this perspective, the entropy function $\tilde{\bf \Phi}$ governs the geometry of statistical families, much like the potential function in Frobenius manifolds or Gromov–Witten theory encodes intersection numbers in moduli spaces of holomorphic curves.

Therefore, we state the following:

\begin{proposition}
The entropy function $\tilde{\bf \Phi}$ of the statistical manifold is intrinsically determined by the Gromov–Witten invariants for statistical manifolds (GWS).
\end{proposition}
More precisely, $\tilde{\bf \Phi}$ arises as a generating function whose coefficients encode the multilinear maps $\tilde{Y_n}$, which define the GWS structure.  These invariants characterize the underlying statistical geometry by capturing the intersection properties of statistical hypersurfaces.


\begin{proof}
Indeed, since $\tilde{\bf \Phi}$, in formula (\ref{E:3}) relies on the polylinear maps $\tilde{Y}_n\in \left(-\sum_{j}\beta^jX_j(\omega)\right)^{\otimes n}$,
 defining the (GWS), the statement follows. 
\end{proof}

\section{Learning}
We consider the tangent fiber bundle over the statistical manifold 
$\sfS$, where $\sfS$ is the space of probability distributions. This bundle structure encodes the infinitesimal geometry of the statistical space, allowing us to describe variations in probability distributions in terms of a Lie group action.
\subsection{Tangent Fiber Bundle Structure}
The tangent fiber bundle is denoted by the quintuple $(T\sfS,\sfS,\pi,G,F)$, where:
\begin{itemize}
    \item $T\sfS$ is the total space of the tangent bundle, consisting of all possible tangent vectors to points in $\sfS$.

    \item $\pi: T\sfS\to \sfS$ is a continuous surjective map that projects each tangent vector to its base point in $\sfS$.
    \item  $F$ is the fiber,  representing the space of allowable tangent vectors at each point of $\sfS$. 
    \item $G$ is a Lie group that acts on the fibers, encoding the parallel transport structure within the statistical manifold.
    
\end{itemize}
\subsection{Tangent Spaces and the Space of Measures}

For any point $\rho\in \sfS$, the tangent space at $\rho$, denoted 
$T_{\rho}\sfS$, is given by: 
\[T_{\rho}S \cong \{\text{bounded, signed measures vanishing on an ideal $I$ of the $\sigma-$algebra}\}.\]


This identification follows from the fact that infinitesimal perturbations of a probability distribution $\rho$ can be described by signed measures that respect the probabilistic constraints imposed by 
$\sfS$. The ideal 
$I$ of the 
$\sigma$-algebra corresponds to the subspace of measures that do not contribute to the variations in probability distributions, ensuring consistency with the underlying measure-theoretic structure.


\subsection{Lie Group Action on the Fibers}
The Lie group $G$  acts freely and transitively on each fiber of $T\sfS$, meaning that every element of the fiber can be transformed into any other through the group action. The action is given by:

\[f\overset{h}{\mapsto} f+h,\]
where:
\begin{itemize}
    \item $f$ is an element of the total space $T\sfS$ i.e., a tangent vector at some $\rho\in \sfS$.

    \item $h$ is a parallel transport within the statistical manifold, representing an infinitesimal displacement in the space of probability distributions.
\end{itemize}


This affine structure on the fibers implies that the action of $G$
acts as a translation group, ensuring a well-defined parallel transport mechanism in the space of probability measures. Such a structure is crucial for describing information geometry, as it encodes how probability distributions evolve under statistical transformations.

%%%%


\begin{lemma}
Consider the fiber bundle $(T\sfS,\sfS,\pi,G,F)$ where: 

Let path $\gamma:\cI\to \sfS$ be a smooth geodesic path in $\sfS$, where $\cI\subset \bbR$. 
The fiber over $\gamma$ is denoted by $F_{\gamma}=\pi^{-1}(\gamma)$, which represents the space of tangent vectors along $\gamma$. 

Then, the fiber \[F_{\gamma}=\gamma^+\sqcup\gamma^{-1}\] consists of two disjoint connected components. Each component $\gamma^+$ and $\gamma^-$ is contained within a totally geodesic submanifold of $T\sfS$, denoted $E^+$ and $E^-$, respectively. 
\end{lemma}
\begin{proof}
Consider the fiber above $\gamma$. Since for any point of $\sfS$, its the tangent space is identified to  module over paracomplex numbers. This space is decomposed into a pair of subspaces (i.e. eigenspaces with eigenvalues $\pm \e$).
The geodesic curve in $\sfS$ is a path such that $\gamma=(\gamma^i(t)): t\in [0,1]\to \sfS$. In local coordinates, the fiber budle is given by $\{\gamma^{ia}e_{a}\}$, and $a\in \{1,2\}$. Therefore, the fiber over $\gamma$ has two components $(\gamma^+,\gamma^-)$. Taking the canonical basis for $\{e_1,e_2\}$, implies that $(\gamma^+,\gamma^-)$ lie respectively in the subspaces $E^+$ and $E^-$. These submanifolds are totally geodesic in virtue of  Lemma 3 in \cite{CoCoNen}. \end{proof}

\subsection{Learning Process via the Ackley--Hinton--Sejnowski method}

We define a learning process in terms of the Ackley--Hinton--Sejnowski method \cite{AHS}, which is based on minimizing the Kullback--Leibler divergence as a measure of distance between probability distributions. This process can be interpreted in a geometric framework as follows:

\begin{proposition}[Geometric Formulation of Learning Process] \label{P:CoNen}
The learning process consists of determining whether there exist intersections between the \emph{paraholomorphic curve} $\gamma^+ $ and the \emph{orthogonal projection} of $\gamma^- $ into the subspace $E^+ $.
\end{proposition}

More precisely, let $\sfS$ be a \emph{statistical manifold} equipped with a fiber bundle structure $(T\sfS, \sfS, \pi, G, F) $. Consider a geodesic path $\gamma: I \to \sfS $, parametrized by an interval $I \subset \mathbb{R}$. The fiber over $\gamma$, denoted by $F_{\gamma}$, is assumed to decompose into two connected components:
\[
    F_{\gamma} = \gamma^+ \sqcup \gamma^-.
\]
Each component $\gamma^+ $ and $\gamma^- $ is contained in a \emph{totally geodesic submanifold} of $TS $, denoted by $E^+ $ and $E^- $, respectively:
\[
    \gamma^+ \subset E^+, \quad \gamma^- \subset E^-.
\]

In particular, the learning process is considered successful whenever the distance between the geodesic $\gamma^+ $ and its orthogonal projection into $E^+ $ decreases towards zero. That is, the process converges if:
\[
    d(\gamma^+, \pi_{E^+}(\gamma^-)) \to 0, \quad \text{as the learning iterations progress}.
\]

This formulation provides a rigorous geometric criterion for assessing the success of learning, leveraging the underlying differential geometry of the statistical manifold.

In other words: 
\begin{proposition}\label{P:CoNen}
The learning process consists in determining if there exist intersections of the paraholomorphic curve $\gamma^+$ with the orthogonal projection of the curve $\gamma^-$ in the subspace $E^+$. 
\end{proposition} 
%%%

More formally, as was depicted in\, \cite{BCN99} (sec. 3) let us denote by $\Upsilon$ the set of (centered) random variables over $(\Omega,\mathcal{F},P_{\theta})$
which admit an expansion in terms of the scores under the following form:
\[\Upsilon_P= \{X\in \mathbb{R}^{\Omega}\, |\, X-\mathbb{E}_P[X]=g^{-1}(\mathbb{E}_P[Xd\ell]), d\ell \}. \]


By direct calculation, one finds that the log-likelihood $\ell= ln\rho$ of the usual (parametric) families of probability distributions belongs to 
$\Upsilon_p$  as well as the difference $\ell -\ell^*$  of log-likelihood of two probabilities of the same family. 

\, 

Being given a family of probability distributions such that $\ell \in  \Upsilon_P$ for any $P$, let $\mathcal{U}_P$, let us denote $P^*$ the set
 such that $\ell-\ell^*\in  \Upsilon_p$. Then, for any $P^*\in  \mathcal{U}_p$, we define $K(P,P^*)=\mathbb{E}_P[\ell - \ell^*]$. 

\begin{theorem}
Let $\sfS$ be  statistical manifold, equipped with a Riemannian metric and an affine connection. Then, the (GWS) determines the evolution of the learning process, through the associated geometric constraints. 
\end{theorem}
\begin{proof}
Whenever there is a successful learning, the distance between the curve $\gamma^+$ and the projection of  $\gamma^-$ on $E^+$ tends to be as small as possible. This implies that $K(P,P^*)=\mathbb{E}_P[\ell - \ell^*]$, so that $K(P,P^*)$ is minimized.

The learning process is by definition given by a {\it deformation} of a pair of geodesics, defined respectively in the pair of totally geodesic manifolds $E^+, E^-$. The (GWS), arise in the $\tilde{Y}_n$ in the potential function $\tilde{\bf \Phi}$, which is directly related to the relative entropy function $K(P,P^*)$. Therefore, it is easy to conclude that the (GWS) determine the learning process.
 \end{proof}

Similarly as in the classical (GW) case, the (GWS) count intersection numbers of the para-holomorphic curves generated by $\gamma^+$ and $\gamma^-$. In fact, we have the following statement:
\begin{corollary}
Let $(T\sfS,\sfS,\pi,G,F)$ be the fiber bundle above, where: 
\begin{itemize}
    \item $\pi:T\sfS\to \sfS$ is the projection map;
    \item $G$ is the Lie group;
    \item $F$ is a fiber.
\end{itemize}
Let $\gamma^-\subset T\sfS$ be a geodesic in the tangent bundle with respect to the affine connection and let $\gamma^+\subset E^+$ be a geodesic in the sub-bundle $E^+\subset T\sfS$. Then, the (GWS) determine the number of intersections of the projection of  $\gamma^-$ onto $E^+$, with the geodesic $\gamma^+$.
\end{corollary}


