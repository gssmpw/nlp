\chapter{Introduction}

In what follows, we shall be concerned with a surprising fusion of ideas, namely, that of differential geometry and probability theory (or, more generally, statistics). At first glance, these two domains might appear unrelated, yet their synthesis gives rise to a highly rich and intricate mathematical structure, leading to the emergence of a relatively young field, known as the \emph{geometry of information}. The relevance of this framework extends beyond pure mathematics, with applications in artificial intelligence (such as in language models like ChatGPT) and machine learning. Given the accelerating pace of developments in these areas, it is imperative to pursue the deeper mathematical underpinnings of this theory.  

\,

One of the great advantages of geometry is that it offers an intuitive and often visual means of understanding complex situations. This feature allows us to circumvent difficulties that might otherwise seem insurmountable. More precisely, at the heart of the geometry of information lies a fundamental object: the manifold of probability distributions, where the probability measures belong to a specific class of distributions. 

\,

To make matters precise, recall that a \emph{probability space} consists of a triple $(\Omega, \mathcal{F}, P)$, where $\Omega$ is the sample space (the set of possible outcomes), $\mathcal{F}$ is a $\sigma$-algebra of measurable subsets of $\Omega$, and $P$ is a probability measure assigning to each event in $\mathcal{F}$ a real number in the interval $[0,1]$. This object $(\Omega, \mathcal{F}, P)$ serves as a rigorous mathematical model for phenomena arising in nature. The challenge we face is to endow this structure with additional geometric data, thereby giving rise to a natural geometric space. However, this task is far from trivial.  

\,

A first step is to recognize that the set $\mathcal{F}$ carries a particular mathematical structure: that of a $\sigma$-algebra. The significance of $\sigma$-algebras lies in their closure properties: if $(\Omega, \mathcal{F})$ is a \emph{measurable space}, then $\mathcal{F}$ is closed under countable unions, countable intersections, and complements. However, when seeking to introduce a geometric framework, it is often more convenient to work directly with probability distributions rather than the space $(\Omega, \mathcal{F})$ itself.  

\,

Probability distributions come in a vast array of different families. To clarify our discussion, let us recall the three principal classes of probability distributions:

\begin{enumerate}[1)]
    \item Discrete probability distributions, either with finite or infinite support.
    \item Absolutely continuous probability distributions, whose support may be:
    \begin{enumerate}[a)]
        \item A bounded interval.
        \item An interval of length $2\pi$ (directional distributions).
        \item A semi-infinite interval, $[0, \infty)$.
        \item The entire real line.
    \end{enumerate}
    \item Probability distributions with variable support.
\end{enumerate}

For our purposes, we restrict attention to the second class, focusing in particular on absolutely continuous distributions supported on intervals of length $2\pi$, namely, \emph{wrapped exponential distributions}. 

A remarkable fact, which has emerged through careful analysis, is that probability distributions of this type possess a hidden geometric structure. The apparatus of differential geometry provides the natural language for uncovering and describing this structure. Objects such as connections, parallel transport, curvature, and flatness serve as fundamental tools in the construction of a geometric framework for probability distributions. This allows for the realization of a novel class of geometric spaces: \emph{the manifolds of probability distributions}. 

More concretely, consider a measurable space $(\Omega, \mathcal{F})$, where $\mathcal{F}$ is a $\sigma$-algebra on $\Omega$. If we consider a family of parametrized probability distributions on $(\Omega, \mathcal{F})$, then the space of such distributions naturally inherits the structure of a manifold. 

\,

The study of these manifolds, particularly in the case of exponential families, leads to unexpected connections with \emph{Topological Field Theory} (TFT). The relation between exponential families and TFT is not accidental; rather, it is deeply encoded in the mathematical structures underlying both domains. Indeed, Topological Field Theory is intimately linked to the celebrated \emph{Witten–Dijkgraaf–Verlinde–}
\emph{Verlinde (WDVV) equations}, which in turn play a central role in the theory of \emph{Frobenius manifolds}, developed and studied by Dubrovin, Manin, Kontsevich, and others. 

\,

A fundamental aspect of our approach, which renders manifest the deep relation between the manifold of probability distributions of exponential type and the (pre-)Frobenius manifold structures, lies in the delicate matter of choosing an appropriate system of coordinates. Indeed, the very possibility of perceiving this connection in an explicit and natural manner is contingent upon the identification of a privileged class of coordinates—ones that reflect, in their very definition, the intrinsic geometry of the underlying structures. The art, then, is not merely to introduce coordinates, but to do so in a manner that unveils (rather than obscures) the hidden algebraic and differential properties inherent in the space.  

\,

Thus, in certain cases, we are able to establish a deep and precise connection between the WDVV equations and the geometry of information. This, in turn, suggests that the study of probability distributions, when viewed through a geometric lens, is far richer than initially expected and might hold profound implications for both mathematical physics and information theory.  

\section{Summary and synthesis}
This introductory textbook is structured in three interrelated parts, each designed to build a solid foundation in modern mathematics and lead the reader toward the emerging field of information geometry.

\, 

\section*{\bf Part 1: Manifolds, Topology, and Geometry}
\, 

\subsection*{\bf Chapter 1: Foundations of General Topology}
\,

The book begins with a modern treatment of general topology, reinterpreting Kuratowski’s early ideas \cite{Ku72} in a contemporary framework. Readers are introduced to essential topological concepts—such as open sets, continuity, convergence, and compactness—establishing the language and tools necessary for later discussions.

\, 

\subsection*{\bf Chapter 2: Topological and Modelled Manifolds}~

Building on the groundwork of topology, the text moves into the realm of manifolds. Chapter 2 explains topological manifolds—spaces that locally resemble Euclidean space—while also presenting modelled manifolds based on S. Lang’s influential works \cite{L95,L99}. For additional reference, the text suggests consulting \cite{Mil65} to cover aspects not fully developed in this book. This chapter serves as a bridge, transitioning from abstract topological spaces to concrete geometric structures.

\,

\subsection*{\bf Chapter 3: Differentiability and Gateaux Derivatives}
Differentiability is explored next with a focus on Gateaux derivatives. This notion is particularly useful in infinite-dimensional settings, especially in probability theory where a norm may not be available. The text carefully contrasts Gateaux differentiability with the stronger concept of Fréchet differentiability, ensuring students understand the subtleties and applications of both approaches.
\,

\subsection*{\bf Chapter 4: Fiber Bundles}
Fiber bundles are introduced as indispensable tools in differential geometry. This chapter outlines the local-to-global perspective that fiber bundles offer, demonstrating how complex geometric structures can be understood by piecing together simpler, locally trivial components.

\,

\subsection*{\bf Chapter 5: Connections, Parallel Transport, and Sheafs}
Delving deeper into the geometric framework, Chapter 5 covers connections, parallel transport, and covariant derivatives—key concepts for understanding how geometric data evolves along a manifold. Classical references such as \cite{KoNo96} and \cite{Sik} provide further reading on these fundamental topics. Additionally, an introduction to sheafs is provided, offering a gentle entry point into their role in modern geometry, with \cite{KS90} recommended for a complete reference. This material lays the groundwork for applying geometric techniques within information geometry later in the book.

\,

\section*{\bf Part 2: Probabilities, Statistics, and Related Topics}
\,

\subsection*{\bf Chapter 6: Basics of Probability and Statistics}~

The second part shifts focus to probability theory and statistics. It introduces standard definitions, theorems, and methodologies, employing familiar examples—such as coin flipping—to illustrate key concepts. The chapter also discusses the Radon–Nikodym derivative, linking measure theory with probabilistic reasoning. References such as \cite{Bi,Pa67} for measure theory and \cite{Fe66} for statistics are suggested for further reading.

\,

\subsection*{\bf  Chapters 7 $\&$ 8: Categorical and Geometrical Structures in Probability}~

These chapters are particularly innovative, blending philosophy with mathematics. Beginning with a discussion inspired by Klein’s geometry and Plato’s ideas, the text presents a novel perspective on how categorical structures naturally arise in the study of probability distributions. By considering manifolds of probability measures and the role of Markov kernels, the reader is guided through a modern generalization of classical geometry into a probabilistic and categorical setting. Key references for this section include \cite{Am85,Am97,MoCh89,MoCh91-1,MoCh91-2}.

\,

\section*{\bf Part 3: Frobenius Manifolds and Information Geometry}
\, 

\subsection*{\bf  Chapters 9,10 $\&$ 11 and Beyond: Frobenius Structures in Modern Research}
The final part of the book centers on Frobenius manifolds—geometric structures that elegantly encapsulate the interplay between algebra and geometry, coming from 2D Topological Field Theory. Here, the text explains how Frobenius structures emerge naturally within the framework of information geometry. For foundational references on Frobenius manifolds and related topics, the book cites \cite{Du,Man99}.

\,

Chapter 11 also integrates cutting-edge research from 2020 onward \cite{CoMa,CoCoNen}, demonstrating the latest developments in the field. A notable highlight is the discussion of learning methods pioneered by researchers such as D. Ackley, G. Hinton, and T. Sejnowski \cite{AHS}, showing how deep learning techniques relate to the broader mathematical framework of information geometry.

\,

{\bf  Conclusion}
This textbook provides a concise, accessible, and modern overview of several core areas of mathematics—topology, geometry, probability, and category theory—culminating in the study of information geometry. Through careful exposition and a judicious selection of topics, it equips students with the necessary tools to explore this young and rapidly evolving branch of mathematics, bridging classical theory with modern research and applications.






