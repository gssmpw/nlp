

\chapter{Foundations in General Topology}
General topology, also known as \emph{point-set topology}, provides the foundational language for modern mathematics by studying the most fundamental properties of sets and their structures. Topology focuses on the intrinsic properties of spaces that remain unchanged under continuous transformations.

A topological space is a set 
$X$ equipped with a topology—a collection of subsets called open sets that satisfy specific axioms and ensuring consistency with notions of continuity and convergence. This framework generalizes many familiar mathematical spaces, including metric spaces and Euclidean spaces.

\, 

Let us mention some key concepts in general topology:

\begin{itemize}
\item Topological spaces: A set equipped with a topology that defines continuity and convergence.
\item Open and closed sets: Fundamental building blocks of topology that generalize intervals in real analysis.
\item Basis and sub-basis: Tools for constructing topologies using (minimal) generating sets.
\item Continuity: A function between topological spaces is continuous if the pre-image of every open set is open.
\item Homeomorphisms: Structure-preserving maps that define when two spaces are topologically equivalent.
\item Compactness: a property that allows to generalize the notion of a closed and bounded subset in an Euclidean space. 
\item Connectedness: A criterion telling whether a space can be separated into disjoint open sets.
Separation axioms: Conditions such as $T_0,T_1,T_2$ (Hausdorff), which control how points can be distinguished in a space.
  \end{itemize}
  
These fundamental concepts serve as the basis for more advanced areas of mathematics, including analysis, geometry, algebraic topology, and functional analysis.

For readers already familiar with general topology, this section can be skipped. However, for those who need a refresher, the following discussion will provide a structured and intuitive approach to these essential topics before moving on to more advanced material.

\section{Topological space}

\begin{definition} [{\bf Topological space}]

 Let $X$ be a set and denote by $\cP(X)$ the \emph{power set}, that is, the set of all its subsets. A topology on $X$ is a distinguished collection of subsets, $\cT \subset \cP(X)$, which we regard as specifying the admissible open sets. This collection satisfies the following axioms:
 \begin{enumerate}
\item The set itself and the empty set belong to the topology: \[\emptyset, X \in  \cT.\]
\item Any arbitrary union of open sets remains open: \[\forall\, \{\cU_{i}\}_{i\in I} \subset  \cT,\quad \bigcup\limits_{i\in I} \cU_{i} \in  \cT.\]
\item Any finite intersection of open sets remains open:  \[\forall \, \cU_{1},\dots,\cU_{N} \in \cT,  \quad \bigcap\limits_{i=1}^{N} \cU_{i} \in  \cT.\]
  \end{enumerate}
   \end{definition}
   
  Thus, a topology provides the language of continuity: it defines what it means for a function to be continuous without reference to distances, relying only on the structure of open sets. 
  
  \, 
  
   The elements of $ \cT$ are called the {\sl open} of the topology; the conditions (1), (2) and (3) form  the axioms of a topology.

\, 

To summarize, we can therefore also state this definition by saying that a topology is a collection of subsets of $X$, called open, which must verify that
\begin{enumerate}
\item the empty set and $X$ are open;
\item any union of open sets is an open;
\item A finite intersection of open sets is an open.
\end{enumerate}

We will sometimes write $(X,\cT)$ to specify that we are considering a set $X$ equipped with its topology $\cT$.

\subsection{Basis}
\begin{definition}[{\bf Basis}]
A basis $\cB$ for a topology $\cT$ is a family of elements of $\cT$ such that every $\cU\in \cT$ is the union of elements of $\cB$. 
\end{definition}
We then refer to this as a topology generated by $\fB$.


\noindent 

An equivalent definition can be given. A basis $\fB$ for a topology $\cT$ is a family of elements of $\cT$ such that for each $x\in X$ and $\cU\in \cT$, with $ x\in \cU$, there exists $\cB\in \fB$ such that $x\in \cB$ and $\cB\subset \cU$.

\begin{example}
A basis for the usual topology of $\bbR$ is provided by the set of open intervals $\{]a,b[ \, \mid\,  a<b, a,b\in \cR\}$.
\end{example}

An example of a topology widely used in practice is the metric topology.

\vspace{5pt}
\begin{definition}[{\bf Metric space}]~\label{D:EspMet} %(We refer to Reed-Simon 1 p.4)
A metric space is a set $M$ endowed with a notion of distance, that is, a function $d : M\times M \to \bbR^{+}$ that satisfies:
\begin{enumerate}
\item $d(x,x)\geq 0, \, \forall \, x\in M$.
\item $d(x,y)= 0 \iff x=y, \, \forall \, x,y\in M$.
\item $d(x,y)=d(y,x), \, \forall \, x,y\in M$.
\item $d(x,z)=d(x,y)+d(y,z), \, \forall \, x,y,z\in M$.
\end{enumerate}
The topology of $M$ is generated by the open balls $B_{r}(a)=\{x\in M \mid d(a,x)<r$\}.
\end{definition}
\begin{ex}\label{Ex:1}
The Cantor ternary set $\cC$
 is created by iteratively deleting the open middle third from a set of line segments. One starts by deleting the open middle third $(\frac{1}{3},\frac {2}{3})$ from the interval 
$[0,1]$, leaving two line segments: 
$[0,\frac {1}{3}]\cup [\frac {2}{3},1]$. One continues iterating the a similar procedure for the remaining line segments. The Cantor set is constituted from all points in the interval $[0,1]$ that are not deleted at any step in this infinite process.
Is the Cantor set a metric space?
\end{ex}
\begin{definition}[{\bf Neighborhood}]
\

\begin{enumerate}
\item A neighborhood of a point $x \in X$ is the set $\cN(x)$ containing an open set that contains the point $x$.
\item A neighborhood of a set $A \subset X$ is the set $\cN(A)$ containing an open set that contains $A$.
\end{enumerate}
\end{definition}

\subsection{Glossary of topologies}
\begin{definition}[{\bf Trivial and discrete topology}]

\

\begin{list}{$\triangleright$}{}
\item {\bf trivial topology}: The trivial topology\index{Topology! Trivial} on a non-empty set $X$ consists in taking as open sets the empty set $\emptyset$ and the entire space $X.$
\item {\bf Discrete topology}: The discrete topology\index{Topology! Discrete} on a non-empty set $X$ consists in taking as open sets the elements of $\cP(X)$ the set of subsets of $X.$
\end{list}
\end{definition}

\begin{definition}[{\bf Coarser and finer topology}]

\

\begin{list}{$\triangleright$}{}

\item {\bf Coarser topology} :\index{Topology!coarse} if $\cT_{1}$ and $\cT_{2}$ are two topologies on $X$ such that $$\cT_{1}\subset \cT_{2},$$ $\cT_{1}$ is said to be coarser than $\cT_{2},$
\item {\bf Fine topology} :\index{Topology!Fine} if $\cT_{1}$ and $\cT_{2}$ are two topologies on $X$ Such that $$\cT_{1}\subset \cT_{2},$$ then $\cT_{2}$ is said to be finer than $\cT_{1}$.
\end{list}
\end{definition}

\noindent $\bullet$ The coarsest topology is the trivial topology.

\noindent $\bullet$ The finest topology is the discrete topology.

\begin{definition}[{\bf Induced topology on a subset}]
Let $(X, \cT_{X })$ be a topological space and $A \subset X$ be a subset. We define a topology $ \cT_{A}$ on $A$ by setting:
\[
\cT_{A}=\{U\cap A\mid U\in \cT_{X}\} .
\]
\end{definition}

In other words, we take as open sets of $A$ the intersections of open sets of $X$ with $A$.

\begin{definition}[{\bf Quotient topology}]
Let $(X,\cT)$ be a topological space and $\cR$ be an equivalence relation on $X$. Let the map 
\[
p : X\to X/\cR,\ x\mapsto [x],
\]
 associate an element $x\in X$ with an equivalence class of $X$. The open sets of the quotient topology on $X/\cR$ are the subsets $\cV\subset X/\cR$ such that $\cV=p^{-1}(\cU)$, where $\cU\in \cT$. \end{definition}

\section{Separated space}
In the study of topological spaces, separation properties play a crucial role in understanding the structure of a space and the behavior of functions defined on it.
\begin{definition}[{\bf Hausdorff  space - $\mathbf{T_2}$}]\index{Space!Hausdorff } 

A topological space $(X,\cT)$ is called Hausdorff (separated or $\mathbf{T_2}$) if for any pair of distinct points ${\scriptstyle M}$ and ${\scriptstyle N}$, we can find two open sets $\cU_{M}, \cU_{N}$ with ${\scriptstyle M}\in\cU_{M} , {\scriptstyle N}\in\cU_{N}$ and $\cU_{M}\cap \cU_{N}=\emptyset$.
\end{definition}

\begin{definition}[{\bf Normal space}]\index{Space!normal}
 A topological space $(X,\cT)$ is normal if it is Hausdorff and if for any pair of disjoint closed sets $ F_{1}$ and $F_{2}$, there exist two disjoint open sets $\cU_{1}$ and $\cU_{2}$ such that $F_{1}$ is included in $\cU_{1}$ and $F_{2}$ in $\cU_{2}$.
\end{definition} 

\begin{theorem} A Hausdorff space $X$ is normal if and only if it satisfies the following condition:

\, 

For every closed subset $F\subset X$ and every open subset $\cU$ containing $F$, there exists an intermediate open subset $\cV$ containing $F$ satisfying %such that the adhesion of $\cV$ is included in $U$:
\[F\subset \cV\subset \overline \cV \subset \cU.\] 
\end{theorem}
Notice that here $\overline \cV $ is properly defined in Sec. \ref{S:topProp} under the terminology of \emph{closure} of a set: $\overline{\{p\}}=\{p\}$.  
\begin{example}
An open set $\cU\subset \bbR^{n}$ is a Hausdorff space and moreover a normal space.
\end{example}

\begin{definition}
    A topological space is ${\bf T_1}$ if for any two distinct points $x$ and $y$, there exist open sets $U$ and 
$V$ such that: 

\begin{itemize}
    \item $x\in U$, $y\notin U$,
    \item $y\in V$, $x\notin V$.
\end{itemize}
In other words, each point is closed (its complement is open).
\end{definition}





  


\begin{ex}\label{Ex:2}
Prove or disprove that any Hausdorff space is a $\mathbf{T_1}$ space. A $\mathbf{T_1}$ space is a space in which any set consisting of one point is closed. 
\end{ex}

$\star$  Warning! There exist ${\bf T_1}$ space that are not ${\bf T_2 }$
\begin{ex}\label{Ex:T_1neqT_2}
Give an example of a ${\bf T_1}$ space that is not ${\bf T_2 }$.
\end{ex}

 \section{Continuity}
\begin{definition}[{\bf Continuity}]\index{Application!Continuity}
Consider two topological spaces $(X,\cT),(X',\cT')$. A map $f:(X,\cT)\to,(X',\cT')$ is continuous if for any open set $\cU'\in \cT'$ its inverse image $g^{-1}(\cU')$ is an open set of the topology $\cT$.
\end{definition}

\begin{theorem}
\

\begin{enumerate}
\item A map $f:(X,\cT)\to(X',\cT')$ is continuous at the point $x_0\in X$ if for every neighborhood $\cN(f(x_0)) \subset X'$  there exists a neighborhood $\cN(x_0 )\subset X)$ of $x\in X$ such that $f(x) \in \cN(f(x_0))$, whenever $x \in\cN(x_0)$.

The map $f$ is continuous on $X$ if it is continuous at all $x\in X$.

\item A map $f$ is continuous if and only if the sequence
$\{f(x_{\beta})\}$ converges to $f(x)$ when the sequence $\{x_{\beta}\}$ converges to $x$.
\end{enumerate}
\end{theorem}

\begin{ex}\label{Ex:cont}
Show that for any pair of continuous functions $f,g:X\to Y$, where $X$ is a topological space $X$ and $Y$ is a Hausdorff space,
the set
    \[\{x\in X \,|\, f(x)=g(x)\} \]
    is closed in $X$. 
\end{ex}

\begin{definition}[{\bf Homeomorphism}]\index{Application!Hom\'eomorphism}
A homeomorphism between two topological spaces $(X,\cT_{X}), (Y,\cT_{Y})$ is a continuous bijective map $h :(X,\cT_{X})\to (Y,\cT_{Y})$ whose inverse map is continuous.
\end{definition}

\begin{definition}[{\bf Open map}]
A continuous map $f: X\to Y$ is called open if the image of any open set of $X$ is an open set of $Y$.
\end{definition}

\begin{definition}[{\bf Closed map}]
A map $f:X\to Y$ between topological spaces is said to be closed if the subset 
\[
\Gamma_f=\{(x,f(x)) \mid x\quad \text{lies in a domain of}\quad  f\}\subset X\times Y,
\] 
The set $\Gamma_f$ is closed in $X\times Y$. The set $\Gamma_f$ is called the \emph{graph} of $f$.
\end{definition}


\section{Topological properties of sets}\label{S:topProp}

\subsection{Open and closed sets}
We will recall some definitions and theorems related to various properties of subsets of a topological space $X$.

\begin{definition}

\

\begin{list}{$\triangleright$}{}

\item {\bf Closed subset} :\index{Set!Subset!Closed}
A set $F \subset X$ is closed if it is the complement of an open set.

\item{\bf Limit point}:\index{Topology! Limit point}
A point $x\in X$ is a limit point of $A \subset X$ if every neighborhood $\cN_{x}$ of $x$ contains at least one point $a\subset A$ different from $x$.

\item {\bf Closure}:\index{Set!Subset!Adherence} \index{Set!Subset!Closure}
The closure $\bar A$ of $A \subset X$ is the union of $A$ and all its limit points. It is the smallest closed set containing $A$.

\item{\bf Dense subset} :\index{Set!Subset!Dense}
The set $A \subset X$ is dense in $ X$ if $\bar A=X$.

\item{\bf Separable} :\index{Set!Separable}
A topological space $X$ is separable if it contains a countable dense subset .

\item {\bf Interior} : \index{Set!Subset!Interior}
The interior $\AA$ of the set $A \subset X$ is the largest open set contained in $A$.

\item {\bf Nowhere dense set} : \index{Set!Subset!Nowhere dense}\label{D:npd}

The set $A \subset X$ is nowhere dense in $X$ if
its adhesion $\bar A$ has an empty interior.
\end{list}
\end{definition}

\begin{definition}[{\bf Baire space}]\label{D:EBaire}
A topological space is called a Baire space if the intersection of any countable family of dense open sets remains dense.

Equivalently, a topological space is Baire if the union of any countable collection of closed sets with empty interior also has an empty interior.
\end{definition}
Complete metric spaces, as well as locally compact Hausdorff spaces, provide natural examples of Baire space. 
\begin{example}
\

\begin{list}{$\triangleright$}{}
\item The set of integers is nowhere dense in the set of real numbers equipped with the usual topology.

\item The set of real numbers whose decimal expansion only includes the digits $0$ or $1$ is nowhere dense in the set of real numbers.
\end{list}
\end{example}

\begin{ex}\label{Ex:3}
    Is the set of rational numbers equipped with the subspace topology a Baire space? Write a proof. 
\end{ex}

\begin{ex}\label{Ex:4} 
Is the set of real numbers equipped with standard topology a Baire space? Write a proof. 
\end{ex}
\begin{theorem}[{\bf Baire's Category Theorem}]
\

\begin{enumerate}
\item Every complete metric space is a Baire space.
\item Every locally compact topological space is a Baire space.
\end{enumerate}
\end{theorem}

\begin{theorem}
\

\begin{enumerate}
\item A subset $A \subset X$ is open if and only if it is a neighborhood of each of its points.

\item A subset $A \subset X$ is closed if and only if it contains all its limit points.
\end{enumerate}
\end{theorem}
 
\subsection{Compactness}
\begin{definition}
\

\begin{list}{$\triangleright$}{}
\item {\bf (Open) Covering}:\index{Set!Covering} A family $\cU$ of open sets $U _{\alpha}\subset X$ is an open covering \index{Set!Covering!Open} if $\displaystyle \bigcup_{\alpha} U_{\alpha}=\cU$.

\item {\bf (Open) Subcovering}: An open subcovering of a covering\index{Set!Covering!Subcovering} $\cU$ is a subset of $\cU$ which is itself an open covering.

\item {\bf Compact }:\index{Set!Covering!Compact} A subset $A \subset X$ is compact if it is Hausdorff
and each covering has a finite subcovering.

The condition of being Hausdorff  can be dropped  in a large number of situations.
\item {\bf Relatively compact}: \index{Set!Covering!Relatively compact}A subset $A \subset X$ is {\bf relatively compact} if its closure $\bar A$ is compact.

\item {\bf Locally compact}: \index{Set!Subset!Locally compact} A subset is locally compact if every point has a compact neighborhood.

Notice that, Euclidean spaces are locally compact but not compact.

\item {\bf Paracompact }: \index{Space!Paracompact} A Hausdorff space is paracompact if every covering $\cU=\{ U _{\alpha}\}$ has a locally finite covering $\cV=\{V _i\}$ such that, for every $V_i$ there exists a $U _{\beta}$ containing it $(V_i\subset U _{\alpha})$.

\item {\bf Compactification}:\index{Topology! Compactification} A compactification of a topological space $X$ is a pair $(h,K)$ where $K$ is a compact space and $h$ is a homeomorphism of $X$ onto a dense subspace of $K$.
\end{list}
\end{definition}

\begin{theorem}
\

\begin{enumerate}
\item A compact subspace of a Hausdorff space is necessarily closed.

\item Every closed subspace of a compact space is compact.

\item {\bf Bolzano-Weierstrass theorem}:\index{Topology!Theorem!Bolzano-Weierstrass} A Hausdorff space is compact if and only if every sequence has a convergent subsequence.

\item {\bf Heine--Borel theorem}\index{Topology!Theorem!Hein-Borel} The compact subsets of $\bbR^n$ are the closed bounded subsets of $\bbR^n$.
\end{enumerate}
\end{theorem}
The Heine-Borel theorem concerns finite-dimensional spaces, it is not necessarily true for an arbitrary topological space. In particular, a closed bounded set with a non-empty interior of an infinite-dimensional normed vector space is  never compact, for the norm topology.
 
\begin{theorem} 
Let $K$ be a compact space. The image of $K$ under a continuous map $f$ is also compact.
\end{theorem}

\begin{corollary} Given a compact space $K$, any continuous function on $K$ attains on $K$ a minimum and a maximum value.
\end{corollary}
\begin{ex}\label{Ex:5}
 Prove that  the Cantor set is compact.
\end{ex}
\begin{ex}\label{Ex:6}
 Consider the set $K$ of all functions $ f: [0, 1] \to [0, 1]$ equipped with the Lipschitz condition i.e. $|f(x) - f(y)| \leq |x - y|$ for all $x, y \in [0,1]$. Consider on $K$ the metric induced by the uniform distance 
\[d(f,g)=\sup_{x\in [0,1]}|f(x)-g(x)|.\] Prove that the space $K$ is compact.   
\end{ex}

 \subsection{Connectedness}
\begin{definition} [{\bf Connectedness}]\index{Space!Connectivity} A topological space $X$ is said to be connected if it cannot be described as the union of two disjoint (non-empty) open sets. Otherwise $X$ is said to be non-connected.

\, 

A subspace of a topological space is said to be connected if it is connected for the induced topology.
\end{definition}
\begin{ex}\label{Ex:7}
Is the General Linear group $GL_n(\bbR)$
(i.e. the space of square matrices of size $n\times n$ with non-null determinant) a connected space? Provide a proof.    
\end{ex}
\begin{theorem}%\index{Topological vector space!Main theorem}
For a topological space $X$ the following conditions are equivalent:
\begin{enumerate}
\item $X$ is connected;
\item $X$ cannot be the union of two non-empty closed sets;
\item The subsets of $X$ that are both open and closed are $X$ and the empty set $\emptyset$;
\item $X$ cannot be the union of two non-empty separable sets;
\item The continuous functions from $X\to \{0,1\}$ are the constant functions.
\end{enumerate}
\end{theorem}

\begin{definition} [{\bf Locally connected}]\index{Space!Locally connected}
A topological space $X$ is said to be locally connected if every neighborhood of every point $x \in X$ contains a connected neighborhood.
\end{definition}

\begin{examples}
~\begin{enumerate}
\item The Euclidean line, the Euclidean plane, the cube $I^{n}$ (without boundary) are connected and locally connected.
\item Any isolated point is a point of local connectivity.
\item The curve $y=\sin\frac{1}{x}$ on the interval $(0,1]$, is connected but not locally connected. This is because the set contains the point $(0,0)$ but it is not possible to connect the function to the origin.
\item Let $\bbR^{2}$ be equipped with the standard topology and let $K=\{\frac{1}{n} \mid n\in \bbN\}$. We call the comb space the set
\[C= (\{0\}\times[0,1])\cup (K\times[0,1])\cup ([0,1]\times\{0\})\]
considered as a subspace of $\bbR^{2}$ equipped with the induced topology. The comb is a connected space that is not locally connected.

\end{enumerate}

\end{examples}

\noindent $\star$ Warning! A space can be connected without being locally connected.

\begin{ex}\label{Ex:8}
 Is the curve $y=\sin\frac{1}{x}$ on the interval $(0,1]$ locally connected?
\end{ex}



\begin{ex}\label{Ex:9}
    Compute how many connected components has the real quartic surface: 
    \[f(u,v)=3.2u-v^2-\frac{1}{20}=0\]
     where $u=x^2+y^2+z^2,v=x^2y^2+y^2z^2+x^2z^2$.

\, 

Do the same for: 
\[f(u,v)=8u-v^2-\frac{1}{20}=0\]
\end{ex}


\begin{definition}[{\bf Covering space}]
The covering of a topological space $X$ is a pair $(\tilde X,f)$, where $\tilde X$ is a connected and locally connected space and $f$ is a continuous map from $\tilde X$ onto $X$, such that for every  neighborhood of $x$, $x\in \cN(x)$, the restriction of $f$ to each connected component $C_{\alpha}$ of $f^{-1}(\cN(x))$ is a homeomorphism from $C_{\alpha}$ to $\cN(x)$.
\end{definition}

\begin{definition}[{\bf Simply connected}]
A topological space $X$ is simply connected if $X$ is connected and locally connected and any covering $(\tilde X,f)$ is isomorphic to the trivial covering $(X,Id)$ where $Id$ is the identity map.
\end{definition}
  
 \begin{example}
\

\begin{list}{$\triangleright$}{}
\item $\bbR$ is simply connected (roughly speaking it has no "holes" and every loop is shrunk to a point).
\item The circle $S^{1}=\bbR/\bbZ$ is not simply connected.
\end{list}
\end{example}


A universal covering space is the "simplest" simply connected space that covers a given space. Rigorously, it is defined as follows.
\begin{definition}[{\bf Universal covering}]
$(\tilde X,f)$ is a universal covering of the space $X$ if it is a covering and if $\tilde X$ is simply connected.
\end{definition}
\begin{example}
  A covering of $S^{1}$ is given by the pair $(\bbR,\pi)$, where $\pi$ is the canonical projection given by \[\pi(t)=\exp{2\pi\imath t}\] and $t$ is a real parameter.    
\end{example}

\begin{ex}\label{Ex:10}
Does the torus $S^1\times S^1$ have a universal covering? If yes, give it explicitly. 
\, 

Demonstrate that the Kähler torus $T^n=\bbC^n/\Lambda$ in $\bbC^n$ where $\Lambda$ is a lattice in $\bbC^n$ (meaning that it is a discrete subgroup isomorphic to $\mathbb{Z}^{2n})$  has a universal covering space.
\end{ex}

\begin{definition}[{\bf Locally simply connected}]
A topological space $X$ is locally simply connected if every point $x\in X$ has at least one simply connected neighborhood.
\end{definition}
\begin{example}
    The sphere $S^2$ is locally simply connected, since every point on the sphere has a simply connected neighborhood. For every point on $S^2$ one can choose a small open neighborhood around that point, homeomorphic to an open disk in $\bbR^2$. 
\end{example}

\begin{ex}\label{Ex:11}
We consider a space which consists of infinitely many circles of decreasing radius, all tangent to a single point. This is called a Hawaiian earring. 
Prove that this space is not locally simply connected.
\end{ex}

 \begin{definition}[{\bf Isomorphism of covering}]
Two coverings $(\tilde X_{1},f_{1})$ and $(\tilde X_{2},f_{2})$ are isomorphic if
\begin{enumerate}
\item $\varphi: \tilde X_{1}\to \tilde X_{2}$ is a homeomorphism.
\item $f_{2}= f_{1}\circ \varphi$.
\end{enumerate}
\end{definition}

\begin{definition}[{\bf Fundamental group}]
Let $X$ be a space that admits a universal covering $(\tilde X,f)$. The group of homeomorphisms $\varphi$ of $\tilde X$ onto itself such that $f\circ \varphi = f$ is called the fundamental group of $\tilde X$.

Since two universal coverings are isomorphic, so are the fundamental groups. The corresponding abstract group is called the fundamental group of $X$.

\end{definition}
\begin{ex}\label{Ex:12}
Compute the fundamental groups of the following topological spaces:
\begin{itemize}
    \item Sphere $S^n$
    \item Torus $T^2$
    \item Real projective space $\bbR P^n$
    \item Hawaiian earring
\end{itemize}
\end{ex}
\begin{theorem}
\

\begin{enumerate}
\item A connected and locally connected space has a universal covering.
\item Two universal coverings are isomorphic.
\item Any covering of a universal covering is trivial. \end{enumerate}
\end{theorem}

\begin{definition}[{\bf Path-connected}]
\

\begin{enumerate}
\item A topological space $X$ is path-connected connected, if for any two points $a$ and $b$ in $X$, there exists a continuous path connecting those points; that is, there exists a continuous function $\gamma : [0,1] \to X$ such that $\gamma(0)=a$ and $\gamma(1)=b$.
\item $X$ is locally path-connected (or pathwise connected) if for each point $x\in X$ and each neighborhood $\cN(x)$ there exists a neighborhood $\cU(x)\subset \cN(x)$ that is path-connected.
\end{enumerate}
\end{definition}

\begin{theorem}
If a topological space is path-connected (locally path-connected) then it is connected (locally connected).
\end{theorem}


\, 

$\star$ Warning! The converse statement is false.

\, 

 \begin{example}
Consider the topologist's sine curve, which is constituted from the graph of the function  $y=sin(\frac{1}{x})$, where $x\in (0,1]$ and the vertical line segment $\{0\}\times [-1,1]$. 
The space is connected however it is not path-connected. 
Indeed, there is no continuous path connecting a point on the vertical segment to a point on the sine curve. In particular, any neighborhood around any point on $\{0\}\times [-1,1]$ contains infinitely many disconnected components of the sine curve. 
 \end{example}     
                       
   \begin{ex}\label{Ex:13}
    Provide an other example where the space is connected but not path-connected.  
\end{ex}

\section{Homotopy}
The notion of homotopy is a purely topological notion which allows to consider classes of topological objects up to some (homotopical) relation. 

To give an everyday life example, up to homotopy, a mug is equivalent to a doughnut (because they are both homotopy equivalent to a torus). 

\begin{definition}[{\bf Homotopic paths}]
Two paths $\gamma_{1}, \gamma_{2}:[0,1]\to X$ are homotopic if there exists a continuous map $F:[0,1]\times [0,1] \to X$ such that $F(t,0)=\gamma_{1}(t)$ and
$F(t,1)=\gamma_{2}(t)$. \end{definition}
\begin{example}
In the figure below, we draw a pair of homotopic planar curves. These curves are the level curves of the real and imaginary parts of a pair of complex polynomials.   
\begin{center}
\includegraphics[scale=0.5]{hom2.png}
\end{center}
Can you tell the degree of those curves?  Give (approximately) the equations of those polynomials.
\end{example}
\begin{definition}[{\bf Homotopic map}]

\

\begin{enumerate}
\item Two continuous maps $f,g:X\to Y$ are homotopic if there exists a continuous map $F:X\times [0,1] \to Y$ such that $F(x,0)=f(x)$ and
$F(x,1)=g(x)$ for all $x\in X$.

\item If $A$ is a subset of $X$, two continuous maps $f,g:X\to Y$ are homotopic with respect to $A$ if there exists a continuous map $F:X\times [0,1] \to Y$ such that $F(x,0)=f(x)$ and
$F(x,1)=g(x)$ for all $x\in X$ and $F(a,t)=f(a)$ for all $a\in A$ and $t\in [0,1]$.
\end{enumerate}
\end{definition}

\begin{theorem}
If $X$ is path-connected and locally path-connected, it is simply connected if every path $\gamma$ in $X$ is homotopic to the constant map.
\end{theorem} 

The notion of homotopy between two functions allows to define an equivalence relation between topological spaces.

\begin{definition}[{\bf Homotopically equivalent spaces}]
Two spaces $X$ and $Y$ are said to be homotopically equivalent (or `` of the same homotopy type '') if there exist two continuous maps $f:X\to Y$ and $g:Y\to X$ such that :
\begin{list}{$\bullet$}{}
\item $g\circ f $ is homotopic to $Id_{X}$, the identity on $X$ ;
\item $f\circ g $ is homotopic to $Id_{Y}$, the identity on $Y$ .
\end{list}
\end{definition}

\begin{definition}[{\bf Contractible space}]\label{D:contractible}
A space is called contractible if it is homotopically equivalent to a point.
\end{definition}
This definition is equivalent to saying that its identity map is homotopic to a constant map.

\begin{example}
The space $\bbR^{n}$ is contractible.
\end{example}

\begin{theorem}
Two homeomorphic topological spaces are homotopically equivalent
\end{theorem}

$\star$ Warning! The converse is false, as shown by the following examples:

\begin{list}{$\triangleright$}{}
\item Except  from the point itself, contractible spaces are not homeomorphic to the point.
\item A circle is homotopically equivalent to $\bbC^{\star}$, that is, a plane without a point, but it is not homeomorphic to it. A circle without two points is not connected, while a plane without three points is.
\item The real interval, the disk in the plane or the ball in $\bbR^{3}$, of radius 1 (open or closed) are all  contractible, therefore homotopically equivalent. However,  pairwise they are non-homeomorphic (by analogous arguments).
\end{list}


\section{Disjoint sum topology}

\subsection{Disjoint sum}
\begin{definition}[{\bf Disjoint sum}]\label{somdisj}

The disjoint sum of an indexed family of sets $\{E_i\}_{i\in I}$ is defined
by:
\begin{equation}
\coprod_{i\in I}E_{i}=\bigcup_{i\in I} \{(x,i) \mid\, i\in I, \,x\in E_i.\}
\end{equation}
\end{definition}

\noindent {\bf Note}: This definition of the disjoint sum allows to take into account the case where the $E_{i}\cap E_{j}\ne \emptyset$. Indeed, the elements of the disjoint sum are ordered pairs $ (x,i)$. Here $i\in I$ serves as an auxiliary index that indicates from which $E_i$ the element $x$ came from.  
Each of the $E_i$ sets  are canonically isomorphic to the set $\tilde E_i=\{(x,i): x\in E_i\}$. 
So, if we define $\varphi_j:E_j \to \coprod_{i\in I}E_i$ by $\varphi_j(x)=(x,j), $ where $j\in I$ then $ \varphi_{j}$ is a bijection from $E_j$ onto $\tilde E_j=\{(x,i)\mid x\in E_j\}$ and the $\tilde E_j$ are  disjoint, even if the $E_i$ are not.  
\[
\coprod_{i\in I}E_{i}=\bigcup_{i\in I} \{(x,i) \mid\, i\in I, \,x\in E_{i}\}=\bigsqcup_{i\in I} \{(x,i) \mid\, i\in I, \,x\in E_{i}\}.
\] 
Usually to state that the union is a disjoint union we replace $\bigcup$ by $ \bigsqcup$.

\subsection{Disjoint sum topology}%(Ronga Geo II) it $(\{E_{i},\cT\}_{i\in I}$ a family of topological space, we propose to put a topology on $\coprod_{i\in I}E_{i}$ making the $\varphi_{i}$ continuous.

 \begin{definition}[{\bf Disjoint sum topology}] The disjoint sum topology on $\coprod_{i\in I}E_{i}$ is defined by 
 \[
 \cT= \left\{\cU\subset \coprod_{i\in I}E_{i} \mid \varphi_{i}^{-1}(\cU)\in \cT_{i} \ \forall \ i\in I. \right\}.
 \]
  \end{definition} 

\begin{proposition}[Fundamental properties of the disjoint sum topology] 

\,

\begin{enumerate}
 \item The set $ \left\{\cU\subset \coprod_{i\in I}E_{i} \mid \varphi_{i}^{-1}(\cU)\in \mathcal{T}_{i} \ \forall \ i\in I \right\}$ defines a topology.

\item[]

\item The maps $\varphi_{j}: X_{j}\to \coprod_{i\in I}E_{i}$ are continuous, open and closed; they are in fact homeomorphisms on
\[\varphi_{j}(X_{j})=\left\{(x,j)\in \coprod_{i\in I}E_{i}\mid x\in E_{j}\right\}.\]

\item[]
\item Let $Y$ be a topological space and $f: \coprod_{i\in I}E_{i}\to Y$ be a map. Then
\[ f\quad \text{is continuous }\quad  \Leftrightarrow f\circ \varphi_{j}:E_{j}\to Y \text{ is continuous }, \forall\, j\in I.\]

\item[]
\item The disjoint sum topology is the finest among those for which the $\varphi_{j},$ for all $\ j\in I$ are continuous.
\end{enumerate}
\end{proposition}


\begin{proof}
\begin{enumerate}
\item It is clear that $\emptyset$ and $\coprod_{i\in I}E_{i}$ belong to $\cT$.

\, 

Axioms (2) and (3) are a consequence of
the fact that the $\cT_{i}$ are topologies and of the following two set-theoretic properties of the operation $\varphi_{j}^{-1}$:

\vspace{5pt}
\begin{enumerate}[a)]
\item if we have a family $\{\cU_{\lambda}\}_{\lambda\in\Lambda}$ of subsets of $\coprod_{i\in I}E_{i} $, then:
\[
\varphi_{j}^{-1}\left(\bigcup_{\lambda\in\Lambda}\cU_{\lambda}\right)=\bigcup_{\lambda\in\Lambda} \varphi_{j}^{-1}(\cU_{\lambda}).
\]

 \vspace{3pt} 
\item if $\cU_{1},\dots,\cU_{N} \subset \coprod_{i\in I}E_{i}$, \[ \varphi_{j}^{-1}(\cU_{1} \cap \dots \cap \cU_{N})=\varphi_{j}^{-1}(\cU_{1} )\cap\dots \cap\varphi_{j}^{-1}(\cU_{N}).\] 
\end{enumerate}

 \vspace{5pt}
 \item By the above statement, the $\varphi_{i},$ where $ i \in I$ are continuous. If $\cU \subset E_{i}$ is open, let us see that $\varphi_{i}(\cU)$ is open. Indeed : 
 \[
 \varphi_{j}^{-1}(\varphi_{i})=\begin{cases} \cU& \text{ if } i=j\\ \emptyset & \text{ otherwise }\end{cases}
 \]
  and therefore it is an open set for the topology that we defined on $ \coprod_{i\in I}E_{i}$.


If $F\subset E_{j}$ is closed, then $\left(A=\coprod_{i\in I}E_i\right)/\varphi_j(F)$. This implies 
\[\varphi_i^{-1}(A)=\begin{cases} E_j/F & \text{ if } i=j\\ E_i & otherwise
\end{cases}\]
and therefore $A$ is open for the topology that we have defined on $\coprod_{i\in I}E_i$, which makes $\varphi_j(F)$ closed. It follows from previous arguments that the $\varphi_{j}$ form homeomorphisms on their image.

\vspace{5pt}
\item If $f: \coprod\limits_{i\in I}E_i\to Y$ is continuous for all $j \in I$, then the composition $f \circ \varphi_j$ is also continuous.

Suppose that $\varphi_j\circ f$ is continuous forall $j\in I$ and that $V \subset Y$ is an open set. Then $ \varphi_j^{-1}(f^{-1}(V))$ is open in
$E_j $, for all $ j \in I$, which means that $f^{-1}(V)$ is open in $ \coprod_{i\in I}E_i$. So, $f$ is continuous.

\vspace{5pt}
\item Let $\cT'$ be a topology on $\coprod_{i\in I}E_i$ for which the $\varphi_j$ are continuous for all$j \in I$. If $\cU\in \cT'$ we therefore have that $\varphi_j^{-1}(\cU')$ is open in $ E_j$, for all $j \in I$, which means that $U' \in \cT$. We thus have $\cT' \subset \cT$, that is to say that $\cT'$ is less fine than $\cT$.
\end{enumerate}
\end{proof}


\section{More Exercises}

\begin{enumerate}

\item Prove or disprove that every Hausdorff space $\mathbf{T_1}$ that is a space in which every set consisting of one point is closed.\\

\item Give an example of $\mathbf{T_1}$ space which is not Hausdorff space.\\

\item For each pair $f,g$ of continuous maps of a topological space $X$ in a Hausdorff space $Y$ the set 
$\{ X\in X: f(x)=g(x)\}$ is closed in space $X$
\end{enumerate}





\section{Hints/ short answers for exercises}

Ex. \ref{Ex:1} $\&$ \ref{Ex:5} The Cantor set is a metric space. 
The Cantor set is an uncountable set with Lebesgue measure zero. As the complement of a union of open sets, it is a closed subset of the real numbers and, consequently, a complete metric space. 
Furthermore, since it is totally bounded, the Heine–Borel theorem ensures that it is compact.
\, 

Ex.\ref{Ex:2}
Let $p$ be a given point of the space $X$. By hypothesis $X$ is a ${\bf T}_2$ space. 
Therefore, it follows that every point $x\neq p$ belongs to an open set $G_x$, that does not contain $p$ and thus that \[X\setminus\{p\}=\cup_{x\neq p}G_x.\] This implies that $X\setminus\{p\}$ is open, in other words $\{p\}$ is closed.

\, 

Ex. \ref{Ex:T_1neqT_2}  Consider the following topological space:
\[
X = \bigg\{ 0 \bigg\} \cup \bigg\{ \frac{1}{n} \;\bigg|\; n \in \mathbb{N}^* \bigg\}.
\]
We equip $X$ with a topology defined as follows:
\begin{itemize}
    \item The open sets that do not contain the point $\{1\}$ are precisely those inherited from the standard topology on $\mathbb{R}$. For example $\{\frac{1}{2},\frac{1}{3}\}$ is open.
    \item Any open set that contains the point $\{1\}$ must necessarily be of the form $U=X\setminus F$, where $F$ is a finite subset of $X$ not containing 1. For example, $U=X\setminus \{\frac{1}{2},\frac{1}{3}\}$.
\end{itemize}

Let us check that $X$ is a ${\bf T_1}$ Space. To show that $X$ is ${\bf T_1}$, we need to prove that every singleton set $\{x\}$ is closed. 
Consider:
\begin{itemize}
    \item For $x=0$: the space $X\setminus\{0\}$ is open. 
    \item For $x=\frac{1}{k}$: the space $X\setminus\{\frac{1}{k}\}$ is open because it can be written as $X\setminus F$, where $F=\{\frac{1}{k}\}$ is finite. 
\end{itemize}
Therefore, every singleton is closed, and $X$ is ${\bf T_1}$.

Let us check if $X$ is ${\bf T_2}$. 
\begin{itemize}
    \item Any open set $U$ containing 0 must be of the form $X\setminus F$, where $F$ does not contain 0. Since $X\setminus F$ is infinite, 
$U$ must contain infinitely many points of the form $\frac{1}{n}$.
\item Any open set $V$ containing 1 must also be of the form $X\setminus F$, where $F$ is finite and does not contain 1. This set $V$ will also contain infinitely many points of the form $\frac{1}{n}$. Thus $U$ and $V$ cannot be disjoint because they both contain infinitely many points $\frac{1}{n}$. Therefore, 0 and 1 cannot be separated by disjoint open sets and $X$ is not ${\bf T_2}$.
\end{itemize}

To summarize,
\begin{itemize}
    \item the space $X$ is ${\bf T_1}$, because every singleton is closed.
    \item The space $X$ is not ${\bf T_2}$,
 because the points 0 and 1 cannot be separated by disjoint open sets.
 \end{itemize}
 
\, 

Ex.\ref{Ex:cont} is is enough to show that the set $$A=\{x\in X: f(x)\neq g(x)\}$$ is open. 
For every $x\in A$ there exists in $Y$ two open sets $U$ and $V$ such that $f(x)\in U$ and $g(x)\in V$ and their intersection is empty. 
The set given by $f^{-1}(U)\cap g^{-1}(V)$ is the neighborhood of the point $x$ and contained in $A$. Therefore, $A$ is an open set. 

Ex.\ref{Ex:3} Let us consider the example of a non-Baire space: the set of rational numbers with the subspace Topology.

The set of rational numbers is countable. In particular, we can write 
\[\bbQ =\cup_{q\in \bbQ}\{q\}.\] Each singleton $\{q\}$ is closed in $\bbQ$ (subspace topology) but has empty interior: thus, $\{q\}$ is nowhere dense. Therefore, $\bbQ$ is a countable union of nowhere dense sets and $\bbQ$ cannot be a Baire space. 

\, 

Ex.\ref{Ex:4} The real numbers $\bbR$, equipped with the standard Euclidean topology, forms a Baire space. 

A space is Baire if the intersection of countably many dense open subsets is dense.
By the Baire Category Theorem, every complete metric space is a Baire space. Therefore, the conclusion follows.

\, 


Ex.\ref{Ex:6} This is an application of the Arzela--Ascoli theorem.
\, 

Ex.\ref{Ex:7} No. This is due to the fact that $GL_n(\bbR)$ splits into two disjoint subsets: the set of matrices with strictly positive determinant and the set of matrices with strictly negative determinant. 

\, 

Ex.\ref{Ex:8} The curve $y=\sin\frac{1}{x}$ on the interval $(0,1]$ is connected but not locally connected nor arc-connected. 
As $x\to 0^+$, $\lim_{x\to 0^+}\frac{1}{x}$ grows without bound, causing $\sin\frac{1}{x}$ to oscillate infinitely between the values -1 and 1. By definition, a space is locally connected if every neighborhood of any point contains a connected open neighborhood.
In the case of $y=\sin\frac{1}{x}$, no matter how small a neighborhood $U$ of $(0,y)$ is chosen, the graph within 
$U$ consists of infinitely many disconnected components (due to the oscillations). Thus, there is no connected open neighborhood around $(0,y)$.

\, 


Ex.\ref{Ex:9} The first quartic has eight non-compact connected components, while the second has only one. This can be shown by leveraging the fact that these quartics are invariant under the Coxeter group $CB_3$, which corresponds to the symmetries of the cube. By decomposing space into Coxeter chambers, the quartic can be analyzed within a single fundamental domain. Applying an appropriate change of variables, $x_i^2\mapsto X_i$ where $i\in \{1,2,3\}$ within this domain reduces the problem to studying degree-2 surfaces, defined in a cone contained in a positive octant (for instance the cone delimited by the equations 
$\{x_1=0\},\{x_2=0\},\{x_3=0\}$ and $\{x_i=x_j\}$ where $i\neq j$. 
Since the classification of quadrics is well understood, this approach simplifies the analysis. See \cite{C18} for an entire classification of such quartics. 

\, 

Ex.\ref{Ex:10} The universal covering space of the torus 
 is the Euclidean plane 
$\bbR^2$, with the covering map: $ \pi:\bbR^2\to T^2,$ 
$ \pi(x,y)=(\exp{2\pi\imath x},\exp{2\pi\imath y})$.

\, 

The Kähler torus (that is a torus equipped with a Kähler structure) admits a universal covering. In fact, the universal covering of a Kähler torus is the same as the universal covering of a standard torus, because the Kähler structure does not affect the underlying topology.
The map $\pi$ projects each point $z\in \bbC^n$ to its equivalence class 
$z+\Lambda$ in the quotient space $\bbC^n/\Lambda$.
This is a smooth, holomorphic map that respects the complex and Kähler structures.
The space $\bbC^n$ is simply connected because it is contractible.

\, 

Ex.\ref{Ex:11} The Hawaiian earring is a classic example of a space that is not locally simply connected. It consists of infinitely many circles of decreasing radius, all tangent to a single point. While each circle is itself simply connected, the entire space fails to be locally simply connected at the point of tangency because any neighborhood of that point contains infinitely many circles, making it impossible to find a simply connected neighborhood.

\, 

Ex.\ref{Ex:12} Fundamental groups. 
\begin{itemize}
    \item $\pi_1(S^1)\cong \bbZ$
    \item $\pi_1(S^n)\cong 0$, if $n\geq 2$
    \item $\pi_1(S^1\times S^1)\cong \bbZ\times \bbZ$
    \item $\pi_1(\bbR P^1)\cong \bbZ$
    \item $\pi_1(\bbR P^n)\cong \bbZ/2\bbZ$
    \item Hawaiian earring: uncountable. 
\end{itemize}

\, 

Ex.\ref{Ex:13} There are many examples. We can take the example of the deleted comb space. The comb space is a subspace of $\bbR^2$ which looks like a comb. A comb space is defined by the set: 
\[\{0\} \times [0,1] ) \cup (K \times [0,1]) \cup ([0,1] \times \{0\}\]
where $K=\{\frac{1}{n}\, |\, n\in\bbN^*\}$.
However, the aim of this exercise is to create your own example, using your imagination. 
















