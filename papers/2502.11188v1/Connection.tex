
\chapter{Connections, Parallel Transport and Sheafs} 

In differential geometry, connections and parallel transport arise from a fundamental necessity: the ability to compare vectors at different points on a curved space, where a naive translation is no longer well-defined.

\begin{minipage}{0.5\textwidth} 
  \centering
  \includegraphics[]{mountain.png}

\end{minipage}

In Euclidean space, one can simply
compare vectors via a parallel transport--shifting,
without change in magnitude or direction.

However, on a curved manifold, the very act of moving a vector must be prescribed by an additional structure, which is provided by a connection.

A connection defines a rule for differentiation that respects the manifold’s geometry, replacing the notion of partial derivative in a given direction with a well-defined covariant derivative. This operator tells us how a vector field changes as we move along the manifold, encoding information about curvature and torsion.

But the true power of connections emerges in the concept of parallel transport. Given a connection, we can transport a vector along a curve, according to the manifold’s structure. This operation is at the heart of many geometric investigations.

In Riemannian geometry, parallel transport reveals the curvature of space, since moving a vector along different paths can lead to different final results. In fact, given a loop $\gamma:[a,b]\to \cM$ where $[a,b]\subset \bbR$ on a smooth manifold $\cM$, where $\gamma(a)=\gamma(b)$ we can transport a vector $\Vec{v}$ along that curve,  starting at the initial point $\gamma(a)$. 

The question that emerges is whether there has been any modification implied on the vector $\Vec{v}$, during this transport, namely does the vector at $\gamma(b)$ coincide with the vector $\Vec{v}$ at $\gamma(a)$ or not. If those vectors coincide then the manifold on is flat on the set where the loop was defined.



In gauge theory, parallel transport describes how fields interact, with connections playing the role of gauge potentials in Yang-Mills theory.

In general relativity, the Levi--Civita connection governs the motion of free-falling observers in curved spacetime.

Thus, a connection is not merely a technical tool but an essential ingredient of geometry, allowing us to define differentiation, curvature, and transport in a way that transcends local coordinates. It is the bridge between the infinitesimal and the global, between algebra and topology, between abstract geometry and the laws of nature.

\section{Linear connection}\index{Connection!Linear}

In Euclidean space,  two vectors of different origin are compared by a parallel translation of the vectors to the one same origin. The derivative of a vector $\Vec{v}$ defined along a curve, is a vector of components $(\partial v^{i}/\partial x^{j}) dx^{j}/dt$ in the Cartesian coordinates. On an arbitrary differentiable manifold, the components  $$\partial v^{i}/\partial x^{j}$$ do not behave as the components of a tensor under a change of coordinates. To solve this difficulty we introduce the notion of  covariant derivative which is of tensorial type. 


\begin{definition}[Linear connection]\label{lincon}
A linear connection on a differentiable manifold $\cM$ is a mapping $\nabla : X\mapsto \nabla X$ from the vector fields on $\cM$ into the differentiable tensor field of type $(1,1)$ on $\cM$ such that
\begin{equation}\begin{aligned}
\nabla(X+Y)&=\nabla(X)+\nabla(Y),\\
\nabla(fX)&=df\otimes X+f\nabla(X), 
\end{aligned}\end{equation}
where $f$ is a differentiable function~\footnote{To give a more precise definition we have to speak  of germ of vector (or tensor) field and germ of function on $\cM$. } on $\cM$.

\vspace{3pt}
The tensor $\nabla X $ is called the \emph{covariant derivative}  of $X$.
\end{definition}


For a $n$-dimensional differentiable manifold $\cM$, equipped with moving a frames (Defintion \ref{D:Movfram}) $\{e_{i}\}_{i=1}^{n},\ e_{i}\in T(\cM)$  and 
$\{\varepsilon^{{i}}\}_{i=1}^{n},\ \varepsilon^{i}\in T^{\star}(\cM)$ (its dual), 
we have that $\nabla e_{i}$ is a $(1,1)$-tensor field which can be written as: 

\begin{equation} 
\nabla e_{i}(e_{k},\varepsilon^{j})=\gamma^{j}_{ki}, \end{equation}
and the $(1,1)$-tensor $\nabla e_{i}$ can be expressed as:
\begin{equation}\label{E:concoef0}
\nabla e_{i}=\gamma^{j}_{ki} \varepsilon^{k}\otimes e_{j},
\end{equation}

\begin{definition}[Connection coefficients]\index{Connection!Coefficients}
The coefficients $\gamma^{j}_{ki}=\nabla e_{i}(e_{k},\varepsilon^{j})$ are called the connection coefficients in the basis $\{e_{i}\}_{i=1}^{n}$.
\end{definition}

It follows from the definition~\ref{lincon} and the equation~\eqref{difffunc} that 
\[\begin{aligned}
\nabla X &= \nabla(X^{i}e_{i})=dX^{i}\otimes e_{i}+X^{i}\nabla(e_{i})\\
&=\big(dX^{i} +\gamma^{i}_{kj} X^{j}\varepsilon^{k}\big) \otimes e_{i}\\
&=\big(e_{k}(X^{i}) +\gamma^{i}_{kj}X^{j} \big) \varepsilon^{k}\otimes e_{i}\\
&=\nabla_{k}X^{i}\varepsilon^{k}\otimes e_{i},
\end{aligned}\]
where  $\nabla_{k}X^{i}=\big (\nabla X \big)^{j}_{k}$ denote the component  of the tensor $\nabla X$.
\begin{equation}\label{concoef1}
\nabla X = \nabla_{k}X^{i}\varepsilon^{k}\otimes e_{i},\quad \text{ with } \quad\nabla_{k}X^{i}=e_{k}(X^{i}) +\gamma^{j}_{ki}X^{i}.
\end{equation}

Furthermore, one can rewrite the $(1,1)$-tensor $\nabla_X$ in the following way:
\begin{equation}\label{concoef2}
\nabla X 
=\big(dX^{i} +\omega^{i}_{j} X^{j}\big) \otimes e_{i},\quad  \omega^{i}_{j}=\gamma^{i}_{kj} \varepsilon^{k}
\end{equation}
making the covariant part explicit.
\begin{definition}[Connection forms]\index{Connection!Connection forms}
The quantities defined by 
\begin{equation}\label{E:conform}
\omega^{i}_{j}=\gamma^{i}_{kj} \varepsilon^{k} .
\end{equation}
are called the connection 1-forms.
\end{definition}
We have the relation
\begin{equation}
\omega^{i}_{j}(e_{k})=\gamma^{i}_{kj} =
\nabla e_{i}(e_{k},\varepsilon^{j}).
\end{equation}
As $\nabla X$ is a $(1,1)$-tensor it is invariant under the change of natural coordinate system. In particular:
\[e_{i}= A_{i}^{j}e'_{j},\quad {\varepsilon}^{k}=  A_{\ell}^{k} {\varepsilon'}^{\ell},\quad  A^{\ell}_{i}A_{\ell}^{j}=\delta _{i}^{j},\]
and \[ X^{i}=A^{i}_{j}{X'}^{j}\quad  \text{ implies }\quad   dX^{i}=A^{i}_{j}d{X'}^{j}+{X'}^{j}dA^{i}_{j}.\]
It follows from \eqref{concoef2} that
\[\nabla X=\big(dX^{i} +\gamma^{i}_{kj} X^{j}\epsilon^{k}\big) \otimes e_{i}=\big(d{X'}^{j} +{\gamma'}^{j}_{k\ell} {X'}^{\ell}{\varepsilon'}^{k}\big) \otimes e'_{j}.\]
Moreover,
\[\begin{aligned}
\nabla X&=\big(dX^{i} +\gamma^{i}_{kj} X^{j}\varepsilon^{k}\big) \otimes e_{j}=\big(dX^{i} +\gamma^{i}_{kj} X^{j}\varepsilon^{k}\big) \otimes A^{k}_{i}e'_{k}\\
&=\big(A^{i}_{j}d{X'}^{j}+ {X'}^{j}dA_{j}^{i}+A^{j}_{\ell}{X'}^{\ell}{\gamma}^{i}_{kj}A^{k}_{m} {\varepsilon'}^{m}\big) \otimes A^{k}_{i}e'_{k}\\
&=\big(d{X'}^{j}+ A^{j}_{k}{X'}^{\ell}dA_{\ell}^{k}+A^{m}_{\ell}{X'}^{\ell}{\gamma}^{i}_{km}A^{j}_{i}A^{k}_{n} {\varepsilon'}^{n}\big) \otimes e'_{j}\\
&=\big(d{X'}^{j}+ (A^{j}_{k}dA_{\ell}^{k}+A^{m}_{\ell}{\gamma}^{i}_{nm}
A^{j}_{i}A^{n}_{k} ){X'}^{\ell}{\varepsilon'}^{k}\big) \otimes e'_{j}\\
\end{aligned}\]
and therefore
\[
{\gamma'}^{j}_{h\ell}=A^{j}_{k}dA_{\ell}^{k}+A^{m}_{\ell}{\gamma}^{i}_{nm}
A^{j}_{i}A^{n}_{h}
\]
which, by equation~\eqref{E:difcoord}, gives us
\begin{equation}\label{E: tconform}
{\gamma'}^{j}_{h\ell}=A^{j}_{k}e'_{h}(A_{\ell}^{k})+A^{n}_{h}A^{m}_{\ell}A^{j}_{i}{\gamma}^{i}_{nm}.
\end{equation}
The connection coefficients are not components of a tensor because of the first term on the right hand side of the equation \eqref{E: tconform}.
\vspace{5pt}

In the  local coordinate system $\{x^{1},\dots,x^{n}\}$, we have:
\begin{equation}\label{E:coeffconloccoor}
\nabla X= \left( \frac{\partial X^{i} }{\partial x^{k}}+\Gamma^{i}_{k\ell}X^{\ell} \right) d x^{k} \otimes \frac{\partial}{\partial x^{i}},
\end{equation}
where  $\Gamma^{i}_{k\ell}X^{\ell}$ is the connection coefficients \eqref{E:concoef0} in the local (natural) coordinate system $\{x^{1},\dots,x^{n}\}$. For the vector field $ \partial_i=\frac{\partial}{\partial x^i}$ the components are all $0$ except for $X^i=1$, and we can express: 
\[
\nabla(\partial_i)= \Gamma^{j}_{ki} dx^k\otimes \partial_j
\]

\begin{definition}[Christoffel symbols]\index{Connection!Christoffel symbols}
The $\Gamma^{j}_{ki}$ are called the Christoffel symbols.
\begin{equation}
\Gamma^{i}_{kj} =
\nabla(\partial_{i})(\partial_{k},dx^{j})=\omega^{j}_{i}(\partial_{k}).
\end{equation}
\end{definition}

\,   

Now, the Christoffel symbols transform under a change of local coordinates system: $\{x^{1},\dots,x^{n}\}\to \{{x'}^{1},\dots,{x'}^{n}\}$ as
\begin{equation}
{\Gamma'}^{j}_{h\ell}= \frac{\partial{x'}^{j}}{\partial x^{i}}  \frac{\partial}{\partial x^{k}} \left(  \frac{\partial{x'}^{i}}{\partial x^{\ell}}\right) +     \frac{\partial x^{m}}{\partial {x'}^{\ell}} \frac{\partial x^{k}}{\partial{x'}^{h}} \frac{\partial{x'}^{j}}{\partial x^{i}} \Gamma^{i}_{k m}.
\end{equation}
Hence, the Christoffel symbols does not define a tensor.

\section{Covariant derivative  in the direction $Y$}\index{Covariant derivative}
Let $X$ and $Y$ be two vector fields. By~\eqref{E:coeffconloccoor}, we get $\nabla X(Y)=Y^{k}\nabla_{k}X^{i}e_{i}=Y^{k} \nabla_{k}X$ which can be identified to the  derivative~\eqref{E:dirder} in the $Y$-direction  

\subsection{\bf Covariant derivative of a vector field in the direction of $Y$}
Let $X$ and $Y$ be two vector fields. By~\eqref{E:coeffconloccoor}, they can be expressed as $$\nabla X(Y)=Y^{k}\nabla_{k}X^{i}e_{i}=Y^{k} \nabla_{k}X,$$ which can be identified to the  derivative~\eqref{E:dirder} in the direction of $Y$.  This leads us to the formulation of a definition of the covariant derivative. 

\begin{definition} [Covariant derivative in the direction $Y$] 
The covariant derivative $\nabla_{_{Y}}X$ of $X$ in the direction $Y$ is defined by
\begin{equation}
\nabla_{_{Y}}X=(\nabla(X))(Y)    \text{ or } \nabla_{_{Y}}X(\cdot)=(\nabla(X))(Y,\cdot) 
\end{equation}
\end{definition}

The covariant derivative $\nabla_{_{Y}}X$ is  \emph{linear} in $Y$
\[
\nabla_{{fY_{1}+gY_{2}}}X=(f\nabla_{_{Y_{1}}}+g\nabla_{_{Y_{2}}})X, \] where \[Y_{i},Y_{2}\in T_{_{M}}(\cM),\quad \text{and}\quad f,g : \cM\to \bbR.
\]

In the frame $\{e_{i}\}_{i=1}^{n}$, we obtain: 
\[
\nabla_{_{Y}}X=Y^{k}\left(e_{k}(X^{i}) + \gamma^{i}_{kj}X^{j}\right)e_{i}=\left(Y(X^{i}) + \gamma^{i}_{kj}Y^{k}X^{j}\right)e_{i},
\]
and thus 
\begin{equation}
\nabla_{e_{i}}e_{k}=  \gamma^{j}_{ik}e_{j},
\end{equation}
where $\nabla_{e_{i}}$ denote the covariant derivative  in the direction $e_i$.

\subsection{\bf The Covariant derivative of a tensor}

\, 

The answer to this question is yes. The notion of a covariant derivative in the  direction of $X\in \cT_{_{M}}(\cM)$ can be extended---in a neightborhood of $M\in \cM$--- to an arbitrary type of tensor field, under the following assumptions: 
\begin{subequations}\label{E:cdt}
\begin{gather}
 \nabla_{X}f=X(f),\quad f\in C^{1}( \cM),\label{E:cdt1}\\
 \nabla_{X}(t+ s)= \nabla_{X}t +\nabla_{X}s,\label{E:cdt2}\\
\nabla_{X}(t\otimes s)= (\nabla_{X}t)\otimes s+ t\otimes (\nabla_{X}s),\label{E:cdt3}\\
\nabla_{X} \text{  commutes with the contracted multiplication.}\label{E:cdt4}
\end{gather}
\end{subequations}
 %\begin{enumerate}
%\item $\nabla_{X}f=X(f),\quad f\in C^{1}( \mathcal{M})$,\\
%\item $ \nabla_{X}(t+ s)= \nabla_{X}(t) +\nabla_{X}( s)$,\\
%\item  $\nabla_{X}(t\otimes s)= \nabla_{X}(t)\otimes s+ t\otimes \nabla_{X}(s)$,\\
%\item $\nabla_{X}$  commute with the contracted multiplication.
%\end{enumerate}

Therefore, if $t$ is a $(p,q)$-tensor its covariant derivative forms a $(q+1,p)$-tensor:
\[(\nabla t)(u,v_{1},\dots,v_{p},\omega_{1},\dots,\omega_{p})=(\nabla_{u} t)(v_{1},\dots,v_{p},\omega_{1},\dots,\omega_{p}),\] where \[ u,v_{i}\in T_{_{M}}(\cM),\quad \text{and}\quad \omega_{j}\in T^{\star}_{_{M}}(\cM).
\]

\vspace{3pt}
\noindent{\bf  (a) Covariant derivative of a 1-form}

\vspace{3pt}
We will study the above definition on the example of a 1-form. By the fourth condition above, we have that 
\[
\nabla_{_{Y}}[\mathbf{\alpha}(X)] = (\nabla_{_{Y}}\mathbf{\alpha})(X)+\mathbf{\alpha}(\nabla_{_{Y}}X),\]
which implies that 
\[(\nabla_{_{Y}} \mathbf{\alpha})(X) =\nabla_{_{Y}}[\mathbf{\alpha}(X)]- \mathbf{\alpha}(\nabla_{_{Y}}X).
\]
If we substitute $X$ by $e_{i}$ this gives
\[
(\nabla_{_{Y}} \mathbf{\alpha})_{i}=Y(\alpha^{i}) - \mathbf{\alpha}(\gamma^{j}_{ki}Y^{k}e_{j})=Y(\alpha^{i})-\gamma^{j}_{ki}Y^{k}\alpha_{j},
\]
and thus
\[
\nabla_{_{Y}} \mathbf{\alpha}=Y^{k}[e_{k}(\alpha_{i})-\gamma^{j}_{ki}\alpha_{j}]\varepsilon^{i},
\]
which means in particular that 
\begin{equation}
\nabla_{_{Y}} \varepsilon^{ i} = -Y^{k}\gamma^{j}_{ki}\varepsilon^{j} \text{ or } \nabla_{e_{k}} \varepsilon^{ i} = -\gamma^{j}_{ki}\varepsilon^{j}.
\end{equation}
This allows us to deduce that
\[
\nabla \mathbf{\alpha}= [e_{k}(\alpha_{i}) -\gamma^{j}_{ki}\alpha_{j}]\epsilon^{k}\otimes \varepsilon^{j}=(d\alpha_{i}+ \alpha_{j}\omega^{j}_{i})\otimes \varepsilon^{i}.
\]
\begin{proposition}[Covariant derivative of 1-form]
Let $\mathbf{\alpha}\in T^{\star}_{_{M}}(\cM)$. Then
 \begin{equation}
(\nabla_{_{Y}} \mathbf{\alpha})(X) =\nabla_{_{Y}}(\mathbf{\alpha}X)- \mathbf{\alpha}(\nabla_{_{Y}}X).
\end{equation}
Additionally, if $\{e_{i}\}_{i=1}^{n}$ is a moving frame and $\{\varepsilon^{i}\}_{i=1}^{n}$ then 
\begin{equation}
\nabla \mathbf{\alpha} =(d\alpha_{i}+ \alpha_{j}\omega^{j}_{i})\otimes \varepsilon^{i}
\end{equation}
\end{proposition}

\begin{proof}
    The proof is left as an exercise. 
\end{proof}


\noindent{\bf  (b) Covariant derivative of a tensor}
We now consider a (2,0)-tensor and explore the notion of covariant derivative of a (2,0)-tensor. Let $g=g_{\mu\nu}\varepsilon^\mu\otimes \varepsilon^{\nu}$ be a $(2,0)$-tensor.

\,

The covariant derivative of $g$ is obtained from the properties (3) and (4) outlined in the list of conditions above. 

Therefore, 
\[\begin{aligned}
\nabla_{_{X}}g&= X(g_{\mu\nu})\epsilon^\mu\otimes \epsilon^{\nu}+g_{\mu\nu}\nabla_{_{X}}\varepsilon^\mu\otimes \epsilon^{\nu}+g_{\mu\nu}\varepsilon^\mu\otimes \nabla_{_{X}}\varepsilon^{\nu}\\
&=X^{\alpha}[e_{\alpha}(g_{\mu\nu})-\gamma^{\beta}_{\alpha\mu} g_{\beta\nu}-\gamma^{\beta}_{\alpha\nu} g_{\mu\beta}]\varepsilon^{\mu}\otimes \varepsilon^{\nu},
\end{aligned}\]
or
\begin{equation}\label{covmetric}
\nabla_\alpha g_{\mu\nu}=e_{\alpha}(g_{\mu\nu})-\gamma^{\beta}_{\alpha\mu} g_{\beta\nu}-\gamma^{\beta}_{\alpha\nu} g_{\mu\beta}.
\end{equation}

\begin{ex}
The generalization is rather straight forward and left as an exercise. 
    
\end{ex}

\begin{ex}
As an example, one can work with a $(2,1)$-tensor $t_{kl}^i$ and show that : 
\[
\nabla_j t^{i}_{k\ell}=e_{j}(t^{i}_{k\ell})+\gamma^{i}_{jm}t^{m}_{k\ell} -\gamma^{m}_{jk} t^{i}_{m\ell}-\gamma^{m}_{j\ell} t^{i}_{km}.
\]
    
\end{ex}

We continue with the next remark. 
\begin{remark}
 The covariant derivative in the direction $e_{i}$ of the tensor product of  $s$ and $t$ of two tensors satisfy, by \eqref{E:cdt3}:
 \[\nabla_{e_{i}}(s\otimes t)=(\nabla_{e_{i}}s)\otimes t+s\otimes ( \nabla_{e_{i}}t).\]
However, it is important to keep in mind that the sum of two tensor products is defined only if the corresponding factors have the same rank and therefore:
\[
 \nabla (s\otimes t)\not=(\nabla s)\otimes t+s\otimes ( \nabla t).
\] 
\end{remark}

\section{Parallel transport - Geodesics}

In Euclidean space, vectors based at different points can be compared by translating them parallelly to a common origin. 

Consequently, if a vector 
$\Vec{v}$ is transported parallelly along a curve $\gamma$, the derivative $dv/dt=0$ vanishes. 



%On a manifold  to speak of the equality of  two vectors at two different points $M\in \mathcal{M}$ and $M'\in \mathcal{M}$, one must be able to assign a uniquely defined frame at $M'$ given a frame at $M$, this can be done by a suitable choice of connection.


On a manifold $\cM$, to compare vectors at two different points $M\in \cM$ and $M'\in \cM$, it is necessary to be able to assign a uniquely defined frame at  $M'$ based on a frame at $M$. 

This is precisely where the notion  of connection plays a key role: by enabling the existence of a covariant derivative, a connection provides a convenient notion of parallel transport.

\subsection{Parallel vector along a curve} 
\begin{definition}[Parallel transport]\index{Connection!Parallel transport}
A vector $X$ is said to be parallel along the curve $\gamma : t\to \gamma(t)$ if
 \begin{equation}
 \nabla_{u}X=0,\quad u=\gamma'\,\frac{d}{dt},
 \end{equation}
 \end{definition}
 
 \begin{remark}
 The vector $u$ is defined only at points lying on the curve $\gamma$. Somehow, it can also be extended to a vector field on a neighborhood of the curve $\gamma$. 
 \end{remark}
 
 If $(\varphi,U)$ is a local chart, the component of a point $\gamma(t)$ on the curve $\gamma$  are \[x^{i} (t)=\varphi^{i}\circ\gamma(t)\] and the components of $u$ are given by $u^{i}=dx^{i}/dt$
\[u^{i}= \varphi^{i}\circ u=u( \varphi^{i}).\]

 \subsection{Geodesics}
 We shortly digress on the notion of geodesics, as they intimately related to the topic discussed in this section. 
 
 \begin{definition} [Affine geodesic]\index{Geodesic!Affine}
 An affine geodesic on $\mathcal{M}$ is a curve \[\gamma : t\to \gamma(t)\]  such that
 \begin{equation}
 \nabla_{u}u=\lambda(t)u,\quad \text{where} \quad u=\frac{d\gamma}{dt},
 \end{equation}
 for some function $\lambda$ on $\bbR$.
 
The curve $\gamma$ is called a geodesic i.e.
 \begin{equation} \nabla_{u}u=0.\end{equation}
 \end{definition}

The concept of geodesics  generalizes the notion of straight line in Euclidean space.

There are some easy computations that can be done given $\gamma:I\to M, \quad I=[a,b]\subset \mathbb{R}$, the length of the smooth curve  is given by
\[L(\gamma):=\int_{a}^b|\dot \gamma(t)|dt.\]
In particular, this notions starts to be interesting as soon as we want to understant better the intrinsic distant between two points on a manifold. 
Assume $\cM\subset \bbR^n$ is an $m$-dim smooth manifold. Take a pair of distinct points $p,q\in \cM$. 
Their Euclidean distance is given by $|p-q|$ in the ambient Euclidean space.
However, this type of distance does not tell us much, from the viewpoint of the manifold $\cM$.
We therefore introduce the notion of an intrinsic distance in $\cM$.
\begin{definition}
The intrinsic distance on $\cM$ between $p, q\in \cM$ is a real number $d(p,q)\geq 0$ 
defined by \[d(p,q):=\inf_{\gamma\in \Omega_{p,q}}L(\gamma),\] 
where $\Omega_{p,q}$ is the space of smooth paths of the unit interval joining $p$ to $q$. 
\end{definition} 

\begin{ex}
Prove that $L(\gamma)\geq |p-q|$.
\end{ex}
\begin{center}
    \includegraphics[]{geod.png}
    \end{center}
 Let us now write the equation of geodesics in local chart $(\varphi,U)$ the component of a point $\gamma(t)$ on the curve $\gamma$  are $x^{i} (t)=\varphi\circ\gamma(t)$ and the component of $u$ are $u^{i}=dx^{i}/dt$



\section{Curvature}
\subsection{Vector bundle - version}
Let $(\mathcal{M}, g)$ be a Riemannian manifold, that is a smooth manifold equipped with a Riemannian metric.
The Riemannian curvature tensor is defined as a map
\[R:\Gamma(\cT\cM) \times \Gamma(\cT\cM)  \times \Gamma(\cT\cM) \to\Gamma(\cT\cM), \] 
characterized by the formula:
\[R(X,Y)Z=\nabla_X\nabla_YZ-\nabla_Y\nabla_YZ-\nabla_{[X,Y]}Z\]
where $[X,Y]$ is a Lie bracket of vector fields and where $\Gamma(\cT\cM)$ are tangent bundle sections.

\begin{example}
    In the flat case, the absence of curvature ensures geodesic parallelism.
\end{example}
\subsection{Sheaf version}
In the scope of using a more modern language, we can consider the notion of Riemannian curvature in a more categorical framework, which relies on the notion of sheaf. To give a rough idea, a sheaf assigns some local data to open sets, and patches these local data together  into a global object.

\begin{definition}
    Let $X$ be a topological space. A presheaf $\cF$ on $X$ is a contravariant functor from the category of open sets of $X$ (denoted $Open(X)$), to a category $\mathscr{C}$, i.e.,
    \[\cF:Open(X)^{op}\to \mathscr{C}.\]
\end{definition}

This definition means that to each open set $U\subset X$, we assign an object $\cF(U$) in the category $\mathscr{C}$. The category $\mathscr{C}$ can be the category of sets, abelian groups, rings etc.

To each inclusion of open sets $V\subset U$, we assign a restriction morphism $\rho_{U,V}:\cF(U)\to \cF(V)$, satisfying: 

\begin{itemize}
    \item {\bf Identity:} $\rho_{U,U}$ is the identity;
    \item {\bf Transitivity:}  $\rho_{V,W}\circ\rho_{U,V}=\rho_{U,W}$ for $W\subset V\subset U$.
\end{itemize}

A sheaf is a presheaf with an extra gluing condition. Namely, for any open covering $U=\bigcup\limits_{\alpha} U_\alpha$ and a family of local sections that agree on overlaps (i.e. {\it compatible local sections}), there exists a unique global section that restricts to them.
In particular, $\cF$ is a sheaf if, for any open cover $U=\bigcup\limits_{\alpha} U_\alpha$, the sequence is exact: 

\[\cF(U)\to \prod_\alpha\cF(U_{\alpha}) \rightrightarrows \prod_{\alpha,\beta}\cF(U_{\alpha}\cap U_{\beta})\]

Let us breakdown this construction: 

\begin{itemize}
    \item The first map sends a global section $s\in \cF(U)$ to its restriction $s|{_{U_\alpha}}$.
    \item The second pair of maps send $(s_{\alpha})$ to their restrictions on overlaps $s_{\alpha}|_{{{U_\alpha}\cap U_{\beta}}}$ and $s_{\beta}|_{{{U_\alpha}\cap U_{\beta}}}$.
\end{itemize}

\subsection{Curvature version 2}
Going back to the definition of curvature, the main differences, between the previous language and the new formalism are depicted below:
\begin{itemize}
    \item the tangent bundle sections are viewed as a \emph{sheaf} $\cE$ of $\cO_\cM$-modules. 

  \item The covariant derivative is interpreted as a \emph{morphism of sheaves} rather than an operation on vector fields.

  \item The curvature tensor is introduced as a functorial transformation of the module structure.
  
  \item The bracket operation is associated with the intrinsic structure of the Lie algebroid given by the sheaf 
$\cE$.

\end{itemize}

Let $(\mathcal{M}, g)$ be a Riemannian manifold, where $\mathcal{M}$ is a smooth manifold equipped with a sheaf $\mathcal{O}_{\mathcal{M}}$ of smooth functions and a sheaf $\mathcal{E} = \mathcal{T}_{\mathcal{M}}$ of sections of the tangent bundle, endowed with a Riemannian metric $g$. The connection $\nabla$ is then a morphism of sheaves of $\mathcal{O}_{\mathcal{M}}$-modules:

\[
\nabla: \mathcal{E} \to \mathcal{E} \otimes_{\mathcal{O}_{\mathcal{M}}} \Omega^1_{\mathcal{M}}
\]

where $\Omega^1_{\mathcal{M}}$ is the sheaf of Kähler differentials on $\mathcal{M}$.

The \emph{Riemannian curvature tensor} is then a natural transformation of the $\mathcal{O}_{\mathcal{M}}$-module functor given by

\[
R: \mathcal{E} \times_{\mathcal{O}_{\mathcal{M}}} \mathcal{E} \times_{\mathcal{O}_{\mathcal{M}}} \mathcal{E} \to \mathcal{E}
\]

defined by the relation

\[
R(X,Y)Z = \nabla_X \nabla_Y Z - \nabla_Y \nabla_X Z - \nabla_{[X,Y]} Z,
\]

where $[X,Y]$ is the Lie bracket of vector fields, arising from the structure of the Lie algebroid associated with $\mathcal{E}$. The operator $[\nabla_X, \nabla_Y]$ is a commutator of differential operators.


Since the right hand side only depends on the values of $X, Y, Z$ at a given point, $R$ is a \emph{tensorial} object, i.e., a section of the sheaf: $ R \in \Gamma(\mathcal{M}, \operatorname{End}(\mathcal{E}) \otimes \Omega^2_{\mathcal{M}})$.


This tensor encodes the \textit{non-commutativity} of the covariant derivative and measures the curvature of the category of $\mathcal{O}_{\mathcal{M}}$-modules endowed with $\nabla$.


\begin{ex}\label{Ex:con1}
Show that we can rewrite it as: $$R(X,Y)=[\nabla_X,\nabla_Y]+\nabla_{[X,Y]}$$
\end{ex}
\begin{ex}\label{Ex:con2}
Symmetries: 
Show that we have the following relations: 
\begin{align}
&R(Y,X)=-R(X,Y) \\
&R(X,Y)Z+R(Y,Z)X+R(Z,X)Y=0\\
&\langle   R(X,Y)Z, W\rangle= \langle   R(Z,W)X, Y\rangle
\end{align}
\end{ex}
\subsection{Sectional curvature}
Let $\cM\subset \bbR^n$ be a smooth $m$-dimensional submanifold. 
Let $p\in \cM$ and let $E\subset \cT\cM$ be a 2-dimensional linear subspace of the tangent space. 
The sectional curvature of $\cM$ at $(p,E)$ is the number 
\[K(p,E)=\frac{\langle R_p(u,v)v,u\rangle }{|u|^2|v|^2-\langle u,v\rangle^2 }\]
where $u,v\in E$ are linearly independent and $R_p$ is the Riemannian curvature tensor. 

\begin{ex}
    Let us consider the manifold of symmetric positive definite matrices i.e. 
    \[\{A\in Mat_{n\times n}(\bbR)\, |\, A^T=A,\,  x^TAx>0 \}.\]
    Compute the sectional curvature for this manifold.
\end{ex}

The space of symmetric positive definite matrices has many interesting applications. 
\begin{itemize}
    \item Optimisation (convex programming).
\item Information geometry and harmonic analysis (via Wishart laws). These are important in random matrix theory and statistics.
\item Medicine (diffusion tensor Imaging) as a model of the anisotropic diffusion of water in the tissues. 
\item Machine learning, one is interested in finding the correlation between given properties (for example one could take weight / volume or weight and height). One models the relationship with a covariance matrix.
\end{itemize}
\begin{definition}
Let $k\in \bbR$ and $m\geq 2$ be an integer. An $m$-manifold has constant sectional curvature 
$k$ if and only if $K(p,E)=k$ for every $p\in \cM$ and every 2-dimensional linear subspace $E\subset \cT_p\cM$.    
\end{definition}

\begin{theorem}
  Let $\cM$ be an $m$-manifold. Fix an element $p\in \cM$ and a real number $k$. Then the following are equivalent:
$K(p,E)=k$ for every 2-dimensional linear subspace $E\subset \cT_p\cM$.
The Riemann curvature tensor of $\cM$ at $p$ is given by:
\[\langle R_p(v_1,v_2)v_3,v_4\rangle=k(\langle v_1,v_4\rangle-\langle v_2,v_3\rangle),\]
for all $v_1,\cdots, v_4 \in \cT_p\cM$.
  
\end{theorem} 

\section{Torsion in Differential Geometry}
\subsection{Torsion tensor}

Let $M$ be a smooth manifold, and let $\nabla$ be an affine connection on $M$. The presence of torsion in $\nabla$ quantifies the extent to which infinitesimal parallel transport fails to be symmetric.

\begin{definition}
The \emph{torsion tensor} $\boldsymbol{T}$ associated with a connection $\nabla$ is the $(1,2)$-tensor defined by
\[
\boldsymbol{T}(X,Y) = \nabla_X (Y) - \nabla_Y (X) - [X,Y],
\]
for any vector fields $X,Y$ on $M$, where $[X,Y]$ denotes the Lie bracket and $\nabla_Y (X)$ is the covariant derivative. The connection $\nabla$ is said to be \emph{torsion-free} if $\boldsymbol{T}=0$.
\end{definition}

\subsection{Local coordinates}
In local coordinates, $\{x^i\}$, the torsion tensor can be expressed in terms of the connection coefficients $\Gamma_{ij}^k$:
\[\boldsymbol{T}^i_{jk}=\Gamma_{jk}^i-\Gamma_{kj}^i,\]

where $\boldsymbol{T}^i_{jk}$ are the components of the torsion tensor, and $\Gamma_{jk}^i$ are the Christoffel symbols of the connection.


\begin{itemize}
    \item The torsion measures the \emph{failure of parallel transport} to preserve the commutativity of vector fields. 

\item Let us mention that the torsion affects the \emph{holonomy} of a connection, which describes how vectors rotate when parallel transported around closed loops.
\end{itemize}

In Riemannian geometry, the Levi-Civita connection is the unique torsion-free connection that is compatible with the metric. This connection is central to the study of curvature and geodesics.

\, 

\subsection{Geodesic Deviations}
In the presence of torsion, the equation of geodesic deviation (which describes how nearby geodesics spread apart or converge) includes additional terms involving the torsion tensor. 

Intuitively, imagine two nearby geodesics on a surface. If the surface is flat (such as a plane), the geodesics are straight lines, and the distance between them remains constant. However, if the surface is curved (for instance, a sphere), the geodesics may converge or diverge over time. Geodesic deviation quantifies this behavior.

\, 

Let $\mathcal{M}$ be a smooth manifold equipped with a sheaf $\mathcal{O}_{\mathcal{M}}$ of smooth functions and a sheaf $\mathcal{E} = \mathcal{T}_{\mathcal{M}}$ of sections of the tangent bundle. A \emph{Riemannian structure} (or a \emph{semi-Riemannian structure} in the case of Lorentzian manifolds) is given by a metric $g: \mathcal{E} \otimes_{\mathcal{O}_{\mathcal{M}}} \mathcal{E} \to \mathcal{O}_{\mathcal{M}}$.

A geodesic on $\mathcal{M}$ is a path satisfying the equation
\[
\nabla_{\dot{\gamma}} \dot{\gamma} = 0,
\]
where $\nabla$ is the Levi-Civita connection associated with $g$.

Now, consider a one-parameter family of geodesics, represented by the morphism
\[
\wp: I \times (-\epsilon, \epsilon) \to \mathcal{M},
\]
where $I$ is an interval in $\mathbb{R}$ parametrizing proper time or arc length, and the second parameter represents a perturbation of the initial geodesic. The \emph{geodesic deviation vector field} is given by
\[
\cJ = \frac{\partial \wp}{\partial s} \Big|_{s=0},
\]
which defines a section of $\mathcal{E}$ along the central geodesic $\gamma(s=0)$. This vector field satisfies the \emph{Jacobi equation}, which in sheaf-theoretic terms can be written as
\[
\nabla_{\dot{\gamma}} \nabla_{\dot{\gamma}} \cJ + R(\dot{\gamma}, \cJ) \dot{\gamma} = 0.
\]
Here, $R: \mathcal{E} \times_{\mathcal{O}_{\mathcal{M}}} \mathcal{E} \to \operatorname{End}(\mathcal{E})$ is the Riemann curvature tensor, viewed as a section of the sheaf $\operatorname{End}(\mathcal{E}) \otimes_{\mathcal{O}_{\mathcal{M}}} \Omega^2_{\mathcal{M}}$.

\subsubsection{Flat vs. Curved Geometry:}
In the case where $(\mathcal{M}, g)$ is flat (i.e., $\mathcal{M}$ is locally isometric to an open subset of $\mathbb{R}^n$ with the Euclidean metric), we have $R = 0$, and hence $\cJ$ satisfies
\[
\nabla_{\dot{\gamma}} \nabla_{\dot{\gamma}} \cJ = 0.
\]
This implies that geodesics remain equidistant, meaning that nearby geodesics neither converge nor diverge.

If $(\mathcal{M}, g)$ is curved, then the curvature tensor $R$ introduces a term that governs the deviation of geodesics. The sign and structure of $R(\dot{\gamma}, \cJ) \dot{\gamma}$ determine whether nearby geodesics converge or diverge, capturing the intrinsic curvature of $\mathcal{M}$ in a functorial manner.

Thus, geodesic deviation measures the sheaf-theoretic failure of parallel transport to be trivial in the presence of curvature.




\subsection{Skew-Symmetry and Vanishing Torsion}


The torsion tensor is skew-symmetric in its lower indices:
\[\boldsymbol{T}(X,Y)=-\boldsymbol{T}(Y,X).\]

\subsubsection{\bf Vanishing Torsion}

If $\boldsymbol{T}=0$, the connection is said to be torsion-free. In this case, the connection coefficients satisfy 
\[\Gamma_{jk}^i=\Gamma_{kj}^i,\] and the connection is symmetric. 
\subsection{Relations to Curvature}
The torsion tensor and the curvature tensor $R$ of a connection are related via the Bianchi identity: 

The \textbf{First Bianchi Identity} for a connection with torsion can be written as:

\begin{equation}
    R^\nabla(X, Y)Z + R^\nabla(Y, Z)X + R^\nabla(Z, X)Y = (\nabla_X \boldsymbol{T})(Y, Z) + (\nabla_Y \boldsymbol{T})(Z, X) + (\nabla_Z \boldsymbol{T})(X, Y).
\end{equation}

In components, this can be rewritten as: 

\begin{align*}
 &R^\lambda_{\;\;\mu\nu\rho} + R^\lambda_{\;\;\nu\rho\mu} + R^\lambda_{\;\;\rho\mu\nu} = \\
 & \nabla_\mu \boldsymbol{T}^\lambda_{\;\;\nu\rho} + \nabla_\nu \boldsymbol{T}^\lambda_{\;\;\rho\mu} + \nabla_\rho \boldsymbol{T}^\lambda_{\;\;\mu\nu} + \boldsymbol{T}^\sigma_{\;\;\mu\nu} \boldsymbol{T}^\lambda_{\;\;\rho\sigma} + \boldsymbol{T}^\sigma_{\;\;\nu\rho} \boldsymbol{T}^\lambda_{\;\;\mu\sigma} + \boldsymbol{T}^\sigma_{\;\;\rho\mu} \boldsymbol{T}^\lambda_{\;\;\nu\sigma}.
\end{align*}


\, 

For a more categorical approach, let $\cE$ be a sheaf of sections of the tangent bundle (or a vector bundle over a space), and let 
$\nabla$ be a connection on $\cE$. The curvature $R^{\nabla}$ and torsion $\boldsymbol{T}$
appear as components of a functorial construction on the Atiyah algebroid 
(the sheaf of first-order differential operators preserving a given structure).

\subsection{Spencer differential}
In particular, using the Lie-algebroid formalism, the first Bianchi identity corresponds to the failure of the \textit{Spencer differential} $d_\nabla$ acting on torsion to be zero, that is:

\begin{equation}
    d_{\nabla} \boldsymbol{T} + R^\nabla \wedge_{\nabla} \operatorname{id} = 0,
\end{equation}

where \begin{itemize}
    \item $d_\nabla$ is the Spencer differential associated with the connection.
    \item $\wedge_{\nabla}$ denotes the wedge operation compatible with the connection.
\end{itemize}


Specifically, if you parallel transport a vector $Y$ along a curve tangent to $X$, 
\begin{example}
The Torsion can be interpreted as an obstruction to the integrability of certain geometric structures,
such as foliations or distributions on a manifold.
\end{example}
\begin{example}
The torsion plays a central role in non-Riemannian geometries. This includes Finsler geometry for instance and 
teleparallel gravity. In those frameworks, the torsion tensor is used to describe the gravitational field, 
replacing the curvature tensor of general relativity.
\end{example}