\chapter{Differentiability and Gateaux Derivatives}\index{Manifold!Differentiable}

In this chapter, we undertake a comprehensive investigation of the notion of differentiability, within an extended framework. This expands beyond elementary calculus to encompass both differentiable functions and the rich structure of differentiable manifolds. Building upon this foundation, we systematically develop the essential constructions of tangent vectors and tangent spaces, alongside their dual counterparts, cotangent spaces. These concepts not only underpin the analytical machinery of differential geometry but also enable far-reaching applications across mathematics—from topology and dynamical systems—and physics, particularly in the geometric formulation of classical mechanics, general relativity, and gauge field theories.

More precisely, the notion of differentiability serves in mathematics in:  
\begin{enumerate}
    \item Local Linear Approximations
    \begin{itemize}
        \item Differentiability allows functions, curves, or maps to be approximated locally by linear objects (e.g., tangent lines, Jacobian matrices). This is used for instance in optimization, where one considers extrema via critical points. This also used in machine learning. 
    \end{itemize}
    \item Smoothness (and other important properties) of manifolds:

    \begin{itemize}
        \item Differentiability defines the notion of "smoothness" of a manifold but also permits to study  vector fields, differential forms and curvature tensors. 
    \end{itemize}
    \item Global/Local analysis: 
    \begin{itemize}
        \item Differentiability connects local properties (e.g., derivatives) to global phenomena for instance using De Rham Cohomology.
        \end{itemize}
    \item Algebraic structures
    \begin{itemize}
        \item
    Differentiability is the gateway to advanced (and more algebraic) frameworks such as 
Jet Bundles; Lie Groups (studying symmetries of differentiable manifolds); Sheaf Theory (formalizing local-to-global properties in algebraic geometry).
\end{itemize}
\end{enumerate}


%Naturally, we recall tangent vectors, tangent vector spaces and their dual space: the cotangent vector spaces. These are foundations of differential geometry and have a very large panel of applications, whether it is in mathematics or in physics. 
\section{Gateaux directional derivation: a definition} Inscribing ourselves in the vein of the previous chapter, the notion of Gateaux's directional derivation appears the most natural to start with. We introduce this notion below. 

\,

Let $\cM$ be a manifold  modeled on a topological vector space $\cE$, and let us assume  that a differentiable structure can be defined on $\cE$. As we have shown $\cE$ to be a Hausdorff (locally convex vector space) one can define a differentiable structure via the \emph{Gateaux directional derivation}.

\, 

\subsubsection{Gateaux derivative}
Suppose that $X$ and $Y$ are locally convex topological vector spaces. Assume $U \subset X$  is open and that we have the map $F: X \to Y$. The Gateaux differential $dF(x;\varphi )$ of $F$ at $x \in U$ in the direction $\varphi$ in $X$ is defined as
\begin{equation}\label{E:Gateaux}
 dF ( x ; \varphi ) = \lim_{t \to 0} \frac{F(x +t\, \varphi) - F ( x )}{ t} = \frac{d}{d t} F( x +t\,\varphi )\vert_{t=0}
 \end{equation}
If this limit exists for all $\varphi$, then $ F$ is said to be \emph{Gateaux differentiable} in $x$.

\, 

The limit  is taken relatively to the topology of $Y$.

Indeed:
\begin{enumerate}
\item if $X$ and $Y$ are {\it real} topological vector spaces, then the limit is taken for {\it real} $t$. 
\item If $X$ and $Y$ are complex topological vector spaces, then the limit above in \eqref{E:Gateaux} is usually taken as $t \to 0 $ in the complex plane, as usually in the definition of complex differentiability.
\item In some cases, a \emph{weak limit} is taken instead of a strong limit, which leads to the notion of a \emph{weak Gateaux derivative.}
\end{enumerate}

\, 

This short introduction to Gateaux's derivative guides us towards the more standard terminology of  differentiable manifolds. 

\subsection{Differentiable manifolds}
We start with a definition on differentiable manifolds.

\begin{definition}[Differentiable manifold]\index{Manifold!Differentiable}
A \(\mathcal{C}^k\)-\textit{differentiable manifold} \(\mathcal{M}\) is a topological manifold where the condition  \((\mathcal{M}3)\) is substituted by a new condition:

\begin{itemize}
    \item[\((\mathcal{M}3')\)] for each pair of indices \(i, j \in I\), the transition map between overlapping coordinate charts,
    \[
    \varphi_j \circ \varphi_i^{-1} : \varphi_i(U_i \cap U_j) \to \varphi_j(U_i \cap U_j),
    \]
    is of class \(\mathcal{C}^{k}\), meaning it is \(k\)-times continuously differentiable. Moreover, for every \(i, j \in I\), the set \(\varphi_i(U_i \cap U_j)\) is open in the model space \(\mathcal{E}\).
\end{itemize}

%\begin{definition}[Differentiable manifold]\index{Manifold!Differentiable}
%A $\cC^k$-differentiable manifold $\cM$ is a topological manifold such that the condition $\cM 3$ is  replaced by the following axiom: 

%\,
%\begin{itemize}
  %  \item  $\cM 3':$ The map
%\[
%\varphi_j\circ\varphi_i^{-1}:\varphi_i(\mathcal{U}_i\cap \mathcal{U}_j) \to \varphi_j(\mathcal{U}_i\cap
%\mathcal{U}_j),
%\]
%is of class $\cC^{k}$, for each pair of indices $i,j$ and for any
%$i,j\in I$
%$\varphi_i(\cU_i\cap \cU_j)$ is open in $\cE$.
%\end{itemize}
\end{definition}

Moreover, we have the following compatibility criterion. 
\begin{definition}[Compatible atlas]\index{Manifold!Atlas!Compatible}
Two $\cC^{k}$ atlases on $\cM$, modeled on  $\cE$ are said to be compatible if their union is an another such atlas.
\end{definition}
Remark that this notion of compatibility is in fact an equivalence relation. 

\begin{ex}
Prove the remark above.
\end{ex}

\begin{definition}[Admissible atlas]\index{Manifold!Atlas!Admisible}
Given a manifold $\cM$,
all atlases, modeled on $\cE$, and lying within the same equivalence class are said  to be \emph{admissible} on $\cM$.
\end{definition} 

It is enough to have an admissible atlas to define the structure of a manifold.

\,
\subsection{Differentiable mappings}
\, 

\begin{definition}[Differentiable mappings]~\label{DiffMap}\index{Mapping!Differentiable}
Consider a pair of $\cC^k$-differentiable manifolds, denoted  $\cM$ and $\cM'$.
\begin{itemize}
\item A mapping $f: \cM \to
\cM'$ is said to be differentiable of class $\cC^r,$ where $\, r\leq k$, if for every chart $(\cU_i,\varphi_i)$ of $\cM$ and every chart $(\cV_j,\psi_j)$ of  $\cM'$ such that $f(\cU_i) \subset \cV_j,$ the mapping $\psi_j
\circ f \circ \varphi_i^{-1}$ 
 of $\varphi_i(\cU_i)$ into $\psi_j(\cV_j)$ is
differentiable, of class $\cC^r$.

\item[ ]
\item A \emph{diffeomorphism} \index{Mapping!Diffeomorphism} $f$ of class $\cC^r$ is a bijective mapping $f$ such that $f$ and $f^{-1}$ are continuously $\cC^r$-differentiable.
\item[ ]

\item If the transition maps $\varphi_j\circ\varphi_i^{-1}$ are diffeomorphisms of class $\cC^\infty$ then the manifold $\cM$ is said to be smooth.


\item[ ]
\item A $\cC^k$-differentiable function $f:\cM \to \bbR$ on $\cM$ is a mapping of class $\cC^k$ from 
$\cM$ to $\bbR$. 
\item[ ]
\item A function $f$ is \emph{differentiable} at a point $ m$ on the manifold $\cM$ if, for some coordinate chart $(U,\phi)$ containing $ m$, the composition $f\circ \varphi^{-1}$ is differentiable at the image point $\varphi(m)$. %{\scshape m} $\in \cM$ on the differentiable manifold $\cM$ if in a chart at
%$\scshape m$, $f\circ \varphi^{-1}$ is differentiable at $\varphi$({\scshape m}).
\end{itemize}
\end{definition}

\noindent

Notice, that these definitions do not depend on the chart.

\,
\begin{remark}
Since the implicit function theorem does not hold for arbitrary locally convex space, this definition has a limited utility if we do not introduce   more information on the topology.
\end{remark}


\section{Manifolds modeled on a normed vector spaces}

Interesting properties are obtained
in the case where the model space $\cE=\fB$ is a Banach vector space a natural  differentiability structure is provided by the Banach derivation~\eqref{E:Bdif}. In this case the implicit function theorem allows to derive interesting properties.



\begin{definition}[Banach manifold]\index{Manifold!Banach}
A $\cC^k$-Banach manifold $\cB$ is a manifold such that condition $\cM 3$ is  replaced by: 

\vspace{3pt}
$\bullet$ $\cM 3''.$ The map
\[
\varphi_j\circ\varphi_i^{-1}:\varphi_i(\mathcal{U}_i\cap \mathcal{U}_j) \to \varphi_j(\mathcal{U}_i\cap
\mathcal{U}_j),
\]
is a $\cC^{k}$-isomorphism on $\fB$ for each pair of indices $i,j$, and for any
$i,j\in I$
$\varphi_i(\cU_i\cap \cU_j)$ is open in $\cB$.
\end {definition}



\begin{definition}[Riemann manifold]\index{Manifold!Riemann}
A Riemannian manifold is a differentiable manifold modeled on a real vector space  $\cE$, the topology of which is given by a scalar product $\langle\cdot|\cdot\rangle$.
\end{definition}
Let us recall what a scalar product is. A \emph{scalar product} on the vector space $\cE$ is a bilinear symmetric form 
\begin{align*}
\langle\cdot,\cdot\rangle:\,  \cE\times \cE \to &  \bbR,  \\
 (x,y)\mapsto&  \langle x, y \rangle  \\
\end{align*}
 where:
\begin{enumerate}
%$(x,y)\in \cE \times \cE  \in  \mapsto \langle x, y \rangle \in \bbR$,
\item  The bi-linearity property is satisfied. For any scalars $\alpha, \beta$ and for any $x,x_1,x_2,y,y_1,y_2\in \cE$ the following holds: 
\begin{align*}
\langle\alpha x_{1} +\beta x_{2}, y\rangle=& \alpha\langle x_{1} ,y \rangle+\beta \langle x_{2},y\rangle\\
\langle x, \alpha y_{1} +\beta y_{2}\rangle=&\alpha\langle x ,y_{1}\rangle +\beta \langle x,y_{2}\rangle.\\
\end{align*}

%\langle\alpha x_{1} +\beta x_{2}, y\rangle= \alpha\langle x_{1} ,y \rangle+\beta \langle x_{2},y\rangle$ and 
%$\langle x, \alpha y_{1} +\beta y_{2}\rangle= \alpha\langle x ,y_{1}\rangle +\beta \langle x,y_{2}\rangle,$
\item The scalar product is symmetric. For any $x,y \in \cE$: \[\langle x,y\rangle=\langle y,x\rangle.\] 
\item The scalar product is positive definite. For any $x,y\in \cE$: \[\langle x, x \rangle \geq 0, \quad  \langle x, x\rangle =0 \,   \iff \, x=0.\]
\end{enumerate}

\, 

A Riemann manifold is said to be a Banach manifold for the norm: \[\Vert x\Vert=\langle x,x\rangle^{\frac{1}{2}},\, x\in \mathcal{E}.\]

In the case where the model space is a Hilbert space, we speak of \emph{ Hilbert manifold}. Hence a Riemann manifold is a real Hilbert manifold.

\,
\subsection{Real Manifolds,  Coordinates, Charts}
\ 

A particularly interesting class of Riemannian manifolds is the manifold endowed with the model space \( \mathcal{E} = \mathbb{R}^{n} \). The usual topology and differentiability structure on \( \mathbb{R}^{n} \) induce a differentiable structure on $ \mathcal{M} $, making it a real $ n $-dimensional differentiable manifold.

In the following, by an $ n$-dimensional (real) manifold $ \mathcal{M} $, we mean a Hausdorff topological space in which every point has a neighborhood homeomorphic to $\mathbb{R}^n$.



\begin{definition}[$n$-dimensional differentiable manifold]\index{Manifold! Finite manifold!Differentiable structure}

An $n$-dimensional real manifold of class $C^k$ is a manifold modeled on $\bbR^{n}$ and such that the condition \(\cM 3\) is replaced by the following property. For each pair of indices $i,j\in I$, the map:
\[
\varphi_j\circ\varphi_i^{-1}:\varphi_i(\cU_i\cap \cU_j) \to \varphi_j(\cU_i\cap
\cU_j),
\]
is a $\cC^{k}$-isomorphism on $\bbR^{n}$. 

Furthermore, for any pairs of indices
$i,j\in I$ the image of the intersection $\cU_i\cap \cU_j$ given by
$\varphi_i(\cU_i\cap \cU_j)$, under the coordinate map $\varphi_i$, is an open subset of $\bbR^{n}$.
\end {definition}
\begin{figure}[h]
\includegraphics[scale=0.6]{chart3_LG23.pdf} 
\caption{ $n$-dimensional real manifold}\label{F:Chart3}
\end{figure}
A natural way to describe the structure of an 
$n$-dimensional differentiable manifold is through local coordinate systems. These coordinate systems are given by charts, which map open subsets of the manifold to open subsets of $\R^n$, allowing us to analyze the manifold locally as if it were a Euclidean space.
\begin{definition}[Local coordinate system]\index{Manifold! Finite manifold!Local coordinate system}
 Let $(\cU,\varphi)$ be a  chart at the point {\scshape m} $\in \cM$, such that the image $\varphi$({\scshape m})$=x=(x^1,\dots,x^n)\in\bbR^{n}$.
Given a basis $\{e_{1},\dots,e_{n}\}$ of the Euclidean space $\mathbb{R}^{n}$, the coordinates $(x^1,\dots,x^n)$ of the image $\varphi$({\scshape m}) $\in\bbR^{n}$ of {\scshape m}\, $\in U\subset\cM$ are called the \emph{coordinates of {\scshape m} in the chart }$(\cU,\varphi)$. The chart $(\cU,\varphi)$ is also called the local coordinate system.
\end{definition}

\,
Local coordinate systems provide a way to describe differentiable manifolds in terms of open subsets $\bbR^{n}$, allowing us to define smooth functions and analyze local geometric properties. 
In particular, this enables the construction of tangent spaces, which capture the local linear structure of the manifold at each point. The dual spaces to these, known as cotangent spaces, naturally arise when considering differential forms and gradients of functions, playing a fundamental role in differential geometry and analysis on manifolds.
\section{Tangent and cotagent spaces}\index{Tangent vector} 
To grasp the essence of the tangent and cotangent spaces, we begin with the most intuitive objects on a manifold: curves. A curve provides a way to move infinitesimally along the manifold, and by examining how functions change along such curves, we arrive at the concept of directional derivatives. These, in turn, will guide us toward a precise definition of tangent vectors. Finally, this will naturally lead us to the construction of a tangent space, the fundamental linear structure underlying the local geometry of the manifold.
\subsection{Curves on a manifold $\&$ Directional derivatives}


\begin{definition}[Curve]\index{Manifold!Curve}
A curve $\gamma:[a,b]\to \cM$ on a Manifold $\cM$ is a $C^p$-map, where $p\geq 1$ mapping an interval $\mathcal{I}=[a,b]\subset \bbR$ into $\cM$. 
\end{definition}

The curve is said to be smooth if it is a $C^\infty$ map.

\, 
\begin{definition}[Tangent vector  to a curve]\index{Tangent vector!Tangent vector to a curve }
Let $\gamma:t\mapsto \gamma(t)$ be a curve of class $C^1$ such that:
\begin{align*}
    &\gamma(t_0)={\scriptstyle M}_0 \in \cM\\
    &x_0:=\varphi({\scriptstyle M}_0) \in\bbR^n.
\end{align*}
%$\gamma(t_0)={\scriptstyle M}_0 \in \cM$  and $x_0:=\varphi({\scriptstyle M}_0) \in\bbR^n$. 
Then, the tangent vector $u_0$ at $\gamma$ in ${\scriptstyle M}_0$ is defined by
\begin{equation}
u_0 =\left. \frac{d \gamma}{dt}\right|_{t_0}.
\end{equation}
 \end{definition}

\begin{definition}[Directional  derivative]\label{D:dirder}\index{Derivative!Directional}

Let \( \mathcal{A}({\scriptstyle M}) \) be the family of \( C^1 \)-functions defined on a neighborhood of \( {\scriptstyle M} \in \mathcal{M} \). Assume \( f \in \mathcal{A}({\scriptstyle M}) \).  

Consider a curve \( \gamma \) of class \( C^1 \),  
\[
\gamma: t \mapsto \gamma(t),
\]
such that at some fixed parameter value \( t_0 \), we have  
\[
\gamma(t_0) = {\scriptstyle M}_0.
\]  
Then, the \emph{directional derivative} of \( f \) in the direction of the curve \( \gamma \) at \( t_0 \) is defined as  


%Let $\cA({\scriptstyle M})$ be the family of $C^1$-functions defined on a neighborhood of ${\scriptstyle M}\in \cM$. Assume $f\in\cA({\scriptstyle M})$.
%Let $\gamma:t\mapsto \gamma(t)$ be a curve of class $C^1$, such that $\gamma(t_0)={\scriptstyle M}_0$. 
%Then, the  (directional) derivative of $f$ in the direction of the curve $\gamma(t)$ at $t=t_0$ is defined by
\begin{equation}\label{E:dirder}
D_{u_{0}}f(t_{0})=\left. \frac{d\,f\circ \gamma}{dt}\right|_{t_0}, \quad u_{0}=\left.\frac{d\gamma}{dt}\right|_{t_{0}},\quad \text{where}\quad f  \in \cA({\scriptstyle M}_0).
\end{equation}
 \end{definition}
 
 More  generally:
\begin{definition}[Derivation]\index{Derivative!Derivation}
A derivation at a point ${\scriptstyle M}$ on $\cM$ is a linear functional \[D:\cA({\scriptstyle M}) \to \bbR,\] where the Leibniz rule holds:
\begin{align}\label{E:Leibniz}\index{Derivative!Leibniz rule}
D(\alpha f + \beta g)({\scriptstyle M})&=\alpha Df ({\scriptstyle M})+ \beta Dg({\scriptstyle M}), \quad \alpha,\, \beta \in \bbR,\\
D(fg)({\scriptstyle M})&= [(Df)g +fDg]({\scriptstyle M}), \quad {\scriptstyle M}\in \cM.
\end{align} 
\end{definition}



\section{Tangent vector - Tangent space}\index{Tangent space}
\ 

One can associate, via the formula~\eqref{E:dirder}, at each point ${\scriptstyle M}\in \cM$ a tangent vector $u$.

\begin{definition}[Tangent vector]\index{Tangent vector!Differentiable manifold}
A \emph{tangent vector} \( X_{\scriptstyle M} \) at a point \( {\scriptstyle M} \in \mathcal{M} \) on a differentiable manifold \( \mathcal{M} \) is a linear map  
\[
X_{\scriptstyle M} : \mathcal{A}({\scriptstyle M}) \to \mathbb{R},
\]  
where \( \mathcal{A}({\scriptstyle M}) \) denotes the space of functions defined and differentiable in some neighborhood of \( {\scriptstyle M} \in \mathcal{M} \).  

This map satisfies the Leibniz rule: for any \( f, g \in \mathcal{A}({\scriptstyle M}) \),  

\begin{equation}\label{E:Leibniz}
\begin{aligned}
X_M(\alpha f + \beta g) &= \alpha X_M (f) + \beta X_M (g), \quad \alpha, \beta \in \mathbb{R},\\
X_M (f g) &=  f(M) X_M (g) + g(M) X_M (f).
\end{aligned} 
\end{equation}
\end{definition}


\,

Notice that two differentiable functions $f_{1}$ and $f_{2}$ which coincide on a neighborhood of  ${\scriptstyle M}$ have the same  tangent vector. Therefore, a tangent vector at ${\scriptstyle M}$ is the same for all functions belonging to   the class of differential functions, coinciding on a neighborhood of ${\scriptstyle M}$.

\, 

The class of functions which coincide on a neighborhood of  ${\scriptstyle M}$ is called a ``germ'' of $f$. A germ forms an algebra under the sum and the pointwise product. A tangent vector (also called contravariant vector) is a derivation on the algebra of germs of differentiable functions at ${\scriptstyle M}$.

\, 
\begin{definition}[Tangent space]
The space of all tangent vectors at \( {\scriptstyle M} \in \mathcal{M} \), equipped with the addition and scalar multiplication defined by  
\begin{equation}
(\alpha X_{\scriptstyle M} + \beta Y_{\scriptstyle M})(f) = \alpha X_{\scriptstyle M} (f) + \beta Y_{\scriptstyle M} (f),
\quad \alpha, \beta \in \mathbb{R}, \quad X_{\scriptstyle M}, Y_{\scriptstyle M} \in \mathcal{T}_{\scriptstyle M}(\mathcal{M}),
\end{equation}
forms a real vector space, called the \emph{tangent space} at \( {\scriptstyle M} \), and denoted by \( \mathcal{T}_{\scriptstyle M}(\mathcal{M}) \).  

\end{definition}

Let $\gamma\in\mathcal{A}({\scriptstyle M})$ be a curve such that $\gamma(t_0)={\scriptstyle M}_0$ and let $u_{0}=\frac{d\gamma}{dt} \vert_{t_{0}}$ be the tangent vector at ${\scriptstyle M}_0$ along $\gamma$. 
A tangent vector $X_{{\scriptstyle M}_0}$ at ${\scriptstyle M}_0\in \cM$  in the direction $u_{0}$ can be written

\begin{equation}\label{E:tdveccov}
X_{{\scriptstyle M}_0}(f)= D_{u_{0}}f=\left.\frac{d\,f\circ \gamma}{dt}\right|_{t_0},  \quad \forall \,f\in \cA({\scriptstyle M}),\quad u_{0=}\left.\frac{d\gamma}{dt}\right|_{t_{0}}.
\end{equation}

\,

 In the case of an $n$-dimensional real manifold $\cM$ \index{Tangent space! finite dimensional}. Let $(\cU,\varphi)$ be a chart at the point ${\scriptstyle M}\in \cM$ and let  $x=(x^1,\dots,x^n) \in \varphi(\cU)\subset \bbR^n$ be a local coordinate system. A tangent vector at $X_{{\scriptstyle M}_0}$ is the set $\{ X_{{\scriptstyle M}_0}(\varphi^{i}) \}_{i=1}^{n}$, called the local coordinates of the tangent vector, where:
 \[ 
 \varphi^i= \pi^i\circ \varphi \quad \text{ and } \quad \pi^i(x^1,\dots,x^n)=x^i ,
  \] 
and where $\pi^i$ is the projection operateur in $\bbR^{n}$. 
  \begin{itemize}
\item For each $i, \, \partial/\partial x^i|_x ,\ x=\varphi({\scriptstyle M})$ satisfies the equation~\eqref{E:Leibniz}

\item[]
\item The
$\{\partial/\partial x^i|_x\}_{i=1}^n$ forms a basis, called the \emph{natural basis}  of $\cT_x( \cM)$. (Note that this  basis associated to the local coordinate system is not an  orthogonal basis).
\end{itemize}

Moreover, for all $C^1$-functions  on a neighborhood of ${\scriptstyle M}\in \cM$,
\begin{equation}\label{E:tgvecnd}
X_{\scriptstyle M}(f) =\left. \sum_{i=1}^n X^i_{\scriptstyle M}\,\frac{\partial f\circ \varphi^{-1}}{\partial x^i}\right\vert_x \ x={\varphi({\scriptstyle M})},\quad X^i_{\scriptstyle M}= X_{\scriptstyle M}(\varphi^i) .
 \end{equation}

The equation~\eqref{E:tgvecnd} can be written in the abbreviated form:

\begin{equation}\label{E:tgvecnd2}
 X_{\scriptstyle M}= \left. \sum_{i=1}^n X_{\scriptstyle M}^i\,\frac{\partial}{\partial x^i}\right|_x  x={\varphi({\scriptstyle M})},\quad X^i_{\scriptstyle M}= X_{\scriptstyle M}(\varphi^i).
 \end{equation}
 
 \begin{remark}
 Any other basis (or frame\index{Frame}) can be obtained from the local coordinate basis. Let $\{e_{i}\}_{i=1}^{n}$ be a basis of $T_{\scriptstyle M}( \cM)$ with: 
  \[
  e_{k}= \sum_{i=1}^n \Phi_{k}^{i}\frac{\partial}{\partial x^i}= \sum_{i=1}^n\Phi_{k}^{i} \partial_{i},
  \]
 where $\Phi_{k}^{i}$ is the matrix corresponding to an invertible linear transformation. 
 We have
  \[X_{\scriptstyle M}= \sum_{k=1}^{n}\chi^{k}_{\scriptstyle M} e_{k}, \text{ with } \chi^{k}_{\scriptstyle M}=\Phi^{k}_{i}X^{i}_{\scriptstyle M}, \]
 the $X^{k}_v$ being the component of $X_{\scriptstyle M}$ in the basis $\{e_{i}\}_{i=1}^{n}$.
 \end{remark}

Tangent and cotangent spaces are dual vector spaces, linked by a canonical pairing. We discuss now the notion of cotangent vector spaces. 
\,

%In the following we identify le point  $ {\scriptstyle M}\in \cM$ with $\varphi({\scriptstyle M})=x\in\bbR^n$, if no confusion is possible and denote $\cA(x):=\cA({\scriptstyle M})$.
%%%%%%%%%%%%%%%%%%%%%%%%
 
\subsection{Cotangent vector, Cotangent space}\index{Cotangent space}
\ 

\,
\subsubsection{\bf Cotangent vectors $\&$ Differential 1-forms}
\begin{definition}[Cotangent vector space]
Let  $\cM$ be a manifold. The dual space $\cT^\star_{\scriptstyle M}(\cM)$ to the tangent
vector space $\cT_{\scriptstyle M}(\cM)$ in ${\scriptstyle M}\in \cM$  is the space of linear forms on  $\cT_{\scriptstyle M}(\cM)$. It is a vector
space  called the \emph{cotangent vector space} to  $\cM$ at ${\scriptstyle M}$. 

\, 

The elements of
$\cT^\star_{\scriptstyle M}(\cM)$ are called cotangent vectors, or covariant vectors, or covectors,\index{Covector} or differential 1-forms. 
\end{definition}


\noindent
Let $\omega_{\scriptstyle M} \in \cT^\star_{\scriptstyle M}(\cM)$ be a differential 1-form and consider $X_{\scriptstyle M} \in \cT_{\scriptstyle M}(\cM)$. Then:
\begin{equation}
\left\{ \begin{aligned} &\omega_{\scriptstyle M}(X_{\scriptstyle M} ) \in \bbR,\\ 
 &X_{\scriptstyle M}(\omega_{\scriptstyle M})= \omega_{\scriptstyle M}(X_{\scriptstyle M}).
\end{aligned}\right. 
\end{equation}

\subsubsection{\bf Finite dimensional real manifold}\index{Cotangent space!Finite dimensional}
In the case of  an $n$-dimensional real manifold,
a useful canonical isomorphism between a space and its dual can be chosen as follows.

\, 

Let
$(e_1,e_2,\dots,e_n)$ be a basis in  $\cT_{\scriptstyle M}(\cM)$.  We may construct its dual
$(\varepsilon^1,\varepsilon^2,\dots,\varepsilon^n)$ by
\begin{equation}
\varepsilon^i(X_{\scriptstyle M})= X^i_{\scriptstyle M}, 
\end{equation}
where $X^i_{\scriptstyle M}$ is the component of $X_{\scriptstyle M}$ in the basis $\{e_j\}$ and 
\begin{equation}
\varepsilon^i(e_j)= \delta^i_j.
\end{equation}

If we have chosen the natural basis $\{e_{i}=\partial/\partial x^i\}_{i=1}^{n}$ then the dual basis is denoted by  $\{\epsilon^{i}=dx_{i}\}_{i=1}^{n}$ with
\begin{equation}
dx^i\left(\frac{\partial}{\partial x^j}\right) =\frac{\partial}{\partial x^j}(dx^i)=\delta^i_j;
\end{equation}
a tangent vector $X_{\scriptstyle M}\in \cT_{\scriptstyle M}$ takes the form
\[
X_{\scriptstyle M} = \sum_{i=1}^n X^{i}_{\scriptstyle M}\frac{\partial}{\partial x^{i}},\quad X^{i}_{\scriptstyle M}=dx_{i}(X_{\scriptstyle M});
\]
whereas a cotangent vector $\omega_{\scriptstyle M} \in \cT_{\scriptstyle M}^\star(\cM)$ in the dual basis is given by
\[
\omega_{\scriptstyle M} = \sum_{i=1}^n \omega_{\scriptstyle M}{}_i\, dx^{i},\qquad  \omega_{\scriptstyle M}{}_i=\omega\left(\frac{\partial}{\partial x^{i}}\right)=\frac{\partial \omega}{\partial x^{i}}.
\]


For a real manifold of finite dimension, we have the following equality:
\begin{equation}
\cT^{\star \star}_{\scriptstyle M}(\cM)= \cT_{\scriptstyle M}(\cM).
\end{equation}

\subsection{Implications of a chart change}

Let $ \cM$ a $n-$dimensional real manifold. A chart $(\cU,\varphi)$ with a local coordinate system $\varphi:{\scriptstyle M}\to \varphi({\scriptstyle M})=x=(x^1,\dots,x^n)\in \bbR^n$  at $M \in \cM$ being given, the local coordinate basis on the tangent space is given by   \[\{e_{i}=\partial/\partial x^{i}\}_{i=1}^{n}\]  and its dual basis is \[\{\epsilon^{i}=dx^{i}\}_{i=1}^{n}.\]  

\, 

Had we chosen a different chart, say  $(\cU',\varphi')$ with $\varphi': {\scriptstyle M}\to  x'=({x'}^1,\dots,{x'}^n)$ at ${\scriptstyle M}\in \cM$,  a local coordinate basis can be given by $ \{e'_{i}=\partial/\partial {x'}^{1}\}_{i=1}^{n}$ as well as its dual basis $ \{{\epsilon'}^{i}=d{x'}^{i}\}_{i=1}^{n}$. We can certainly express one local coordinate basis in terms of the other on the open set $\varphi(\cU)\cap\varphi'(\cU')$.

\, 

Let us break this statement down. Under a given change of a coordinates, the local coordinates of a tangent vector are described as follows:  
\[\begin{aligned}
 X_{\scriptstyle M}= \sum_{i=1}^n X_{\scriptstyle M}^i \frac{\partial}{\partial x^{i}}=& \sum_{i=1}^n {X'}_{\scriptstyle M}^i \frac{\partial}{\partial {x'}^{i}},\\
X_{\scriptstyle M}^i=\varphi^i({\scriptstyle M}), \quad  {X'}_{\scriptstyle M}^i={\varphi'}^i({\scriptstyle M}),\quad 
&\frac{\partial}{\partial x^i}= \sum_{j=1}^n\left. \frac{\partial {x'}^j}{\partial {x}^i}\right|_{\varphi({\scriptstyle M})} \frac{\partial}{\partial {x'}^{i}},
\end{aligned}
 \]
So, the vector transformation law is given by: 
\begin{equation}
{X'}_{\scriptstyle M}^i= \sum_{j=1}^n\left. \frac{\partial {x'}^i}{\partial {x}^j}\right|_{\varphi({\scriptstyle M})} X_{\scriptstyle M}^j.
\end{equation}

\, 

Similarly, for the covector we have
\[ 
\omega_{\scriptstyle M}= \sum_{i=1}^n \omega_idx^{i}= \sum_{i=1}^n \omega'_i d{x'}^{i},\qquad  \omega_i= \omega\left(\frac{\partial}{\partial x^{i}}\right)\ \omega'_i= \omega\left(\frac{\partial}{\partial {x'}^{i}}\right)\]
\begin{equation}
\begin{aligned}\label{E:chartchange}
dx^{i}&= \sum_{j=1}^n\left. \frac{\partial {x}^i}{\partial {x'}^j}\right|_{\varphi({\scriptstyle M})} d{x'}^{j}\\
\omega'_{i}&= \sum_{j=1}^n\left. \frac{\partial {x}^j}{\partial {x'}^i}\right|_{\varphi({\scriptstyle M}} \omega_{j}.
\end{aligned}
\end{equation}

Having explored the implications of a changing the chart for vectors and covectors, a logical next step is to establish a rigorous definition of a function's differential.

\subsection{Differential of a function}\index{Differential}
\begin{definition}
Let $f$ be a differentiable function.
The differential $df|_{\scriptstyle M}$ of $f$ on $\cM$ in the neighborhood of a point ${\scriptstyle M}$ is defined by the 1-form 
\begin{equation}\label{difffunc}
df|_{\scriptstyle M}(X_{\scriptstyle M})=X_{\scriptstyle M}(f).
\end{equation}
\end{definition}

In the natural basis
\[
df|_{\scriptstyle M}(X_{\scriptstyle M})= \left. \sum_i X_{\scriptstyle M}^{i}\frac{\partial f}{\partial x^{i}} \right\vert _{\scriptstyle M}.
\]

More generally, let $\{e_{i}\}_{i=1}^{n}$ be a basis in $\cT_{\scriptstyle M}(\cM)$ and let $\{\epsilon_{i}\}_{i=1}^{n}$ be its dual basis. The value of
 $df|_{\scriptstyle M}\in \cT^{\star}_{_{M}}(\cM)$ at $X_{\scriptstyle M}\in\cT_{\scriptstyle M}(\cM)$ is given by 
 
 \[\begin{aligned}   
df|_{\scriptstyle M}(X_{\scriptstyle M})=&X_{\scriptstyle M}(f)=\\
\sum_i X^{i}e_{i}(f)=& \sum_i e_{i}(f)\epsilon^{i}(X_{\scriptstyle M}).
\end{aligned}\]


Hence,
\begin{equation} \label{E:difcoord}
df|_{\scriptstyle M}= \sum_i e_{i}(f)\epsilon^{i}.
\end{equation}


\,
Having previously established the fundamental geometric notions of vectors, differential forms (linear maps on vectors), along with differentials of functions (a type of 1-form), we naturally arrive to the notion of tensors, which encode multilinear maps between these objects. 


\section{Tensor at a point of a manifold}\label{S:tensor}\index{Tensor}

For simplicity, in this section, we assume that the manifold is finite dimensional.

\, 


\subsection{Multilinear maps $\&$ Tensors of type $(r,s)$}\index{Manifold!Tensor at a point}
\begin{definition}
Given a point ${\scriptstyle M}\in \cM$, a tensor $T^{(r,s)}_{\scriptscriptstyle M}$ of type $(r,s)$ is a multi-linear form on the space
$\underbrace{ \cT_{\scriptstyle M}\times \dots \times \cT_{\scriptstyle M}}_{r}\times \underbrace{\cT_{\scriptstyle M}^\star \times\dots \times \cT_{\scriptstyle M}^\star }_{s}= \cT_{\scriptstyle M}^{\otimes r}\times  {\cT_{\scriptstyle M}^\star}^{\otimes s},$

defined by

\begin{align*}
T_{\scriptscriptstyle M}:\underbrace{ \cT_{\scriptstyle M}\times \dots \times \cT_{\scriptstyle M} }_{r}\times \underbrace{\cT_{\scriptstyle M}^\star \times\dots \times \cT_{\scriptstyle M}^\star }_{s}\to & \bbR.\\
(X_{1},\quad, \dots,\quad \, X_{r}, \omega_{1},\quad\cdots,\quad\omega_{s})\mapsto & T_{\scriptscriptstyle M}(X_{1},\dots, X_{r},\omega_{1},\dots,\omega_{s}).\\
\end{align*}

Furthermore, this object adheres to the following properties. 
 \[
\begin{aligned} 
\hspace{.2cm}T_{\scriptscriptstyle M}(X_{1},\dots,&\alpha X_{j}+ \beta Y_{j},\dots, X_{r},\omega_{1},\dots,\omega_{s})\\&=
\alpha T_{\scriptscriptstyle M}(X_{1},\dots, X_{j},\dots, X_{r},\omega_{1},\dots,\omega_{s})+\beta T_{\scriptscriptstyle M}(X_{1},\dots, Y_{j},\dots, X_{r},\omega_{1},\dots,\omega_{s}),\\
\hspace{.2cm}T(X_{1},\dots,&, X_{r},\omega_{1}\dots,\alpha\omega_{k}+\beta \eta_{k},\dots,\omega_{s})\\&=
\alpha T_{\scriptscriptstyle M}(X_{1},\dots, X_{r},\omega_{1},\dots,\omega_{k},\dots,\omega_{s})+\beta T_{\scriptscriptstyle M}(X_{1},\dots,  X_{r},\omega_{1},\dots,\eta_{k},\dots,\omega_{s}),\\
\end{aligned}\]
for scalars $\alpha,\beta \in \bbR$ and all $1\leq j\leq r$ as well as $1\leq k\leq s$.

Moreover, if $T_{\scriptscriptstyle M}$ and $S_{\scriptscriptstyle M}$ are two tensor of the same type $(r,s)$, then
\begin{equation}
\begin{aligned}
(\alpha T_{\scriptscriptstyle M}+ &\beta S_{\scriptscriptstyle M})(X_{1},\dots, X_{r},\omega_{1},\dots,\omega_{s})\\
&=\alpha T_{\scriptscriptstyle M}(X_{1},\dots, X_{r},\omega_{1},\dots,\omega_{s})+ \beta S_{\scriptscriptstyle M}(X_{1},\dots, X_{r},\omega_{1},\dots,\omega_{s} ), \end{aligned}\end{equation}
and tensors of a given type $(r,s)$ span a linear vector space of dimension $n^{r+s}$.
\end{definition}


\,

The following denominations are often used in the literature. 
\begin{itemize}
\item A (covariant) tensor of order $r$ at $\scriptstyle M$ is a tensor on $\cT_{\scriptstyle M}(\cM)^{\otimes r}$. It is of type $(r,0)$\\
\item  A contravariant tensor of order $s$ at $\scriptstyle M$, is a tensor on ${\cT^\star _{\scriptstyle M}(\cM)}^{\otimes s}$. It is of type $(0,s)$.
\end{itemize}

\,
\subsection{Tensor products}
\begin{definition}[Tensor product]\index{Tensor!Tensor product}
Let $T_{\scriptscriptstyle M}$ be a tensor at the point $\scriptstyle M\in \cM$ of type $(r,s)$ and let $S_{\scriptscriptstyle M}$ be a tensor of type $(p,q)$. Then, the tensor $T_{\scriptscriptstyle M}\otimes S_{\scriptscriptstyle M}$ defined by
\begin{equation}\begin{aligned}
T_{\scriptscriptstyle M}&\otimes S_{\scriptscriptstyle M}(X_{1},\dots,X_{r},\dots,X_{r+p},\omega_{1},\dots,\omega_{s},\dots,\omega_{s+q})\\
&=T_{\scriptscriptstyle M}(X_{1},\dots,X_{r},\omega_{1},\dots,\omega_{s})\times  S_{\scriptscriptstyle M}(X_{r+1},\dots,X_{r+p},\dots,\omega_{s+1},\dots,\omega_{s+q}).
\end{aligned}\end{equation}
is called the tensorial product of $T_{\scriptscriptstyle M}$ and $S_{\scriptscriptstyle M}$.
\end{definition}


Admitting this definition, an $(r,s)$-tensor $T$ can be seen as the tensorial product of $r$ covariant vectors and $s$ tangent vectors at $\scriptstyle M \in \cM$ as we can see: 
\[
T(X_{1},\dots,X_{r},\omega_{1},\dots,\omega_{s})= T_{(1)}\otimes\dots\otimes T_{(r)}\otimes T^{(1)}\otimes\dots\otimes T^{(s)}(X_{1},\dots,X_{r},\omega_{1},\dots,\omega_{s}),\]
where the $T_{(i)}$ are covectors and the $T^{(k)}$ are contravectors. 

BY abuse of notation, we have omitted the index $\scriptscriptstyle M$ to have a more convenient formula. However, it is important to remember that the definition is local.

\,

To conclude, the space of tensors of type $(r,s)$ can be identified with the tensorial product space:	
\[ \cT_{\scriptstyle M}^{\star}(\cM)^{\otimes r} \otimes T_{\scriptstyle M}(\cM)^{\otimes s }.\]	

\,

%Assiume that we have the basis  $\{e_{i}\}_{i=1}^{n}$	of $T_{_{M}}(\cM)$. 

Let us choose a basis of $ T_{_{M}}^{\star}(\cM)^{\otimes r} \otimes T_{_{M}}(\cM)^{\otimes s }$ given by the $n^{(r+s)}$ vectors:

\begin{equation}
 \mathbf{e}^{i_{1}\dots i_{r}}_{\phantom{i_{1}\dots i_{r}}j_{1}\dots j_{s}}= \epsilon^{i_{1}}\otimes\dots\otimes\epsilon^{i_{r}}\otimes e_{j_{1}}\otimes\dots\otimes e_{j_{n}}\end{equation}  
where $\{\epsilon^{i}\}_{i=1}^{n}$ is the dual basis of  $\{e_{j}\}_{j=1}^{s}$.

\vspace{3pt}

If we work in the framework of local coordinate basis the associate basis of the $(r,s)$-tensor space is
\[
\mathbf{e}^{i_{1}\dots i_{r}}_{\phantom{i_{1}\dots i_{r}}j_{1}\dots j_{s}}=dx^{i_{1}}\otimes\dots\otimes dx^{i_{r}}\otimes \partial_{j_{1}}\otimes\dots\otimes \partial_{j_{n}}=\bigotimes_{k=1}^{r}dx^{i_{k}} \bigotimes_{\ell=1}^{s}\partial_{j_{\ell}}, \quad 1\leq i_{_{k}}, j_{_{\ell}}\leq n,
\]
 then the  $(r,s)$-tensor $T$ is given by:

\begin{equation}
T=\sum_{i_{k}=1\atop 1\leq k\leq r}\sum_{j_{\ell}=1\atop 1\leq\ell\leq s}T_{i_{1}\dots i_{r}}^{\phantom{i_{1}\dots i_{r}}j_{1}\dots j_{s}}dx^{i_{1}}\otimes\dots\otimes dx^{i_{r}}\otimes \partial_{j_{1}}\otimes\dots\otimes \partial_{j_{n}}.
\end{equation}

To enhance the clarity, it is advantageous to adopt the Einstein summation convention, where repeated upper and lower indices are implicitly summed over. In the following formula, we illustrate this convention.

\begin{equation}
T=T_{i_{1}\dots i_{r}}^{\phantom{i_{1}\dots i_{r}}j_{1}\dots j_{s}}dx^{i_{1}}\otimes\dots\otimes dx^{i_{r}}\otimes \partial_{j_{1}}\otimes\dots\otimes \partial_{j_{n}},
\end{equation}

where the components are given by,
\begin{equation}
T_{i_{1}\dots i_{r}}^{\phantom{i_{1}\dots i_{r}}j_{1}\dots j_{s}}=\partial_{i_{1}}\otimes\dots\otimes \partial_{i_{r}}\otimes dx^{j_{1}}\otimes\dots\otimes dx^{j_{n}}(T)= \bigotimes_{k=1}^{r}\partial_{i_{k}} \bigotimes_{\ell=1}^{s}dx^{j_{\ell}}(T),
\end{equation}

Under a change of coordinates, an $(r,s)$-tensor transforms via 
$r$ applications of the Jacobian (for covariant indices) and 
$s$ applications of its inverse (for contravariant indices):

%contracting its components with $r$ copies of the Jacobian matrix and $s$ copies of the inverse Jacobian matrix

%Under a change of charts, an $(r,s)$-tensor is transformed as a product of $r$ covariant vector and $s$ contravariant vectors
\begin{equation}
T_{i'_{1}\dots i'_{r}}^{\phantom{i'_{1}\dots i'_{r}}j'_{1}\dots j'_{s}}=\prod_{k=1}^{r}\frac{\partial x^{i_{k}}}{\partial x^{i'_{k}}} \prod_{\ell=1}^{s}\frac{\partial x^{j'_{\ell}}}{\partial x^{j_{\ell}}}T_{i_{1}\dots i_{r}}^{\phantom{i_{1}\dots i_{r}}j_{1}\dots j_{s}},
\end{equation}


We leverage this framework to study  symmetries of tensors. The study of tensor symmetries within this framework reveals fundamental insights into their invariant properties and multilinear algebraic behaviour.

\subsection{Symmetry properties of tensors}\index{Tensor!Symmetry properties}

Let us first consider a (covariant) tensor of order $r$. Let $S_{r} $ be the group of permutations of the $r$ integers $\{1,2,\dots,r\}$. By definition, it acts on the tangent space at $\scriptstyle M \in \cM$ in the following way:
\[ (\sigma T)(X_{1},\dots,X_{r})= T(X_{\sigma (1)},\dots,X_{\sigma (r)}), \quad \sigma \in S_{r},\ X_{k}\in T_{_{M}}(\mathcal{ M}),\]
or in local coordinates
\[   (\sigma T)_{i_{1}\dots i_{r}}= T_{\sigma(i_{1})\dots \sigma (i_{r})}, \]
so that $T$ has the symmetry defined by $\sigma$.
\begin{definition} 

\ 

$\bullet$   If $\sigma T=T$  then the tensor $T$ is said to be symmetric.

\vspace{3pt}
$\bullet$  If $\sigma T=sign(\sigma)\,T$, with $sign(\sigma)=\pm1$ (depending on  whether the permutation is even or odd)  the tensor $T$ is said to be antisymmetric.
\end{definition}

One can define a symmetrization operator $S$ as well as an antisymmetrization operator $A$ on the (covariant) tensor $T$. This is done below: 

\begin{equation}\begin{aligned}
ST&=\frac{1}{r!}\sum\limits_{\sigma\in S_{r}} \sigma\, T:\,  \text{ a completely symmetric tensor.}\\\\
AT&= \frac{1}{r!}\sum\limits_{\sigma\in S_{r}} sign(\sigma)\,\sigma \,T:\,  \text{ a completely antisymmetric tensor,}
\end{aligned}\end{equation}

which in local coordinates is defined by 

\begin{equation}(AT)_{i_{1}\dots i_{r}}=\frac{1}{r!}\epsilon_{i_{1}\dots i_{r}}^{k_{1}\dots k_{r}}T_{k_{1}\dots k_{r}}\end{equation} where we have used the Einstein summation rule and 
\[ 
\epsilon_{i_{1}\dots i_{r}}^{k_{1}\dots k_{r}}=\begin{cases}
\phantom{+}0& \text{if } ( k_{1}\dots k_{r}) \text{ is not a permutation of } (i_{1}\dots i_{r})\\
+1& \text{if } ( k_{1}\dots k_{r}) \text{ is an even permutation of }(i_{1}\dots i_{r})\\
-1& \text{if } ( k_{1}\dots k_{r}) \text{ is an odd permutation of }(i_{1}\dots i_{r})\\
\end{cases}
\] 
is the Kronecker tensor.

\,

The symmetry properties of a contravariant tensor is defined similarly. 

\begin{remark}
The permutation $\sigma$ permutes only the labels of objects of the same nature. For a $(r,s)$-tensor, the symmetry and antisymmetry properties are defined only for indices of the same nature. 
It is usual to use square brackets $[\cdot]$ for the antisymmetry and   round brackets ($\cdot$) for symmetry. For example, the $(4,3)$-tensor $T_{[abc]d}^{\phantom{[abc]d}(ef)g}$ is antisymmetric in $a,b,$ and symmetric in $e,f$.
\end{remark}


