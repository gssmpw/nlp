

\chapter{Topological and Modelled manifolds}
\begin{center}
    \begin{minipage}{0.6\textwidth} % 'r' for right, width of the figure
  \centering
  \includegraphics[scale=0.5]{charts.png}
\end{minipage}
\end{center}

A topological manifold is a fundamental object in mathematics, serving as an abstraction of spaces that locally resemble Euclidean space but  exhibit additional properties. Manifolds provide the natural setting for geometry and topology, forming the foundation for different disciplines in mathematics.

Formally, an $n$-dimensional topological manifold is a Hausdorff topological space that is locally homeomorphic to $\bbR^n$. This means that around every point, there exists a neighborhood that behaves like an open subset of the Euclidean space, allowing us to use an Euclidean space type of intuition, while still permitting the existence of global structures (namely curvature, holes, or any other topological feature).

\, 

Topological manifolds can be studied with some extra structures:

\begin{itemize}
    \item Differentiable manifolds, where smooth functions and derivatives can be defined, leading to differential geometry.
    \item Riemannian manifolds, which are equipped with a Riemannian metric.
    \item Lie groups: these are topological spaces that also possess a group structure, playing a crucial role in symmetry and mathematical physics.
\end{itemize}
The simplest examples of topological manifolds include familiar spaces such as the circle $S^1$, the sphere $S^n$ the torus $T^n$, as well as more sophisticated objects like projective spaces.
\section{Definitions}

\, 

\begin{definition}[Topological Manifolds]\label{D:topman}\index{Manifold!Topological Manifold}
Let $\cM$ be a set, and let $(\cE_i)_{i\in I}$ be a family of Hausdorff topological vector spaces. Consider a collection $\cA$ of pairs $\{(\cU_i, \varphi_i)\}_{i \in I}$, indexed by some set $I$, where each $\cU_i$ is a subset of $\cM$ and each $\varphi_i$ is a mapping associated to it, subject to the following axioms:

\begin{itemize}
    \item[$\mathbf{(\cM1)}$] The sets $\cU_i$ form an open cover of $\cM$:
    \[
    \cM = \bigcup_{i \in I} \cU_i, \quad \cU_i \subset \cM.
    \]
    
    \item[$\mathbf{(\cM2)}$] Each $\varphi_i$ is a bijection between $\cU_i$ and an open subset of $\cE_i$:
    \[
    \varphi_i: \cU_i \to \varphi_i(\cU_i) \subset \cE_i.
    \]

    \item[$\mathbf{(\cM3)}$] For each pair of indices $i, j \in I$, the transition maps
    \[
    \varphi_j \circ \varphi_i^{-1}: \varphi_i(\cU_i \cap \cU_j) \to \varphi_j(\cU_i \cap \cU_j)
    \]
    define homeomorphisms between the respective open subsets of $\cE_i$ and $\cE_j$.

    \item[$\mathbf{(\cM4)}$] Unless otherwise specified, we assume that each $\cE_i$ is a locally convex Hausdorff topological vector space.
\end{itemize}

The data $(\cM, \cA)$ thus defines a \textit{topological manifold} modeled on the spaces $\cE_i$.
\end{definition}



%****
%\begin{definition}[Topological manifolds]\label{D:topman}\index{Manifold!Topological Manifold}
%Let $\cM$ be a set, $\cE_{i}$ a family of topological Hausdorff vector spaces. Let $\cA$  a collection of pairs $\{(\cU_i,\varphi_i)\}_{i\in I}$, for some index set $I$, of subsets of  $\cM$, satisfying the following conditions:

%\begin{itemize}
%\item $\cM 1.$ Each $\cU_i$ is a subset $\cM$ and the 
%$\cU_i$ cover $\cM$
%\[
%\cM=\bigcup_{i\in I} \cU_i,\quad %\cU_i \subset \cM.
%\] 
%\item $\cM 2.$ Each $\varphi_i$ is a bijection of $\cU_i$ onto an open subset
%$\varphi_i(\cU_i)$ of $\cE_{i}$.

%\item $\cM 3.$ The map
%\[
%\varphi_j\circ\varphi_i^{-1}:\varphi_i(\cU_i\cap \cU_j) \to \varphi_j(\cU_i\cap
%\cU_j),
%\]
%is an homeomorphism\index{Topology!Homeomorphism}\index{Mapping!Homeomorphism}  from $\cE_{i}$ to $\cE_{j}$ for each pair of indices $i,j$, and for any $i,j\in I$ $\varphi_i(\cU_i\cap \cU_j)$ is open in $\cE_i$.

%\item $\cM 4.$ In the following, unless special specification, we ask that the topological vector spaces
 %was Hausdorff locally convex topological space.
%\end{itemize}
%\end{definition}
\begin{figure}[h]
\includegraphics[scale=0.45]{chart1_LG23.pdf} 
\caption{Charts and atlas.}\label{F:chart1}
\end{figure}

\section{Charts, Atlas}

\begin{definition}[Chart]\index{Manifold!Chart}
Let $\cM$ be a topological manifold. Each pair $(\mathcal{U}_i, \varphi_i) \in \mathcal{A}$, where $\mathcal{U}_i$ is an open subset of $\cM$ and $\varphi_i$ is a mapping of $\mathcal{U}_i$ onto an open subset of a topological vector space, is called a \textit{chart} (or \textit{coordinate system}) of $\cM$.

If a point \textsc{m} $\in \mathcal{M}$ lies in $\mathcal{U}_i$, we say that $(\mathcal{U}_i, \varphi_i)$ is a chart at \textsc{m}.
\end{definition}

\begin{definition}[Atlas]\index{Manifold!Atlas}
An \textit{atlas} on a manifold $\cM$ is a collection of charts $\{(\mathcal{U}_i, \varphi_i)\}_{i \in I}$ such that:
\begin{itemize}
    \item[$\mathbf{(A1)}$] The sets $\mathcal{U}_i$ cover $\cM$:
    \[
    \cM = \bigcup_{i \in I} \mathcal{U}_i.
    \]
    \item[$\mathbf{(A2)}$] The transition maps
    \[
    \varphi_j \circ \varphi_i^{-1}: \varphi_i(\mathcal{U}_i \cap \mathcal{U}_j) \to \varphi_j(\mathcal{U}_i \cap \mathcal{U}_j)
    \]
    are homeomorphisms for all $i, j$ for which $\mathcal{U}_i \cap \mathcal{U}_j \neq \emptyset$.
\end{itemize}
\end{definition}

It is natural to assume that the topology on $\cM$ is given a priori. In many cases, one requires that each chart $\varphi_i$ be a homeomorphism onto its image. As a first consequence of this definition, we obtain the following structural result:
\begin{proposition} One can give to $\cM$ a topology in a unique way such that each
$\cU_i$ is open, and the $\varphi_i$ are topological isomorphisms.
\end{proposition}

%\begin{proposition}
%There exists a unique topology on $\cM$ for which each $\mathcal{U}_i$ is open and the maps $\varphi_i$ are topological isomorphisms.
%\end{proposition}

%******
%\begin{definition}[Chart]\index{Manifold!Chart}
%Let $\cM$ be a topological manifold. Each pair $(\mathcal{U}_i,\varphi_i)\in \cA$ is called a chart\index{Chart} (or coordinate system).

%IF a point {\scshape m} $ \in \mathcal{M}$ lies in $\mathcal{U}_i$, we say that $(\mathcal{U}_i,\varphi_i)$ is
%a chart at {\scshape m}.
%\end{definition}

%\begin{definition}[Atlas]\index{Manifold!Atlas}
%An atlas\index{Atlas} on a manifold $\cM$ is a set $\{(U_{i},\varphi_i)\}$ of charts of  $\cM$ such that the  $\{U_{i}\}$ is a covering of $\cM$ and verify the compatibility $\cM 3$.
%\end{definition}



%For a general definition it is natural to assume that the topology %on $\cM$ is given. In many cases the  %topology on the manifold $\cM$ is %given by asking that all the charts %$\varphi_{\alpha}$  be homeomorphisms.
%Then as a first consequence the %definition we  can always  seen the %manifold $\cM$ as a topological space:

%\begin{proposition} One can give to %$\cM$ a topology in a unique way such %that each
%$\cU_i$ is open, and the $\varphi_i$ %are topological isomorphisms.
%\end{proposition}


\begin{remark}
The condition that $\cM$ is  Hausdorff,  is not necessary.
This condition plays no role in the formal development of manifold.  

However, in practical applications,  $\cM$ is Hausdorff.
\end{remark}
\begin{ex}\label{Ex:smfd}
    Prove that the Special Linear group $SL_n(\R)$ is a smooth submanifold of $\bbR^{n\times n}$.
\end{ex}
\begin{ex}\label{Ex:sing}
Is the set $\{{x,y}\in \bbR^2\, |\, xy=0\}$ a submanifold of $\bbR^2$ ?
\end{ex}

\section{Modeled Manifolds}
\subsection*{Hausdorff Condition and Separation Axioms}
$\star$ In what follows, we impose the Hausdorff condition on all manifolds under consideration. Furthermore, any construction performed subsequently—such as products, tangent bundles, or fibered structures—will be required to yield spaces that remain Hausdorff. This ensures a well-behaved topological framework in which geometric operations preserve separation properties.

\subsection{\(\mathcal{E}\)-modeled manifolds}
%\subsection{Locally Convex Topological Vector Spaces}

In the formulation given above, particularly in condition $\cM2$, no global assumption was imposed on the structure of the topological vector spaces $\mathcal{E}_i$ used as local models. In particular, we did not require that all $\mathcal{E}_i$ be the same, nor that there exist continuous linear isomorphisms between them.

However, one of the most natural and frequently encountered cases in practice is that in which there exists a fixed topological vector space $\mathcal{E}$ such that each $\mathcal{E}_i$ is isomorphic to $\mathcal{E}$, allowing a uniform model for the local structure of $\mathcal{M}$. In such a setting, one can regard $\mathcal{M}$ as being locally modeled on a single space $\mathcal{E}$, which provides a more rigid but also more manageable framework for various constructions.


%%%%%%
 %In the following,  we impose the Hausdorff condition, and then  impose a separation condition to any construction which we perform in the following (like: products, tangent bundle, etc.) would yield Hausdorff spaces.

%\section{Modeled manifold}

%\subsection{Locally convex topological modeled manifolfd}

% In condition $\cM 2$, we did not require that the topological vector spaces $\mathcal{E}_i$ be the 
%same for all indices $i$, or even that there exist  continuous linear isomorphism between them.
%However the case where  $\cE_i$ isomorphic to $\cE$ for all $i$ being one of the most used in practice.

\vspace{3pt}
This remark leads us to the notion of modeled manifold:
\begin{figure}[h]
\includegraphics[scale=0.5]{chart2_LG23.pdf}
\caption{Modeled manifold }\label{F:Chart2}
\end{figure}

\begin{definition}[Modeled manifold]\index{Manifold!Modeled manifold}

\noindent
Let \(\mathcal{M}\) be a topological space. An \(\mathcal{E}\)-\textit{atlas} on \(\mathcal{M}\) consists of a collection of charts \(\{(U_i, \varphi_i)\}_{i \in I}\), where:

\begin{enumerate}
    \item \(\{U_i\}_{i \in I}\) is an open cover of \(\mathcal{M}\),
    \item Each \(\varphi_i: U_i \to \mathcal{E}_i\) is a homeomorphism onto its image, where \(\mathcal{E}_i\) is a topological space,
    \item All the spaces \(\mathcal{E}_i\) are homeomorphic to a fixed topological space \(\mathcal{E}\), i.e., there exist homeomorphisms \(h_i: \mathcal{E}_i \to \mathcal{E}\).
\end{enumerate}

If an \(\mathcal{E}\)-atlas is given, we say that \(\mathcal{M}\) is an \(\mathcal{E}\)-\textit{modeled manifold} (or simply an \(\mathcal{E}\)-\textit{manifold}).


%If the $\cE_i$ are isomorphic, for all $i$, to a topological space $ \cE$  we say that the atlas is an $\cE$-atlas. The manifold $\cM$ is said a modeled manifold (or $\cE$-manifold).
\end{definition}

\noindent

If the transition maps (i.e. maps of the type \(\varphi_j \circ \varphi_i^{-1}\)) satisfy additional compatibility conditions (such as differentiability, smoothness, or analyticity), then  the \(\mathcal{E}\)-manifold is endowed with some extra structure.


\vspace{5pt}% W: Locally convex topological vector space 

In the following, we primarily focus on modeled manifolds whose model space is a vector space that admits a well-defined differentiable structure. This is the case for manifolds modeled on Banach spaces or Hilbert spaces, where a differentiable structure can be naturally defined. However, we are also interested in a generalization of normed spaces, namely locally convex topological vector spaces. 

\, 
\begin{example}
A Hilbert manifold is a manifold modeled on a Hilbert space, which is a complete inner product space. 
Hilbert manifolds generalize finite-dimensional smooth manifolds to infinite dimensions.

\begin{itemize}
    \item {\bf Model Space:} Assume $H$ is a separable Hilbert space. One can take for instance the space of square integrable functions $L^2(\R)$.
    \item {\bf Charts:} A Hilbert manifold $M$ is a topological space equipped with an atlas of charts where $U_a\subseteq M$ is an open set and $\phi_a:U_a\to H$ is a homeomorphism onto an open subset of $H$.
    \item {\bf Transition Maps:} Given a pair of charts $(U_a,\phi_a)$ and $(U_b,\phi_b)$, the transition map 
    \[\phi_b\circ \phi_a^{-1}:\phi_a(U_a\cap U_b)\to \phi_b(U_a\cap U_b) \]
    is a smooth map between open subsets of $H$.
    \item {\bf Local Structure:} At every point $p\in M$, the tangent space $T_pM$ is isomorphic to $H$, the Hilbert space. 
\end{itemize}
\end{example}

\,

\begin{ex}\label{Ex:LoopSpace}
Consider a smooth finite-dimensional manifold $M$. The loop space $LM$ is the space of all smooth maps:

\[\gamma:S^1\to M,\]
where $S^1$ is the unit circle. Is the loop space $LM$ a modeled manifold? 
\end{ex}

\begin{ex}\label{Ex:BanSpace}
Let $M$ and $N$ be finite dimensional smooth manifolds. Let $C^k(M,N)$ be the space of $C^k$-differentiable maps from $M$ to $N$. Prove that  this space can be given the structure of a Banach manifold modeled on a Banach space.
\end{ex}
%Functions are defined as being differentiable in some open neighbourhood of  $x$, rather than at individual points, as not doing so tends to lead to many pathological counterexamples.




%In the following we are mainly interested by modeled manifold on vector spaces on which a differentiable structure can be defined. This is the case for manifold modeled on a Banach space or on a Hilbert space. But we are also interested by a generalization of normed space more precisely by  locally topological vector spaces. However the existence  a differentiable structure does not exists for all arbitrary locally convex topological space. But a class of  locally topological vector spaces, the Fr\'echet spaces,  play an important role in statistics or partial differential equations. A Fr\'echet space $X$  is defined to be a locally convex metrizable topological vector space that is complete, meaning that every Cauchy sequence in  $X$ converges to some point in $X$.

\subsection{Recollections on Locally Convex  Spaces}
%A \emph{Fréchet space} $X$ is defined as a locally convex, metrizable, and complete topological vector space, meaning that every Cauchy sequence in $X$ converges to a point in $X$.

\, 

Let us recall that a {\bf locally convex  space} $X$ is a vector space over $\bbK$, a (sub)field of the complex numbers (it can be $\bbC$ itself or $\bbR$ for instance).

A locally convex space is defined either in terms of convex sets or equivalently in terms of seminorms.
In fact, a topological vector space $X$ is said to be locally convex if it verifies one of the following two equivalent properties. We start with the convex set definition.

\begin{definition}[{\bf Convex sets definition}]
A topological vector space $X$ is said to be locally convex if there exists  a neighborhood basis (that is, a local base) at the origin, consisting of balanced convex sets. 

We elaborate on those two notions of convexity and of balanced sets. 

\begin{enumerate}
\item A subset $C\in  X$ is called \emph{convex} if for all $ x,y\in C$, and $0\leq t\leq 1$ we have 
\[tx+(1-t)y\in C.\] 
\item A convex subset $C\in  X$ is called 
\emph{balanced} if for all $x\in C$ and scalars $s\in \bbK$, the condition $|s|\leq 1$ implies that $ sx\in C$.
\item A convex subset $C\in  X$ is called a \emph{cone} (we assume here that the underlying field is ordered) if for all  $x\in C$ and 
$ t\geq 0$ we have $tx\in C$.
\item A convex subset $C\in  X$ is called  \emph{absorbent}  if for every $ x\in X$ there exists $r>0$ such that 
\[ x\in t\cdot C\] for all $t\in \bbK$, and  $|t|>r$. The set 
$C$ can be scaled out by any "large" value to absorb every point in the space.
\end{enumerate} 
\end{definition}
In any topological vector space, every neighborhood of the origin is absorbent.

\, 

A second possible viewpoint on this notion can be achieved using seminorms. A \emph{seminorm} on $X$ is a function \[p: X \to \bbR\] such that:

\begin{enumerate}
\item {\bf Non-negativity:} $p$ is nonnegative or positive semidefinite i.e. \[p(x)\geq 0,\quad \text{for all}\quad x\in X.\]

\item {\bf Scaling property}: $p$ is positive homogeneous or positive scalable: \[p(sx)=|s|\cdot p(x),\quad \text{for every scalar} \quad s \, \text{and}\, x\in X.\] So, in particular, $p(0)=0$,

\item {\bf Subadditivity}: $ p$ is subadditive and it satisfies the triangle inequality: \[p(x+y)\leq p(x)+p(y), \quad \text{for all}\quad x,y\in X.\]
\end{enumerate}

\begin{definition}{\bf (Seminorms definition)} A topological vector space $X$ over a field $\bbC$ or $\bbR$ is said to be locally convex if there exists a family $\cP$ of \emph{seminorms} on $X$.
Let $\{p_i\}_{i\in I}$ be a family of semi-norms on 
$X$, where $I$ is an index set. Each semi-norm $p_i:X\to \bbR$ satisfies the properties of a semi-norm (non-negativity, absolute homogeneity, and subadditivity).
\end{definition}
\,

\begin{remark}
    Although the definition in terms of a neighborhood base gives a better geometric picture and intuition the definition in terms of seminorms is easier to work with, in practice.

\end{remark}
    
\begin{ex}\label{Ex:seminorm}
Consider the space of continuous functions $C(\bbR)$ on $\bbR$. Verify whether the family of functions $\{p_n\}_{n\in \bbN}$,
given by: \[p_n(f)=\sup_{x\in [-n,n]}|f(x)|\] forms a family of seminorms.  
Is $C(\bbR)$ locally convex? Do the $\{p_n(f)\}$ generate a locally convex topology?
\end{ex}
\subsection{Bridging those two definitions}

\,
 As mentioned earlier, both definitions are equivalent.  We outline a sketch of the proof showing this equivalence. 

\, 

$\star$ The equivalence of those two definitions follows from a construction known as the \emph{Minkowski functional} or Minkowski gauge. The key feature of seminorms which ensures the convexity of their $\varepsilon$-balls is the {\it triangle inequality}.

\vspace{3pt}

For an absorbing set $C$ such that: if $x\in C$ then $ t\cdot x\in C$, whenever $ 0\leq t\leq 1$, let us define the Minkowski functional of $C$ to be \[\mu _{C}(x)=\inf\{r>0 : x \in r\, C\}.\]

\vspace{3pt}

From this definition, it follows that $\mu _{C}$ is a seminorm if $C$ is balanced and convex (it is also absorbent by assumption). 

\, 

Conversely, given a family of seminorms, the sets
\[\left\{x:p_{\alpha _{1}}(x)<\varepsilon _{1},\ldots ,p_{\alpha _{n}}(x)<\varepsilon _{n}\right\}\] form a base of convex absorbent balanced sets.


\vspace{5pt}
Let us mention some interesting properties.
\begin{itemize}
\item If $p$ is positive definite (which states that if $p(x)=0$ then 
 $x=0$) it implies that $p$ is a norm. 

\,

$\star$ While in general seminorms {\it need not} be norms, there is an analogue of this criterion for families of seminorms and separatedness, defined below.\\

\item If $X$ is a vector space and $\cP$ is a family of seminorms on $X$ then a subset $\mathcal{Q} \subset \cP$ is called a base of seminorms for $\cP$ if for all $p\in \cP$ there exists a $q\in \mathcal{Q} $ and a real number $r>0$ such that $p\leq rq$.\\

\item If $X$ is a real vector space and $C$ a convex subspace, if $x\in C$, then $C$ contains the line segment between $x$ and $-x$.\\
 
\item If $X$ is a complex vector space and $C$ a convex subspace,  for any $ x\in C$ convex space, contains the disk with $x$ on its boundary, centered at the origin, in the one-dimensional complex subspace generated by $x$.
\end{itemize}


\noindent
\subsection{Relations to the Banach Space Implicit Function Theorem}
A \textit{locally convex topological vector space} is a topological vector space in which every neighborhood of \(0\) contains an open neighborhood \(U\) of \(0\) such that, for all \(x, y \in U\) and \(0 \leq t \leq 1\),
\[
t\cdot x + (1-t)\cdot y \in U.
\]
This property turns out to be essential in the context of the implicit function theorem, for Banach spaces.

\vspace{0.5cm}

\noindent
The latter result, known as the \textbf{Banach Space Implicit Function Theorem}, can be stated as follows:

\, 

Let \(X\), \(Y\) and \(Z\) be Banach spaces and let \(\mathcal{U}\) be an open subset of \(X \times Y\). Suppose that \(f : \mathcal{U} \to Z\) is a continuously differentiable (\(C^1\)) mapping such that \(f(a, b) = 0\) for some \((a, b) \in \mathcal{U}\) and the partial derivative \(D_y f(a, b)\) is a linear isomorphism from \(Y\) onto \(Z\).

Then, there exists:

\begin{itemize}
    \item an open set \(\mathcal{W} \subset X\) with \(a \in \mathcal{W}\);
    \item an open set \(\mathcal{V} \subset \mathcal{U} \subset X \times Y\) with \((a, b) \in \mathcal{V}\);
    \item a continuously differentiable (\(C^1\)) mapping \(g : \mathcal{W} \to Y\),
\end{itemize}

such that
\[
(x, y) \in \mathcal{V},\, f(x, y) = 0 \iff x \in \mathcal{W},\, y = g(x).
\]

%A locally convex topological space is a locally convex space and for such a space, every neighborhood of $0$ contains an open neighborhood $U$ of $0$, such that: if $x$ and $y$ are in $U$ and $0\leq t\leq 1$, then \[tx+(1-t)y\] also lies in $U$. This is because of the implicit function theorem.

%\, 

%The latter---known as the {\bf %Banach space implicit function %theorem}---goes as follows. 

%Let $X,Y,Z$ be Banach spaces; let %$\cU$ be an open subset of $X\times %Y$. Let $f:\cU\to Z$ be a $C^1$-%mapping, such that $f(a,b)=0$ for %$(a,b)\in \cU$ and the partial %derivative $f_y''(a,b)$ is a %linear isomorphism de $Y$ onto $Z$. Then, there exists an open %%set$\cW\subset X$, $a\in \cW$, an %open set $\cV\subset \cU \subset %X\times Y $, $(a,b)\in \cV$ and a %$\cC^1$-mapping$g:\cW\to Y$ such %%that
%\[
%(x,y)\in \cV,\, f(x,y)=0 %\Leftrightarrow x\in \cW,\, y=g(x).
%\] 


In other words, the implicit equation $f(x,y)=0$ has for $x\in W$ a solution $y=g(x)$ of class $C^1$
such that $(x,y)\in \cV$. 

This solution is unique in an open set $\cW'\subset \cW$.

\medskip 


$\star{263C}$ Warning! It is important to note that \emph{not} every locally convex topological vector space admits a differentiable structure in a meaningful way. Nevertheless, a particular class of locally convex topological vector spaces, known as \emph{Fréchet spaces}, plays a fundamental role in various areas of mathematics, such as statistics and the theory of partial differential equations.

\, 

A \emph{Fréchet space} $X$ is defined as a locally convex, metrizable, and complete topological vector space, meaning that every Cauchy sequence in $X$ converges to a point in $X$. 

 For general normed vector space $X,Y$,  the Fr\'echet directional derivative  of a function  $ f:\cU\to Y$, where $ \cU$ is an open subset of $X$, exists at $ x\in \cU$  if there exists a bounded linear operator $A:X\to Y$ such that
\[
 \lim _{\|h\|\to 0}{\frac {\|f(x+h)-f(x)-Ah\|_{Y}}{\|h\|_{X}}}=0.
 \]
The Fréchet derivative in finite-dimensional spaces is the usual derivative.  it is represented in coordinates by the Jacobian matrix.

\section{Exercises}

Ex. \ref{Ex:LoopSpace} The loop space $LM$ is modeled on the Hilbert space $H=L^2(S^1,\bbR^n)$, where $n=dim(M)$. 
Each loop $\gamma$ can be locally approximated by functions in $L^2(S^1,\bbR^n)$.
For a fixed loop $\gamma\in LM$, a chart around $\gamma$ can be constructed using the exponential map on $M$. For a small neighborhood $U$ of $\gamma$, the chart maps $U$ to an open subset of $H$.

\, 

Ex. \ref{Ex:BanSpace} This space can be given the structure of a Banach manifold, modeled on a Banach space of sections of a vector bundle. The model space is the Banach space $C^k(M,\bbR^n)$ where $n=dim(N)$. For a fixed map $f\in C^k(M,N)$, a chart around $f$ can be constructed using the exponential map on $N$. Specifically, for a small neighborhood $U$ of $f$, the chart maps $U$ to an open subset of $C^k(M,\bbR^n)$.

\, 

Ex. \ref{Ex:seminorm}. Yes, each $p_n$ forms a seminorm. The family $\{p_n\}_{n\in \bbN}$ generates a locally convex topology on $C(\bbR)$. In this topology, a sequence of functions $f_k$ converges to $f$ if and only if $p_n(f_k-f)\to 0$, for every $n\in \bbN$.  

\, 

Ex. \ref{Ex:smfd}.
This exercise is slightly more advanced since this statement can be shown using the Regular Value Theorem, which we have not considered in this book. We will use this exercise as a possibility to mention and illustrate this theorem! 

\, 
\begin{enumerate}
    \item \begin{itemize}
    \item A first step, is to identify the space of all $n\times n$ matrices to the Euclidean space $\bbR^{n^2}$. 
\item The determinant function $\det$ is a smooth (infinitely differentiable) function since it is a polynomial in the entries of the matrix.
\item By definition,  the special linear group is a set of matrices of size $n\times n$ where the determinant is equal to 1. In other words, this is the preimage of the value 1, under the determinant function.
\end{itemize}
\item The Regular Value Theorem states that if \[f:M\to N\] is a smooth map between two smooth manifolds and $y\in N$ is a regular value of $f$, then $f^{-1}(y)$ is a smooth submanifolds of $M$. In our case, we have $M=Mat_n(\bbR)=\bbR^{n^2}$, $N=\bbR$ and $f=\det$. One needs to prove t hat 1 is a regular value of $\det$. This can be done by showing that for every matrix $A\in Sl_n(\bbR)$ the derivative $d(\det)_A$ is surjective. 
\item We now compute the derivative of the determinant. The derivative of the determinant function at a matrix $A\in Mat_n(\bbR)$ is given by the following formula: 
\[d(\det)_A:Mat_n(\bbR)\to \bbR,\quad d(\det)_A(H)=tr(adj(A)\cdot H),\]
where $adj$ stands for the classical adjoint of the square matrix $A$ (it is defined as the transpose of the cofactor matrix of $A$) and $tr $is the trace operator. If $A\in Sl_n(\bbR)$ then the formula becomes
\[d(\det)_A(H)=tr(A^{-1}\cdot H),\] since $adj(A)=A^{-1}$ if $A\in Sl_n(\bbR)$.

\item Finally, the last step is to show the surjectivity of the derivative.  In other we need to demonstrate that for any real number $c\in \bbR$, there exists a matrix $H\in Mat_n(\bbR)$ such that:
\[tr(A^{-1}H)=c.\]
An easy example of such matrices is given for $H=cA$. 
Indeed, \[tr(A^{-1}H)=tr(A^{-1}(cA))=c\cdot tr(I_n)=c\cdot n.\]
Therefore, $d(\det)_A$ is surjective for all $A\in Sl_n(\bbR)$.
\item Applying the theorem leads to the conclusion. 
\end{enumerate}
\, 

Ex. \ref{Ex:sing}. The equation $f(x,y)=xy=0$ in the Euclidean plane $\bbR^2$ describes a union of the two coordinate axes:
the $x$-axis (corresponding to $y=0$) and $y$-axis corresponding to ($x=0$). This set is not a submanifold of $\bbR^2$ in the usual sense of differential geometry. 
\begin{enumerate}
    \item Observe that there exists a \emph{singular point} at $(0,0)$. The set 
    \[S=\{x,y\in \bbR^2\, |\, xy=0\}\] consists of two connected components (the $x$-axis and the $y$-axis), which intersect precisely at the origin. Intuitively, a smooth submanifold of $\bbR^2$ must locally resemble a smooth curve (or a single point). Away from the origin, the set $S$ consists of a pair of smooth lines, but the origin, the submanifold condition fails. 
    \item To verify this, we compute the partial derivatives of the function $f(x,y)=xy$:
    \[\frac{\partial f(x,y)}{\partial x}=y\quad \text{and} \quad \frac{\partial f(x,y)}{\partial y}=x.\]
    At $(0,0)$, both derivatives  vanish:
    \[\frac{\partial f(x,y)}{\partial x}(0,0)=0 \quad \frac{\partial f(x,y)}{\partial y}(0,0)=0.\]
    This confirms that (0,0) is indeed a singular point.  
\end{enumerate}
This illustrates an example of a singular algebraic variety, where the smooth parts (the axes away from the origin) are manifolds, but the overall structure is not a manifold due to the singularity at the origin.
