\section{Related Work}
\subsection{Consistency Regularization}
Most object detection methods rely on large-scale and well-annotated datasets to achieve cutting-edge performance \cite{russakovsky2015imagenet,lin2014microsoft}. However, the high cost of acquiring labeled data drives the development of semi-supervised learning (SSL) methods, which leverage unlabeled data to significantly improve model performance \cite{berthelot2019remixmatch,sohn2020fixmatch,xie2020unsupervised,pham2021meta}.

Consistency regularization is among the most widely used techniques in SSL and requires models to produce consistent predictions across augmented versions of the same input \cite{bachman2014learning,sajjadi2016regularization}. Expanding on this concept, numerous studies optimize model performance through pseudo-label generation and distribution alignment.

For example, FixMatch \cite{sohn2020fixmatch} introduces a high-confidence pseudo-labeling strategy, retaining only pseudo-labels with sufficient confidence to improve efficiency. FlexMatch \cite{zhang2021flexmatch} dynamically adjusts thresholds for different classes to accommodate varying learning difficulties. LaplaceNet \cite{sellars2022laplacenet} leverages graph structures to optimize pseudo-label generation, while DC-SSL \cite{zhao2022dc} aligns predicted distributions with true distributions to improve pseudo-label accuracy. MW-FixMatch \cite{zheng2025mw} addresses class imbalance in SSL by incorporating a weight network to balance contributions from labeled and unlabeled data, which improves performance on imbalanced datasets.

Although these methods enhance model performance on unlabeled data, their effectiveness in specific scenarios, such as low-light environments, remains underexplored. Furthermore, pseudo-label quality is constrained by issues like confirmation bias, which may lead models to overfit on incorrect pseudo-labels \cite{wang2023conflict}.

To address these challenges, this study proposes a semi-supervised learning method, specifically designed for low-light pedestrian tracking, based on consistency regularization. Nighttime data augmentation strategies generate effective perturbations to enhance the model’s adaptability to low-light conditions. By jointly leveraging labeled and unlabeled data, the proposed method enhances both the robustness and accuracy of object detection and tracking in low-light scenarios. Compared to existing methods, this work introduces a specialized data augmentation strategy tailored for low-light conditions. It also improves pseudo-label generation and utilization within the consistency regularization framework, effectively overcoming the challenges of leveraging unlabeled data in such environments.

\subsection{Multi Object Tracking}
Current multi-object tracking algorithms primarily focus on improving the algorithms, paying little attention to data-related issues and diverse application scenarios. Detection-based tracking (DBT) is one of the most widely studied approaches for pedestrian tracking, detecting objects and performing data association across frames. DBT methods often combine motion and appearance features to improve tracking robustness.

Motion-based association methods predict object trajectories over time to match detections across frames. For example, the classic SORT\cite{bewley2016simple} algorithm uses Kalman filtering and IoU for target matching. Subsequent methods like BoT-SORT\cite{aharon2022bot} improve robustness to camera motion, while ByteTrack\cite{zhang2022bytetrack} enhances low-confidence target association.
Appearance-based association methods use visual features such as color, texture, and shape to match and track objects. However, they are less common due to their sensitivity to lighting and pose changes. QDTrack\cite{fischer2023qdtrack} employs contrastive learning to improve object association accuracy without relying on motion priors.
Combined motion-appearance methods integrate motion and appearance cues, yielding better performance in handling occlusions and complex motion patterns\cite{rajapaksha2024consistency}. For example, DeepSORT\cite{wojke2017simple} combines appearance features with motion cues, while FairMOT\cite{zhang2021fairmot} unifies detection and re-identification tasks to improve both accuracy and real-time efficiency. DetTrack\cite{gao2023dettrack} uses spatiotemporal features to mitigate the negative effects of occlusion. GeneralTrack\cite{qin2024towards} introduces a "point-wise to instance-wise relation" framework, enhancing generalizability across diverse MOT scenarios by improving object association robustness.
Additionally, unified cue integration methods\cite{li2025unisort} enhance performance in crowded and occlusion-heavy scenarios. These approaches leverage weak cues (e.g., confidence, pseudo-depth, and height) to compensate for failures of strong cues and refine detection associations, improving overall robustness.

Despite significant algorithmic advancements, DBT methods remain constrained by the performance of the detector, which is inherently constrained by the quality and scale of annotated datasets. Collecting and annotating pedestrian tracking datasets is particularly challenging due to large deformations, frequent occlusions, and numerous small objects. For nighttime pedestrian datasets, noise and blur significantly complicate data collection and reduce data quality.

To mitigate the challenges posed by limited annotated data, semi-supervised learning (SSL) methods are extensively studied in recent years. These methods combine labeled and unlabeled data to enhance performance while reducing the reliance on fully labeled datasets. For instance, some approaches use tracking information to iteratively train classifiers and extract useful samples from unlabeled data. These methods achieve accuracy comparable to fully supervised approaches while requiring significantly fewer labeled samples\cite{teichman2012tracking}. Additionally, multi-classifier systems that integrate supervised and semi-supervised models prove effective in enhancing tracking robustness under challenging scenarios, such as occlusions or visually similar objects. These systems also help reduce drifting errors\cite{stalder2009beyond}.

SSL methods also leverage temporal and spatial features to improve appearance model learning. For example, contrastive loss-based embedding methods learn from partially labeled videos, producing discriminative appearance features that enhance tracking robustness\cite{li2021semi}. Similarly, super-trajectory-based video segmentation approaches group trajectories with consistent motion patterns. These methods enable reliable propagation of initial annotations and improve robustness in handling occlusions\cite{wang2018semi}. However, these methods often rely on trajectory consistency, but this consistency is often unreliable in low-light or noisy environments, limiting their adaptability.

Existing nighttime multi-object tracking methods\cite{yi2024comprehensive,wang2024multi} often use data augmentation on limited datasets to simulate low-light conditions, but they lack real-world scene data and are constrained by dataset size. To address these issues, we propose a method for pedestrian tracking in low-light conditions that leverages annotated and unannotated data during training. Drawing inspiration from SSL methods such as iterative training strategies\cite{teichman2012tracking} and contrastive learning mechanisms\cite{li2021semi}, our approach addresses their limitations in noisy environments. It also improves adaptability to real-world scenarios, overcoming the constraints imposed by dataset size.