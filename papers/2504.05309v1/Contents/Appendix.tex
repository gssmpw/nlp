\subsection{Rewrite Direction}\label{app:rewrite-direction}
\begin{table}[!htbp]
\centering
\resizebox{\linewidth}{!}{
\begin{tabular}{p{0.25\linewidth}|p{0.25\linewidth}|p{0.25\linewidth}|p{0.25\linewidth}} \hline \hline
    Direction & Definition & Positive Cases & Negative Cases \\ \hline
    Key Word Extraction & Extract the core content from the query, which must be entirely contained within the query & Animal cream birthday cake $\to$ cake,  birthday cake & Linwei's Chuan $\to$ barbecue \\ \hline
    Correction & Rewrite the query to use correct wording. & Wontom $\to$ Wonton & HK $\to$ HK style Cafe \\ \hline
    Alias \& Synonyms & Rewrite the query to commonly used wording. & kfc $\to$ Kentucky Fried Chicken & kfc $\to$ McDonald
    \\ \hline
    Main Dish & Rewrite the query to cuisine that are short, common, and highly related to the original query; cannot directly use the provided query. & Daifuku Spicy Kimchi $\to$ Kimchi Soup, Spicy Pickled Cabbage & Daifuku Spicy Kimchi $\to$ Bibimbap \\ \hline
    Low Relevance & Rewrite the query to similar cuisine. & Big Pork Bone $\to$ Steak & Big Pork Bone $\to$ Northeast Braised Pork Bones  \\ \hline \hline
\end{tabular}}
\caption{Rewrite direction}
\label{tab:rewrite-direction}
\end{table}

\subsection{Prompts for Rewrite Generation}\label{app:rewrite-prompts}
\begin{tcolorbox}[title = {Prompt for Rewrite Generation for Tail Query}]
\texttt{Instruction}: 

You are a query analysis expert for a food delivery platform.
You are provided with user search queries, along with the standard names of the restaurant and cuisine that have historically received the most clicks and purchases for those queries. Please:

1) First, analyze the meaning of the query. If there is an input error, typo in the query, please correct it. If no correction is needed, output ``None'';

2) Then, determine the search intent. Do user tend to match cuisine names or restaurant names under the query? Choose one from ``Cuisine'', ``Restaurant'', or ``Neither'';

3) After that, provide N query rewrites for the original query, and output them in order of rewrite efficiency from high to low. The query rewrite can include the following types:

- Key word extraction: Extract the core content from the query, which must be fully contained within the query;

- Alias \& Synonyms: To search for more similar dishes, the terms should be short and common, and should not directly use the provided dish names;

- Main Dish: For example, ``burger'', ``cake'', ``noodles''. Note that dish category terms should be specific and not too general or vague;

- Related cuisine (Low Relevance): The rewrite should be short, common, and highly related to the original query, and should not directly use the provided restaurant and cuisine names;

Finally, output only in the following format: {``Query meaning'': ``Summarize in less than 30 words'', ``Correction'': ``Correction result'', ``Search intent'': ``Cuisine/Restaurant/Neither'', ``Rewrite'': ``rewrite1, rewrite2, ...''}.

\texttt{User}:

Query:\{\} 

Associated restaurant/Cuisine:\{\}

Query Explanation: It is highly possible that the query contain type, synonyms,  particular local food \/ restaurant, equivocal search intention, natural language and etc, leading to a low appearance frequency. Try to infer the actual meaning of the query.

\texttt{Assistant}: 

Output: \{\}
\end{tcolorbox}

Different categories of query applies to different query explanation:

\begin{tcolorbox}[title = {Query explanation for Query with High Frequency}]
This query is commonly used, it may represent a board category or common cuisine. Try to extract the key word and find cuisine that are with low relevance for interest exploration.
\end{tcolorbox}

\begin{tcolorbox}[title = {Query explanation for Query with Mid Frequency}]
This query may be a local food or specific brand name. Try to find out the main dish the query contain.
\end{tcolorbox}

\subsection{Post-Training}

\subsubsection{Rewrite Quality}\label{app:post-training:rewrite-quality}
Training Sample for Rewrite Quality:
\begin{tcolorbox}[title = {Training Sample for Rewrite Quality}]
\texttt{Instruction}:

You are a query rewrite evaluation expert for a food delivery platform. Your task is to determine whether a given query rewrite is a good rewrite based on the user's input query, names of the restaurant and cuisine with the most historical clicks and purchases for that query.

A good query rewrite should ensure strong relevance to the original query term while also enabling the incremental recall of more dishes that users might click and order.

You will be provided with a group of two queries and corresponding information each time. Please evaluate them in sequence and respond with ``Yes'' or ``No''.

\texttt{User}:

Query1: \{Query\}; Associated restaurant/Cuisine1:\{\}


Query2: \{Query\}; Associated restaurant/Cuisine2:\{\}

\texttt{Assistant}:

Output: \{1.Yes. 2.No\}
\end{tcolorbox}

\subsubsection{Relevance}\label{app:post-training:relevance}
Training Sample for Relevance:
\begin{tcolorbox}[title = {Training Sample for Relevance}]
\texttt{Instruction}:

You are an expert in a food delivery platform. Your task is to determine the relevance level of user search query to the products based on the following scoring criteria, categorized into three levels: High, Low, and None. You will be given user search query, names of restaurant and cuisine.

First, you need to analyze the user's search intent based on the user query. Extract the type of restaurant and cuisine the user wants to buy and any specific attributes they require (such as ingredients, taste, preparation method, size).

High Relevance: If the restaurant and cuisine is of the type the user wants to buy, and the user does not have specific attribute requirements, or the restaurant and cuisine meets all the user's attribute requirements, then the restaurant and cuisine fully satisfies the user's needs.

Low Relevance: If the restaurant and cuisine is of the type of cuisine the user wants to buy but does not fully meet the user's attribute requirements; or if the user does not have specific attribute requirements, and although the restaurant and cuisine is not of the type of dish the user wants to buy, it can still meet the user's needs based on the intended use.

No Relevance: If the restaurant and cuisine is not of the type of dish the user wants to buy and cannot meet the user's needs based on the intended use, then the restaurant and cuisine does not satisfy the user's requirements.

You will be provided with three groups each time. Please evaluate them in sequence and respond with the relevance level: ``High'', ``Low'', or ``None''.

\texttt{User}:

Query1: \{Query\}; Restaurant1: \{\}; Cuisine1: \{\}


Query2: \{Query\}; Restaurant1: \{\}; Cuisine2: \{\}


Query3: \{Query\}; Restaurant1: \{\}; Cuisine3: \{\}

\texttt{Assistant}:

Output: \{1.Low. 2.High 3. None\}
\end{tcolorbox}