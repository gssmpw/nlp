\section{Summary}\label{sec-con}

\textbf{Research Objectives and Contributions: }Our paper explores the integration of Large Language Models (LLMs) in software engineering education, focusing on how student teams interact with AI tools throughout a multi-milestone academic project. We analyzed tool usage, code complexity, refinement, and student prompting behavior to uncover patterns in AI-aided code development throughout the educational process. Our study provides actionable insights for educators to optimize AI tool usage in Software Engineering curricular.

\textbf{Summary of Findings: } Most of the teams utilized AI during development. Copilot was preferred for auto-completion, while ChatGPT excelled in iterative refinement of more complex solutions. AI usage declined across milestones, as students relied on LLMs more at the early stages of the project. Copilot's outputs were often more complex, while ChatGPT produced more concise and understandable solutions. The AI-generated code showed increasing alignment with project goals over time, showcasing improved prompt engineering. Early prompts were exploratory and less precise, later students gained experience and improve on this skill. Sentiment analysis highlighted initial positivity, occasional mid-conversation frustration, and eventual resolution, underscoring the iterative value of AI-assisted coding.

\textbf{Evolution of Student Engagement with LLMs: } Over the course students demonstrated notable growth in their use of LLMs, with improved prompt engineering and more efficient workflows compared to our previous study \cite{Rasnayaka2024}. Access to paid LLMs enabled broader integration of AI tools, encouraging deeper engagement in AI-assisted problem-solving. Increased prevelance of LLM use highlights key pedagogical implications, including the enhanced critical assessment and integration of AI in software development.

\textbf{Implications for Educators: } Experiential learning of Prompt Engineering is effective to enhance code quality and reduce refinement effort. Providing avenues to critically assess AI-generated outputs mitigates over-reliance. Therefore, integration of AI tools in software engineering curricula is essential to maximize the benefits of LLMs. This study highlights the potential of LLMs to transform educational practices, fostering both productivity and deeper understanding in software development processes.
