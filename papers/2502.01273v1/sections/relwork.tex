

\section{Related Work}\label{sec-lit}
% In recent years, there has been a marked increase in research on the application of large language models (LLMs) in software engineering (LLM4SE), education (LLM4Edu), as well as their intersection in SE Education (LLM4SE Edu).

\subsection{LLMs in SE Education (LLM4SE Edu)}

The increased popularity and accessibility of LLMs are prompting significant changes to approaches to software engineering education, with an emphasis on adaptive learning strategies and ethical considerations. \cite{3626252.3630927} underscores the need for SE education to evolve in response to LLM advancements, advocating for combining technical skills, ethical awareness, and adaptable learning strategies.  
AI-powered tutors, such as those based on LLMs, have also shown promise in delivering timely and personalized feedback in programming courses.
\cite{savelka2023efficientclassificationstudenthelp} has also found LLMs to be feasible in classifying student needs in SE educational courses, presenting a cost effective alternative to traditional tutor support demand.
However, \cite{3639474.3640061} highlights challenges such as generic responses and potential student dependency on AI, warranting further discussions on the cost-effectiveness of using LLMs in SE education.  
Similarly, \cite{3661167.3661273} finds that Gamified learning environments, when augmented with LLMs, can boost student engagement but may inadvertently lead to over-reliance, undermining the learning process. 
The StudentEval benchmark also introduces novice prompts, shedding light on non-expert interactions and revealing critical insights into user behavior and model performance \cite{llm4code95}.
Work has also been done on programming assistants that do not directly reveal code solutions \cite{10.1145/3613904.3642773}, providing design considerations for future AI education assistants.


\subsection{LLMs in Software Engineering (LLM4SE)}

% LLMs have shown immense potential in automating and enhancing various software engineering tasks, including code generation, testing, and debugging.

LLMs have been employed in tools designed to improve code comprehension directly within integrated development environments (IDEs). These tools utilize contextualized, prompt-free interactions to enhance task efficiency, as shown in \cite{3597503.3639187}.  
In the realm of automated unit test generation, ChatGPT has demonstrated competitive performance against traditional tools like Pynguin, particularly when enhanced through prompt engineering techniques \cite{llm4code38}.  
LLMs have also been leveraged to generate insightful questions that bridge gaps between data and corresponding code, improving semantic alignment and comprehension \cite{llm4code21}.  
Automated Program Repair (APR) is another area where LLMs have proven effective, showcasing their ability to fix bugs in both human-written and machine-generated code \cite{10172854}.  
Additionally, \cite{2409.02977} provides a comprehensive survey of LLM-based agents, emphasizing their utility in addressing complex software engineering challenges through human and tool integration.
Despite these promising advancements, \cite{3639476.3639764} highlights critical challenges in ensuring the validity and reproducibility of LLM-based SE research, proposing guidelines to mitigate risks such as data leakage and model opacity.  


\subsection{LLMs in Education}
LLMs are promising to reshape pedagogy, by offering solutions for personalized learning and scalable assessment practices. A systematic review of LLM applications in smart education highlights their role in enabling personalized learning pathways, intelligent tutoring systems, and automated educational assessments \cite{2311.13160}. LLMs have also been evaluated for their utility in grading programming assignments, with research demonstrating that ChatGPT provides scalable and consistent grading, rivaling traditional human evaluators \cite{llm4code5}.


% \subsection{Emerging Themes and Common Use Cases}

% Several common themes emerge across these domains. LLMs consistently enhance task efficiency in software engineering, enable personalized and scalable learning experiences, and foster innovative teaching methodologies. Key use cases include AI tutors, automated graders, and intelligent coding assistants, illustrating the versatility of LLMs in transforming both SE and education. Despite these opportunities, challenges such as prompt engineering, over-reliance on AI systems, and ethical considerations remain central to ongoing discussions. Addressing these challenges will be crucial in maximizing the potential of LLMs in SE Education.



Our work extends beyond these existing work in the following aspects: we performed a study on the interaction between LLMs and Software Engineering students working on a complex project, conducting a comprehensive suite of analyses on both the prompts and generated code produced in these interactions, differing from the existing literature in the scope of analysis, a focus on the effects of the conversational nature of LLM code generators, as well as the examination of user sentiments via prompts they used to generate code.
