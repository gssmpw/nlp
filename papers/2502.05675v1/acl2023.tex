% This must be in the first 5 lines to tell arXiv to use pdfLaTeX, which is strongly recommended.
\pdfoutput=1
% In particular, the hyperref package requires pdfLaTeX in order to break URLs across lines.

\documentclass[11pt]{article}

% Change "review" to "final" to generate the final (sometimes called camera-ready) version.
% Change to "preprint" to generate a non-anonymous version with page numbers.
\usepackage[final]{acl2023}
% review
% Standard package includes
\usepackage{times}
\usepackage{latexsym}
\usepackage{multirow}
\usepackage{cite} % Optional, for improved citation handling
% For proper rendering and hyphenation of words containing Latin characters (including in bib files)
\usepackage[T1]{fontenc}
% For Vietnamese characters
% \usepackage[T5]{fontenc}
% See https://www.latex-project.org/help/documentation/encguide.pdf for other character sets

% This assumes your files are encoded as UTF8
\usepackage[utf8]{inputenc}

% This is not strictly necessary, and may be commented out,
% but it will improve the layout of the manuscript,
% and will typically save some space.
\usepackage{microtype}

% This is also not strictly necessary, and may be commented out.
% However, it will improve the aesthetics of text in
% the typewriter font.
\usepackage{inconsolata}
\usepackage{adjustbox}
\usepackage{booktabs}
\usepackage{longtable}
\usepackage{tcolorbox}
\usepackage{stackengine}
\usepackage{multirow}
\usepackage{graphicx}
\usepackage{wrapfig}
\usepackage{hyperref}       % hyperlinks
\usepackage{url}            % simple URL typesetting
\usepackage{amsfonts}       % blackboard math symbols
\usepackage{nicefrac}       % compact symbols for 1/2, etc.
\usepackage{microtype}      % microtypography
\usepackage{xcolor,colortbl}         % colors
\usepackage{times}
\usepackage{latexsym}
\usepackage{multicol}
\usepackage{blindtext}
\usepackage{tabu}
\usepackage{amsmath, bm}
\newcommand{\sub}{\textsubscript}
\renewcommand{\UrlFont}{\ttfamily\small}
\usepackage{subcaption}
\usepackage{caption}
\usepackage[normalem]{ulem}
\usepackage{soul}
\usepackage[shortlabels]{enumitem}
\usepackage{array}
\usepackage{pgffor}
\usepackage{textcomp}
\usepackage{amssymb}
\usepackage{pifont}
\usepackage{booktabs}
\usepackage{tabularx}
\usepackage{color}
\usepackage{xcolor}
\usepackage{mwe}
\usepackage{float}
\newcommand{\xmark}{\ding{55}}
\definecolor{lightgreen}{rgb}{0.8, 0.95, 0.8}
\definecolor{lightred}{rgb}{0.95, 0.8, 0.8}
\definecolor{naplesyellow}{rgb}{0.98, 0.85, 0.37}
\definecolor{pastelyellow}{rgb}{0.99, 0.99, 0.59}

% \newcommand{\Venkatesh}[1]{\textbf{\small {\color{blue}[#1 -Venkatesh]}}}
\newcommand{\mihir}[1]{\textcolor{cyan}{[Mihir: #1]}}

% If the title and author information does not fit in the area allocated, uncomment the following
%
%\setlength\titlebox{<dim>}
%
% and set <dim> to something 5cm or larger.

% \title{Investigating the Shortcomings of LLMs in Step-by-Step Reasoning using Context-Rich Legal Scenarios}
\title{Investigating the Shortcomings of LLMs in Step-by-Step Legal Reasoning}

% \title{Investigating the Shortcomings of Large Language Models in Step-by-Step Reasoning using Context-Rich Legal Scenarios}

% Author information can be set in various styles:
% For several authors from the same institution:
% \author{Author 1 \and ... \and Author n \\
%         Address line \\ ... \\ Address line}
% if the names do not fit well on one line use
%         Author 1 \\ {\bf Author 2} \\ ... \\ {\bf Author n} \\
% For authors from different institutions:
% \author{Author 1 \\ Address line \\  ... \\ Address line
%         \And  ... \And
%         Author n \\ Address line \\ ... \\ Address line}
% To start a separate ``row'' of authors use \AND, as in
% \author{Author 1 \\ Address line \\  ... \\ Address line
%         \AND
%         Author 2 \\ Address line \\ ... \\ Address line \And
%         Author 3 \\ Address line \\ ... \\ Address line}

% \author{First Author \\
%   Affiliation / Address line 1 \\
%   Affiliation / Address line 2 \\
%   Affiliation / Address line 3 \\
%   \texttt{email@domain} \\\And
%   Second Author \\
%   Affiliation / Address line 1 \\
%   Affiliation / Address line 2 \\
%   Affiliation / Address line 3 \\
%   \texttt{email@domain} \\}

\author{Venkatesh Mishra$^{1*}$ \quad Bimsara Pathiraja$^1$\thanks{\ \ Equal Contribution} \quad Mihir Parmar$^1$ \quad Sat Chidananda$^1$  \\ \textbf{Jayanth Srinivasa}$^2$\quad  \textbf{Gaowen Liu}$^2$ \quad \textbf{Ali Payani}$^2$ \quad \textbf{Chitta Baral}$^1$ \\\\ 
$^1$Arizona State University \quad $^2$Cisco Research\\
\small{\texttt{\{vmishr23, bpathir1, chitta\}@asu.edu}}
}
\begin{document}
\maketitle
\begin{abstract}

Reasoning abilities of LLMs have been a key focus in recent years. One challenging reasoning domain with interesting nuances is legal reasoning, which requires careful application of rules, and precedents while balancing deductive and analogical reasoning, and conflicts between rules.  Although there have been a few works on using LLMs for legal reasoning, their focus has been on overall accuracy. In this paper, we dig deeper to do a step-by-step analysis and figure out where they commit errors. We use the college-level Multiple Choice Question-Answering (MCQA) task from the \textit{Civil Procedure} dataset and propose a new error taxonomy derived from initial manual analysis of reasoning chains with respect to several LLMs, including two objective measures: soundness and correctness scores. We then develop an LLM-based automated evaluation framework to identify reasoning errors and evaluate the performance of LLMs. The computation of soundness and correctness on the dataset using the auto-evaluator framework reveals several interesting insights. Furthermore, we show that incorporating the error taxonomy as feedback in popular prompting techniques marginally increases LLM performance. Our work will also serve as an evaluation framework that can be used in detailed error analysis of reasoning chains for logic-intensive complex tasks\footnote{Data and source code are available at \url{https://github.com/VenkyMishra/legal_reasoning}}.


%With the advent of Large Language Models (LLMs) and their adoption in the legal domain, it is important to analyze their legal reasoning abilities. Since legal reasoning requires these models to reason logically, evaluating it in LLMs also provides valuable insight into their logical reasoning skills. To the best of authors' knowledge, most existing work evaluating LLMs' legal reasoning primarily focus on their ability to predict the correct final answer, and there have not been much attention paid to analyzing its step-by-step reasoning process and where they falter. Motivated by this, we systematically evaluate the reasoning processes of LLMs on college-level Multiple Choice Question-Answering (MCQA) task from the \textit{Civil Procedure} dataset. We propose a new error taxonomy derived from initial manual analysis of reasoning chains from LLMs including Mistral-7B, Llama-3-8B, GPT-3.5, GPT-4-Turbo, and GPT-4o. Leveraging this taxonomy, we develop an LLM-based automated evaluation framework to identify reasoning errors, akin to human error identification. Additionally, we evaluate the performance of LLMs using two objective measures, soundness and correctness scores. Furthermore, we propose that incorporating the error taxonomy as feedback in existing prompting methods enhances LLM reasoning abilities and improves their final performance. We believe that this work provides evaluation framework for future exploration of detailed error analysis of reasoning chains for more logic-intensive complex tasks\footnote{Data and source code are available at <anonymous link>}.     

% Legal reasoning requires highly critical multi-step analytical and logical reasoning while considering the subtle differences and nuances of different legal scenarios. Evaluating the legal reasoning capabilities of large language models (LLMs) can serve as a valuable measure of its overall reasoning abilities. While almost all of the existing work on legal reasoning evaluate the performance of LLMs based on their final answer, in-depth analysis of the most common errors made by LLMs in their intermediate reasoning steps is crucial to improve the reasoning abilities of LLMs. Thus, in this work, we critically analyze the reasoning capabilities of various LLMs on a context-rich multiple choice question answering dataset containing 175 college-level law problems from Civil Procedure encompassed in the US federal and state laws. The study aims to assess the correctness of these models' reasoning processes, highlighting their strengths and limitations in scenarios where they have access to all the necessary knowledge. Based on meticulous human evaluation, we have created an error taxonomy of the common errors committed by five LLMs (Mistral-7B-v2-instruct, Llama-3-8B-instruct, GPT-3.5-turbo, GPT-4-turbo, and GPT-4o) while generating the reasoning chains. Our proposed automated evaluator for reasoning step error identification helps scaling up the evaluation on the entire dataset. The prompt-based error mitigation strategies which leverage the proposed error taxonomy \textit{outperform}\footnote{Mitigation of errors investigation currently in progress} prompting methods which do not incorporate this framework. 
% We have proposed various mitigation strategies to tackle and prevent fallacies of these X categories from occurring during the logical reasoning process. These strategies demonstrate that LLMs can improve their reasoning abilities, even on challenging datasets such as LRBENCH and tasks beyond legal reasoning.

%Now, we utilize these manually annotated samples based on error categories as few-shot exemplars for GPT-4 and explore its capabilities to evaluate reasoning chains. On the puzzle-solving task, we show that our proposed method easily identifies various error types. Based on various error categories, we propose to utilize some base prompting methods to improve the performance of LLMs on puzzle-solving. We believe that our framework for error categorization can be useful to many reasoning tasks in improving performance beyond puzzle solving.

%annotate different errors in reasoning chains for puzzle-solving task. Based on that, we propose a guideline to 
\end{abstract}

\begin{figure}[ht]
    \centering
    \includegraphics[width=0.8\linewidth]{graphs/greater_than_naive.pdf}
    \vspace{0.5cm}
    \includegraphics[width=0.8\linewidth]{graphs/p1_bottom.png}
    \vspace{-5pt}
    \caption{\textcolor{positional}{Positional} vs.\ \textcolor{nonpositional}{non-positional} circuits. In a \textcolor{nonpositional}{non-positional} circuit, the same edges must be included at all positions. A \textcolor{positional}{positional} circuit can distinguish between the same edge at different positions. This specificity yields better trade-offs between circuit size and faithfulness. It can also increase both precision and recall.}
    \label{fig:p1}
    \vspace{-5pt}
\end{figure}

\section{Introduction}

\looseness=-1
A primary goal of interpretability research is to characterize the internal mechanisms in language models (LMs) and other NLP models. 
A core approach in this area is \textbf{circuit discovery}---identifying the minimal subgraph within the model's computation graph that performs a specific task \citep{olah2021framework,olah-mech}.
Typically, the nodes of a circuit represent model components (e.g., attention heads, neurons, or layers).
While manual circuit discovery methods can yield position-specific insights \citep{wanginterpretability,goldowskydill2023localizingmodelbehaviorpath}, \emph{automatic methods often overlook positional information}, treating components as uniformly relevant across all input token positions \citep{conmytowards,syed2023attribution}. 
For instance, if an attention head is included in a circuit, it is assumed to contribute equally to the computation for every position in the input sequence.
The assumption that circuits are position-invariant ignores the fact that different positions often require distinct computations.
By ignoring positions, current methods limit their ability to capture mechanisms that operate across positions, such as interactions between attention heads across positions.

In this study, we start by demonstrating that positional agnosticism is a significant limitation (\S\ref{sec:motivating}). Then, to address these limitations, we introduce a new approach: position-aware edge attribution patching (PEAP; \S\ref{sec:full_circ_discovery}; Figure~\ref{fig:p1}). Current approaches  assume that if an edge is in a circuit, then the same edge will be in the circuit at all positions, thus leading to low precision. It is also assumed that an edge's importance should be aggregated across positions before deciding whether it should be included in the circuit; this can lead to cancellation effects, and thus low recall. PEAP instead allows us to compute the importance of cross-positional edges, and separately evaluates edge importance at each position. We show that this leads to smaller and more accurate circuits; see Figure~\ref{fig:p1}.

Incorporating positional information into circuit discovery is straightforward when inputs have the same length and structure across examples.

However, realistic datasets are not nearly this templatic.
How, then, can we incorporate positional information into automatic circuit discovery?
To address this challenge, we propose \textbf{schemas} (\S\ref{sec:schema}). 
Schemas assign semantic labels to spans of tokens, enabling information aggregation across examples even when the spans differ in length.

For example, in the input ``The \textcolor{positional}{war} lasted from 1453 to 14\underline{\hspace{1em}},'' the span ``\textcolor{positional}{war}'' could be labeled as ``\emph{Subject}''.
This enables handling spans with varying lengths: the phrase ``\textcolor{positional}{Black Plague}'' in another example can be treated as a single positional span with the same role as ``\textcolor{positional}{war}''.
In experiments with two LMs and three tasks, we find that circuits discovered using schemas achieve a better trade-off between circuit size and faithfulness to the model's behavior than position-agnostic circuits.
Importantly, position-aware circuits offer a more precise representation of the underlying mechanisms, providing a more concise foundation for mechanistic explanations.

We also present a fully automated pipeline for schema generation and application (\S\ref{sec:schema-generation}) using large language models (LLMs). 
We evaluate the quality of the generated schemas and their utility in discovering position-aware circuits (\S\ref{sec:schema-eval}).
Notably, circuits derived using automatically generated and applied schemas achieve comparable faithfulness scores to circuits discovered with human-designed and manually applied schemas.

We summarize our contributions as follows:
\begin{itemize}[noitemsep,leftmargin=*,topsep=1pt,parsep=1pt]
    \item Introduce a position-aware circuit discovery method, which obtains better faithfulness than position-agnostic discovery.  
    \item Introduce dataset schemas,  facilitating positional circuit discovery in more naturalistic settings. 
    \item Develop an automated schema generation and application pipeline with LLMs, yielding schemas that are comparable to manually-annotated ones.
\end{itemize}



\section{Related work}


Recent advances in single-image animatable head avatar generation can be categorized into mainly 2D-based and 3D-based approaches. 

\paragraph{\bf Image to 2D Animatable Avatar.}
2D-based methods, leveraging the power of convolutional neural networks (CNNs)~\cite{DBLP:conf/cvpr/KarrasLAHLA20,DBLP:conf/cvpr/IsolaZZE17,DBLP:conf/nips/GoodfellowPMXWOCB14}, often employ generative adversarial networks (GANs)~\cite{DBLP:conf/cvpr/StyleGAN} for direct image synthesis. Early approaches~\cite{DBLP:conf/cvpr/WangDYSW23,DBLP:conf/cvpr/BurkovPGL20,DBLP:conf/iccv/ZakharovSBL19} focus on injecting expression and pose features into the generator network, often utilizing architectures like U-Net or StyleGAN~\cite{DBLP:conf/cvpr/StyleGAN}.
Some other 2D methods~\cite{DBLP:journals/corr/abs-2407-03168,DBLP:conf/cvpr/ZhangQZZW0CW023,DBLP:conf/cvpr/HongZS022,DBLP:conf/mm/DrobyshevCKILZ22,DBLP:conf/cvpr/BurkovPGL20,DBLP:conf/nips/SiarohinLT0S19} represent expressions and poses as warping fields applied to the source image. 
Benefiting from advances in image and video diffusion networks, more recent 2D-based works~\cite{DBLP:journals/corr/abs-2410-07718,DBLP:journals/corr/abs-2406-08801,DBLP:conf/eccv/TianWZB24} get improved results with diffusion techniques. 
However, these methods still face challenges related to long generation times and significant computational resource demands. Audio-driven 2D control methods~\cite{DBLP:conf/cvpr/ZhangCWZSGSW23,DBLP:journals/corr/abs-2211-12368,DBLP:conf/iccv/GuoCLLBZ21} are easy to use but cannot explicitly control facial expressions and poses. 2D-based techniques often struggle with large pose or expression variations due to the lack of an explicit 3D structure, sometimes producing unrealistic distortions or identity changes. While some 2D methods~\cite{SadTalker,StyleHEAT,Pirenderer,DBLP:conf/cvpr/WangM021,MegaPortraits} incorporate 3D Morphable Models (3DMMs)~\cite{DBLP:conf/fgr/GerigMBELSV18,DBLP:journals/tog/LiBBL017,DBLP:conf/avss/PaysanKARV09,DBLP:conf/siggraph/BlanzV99} to mitigate these issues, they typically cannot achieve free-viewpoint rendering. 

\vspace{-0.1in}

\begin{figure*}[h]
    \centering
    \includegraphics[width=0.9\linewidth]{images/framework.pdf}
    \caption{\textbf{Overall Framework.} Our framework utilizes learnable query features attached to FLAME vertices to perform cross-attention with the extracted multi-level image features. The extracted features are then decoded to reconstruct the Gaussian avatar in the canonical space, which can be animated utilizing standard linear blend skinning (LBS) and corrective blendshapes as the FLAME model did and rendered in real-time on various platforms.}
    \label{fig:framework}
\end{figure*}

\paragraph{\bf Image to 3D Animatable Avatar.}
3D-aware methods offer improved geometric consistency and free-viewpoint rendering capabilities. Early 3D approaches~\cite{DBLP:conf/eccv/KhakhulinSLZ22,DBLP:conf/cvpr/XuYCWDJT20} utilize 3DMMs for head avatar reconstruction. With the advent of Neural Radiance Fields (NeRFs)~\cite{DBLP:conf/eccv/MildenhallSTBRN20}, many recent methods~\cite{DBLP:conf/siggraph/YuFZWYBCSWSW23,DBLP:conf/cvpr/MaZQLZ23,DBLP:conf/cvpr/LiZWZ0CZWB023,GPAvatar,ye2024real3d,deng2024portrait4d,deng2024portrait4d2,DBLP:conf/eccv/KiMC24,DBLP:conf/cvpr/BaiFWZSYS23,PointAvatar,Nerfies,INSTA} have adopted this representation for higher fidelity, particularly in modeling fine details like hair. However, NeRF-based~\cite{DBLP:conf/cvpr/ZhangZLHLWGCL024,HAvatar,DBLP:conf/cvpr/BaiTHSTQMDDOPTB23,AD-NeRF,DBLP:journals/tog/GaoZXHGZ22,DBLP:journals/tog/ParkSHBBGMS21,DBLP:conf/cvpr/AtharXSSS22,DBLP:journals/corr/abs-2112-05637,DBLP:conf/iccv/TretschkTGZLT21,DBLP:conf/cvpr/GafniTZN21,DBLP:conf/eccv/KiMC24,DBLP:conf/cvpr/BaiFWZSYS23,PointAvatar,Nerfies,DBLP:conf/siggraph/YuFZWYBCSWSW23,DBLP:conf/cvpr/MaZQLZ23,DBLP:conf/cvpr/LiZWZ0CZWB023} approaches often require extensive training data, including multi-view or single-view videos, raising privacy concerns and limiting generalization to unseen identities. Some methods~\cite{DBLP:conf/cvpr/SunWWLZZL23,DBLP:conf/3dim/ZhuangMKS22,DBLP:journals/pami/SunWZHWL24,DBLP:journals/tvcg/TangZYZCMW24,DBLP:conf/iclr/XuZLZBFS23} bypass this data requirement by training generators with random noise and then inverting them for identity-specific reconstruction, but inversion accuracy remains a challenge. Test-time optimization offers another alternative, but its computational cost limits practical applications. Several recent works~\cite{goha2023,hidenerf2023,gpavatar2024,ye2024real3d,ma2024cvthead,deng2024portrait4d,deng2024portrait4d2,GGHead} have explored one-shot 3D head reconstruction to address the limitations of data requirements and computational cost. These methods employ various techniques, such as tri-plane features, deformation fields, point-based expression fields, and vertex-feature transformers. Despite these advancements, NeRF-based methods often struggle with real-time rendering. 
Recently, 3D Gaussian Splatting~\cite{GaussianSplatting} has emerged as a promising alternative, offering both high-quality results and fast rendering speeds. However, existing Gaussian Splatting methods~\cite{GaussianAvatar,DBLP:conf/cvpr/XuCL00ZL24} typically rely on video data for training for each person, limiting their ability to generalize to new identities. Instead, the most recent work, GAGAvatar~\cite{GAGAvatar}, proposes a one-shot 3D Gaussian-based head avatar generation method. However, it still relies heavily on complex 2D neural post-processing to achieve optimal animation outcomes, thus it is not a pure 3D solution and the extra neural network hinders its application on various platforms. In contrast, our work generates Gaussian heads that are immediately animatable and renderable without additional networks or post-processing steps, enabling seamless integration into existing rendering pipelines for real-time animation and rendering across a wide range of platforms, including mobile phones. 

\begin{table*}[ht]
\small
\centering
\begin{tabular}{p{4cm}|p{11cm}}
\toprule
\textbf{Category}                               & \multicolumn{1}{c}{\textbf{Description}}                                     \\ \midrule
Misinterpretation (associated with Error of Law)                               & The LLM misinterprets or omits some part/entirety of the legal context, question or the options (or a combination of them). This usually leads to the wrong reasoning and selection of wrong conclusion. The following error instances fall under the taxon of misinterpretation: 1. Misunderstanding the legal rules. 2. Misunderstanding the legal situation/issue at hand. 3. Omission of parts of the provided context while reasoning. 4. Incompletely applying a legal rule. 5. Incorrectly applying the legal rule. 6. Wrong assumptions derived from the provided context.                                                                             \\ \midrule
Irrelevant Premise (associated with Error of Law)                             & The LLM generates a premise which is not relevant in solving the question or that it may divert the reasoning chain from solving the question correctly. An premise may be logically valid and factually true but the absence of this premise can still lead to the correct conclusion.                         \\ \midrule
Factual Hallucination (associated with Error of Fact)                          & This error category covers instances where the LLM, during its reasoning process, generates information that is either inconsistent with the facts of the given legal scenario or is entirely fabricated with no basis in reality.                                                                                                                \\ \bottomrule
\end{tabular}
\caption{Error taxonomy for the Premise-level steps. The taxonomy has been developed with consideration for the types of errors that a human reasoner might commit when constructing a rationale for a given legal scenario. Error of Law and Error of Fact are explained in \citep{cornell2024mistake, cornell2024mistake_fact, oreilly2012errors, wilberg2023mistake}. Some fine-grained error instances of the `Misinterpretation' category are shown in Tables \ref{table:human_annotation_example_initial}, \ref{table: misinterpretation-2}, \ref{table: misinterpretation-3} and \ref{table: misinterpretation-4}.}.
\label{table:premise_errors}
\end{table*}
\section{Evaluation of Reasoning Chains}
\label{sec:error_category}

\subsection{The \textit{Civil Procedure} Dataset}
The dataset has been sourced from MCQs present in the `The Glannon Guide To Civil Procedure' \citep{glannon2013guide}. We compile the \textit{Civ. Pro.} dataset with 175 samples of college-level law multiple-choice questions from the US Civil Procedure laws. The questions are primarily designed to evaluate the ability of university-level law students to reason about various legal scenarios about US Civil Procedure laws and provide their final judgment by choosing the most correct option as an answer. The dataset includes relevant legal context, multiple-choice questions, and expert answers with correct explanations provided by legal experts. These elements were extracted and converted into a prompt-based format suitable for LLM inference and the generation of reasoning chains. The \textit{Civ. Pro.} dataset consists of samples comprising of $\mathcal{D} = {<lc_{n}, q_{n}, o_{n}, e_{n}>}$, where $lc_n$, $q_{n}$, $o_{n}$ and $e_{n}$ denote the $n^{th}$ legal context, question, option-set and expert-answer respectively.  


 % The questions range from straightforward applications of relevant rules to find the answers to complex legal scenarios involving scenarios with a multitude of combination of statutes, precedents and exceptions
 % To find out the commonly occurring errors in the reasoning chains generated by the LLMs, we performed a detailed sentence-level analysis of the chains. We began by segmenting the chains into the individual statements which form part of the argument, comprising of set of premises and a conclusion. To avoid confusion and maintain simplicity, we specify all intermediate sentences in the reasoning chain to be premises and the final sentence which provides the final answer by choosing an option from given options as the conclusion. The premises may be single declarative sentences or combination of declarative sentences and inferences made from those declarations.
 
 % For sound and valid argument construction, the LLM must accurately deduce applicable laws, follow correct reasoning steps, and reach an appropriate conclusion. Similar to solving a deductive reasoning problem, our focus is on evaluating the soundness of arguments, particularly in the context of informal legal reasoning. 
\subsection{Manual Evaluation Of Reasoning Chains}
\label{section:manual-eval-sec}
Human evaluators are instructed to find flaws in a reasoning chain and explain the flaws in natural descriptive language. To solve a given legal question in \textit{Civ. Pro.}, an LLM generates a set of statements $<$$A: s_1, s_2, ..., s_k, c$$>$, where $A$ represents the legal argument/rationale put forward to solve the problem, with $s_1, s_2...s_k$ being the `\textit{k}' number of intermediate steps generated to reason towards the final conclusion $c$. Each step in the reasoning-chain, including the final conclusion, is separately evaluated for the presence/absence of errors. To create an error taxonomy, we adopt an exhaustive approach, continuously updating the taxonomy until no new errors are identified. Specifically, 120 reasoning chains containing approximately 537 reasoning steps are used for evaluation (generated as responses by four LLMs: Mistral-7B-v2-Instruct, Llama-3-8B-Instruct, GPT-3.5-turbo and GPT-4-turbo, to the same 30 data sample subset). This evaluation helped to solidify our proposed taxonomy as described in \textsection \ref{section:error_taxonomy}. Detailed statistics of the human-evaluations are provided in Tables \ref{table:manual-evaluation-premise} and \ref{table:manual-evaluation-conclusion} of Appendix \ref{section:manual-eval}. Further details regarding annotation guidelines and process, inter-annotator agreement statistics using Cohen's kappa coefficient \citep{cohen1960kappa} and annotation examples are provided in Appendix \ref{section:annotation_guidelines} and Appendix \ref{section:human_annot} (Tables \ref{table:human_annotation_example_initial}-\ref{table:human_annotation_example_final}).

%These initial annotations and analyses gave rise to the formulation of an error taxonomy on a fine-grained level.
% divided classifying the errors on two-broad levels: finding errors in 1. Premise-level, the intermediate step level of the rationale, and 2. Final Conclusion level, where an option is chosen as the answer to the legal MCQ. Futher details and examples of the process are provided in  

% Premise-level errors were divided into: 1. Misinterpretation 2. Factual Hallucinations and 3. Irrelevant premises. Conclusion-level errors were divided into: 1. Wrong conclusion from False Premises. 2. Wrong Conclusion from Incomplete Premises 3. Correct Conclusion from False Premises 4. Correct Conclusion from Incomplete Premises and 5. Correct Conclusion with Hallucinated Output 

\subsection{Proposed Error Taxonomy}
\label{section:error_taxonomy}
The error taxonomy is designed to mirror the types of errors humans make when reasoning about passage comprehension and constructing rational arguments. It classifies errors into two levels: 1. Premise-level and 2. Conclusion-level errors. Premise-level errors are based on `Errors of Law' and `Errors of Fact' grounded in the legal domain \citep{cornell2024mistake, cornell2024mistake_fact, oreilly2012errors, wilberg2023mistake}. While premise-level errors often influence errors at the conclusion level, many conclusion-level errors occur independently. Conclusion-level errors serve as indicators of the  overall decision-making ability of LLMs in generating the final answer to a legal question. 

\paragraph{Premise-level Errors}
These errors have occurred in one of the premises of the reasoning chain. They highlight the core issue with LLMs that ineffectively prioritize relevant parts of the prior context and incorrectly identify important information. We categorize these errors as shown in Table \ref{table:premise_errors}.
 
% \paragraph{Category 1: Misinterpretation}
% This is the dominant category of error which occurs in in the premise-level of the step-by-step rationale generated by LLMs. The underlying issue with LLMs is their inability to dynamically prioritize specific parts of prior context and 'intelligently' discern what information is relevant and what can be disregarded. 
% The following error instances fall under the taxon of misinterpretation: 1. Misunderstanding the legal rules. 2. Misunderstanding the legal situation/issue at hand. 3. Incorrectly applying the legal rule. 4. Incompletely applying a legal rule. 5. Omission of parts of the provided context while reasoning. 6. Wrong assumptions derived from the the provided context. 

% \paragraph{Category 2: Factual Hallucinations} This error category includes instances where the LLM, during its reasoning process, cites information that is either inconsistent with the facts of the given legal scenario or is entirely fabricated with no basis in reality. 

% \paragraph{Category 3: Irrelevant Premises}
% The error occurs when the LLM reasoner generates a premise which is not directly useful in reasoning towards an answer. These premises might be factually true and logically valid but they do not contribute towards proving the final conclusion. These errors can deviate the line of reasoning towards focusing on some aspect which would draw wrong conclusions.
\begin{table*}
\small
\begin{tabular}{l|p{3.8cm}|p{8.3cm}}
\toprule
\textbf{Broad-Category}           & \textbf{Sub-Category}                     & \multicolumn{1}{c}{\textbf{Description}}
                                                                                         \\ \midrule
\multirow{4}{*}{Wrong Conclusion} & Wrong Conclusion from False Premise(s)      & This error primarily occurs when the step-by-step rationale generated includes premises that are logically invalid, factually incorrect, irrelevant to solving the question posed, or a combination of these issues.                                                                                 \\ \cmidrule{2-3} 
                                  & Wrong Conclusion from Incomplete Premise(s) & This error occurs when valid and sound premises are provided but fail to fully support the reasoning, leading to the wrong conclusion. A special example is 'Wrong Conclusion from Correct Premises,' where sufficient premises still result in an incorrect conclusion.                 \\ \midrule
Right Conclusion                  & Right Conclusion from False Premise(s)      & This error occurs when the LLM reasons to the correct option while providing a wrong argument. One or multiple premises contain errors which fall under one of the three premise-level error categories and yet lead to the reasoning path choosing the correct option as its final answer. \\ \cmidrule{2-3}
                                  & Right Conclusion from Incomplete Premise(s) & This error occurs when the correct final option is selected, even though the premises provided are incomplete or insufficient to fully justify that conclusion.
                                  
                                                                                                    \\ \cmidrule{2-3}
                                  & Right Conclusion with Hallucinated Content & This error occurs when the LLM selects the correct option but the generated content does not semantically match the provided options. For example, the LLM might output 'Option D. The suspect is \textit{X},' when the actual content is 'Option D. The suspect is \textit{Y} and committed crime \textit{Z},' due to an LLM hallucination. 
                                  \\ \midrule
\end{tabular}
\caption{Error taxonomy for the Conclusion.}
\label{table:conclusion_errors}
\end{table*}

% \textit{Our current auto-evaluator system cannot detect this, as it cannot distinguish it from a 'Correct Conclusion from Correct Premises' scenario.}
% Wrong Conclusion being reached due to a contributing premise being untrue or being true but irrelevant in solving the question.

% Wrong Conclusion/Premature Conclusion being reached due to incomplete reasoning/premises not being sufficient to reach a correct conclusion OR Wrong Conclusion being reached even when the premises are correct and sufficient.

% Right conclusion being reached even when some premise is incorrect. This means that reasoning chain is still faulty and there is some misinterpretation in some part of the reasoning chain.

% Right conclusion being reached even when premises are insufficient. This means that reasoning chain is still faulty and there is some misinterpretation in some part of the reasoning chain.

% While the correct option is chosen, the content of the option generated by the LLM reasoner does not match with that of the actual option provided in the legal context.
\begin{table}[t]
\small
\scriptsize  % Use a smaller font size
\resizebox{0.9\linewidth}{!}{
\begin{tabular}{p{0.85\columnwidth} }
\toprule
    \textbf{Examples of a reasoning step and final conclusion evaluated by GPT-4o based auto-evaluator} \\ \midrule

       \multirow{3}{*}{}{\textbf{Premise 4:} Marla's Intent: Marla's intent is somewhat ambiguous. She is attending a two-year program and is unsure if she wants to be a stylist. She has considered staying in Denver to work if she likes the program, or leaving the program and looking for work in Denver if she doesn't like it.} \\ \textbf{Auto-Evaluator Evaluation and Error Explanation:} Premise 4: Misinterprets Marla's intent.\\The premise incorrectly suggests ambiguity in Marla's intent. According to the legal context, Marla's plans are open-ended, meaning she does not have definite plans to leave Denver. The ambiguity mentioned does not align with the legal context's explanation of "indefinite intent". \\ \textbf{Error Category Label:} "Misinterpretation" \\ 
    \midrule

        \multirow{4}{*}{}{\textbf{Predicted Conclusion:} Given that Marla resides in Colorado and has no definite plans to leave, she meets the domicile test's requirement of residing with the intent to remain indefinitely.

Final answer: D,}\\ \textbf{Extracted Ground-Truth Conclusion:} D.  is domiciled in Colorado, because she resides there with the intent to remain indefinitely. \\ \textbf{Auto-Evaluator Evaluation and Error Explanation:} Based on the provided conditions and the information given:

\\- The extracted conclusion's final answer is D.
\\- The ground truth option is also D.
\\- The content of the extracted conclusion and the ground truth option match: both state that Marla is domiciled in Colorado because she resides there with the intent to remain indefinitely.
\\- There is an error in the premises (misinterpretation of Marla's intent).\\\\
Given these points, the appropriate classification is:
\\
\textbf{CORRECT CONCLUSION FROM FALSE PREMISES} \\ \textbf{Error Category Label:} "Correct Conclusion from False Premises"\\

\bottomrule
\end{tabular}}
\caption{Example of reasoning step (premise) and conclusion evaluated by LLM-based `Auto-evaluator' (GPT-4o). The error category labels are extracted from the detailed explanations using an LLM prompted to extract error keywords.}
\label{table:example_annot}
\end{table}


\paragraph{Conclusion-level Errors} 
Conclusion-level errors indicate issues with deductive reasoning, reflecting the LLM's ability to follow premises to reach the correct conclusion. They also reveal how much the decision-making process is influenced by intermediate premises in choosing the final answer. We categorize these errors as shown in Table \ref{table:conclusion_errors}.  
% \paragraph{Category 1: Wrong Conclusion from False Premise(s)}
% This error primarily occurs when the step-by-step rationale generated includes premises that are logically invalid, factually incorrect, irrelevant to solving the question posed, or a combination of these issues.   
% These factors render the intermediate step as being 'False' and therefore lead the line of reasoning towards a wrong conclusion. 

% \paragraph{Category 2: Wrong Conclusion from Incomplete Premise(s)}
% This error occurs when the step-by-step generated rationale includes premises which, while correct and valid towards the answering question, fall short of providing the entire line of reasoning which could lead to the correct conclusion. As a result, the wrong conclusion gets selected as the answer. A special case of this error is the 'Wrong Conclusion from Correct Premises' where even though the premises generated are true and sufficient to find the correct conclusion, the LLM reasoner still spuriously chooses a wrong conclusion as its answer.

% \paragraph{Category 3: Correct Conclusion from False Premises}
% This error occurs when the LLM reasons to the correct option while providing a wrong argument. One or multiple premises contain errors which fall under one of the three premise-level error categories and yet lead to the reasoning path choosing the correct option as its final answer.  

\begin{figure*}[t]
    \centering     
    \includegraphics[width=0.95\linewidth]{Images/Autoeval.pdf}
    \caption{The overall schematic representation of the LLM-based error-detection and evaluation system and the calculation of the metrics. The reasoning chains are produced by 5 LLMs and the expert answer is referenced from the \textit{Civ. Pro.} dataset}
    \label{fig:autoeval}
\end{figure*}

\paragraph{Conclusion from Incomplete Premises \textit{vs.} Correct Premises}
We argue that a `Wrong Conclusion from Correct Premises' is essentially a `Wrong Conclusion from Incomplete Premises' because either the premises, though correct, are incomplete and lead to a wrong conclusion, or the LLM fails to explicitly generate a key premise. This poses a challenge for LLM-based auto-evaluators, as discussed in \textsection \ref{section:llm_aided_eval}, which struggle to assess whether the rationale is sufficient or inadequate.

\subsection{LLM-aided Automatic Evaluation}
\label{section:llm_aided_eval}
Manual analysis of reasoning chains provided a detailed categorization of errors; however, it was time-consuming and, therefore, challenging to scale for the entire dataset. Thus, we develop an alternate approach to leverage LLMs to evaluate the errors in the reasoning chains akin to human evaluation. Specifically, we use GPT-4o as the LLM backbone of the `auto-evaluator' system to identify and label the errors. The auto-evaluator assesses a total of 875 reasoning chains, encompassing approximately 4,844 individual reasoning steps, which include both premise-level and conclusion-level steps (refer Table \ref{table:average_steps}). The details of the implementation are described in Appendix \ref{section:auto-eval} and an example snippet of LLM-aided annotation is provided in Table \ref{table:example_annot}. We develop two approaches for error evaluation:     

\paragraph{Exact Error Label Match} In this approach, we task the `auto-evaluator' with identifying the exact error category labels which the human evaluators had labeled a particular premise/conclusion of a reasoning chain. Experiments revealed significant mislabeling between the auto-evaluator and human evaluators, with many `Misinterpretation' errors at the premise level being labeled as `Irrelevant Premises' or `Factual Hallucination' (Refer Appendix \ref{section:error_disambiguation}) by the auto-evaluator, and vice versa. Hence, we make changes to the auto-evaluators to include error explanations along with the labels.

Another significant challenge was the low error detection rate of factual hallucinations with the help of single-call LLM auto-evaluators. Motivated by \citet{varshney2023stitchtimesavesnine, dhuliawala2023chainofverificationreduceshallucinationlarge}, we develop a multi-call LLM system, consisting of two separate LLM calls,  in which one LLM call creates verification questions to probe various aspects of a premise and another LLM call answers them citing the provided legal context for factuality. A premise is considered to contain factual hallucination if the answers to any of the verification questions contradicts the content of the premise directly.  

\paragraph{Semantic Error Explanation Match} As an alternative approach to the above problems, we develop a multi-analyzer system consisting of three `single-call' and one `multi-call' LLM-based pipeline focused on providing explanation of errors at the premise-level. A `summarizer' LLM (Refer Appendix \ref{section: summ_agg_llm}) combines the individual analyses of all analyzers into a single error explanation for a premise. This enables the pipeline to detect and label multiple errors in a single premise (e.g., a premise containing both misinterpretation and factual hallucination).

To validate the effectiveness of the auto-evaluator, we sample 120 reasoning chains from the manually evaluated set of four LLMs (Mistral-7B-v2-Instruct, Llama-3-8B-Instruct, GPT-3.5-turbo and GPT-4-turbo). The human evaluators then compare their error category assignments as well as explanations to those provided by the auto-evaluator. The recall percentage of detecting an error at the premise level across four LLMs ranged from 83.87$\%$ to 90.6\%. The recall percentage range for detecting an error-free premise step ranged from 86.17$\%$ to 93.85$\%$. The details of the autoevaluator performance statistics are present in Tables \ref{table:auto-eval-agreement-premise} and \ref{table:auto-eval-agreement-conclusion} of Appendix \ref{section:auto-eval-agree}. Figure \ref{fig:autoeval-system} shows the pipeline of the error detection implemented using GPT-4o. 

% The agreement score (recall percentage) for detecting the presence of an error (through explanation match) in reasoning steps between the manual evaluation and the GPT-4o evaluation was $\sim$85.3$\%$. The agreement score (recall percentage) for detecting the absence of an error (through explanation match) in reasoning steps between the manual evaluation and the GPT-4o evaluation was $\sim$86$\%$

\begin{table}[t]
\centering
\caption{Results over the benchmark datasets. The mIoU is reported. %
}
\label{tab:sota_results}
\resizebox{\columnwidth}{!}{
\begin{tabular}{ccccccc}
\toprule
\textbf{Method} & \begin{tabular}[c]{@{}c@{}}Inference\\ Vocab. \end{tabular} & A-847 & PC-459 & A-150 & PC-59 & VOC-20 \\ \midrule
SAN \cite{xu2023side} & \checkmark & 12.4 & 15.7 & 27.5 & 53.8 & 94.0 \\
AttrSeg \cite{ma2024open} & \checkmark & -- & -- & -- & 56.3 & 91.6 \\
SCAN \cite{liu2024open} & \checkmark & 14.0 & 16.7 & 30.8 & \textbf{58.4} & \textbf{97.0} \\
EBSeg \cite{shan2024open} & \checkmark & 13.7 & 21.0 & 30.0 & 56.7 & 94.6 \\
SED \cite{xie2024sed} & \checkmark & 11.4 & 18.6 & 31.6 & 57.3 & 94.4 \\
CAT-Seg \cite{cho2024cat} & \checkmark & \textbf{16.0} & \textbf{23.8} & \textbf{31.8} & 57.5 & 94.6 \\ \midrule
CaSED + SAM \cite{conti2024vocabulary} & \xmark & -- & -- & 6.1 & 7.5 & 13.7 \\
CaSED + SAN \cite{conti2024vocabulary} & \xmark & -- & -- & 7.2 & 15.5 & 26.9 \\
DenseCaSED \cite{conti2024vocabulary} & \xmark & -- & -- & 8.6 & 13.4 & 20.5 \\
\textbf{Chick.-and-egg} (CaSED) & \xmark & 3.2 & 4.4 & 9.7 & \textbf{23.1} & \textbf{47.6} \\
\textbf{Chick.-and-egg} (RAM) & \xmark & \textbf{3.7} & \textbf{7.1} & \textbf{15.6} & 23.0 & 47.5  \\
\bottomrule
\end{tabular}
}
\end{table}

\section{Experiments}
\label{ch:results}

We conduct a comprehensive experimental analysis to investigate how different components affect VSS performance.
First, we evaluate the proposed two-stage approach on standard benchmarks to establish the baseline (\Cref{sec:benchmark}). We then present an in-depth analysis of the text encoder's behaviour and its impact on segmentation quality (\Cref{sec:text}). To better understand the relationship between the two-stages, we examine the image tagging accuracy and its influence on the segmentation task (\Cref{sec:tagging}). Finally, we study how different assignment thresholds in the evaluation protocol affect the reported performance (\Cref{sec:thresholds}).

\textbf{Implementation Details:}
The model is trained on the COCO-Stuff dataset \cite{caesar2018coco}, which contains 118k annotated images across 171 categories, following \cite{cho2024cat}. All results are based on CLIP \cite{radford2021learning} with a ViT-B/16 backbone. The image encoder and cost aggregation module are trained with per-pixel binary cross-entropy loss. 
The training parameters follow \cite{cho2024cat}. The batch size is 4, and models are trained for 80k iterations.
We performed image tagging and instance description using a frozen VLM model not trained on the testing dataset. More in detail, we examined the robustness of two models RAM \cite{zhang2024recognize} and Llava-1.6  \cite{liu2024llavanext}.

\textbf{Test Datasets:} The evaluation covers several datasets to ensure comprehensive testing. We used ADE20K \cite{zhou2019semantic} with both 150 and 847 class configurations, Pascal Context \cite{mottaghi2014role} with 59 and 459 class setups, and Pascal VOC \cite{everingham2010pascal} with its 20 classes. 

\begin{figure*}[t]
    \centering
    \resizebox{\textwidth}{!}{%
    \begin{tabular}{@{}ccccc@{}}
        
        
        \includegraphics[width=0.25\textwidth]{fig/qualitative/ADE_val_00000049_img.png} &
        \includegraphics[width=0.25\textwidth]{fig/qualitative_new/ADE_val_00000049_zeroseg.png} &
        \includegraphics[width=0.25\textwidth]{fig/qualitative/ADE_val_00000049_cased_labels_bigger.png} &
        \includegraphics[width=0.25\textwidth]{fig/qualitative/ADE_val_00000049_ours_labels_bigger.png} &
        \includegraphics[width=0.25\textwidth]{fig/qualitative/ADE_val_00000049_labels_bigger.png} \\[0.2cm]
        



        \includegraphics[width=0.25\textwidth] {fig/qualitative_new/ADE_val_00000683_img.png} &
        \includegraphics[width=0.25\textwidth] {fig/qualitative_new/ADE_val_00000683_zero_seg.png} &
        \includegraphics[width=0.25\textwidth]{fig/qualitative_new/ADE_val_00000683_real_image_cased.png} &
        \includegraphics[width=0.25\textwidth]{fig/qualitative_new/ADE_val_00000683_real_image.png} &
        \includegraphics[width=0.25\textwidth]{fig/qualitative_new/ADE_val_00000683_ground_truth.png} \\[0.2cm]


        \includegraphics[width=0.25\textwidth] {fig/qualitative_new/2007_008415_img.png} &
        \includegraphics[width=0.25\textwidth] {fig/qualitative_new/2007_008415_zeroseg.png} &
        \includegraphics[width=0.25\textwidth]{fig/qualitative_new/2007_008415_real_image_cased.png} &
        \includegraphics[width=0.25\textwidth]{fig/qualitative_new/2007_008415_real_image.png} &
        \includegraphics[width=0.25\textwidth]{fig/qualitative_new/2007_008415_ground_truth.png} \\[0.2cm]
        
        
        \textbf{Image} &
        \textbf{ZeroSeg} & 
        \textbf{Chicken-and-egg} (CaSED) & 
        \textbf{Chicken-and-egg} (RAM) & 
        \textbf{GT}
    \end{tabular}
    }
    \caption{Comparison of segmentation results across ZeroSeg \cite{rewatbowornwong2023zero} and \textbf{Chicken-and-Egg} (CaSED \cite{conti2024vocabulary} and RAM \cite{zhang2024recognize}), and ground-truth labels.}
    \label{fig:sota_qualitative_comparison}
\end{figure*}





\begin{table*}[t]
\centering
\caption{Results over the benchmark datasets by using soft assignment. † Results come from their original work. * mapped with Llama-2 \cite{ulger2024autovocabularysemanticsegmentation} rather than Sentence-BERT \cite{reimers2019sentence}. The soft assignment has threshold zero (i.e., all the words are assigned to a class in the evaluation vocabulary).}
\label{tab:mapping_results}
\resizebox{.99\textwidth}{!}{%
    \begin{tabular}{l|cc|cc|cccccccccc}
    \toprule
    \multirow{2}{*}{\textbf{Method}} & \multirow{2}{*}{\begin{tabular}[c]{@{}c@{}}\textbf{Vision}\\ \textbf{Backbone}\end{tabular}} & \multirow{2}{*}{\textbf{Stages}} &\multicolumn{2}{c}{\textbf{Components}} & A-847 & PC-459 & A-150 & PC-59 & VOC-20 \\
     & & & Tagging & Segmentation & &&&&& \\ \midrule
    Zero-Seg† \cite{rewatbowornwong2023zero} & ViT-B/16 & Mask2Tag & CLIP+GPT-2 & DINO & -- & -- & -- & 11.2 & 8.1 \\
    Auto-Seg† \cite{ulger2024autovocabularysemanticsegmentation} & ViT-L/16 & Tag2Mask & BLIP-2 & X-Decoder & 5.9* & -- & -- & 11.7* & \textbf{87.1}* \\
    TAG† \cite{kawano2024tag} & ViT-L/14 & Mask2Tag &CLIP+DB & DINO & -- & -- & 6.6 & 20.2 & 56.9 \\
    \textbf{Chicken-and-egg} (CaSED) & ViT-B/16 & Tag2Mask & CLIP+DB & CAT-Seg & 4.3 & 3.1 & 7.8 & \textbf{27.9} & 82.3 \\
    \textbf{Chicken-and-egg} (RAM) & ViT-B/16 & Tag2Mask & CLIP+Swin & CAT-Seg & \textbf{6.7} & \textbf{8.0} & \textbf{18.8} & 27.8 & 81.8 \\
        \bottomrule
    \end{tabular}
}
\end{table*}

\subsection{Benchmark Evaluation}\label{sec:benchmark}
We first conducted a comprehensive benchmark evaluation comparing existing approaches to establish a strong foundation for VSS and identify the most promising direction. This analysis served two key purposes: (1) to understand the current state-of-the-art performance in VSS and (2) to determine which baseline architecture would be the most suitable.

\textbf{Quantitatives:} \Cref{tab:sota_results} compares the mIoU across the Open-Vocabulary benchmarks. The proposed pipeline outperforms the previous VSS methods by a constant margin in all the datasets. To better accommodate VSS methods, they adopt a class remapping strategy that reduces penalization in cases where an exact class match is not found. This approach is reflected in \Cref{tab:mapping_results}, where the soft evaluation assignment takes place as described in \Cref{sec:assignment}.

\textbf{Qualitatives:} As shown in \Cref{fig:sota_qualitative_comparison}, the current approach fills the gap between the predictions and original dataset labels without a predefined vocabulary, offering finer-grained details across diverse scenarios (indoor and outdoor). %
The maps obtained suggest that current evaluation metrics might be overly pessimistic about the qualitative performance of the results. This issue arises from dataset limitations, where many instances struggle to find appropriate matches (e.g., in the third image, "husky" instead of "dog"). Mask2Tag methods like ZeroSeg \cite{rewatbowornwong2023zero} tend to over-segment the instances, getting improper text matches. On the other hand, Chicken-and-egg with CaSED tends to limit the number of predicted tags, %
while coupled with RAM it reaches the best compromise.




\subsection{Segmentatation Analysis} \label{sec:text}
\textbf{Perfect Tagger:} Our empirical results on the OVSS task - presented in \Cref{tab:gt_labels} - revealed that providing only image-specific text labels, %
rather than the entire vocabulary, during training led to improved segmentation performance when applying the same adjustment at inference. Although having access to inference labels is unrealistic, this setup represents the best achievable performance if tagger predictions were 100\% accurate. 
More in detail, in \Cref{tab:gt_labels}, the set of class names is defined for each batch as \(\mathcal{C}_b \subset \mathcal{C}\) during training, where \(\mathcal{C}_b\) represents the batch-specific subset of the entire class vocabulary \(\mathcal{C}\), dynamically selected based on the batch's unique context or requirements. This subset approach allows the model to focus on relevant classes without being overwhelmed by the entire vocabulary. However, we observed no gain when the text labels in inference are predicted from an image tagger. Nevertheless, this represents the upper bound currently obtainable with the state-of-the-art open-vocabulary method \cite{cho2024cat}. Moreover, we show in \Cref{tab:attvsadj} that when performing inference on perfect predictions (100\% accuracy from the tagger) we can boost performance by providing additional textual information.
\begin{table}[t]
\centering
\caption{\textbf{Comparison using CAT-Seg \cite{cho2024cat}, using ground truth classes as text embeddings at different stages}, where $T$ represents training and $I$ represents inference. The mIoU is reported on ADE-20K (A)\cite{zhou2019semantic}, Pascal Context (PC)\cite{mottaghi2014role}, and Pascal VOC (VOC) \cite{everingham2010pascal}.
}
\label{tab:gt_labels}
\resizebox{\columnwidth}{!}{%
\begin{tabular}{ccccccccc}
\toprule
\textbf{Method} & \multicolumn{2}{c}{\textbf{\begin{tabular}[c]{@{}c@{}}Only GT\\ Text Labels\end{tabular}}} & COCO & A-847 & PC-459 & A-150 & PC-59 & VOC-20 \\ \cmidrule{2-3}
 & T & I &  &  &  &  &  &  \\ \midrule
Base &  &  & 47.11 & 11.95 & 18.95 & 31.78 & 57.20 & 95.30 \\
L.Bound & \checkmark &  & 43.73 & 10.89 & 16.63 & 30.29 & 55.99 & 94.20 \\
U.Bound (I) &  & \checkmark & 56.15 & 12.38 & 18.38 & 45.53 & 69.77 & \textbf{95.87} \\
U.Bound & \checkmark & \checkmark & \textbf{64.03} & \textbf{13.98} & \textbf{24.04} & \textbf{51.21} & \textbf{72.79} & 94.38 \\
\bottomrule
\end{tabular}
}
\end{table}

\begin{table}[t]
\centering
\caption{All methods are based on \cite{cho2024cat}, changing textual descriptors, while performing inference on GT classes. (a)-(c) are trained using the predicted VLM information on COCO dataset.
}
\label{tab:attvsadj}
\resizebox{\columnwidth}{!}{%
\begin{tabular}{ccccccccc}
\toprule
\textbf{Method} & \multicolumn{2}{c}{\textbf{VLM input}} & COCO & A-847 & PC-459 & A-150 & PC-59 \\ \cmidrule{2-3}
& $Image$& $Text$ & & & \\ \midrule
Baseline \cite{cho2024cat} & & & 56.15 & 12.38 & 18.38 %
& 45.53 & 69.77 &\\ %
(a) Caption & \checkmark & & 58.17 & 12.71 & 17.07 %
& 47.09 & 71.04 &\\ %
(b) Class Adjectives & & \checkmark & 62.33 & 14.96 & 19.13 & 48.77 & 60.47 \\ %
(c) Instance Adjectives & \checkmark & \checkmark & \textbf{65.13} & \textbf{15.40} & \textbf{23.20} & \textbf{54.43} & \textbf{72.04} \\ %
\bottomrule
\end{tabular}
}
\end{table}

\begin{table*}[t]
\centering
\caption{Prompts for different algorithms for \cref{tab:attvsadj} results.}
\label{tab:prompt}
\resizebox{.9\textwidth}{!}{%
\begin{tabular}{cc}
\toprule
\textbf{\begin{tabular}[c]{@{}c@{}}Description\\ Level\end{tabular}} & \textbf{Prompts} \\ \midrule
Class & \begin{tabular}[c]{@{}c@{}}1. "Please group the classes in this list $<$dataset-class-list$>$ into groups of classes that are similar to each other \\ meaning they could be confused in an image. Every class should be in one group and only in one group. \\ Make sure there are no classes from the original list missing in your grouping. \\ This is an example of how the output should look: dog, cat, kitten, bird -- couch, desk, sofa, lamp -- knife, fork, plate"\\ 2. " The classes in the group are: $<$group$>$. Please generate a short list of adjectives for each class \\ that describe how the object looks in an image. The adjectives should be distinctive within each group meaning that \\ the same attribute should not appear for two classes in the same group. Generate at least one adjective for each class. \\ This is an example how the output should look. {giraffe: [tall, brown, spotted, yellow], tree: [tall, green], armchair: [comfortable]}\end{tabular}\\ \midrule
Instance & \begin{tabular}[c]{@{}c@{}}"The objects in the image are: $<$dataset-class-list$>$. Please generate a short list of adjectives\\ for each object that describes how the object looks in the image. \\ This is an example of how the output should look. \{giraffe: {[}tall, brown, spotted, interacting{]}, tree: {[}tall, green, leafy{]}\}"\end{tabular} \\ \bottomrule
\end{tabular}
}
\end{table*}

\textbf{Aiding Text Encoder with Descriptions:} Previous works \cite{ma2024open} used adjectives with the assumption to find the common class features that better describe each class. For example, a "dalmata" could be described as "a white dog with black spots". However, in typical recognition tasks, the categories are much broader, such as simply "dog", and a "dog" could be described very differently in terms of color and size. Hence, AttrSeg \cite{ma2024open} have focused on training strategies to find the optimal set of descriptions that could enhance class distinguishability while still being able to represent each class. While this approach has merit, it can result in the loss of fine-grained details. For instance, a "table" or "hat" could be of any size or color, and even a "wall" that is typically "white" could be "bricked" or some other texture.
Zhao et al. \cite{zhao2024gradient} experimented on CLIP's ability to identify different types of object attributes, including shape, material, color, size, and position. 
For shape and material attributes, CLIP showed a certain but limited knowledge, with the heat maps highlighting partial correct attention on obvious objects, but also exhibiting false positive and false negative errors. For color attributes, the results further verified that CLIP has a good ability to distinguish different colors.
For comparative attributes like size and position, CLIP produced some erroneous results, demonstrating that it relies more on the primary object (e.g., "cube", "red") rather than the comparative attribute (e.g., "small", "left"). Overall, their analysis suggests that CLIP has advantages with common perceptual attributes.
Therefore, we adopted a pre-trained VLM to find the corresponding descriptions given each image and its specific set of class names - the text labels of each image-, and we tried to enforce general language descriptions.
The prompts used are shown in \Cref{tab:prompt}. For generating captions, we employed the BLIP-2 model \cite{li2023blip} without any query input, whereas for the multimodal model, we utilized Llava-1.6 \cite{liu2024llavanext}. These models were selected because they both incorporate CLIP as their text encoder.
The text embedding of the captions is employed as a query within an additional cross-attention module, linking it to the embeddings of the classes. In the case of the adjectives, they are sampled and used within the template "A photo of a \{adjective\} \{class name\}".
We report the results in table \Cref{tab:attvsadj}, where adding image-specific content results beneficial, specially for large numbers of classes.
It is important to notice that, when using predicted labels from the image tagger or applying the complete set of image labels during inference, we did not observe the same benefit. %
In the VSS scenario, ambiguities with other classes are largely resolved during the CLIP segmentation stage by directly predicting the image's content using the image tagger. However, misclassifications may still occur at this stage, a behaviour explored in the next paragraph.
\begin{table}[t]
\centering
\caption{Class recognition accuracy of different VLMs with $T_\text{SBERT}$=$0.0$. \\ * using vocabulary.
\# FN = average number of missed classes, \# FP = average number of classes predicted but not in the ground truth.}
\label{tab:accuracy}
\resizebox{.5\textwidth}{!}{%
\begin{tabular}{ccccccccccc}
\toprule
\multirow{2}{*}{\textbf{\begin{tabular}[c]{@{}c@{}}Predicted\\ Classes\end{tabular}}} & \multirow{2}{*}{\textbf{\begin{tabular}[c]{@{}c@{}}Mapping\\ Model\end{tabular}}} & \multicolumn{3}{c}{A-150} & \multicolumn{3}{c}{PC-59} & \multicolumn{3}{c}{VOC-20} \\ \cmidrule{3-11} 
 &  & Acc & \#FP & \#FN & Acc & \#FP & \#FN & Acc & \#FP & \#FN \\ \midrule
CaSED & - & 10 & 10.7 & 7.8 & 22 & 9.3 & 4.0 & 50 & 9.5 & 0.9 \\
CaSED & SBERT & 23 & 7.4 & 6.8 & 42 & 5.4 & 3.1 & 84 & 4.2 & 0.3 \\
Llava-1.6 & - & 26 & 4.9 & 6.3 & 29 & 3.7 & 3.5 & 53 & \textbf{3.5} & 0.8 \\
Llava-1.6 & SBERT & 39 & \textbf{2.6} & 5.2 & 47 & \textbf{1.8} & 2.7 & 91 & 1.9 & 0.2 \\
RAM & - & 34 & 10.4 & 5.9 & 41 & 11.8 & 3.1 & 68 & 12.2 & 0.5 \\
RAM & SBERT & \textbf{46} & 5.7 & \textbf{4.8} & \textbf{61} & 5.4 & \textbf{2.2} & \textbf{96} & 4.8 & \textbf{0.1} \\
\midrule
RAM* & - & 79 & 16.7 & 1.95 & 80 & 6.2 & 1.1 & 97 & 1.5 & 0.1  \\
\bottomrule
\end{tabular}
}
\end{table}

\subsection{Image Tagging Analysis}\label{sec:tagging}
In \Cref{tab:accuracy}, we investigated various image tagging methods to understand how different types of errors affect the sensitivity of the segmentation module, particularly the text encoder since we use the tags as input to CLIP. We evaluated the impact of three architectures: a training-free method, CaSED \cite{conti2024vocabulary}, a multi-step trained method, RAM \cite{zhang2024recognize}, and a general-purpose multimodal model, Llava \cite{liu2024llavanext}.
CaSED %
uses a pre-trained vision-language model and an external database to extract candidate categories and assign the image to the best match. 
On the other hand, RAM %
generates large-scale image tags through automatic semantic parsing, followed by training a model to annotate images using both captioning and tagging tasks. A data engine then refines these annotations, and the model is retrained on this enhanced data, with final fine-tuning on a higher-quality subset.
\Cref{tab:accuracy} shows that using SBERT for evaluation avoids discarding words merely due to the absence of an exact match with the chosen word by the annotators. RAM achieves the best overall results across the evaluated datasets. In the table, the performance of Llava \cite{liu2024llavanext} %
demonstrates the versatility of powerful vision-language architectures. Note that the current baseline, RAM, does not reach a perfect accuracy even when the whole list of desired classes (i.e., non-vocabulary free), hence this represents the current limitation of such an approach. 
Furthermore, compared to CaSED, RAM demonstrates higher class recognition accuracy, but with more false positives on average. To investigate this further, we examined in \Cref{fig:miss_vs_false_sim} how the model is influenced by simulating a drop rate and false positives on top of the ground truth text classes in each image. In the table, the false positives are randomly selected from the vocabulary set. The influence of false negatives deeply influences the performance, while introducing false positives only leads to marginal degradation. These results confirm why RAM outperforms current alternatives: it has the fewest misclassifications, despite having a higher rate of false positives.

\begin{figure}
    \centering
    \includegraphics[width=.9\columnwidth, trim=0cm 0.55cm 0cm 0.75cm, clip]{fig/fpfn_full_stefano.png}
    \caption{Simulating missing classes or adding wrong ones over the OVSS baseline by assuming the labels are known at inference time.}
    \label{fig:miss_vs_false_sim}
\end{figure}


\subsection{Evaluation Assignment Thresholds} \label{sec:thresholds}
In \Cref{fig:thresh} we show the effect of providing different values for $T_\text{SBERT}$. Unlike Zero-Seg \cite{rewatbowornwong2023zero}, we did not observe a consistent trend in the optimal threshold across datasets. Respectively, $0.6-0.7$ for A-847, PC-459 and A-150, $0.5$ for PC-59 and $0.1$ for VOC-20. 
Our findings suggest that as the number of classes increases, we need to be more confident in the assignment, hence a higher threshold leads to a better score.

\begin{figure}
    \centering
    \includegraphics[width=.9\columnwidth, trim=0cm 0.55cm 0cm 0.75cm, clip]{fig/thresholds_stefano.png}
    \caption{Ablation over different thresholds for the evaluation mapping.%
    }
    \label{fig:thresh}
\end{figure}





\section{Results and Analysis}
\label{sec:results}

\subsection{Objective Evaluation}

\paragraph{Soundness metrics are high but correctness scores are low}
Table \ref{table:metrics} shows that the majority of the premises are error-free (with the highest being GPT-4o having 78.4\% of the generated premises being error-free). In contrast, Figure \ref{fig:prem_conc_corr} reveals that an average of $\sim$96\% of reasoning chains leading to conclusions from false premises have one or more misinterpretation errors in the intermediate premises. This finding, aided by empirical human analysis, suggests that much of the LLM-generated reasoning chain re-iterates existing context, while most errors occur in the smaller portion where new `decision-making' inferences are generated. The similarity in correctness score in Mistral-7B-v2-Instruct and Llama-3-8B-Instruct in contrast to the higher accuracy of Llama-3-8B-Instruct could be attributed to the lesser number of steps (see Table \ref{table:average_steps} (Appendix \ref{section:avg_steps_generated})) on average in the reasoning chain of Llama-3-8B-Instruct when compared to Mistral-7B-v2-Instruct. 
\begin{table}[!htbp]
\centering
\begin{tabular}{lccc}
\toprule
Model   & S ($\uparrow$) & A ($\uparrow$)  & C ($\uparrow$)\\ 
\midrule
Mistral-7B-v2-Instruct   & 0.623  & 0.371 & 0.131\\ 
Llama-3-8B-Instruct   & 0.493  & 0.451 & 0.137\\ 
GPT-3.5-turbo   & 0.607  & 0.417 & 0.217\\ 
GPT-4-turbo     & 0.738  & 0.725 & 0.417\\ 
GPT-4o   & \textbf{0.784}  & \textbf{0.737} & \textbf{0.445}\\ 
\bottomrule
\end{tabular}
\caption{The results for soundness, accuracy, and correctness metrics for all LLMs on the \textit{Civ. Pro.} dataset. Here `S' denotes the Soundness, `A' denotes the Accuracy, and `C' denotes the Correctness.}
\label{table:metrics}
\end{table}

\paragraph{Accuracy vs. Correctness Score} Table \ref{table:metrics} and Figure \ref{fig:acc_correctness_results} show a sharp decrease (an average of $\sim$27\%) in the scores of accuracy to correctness across all LLMs. The highest fall in percentage is observed in Llama-3-8B-Instuct (31.4\% decrease). This is significant as it shows that while LLMs can arrive at the correct conclusion, there are a lot of cases where the reasoning chain they generate is not entirely error-free. These results also suggest that LLMs often rely on superficial correlations and patterns, likely learned in the training stages, to arrive at correct conclusions, rather than through genuine reasoning. In high-stakes domains such as legal, financial, and medical fields, it is imperative that the reasoning generated by LLMs is completely error-free as even minor inaccuracies in these critical areas can lead to significant consequences. This also underscores the necessity for robust evaluation mechanisms to ensure the reliability and correctness of model outputs.   

\begin{figure}[h]
    \centering
    \includegraphics[width=1.0\linewidth]{Images/accuracy_and_correctness.pdf}
    \caption{Performance of 5 LLMs in terms of Accuracy vs. Correctness on the \textit{Civ. Pro.} dataset. Here, Mistral stands for Mistral-7B-v2-Instruct, Llama stands for Llama-3-8B-Instruct, GPT-3.5t and GPT-4t stand for GPT-3.5-turbo and GPT-4-turbo respectively.}
    \label{fig:acc_correctness_results}
\end{figure}

\begin{table}[ht]
\small
\centering
\begin{tabular}{lcccc}
\toprule
% \multirow{2}{*}{} & \multicolumn{4}{c}{Prompting Strategies} \\
% \cmidrule(lr){2-5}
Prompting               & B & PS & SC & SD \\
\midrule
\textbf{Gemini} & \multicolumn{4}{c}{} \\
w/o feedback     & 63.31    & 59.17          & \textbf{61.54}        & \textbf{64.50}          \\
w/ feedback      & \textbf{64.50}    & \textbf{62.13}          & 59.76        & 63.31          \\
\midrule
\textbf{Llama} & \multicolumn{4}{c}{} \\
w/o feedback     & 53.71    & 50.29          & 48.58        & 47.42          \\
w/ feedback      & \textbf{57.14}    & \textbf{52.00}          & \textbf{52.57  }      & \textbf{49.14}          \\
\bottomrule
\end{tabular}
\caption{Comparison of accuracy metric for models under different prompting strategies with and without feedback. The models are Gemini-1.5-Flash and Llama-3-8B-Instruct. The prompting strategies abbreviations stands for B: Baseline (CoT), PS: Plan-and-Solve, SC: Self-Correct, and SD: Self-Discovery.}
\label{table:prompt_accuracy}
\end{table}
\paragraph{Larger, proprietary models `reason' better than smaller, open-source models}
Figure \ref{fig:acc_correctness_results} and Table \ref{table:metrics} convey that proprietary models generate more error-free reasoning steps and arrive at the correct conclusion more often than the open-source LLMs. An exception is GPT-3.5-turbo, which performs comparably to Llama-3-8B-Instruct and Mistral-7B-v2-Instruct, suggesting that training data and methods might play a more significant role in enhancing reasoning than merely scaling model parameters. 

\subsection{Reasoning Chain Evaluation}
\paragraph{`Misinterpretations' are the dominant category of errors at premise-level}
Figure \ref{fig:prem-level-cat} (Appendix \ref{section: premise-level-error-dist}) and Figure \ref{fig:prem_conc_corr} reveal that `Misinterpretation' is the most dominant category of error which occurs in the reasoning chains at the premise-level. This highlights that  LLMs struggle to fully grasp the nuanced complexities of legal scenarios requiring the demonstration of critical analysis in zero-shot CoT settings.  

% This aptly goes on to show that LLMs, while trained on vast amounts of data, do not represent and effectively `understand' the complex nuances of legal scenarios (or any other complex context requiring critical analysis) when used in a zero-shot setting which is equivalent to an experienced human reasoner (a lawyer in the scenario of legal reasoning) being able to capture the nuances and correctly solve the problem with minimal supervision/guidance. The frequent occurrence of misinterpretations within reasoning chains, which lead to conclusion-level errors, suggests that they are a primary factor contributing to flaws or failures in the reasoning process.
\paragraph{`Wrong Conclusion from False Premises' is the dominant category of error at conclusion-level} The prevalence of `Wrong Conclusion from False Premises' (Figure \ref{fig:conc-level-cat} (Appendix \ref{section: conclusion-level-error-dist})) in conclusion-level errors results from premise-level mistakes leading to incorrect conclusions. However, in GPT-4-turbo and GPT-4o, the dominant error is `Correct Conclusion from False Premises,' suggesting these models may be relying on patterns of similar examples from their training.  

\subsection{Discussion on Error-Mitigation Strategies}
\label{section:mitigation_discussion}
We carry out several experiments on the \textit{Civ. Pro.} dataset, employing widely used prompting techniques alongside the most frequently observed errors we found through \textsection \ref{section:error_taxonomy} with the aim to explore the possibility of enhancing the reasoning capabilities of both closed-source and open-source LLMs. Four prompting techniques are utilized: (1) Chain-of-Thought \citep{wei2022chainofthought} (2) Plan-and-Solve \citep{wang2023plan} (3) Self-Correct  \citep{zhang2024smalllanguagemodelsneed} and (4) Self-Discovery  \citep{zhou2024selfdiscoverlargelanguagemodels}. These techniques are tested with and without incorporating error definitions as feedback, following the Feedback-Learning method \citep{tyagi2024stepbystepreasoningsolvegrid}. Detailed descriptions of the prompting strategies can be found in Appendix \ref{section:prompting_strategies}.

The error definitions are provided in three styles: generic, short, and long. The generic version uses the error definitions from the Feedback-Learning method, while the short and long versions are derived from the error taxonomy described in \textsection\ref{section:error_taxonomy}. All experiments are conducted in a zero-shot setting, and we evaluate each prompting technique based on the accuracy metric. We test one closed-source model, Gemini-1.5-Flash, and one open-source model, Llama-3-8B-Instruct.

As shown in Table \ref{table:prompt_accuracy}, adding the error definitions as feedback showed improvement in accuracy up to ~4\%. For Llama-3-8B-instruct, accuracy improved across all prompting techniques, whereas for Gemini-1.5-Flash, the accuracy increased only for the Chain-of-Thought and Plan-and-Solve methods. From our observations, the decrease in accuracy for these strategies with Gemini resulted due to self-doubting \citep{krishna2023intersectionselfcorrectiontrustlanguage} nature of LLMs. These findings suggest that while feedback on errors provides marginal improvements in LLM performance, there is a need to develop more effective frameworks beyond prompting, such as agent-based methods, that account for these errors and enhance the model’s legal reasoning capabilities.

% These findings show The minor improvements highlight the need for strategies beyond prompting to achieve further performance gains. 

% This finding points towards the possibility that while feedback of errors committed helps in marginally improving LLM performance, they are not strong supervision signals for the LLMs to rectify their reasoning traces.

% \mihir{Write 1-2 lines about what are the takeaways. Why these experiments add value to current study?} 

% We used the zero-shot CoT method as the baseline method in which the LLM is prompted to provide the final answer along with step-by-step reasoning. Plan-and-Solve prompts the LLM first to generate a plan to solve the problem without solving it and after that the LLM carries out the self-suggested plan to get the final answer. Self-Correct uses self-verification and self-refining to improve the reasoning ability of the LLMs. Self-Discover utilizes self-discover reasoning modules to create an explicit reasoning structure to follow to solve the problem. Lastly, we follow Feedback-Learning which provides the definitions and methods to navigate through the identified error taxonomy. The detailed prompt structure for each technique is illustrated in the Appendix. 

% and GPT-4o outperforms Llama3-70B in every prompting strategy. It is also should be noted that Self-Discovery is the most costly prompting technique due to the use of more tokens and the 4-step structure to get the final answer.

% \begin{table*}[!htbp]
% \small
% \centering
% \begin{tabular}{lcccccc}
% \toprule
% Model   & Baseline & Plan-and-Solve  & Self-Correct & Self-Discovery & Feedback-Learning & Self-Ranking \\ 
% \midrule
% GPT-4o   & 78.85  & 80.00 & 80.00 & 76.57 & 79.43 & 80.57 \\ 
% Llama-3-70B   & 50.29  & 52.00 & 51.43 & 48.00 & 52.00 & 57.71 \\ 
% \bottomrule
% \end{tabular}
% \caption{The results for accuracy for GPT-4o and Llama-3-70B on the legal reasoning dataset.}
% \label{table:accuracy}
% \end{table*}

% \st{sheds light on the finding} \mihir{do not use words like this}

\section{Conclusion and future work}
In this study, we examined the ability of LLMs to produce self-generated counterfactual explanations (SCEs).
We design a prompt-based setup for evaluating the efficacy of \SCEs.
Our results show that LLMs consistently struggle with generating valid \SCEs. In many cases model prediction on a \SCE does not yield the same target prediction for which the model crafted the \SCE.
Surprisingly, we find that LLMs put significant emphasis on the context---the prediction on \SCE is significantly impacted by the presence of original prediction and instructions for generating the \SCE.
Based on this empirical evidence, we argue that LLMs are still far from being able to explain their own predictions counterfactually.
Our findings add to similar insights from recent studies on other forms of self-explanations~\cite{lanham2023measuring,tanneru2024quantifying}.



Our work opens several avenues for future work. Inspired by counterfactual data augmentation~\cite{sachdeva2023catfood}, one could include the counterfactual explanation capabilities a part of the LLM training process. This inclusion may enhance the counterfactual reasoning capabilities of the LLM. Follow ups should also explore the effect of prompt tuning, specifically, model-tailored prompts for generating \SCEs. These approaches might lead to better quality \SCEs.


We limited our investigation to open source models of upto 70B parameters. Extending our analysis to larger and more recent models, \eg, DeepSeek R1 671B, and closed source models like OpenAI o3 would be an interesting avenue for future work.

Finally, our experiments were limited to relatively simple tasks: classification and mathematics problems where the solution is an integer. This limitation was mainly due to the fact that it is difficult to automatically judge validity of answers for more open-ended language generation tasks like search and information retrieval. Scaling our analysis to such tasks would require significant human-annotation resources, and is an important direction for future investigations.



% \section*{Ethics Statement}
% We have utilized AI assistants, specifically Grammarly and ChatGPT, to correct grammatical errors and rephrase sentences

% \section*{Acknowledgement}

% We extend our gratitude to the Research Computing (RC), and Enterprise Technology at ASU for providing computing resources, and access to the ChatGPT enterprise version for experiments. We acknowledge the support of a CISCO research grant.

% Entries for the entire Anthology, followed by custom entries
\bibliography{references}
\bibliographystyle{acl_natbib}

\clearpage

\appendix
\newpage
\appendix
\onecolumn

\renewcommand{\thetable}{A\arabic{table}} % Prefix table numbers with 'A'
\renewcommand{\thefigure}{A\arabic{figure}} % Prefix figure numbers with 'A'
\renewcommand{\theequation}{A\arabic{equation}} % Prefix equation numbers with 'A'

\setcounter{table}{0} % Reset table counter
\setcounter{figure}{0} % Reset figure counter
\setcounter{equation}{0} % Reset equation counter

\section*{Appendix}

\section{Optimal Brain Surgeon Derivation}
\label{OBS_ALGORITHM}

In the original setup in OBS, we have a local quadratic model for the loss $L$ given by:
$$
    \delta L = L(w + \delta w) \approx L(w) + \nabla_w L^T \delta w + \frac{1}{2} \delta w^T H \delta w
$$
Since OBS is a pruning-after-training approach, they discarded the 1-st order component. Reducing the expression for saliency as:
$$
    \delta L = \frac{1}{2} \delta w^T H \delta w
$$
To remove a single parameter, the authors of OBS introduced the constraint $e_q^T \delta w + w_q = 0$, with $e_q$ being the $q^{\text{th}}$ canonical basis vector. The pruning is defined as a constrained optimization problem of the form:
$$
    \min_{\delta w \in \mathbb{R^d}} \left( \frac{1}{2} \delta w^T H \delta w\right),
    ~~\text{s.t}~~
    e_q^T \delta w + w_q = 0.
$$
And the choice of which parameter to remove becomes:
$$
    \min_{q \in \mathcal{Q}} \left\{
        \min_{\delta w \in \mathbb{R^d}} \left( \frac{1}{2} \delta w^T H \delta w\right),
        ~~\text{s.t}~~
        e_q^T \delta w + w_q = 0
    \right\}.
$$
To solve the internal problem, we use a Lagrange multiplier $\lambda$ to write the problem as an unconstrained optimization case as follows:
$$
    \mathcal{L}(\delta w, \lambda) =
    \frac{1}{2} \delta w^T H \delta w +
    \lambda(e_q^T \delta w + w_q).
$$
Then, to find the stationary conditions, we compute the partial derivatives with respect to $\delta w$ and $\lambda$, and equate them to 0, obtaining:
$$
    \nabla_{\delta w} \mathcal{L} = 
    H \delta w + \lambda e_q = 0 
    \rightarrow
    \delta w = - \lambda H^{-1} e_q
$$
$$
    \nabla_{\lambda} \mathcal{L} =
    e_q^T \delta w + w_q = 0
    \rightarrow
    e_q^T \delta w = -w_q
$$
With some replacements, we get:
$$
    e_q^T \delta w = -w_q
    \rightarrow
    e_q^T \left( 
        - \lambda H^{-1} e_q
    \right) = -w_q
    \rightarrow
    - \lambda e_q^T H^{-1} e_q = -w_q
    \rightarrow
    \lambda = \frac{w_q}{e_q^T H^{-1} e_q} = \frac{w_q}{[H^{-1}]_{qq}}
$$
$$
    \delta w = - \frac{w_q H^{-1} e_q}{[H^{-1}]_{qq}}
$$
Replacing the expression for $\delta w$ in the saliency expression, we have:
\begin{align*}
    \delta L = \frac{1}{2} \delta w^T H \delta w
    &= \frac{1}{2}\left(
        - \frac{w_q H^{-1} e_q}{[H^{-1}]_{qq}}
    \right)^T
    H
    \left(
        - \frac{w_q H^{-1} e_q}{[H^{-1}]_{qq}}
    \right)
    \nonumber \\
    &= 
    \frac{w_q^2}{2[H^{-1}]_{qq}^2}
    \left(
        H^{-1} e_q
    \right)^T
    H
    \left(
        H^{-1} e_q
    \right)
    \nonumber \\
    &= 
    \frac{w_q^2}{2[H^{-1}]_{qq}^2}
    e_q ^T
    H^{-1}
    e_q
    = 
    \frac{w_q^2}{2[H^{-1}]_{qq}^2}
    [H^{-1}]_{qq}
    = 
    \frac{w_q^2}{2[H^{-1}]_{qq}}
    \nonumber \\
\end{align*}
%------------------------------------------------------------------------------------------------
\newpage
\section{Fisher Brain Surgeon Sensitivity Derivation}
\label{FBSS_ALGORITHM}
As we considered a PBT setting, it is not possible to ignore the first-order term in the local quadratic approximation of the error as it could still be informative. In this case, our model for sensitivity is given by: 
$$
    \delta L = \nabla_w L^T \delta w + \frac{1}{2} \delta w^T H \delta w
$$
The process to remove a single parameter remains similar; the constraint $e_q^T \delta w + w_q = 0$, with $e_q$ is still valid, redefining the optimization problem as:
$$
    \min_{\delta w \in \mathbb{R^d}} \left(
        \nabla_w L^T \delta w +  \frac{1}{2} \delta w^T H \delta w
    \right),
    ~~\text{s.t}~~
    e_q^T \delta w + w_q = 0.
$$
And the choice of which parameter to remove becomes:
$$
    \min_{q \in \mathcal{Q}} \left\{
        \min_{\delta w \in \mathbb{R^d}} \left(
            \nabla_w L^T \delta w + \frac{1}{2} \delta w^T H \delta w
        \right),
        ~~\text{s.t}~~
        e_q^T \delta w + w_q = 0
    \right\}.
$$
Using a Lagrange multiplier $\lambda$ as in the reference case, we solve the following unconstrained optimization problem:
$$
    \mathcal{L}(\delta w, \lambda) =
    \nabla_w L^T \delta w + 
    \frac{1}{2} \delta w^T H \delta w +
    \lambda(e_q^T \delta w + w_q).
$$
With the following stationary conditions:
$$
    \nabla_{\delta w} \mathcal{L} = 
    \nabla_w L + H \delta w + \lambda e_q = 0 
    \rightarrow
    \delta w = - (\lambda H^{-1}e_q + H^{-1} \nabla_w L)
$$
$$
    \nabla_{\lambda} \mathcal{L} =
    e_q^T \delta w + w_q = 0
    \rightarrow
    e_q^T \delta w = -w_q
$$
The expression for $\lambda$ is redefined as follows:
\begin{align*}
    e_q^T \left(
        - (\lambda H^{-1}e_q + H^{-1} \nabla_w L)
    \right) 
    &= -w_q
    \nonumber \\
    \lambda e_q^T H^{-1} e_q + e_q^T H^{-1} \nabla_w L
    &= w_q
    \nonumber \\
    \lambda [H^{-1}]_{qq} 
    &= w_q - e_q^T H^{-1} \nabla_w L
    \nonumber \\
    \lambda
    &= \frac{w_q - e_q^T H^{-1} \nabla_w L}{[H^{-1}]_{qq}}
\end{align*}
Replacing the expression for $\delta w$ in our sensitivity expression, we have:
\begin{align*}
    \delta L = \nabla_w L^T \delta w + \frac{1}{2} \delta w^T H \delta w
    &= 
    \nabla_w L^T \left[
        - (\lambda H^{-1}e_q + H^{-1} \nabla_w L)
    \right]
    \nonumber \\
    &+
    \frac{1}{2}\left[
        - (\lambda H^{-1}e_q + H^{-1} \nabla_w L)
    \right]^T
    H
    \left[
        - (\lambda H^{-1}e_q + H^{-1} \nabla_w L)
    \right]
    \nonumber \\
    &= 
    - \lambda \nabla_w L^T H^{-1}e_q - \nabla_w L^T H^{-1} \nabla_w L
    \nonumber \\
    &+
    \frac{1}{2}\left[
        (\lambda H^{-1}e_q)^T + (H^{-1} \nabla_w L)^T
    \right]
    \left[
        \lambda H H^{-1}e_q + H H^{-1} \nabla_w L)
    \right]
    \nonumber \\
    &= 
    - \lambda \nabla_w L^T H^{-1}e_q - \nabla_w L^T H^{-1} \nabla_w L
    \nonumber \\
    &+
    \frac{1}{2}\left[
        (\lambda H^{-1}e_q)^T + (H^{-1} \nabla_w L)^T
    \right]
    \left[
        \lambda e_q + \nabla_w L
    \right]
    \nonumber \\
    &= 
    - \lambda \nabla_w L^T H^{-1}e_q - \nabla_w L^T H^{-1} \nabla_w L
    \nonumber \\
    &+
    \frac{1}{2}\left[
        (\lambda H^{-1}e_q)^T \lambda e_q
        + (H^{-1} \nabla_w L)^T \lambda e_q
        + (\lambda H^{-1}e_q)^T \nabla_w L
        + (H^{-1} \nabla_w L)^T \nabla_w L
    \right]
    \nonumber \\
    &= 
    - \lambda \nabla_w L^T H^{-1}e_q - \nabla_w L^T H^{-1} \nabla_w L
    \nonumber \\
    &+
    \frac{1}{2}\left[
        \lambda^2 e_q^T H^{-1} e_q
        + \lambda \nabla_w L^T H^{-1} e_q
        + \lambda e_q^T H^{-1} \nabla_w L
        + \nabla_w L^T H^{-1} \nabla_w L
    \right]
    \nonumber \\
    &= 
    \frac{1}{2}\left[
        \lambda^2 [H^{-1}]_{qq}
        - \lambda \nabla_w L^T H^{-1} e_q
        + \lambda e_q^T H^{-1} \nabla_w L
        - \nabla_w L^T H^{-1} \nabla_w L
    \right]
    \nonumber \\
\end{align*}
Finally, replacing the $\lambda$:
\begin{align*}
    \delta L 
    &= 
    \frac{1}{2}\left[
        \lambda^2 [H^{-1}]_{qq}
        - \lambda \nabla_w L^T H^{-1} e_q
        + \lambda e_q^T H^{-1} \nabla_w L
        - \nabla_w L^T H^{-1} \nabla_w L
    \right]
    \nonumber \\
    &= 
    \frac{1}{2[H^{-1}]_{qq}}\left[
        (w_q - e_q^T H^{-1} \nabla_w L)^2 
        + (w_q - e_q^T H^{-1} \nabla_w L)(e_q^T H^{-1} \nabla_w L - \nabla_w L^T H^{-1} e_q)
        - \nabla_w L^T H^{-1} \nabla_w L
    \right]
    \nonumber \\
    &= 
    \frac{1}{2[H^{-1}]_{qq}}[
        w_q^2
        - 2 w_q (e_q^T H^{-1} \nabla_w L)
        + (e_q^T H^{-1} \nabla_w L)^2
        + w_q (e_q^T H^{-1} \nabla_w L)
    \nonumber \\
        &- w_q (\nabla_w L^T H^{-1} e_q)
        - (e_q^T H^{-1} \nabla_w L)(e_q^T H^{-1} \nabla_w L)
        + (e_q^T H^{-1} \nabla_w L)(\nabla_w L^T H^{-1} e_q)
        - \nabla_w L^T H^{-1} \nabla_w L
    ]
    \nonumber \\
    &= 
    \frac{1}{2[H^{-1}]_{qq}}[
        w_q^2
        - w_q (e_q^T H^{-1} \nabla_w L)
        + (e_q^T H^{-1} \nabla_w L)^2
    \nonumber \\
        &- w_q (\nabla_w L^T H^{-1} e_q)
        - (e_q^T H^{-1} \nabla_w L)^2
        + (e_q^T H^{-1} \nabla_w L)(\nabla_w L^T H^{-1} e_q)
        - \nabla_w L^T H^{-1} \nabla_w L
    ]
    \nonumber \\
    &= 
    \frac{1}{2[H^{-1}]_{qq}}\left[
        w_q^2
        - 2 w_q (e_q^T H^{-1} \nabla_w L)
        + (e_q^T H^{-1} \nabla_w L)^2
        - \nabla_w L^T H^{-1} \nabla_w L
    \right]
    \nonumber \\
    &= 
    \frac{1}{2[\hat{F}^{-1}]_{qq}}
    \left[
        w_q - (e_q^T \hat{F}^{-1} \nabla \mathcal{L}(w_0))
    \right]^2
\end{align*}

%------------------------------------------------------------------------------------------------

\newpage
\section{Training and Testing Details}
\label{appendix:training_parameters}

We perform an 80:20 stratified split, with a constant seed, on the CIFAR10/100 training dataset to obtain a validation set with the same class distribution. For both datasets, we have a training set with 40,000 samples, a validation set with 10,000 samples, and a testing set of 10,000 samples. Validation is performed after each training step, and the weights of the best-performing validation step (based on top-1 accuracy) are utilized for the final evaluation on the testing set. Table \ref{tab:table_training_parameters} summarizes the training parameters.

\begin{table}[h]
\caption{Training parameters used for ResNet18 and VGG19 on the CIFAR-10/100 datasets.}
\label{tab:table_training_parameters}
\vskip 0.15in
\begin{center}
\begin{small}
\begin{sc}
\begin{tabular}{lcc}
\toprule
Parameter & ResNet18 & VGG19 \\
\midrule
Number of steps       & 160 & 160 \\
Criterion             & CE & CE \\
Optimizer             & SGD & SGD \\
Learning rate         & 0.01 & 0.1 \\
Momentum              & 0.9 & 0.9 \\
Weight decay          & $5 \times 10^{-4}$ & $1 \times 10^{-4}$ \\
Learning rate drops   & [60, 120] & [60, 120] \\
Learning rate drop factor & 0.2 & 0.1 \\
\bottomrule
\end{tabular}
\end{sc}
\end{small}
\end{center}
\vskip -0.1in
\end{table}

%------------------------------------------------------------------------------------------------

\newpage
\section{Results CIFAR10}
\subsection{ResNet18}
\label{appendix:CIFAR10_ResNet18}

\begin{table}[h]
\caption{Performance of different sensitivity methods for pruning evaluated using ResNet18 on the CIFAR-10 testset. The right side of the table presents our proposed criteria. The mean accuracy and standard deviation are reported across three initialization seeds for various sparsity levels. Baseline, no pruning: $91.78 \pm 0.09$.}
\label{tab:resnet18_cifar10_compressors}
\vskip 0.15in
\begin{center}
\begin{small}
\begin{sc}
\resizebox{\textwidth}{!}{%
\begin{tabular}{lccccc|cccc}
\toprule
Sparsity  & Random & Magnitude & GN & SNIP & GraSP & FD & FP & FTS & FBSS \\
\midrule
0.10  & 91.71 ± 0.21 & 91.72 ± 0.07 & 91.57 ± 0.15 & 91.72 ± 0.07 & 89.16 ± 0.05 & 91.87 ± 0.13 & 91.63 ± 0.21 & 91.53 ± 0.12 & 91.76 ± 0.08 \\
0.20  & 91.63 ± 0.11 & 91.42 ± 0.12 & 91.51 ± 0.09 & 91.64 ± 0.16 & 88.69 ± 0.34 & 91.50 ± 0.12 & 91.65 ± 0.14 & 91.53 ± 0.15 & 91.54 ± 0.13 \\
0.30  & 91.45 ± 0.18 & 91.61 ± 0.13 & 91.68 ± 0.20 & 91.65 ± 0.08 & 88.67 ± 0.26 & 91.65 ± 0.18 & 91.44 ± 0.27 & 91.49 ± 0.05 & 91.62 ± 0.07 \\
0.40  & 91.59 ± 0.18 & 91.06 ± 0.16 & 91.61 ± 0.09 & 91.55 ± 0.08 & 88.24 ± 0.33 & 91.51 ± 0.05 & 91.38 ± 0.13 & 91.56 ± 0.28 & 91.39 ± 0.05 \\
0.50  & 91.60 ± 0.06 & 91.32 ± 0.13 & 91.44 ± 0.13 & 91.22 ± 0.07 & 87.69 ± 0.15 & 91.30 ± 0.18 & 91.58 ± 0.16 & 91.46 ± 0.19 & 91.41 ± 0.05 \\
0.60  & 91.10 ± 0.16 & 91.18 ± 0.16 & 91.59 ± 0.13 & 91.24 ± 0.04 & 87.48 ± 0.55 & 91.34 ± 0.07 & 91.35 ± 0.16 & 91.40 ± 0.11 & 91.38 ± 0.18 \\
0.70  & 91.17 ± 0.04 & 91.07 ± 0.07 & 91.19 ± 0.17 & 91.33 ± 0.18 & 87.26 ± 0.34 & 91.34 ± 0.23 & 91.42 ± 0.23 & 91.18 ± 0.18 & 91.27 ± 0.14 \\
0.80  & 90.78 ± 0.08 & 91.10 ± 0.12 & 90.95 ± 0.35 & 90.74 ± 0.10 & 87.18 ± 0.51 & 90.95 ± 0.11 & 91.08 ± 0.06 & 90.94 ± 0.22 & 90.73 ± 0.33 \\
0.90  & 89.35 ± 0.13 & 89.88 ± 0.28 & 90.39 ± 0.23 & 90.36 ± 0.34 & 86.60 ± 0.51 & 90.04 ± 0.21 & 90.20 ± 0.08 & 90.55 ± 0.23 & 89.22 ± 0.30 \\
0.95  & 87.59 ± 0.11 & 89.23 ± 0.19 & 89.00 ± 0.05 & 89.31 ± 0.17 & 86.50 ± 0.05 & 88.61 ± 0.28 & 89.50 ± 0.18 & 89.47 ± 0.32 & 87.58 ± 0.25 \\
0.98  & 83.47 ± 0.20 & 85.70 ± 0.33 & 86.43 ± 0.05 & 87.26 ± 0.28 & 85.99 ± 0.08 & 85.61 ± 0.20 & 86.97 ± 0.22 & 87.24 ± 0.32 & 83.40 ± 0.74 \\
0.99  & 78.28 ± 0.45 & 71.99 ± 0.28 & 83.47 ± 0.15 & 84.54 ± 0.04 & 84.56 ± 0.46 & 82.13 ± 0.28 & 83.74 ± 0.48 & 84.85 ± 0.18 & 77.60 ± 1.02 \\
\bottomrule
\end{tabular}}
\end{sc}
\end{small}
\end{center}
\vskip -0.1in
\end{table}

%------------------------------------------------------------------------------------------------
\clearpage
\subsection{VGG19}
\label{appendix:CIFAR10_VGG19}

As discussed earlier, introducing a warm-up phase effectively mitigates layer collapse in data-dependent pruning methods. Here, we evaluate the impact of different warm-up durations by comparing no warm-up, a single warm-up epoch, and five warm-up epochs. Table \ref{tab:VGG19_cifar10_compressors} demonstrates how performance drastically degrades with increasing sparsity, ultimately leading to layer collapse at 0.90 sparsity. However, as shown in the results, a single warm-up epoch is sufficient to prevent collapse and stabilize pruning performance. Moreover, as seen in Table \ref{tab:VGG19_cifar10_compressors_warmup5}, increasing the warm-up period to five epochs provides no substantial additional improvement. This indicates that prolonged warm-up training is not necessary; a single training step is enough to achieve gradient stabilization and overcome layer collapse.

\begin{table}[h]
\caption{Performance of different sensitivity methods for pruning evaluated using VGG19 on the CIFAR-10 test set. The right side of the table presents our proposed criteria. The mean accuracy and standard deviation are reported across three initialization seeds for various sparsity levels. Baseline, no pruning: $89.21 \pm 0.22$.}
\label{tab:VGG19_cifar10_compressors}
\vskip 0.15in
\begin{center}
\begin{small}
\begin{sc}
\resizebox{\textwidth}{!}{%
\begin{tabular}{lccccc|cccc}
\toprule
Sparsity  & Random & Magnitude & GN & SNIP & GraSP & FD & FP & FTS & FBSS \\
\midrule
0.10  & 88.40 ± 0.95 & 89.12 ± 0.55 & 90.14 ± 0.10 & 90.16 ± 0.18 & 87.81 ± 1.66 & 90.20 ± 0.29 & 90.21 ± 0.37 & 90.25 ± 0.38 & 89.06 ± 0.75 \\
0.20  & 89.19 ± 0.22 & 89.65 ± 0.60 & 89.59 ± 0.69 & 90.06 ± 0.04 & 89.57 ± 0.34 & 89.91 ± 0.28 & 90.28 ± 0.55 & 89.80 ± 0.28 & 88.89 ± 0.76 \\
0.30  & 88.93 ± 0.83 & 88.77 ± 1.07 & 90.23 ± 0.09 & 89.88 ± 0.59 & 89.14 ± 0.19 & 90.25 ± 0.09 & 89.97 ± 0.26 & 90.46 ± 0.41 & 89.06 ± 0.36 \\
0.40  & 88.28 ± 1.08 & 89.38 ± 0.53 & 90.50 ± 0.23 & 89.79 ± 0.67 & 88.20 ± 0.31 & 90.51 ± 0.12 & 90.37 ± 0.24 & 90.23 ± 0.14 & 10.00 ± 0.00 \\
0.50  & 88.96 ± 0.82 & 89.03 ± 0.59 & 90.46 ± 0.60 & 90.38 ± 0.25 & 88.67 ± 0.23 & 89.54 ± 0.86 & 90.47 ± 0.52 & 90.19 ± 0.31 & 10.00 ± 0.00 \\
0.60  & 88.15 ± 0.68 & 89.47 ± 0.18 & 89.95 ± 0.30 & 90.32 ± 0.25 & 88.82 ± 0.32 & 90.02 ± 0.40 & 90.18 ± 0.33 & 90.14 ± 0.36 & 10.00 ± 0.00 \\
0.70  & 88.02 ± 0.53 & 89.63 ± 0.44 & 89.69 ± 0.42 & 89.23 ± 0.19 & 89.62 ± 0.81 & 89.85 ± 0.08 & 90.01 ± 0.34 & 10.00 ± 0.00 & 10.00 ± 0.00 \\
0.80  & 88.28 ± 0.34 & 89.62 ± 0.91 & 85.72 ± 0.63 & 89.39 ± 0.43 & 88.82 ± 0.14 & 10.00 ± 0.00 & 88.29 ± 0.11 & 10.00 ± 0.00 & 10.00 ± 0.00 \\
0.90  & 85.82 ± 0.19 & 89.29 ± 0.79 & 10.00 ± 0.00 & 80.85 ± 0.62 & 24.28 ± 20.2 & 10.00 ± 0.00 & 10.00 ± 0.00 & 10.00 ± 0.00 & 10.00 ± 0.00 \\
0.95  & 84.41 ± 0.05 & 10.00 ± 0.00 & 10.00 ± 0.00 & 10.00 ± 0.00 & 10.00 ± 0.00 & 10.00 ± 0.00 & 10.00 ± 0.00 & 10.00 ± 0.00 & 10.00 ± 0.00 \\
0.98  & 80.04 ± 0.90 & 10.00 ± 0.00 & 10.00 ± 0.00 & 10.00 ± 0.00 & 10.00 ± 0.00 & 10.00 ± 0.00 & 10.00 ± 0.00 & 10.00 ± 0.00 & 10.00 ± 0.00 \\
0.99  & 76.89 ± 0.26 & 10.00 ± 0.00 & 10.00 ± 0.00 & 10.00 ± 0.00 & 10.00 ± 0.00 & 10.00 ± 0.00 & 10.00 ± 0.00 & 10.00 ± 0.00 & 10.00 ± 0.00 \\
\bottomrule
\end{tabular}}
\end{sc}
\end{small}
\end{center}
\vskip -0.1in
\end{table}
\newpage
%------------------------------------------------------------------------------------------------
\begin{table*}[h]
\caption{Performance of different compression methods evaluated after 1 warmup epoch using VGG19 on the CIFAR-10 dataset. We report the mean accuracy between three initialization seeds across various sparsity levels. Baseline, no pruning: $89.21 \pm 0.22$.}
\label{tab:VGG19_cifar10_compressors_warmup1}
\vskip 0.15in
\begin{center}
\begin{small}
\begin{sc}
\resizebox{\textwidth}{!}{%
\begin{tabular}{lccccc|cccc}
\toprule
Sparsity  & Random & Magnitude & GN & SNIP & GraSP & FD & FP & FTS & FBSS \\
\midrule
0.80  & 88.73 ± 0.38 & 88.35 ± 0.54 & 86.76 ± 0.27 & 87.39 ± 0.66 & 87.24 ± 0.25 & 87.14 ± 0.45 & 87.00 ± 0.87 & 87.68 ± 0.33 & 64.33 ± 15.91 \\
0.90  & 87.26 ± 0.42 & 88.62 ± 0.49 & 85.96 ± 0.75 & 86.75 ± 0.76 & 87.47 ± 0.33 & 86.69 ± 0.72 & 87.09 ± 0.31 & 87.42 ± 0.21 & 46.16 ± 7.62 \\
0.95  & 85.47 ± 0.64 & 87.68 ± 0.49 & 86.66 ± 0.27 & 86.00 ± 1.10 & 86.71 ± 1.24 & 85.71 ± 1.35 & 86.73 ± 0.36 & 87.56 ± 0.62 & 46.30 ± 5.32 \\
0.98  & 80.44 ± 0.30 & 86.61 ± 0.62 & 84.72 ± 1.69 & 87.22 ± 0.23 & 86.45 ± 0.64 & 80.34 ± 6.43 & 86.07 ± 0.39 & 86.36 ± 0.29 & 49.05 ± 4.31 \\
0.99  & 77.24 ± 0.73 & 83.69 ± 1.36 & 80.28 ± 2.04 & 83.49 ± 1.77 & 85.39 ± 0.43 & 75.11 ± 7.80 & 84.40 ± 1.27 & 85.35 ± 1.05 & 47.10 ± 4.41 \\
\bottomrule
\end{tabular}}
\end{sc}
\end{small}
\end{center}
\vskip -0.1in
\end{table*} 
%------------------------------------------------------------------------------------------------

\begin{table}[h]
\caption{Performance of different sensitivity methods for pruning evaluated after 5 warmup epochs using VGG19 on the CIFAR-10 testset. The right side of the table presents our proposed criteria. The mean accuracy and standard deviation are reported across three initialization seeds for various sparsity levels. Baseline, no pruning: $89.21 \pm 0.22$.}
\label{tab:VGG19_cifar10_compressors_warmup5}
\vskip 0.15in
\begin{center}
\begin{small}
\begin{sc}
\resizebox{\textwidth}{!}{%
\begin{tabular}{lccccc|cccc}
\toprule
Sparsity  & Random & Magnitude & GN & SNIP & GraSP & FD & FP & FTS & FBSS \\
\midrule
0.80  & 88.84 ± 0.43 & 88.41 ± 0.47 & 87.58 ± 0.52 & 88.15 ± 1.09 & 86.77 ± 1.14 & 87.28 ± 0.90 & 88.22 ± 0.82 & 86.68 ± 0.61 & 70.52 ± 9.25 \\
0.90  & 87.56 ± 0.62 & 88.60 ± 0.93 & 86.73 ± 0.37 & 87.89 ± 0.25 & 87.10 ± 0.47 & 87.50 ± 1.42 & 88.18 ± 0.47 & 86.98 ± 0.14 & 47.78 ± 1.26 \\
0.95 & 85.51 ± 0.69 & 87.66 ± 1.19 & 87.44 ± 0.46 & 87.71 ± 0.82 & 87.05 ± 0.16 & 86.83 ± 1.47 & 87.36 ± 0.52 & 87.00 ± 0.74 & 48.83 ± 2.52 \\
0.98 & 82.09 ± 0.17 & 86.24 ± 0.52 & 84.66 ± 1.33 & 86.55 ± 0.84 & 86.04 ± 0.66 & 85.44 ± 0.64 & 86.64 ± 0.13 & 84.89 ± 0.51 & 49.48 ± 0.85 \\
0.99 & 77.22 ± 1.03 & 83.93 ± 1.80 & 81.62 ± 2.17 & 84.53 ± 0.70 & 81.33 ± 5.77 & 81.71 ± 1.41 & 85.02 ± 0.69 & 83.78 ± 0.80 & 41.24 ± 1.55 \\
\bottomrule
\end{tabular}}
\end{sc}
\end{small}
\end{center}
\vskip -0.1in
\end{table}

%------------------------------------------------------------------------------------------------

\newpage
\section{Results CIFAR100}
\subsection{ResNet18}
\label{sec:resnet_cifar-100}

CIFAR-100 results exhibit a similar trend to those observed on CIFAR-10, further reinforcing the robustness of our proposed Fisher-Taylor Sensitivity (FTS) criterion. Across all evaluated sparsity levels, FTS consistently maintains strong performance, frequently ranking among the top-performing methods. This trend is particularly evident at extreme sparsities, where many pruning approaches suffer significant performance degradation. The stability of FTS across both datasets highlights its effectiveness in preserving network expressivity despite aggressive pruning.

\begin{table}[h]
\caption{Performance of different compression methods evaluated using ResNet18 on the CIFAR-100 dataset. We report the mean accuracy between three initialization seeds across various sparsity levels. Baseline, no pruning: $69.57 \pm 0.19$.}
\label{tab:resnet18_cifar100_compressors}
\vskip 0.15in
\begin{center}
\begin{small}
\begin{sc}
\resizebox{\textwidth}{!}{%
\begin{tabular}{lccccc|cccc}
\toprule
Sparsity  & Random & Magnitude & GN & SNIP & GraSP & FD & FP & FTS & FBSS \\
\midrule
0.10  & 69.16 ± 0.11 & 69.37 ± 0.14 & 69.63 ± 0.34 & 69.42 ± 0.07 & 64.26 ± 0.27 & 69.66 ± 0.30 & 69.08 ± 0.21 & 69.16 ± 0.11 & 69.07 ± 0.10 \\
0.20  & 69.16 ± 0.30 & 69.06 ± 0.24 & 69.19 ± 0.11 & 69.30 ± 0.08 & 63.28 ± 0.58 & 69.60 ± 0.30 & 69.35 ± 0.35 & 69.41 ± 0.43 & 69.07 ± 0.20 \\
0.30  & 69.36 ± 0.18 & 68.58 ± 0.36 & 69.37 ± 0.13 & 68.82 ± 0.17 & 62.02 ± 0.43 & 69.24 ± 0.40 & 68.84 ± 0.13 & 68.80 ± 0.55 & 68.96 ± 0.11 \\
0.40  & 69.41 ± 0.20 & 68.50 ± 0.29 & 69.16 ± 0.26 & 68.95 ± 0.19 & 61.18 ± 0.19 & 69.17 ± 0.16 & 68.88 ± 0.25 & 69.02 ± 0.21 & 68.92 ± 0.25 \\
0.50  & 69.12 ± 0.46 & 68.17 ± 0.20 & 68.94 ± 0.20 & 68.63 ± 0.11 & 61.11 ± 0.40 & 69.13 ± 0.13 & 68.68 ± 0.12 & 68.71 ± 0.12 & 68.71 ± 0.57 \\
0.60  & 68.66 ± 0.27 & 67.78 ± 0.35 & 68.77 ± 0.17 & 68.63 ± 0.42 & 61.40 ± 0.78 & 68.34 ± 0.43 & 67.98 ± 0.23 & 68.41 ± 0.14 & 68.60 ± 0.15 \\
0.70  & 67.95 ± 0.43 & 67.51 ± 0.24 & 68.29 ± 0.39 & 68.08 ± 0.18 & 59.43 ± 0.76 & 68.03 ± 0.46 & 67.96 ± 0.15 & 68.29 ± 0.06 & 68.16 ± 0.07 \\
0.80  & 67.26 ± 0.48 & 66.55 ± 0.19 & 67.20 ± 0.37 & 67.21 ± 0.38 & 59.08 ± 0.22 & 66.70 ± 0.05 & 67.05 ± 0.06 & 66.77 ± 0.65 & 66.62 ± 0.43 \\
0.90  & 64.75 ± 0.16 & 64.48 ± 0.18 & 64.87 ± 0.27 & 65.70 ± 0.08 & 59.16 ± 0.91 & 64.74 ± 0.44 & 65.46 ± 0.30 & 65.41 ± 0.13 & 63.90 ± 0.31 \\
0.95  & 61.01 ± 0.32 & 62.20 ± 0.06 & 62.20 ± 0.23 & 63.20 ± 0.20 & 57.91 ± 0.09 & 62.14 ± 0.42 & 63.22 ± 0.25 & 63.21 ± 0.47 & 61.25 ± 0.44 \\
0.98  & 54.72 ± 0.22 & 55.44 ± 0.18 & 57.34 ± 0.31 & 58.83 ± 0.35 & 54.85 ± 0.35 & 55.57 ± 0.17 & 58.05 ± 0.18 & 58.59 ± 0.12 & 55.02 ± 0.34 \\
0.99  & 45.62 ± 0.55 & 40.39 ± 0.36 & 50.46 ± 0.61 & 52.96 ± 0.10 & 49.13 ± 0.19 & 48.02 ± 0.32 & 49.98 ± 0.60 & 52.85 ± 0.24 & 44.91 ± 0.52 \\
\bottomrule
\end{tabular}}
\end{sc}
\end{small}
\end{center}
\vskip -0.1in
\end{table}

%------------------------------------------------------------------------------------------------
\clearpage
\subsection{VGG19}
The results on VGG19 with CIFAR-100 exhibit a similar trend to those observed on CIFAR-10, reinforcing the effectiveness of our proposed approach. Once again, we identify the occurrence of layer collapse at extreme sparsities when no warm-up is applied, leading to a significant drop in accuracy. Introducing a single warm-up epoch effectively resolves this issue, restoring pruning performance across all evaluated criteria. However, increasing the warm-up phase to five epochs does not yield any additional advantage, indicating that a brief warm-up period is sufficient to stabilize gradient-based importance scores and prevent collapse.

\label{sec:vgg_cifar-100}

\begin{table}[h]
\caption{Performance of different compression methods evaluated using VGG19 on the CIFAR-100 dataset. We report the mean accuracy between three initialization seeds across various sparsity levels. Baseline, no pruning: $58.96 \pm 2.30$.}
\label{tab:VGG19_cifar100_compressors}
\vskip 0.15in
\begin{center}
\begin{small}
\begin{sc}
\resizebox{\textwidth}{!}{%
\begin{tabular}{lccccc|cccc}
\toprule
Sparsity & Random & Magnitude & GN & SNIP & GraSP & FD & FP & FTS & FBSS \\
\midrule
0.10  & 60.31 ± 0.40 & 59.13 ± 1.29 & 61.93 ± 0.48 & 61.98 ± 0.29 & 59.32 ± 0.63 & 62.13 ± 0.61 & 60.45 ± 3.47 & 61.56 ± 1.04 & 58.79 ± 0.98 \\
0.20  & 60.43 ± 1.14 & 59.27 ± 0.34 & 62.64 ± 0.21 & 62.68 ± 0.24 & 61.21 ± 0.41 & 63.04 ± 0.43 & 62.71 ± 1.02 & 62.24 ± 0.44 & 60.48 ± 0.48 \\
0.30  & 58.32 ± 0.60 & 59.35 ± 1.43 & 62.61 ± 0.23 & 63.11 ± 0.35 & 59.30 ± 0.43 & 62.85 ± 0.42 & 61.43 ± 0.61 & 62.65 ± 0.54 & 58.77 ± 1.02 \\
0.40  & 56.50 ± 3.20 & 60.04 ± 1.02 & 62.36 ± 0.02 & 62.39 ± 0.55 & 56.34 ± 1.49 & 62.38 ± 0.75 & 61.56 ± 1.25 & 62.67 ± 0.06 & 1.00 ± 0.00 \\
0.50  & 58.47 ± 1.49 & 61.49 ± 1.22 & 62.02 ± 0.64 & 62.76 ± 0.50 & 54.43 ± 0.84 & 62.84 ± 0.33 & 62.25 ± 0.33 & 62.47 ± 0.42 & 1.00 ± 0.00 \\
0.60  & 57.54 ± 0.74 & 61.50 ± 0.30 & 62.55 ± 0.13 & 63.08 ± 0.55 & 56.76 ± 0.69 & 62.40 ± 0.57 & 62.70 ± 0.63 & 62.17 ± 0.23 & 1.00 ± 0.00 \\
0.70  & 57.63 ± 0.80 & 61.71 ± 0.25 & 60.85 ± 0.79 & 60.58 ± 0.39 & 57.76 ± 0.84 & 60.44 ± 0.34 & 60.92 ± 0.41 & 60.51 ± 1.67 & 1.00 ± 0.00 \\
0.80  & 57.84 ± 0.57 & 61.89 ± 1.02 & 55.09 ± 0.49 & 59.84 ± 0.29 & 58.39 ± 0.74 & 1.00 ± 0.00 & 43.16 ± 1.02 & 58.66 ± 2.28 & 1.00 ± 0.00 \\
0.90  & 58.41 ± 0.41 & 62.60 ± 0.91 & 1.00 ± 0.00 & 8.35 ± 10.39 & 42.88 ± 1.64 & 1.00 ± 0.00 & 1.00 ± 0.00 & 8.87 ± 11.13 & 1.00 ± 0.00 \\
0.95  & 54.84 ± 1.08 & 1.00 ± 0.00 & 1.00 ± 0.00 & 1.00 ± 0.00 & 1.00 ± 0.00 & 1.00 ± 0.00 & 1.00 ± 0.00 & 1.00 ± 0.00 & 1.00 ± 0.00 \\
0.98  & 50.21 ± 0.72 & 1.00 ± 0.00 & 1.00 ± 0.00 & 1.00 ± 0.00 & 1.00 ± 0.00 & 1.00 ± 0.00 & 1.00 ± 0.00 & 1.00 ± 0.00 & 1.00 ± 0.00 \\
0.99  & 46.69 ± 0.45 & 1.00 ± 0.00 & 1.00 ± 0.00 & 1.00 ± 0.00 & 1.00 ± 0.00 & 1.00 ± 0.00 & 1.00 ± 0.00 & 1.00 ± 0.00 & 1.00 ± 0.00 \\
\bottomrule
\end{tabular}}
\end{sc}
\end{small}
\end{center}
\vskip -0.1in
\end{table}

%------------------------------------------------------------------------------------------------

\begin{table}[h]
\caption{Performance of different compression methods evaluated after 1 warmup epoch using VGG19 on the CIFAR-100 dataset. We report the mean accuracy between three initialization seeds across various sparsity levels. Baseline, no pruning: $58.96 \pm 2.30$.}
\label{tab:VGG19_cifar100_compressors_warmup1}
\vskip 0.15in
\begin{center}
\begin{small}
\begin{sc}
\resizebox{\textwidth}{!}{%
\begin{tabular}{lccccc|cccc}
\toprule
Sparsity & Random & Magnitude & GN & SNIP & GraSP & FD & FP & FTS & FBSS \\
\midrule
0.80  & 60.39 ± 1.16 & 58.91 ± 0.41 & 52.81 ± 1.32 & 55.62 ± 2.27 & 55.15 ± 2.25 & 56.71 ± 0.31 & 58.03 ± 0.93 & 52.41 ± 3.07 & 52.74 ± 5.16 \\
0.90  & 58.90 ± 0.98 & 60.95 ± 0.81 & 50.56 ± 4.59 & 55.89 ± 2.05 & 56.01 ± 1.58 & 52.07 ± 3.24 & 53.65 ± 0.57 & 52.45 ± 3.75 & 19.65 ± 1.68 \\
0.95  & 56.10 ± 0.85 & 57.64 ± 2.63 & 50.34 ± 1.00 & 53.70 ± 3.60 & 56.16 ± 0.41 & 54.44 ± 1.38 & 53.24 ± 3.54 & 53.56 ± 1.26 & 17.24 ± 0.44 \\
0.98  & 50.97 ± 0.40 & 54.66 ± 2.56 & 43.43 ± 5.32 & 50.19 ± 1.59 & 54.64 ± 1.50 & 42.75 ± 1.91 & 50.59 ± 3.39 & 48.56 ± 5.25 & 16.42 ± 0.64 \\
0.99  & 46.52 ± 0.45 & 43.33 ± 5.83 & 33.90 ± 5.35 & 42.65 ± 5.32 & 45.98 ± 4.48 & 29.67 ± 8.49 & 49.11 ± 3.46 & 48.70 ± 2.59 & 13.25 ± 0.84 \\
\bottomrule
\end{tabular}}
\end{sc}
\end{small}
\end{center}
\vskip -0.1in
\end{table}


%------------------------------------------------------------------------------------------------

\begin{table}[h]
\caption{Performance of different compression methods evaluated after 5 warmup epochs using VGG19 on the CIFAR-100 dataset. We report the mean accuracy between three initialization seeds across various sparsity levels. Baseline, no pruning: $58.96 \pm 2.30$.}
\label{tab:VGG19_cifar100_compressors_warmup5}
\vskip 0.15in
\begin{center}
\begin{small}
\begin{sc}
\resizebox{\textwidth}{!}{%
\begin{tabular}{lccccc|cccc}
\toprule
Sparsity & Random & Magnitude & GN & SNIP & GraSP & FD & FP & FTS & FBSS \\
\midrule
0.80  & 60.41 ± 1.39 & 58.38 ± 0.85 & 60.86 ± 0.79 & 61.63 ± 0.45 & 56.25 ± 0.49 & 59.59 ± 0.76 & 59.37 ± 3.50 & 60.86 ± 0.53 & 46.93 ± 9.04 \\
0.90  & 60.32 ± 0.09 & 57.74 ± 1.64 & 57.77 ± 2.41 & 58.23 ± 4.07 & 56.27 ± 1.02 & 60.19 ± 0.63 & 61.23 ± 0.50 & 60.52 ± 0.37 & 21.66 ± 1.95 \\
0.95 & 57.86 ± 0.53 & 59.55 ± 1.15 & 56.09 ± 0.97 & 58.83 ± 0.65 & 55.26 ± 1.25 & 55.80 ± 2.77 & 59.83 ± 0.94 & 58.52 ± 1.32 & 19.98 ± 2.62 \\
0.98 & 51.75 ± 0.43 & 47.75 ± 7.63 & 52.26 ± 4.06 & 55.27 ± 1.69 & 54.59 ± 0.96 & 49.46 ± 4.98 & 57.40 ± 1.26 & 56.00 ± 1.08 & 17.59 ± 1.36 \\
0.99 & 47.59 ± 0.80 & 42.46 ± 7.95 & 46.58 ± 2.00 & 53.13 ± 0.84 & 53.91 ± 1.53 & 42.87 ± 4.63 & 53.17 ± 1.18 & 53.05 ± 2.14 & 13.92 ± 0.14 \\
\bottomrule
\end{tabular}}
\end{sc}
\end{small}
\end{center}
\vskip -0.1in
\end{table}


%------------------------------------------------------------------------------------------------
\clearpage

\section{Mask Batch Size for Other Sparsities}
The Effect of batch size on pruning performance across different sparsities. 
As sparsity increases, the effect of batch size on pruning performance becomes more pronounced. 
At lower sparsities (0.90, 0.95), the differences across batch sizes are less evident, suggesting that even smaller batches provide a reasonable estimation of parameter importance. However, at extreme sparsities (0.98, 0.99), we observe a clear trend where larger batch sizes consistently lead to better parameter selection, ultimately improving accuracy. This aligns with our hypothesis that larger batches help reduce variance in gradient estimation, leading to more stable and effective pruning decisions. 
\label{batch_size_heatmaps}

\begin{figure}[h]
    \centering
    \includegraphics[width=0.8\linewidth]{imgs/cifar10_resnet18_heatmap_warmup_0.png}
    \caption{Effect of batch size on pruning performance at increasing sparsities.}
    \label{fig:enter-label}
\end{figure}

%------------------------------------------------------------------------------------------------

\clearpage
\section{Comparison of our criteria with magnitude-based pruning}

Figure \ref{fig:our_criterion_vs_magnitude} illustrates the relationship between parameter magnitude and different sensitivity-based pruning metrics. Each point represents a model parameter, with red points indicating the top-ranked parameters selected for retention by each criterion. The green dashed line marks the 99th percentile of parameter magnitudes.

A key observation is that the most effective pruning criteria, such as Fisher-Taylor Sensitivity, tend to retain parameters with a broad range of magnitudes, including many that are relatively small (left of the green line). This shows that the estimated importance does not always prioritize parameters based on their magnitude. 


\begin{figure}[htp]
    \centering
    \includegraphics[width=0.9\linewidth]{imgs/cifar_10_mag_vs_criteria_s_99.png}
    \caption{Our criteria vs. Magnitude parameter selection for 99\% sparsity (ResNet18, CIFAR-10, Seed 0)} 
    \label{fig:our_criterion_vs_magnitude}
\end{figure}

\end{document}
