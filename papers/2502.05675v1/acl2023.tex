% This must be in the first 5 lines to tell arXiv to use pdfLaTeX, which is strongly recommended.
\pdfoutput=1
% In particular, the hyperref package requires pdfLaTeX in order to break URLs across lines.

\documentclass[11pt]{article}

% Change "review" to "final" to generate the final (sometimes called camera-ready) version.
% Change to "preprint" to generate a non-anonymous version with page numbers.
\usepackage[final]{acl2023}
% review
% Standard package includes
\usepackage{times}
\usepackage{latexsym}
\usepackage{multirow}
\usepackage{cite} % Optional, for improved citation handling
% For proper rendering and hyphenation of words containing Latin characters (including in bib files)
\usepackage[T1]{fontenc}
% For Vietnamese characters
% \usepackage[T5]{fontenc}
% See https://www.latex-project.org/help/documentation/encguide.pdf for other character sets

% This assumes your files are encoded as UTF8
\usepackage[utf8]{inputenc}

% This is not strictly necessary, and may be commented out,
% but it will improve the layout of the manuscript,
% and will typically save some space.
\usepackage{microtype}

% This is also not strictly necessary, and may be commented out.
% However, it will improve the aesthetics of text in
% the typewriter font.
\usepackage{inconsolata}
\usepackage{adjustbox}
\usepackage{booktabs}
\usepackage{longtable}
\usepackage{tcolorbox}
\usepackage{stackengine}
\usepackage{multirow}
\usepackage{graphicx}
\usepackage{wrapfig}
\usepackage{hyperref}       % hyperlinks
\usepackage{url}            % simple URL typesetting
\usepackage{amsfonts}       % blackboard math symbols
\usepackage{nicefrac}       % compact symbols for 1/2, etc.
\usepackage{microtype}      % microtypography
\usepackage{xcolor,colortbl}         % colors
\usepackage{times}
\usepackage{latexsym}
\usepackage{multicol}
\usepackage{blindtext}
\usepackage{tabu}
\usepackage{amsmath, bm}
\newcommand{\sub}{\textsubscript}
\renewcommand{\UrlFont}{\ttfamily\small}
\usepackage{subcaption}
\usepackage{caption}
\usepackage[normalem]{ulem}
\usepackage{soul}
\usepackage[shortlabels]{enumitem}
\usepackage{array}
\usepackage{pgffor}
\usepackage{textcomp}
\usepackage{amssymb}
\usepackage{pifont}
\usepackage{booktabs}
\usepackage{tabularx}
\usepackage{color}
\usepackage{xcolor}
\usepackage{mwe}
\usepackage{float}
\newcommand{\xmark}{\ding{55}}
\definecolor{lightgreen}{rgb}{0.8, 0.95, 0.8}
\definecolor{lightred}{rgb}{0.95, 0.8, 0.8}
\definecolor{naplesyellow}{rgb}{0.98, 0.85, 0.37}
\definecolor{pastelyellow}{rgb}{0.99, 0.99, 0.59}

% \newcommand{\Venkatesh}[1]{\textbf{\small {\color{blue}[#1 -Venkatesh]}}}
\newcommand{\mihir}[1]{\textcolor{cyan}{[Mihir: #1]}}

% If the title and author information does not fit in the area allocated, uncomment the following
%
%\setlength\titlebox{<dim>}
%
% and set <dim> to something 5cm or larger.

% \title{Investigating the Shortcomings of LLMs in Step-by-Step Reasoning using Context-Rich Legal Scenarios}
\title{Investigating the Shortcomings of LLMs in Step-by-Step Legal Reasoning}

% \title{Investigating the Shortcomings of Large Language Models in Step-by-Step Reasoning using Context-Rich Legal Scenarios}

% Author information can be set in various styles:
% For several authors from the same institution:
% \author{Author 1 \and ... \and Author n \\
%         Address line \\ ... \\ Address line}
% if the names do not fit well on one line use
%         Author 1 \\ {\bf Author 2} \\ ... \\ {\bf Author n} \\
% For authors from different institutions:
% \author{Author 1 \\ Address line \\  ... \\ Address line
%         \And  ... \And
%         Author n \\ Address line \\ ... \\ Address line}
% To start a separate ``row'' of authors use \AND, as in
% \author{Author 1 \\ Address line \\  ... \\ Address line
%         \AND
%         Author 2 \\ Address line \\ ... \\ Address line \And
%         Author 3 \\ Address line \\ ... \\ Address line}

% \author{First Author \\
%   Affiliation / Address line 1 \\
%   Affiliation / Address line 2 \\
%   Affiliation / Address line 3 \\
%   \texttt{email@domain} \\\And
%   Second Author \\
%   Affiliation / Address line 1 \\
%   Affiliation / Address line 2 \\
%   Affiliation / Address line 3 \\
%   \texttt{email@domain} \\}

\author{Venkatesh Mishra$^{1*}$ \quad Bimsara Pathiraja$^1$\thanks{\ \ Equal Contribution} \quad Mihir Parmar$^1$ \quad Sat Chidananda$^1$  \\ \textbf{Jayanth Srinivasa}$^2$\quad  \textbf{Gaowen Liu}$^2$ \quad \textbf{Ali Payani}$^2$ \quad \textbf{Chitta Baral}$^1$ \\\\ 
$^1$Arizona State University \quad $^2$Cisco Research\\
\small{\texttt{\{vmishr23, bpathir1, chitta\}@asu.edu}}
}
\begin{document}
\maketitle
\begin{abstract}

Reasoning abilities of LLMs have been a key focus in recent years. One challenging reasoning domain with interesting nuances is legal reasoning, which requires careful application of rules, and precedents while balancing deductive and analogical reasoning, and conflicts between rules.  Although there have been a few works on using LLMs for legal reasoning, their focus has been on overall accuracy. In this paper, we dig deeper to do a step-by-step analysis and figure out where they commit errors. We use the college-level Multiple Choice Question-Answering (MCQA) task from the \textit{Civil Procedure} dataset and propose a new error taxonomy derived from initial manual analysis of reasoning chains with respect to several LLMs, including two objective measures: soundness and correctness scores. We then develop an LLM-based automated evaluation framework to identify reasoning errors and evaluate the performance of LLMs. The computation of soundness and correctness on the dataset using the auto-evaluator framework reveals several interesting insights. Furthermore, we show that incorporating the error taxonomy as feedback in popular prompting techniques marginally increases LLM performance. Our work will also serve as an evaluation framework that can be used in detailed error analysis of reasoning chains for logic-intensive complex tasks\footnote{Data and source code are available at \url{https://github.com/VenkyMishra/legal_reasoning}}.


%With the advent of Large Language Models (LLMs) and their adoption in the legal domain, it is important to analyze their legal reasoning abilities. Since legal reasoning requires these models to reason logically, evaluating it in LLMs also provides valuable insight into their logical reasoning skills. To the best of authors' knowledge, most existing work evaluating LLMs' legal reasoning primarily focus on their ability to predict the correct final answer, and there have not been much attention paid to analyzing its step-by-step reasoning process and where they falter. Motivated by this, we systematically evaluate the reasoning processes of LLMs on college-level Multiple Choice Question-Answering (MCQA) task from the \textit{Civil Procedure} dataset. We propose a new error taxonomy derived from initial manual analysis of reasoning chains from LLMs including Mistral-7B, Llama-3-8B, GPT-3.5, GPT-4-Turbo, and GPT-4o. Leveraging this taxonomy, we develop an LLM-based automated evaluation framework to identify reasoning errors, akin to human error identification. Additionally, we evaluate the performance of LLMs using two objective measures, soundness and correctness scores. Furthermore, we propose that incorporating the error taxonomy as feedback in existing prompting methods enhances LLM reasoning abilities and improves their final performance. We believe that this work provides evaluation framework for future exploration of detailed error analysis of reasoning chains for more logic-intensive complex tasks\footnote{Data and source code are available at <anonymous link>}.     

% Legal reasoning requires highly critical multi-step analytical and logical reasoning while considering the subtle differences and nuances of different legal scenarios. Evaluating the legal reasoning capabilities of large language models (LLMs) can serve as a valuable measure of its overall reasoning abilities. While almost all of the existing work on legal reasoning evaluate the performance of LLMs based on their final answer, in-depth analysis of the most common errors made by LLMs in their intermediate reasoning steps is crucial to improve the reasoning abilities of LLMs. Thus, in this work, we critically analyze the reasoning capabilities of various LLMs on a context-rich multiple choice question answering dataset containing 175 college-level law problems from Civil Procedure encompassed in the US federal and state laws. The study aims to assess the correctness of these models' reasoning processes, highlighting their strengths and limitations in scenarios where they have access to all the necessary knowledge. Based on meticulous human evaluation, we have created an error taxonomy of the common errors committed by five LLMs (Mistral-7B-v2-instruct, Llama-3-8B-instruct, GPT-3.5-turbo, GPT-4-turbo, and GPT-4o) while generating the reasoning chains. Our proposed automated evaluator for reasoning step error identification helps scaling up the evaluation on the entire dataset. The prompt-based error mitigation strategies which leverage the proposed error taxonomy \textit{outperform}\footnote{Mitigation of errors investigation currently in progress} prompting methods which do not incorporate this framework. 
% We have proposed various mitigation strategies to tackle and prevent fallacies of these X categories from occurring during the logical reasoning process. These strategies demonstrate that LLMs can improve their reasoning abilities, even on challenging datasets such as LRBENCH and tasks beyond legal reasoning.

%Now, we utilize these manually annotated samples based on error categories as few-shot exemplars for GPT-4 and explore its capabilities to evaluate reasoning chains. On the puzzle-solving task, we show that our proposed method easily identifies various error types. Based on various error categories, we propose to utilize some base prompting methods to improve the performance of LLMs on puzzle-solving. We believe that our framework for error categorization can be useful to many reasoning tasks in improving performance beyond puzzle solving.

%annotate different errors in reasoning chains for puzzle-solving task. Based on that, we propose a guideline to 
\end{abstract}

\section{Introduction}
\label{section:introduction}

% redirection is unique and important in VR
Virtual Reality (VR) systems enable users to embody virtual avatars by mirroring their physical movements and aligning their perspective with virtual avatars' in real time. 
As the head-mounted displays (HMDs) block direct visual access to the physical world, users primarily rely on visual feedback from the virtual environment and integrate it with proprioceptive cues to control the avatar’s movements and interact within the VR space.
Since human perception is heavily influenced by visual input~\cite{gibson1933adaptation}, 
VR systems have the unique capability to control users' perception of the virtual environment and avatars by manipulating the visual information presented to them.
Leveraging this, various redirection techniques have been proposed to enable novel VR interactions, 
such as redirecting users' walking paths~\cite{razzaque2005redirected, suma2012impossible, steinicke2009estimation},
modifying reaching movements~\cite{gonzalez2022model, azmandian2016haptic, cheng2017sparse, feick2021visuo},
and conveying haptic information through visual feedback to create pseudo-haptic effects~\cite{samad2019pseudo, dominjon2005influence, lecuyer2009simulating}.
Such redirection techniques enable these interactions by manipulating the alignment between users' physical movements and their virtual avatar's actions.

% % what is hand/arm redirection, motivation of study arm-offset
% \change{\yj{i don't understand the purpose of this paragraph}
% These illusion-based techniques provide users with unique experiences in virtual environments that differ from the physical world yet maintain an immersive experience. 
% A key example is hand redirection, which shifts the virtual hand’s position away from the real hand as the user moves to enhance ergonomics during interaction~\cite{feuchtner2018ownershift, wentzel2020improving} and improve interaction performance~\cite{montano2017erg, poupyrev1996go}. 
% To increase the realism of virtual movements and strengthen the user’s sense of embodiment, hand redirection techniques often incorporate a complete virtual arm or full body alongside the redirected virtual hand, using inverse kinematics~\cite{hartfill2021analysis, ponton2024stretch} or adjustments to the virtual arm's movement as well~\cite{li2022modeling, feick2024impact}.
% }

% noticeability, motivation of predicting a probability, not a classification
However, these redirection techniques are most effective when the manipulation remains undetected~\cite{gonzalez2017model, li2022modeling}. 
If the redirection becomes too large, the user may not mitigate the conflict between the visual sensory input (redirected virtual movement) and their proprioception (actual physical movement), potentially leading to a loss of embodiment with the virtual avatar and making it difficult for the user to accurately control virtual movements to complete interaction tasks~\cite{li2022modeling, wentzel2020improving, feuchtner2018ownershift}. 
While proprioception is not absolute, users only have a general sense of their physical movements and the likelihood that they notice the redirection is probabilistic. 
This probability of detecting the redirection is referred to as \textbf{noticeability}~\cite{li2022modeling, zenner2024beyond, zenner2023detectability} and is typically estimated based on the frequency with which users detect the manipulation across multiple trials.

% version B
% Prior research has explored factors influencing the noticeability of redirected motion, including the redirection's magnitude~\cite{wentzel2020improving, poupyrev1996go}, direction~\cite{li2022modeling, feuchtner2018ownershift}, and the visual characteristics of the virtual avatar~\cite{ogawa2020effect, feick2024impact}.
% While these factors focus on the avatars, the surrounding virtual environment can also influence the users' behavior and in turn affect the noticeability of redirection.
% One such prominent external influence is through the visual channel - the users' visual attention is constantly distracted by complex visual effects and events in practical VR scenarios.
% Although some prior studies have explored how to leverage user blindness caused by visual distractions to redirect users' virtual hand~\cite{zenner2023detectability}, there remains a gap in understanding how to quantify the noticeability of redirection under visual distractions.

% visual stimuli and gaze behavior
Prior research has explored factors influencing the noticeability of redirected motion, including the redirection's magnitude~\cite{wentzel2020improving, poupyrev1996go}, direction~\cite{li2022modeling, feuchtner2018ownershift}, and the visual characteristics of the virtual avatar~\cite{ogawa2020effect, feick2024impact}.
While these factors focus on the avatars, the surrounding virtual environment can also influence the users' behavior and in turn affect the noticeability of redirection.
This, however, remains underexplored.
One such prominent external influence is through the visual channel - the users' visual attention is constantly distracted by complex visual effects and events in practical VR scenarios.
We thus want to investigate how \textbf{visual stimuli in the virtual environment} affect the noticeability of redirection.
With this, we hope to complement existing works that focus on avatars by incorporating environmental visual influences to enable more accurate control over the noticeability of redirected motions in practical VR scenarios.
% However, in realistic VR applications, the virtual environment often contains complex visual effects beyond the virtual avatar itself. 
% We argue that these visual effects can \textbf{distract users’ visual attention and thus affect the noticeability of redirection offsets}, while current research has yet taken into account.
% For instance, in a VR boxing scenario, a user’s visual attention is likely focused on their opponent rather than on their virtual body, leading to a lower noticeability of redirection offsets on their virtual movements. 
% Conversely, when reaching for an object in the center of their field of view, the user’s attention is more concentrated on the virtual hand’s movement and position to ensure successful interaction, resulting in a higher noticeability of offsets.

Since each visual event is a complex choreography of many underlying factors (type of visual effect, location, duration, etc.), it is extremely difficult to quantify or parameterize visual stimuli.
Furthermore, individuals respond differently to even the same visual events.
Prior neuroscience studies revealed that factors like age, gender, and personality can influence how quickly someone reacts to visual events~\cite{gillon2024responses, gale1997human}. 
Therefore, aiming to model visual stimuli in a way that is generalizable and applicable to different stimuli and users, we propose to use users' \textbf{gaze behavior} as an indicator of how they respond to visual stimuli.
In this paper, we used various gaze behaviors, including gaze location, saccades~\cite{krejtz2018eye}, fixations~\cite{perkhofer2019using}, and the Index of Pupil Activity (IPA)~\cite{duchowski2018index}.
These behaviors indicate both where users are looking and their cognitive activity, as looking at something does not necessarily mean they are attending to it.
Our goal is to investigate how these gaze behaviors stimulated by various visual stimuli relate to the noticeability of redirection.
With this, we contribute a model that allows designers and content creators to adjust the redirection in real-time responding to dynamic visual events in VR.

To achieve this, we conducted user studies to collect users' noticeability of redirection under various visual stimuli.
To simulate realistic VR scenarios, we adopted a dual-task design in which the participants performed redirected movements while monitoring the visual stimuli.
Specifically, participants' primary task was to report if they noticed an offset between the avatar's movement and their own, while their secondary task was to monitor and report the visual stimuli.
As realistic virtual environments often contain complex visual effects, we started with simple and controlled visual stimulus to manage the influencing factors.

% first user study, confirmation study
% collect data under no visual stimuli, different basic visual stimuli
We first conducted a confirmation study (N=16) to test whether applying visual stimuli (opacity-based) actually affects their noticeability of redirection. 
The results showed that participants were significantly less likely to detect the redirection when visual stimuli was presented $(F_{(1,15)}=5.90,~p=0.03)$.
Furthermore, by analyzing the collected gaze data, results revealed a correlation between the proposed gaze behaviors and the noticeability results $(r=-0.43)$, confirming that the gaze behaviors could be leveraged to compute the noticeability.

% data collection study
We then conducted a data collection study to obtain more accurate noticeability results through repeated measurements to better model the relationship between visual stimuli-triggered gaze behaviors and noticeability of redirection.
With the collected data, we analyzed various numerical features from the gaze behaviors to identify the most effective ones. 
We tested combinations of these features to determine the most effective one for predicting noticeability under visual stimuli.
Using the selected features, our regression model achieved a mean squared error (MSE) of 0.011 through leave-one-user-out cross-validation. 
Furthermore, we developed both a binary and a three-class classification model to categorize noticeability, which achieved an accuracy of 91.74\% and 85.62\%, respectively.

% evaluation study
To evaluate the generalizability of the regression model, we conducted an evaluation study (N=24) to test whether the model could accurately predict noticeability with new visual stimuli (color- and scale-based animations).
Specifically, we evaluated whether the model's predictions aligned with participants' responses under these unseen stimuli.
The results showed that our model accurately estimated the noticeability, achieving mean squared errors (MSE) of 0.014 and 0.012 for the color- and scale-based visual stimili, respectively, compared to participants' responses.
Since the tested visual stimuli data were not included in the training, the results suggested that the extracted gaze behavior features capture a generalizable pattern and can effectively indicate the corresponding impact on the noticeability of redirection.

% application
Based on our model, we implemented an adaptive redirection technique and demonstrated it through two applications: adaptive VR action game and opportunistic rendering.
We conducted a proof-of-concept user study (N=8) to compare our adaptive redirection technique with a static redirection, evaluating the usability and benefits of our adaptive redirection technique.
The results indicated that participants experienced less physical demand and stronger sense of embodiment and agency when using the adaptive redirection technique. 
These results demonstrated the effectiveness and usability of our model.

In summary, we make the following contributions.
% 
\begin{itemize}
    \item 
    We propose to use users' gaze behavior as a medium to quantify how visual stimuli influences the noticebility of redirection. 
    Through two user studies, we confirm that visual stimuli significantly influences noticeability and identify key gaze behavior features that are closely related to this impact.
    \item 
    We build a regression model that takes the user's gaze behavioral data as input, then computes the noticeability of redirection.
    Through an evaluation study, we verify that our model can estimate the noticeability with new participants under unseen visual stimuli.
    These findings suggest that the extracted gaze behavior features effectively capture the influence of visual stimuli on noticeability and can generalize across different users and visual stimuli.
    \item 
    We develop an adaptive redirection technique based on our regression model and implement two applications with it.
    With a proof-of-concept study, we demonstrate the effectiveness and potential usability of our regression model on real-world use cases.

\end{itemize}

% \delete{
% Virtual Reality (VR) allows the user to embody a virtual avatar by mirroring their physical movements through the avatar.
% As the user's visual access to the physical world is blocked in tasks involving motion control, they heavily rely on the visual representation of the avatar's motions to guide their proprioception.
% Similar to real-world experiences, the user is able to resolve conflicts between different sensory inputs (e.g., vision and motor control) through multisensory integration, which is essential for mitigating the sensory noise that commonly arises.
% However, it also enables unique manipulations in VR, as the system can intentionally modify the avatar's movements in relation to the user's motions to achieve specific functional outcomes,
% for example, 
% % the manipulations on the avatar's movements can 
% enabling novel interaction techniques of redirected walking~\cite{razzaque2005redirected}, redirected reaching~\cite{gonzalez2022model}, and pseudo haptics~\cite{samad2019pseudo}.
% With small adjustments to the avatar's movements, the user can maintain their sense of embodiment, due to their ability to resolve the perceptual differences.
% % However, a large mismatch between the user and avatar's movements can result in the user losing their sense of embodiment, due to an inability to resolve the perceptual differences.
% }

% \delete{
% However, multisensory integration can break when the manipulation is so intense that the user is aware of the existence of the motion offset and no longer maintains the sense of embodiment.
% Prior research studied the intensity threshold of the offset applied on the avatar's hand, beyond which the embodiment will break~\cite{li2022modeling}. 
% Studies also investigated the user's sensitivity to the offsets over time~\cite{kohm2022sensitivity}.
% Based on the findings, we argue that one crucial factor that affects to what extent the user notices the offset (i.e., \textit{noticeability}) that remains under-explored is whether the user directs their visual attention towards or away from the virtual avatar.
% Related work (e.g., Mise-unseen~\cite{marwecki2019mise}) has showcased applications where adjustments in the environment can be made in an unnoticeable manner when they happen in the area out of the user's visual field.
% We hypothesize that directing the user's visual attention away from the avatar's body, while still partially keeping the avatar within the user's field-of-view, can reduce the noticeability of the offset.
% Therefore, we conduct two user studies and implement a regression model to systematically investigate this effect.
% }

% \delete{
% In the first user study (N = 16), we test whether drawing the user's visual attention away from their body impacts the possibility of them noticing an offset that we apply to their arm motion in VR.
% We adopt a dual-task design to enable the alteration of the user's visual attention and a yes/no paradigm to measure the noticeability of motion offset. 
% The primary task for the user is to perform an arm motion and report when they perceive an offset between the avatar's virtual arm and their real arm.
% In the secondary task, we randomly render a visual animation of a ball turning from transparent to red and becoming transparent again and ask them to monitor and report when it appears.
% We control the strength of the visual stimuli by changing the duration and location of the animation.
% % By changing the time duration and location of the visual animation, we control the strengths of attraction to the users.
% As a result, we found significant differences in the noticeability of the offsets $(F_{(1,15)}=5.90,~p=0.03)$ between conditions with and without visual stimuli.
% Based on further analysis, we also identified the behavioral patterns of the user's gaze (including pupil dilation, fixations, and saccades) to be correlated with the noticeability results $(r=-0.43)$ and they may potentially serve as indicators of noticeability.
% }

% \delete{
% To further investigate how visual attention influences the noticeability, we conduct a data collection study (N = 12) and build a regression model based on the data.
% The regression model is able to calculate the noticeability of the offset applied on the user's arm under various visual stimuli based on their gaze behaviors.
% Our leave-one-out cross-validation results show that the proposed method was able to achieve a mean-squared error (MSE) of 0.012 in the probability regression task.
% }

% \delete{
% To verify the feasibility and extendability of the regression model, we conduct an evaluation study where we test new visual animations based on adjustments on scale and color and invite 24 new participants to attend the study.
% Results show that the proposed method can accurately estimate the noticeability with an MSE of 0.014 and 0.012 in the conditions of the color- and scale-based visual effects.
% Since these animations were not included in the dataset that the regression model was built on, the study demonstrates that the gaze behavioral features we extracted from the data capture a generalizable pattern of the user's visual attention and can indicate the corresponding impact on the noticeability of the offset.
% }

% \delete{
% Finally, we demonstrate applications that can benefit from the noticeability prediction model, including adaptive motion offsets and opportunistic rendering, considering the user's visual attention. 
% We conclude with discussions of our work's limitations and future research directions.
% }

% \delete{
% In summary, we make the following contributions.
% }
% % 
% \begin{itemize}
%     \item 
%     \delete{
%     We quantify the effects of the user's visual attention directed away by stimuli on their noticeability of an offset applied to the avatar's arm motion with respect to the user's physical arm. 
%     Through two user studies, we identified gaze behavioral features that are indicative of the changes in noticeability.
%     }
%     \item 
%     \delete{We build a regression model that takes the user's gaze behavioral data and the offset applied to the arm motion as input, then computes the probability of the user noticing the offset.
%     Through an evaluation study, we verified that the model needs no information about the source attracting the user's visual attention and can be generalizable in different scenarios.
%     }
%     \item 
%     \delete{We demonstrate two applications that potentially benefit from the regression model, including adaptive motion offsets and opportunistic rendering.
%     }

% \end{itemize}

\begin{comment}
However, users will lose the sense of embodiment to the virtual avatars if they notice the offset between the virtual and physical movements.
To address this, researchers have been exploring the noticing threshold of offsets with various magnitudes and proposing various redirection techniques that maintain the sense of embodiment~\cite{}.

However, when users embody virtual avatars to explore virtual environments, they encounter various visual effects and content that can attract their attention~\cite{}.
During this, the user may notice an offset when he observes the virtual movement carefully while ignoring it when the virtual contents attract his attention from the movements.
Therefore, static offset thresholds are not appropriate in dynamic scenarios.

Past research has proposed dynamic mapping techniques that adapted to users' state, such as hand moving speed~\cite{frees2007prism} or ergonomically comfortable poses~\cite{montano2017erg}, but not considering the influence of virtual content.
More specifically, PRISM~\cite{frees2007prism} proposed adjusting the C/D ratio with a non-linear mapping according to users' hand moving speed, but it might not be optimal for various virtual scenarios.
While Erg-O~\cite{montano2017erg} redirected users' virtual hands according to the virtual target's relative position to reduce physical fatigue, neglecting the change of virtual environments. 

Therefore, how to design redirection techniques in various scenarios with different visual attractions remains unknown.
To address this, we investigate how visual attention affects the noticing probability of movement offsets.
Based on our experiments, we implement a computational model that automatically computes the noticing probability of offsets under certain visual attractions.
VR application designers and developers can easily leverage our model to design redirection techniques maintaining the sense of embodiment adapt to the user's visual attention.
We implement a dynamic redirection technique with our model and demonstrate that it effectively reduces the target reaching time without reducing the sense of embodiment compared to static redirection techniques.

% Need to be refined
This paper offers the following contributions.
\begin{itemize}
    \item We investigate how visual attractions affect the noticing probability of redirection offsets.
    \item We construct a computational model to predict the noticing probability of an offset with a given visual background.
    \item We implement a dynamic redirection technique adapting to the visual background. We evaluate the technique and develop three applications to demonstrate the benefits. 
\end{itemize}



First, we conducted a controlled experiment to understand how users perceived the movement offset while subjected to various distractions.
Since hand redirection is one of the most frequently used redirections in VR interactions, we focused on the dynamic arm movements and manually added angular offsets to the' elbow joint~\cite{li2022modeling, gonzalez2022model, zenner2019estimating}. 
We employed flashing spheres in the user's field of view as distractions to attract users' visual attention.
Participants were instructed to report the appearing location of the spheres while simultaneously performing the arm movements and reporting if they perceived an offset during the movement. 
(\zhipeng{Add the results of data collection. Analyze the influence of the distance between the gaze map and the offset.}
We measured the visual attraction's magnitude with the gaze distribution on it.
Results showed that stronger distractions made it harder for users to notice the offset.)
\zhipeng{Need to rewrite. Not sure to use gaze distribution or a metric obtained from the visual content.}
Secondly, we constructed a computational model to predict the noticing probability of offsets with given visual content.
We analyzed the data from the user studies to measure the influence of visual attractions on the noticing probability of offsets.
We built a statistical model to predict the offset's noticing probability with a given visual content.
Based on the model, we implement a dynamic redirection technique to adjust the redirection offset adapted to the user's current field of view.
We evaluated the technique in a target selection task compared to no hand redirection and static hand redirection.
\zhipeng{Add the results of the evaluation.}
Results showed that the dynamic hand redirection technique significantly reduced the target selection time with similar accuracy and a comparable sense of embodiment.
Finally, we implemented three applications to demonstrate the potential benefits of the visual attention adapted dynamic redirection technique.
\end{comment}

% This one modifies arm length, not redirection
% \citeauthor{mcintosh2020iteratively} proposed an adaptation method to iteratively change the virtual avatar arm's length based on the primary tasks' performance~\cite{mcintosh2020iteratively}.



% \zhipeng{TO ADD: what is redirection}
% Redirection enables novel interactions in Virtual Reality, including redirected walking, haptic redirection, and pseudo haptics by introducing an offset to users' movement.
% \zhipeng{TO ADD: extend this sentence}
% The price of this is that users' immersiveness and embodiment in VR can be compromised when they notice the offset and perceive the virtual movement not as theirs~\cite{}.
% \zhipeng{TO ADD: extend this sentence, elaborate how the virtual environment attracts users' attention}
% Meanwhile, the visual content in the virtual environment is abundant and consistently captures users' attention, making it harder to notice the offset~\cite{}.
% While previous studies explored the noticing threshold of the offsets and optimized the redirection techniques to maintain the sense of embodiment~\cite{}, the influence of visual content on the probability of perceiving offsets remains unknown.  
% Therefore, we propose to investigate how users perceive the redirection offset when they are facing various visual attractions.


% We conducted a user study to understand how users notice the shift with visual attractions.
% We used a color-changing ball to attract the user's attention while instructing users to perform different poses with their arms and observe it meanwhile.
% \zhipeng{(Which one should be the primary task? Observe the ball should be the primary one, but if the primary task is too simple, users might allocate more attention on the secondary task and this makes the secondary task primary.)}
% \zhipeng{(We need a good and reasonable dual-task design in which users care about both their pose and the visual content, at least in the evaluation study. And we need to be able to control the visual content's magnitude and saliency maybe?)}
% We controlled the shift magnitude and direction, the user's pose, the ball's size, and the color range.
% We set the ball's color-changing interval as the independent factor.
% We collect the user's response to each shift and the color-changing times.
% Based on the collected data, we constructed a statistical model to describe the influence of visual attraction on the noticing probability.
% \zhipeng{(Are we actually controlling the attention allocation? How do we measure the attracting effect? We need uniform metrics, otherwise it is also hard for others to use our knowledge.)}
% \zhipeng{(Try to use eye gaze? The eye gaze distribution in the last five seconds to decide the attention allocation? Basically constructing a model with eye gaze distribution and noticing probability. But the user's head is moving, so the eye gaze distribution is not aligned well with the current view.)}

% \zhipeng{Saliency and EMD}
% \zhipeng{Gaze is more than just a point: Rethinking visual attention
% analysis using peripheral vision-based gaze mapping}

% Evaluation study(ideal case): based on the visual content, adjusting the redirection magnitude dynamically.

% \zhipeng{(The risk is our model's effect is trivial.)}

% Applications:
% Playing Lego while watching demo videos, we can accelerate the reaching process of bricks, and forbid the redirection during the manipulation.

% Beat saber again: but not make a lot of sense? Difficult game has complicated visual effects, while allows larger shift, but do not need large shift with high difficulty




\section{Related Work}
\label{lit_review}

\begin{highlight}
{

Our research builds upon {\em (i)} Assessing Web Accessibility, {\em (ii)} End-User Accessibility Repair, and {\em (iii)} Developer Tools for Accessibility.

\subsection{Assessing Web Accessibility}
From the earliest attempts to set standards and guidelines, web accessibility has been shaped by a complex interplay of technical challenges, legal imperatives, and educational campaigns. Over the past 25 years, stakeholders have sought to improve digital inclusion by establishing foundational standards~\cite{chisholm2001web, caldwell2008web}, enforcing legal obligations~\cite{sierkowski2002achieving, yesilada2012understanding}, and promoting a broader culture of accessibility awareness among developers~\cite{sloan2006contextual, martin2022landscape, pandey2023blending}. 
Despite these longstanding efforts, systemic accessibility issues persist. According to the 2024 WebAIM Million report~\cite{webaim2024}, 95.9\% of the top one million home pages contained detectable WCAG violations, averaging nearly 57 errors per page. 
These errors take many forms: low color contrast makes the interface difficult for individuals with color deficiency or low vision to read text; missing alternative text leaves users relying on screen readers without crucial visual context; and unlabeled form inputs or empty links and buttons hinder people who navigate with assistive technologies from completing basic tasks. 
Together, these accessibility issues not only limit user access to critical online resources such as healthcare, education, and employment but also result in significant legal risks and lost opportunities for businesses to engage diverse audiences. Addressing these pervasive issues requires systematic methods to identify, measure, and prioritize accessibility barriers, which is the first step toward achieving meaningful improvements.

Prior research has introduced methods blending automation and human evaluation to assess web accessibility. Hybrid approaches like SAMBA combine automated tools with expert reviews to measure the severity and impact of barriers, enhancing evaluation reliability~\cite{brajnik2007samba}. Quantitative metrics, such as Failure Rate and Unified Web Evaluation Methodology, support large-scale monitoring and comparative analysis, enabling cost-effective insights~\cite{vigo2007quantitative, martins2024large}. However, automated tools alone often detect less than half of WCAG violations and generate false positives, emphasizing the need for human interpretation~\cite{freire2008evaluation, vigo2013benchmarking}. Recent progress with large pretrained models like Large Language Models (LLMs)~\cite{dubey2024llama,bai2023qwen} and Large Multimodal Models (LMMs)~\cite{liu2024visual, bai2023qwenvl} offers a promising step forward, automating complex checks like non-text content evaluation and link purposes, achieving higher detection rates than traditional tools~\cite{lopez2024turning, delnevo2024interaction}. Yet, these large models face challenges, including dependence on training data, limited contextual judgment, and the inability to simulate real user experiences. These limitations underscore the necessity of combining models with human oversight for reliable, user-centered evaluations~\cite{brajnik2007samba, vigo2013benchmarking, delnevo2024interaction}. 

Our work builds on these prior efforts and recent advancements by leveraging the capabilities of large pretrained models while addressing their limitations through a developer-centric approach. CodeA11y integrates LLM-powered accessibility assessments, tailored accessibility-aware system prompts, and a dedicated accessibility checker directly into GitHub Copilot---one of the most widely used coding assistants. Unlike standalone evaluation tools, CodeA11y actively supports developers throughout the coding process by reinforcing accessibility best practices, prompting critical manual validations, and embedding accessibility considerations into existing workflows.
% This pervasive shortfall reflects the difficulty of scaling traditional approaches---such as manual audits and automated tools---that either demand immense human effort or lack the nuanced understanding needed to capture real-world user experiences. 
%
% In response, a new wave of AI-driven methods, many powered by large language models (LLMs), is emerging to bridge these accessibility detection and assessment gaps. Early explorations, such as those by Morillo et al.~\cite{morillo2020system}, introduced AI-assisted recommendations capable of automatic corrections, illustrating how computational intelligence can tackle the repetitive, common errors that plague large swaths of the web. Building on this foundation, Huang et al.~\cite{huang2024access} proposed ACCESS, a prompt-engineering framework that streamlines the identification and remediation of accessibility violations, while López-Gil et al.~\cite{lopez2024turning} demonstrated how LLMs can help apply WCAG success criteria more consistently---reducing the reliance on manual effort. Beyond these direct interventions, recent work has also begun integrating user experiences more seamlessly into the evaluation process. For example, Huq et al.~\cite{huq2024automated} translate user transcripts and corresponding issues into actionable test reports, ensuring that accessibility improvements align more closely with authentic user needs.
% However, as these AI-driven solutions evolve, researchers caution against uncritical adoption. Othman et al.~\cite{othman2023fostering} highlight that while LLMs can accelerate remediation, they may also introduce biases or encourage over-reliance on automated processes. Similarly, Delnevo et al.~\cite{delnevo2024interaction} emphasize the importance of contextual understanding and adaptability, pointing to the current limitations of LLM-based systems in serving the full spectrum of user needs. 
% In contrast to this backdrop, our work introduces and evaluates CodeA11y, an LLM-augmented extension for GitHub Copilot that not only mitigates these challenges by providing more consistent guidance and manual validation prompts, but also aligns AI-driven assistance with developers’ workflows, ultimately contributing toward more sustainable propulsion for building accessible web.

% Broader implications of inaccessibility—legal compliance, ethical concerns, and user experience
% A Historical Review of Web Accessibility Using WAVE
% "I tend to view ads almost like a pestilence": On the Accessibility Implications of Mobile Ads for Blind Users

% In the research domain, several methods have been developed to assess and enhance web accessibility. These include incorporating feedback into developer tools~\cite{adesigner, takagi2003accessibility, bigham2010accessibility} and automating the creation of accessibility tests and reports for user interfaces~\cite{swearngin2024towards, taeb2024axnav}. 

% Prior work has also studied accessibility scanners as another avenue of AI to improve web development practices~\cite{}.
% However, a persistent challenge is that developers need to be aware of these tools to utilize them effectively. With recent advancements in LLMs, developers might now build accessible websites with less effort using AI assistants. However, the impact of these assistants on the accessibility of their generated code remains unclear. This study aims to investigate these effects.

\subsection{End-user Accessibility Repair}
In addition to detecting accessibility errors and measuring web accessibility, significant research has focused on fixing these problems.
Since end-users are often the first to notice accessibility problems and have a strong incentive to address them, systems have been developed to help them report or fix these problems.

Collaborative, or social accessibility~\cite{takagi2009collaborative,sato2010social}, enabled these end-user contributions to be scaled through crowd-sourcing.
AccessMonkey~\cite{bigham2007accessmonkey} and Accessibility Commons~\cite{kawanaka2008accessibility} were two examples of repositories that store accessibility-related scripts and metadata, respectively.
Other work has developed browser extensions that leverage crowd-sourced databases to automatically correct reading order, alt-text, color contrast, and interaction-related issues~\cite{sato2009s,huang2015can}.

One drawback of collaborative accessibility approaches is that they cannot fix problems for an ``unseen'' web page on-demand, so many projects aim to automatically detect and improve interfaces without the need for an external source of fixes.
A large body of research has focused on making specific web media (e.g., images~\cite{gleason2019making,guinness2018caption, twitterally, gleason2020making, lee2021image}, design~\cite{potluri2019ai,li2019editing, peng2022diffscriber, peng2023slide}, and videos~\cite{pavel2020rescribe,peng2021say,peng2021slidecho,huh2023avscript}) accessible through a combination of machine learning (ML) and user-provided fixes.
Other work has focused on applying more general fixes across all websites.

Opportunity accessibility addressed a common accessibility problem of most websites: by default, content is often hard to see for people with visual impairments, and many users, especially older adults, do not know how to adjust or enable content zooming~\cite{bigham2014making}.
To this end, a browser script (\texttt{oppaccess.js}) was developed that automatically adjusted the browser's content zoom to maximally enlarge content without introducing adverse side-effects (\textit{e.g.,} content overlap).
While \texttt{oppaccess.js} primarily targeted zoom-related accessibility, recent work aimed to enable larger types of changes, by using LLMs to modify the source code of web pages based on user questions or directives~\cite{li2023using}.

Several efforts have been focused on improving access to desktop and mobile applications, which present additional challenges due to the unavailability of app source code (\textit{e.g.,} HTML).
Prefab is an approach that allows graphical UIs to be modified at runtime by detecting existing UI widgets, then replacing them~\cite{dixon2010prefab}.
Interaction Proxies used these runtime modification strategies to ``repair'' Android apps by replacing inaccessible widgets with improved alternatives~\cite{zhang2017interaction, zhang2018robust}.
The widget detection strategies used by these systems previously relied on a combination of heuristics and system metadata (\textit{e.g.,} the view hierarchy), which are incomplete or missing in the accessible apps.
To this end, ML has been employed to better localize~\cite{chen2020object} and repair UI elements~\cite{chen2020unblind,zhang2021screen,wu2023webui,peng2025dreamstruct}.

In general, end-user solutions to repairing application accessibility are limited due to the lack of underlying code and knowledge of the semantics of the intended content.

\subsection{Developer Tools for Accessibility}
Ultimately, the best solution for ensuring an accessible experience lies with front-end developers. Many efforts have focused on building adequate tooling and support to help developers with ensuring that their UI code complies with accessibility standards.

Numerous automated accessibility testing tools have been created to help developers identify accessibility issues in their code: i) static analysis tools, such as IBM Equal Access Accessibility Checker~\cite{ibm2024toolkit} or Microsoft Accessibility Insights~\cite{accessibilityinsights2024}, scan the UI code's compliance with predefined rules derived from accessibility guidelines; and ii) dynamic or runtime accessibility scanners, such as Chrome Devtools~\cite{chromedevtools2024} or axe-Core Accessibility Engine~\cite{deque2024axe}, perform real-time testing on user interfaces to detect interaction issues not identifiable from the code structure. While these tools greatly reduce the manual effort required for accessibility testing, they are often criticized for their limited coverage. Thus, experts often recommend manually testing with assistive technologies to uncover more complex interaction issues. Prior studies have created accessibility crawlers that either assist in developer testing~\cite{swearngin2024towards,taeb2024axnav} or simulate how assistive technologies interact with UIs~\cite{10.1145/3411764.3445455, 10.1145/3551349.3556905, 10.1145/3544548.3580679}.

Similar to end-user accessibility repair, research has focused on generating fixes to remediate accessibility issues in the UI source code. Initial attempts developed heuristic-based algorithms for fixing specific issues, for instance, by replacing text or background color attributes~\cite{10.1145/3611643.3616329}. More recent work has suggested that the code-understanding capabilities of LLMs allow them to suggest more targeted fixes.
For example, a study demonstrated that prompting ChatGPT to fix identified WCAG compliance issues in source code could automatically resolve a significant number of them~\cite{othman2023fostering}. Researchers have sought to leverage this capability by employing a multi-agent LLM architecture to automatically identify and localize issues in source code and suggest potential code fixes~\cite{mehralian2024automated}.

While the approaches mentioned above focus on assessing UI accessibility of already-authored code (\textit{i.e.,} fixing existing code), there is potential for more proactive approaches.
For example, LLMs are often used by developers to generate UI source code from natural language descriptions or tab completions~\cite{chen2021evaluating,GitHubCopilot,lozhkov2024starcoder,hui2024qwen2,roziere2023code,zheng2023codegeex}, but LLMs frequently produce inaccessible code by default~\cite{10.1145/3677846.3677854,mowar2024tab}, leading to inaccessible output when used by developers without sufficient awareness of accessibility knowledge.
The primary focus of this paper is to design a more accessibility-aware coding assistant that both produces more accessible code without manual intervention (\textit{e.g.,} specific user prompting) and gradually enables developers to implement and improve accessibility of automatically-generated code through IDE UI modifications (\textit{e.g.}, reminder notifications).

}
\end{highlight}



% Work related to this paper includes {\em (i)} Web Accessibility and {\em (ii)} Developer Practices in AI-Assisted Programming.

% \ipstart{Web Accessibility: Practice, Evaluation, and Improvements} Substantial efforts have been made to set accessibility standards~\cite{chisholm2001web, caldwell2008web}, establish legal requirements~\cite{sierkowski2002achieving, yesilada2012understanding}, and promote education and advocacy among developers~\cite{sloan2006contextual, martin2022landscape, pandey2023blending}. In the research domain, several methods have been developed to assess and enhance web accessibility. These include incorporating feedback into developer tools~\cite{adesigner, takagi2003accessibility, bigham2010accessibility} and automating the creation of accessibility tests and reports for user interfaces~\cite{swearngin2024towards, taeb2024axnav}. 
% % Prior work has also studied accessibility scanners as another avenue of AI to improve web development practices~\cite{}.
% However, a persistent challenge is that developers need to be aware of these tools to utilize them effectively. With recent advancements in LLMs, developers might now build accessible websites with less effort using AI assistants. However, the impact of these assistants on the accessibility of their generated code remains unclear. This study aims to investigate these effects.

% \ipstart{Developer Practices in AI-Assisted Programming}
% Recent usability research on AI-assisted development has examined the interaction strategies of developers while using AI coding assistants~\cite{barke2023grounded}.
% They observed developers interacted with these assistants in two modes -- 1) \textit{acceleration mode}: associated with shorter completions and 2) \textit{exploration mode}: associated with long completions.
% Liang {\em et al.} \cite{liang2024large} found that developers are driven to use AI assistants to reduce their keystrokes, finish tasks faster, and recall the syntax of programming languages. On the other hand, developers' reason for rejecting autocomplete suggestions was the need for more consideration of appropriate software requirements. This is because primary research on code generation models has mainly focused on functional correctness while often sidelining non-functional requirements such as latency, maintainability, and security~\cite{singhal2024nofuneval}. Consequently, there have been increasing concerns about the security implications of AI-generated code~\cite{sandoval2023lost}. Similarly, this study focuses on the effectiveness and uptake of code suggestions among developers in mitigating accessibility-related vulnerabilities. 


% ============================= additional rw ============================================
% - Paulina Morillo, Diego Chicaiza-Herrera, and Diego Vallejo-Huanga. 2020. System of Recommendation and Automatic Correction of Web Accessibility Using Artificial Intelligence. In Advances in Usability and User Experience, Tareq Ahram and Christianne Falcão (Eds.). Springer International Publishing, Cham, 479–489
% - Juan-Miguel López-Gil and Juanan Pereira. 2024. Turning manual web accessibility success criteria into automatic: an LLM-based approach. Universal Access in the Information Society (2024). https://doi.org/10.1007/s10209-024-01108-z
% - s
% - Calista Huang, Alyssa Ma, Suchir Vyasamudri, Eugenie Puype, Sayem Kamal, Juan Belza Garcia, Salar Cheema, and Michael Lutz. 2024. ACCESS: Prompt Engineering for Automated Web Accessibility Violation Corrections. arXiv:2401.16450 [cs.HC] https://arxiv.org/abs/2401.16450
% - Syed Fatiul Huq, Mahan Tafreshipour, Kate Kalcevich, and Sam Malek. 2025. Automated Generation of Accessibility Test Reports from Recorded User Transcripts. In Proceedings of the 47th International Conference on Software Engineering (ICSE) (Ottawa, Ontario, Canada). IEEE. https://ics.uci.edu/~seal/publications/2025_ICSE_reca11.pdf To appear in IEEE Xplore
% - Achraf Othman, Amira Dhouib, and Aljazi Nasser Al Jabor. 2023. Fostering websites accessibility: A case study on the use of the Large Language Models ChatGPT for automatic remediation. In Proceedings of the 16th International Conference on PErvasive Technologies Related to Assistive Environments (Corfu, Greece) (PETRA ’23). Association for Computing Machinery, New York, NY, USA, 707–713. https://doi.org/10.1145/3594806.3596542
% - Zsuzsanna B. Palmer and Sushil K. Oswal. 0. Constructing Websites with Generative AI Tools: The Accessibility of Their Workflows and Products for Users With Disabilities. Journal of Business and Technical Communication 0, 0 (0), 10506519241280644. https://doi.org/10.1177/10506519241280644
% ============================= additional rw ============================================

\begin{table*}[ht]
\small
\centering
\begin{tabular}{p{4cm}|p{11cm}}
\toprule
\textbf{Category}                               & \multicolumn{1}{c}{\textbf{Description}}                                     \\ \midrule
Misinterpretation (associated with Error of Law)                               & The LLM misinterprets or omits some part/entirety of the legal context, question or the options (or a combination of them). This usually leads to the wrong reasoning and selection of wrong conclusion. The following error instances fall under the taxon of misinterpretation: 1. Misunderstanding the legal rules. 2. Misunderstanding the legal situation/issue at hand. 3. Omission of parts of the provided context while reasoning. 4. Incompletely applying a legal rule. 5. Incorrectly applying the legal rule. 6. Wrong assumptions derived from the provided context.                                                                             \\ \midrule
Irrelevant Premise (associated with Error of Law)                             & The LLM generates a premise which is not relevant in solving the question or that it may divert the reasoning chain from solving the question correctly. An premise may be logically valid and factually true but the absence of this premise can still lead to the correct conclusion.                         \\ \midrule
Factual Hallucination (associated with Error of Fact)                          & This error category covers instances where the LLM, during its reasoning process, generates information that is either inconsistent with the facts of the given legal scenario or is entirely fabricated with no basis in reality.                                                                                                                \\ \bottomrule
\end{tabular}
\caption{Error taxonomy for the Premise-level steps. The taxonomy has been developed with consideration for the types of errors that a human reasoner might commit when constructing a rationale for a given legal scenario. Error of Law and Error of Fact are explained in \citep{cornell2024mistake, cornell2024mistake_fact, oreilly2012errors, wilberg2023mistake}. Some fine-grained error instances of the `Misinterpretation' category are shown in Tables \ref{table:human_annotation_example_initial}, \ref{table: misinterpretation-2}, \ref{table: misinterpretation-3} and \ref{table: misinterpretation-4}.}.
\label{table:premise_errors}
\end{table*}
\section{Evaluation of Reasoning Chains}
\label{sec:error_category}

\subsection{The \textit{Civil Procedure} Dataset}
The dataset has been sourced from MCQs present in the `The Glannon Guide To Civil Procedure' \citep{glannon2013guide}. We compile the \textit{Civ. Pro.} dataset with 175 samples of college-level law multiple-choice questions from the US Civil Procedure laws. The questions are primarily designed to evaluate the ability of university-level law students to reason about various legal scenarios about US Civil Procedure laws and provide their final judgment by choosing the most correct option as an answer. The dataset includes relevant legal context, multiple-choice questions, and expert answers with correct explanations provided by legal experts. These elements were extracted and converted into a prompt-based format suitable for LLM inference and the generation of reasoning chains. The \textit{Civ. Pro.} dataset consists of samples comprising of $\mathcal{D} = {<lc_{n}, q_{n}, o_{n}, e_{n}>}$, where $lc_n$, $q_{n}$, $o_{n}$ and $e_{n}$ denote the $n^{th}$ legal context, question, option-set and expert-answer respectively.  


 % The questions range from straightforward applications of relevant rules to find the answers to complex legal scenarios involving scenarios with a multitude of combination of statutes, precedents and exceptions
 % To find out the commonly occurring errors in the reasoning chains generated by the LLMs, we performed a detailed sentence-level analysis of the chains. We began by segmenting the chains into the individual statements which form part of the argument, comprising of set of premises and a conclusion. To avoid confusion and maintain simplicity, we specify all intermediate sentences in the reasoning chain to be premises and the final sentence which provides the final answer by choosing an option from given options as the conclusion. The premises may be single declarative sentences or combination of declarative sentences and inferences made from those declarations.
 
 % For sound and valid argument construction, the LLM must accurately deduce applicable laws, follow correct reasoning steps, and reach an appropriate conclusion. Similar to solving a deductive reasoning problem, our focus is on evaluating the soundness of arguments, particularly in the context of informal legal reasoning. 
\subsection{Manual Evaluation Of Reasoning Chains}
\label{section:manual-eval-sec}
Human evaluators are instructed to find flaws in a reasoning chain and explain the flaws in natural descriptive language. To solve a given legal question in \textit{Civ. Pro.}, an LLM generates a set of statements $<$$A: s_1, s_2, ..., s_k, c$$>$, where $A$ represents the legal argument/rationale put forward to solve the problem, with $s_1, s_2...s_k$ being the `\textit{k}' number of intermediate steps generated to reason towards the final conclusion $c$. Each step in the reasoning-chain, including the final conclusion, is separately evaluated for the presence/absence of errors. To create an error taxonomy, we adopt an exhaustive approach, continuously updating the taxonomy until no new errors are identified. Specifically, 120 reasoning chains containing approximately 537 reasoning steps are used for evaluation (generated as responses by four LLMs: Mistral-7B-v2-Instruct, Llama-3-8B-Instruct, GPT-3.5-turbo and GPT-4-turbo, to the same 30 data sample subset). This evaluation helped to solidify our proposed taxonomy as described in \textsection \ref{section:error_taxonomy}. Detailed statistics of the human-evaluations are provided in Tables \ref{table:manual-evaluation-premise} and \ref{table:manual-evaluation-conclusion} of Appendix \ref{section:manual-eval}. Further details regarding annotation guidelines and process, inter-annotator agreement statistics using Cohen's kappa coefficient \citep{cohen1960kappa} and annotation examples are provided in Appendix \ref{section:annotation_guidelines} and Appendix \ref{section:human_annot} (Tables \ref{table:human_annotation_example_initial}-\ref{table:human_annotation_example_final}).

%These initial annotations and analyses gave rise to the formulation of an error taxonomy on a fine-grained level.
% divided classifying the errors on two-broad levels: finding errors in 1. Premise-level, the intermediate step level of the rationale, and 2. Final Conclusion level, where an option is chosen as the answer to the legal MCQ. Futher details and examples of the process are provided in  

% Premise-level errors were divided into: 1. Misinterpretation 2. Factual Hallucinations and 3. Irrelevant premises. Conclusion-level errors were divided into: 1. Wrong conclusion from False Premises. 2. Wrong Conclusion from Incomplete Premises 3. Correct Conclusion from False Premises 4. Correct Conclusion from Incomplete Premises and 5. Correct Conclusion with Hallucinated Output 

\subsection{Proposed Error Taxonomy}
\label{section:error_taxonomy}
The error taxonomy is designed to mirror the types of errors humans make when reasoning about passage comprehension and constructing rational arguments. It classifies errors into two levels: 1. Premise-level and 2. Conclusion-level errors. Premise-level errors are based on `Errors of Law' and `Errors of Fact' grounded in the legal domain \citep{cornell2024mistake, cornell2024mistake_fact, oreilly2012errors, wilberg2023mistake}. While premise-level errors often influence errors at the conclusion level, many conclusion-level errors occur independently. Conclusion-level errors serve as indicators of the  overall decision-making ability of LLMs in generating the final answer to a legal question. 

\paragraph{Premise-level Errors}
These errors have occurred in one of the premises of the reasoning chain. They highlight the core issue with LLMs that ineffectively prioritize relevant parts of the prior context and incorrectly identify important information. We categorize these errors as shown in Table \ref{table:premise_errors}.
 
% \paragraph{Category 1: Misinterpretation}
% This is the dominant category of error which occurs in in the premise-level of the step-by-step rationale generated by LLMs. The underlying issue with LLMs is their inability to dynamically prioritize specific parts of prior context and 'intelligently' discern what information is relevant and what can be disregarded. 
% The following error instances fall under the taxon of misinterpretation: 1. Misunderstanding the legal rules. 2. Misunderstanding the legal situation/issue at hand. 3. Incorrectly applying the legal rule. 4. Incompletely applying a legal rule. 5. Omission of parts of the provided context while reasoning. 6. Wrong assumptions derived from the the provided context. 

% \paragraph{Category 2: Factual Hallucinations} This error category includes instances where the LLM, during its reasoning process, cites information that is either inconsistent with the facts of the given legal scenario or is entirely fabricated with no basis in reality. 

% \paragraph{Category 3: Irrelevant Premises}
% The error occurs when the LLM reasoner generates a premise which is not directly useful in reasoning towards an answer. These premises might be factually true and logically valid but they do not contribute towards proving the final conclusion. These errors can deviate the line of reasoning towards focusing on some aspect which would draw wrong conclusions.
\begin{table*}
\small
\begin{tabular}{l|p{3.8cm}|p{8.3cm}}
\toprule
\textbf{Broad-Category}           & \textbf{Sub-Category}                     & \multicolumn{1}{c}{\textbf{Description}}
                                                                                         \\ \midrule
\multirow{4}{*}{Wrong Conclusion} & Wrong Conclusion from False Premise(s)      & This error primarily occurs when the step-by-step rationale generated includes premises that are logically invalid, factually incorrect, irrelevant to solving the question posed, or a combination of these issues.                                                                                 \\ \cmidrule{2-3} 
                                  & Wrong Conclusion from Incomplete Premise(s) & This error occurs when valid and sound premises are provided but fail to fully support the reasoning, leading to the wrong conclusion. A special example is 'Wrong Conclusion from Correct Premises,' where sufficient premises still result in an incorrect conclusion.                 \\ \midrule
Right Conclusion                  & Right Conclusion from False Premise(s)      & This error occurs when the LLM reasons to the correct option while providing a wrong argument. One or multiple premises contain errors which fall under one of the three premise-level error categories and yet lead to the reasoning path choosing the correct option as its final answer. \\ \cmidrule{2-3}
                                  & Right Conclusion from Incomplete Premise(s) & This error occurs when the correct final option is selected, even though the premises provided are incomplete or insufficient to fully justify that conclusion.
                                  
                                                                                                    \\ \cmidrule{2-3}
                                  & Right Conclusion with Hallucinated Content & This error occurs when the LLM selects the correct option but the generated content does not semantically match the provided options. For example, the LLM might output 'Option D. The suspect is \textit{X},' when the actual content is 'Option D. The suspect is \textit{Y} and committed crime \textit{Z},' due to an LLM hallucination. 
                                  \\ \midrule
\end{tabular}
\caption{Error taxonomy for the Conclusion.}
\label{table:conclusion_errors}
\end{table*}

% \textit{Our current auto-evaluator system cannot detect this, as it cannot distinguish it from a 'Correct Conclusion from Correct Premises' scenario.}
% Wrong Conclusion being reached due to a contributing premise being untrue or being true but irrelevant in solving the question.

% Wrong Conclusion/Premature Conclusion being reached due to incomplete reasoning/premises not being sufficient to reach a correct conclusion OR Wrong Conclusion being reached even when the premises are correct and sufficient.

% Right conclusion being reached even when some premise is incorrect. This means that reasoning chain is still faulty and there is some misinterpretation in some part of the reasoning chain.

% Right conclusion being reached even when premises are insufficient. This means that reasoning chain is still faulty and there is some misinterpretation in some part of the reasoning chain.

% While the correct option is chosen, the content of the option generated by the LLM reasoner does not match with that of the actual option provided in the legal context.
\begin{table}[t]
\small
\scriptsize  % Use a smaller font size
\resizebox{0.9\linewidth}{!}{
\begin{tabular}{p{0.85\columnwidth} }
\toprule
    \textbf{Examples of a reasoning step and final conclusion evaluated by GPT-4o based auto-evaluator} \\ \midrule

       \multirow{3}{*}{}{\textbf{Premise 4:} Marla's Intent: Marla's intent is somewhat ambiguous. She is attending a two-year program and is unsure if she wants to be a stylist. She has considered staying in Denver to work if she likes the program, or leaving the program and looking for work in Denver if she doesn't like it.} \\ \textbf{Auto-Evaluator Evaluation and Error Explanation:} Premise 4: Misinterprets Marla's intent.\\The premise incorrectly suggests ambiguity in Marla's intent. According to the legal context, Marla's plans are open-ended, meaning she does not have definite plans to leave Denver. The ambiguity mentioned does not align with the legal context's explanation of "indefinite intent". \\ \textbf{Error Category Label:} "Misinterpretation" \\ 
    \midrule

        \multirow{4}{*}{}{\textbf{Predicted Conclusion:} Given that Marla resides in Colorado and has no definite plans to leave, she meets the domicile test's requirement of residing with the intent to remain indefinitely.

Final answer: D,}\\ \textbf{Extracted Ground-Truth Conclusion:} D.  is domiciled in Colorado, because she resides there with the intent to remain indefinitely. \\ \textbf{Auto-Evaluator Evaluation and Error Explanation:} Based on the provided conditions and the information given:

\\- The extracted conclusion's final answer is D.
\\- The ground truth option is also D.
\\- The content of the extracted conclusion and the ground truth option match: both state that Marla is domiciled in Colorado because she resides there with the intent to remain indefinitely.
\\- There is an error in the premises (misinterpretation of Marla's intent).\\\\
Given these points, the appropriate classification is:
\\
\textbf{CORRECT CONCLUSION FROM FALSE PREMISES} \\ \textbf{Error Category Label:} "Correct Conclusion from False Premises"\\

\bottomrule
\end{tabular}}
\caption{Example of reasoning step (premise) and conclusion evaluated by LLM-based `Auto-evaluator' (GPT-4o). The error category labels are extracted from the detailed explanations using an LLM prompted to extract error keywords.}
\label{table:example_annot}
\end{table}


\paragraph{Conclusion-level Errors} 
Conclusion-level errors indicate issues with deductive reasoning, reflecting the LLM's ability to follow premises to reach the correct conclusion. They also reveal how much the decision-making process is influenced by intermediate premises in choosing the final answer. We categorize these errors as shown in Table \ref{table:conclusion_errors}.  
% \paragraph{Category 1: Wrong Conclusion from False Premise(s)}
% This error primarily occurs when the step-by-step rationale generated includes premises that are logically invalid, factually incorrect, irrelevant to solving the question posed, or a combination of these issues.   
% These factors render the intermediate step as being 'False' and therefore lead the line of reasoning towards a wrong conclusion. 

% \paragraph{Category 2: Wrong Conclusion from Incomplete Premise(s)}
% This error occurs when the step-by-step generated rationale includes premises which, while correct and valid towards the answering question, fall short of providing the entire line of reasoning which could lead to the correct conclusion. As a result, the wrong conclusion gets selected as the answer. A special case of this error is the 'Wrong Conclusion from Correct Premises' where even though the premises generated are true and sufficient to find the correct conclusion, the LLM reasoner still spuriously chooses a wrong conclusion as its answer.

% \paragraph{Category 3: Correct Conclusion from False Premises}
% This error occurs when the LLM reasons to the correct option while providing a wrong argument. One or multiple premises contain errors which fall under one of the three premise-level error categories and yet lead to the reasoning path choosing the correct option as its final answer.  

\begin{figure*}[t]
    \centering     
    \includegraphics[width=0.95\linewidth]{Images/Autoeval.pdf}
    \caption{The overall schematic representation of the LLM-based error-detection and evaluation system and the calculation of the metrics. The reasoning chains are produced by 5 LLMs and the expert answer is referenced from the \textit{Civ. Pro.} dataset}
    \label{fig:autoeval}
\end{figure*}

\paragraph{Conclusion from Incomplete Premises \textit{vs.} Correct Premises}
We argue that a `Wrong Conclusion from Correct Premises' is essentially a `Wrong Conclusion from Incomplete Premises' because either the premises, though correct, are incomplete and lead to a wrong conclusion, or the LLM fails to explicitly generate a key premise. This poses a challenge for LLM-based auto-evaluators, as discussed in \textsection \ref{section:llm_aided_eval}, which struggle to assess whether the rationale is sufficient or inadequate.

\subsection{LLM-aided Automatic Evaluation}
\label{section:llm_aided_eval}
Manual analysis of reasoning chains provided a detailed categorization of errors; however, it was time-consuming and, therefore, challenging to scale for the entire dataset. Thus, we develop an alternate approach to leverage LLMs to evaluate the errors in the reasoning chains akin to human evaluation. Specifically, we use GPT-4o as the LLM backbone of the `auto-evaluator' system to identify and label the errors. The auto-evaluator assesses a total of 875 reasoning chains, encompassing approximately 4,844 individual reasoning steps, which include both premise-level and conclusion-level steps (refer Table \ref{table:average_steps}). The details of the implementation are described in Appendix \ref{section:auto-eval} and an example snippet of LLM-aided annotation is provided in Table \ref{table:example_annot}. We develop two approaches for error evaluation:     

\paragraph{Exact Error Label Match} In this approach, we task the `auto-evaluator' with identifying the exact error category labels which the human evaluators had labeled a particular premise/conclusion of a reasoning chain. Experiments revealed significant mislabeling between the auto-evaluator and human evaluators, with many `Misinterpretation' errors at the premise level being labeled as `Irrelevant Premises' or `Factual Hallucination' (Refer Appendix \ref{section:error_disambiguation}) by the auto-evaluator, and vice versa. Hence, we make changes to the auto-evaluators to include error explanations along with the labels.

Another significant challenge was the low error detection rate of factual hallucinations with the help of single-call LLM auto-evaluators. Motivated by \citet{varshney2023stitchtimesavesnine, dhuliawala2023chainofverificationreduceshallucinationlarge}, we develop a multi-call LLM system, consisting of two separate LLM calls,  in which one LLM call creates verification questions to probe various aspects of a premise and another LLM call answers them citing the provided legal context for factuality. A premise is considered to contain factual hallucination if the answers to any of the verification questions contradicts the content of the premise directly.  

\paragraph{Semantic Error Explanation Match} As an alternative approach to the above problems, we develop a multi-analyzer system consisting of three `single-call' and one `multi-call' LLM-based pipeline focused on providing explanation of errors at the premise-level. A `summarizer' LLM (Refer Appendix \ref{section: summ_agg_llm}) combines the individual analyses of all analyzers into a single error explanation for a premise. This enables the pipeline to detect and label multiple errors in a single premise (e.g., a premise containing both misinterpretation and factual hallucination).

To validate the effectiveness of the auto-evaluator, we sample 120 reasoning chains from the manually evaluated set of four LLMs (Mistral-7B-v2-Instruct, Llama-3-8B-Instruct, GPT-3.5-turbo and GPT-4-turbo). The human evaluators then compare their error category assignments as well as explanations to those provided by the auto-evaluator. The recall percentage of detecting an error at the premise level across four LLMs ranged from 83.87$\%$ to 90.6\%. The recall percentage range for detecting an error-free premise step ranged from 86.17$\%$ to 93.85$\%$. The details of the autoevaluator performance statistics are present in Tables \ref{table:auto-eval-agreement-premise} and \ref{table:auto-eval-agreement-conclusion} of Appendix \ref{section:auto-eval-agree}. Figure \ref{fig:autoeval-system} shows the pipeline of the error detection implemented using GPT-4o. 

% The agreement score (recall percentage) for detecting the presence of an error (through explanation match) in reasoning steps between the manual evaluation and the GPT-4o evaluation was $\sim$85.3$\%$. The agreement score (recall percentage) for detecting the absence of an error (through explanation match) in reasoning steps between the manual evaluation and the GPT-4o evaluation was $\sim$86$\%$

\section{Experiments}
In this section, we present the data collection and pre-processing, implementation details, and experimental results of the proposed approach. 

\subsection{Data Collection and Implementation Details}
\subsubsection{Dataset}
To the best of our knowledge, there are no existing datasets that demonstrate the stroke-by-stroke evolution of visual art from the initial sketch to the final painting. The closest available dataset for ordered stroke sequences is QuickDraw \cite{ha2017neural}, which consists of simple hand-drawn vector sketches of common objects, created in an online game where players are to draw specific objects within 20 seconds. However, these sketches do not align with our objective of generating the evolution of complex real-world artworks.

For our purposes, we curate a sample dataset from WikiArt \cite{saleh2015large} that includes a diverse range of artworks by renowned artists. In particular, we sampled 500 artworks from various artists to evaluate the effectiveness of the proposed method. Additionally, to investigate the proposed method in diverse settings, we harvested 90 sketch images and 70 RGB images, which include line art, face sketches, and natural images. We randomly sampled face sketches from FS2K-SDE Dataset \cite{dai2023sketch}, line art sketches from \cite{lineartweb}, and natural images from \cite{tong2021sketch}. 

\subsubsection{Implementation details}
To obtain sketches from paintings and natural images, we leverage the line drawing method \cite{chan2022learning} that trained on sampled COCO dataset using CLIP features. Further, to attain a vector image of the input sketch or image, we convert pixel image into vector curves through SVG conversion via the vectorizing tool \cite{vectwebsite}. We impose no restrictions on the dimensions of image inputs or the number of strokes within each image. And, the proximity distance is treated as a hyper-parameter. Here, the number of clusters and the number of strokes per cluster are determined based on this proximity distance. We found that setting proximity distance to approximately 
$\max(Input_{width}, Input_{height})/8$ yields compact clusters that provide a good balance between the number of clusters and the number of strokes per cluster.

\begin{figure}[!t]
    \centering
    \includegraphics[width=\textwidth]{Samples/Fig5_Evolution_WikiArt.pdf}
    \caption{Sketch \& Paint stroke evolution sequences on WikiArt samples.}
    \label{fig:wikiart_results}
\end{figure}

 
\subsection{Results}
We extensively test our algorithm on inputs with varying degrees of complexity and structure. Since there are no formal quantitative measures to gauge the valid stroke sequence evolution on input images, we principally assess the effectiveness of the proposed model qualitatively.

\subsubsection{Results on WikiArt}
 Figure \ref{fig:wikiart_results} shows some examples of stroke-by-stroke ordering on the WikiArt dataset. From this figure, we can observe that the proposed algorithm successfully composes stroke sequence evolution from sketch to paint. Additionally, it effectively handles images with varied resolutions, intricate details, numerous strokes, and diverse color palettes. 


\begin{figure}[!t]

    \centering
    \includegraphics[width=\textwidth]{Samples/Fig6_Evolution_DiverseData.pdf}   
    \caption{Demonstration of stroke evolution on various other input data types such as (A-B) Simple line art \cite{lineartweb} (C-D) Face sketches sampled from FS2K-SDE \cite{dai2023sketch}, (E-F) Natural images from \cite{tong2021sketch}.}
    \label{fig:aaai_results}
\end{figure}

To evaluate the robustness of the proposed method, we further apply it to other forms of data such as line art, face sketches, and natural images. Figure \ref{fig:aaai_results} presents sampled sequences from these diverse input images. The results demonstrate that the predicted stroke sequence order closely mirrors a pragmatic drawing process, regardless of the input type. Specifically, the algorithm produces plausible drawing sequences for less complex images like line art (Figure \ref{fig:aaai_results}, A-B) and face sketches (Figure \ref{fig:aaai_results}, C-D). These predicted sequences follow a logical and intuitive order, closely mirroring the natural progression an artist is likely to adopt. As seen in Figure \ref{fig:aaai_results} (E-F), our method can also effectively interpret natural images that are complex in terms of resolution, detail, and stroke count. From all the results presented above, we can infer that our algorithm can comprehend a variety of input images and produce a pragmatic drawing process. Dynamic examples of drawing evolution, from sketch to painting across various inputs, can be viewed at \cite{youtube_video}. 


\subsubsection{Comparison with other methods}
In this section, we analyze the effectiveness of our algorithm relative to other state-of-the-art methods. Figure \ref{fig:comparison} illustrates the comparative evolution of our method against prominent techniques \cite{tong2021sketch, liu2021paint}. 
For a fair assessment, we qualitatively compare only our image-to-sketch translation with VectorFlow’s \cite{tong2021sketch} image-to-pencil translation, and only our colored paint stroke sequence with Paint Transformer’s \cite{liu2021paint} paint sequence. 
In other words, to ensure fair comparability, we omit color sequencing for the VectorFlow comparison and sketch sequencing for the Paint Transformer comparison. From Figure \ref{fig:comparison}, we can infer that our proposed method can provide systematic sequencing rather than projecting strokes in random order as in \cite{liu2021paint,tong2021sketch}.

\begin{figure}[!t]
    \centering
    \includegraphics[width=\textwidth]{Samples/Fig7_Comparison_RelatedWork.pdf}
    \caption{Qualitative comparison of our method over Vector Flow \cite{tong2021sketch} and Paint Transformer \cite{liu2021paint} (We only include the relevant corresponding portions of generated sequence from our method).} 
    \label{fig:comparison}
\end{figure}

\begin{figure}[!t]
    \centering
    \includegraphics[width=\textwidth]{Samples/Fig8_UserSurvey.pdf}
    \caption{User-survey results for quality evaluation of our method, Paint \& Sketch, in comparison with  Vector Flow (A) \cite{tong2021sketch} and Paint Transformer (B). \cite{liu2021paint} }
    \label{fig:user_survey}
\end{figure}

Additionally, we conducted an initial user study with limited participants (5), each evaluating 5 image generations with VectorFlow \cite{tong2021sketch} and 3 image generations with PaintTransformer \cite{liu2021paint} along with corresponding generations through our method. Participants rated the systems on naturalness, user engagement, stroke quality, and overall experience, with scores ranging from 1 to 10. The survey asked the participants about how accurately the system emulated a pragmatic drawing process, how engaged they felt during the interaction, what the quality of stroke texture and tone was, and what their overall satisfaction was. The bar chart depicting the average user study results is shown in Figure \ref{fig:user_survey}. These results demonstrate that our method provides a more engaging and satisfying user experience than other related methods.

\subsubsection{Limitations} The proposed approach has some known limitations: (1) The method depends upon the quality of line drawing for sketch generation. Hence, it may fail to extract contour lines when the image is dominated by black-intensity regions. (2) There is no mechanism that identifies and learns from the generated sequences that are better aligned to human drawing processes. (3) The paint strokes are coarse and typically not in the form of strokes from any particular drawing medium such as painting brushes.

\section{Results and Analysis}
\label{sec:results}

\subsection{Objective Evaluation}

\paragraph{Soundness metrics are high but correctness scores are low}
Table \ref{table:metrics} shows that the majority of the premises are error-free (with the highest being GPT-4o having 78.4\% of the generated premises being error-free). In contrast, Figure \ref{fig:prem_conc_corr} reveals that an average of $\sim$96\% of reasoning chains leading to conclusions from false premises have one or more misinterpretation errors in the intermediate premises. This finding, aided by empirical human analysis, suggests that much of the LLM-generated reasoning chain re-iterates existing context, while most errors occur in the smaller portion where new `decision-making' inferences are generated. The similarity in correctness score in Mistral-7B-v2-Instruct and Llama-3-8B-Instruct in contrast to the higher accuracy of Llama-3-8B-Instruct could be attributed to the lesser number of steps (see Table \ref{table:average_steps} (Appendix \ref{section:avg_steps_generated})) on average in the reasoning chain of Llama-3-8B-Instruct when compared to Mistral-7B-v2-Instruct. 
\begin{table}[!htbp]
\centering
\begin{tabular}{lccc}
\toprule
Model   & S ($\uparrow$) & A ($\uparrow$)  & C ($\uparrow$)\\ 
\midrule
Mistral-7B-v2-Instruct   & 0.623  & 0.371 & 0.131\\ 
Llama-3-8B-Instruct   & 0.493  & 0.451 & 0.137\\ 
GPT-3.5-turbo   & 0.607  & 0.417 & 0.217\\ 
GPT-4-turbo     & 0.738  & 0.725 & 0.417\\ 
GPT-4o   & \textbf{0.784}  & \textbf{0.737} & \textbf{0.445}\\ 
\bottomrule
\end{tabular}
\caption{The results for soundness, accuracy, and correctness metrics for all LLMs on the \textit{Civ. Pro.} dataset. Here `S' denotes the Soundness, `A' denotes the Accuracy, and `C' denotes the Correctness.}
\label{table:metrics}
\end{table}

\paragraph{Accuracy vs. Correctness Score} Table \ref{table:metrics} and Figure \ref{fig:acc_correctness_results} show a sharp decrease (an average of $\sim$27\%) in the scores of accuracy to correctness across all LLMs. The highest fall in percentage is observed in Llama-3-8B-Instuct (31.4\% decrease). This is significant as it shows that while LLMs can arrive at the correct conclusion, there are a lot of cases where the reasoning chain they generate is not entirely error-free. These results also suggest that LLMs often rely on superficial correlations and patterns, likely learned in the training stages, to arrive at correct conclusions, rather than through genuine reasoning. In high-stakes domains such as legal, financial, and medical fields, it is imperative that the reasoning generated by LLMs is completely error-free as even minor inaccuracies in these critical areas can lead to significant consequences. This also underscores the necessity for robust evaluation mechanisms to ensure the reliability and correctness of model outputs.   

\begin{figure}[h]
    \centering
    \includegraphics[width=1.0\linewidth]{Images/accuracy_and_correctness.pdf}
    \caption{Performance of 5 LLMs in terms of Accuracy vs. Correctness on the \textit{Civ. Pro.} dataset. Here, Mistral stands for Mistral-7B-v2-Instruct, Llama stands for Llama-3-8B-Instruct, GPT-3.5t and GPT-4t stand for GPT-3.5-turbo and GPT-4-turbo respectively.}
    \label{fig:acc_correctness_results}
\end{figure}

\begin{table}[ht]
\small
\centering
\begin{tabular}{lcccc}
\toprule
% \multirow{2}{*}{} & \multicolumn{4}{c}{Prompting Strategies} \\
% \cmidrule(lr){2-5}
Prompting               & B & PS & SC & SD \\
\midrule
\textbf{Gemini} & \multicolumn{4}{c}{} \\
w/o feedback     & 63.31    & 59.17          & \textbf{61.54}        & \textbf{64.50}          \\
w/ feedback      & \textbf{64.50}    & \textbf{62.13}          & 59.76        & 63.31          \\
\midrule
\textbf{Llama} & \multicolumn{4}{c}{} \\
w/o feedback     & 53.71    & 50.29          & 48.58        & 47.42          \\
w/ feedback      & \textbf{57.14}    & \textbf{52.00}          & \textbf{52.57  }      & \textbf{49.14}          \\
\bottomrule
\end{tabular}
\caption{Comparison of accuracy metric for models under different prompting strategies with and without feedback. The models are Gemini-1.5-Flash and Llama-3-8B-Instruct. The prompting strategies abbreviations stands for B: Baseline (CoT), PS: Plan-and-Solve, SC: Self-Correct, and SD: Self-Discovery.}
\label{table:prompt_accuracy}
\end{table}
\paragraph{Larger, proprietary models `reason' better than smaller, open-source models}
Figure \ref{fig:acc_correctness_results} and Table \ref{table:metrics} convey that proprietary models generate more error-free reasoning steps and arrive at the correct conclusion more often than the open-source LLMs. An exception is GPT-3.5-turbo, which performs comparably to Llama-3-8B-Instruct and Mistral-7B-v2-Instruct, suggesting that training data and methods might play a more significant role in enhancing reasoning than merely scaling model parameters. 

\subsection{Reasoning Chain Evaluation}
\paragraph{`Misinterpretations' are the dominant category of errors at premise-level}
Figure \ref{fig:prem-level-cat} (Appendix \ref{section: premise-level-error-dist}) and Figure \ref{fig:prem_conc_corr} reveal that `Misinterpretation' is the most dominant category of error which occurs in the reasoning chains at the premise-level. This highlights that  LLMs struggle to fully grasp the nuanced complexities of legal scenarios requiring the demonstration of critical analysis in zero-shot CoT settings.  

% This aptly goes on to show that LLMs, while trained on vast amounts of data, do not represent and effectively `understand' the complex nuances of legal scenarios (or any other complex context requiring critical analysis) when used in a zero-shot setting which is equivalent to an experienced human reasoner (a lawyer in the scenario of legal reasoning) being able to capture the nuances and correctly solve the problem with minimal supervision/guidance. The frequent occurrence of misinterpretations within reasoning chains, which lead to conclusion-level errors, suggests that they are a primary factor contributing to flaws or failures in the reasoning process.
\paragraph{`Wrong Conclusion from False Premises' is the dominant category of error at conclusion-level} The prevalence of `Wrong Conclusion from False Premises' (Figure \ref{fig:conc-level-cat} (Appendix \ref{section: conclusion-level-error-dist})) in conclusion-level errors results from premise-level mistakes leading to incorrect conclusions. However, in GPT-4-turbo and GPT-4o, the dominant error is `Correct Conclusion from False Premises,' suggesting these models may be relying on patterns of similar examples from their training.  

\subsection{Discussion on Error-Mitigation Strategies}
\label{section:mitigation_discussion}
We carry out several experiments on the \textit{Civ. Pro.} dataset, employing widely used prompting techniques alongside the most frequently observed errors we found through \textsection \ref{section:error_taxonomy} with the aim to explore the possibility of enhancing the reasoning capabilities of both closed-source and open-source LLMs. Four prompting techniques are utilized: (1) Chain-of-Thought \citep{wei2022chainofthought} (2) Plan-and-Solve \citep{wang2023plan} (3) Self-Correct  \citep{zhang2024smalllanguagemodelsneed} and (4) Self-Discovery  \citep{zhou2024selfdiscoverlargelanguagemodels}. These techniques are tested with and without incorporating error definitions as feedback, following the Feedback-Learning method \citep{tyagi2024stepbystepreasoningsolvegrid}. Detailed descriptions of the prompting strategies can be found in Appendix \ref{section:prompting_strategies}.

The error definitions are provided in three styles: generic, short, and long. The generic version uses the error definitions from the Feedback-Learning method, while the short and long versions are derived from the error taxonomy described in \textsection\ref{section:error_taxonomy}. All experiments are conducted in a zero-shot setting, and we evaluate each prompting technique based on the accuracy metric. We test one closed-source model, Gemini-1.5-Flash, and one open-source model, Llama-3-8B-Instruct.

As shown in Table \ref{table:prompt_accuracy}, adding the error definitions as feedback showed improvement in accuracy up to ~4\%. For Llama-3-8B-instruct, accuracy improved across all prompting techniques, whereas for Gemini-1.5-Flash, the accuracy increased only for the Chain-of-Thought and Plan-and-Solve methods. From our observations, the decrease in accuracy for these strategies with Gemini resulted due to self-doubting \citep{krishna2023intersectionselfcorrectiontrustlanguage} nature of LLMs. These findings suggest that while feedback on errors provides marginal improvements in LLM performance, there is a need to develop more effective frameworks beyond prompting, such as agent-based methods, that account for these errors and enhance the model’s legal reasoning capabilities.

% These findings show The minor improvements highlight the need for strategies beyond prompting to achieve further performance gains. 

% This finding points towards the possibility that while feedback of errors committed helps in marginally improving LLM performance, they are not strong supervision signals for the LLMs to rectify their reasoning traces.

% \mihir{Write 1-2 lines about what are the takeaways. Why these experiments add value to current study?} 

% We used the zero-shot CoT method as the baseline method in which the LLM is prompted to provide the final answer along with step-by-step reasoning. Plan-and-Solve prompts the LLM first to generate a plan to solve the problem without solving it and after that the LLM carries out the self-suggested plan to get the final answer. Self-Correct uses self-verification and self-refining to improve the reasoning ability of the LLMs. Self-Discover utilizes self-discover reasoning modules to create an explicit reasoning structure to follow to solve the problem. Lastly, we follow Feedback-Learning which provides the definitions and methods to navigate through the identified error taxonomy. The detailed prompt structure for each technique is illustrated in the Appendix. 

% and GPT-4o outperforms Llama3-70B in every prompting strategy. It is also should be noted that Self-Discovery is the most costly prompting technique due to the use of more tokens and the 4-step structure to get the final answer.

% \begin{table*}[!htbp]
% \small
% \centering
% \begin{tabular}{lcccccc}
% \toprule
% Model   & Baseline & Plan-and-Solve  & Self-Correct & Self-Discovery & Feedback-Learning & Self-Ranking \\ 
% \midrule
% GPT-4o   & 78.85  & 80.00 & 80.00 & 76.57 & 79.43 & 80.57 \\ 
% Llama-3-70B   & 50.29  & 52.00 & 51.43 & 48.00 & 52.00 & 57.71 \\ 
% \bottomrule
% \end{tabular}
% \caption{The results for accuracy for GPT-4o and Llama-3-70B on the legal reasoning dataset.}
% \label{table:accuracy}
% \end{table*}

% \st{sheds light on the finding} \mihir{do not use words like this}


\section{Conclusion}
\label{sec:Conclusion}
In this paper, we proposed a complete real-time planning and control approach for continuous, reliable, and fast online generation of dynamically feasible Bernstein trajectories and control for FW aircrafts. The generated trajectories span kilometers, navigating through multiple waypoints. By leveraging differential flatness equations for coordinated flight, we ensure precise trajectory tracking. Our approach guarantees smooth transitions from simulation to real-world applications, enabling timely field deployment. The system also features a user-friendly mission planning interface. Continuous replanning  maintains the rajectory curvature 
$\kappa$ within limits, preventing abrupt roll changes.

Future works will include the ability to add  a higher-level kinodynamic path planner to optimize waypoint spatial allocation and improve replanning success, and enhancing the trajectory-tracking algorithm by refining the aerodynamic coefficient estimation. 



% \section*{Ethics Statement}
% We have utilized AI assistants, specifically Grammarly and ChatGPT, to correct grammatical errors and rephrase sentences

% \section*{Acknowledgement}

% We extend our gratitude to the Research Computing (RC), and Enterprise Technology at ASU for providing computing resources, and access to the ChatGPT enterprise version for experiments. We acknowledge the support of a CISCO research grant.

% Entries for the entire Anthology, followed by custom entries
\bibliography{references}
\bibliographystyle{acl_natbib}

\clearpage

\appendix
%%%%%%%%%%%%%%%%%%%%%%%%%%%%%%%%%%%%%%%%%%%%%%%%%%%%%%%%%%%%%%%%%%%%%%%%%%%%%%%
%%%%%%%%%%%%%%%%%%%%%%%%%%%%%%%%%%%%%%%%%%%%%%%%%%%%%%%%%%%%%%%%%%%%%%%%%%%%%%%
% APPENDIX
%%%%%%%%%%%%%%%%%%%%%%%%%%%%%%%%%%%%%%%%%%%%%%%%%%%%%%%%%%%%%%%%%%%%%%%%%%%%%%%
%%%%%%%%%%%%%%%%%%%%%%%%%%%%%%%%%%%%%%%%%%%%%%%%%%%%%%%%%%%%%%%%%%%%%%%%%%%%%%%
\newpage
\appendix
\onecolumn

\section{Related Work} \label{app:related_work}
\textbf{Personalized Generation} 
Due to the considerable success of large text-to-image models \cite{ramesh2022hierarchical, ramesh2021zero, saharia2022photorealistic, rombach2022high}, the field of personalized generation has been actively developed. The challenge is to customize a text-to-image model to generate specific concepts that are specified using several input images. Many different approaches \cite{DB, TI, CD, svdiff, ortogonal, profusion, elite, r1e} have been proposed to solve this problem and can be divided into the following groups: pseudo-token optimization \cite{TI, profusion, disenbooth, r1e}, diffusion fune-tuning \cite{DB, CD, profusion}, and encoder-based \cite{elite}. The pseudo-token paradigm adjusts the text encoder to convert the concept token into the proper embedding for the diffusion model. Such embedding can be optimized directly \cite{TI, r1e} or can be generated by other neural networks \cite{disenbooth, profusion}. Such approaches usually require a small number of parameters to optimize but lose the visual features of the target concept. Diffusion fine-tuning-based methods optimize almost all \cite{DB} or parts \cite{CD} of the model to reconstruct the training images of the concept. This allows the model to learn the input concept with high accuracy, but the model due to overfitting may lose the ability to edit it when generated with different text prompts. To reduce overfitting and memory usage, lightweight parameterizations \cite{svdiff, r1e, lora} have been proposed that preserve edibility but at the cost of degrading concept fidelity. Encoder-based methods \cite{elite} allow one forward pass of an encoder that has been trained on a large dataset of many different objects to embed the input concept. This dramatically speeds up the process of learning a new concept and such a model is highly editable, but the quality of recovering concept details may be low. Generally, the main problem with existing personalized generation approaches is that they struggle to simultaneously recover a concept with high quality and generate it in a variety of scenes.

\textbf{Sampling strategies}
Much research has been devoted to sampling techniques for text-to-image diffusion models, focusing not only on personalized generation but also on image editing. In this paper, we address a more specific question: how can the two trajectories -- superclass and concept -- be optimally combined to achieve both high concept fidelity and high editability? The ProFusion paper \cite{profusion} considered one way of combining these trajectories (Mixed sampling), which we analyze in detail in our paper (see Section \ref{sec:mixed_sampling}) and show its properties and problems. In ProFusion, authors additionally proposed a more complex sampling procedure, which we observed to be redundant compared to Mixed sampling, as can be seen in our experiments (see Section \ref{sec:experiments}). In Photoswap \cite{photoswap}, authors consider another way of combining trajectories by superclass and concept, which turns out to be almost identical to the Switching sampling strategy that we discuss in detail in Section \ref{sec:switching_sampling}. We show why this strategy fails to achieve simultaneous improvements in concept reconstruction and editability. In the paper, we propose a more efficient way of combining these two trajectories that achieves an optimal balance between the two key features of personalized generation: concept reconstruction and editability.

\section{Training details} \label{sec:training-details}
The Stable Diffusion-2-base model is used for all experiments. For the Dreambooth, Custom Diffusion, and Textual Inversion methods, we used the implementation from \url{https://github.com/huggingface/diffusers}.

\textbf{SVDiff} We implement the method based on \url{https://github.com/mkshing/svdiff-pytorch}. The parameterization is applied to all Text Encoder and U-Net layers. The models for all concepts were trained for $1600$ using Adam optimizer with $\text{batch size} = 1$, $\text{learning rate} = 0.001$, $\text{learning rate 1d} = 0.000001$, $\text{betas} = (0.9, 0.999)$, $\text{epsilon} = 1e\!-\!8$, and $\text{weight decay} = 0.01$. 

\textbf{Dreambooth} All query, key, and value layers in Text Encoder and U-Net were trained during fine-tuning. The models for all concepts were trained for $400$ steps using Adam optimizer with $\text{batch size} = 1$, $\text{learning rate} = 2e\!-\!5$, $\text{betas} = (0.9, 0.999)$, $\text{epsilon} = 1e\!-\!8$, and $\text{weight decay} = 0.01$. 

\textbf{Custom Diffusion} The models for all concepts were trained for $1600$ steps using Adam optimizer with $\text{batch size} = 1$, $\text{learning rate} = 0.00001$, $\text{betas} = (0.9, 0.999)$, $\text{epsilon} = 1e\!-\!8$, and $\text{weight decay} = 0.01$. 

\textbf{Textual Inversion} The models for all concepts were trained for $10000$ steps using Adam optimizer with $\text{batch size} = 1$, $\text{learning rate} = 0.005$, $\text{betas} = (0.9, 0.999)$, $\text{epsilon} = 1e\!-\!8$, and $\text{weight decay} = 0.01$. 

\textbf{ELITE} We used the pre-trained model from the official repo \url{https://github.com/csyxwei/ELITE} with $\lambda=0.6$ and inference hyperparams from the original paper.

\clearpage
\section{Superclass and concept trajectory choice}\label{app:hyper_theta}

 \begin{wrapfigure}{r}{0.45\textwidth}
    % \centering
    \includegraphics[trim={3cm 10cm 3cm 10cm},clip,width=\linewidth]{imgs/mixed_noft_nosup.pdf}
    \caption{The Pareto frontiers for original Mixed sampling and Mixed sampling in the Superclass, NoFT, and Empty Prompt setups. Mixed NoFT and Mixed Empty Prompt configurations overlap with the Pareto frontier of the original mixed sampling, but primarily in regions associated with low image similarity, which compromises concept fidelity.} \label{fig:mixed_noft_ep}
    \vspace{-0.14in}
\end{wrapfigure}

There are multiple ways to define sampling with maximized textual alignment to the prompt. However, the arbitrary choice can harm the alignment between Base sampling~\ref{eq:concept_sampling} and the selected trajectory. We use the Sampling with superclass (\ref{eq:superclass_sampling}) as it's the default choice in the literature and guarantees the maximized alignment between noise predictions $\tilde{\varepsilon}_{\theta}(p^C)$ and $\tilde{\varepsilon}_{\theta}(p^S)$. 

The several natural ways to adjust Sampling with superclass can be presented by varying $\theta$ and $p^{S}$ in (\ref{eq:superclass_sampling}). We explore two additional options with decreased alignment with (\ref{eq:concept_sampling}): (1) NoFT -- weights of base model $\theta^{\text{orig}}$ instead fine-tuned weights, (2) Empty Prompt -- prompt without any reference to a concept, even to its superclass category, i.e. $p^{\hat{S}} = \textit{"with a city in the background"}$ instead of $p^{S} = \textit{"a backpack with a city in the background"}$.

To validate the robustness of our framework for sampling method selection, we employ the original experimental protocol, supplementing the results shown in Figures~\ref{fig:examples} and~\ref{fig:profusion-photoswap}. Our analysis of Figures~\ref{fig:multi-stage_noft_ep} and~\ref{fig:masked_noft_ep} reveals that trajectories generated under the NoFT and Empty Prompt configurations (second and third columns, respectively) maintain identical method ordering to those produced by Superclass sampling ((\ref{eq:superclass_sampling}), first column).

Notably, Figure~\ref{fig:mixed_noft_ep} shows that Empty Prompt configuration demonstrates weaker alignment with Base sampling compared to NoFT, particularly at higher values of the superclass guidance scale $\omega_{s}$. This divergence manifests as reduced concept fidelity for Empty Prompt under large $\omega_{s}$. These findings highlight a practical adjustment: prioritizing smaller $\omega_{s}$ values in Empty Prompt setup preserves concept fidelity without altering the framework’s core selection logic. 

A key limitation of increased misalignment is the gradual erosion of superclass category information from generated images, which can lead to semantically inconsistent outputs. For instance, Figure~\ref{fig:examples_noft_ep} illustrates how the Mixed Empty Prompt setup, despite the strong animal prior in Base sampling, can produce human-like features in an image of a cat described as \textit{"in a chef outfit"}. This suggests that when superclass information is weakened, the model may introduce unexpected visual artifacts, impacting the fidelity of the intended concept.

Concept sampling (\ref{eq:concept_sampling}) can also be adjusted to better capture a concept’s visual characteristics, further decoupling fidelity from editability. For example, this can be achieved by (1) using the weights of a highly overfitted model (e.g., DreamBooth) or (2) selecting a prompt that omits contextual details, such as $p^{\hat{C}} = \textit{"a photo of V*"}$ instead of $p^{C} = \textit{"a V* with a city in the background"}$. Combining superclass sampling under NoFT or Empty Prompt with Base sampling configured via (1) or (2) could enhance both image and text similarity. We leave this direction for future work.

\begin{figure}[b]
    \centering
    \includegraphics[trim={3cm 10cm 3cm 10cm},clip,width=0.32\linewidth]{imgs/multi-stage_original.pdf}
    \hfill
    \includegraphics[trim={3cm 10cm 3cm 10cm},clip,width=0.32\linewidth]{imgs/multi-stage_noft.pdf}
    \hfill
    \includegraphics[trim={3cm 10cm 3cm 10cm},clip,width=0.32\linewidth]{imgs/multi-stage_nosup.pdf}
    \caption{Pareto Frontier Curves for Mixed, Switching, and Multi-Stage Sampling Methods in the Superclass, NoFT and Empty Prompt setups.
The NoFT and Empty Prompt configurations (second and third columns, respectively) preserve the same method ordering as those produced by Superclass sampling (first column).} \label{fig:multi-stage_noft_ep}
\end{figure}
\begin{figure}[t]
    \centering
    \includegraphics[trim={3cm 10cm 3cm 10cm},clip,width=0.32\linewidth]{imgs/masked_profusion.pdf}
    \hfill
    \includegraphics[trim={3cm 10cm 3cm 10cm},clip,width=0.32\linewidth]{imgs/masked_noft.pdf}
    \hfill
    \includegraphics[trim={3cm 10cm 3cm 10cm},clip,width=0.32\linewidth]{imgs/masked_nosup.pdf}
    \caption{Pareto Frontier Curves for Mixed, Switching, Masked, and ProFusion Sampling Methods in the Superclass, NoFT, and Empty Prompt setups.
The NoFT and Empty Prompt configurations (second and third columns, respectively) preserve the same method ordering as those produced by Superclass sampling (first column).} \label{fig:masked_noft_ep}
\end{figure}

\begin{figure}[b]
    \centering
    \includegraphics[width=\linewidth]{imgs/examples_noft_ep.pdf}
    \caption{Examples of the generation outputs for Mixed and ProFusion sampling methods for their optimal metrics point in the Superclass, NoFT, and Empty Prompt (EP) setups.} \label{fig:examples_noft_ep}
\end{figure}

\clearpage

\begin{figure}[ht!]
  \centering
  \includegraphics[trim={0 5cm 0 5cm},clip,width=0.95\linewidth]{imgs/us_example_new.pdf}
  \caption{An example of a task in the user study}
  \label{fig:us_ex}
  \vspace{-0.19in}
\end{figure}

\section{Data preparation}\label{app:data}
For each concept, we used inpainting augmentations to create the training dataset. We took an original image and automatically segmented it using the Segment Anything model on top of the CLIP cross-attention maps. Then we crop the concept from the original image, apply affine transformations to it, and inpaint the background. We used $10$ augmentation prompts, different from the evaluation prompts, and sampled $3$ images per prompt, resulting in a total of $30$ training images per concept. We commit to open-source the augmented datasets for each concept after publication.

\section{User Study}\label{app:us}

An example task from the user study is shown in Figure~\ref{fig:us_ex}. In total, we collected 48,864 responses from 200 unique users for 16,000 unique pairs. For each task, users were asked three questions: 1) "Which image is more consistent with the text prompt?" 2) "Which image better represents the original image?" 3) "Which image is generally better in terms of alignment with the prompt and concept identity preservation?" For each question, users selected one of three responses: "1", "2", or "Can't decide."

\section{Complex Prompts Setting}\label{app:long_prompts}

We conduct a comparison of different sampling methods using a set of complex prompts. For this analysis, we collected 10 prompts, each featuring multiple scene changes simultaneously, including stylization, background, and outfit:

\adjustbox{max width=\linewidth}{
\begin{lstlisting}
live_long = [
  "V* in a chief outfit in a nostalgic kitchen filled with vintage furniture and scattered biscuit",
  "V* sitting on a windowsill in Tokyo at dusk, illuminated by neon city lights, using neon color palette",
  "a vintage-style illustration of a V* sitting on a cobblestone street in Paris during a rainy evening, showcasing muted tones and soft grays",
  "an anime drawing of a V* dressed in a superhero cape, soaring through the skies above a bustling city during a sunset",
  "a cartoonish illustration of a V* dressed as a ballerina performing on a stage in the spotlight",
  "oil painting of a V* in Seattle during a snowy full moon night",
  "a digital painting of a V* in a wizard's robe in a magical forest at midnight, accented with purples and sparkling silver tones",
  "a drawing of a V* wearing a space helmet, floating among stars in a cosmic landscape during a starry night",
  "a V* in a detective outfit in a foggy London street during a rainy evening, using muted grays and blues",
  "a V* wearing a pirate hat exploring a sandy beach at the sunset with a boat floating in the background",
]

object_long = [
  "a digital illustration of a V* on a windowsill in Tokyo at dusk, illuminated by neon city lights, using neon color palette",
  "a sketch of a V* on a sofa in a cozy living room, rendered in warm tones",
  "a watercolor painting of a V* on a wooden table in a sunny backyard, surrounded by flowers and butterflies",
  "a V* floating in a bathtub filled with bubbles and illuminated by the warm glow of evening sunlight filtering through a nearby window",
  "a charcoal sketch of a giant V* surrounded by floating clouds during a starry night, where the moonlight creates an ethereal glow",
  "oil painting of a V* in Seattle during a snowy full moon night",
  "a drawing of a V* floating among stars in a cosmic landscape during a starry night with a spacecraft in the background",
  "a V* on a sandy beach next to the sand castle at the sunset with a floaing boat in the background",
  "an anime drawing V* on top of a white rug in the forest with a small wooden house in the background",
  "a vintage-style illustration of a V* on a cobblestone street in Paris during a rainy evening, showcasing muted tones and soft grays",
]
\end{lstlisting}
}

The results of this comparison are presented in Figures~\ref{fig:add_long},~\ref{fig:add_long_metrics}. We observe that Base sampling may struggle to preserve all the features specified by the prompts, whereas advanced sampling techniques effectively restore them. The overall arrangement of methods in the metric space closely mirrors that observed in the setting with simple prompts.

\begin{figure}[h!]
  \centering
  \includegraphics[width=\linewidth]{imgs/long_prompts_examples.pdf}
  \caption{Additional examples of the generation outputs for different sampling methods with \textbf{complex prompts}. We highlight parts of the prompt that are missing in Base sampling while appearing in other methods.}
  \label{fig:add_long}
\end{figure}

\clearpage
\section{Dreambooth results}\label{app:dreambooth}

We conduct additional analysis of different sampling methods in combination with Dreambooth. Figure~\ref{fig:add_db_metrics} shows that Mixed Sampling still overperforms Switching and Photoswap,  while Multi-stage and Masked struggle to provide an additional improvement over the simple baseline. Figure~\ref{fig:add_db} shows that all methods allow for improvement TS with a negligent decrease in IS while Mixed Sampling provides the best IS among all samplings.

\begin{figure*}[!ht]
\centering
\begin{minipage}{.477\textwidth}
  \centering
  \includegraphics[trim={3cm 10cm 3cm 10cm},clip,width=\linewidth]{imgs/long_prompts.pdf}
  \caption{CLIP metrics for different sampling methods estimated on \textbf{complex prompts}.}
  \label{fig:add_long_metrics}
\end{minipage}
\hfill
\begin{minipage}{.477\textwidth}
  \centering
  \includegraphics[trim={3cm 10cm 3cm 10cm},clip,width=\linewidth]{imgs/db_samplings.pdf}
  \caption{CLIP metrics for different sampling strategies on top of a Dreambooth fine-tuning method.}
  \label{fig:add_db_metrics}
\end{minipage}
\end{figure*} 

\begin{figure}[h!]
  \centering
  \includegraphics[trim={0 1cm 0 1cm},clip,width=\linewidth]{imgs/db_sampling_examples.pdf}
  \caption{Additional examples of the generation outputs for different sampling methods on top of a Dreambooth fine-tuning method.}
  \label{fig:add_db}
\end{figure}

\section{PixArt-alpha \& SD-XL}\label{app:add_backbones}
We conducted a series of experiments using different backbones. For SD-XL~\cite{podell2023sdxlimprovinglatentdiffusion}, we used SVDDiff as the fine-tuning method, while PixArt-alpha~\citep{chen2023pixartalphafasttrainingdiffusion} employed standard Dreambooth training. Hyperparameters for Switching, Masked, and ProFusion were selected in the same manner as in the experiments with SD2.

Figures~\ref{fig:pixart} and~\ref{fig:sdxl} demonstrate that Mixed Sampling follows a similar pattern to SD2, improving TS without a significant loss in IS. Notably, Mixed Sampling for SD-XL achieves simultaneous improvements in both IS and TS. ProFusion exhibits behavior consistent with SD2, enhancing IS more effectively than Mixed Sampling but performing worse at improving TS while also requiring twice the computational resources. 

\begin{figure}[h]
\centering
\begin{minipage}{.49\textwidth}
  \centering
  \includegraphics[trim={3cm 10cm 3cm 10cm},clip,width=\linewidth]{imgs/pixart.pdf}
  \captionof{figure}{CLIP metrics for different sampling methods estimated on PixArt model.}
  \label{fig:pixart}
\end{minipage}%
\hfill
\begin{minipage}{.49\textwidth}
  \centering
  \includegraphics[trim={3cm 10cm 3cm 10cm},clip,width=\linewidth]{imgs/sdxl.pdf}
  \captionof{figure}{CLIP metrics for different sampling methods estimated on SD-XL model.}
  \label{fig:sdxl}
\end{minipage}
\end{figure}

\clearpage
\section{Cross-Attention Masks}\label{app:cross_attn}

\begin{figure}[h!]
  \centering
  \includegraphics[trim={3cm 0cm 3cm 0cm},clip,width=\linewidth]{imgs/cross_attention_masks.pdf}
  \caption{Visualization of the cross-attention masks for Masked sampling examples. Here, $q$ defines the thresholding quantile and $t$ the denoising step.}
  \label{fig:cross_attn_add_ex}
\end{figure}

\clearpage
\section{Additional Examples}\label{app:add_example}

\begin{figure}[h!]
  \centering
  \includegraphics[trim={0 2cm 0 2cm},clip,width=\linewidth]{imgs/additional_examples.pdf}
  \caption{Additional examples of the generation outputs for different sampling methods.}
  \label{fig:add_ex}
\end{figure}

\begin{figure}[h!]
  \centering
  \includegraphics[trim={0 2cm 0 2cm},clip,width=\linewidth]{imgs/additional_examples_all.pdf}
  \caption{Additional examples of the generation outputs for Mixed and ProFusion sampling methods in comparison to the main personalized generation baselines.}
  \label{fig:add_ex_all}
\end{figure}

\clearpage
\section{DINO Image Similarity}\label{app:add_dino}

We compare CLIP-IS (left column) and DINO-IS~\citep{oquab2024dinov2learningrobustvisual} (right column) in Figures~\ref{fig:profusion_photoswap_dino},~\ref{fig:all_methods_dino}. We observe that despite the choice of metric, different sampling techniques and finetuning strategies have the same arrangement. The most noticeable difference is that SVDDiff superiority over ELITE and TI is more pronounced. That strengthens our motivation to select SVDDiff as the main backbone.

\begin{figure}[h]
\centering
\begin{minipage}{.49\textwidth}
  \centering
  \includegraphics[trim={3cm 10cm 3cm 10cm},clip,width=\linewidth]{imgs/profusion_photoswap.pdf}
\end{minipage}%
\hfill
\begin{minipage}{.49\textwidth}
  \centering
  \includegraphics[trim={3cm 10cm 3cm 10cm},clip,width=\linewidth]{imgs/profusion_photoswap_dino.pdf}
\end{minipage}
\caption{Pareto frontiers curves for Photoswap~\citep{photoswap} and ProFusion~\citep{profusion}.}\label{fig:profusion_photoswap_dino}
\end{figure}

\begin{figure}[h]
\centering
\begin{minipage}{.49\textwidth}
  \centering
  \includegraphics[trim={3cm 10cm 3cm 10cm},clip,width=\linewidth]{imgs/all_methods.pdf}
\end{minipage}%
\hfill
\begin{minipage}{.49\textwidth}
  \centering
  \includegraphics[trim={3cm 10cm 3cm 10cm},clip,width=\linewidth]{imgs/all_methods_dino.pdf}
\end{minipage}
\caption{The overall results of different sampling methods against main personalized generation baselines.}\label{fig:all_methods_dino}
\end{figure}

\end{document}
