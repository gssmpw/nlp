\begin{table*}[ht]
\small
\centering
\begin{tabular}{p{4cm}|p{11cm}}
\toprule
\textbf{Category}                               & \multicolumn{1}{c}{\textbf{Description}}                                     \\ \midrule
Misinterpretation (associated with Error of Law)                               & The LLM misinterprets or omits some part/entirety of the legal context, question or the options (or a combination of them). This usually leads to the wrong reasoning and selection of wrong conclusion. The following error instances fall under the taxon of misinterpretation: 1. Misunderstanding the legal rules. 2. Misunderstanding the legal situation/issue at hand. 3. Omission of parts of the provided context while reasoning. 4. Incompletely applying a legal rule. 5. Incorrectly applying the legal rule. 6. Wrong assumptions derived from the provided context.                                                                             \\ \midrule
Irrelevant Premise (associated with Error of Law)                             & The LLM generates a premise which is not relevant in solving the question or that it may divert the reasoning chain from solving the question correctly. An premise may be logically valid and factually true but the absence of this premise can still lead to the correct conclusion.                         \\ \midrule
Factual Hallucination (associated with Error of Fact)                          & This error category covers instances where the LLM, during its reasoning process, generates information that is either inconsistent with the facts of the given legal scenario or is entirely fabricated with no basis in reality.                                                                                                                \\ \bottomrule
\end{tabular}
\caption{Error taxonomy for the Premise-level steps. The taxonomy has been developed with consideration for the types of errors that a human reasoner might commit when constructing a rationale for a given legal scenario. Error of Law and Error of Fact are explained in \citep{cornell2024mistake, cornell2024mistake_fact, oreilly2012errors, wilberg2023mistake}. Some fine-grained error instances of the `Misinterpretation' category are shown in Tables \ref{table:human_annotation_example_initial}, \ref{table: misinterpretation-2}, \ref{table: misinterpretation-3} and \ref{table: misinterpretation-4}.}.
\label{table:premise_errors}
\end{table*}