\begin{table}[t]
\small
\scriptsize  % Use a smaller font size
\resizebox{0.9\linewidth}{!}{
\begin{tabular}{p{0.85\columnwidth} }
\toprule
    \textbf{Examples of a reasoning step and final conclusion evaluated by GPT-4o based auto-evaluator} \\ \midrule

       \multirow{3}{*}{}{\textbf{Premise 4:} Marla's Intent: Marla's intent is somewhat ambiguous. She is attending a two-year program and is unsure if she wants to be a stylist. She has considered staying in Denver to work if she likes the program, or leaving the program and looking for work in Denver if she doesn't like it.} \\ \textbf{Auto-Evaluator Evaluation and Error Explanation:} Premise 4: Misinterprets Marla's intent.\\The premise incorrectly suggests ambiguity in Marla's intent. According to the legal context, Marla's plans are open-ended, meaning she does not have definite plans to leave Denver. The ambiguity mentioned does not align with the legal context's explanation of "indefinite intent". \\ \textbf{Error Category Label:} "Misinterpretation" \\ 
    \midrule

        \multirow{4}{*}{}{\textbf{Predicted Conclusion:} Given that Marla resides in Colorado and has no definite plans to leave, she meets the domicile test's requirement of residing with the intent to remain indefinitely.

Final answer: D,}\\ \textbf{Extracted Ground-Truth Conclusion:} D.  is domiciled in Colorado, because she resides there with the intent to remain indefinitely. \\ \textbf{Auto-Evaluator Evaluation and Error Explanation:} Based on the provided conditions and the information given:

\\- The extracted conclusion's final answer is D.
\\- The ground truth option is also D.
\\- The content of the extracted conclusion and the ground truth option match: both state that Marla is domiciled in Colorado because she resides there with the intent to remain indefinitely.
\\- There is an error in the premises (misinterpretation of Marla's intent).\\\\
Given these points, the appropriate classification is:
\\
\textbf{CORRECT CONCLUSION FROM FALSE PREMISES} \\ \textbf{Error Category Label:} "Correct Conclusion from False Premises"\\

\bottomrule
\end{tabular}}
\caption{Example of reasoning step (premise) and conclusion evaluated by LLM-based `Auto-evaluator' (GPT-4o). The error category labels are extracted from the detailed explanations using an LLM prompted to extract error keywords.}
\label{table:example_annot}
\end{table}
