\section{Related Work}
Several definitions of a hypothesis have been proposed on the basis of first-order logic \cite[Section 2.2]{paul1993approaches}.
In particular, a hypothesis graph in weighted abduction is a proof in first-order logic consisting of literals \cite{stickel1991prolog}.
Thus, the edges between literals of a hypothesis graph express implication.
Extending this definition, we propose a definition of a hypothesis graph that is allowed to contain many-sorted constants and variables and arbitrarily labelled hyperedges between literals.

Dialogue protocols have also been proposed in which a system and a user cooperate in seeking for a proof or a plausible hypothesis.
An inquiry dialogue \cite{black2009inquiry} is one of such protocols. Its goal is to construct a proof of a given query such as `he is guilty' in propositional logic.
Using this protocol, the user can obtain all minimal proofs and find the one that seems most plausible to him.
However, this cannot be applied to extended hypotheses/hypothesis graphs, because there can be an infinite number of possible extended hypotheses/hypothesis graphs (see also Examples \ref{example: semantics-based protocol not terminate} and \ref{example: infinite syntax-based dialogue}, \textit{infra}).
The work \cite{motoura2021cooperative} proposes user-feedback functions on nodes in extended hypothesis graphs presented by the abductive reasoner.
However, these functions lack theoretical and empirical supports. 
In contrast, we theoretically prove that our protocols enjoy the halting and the convergence properties.