\begin{lstlisting}[language=json,numbers=none]
Reformat Answer:

**Clear Cell Tumors: An Overview**
Clear cell tumors are a subset of surface epithelial-stromal tumors, accounting for approximately 6% of ovarian cancer cases. These neoplasms can also occur in the pancreas and salivary glands.
**Key Characteristics:**
1. **Malignancy:** Most clear cell tumors are malignant, with benign and borderline variants being rare.
2. **Cystic Neoplasms:** Typically, clear cell tumors present as cystic neoplasms with polypoid masses that protrude into the cyst.
3. **Microscopic Pathology:** On microscopic examination, these tumors are composed of cells with clear cytoplasm, which contains glycogen. Hobnail cells, characterized by the secretion of glycogen, are also present.
4. **Histological Patterns:** The pattern of clear cell tumors may be glandular, papillary, or solid.
**Diagnostic Considerations:**
When diagnosing clear cell tumors, it is essential to consider the following:
1. **Imaging Studies:** Imaging studies, such as ultrasound or CT scans, can help identify the presence of cystic neoplasms with polypoid masses.



Scores:
SkyworkLlama:-26.6250
SkyworkGemma:-6.9687
URM:-3.3092
QRM:4.6931
Internlm2:1.0605
\end{lstlisting}