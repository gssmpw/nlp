\begin{lstlisting}[language=json,numbers=none]
Original Answer:

Clear cell tumors are part of the surface epithelial-stromal tumor group of Ovarian cancers, accounting for 6% of these neoplastic cases. Clear cell tumors are also associated with the pancreas and salivary glands.
Benign and borderline variants of this neoplasm are rare, and most cases are malignant.
Typically, they are cystic neoplasms with polypoid masses that protrude into the cyst.
On microscopic pathological examination, they are composed of cells with clear cytoplasm (that contains glycogen) and hob nail cells (from which the glycogen has been secreted).
The pattern may be glandular, papillary or solid.


















Scores:
SkyworkLlama:-19.7500
SkyworkGemma:-6.6562
URM:3.4967
QRM:17.7778
Internlm2:0.1964
\end{lstlisting}