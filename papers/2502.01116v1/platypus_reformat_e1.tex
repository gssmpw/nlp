\begin{lstlisting}[language=json,numbers=none]
Reformat Answer:

To find the probability of the spinner landing on $C$, we need to subtract the probabilities of the spinner landing on $A$ and $B$ from $1$, since the sum of the probabilities of all possible outcomes is $1$. We can write this as an equation: $P(C) = 1 - P(A) - P(B)$.
Given that $P(A) = \frac{1}{3}$ and $P(B) = \frac{5}{12}$, we can plug these values into the equation and simplify.
1. **Subtract the probabilities of A and B from 1:**
   - $P(C) = 1 - P(A) - P(B)$
   - $P(C) = 1 - \frac{1}{3} - \frac{5}{12}$
2. **Find a common denominator for the fractions:**
   - The least common multiple of $3$ and $12$ is $12$, so we can rewrite $\frac{1}{3}$ as $\frac{4}{12}$.
   - $P(C) = 1 - \frac{4}{12} - \frac{5}{12}$
3. **Combine the fractions:**
   - $P(C) = \frac{12}{12} - \frac{4}{12} - \frac{5}{12}$
   - $P(C) = \frac{12 - 4 - 5}{12}$
4. **Simplify the fraction:**
   - $P(C) = \frac{3}{12}$
   - We can reduce this fraction by dividing the numerator and denominator by $3$.
5. **Reduce the fraction:**
   - $P(C) = \frac{1}{4}$
Therefore, the probability of the spinner landing on $C$ is $\frac{1}{4}$.

Scores:
SkyworkLlama:12.0625
SkyworkGemma:-4.8437
URM:8.7929
QRM:30.5980
Internlm2:2.5214
\end{lstlisting}