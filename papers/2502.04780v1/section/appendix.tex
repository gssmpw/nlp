\appendix
\section{Detailed Pipeline}
\label{ap:pipeline}
% \subsection{Pipeline of Problem Solving Setting}

Given the wrong answer problem set $\mathcal{W} = \{(x_i, y_i)\}_{i=1}^w$,In each iteration, we first select the agent to be optimized. For instance, as shown in the diagram, the selected agent is the physicist ($A$). The external agent provides feedback 
$
f_i = P_{\theta^{(\text{ext})}}(\cdot | x_i, \hat{a}_i, y_i)
$ 
based on the question $x_i$, the original response $\hat{a}_i$, and the correct answer $y_i$.

The physicist then regenerates the solution by incorporating the feedback:  
$
\hat{a}_i^{r} = P_{\theta^{(A)}}(\cdot | x_i, \hat{y}_i, f_i).
$  

To ensure clarity and coherence, the regenerated response $\hat{a}_i^{r}$ is subsequently rephrased to produce $\hat{y}_i^{\text{final}}$, making it appear as if derived directly through problem-solving without mentioning any modifications or feedback. This updated response is then used in subsequent collaborations with other agents to refine the overall solution further.

\begin{algorithm}[htbp]
\caption{Detailed Pipeline of \model{}}
\begin{algorithmic}[1]
\STATE \textbf{Input:} A group of agents $A^{(1)},\cdots,A^{(K)}$, the system's  topological graph $\mathcal{G}$, maximum solution generation tries $\max_\text{sol}$, maximum feedback generation tries $\max_\text{f}$, maximum  regeneration tries $\max_\text{re}$. 
An initial dataset of problems $x$ with answer $y:\mathcal{D} = \{(x_i, y_i)\}_{i=1}^D$, total number of fine-tuning Iterations $T$.
\STATE \textbf{Initialize:} Initialize policy parameters $\theta^{(k)}$ for each agent $A^{(k)}$, $k = 1, 2, \dots, K$. $\theta^{(c)}$  for Critic Agent $A^{(c)}$

\FOR{Fine-tuning Iteration $\text{t}_\text{ft} =1,\cdots,T$}    
    \WHILE{$\text{t}_\text{sol} \leq \max_\text{sol}$}
         \STATE $a_i^{(k)}=\mathcal{P}_{\theta^{(k)}}(\cdot|x_i,\textbf{a}_i^{\mathrm{Pre}(A^{(k)})})$.
        
        \STATE $\hat{y}_i=a_i^{(K)}$    
        \FOR{each agent $k = 1, 2, \dots, K$}
            
        
            \STATE $\mathcal{C}_{\text{t}_\text{ft}}^{(k)}\leftarrow \{(x_i, a_i^{(k)}| i \in [1,D] \land \hat{y}_i=y_i )\}$ 
            \STATE $\mathcal{W}_{\text{t}_\text{ft}}^{(k)}\leftarrow \{(x_i, a_i^{(k)}| i \in [1,D] \land \hat{y}_i\neq y_i )\}$ 
            \FOR {$x_i \in W_t^{(k)}$}
                \WHILE{$\text{t}_\text{f} \leq \max_\text{f}$}
                
                    \STATE $ f_i^{(k)}=\mathcal{P}_{\theta^{(c)}}(\cdot | x_i,a_i^{(k)},y_i)$
                    \WHILE{$\text{t}_\text{re} \leq \max_\text{re}$}
                   
                    \STATE {\small $ a_i^{(k),re}=\mathcal{P}_{\theta^{(k)}}(\cdot | x_i,a_i^{(k)},f_i^{(k)})$}
                    \STATE { \scriptsize $ \mathcal{S}_j=\mathrm{Sus}(A^{(k)}) \cap \mathrm{Pre}(A^{(j)})$, $j \in \mathrm{Sus}(A^{(k)}) $}
                    \STATE { \scriptsize $ a_i^{(j),re}=\mathcal{P}_{\theta^{(j)}}(\cdot | x_i,\textbf{a}_i^{\mathrm{Pre}(A^{(j)})\setminus\mathcal{S}_j} \cup \textbf{a}_i^{\mathcal{S}_j,re})$}
                    \STATE { $\hat{y}_i^{re}= a_i^{(K),re}$}
                    \IF{ $\hat{y}_i^{re}=y_i$}
                        \STATE $\mathcal{C}_{\text{t}_\text{ft}}^{(j)}\leftarrow \{(x_i, a_i^{(j),re} \}$, $j=k, \cdots, K$
                        \STATE \textbf{break while}
                    \ENDIF
                    \ENDWHILE
                \ENDWHILE
            \ENDFOR
        \ENDFOR
        
    \ENDWHILE
    \STATE $\theta^{(k)}_{\text{t}_\text{ft}} \leftarrow \textbf{Standard SFT on }\mathcal{C}_{\text{t}_\text{ft}}^{(k)}$, $k=1,\cdots,K$
\ENDFOR
\end{algorithmic}
\end{algorithm}


% \subsection{Pipeline of Actor-Critic Setting}
% Given an initial dataset $\mathcal{D} = \{(x_i, y_i)\}_{i=1}^D$, the \textbf{Actor Agent} $A$ generates an initial solution $\hat{y}_i = P_{\theta^{(A)}}(\cdot | x_i)$. The Critic Agent evaluate the initial solutions, framing the problem as a binary classification problem.

% For  \textbf{positive true (PT)}  solutions and judgments, the data are added to Actor Agent's training dataset $\mathcal{C}^{(A)}$ and Critic Agent's training dataset $\mathcal{C}^{(C)}$.  

% For  \textbf{positive false (PF)}  cases, the correct solution is added to Actor Agent's training dataset $\mathcal{C}^{(A)}$. Since Critic Agent mistakenly judge this solution as wrong, an external agent is prompted to inform the critic of its incorrect judgment, which leads Critic Agent to correct and regenerate its response.

% For \textbf{ negative true (NT)}  cases, Critic Agent $C$ generates feedback $f_i = P_{\theta^{(C)}}(\cdot | x_i, \hat{y}_i)$. This feedback is used by the actor to regenerate the solution as $\hat{y}_i^{r} = P_{\theta^{(A)}}(\cdot | x_i, \hat{y}_i, f_i)$.  


% For \textbf{negative false (NF)} cases, since Critic Agent fails to identify errors in the solution, an external agent is prompted to notify Critic Agent of its incorrect judgment, encouraging it to regenerate its evaluation.
% \begin{figure}[htbp]
%     \centering
%     \includegraphics[width=0.6\linewidth]{figure/actor_critic.pdf}
%     \caption{Pipeline of Actor-Critic Setting}
%     \label{fig:actor_critic}
% \end{figure}
\section{Detailed Competitive Settings}
We follow the settings of {\color{mediumpurple}\textsc{NegotiationArena}} Platform~\citep{bianchi2024well}. 
\subsection{Resource Exchange Scenario}
In this game, each agent has access to a set of resources and a goal. For example, an agent has access to resources 25 Xs and 5 Ys. The agent might have the goal of maximizing its total resources. Since this goal is very general, it could bring the models to employ different strategies (e.g., a model might want to diversify the resources it has or maximize only an individual resource). Both agents have multiple turns that they can use to make each other proposals until one of the two accepts a proposal. The game ends on acceptance or when the maximum number of turns finishes.

\subsection{Multi-Turn Ultimatum Game}
The Ultimatum game~\citep{sanfey2003neural} is a classical game used in economics to study aspects of human behavior, such as fairness and rationality. It involves two agents agreeing on a split of resources (often money). One agent is given all the game's resources and proposes a split of the resources. The second agent can either accept or reject the proposal, which means both agents lose all resources. In the classical Ultimatum game the rational actions correspond to (1) the first agent offering to give 1 unit of resource (i.e., the bare minimum) and (2) the second agent accepting any proposal that is greater than 0 units. The classical Ultimatum game has one round of negotiation (i.e. agent 2 can only decide whether or not to accept agent 1's first offer). In our version of the game, the game can go on for more turns (e.g. agents can make multiple counteroffers) and both players can accept the opponent's offer.

\subsection{Seller and Buyer Scenario}
We introduce a seller and buyer game involving two agents, one looking to sell a set of resources and one looking to buy them, similar to other approaches in the literature (e.g., \citep{he2018decoupling}). We imbue agents with some beliefs about the object being sold, but unlike the ultimatum game, the seller and buyer game is an incomplete information game, i.e., players do not have complete information about other players (e.g., their beliefs). Only the seller is aware of the production cost of the object, and only the buyer is assigned and is aware of their willingness to pay for the object. Given these beliefs, the seller and the buyer are prompted to sell and buy the object, respectively. The seller starts first: reproducing a scenario in which the object is already on sale.




\section{Dataset Details}\label{ap:dataset}
\subsection{Dataset Split Statistics}
In this work, we use three datasets for evaluating the performance of our model: Massive Multitask Language Understanding (MMLU)~\citep{hendrycks2020measuring}, Graduate-Level Google-Proof Q\&A (GPQA)~\citep{rein2023gpqa}, and Theorem-Driven Question Answering (TheoremQA)~\citep{chen2023theoremqa}. These datasets contain a variety of question types, with a focus on college-level physics and chemistry problems that remain difficult and present room for improvement in performance with large language models.


The dataset was split into training and test sets with a 2:1 ratio, and the data distribution for each dataset is shown in Table~\ref{tab:dataset_split}.
\begin{table}[htbp]
\centering
\caption{Dataset Split Statistics.}
\begin{tabular}{l|cc|cc}
\toprule
\textbf{Task}          &\multicolumn{2}{c|}{College Physics}  &\multicolumn{2}{c}{College Chemistry}   \\ \midrule
\textbf{Dataset} & \textbf{Train Size} & \textbf{Test Size}& \textbf{Train Size} &\textbf{Test Size} \\ \midrule
MMLU             &         68          &          34         &         66           &    34        \\
GPQA             &         57          &          29         &         62           &    31        \\ 
TheoremQA        &         87          &          44         &          -           &     -       \\ 
\bottomrule
\end{tabular}

\label{tab:dataset_split}
\end{table}
\subsection{Finetuning Dataset Statistics}
For each experiment, we specify the Trajectories Augmentation Ratio and whether ground truth answers are used during the training process.
We summarize the setup for each experiment in Table~\ref{tab:problem-solving}.
\begin{table}[htbp]
\centering
\caption{Finetuning Dataset Statistics.}
\label{tab:problem-solving}
\begin{tabular}{l|l|cc}
\toprule
\textbf{Model   }              & \textbf{Task} &\textbf{Augmentation Ratio} & \textbf{Ground Truth Used} \\ \midrule

\multirow{4}{*}{GPT-3.5-turbo} & Problem-Solving(College-Physics)  & 108.93\% & Yes\\ 
                               & Problem-Solving(College-Chemistry)& 157.78\% & Yes  \\ 
                               & Problem-Solving(PubMedQA)         & 13.09\%  & Yes   \\ 
                               & Actor-Critic                      & 136.46\%  & No    \\ 
\midrule
\multirow{4}{*}{GPT-4o-mini}   & Problem-Solving(College-Physics)  & 38.89\%  & Yes    \\ 
                               & Problem-Solving(College-Chemistry)& 63.79\%  & Yes   \\ 
                               & Problem-Solving(PubMedQA)         & 12.85\%  & Yes    \\
                               & Actor-Critic                      & 14.94\%  & No      \\ 
\bottomrule
\end{tabular}
\end{table}


\section{Additional Experiment Result}\label{ap:result}
In this section, we present additional experiments conducted in a competitive setting to assess the generalization of \model{}. These results demonstrate the adaptability of \model{} across various configurations. 
\begin{figure}[h!]
    \centering
    \includegraphics[width=0.7\linewidth]{figure/resource_35_15.pdf}
    \caption{Resource Exchange Game with Initial Resource Player 1: 35Xs + 15Ys, Player 2: 15Xs + 35Ys. Win Rate in decisive games and Payoff in all games. We show Player 2 Win rate/payoff in all cells.}
    \label{fig:resource_35_15}
\end{figure}

\begin{minipage}{0.48\textwidth}
    \centering
    \includegraphics[width=0.9\linewidth]{figure/buysell_30_70.pdf}
    \captionof{figure}{Final Selling Price for a Seller\&Buyer with object valuations of 30 and 70. A higher number means the seller gets a greater payoff.}
    \label{fig:buysell_30_70}
\end{minipage}
\hfill
\begin{minipage}{0.48\textwidth}
    \centering
    \includegraphics[width=0.9\linewidth]{figure/ultimate_bar_1000.pdf}
    \captionof{figure}{Player 1's payoff in the Ultimatum game with Initial Resource settings of 1000. \model{} as Player 1 can effectively secure a higher share of the split.}
    \label{fig:ultimate_1000}
\end{minipage}


% \begin{figure}[h!]
%     \centering
%     \includegraphics[width=0.5\linewidth]{figure/buysell_30_70.pdf}
%     \caption{Final Selling Price for a Seller\&Buyer with object valuations of 30 and 70. A higher number means the seller gets a greater payoff.}
%     \label{fig:buysell_30_70}
% \end{figure}

% \begin{figure}[h!]
%     \centering
%     \includegraphics[width=0.5\linewidth]{figure/ultimate_bar_1000.pdf}
%     \caption{Player 1's payoff in the Ultimatum game with Initial Resource settings of 1000. \model{} as Player 1 can effectively secure a higher share of the split.}
%     \label{fig:ultimate_1000}
% \end{figure}




\section{Agent Prompts}
\subsection{Problem Solving Setting}
\begin{codebox}[title= Prompts for College-Physics Task]

\textbf{System\_prompt} = '''You are part of a team with multiple experts from different disciplines. Your team aims to solve a given cross-discipline problem collectively.

The team is composed of three experts:

1. The Physicist

    Role Definition: You are a physicist with a specialization in the field of college-level physics. Your vast knowledge covers multiple aspects of physics including classical mechanics, thermodynamics, electromagnetism, quantum mechanics, and statistical physics. You understand these topics in depth and have the ability to explain them in a way that is easily comprehensible to those less familiar with them.

    Responsibility: Focus on contributing physics-specific insights and collaborate with the mathematician to help develop and validate mathematical models.**Do not perform calculations or solve the entire problem**. Your goal is to provide a clear explanation of the physics, leaving calculations to the mathematician.

    Principles: Emphasize empirical, systematic, and data-driven approaches while fostering curiosity, innovation, and ethical scientific practices.

2. The Mathematician

    Role Definition: You are a mathematician, specializing in the broad and complex field of mathematics at the college level. Your expertise ranges from pure mathematical theory, including algebra, calculus, geometry, number theory, and statistics, to applied mathematics such as optimization and probability theory. You have an innate ability to abstract and generalize problems, solving them with elegance and precision. You excel at creating mathematical models that represent real-world situations and can interpret the implications of those models. You are not only well-versed in complex equations and proofs, but also experienced in conveying these concepts to others through teaching.

    Responsibilities: Apply mathematical reasoning to analyze and address complex, cross-disciplinary problems; Collaborate with the physicist to refine mathematical models and validate their conclusions; Convey mathematical insights in a clear manner to facilitate team decision making.

    Principles: Foster a culture of analytical thinking and evidence-based decisions; Encourage an atmosphere of curiosity, innovation, and continuous learning; Maintain high mathematical integrity and respect for varying perspectives.

3. The Final Answer Synthesizer

    Role Definition: You are the Final Answer Synthesizer, an integrative role in the team responsible for coalescing the insights provided by the experts. With a clear understanding of the different disciplines, you effectively distill the responses from the physicist and the mathematician into a coherent, final solution. Your role involves keenly interpreting expert input, synthesizing various problem-solving approaches, and presenting a clear, well-rounded answer that incorporates the collective wisdom of the team. 
    
    Responsibility: summarize the solutions; give a final answer.
    
    Principles: make sure to give a specific answer to the given task.'''
\vspace{1em}

\textbf{Physicist\_prompt} =  '''Your role is the physicist.
Here is the given problem:
"{question}"
Your task is **only to explain** the relevant physics concepts and principles that apply to this problem. '''

\vspace{1em}
\textbf{Mathematician\_prompt} = '''Your role is the mathematician. 
Here is the given problem:
"{question}"
Here is the response from the physicist:
"\{agent\_1\_response\}"
Please give your opinion on how to solve the problem in consideration of the response from the physicist.'''

\vspace{1em}
\textbf{Summarizer\_prompt }= '''Your role is the Final Answer Synthesizer. 
Here is the given problem:
"{question}"
Here is the response from the physicist:
"\{agent\_1\_response\}"
Here is the response from the mathematician:
"\{agent\_2\_response\}"

Please provide a final answer to the given problem. \{format\_prompt\}'''
\end{codebox}


\begin{codebox}[title= Prompts for College-Chemistry Task]

\textbf{System\_prompt} = '''You are part of a team with multiple experts from different disciplines. Your team aims to solve a given cross-discipline problem collectively.

The team is composed of three experts:

1. The Chemist

    Role Definition: You are a chemist with a specialization in the field of college-level chemistry. Your vast knowledge covers multiple aspects of chemistry including organic, inorganic, physical, analytical, and biochemistry. You understand these topics in depth and have the ability to explain them in a way that is easily comprehensible to those less familiar with them.

    Responsibility: Focus on contributing chemistry-specific insights and collaborate with the mathematician to help develop and validate mathematical models.**Do not perform calculations or solve the entire problem**. Your goal is to provide a clear explanation of the chemistry concepts, leaving calculations to the mathematician.

    Principles: Emphasize empirical, systematic, and data-driven approaches while fostering curiosity, innovation, and ethical scientific practices.

2. The Mathematician

    Role Definition: You are a mathematician, specializing in the broad and complex field of mathematics at the college level. Your expertise ranges from pure mathematical theory, including algebra, calculus, geometry, number theory, and statistics, to applied mathematics such as optimization and probability theory. You have an innate ability to abstract and generalize problems, solving them with elegance and precision. You excel at creating mathematical models that represent real-world situations and can interpret the implications of those models. You are not only well-versed in complex equations and proofs, but also experienced in conveying these concepts to others through teaching.

    Responsibilities: Apply mathematical reasoning to analyze and address complex, cross-disciplinary problems; Collaborate with the chemist to refine mathematical models and validate their conclusions; Convey mathematical insights in a clear manner to facilitate team decision making.

    Principles: Foster a culture of analytical thinking and evidence-based decisions; Encourage an atmosphere of curiosity, innovation, and continuous learning; Maintain high mathematical integrity and respect for varying perspectives.

3. The Final Answer Synthesizer

    Role Definition: You are the Final Answer Synthesizer, an integrative role in the team responsible for coalescing the insights provided by the experts. With a clear understanding of the different disciplines, you effectively distill the responses from the chemist and the mathematician into a coherent, final solution. Your role involves keenly interpreting expert input, synthesizing various problem-solving approaches, and presenting a clear, well-rounded answer that incorporates the collective wisdom of the team. 
    
    Responsibility: Summarize the solutions; give a final answer.
    
    Principles: Make sure to give a specific answer to the given task.'''
\vspace{1em}

\textbf{Chemist\_prompt} =  '''Your role is the chemist.
Here is the given problem:
"{question}"
Your task is **only to explain** the relevant chemistry concepts and principles that apply to this problem. **Do not** perform any calculations or try to find the final solution. Your role is to explain the chemical reasoning, such as reactions or principles, but refrain from solving the equations or completing the solution. Leave the mathematical work to the mathematician.'''

\vspace{1em}
\textbf{Mathematician\_prompt} = '''Your role is the mathematician. 
Here is the given problem:
"{question}"
Here is the response from the physicist:
"\{agent\_1\_response\}"
Please give your opinion on how to solve the problem in consideration of the response from the physicist.'''

\vspace{1em}
\textbf{Summarizer\_prompt }= '''Your role is the Final Answer Synthesizer. 
Here is the given problem:
"{question}"
Here is the response from the physicist:
"\{agent\_1\_response\}"
Here is the response from the mathematician:
"\{agent\_2\_response\}"

Please provide a final answer to the given problem. \{format\_prompt\}'''
\end{codebox}


\begin{codebox}[title= Prompts for PubMedQA Task]

\textbf{System\_prompt} = '''You are part of a team of experts working collaboratively to solve science-related yes/no questions using contextual evidence. The goal is to analyze the provided question and context thoroughly to determine the correct answer.  

The team is composed of two roles:  

1. The Context Analyst  

**Role Definition:** You are the Context Analyst, skilled in extracting and summarizing key information from the given context to address the question.  

**Responsibility:** Read the provided question and context carefully, then summarize the most relevant information needed to answer the question. Your summary should focus on the evidence directly supporting or refuting the question’s claim.  

**Principles:** Prioritize clarity and relevance. Extract only the essential details from the context that will help guide the next agent in making an evidence-based decision.  

2. The Problem Solver

**Role Definition:** You are the Problem Solver, responsible for interpreting the Context Analyst's summary and determining the correct yes/no answer based on evidence.  

**Responsibility:** Review the question and the Context Analyst's summary, analyze the evidence, and construct a concise final response (yes or no) supported by clear reasoning. If the context does not provide sufficient evidence to make a confident decision, clearly state that the evidence is inconclusive. 

**Principles:** Ensure logical coherence, accuracy, and completeness. Justify your answer with reasoning directly tied to the summarized evidence.  
'''
\vspace{1em}

\textbf{Analyst\_prompt} =  '''Your role is the Context Analyst.

Here is the provided context:
"\{context\}"

Your task is to carefully read through this context and summarize the main points relevant to the question. Only provide essential information that would help address the question.'''


\vspace{1em}
\textbf{Solver\_prompt} =  '''Your role is the Problem Solver.

Here is the question:
"\{question\}"

Here is the summary from the Context Analyst:
"\{agent\_1\_response\}"

Please analyze the question, using the summary to answer the problem.  \{format\_prompt\}'''
\end{codebox}

\subsection{Actor-Critic Setting}
\begin{codebox}[title= Prompts for Actor Agent and Regeneration]

\textbf{System\_prompt }='''You are a scientist working on solving science-related yes/no questions using contextual evidence. '''
\vspace{1em}

\textbf{Actor\_prompt} = '''You are supposed to provide a solution to a given problem. 

Here is the given context:
"\{context\}"

Problem:
"\{question\}"

Please provide yes, no or maybe to the given problem. \{format\_prompt\}'''

\vspace{1em}

\textbf{Actor\_regenerate\_prompt} = '''You are supposed to provide a solution to a given problem. 
Here is the given context: "\{context\}"

Problem: "\{question\}"

Here is your original response:
\{original\_response\}

Here is the feedback for your original response:
"\{feedback\}"

Please first consider the feedback and then update your opinion on how to solve the problem.
Please provide a final answer to the given problem. \{format\_prompt\}'''


\end{codebox}


\begin{codebox}[title= Prompts for Judgment Agent]

\textbf{System\_prompt }= '''Below is a yes/no question and a prediction. 
You are a critical and creative scientist tasked with evaluating the prediction. Your responsibility is to thoroughly investigate the reasoning behind the prediction. If the original response is entirely correct, output "True." If you identify any errors, inconsistencies, or flaws in the reasoning, output "False."
'''

\vspace{1em}
\textbf{Judgment\_prompt} = '''Here is the given context: "\{context\}"

Problem: "\{question\}"

Original response: \{original\_response\}

Provide your response in the following format:

1. Analysis: 
Provide a detailed and objective critique of the reasoning in the language model’s answer. Discuss whether the logic, assumptions, and conclusions are valid. Highlight any errors, alternative perspectives, or missing considerations.

2. Decision: 
'Opinion: True or False' (without quotes) where Opinion is your final Decision based on your analysis. Your Decision should be either "True" or "False".
Ensure this conclusion directly reflects the correctness of the reasoning in the language model’s answer.
'''


\end{codebox}

\begin{codebox}[title= Prompts for Critic Agent]

\textbf{System\_prompt }= '''Below is a biomedical yes/no question, the context, and a prediction.
You are a critical and creative scientist. Your job is to investigate the prediction. Critically go through reasoning steps, and see if there is a
reason why the prediction could be incorrect. Use the Janusian Process, think about whether alternative answers could be true.'''

\vspace{1em}
\textbf{Critic\_prompt} = '''Here is the given context: "\{context\}"

Question:  "\{question\}"

Answer by the language model:  \{original\_response\}
'''
\end{codebox}
\begin{codebox}[title= Prompts for Rephrasing]

\textbf{System\_prompt }=  '''Rephrase the following solution process to ensure that it appears as though the solution was arrived at directly, with no traces of mistakes or corrections. Retain all key steps and avoid generating any new content. The focus should be on smoothing the flow and ensuring logical consistency, without altering the meaning or introducing additional information.
'''
\vspace{1em}

\textbf{Rephrase\_prompt} = '''Here is the problem and the original solution process:
Problem: \{question\}

Original Solution Process:\{original\_response\}

Please output the rephrased solution process'''
\end{codebox}

\subsection{Competitive Setting}
\begin{codebox}[title= Prompts for Resource Exchange]
              \textbf{System\_prompt }= '''You are playing a strategic game of trading resources with another player whose resources you have no knowledge about.\\RULES:\\```\\1. You can either:\\A) Accept the trade by saying:\\<player answer> ACCEPT </player answer>\\<newly proposed trade> NONE </newly proposed trade>\\B) Reject and propose a new trade (you can only trade integer amounts, not decimals):\\<player answer> NONE </player answer>\\<newly proposed trade> Player RED Gives item1: amount, item2: amount, ... | Player BLUE Gives item1: amount, item2: amount, ... </newly proposed trade>\\C) Don't accept or propose anything and wait for a new offer:\\<player answer> NONE </player answer>\\<newly proposed trade> NONE </newly proposed trade>\\Note: the game will end if one of the players accepts. This means that you have to be careful about both accepting and proposing a trade.\\2. You are allowed at most 3 proposals of your own to complete the game, after which you can only ACCEPT or NONE.\\DO NOT propose a new trade after 3 proposals. Your limit for proposals is 3.\\3. You can reason step by step by using the following format:\\<reason> [add reasoning] </reason>\\Add as much text as you want. This information will not be sent to the other player. \\It is just for you to keep track of your reasoning.\\4. At each turn send messages to each other by using the following format:\\<message> [add message] </message>\\You can decide if you want disclose your resources and goals in the message.\\```\\Here is what you have access to:\\```\\Resources available in the game: X, Y\\<my resources> X: 25, Y: 5 </my resources>\\<my goals> X: 15, Y: 15 </my goals>\\<my name> Player RED </my name>\\```\\All the responses you send should contain the following and in this order:\\```\\<my name> [add here] </my name>\\<my resources> [add here] </my resources>\\<my goals> [add here] </my goals>\\<reason> [add here] </reason>\\<player answer> [add here] </player answer>\\<message> [add here] </message\\<newly proposed trade> [add here] </newly proposed trade>\\```\\Please be sure to include all.\\More resources in general are always better.\\This is the last round of trading. There are no future rounds after this one.\\\\'''
\end{codebox}
\begin{codebox}[title= Prompts for Sell\&Buy]
              \textbf{System\_prompt }= '''You are playing game where you are buying or selling an object. There is only one object for sale/purcahse.\\Player RED is going to sell one object. Player BLUE gives ZUP to buy resources.\\RULES:\\```\\1. You must always respond with:\\    A) Propose a trade with (you can only trade in integer amounts, not decimals):\\    <player answer> PROPOSAL </player answer>\\    <newly proposed trade> Player RED Gives X: amount, ...| Player BLUE Gives ZUP: amount </newly proposed trade>\\    B) Accept the trade by saying:\\    <player answer> ACCEPT </player answer>\\    <newly proposed trade> NONE </newly proposed trade>\\    C) Reject and end the game:\\    <player answer> REJECT </player answer>\\    <newly proposed trade> NONE </newly proposed trade>\\    Note: The game will end if one of the players ACCEPT OR REJECT. This means that you have to be careful about both accepting, rejecting and proposing a trade.\\2. You are allowed at most 4 proposals of your own to complete the game, after which you can only reply with ACCEPT or REJECT.\\DO NOT propose a new trade after 4 proposals. Your limit for proposals is 4.\\3. You can reason step by step on why you are A) proposing, B) rejecting and C) accepting a trade with:\\<reason> [add reasoning] </reason> add as much text as you want\\This information will not be sent to the other player. It is just for you to keep track of your reasoning.\\4. At each turn send messages to each other by using the following format:\\<message>your message here</message>\\You can decide if you want disclose your resources, goals, cost and willingness to pay in the message.\\```\\Here is what you have access to:\\```\\Object that is being bought/sold: X\\<my resources> ZUP: 100 </my resources>\\<my goals> Buy resources with <ZUP>. You are willing to pay at most X: 70 ZUP for the resources. </my goals>,\\```\\All the responses you send should contain the following and in this order:\\```\\<proposal count> [add here (inclusive of current)] </proposal count>\\<my resources> [add here] </my resources>\\<my goals> [add here] </my goals>\\<reason> [add here] </reason>\\<player answer> [add here] </player answer>\\<newly proposed trade> [add here] </newly proposed trade>\\<message> [add here] </message\\```\\Please be sure to include all.\\\\You are Player BLUE.'''
\end{codebox}
\begin{codebox}[title= Prompts for Ultimatum]
              \textbf{System\_prompt }= '''You are playing a game involving a split of resources. \\Player RED starts with Dollars: 100, Player BLUE has none to trade.\\RULES:\\```\\1. Player RED has 4 moves while Player BLUE has 4 moves.\\2. You must either:\\    A) Accept the current proposal by saying:\\        <player answer> ACCEPT </player answer>\\        <newly proposed trade> NONE </newly proposed trade>\\    B) Reject the current proposal, whereby both players lose all their resources and the game ends immediately:\\        <player answer> REJECT </player answer>\\        <newly proposed trade> NONE </newly proposed trade>\\    C) Propose a new trade:\\        <player answer> NONE </player answer>\\        <newly proposed trade> Player RED Gives item1: amount | Player BLUE Gives item1: 0 </newly proposed trade>\\    Player BLUE cannot do (C) on move 4/4 and MUST ONLY answer with ACCEPT (A) or REJECT (B) but Player RED is not affected by this condition.        \\    \\    Note: the game ends immediately if you ACCEPT or REJECT.\\    \\3. If either player REJECTS, both players lose all their resources and the ends immediately.\\4. There are no future games.\\5. You can reason step by step on why you are A) proposing, B) rejecting and C) accepting a trade with:\\<reason> [add reasoning] </reason> add as much text as you want\\This information will not be sent to the other player. It is just for you to keep track of your reasoning.\\6. At each turn send messages to each other by using the following format:\\<message>your message here</message>\\You can disclose anything in the message.\\```\\Here is what you have access to:\\```\\Resources available in the game: Dollars\\<my resources> Dollars: 100 </my resources>\\```\\All the responses you send should contain the following and in this order:\\```\\<my name> [add here] </my name>\\<move> [add here] / [add here]  </move> \\<my resources> [add here] </my resources>\\<reason> [add here] </reason>\\<player answer> [add here] </player answer>\\<message> [add here] </message\\<newly proposed trade> [add here] </newly proposed trade>\\```\\Please be sure to include all.\\\\"
'''
\end{codebox}



