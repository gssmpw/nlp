\section{Related Work}
While several clinical QA datasets exist \citep{pampari-etal-2018-emrqa, lehman-etal-2022-learning, soni-etal-2022-radqa, bardhan-etal-2022-drugehrqa, dada2024information, kweon2024ehrnoteqa}, none, to the best of our knowledge, are explicitly designed for patient-oriented use. 

Prior research has explored medical text simplification, but did not focus on helping patients understand clinical documents in a QA format. \citet{aali2024dataset} developed a public dataset that converts MIMIC hospital course summaries into concise discharge letters. \citet{campillos2022building} created a Spanish dataset for simplifying clinical trial texts, demonstrating the importance of multilingual resources. \citet{trienes-etal-2022-patient} focused on making pathology reports more understandable for patients, though their dataset remains private and does not address everyday clinical questions. Similarly, while \citet{ben-abacha-demner-fushman-2019-summarization}'s MeQSum dataset transforms consumer health questions into brief medical queries, it lacks strong clinical focus.

Our work addresses these limitations by introducing a public, patient-centered QA dataset based on clinical MIMIC-IV discharge summaries, creating a benchmark to evaluate LLMs.