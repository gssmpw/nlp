\section{Related work}
In~\cite{SatoshiIPSJ96}, a new connective $\Delta$ is introduced to
express non-deter\-min\-ism in the context of multiplicative additive
linear logic and proof nets. This connective is self-dual which
suggests that it is a kind of additive disjunction and conjonction at
the same time. This produces non-determinism in the cut elimination.

An attempt to explore quantum superposition in logic is found in
Propositional Superposition Logic~\cite{TzouvarasIGPL17}. This logic
introduces the binary connective $|$ to capture the logical interplay
between conjunction and disjunction through the notion of
superposition. This framework was subsequently extended to First-Order
Superposition Logic~\cite{TzouvarasIGPL19}, which adapts these
ideas to a quantified logic setting.

In \cite{DiazcaroDowekTCS23} a new connective $\odot$ for
superposition is also introduced. It has the introduction rule of the
conjunction (similar to $\vee$-i3) and the elimination rule of the
disjunction ($\vee$-e), but no other introduction rules.  This $\odot$
connective is introduced in the context of quantum computing, and its
intuitionistic linear logic version discussed
in~\cite{DiazcaroDowekMSCS24}. The introduction of $\odot$ is the same
as that of $\with$ (or $\wedge$), the same as $\Delta$. However,
$\odot$ has an elimination rule which is the same as that of $\oplus$ (or $\vee$),
which renders it non-deterministic.

All these works introduce a new connective, while the logic presented
in this paper does not.

Among these papers \cite{SatoshiIPSJ96} and \cite{DiazcaroDowekTCS23}
have an explicit proof language  but not \cite{TzouvarasIGPL17} and
\cite{TzouvarasIGPL19}.
Among those that have an explicit proof language, 
\cite{DiazcaroDowekTCS23} introduce a sum rule, but not
\cite{SatoshiIPSJ96}.
In addition, these works do not address the problem of commuting cuts.

Independently, in~\cite{MogbilFOPARA09} a sum rule is added to multiplicative linear logic to model non-deterministic choices and introduce parallel reduction strategies for proof nets.  However, since its objectives are different, it does not includes additives and thus the relation of the rule sum with the additive disjonction is not treated.


%nor propose a proof system. Moreover, their focus aligns more closely with
%the introduction of a new connective, as with $\odot$, rather than our
%approach, which extends the disjunction's introduction rules without
%requiring a new connective.