In the preceding sections, we delved into the six dimensions of Trustworthy RAG systems, outlining their foundational principles and challenges. Building on this foundation, we now extend the discussion to their practical applications across high-stakes domains. Specifically, we focus on Healthcare, Legal, and Education. For each domain, we provide a summary of the current landscape and highlight recent advancements. Additionally, we identify domain-specific open problems that present opportunities for further research and development.

\subsection{Healthcare}
\subsubsection{RAG Use Cases in Healthcare}

\paragraph{Clinical Decision Support.}
One of the primary healthcare applications of LLMs is in Clinical Decision Support (CDS) for medical professionals~\cite{LEVRA2025113}. These systems enhance the decision-making process by providing access to diagnostic guidelines, treatment plans, and patient history for medical professionals. Although the majority of clinical models leverage fine-tuning for knowledge injection, recent results have shown that by integrating the retrieval of relevant medical documents and personalized patient data, RAG systems can improve the performance of the model to make more informed and accurate decisions~\cite{xiong2024benchmarkingretrievalaugmentedgenerationmedicine}. 

\paragraph{Patient Communication.}
While CDS systems are geared toward assisting medical professionals, RAG systems can also play a vital role in patient-facing communication. There are mainly two categories of patient-facing communication: question answering (QA) and dialogue support~\cite{HE2025102963}. The Centers for Disease Control and Prevention (CDC) has shown that 58\% of US adults have used the internet to search for medical information~\cite{cdc2023internet}. This widespread reliance on online health information underscores the need for personalized and reliable medical guidance. RAG systems, by integrating structured retrieval with generative capabilities, can enhance the accuracy and relevance of patient interactions. For instance, a RAG-based chatbot might help patients assess symptoms and recommend next steps, such as scheduling a medical consultation or seeking immediate care. 

\paragraph{Knowledge Discovery.}
AI has been extensively used in various medical knowledge discovery tasks such as drug discovery. Traditional works prioritize graph-based approaches for analyzing molecular structures and drug-drug interactions, leveraging methods such as GNN~\cite{Rohani2019, Rabeah2022}. Recently, researchers started to look at LLM for science and explore the application of LLMs in medical-related fields, particularly for accelerating drug discovery and development through improved information processing and hypothesis generation~\cite{Pal2023}. RAG systems can significantly accelerate the research process by retrieving the latest research findings, publications, and clinical trial data~\cite{Yang2025}. For example, research has shown that RAG can greatly enhance the prediction on Nephrology~\cite{Miao2024RAG}. By synthesizing large volumes of medical literature into a retrieval database, these systems provide valuable insights that can inform future studies, improve hypotheses, and guide the direction of medical innovation~\cite{Roy2024}.

\paragraph{Precharting.}
Precharting is a crucial healthcare application that involves a physician reviewing a patient's medical records before a visit~\cite{Bowman2021EMR}. Studies have demonstrated that LLMs encode extensive medical knowledge~\cite{Singhal2023LLM}, while retrieval-augmented generation (RAG) can enhance personalization and contextual relevance in clinical settings~\cite{Yang2025}. Although the integration of RAG into precharting remains largely unexplored, these observations suggest its potential to transform the process by improving efficiency and reducing load of healthcare professionals. Future research could further investigate how RAG can optimize precharting workflows and enhance patient care.

\subsubsection{Trustworthiness Challenges in Healthcare}
\paragraph{Reliability.}
 Previous research have identified reliability as one of the main challenge to healthcare applications~\cite{HE2025102963}. Healthcare applications demand an exceptionally high level of reliability due to the high-stakes impact of decisions made using RAG outputs. Uncertainty quantification has been extensively used in various healthcare applications~\cite{SEONI2023107441}, but its use in LLM-based applications remains largely unexplored. Applications such as CDS and Patient Communication should include uncertainty quantification measures to enable end-users to make informed decisions by understanding the confidence levels of the outputs. Furthermore, these systems require robustness to ensure consistent performance under challenging or unexpected input. Current research often focuses on general question answering~\cite{ni2024trustworthyknowledgegraphreasoning,luo2024rog, sun2023thinkongraph}; however, we advocate for increased focus on domain-specific and individual difference challenges in medical knowledge. Tailoring RAG systems to the requirements of healthcare can enhance their reliability and effectiveness in real-world applications.

\paragraph{Privacy.}
Another critical challenge in healthcare RAG applications is maintaining the privacy of sensitive medical data. The personal nature of medical information makes it imperative to safeguard against data breaches and misuse. Zeng et al.~\cite{privacy_rag_2024} highlight the risks of personal identification information leakage in medical data, representing an initial effort to address this issue. Since applications such as precharting heavily rely on the retrieval of personal medical records, ensuring the privacy of the data is imperative given the high stakes~\cite{Bowman2021EMR}. Future research should extend these efforts to more complex, real-world scenarios where multiple stakeholders, data sources, and regulatory frameworks intersect. Developing privacy-preserving mechanisms tailored to healthcare RAG systems will be crucial to fostering trust and ensuring compliance with legal and ethical standards.

\paragraph{Others.}
While Reliability and Privacy are particularly critical in healthcare applications, the other dimensions of trustworthiness also play significant roles. For instance, healthcare RAG systems must be safe and secure, i.e., resilient to adversarial attacks, as malicious actors could manipulate critical outputs, potentially leading to harmful consequences for patients~\cite{GhaffariLaleh2022Adversarial}. Fairness is equally essential, as healthcare is a fundamental right, and biases in RAG outputs could disproportionately disadvantage certain populations or exacerbate health disparities~\cite{HE2025102963}. Additionally, Explainability and Accountability are crucial to building trust with both medical professionals and patients. Explainable systems allow users to understand the rationale behind recommendations, while accountability mechanisms ensure that errors or unintended consequences can be traced and addressed effectively. Together, these dimensions create a comprehensive framework for trustworthy healthcare RAG systems.

\subsection{Law}
\subsubsection{RAG Use Cases in Law}
\paragraph{Legal Question Answering.}
One of the key applications of RAG systems in the legal field is a legal question answering (LQA)~\cite{lai2023largelanguagemodelslaw, chen2024survey}. LQA systems aim to provide answers to queries related to law, cases, and theoretical analysis. Due to the idiosyncratic nature of the laws in different jurisdictions, current methods focus on fine-tuning existing language models on specific laws or jurisdictions~\cite{ahmad-etal-2020-policyqa, mansouri2023falqufindinganswerslegal, abdallah2023exploring}. These systems streamline the process of finding precedents, statutes, and relevant case law, enabling legal professionals to build stronger arguments and make informed decisions. Recent research has shown that RAG can also help provide useful context for LQA~\cite{Wiratunga2023CBR} besides fine-tuning. Given the highly specialized nature of law, future research should focus on effectively integrating context to LQA through various methods (Fine-tuning, RAG, etc.). 

\paragraph{Legal Document Summarization.}
Legal Document Summarization (LDS) has become an increasingly popular use case of large language models (LLMs) in the legal community~\cite{Anthropic_Legal_Summarization, chen2024survey}. Legal documents are often long and complex, making manual summarization time-consuming and error-prone. LDS assists legal professionals and researchers by enhancing efficiency in legal analysis. Commercial models such as Claude have incorporated rule-based summarization techniques alongside state-of-the-art LLMs to improve the quality and relevance of summaries. However, the integration of RAG into LDS remains underexplored. By incorporating RAG into LDS systems, legal summaries can better capture relevant case law, statutes, and contextual references, leading to improved factual consistency and enhanced robustness~\cite{liu2024robustretrievalbasedsummarization}. 

\paragraph{Legal Judgment Prediction.}
Legal Judgment Prediction (LJP) aims to forecast court rulings based on case fact descriptions~\cite{chen2024survey}. It plays a crucial role in legal decision support, assisting judges, lawyers and clients in analyzing case outcomes. Early research formulated LJP as a classification task using traditional machine learning models~\cite{cui2022surveylegaljudgmentprediction}. However, to better reflect real-world judicial reasoning, recent studies have integrated external legal databases and RAG techniques to incorporate precedent cases and legal statutes~\cite{wu-etal-2023-precedent}, enhancing the interpretability and reliability of predictions. 

\subsubsection{Trustworthiness Challenges in Legal Applications}
\paragraph{Fairness.}
Fairness is a critical concern in legal RAG applications, as biases in data or algorithms can lead to unjust or discriminatory outcomes. For instance, biases in training data may disproportionately affect marginalized groups by retrieving prejudiced legal precedents or overlooking relevant cases. Existing research has shown that LLMs and RAG systems are susceptible to both explicit and implicit biases~\cite{shrestha2024fairrag, wan2023genderbiases}. While initial methods have been proposed to mitigate these biases~\cite{10.1162/coli_a_00524}, dedicated research in legal contexts remains limited. Addressing fairness in legal RAG systems requires not only debiasing training data but also designing algorithms that promote equity in document retrieval and output generation. Future research should explore domain-specific challenges to ensure just and transparent legal RAG systems.

\paragraph{Explainability.}
Explainability is essential in legal RAG applications due to the complexity and high-stakes nature of legal decision-making~\cite{cui2022surveylegaljudgmentprediction, chen2024survey}. Legal professionals require transparent insights into how these systems retrieve and synthesize information. For instance, when generating judicial opinions, it is crucial to trace the origins of retrieved precedents and legal arguments to assess their relevance, credibility, and potential biases. While existing research has explored explainability in general LLMs, recent studies have specifically examined interpretable methods for long-form legal question answering~\cite{louis2023interpretablelongformlegalquestion}. Future work should focus on developing domain-specific explainability frameworks that ensure legal RAG systems provide justifications aligned with legal reasoning.

\paragraph{Others.}
In addition to Fairness and Explainability, other dimensions of trustworthiness play a vital role in legal RAG applications. Reliability needs to be ensured as the output of the legal applications needs to be factually grounded with reliable uncertainty and robustness estimation. Privacy is paramount when handling sensitive client information or confidential case details, necessitating the development of privacy-preserving mechanisms compliant with legal and ethical standards. Furthermore, similar to the healthcare applications, accountability are essential to fostering trust among legal professionals and clients. Accountability mechanisms ensure that errors, biases, or unintended consequences can be identified and addressed. Together, these dimensions establish a comprehensive foundation for trustworthy RAG systems in legal applications.

\subsection{Education}
\subsubsection{RAG Use Cases in Education}
\paragraph{Personalized Learning.}
LLMs have demonstrated significant potential in education~\cite{wang2024largelanguagemodelseducation}, with personalized learning being one of their key applications. Recent advancements in educational research highlight the effectiveness of LLMs in learning path planning~\cite{ng2024educationalpersonalizedlearningpath}. However, existing approaches often lack dynamic retrieval mechanisms to adapt to different student needs and contextual knowledge gaps. Integrating RAG can enhance personalized learning by incorporating up-to-date information, ensuring more adaptive and tailored learning experiences. This approach has the potential to improve student engagement, optimize learning trajectories, and foster deeper comprehension.

\paragraph{Student Support.}
RAG systems can assist students by providing real-time answers to academic queries and offering guidance on educational content~\cite{wang2024largelanguagemodelseducation, dakshit2024facultyperspectivespotentialrag}. For example, a RAG-powered chatbot could help students understand complex concepts that might not be necessarily present in the training material. By addressing common questions and providing actionable insights, these systems can reduce the burden on teachers and counselors while providing valuable assistance to the students. 

\paragraph{Teacher Support.}
Another valuable application of RAG systems is assisting teachers in classroom instruction by integrating external databases to provide up-to-date knowledge and resources. This can help educators generate lesson materials and answer student queries based on real-time information. Current research has shown promising preliminary results in leveraging LLMs for classroom support using reddit as a data source~\cite{mullins2024enhancingclassroomteachingllms}. However, there remains a gap in developing high-quality, domain-specific datasets and more sophisticated models tailored for real-world classroom settings. Future research should focus on refining dataset curation and improving model adaptability to better integrate RAG into educational practices.

\subsubsection{Trustworthiness Challenges in Educational Applications}
\paragraph{Fairness.}
Recent research has highlighted that LLMs often exhibit explicit and implicit biases~\cite{wan2023genderbiases, sheng2021nice, sheng2021societal}. In the context of educational applications, ensuring fairness is crucial to provide all students with equal access to learning opportunities and unbiased content. For example, a RAG-based learning system might retrieve or synthesize information influenced by biases present in its training data, potentially reinforcing stereotypes or disadvantaging students from underrepresented groups. Furthermore, if the system inadvertently retrieves or generates content based on protected attributes such as race, gender, or socioeconomic status, it could lead to unequal treatment or negative educational outcomes. 

\paragraph{Safety.}
As shown in our previous discussion, studies have demonstrated that RAG systems are vulnerable to various adversarial attacks~\cite{deng2024pandorajailbreakgptsretrieval, xue2024badrag, wei2024jailbroken}. These vulnerabilities present significant safety concerns in educational settings. For instance, a student using a RAG-based learning platform could be exposed to harmful or misleading content if a malicious actor successfully jailbreaks the system to bypass safety filters. Such breaches could lead to the dissemination of inappropriate or even dangerous materials, undermining the utility of the system and harming the minors. To mitigate these risks, robust adversarial defense mechanisms tailored to the application are essential to ensure that educational RAG applications remain secure and reliable for learners.

\paragraph{Others.}
Beyond Fairness and Safety, other dimensions of trustworthiness are equally critical for educational applications of RAG systems. First, Reliability must be ensured to emphasize the accuracy and consistency of retrieved or synthesized content. Next, Privacy is a vital consideration, as educational platforms often handle sensitive personal information such as student performance, learning behaviors, and demographic data. A failure to protect privacy not only undermines trust but could also expose students to risks such as data breaches or identity theft. Lastly, Robustness is crucial for maintaining system trustworthiness in dynamic educational environments where students and learning objectives might constantly change. Educational RAG systems must adapt to diverse user queries, varying levels of prior knowledge, and potential ambiguities in input while ensuring stable performance. 
