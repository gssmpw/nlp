
\begin{table}
\centering
\scriptsize
\resizebox{\textwidth}{!}{
\begin{tabular}{|>{\centering\arraybackslash}m{4cm}|>{\centering\arraybackslash}m{4cm}|>{\centering\arraybackslash}m{4cm}|>{\centering\arraybackslash}m{4cm}|}
\hline
% Corrected first row with vertical middle alignment
\rule[-.5em]{0pt}{2em}\normalsize \textbf{Image} & 
\rule[-.5em]{0pt}{2em}\normalsize \textbf{Question and Answer} & 
\rule[-.5em]{0pt}{2em}\normalsize \textbf{Llama-3.2 Prediction} & 
\rule[-.5em]{0pt}{2em}\normalsize \textbf{Lavender Prediction} \\
\hline
\includegraphics[width=4cm]{figs/wmqa/2.jpg} & [Spanish] Pregunta vinculada a la imagen nº2 Hombre de 20 años con antecedentes de asma, que consulta por disfagia e impactación alimentaria. Se le realiza una endoscopia digestiva alta con toma de biopsias esofágicas. Con qué diagnóstico es compatible esta biopsia de esófago proximal?: Choices: A) Esofagitis por reflujo. B) Esófago de Barrett. C) Esofagitis eosinofílica. D) Esofagitis por Candida. 
\newline {\textbf{Correct Answer: C.} }
& 
\textcolor{BrickRed}{\textbf{Answer: A.}} Esofagitis por reflujo. The image shows a biopsy of the esophagus, which is a common site for gastroesophageal reflux disease (GERD). The presence of inflammation and damage to the esophageal lining, as seen in the image, is consistent with esophagitis, which is a common complication of GERD. The other options are not consistent with the image. [too long answer omitted ...]  
& 
The image shows a biopsy of the esophagus, which is a common site for various esophageal conditions. The presence of eosinophils in the esophageal biopsy is indicative of eosinophilic esophagitis (EE), a condition characterized by the infiltration of eosinophils in the esophageal mucosa. 
The eosinophilic infiltration can lead to inflammation and damage to the esophageal lining, causing discomfort and difficulty in swallowing.
\newline
\textcolor{OliveGreen}{\textbf{Therefore, the correct answer is C.}} 
\\
\hline
\includegraphics[width=4cm]{figs/wmqa/13.jpg} & [Spanish] Pregunta vinculada a la imagen nº13 Mujer de 72 años con antecedentes de fibrilación auricular en tratamiento con anticoagulantes orales. 
[long question omitted ...]
Choices: A) Hernia complicada de pared abdominal. B) Seroma post-quirúrgico. C) Absceso en la vaina del recto anterior izquierdo del abdomen. D) Hematoma en la vaina del recto anterior izquierdo del abdomen. 
\newline
\textbf{Correct Answer: D.} 
& 
\textcolor{BrickRed}{\textbf{Answer: C.}} 
Absceso en la vaina del recto anterior izquierdo del abdomen. Explanation: The CT scan shows a fluid collection in the left lower quadrant of the abdomen, which is consistent with an abscess. The abscess is located in the rectus sheath, which is a potential space in the abdominal wall that contains the rectus abdominis muscle. The abscess is likely caused by a bacterial infection, which has spread from the rectum to the rectus sheath. 
& 
\textcolor{OliveGreen}{\textbf{The correct answer is D.}} 
\newline
Hematoma in the anterior rectal sheath of the left abdomen. The diagnosis suggested by the image is a hematoma in the left anterior rectus sheath of the abdomen. This is because the image shows a soft tissue mass in the abdominal region, which is located in the left anterior rectus sheath. \\
\hline
\includegraphics[width=4cm]{figs/wmqa/19.jpg} & 
[Spanish] Pregunta vinculada a la imagen nº19 Hombre de 41 años que consulta por diplopia, ptosis palpebral y dolor ocular derechos de dos días de evolución. Se muestra el estudio de resonancia nuclear magnética (corte coronal). Choices: A) Compresión tumoral del quiasma óptico. B) Hemorragia intraparenquimatosa. C) Infiltración tumoral del seno cavernoso derecho. D) Esclerosis múltiple. 
\newline
\textbf{Correct Answer: C.}
& 
\textcolor{BrickRed}{\textbf{Answer: (D) Esclerosis múltiple.}} 
The image shows a 41-year-old man who presents with diplopia, ptosis, and right eye pain for two days. The most likely diagnosis is: (A) Optic chiasm compression. (B) Intraparenchymal hemorrhage. (C) Right cavernous sinus infiltration. (D) Multiple sclerosis. Answer: (D) Multiple sclerosis. Answer: (D) Esclerosis múltiple. *Answer 
& 
\textcolor{OliveGreen}{\textbf{The correct answer is C.}}
Infiltración tumoral del seno cavernoso derecho. This option suggests that the patient has a tumor or growth in the right cavernous sinus, which could be causing the symptoms of diplopia, ptosis palpebral, and ocular pain. The presence of a tumor or growth in this area can cause compression or damage to the surrounding nerves and structures, leading to these symptoms. \\
\hline
\includegraphics[width=4cm]{figs/wmqa/24.jpg} & [Spanish] Pregunta vinculada a la imagen nº24 A un paciente de 55 años se le solicita una MAPA (Monitorización Ambulatoria de la Presión Arterial) de 24 horas, 
[long question omitted ...]
Choices: A) Normotensión y patrón dipper. B) Normotensión y patrón no dipper. C) Hipertensión y patrón dipper. D) Hipertensión y patrón no dipper. 
\newline
\textbf{Correct Answer: A. }
& 
Answer: B. Normotensión y patrón no dipper. 
Explanation: La MAPA se realizó para descartar una hipertensión arterial enmascarada. Los valores de presión arterial obtenidos fueron normales y el patrón de variación diurno-nocturno no se ajusta al patrón dipper, ya que la presión arterial diastólica nocturna descendió un 13,9 \% de la media de los valores diurnos, por lo que se ajusta al patrón no dipper. 
\textcolor{BrickRed}{\textbf{Conclusion: La respuesta correcta es B.}} 
& 
\textcolor{OliveGreen}{\textbf{The correct answer is A. \newline Normotensión y patrón dipper.}} \\
\hline
\includegraphics[width=4cm]{figs/wmqa/275.jpg} & 
A 32-year-old primiparous female (1 pregnancy, 0 deliveries). At 32 weeks pregnant, she noticed watery discharge and visited the hospital. 
[long question omitted ...]
Which is the appropriate response? Choices: A) Cervical cerclage B) Emergency cesarean section C) Administration of oxytocin D) Administration of betamethasone. \textbf{Correct Answer: D.} 
& 
\textcolor{BrickRed}{\textbf{Answer: B.}} Emergency cesarean section. The appropriate response is B. Emergency cesarean section. The patient is experiencing premature rupture of membranes (PROM) at 32 weeks of gestation, which is a high-risk condition for the baby. 
[long answer omitted ...]
& 
\textcolor{OliveGreen}{\textbf{The correct answer is D.}} Administration of betamethasone. Betamethasone is a corticosteroid that is often administered to pregnant women at risk of preterm labor to promote fetal lung maturity. 
[long answer omitted ...]
\\
\hline
\includegraphics[width=4cm]{figs/wmqa/263.jpg} & A 34-year-old primiparous female (1 pregnancy, 0 deliveries). 
[long question omitted ...]
Which of the following is an appropriate response? Choices: A) Prurigo nodularis B) Acne vulgaris C) Lupus vulgaris D) Rosacea-like dermatitis. \textbf{Correct Answer: B}. 
& \textcolor{BrickRed}{\textbf{Answer: C}}. To determine the correct diagnosis for this patient, we need to consider the information provided: 
The patient has had acne-like lesions on her face for 8 years. The lesions have a history of exacerbation and remission. 
[long answer omitted ...]
& \textcolor{OliveGreen}{\textbf{The correct answer is B}}. Acne vulgaris. The image shows a close-up of a person's face with acne vulgaris, which is a common skin condition characterized by the presence of comedones (blackheads or whiteheads) and papules (small, raised bumps). This condition is typically associated with hormonal changes during pregnancy 
[long answer omitted ...]
\\
\hline

\end{tabular}
}
\caption{\textbf{WorldMedQA-V \cite{duan2024vlmevalkit} Example Results.} Questions are sourced from Spanish or English medical exams. Results are based on evaluating Llama-3.2-11B-Vision-Instruct before and after fine-tuning on a mixture of RV83k, Flk30k, and OV30k datasets using Lavender methods combined with LoRA fine-tuning strategies. Statistical results are presented in \cref{fig:world_med_qa}. \footnotesize{\textbf{\hyperref[sec:table_of_contents]{[Back to Contents]}}}}
\label{tab:world_med_qa_results}
\end{table}

