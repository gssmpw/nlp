\section{Related Work}
%%%%%%%%%%%%%%%%%%%%%%%%%%%%%%%%%%%%%%%%%%%%%

A variety of deep generative models have been developed to generate molecules using various molecule representations, 
including generating SMILES string representations____, or 2D molecular graph representations____.
%
{However, these representations fall short in capturing the 3D structures of molecules, which are critical for understanding their biological activities and certain properties.}
%
Recent efforts have been dedicated to the generation of 3D molecules. 
%
For example, Hoogeboom \etal____ developed an equivariant diffusion model 
in which an equivariant network is employed to
jointly predict both the positions and features of all atoms.
%
In 3D molecule generation, two types of methods have been developed.
%
Motivated by LBDD, the first type of methods, referred to as shape-conditioned molecule generation (SMG) methods, generates molecules with similar shapes to condition molecules (e.g., ligands).
% 
The second type of methods, referred to as pocket-conditioned molecule generation (PMG) methods, is motivated by SBDD and generates binding molecules to a target protein pocket.
%



%===================================================================
\subsection*{SMG Methods}
%===================================================================

%In shape-conditioned molecule generation, previous work____ has been focused on generating molecules with similar shapes to condition molecules.
%
%This effort is motivated by the principle of LBDD____ that molecules with similar shapes tend to have similar binding activities.
%
%Consequently, when using the shapes of known ligands as the condition, shape-conditioned molecule generation methods are capable of identifying promising drug candidates that have similar binding activities to known ligands.
%
Previous SMG methods____ generally leverage shapes as conditions and use generative models such as variational autoencoders (VAE)____ to generate potentially binding molecules.  
%
Among SMG methods, Adams and Coley____ developed \squid, which consists of a fragment-based generative model and a rotatable-bond scoring model.
%
The former generates molecules using VAE and sequentially decodes fragments based on the shapes of condition molecules (e.g., ligands), while the latter adjusts the angles of rotatable bonds between fragments to adapt to the condition shapes.
%
%It also neglects the bonding geometries in real molecules and generates molecules with fixed bond lengths and bond angles.
%
Long \etal____ developed an encoder-decoder framework, referred to as $\mathsf{Shape2Mol}$, which first encodes 3D shapes of molecules into latent embeddings and then generates fragments sequentially based on these embeddings to build molecules.
%
{In our preliminary work____, we also demonstrated the potential of diffusion models for generating binding molecules conditioned on shapes. 
%
By improving the shape-conditioned molecule prediction module (Section ``Shape-conditioned Molecule Prediction''), we have significantly enhanced the performance of \method.}


%Our model \method also generates molecules with similar shapes to condition molecules.
%
It is worth noting that \method is fundamentally different from \squid.
%
%As will be shown in Section ``Methods'', \method directly arranges atoms in the 3D space to closely resemble the shapes of condition molecules using a diffusion model.
%
\squid, as a fragment-based method, generates molecules by sequentially adding fragments. 
%
When predicting the next fragments, however, \squid fails to consider the effects of their various poses, and thus, could lead to inaccurate fragment predictions.
%
Due to the sequential nature, the prediction errors will be cumulated and could substantially degrade the generation performance.
%
Different from \squid, \method generates molecules by directly arranging atoms in the 3D space using diffusion models.
%
This design explicitly considers the influence of varying 3D atom positions in the generation process, leading to effective generations. 
%
%\ziqi{\squid fails to consider the optimal postures of fragments in 3D space when generating }
%In contrast, \squid sequentially predicts fragments to construct molecules in an autoregressive manner.
%
%It then integrates these fragments into molecules by adjusting their angles to previously added fragments. %using a rotatable-bond scoring model.
%
%However, this two-step approach might 
%\bo{\st{lead {\squid} to overlook}} 
%\bo{
%render \squid struggle to capture
%}
%the structural implications of each fragment prediction on the generated molecule.
%
%{This issue}
%\bo{\st{Such oversight}} 
%could result in a misalignment between the generated molecule and the shape condition, thereby decreasing the effectiveness of \squid in generating molecules that closely follow the shape condition.
%
%Different from \squid, \method enables better shape following by directly arranging atoms.}
%\hl{when predicting fragments to construct molecules, {\squid} does not consider the alignment of these fragments with the shape condition or their resultant structures.}
%\bo{
%@Ziqi rephrase. It seems this sentence has two points (fragments and XXX). It could be better to present them in two sentences. And the other point "XXX" is not clear.
%}
%resulting structures with the predicted fragments and their alignment with the shape condition.
%
%Instead, \squid relies on an additional rotatable-bond scoring model to determine the resultant structures by predicting angles of predicted fragments to \hl{maximize the shape following}. %3D structures for better shape following by adjusting the angles of the predicted fragments.
%
%This approach could \hl{degrade {\squid's} performance in shape following} \bo{@Ziqi, you just said to maximize shape following... You may want to rephrase the sentence before}, especially when fragments have distinct shapes from the condition. %, %and relies on a rotatable-bond scoring model to ensure the shape following by adjusting angles for connecting the added fragments, 
%which potentially degrades the shape following of generated molecules. %and relies on a rotatable-bond scoring model to adjust the angles for connecting the added fragments.
%
%\bo{\st{As a result, {\method} can enable better shape following of generated molecules than {\squid}.}}
%This strategy used by \squid can degrade the shape following of generated molecules when the generated fragments cannot fit into the shape condition.  
%
%relies on a rotatable-bond scoring model to control the 3D structures of generated molecules.
%
%The strategy adopted by \squid could degrade the shape following of generated molecules when the generated fragments are far from the shape condition and cannot be connected to resemble the shape condition.  
%
%\ziqi{In addition, \squid suffers from the limited diversity of generated molecules by using only fragments in the predefined library, 
%while \method enhances the diversity by allowing for the generation of any fragments. }
%
%\bo{
In addition, by using only fragments in a predefined library, \squid could struggle to generate diverse molecules, 
while \method ensures superior diversity by allowing for the generation of any fragments.
%}
%{{\method} also enables the discovery of new fragments, while {\squid} is restricted to generating molecules with only fragments in the library.}
%
%\bo{
%@Ziqi Is this still true if atoms are used as fragments? 
%}
%
%In contrast, \squid is restricted to generate molecules with fragments in the library, potentially degrading the novelty and diversity of generated molecules.
%
%In addition, %\method considers the {flexible bonding geometries}
%
%\bo{
\method also captures the flexibility of bonding geometries
%@Ziqi is bonding geometry a widely used term? ziqi: yes
%}
%\bo{
%@Ziqi do not look right to me
%}
in real 3D molecules by generating molecules with flexible bond lengths and angles.
%
However, \squid can only generate molecules with fixed bond lengths and angles, leading to the discrepancy in 3D structures between the %\bo{\st{their}} 
generated molecules and real molecules. 

\method %\bo{@Ziqi think of a better name. It should not be \method}
is also different from $\mathsf{Shape2Mol}$. 
%
\method is specifically designed to be equivariant under any rotations and translations of the shape condition, allowing for better sampling efficiency____.
%
Conversely, $\mathsf{Shape2Mol}$ is not equivariant, and thus, suffers from limited training efficiency. %requires resource-intensive training to learn knowledge shared across different rotations or translations. 
%
In addition, different from \method, $\mathsf{Shape2Mol}$ is a fragment-based approach and could suffer from the same issues as discussed above for \squid. % limited to the fragments in the {pre-defined} library.
%

%
%Papadopoulos \etal____ developed a reinforcement learning method 
%to generate SMILES strings of molecules that are similar to known antagonists of DRD2 receptors 
%in 3D shapes and pharmacophores.
%molecules similar to known antagonists of the DRD2 receptor in 3D shape and pharmacophore similarity.} \xia{grammar!}
%
%Imrie \etal____ generated 2D molecular graphs conditioned on 3D pharmacophores using a graph-based autoencoder.
%
%However, there is limited work that generates 3D molecule structures conditioned on 3D shapes.
%
% the angles of rotatable bonds between fragments to maximize shape similarity.
%\squid uses a variational autoencoder to jointly encode the 3D shape and the 2D molecular graph of an input molecule
%into a shape latent embedding. % and a posterior distribution of molecule embeddings.
%
%It then decodes molecules from the shape latent embeddings and the sampled molecule embeddings, by sequentially attaching fragments with fixed bond lengths and angles.
%
%The angles of rotatable bonds between fragments are adjusted by an independent rotatable bond scoring network.  
%to maximize the 
%%3D shape similarity.
%
%One limitation of \squid is that it only generates artificial molecules with fixed local bond lengths and angles,
%and ignores the distorted bonding geometries in real molecules.
%
 
 %===================================================================
\subsection*{PMG Methods}
%===================================================================

%\xia{this subsection needs to be extended: include more related work. Need to explicitly point out 
%the current challenges and issues, in response to the innovations you will claim about your method. }

For PMG, previous work____ has been focused on directly utilizing protein pockets as a condition and generating molecules binding towards these pockets.
%
These methods can be grouped into three categories: VAE-based, autoregressive model-based, and diffusion model-based.
%
Among VAE-based methods, Ragoza \etal____ developed a conditional VAE model to generate atomic density grids based on the density grids of protein pockets.
%
The generated atomic density grids are then converted to molecules.
%
Several autoregressive models____ also have been developed to generate binding molecules by sequentially adding atoms into the 3D space conditioned on 
protein pocket atoms. 
%
Particularly, Luo \etal____ developed an autoregressive model \AR to estimate the probability density of atoms' occurrences in the 3D space conditioned on protein pockets.
%
\AR sequentially adds atoms based on these estimations to construct molecules. 
%
Peng \etal____ improved \AR into \pockettwomol by incorporating a more efficient atom sampling strategy.
%
\pockettwomol determines the positions of newly added atoms by predicting their relative positions to previously added atoms.
%
Diffusion models are also very popular in PMG.
%
Guan \etal____ developed a conditional diffusion model \targetdiff that generates molecules based on protein pockets by sequentially denoising both continuous atom coordinates and categorical atom types in noisy molecules. 
%
Guan \etal____ further improved \targetdiff into \decompdiff by utilizing data-dependent prior distributions over molecular arms and scaffolds.
%
These priors are derived from either known ligands or protein pockets.

%\decompdiff derived the decomposed priors from either known ligands or pockets.
%
Though promising, %\bo{@Ziqi is this the right word? I only see this for drugs}, 
PMG methods require protein-ligand complex data for training.
%
However, such data is expensive and thus highly limited.
%
%Note that protein-ligand complexes are expensive and thus highly limited.
%
%\bo{\st{The limited availability of ligands in complexes}} %\bo{complexes} ligands\bo{?} 
{The sparse ground-truth binding ligands}
%
%\bo{\st{for training}} 
confine %\bo{\st{the ability of}} 
these methods in exploring a wide range of molecules with desired properties.  
%
In contrast, \method can learn from rich molecule data, improving its ability to generate effective and novel binding molecules.

%\hl{avoid using many "conditioned on"}
%encodes and includes a sequential fragment-based generative model based on v
%generates fragments with predefined local bond lengths and angles, 
%and a rotatable bond scoring network that adjusts bond angles of rotatable bonds to maximize 3D shape similarity.
% 
%However, a limitation of this framework is its ability to only generate artificial molecules with fixed local bond lengths and angles.
%
%\xia{need to have a summary here saying that shape-conditioned methods are still less developed overall. }


%===================================================================
%\subsection{Diffusion Models}
%===================================================================
\begin{comment}
%%%%%%%%%%%%%%%%%%%%%%%%%%%%%%%%%%%%%%%%%%%%%%