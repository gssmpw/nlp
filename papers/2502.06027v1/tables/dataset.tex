\begin{table*}[!t]
  \centering
      \caption{Data Statistics for SMG and PMG}
  \label{tbl:data}
  \begin{threeparttable}
 \begin{scriptsize}
      \begin{tabular}{
	@{\hspace{2pt}}l@{\hspace{10pt}}
	@{\hspace{10pt}}l@{\hspace{10pt}} 
	@{\hspace{10pt}}l@{\hspace{10pt}} 
	@{\hspace{10pt}}r@{\hspace{2pt}}         
	}
        \toprule
        %Notation &
        Task & Dataset & Description & Statistics \\
        \midrule
        %$t_a$    & 
        \multirow{3}{*}{SMG} &  \multirow{3}{*}{MOSES}
         %hidden layer dimension         & \{16, 32, 64, 128\} \\
         %atom/node embedding dimension &  \{16, 32, 64, 128\} \\
         %$\latent^{\add}$/$\latent^{\delete}$ dimension        & \{8, 16, 32, 64\} \\
         & \#training molecules            & 1,592,653 \\
         & & \#validation molecules    & 1,000 \\
         & & \#test molecules        &  1,000 \\
         \midrule
         PMG & CrossDocked2020
          %\multirow{3}{*}{PMG}
         %\#training protein-ligand complexes for PMG baselines   & \multirow{6}{*}{CrossDocked2020} & 100,000 \\%1024 \\%\bo{\{1024\}}\\
         %\#training ligands for PMG baselines             &  &  11,915        \\
         %\#training proteins for PMG baselines           &  & 15,207           \\
         & \#test protein-ligand complexes &  72          \\
         %& \#test proteins & 72 \\
         %& \#test ligands & 72 \\
        \bottomrule
      \end{tabular}
%  	\begin{tablenotes}[normal,flushleft]
%  		\begin{footnotesize}
%  	
%  	\item In this table, hidden dimension represents the dimension of hidden layers and 
%  	atom/node embeddings; latent dimension represents the dimension of latent embedding \latent.
%  	\par
%  \end{footnotesize}
%  
%\end{tablenotes}
%      \begin{tablenotes}
%      \item 
%      \par
%      \end{tablenotes}
\end{scriptsize}
  \end{threeparttable}
\end{table*}