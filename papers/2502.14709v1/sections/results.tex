\section{Evaluation Results}

% \cx{be more concise in section outline intro. (cut length by half)}
% In this section, we first evaluate the performance of Group-MATES as well as baseline methods on DCLM (§~\ref{sec:performance}). We further demonstrate the effectiveness of group-level data influence modeling (§~\ref{sec:eff-modeling}), bootstrapping data influence models (§~\ref{sec:eff-bootstrapping}), and influence-aware clustering (§~\ref{sec:eff-clustering}). Finally, we conduct comprehensive ablation studies to analyze the contributions of each component in our approach (§~\ref{sec:ablation}).

In this section, we evaluate Group-MATES and baselines on DCLM (§\ref{sec:performance}), conduct ablation studies (§\ref{sec:ablation}), and demonstrate the effectiveness of group-level data influence modeling (§\ref{sec:eff-modeling}), training relational data influence models (§\ref{sec:eff-bootstrapping}), and influence-aware clustering (§\ref{sec:eff-clustering}). Additional results and analyses can be found in Appendix~\ref{sec:additional}.
% with ablation studies to assess the contribution of each component (§~\ref{sec:ablation}).

\subsection{Overall Performance}
\label{sec:performance}

\begin{figure}[t]
    \centering
    % \begin{subfigure}{0.241\textwidth}
    % \centering
    % \includegraphics[width=1.0\linewidth]{figures/overlap_dim.pdf}
    % \caption{Selected data overlap.}
    % \label{fig:overlap-dim}
    % \end{subfigure}
    % ~
    \begin{subfigure}{0.218\textwidth}
    \centering
    \includegraphics[width=1.0\linewidth]{figures/upperbound_dim.pdf}
    \caption{Reference loss.}
    \label{fig:upperbound-dim}
    \end{subfigure}
    % ~
    \begin{subfigure}{0.236\textwidth}
    \centering
    \includegraphics[width=1.0\linewidth]{figures/eval_upperbound.pdf}
    \caption{Downstream evaluation.}
    \label{fig:upperbound-eval}
    \end{subfigure}
    \vspace{-0.35cm}
    \caption{Reference loss trajectories (a) and evaluation results (b) of data selected by upper bound group-level influence (Group), our relational data influence model (Rel), individual data influence model~\cite{yu2024mates} (Indiv), and random (Rand).}
    \vspace{-0.3cm}
    \label{fig:pairwise}
\end{figure}

% \cx{We should emphasize more on the benefits/behavior of set/pair level data influence modeling, with 1) ablation to show its empirical benefits in end results 2) analysis to show what different data it is able to pick up compared to MATES, both in terms of quantative overlap but also case studies? 3) the approximation to upper-bound inference in the next subsection.}

Table~\ref{tab:main} presents the overall performance on the DCLM benchmark. 
Group-MATES significantly outperforms random selection, achieving relative improvements of 10.1\% and 4.2\% core scores across 22 downstream tasks in 400M-4x and 1B-1x settings, respectively. 
These performance gains are substantial, as even the strong Edu Classifier—recognized for its effectiveness on less curated datasets~\cite{penedo2024fineweb}—fails to surpass random selection on 1B-1x setup. 
% \jie{do we want to reiterate DCLM has already gone through a set of rigorous selection steps, thus `random` here is not truly random. Easiest way is to refer readers to Sec.4}.
Furthermore, Group-MATES demonstrates superior generalization capabilities compared to other individual-influence-based data selection baselines, such as MATES and Quad, highlighting the benefits of modeling group-level influences in data curation.
% \cx{add the number of tasks won in table 1? if ours are stable enough, especially in 1b setting.}

% On commonsense reasoning tasks, Group-MATES seems to lag behind other baseline methods in the 400M-4x setting, while the reason is that 400M models demonstrate very unstable performances on CommonsenseQA, which dominates the score in this group. 
% This phenomenon does not affect the 1B model too much. 
On world knowledge tasks, Edu Classifier performs the best since it is specifically optimized for educational value that prefers knowledge-related data. As a price of favoring too much educational value, Edu Classifier performs worse than random selection on the majority of other tasks. This finding suggests that on a well-curated dataset (e.g., DCLM), simple heuristic-based curation, such as the educational value, may tend to overfit. In contrast, Group-MATES evaluates data utility based on model preferences and thus curates more effective data to enhance its generalization capabilities.
% Model-aware approaches, like Group-MATES, will consider the model's evolving data preferences, and thus curate more effective data to enhance its generalization capabilities. 

% In Figure~\ref{fig:flops}, we also plot the scaling curves of model performance w.r.t. total FLOPs, including both pretraining and data selection costs. 
We also provide a detailed breakdown of total FLOPs for Group-MATES in Table~\ref{tab:breakdown}. Notably, the data selection procedure of Group-MATES only accounts for 12.2\% and 6.7\% of the total FLOPs in 400M-4x and 1B-1x settings, respectively. The relative selection cost in larger setups is generally smaller because their pretraining FLOPs dominates the total computation, while the training and inference costs of our data influence model remain stable. Considering the remarkable improvements our method achieves on the DCLM benchmark, the associated cost becomes negligible.

\begin{figure}[t]
    \centering
    % \vspace{-0.3cm}
    \begin{subfigure}{0.212\textwidth}
    \centering
    \includegraphics[width=1.0\linewidth]{figures/all_pw.pdf}
    \caption{Relationship distribution.}
    \label{fig:dis-rel}
    \end{subfigure}
    \begin{subfigure}{0.212\textwidth}
    \centering
    \includegraphics[width=1.0\linewidth]{figures/all_overlap.pdf}
    \caption{Data overlap.}
    \label{fig:overlap-rel}
    \end{subfigure}
    \vspace{-0.35cm}
    \caption{Distributions of relationship terms (a) and data overlap (b) between our relational data influence model (DIM) and oracle.} 
    \label{fig:relationship}
    \vspace{-0.3cm}
\end{figure}












\section{Case studies}\label{sec:cases}
We now review the relationship between
merge-width and previously studied graph parameters.
First, it is not difficult to check that 
$\mw_\infty(G)$ is functionally equivalent to the clique-width of $G$. 
\begin{theorem}\label{thm:cw}
  A graph class $\CC$ has bounded clique-width if and only if $\mw_\infty(\CC)<\infty$.
\end{theorem}


    It is not difficult and quite instructive to prove the result directly,
    by translating between construction sequences and clique-width expressions, and vice-versa.
    Such a proof would provide explicit (linear) bounds in each direction.
    To be more succinct, we derive the statement without explicit bounds, by deriving it from results proved later in \Cref{sec:cases}.

\begin{proof}[Proof sketch]

    For the left-to-right implication,
    let $G$ be a graph of clique-width $k$.
    By \cite[Thm. 1]{tww6}, $G$ has a contraction sequence $\cal P_1,\ldots,\cal P_n$, such that at every 
    step $i\in\set{1,\ldots,n}$ of the sequence,
    every connected component of the red graph of $\cal P_i$ contains 
    at most $k'$ parts, for some constant $k'\in\N$ depending on $k$.
    By transforming the contraction sequence into a merge sequence as in the proof of \Cref{lem:tww} below, by \eqref{eq:tww-mw} we derive that the resulting construction sequence has radius-$\infty$ width at most $k'$. Thus, $\mw_\infty(G)\le k'$.
    This proves the left-to-right implication.

    Conversely, suppose that $\mw_\infty(\CC)<\infty$.
    It follows from \Cref{thm:fw-cases} below that $\fw_\infty(\CC)<\infty$.
    Therefore, $\CC$ has bounded clique-width, by \cite[Thm. II.6]{flip-width}.
\end{proof}






\noindent Thus, in a sense, the finite-radius merge-width parameters are local variants of clique-width.













\subsection{Merge-width and twin-width}\label{sec:tww}
We first discuss the relationship of merge-width and twin-width,
whose definition we recall now.
Recall that a \emph{contraction sequence} of a graph $G$ is a sequence of merge operations, which starts with the partition of $V(G)$ into singletons, and ends with the partition with one part.
Two sets $A,B\subset V(G)$ of vertices are \emph{homogeneous} if $AB\subset E(G)$ or $AB\cap E(G)=\emptyset$.
The \emph{red graph} of a partition $\cal P$
is the graph with vertices $\cal P$ and edges connecting pairs $\set{A,B}\in{\cal P\choose 2}$ 
which are not homogeneous. The twin-width of a graph $G$ is the minimum number $d$ 
such that $G$ admits a contraction sequence such that at every step, the red graph of the current partition has maximum degree at most $d$.


\twwintro*
\Cref{thm:tww} follows immediately from the next lemma.
\begin{lemma}\label{lem:tww}
  Fox every $r,d\in \N$ and graph $G$ of twin-width $d$, we have 
  $$\mw_r(G)\le 2+d+\ldots+d^r.$$
\end{lemma}
\begin{proof}
Fix a contraction sequence $\cal P_1,\ldots,\cal P_n$ of $G$, such that
the red graph of each partition $\cal P_t$ has maximum degree at most $d$.
For a vertex \(v\), denote by \(\cal P_t(v)\) the part of \(\cal P_t\) containing \(v\).
For $t\in[n]$, let $R_t$ consist of those pairs $ab\in {V\choose 2}$ such that $\cal P_t(a),\cal P_t(b)$ are either equal, or not homogeneous in $G$.
Then 
  $(\cal P_1,R_1),\ldots,(\cal P_n,R_n),$
  is a merge sequence of $G$. We bound its radius-$r$ width, for fixed $r\in\N$.

  For every $t\in[n]$, part $A\in\cal P_t$ and vertex $a\in A$, 
  we have that 
  \begin{align}\label{eq:tww-mw}
  B^r_{R_t}(a)\subset \bigcup B^r_{\cal P_t}(A),  
  \end{align}
  where $B^r_{R_t}(a)$ is the ball of radius $r$ around $a$ in the graph $(V,R_t)$, 
  and $B^r_{\cal P_t}(A)$ is the ball of radius $r$ around $A$ in the red graph of $\cal P_t$.
  In particular, $B^r_{R_t}(a)$ intersects at most $|B^r_{\cal P_t}(A)|$ parts of $\cal P_t$.
  As degree in the red graph is at most \(d\), the radius-$r$ width of $(\cal P_t,R_t)$ is at most $|B^r_{\cal P_t}(A)|\le 1+d+\ldots+d^r$.
  Since $\cal P_{t-1}$ is a refinement of $\cal P_t$ with exactly one more part,
  it follows that 
  the radius-$r$ width of $(\cal P_{t-1},R_t)$ is at most $2+d+\ldots+d^r$.
Consequently,
$\mw_r(G)\le 2+d+\ldots+d^r.$
\end{proof}


\subsection{Merge-width and Sparsity}\label{sec:sparsity}
The converse to \Cref{thm:tww} is false. 
For instance,  the class 
of graphs of maximum degree bounded by $d$ (a class which has unbounded twin-width already for $d=3$) 
has bounded merge-width.
Indeed, there is a trivial merge sequence for such a graph $G$: 
 $$\bigl(\textit{partition into singletons},\emptyset\bigr),\quad \bigl(\textit{partition with one part},E(G)\bigr).$$
The radius-$r$ width of this sequence is 
the maximum size of a radius-$r$ ball in $G$,
which is at most $1+d+\cdots+d^r$, 
so 
$\mw_r(G)\le 1+d+\cdots+d^r$.



If the above merge sequence for graphs of bounded degree seems uninsightful,
this reflects the fact that graphs of bounded degree are trivial from the perspective of Sparsity theory.
The next result is slightly more 
illuminating.
The proof is already given in \Cref{ex:degeneracy}.




\begin{theorem}\label{thm:deg}
  If $G$ is a $d$-degenerate graph then 
  $\mw_1(G)\le d+2.$
\end{theorem}


Next, we prove:
\beintro*

The simple construction from \Cref{thm:deg} does not trivially generalize to bound the radius-$r$ merge-width 
of graphs in classes of bounded expansion,
as it appears that to bound $\mw_r(G)$ for $r>1$, a more involved construction 
is needed, which we now present.

\medskip

We first recall the definition of classes of bounded expansion for completeness only, as we will not use the original definition, only the characterization given in the upcoming \Cref{fact:be}.
By definition,  a graph class $\CC$ has \emph{bounded expansion} if and only if for every $r\in\N$,
there is some $k\in\R$ such that for every graph $G$
which can be obtained from some graph in $\CC$ by first removing vertices and edges, and then contracting 
some vertex subsets of radius ${\le r}$ to single vertices, 
we have that $|E(G)|\le k\cdot |V(G)|$.
Classes of bounded expansion include every class of bounded maximum degree,
the class of planar graphs, and every graph class that excludes some graph  as a  minor, or as a topological minor.


Fix a graph $G$, number $r\in\N$, and total order $\le$ on $V(G)$. 
A vertex \(u\) is \emph{weakly \(r\)-reachable} from a vertex \(v\) if there is a path of length at most \(r\) from \(v\) 
to \(u\) and all vertices on that path are at least as large as \(u\).
Let \(\wreach_r(G,\le,v)\) be the set of vertices that are weakly \(r\)-reachable from \(v\) with respect to the ordering.
Then the weak \(r\)-coloring number of \(G\) is defined as
\[
\wcol_r(G) \coloneqq \min_{\le} \max_{v \in V(G)} |\wreach_r(G,\le,v)|,
\]
where the minimum ranges over all total orders $\le$ of $V$.

\begin{lemma}[\cite{zhu2009colouring}]\label{fact:be}
A graph class $\CC$ has bounded expansion if and only if $\wcol_r(\CC)<\infty$, for all~${r\in\N}$.
\end{lemma}













\Cref{thm:be} follows immediately from the next lemma,
which 
shows that the 
 radius-$r$ merge-width 
is bounded in terms of the weak $(r+1)$-coloring number. 


\begin{lemma}\label{lem:wcol}
  For every $k,r\in\N$ and graph $G$ with $\wcol_{r+1}(G)\le k$, we have
  $$\mw_r(G)\le 3\cdot 2^{k} .$$
\end{lemma}
In particular, we show that $\mw_2(G)$  is bounded in terms of $\wcol_3(G)$,
and we do not know whether it is bounded in terms of $\wcol_2(G)$.

\begin{proof}Fix $r\in\N$, and a graph $G$ with vertex set $V$.
Let $\le$ be an ordering of the vertex set $V$ of $G$ 
such that for every vertex \(v\), the weak $r$-reachability set $\wreach_{r+1}(G,\le,v)$ has size at most $k$.

For $t=1,\ldots,n$, define the following.
\begin{itemize}
  \item Let $L_t$ comprise the $t$ largest elements of $V$ with respect to $\le$, and let $S_t\coloneqq V-L_t$.
  \item Let  $\cal P_t$ be the partition of $V$ into \emph{atomic types}
  over the set $S_t$. That is, $\cal P_t$ partitions $S_t$ into singletons, and partitions vertices  $v\in L_t$ according to their neighborhood $N(v)\cap S_t$~in~$S_t$.
  \item Let $R_t \coloneqq\setof{ab\in E(G)}{a,b\in L_t}$ be the set of all edges in $G$ with both endpoints in $L_t$.
\end{itemize}
We verify that $(\cal P_1,R_1),\ldots,(\cal P_n,R_n)$ is a merge sequence of $G$. Clearly, $\cal P_1$ is the partition into singletons, 
and $\cal P_n$ has one part.
Observe that for $t\in[n]$ and any two parts $A,B$ of $\cal P_t$,
either $A,B\subset L_t$, and therefore  $E(G)\cap AB\subset R_t$,
or, otherwise, at least one of $A,B$ is contained in $S_t$, and then $A$ and $B$ are homogeneous,
as both $A$ and  $B$ are atomic types over $S_t$. 
In any case, $AB-R_t\subset E(G)$ or 
$AB-R_t\subset {V\choose 2}-E(G)$, so the conditions of a merge sequence are satisfied.


Fix $t\in[n]$. Fix two vertices $v,w\in L_t$, such that 
there is a path of length at most $r$ in $(V,R_t)$ from $v$ to $w$.
Then there is a path $\pi$ of length at most \(r\) from $v$ to $w$ in $G$
contained in $L_t$.
In particular, for every vertex $u\in N_G(w)\cap S_t$ 
there is a path from $v$ to $u$ of length at most $r+1$ in \(G\) (namely the path $\pi$ followed by the edge $wu$) such that every inner vertex on that path is in $L_t$.
In other words, $N_G(w)\cap S_t \subseteq \wreach_{r+1}(G,\le,v)$.
 As $|\wreach_{r+1}(G,\le,v)|\le k$ by assumption, and 
 the atomic type of $w$ over $S_t$ is uniquely 
determined by $N(w)\cap S_t$, it follows that 
there are at most $2^k$ parts of $\cal P_t$ 
that are reachable by a path of length at most $r$ from $v$.
Thus, the radius-$r$ width of $(\cal P_t,R_t)$ is at most
$2^k$.



To bound the radius-$r$ width of $(\cal P_{t-1},R_t)$, for $t>1$, notice that every part in $\cal P_{t}$ 
is a union of at most three parts in $\cal P_{t-1}$.
 Namely, if $v$ is the unique vertex with $v\in S_{t-1}-S_{t}$, then the atomic type $A$ of a vertex $u$ over $S_{t}$ 
 is uniquely determined by the atomic type of $u$ over $S_{t-1}$ and the atomic type of $u$ over $\set v$. 
 There are three 
 possible atomic types over $\set v$, corresponding 
 to
 whether $u=v$, or $(u\neq v)\land E(u,v)$, or 
$(u\neq v)\land \neg E(u,v)$. This proves that $A$ is a union of at most three parts in $\cal P_{t-1}$.

It follows that the radius-$r$ width of $(\cal P_{t-1},R_t)$ is at most $3\cdot 2^k$, and we have constructed a merge sequence of $G$ 
of radius-$r$ width at most $3\cdot 2^k$.
\end{proof}























 






\subsection{Merge-width and flip-width}\label{sec:fw}
We study the relationship between merge-width and flip-width, whose definition we now recall.

For a graph $G$ and 
two sets $A,B\subset V(G)$,
\emph{flipping} the pair $(A,B)$ in $G$ 
results in the graph $G'$ 
with edges $E(G')=E(G)\triangle AB$, where \(\triangle\) denotes the symmetric difference.
Given a partition $\cal P$ of $V(G)$, 
a \emph{$\cal P$-flip} of $G$ is a graph $G'$
obtained from $G$ by flipping arbitrary pairs $A,B\in\cal P$ (possibly with $A=B$).
Thus, a graph $G'$ is a $\cal P$-flip of $G$ if and only if for every pair $A,B\in\cal P$, 
either $E(G')\cap AB=E(G)\cap AB$ or $E(G')\cap AB=AB-E(G)$.
A \(k\)-flip is a \(\cal P\)-flip with \(|\cal P| \le k\).

We recall the definition of flip-width from \cite{flip-width}.
Fix $k,r\in\N$.
The \emph{flip-width game} of radius $r$ and width $k$ is played on a graph $G$ between two players, flipper and runner.
Initially, runner picks a vertex $v_0\in V(G)$.
In round $t=1,2,\ldots$,
    flipper announces a $k$-flip $G_t$ of $G$;
    then runner picks $v_t\in B^r_{G_{t-1}}(v_{t-1})$ -- that is, they traverse 
    a path of length at most $r$ in the \emph{previous} graph $G_{t-1}$.
    Flipper wins if $v_t$ is isolated in $G_t$.
The \emph{radius-$r$ flip-width} of $G$, denoted $\fw_r(G)$,
is the least number $k$ such that flipper has a winning strategy for the flip-width game of width $k$ and radius $r$ on $G$.
We prove:
\begin{theorem}\label{thm:fw-cases}
  Every class of bounded merge-width has bounded flip-width.
  More precisely, for all $r\in\N$ and every graph $G$,
  $$\fw_r(G)\le 4^{\mw_{2r-1}(G)}.$$
\end{theorem}

\Cref{thm:fw-cases} allows to deduce several results about classes of bounded merge-width
from the corresponding results for classes of bounded flip-width.
For instance, we obtain the two corollaries stated in the introduction:
\corbe*
The forward implication in \Cref{cor:be} follows from \Cref{thm:be} and the trivial fact that each class of bounded expansion excludes some biclique as a subgraph. The backwards direction follows from \Cref{thm:fw-cases} and from the analogous statement for classes of bounded flip-width, proved in \cite[Thm. VI.3]{flip-width}.


\cortww*
In \Cref{cor:tww-mw}, we use the notion of merge-width for binary structures, specifically for \emph{ordered graphs} --  graphs equipped with a total order on the vertex set. This is defined in \Cref{sec:computing}.  \Cref{thm:fw-cases} applies to classes binary structures, allowing to derive \Cref{cor:tww-mw} by a similar reasoning as for \Cref{cor:be} above:
The forward direction uses \Cref{thm:tww}and the fact that every graph can be extended with a total order, without increasing its twin-width \cite{tww4}, and the backwards direction uses \Cref{thm:fw-cases} and the analogue of \Cref{cor:tww-mw} proved in \cite[Thm. VII.3]{flip-width}.
For simplicity, below we present the proof of \Cref{thm:fw-cases} only for graph classes. The more general case is analogous, 
but the requires notion of flips and of flip-width for binary structures as defined in \cite[Sec. V.B]{flip-width}.

We state another corollary that follows from \Cref{thm:deg} and an analogous result from \cite[Thm. VI.1]{flip-width}.
\begin{corollary}\label{cor:deg}
  Let $\CC$ be a graph class. Then $\CC$ has bounded degeneracy if and only if $\mw_1(\CC)<\infty$ and $\CC$ excludes some biclique $K_{t,t}$ as a subgraph.
\end{corollary}




\medskip
To prove \Cref{thm:fw-cases},
we start with a simple observation that allows us to rephrase the main condition from the definition 
of a merge sequence in terms of flips, as follows.


\begin{lemma}\label{lem:flip}
    Let $G=(V,E)$ be a graph, let $\cal P$ be a partition of $V$, and let $R\subset {V\choose 2}$.
    The following conditions are equivalent:
    \begin{enumerate}
        \item for all $A,B\in\cal P$, either $AB-R\subset E$, or $AB-R\subset AB - E$,
        \item  there is a $\cal P$-flip $G'$ of $G$ with $E(G')\subset R$.
    \end{enumerate}
    \end{lemma}
    \begin{proof}
        (1$\rightarrow$2).
        The $\cal P$-flip $G'$
        is obtained from $G$ by flipping each pair $A,B\in\cal P$ such that $AB-R\subset E$. As a result $AB-R\subset AB-E(G')$, 
        and therefore $E(G')\cap AB\subset R$ for each $A,B\in\cal P_t$.
        Altogether, $E(G')\subset R$.
    
        (2$\rightarrow$1). Fix $A,B\in\cal P$. We have that
        $E(G')\cap AB\subset R\cap AB$,
        so $AB-R\subset AB-E(G')$.
        As $G'$ is a $\cal P$-flip of $G$,
        we have that either $E(G')\cap AB=E(G)\cap AB$ or $E(G')\cap AB=AB-E(G)$.
        Altogether, either $AB-R\subset E(G)\cap AB$, or  $AB-R\subset AB-E(G)$.
    \end{proof}
    
Using \Cref{lem:flip}, we may therefore redefine merge-width in terms of flips, as follows.




\begin{definition}Let $G$ be a graph.
A \emph{restrained flip sequence} for $G$ is a sequence
\begin{align}\label{eq:mono-fw}
    (\cal P_1,R_1,G_1),\ldots,(\cal P_m,R_m,G_m)    
\end{align}
such that:
\begin{itemize}
    \item $\cal P_1,\ldots,\cal P_m$ is a refining sequence of partitions, starting with the  partition $\cal P_1$ of $V(G)$ with one part, and ending with the partition $\cal P_m$ into singletons,
    \item the graph $G_t$ is a $\cal P_{t}$-flip of $G$, for $t\in [m]$,
    \item ${V\choose 2}= R_1\supseteq \cdots \supseteq R_m=\emptyset$, and
    \item  $E(G_t)\subset R_t$ for $t\in[m]$.
\end{itemize}
The radius-$r$ width of the restrained flip sequence is 
the maximum, over $t\in[m-1]$,
of the radius-$r$ width of $(\cal P_{t+1},R_{t})$.
\end{definition}

Note that in the reformulation above,
the partitions $\cal P_1,\ldots,\cal P_m$ are becoming finer with each step, instead of becoming coarser as in merge sequences. In each step of a restrained flip sequence, we provide a $\cal P_t$-flip $G_t$.
The sequence $R_1\supseteq \cdots\supseteq R_m$
is a descending sequence of subsets of $V\choose 2$ with $R_m=\emptyset$. The set $R_t$ is called the \emph{restraint} at time $t$, as we require that $E(G_s)\subset R_t$ for all $s\in[m]$ with $s\ge t$, thus restraining all the future graphs $G_t,G_{t+1},\ldots,G_m$.  Note that  $E(G_m)\subset R_m=\emptyset$, so $G_m$ is edgeless.

As $G_t$ is a $\cal P_t$-flip of $G$, 
$G_{t-1}$ is a $\cal P_{t-1}$-flip of $G$, and $\cal P_{t-1}$ is coarser than $\cal P_t$,
we may equivalently require that $G_t$ as a $\cal P_{t}$-flip of $G_{t-1}$, rather than of $G$.
Therefore, in the sequence $G_1,G_2,G_3,\ldots$ 
each subsequent graph is a flip of the previous graph.
This is similar to the setting of flipper games considered in \cite{flippers}.




\begin{lemma}\label{lem:mw-flip seq}
    For every graph $G$,
    there is a correspondence between merge sequences and restrained flip sequences for $G$,
    which preserves the length, and the radius-$r$ width of the sequences, for each $r\in\N\cup\set{\infty}$.
\end{lemma}




\begin{proof}
    Fix a merge sequence 
    $(\cal P_1,R_1),\ldots,(\cal P_m,R_m)$
    of $G$.
 By \Cref{lem:flip} (1$\rightarrow$2),
 for each $t\in[m]$,
    there is a  $\cal P_t$-flip $G_t$ of $G$ such that 
$E(G_t)\subset R_t$.
    The sequence 
    $(\cal P_m,R_m,G_m),\ldots,(\cal P_1,R_1,G_1)$
    is a restrained flip sequence for $G$.
    
    Conversely,
    given a restrained flip sequence for $G$ of the form $(\cal P_1,R_1,G_1),\ldots,(\cal P_m,R_m,G_m)$,
    by \Cref{lem:flip} (2$\rightarrow$1) we conclude that 
    the sequence $(\cal P_m,R_m),\ldots,(\cal P_1,R_1)$ is a merge sequence of $G$.
\end{proof}










The next lemma shows that radius-$r$ flip-width is bounded in terms of radius-$(2r-1)$ merge-width, thus proving \Cref{thm:fw-cases}.




\begin{lemma}\label{lem:fw}
    Fix $r,s\in\N$. 
    For every graph $G$, if $\mw_{2r-1}(G)\le s$ then  $\fw_r(G)\le 4^s$. 
\end{lemma}

\begin{proof}
    Let $G$ be a graph with $\mw_{2r-1}(G)\le s$. Then $G$ has a merge sequence of radius-$(2r-1)$ merge-width at most $s$.
    By \Cref{lem:mw-flip seq}, $G$ has a 
 restrained flip sequence of radius-$(2r+1)$ width at most \(s\) of the form
$$(R_1,\cal P_1,G_1),\ldots,(R_n,\cal P_n,G_n).$$

The following lemma will allow us to construct a strategy for flipper in the flip-width game.

\begin{lemma}\label{lem:strategy}
    Fix $t\in [2,n]$, and 
    let $s$ be the radius-$(2r-1)$ width of $(\cal P_t,R_{t-1})$.
    For every $v\in V(G)$ there is a $4^s$-flip $G_t'$ of $G$ such that:
$$B^r_{G_t'}(w)\subset B^r_{G_t}(w)\quad\text{for every $w\in B^r_{G_{t-1}}(v)$.}$$
\end{lemma}
Before proving \Cref{lem:strategy}, we first show how it yields a winning strategy
for flipper in the flip-width game of radius $r$
    and width $4^s$.
Flipper's strategy will be to announce graphs $G_1',G_2',\ldots$, ensuring  that following invariant holds after round $t\ge 1$ of the game:
    \begin{align}\label{eq:fw-invariant}
    B^r_{G_{t}'}(v_t)\subset B^r_{G_t}(v_t).    
    \end{align}
    
    In the first round, when $t=1$, flipper announces $G_1'=G_1$ and runner picks a vertex 
    $v_1$,
    and the invariant holds trivially.
    Suppose the invariant is satisfied after round $t-1$ of the game,
    that is, 
    $$B^r_{G_{t-1}'}(v_{t-1})\subset B^r_{G_{t-1}}(v_{t-1}).$$
    Now flipper announces the $4^s$-flip $G_t'$ of $G$
    given by \Cref{lem:strategy} for $v \coloneqq v_{t-1}$.
    Next, runner picks a vertex $v_t\in B^r_{G_{t-1}'}(v_{t-1})$.
In particular, $v_t\in B^r_{G_{t-1}}(v_{t-1})$, so 
 \eqref{eq:fw-invariant} holds by \Cref{lem:strategy}, and the invariant is fulfilled.

    Playing according to this strategy, flipper wins within $n$ rounds,
    since $E(G_n)=\emptyset$, and therefore $v_n$ is isolated in $G_n'$
    by \eqref{eq:fw-invariant}. This proves that $\fw_r(G)\le 4^s$, and thus \Cref{lem:fw}.
\end{proof}

\begin{remark}\label{rem:duration}
    Observe that the number of rounds needed by flipper to win in the flip-width game of radius $r$ and width $4^s$ can be bounded by $|V(G)|$.
    Namely, the proof of \Cref{lem:fw} above shows that 
    the number of rounds is (at most) equal to the length $n$ of a restrained flip sequence 
     of radius-$(2r+1)$ width at most \(s\) for $G$.
     By \Cref{lem:mw-flip seq}, this corresponds to the length of a radius-$(2r+1)$ merge sequence of width $s$. Any merge sequence of $G$ can be converted into one of length $n=|V(G)|$, while preserving its radius-$r$ width for each $r\in\N$, since if for two consecutive pairs $(\cal P_i, R_i),(\cal P_{i+1},R_{i+1})$ in a merge sequence we have $\cal P_i=\cal P_{i+1}$, then $(\cal P_{i+1},R_{i+1})$ can be dropped from the merge sequence.
\end{remark}

We now prove \Cref{lem:strategy}. 



    \begin{proof}[Proof of \Cref{lem:strategy}]
        Fix $v\in V(G)$. For $i=0,\ldots,2r-1$, let 
 $\cal Q_i\subset \cal P_t$ consist of all parts $A\in\cal P_t$ that can be reached by a path of length at most $i$ by from $v$ in the graph $(V,R_{t-1})$.
In particular, $|\cal Q_{2r-1}|\le s$.

Let $G_t'=(V,E')$ be the graph 
such that for any two parts $A,B\in\cal P_t$:
$$E'\cap AB=
\begin{cases}
    E(G)\cap AB&\text{if $A,B\notin\cal Q_{2r-1}$}\\
    E(G_t)\cap AB&\text{otherwise}.
\end{cases}$$

\begin{claim}\label{cl:k-flip}
    $G_t'$ is a $4^s$-flip of $G$.
\end{claim}
\begin{claimproof}
    As $G_t$ is a $\cal P_t$-flip of $G$, so is $G_t'$.
    Therefore, for any two parts $A,B\in\cal P_t$, 
    the set $E'\cap AB$ is either equal to $E\cap AB$,
    or to $AB-E$.

    For every part $A\in\cal Q_{2r-1}$, let $U(A)$ be the union of all parts 
    $B\in\cal P_t$ such that $E'\cap AB= AB-E$.
    Then for every $B$ with $B\subset U(A)$ or $B\subset V- U(A)$,
    we have that
    $E'\cap AB=E\cap AB$ or 
     $E'\cap AB=AB-E$.

    Consider the set family:
    $$\cal F\coloneqq \cal Q_{2r-1} \cup \setof{U(A)}{A\in \cal Q_{2r-1}}.$$ Then $|\cal F|\le 2|\cal Q_{2r-1}|\le 2s$.
    
    Let $\cal Q$ be the partition 
    of $V$ such that 
    two vertices $a,b$ are in the same part of $\cal Q$ if and only if $a$ and $b$ belong to the same sets in $\cal F$.
    Then $|\cal Q|\le 2^{|\cal F|}\le 4^{s}$.
    Moreover, for every $A,B\in \cal Q$,
    either $E'\cap AB=E\cap AB$,
    or $E'\cap AB=AB-E$.
    It follows that 
    $G_t'$ is a $4^s$-flip of $G$.
\end{claimproof}



\begin{claim}\label{cl:balls}
    For all $w\in B^r_{G_{t-1}}(v)$
 and $i=0,\ldots,r$ we have:
    $$B^i_{G_t'}(w)\subset B^i_{G_t}(w).$$
\end{claim}

\begin{claimproof}
    We induct on $i$. For $i=0$ the statement is trivial.
In the inductive step, fix $1\le i\le r$ and $u\in B^i_{G'}(w)$ with $u\neq w$.
Then there is some  $a\in B^{i-1}_{G'}(w)$ with $ua\in E(G')$.
By inductive assumption, $a\in B^{i-1}_{G_t}(w)$.
From  $i\le r$ and  $E(G_t)\subset R_{t-1}$
we get that $a\in B^{r-1}_{R_{t-1}}(w)$,
which together with $w\in B^{r}_{G_{t-1}}(v)\subset B^{r}_{R_{t-1}}(v)$ implies $a\in B^{2r-1}_{R_{t-1}}(v)$. In particular, $a\in \bigcup \cal Q_{2r-1}$.
By definition of $E'$, this implies $ua\in E(G_t)$. Together with 
$a\in B^{i-1}_{G_t}(w)$, this implies that $u\in B^{i}_{G_t}(w)$, as required.
\end{claimproof}
\Cref{cl:k-flip} and \Cref{cl:balls} together prove \Cref{lem:strategy}.
\end{proof}




\subsection{Almost bounded merge-width}\label{sec:abmw}
A graph class $\CC$ has \emph{almost bounded merge-width}
if for every $r\in\N$ and $\eps>0$,
we have that $\mw_r(G)\le O_{\CC,r,\eps}(|V(G)|^\eps)$, for all $n\in\N$ and all $n$-vertex graphs $G\in\CC$.

Clearly, classes of bounded merge-width have almost bounded merge-width.
The following result states that all nowhere dense classes have almost bounded merge-width. We will prove this later below, by inspecting the proof of \Cref{thm:be}.

\thmnwd*

Classes of \emph{almost bounded flip-width} are defined analogously as classes of almost bounded merge-width, with $\mw_r(G)$ replaced with $\fw_r(G)$.
The following is the main result of this section.

\thmabmw*

As every hereditary class of almost bounded flip-width is monadically dependent \cite[Thm. 2.12]{flip-breakability}, we obtain the following.
\begin{corollary}\label{cor:abmw-mNIP}
    Every hereditary class of almost bounded merge-width is monadically dependent.
\end{corollary}

\Cref{thm:nwd} and \Cref{cor:abmw-mNIP} together justify our \Cref{conj:almostboundedmergewidth}
that almost bounded merge-width coincides with monadic dependence on hereditary classes.










\subsection{Preliminaries}
Before proving \Cref{thm:nwd} and \Cref{thm:abmw}, 
we first recall some basic notions.

\medskip
A \emph{set system} is a pair $(X,\cal F)$ with $\cal F\subset 2^X$.
Its \emph{VC-dimension} is the maximal size of a subset $Y\subset X$ such that 
$\setof{Y\cap F}{F\in\cal F}=2^Y$.
We recall the fundamental Sauer-Shelah-Perles lemma~\cite{sauer,shelah-sauer-lemma}.

\begin{lemma}[Sauer-Shelah-Perles lemma]\label{lem:sauer-shelah-perles}
  Let $(X,\cal F)$ be a set system of VC-dimension~$d$.
  Then $|\cal F|\le O(|X|^d)$.
\end{lemma}

The \emph{VC-dimension} of a graph $G$, denoted $\VCdim(G)$, 
is defined as the VC-dimension of the set system $\bigl(V(G),\setof{N(v)}{v\in V(G)}\bigr)$.
More explicitly, $\VCdim(G)$ is the maximal size of a subset $X\subset V(G)$
such that $\setof{N(v)\cap X}{v\in V(G)}=2^X$.

Define the \emph{atomic complexity} of a graph $G$, as the function $\pi_G\from\N\to\N$ defined as
 $$\pi_G(s)\coloneqq\max\Bigl\{|S|+\bigl|\{N_G(v)\cap S\,:\,v\in V(G)-S\}\bigr|\,:\, S\subset V(G), |S|\le s\Bigr\} \quad\text{for all }s \in \N.$$
(Equivalently, $\pi_G(s)$ is the maximal number of atomic types over a set $S$ of size at most $s$ in $G$. This is essentially equal to the \emph{neighborhood complexity}, up to an additive factor of $s$.) 
In particular, $\pi_G(s)\le s+2^s$, for all graphs $G$ and all $s\in\N$.
From \Cref{lem:sauer-shelah-perles}, we get the following.
\begin{corollary}\label{cor:sauer-shelah}
    Let $G$ be a graph of VC-dimension $d$.
    Then $\pi_G(s)\le O(s^d)$, for all $s\in \N$.
\end{corollary}



\subsection{Proof of \Cref{thm:nwd}}


To prove \Cref{thm:nwd}, we use the following result.
\begin{fact}[\cite{zhu2009colouring,sparsity-book}]\label{fact:nd-wcol}
    Let $\CC$ be a nowhere dense graph class, and fix $r\in\N$ and $\eps>0$.
    Then for every graph $G\in\CC$
    $$\wcol_r(G)\le O_{\CC,r,\eps}(|V(G)|^{\eps}).$$
\end{fact}

\begin{lemma}\label{lem:nd-VC}
    Let $\CC$ be a nowhere dense graph class. 
    Then there is some $d\in \N$ such that all graphs $G\in \CC$ have VC-dimension less than $d$.
\end{lemma}
\begin{proof}
    As $\CC$ is nowhere dense, there is some $d\in\N$ such that 
    no graph $G\in \CC$ contains the $1$-subdivision of the clique $K_d$, as a subgraph.
    It follows that every graph $G\in \CC$ has VC-dimension less than $d$.
\end{proof}


\begin{lemma}\label{lem:wcol-poly}
    For every $r\in\N\cup\set{\infty}$ and graph $G$, we have
    $$\mw_r(G)\le 3\cdot\pi_G(\wcol_{r+1}(G)).$$
  \end{lemma}
  \begin{proof}[Proof sketch]
    We follow the proof of \Cref{lem:wcol}, and use the same notation. 
    Let \(k\coloneqq\wcol_{r+1}(G)\).
    Fix $t\in [n]$, and let $L_t,S_t,\cal P_t,R_t,\le$ be as defined in the proof.
    Fix $v\in L_t$.
    It was observed that for every $w\in V(G)$ which is reachable from $v$ by a path of length $r$
in the graph $(V,R_t)$, 
 the atomic type of $w$ over $S_t$ is uniquely 
determined by $N_G(w)\cap S_t \subseteq \wreach_{r+1}(G,\le,v)$,
and moreover, $|\wreach_{r+1}(G,\le,v)|\le k$.
Since $$\Big|\setof{N_G(w)\cap \wreach_{r+1}(G,\le,v)}{w\in V(G)}\Big|\le  \pi_G(k),$$
we conclude that are at most $\pi_G(k)$ parts of $\cal P_t$ 
that are reachable by a path of length at most $r+1$ from $v$.
Thus, the radius-$r$ width of $(\cal P_t,R_t)$ is at most
$\pi_G(k)$. The rest of the argument remains unchanged, yielding the final bound $\mw_r(G)\le 3\cdot \pi_G(k)\le 3\cdot \pi_G(\wcol_{r+1}(G))$.
  \end{proof}
  
  \Cref{thm:nwd} now follows by combining the previous insights.
  \begin{proof}[Proof of \Cref{thm:nwd}]
    By \Cref{lem:nd-VC}, there is some $d\in\N$ such that every $G\in\CC$ has VC-dimension at most $d$.
    Fix $r\in\N$ and $\eps>0$.
Then, for every graph $G\in\CC$, by \Cref{lem:wcol-poly} and \Cref{fact:nd-wcol}, we have 
$$\mw_r(G)\le 3\cdot \pi_G\bigl(\wcol_{r+1}(G)\bigr)\le  3\cdot \pi_G\bigl(O_{\CC,r,\eps}(|V(G)|^\eps)\bigr)\le O_{\CC,r,\eps}\bigl(|V(G)|^{\eps\cdot d}\bigr).$$
Since $\eps>0$ is arbitrary, this implies that $\CC$ has almost bounded merge-width.
  \end{proof}

  \subsection{Proof of \Cref{thm:abmw}}


We start by proving that hereditary classes of almost bounded merge-width have bounded VC-dimension. 
In fact, we prove a stronger result, expressed using the following notion.

\newcommand{\ntn}{\mathrm{ntn}}

A pair $u,v$ of distinct vertices in a graph $G$ is a pair of \emph{$k$-near-twins} if $|N_G(u)\triangle N_G(v)|\le k$.
 For a bipartite graph $G=(X,Y,E)$, let the near-twin number $\ntn(G)$ denote 
 the smallest number $k$ such that 
for all $X'\subset X,Y'\subset Y$ with $|X'|+|Y'|>2$,
the bipartite subgraph $G[X',Y']$ induced by $X'$ and $Y'$  in $G$
has a pair of $k$-near-twins contained either in $X'$, or in $Y'$.

\begin{lemma}\label{lem:twins-bip}
    Let $G=(X,Y,E)$ be a bipartite graph with $\mw_1(G)\le k$
    and with $|X|+|Y|>2$.
    Then $G$ contains a pair of $2k$-near-twins
     contained in a single part of $G$. In particular, $\ntn(G)\le 2k$.
\end{lemma}
\begin{proof}
    Fix a construction sequence of $G$ of radius-$1$ width $k$. Consider the first moment when some part $A$ of the current partition $\cal P$ contains two vertices $a,b$ that belong to the same part of the bipartition $\set{X,Y}$ of $G$. Suppose $a,b\in X$. Then $N_G(a)\triangle N_G(b)\subset (N_R(a)\cup N_R(b))\cap Y$,
    where $N_R(\cdot)$ denotes the neighborhood in the graph $(V,R)$  of currently resolved pairs.
    Note that $|N_R(a)\cap Y|\le k$,
    since no two vertices of $Y$ are in a single part of the current partition $\cal P$, and $N_R(a)$ is contained in at most $k$ parts of $\cal P$. Similarly, $|N_R(b)\cap Y|\le k$.
    It follows that $|N_G(a)\triangle N_G(b)|\le 2k$.
\end{proof}

The following lemma is implicit in \cite{flip-width}
(it follows from 
\cite[Lem. 5.25]{flip-width-arxiv}).
\begin{lemma}\label{lem:ntn}
    If $\CC$ is a hereditary class of bipartite graphs such that $\ntn(G)\le o(|V(G)|)$ for $G \in \CC$, then $\VCdim(\CC)<\infty$.
\end{lemma}





\begin{corollary}\label{lem:abmw-vc}
    If $\CC$ is a hereditary class of graphs such that $\mw_1(G)\le o(|V(G)|)$ for $G \in \CC$, then $\VCdim(\CC)<\infty$.
\end{corollary}
    \begin{proof}
For a graph $G$ define the bipartite graph $B(G)$, with two parts of size $V(G)$,
representing the binary relation $E(G)\subset V(G)\times V(G)$.
Then $\mw_1(B(G))\le O (\mw_1(G))$, as a construction sequence of $G$ can
be easily converted to a construction sequence for $B(G)$.
In particular, $\mw_1(B(G))\le o(|V(G)|)$ for $G\in \CC$.
Moreover, $\VCdim(G)=\VCdim(B(G))$.
The conclusion follows from \Cref{lem:twins-bip}
and \Cref{lem:ntn}, applied to the class $\setof{B(G)}{G\in\CC}$.
    \end{proof}




    Recall that by the Sauer-Shelah-Perles Lemma (see \Cref{lem:sauer-shelah-perles}), 
    if $G$ is a graph of VC-dimension at most $d$,
    then $\pi_G(s)\le O(s^d)$ for all $s\in\N$. The following variant of \Cref{lem:fw}
     therefore gives -- for graphs of fixed VC-dimension -- a polynomial bound on the flip-width parameters, in terms of the merge-width parameters.

    
    \begin{lemma}\label{lem:fw-poly}Fix $r\in\N$.
        For every graph $G$,  $$\fw_r(G)\le \pi_G(\mw_{2r}(G)).$$
    \end{lemma}
\Cref{lem:fw-poly} strengthens 
 \Cref{lem:fw} by replacing the upper bound $4^s$ by $\pi_G(s)$.
    However, \Cref{lem:fw-poly} gives a slightly better dependency on the radius, as the upper bound involves $\mw_{2r-1}(G)$, rather than $\mw_{2r}(G)$.
    
    \Cref{lem:fw-poly} follows from the lemma below, 
    exactly in the same way as \Cref{lem:fw} follows from \Cref{lem:strategy}.
    
    \begin{lemma}\label{lem:strategy-poly}
        Fix $t\in [2,n]$, and 
        let $s$ be the radius-$2r$ width of $(\cal P_t,R_{t-1})$.
        For every $v\in V(G)$ there is a $\pi_G(s)$-flip $G_t'$ of $G$ such that:
    $$B^r_{G_t'}(w)\subset B^r_{G_t}(w)\quad\text{for every $w\in B^r_{G_{t-1}}(v)$.}$$
    \end{lemma}
    
    
    
    
        \begin{proof}Denote $s\coloneqq \mw_{2r}(G)$ and $V\coloneqq V(G)$.
            Fix $v\in V$. 
            For $i=0,\ldots,2r$, let 
            $B_i$ denote the set of vertices  that can be reached by a path of length at most $i$ by from $v$ in the graph $(V,R_{t-1})$.
            Let $\cal Q_i=\setof{A\cap B_{2r}}{A\in\cal P_t,A\cap B_{2r}\subset B_i}$.
    In particular, $|\cal Q_{2r}|\le s$.
    
    
    Observe that
    there are no edges in $G_{t}$ between  $B_{2r-1}$ and $V-B_{2r}$,
    since an edge  $uw\in E(G_t)$ with  $u\in B_{2r-1}$  implies $w\in B_{2r}$,
    as $E(G_{t})\subset R_t\subset R_{t-1}$.
    Since each part of $\cal Q_{2r-1}$ is contained in $B_{2r-1}$ and in some part of $\cal P_t$,
     and $G_t$ is a $\cal P_t$-flip of $G$, 
    it follows that each part of $\cal Q_{2r-1}$ is homogeneous in $G$ towards 
    every vertex $w\in V(G)-B_{2r}$.
    
    
    Let $\cal R$ be the partition of $V(G)-B_{2r}$ which partitions vertices according to their neighborhood in $B_{2r-1}$, in $G$.
    Denote $\cal Q\coloneqq \cal Q_{2r}\cup\cal R$. Then $\cal Q$ is a partition of $V$.
    
    \begin{claim}\label{cl:small-q-poly}
        $|\cal Q|\le \pi_G(s).$
       \end{claim}
       \begin{claimproof}
           For every part $A\in \cal Q_{2r-1}$ pick a vertex $v_A\in B_{2r-1}\cap A$,
           and let $S=\setof{v_A}{A\in\cal Q_{2r-1}}$. In particular, $|S|\le s$.
           It follows from the above that, in the graph $G$, for every vertex $w\in V(G)-B_{2r}$,
           the neighborhood  of $w$ in $B_{2r-1}$ is uniquely determined by the neighborhood of $w$ in $S$; moreover, $V(G)-B_{2r}$ is disjoint from $S$.
           Therefore, $$|\cal R|\le  \pi_G(s)-s.$$
           Together with $|\cal Q_{2r}|\le s$, this yields the conclusion.
       \end{claimproof}
           
    
    Consider the graph $G'$ with vertex set $V$,
    such that for any two parts $A,B\in\cal Q$:
    $$E(G')\cap AB=
    \begin{cases}
        E(G)\cap AB&\text{if $A,B\notin \cal Q_{2r-1}$},\\
        E(G_t)\cap AB&\text{otherwise.}\\
    \end{cases}$$
    
    \begin{claim}\label{cl:k-flip-poly}
        $G'$ is a $\cal Q$-flip of $G$.
    \end{claim}
    \begin{claimproof}
        
        It is enough to show that for all   $A,B\in \cal Q$, 
        \begin{align}
            E(G')\cap AB=E(G)\cap AB\qquad\textit{or}\qquad
            E(G')\cap AB= AB-E(G).\label{eq:flip-poly}    
        \end{align}
        We consider several cases.
    \begin{itemize}
        \item If $A,B\notin \cal Q_{2r-1}$, this is clear by definition of $E(G')$.
        \item Suppose that $A\in\cal Q_{2r-1}$ and $B\in \cal Q_{2r}$.
        Then \eqref{eq:flip-poly} follows, as $E(G')\cap AB=E(G_t)\cap AB$ by definition,
        and $G_t$ is a $\cal P_t$-flip of $G$, and each of the parts $A,B$ is contained in some part of $\cal P_t$.
        \item Suppose that $A\in\cal Q_{2r-1}$ and $B\notin\cal Q_{2r}$. Then 
        $A\subset B_{2r-1}$ and 
        $B\subset V(G)-B_{2r}$, and $B\in\cal R$. By the previous discussion, $E(G_t)\cap AB=\emptyset$. Moreover, each vertex of $b\in B$ is homogeneous in $G$ towards $A$,
        that is, each $b$ is either complete, or anti-complete towards $A$ in $G$.
        Since all vertices of $B$ have equal neighborhoods in $A$
        (by definition of $\cal R$, as $B\in\cal R$ and $A\subset B_{2r-1}$),
        it follows that $B$ is homogeneous in $G$ towards $A$.
    Together with $E(G_t)\cap AB=\emptyset$, this yields \eqref{eq:flip-poly}.
    \end{itemize}
    Up to exchanging the roles of $A$ and $B$, this covers all the cases.
    \end{claimproof}
    
    
    
    
    \begin{claim}\label{cl:balls-poly}
        For all $w\in B^r_{G_{t-1}}(v)$
     and $i=0,\ldots,r$ we have:
        $$B^i_{G'}(w)\subset B^i_{G_t}(w).$$
    \end{claim}
    
    \begin{claimproof}
        We induct on $i$. For $i=0$ the statement is trivial.
    In the inductive step, fix $1\le i\le r$ and $u\in B^i_{G'}(w)$ with $u\neq w$.
    Then there is some  $a\in B^{i-1}_{G'}(w)$ with $ua\in E(G')$.
    By inductive assumption, $a\in B^{i-1}_{G_t}(w)$.
    From  $i\le r$ and  $E(G_t)\subset R_{t-1}$
    we get that $a\in B^{r-1}_{R_{t-1}}(w)$,
    which together with $w\in B^{r}_{G_{t-1}}(v)\subset B^{r}_{R_{t-1}}(v)$ implies $a\in B^{2r-1}_{R_{t-1}}(v)=B_{2r-1}$. In particular, $a\in \bigcup \cal Q_{2r-1}$.
    As $ua\in E(G')$, this implies $ua\in E(G_t)$, by definition of $E(G')$. Together with 
    $a\in B^{i-1}_{G_t}(w)$, this implies that $u\in B^{i}_{G_t}(w)$, as required.
    \end{claimproof}
    \Cref{cl:k-flip-poly,cl:small-q-poly,cl:balls-poly} together prove \Cref{lem:strategy-poly}.
    \end{proof}
   As mentioned, \Cref{lem:strategy-poly} implies \Cref{lem:fw-poly},
   analogously as in the proof of \Cref{lem:fw}.


\thmabmw*
\begin{proof}
    Let $\CC$ be a hereditary class of almost bounded merge-width.
    By \Cref{lem:abmw-vc}, $\CC$  has VC-dimension bounded by some $d\in\N$.
By \Cref{lem:sauer-shelah-perles}, we have that $\pi_G(s)\le O(s^d)$ for all $G\in\CC$ and $s\in\N$.

Fix $r\in\N$ and $\eps>0$.
Then, for every graph $G\in\CC$, we have 
$$\fw_r(G)\le \pi_G(\mw_{2r}(G))\le \pi_G(O_{\CC,r,\eps}(|V(G)|^\eps))\le O_{\CC,r,\eps}(|V(G)|^{\eps\cdot d}).$$
Since $\eps>0$ is arbitrary, this implies that $\CC$ has almost bounded flip-width.
\end{proof}

























    





\begin{figure*}
    \centering
    \vspace{-0.2cm}
    \begin{subfigure}{0.238\textwidth}
    \centering
    \includegraphics[width=1.0\linewidth]{figures/bs_distribution.pdf}
    \caption{Group distribution.}
    \label{fig:pairwise-dis}
    \end{subfigure}
    ~
    \begin{subfigure}{0.245\textwidth}
    \centering
    \includegraphics[width=1.0\linewidth]{figures/bs_distribution_1.pdf}
    \caption{Individual distribution.}
    \label{fig:individual-dis}
    \end{subfigure}
    ~
    \begin{subfigure}{0.232\textwidth}
    \centering
    \includegraphics[width=1.0\linewidth]{figures/bs_ablation.pdf}
    \caption{Group approximation.}
    \label{fig:pairwise-approx}
    \end{subfigure}
    ~
    \begin{subfigure}{0.232\textwidth}
    \centering
    \includegraphics[width=1.0\linewidth]{figures/pw_ablation.pdf}
    \caption{Individual approximation.}
    \label{fig:individual-approx}
    \end{subfigure}
    \caption{Distributions of oracle group (a) and individual (b) influences across different numbers of bootstrapping iterations. Performance of group (c) and individual (d) data influence approximations.}
    \label{fig:pwdim}
    \vspace{-0.3cm}
\end{figure*}

\subsection{Ablation Studies}
\label{sec:ablation}

% \cx{It feels smoother if we start with this ablation right after overall, and then three analyses afterwwards, each illustrating benefit of one tech.}\zichun{I think this part is not as insightful as the other three to be prioritized? The most important message here may be just the importance of relationship modeling.} \cx{it is not insightful thus we can treat it as a lead/entry point to later more insightful ones. end the findings with the least insightful experiemtns is not ideal.}

Table~\ref{tab:ablation} shows the ablation studies of three key components in Group-MATES. When we remove relationship terms in the selection and only consider individual influences, the overall performance gain over random selection decreases by 3.6\%;  discarding the bootstrapping technique and replacing influence-aware clustering with BGE clustering also yield observable performance loss, but the drop is not as significant as removing relationship terms. This experiment highlights the benefits of having relationship measurements between training points in data-efficient pretraining.

\subsection{Analysis on Group-Level Data Influence Modeling}
\label{sec:eff-modeling}

% \zichun{I still tend to show the amplification/cancellation effect identified by our DIM since this is the most direct way to show our effectiveness? We can state that there do exist cases that are identified by oracle as amplification, but our DIM cannot capture well, while from our observation these cases are complex and even hard for human to interpret.}

% To study the characteristics of group-level influence modeling, we compare the overlap and reference loss trajectories of data selected by our relational data influence model with those selected by the individual data influence model~\cite{yu2024mates}.
To illustrate the effectiveness of group-level influence modeling, we compare the reference loss trajectories and the evaluation results of data selected by greedily maximizing group influences~\cite{broderick2020automatic}, our relational data influence model, individual data influence model~\cite{yu2024mates}, and random.
In this experiment, we set the selection ratio to 20\% and utilize the 100-step decay stage of pretraining~\cite{hu2024minicpm} to enlarge the gap between different selections. All analyses are run with five different data splits, and we calculate the mean and std of the metrics to ensure the stability of observations. 
% As shown in Figure~\ref{fig:overlap-dim}, the selected data from these two models quickly diverges as the scale of the data pool increases. 

As shown in Figure~\ref{fig:upperbound-dim}, the subset selected by our relational data influence model consistently achieves a lower reference loss than the individual one after the initial steps. The evaluation results in Figure~\ref{fig:upperbound-eval} further validate the superiority of our relational data influence model, with a 6\% relative performance gain compared to individual selection after the decay. We also emphasize the significant potential of optimal group-level selection, which nearly doubles the performance gain even in the short decay stage. Our method represents a critical step toward efficiently approaching optimal performance and has demonstrated its effectiveness. 

% While a gap remains between our selected data and the optimal group obtained by greedily minimizing reference loss at each step~\cite{nemhauser1978analysis}, our method effectively approaches the optimal performance with the relationship approximation.

\begin{figure*}
    \centering
    \vspace{-0.3cm}
    \begin{subfigure}{0.22\textwidth}
    \centering
    \includegraphics[width=1.0\linewidth]{figures/box_bge.pdf}
    \caption{BGE clustering.}
    \label{fig:box-bge}
    \end{subfigure}
    ~
    \begin{subfigure}{0.22\textwidth}
    \centering
    \includegraphics[width=1.0\linewidth]{figures/box_pairwise.pdf}
    \caption{Influence-aware clustering.}
    \label{fig:box-pairwise}
    \end{subfigure}
    ~
    \begin{subfigure}{0.228\textwidth}
    \centering
    \includegraphics[width=1.0\linewidth]{figures/cluster_corr.pdf}
    \caption{Influence correlation.}
    \label{fig:cluster-corr}
    \end{subfigure}
    ~
    \begin{subfigure}{0.243\textwidth}
    \centering
    \includegraphics[width=1.0\linewidth]{figures/intra_inter.pdf}
    \caption{Relationship distribution.}
    \label{fig:intra-inter}
    \end{subfigure}
    \caption{Comparison of individual influence distributions of BGE (a) and influence-aware clustering (b). Correlation between cluster-averaged influences and original individual influences (c). Relationship term distribution in intra-cluster and inter-cluster scenarios (d).}
    \label{fig:clustering}
    \vspace{-0.5cm}
\end{figure*}

We further compare the distributions of relationship terms identified by our relational data influence model and the oracle. Following~\citet{hu2024most}, we classify these relationships into two types: cancellation and amplification. Cancellation occurs when the relationship term, $\text{sim}(\textbf{h}_{x_{i1}},\textbf{h}_{x_{i2}})$, is greater than 0, whereas amplification occurs when it is lower than 0.
Figure~\ref{fig:dis-rel} shows a strong similarity between our relational data influence model and the oracle in relationship measurements. Figure~\ref{fig:overlap-rel} further reveals that our data influence model effectively captures cancellation effects, but struggles with amplification, particularly in the [-1, -0.5) interval (i.e., tail amplification cases). Explaining and modeling these challenging amplification effects will be an interesting direction for future work. 
% \jie{add some discussions on possible future directions to close the gap between our methods and oracle.}

% While our relational data influence model can represent these interactions, differences remain in the distributions of relationship terms between our model prediction and the oracle, as shown in Figure~\ref{fig:dis-rel}. Here, the oracle relationship term is calculated in the same way as the prediction formula $\Theta(x_{i1}, x_{i2})$, with the predicted individual influence replaced by the oracle. 
% Figure~\ref{fig:overlap-rel} further reveals that the relationship terms that fall in the [-1, -0.5) interval (i.e., tail amplification cases) are the most challenging for our data influence model to capture. 
% We manually observe these cases, but cannot interpret them comprehensively. 

% \cx{show the distribution of learned pariwise weights and oracle pairwise weight-ish, to show how many are cancelation and how many are amplificaiton, then do a case study of each of the four (oracle|learned by cancel|amplify. If learned amplify is not so good, discuss why you think so and make it a potential future work. This is the most important experiment missing. The next one is oracle performances comparisons of group versus individual.}

We also present a case study in Table~\ref{tab:cases}. The cancellation effect in the first example arises from misaligned perspectives on education, where data 1 emphasizes parental influence and data 2 highlights teachers' critical roles. In contrast, the amplification effect in the second example emerges from complementary concepts: data 1 requires gcd for its problem solution, while data 2 provides a formal definition of gcd.
% of artistic exploration, with one data reflecting personal growth through acting and the other analyzing directorial creativity. They complement each other to build a broader scope of the performing arts. 
The case study provides an in-depth analysis of how the relationship term in our relational data influence model shapes complex interactions between training points.

\subsection{Effectiveness of Bootstrapping Influence Models}
\label{sec:eff-bootstrapping}

% Figure: make the probing data distribution more spread-out and thus better for the data influence model to learn. (+ both first data \& second data sampling) (+ multiple iterations)

This set of experiments analyzes the effectiveness of bootstrapping relational data influence models. As shown in Figure~\ref{fig:pairwise-dis} and ~\ref{fig:individual-dis}, with bootstrapping, the sampled oracle distributions of both individual and group influences include more edge cases. This validates that bootstrapping effectively identifies diverse data from the tails to collect oracles. As a result, our relational data influence model better approximates the oracle, as illustrated in Figure~\ref{fig:pairwise-approx}. Specifically, a single iteration of bootstrapping can significantly enhance the upper bound of validation Spearman correlation by 0.2. 
The second iteration offers limited additional benefit, as the first iteration is already effective at mining tail cases. 
Considering the additional cost of the bootstrapping process, we adopt only one iteration of bootstrapping in our final setup. 

% We further investigate the fundamental reasons underlying the effectiveness of bootstrapping.

In Figure~\ref{fig:individual-approx}, we also compare our relational data influence model with the individual one in MATES~\cite{yu2024mates} in terms of individual data influence approximation. Interestingly, although our relational data influence model is not directly optimized to approximate individual data influence, it actually performs better than the individual model. We hypothesize that modeling relational information can refine influence representations $\textbf{h}_{x}$ and thus better capture individual influences. Our findings suggest that the relational data influence model can be a more effective way to approximate the influence, either at the individual or group level.

\subsection{Effectiveness of Influence-Aware Clustering}
\label{sec:eff-clustering}

This experiment demonstrates the advantages of using influence-aware clustering compared to vanilla semantic clustering (e.g., BGE~\cite{xiao2024bge}) in efficient inference. 
% To study whether influence-aware clustering can better group the data with similar influences,
% We first show the influence distributions of the lowest- and highest-influence clusters. 
As shown in Figure~\ref{fig:box-bge} and ~\ref{fig:box-pairwise}, influence-aware clustering can significantly reduce the variance of influence distributions within each cluster compared to BGE. 
% This suggests that data points within the same influence-aware cluster will likely have similar influence representations.
To quantify how well influence-aware clustering can group the data with similar influences, we assign each data with its cluster-averaged influence and calculate the Spearman correlation with the original influences. As illustrated in Figure~\ref{fig:cluster-corr}, influence-aware clustering has a consistently better correlation than BGE clustering with different numbers of clusters. Our results suggest that influence-aware clustering effectively groups similar data based on their influences.

We also examine the distributions of the relationship term (i.e., $\text{sim}(\textbf{h}_{x_{i1}},\textbf{h}_{x_{i2}})$) in intra-cluster and inter-cluster scenarios. As shown in Figure~\ref{fig:intra-inter}, the relationship terms are generally higher in the intra-cluster scenario, implying a greater cancellation effect for the data in each cluster. In contrast, inter-cluster relationship terms are distributed around 0 and less significant than the intra-cluster ones. 
This study shows that influence-aware clustering effectively approximates the relationship computation over the full dataset.

% \jie{do we want to briefly mention that we have some important ablations in the appendix, such as Sec B.1, etc.?}

% Clustering-based selection vs. Gumbel-Top-$k$ or brute-force sequential selection (not efficient), e.g., how many data points are selected within the clusters of highest/lowest influence. We should have better diversity.