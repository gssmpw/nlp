% CVPR 2025 Paper Template; see https://github.com/cvpr-org/author-kit

\documentclass[10pt,twocolumn,letterpaper]{article}

%%%%%%%%% PAPER TYPE  - PLEASE UPDATE FOR FINAL VERSION
% \usepackage{cvpr}              % To produce the CAMERA-READY version
% \usepackage[review]{cvpr}      % To produce the REVIEW version
\usepackage[pagenumbers]{cvpr} % To force page numbers, e.g. for an arXiv version

% Import additional packages in the preamble file, before hyperref
\newcommand{\CG}{\mathcal{G}\xspace}
\newcommand{\CV}{\mathcal{V}\xspace}
\newcommand{\CE}{\mathcal{E}\xspace}
\newcommand{\CA}{\mathcal{A}\xspace}
\newcommand{\CF}{\mathcal{F}\xspace}
\newcommand{\CR}{\mathcal{R}\xspace}
\newcommand{\CB}{\mathcal{B}\xspace}
\newcommand{\CX}{\mathcal{X}\xspace}
\newcommand{\CK}{\mathcal{K}\xspace}
\newcommand{\CM}{\mathcal{M}\xspace}
\newcommand{\CC}{\mathcal{C}\xspace}
\newcommand{\CL}{\mathcal{L}\xspace}
\newcommand{\CI}{\mathcal{I}\xspace}
\newcommand{\CQ}{\mathcal{Q}\xspace}
\newcommand{\CO}{\mathcal{O}\xspace}
\newcommand{\CP}{\mathcal{P}\xspace}
\newcommand{\CS}{\mathcal{S}\xspace}
\newcommand{\CT}{\mathcal{T}\xspace}
\newcommand{\CJ}{\mathcal{J}\xspace}
\usepackage[para]{footmisc}
\usepackage{subfig}
% \usepackage{subcaption}
% \usepackage{array}
% \usepackage{colortbl}


% Useful packages
% Begin RSS packages
\usepackage{times}
% numbers option provides compact numerical references in the text.
\usepackage[numbers]{natbib}
\usepackage{multicol,multirow,array}
%\usepackage[bookmarks=true]{hyperref}
% End RSS packages

% https://tex.stackexchange.com/questions/342312/citation-error-in-using-bibtex-paragraph-ended-before-citex-was-complete
% \usepackage{cite}

\usepackage{float}
\usepackage{amsmath,amssymb,amsfonts,bm,mathtools}
\usepackage{graphicx}
\usepackage{textcomp}
\usepackage{xcolor}
\usepackage{soul}
\usepackage{lipsum}
\usepackage{algorithm, setspace}
\usepackage{algpseudocode}
\usepackage{arydshln}
\usepackage{pifont}

\newcommand{\codecomment}[1]{{\color{Teal}\#~\textit{#1}}}

\def\BibTeX{{\rm B\kern-.05em{\sc i\kern-.025em b}\kern-.08em
    T\kern-.1667em\lower.7ex\hbox{E}\kern-.125emX}}

\usepackage{gensymb}
\usepackage{xspace}
\usepackage{subcaption}
\usepackage{pgfmath} % or tikz
\usepackage{dblfloatfix} % Fix for two-column float placement

\newcommand\eq[1]{Eq.~\eqref{#1}}
\newcommand\fig[1]{Fig.~\ref{#1}}
\newcommand\sect[1]{Section~\ref{#1}}
\newcommand\tab[1]{Table~\ref{#1}}
\newcommand\alg[1]{Algorithm~\ref{#1}}
\newcommand\lin[1]{Line~\ref{#1}}
\newcommand\figref[1]{Fig. #1}
\newcommand\tabref[1]{Table #1}

\DeclareMathOperator*{\argmax}{arg\,max}
\DeclareMathOperator*{\argmin}{arg\,min}
\usepackage{wrapfig}
\makeatletter
\DeclareRobustCommand\onedot{\futurelet\@let@token\@onedot}
\def\@onedot{\ifx\@let@token.\else.\null\fi\xspace}
% \newcommand{\ie}{\textit{i.e.\xspace}}
% \newcommand{\eg}{\textit{e.g.\xspace}}
% \newcommand{\etc}{\textit{etc.\xspace}}
\def\etc{\textit{etc}\onedot}
% \newcommand{\etal}{\textit{et al.\xspace}}

\usepackage{makecell}
\long\def\invis#1{}

% https://tex.stackexchange.com/questions/170772/command-labelindent-already-defined. Only for IEEE style
% \let\labelindent\relax
\usepackage[shortlabels]{enumitem}

% Phantom zeros
\newcommand\z{\phantom{0}}

% List items
\newcommand{\li}[1]{\noindent\textbf{#1}}

% Inline Prompt Styles
\newcommand\DOTS{\hbox to 1em{.\hss.\hss.}}
\newcommand{\Prompt}[1]{``{\tt #1}''}
\newcommand{\DscPrompt}[1]{\texttt{\small{\textbf{[#1]}}}}
\newcommand{\SttPrompt}[1]{\texttt{\small{\textbf{(#1)}}}}
\newcommand{\ObjPrompt}[1]{\texttt{\small{\textbf{\{#1\}}}}}
\newcommand{\ttt}[1]{\texttt{\small{\textbf{#1}}}}
\newcommand{\DIV}{\allowbreak |\allowbreak}

\newcommand{\TODO}[1]{{\color{red}\underline{{\color{gray}{#1}}~\textbf{\textsubscript{[TODO]}}}}}
\newcommand{\NeedCitations}{\TODO{[...]} }
\newcommand{\draft}[1]{{\color{blue}{#1}}}

\newcommand{\Internal}[1]{#1}
\newcommand\JI[1]{\Internal{\noindent{\color{red}{Dr. Islam: #1}}}}
\newcommand\SK[1]{\Internal{\noindent{\color{red}{Dr. Koppal: #1}}}}
\newcommand\YZ[1]{\Internal{\noindent{\color{cyan}{Yuxuan: #1}}}}

% UNCOMMENT COMMANDS BELOW FOR SUBMISSION
% \renewcommand{\Internal}[1]{}
% \renewcommand{\draft}[1]{#1}
\usepackage[colorlinks,bookmarksopen,bookmarksnumbered,citecolor=green,urlcolor=red]{hyperref}
\usepackage{pifont}

\definecolor{darkpastelred}{rgb}{0.76, 0.23, 0.13}

\newcommand{\cmark}{\ding{51}}%
\newcommand{\magma}{\texttt{Magma}\xspace}
\newcommand{\magmapt}{Multimodal Agentic Pretraining}

% \usepackage[table]{xcolor}

% For author commenting
\newcommand{\jw}[1]{\textcolor{red}{[Jianwei: #1]}} %
\newcommand{\qh}[1]{\textcolor{blue}{[Qianhui: #1]}}

\newcommand{\ensuretext}[1]{#1}
\newcommand{\marker}[2]{\ensuremath{^{\textsc{#1}}_{\textsc{#2}}}}
\newcommand{\arkcomment}[3]{\ensuretext{\textcolor{#3}{[#1 #2]}}}
\newcommand{\rt}[1]{\arkcomment{\marker{R}{T}}{#1}{olive}}
\newcommand\blfootnote[1]{\begingroup\renewcommand\thefootnote{}\footnote{#1}\addtocounter{footnote}{-1}\endgroup}

% It is strongly recommended to use hyperref, especially for the review version.
% hyperref with option pagebackref eases the reviewers' job.
% Please disable hyperref *only* if you encounter grave issues, 
% e.g. with the file validation for the camera-ready version.
%
% If you comment hyperref and then uncomment it, you should delete *.aux before re-running LaTeX.
% (Or just hit 'q' on the first LaTeX run, let it finish, and you should be clear).
\definecolor{commentcolor}{RGB}{34,139,34}
\definecolor{cvprblue}{rgb}{0.21,0.49,0.74}
\usepackage[pagebackref,breaklinks,colorlinks,allcolors=cvprblue]{hyperref}

%%%%%%%%% PAPER ID  - PLEASE UPDATE
\def\paperID{1545} % *** Enter the Paper ID here
\def\confName{CVPR}
\def\confYear{2025}

%%%%%%%%% TITLE - PLEASE UPDATE
% \title{Multimodal Agentic Foundation that Can See, Talk and Act}
\title{\includegraphics[height=0.8cm]{figures/magma_logo.png} Magma: A Foundation Model for Multimodal AI Agents}
% \title{Multimodal Agentic Pretraining with Spatial-Temporal Intelligence}
% \title{Multimodal Agentic Pretraining}

% pre-training with spatial-temporal marks for ...

%%%%%%%%% AUTHORS - PLEASE UPDATE
%\author{
%Jianwei Yang\textsuperscript{1*$\dagger$}~\hspace{3pt}
%Reuben Tan\textsuperscript{1$\dagger$}~\hspace{3pt}
%Qianhui Wu\textsuperscript{1$\dagger$}~\hspace{3pt}
%Ruijie Zheng\textsuperscript{2$\ddagger$}~\hspace{3pt}
%Baolin Peng\textsuperscript{2$\ddagger$}~\hspace{3pt}
%Yongyuan Liang\textsuperscript{2$\ddagger$}~\hspace{3pt} \\
%Yu Gu\textsuperscript{1}~\hspace{5pt}
%Mu Cai\textsuperscript{3}~\hspace{5pt}
%Seonghyeon Ye\textsuperscript{4}~\hspace{5pt}
%Joel Jang\textsuperscript{5}~\hspace{5pt}
%Yuquan Deng\textsuperscript{5}~\hspace{5pt}
%Lars Liden\textsuperscript{1}~\hspace{5pt}
%Jianfeng Gao\textsuperscript{1$\bigtriangledown$}~\hspace{5pt}
%\\
%{\small\textsuperscript{1}Microsoft Research\hspace{5pt} 
%\small\textsuperscript{2}University of Maryland\hspace{5pt} 
%\small\textsuperscript{3}University of Wisconsin-Madison\hspace{5pt}
%\\
%\small\textsuperscript{4}KAIST\hspace{5pt}
%\small\textsuperscript{5}University of Washington}\\ 
%\url{https://microsoft.github.io/Magma}
%}

\author{Jianwei Yang\textsuperscript{1*$\dagger$} \ \ \ Reuben Tan\textsuperscript{1$\dagger$} \ \ \ Qianhui Wu\textsuperscript{1$\dagger$} 
\ \ \ Ruijie Zheng\textsuperscript{2$\ddagger$} \ \ \  Baolin Peng\textsuperscript{2$\ddagger$} \ \ \ Yongyuan Liang\textsuperscript{2$\ddagger$} \\
Yu Gu\textsuperscript{1} \ \ \ Mu Cai\textsuperscript{3} \ \ \ Seonghyeon Ye\textsuperscript{4} \ \ \ Joel Jang\textsuperscript{5} \ \ \ Yuquan Deng\textsuperscript{5} \ \ \ Lars Liden\textsuperscript{1} \ \ \ Jianfeng Gao\textsuperscript{1$\bigtriangledown$} \\
$^{1}$\small{Microsoft Research}, $^{2}$University of Maryland, $^{3}$University of Wisconsin-Madison
\\
$^{4}$\small{KAIST}, $^{5}$University of Washington \\
\url{https://microsoft.github.io/Magma}
\\
}


\begin{document}
%\maketitle
%\renewcommand\twocolumn[1][]{#1}
\twocolumn[{%
\renewcommand\twocolumn[1][]{#1}%
\maketitle
\begin{center}
   \captionsetup{type=figure}
    \vspace{-0.8cm}    
\includegraphics[width=1.0\linewidth]{figures/magma_teaser.pdf}    
    \vspace{-0.5cm}
    \caption{We introduce \magma, the \textit{first} foundation model that is capable of interpreting and grounding multimodal inputs within its environment. Given a described goal, \magma is able to formulate plans and execute actions to achieve it. By effectively transferring knowledge from freely available visual and language data, \magma bridges verbal and spatial intelligence to navigate complex tasks.}
    \label{fig:teaser} 
\end{center}   
}]

\blfootnote{$^{\dagger}$ First Authors; $^{\ddagger}$ Second Authors; $^*$ Project Lead; $^{\bigtriangledown}$ Leadership}

% \begin{figure*}[t!]% Use '!t' instead of just 't'
%     \centering
%     \includegraphics[width=1.0\linewidth]{figures/images/magma_fig_1_2_3_reuben_qianhui.pdf}
%     \vspace{-10pt}
%     \caption{We introduce our Multimodal Agentic Model (\magma), that is capable of interpreting and grounding multimodal inputs within its environment. Given a described goal, \magma is able to formulate plans and execute actions to achieve it. By effectively transferring knowledge from freely available visual and language data, \magma bridges verbal and spatial intelligence to navigate complex tasks and settings. \jw{polish this teaser figure to add a radar plot}}
%     \label{fig:teaser}
% \end{figure*}

\begin{abstract}

% Recent works to jointly reconstruct 3D human and object from a single RGB image, are mostly model-based, that fail to capture the fine details of the clothed human body and object surface. In this paper, we introduce ReCHOR, a novel, model-free, first-method to produce realistic clothed human-object reconstructions from a monocular view. This is extremely challenging due to human-object occlusions, diverse interactions and depth ambiguity, as it needs to infer both 3D spatial awareness and high resolution details. Our core idea is based on estimating neural implicit representations for human and object respectively by an attention-based neural implicit model that attends to pixel-aligned features from both the global human-object image for spatial awareness and  the local separate view of human and object images for high quality details. Additionally, the network is conditioned on semantic features from an initial estimated human-object pose prior and a generative diffusion model that inpaints occluded regions, thus enabling the retrieval of details from them.
% We also propose a synthetic dataset with rendered scenes of diverse, inter-occluded 3D human and object scans, to train our network. We evaluate our method on the synthetic and real world BEHAVE dataset. Our experiments show that our method outperforms the SOTA in achieving realistic clothed human-object reconstructions.
Recent approaches to jointly reconstruct 3D humans and objects from a single RGB image represent 3D shapes with template-based or coarse models, which fail to capture details of loose clothing on human bodies. In this paper, we introduce a novel implicit approach for jointly reconstructing realistic 3D clothed humans and objects from a monocular view. For the first time, we model both the human and the object with an implicit representation, allowing to capture more realistic details such as clothing. This task is extremely challenging due to human-object occlusions and the lack of 3D information in 2D images, often leading to poor detail reconstruction and depth ambiguity. To address these problems, we propose a novel attention-based neural implicit model that leverages image pixel alignment from both the input human-object image for a global understanding of the human-object scene and from local separate views of the human and object images to improve realism with, for example, clothing details. Additionally, the network is conditioned on semantic features derived from an estimated human-object pose prior, which provides 3D spatial information about the shared space of humans and objects. To handle human occlusion caused by objects, we use a generative diffusion model that inpaints the occluded regions, recovering otherwise lost details. For training and evaluation, we introduce a synthetic dataset featuring rendered scenes of inter-occluded 3D human scans and diverse objects. Extensive evaluation on both synthetic and real-world datasets demonstrates the superior quality of the proposed human-object reconstructions over competitive methods.
\end{abstract}
\section{Introduction}
\label{sec:intro}
% Image editing methods in diffusion models depend on user-defined control directions - users can unlock their creativity using these methods by specifying the desired manipulation through prompts~\cite{gandikota2023concept}, reference images~\cite{ruiz2022dreambooth, kumari2022customdiffusion, gal2022image, chen2024trainingfreeregionalpromptingdiffusion}, or attribute vectors~\cite{parmar2023zero,hertz2022prompt}. In this work, we ask a fundamentally different question: \emph{Can we automatically discover the underlying visual structure of a concept within diffusion model's knowledge?} %Rather than requiring user-specified controls, we aim to decompose the model's internal knowledge into meaningful directions.

% This question touches on a fundamental limitation in how we interact with diffusion models. Current control methods ~\cite{zhang2023addingconditionalcontroltexttoimage, gandikota2023concept, ye2023ipadaptertextcompatibleimage,ye2023ipadaptertextcompatibleimage, hertz2024stylealignedimagegeneration, li2023photomaker, shi2024instantbooth, chen2024trainingfreeregionalpromptingdiffusion} require users to specify their desired manipulations in advance, limiting interactive creativity. This contrasts with natural human artistic workflows, where creators dynamically explore creative ideas while jointly refining them toward meaningful artistic outcomes~\cite{hoffmann2016modeling}. This synergy between specification and exploration is not new to generative models. Early GAN architectures naturally developed disentangled latent spaces that enabled continuous\cite{harkonen2020ganspace,radford2015unsupervised, wu2021stylespace, shen2020interfacegan}, compositional control over generated images. Users could explore these spaces to discover interesting variations that would be difficult to describe in words~\cite{wu2021stylespace}, then combine them to achieve their creative goals~\cite{grabe2022towards}. 


% While diffusion models have largely superseded GANs in conditional image synthesis~\cite{dhariwal2021diffusion},  their underlying structure remains less understood. Diffusion models achieve remarkable diversity through high-dimensional latents, unlike GANs' compact latent spaces.  With a single prompt, diffusion models can generate radically different variations through different random initializations of input noise. We ask - Is it possible to discover interpretable structure within this vast space of variations?

Text-to-image diffusion models are capable of generating remarkable visual variations from a single prompt through different random initializations. However, this vast creative potential remains largely opaque to users---while we can generate diverse images, we lack understanding of the underlying structure of these variations. This presents a fundamental challenge: how can we discover and expose the latent visual capabilities encoded within these models?

\let\thefootnote\relax \footnote{$^{*}$Correspondence to \texttt{gandikota.ro@northeastern.edu}}

The challenge touches on a key limitation in how we interact with diffusion models today. Current control methods require users to explicitly specify their desired edits in advance through prompts~\cite{gandikota2023concept}, reference images~\cite{zhang2023addingconditionalcontroltexttoimage, chen2024trainingfreeregionalpromptingdiffusion, ruiz2022dreambooth,kumari2022customdiffusion, Ryu_lora, hu2021lora}, or attribute vectors~\cite{ye2023ipadaptertextcompatibleimage, hertz2024stylealignedimagegeneration, li2023photomaker, shi2024instantbooth,parmar2023zero,hertz2022prompt}. That contrasts sharply with natural human creative workflows, where artists dynamically explore creative ideas and jointly refine them toward meaningful artistic outcomes~\cite{hoffmann2016modeling}. The need for pre-specified controls creates a barrier between users and the full creative potential of these models.

Interestingly, earlier generative models like GANs~\cite{gans,karras2019style,brock2018large} naturally developed more interpretable internal structures. Their compact latent spaces often exhibited emergent disentanglement~\cite{harkonen2020ganspace,radford2015unsupervised, wu2021stylespace, shen2020interfacegan}, enabling continuous and compositional control over generated images. Users could explore these spaces to discover interesting variations that would be difficult to describe in words~\cite{wu2021stylespace}, then combine them to achieve their creative goals~\cite{grabe2022towards}.

Diffusion models have largely superseded GANs in conditional image synthesis~\cite{dhariwal2021diffusion}, achieving greater diversity through much higher-dimensional latents. And yet an understanding of the underlying structure of these larger latent spaces has remained elusive. In this work, we ask a fundamental question: \emph{Can we automatically discover the visual structure within a diffusion model's knowledge of a concept?} Rather than requiring user-specified controls, we aim to decompose the model's internal representations into expressive directions that users can explore and combine.

To address these needs, we present \textbf{SliderSpace}, a framework that brings systematic explorability to diffusion models. Given just a text prompt, SliderSpace discovers a canonical set of meaningful, diverse, and controllable directions within the model's knowledge of that concept. Each direction is implemented as a low-rank adapter~\cite{hu2021lora} that can be scaled and composed with others, allowing users to explore and smoothly combine different aspects of variation, as shown in Figure~\ref{fig:intro}.

We ground SliderSpace discovery in three key requirements for meaningful decomposition of a diffusion model's visual manifold: 
\begin{enumerate}
    \item \textbf{Unsupervised Discovery:} The decomposition process should emerge from the intrinsic structure of the model's learned representation, rather than being guided by predefined attributes. This ensures we capture the true topology of the model's knowledge space rather than projecting our assumptions onto it.
    
    \item \textbf{Semantic Orthogonality:} Each discovered control must represent a distinct semantic direction. This is enforced in a semantic feature space, like CLIP, where every slider has an orthogonal effect in embeddings. This prevents discovering multiple controls that create similar semantic effects, making the system more efficient and easier.
    
    \item \textbf{Distribution Consistency:} Directions must induce consistent transformations across both random seeds and prompt variations. 
\end{enumerate}

These requirements naturally lead to our proposed framework, which we formalize in Section~\ref{sec:method}. As we show in our experiments, SliderSpace is architecture-agnostic, working with both conventional U-Net based models like Stable Diffusion~\cite{rombach2022high, rombach2022sd20, podell2023sdxl, turbo, dmd} and recent transformer-based architectures like Flux~\cite{flux}.

We demonstrate the expressiveness of SliderSpace through three applications: First, we show how SliderSpace can decompose high-level concepts into diverse and expressive components, revealing the natural axes of variation in the model's understanding. Second, we explore artistic style variation, where SliderSpace discovers directions that match or exceed the diversity of manually curated artist lists while being judged more useful by human evaluators. Finally, we show how SliderSpace can help reverse the mode collapse commonly observed in distilled diffusion models, restoring diversity while maintaining generation speed.

Beyond providing practical creative control, SliderSpace opens new avenues for understanding and utilizing the latent capabilities of diffusion models. By mapping these models' visual potential into intuitive, composable directions, we take a step toward making their creative possibilities more accessible and interpretable to users.

% Image editing methods in diffusion models unlock the creativity of users. In this work we ask an alternate question: \emph{Can we organize and expose what of the diffusion model is already capable of?}.
% Existing methods for controlling image generation typically require users to manually specify edit directions for desired changes. This process is time-consuming, requires technical expertise, and limits the spontaneity of the creative process. For instance, if a user wants to adjust the smile of a generated person, they must explicitly request this edit, often through imprecise prompt engineering or model fine-tuning. This approach of predefined controls or manual specifications restricts users from fully exploring the latent capabilities of the model. There may be interesting stylistic variations or attributes that the model can generate, but users have no easy way to discover or utilize these.

% Natural visual disentanglement was an emergent property in the latent space of Generative Adversarial Models (GANs) \cite{harkonen2020ganspace,radford2015unsupervised, wu2021stylespace, shen2020interfacegan}. In particular, it has been observed that StyleGAN~\cite{karras2019style} stylespace neurons offer detailed control over many meaningful aspects of images that would be difficult to describe in words~\cite{wu2021stylespace}. However, diffusion models do not share such a compact latent space~\cite{park2023unsupervised}; and efforts to uncover such a space in the semantic embeddings of the text conditioning have met with limited success \nik{Nick - is there a specific citation you were thinking about?}.

% In this work we introduce \textbf{SliderSpace}, which takes a step towards uncovering an analogous low dimensional representation of diffusion models' visual breadth; in essence treating the diffusion model as many generators sharing parameters, where a particular generator is defined by a specific prompt. For a given prompt we sample many random seeds (and optionally prompt expansions using an LLM), generate the corresponding images, and apply an off the shelf feature extractor (in this work CLIP, but our method can be applied to any differentiable feature extractor). We use PCA to analyze these features, and for each of the leading $k$ principal components we train a LoRA \cite{} which causes the diffusion model to produces images which increase the feature magnitude along that component when passed back through the same feature extractor. This leads to a 'Slider' for each principal component, because each LoRA can be scaled and applied to the original diffusion model, continuously varying those visual features in the generated results (as measured, in our case, by CLIP).

% There are many other works that enhance the controllability of diffusion models. One common approach is enabling users to add spatial constraints to a generation either manually, or via a reference image \cite{zhang2023addingconditionalcontroltexttoimage, chen2024trainingfreeregionalpromptingdiffusion}, a second is leveraging more abstract embeddings (e.g. identity, style) extracted from a reference image \cite{ye2023ipadaptertextcompatibleimage, hertz2024stylealignedimagegeneration, li2023photomaker, shi2024instantbooth}, a third is finetuning a foundation model to better generate a concept important to the user \cite{ruiz2022dreambooth, kumari2022customdiffusion, Ryu_lora, hu2021lora}, and a fourth (most relevant to this work) is finding low-rank adaptors of the model based on a prompt or small training set which can be scaled to provide continous control over one aspect of generated image (e.g. night vs day, basic vs luxury, etc.) \cite{gandikota2023concept}. SliderSpace is complementary to all of these methods and offers something distinct. All of the other methods we are aware require the user (and / or model designer) to know in advance what type of control they want. In contrast SliderSpace assists users in discovering and controlling hidden capabilities present in the diffusion model's distribution of possible generations.

%We propose that truly intuitive creative control in a text-to-image model should meet three key criteria: \emph{discoverability}, \emph{intuitiveness}, and \emph{specificity}. The model should reveal controllable attributes that may not be immediately obvious, offer controls that are easy to understand and manipulate, and ensure each control affects a distinct attribute of the generated image.

% We demonstrate the utility and power of SliderSpace using three applications built on top of SDXL-DMD \cite{dmd}, because its fast generation speed lends itself well to the continuous control offered by SliderSpace.

% First, we study concept decomposition (Section \ref{sec:concept_exp}), where we learn sliders for a specific concept (e.g. 'monster', 'waterfall', 'car'). Through quantitative metrics of diversity and text alignment we demonstrate that the learned sliders dramatically boost the diversity of generations when randomly applied without harming text alignment; we also ask humans to qualitatively judge these results in a user study where they find the SliderSpace results to be more 'Diverse', 'Useful', and 'Creative' than our baselines.

% Second, we attempt to compare the automatic discoveries of SliderSpace to a large scale manual study of artistic styles (Section \ref{sec:art_exp}), open-sourced by ParrotZone \cite{parrotzone}. In this study SDXL was prompted with over 4300 artist names,  and based on visual inspection the cases of successful stylistic mimicry recorded. Quantitatively SliderSpace more closely matches the distribution of artistic variation discovered by ParrotZone than other baselines, and in our user studies was judged to be significantly more 'Diverse' and 'Useful' than the baselines. To our surprise humans even judged SliderSpace results to be slightly more 'Diverse' than the results generated by the manually discovered artist names of \cite{parrotzone}.

% Third, we attempt to use SliderSpace to reverse the mode collapse commonly observed in distilled few-step diffusion models relative to the original teacher model (Section \ref{sec:diverse_exp}). We quantitatively demonstrate that applying SliderSpace to SDXL-DMD leads to more closely matching the distribution of images by the original teacher, SDXL.

%Through extensive experiments on various state-of-the-art text-to-image models, we demonstrate that SliderSpace significantly enhances user control and creative expression in AI-assisted image generation tasks. Our method enables a range of applications, including concept decomposition and control, diversity improvement in generated images, customization dissection and edits, and the exploration of artistic styles inherent in the model.

% SliderSpace goes beyond providing a practical tool for enhanced creative control. By mapping the visual potential of diffusion models it can open new avenues for generative creativity and deepens our understanding of each model's hidden potential.
\section{Related Work}

\paragraph{LLMs for Agent tasks.}

Our research is related to deploying large language models (LLMs) as agents for decision-making tasks in interactive environments~\citep{liu2023agentbench,zhou2023webarena,shridhar2020alfred,toyama2021androidenv}. Earlier works, such as~\citep{yao2023webshopscalablerealworldweb}, fine-tuned models like BERT~\citep{devlin2019bertpretrainingdeepbidirectional} for decision-making in simplified environments, such as online shopping or mobile phone manipulation. With the advent of large language models~\citep{brown2020languagemodelsfewshotlearners,openai2024gpt4technicalreport}, it became feasible to perform decision-making tasks through zero-shot or few-shot in-context learning. To better assess the capabilities of LLMs as agents, several models have been developed~\citep{deng2024mind2web,xiong2024watch,hong2023cogagent,yan2023gpt}. Most approaches~\citep{zheng2024seeact,deng2024mind2web} provide the agent with observation and action history, and the language model predicts the next action via in-context learning. Additionally, some methods~\citep{zhang2023building,li2023camel,song2024trial} attempt to distill trajectories from state-of-the-art language models to train more effective policy models. In contrast, our paper introduces a novel framework that automatically learns a reward model from LLM agent navigation, using it to guide the agents in making more effective plans.

\textbf{LLM Planning.} Our paper is also related to planning with large language models. Early researchers~\citep{brown2020languagemodelsfewshotlearners} often prompted large language models to directly perform agent tasks. Later, \citet{yao2022react} proposed ReAct, which combined LLMs for action prediction with chain-of-thought prompting~\citep{wei2022chain}. Several other works~\citep{yao2023treethoughtsdeliberateproblem,hao2023reasoning,zhao2023large,qiao2024agentplanningworldknowledge} have focused on enhancing multi-step reasoning capabilities by integrating LLMs with tree search methods. Our model differs from these previous studies in several significant ways. First, rather than solely focusing on text generation tasks, our pipeline addresses multi-step action planning tasks in interactive environments, where we must consider not only historical input but also multimodal feedback from the environment. Additionally, our pipeline involves automatic learning of the reward model from the environment without relying on human-annotated data, whereas previous works rely on prompting-based frameworks that require large commercial LLMs like GPT-4~\citep{openai2024gpt4technicalreport} to learn action prediction. Furthermore, \Model supports a variety of planning algorithms beyond tree search.

\textbf{Learning from AI Feedback.} In contrast to prior work on LLM planning, our approach also draws on recent advances in learning from AI feedback~\citep{bai2022constitutional,lee2023rlaif,yuan2024self,sharma2024critical,pan2024autonomous,koh2024tree}. These studies initially prompt state-of-the-art large language models to generate text responses that adhere to predefined principles and then potentially fine-tune the LLMs with reinforcement learning. Like previous studies, we also prompt large language models to generate synthetic data. However, unlike them, we focus not on fine-tuning a better generative model but on developing a classification model that evaluates how well action trajectories fulfill the intended instructions. This approach is simpler, requires no reliance on state-of-the-art LLMs, and is more efficient. We also demonstrate that our learned reward model can integrate with various LLMs and planning algorithms, consistently improving their performance.

\textbf{Inference-Time Scaling.} ~\citet{snell2024scaling} validates the efficacy of inference-time scaling for language models. Based on inference-time scaling, various methods have been proposed, such as random sampling~\citep{wang2022self} and tree-search methods~\citep{hao2023reasoning, zhang2024accessing, guan2025rstar}. Concurrently, several works have also leveraged inference-time scaling to improve the performance of agentic tasks. ~\citet{koh2024tree} adopts a training-free approach, employing MCTS to enhance policy model performance during inference and prompting the LLM to return the reward. ~\citet{gu2024your} introduces a novel speculative reasoning approach to bypass irreversible actions by leveraging LLMs or VLMs. It also employs tree search to improve performance and prompts an LLM to output rewards. ~\citet{yu2024exact} proposes Reflective-MCTS to perform tree search and fine-tune the GPT model, leading to improvements in ~\citet{koh2024visualwebarena}. ~\citet{putta2024agent} also utilizes MCTS to enhance performance on web-based tasks such as ~\citet{yao2023webshopscalablerealworldweb} and real-world booking environments. ~\cite{lin2025qlass} utilizes the stepwise reward to give effective intermediate guidance across different agentic tasks. Our work differs from previous efforts in two key aspects: (1) Broader Application Domain. Unlike prior studies that primarily focus on tasks from a single domain, our method demonstrates strong generalizability across web agents, mathematical reasoning, and scientific discovery domains, further proving its effectiveness. (2) Flexible and Effective Reward Modeling. Instead of simply prompting an LLM as a reward model, we finetune a small scale VLM~\citep{lin2023vila} to evaluate input trajectories. %Our reward scores range continuously between 0 and 1, in contrast to existing methods that rely on discrete scoring (e.g., 0 and 1, or 0, 0.5, and 1) through direct LLM prompting.

% Concurrently, several works have also leveraged inference-time scaling to improve the performance of agentic tasks. ~\citet{pan2024autonomous} demonstrates that LLMs and VLMs, such as the GPT series, can function as evaluators or reward models to provide guidance for fine-tuning or reflection, thereby enhancing digital agents. This lays the groundwork for subsequent studies that directly prompt LLMs as reward models. ~\citet{koh2024tree} adopts a training-free approach, employing MCTS to enhance policy model performance during inference. However, it is limited to web environments~\citep{koh2024visualwebarena}. Moreover, its value function relies on prompting an LLM, which is less effective than our proposed method. We validate our approach through ablation studies, demonstrating that our fine-tuned reward model is more effective. ~\citet{gu2024your} introduces a novel speculative reasoning approach to bypass irreversible actions, such as purchasing a product, by leveraging LLMs or VLMs. It also employs tree search to improve performance, but it remains restricted to the web domain~\citep{koh2024visualwebarena, deng2024mind2web}. Additionally, it lacks reward modeling and instead prompts an LLM to output rewards. ~\citet{yu2024exact} proposes Reflective-MCTS to perform tree search and fine-tune the GPT model, leading to improvements in ~\citep{koh2024visualwebarena}. However, this work focuses solely on a single web agent task, and its reward modeling is derived from multi-agent debate, differing from our more effective and efficient reward modeling approach. ~\citet{putta2024agent} also utilizes MCTS to enhance performance, but it is limited to web-based tasks such as ~\citep{yao2023webshopscalablerealworldweb} and real-world booking environments.


\section{Methodology}
\paragraph{Preliminaries.}
We primarily focus on the homologous model merging, in which $\boldsymbol{\theta}_i$ all come from the same base model $\boldsymbol{\theta}_{\rm{base}}$. Given $K$ tasks $\{T_1,T_2,\cdots,T_K\}$ and $K$ corresponding fine-tuned models with parameters $\{\boldsymbol{\theta}_1,\boldsymbol{\theta}_2,\cdots,\boldsymbol{\theta}_K\}$, model merging aims to combine $K$ fine-tuned models into one single model simultaneously performing on $\{T_1,T_2,\cdots,T_K\}$ without post-training~\cite{method_p1_1,method_p1_2}.
Task vector~\cite{ilharco2023editing,yang2024adamerging} is a key element in merging method which could enhances the base model‘s ability or enable the model to handle other tasks. Specifically, for task $T_i$, the task vector $\boldsymbol\tau_i\in \mathbb{R}^D$ is defined as the vector obtained by subtracting the SFT weights $\boldsymbol{\theta}_i$ from the base model weight
$\boldsymbol{\theta}_{\rm{base}}$, \emph{i.e.}, $\boldsymbol\tau_i=\boldsymbol{\theta}_i-\boldsymbol{\theta}_{\rm{base}}$. The merged model could be denoted as $\boldsymbol{\theta}_m=\boldsymbol{\theta}_{\rm{base}}+\sum_i \lambda_i\boldsymbol{\tau}_i$, which $\lambda_i$ is the scaling factor measuring the importance of task vector. For clarification, we also denote the neuron set in $\boldsymbol{\theta}_i$ as $\mathcal{N}_i$, the neuron set in $\boldsymbol{\tau}_i$ as $\mathcal{T}_i$.



\begin{algorithm}[!ht]
    \caption{LED-Merging}
    \label{alg1}
    \begin{algorithmic}[1]
        \REQUIRE  base model $\boldsymbol{\theta}_{\rm{base}}$, SFT models $\{\boldsymbol{\theta}_{i}\mid i\in [K]\}$, mask ratios \{$r_{i} \mid i\in [K]\}$, scaling factors $\{\lambda_i\mid i\in[K]\}$, location datasets $\{\mathcal{X}_{i}\mid i\in[K]\}$
        \ENSURE merged parameter $\boldsymbol{\theta}_{m}$
        \STATE $\mathcal{M}\leftarrow\phi$
        \STATE $\boldsymbol{\theta}_{m}\leftarrow \boldsymbol{\theta}_{\rm{base}}$
        \FOR{$i\in [K]$}
        \STATE $I(\boldsymbol{\theta}_i)=\mathbb{E}_{x\sim \mathcal{X}_i}|\boldsymbol{\theta}_{i}\odot \nabla_{\boldsymbol{\theta}_i}\mathcal{L}(x)|$
        \STATE $I(\boldsymbol{\theta}_{\rm{base}})=\mathbb{E}_{x\sim \mathcal{X}_i}|\boldsymbol{\theta}_{\rm{base}}\odot \nabla_{\boldsymbol{\theta}_{\rm{base}}}\mathcal{L}(x)|$
        
        \STATE calculate $\mathcal{T}^{r_i}_{i}$ following Equation \ref{vote}
        \STATE  $\mathcal{M}\leftarrow \mathcal{M}\cup\{\mathcal{T}^{r_i}_i\}$
       
        
   
        
        
        \ENDFOR  
        \FOR{$i\in [K]$}
        
        \STATE calculate $\text{Disjoint}(\mathcal{T}_i^{r_i})$ use Equation~\ref{disjoint_safety}
        \STATE $\boldsymbol{m}_i \leftarrow \boldsymbol{0}$
        \FOR{$d\in \mathcal{T}_i^{r_i}$}
        \STATE $\boldsymbol{m}_{i,d}=1$
        \ENDFOR
        \STATE $\boldsymbol{\theta}_{m}\leftarrow \boldsymbol{\theta}_{m}+\lambda_i \boldsymbol{\tau}_i\odot \boldsymbol{m}_{i}$
        \ENDFOR
    \end{algorithmic}
\end{algorithm}
    %\vspace{-5pt}
\begin{figure*}[h!]
    \centering
    \includegraphics[width=\linewidth]{figs/pipeline_v2.pdf}
    \vspace{-40mm}
    \caption{Overview of our two-stage training pipeline {\ours}.}
    \label{fig:pipeline}
\end{figure*}


\paragraph{LED-Merging: Location, Election, and Disjoint Merging}
To address the neuron misidentification and interference issues in existing model merging methods, we propose LED-Merging (Location, Election, and Disjoint Merging). Specifically, previous studies \cite{modelstock, ilharco2023editing, tiesmerging} fail to accurately identify safety-related neurons in task vectors with a single magnitude score, namely \textit{neuron misidentification}. Meanwhile, there exists an interference between safety-related and utility-related task vector neurons during the merging process, namely \textit{neuron interference}. To address neuron misidentification, we first locate important neurons both in the base and fine-tuned models and then elect neurons from the task vector considering these two scores together. Subsequently, to mitigate the interference, we introduce a disjoint step, isolating these important neurons so that they influence different base neurons. The whole process is illustrated in Figure~\ref{fig:method}. 




In the location and election step, we consider the importance score from base and fine-tuned models simultaneously to locate task-specific neurons. In this way, it is more accurate than relying on the magnitude score alone because task-specific neurons with high importance score in the fine-tuned model may not necessarily score high in the base model, and vice versa.

{\textbf{Location}}.  We first calculate importance scores for each neuron in a base/fine-tuned model. Given a location dataset $\mathcal{X}_i=\{(x,y)_k\}$, where $x$ is the question and $y$ is the answer, we calculate the importance scores for the weight $\boldsymbol{\theta}_i\in\mathbb{R}^D$ in any  layer as follows~\cite{snip,spareseGPT,sun2024a}:
\begin{equation}
    I(\boldsymbol{\theta}_i)=\mathbb{E}_{x\sim \mathcal{X}_i}[\boldsymbol{\theta}_i\odot \nabla _{\boldsymbol{\theta}_i}\mathcal{L}(x)],
    \label{location}
\end{equation}
which $\mathcal{L}(x)=-\log p(y\mid x)$ is the conditional negative log-likelihood loss. We choose the SNIP score~\cite{snip} because it balances computational efficiency and performance~\cite{cq}. Please refer to Sec.~\ref{sec:ablation} for the comparison between different location methods. After computing importance scores, we choose top-$r_i$ neurons as the important neuron subset $\mathcal{N}_{i}^{r_i}$ from $I(\boldsymbol{\theta}_i)$.
 
 % After computing locating scores, we select the neurons scoring both high in base and fine-tuned models as important neurons in task vectors. Then in the disjoint step,  with preventing  polysemantic neurons  from receiving gradient updates towards different directions,
 % we use set difference to isolate the safety   and utility-related neurons  and construct corresponding masks for merging process,

{\textbf{Election}}. A natural question is how to select important neurons in the task vector $\boldsymbol{\tau}_i$ based on $I(\boldsymbol{\theta}_{\rm{base}})$ and $I(\boldsymbol{\theta}_{i})$. The important neurons in the base model may be different from neurons in the fine-tuned model. Therefore, we introduce the following election strategy to select neurons with high scores in both base and fine-tuned models:
\begin{equation}
    \mathcal{T}_i^{r_i}=\mathcal{N}_i^{r_i}\cap \mathcal{N}_{\rm{base}}^{r_i}.
    \label{vote}
\end{equation}
\emph{Remark}. We compare different choosing methods, including scoring low or high in base or fine-tuned model in Section~\ref{sec:ablation} and find that Equation \ref{vote} achieves the best performance.





{\textbf{Disjoint}}. As important neurons from different task vectors may conflict with each other at the same position, we use the set difference to disjoint the neurons from others to prevent interference:
\begin{equation}
    \text{Disjoint}(\mathcal{T}^{r_i}_{i})=\mathcal{T}^{r_i}_{i}-\mathop{\cup}\limits_{{J}\subsetneqq [K],|J|\geq 2}\mathop{\cap}\limits_{j\in {J}}\mathcal{T}^{r_j}_{j}.
    \label{disjoint_safety}
\end{equation}

Next, we construct a mask $\boldsymbol{m}_i\in\mathbb{R}^D$ to implement disjoint in the merging process. Specifically, this mask $\boldsymbol{m}_i$ is used to select neurons from $\mathcal{T}_i$. The mask ratio is $r_i$, where $r\in(0,1]$. The mask $\boldsymbol{m}_i$ can be derived from:
\begin{equation}
    \boldsymbol{m}_{i,d}=\begin{aligned} &\left\{ \begin{array}{ll} 1, & \text{if } d\in \text{Disjoint}(\mathcal{T}_{i}^{r_i}), \\ 0, & \text{otherwise}. \end{array} \right. \end{aligned}
    \label{mask_safety}
\end{equation}


% \subsection{Merging Models with Masks}
{\textbf{Merging}}. The final
merged task vector $\boldsymbol{\tau}_m$ is as follows:
\begin{equation}
    \boldsymbol{\tau}_m= \sum_i \lambda_i\boldsymbol{\tau}_{i}\odot\boldsymbol{m}_i.
    \label{merged_task_vector}
\end{equation}
We summarize the workflow in Algorithm \ref{alg1}.



\section{Experiments}
\label{sec:experiment}

Experiments are carried out on NVIDIA RTX4090 GPUs using PyTorch 2.2.0 \cite{paszke2019pytorch} and the rotation detection tool kits: MMRotate 1.0.0 \cite{zhou2022mmrotate}. All the experiments follow the same hyper-parameters (learning rate, batch size, optimizer, etc.).

Average precision (AP) is adopted as the primary metric. All the models are configured upon ResNet50 \cite{he2016deep} and trained with AdamW \cite{loshchilov2018decoupled}.
\textbf{1) Learning rate.} Initialized at 5e-5, warm-up for 500 iterations, and divided by ten at each decay step. 
\textbf{2) Epochs.} 72 for HRSC; 12 for the others.
\textbf{3) Augmentation.} Random rotation/flip for HRSC; random flip for the others.
\textbf{4) Image size.} Split into 1,024 $\times$ 1,024 with an overlap of 200 for DOTA/FAIR1M/STAR; scaled to 800 $\times$ 800 for others.
\textbf{5) Multi-scale.} All experiments evaluated without multi-scale technique \cite{zhou2022mmrotate}. 
\textbf{6) Datasets.} Six remote sensing and one retail scene datasets, covering all datasets used by the main counterparts \cite{yu2024point2rbox, luo2024pointobb, cao2023p2rbox}:

\begin{table*}[!tb]
\fontsize{8.5pt}{10pt}\selectfont
\setlength{\tabcolsep}{0.65mm}
\setlength{\aboverulesep}{0.4ex}
\setlength{\belowrulesep}{0.4ex}
\setlength{\abovecaptionskip}{1.5mm}
\centering
\begin{tabular}{l|c|c|c|c|c|c|c|c|c|c}
\toprule
{\textbf{Methods}} & {*} & {\textbf{\,DOTA-v1.0\,}} & {\textbf{\,DOTA-v1.5\,}} & {\textbf{\,DOTA-v2.0\,}} & {\textbf{~~DIOR~~}} & {\textbf{~~HRSC~~}} & {\textbf{\,FAIR1M\,}} & {\textbf{~~STAR~~}} & {\textbf{\,SKU110K\,}} & {\textbf{~~RSAR~~}} \\
\hline
\rowcolor{gray!20} \multicolumn{11}{l}{$\blacktriangledown$ \textit{RBox-supervised OOD}} \\ \hline
RetinaNet (2017) \cite{lin2017focal} & \checkmark & 68.69 & 60.57        & 47.00 & 54.96 & 84.49   & 37.67   & 21.80 & 78.50 & 57.67  \\
GWD (2021) \cite{yang2021rethinking} & \checkmark & 71.66 & 63.27        & 48.87 & 57.60 & 86.67   & 39.11   & 25.30 & 79.16 & 57.80 \\
FCOS (2019) \cite{tian2019fcos} & \checkmark & 72.44 & 64.53        & 51.77    &  59.83  & 88.99  & 41.25   & \textbf{28.10} & 80.09 & \textbf{66.66} \\
S$^2$A-Net (2022) \cite{han2022align} & \checkmark & \textbf{75.81} & \textbf{66.53} & \textbf{52.39} & \textbf{61.41} & \textbf{90.10} & \textbf{42.44}   & 27.30 & \textbf{80.36} & 66.47 \\
\hline
\rowcolor{gray!20} \multicolumn{11}{l}{$\blacktriangledown$ \textit{HBox-supervised OOD}} \\ \hline
Sun et al. (2021) \cite{sun2021oriented} & $\times$ & 38.60 & - & - & - & - & - & - & - & - \\
KCR (2023) \cite{zhu2023knowledge} & \checkmark & - & - & - & - &  79.10  & -  & - & - & -  \\
H2RBox (2023) \cite{yang2023h2rbox} & \checkmark & 70.05 & 61.70        & 48.68    & 57.80 &  7.03  & 35.94  & 17.20 & 57.15 & 49.92    \\
H2RBox-v2 (2023) \cite{yu2023h2rboxv2} & \checkmark & 72.31 & 64.76 & 50.33 & 57.64 & \textbf{89.66} & \textbf{42.27} & \textbf{27.30} & \textbf{70.70} & \textbf{65.16} \\
AFWS (2024) \cite{lu2024afws} & \checkmark & \textbf{72.55} & \textbf{65.92} & \textbf{51.73} & \textbf{59.07} & - & 41.80 & - & - & - \\
\hline
\rowcolor{gray!20} \multicolumn{11}{l}{$\blacktriangledown$ \textit{Point-supervised OOD}} \\ \hline
P2RBox (2024) \cite{cao2023p2rbox}$^\dagger$ & $\times$ & \underline{59.04} & -        & - & - & -   & -  & -  & - & -  \\
PointSAM (2024) \cite{liu2024pointsam}$^\dagger$ & $\times$ & - & - & - & \textbf{46.20} & -   & -  & -  & - & - \\
PointOBB (2024) \cite{luo2024pointobb} & $\times$ & 30.08 & 10.66        & 5.53     &  37.31  & -   & 11.19 & 9.19  & - & 13.80    \\
Point2RBox+SK (2024) \cite{yu2024point2rbox}$^\dagger$ & \checkmark & 40.27 & 30.51        & 23.43    & 27.34 & 79.40   & 20.03 & 7.86  & 3.41 & 27.81    \\
PointOBB-v2 (2025) \cite{ren2024pointobbv2} & $\times$ & 41.68 & 30.59        & 20.64    &  39.56  & -   & 13.36 & 9.00  & 56.63 & 18.99   \\
PointOBB-v3 (2025) \cite{zhang2025pointobbv3} & $\checkmark$ & 41.20 & 31.25 & 22.82 & 37.60 & - & 11.42  & 11.31 & - & 15.84 \\
PointOBB-v3 (2025) \cite{zhang2025pointobbv3} & $\times$ & 49.24 & 33.79 & 23.52 & 40.18 & - & 18.35 & \underline{12.85} & - & 22.60 \\
\rowcolor{gray!20} Point2RBox-v2 (ours) & \checkmark & 51.00 & \underline{39.45} & \underline{27.11} & 34.70 & \underline{82.67} & \underline{25.72} & 7.80 & \underline{64.00} & \underline{28.60}
 \\
\rowcolor{gray!20} Point2RBox-v2 (ours) & $\times$ & \textbf{62.61} & \textbf{54.06}        & \textbf{38.79}   & \underline{44.45}  & \textbf{86.15}   & \textbf{34.71}  & \textbf{14.20} & \textbf{65.64} & \textbf{30.90}    \\
\bottomrule
\specialrule{0pt}{2pt}{0pt}
\multicolumn{11}{l}{$^*$Comparison tracks: \checkmark = End-to-end training and testing; $\times$ = Generating pseudo labels to train the FCOS detector (two-stage training).} \\
\multicolumn{11}{l}{$^\dagger$Using additional priors. P2RBox/PointSAM: Pre-trained SAM model; Point2RBox+SK: One-shot sketches for each class.} \\
\bottomrule
\end{tabular}
\caption{Accuracy (AP$_{50}$) comparisons on the DOTA-v1.0/1.5/2.0, DIOR, HRSC, FAIR1M, STAR, SKU110K, and RSAR datasets.}
\label{tab:exp_other}
\vspace{-4pt}
\end{table*}

\begin{itemize}
    \item \textbf{DOTA \cite{xia2018dota}.} DOTA-v1.0 has 2,806 aerial images annotated with 15 categories, while DOTA-v1.5/2.0 are the extended versions with more small objects and categories.
    
    \item \textbf{DIOR \cite{cheng2022anchor}.} It is an aerial image dataset re-annotated with RBoxes based on its original HBox version \cite{li2020object}, with a high variation in object size and high intra‐class diversity. 

    \item \textbf{HRSC \cite{liu2017hrsc}.} It contains ship instances on the sea and inshore. The train/val/test set includes 436/181/444 images.

    \item \textbf{FAIR1M \cite{sun2022fair1m}.} It has more than 1 million instances and more than 40,000 images for fine-grained object recognition in remote sensing imagery, annotated with 37 categories. The results are evaluated on FAIR1M-1.0.

    \item \textbf{STAR \cite{li2024star}.} It is extensive for scene graph generation, covering more than 210,000 objects with diverse spatial resolutions, classified into 48 fine-grained categories and precisely annotated with oriented bounding boxes. 

    \item \textbf{SKU110K \cite{pan2020dynamic}.} It focuses on the detection of densely packed retail scenes with 110,712 objects in 11,762 images. The density reaches 86 instances per image. 

    \item \textbf{RSAR \cite{zhang2025rsar}.} It is a remote sensing dataset based on Synthetic Aperture Radar (SAR) imagery with 6 categories.

\end{itemize}

\begin{table*}[!tb]
\fontsize{8.5pt}{10pt}\selectfont
\setlength{\tabcolsep}{2.08mm}
\setlength{\aboverulesep}{0.4ex}
\setlength{\belowrulesep}{0.4ex}
\setlength{\abovecaptionskip}{1.5mm}
\hspace{1pt}
\begin{minipage}[t]{0.315\linewidth}
\centering
\begin{tabular}{c|cc|cc}
\toprule
\multirow{2}{*}{$w_\text{O}$} & \multicolumn{2}{c|}{\textbf{DOTA}} & \multicolumn{2}{c}{\textbf{HRSC}} \\
                  & {E2E} & {FCOS} & {E2E} & {FCOS} \\ \midrule
3  & 48.76 & 61.62 & 81.85 & 84.36 \\
5  & 49.81 & 62.44 & 82.46 & 85.76 \\
\rowcolor{gray!20} 10 & \textbf{51.00} & \textbf{62.61} & \textbf{82.67} & \textbf{86.15} \\
30 & 45.88 & 57.83 & 81.56 & 85.61 \\
\bottomrule
\end{tabular}
\caption{Ablation with the weight of $\mathcal{L}_\text{O}$.}
\label{tab:abl_lo}
\end{minipage}
\quad
\begin{minipage}[t]{0.315\linewidth}
\centering
\begin{tabular}{c|cc|cc}
\toprule
\multirow{2}{*}{$w_\text{W}$} & \multicolumn{2}{c|}{\textbf{DOTA}} & \multicolumn{2}{c}{\textbf{HRSC}} \\
                  & {E2E} & {FCOS} & {E2E} & {FCOS} \\ \midrule
3  & 50.85 & 56.78 & 78.42 & 83.49 \\
\rowcolor{gray!20} 5  & \textbf{51.00} & \textbf{62.61} & \textbf{82.67} & \textbf{86.15} \\
10 & 49.15 & 60.54 & 30.37 & 35.13 \\
30 & 42.84 & 52.53 & 23.89 & 25.91 \\
\bottomrule
\end{tabular}
\caption{Ablation with the weight of $\mathcal{L}_\text{W}$.}
\label{tab:abl_lw}
\end{minipage}
\quad
\begin{minipage}[t]{0.315\linewidth}
\setlength{\tabcolsep}{2.04mm}
\centering
\begin{tabular}{c|cc|cc}
\toprule
\multirow{2}{*}{$w_\text{E}$} & \multicolumn{2}{c|}{\textbf{DOTA}} & \multicolumn{2}{c}{\textbf{HRSC}} \\
                  & {E2E} & {FCOS} & {E2E} & {FCOS} \\ \midrule
0.1 & 48.75 & 57.62 & 34.71 & 39.45 \\
\rowcolor{gray!20} 0.3 & 51.00 & 62.61 & \textbf{82.67} & \textbf{86.15} \\
0.5 & \textbf{51.36} & \textbf{62.63} & 76.85 & 85.22 \\
1.0 & 49.05 & 60.63 & 56.59 & 59.59 \\
\bottomrule
\end{tabular}
\caption{Ablation with the weight of $\mathcal{L}_\text{E}$.}
\label{tab:abl_le}
\end{minipage}
\vspace{-4pt}
\end{table*}

\begin{table*}[!tb]
\fontsize{8.5pt}{10pt}\selectfont
\setlength{\tabcolsep}{2.04mm}
\setlength{\aboverulesep}{0.4ex}
\setlength{\belowrulesep}{0.4ex}
\setlength{\abovecaptionskip}{1.5mm}
\hspace{1pt}
\begin{minipage}[t]{0.315\linewidth}
\centering
\begin{tabular}{c|cc|cc}
\toprule
\multirow{2}{*}{$w_\text{ss}$} & \multicolumn{2}{c|}{\textbf{DOTA}} & \multicolumn{2}{c}{\textbf{HRSC}} \\
                  & {E2E} & {FCOS} & {E2E} & {FCOS} \\ \midrule
0.1 & 49.28 & 59.66 & 73.66 & 78.92 \\
\rowcolor{gray!20} 1.0 & \textbf{51.00} & \textbf{62.61} & \textbf{82.67} & \textbf{86.15} \\
3.0 & 49.15 & 59.20 & 1.30  & 1.65 \\
\bottomrule
\end{tabular}
\caption{Ablation with the weight of $\mathcal{L}_\text{ss}$.}
\label{tab:abl_lss}
\end{minipage}
\quad
\begin{minipage}[t]{0.647\linewidth}
\setlength{\tabcolsep}{3.05mm}
\centering
\begin{tabular}{c|c|c||c|c|c}
\toprule
{R / F / S} & {\textbf{DOTA}} & {\textbf{HRSC}} & {R / F / S} & {\textbf{DOTA}} & {\textbf{HRSC}} \\
 \midrule
90\% / 10\% / 0\% & 60.42 & 85.46 & 80\% / 20\% / 0\%  & 59.46 & 84.73 \\
75\% / 0\% / 25\% & 60.79 & 86.22 & 60\% / 15\% / 25\% & 62.38 & 84.21 \\
\cellcolor{gray!20}68\% / 7\% / 25\% & \cellcolor{gray!20}\textbf{62.61} & \cellcolor{gray!20}\textbf{86.15} & 38\% / 37\% / 25\% & 45.87 & 8.56  \\
45\% / 5\% / 50\% & 60.55 & 85.34 & 40\% / 10\% / 50\% & 60.49 & 10.74 \\
\bottomrule
\end{tabular}
\caption{Ablation with the proportion of augmented views in self-supervision.}
\label{tab:abl_pro}
\end{minipage}
\vspace{-10pt}
\end{table*}

\subsection{Main Results on DOTA-v1.0}
\label{sec:experiment-main}

Table \ref{tab:exp_dota} compares Point2RBox-v2 with the state-of-the-art methods, which can be categorized into two tracks: 

\textbf{1) End-to-end training.} These methods apply the trained weakly-supervised detector directly to the test set. Without relying on priors, our approach demonstrates an improvement of 16.93\% (51.00\% vs. 34.07\%) compared to Point2RBox. Even when compared to Point2RBox+SK, which incorporates additional data-side priors (i.e. one-shot examples for each class), our method still outperforms it by 10.73\% (51.00\% vs. 40.27\%).

\textbf{2) Two-stage training.} These methods generate RBox labels on train/val sets, with which the FCOS detector is trained. In this two-stage mode, Point2RBox-v2 achieves an accuracy of 62.61\%, considerably surpassing PointOBB series. Remarkably, it even outperforms the SAM-powered method P2RBox by 3.57\% (62.61\% vs. 59.04\%).

\textbf{Class-wise analysis.} The FCOS detector trained with labels generated by Point2RBox-v2 achieves accuracy nearly equivalent to RBox-supervised FCOS across six high-density categories: SH (86.9\% vs. 87.1\%), SV (79.6\% vs. 79.8\%), LV (76.3\% vs. 79.8\%), PL (88.0\% vs. 89.1\%), ST (82.9\% vs. 84.6\%), and TC (89.1\% vs. 90.4\%). Interestingly, these six high-density categories account for 88\% of DOTA instances. By annotating these categories with points and generating RBoxes using Point2RBox-v2 while labeling the other sparse categories with RBoxes, we can significantly reduce annotation labor without sacrificing much accuracy, highlighting the valuable role our method can play.

\begin{figure*}[t!]
\setlength{\abovecaptionskip}{1.2mm}
\centering
\includegraphics[width=0.96\linewidth]{figs/case.pdf}
\caption{Qualitative analysis on failed cases and overlap cases.}
\label{fig:case}
\vspace{-6pt}
\end{figure*}

\subsection{Results on More Datasets}

The results are displayed in Table \ref{tab:exp_other}.
On more challenging DOTA-v1.5/2.0, Point2RBox-v2 presents a similar trend, 23.47\%/18.15\% higher than PointOBB-v2 in the pseudo-generation track. 
On the ship detection dataset HRSC, the gap between Point2RBox-v2 and RBox-supervised FCOS is only 2.84\% (86.15\% vs. 88.99\%).
DIOR is relatively sparse, leading to less improvement with our methods---lower than PointSAM (44.45\% vs. 46.20\%) but still higher than methods that do not use SAM. 
Our method also provides competitive performance on fine-grained datasets FAIR1M and STAR. 
In addition to remote sensing scenarios, we carry out experiments on SKU110K for densely packed retail scenes. Existing point-supervised methods struggle in this case, whereas Point2RBox-v2 achieves performance on par with HBox-supervised H2RBox (65.64\% vs. 57.15\%).

\begin{table}[!tb]
\fontsize{8.5pt}{10pt}\selectfont
\setlength{\tabcolsep}{1.78mm}
\setlength{\aboverulesep}{0.4ex}
\setlength{\belowrulesep}{0.4ex}
\setlength{\abovecaptionskip}{1.5mm}
\centering
\begin{tabular}{ccccc|cc|cc}
\toprule
\multicolumn{5}{c|}{\textbf{Modules}} & \multicolumn{2}{c|}{\textbf{DOTA}} & \multicolumn{2}{c}{\textbf{HRSC}} \\
$\mathcal{L}_\text{O}$ & $\mathcal{L}_\text{W}$ & $\mathcal{L}_\text{ss}$ & $\mathcal{L}_\text{E}$ & \textit{CP} & {E2E} & {FCOS} & {E2E} & {FCOS} \\ \midrule
\checkmark & & & & & 0.00 & 0.00 & 0.00 & 0.00 \\
\checkmark & \checkmark & & & & 41.54 & 52.98 & 17.96 & 19.64 \\
\checkmark & \checkmark & \checkmark & & & 46.64 & 54.26 & 18.10 & 22.13 \\
\checkmark & \checkmark & \checkmark & \checkmark & & 49.55 & 61.88 & 78.79 & 83.79 \\
& \checkmark & \checkmark & \checkmark & \checkmark & 48.58 & 59.56 & 20.35 & 24.76 \\
\checkmark & & \checkmark & \checkmark & \checkmark & 38.94 & 48.44 & 11.64 & 14.93 \\
\checkmark & \checkmark & \checkmark & & \checkmark & 47.08 & 55.05 & 19.58 & 21.78 \\
\rowcolor{gray!20} \checkmark & \checkmark & \checkmark & \checkmark & \checkmark & \textbf{51.00} & \textbf{62.61} & \textbf{82.67} & \textbf{86.15} \\
\bottomrule
\end{tabular}
\caption{Ablation with incremental addition of modules.}
\label{tab:abl_mod}
\vspace{-4pt}
\end{table}

\begin{table}[!tb]
\fontsize{8.5pt}{10pt}\selectfont
\setlength{\tabcolsep}{2.85mm}
\setlength{\aboverulesep}{0.4ex}
\setlength{\belowrulesep}{0.4ex}
\setlength{\abovecaptionskip}{1.5mm}
\centering
\begin{tabular}{c|c|c||c|c|c}
\toprule
16 & \cellcolor{gray!20}$K\!=\!24$ & 32 & 1.2 & \cellcolor{gray!20}$\beta\!=\!1.6$ & 2.0 \\ \midrule
50.87 & \cellcolor{gray!20}\textbf{51.00} & 48.08 & 48.14 & \cellcolor{gray!20}51.00 & \textbf{51.33} \\
\bottomrule
\end{tabular}
\caption{Ablation with $K$ and $\beta$ in edge loss on DOTA (E2E).}
\label{tab:abl_edgeparam}
\vspace{-4pt}
\end{table}

\begin{table}[!tb]
\fontsize{8.5pt}{10pt}\selectfont
\setlength{\tabcolsep}{1.75mm}
\setlength{\aboverulesep}{0.4ex}
\setlength{\belowrulesep}{0.4ex}
\setlength{\abovecaptionskip}{1.5mm}
\centering
\begin{tabular}{c|cc|cc|cc}
\toprule
\multirow{2}{*}{$\sigma$} & \multicolumn{2}{c|}{Point2RBox} & \multicolumn{2}{c|}{PointOBB-v2} & \multicolumn{2}{c}{Point2RBox-v2} \\
 & {\textbf{DOTA}} & {\textbf{HRSC}} & {\textbf{DOTA}} & {\textbf{HRSC}} & {\textbf{DOTA}} & {\textbf{HRSC}} \\ \midrule
0\%  & 40.27 & 79.40 & 44.85 & - & 62.61 & 86.15 \\
10\% & 39.60 & 78.81 & 42.30 & - & 61.58 & 85.76 \\
30\% & 38.42 & 78.28 & 38.46 & - & 60.31 & 85.71 \\
\bottomrule
\end{tabular}
\caption{Ablation with the inaccuracy in point annotations.}
\label{tab:abl_noise}
\vspace{-10pt}
\end{table}

\subsection{Ablation Studies}
\label{sec:experiment-ablation}

Tables \ref{tab:abl_lo}-\ref{tab:abl_noise} display the ablation studies on DOTA-v1.0 and HRSC. ``E2E'' denotes end-to-end training; ``FCOS'' denotes two-stage training (i.e. generating pseudo labels to train FCOS). The final values adopted are highlighted in gray.

\textbf{Weight of each loss.} Tables \ref{tab:abl_lo}-\ref{tab:abl_le} determine the weights of the proposed losses. Based on these experiments, the weights $(w_\text{O},w_\text{W},w_\text{E},w_\text{ss})$ are set to $(10, 5, 0.3, 1)$.

\textbf{Proportion of augmented views.} Table \ref{tab:abl_pro} studies the proportion between rotation, flip, and scale. The results are reported with two-stage training (FCOS). Based on the results, the proportion is set to 68\%, 7\%, and 25\%.

\textbf{Incremental addition of modules.} Table \ref{tab:abl_mod} demonstrates the constraints from Gaussian and Voronoi achieve an accuracy of 52.98\% on DOTA. Adding consistency loss and edge loss further boosts it to 54.26\% and 61.88\%, respectively, whereas the improvement from copy-paste is 0.73\%. We also demonstrate the impact of omitting each core loss.

\textbf{Edge loss parameters.} We set $K=24$ and $\beta=1.6$ as they are observed to discern the correct edges during code development. Table \ref{tab:abl_edgeparam} provides a more precise ablation.

\textbf{Annotation inaccuracy.} We offset the annotated points by a noise from the uniform distribution $\left[-\sigma H, +\sigma H \right ]$, where $H$ is the height of objects. Table \ref{tab:abl_noise} shows that the AP$_{50}$ of Point2RBox-v2 decreases by less than 3\% when noise is added to point annotations, demonstrating the robustness of the proposed learning mechanisms.

\subsection{More Discussions}
\label{sec:experiment-discussions}

The qualitative analysis on the failed/overlap cases is shown in Fig. \ref{fig:case}. \textbf{1) Failed cases.} Although our method performs well overall, it struggles with certain categories that are sparse and not constrained by other objects. \textbf{2) Overlap cases.} 
Minimizing overlap as a soft constraint during training does not entirely eliminate overlap. Once trained, the model remains robust to some overlap during inference.

\section{Conclusion}

%In this paper, w
We propose a new PEFT method called DiffoRA, which enables efficient and adaptive LLM fine-tuning based on LoRA. 
Instead of adjusting every interior rank, 
%of the decomposition matrices 
%of all modules, 
we argue that adopting LoRA module-wisely is sufficient. 
To achieve this, we construct a DAM to select the modules that are most suitable and essential to fine-tune. We theoretically analyze how the DAM impacts the convergence rate and generalization capability.
%of the pre-trained model. 
Furthermore, we adopt continuous relaxation and discretization to establish DAM.
%for each task. 
To alleviate the issue of discretization discrepancy, we utilize the weight-sharing strategy for optimization. 
%We fully implement our method and t
The experimental results demonstrate that our DiffoRA works consistently better than the baselines across all benchmarks. 

{
    \small
    \bibliographystyle{ieeenat_fullname}
    \bibliography{main}
}

\clearpage
% \setcounter{page}{1}
% \maketitlesupplementary
\begin{center}
Supplementary Material
\end{center}

% {
%     \onecolumn
%     \centering
%     \Large
%     \textbf{\thetitle}\\
%     \vspace{0.5em}Supplementary Material \\
%     \vspace{1.0em}
% }

\section{Proof of \cref{theorem:dr}}
We require some additional regularity assumptions:
\begin{assumption} 1) The number of classes $C$ is bounded w.r.t the number of samples $N$, 2) the missingness mechanism $P(A=1|Y,\theta)$, as well as its estimated counterpart $P(A=1|Y,\theta)$, are bounded below by some constant $\epsilon > 0$, 3) the quantities $P(Y|X,\theta)$ and $P(A|Y,\theta)$ are estimated using auxiliary samples independent of samples used for the sample averaging.
\label{assumption:extra}
\end{assumption}
Assumptions 1 and 2 are natural. For the missingness mechanism, the ground truth being bounded means that there is a non-vanishing proportion of samples for every class. The boundedness of the estimate can be enforced by clipping the estimate. Assumption 3 is called sample splitting in \cite{kennedy-dr}.

For convenience we use operator $\E_N$ to denote the average of $N$ samples i.e. $\frac{1}{N}\sum_{i=1}^N$. Note that this is by itself a random variable, in contrast to $\E$ which is a fixed number.

\begin{proof}[Proof of \cref{theorem:dr}] Because $C$ is bounded (assumption \ref{assumption:extra}), we can fix a class $c$ and prove the theorem.
Let us define the influence function $\phi$, parameterized by $\theta$, as
\begin{equation}
\phi(O | \theta)(c) = P(Y=c|X,\theta) + \frac{\one(A=1)}{P(A=1|Y,\theta)} (\one(Y=c) - P(Y=c|X,\theta)) - P(Y=c)
\end{equation}
As we have done in the main text, we use $\phi(O)$ to denote the same function but all estimated quantities are replaced with their truths. In other words, we use $\phi(O)$ for $\phi(O|\theta_0)$ where $\theta_0$ is the truth, given that our model contains $\theta_0$ e.g. when the model is consistent.

Recall that:
\begin{equation}
\begin{aligned}
\Psi_{dr}(\theta)(c) &= \frac{1}{N}\sum_{i=1}^N \left\{P(Y=c|X,\theta) + \frac{\one(A=1)}{P(A=1|Y,\theta)} (\one(Y=c) - P(Y=c|X,\theta))\right\}\\
&= \E_N [\phi(O|\theta)(c)] + P(Y=c)
\end{aligned}
\end{equation}

We will show that:
\begin{equation}
\Psi_{dr}(\theta)(c) - P(Y=c) = (\E_N - \E)[\phi(O)(c)] + o_P(N^{-1/2})
\label{eq:proof-linearity}
\end{equation}
To do that, we use the following decomposition
\begin{equation}
\begin{aligned}
\Psi_{dr}(\theta)(c) - P(Y=c) &= \E_N [\phi(O|\theta)(c)] \\
&= (\E_N - \E)[\phi(O)(c)] + (\E_N - \E)[\phi(O|\theta)(c) - \phi(O)(c)] + \E[\phi(O|\theta)(c)]
% &+ (\E_n - \E)[\phi(O;\theta) - \phi(O)]\\
% &+ \E[P(Y=c|X,\theta)] - \E[P(Y=c|X)] + \E[\phi(O,\theta)]
\end{aligned}
\end{equation}
and analyze the second and third term. The third term is:
\begin{equation}
\begin{aligned}
\E[\phi(O|\theta)(c)] &= \E[P(Y=c|X,\theta)] + \E\left[\frac{\one(A=1)}{P(A=1|Y,\theta)}(\one(Y=c) - P(Y=c|X,\theta))\right]- P(Y=c) \\
&= \E\left[P(Y=c|X,\theta) + \frac{P(A=1|Y)}{P(A=1|Y,\theta)}(P(Y=c|X) - P(Y=c|X,\theta))\right] - \E[P(Y=c|X)]\\
&= \E\left[(P(Y=c|X,\theta) - P(Y=c|X)) (P(A=1|Y,\theta) -P(A=1|Y)) \frac{1}{P(A=1|Y,\theta)}\right]\\
\end{aligned}
\end{equation}
by Cauchy-Schwarz inequality:
\begin{equation}
\begin{aligned}
\E[\phi(O|\theta)(c)] &\le \frac{1}{\epsilon} \|P(A=1|Y,\theta) - P(A=1|Y)\|_2 \|P(Y=c|X,\theta) - P(Y=c|X)\|_{L_2(P)}\\
&= \frac{1}{\epsilon} o_P(N^{-1/4} N^{-1/4}) = o_P(N^{-1/2})
\end{aligned}
\end{equation}
by assumption \ref{assumption:4th-root-n} and that $P(A=1|Y,\theta) > \epsilon$ (assumption \ref{assumption:extra}). The second term can be bounded by Chebyshev inequality
% \begin{equation}
% \begin{aligned}
% \E[\E_N[\phi(O|\theta)(c) - \phi(O)(c)]] &= \E[\phi(O|\theta)(c) - \phi(O)(c)]\\
% \var[\E_N[\phi(O|\theta)(c) - \phi(O)(c)]] &= \frac{1}{N}\var[\phi(O|\theta)(c) - \phi(O)(c)] \le 
% \end{aligned}
% \end{equation}
\begin{equation}
P(|(\E_N - \E)[\phi(O|\theta)(c) - \phi(O)(c)]| \ge t) \le \frac{\var[\E_N[\phi(O|\theta)(c) - \phi(O)(c)]]}{t^2} = \frac{\var[\phi(O|\theta)(c) - \phi(O)(c)]}{Nt^2}
\end{equation}
note here that $\theta$ is independent of the samples used for $\E_N$ by assumption \ref{assumption:extra}. For any $\varepsilon > 0$, by picking $t = \frac{1}{\sqrt{N\varepsilon}}$ we get
\begin{equation}
P\left(\left|\frac{(\E_N - \E)[\phi(O|\theta)(c) - \phi(O)(c)]}{N^{-1/2}}\right| \ge \frac{1}{\sqrt{\varepsilon}}\right) \le \varepsilon \var[\phi(O|\theta)(c) - \phi(O)(c)]
\end{equation}
by the definition of $O_P$, we then get
\begin{equation}
(\E_N - \E)[\phi(O|\theta)(c) - \phi(O)(c)] = O_P(N^{-1/2}\var[\phi(O|\theta)(c) - \phi(O)(c)])
\end{equation}
Because $\phi$ is a continuous function of $P(Y|X,\theta)$ and $P(A|Y,\theta)$ (given $P(A|Y,\theta) > \epsilon$, assumption \ref{assumption:extra}), by the continuous mapping theorem and the fact that $P(Y|X,\theta)$ and $P(A|Y,\theta)$ are convergent in probability (assumption \ref{assumption:4th-root-n}), we get $\var[\phi(O|\theta)(c) - \phi(O)(c)] = o_P(1)$. This gives
\begin{equation}
(\E_N - \E)[\phi(O|\theta)(c) - \phi(O)(c)] = o_P(N^{-1/2})
\end{equation}
Therefore, we have shown that the second and third term are both $o_P(N^{-1/2})$, proving \cref{eq:proof-linearity}. As the final step, multiply both sides of this equation by $\sqrt{N}$ we get:
\begin{equation}
\sqrt{N}(\Psi_{dr}(\theta)(c) - P(Y=c)) = \sqrt{N} (\E_N - \E)[\phi(O)(c)] + o_P(1) \rightsquigarrow \mathcal{N}(0, \var[\phi(O)(c)])
\end{equation}
by the central limit theorem, and $\var[\phi(O)(c)] = \E[\phi(O)(c)^2]$ because $\E[\phi(O)(c)] = 0$.
\end{proof}

While we started with the definition of $\phi$, \cref{eq:proof-linearity} shows that $\phi$ is indeed an influence function. Now we show that $\phi$ is also the efficient influence function, by using the characterization of the model's tangent space \cite{tsiatis-missingdata}. Note that the joint probability factorizes as $P(X,A,Y) = P(X)P(Y|X)P(A|Y)$, therefore the tangent space $\mathcal{T}$ factorizes as $\mathcal{T} = \mathcal{T}_{X} \oplus \mathcal{T}_{Y|X} \oplus \mathcal{T}_{A|Y}$ where $\mathcal{T}_X = \{h(X): \E[h] = 0\}$, $\mathcal{T}_{Y|X} = \{h(X,Y): \E[h|X] = 0\}$, $\mathcal{T}_{A|Y} = \{h(A,Y): \E[h|Y] = 0\}$, and the 3 subspaces are pairwise orthogonal. All influence functions are orthogonal to the tangent space, but the influence function that is also in the tangent space has the smallest variance and is called the efficient influence function. As $\phi$ is already an influence function, we need only show that $\phi$ is in $\mathcal{T}$. We write $\phi$ as
\begin{equation}
\phi(O)(c) = (P(Y=c|X) - P(Y=c)) + \left[\frac{\one(A=1)}{P(A=1|Y)} - 1\right](\one(Y=c) - P(Y=c|X)) + (\one(Y=c) - P(Y=c|X))
\end{equation}
and note that the first, second and third term are in $\mathcal{T}_X$, $\mathcal{T}_{A|Y}$ and $\mathcal{T}_{Y|X}$ respectively. Therefore, $\phi$ is indeed in $\mathcal{T}$. The efficient influence function has the smallest variance of all influence function, and therefore our estimator being asymptotically linear in $\phi$ (\cref{eq:proof-linearity}) has the smallest mean squared error in a local asymptotic minimax sense \cite{kennedy-dr, asymptoticstatistics}

\section{Further background and related work}
\paragraph{Discussion on semi-supervised EM.}
It appears that semi-supervised EM was first used for parameter estimation when the missingness mechanism is non-ignorable in \cite{ibrahim1996parameter}, but has not been used for label shift estimation.
Perhaps this is because the semi-supervised situation where additional unlabeled data is available during training is rarer than the test-time adaptation case. EM is well suited to take advantage of the extra unlabeled data to improve the classifier under very scarce and long-tailed labeled data. While the connection between pseudo-labeling and EM has been explored before \cite{entropyminimization}, the situation with label shift has not until recently \cite{simpro}. Here the application of EM is much more interesting, because other than simply giving pseudo-labeling a rigorous formulation, EM also estimates the missingness mechanism (equivalently the label distribution shift), which is important for shift correction and thus high-quality pseudo-labels \cite{acr}. The application of confidence thresholding can be seen as a sparse variant of EM \cite{neal1998view}.

\paragraph{The doubly-robust risk.} 
\label{subsec:dr-risk}
A technique that also derives from the theory of semi-parametric efficiency is orthogonal statistical learning \citep{foster2023orthogonal}. The idea is to minimize the doubly-robust risk:
\label{subsec:method-dr-risk}
\begin{equation}
\label{eq:dr-risk}
\mathcal{R}(\theta_2) = \frac{1}{N} \sum_{i=1}^N \Bigg[ l(x_i, \hat y_i|\theta_2) + \frac{\one(a_i=1)}{P(A=a_i|Y=y_i, \theta_1)} (l(x_i, y_i | \theta_2) - l(x_i, \hat y_i | \theta_2))\Bigg]
\end{equation}
where $l(x,y|\theta) = -\sum_{c=1}^C [y]_c \log P(Y=c|X=x,\theta)$ is the negative cross-entropy. 
The notation $[y]_c$ means that we are using the $c$-entry in a C-dimension probability vector $y$. 
Thus, $y_i$ denotes the one-hot label of observation $i$, while $\hat y_i$ denotes the pseudo-label, which can be one-hot or all-zero. 
Finally, we use $\theta_1$ to denote that $P(a|y,\theta_1)$ is an estimation from a previous stage, but it can be estimated with $\theta_2$ as well. 
The risk $\mathcal{R}(\theta_2)$ can be used as a training loss in a straightforward fashion. 
Similar to the doubly robust estimation of $P(Y)$, the doubly robust risk provides approximately unbiased estimation of the risk. 
This property has been used in \citep{arelabelsinformative, onnonrandommissinglabels, drst} also in the semi-supervised learning setting.
More broadly, it is at the heart of one of the core techniques in heterogenous treatment effect estimation in causal estimation \cite{kennedy2023towards, foster2023orthogonal, wager2018estimation}. 
The focus here is not the estimation of $\mathcal{R}(\theta_2)$ per se, but the quality of the learned model \cite{foster2023orthogonal}.
By using the doubly-robust risk, we can achieve an optimality result similar in spirit to our theorem \cref{theorem:dr}, but for the generalization error.
While this is appealing, in practice there are 2 problems with this approach. First, the inverse probability weight $P(A=a_i|Y=y_i,\theta_1)$ can be very large if the class ratio is highly unlabeled, making training unstable \cite{kallus2020deepmatch, pham2023stable}. 
This problem exists for our estimation as well. However, it is much easier to control for estimation than for training because of the iterative nature of model update. Secondly, we can further write $\mathcal{R}$ as:
\begin{equation}
\mathcal{R}(\theta_2) = \frac{1}{N}\sum_{i=1}^N l\left(x_i, \hat y_i + \frac{\one(a_i=1)}{P(A=a_i|Y=y_i,\theta_1)} (y_i - \hat y_i)\Bigg\vert\theta_2\right)
\end{equation}
which is a cross-entropy loss with new meta-pseudo-labels. However, these labels are not meant to be learned exactly, and furthermore they can be negative. Thus, theoretical works have to put stringent assumptions on the models. In \cref{subsec:ablation-1}, we show that experimentally that the instability problem makes doubly-robust risk performance worse than our 2-stage approach.

\section{Training and hyperparameter settings.}
\label{subsec:training-setting}
For neural network training, we follow the implementation and hyperparameter settings of \cite{simpro}. In particular, we adapt the core code of SimPro for Supervised, MLE and EM. For MLE, we update $P(A|Y)$ using the Adam optimizer with learning rate 1e-3, while for EM we use a momentum update similar to SimPro's update of $P(Y|A)$ because it has a a closed-form solution at each mini-batch. We use Wide ResNet-28-2 on all methods and all datasets in this section, including Imagenet-127, because we are motivated by the fact that stage-1's goal is not classification accuracy but the estimation of a finite-dimensional parameter. When using Wide ResNet-28-2 for Imagenet-127, we use the hyperparameters of CIFAR-100, except we lower the batch size of unlabeled data to 2 times that of labeled data instead of 8 for memory reason. We do not perform additional hyperparameter tuning. All experiments can be performed on 1 A6000 RTX GPU, and are run 3 times. We report the total variation distance between the estimated and the ground truth unlabeled class distribution, similar to its usage in Theorem 3.1 of \cite{lsc}, and the top-1 classification accuracy.

In the second stage of our algorithm, we freeze our estimation and plug it in SimPro and BOAT.
We keep exactly the same hyperparameter settings that SimPro and BOAT use. In particular, for Imagenet-127, we now use ResNet-50 and run each experiment once.
In SimPro, we set the unlabeled class distribution $P(Y|A=0)$ at the E-step;  however, we still keep a running estimate of the class distribution $P(Y)$ in the logit adjustment loss \cref{eq:simpro-la-loss}. While it is possible to use the first stage estimate in the logit adjustment loss, we observe that doing so results in lower accuracy than using the the running average. This is conceptually consistent with the role of the running average - serving not as an accurate estimate of $P(Y)$ but to make the classifier's class distribution uniform through the logit adjustment loss, which is good for the test set. Similarly, in BOAT, we only replace $\Delta_c = \log P(Y|A=1) - \log P(Y|A=0)$ in equation (4) of \cite{boat}, which is adjusting a classifier's predictions from the labeled to the unlabeled class distribution, with our SimPro + DR estimate instead of their on-the-fly estimate. 


% \section{Additional experiments}
% % \begin{table*}[t]
\centering
\caption{Total Variation Distance on CIFAR-10-LT ($N_l = 500$, $M_l = 4000$) with different class imbalance ratios $\gamma_l$ and $\gamma_u$ under five different unlabeled class distributions.}
\label{tab:cifar10-tv}
\resizebox{\textwidth}{!}{
\begin{tabular}{lccccccccccc}
\toprule
& & \multicolumn{2}{c}{consistent} & \multicolumn{2}{c}{uniform} & \multicolumn{2}{c}{reversed} & \multicolumn{2}{c}{middle} & \multicolumn{2}{c}{head-tail} \\
\cmidrule(lr){3-4} \cmidrule(lr){5-6} \cmidrule(lr){7-8} \cmidrule(lr){9-10} \cmidrule(lr){11-12}
& & $\gamma_l = 150$ & $\gamma_l = 100$ & $\gamma_l = 150$ & $\gamma_l = 100$ & $\gamma_l = 150$ & $\gamma_l = 100$ & $\gamma_l = 150$ & $\gamma_l = 100$ & $\gamma_l = 150$ & $\gamma_l = 100$ \\
Model & Estimator & $\gamma_u = 150$ & $\gamma_u = 100$ & $\gamma_u = 1$ & $\gamma_u = 1$ & $\gamma_u = 1/150$ & $\gamma_u = 1/100$ & $\gamma_u = 150$ & $\gamma_u = 100$ & $\gamma_u = 150$ & $\gamma_u = 100$ \\
\midrule
Supervised & MLLS & 0.269 ± 0.252 & 0.038 ± 0.006 & 0.251 ± 0.046 & 0.255 ± 0.060 & 0.429 ± 0.028 & 0.493 ± 0.050 & 0.333 ± 0.042 & 0.320 ± 0.009 & 0.457 ± 0.034 & 0.444 ± 0.043 \\
Supervised & RLLS & 0.043 ± 0.001 & 0.044 ± 0.010 & 0.348 ± 0.034 & 0.305 ± 0.068 & 0.769 ± 0.016 & 0.678 ± 0.028 & 0.430 ± 0.008 & 0.368 ± 0.013 & 0.539 ± 0.018 & 0.503 ± 0.020 \\
\midrule
MLE & IPW & 0.027 ± 0.001 & 0.027 ± 0.000 & 0.319 ± 0.072 & 0.243 ± 0.010 & 0.674 ± 0.020 & 0.646 ± 0.041 & 0.438 ± 0.020 & 0.454 ± 0.026 & 0.547 ± 0.049 & 0.491 ± 0.059 \\
MLE & OR & 0.045 ± 0.004 & 0.042 ± 0.000 & 0.215 ± 0.026 & 0.203 ± 0.032 & 0.433 ± 0.017 & 0.395 ± 0.033 & 0.193 ± 0.006 & 0.209 ± 0.037 & 0.307 ± 0.147 & 0.249 ± 0.130 \\
MLE & DR & 0.090 ± 0.002 & 0.079 ± 0.000 & 0.407 ± 0.027 & 0.360 ± 0.007 & 0.425 ± 0.007 & 0.421 ± 0.029 & 0.256 ± 0.001 & 0.286 ± 0.031 & 0.435 ± 0.136 & 0.362 ± 0.122 \\
\midrule
EM & IPW & 0.035 ± 0.002 & 0.040 ± 0.001 & 0.021 ± 0.001 & 0.029 ± 0.015 & 0.303 ± 0.187 & 0.091 ± 0.010 & 0.119 ± 0.011 & 0.105 ± 0.022 & 0.104 ± 0.026 & 0.104 ± 0.051 \\
EM & OR & 0.037 ± 0.003 & 0.042 ± 0.002 & 0.016 ± 0.001 & 0.024 ± 0.012 & 0.269 ± 0.183 & 0.090 ± 0.008 & 0.122 ± 0.012 & 0.103 ± 0.022 & 0.072 ± 0.012 & 0.073 ± 0.024 \\
EM & DR & 0.034 ± 0.004 & 0.037 ± 0.001 & 0.014 ± 0.001 & 0.027 ± 0.020 & 0.264 ± 0.191 & 0.092 ± 0.005 & 0.111 ± 0.019 & 0.097 ± 0.026 & 0.077 ± 0.016 & 0.073 ± 0.028 \\
\midrule
SimPro & IPW & 0.070 ± 0.011 & 0.058 ± 0.000 & 0.046 ± 0.001 & 0.049 ± 0.005 & 0.254 ± 0.074 & 0.223 ± 0.098 & 0.097 ± 0.025 & 0.067 ± 0.002 & 0.105 ± 0.066 & 0.110 ± 0.079 \\
SimPro & OR & 0.071 ± 0.012 & 0.058 ± 0.000 & 0.045 ± 0.001 & 0.049 ± 0.006 & 0.040 ± 0.003 & 0.059 ± 0.017 & 0.074 ± 0.006 & 0.075 ± 0.002 & 0.033 ± 0.003 & 0.033 ± 0.003 \\
SimPro & DR & 0.017 ± 0.004 & 0.026 ± 0.001 & 0.019 ± 0.002 & 0.018 ± 0.003 & 0.039 ± 0.003 & 0.058 ± 0.025 & 0.091 ± 0.007 & 0.031 ± 0.001 & 0.015 ± 0.003 & 0.019 ± 0.007 \\
\bottomrule
\end{tabular}
}
\end{table*}
% 

\begin{table*}[t]
\centering
\caption{Total Variation Distance on CIFAR-100-LT ($N_l = 50$, $M_l = 400$) with different class imbalance ratios $\gamma_l$ and $\gamma_u$ under five different unlabeled class distributions.}
\label{tab:cifar100-tv}
\resizebox{\textwidth}{!}{
\begin{tabular}{lccccccccccc}
\toprule
& & \multicolumn{2}{c}{consistent} & \multicolumn{2}{c}{uniform} & \multicolumn{2}{c}{reversed} & \multicolumn{2}{c}{middle} & \multicolumn{2}{c}{head-tail} \\
\cmidrule(lr){3-4} \cmidrule(lr){5-6} \cmidrule(lr){7-8} \cmidrule(lr){9-10} \cmidrule(lr){11-12}
& & $\gamma_l = 20$ & $\gamma_l = 10$ & $\gamma_l = 20$ & $\gamma_l = 10$ & $\gamma_l = 20$ & $\gamma_l = 10$ & $\gamma_l = 20$ & $\gamma_l = 10$ & $\gamma_l = 20$ & $\gamma_l = 10$ \\
Model & Estimator & $\gamma_u = 20$ & $\gamma_u = 10$ & $\gamma_u = 1$ & $\gamma_u = 1$ & $\gamma_u = 1/20$ & $\gamma_u = 1/10$ & $\gamma_u = 20$ & $\gamma_u = 10$ & $\gamma_u = 20$ & $\gamma_u = 10$ \\
\midrule
Supervised & MLLS & 0.707 ± 0.016 & 0.313 ± 0.100 & 0.445 ± 0.172 & 0.309 ± 0.119 & 0.383 ± 0.075 & 0.397 ± 0.006 & 0.570 ± 0.001 & 0.373 ± 0.107 & 0.543 ± 0.009 & 0.231 ± 0.057 \\
Supervised & RLLS & 0.520 ± 0.007 & 0.133 ± 0.003 & 0.337 ± 0.125 & 0.253 ± 0.082 & 0.424 ± 0.060 & 0.463 ± 0.003 & 0.454 ± 0.021 & 0.306 ± 0.074 & 0.460 ± 0.028 & 0.241 ± 0.040 \\
\midrule
MLE & IPW & 0.075 ± 0.000 & 0.071 ± 0.001 & 0.229 ± 0.001 & 0.167 ± 0.002 & 0.565 ± 0.005 & 0.443 ± 0.007 & 0.415 ± 0.000 & 0.311 ± 0.005 & 0.343 ± 0.000 & 0.280 ± 0.001 \\
MLE & OR & 0.065 ± 0.002 & 0.061 ± 0.001 & 0.200 ± 0.007 & 0.143 ± 0.001 & 0.526 ± 0.011 & 0.399 ± 0.023 & 0.360 ± 0.003 & 0.256 ± 0.012 & 0.328 ± 0.003 & 0.266 ± 0.005 \\
MLE & DR & 0.149 ± 0.019 & 0.145 ± 0.010 & 0.243 ± 0.004 & 0.214 ± 0.019 & 0.568 ± 0.005 & 0.464 ± 0.014 & 0.403 ± 0.014 & 0.309 ± 0.012 & 0.365 ± 0.007 & 0.320 ± 0.004 \\
\midrule
EM & IPW & 0.097 ± 0.008 & 0.092 ± 0.004 & 0.239 ± 0.007 & 0.179 ± 0.003 & 0.478 ± 0.012 & 0.329 ± 0.020 & 0.262 ± 0.016 & 0.202 ± 0.003 & 0.312 ± 0.002 & 0.227 ± 0.001 \\
EM & OR & 0.121 ± 0.007 & 0.108 ± 0.005 & 0.261 ± 0.007 & 0.189 ± 0.004 & 0.489 ± 0.013 & 0.335 ± 0.020 & 0.274 ± 0.016 & 0.211 ± 0.004 & 0.336 ± 0.003 & 0.235 ± 0.001 \\
EM & DR & 0.125 ± 0.005 & 0.111 ± 0.004 & 0.269 ± 0.007 & 0.194 ± 0.005 & 0.497 ± 0.010 & 0.336 ± 0.024 & 0.281 ± 0.019 & 0.219 ± 0.008 & 0.336 ± 0.007 & 0.233 ± 0.004 \\
\midrule
SimPro & IPW & 0.125 ± 0.001 & 0.100 ± 0.005 & 0.166 ± 0.007 & 0.141 ± 0.009 & 0.353 ± 0.023 & 0.261 ± 0.008 & 0.202 ± 0.003 & 0.158 ± 0.005 & 0.277 ± 0.009 & 0.197 ± 0.003 \\
SimPro & OR & 0.133 ± 0.005 & 0.100 ± 0.004 & 0.160 ± 0.007 & 0.138 ± 0.010 & 0.322 ± 0.014 & 0.253 ± 0.008 & 0.202 ± 0.003 & 0.156 ± 0.005 & 0.269 ± 0.006 & 0.191 ± 0.004 \\
SimPro & DR & 0.122 ± 0.003 & 0.106 ± 0.006 & 0.188 ± 0.009 & 0.149 ± 0.006 & 0.343 ± 0.023 & 0.257 ± 0.007 & 0.219 ± 0.010 & 0.172 ± 0.002 & 0.279 ± 0.007 & 0.198 ± 0.004 \\
\bottomrule
\end{tabular}
}
\end{table*}

% WARNING: do not forget to delete the supplementary pages from your submission 

\end{document}
