In the field of computer vision, object detection stands as a fundamental capability that allows computers to locate and classify specific objects of interest within the image or video. This widely used technology has impacted numerous fields, like autonomous vehicle navigation, medical image analysis, and intelligent surveillance systems. Object detection is an important step towards enabling machines to ``see" and understand the visual world around them, which assists them in decision-making processes.

Despite its extensive applications, object detection faces inherent challenges. This occurs when dealing with complex environments and cluttered scenes, where many different objects might be present. Moreover, factors like variations in scale, angle, and lighting can further complicate the task.  Researchers are constantly developing and improving algorithms to robustly tackle these challenges.

One specific area that faces these challenges is aircraft detection.  Being able to find and identify aircraft quickly and accurately is crucial in various sectors, including airspace security, airport traffic management, and military applications \cite{Arwin:09}. This is usually done using IFF system, which uses electromagnetic and RF devices to communicate with the aircraft. However, they are susceptible to radar jamming or equipment failures, which can compromise the ability to detect and identify aircraft. 

To address these limitations, AAR using image-based detection and classification has emerged as a promising alternative. AAR offers several advantages, including its passive nature, which makes it less susceptible to jamming, and its cost-effectiveness compared to deploying additional radar systems. While AAR offers significant benefits, it remains an evolving field requiring further research and development \cite{zhao:21}. This project aims to contribute to this ongoing effort by developing a deep learning model for accurate aircraft detection and classification (military vs. commercial) using ground-based cameras.

However, a significant challenge in developing this deep learning model lies in the availability of training data.  Large datasets are required to train these models effectively, and publicly available data for aircraft detection is limited. To overcome this challenge, a two-pronged approach involving data augmentation and image compositing will be implemented.

Data augmentation involves artificially expanding the training dataset by creating synthetic variations of existing images. This will be achieved through various techniques, such as flipping images horizontally or vertically, adjusting brightness and contrast, adding noise or blur to simulate different environmental conditions, and rotating the images at various angles. By introducing these variations, the model learns to become more robust to changes in the appearance of aircraft and can generalise better to real-world scenarios.

Image compositing involves combining elements from multiple images to create a new, synthetic image. In this project, Python scripting will be used to automate the process of compositing aircraft images with various sky backgrounds, generating synthetic variations in aircraft size, position, and lighting conditions within diverse backgrounds. 

By employing these techniques, this project aims to develop a robust deep learning model for AAR that can accurately detect and classify (military vs. commercial) aircraft using ground-based cameras, even with limited training data. This advancement has the potential to significantly contribute to airspace security, airport traffic management, and various other applications, including serving as a tool for aviation enthusiasts with the integration of the model into a mobile application.