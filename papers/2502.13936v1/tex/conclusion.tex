\section{\uppercase{conclusion}}
\label{sec:conclusion}
In this research, we proposed a comprehensive framework for improving object detection performance using various data augmentation techniques. Our approach leverages a combination of classical augmentation methods, image compositing, and advanced models like Stable Diffusion XL and ControlNet to augment the dataset. By augmenting the dataset in different ways, we were able to improve model robustness and generalization, addressing the challenges of limited annotated data in object detection tasks. 

Through rigorous experiments on a custom dataset involving both commercial and military aircraft, we demonstrated that different augmentation techniques provide varying degrees of improvement in detection accuracy, as measured by precision, recall, and mAP@0.50. Among the methods evaluated, image compositing stood out as the most effective in terms of performance, achieving the highest precision and recall scores, as well as the best mAP. 

Our results validate the hypothesis that data augmentation can significantly enhance the performance of object detection models, even in the presence of complex and imbalanced datasets. Moving forward, we plan to further refine and optimize the augmentation strategies, combining them with cutting-edge techniques such as generative adversarial networks and semi-supervised learning methods. Additionally, extending our approach to larger datasets and applying it across other domains, such as autonomous vehicles and medical imaging, presents an exciting direction for future work. Our ultimate goal is to continue advancing the state-of-the-art in object detection, improving both model accuracy and computational efficiency.