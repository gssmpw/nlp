\section{Prompts Used during Data Extraction}
\label{appendix:prompt_examples}

We provide the prompts used during the data extraction in Table~\ref{tab:prompt_example} and Table~\ref{tab:prompt_example_context_augmentation}.



\begin{table*}[ht]  
    \centering  
    \small
    \renewcommand{\arraystretch}{1.3} 
    \begin{tabular}{p{0.15\textwidth}|p{0.8\textwidth}}
    \toprule
    \textbf{Component} & \textbf{Prompt} \\ \midrule  
    Identifying Leaderboard Tables & \texttt {Determine if the given Table LaTeX represents a leaderboard table.}
    
\texttt {Leaderboard tables showcase the main results of the paper on a specific benchmark, often comparing these results with those from other studies.}

\texttt {Tables are NOT considered leaderboard tables if they focus on ablation studies, hyperparameter tuning, dataset statistics, or other supplementary experiments.}

\texttt {Respond with 'true' if the table is a leaderboard table, and 'false' otherwise, with no additional explanation.}

\textbf
{Input:}  

\texttt {Table LaTeX: }  

\textbf
{Output:}  

\texttt
{Classification Output: true/false}  \\ \hline

Schema-Driven Extraction & 

\texttt {Your task is to extract all numeric cells representing the experimental results of a specified target model from a table LaTeX source, following the provided template. When extracting results for the target model, exclude results from its variant models.} \newline

\texttt{For example:}

\texttt{- For GPT-4, exclude GPT-4o, GPT-4-v, or Deplot + GPT-4.}

\texttt{- For Claude3 Opus, exclude Claude3 Sonnet, Claude3 Haiku, Claude2, Claude 3.5.}

\texttt{- For Gemini 1.0 Pro, exclude Gemini 1.5, Gemini 1.5 Pro, Gemini Ultra, and Gemini Flash.}

\texttt{- For GPT-4o, exclude GPT-4, GPT-4-o1, GPT4-Turbo, and GPT4-V.} \newline

\texttt{However, include results from different versions of the target model across time periods (e.g., GPT-4, GPT-4-0828, GPT-4-0623, GPT-4-0314). If the table contains results where the target model is used for generation or evaluation, exclude those results. The goal is to extract results about the target model itself, not those where it is used as a tool. If no numeric cells related to the target model are found, output "<FAILED>".}

\texttt{During output, return only the extracted results in the following template. Do not provide explanations. For any unanswerable attributes, leave their value as "xx".}

\texttt{Template:
\{"value": "xx", "dataset": "xx", "dataset\_citation\_tag": "xx", "subset": "xx", "model\_name": "xx", "metric": "xx", "prompting\_method": "xx", "number\_of\_shots": "xx"\}}

\texttt{Field Descriptions:}

\texttt{- value: Extracted numeric cell value.}

\texttt{- dataset: Name of the dataset or benchmark (must be a proper noun, e.g., "synthetic dataset" is not acceptable).}

\texttt{- dataset\_citation\_tag: Citation tag for the dataset.}

\texttt{- subset: Dataset subset used (e.g., subtask, domain, split).}

\texttt{- model\_name: Name of the model used in the experiment.}

\texttt{- metric: Evaluation metric used.}

\texttt{- prompting\_method: Prompting method used (do not include shot count here).}

\texttt{- number\_of\_shots: Integer value representing the number of shots used.}

\textbf{Input:}

\texttt{Target Model: }

\texttt{Table LaTeX Source: }

\textbf{Output}
\texttt{Extracted Results:} \\ 

    \bottomrule
    \end{tabular}
    \caption{Prompts for identifying leaderboard tables and schema-driven data extraction, where the model needs to identify if the table contains the experimental results of target models.}
    \label{tab:prompt_example}  
\end{table*}  


\begin{table*}[ht]  
    \centering  
    \small
    \renewcommand{\arraystretch}{1.3} 
    \begin{tabular}{p{0.15\textwidth}|p{0.8\textwidth}}
    \toprule
    \textbf{Component} & \textbf{Prompt} \\ \midrule  

Context Augmentation & \texttt{Augment the extracted records from the table's LaTeX source by incorporating additional context from the text source to enrich and complete the records.}  

\texttt{Extracted Record Template: \{"value": "xx", "dataset": "xx", "dataset\_citation\_tag": "xx", "subset": "xx", "model\_name": "xx", "metric": "xx", "prompting\_method": "xx", "number\_of\_shots": "xx"\}}  \newline

\texttt{To accurately augment and enrich the extracted records, follow these steps systematically:}  

\texttt{1. value: Referencing the table source, if a numeric value is only partially extracted from a table cell, ensure that the entire content of the cell is used to update the value.}  

\texttt{2. dataset: Referencing the table source, table caption, and source text, locate the full name of the dataset and update the name accordingly.}  

\texttt{3. dataset\_citation\_tag: Referencing the table source, table caption, and source text, identify the dataset citation tag to the extracted record. Avoid using LaTeX syntax (e.g., cite and curly brackets); return only the tag name contained within.}  

\texttt{4. subset: Referencing the table source, table caption, and source text, identify specific subsets of the dataset, such as subtasks, domains, splits, or language pairs, and provide detailed descriptions of each subset. Prioritize the use of column information from the table source to identify the subset. If the subset is not explicitly mentioned in the table source, refer to the table caption or source text to identify the subset.}  

\texttt{5. model\_name: Referencing the table source, if a model name is partially extracted, revisit the corresponding table cell to ensure the entire content is included.}  

\texttt{6. metric: Referencing the table source, table caption, and source text, extract the metrics used in the experiment along with detailed descriptions and any additional information about the evaluation protocol.}  

\texttt{7. prompting\_method: Referencing the table source, table caption, and source text, search for and identify the prompting technique (e.g., direct, CoT, etc.) applied in the experiment and provide a detailed explanation of it. Do not include any information related to the number of shots (e.g., few-shot, zero-shot, three-shot) in this field.}  

\texttt{8. number\_of\_shots: Referencing the table source, table caption, and source text, specify the number of shots used in the experiment. This must be an integer value.}  \newline

\texttt{During output, output only the template following extracted results. Do not output any explanations or use LaTeX grammar. For any unanswerable attributes in the templates, leave their value as "xx".} 

\textbf{Input}

\texttt{Extracted Records: }  

\texttt{Table LaTeX Source: }  

\texttt{Text Source: }  

\textbf{Output}

\texttt{Augmented Extracted Records: }  \\
    \bottomrule
    \end{tabular}
    \caption{Prompt for context augmentation.}  
    \label{tab:prompt_example_context_augmentation}  
\end{table*}  



% \jp{TODO: add prompt}

% \paragraph{Prompt for Context Augmentation} We provide the prompt for context-augmentation, designed to extract~\&~augmented attribute information which is not from a given source table but from the paper content.

% \jp{TODO: add prompt}


% \paragraph{Prompt for Dataset Description Generation} We provide the prompt for dataset description generation using LLM's internal knowledge~\citep{hurst2024gpt}.

% \jp{TODO: add prompt}

% \paragraph{Prompt for Dataset Description Extraction} We provide the prompt for dataset extraction referencing the source papers~\citep{hurst2024gpt}.

% \jp{TODO: add prompt (table from Fan's latex)}
