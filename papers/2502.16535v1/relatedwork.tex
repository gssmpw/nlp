\section{Related Work}
Several reviews and surveys have explored the role of GANs in addressing data imbalance with a focus on specific application domains~\cite{52},~\cite{53},~\cite{54},~\cite{55}, improved methodologies~\cite{56},~\cite{57},~\cite{58}, and architecture adaptations~\cite{59},~\cite{60},~\cite{61},~\cite{62}. The research methodology identified four major reviews and surveys~\cite{63},~\cite{64},~\cite{65},~\cite{66}. However, these studies have some notable limitations and research gaps that give scope for further exploration.\\
Sampath et al.~\cite{63} provided a detailed taxonomy of GAN-based solutions for addressing imbalance in computer vision tasks. It categorizes issues at the image, object, and pixel levels. While the study offered insights into classification, detection, and segmentation challenges, it did not extend its analysis to other crucial domains, such as structured data or non-visual tasks. 
Cole and Khoshgoftaar~\cite{64} focused on the using GANs in tabular datasets, primarily focusing on network traffic classification and financial transactions. Their survey covers the prevalence of CGAN architectures and CNN-based learners having improvements in accuracy. However, their scope was confined to structured data and did not address GAN applications in broader or hybrid frameworks. 
Also, in another study, Cole et al.~\cite{65} adopted a practical analysis of 18 GitHub repositories implementing GAN for imbalanced datasets. Their methodology focused on code structure, library reliance, and best practices for researchers. However, the study lacked a theoretical framework or categorization of GAN techniques across application areas.   
Nayak et al.~\cite{66} examined GAN methodologies in computer vision with a particular focus on image enhancement tasks. Their systematic literature review compared GANs with traditional machine learning and MATLAB-based approaches, proving GANs superior performance in tasks like image denoising and enhancement. While the studyfocused on image quality, it offered limited insights into the broader challenges of dataset imbalance in various domains.
The existing literature reveals critical gaps in the study of GANs in addressing data imbalance. Most reviews have a narrow focus on domains while they lack attention in high-impact areas like healthcare~\cite{67},~\cite{68},~\cite{69}, finance~\cite{70},~\cite{71},~\cite{72},~\cite{73} and cybersecurity~\cite{74},~\cite{75},~\cite{76},~\cite{77} unexplored. Also, the literature fails to provide systematic mapping of GAN techniques, variants, and their application strengths, which are essential for a deeper understanding across domains.

Our methodology addresses these gaps by systematically analyzing studies. This study spans a broader range of application areas like healthcare, finance, cybersecurity, and others. It introduces unique categorization mappings that systematically organize GAN techniques, variants, and their domain-specific applications. It offers a more comprehensive perspective than previous reviews. The research highlights a scope for future research approaches like diffusion models and reinforcement learning by identifying the existing absence of hybridized GAN frameworks. Through these contributions, the study strengthens the existing knowledge and provides a foundation for addressing data imbalance.