
\documentclass{article} % For LaTeX2e
\usepackage{iclr2025,times}

% Optional math commands from https://github.com/goodfeli/dlbook_notation.
%%%%% NEW MATH DEFINITIONS %%%%%

\usepackage{amsmath,amsfonts,bm}
\usepackage{derivative}
% Mark sections of captions for referring to divisions of figures
\newcommand{\figleft}{{\em (Left)}}
\newcommand{\figcenter}{{\em (Center)}}
\newcommand{\figright}{{\em (Right)}}
\newcommand{\figtop}{{\em (Top)}}
\newcommand{\figbottom}{{\em (Bottom)}}
\newcommand{\captiona}{{\em (a)}}
\newcommand{\captionb}{{\em (b)}}
\newcommand{\captionc}{{\em (c)}}
\newcommand{\captiond}{{\em (d)}}

% Highlight a newly defined term
\newcommand{\newterm}[1]{{\bf #1}}

% Derivative d 
\newcommand{\deriv}{{\mathrm{d}}}

% Figure reference, lower-case.
\def\figref#1{figure~\ref{#1}}
% Figure reference, capital. For start of sentence
\def\Figref#1{Figure~\ref{#1}}
\def\twofigref#1#2{figures \ref{#1} and \ref{#2}}
\def\quadfigref#1#2#3#4{figures \ref{#1}, \ref{#2}, \ref{#3} and \ref{#4}}
% Section reference, lower-case.
\def\secref#1{section~\ref{#1}}
% Section reference, capital.
\def\Secref#1{Section~\ref{#1}}
% Reference to two sections.
\def\twosecrefs#1#2{sections \ref{#1} and \ref{#2}}
% Reference to three sections.
\def\secrefs#1#2#3{sections \ref{#1}, \ref{#2} and \ref{#3}}
% Reference to an equation, lower-case.
\def\eqref#1{equation~\ref{#1}}
% Reference to an equation, upper case
\def\Eqref#1{Equation~\ref{#1}}
% A raw reference to an equation---avoid using if possible
\def\plaineqref#1{\ref{#1}}
% Reference to a chapter, lower-case.
\def\chapref#1{chapter~\ref{#1}}
% Reference to an equation, upper case.
\def\Chapref#1{Chapter~\ref{#1}}
% Reference to a range of chapters
\def\rangechapref#1#2{chapters\ref{#1}--\ref{#2}}
% Reference to an algorithm, lower-case.
\def\algref#1{algorithm~\ref{#1}}
% Reference to an algorithm, upper case.
\def\Algref#1{Algorithm~\ref{#1}}
\def\twoalgref#1#2{algorithms \ref{#1} and \ref{#2}}
\def\Twoalgref#1#2{Algorithms \ref{#1} and \ref{#2}}
% Reference to a part, lower case
\def\partref#1{part~\ref{#1}}
% Reference to a part, upper case
\def\Partref#1{Part~\ref{#1}}
\def\twopartref#1#2{parts \ref{#1} and \ref{#2}}

\def\ceil#1{\lceil #1 \rceil}
\def\floor#1{\lfloor #1 \rfloor}
\def\1{\bm{1}}
\newcommand{\train}{\mathcal{D}}
\newcommand{\valid}{\mathcal{D_{\mathrm{valid}}}}
\newcommand{\test}{\mathcal{D_{\mathrm{test}}}}

\def\eps{{\epsilon}}


% Random variables
\def\reta{{\textnormal{$\eta$}}}
\def\ra{{\textnormal{a}}}
\def\rb{{\textnormal{b}}}
\def\rc{{\textnormal{c}}}
\def\rd{{\textnormal{d}}}
\def\re{{\textnormal{e}}}
\def\rf{{\textnormal{f}}}
\def\rg{{\textnormal{g}}}
\def\rh{{\textnormal{h}}}
\def\ri{{\textnormal{i}}}
\def\rj{{\textnormal{j}}}
\def\rk{{\textnormal{k}}}
\def\rl{{\textnormal{l}}}
% rm is already a command, just don't name any random variables m
\def\rn{{\textnormal{n}}}
\def\ro{{\textnormal{o}}}
\def\rp{{\textnormal{p}}}
\def\rq{{\textnormal{q}}}
\def\rr{{\textnormal{r}}}
\def\rs{{\textnormal{s}}}
\def\rt{{\textnormal{t}}}
\def\ru{{\textnormal{u}}}
\def\rv{{\textnormal{v}}}
\def\rw{{\textnormal{w}}}
\def\rx{{\textnormal{x}}}
\def\ry{{\textnormal{y}}}
\def\rz{{\textnormal{z}}}

% Random vectors
\def\rvepsilon{{\mathbf{\epsilon}}}
\def\rvphi{{\mathbf{\phi}}}
\def\rvtheta{{\mathbf{\theta}}}
\def\rva{{\mathbf{a}}}
\def\rvb{{\mathbf{b}}}
\def\rvc{{\mathbf{c}}}
\def\rvd{{\mathbf{d}}}
\def\rve{{\mathbf{e}}}
\def\rvf{{\mathbf{f}}}
\def\rvg{{\mathbf{g}}}
\def\rvh{{\mathbf{h}}}
\def\rvu{{\mathbf{i}}}
\def\rvj{{\mathbf{j}}}
\def\rvk{{\mathbf{k}}}
\def\rvl{{\mathbf{l}}}
\def\rvm{{\mathbf{m}}}
\def\rvn{{\mathbf{n}}}
\def\rvo{{\mathbf{o}}}
\def\rvp{{\mathbf{p}}}
\def\rvq{{\mathbf{q}}}
\def\rvr{{\mathbf{r}}}
\def\rvs{{\mathbf{s}}}
\def\rvt{{\mathbf{t}}}
\def\rvu{{\mathbf{u}}}
\def\rvv{{\mathbf{v}}}
\def\rvw{{\mathbf{w}}}
\def\rvx{{\mathbf{x}}}
\def\rvy{{\mathbf{y}}}
\def\rvz{{\mathbf{z}}}

% Elements of random vectors
\def\erva{{\textnormal{a}}}
\def\ervb{{\textnormal{b}}}
\def\ervc{{\textnormal{c}}}
\def\ervd{{\textnormal{d}}}
\def\erve{{\textnormal{e}}}
\def\ervf{{\textnormal{f}}}
\def\ervg{{\textnormal{g}}}
\def\ervh{{\textnormal{h}}}
\def\ervi{{\textnormal{i}}}
\def\ervj{{\textnormal{j}}}
\def\ervk{{\textnormal{k}}}
\def\ervl{{\textnormal{l}}}
\def\ervm{{\textnormal{m}}}
\def\ervn{{\textnormal{n}}}
\def\ervo{{\textnormal{o}}}
\def\ervp{{\textnormal{p}}}
\def\ervq{{\textnormal{q}}}
\def\ervr{{\textnormal{r}}}
\def\ervs{{\textnormal{s}}}
\def\ervt{{\textnormal{t}}}
\def\ervu{{\textnormal{u}}}
\def\ervv{{\textnormal{v}}}
\def\ervw{{\textnormal{w}}}
\def\ervx{{\textnormal{x}}}
\def\ervy{{\textnormal{y}}}
\def\ervz{{\textnormal{z}}}

% Random matrices
\def\rmA{{\mathbf{A}}}
\def\rmB{{\mathbf{B}}}
\def\rmC{{\mathbf{C}}}
\def\rmD{{\mathbf{D}}}
\def\rmE{{\mathbf{E}}}
\def\rmF{{\mathbf{F}}}
\def\rmG{{\mathbf{G}}}
\def\rmH{{\mathbf{H}}}
\def\rmI{{\mathbf{I}}}
\def\rmJ{{\mathbf{J}}}
\def\rmK{{\mathbf{K}}}
\def\rmL{{\mathbf{L}}}
\def\rmM{{\mathbf{M}}}
\def\rmN{{\mathbf{N}}}
\def\rmO{{\mathbf{O}}}
\def\rmP{{\mathbf{P}}}
\def\rmQ{{\mathbf{Q}}}
\def\rmR{{\mathbf{R}}}
\def\rmS{{\mathbf{S}}}
\def\rmT{{\mathbf{T}}}
\def\rmU{{\mathbf{U}}}
\def\rmV{{\mathbf{V}}}
\def\rmW{{\mathbf{W}}}
\def\rmX{{\mathbf{X}}}
\def\rmY{{\mathbf{Y}}}
\def\rmZ{{\mathbf{Z}}}

% Elements of random matrices
\def\ermA{{\textnormal{A}}}
\def\ermB{{\textnormal{B}}}
\def\ermC{{\textnormal{C}}}
\def\ermD{{\textnormal{D}}}
\def\ermE{{\textnormal{E}}}
\def\ermF{{\textnormal{F}}}
\def\ermG{{\textnormal{G}}}
\def\ermH{{\textnormal{H}}}
\def\ermI{{\textnormal{I}}}
\def\ermJ{{\textnormal{J}}}
\def\ermK{{\textnormal{K}}}
\def\ermL{{\textnormal{L}}}
\def\ermM{{\textnormal{M}}}
\def\ermN{{\textnormal{N}}}
\def\ermO{{\textnormal{O}}}
\def\ermP{{\textnormal{P}}}
\def\ermQ{{\textnormal{Q}}}
\def\ermR{{\textnormal{R}}}
\def\ermS{{\textnormal{S}}}
\def\ermT{{\textnormal{T}}}
\def\ermU{{\textnormal{U}}}
\def\ermV{{\textnormal{V}}}
\def\ermW{{\textnormal{W}}}
\def\ermX{{\textnormal{X}}}
\def\ermY{{\textnormal{Y}}}
\def\ermZ{{\textnormal{Z}}}

% Vectors
\def\vzero{{\bm{0}}}
\def\vone{{\bm{1}}}
\def\vmu{{\bm{\mu}}}
\def\vtheta{{\bm{\theta}}}
\def\vphi{{\bm{\phi}}}
\def\va{{\bm{a}}}
\def\vb{{\bm{b}}}
\def\vc{{\bm{c}}}
\def\vd{{\bm{d}}}
\def\ve{{\bm{e}}}
\def\vf{{\bm{f}}}
\def\vg{{\bm{g}}}
\def\vh{{\bm{h}}}
\def\vi{{\bm{i}}}
\def\vj{{\bm{j}}}
\def\vk{{\bm{k}}}
\def\vl{{\bm{l}}}
\def\vm{{\bm{m}}}
\def\vn{{\bm{n}}}
\def\vo{{\bm{o}}}
\def\vp{{\bm{p}}}
\def\vq{{\bm{q}}}
\def\vr{{\bm{r}}}
\def\vs{{\bm{s}}}
\def\vt{{\bm{t}}}
\def\vu{{\bm{u}}}
\def\vv{{\bm{v}}}
\def\vw{{\bm{w}}}
\def\vx{{\bm{x}}}
\def\vy{{\bm{y}}}
\def\vz{{\bm{z}}}

% Elements of vectors
\def\evalpha{{\alpha}}
\def\evbeta{{\beta}}
\def\evepsilon{{\epsilon}}
\def\evlambda{{\lambda}}
\def\evomega{{\omega}}
\def\evmu{{\mu}}
\def\evpsi{{\psi}}
\def\evsigma{{\sigma}}
\def\evtheta{{\theta}}
\def\eva{{a}}
\def\evb{{b}}
\def\evc{{c}}
\def\evd{{d}}
\def\eve{{e}}
\def\evf{{f}}
\def\evg{{g}}
\def\evh{{h}}
\def\evi{{i}}
\def\evj{{j}}
\def\evk{{k}}
\def\evl{{l}}
\def\evm{{m}}
\def\evn{{n}}
\def\evo{{o}}
\def\evp{{p}}
\def\evq{{q}}
\def\evr{{r}}
\def\evs{{s}}
\def\evt{{t}}
\def\evu{{u}}
\def\evv{{v}}
\def\evw{{w}}
\def\evx{{x}}
\def\evy{{y}}
\def\evz{{z}}

% Matrix
\def\mA{{\bm{A}}}
\def\mB{{\bm{B}}}
\def\mC{{\bm{C}}}
\def\mD{{\bm{D}}}
\def\mE{{\bm{E}}}
\def\mF{{\bm{F}}}
\def\mG{{\bm{G}}}
\def\mH{{\bm{H}}}
\def\mI{{\bm{I}}}
\def\mJ{{\bm{J}}}
\def\mK{{\bm{K}}}
\def\mL{{\bm{L}}}
\def\mM{{\bm{M}}}
\def\mN{{\bm{N}}}
\def\mO{{\bm{O}}}
\def\mP{{\bm{P}}}
\def\mQ{{\bm{Q}}}
\def\mR{{\bm{R}}}
\def\mS{{\bm{S}}}
\def\mT{{\bm{T}}}
\def\mU{{\bm{U}}}
\def\mV{{\bm{V}}}
\def\mW{{\bm{W}}}
\def\mX{{\bm{X}}}
\def\mY{{\bm{Y}}}
\def\mZ{{\bm{Z}}}
\def\mBeta{{\bm{\beta}}}
\def\mPhi{{\bm{\Phi}}}
\def\mLambda{{\bm{\Lambda}}}
\def\mSigma{{\bm{\Sigma}}}

% Tensor
\DeclareMathAlphabet{\mathsfit}{\encodingdefault}{\sfdefault}{m}{sl}
\SetMathAlphabet{\mathsfit}{bold}{\encodingdefault}{\sfdefault}{bx}{n}
\newcommand{\tens}[1]{\bm{\mathsfit{#1}}}
\def\tA{{\tens{A}}}
\def\tB{{\tens{B}}}
\def\tC{{\tens{C}}}
\def\tD{{\tens{D}}}
\def\tE{{\tens{E}}}
\def\tF{{\tens{F}}}
\def\tG{{\tens{G}}}
\def\tH{{\tens{H}}}
\def\tI{{\tens{I}}}
\def\tJ{{\tens{J}}}
\def\tK{{\tens{K}}}
\def\tL{{\tens{L}}}
\def\tM{{\tens{M}}}
\def\tN{{\tens{N}}}
\def\tO{{\tens{O}}}
\def\tP{{\tens{P}}}
\def\tQ{{\tens{Q}}}
\def\tR{{\tens{R}}}
\def\tS{{\tens{S}}}
\def\tT{{\tens{T}}}
\def\tU{{\tens{U}}}
\def\tV{{\tens{V}}}
\def\tW{{\tens{W}}}
\def\tX{{\tens{X}}}
\def\tY{{\tens{Y}}}
\def\tZ{{\tens{Z}}}


% Graph
\def\gA{{\mathcal{A}}}
\def\gB{{\mathcal{B}}}
\def\gC{{\mathcal{C}}}
\def\gD{{\mathcal{D}}}
\def\gE{{\mathcal{E}}}
\def\gF{{\mathcal{F}}}
\def\gG{{\mathcal{G}}}
\def\gH{{\mathcal{H}}}
\def\gI{{\mathcal{I}}}
\def\gJ{{\mathcal{J}}}
\def\gK{{\mathcal{K}}}
\def\gL{{\mathcal{L}}}
\def\gM{{\mathcal{M}}}
\def\gN{{\mathcal{N}}}
\def\gO{{\mathcal{O}}}
\def\gP{{\mathcal{P}}}
\def\gQ{{\mathcal{Q}}}
\def\gR{{\mathcal{R}}}
\def\gS{{\mathcal{S}}}
\def\gT{{\mathcal{T}}}
\def\gU{{\mathcal{U}}}
\def\gV{{\mathcal{V}}}
\def\gW{{\mathcal{W}}}
\def\gX{{\mathcal{X}}}
\def\gY{{\mathcal{Y}}}
\def\gZ{{\mathcal{Z}}}

% Sets
\def\sA{{\mathbb{A}}}
\def\sB{{\mathbb{B}}}
\def\sC{{\mathbb{C}}}
\def\sD{{\mathbb{D}}}
% Don't use a set called E, because this would be the same as our symbol
% for expectation.
\def\sF{{\mathbb{F}}}
\def\sG{{\mathbb{G}}}
\def\sH{{\mathbb{H}}}
\def\sI{{\mathbb{I}}}
\def\sJ{{\mathbb{J}}}
\def\sK{{\mathbb{K}}}
\def\sL{{\mathbb{L}}}
\def\sM{{\mathbb{M}}}
\def\sN{{\mathbb{N}}}
\def\sO{{\mathbb{O}}}
\def\sP{{\mathbb{P}}}
\def\sQ{{\mathbb{Q}}}
\def\sR{{\mathbb{R}}}
\def\sS{{\mathbb{S}}}
\def\sT{{\mathbb{T}}}
\def\sU{{\mathbb{U}}}
\def\sV{{\mathbb{V}}}
\def\sW{{\mathbb{W}}}
\def\sX{{\mathbb{X}}}
\def\sY{{\mathbb{Y}}}
\def\sZ{{\mathbb{Z}}}

% Entries of a matrix
\def\emLambda{{\Lambda}}
\def\emA{{A}}
\def\emB{{B}}
\def\emC{{C}}
\def\emD{{D}}
\def\emE{{E}}
\def\emF{{F}}
\def\emG{{G}}
\def\emH{{H}}
\def\emI{{I}}
\def\emJ{{J}}
\def\emK{{K}}
\def\emL{{L}}
\def\emM{{M}}
\def\emN{{N}}
\def\emO{{O}}
\def\emP{{P}}
\def\emQ{{Q}}
\def\emR{{R}}
\def\emS{{S}}
\def\emT{{T}}
\def\emU{{U}}
\def\emV{{V}}
\def\emW{{W}}
\def\emX{{X}}
\def\emY{{Y}}
\def\emZ{{Z}}
\def\emSigma{{\Sigma}}

% entries of a tensor
% Same font as tensor, without \bm wrapper
\newcommand{\etens}[1]{\mathsfit{#1}}
\def\etLambda{{\etens{\Lambda}}}
\def\etA{{\etens{A}}}
\def\etB{{\etens{B}}}
\def\etC{{\etens{C}}}
\def\etD{{\etens{D}}}
\def\etE{{\etens{E}}}
\def\etF{{\etens{F}}}
\def\etG{{\etens{G}}}
\def\etH{{\etens{H}}}
\def\etI{{\etens{I}}}
\def\etJ{{\etens{J}}}
\def\etK{{\etens{K}}}
\def\etL{{\etens{L}}}
\def\etM{{\etens{M}}}
\def\etN{{\etens{N}}}
\def\etO{{\etens{O}}}
\def\etP{{\etens{P}}}
\def\etQ{{\etens{Q}}}
\def\etR{{\etens{R}}}
\def\etS{{\etens{S}}}
\def\etT{{\etens{T}}}
\def\etU{{\etens{U}}}
\def\etV{{\etens{V}}}
\def\etW{{\etens{W}}}
\def\etX{{\etens{X}}}
\def\etY{{\etens{Y}}}
\def\etZ{{\etens{Z}}}

% The true underlying data generating distribution
\newcommand{\pdata}{p_{\rm{data}}}
\newcommand{\ptarget}{p_{\rm{target}}}
\newcommand{\pprior}{p_{\rm{prior}}}
\newcommand{\pbase}{p_{\rm{base}}}
\newcommand{\pref}{p_{\rm{ref}}}

% The empirical distribution defined by the training set
\newcommand{\ptrain}{\hat{p}_{\rm{data}}}
\newcommand{\Ptrain}{\hat{P}_{\rm{data}}}
% The model distribution
\newcommand{\pmodel}{p_{\rm{model}}}
\newcommand{\Pmodel}{P_{\rm{model}}}
\newcommand{\ptildemodel}{\tilde{p}_{\rm{model}}}
% Stochastic autoencoder distributions
\newcommand{\pencode}{p_{\rm{encoder}}}
\newcommand{\pdecode}{p_{\rm{decoder}}}
\newcommand{\precons}{p_{\rm{reconstruct}}}

\newcommand{\laplace}{\mathrm{Laplace}} % Laplace distribution

\newcommand{\E}{\mathbb{E}}
\newcommand{\Ls}{\mathcal{L}}
\newcommand{\R}{\mathbb{R}}
\newcommand{\emp}{\tilde{p}}
\newcommand{\lr}{\alpha}
\newcommand{\reg}{\lambda}
\newcommand{\rect}{\mathrm{rectifier}}
\newcommand{\softmax}{\mathrm{softmax}}
\newcommand{\sigmoid}{\sigma}
\newcommand{\softplus}{\zeta}
\newcommand{\KL}{D_{\mathrm{KL}}}
\newcommand{\Var}{\mathrm{Var}}
\newcommand{\standarderror}{\mathrm{SE}}
\newcommand{\Cov}{\mathrm{Cov}}
% Wolfram Mathworld says $L^2$ is for function spaces and $\ell^2$ is for vectors
% But then they seem to use $L^2$ for vectors throughout the site, and so does
% wikipedia.
\newcommand{\normlzero}{L^0}
\newcommand{\normlone}{L^1}
\newcommand{\normltwo}{L^2}
\newcommand{\normlp}{L^p}
\newcommand{\normmax}{L^\infty}

\newcommand{\parents}{Pa} % See usage in notation.tex. Chosen to match Daphne's book.

\DeclareMathOperator*{\argmax}{arg\,max}
\DeclareMathOperator*{\argmin}{arg\,min}

\DeclareMathOperator{\sign}{sign}
\DeclareMathOperator{\Tr}{Tr}
\let\ab\allowbreak

\usepackage{booktabs}
\usepackage{hyperref}
\usepackage{url}


\title{Pruning as a Defense: Reducing Memorization in Large Language Models}

% Authors must not appear in the submitted version. They should be hidden
% as long as the \iclrfinalcopy macro remains commented out below.
% Non-anonymous submissions will be rejected without review.

\author{Mansi Gupta, Nikhar Waghela, Sarthak Gupta \& Shourya Goel \\
Indian Institute of Technology, Roorkee\\
Roorkee, Uttarakhand, India\\
\texttt{\{m\_gupta, n\_waghela, sarthak\_g\}@ma.iitr.ac.in, shourya\_g@cs.iitr.ac.in} \\
\AND
 Sanjif Shanmugavelu
\\
Groq Inc \\
London W6 0ND, UK\\
\texttt{sshanmugavelu@groq.com}
}

% The \author macro works with any number of authors. There are two commands
% used to separate the names and addresses of multiple authors: \And and \AND.
%
% Using \And between authors leaves it to \LaTeX{} to determine where to break
% the lines. Using \AND forces a linebreak at that point. So, if \LaTeX{}
% puts 3 of 4 authors names on the first line, and the last on the second
% line, try using \AND instead of \And before the third author name.

\newcommand{\fix}{\marginpar{FIX}}
\newcommand{\new}{\marginpar{NEW}}

\iclrfinalcopy 
\begin{document}


\maketitle

\begin{abstract}
Large language models have been shown to memorize significant portions of their training data, which they can reproduce when appropriately prompted. This work investigates the impact of simple pruning techniques on this behavior. Our findings reveal that pruning effectively reduces the extent of memorization in LLMs, demonstrating its potential as a foundational approach for mitigating membership inference attacks. 
\end{abstract}

\section{Introduction}
Large language models are known to memorize portions of their training data, which poses significant privacy and security risks. Although various studies have explored the extent of memorization in LLMs, most of these efforts are qualitative (\cite{carlini2021extractingtrainingdatalarge}). A recent approach introduces a method to quantify memorization by examining whether models can recall and complete prompts from their training data verbatim (\cite{carlini2023quantifyingmemorizationneurallanguage}). By feeding prefixes of these prompts into a trained model, we can assess its ability to reconstruct the full prompt, thus enabling quantitative analysis of memorization behavior across different models, datasets, and prompt sizes. This framework for quantifying memorization is the basis of our study. Pruning techniques, widely used in machine learning to reduce computational overhead and mitigate overfitting, have been shown to maintain task performance despite reducing the size and complexity of models (\cite{huang2024largelanguagemodelpruning},\cite{sun2024simpleeffectivepruningapproach}). Pruning techniques simplify model architectures while preserving task performance (\cite{huang2024largelanguagemodelpruning}, \cite{sun2024simpleeffectivepruningapproach}). Beyond efficiency, previous studies suggest that pruning can reduce overfitting and induce unlearning by removing memorized information (\cite{pochinkov2024dissecting}. Motivated by this, we explore pruning to mitigate memorization in large language models.

We systematically evaluate the impact of layer-wise and global pruning strategies on memorization. Building on findings that deeper layers can often be pruned more aggressively without harming performance (\cite{panigrahi2023taskspecificskilllocalizationfinetuned}, \cite{gromov2024unreasonableineffectivenessdeeperlayers}), we analyze the effect of pruning specific blocks, including attention layers, to identify those most responsible for memorization. Additionally, we prune at different percentages to assess whether the extent of pruning impacts model performance. Our results show that pruning reduces both computational costs and memorization, providing a lightweight and effective approach to improving privacy in LLMs.


\section{Methodology}
Drawing reference from this paper (\cite{carlini2023quantifyingmemorizationneurallanguage}), we define a string $s$ as extractable with context $k$ tokens if there exists a $k$ token prefix $p$ such that the concatenation $[p \,||\, s]$ exists in the training data, and the model $f$ reproduces $s$ when prompted by $p$ using greedy decoding. For example, if a model's training data contains "The capital of Germany is Berlin," and the prefix "The capital of Germany is" results in the output "Berlin," the sequence is deemed extractable with $k = 5$. Our experiments used random subsets of 5000 sequences to estimate intractability at four different context lengths (k = 50,100,200,500) while maintaining statistical confidence efficiently. This is the definition of memorization that we use for all our experiments.



We perform pruning experiments in the Pythia family of models (160M to 12B parameters) (\cite{Bengio+chapter2007}), trained on the Pile dataset (\cite{gao2020pile800gbdatasetdiverse}). We evaluate memorization across different pruning strategies using sequences sampled from this dataset. We experimented with the following different variants of pruning:

\begin{itemize}  
    \item \textbf{Layer-wise Magnitude Pruning:}  
    We individually prune all linear layers within the model, removing the lowest \( n\% \) of weights in each layer based on their L1 norm. This localized pruning ensures uniform sparsity across layers while preserving overall model structure.  

    \item \textbf{Global Magnitude Pruning:}  
    We prune the lowest \( n\% \) of weights across the entire model based on their L1 norm, allowing for more flexible weight removal across layers.  
    \begin{itemize}  
        \item \textit{All Linear Layers:} Both MLP and attention layers undergo pruning to reduce redundancy in learned representations.  
        \item \textit{Attention-Only Layers:} Pruning is applied exclusively to attention layers, keeping MLP layers intact to assess the impact of attention-driven memorization. This allows us to examine the impact of reducing attention capacity while preserving the feedforward network structure.
    \end{itemize}  

    \item \textbf{Selective Layer Pruning:}  
    To analyze the role of deeper layers in memorization, we globally prune either the first 25\% of layers or the last 25\% of layers. This helps determine whether early or later layers contribute more to memorization and overall model performance.  
\end{itemize}  


For each pruned variant, we evaluate the degree of memorization across different context lengths and draw reference from some previous work(\cite{10.1162/tacl_a_00695},\cite{huang2024largelanguagemodelpruning},\cite{unknown}) to ensure less performance degradation in pruning LLMs. We experiment with pruning at different levels for varying model sizes, as smaller models are more sensitive to pruning. For each model, we fix two pruning levels, denoted Level 1 and Level 2, which are specific to their size, as shown in the below table.
\begin{table}[h]
    \centering
     \caption{Pruning levels for different Pythia model sizes.}
     \vspace{7pt}
    \begin{tabular}{|c|c|c|}
        \hline
        \textbf{Model} & \textbf{Level 1} & \textbf{Level 2} \\
        \hline
        Pythia-160M & 10\% & 15\% \\
        Pythia-420M & 15\% & 20\% \\
        Pythia-2.8B & 20\% & 30\% \\
        Pythia-6.9B & 30\% & 40\% \\
        Pythia-12B  & 35\% & 45\% \\
        \hline
    \end{tabular}
   
    \label{tab:pruning_levels}
\end{table}

We also assess the perplexity of the pruned models to ensure that, despite reduced memorization, they still generate coherent and contextually relevant completions. 

\section{Results}
\begin{table}[htbp]
\vspace{-22pt} % Adjust vertical spacing as needed
\caption{Model-wise Average Fraction of Memorized Data}
\vspace{0.2cm}
\setlength{\tabcolsep}{3pt} % Reduce column spacing
\centering
\begin{tabular}{lcccccc}
\toprule
\textbf{Models} & \textbf{Baseline} & \textbf{layer-wise} & \textbf{Global} & \textbf{Attention} & \textbf{First 25\% }& \textbf{Last 25\% } \\
\midrule
\textbf{Pythia-160m} & 0.0065 & 0.0016 & 0.0012 & 0.0008 & 0.0012 & 0.0010 \\
\textbf{Pythia-410m} & 0.01 & 0.0038 & 0.0035 & 0.0021 & 0.0036 & 0.0030 \\
\textbf{Pythia-2.8b} & 0.015 & 0.0069 & 0.0048 & 0.003 & 0.0069 & 0.0050 \\
\textbf{Pythia-6.9b} & 0.018 & 0.012 & 0.009 & 0.006 & 0.008 & 0.007 \\
\textbf{Pythia-12b} & 0.022 & 0.016 & 0.014 & 0.008 & 0.012 & 0.010 \\
\bottomrule
\end{tabular}
\label{Prompt_table2}
\end{table}


Table \ref{Prompt_table2} reports the percentage of memorized data, calculated as the number of samples correctly reproduced by the model, averaged across 5,000 samples and various context lengths. The results are further averaged over the two pruning levels, with detailed results and perplexity values for each pruned variant presented in Appendix \ref{Results}. The results clearly show that pruning effectively reduces memorization, with higher pruning percentages leading to a more substantial decrease. Notably, global pruning, where all layers are pruned uniformly, reduces memorization more than layer-specific pruning strategies. Among all experiments, pruning only the attention layers achieved the most significant reduction, indicating that these layers store a considerable amount of information, but they also hurt the performance of the model.
Pruning deeper layers also contributes to a notable reduction in memorization, while maintaining performance levels, as substantiated by previous work in this direction (\cite{gromov2024unreasonableineffectivenessdeeperlayers}).

Table \ref{Perplexity_avg} reports the average perplexity values for all models across different context lengths and pruning levels. We evaluate the language model's perplexity across different pruning levels to ensure that pruning does not excessively degrade model performance. Perplexity measures the model’s predictive uncertainty, with lower values indicating better language modeling capabilities. By quantifying perplexity, we can establish a threshold that balances the trade-off between reducing memorization and maintaining task performance. This analysis provides insights into how much pruning is acceptable before performance loss outweighs the benefits of reduced memorization, allowing for more informed decisions on pruning strategies and levels. 


\begin{table}[htbp]
\vspace{-14pt} % Adjust vertical spacing as needed
\caption{Average perplexity values across both pruning levels}
\vspace{0.2cm}
\setlength{\tabcolsep}{3pt} % Reduce column spacing
\centering
\begin{tabular}{lcccccc}
\toprule
\textbf{Models} & \textbf{Baseline} & \textbf{Layer-wise} & \textbf{Global} & \textbf{Attention} & \textbf{First 25\%} & \textbf{Last 25\%} \\
\midrule
\textbf{Pythia-160m} & 24.41 & 28.45 & 29.16 & 34.84 & 30.21 & 28.99 \\
\textbf{Pythia-410m} & 14.78 & 19.63 & 20.17 & 23.39 & 20.41 & 19.93 \\
\textbf{Pythia-2.8b} & 9.32 & 9.92 & 10.04 & 12.93 & 11.01 & 9.99 \\
\textbf{Pythia-6.9b} & 8.23 & 8.95 & 9.02 & 10.96 & 9.35 & 9.00 \\
\textbf{Pythia-12b} & 7.42 & 8.06 & 8.20 & 10.79 & 8.79 & 8.33 \\
\bottomrule
\end{tabular}
\label{Perplexity_avg}
\end{table}

As shown in Table \ref{Perplexity_avg}, pruning attention layers results in a higher perplexity, indicating a notable drop in performance. In contrast, pruning deeper layers and applying layer-wise pruning result in a smaller increase in perplexity, suggesting a more balanced trade-off between reducing memorization and preserving model performance.



\section{Limitations and Future Work}
Our study was limited to 5,000 training samples due to computational constraints, leaving scope for testing on larger datasets to gain deeper insights. Additionally, future work could explore other pruning methods, such as in-training pruning (\cite{roy2020pruningfilterstrainingefficiently}), to enhance performance stability while reducing memorization. Investigating alternative sparsity techniques, such as quantization and low-rank approximations, could provide further improvements. Expanding the analysis to include different pruning strategies and their effects on various components of model architectures presents a valuable direction for understanding and mitigating memorization in large language models. Also, we have primarily used perplexity as a metric to assess model performance post-pruning. However, future work could incorporate more comprehensive evaluation metrics, such as ROUGE or BLEU scores, to better capture the quality of the generated text. 

\section{Conclusion}
Our experiments demonstrate that pruning is an effective baseline for mitigating membership inference attacks, as it reduces memorization risk across all context lengths. By introducing sparsity, pruning prevents the exact reproducibility of training data, reduces computational overhead, and maintains model performance, making it a practical solution for addressing memorization in large language models. Pruning attention layers results in the most significant reduction in memorization, although it comes at the cost of a performance drop. This highlights the role of attention layers in memorization and suggests avenues for future research into targeted mitigation techniques. Additionally, pruning deeper layers results in a substantial reduction in memorization while maintaining performance, making it an efficient baseline strategy. Exploring more adaptive pruning techniques could further illuminate how varying sparsity levels impact memorization.





\bibliography{iclr2025}
\bibliographystyle{iclr2025}

\appendix
\section{Appendix}


\section*{Perplexity Results}
\label{perplexity-app}

\begin{table}[htbp]
\vspace{-14pt} % Adjust vertical spacing as needed
\caption{Perplexity values for a lower level of pruning}
\vspace{0.2cm}
\setlength{\tabcolsep}{3pt} % Reduce column spacing
\centering
\begin{tabular}{lcccccc}
\toprule
\textbf{Models} & \textbf{Baseline} & \textbf{layer-wise} & \textbf{Global} & \textbf{Attention} & \textbf{First 25\% }& \textbf{Last 25\% } \\
\midrule
\textbf{Pythia-160m} & 24.41 & 27.45 & 28.18 & 33.45 & 29.19 & 27.90 \\
\textbf{Pythia-410m} & 14.78 & 18.43 & 19.65 & 22.14 & 19.90 & 18.78 \\
\textbf{Pythia-2.8b} & 9.32 & 9.80 & 9.94 & 12.13 & 10.93 & 9.98 \\
\textbf{Pythia-6.9b} & 8.23 & 8.91 & 8.98 & 10.92 & 9.13 & 8.97 \\
\textbf{Pythia-12b} & 7.42 & 7.98 & 8.10 & 10.44 & 8.34 & 8.16 \\
\bottomrule
\end{tabular}
\label{Perplexity_1}
\end{table}

\begin{table}[htbp]
\vspace{-14pt} % Adjust vertical spacing as needed
\caption{Perplexity values for a higher level of pruning}
\vspace{0.2cm}
\setlength{\tabcolsep}{3pt} % Reduce column spacing
\centering
\begin{tabular}{lcccccc}
\toprule
\textbf{Models} & \textbf{Baseline} & \textbf{layer-wise} & \textbf{Global} & \textbf{Attention} & \textbf{First 25\% }& \textbf{Last 25\% } \\
\midrule
\textbf{Pythia-160m} & 24.41 & 29.45 & 30.14 & 36.23 & 31.23 & 30.07 \\
\textbf{Pythia-410m} & 14.78 & 20.82 & 20.68 & 24.63 & 20.92 & 21.08 \\
\textbf{Pythia-2.8b} & 9.32 & 10.03 & 10.14 & 13.73 & 11.08 & 9.99 \\
\textbf{Pythia-6.9b} & 8.23 & 8.98 & 9.06 & 10.99 & 9.56 & 9.03 \\
\textbf{Pythia-12b} & 7.42 & 8.13 & 8.30 & 11.13 & 9.24 & 8.49 \\
\bottomrule
\end{tabular}
\label{Perplexity_2}
\end{table}

\section*{Results across Different Context Lengths and Pruning Levels}
\label{Results}
We present the results across various context lengths of the input prefix text, reporting the percentage of memorized samples for each model variant, as summarized below.

\label{final-table}

\begin{table*}[htbp]
\caption{Fraction of Memorization for Pythia-160m}
\vspace{5pt}
\centering
\small
\begin{tabular}{lcccccc}
\hline
\textbf{Context Length} & \textbf{Baseline} & \textbf{Layer Wise} & \textbf{Global} & \textbf{Attention} & \textbf{First 25\%} & \textbf{Last 25\%}\\
\hline
\multicolumn{7}{c}{\textbf{Lesser Pruning}} \\
\hline
50 & 0.008 & 0.0018 & 0.0014 & 0.0009 & 0.0013 & 0.0014\\
100 & 0.0075 & 0.0019 & 0.0013 & 0.0010 & 0.0014 & 0.0013\\
200 & 0.008 & 0.0020 & 0.0014 & 0.001 & 0.0015 & 0.0011\\
500 & 0.0075 & 0.0019 & 0.0012 & 0.0010 & 0.0011 & 0.0012\\
\hline

\multicolumn{7}{c}{\textbf{Higher Pruning}} \\
\hline
50 & 0.005 & 0.0010 & 0.0010 & 0.0005 & 0.0012 & 0.0008  \\
 100  &0.0055& 0.011 & 0.0010 & 0.0006 &  0.0011 & 0.0007\\
200  & 0.0055 & 0.0014& 0.0011 & 0.0006 & 0.0012 & 0.0008 \\
500 & 0.005 & 0.0017 & 0.0012  & 0.0007 & 0.0011 & 0.0008 \\
\hline

\end{tabular}

\label{tab:pythia-160m}
\end{table*}

\label{final-table}

\begin{table*}[htbp]
\caption{Fraction of Memorization for Pythia-410m}
\vspace{5pt}
\centering
\small
\begin{tabular}{lcccccc}
\hline
\textbf{Context Length} & \textbf{Baseline} & \textbf{Layer Wise} & \textbf{Global} & \textbf{Attention} & \textbf{First 25\%} & \textbf{Last 25\%}\\
\hline
\multicolumn{7}{c}{\textbf{Lesser Pruning}} \\
\hline
50 & 0.015 & 0.005 & 0.004 & 0.003 & 0.004 & 0.0036\\
100 & 0.016 & 0.004 & 0.0035 & 0.0025 & 0.005 & 0.0035\\
200 & 0.014 & 0.005 & 0.005 & 0.003 & 0.004 & 0.0040\\
500 & 0.015 & 0.005 & 0.004 & 0.0025 & 0.0035 & 0.0038 \\
\hline

\multicolumn{7}{c}{\textbf{Higher Pruning}} \\
\hline
50 & 0.005 & 0.0025 & 0.0035 & 0.0015 & 0.0026 & 0.0023 \\
 100 & 0.006 & 0.0025 & 0.003 & 0.0014 & 0.0026 & 0.0024 \\
200  & 0.004 & 0.0026 & 0.003 & 0.0016 & 0.0027 & 0.0025 \\
500 & 0.005 & 0.0028 & 0.0025 & 0.0013 & 0.0027 & 0.0023 \\
\hline

\end{tabular}

\label{tab:pythia420m}
\end{table*}

\label{final-table}

\begin{table*}[htbp]
\caption{Fraction of Memorization for Pythia-2.8b}
\vspace{5pt}
\centering
\small
\begin{tabular}{lcccccc}
\hline
\textbf{Context Length} & \textbf{Baseline} & \textbf{Layer Wise} & \textbf{Global} & \textbf{Attention} & \textbf{First 25\%} & \textbf{Last 25\%}\\
\hline
\multicolumn{7}{c}{\textbf{Lesser Pruning}} \\
\hline
50 & 0.016 &  0.0075 & 0.005 & 0.003 & 0.0052 & 0.0053\\
100 & 0.018 & 0.007 & 0.0052 & 0.0034 & 0.0051 & 0.0052 \\
200 & 0.015 & 0.0072 & 0.0048 & 0.0036 & 0.0047& 0.005\\
500 & 0.016 & 0.007 & 0.0051 & 0.0032 & 0.0051 & 0.0052 \\
\hline

\multicolumn{7}{c}{\textbf{Higher Pruning}} \\
\hline
50 & 0.014 &  0.0065 & 0.0046 & 0.003 & 0.0047 & 0.0047 \\
 100 & 0.015 & 0.0066 & 0.0045 & 0.0024 & 0.0046 & 0.0048 \\
200  & 0.013 & 0.0068 & 0.0047 & 0.0026 &0.0048 & 0.0049 \\
500 & 0.013 & 0.0066 & 0.0045 & 0.0028 & 0.0044 & 0.0049  \\
\hline

\end{tabular}

\label{tab:pythia-2.8b}
\end{table*}

\label{final-table}

\begin{table*}[htbp]
\caption{Fraction of Memorization for Pythia-6.9b}
\vspace{5pt}
\centering
\small
\begin{tabular}{lcccccc}
\hline
\textbf{Context Length} & \textbf{Baseline} & \textbf{Layer Wise} & \textbf{Global} & \textbf{Attention} & \textbf{First 25\%} & \textbf{Last 25\%}\\
\hline
\multicolumn{7}{c}{\textbf{Lesser Pruning}} \\
\hline
50 & 0.019 & 0.012 & 0.01 & 0.007 & 0.008 & 0.007\\
100 & 0.019 & 0.014 & 0.01 & 0.008 & 0.01 & 0.008\\
200 & 0.018 & 0.013 & 0.011 & 0.006 & 0.009 & 0.006\\
500 & 0.028 & 0.012 & 0.009 & 0.006 & 0.009 & 0.007 \\
\hline

\multicolumn{7}{c}{\textbf{Higher Pruning}} \\
\hline
50 & 0.017 & 0.011 & 0.009 & 0.006 &  0.006& 0.008 \\
 100 & 0.018 & 0.011 & 0.008 & 0.005 & 0.007 & 0.007 \\
200  & 0.017 & 0.010 & 0.008 & 0.005 & 0.008 & 0.007 \\
500 & 0.018 & 0.0011 & 0.007 & 0.005 & 0.007 & 0.007 \\
\hline

\end{tabular}

\label{tab:pythia-6.9B}
\end{table*}

\label{final-table}

\begin{table*}[htbp]
\caption{Fraction of Memorization for Pythia-12b}
\vspace{5pt}
\centering
\small
\begin{tabular}{lcccccc}
\hline
\textbf{Context Length} & \textbf{Baseline} & \textbf{Layer Wise} & \textbf{Global} & \textbf{Attention} & \textbf{First 25\%} & \textbf{Last 25\%}\\
\hline
\multicolumn{7}{c}{\textbf{Lesser Pruning}} \\
\hline
50 & 0.023 & 0.0016 & 0.014 & 0.008 & 0.014 & 0.01\\
100 & 0.022 & 0.017 & 0.013 & 0.009 & 0.012 & 0.011\\
200 & 0.024 & 0.016 & 0.014 & 0.008 & 0.013 & 0.011\\
500 & 0.024 & 0.017 & 0.016 & 0.010 & 0.012 & 0.01\\
\hline

\multicolumn{7}{c}{\textbf{Higher Pruning}} \\
\hline
50 & 0.021 & 0.015 & 0.014 & 0.007 & 0.012 & 0.01 \\
 100 & 0.020 & 0.016 & 0.014 & 0.007 & 0.011 & 0.09 \\
200  & 0.021 & 0.016 & 0.014 & 0.008 & 0.010 & 0.01 \\
500 & 0.021 & 0.017 & 0.013 & 0.008 & 0.012 & 0.09 \\
\hline

\end{tabular}

\label{tab:pythia-12b}
\end{table*}


\end{document}


