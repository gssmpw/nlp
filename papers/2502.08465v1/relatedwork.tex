\section{Related Work}
\label{rw} 
 % \cite{castro1999practical} and Tendermint \cite{buchman2016tendermint,buchman2018latest},
 % To mention: PBFT,  Zyzzyva, Tendermint, Hotstuff, Hashgraph, Aleph, Narwhal, DAG-Rider, Tusk, Bullshark, Cordial Miners, Mysticeti, BBCA-chain, Shaol, Shaol++ (with discussion of relationship to Sailfish), Sailfish, Motorway, Autobahn (and some differences with Motorway), GradedDAG [17] and LightDAG [15] (which use consistent broadcast rather than RB), Avalanche
 % Reliable broadcast (two references) 

 Morpheus uses a PBFT \cite{castro1999practical} style approach to view changes, while consistency between finalised transaction blocks within the same view uses an approach similar to Tendermint \cite{buchman2016tendermint,buchman2018latest} and Hotstuff \cite{yin2019hotstuff}. As noted in Section \ref{metrics}, Hotstuff's approach of relaying all messages via the leader could be used by Morpheus during low throughput to decrease communication complexity, but this is unlikely to lead to a decrease in `real' latency (i.e.\ actual finalisation times). As also noted in Section \ref{metrics}, the optimistic `fast commit' of Zyzzyva \cite{gueta2019sbft,kotla2007zyzzyva} can also be applied as a further optimisation. 

 \vspace{0.2cm} 
 Morpheus transitions between being a leaderless `linear' blockchain during low throughput to a leader-based DAG-protocol during high throughput. DAG protocols have been studied for a number of years, Hashgraph \cite{baird2016swirlds} being an early example. Hashgraph builds an unstructured DAG and suffers from latency exponential in the number of processes. Spectre was another early DAG protocol, designed for the `permissionless'' setting \cite{sompolinsky2016spectre}, with proof-of-work as the mechanism for sybil resistance. The protocol implements a `payment system', but does not totally order transactions.  Aleph \cite{gkagol2019aleph} is more similar to most recent DAG protocols in that it builds a structured DAG in which each process proceeds to the next `round' after receiving blocks from $2f+1$ processes corresponding to the previous round, but still has greater  latency than modern DAG protocols.  

 \vspace{0.2cm}
 More recent DAG protocols use a variety of approaches to consensus. Narwhal \cite{danezis2022narwhal} builds a DAG for the purpose of ensuring data availability, from which (one option is that) a protocol like Hotstuff or PBFT can then be used to efficiently establish a total ordering on transactions. DAG-Rider \cite{keidar2021all}, on the other hand, builds the DAG in such a way that a total ordering can be extracted from the structure of the DAG, with zero further communication cost. The protocol proceeds in `waves', where each wave consists of four rounds, each round building one `layer' of the DAG. In each round, each process uses an instance of Reliable Broadcast (RBC) to disseminate their block for the round. Each wave has a leader and an 
 expected six rounds (6 sequential RBCs) are required to finalise
the leader's block for the first round of the wave. This finalises all blocks observed by that leader block, but other blocks (such as those in the same round as the leader block) may have signicantly greater latency. Tusk \cite{danezis2022narwhal} is an implementation based on DAG-Rider. 

 \vspace{0.2cm} 
 Given the ability of DAG-Rider to handle significantly higher throughput in many settings, when compared to protocols like PBFT that build a linear blockchain, much subsequent work has taken a similar approach, while looking to improve on latency. While DAG-Rider functions in asynchrony, Bullshark \cite{spiegelman2022bullshark} is designed to achieve lower latency in the partially synchronous setting. 
 GradedDAG \cite{dai2023gradeddag} and LightDAG \cite{dai2024lightdag} function in asynchrony, but look to improve latency by replacing RBC \cite{bracha1987asynchronous} with weaker primitives, such as consistent broadcast \cite{srikanth1987simulating}. This means that those protocols solve Extractable SMR (as defined in Section \ref{esmr}), rather than SMR, and that further communication may be required to ensure full block dissemination in executions with faulty processes. Cordial Miners \cite{keidar2022cordial} has versions for both partial synchrony and asynchrony and further decreases latency 
by using the DAG structure (rather than any primitive such as Consistent or Reliable Broadcast) for equivocation exlcusion. Mysticeti \cite{babel2023mysticeti} builds on Cordial Miners and establishes a mechanism to  accommodate multiple leaders within a single round. Shaol \cite{spiegelman2023shoal} and Shoal++ \cite{arun2024shoal++} extend Bullshark by establishing a `pipelining approach' that implements simultaneous instances of Bullshark with a leader in each round. This reduces latency in the good case because one is required to wait less time before reaching a round in which a leader block is finalised. Both of these papers, however, use a `reputation' system to select leaders, which comes its own trade-offs. Sailfish \cite{shrestha2024sailfish} similarly describes a mechanism where each round has a leader, but does not make use of a reputation system. As noted previously, the protocol most similar to Morpheus during high throughput is Autobahn \cite{giridharan2024autobahn}. One of the major distinctions between Autobahn and those previously discussed, is that most blocks are only required to point to a single parent. This significantly decreases communication complexity when the number of processes is large and allows one to achieve linear ammortised communction complexity without the use of erasure coding \cite{alhaddad2022balanced,nayak2020improved} or batching \cite{miller2016honey}. 

 

 
 
 \bibliographystyle{plainurl}

\begin{thebibliography}{10}

\bibitem{alhaddad2022balanced}
Nicolas Alhaddad, Sourav Das, Sisi Duan, Ling Ren, Mayank Varia, Zhuolun Xiang,
  and Haibin Zhang.
\newblock Balanced byzantine reliable broadcast with near-optimal communication
  and improved computation.
\newblock In {\em Proceedings of the 2022 ACM Symposium on Principles of
  Distributed Computing}, pages 399--417, 2022.

\bibitem{arun2024shoal++}
Balaji Arun, Zekun Li, Florian Suri-Payer, Sourav Das, and Alexander
  Spiegelman.
\newblock Shoal++: High throughput dag bft can be fast!
\newblock {\em arXiv preprint arXiv:2405.20488}, 2024.

\bibitem{babel2023mysticeti}
Kushal Babel, Andrey Chursin, George Danezis, Anastasios Kichidis, Lefteris
  Kokoris-Kogias, Arun Koshy, Alberto Sonnino, and Mingwei Tian.
\newblock Mysticeti: Reaching the limits of latency with uncertified dags.
\newblock {\em arXiv preprint arXiv:2310.14821}, 2023.

\bibitem{baird2016swirlds}
Leemon Baird.
\newblock The swirlds hashgraph consensus algorithm: Fair, fast, byzantine
  fault tolerance.
\newblock {\em Swirlds Tech Reports SWIRLDS-TR-2016-01, Tech. Rep}, 34:9--11,
  2016.

\bibitem{bollobas2013modern}
B{\'e}la Bollob{\'a}s.
\newblock {\em Modern graph theory}, volume 184.
\newblock Springer Science \& Business Media, 2013.

\bibitem{boneh2001short}
Dan Boneh, Ben Lynn, and Hovav Shacham.
\newblock Short signatures from the weil pairing.
\newblock In {\em International conference on the theory and application of
  cryptology and information security}, pages 514--532. Springer, 2001.

\bibitem{bracha1987asynchronous}
Gabriel Bracha.
\newblock Asynchronous byzantine agreement protocols.
\newblock {\em Information and Computation}, 75(2):130--143, 1987.

\bibitem{buchman2016tendermint}
Ethan Buchman.
\newblock {\em Tendermint: Byzantine fault tolerance in the age of
  blockchains}.
\newblock PhD thesis, 2016.

\bibitem{buchman2018latest}
Ethan Buchman, Jae Kwon, and Zarko Milosevic.
\newblock The latest gossip on bft consensus.
\newblock {\em arXiv preprint arXiv:1807.04938}, 2018.

\bibitem{castro1999practical}
Miguel Castro, Barbara Liskov, et~al.
\newblock Practical byzantine fault tolerance.
\newblock In {\em OsDI}, volume~99, pages 173--186, 1999.

\bibitem{dai2024lightdag}
Xiaohai Dai, Guanxiong Wang, Jiang Xiao, Zhengxuan Guo, Rui Hao, Xia Xie, and
  Hai Jin.
\newblock Lightdag: A low-latency dag-based bft consensus through lightweight
  broadcast.
\newblock {\em Cryptology ePrint Archive}, 2024.

\bibitem{dai2023gradeddag}
Xiaohai Dai, Zhaonan Zhang, Jiang Xiao, Jingtao Yue, Xia Xie, and Hai Jin.
\newblock Gradeddag: An asynchronous dag-based bft consensus with lower
  latency.
\newblock In {\em 2023 42nd International Symposium on Reliable Distributed
  Systems (SRDS)}, pages 107--117. IEEE, 2023.

\bibitem{danezis2022narwhal}
George Danezis, Lefteris Kokoris-Kogias, Alberto Sonnino, and Alexander
  Spiegelman.
\newblock Narwhal and tusk: a dag-based mempool and efficient bft consensus.
\newblock In {\em Proceedings of the Seventeenth European Conference on
  Computer Systems}, pages 34--50, 2022.

\bibitem{gkagol2019aleph}
Adam G{\k{a}}gol, Damian Le{\'s}niak, Damian Straszak, and Micha{\l}
  {\'S}wi{\k{e}}tek.
\newblock Aleph: Efficient atomic broadcast in asynchronous networks with
  byzantine nodes.
\newblock In {\em Proceedings of the 1st ACM Conference on Advances in
  Financial Technologies}, pages 214--228, 2019.

\bibitem{giridharan2024autobahn}
Neil Giridharan, Florian Suri-Payer, Ittai Abraham, Lorenzo Alvisi, and Natacha
  Crooks.
\newblock Autobahn: Seamless high speed bft.
\newblock In {\em Proceedings of the ACM SIGOPS 30th Symposium on Operating
  Systems Principles}, pages 1--23, 2024.

\bibitem{gueta2019sbft}
Guy~Golan Gueta, Ittai Abraham, Shelly Grossman, Dahlia Malkhi, Benny Pinkas,
  Michael Reiter, Dragos-Adrian Seredinschi, Orr Tamir, and Alin Tomescu.
\newblock Sbft: A scalable and decentralized trust infrastructure.
\newblock In {\em 2019 49th Annual IEEE/IFIP international conference on
  dependable systems and networks (DSN)}, pages 568--580. IEEE, 2019.

\bibitem{keidar2021all}
Idit Keidar, Eleftherios Kokoris-Kogias, Oded Naor, and Alexander Spiegelman.
\newblock All you need is dag.
\newblock In {\em Proceedings of the 2021 ACM Symposium on Principles of
  Distributed Computing}, pages 165--175, 2021.

\bibitem{keidar2022cordial}
Idit Keidar, Oded Naor, Ouri Poupko, and Ehud Shapiro.
\newblock Cordial miners: Fast and efficient consensus for every eventuality.
\newblock {\em arXiv preprint arXiv:2205.09174}, 2022.

\bibitem{kotla2007zyzzyva}
Ramakrishna Kotla, Lorenzo Alvisi, Mike Dahlin, Allen Clement, and Edmund Wong.
\newblock Zyzzyva: speculative byzantine fault tolerance.
\newblock In {\em Proceedings of twenty-first ACM SIGOPS symposium on Operating
  systems principles}, pages 45--58, 2007.

\bibitem{miller2016honey}
Andrew Miller, Yu~Xia, Kyle Croman, Elaine Shi, and Dawn Song.
\newblock The honey badger of bft protocols.
\newblock In {\em Proceedings of the 2016 ACM SIGSAC conference on computer and
  communications security}, pages 31--42, 2016.

\bibitem{nayak2020improved}
Kartik Nayak, Ling Ren, Elaine Shi, Nitin~H Vaidya, and Zhuolun Xiang.
\newblock Improved extension protocols for byzantine broadcast and agreement.
\newblock {\em arXiv preprint arXiv:2002.11321}, 2020.

\bibitem{shoup2000practical}
Victor Shoup.
\newblock Practical threshold signatures.
\newblock In {\em International Conference on the Theory and Applications of
  Cryptographic Techniques}, pages 207--220. Springer, 2000.

\bibitem{shrestha2024sailfish}
Nibesh Shrestha, Rohan Shrothrium, Aniket Kate, and Kartik Nayak.
\newblock Sailfish: Towards improving latency of dag-based bft.
\newblock {\em Cryptology ePrint Archive}, 2024.

\bibitem{sompolinsky2016spectre}
Yonatan Sompolinsky, Yoad Lewenberg, and Aviv Zohar.
\newblock Spectre: A fast and scalable cryptocurrency protocol.
\newblock {\em Cryptology ePrint Archive}, 2016.

\bibitem{spiegelman2023shoal}
Alexander Spiegelman, Balaji Arun, Rati Gelashvili, and Zekun Li.
\newblock Shoal: Improving dag-bft latency and robustness.
\newblock {\em arXiv preprint arXiv:2306.03058}, 2023.

\bibitem{spiegelman2022bullshark}
Alexander Spiegelman, Neil Giridharan, Alberto Sonnino, and Lefteris
  Kokoris-Kogias.
\newblock Bullshark: Dag bft protocols made practical.
\newblock In {\em Proceedings of the 2022 ACM SIGSAC Conference on Computer and
  Communications Security}, pages 2705--2718, 2022.

\bibitem{srikanth1987simulating}
TK~Srikanth and Sam Toueg.
\newblock Simulating authenticated broadcasts to derive simple fault-tolerant
  algorithms.
\newblock {\em Distributed Computing}, 2(2):80--94, 1987.

\bibitem{yin2018hotstuff}
Maofan Yin, Dahlia Malkhi, Michael~K Reiter, Guy~Golan Gueta, and Ittai
  Abraham.
\newblock Hotstuff: Bft consensus in the lens of blockchain.
\newblock {\em arXiv preprint arXiv:1803.05069}, 2018.

\bibitem{yin2019hotstuff}
Maofan Yin, Dahlia Malkhi, Michael~K Reiter, Guy~Golan Gueta, and Ittai
  Abraham.
\newblock Hotstuff: Bft consensus with linearity and responsiveness.
\newblock In {\em Proceedings of the 2019 ACM Symposium on Principles of
  Distributed Computing}, pages 347--356, 2019.

\end{thebibliography}



 
 
 



%
%      \vspace{0.1cm} 
% \noindent \textbf{Signatures}. We write $m_{p_i}$ to denote\footnote{While the subfix $p_i$ is sometimes used to indicate a value specific to $p_i$ (such as in the case of $\mathtt{log}_{p_i}$), no ambiguity will result. The subfix $p_i$ only indicates a signature when applied to individual messages.} the message $m$ signed by $p_i$. 
%