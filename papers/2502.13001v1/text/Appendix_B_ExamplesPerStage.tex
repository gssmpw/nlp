This appendix shows intermediate results of \pipeline{} with a GPT backbone generating a meeting transcript from the ``Pandemics'' Wikipedia article.
Given this input to Stage 1: Content Brainstorming, the pipeline extracts the topics of the article and generates a target summary (\Cref{fig:stage1_scouting}).
Informed about the summary, the pipeline defines during Stage 2: Casting a set of participants such as in \Cref{fig:participant_figure} and further extends the summary into a meeting outline (\Cref{fig:outline_figure}) during Stage 3: Scripting.
The participants are then orchestrated to discuss the points on the outline turn-by-turn, producing one scene per topic such as \Cref{fig:stage4_filming}.
This raw scene undergoes a refinement step during Stage 5: Quality assuring from the director LLM which provides a thorough feedback (\Cref{fig:stage5_feedback}).
If the director LLM approves a scene, the pipeline may inject, with a probability of 25\%, a special effect into the scene (\Cref{fig:stage6_special_effects}) during Stage 6: Special effects.
Finally, during Stage 7: Editing, the scene is assessed by an editorial LLM and a detector-revision LLM to polish AI content %, which both provide feedback (\Cref{A,B}) based on which 
and generate the final scene (\Cref{fig:stage7_post_processing}).
After all scenes undergo this procedure, the whole meeting is generated.


\begin{figure*}[t]
    \begin{AIbox}{Stage 1: Content Brainstorming - Target Summary}
    \parbox[t]{\textwidth}{
        \textbf{Meeting Summary}\\
        The meeting focused on understanding pandemics, their historical impact, and strategies for prevention and management. It was noted that pandemics are characterized by a sudden increase in cases of an infectious disease spreading across large regions or globally. Historical examples include the Black Death and the 1918 Spanish flu, with recent instances being HIV/AIDS, swine flu, and COVID-19.
    
        Key points included the ongoing efforts by WHO to establish an International Treaty on Pandemic Prevention, Preparedness, and Response. The White House has also released policies to manage medical research involving hazardous pathogens. Definitions of pandemics were clarified, emphasizing that they must be infectious and widespread.
    
        The importance of international collaboration was highlighted through initiatives like WHO's Pandemic Hub and the International Pathogen Surveillance Network. Efforts by CEPI to expedite vaccine development were discussed alongside NIAID's Pandemic Preparedness Plan focusing on diagnostics and therapies.
    
        Strategies for pandemic control were reviewed, including containment measures like contact tracing and isolation, as well as mitigation efforts to slow disease spread. Non-pharmaceutical interventions such as social distancing and enhanced air quality measures were considered crucial.
    
        Ethical issues during pandemics were addressed, including resource allocation and vaccination mandates. The need for robust public awareness campaigns to combat misinformation was emphasized. Finally, concerns about antibiotic-resistant microorganisms and climate change's impact on disease spread were acknowledged as future challenges requiring attention.
    
        \textbf{Article Tags}: \\
        $\bigr[$"Pandemics", "Infectious diseases", "World Health Organization (WHO)", "COVID-19 pandemic", "Public health preparedness"$\bigr]$
    }
    \end{AIbox}
    \caption{MIP prompt template in the format for multi-instance usage. In the single-instance setup, the definition and example blocks are repeated for every error type.}
    \label{fig:stage1_scouting}
\end{figure*}



\begin{figure*}
    \centering
    \includegraphics[width=0.9\linewidth]{figure/storage/new_participant.png}
    \caption{Example of a participant profile.}
    \label{fig:participant_figure}
\end{figure*}


\begin{figure*}
    \centering
    \includegraphics[width=0.9\linewidth]{figure/storage/new_meeting_scenes.png}
    \caption{Example of an outline.}
    \label{fig:outline_figure}
\end{figure*}


\begin{figure*}[t]
    \begin{AIbox}{Stage 4: Filming - Raw Meeting Scene}
    \parbox[t]{\textwidth}{
    >>\textbf{Virologist}: According to the latest research, CEPI's initiative to condense vaccine development timelines to 100 days is groundbreaking! However, rapid vaccine development alone isn't enough—NIAID's Pandemic Preparedness Plan, which focuses on diagnostics and therapies, is equally vital.
    
    >>\textbf{Public Health Policy Maker}: To address this issue, we must integrate NIAID's Pandemic Preparedness Plan with CEPI's efforts. Diagnostics and therapies should complement rapid vaccine deployment for a comprehensive response.
    
    >>\textbf{Epidemiologist}: Integration is essential for a seamless system where early detection leads to immediate action, reducing the spread of emerging threats.
    
    >>\textbf{Sociologist}: From a sociological perspective, robust systems for detection and treatment boost public trust and compliance with vaccination programs.
    
    >>\textbf{Public Health Policy Maker}: We must ensure measures are scalable and adaptable to low-resource settings, addressing disparities effectively.
    
    >>\textbf{Epidemiologist}: Integration is about synergy—early detection should lead to immediate action. This approach can significantly curb emerging threats.

    >>\textbf{Public Health Policy Maker}: Indeed! The integration of diagnostics and therapies is crucial, but we must also ensure that these measures are scalable and adaptable to different regions. It's imperative that we consider the unique challenges faced by low-resource settings and develop policies that address these disparities effectively.
    
    >>\textbf{Epidemiologist}: It's crucial to understand that the integration of diagnostics and therapies, as outlined in NIAID's Pandemic Preparedness Plan, is not just about having tools ready; it's about creating a seamless system where early detection leads to immediate action. This synergy can significantly reduce the spread and impact of emerging threats.
    

    ...
  


    ...
    
    >>\textbf{Public Health Policy Maker}: To address this issue, it's imperative that we consider the integration of NIAID's Pandemic Preparedness Plan with CEPI's vaccine development efforts. What we need to understand is how diagnostics and therapies can complement rapid vaccine deployment to ensure a comprehensive response.
    
    >>\textbf{Epidemiologist}: From an epidemiological standpoint, it's crucial to understand that the integration of diagnostics and therapies is not just about having tools ready; it's about creating a seamless system where early detection leads to immediate action. This synergy can significantly reduce the spread and impact of emerging threats.
    
    >>\textbf{Public Health Policy Maker}: To address this issue, it's imperative that we consider the integration of NIAID's Pandemic Preparedness Plan with CEPI's vaccine development efforts. What we need to understand is how diagnostics and therapies can complement rapid vaccine deployment to ensure a comprehensive response.

    }
    \end{AIbox}
    \caption{Stage 4: Raw meeting scene, focusing on the discussion around vaccine development and diagnostics integration.}
    \label{fig:stage4_filming}
\end{figure*}




\begin{figure*}[t]
    \begin{AIbox}{Stage 5: Director - Feedback}
    \parbox[t]{\textwidth}{
    The summary outlines that the scene should discuss initiatives for vaccine development and pandemic preparedness, highlighting CEPI's efforts to expedite vaccine development and NIAID's Pandemic Preparedness Plan, which focuses on diagnostics and therapies.

    \textbf{Strengths:}\\
    - The transcript covers CEPI's initiative to condense vaccine development timelines to 100 days, meeting a core requirement.

    \textbf{Areas for Improvement:}\\
    - The dialogue omits details about NIAID's Pandemic Preparedness Plan and its focus on diagnostics and therapies, which are essential components of the scene.\\
    - The conversation becomes repetitive, reiterating the same points without adding depth or new insights, which reduces engagement and coherence.
    }
    \end{AIbox}
    \caption{Stage 5: Director's feedback on the raw meeting scene, focusing on content gaps and dialogue improvement.}
    \label{fig:stage5_feedback}
\end{figure*}


\begin{figure*}[t]
    \begin{AIbox}{Stage 6: Special Effects}
    \parbox[t]{\textwidth}{
    >>\textbf{Virologist}: According to the latest research, CEPI's initiative to condense vaccine development timelines to 100 days is groundbreaking! However, rapid vaccine development alone isn't enough—NIAID's Pandemic Preparedness Plan, which focuses on diagnostics and therapies, is equally vital for a comprehensive response.

    >>\textbf{Public Health Policy Maker}: Integrating NIAID's plan with CEPI's efforts is imperative. Diagnostics and therapies should complement rapid vaccine deployment for an effective response.

    >>\textbf{Epidemiologist}: Um, integration is essential—these elements must work synergistically to address emerging threats effectively.

    >>\textbf{Public Health Policy Maker}: Absolutely! We must consider logistical challenges to ensure equitable access to diagnostics and therapies alongside rapid vaccine deployment.

    >>\textbf{Sociologist}: From a sociological perspective, integration boosts public trust—people are more likely to comply with health policies when they know robust systems for early detection and treatment are in place.

    >>\textbf{Epidemiologist}\textbf{}: You know, creating a seamless system where early detection triggers immediate action is critical to reduce the impact of emerging threats.

    \textit{Phone rings briefly}

    >>\textbf{Virologist}: Sorry about that interruption. As I was saying...

   \textit{Participants chuckle lightly}

    >>\textbf{Virologist}: No problem at all! Let's continue.

    >>\textbf{Epidemiologist}: Right, from an epidemiological standpoint, it’s about more than tools—it’s about synergy, where early detection leads to immediate action, reducing threat impact.

    >>\textbf{Public Health Policy Maker}: Indeed! Adaptability to various regional contexts is vital, with policies addressing disparities in low-resource settings.
    }
    \end{AIbox}
    \caption{Stage 6: Special Effects — Full dialogue with interruptions and interactions.}
    \label{fig:stage6_special_effects}
\end{figure*}





\begin{figure*}[t]
    \begin{AIbox}{Stage 7: Post-processing}
    \parbox[t]{\textwidth}{
    >>\textbf{Virologist}: According to the latest research, CEPI's initiative to condense vaccine development timelines to 100 days is groundbreaking! However, we need a comprehensive approach beyond rapid vaccine development.

    >>\textbf{Public Health Policy Maker}: Absolutely! We must ensure diagnostics and therapies are available during vaccine rollout. Integrating NIAID's Pandemic Preparedness Plan with CEPI's efforts is key.

    >>\textbf{Epidemiologist}: Right. A robust diagnostic system can help prioritize vaccine distribution during outbreaks. Last flu season, regions with better diagnostics managed outbreaks more effectively.

    >>\textbf{Sociologist}: From a sociological angle, early detection and treatment systems build public trust. However, public resistance due to misinformation is a concern.

    >>\textbf{Public Health Policy Maker}: Agreed. Equitable access and readiness are crucial. We must address potential distribution bottlenecks, especially in low-resource settings.

    \textit{Phone rings briefly}

    >>\textbf{Virologist}: Sorry about that interruption.

    \textit{Participants chuckle lightly}

    >>\textbf{Public Health Policy Maker}: No worries, it happens!

    >>\textbf{Virologist}: Thanks! So, as I was saying...

    >>\textbf{Epidemiologist}: Creating a seamless system where early detection triggers immediate action can reduce the spread of emerging threats.

    >>\textbf{Sociologist}: Integration is promising, but misinformation remains a barrier. How do we tackle it?

    >>\textbf{Public Health Policy Maker}: Indeed! Plans must be adaptable across regions and include strategies to counter misinformation effectively.
    }
    \end{AIbox}
    \caption{Stage 7: Post-processing — Full dialogue with context and interactions.}
    \label{fig:stage7_post_processing}
\end{figure*}
