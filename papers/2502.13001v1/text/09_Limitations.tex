The quality of generated meetings partly depends on the underlying LLM's capabilities and biases.
Models with smaller context windows or different linguistic styles may produce less coherent dialogues or more frequent artifacts.
Nevertheless, our ablation study shows that even mid-scale LLMs (e.g., Llama 3.3) can produce high-quality transcripts, aided by our feedback loops and refinement stages, to address major flaws.
Although \dataset{} covers seven broad Wikipedia domains and features a German subset, it does not encompass all real-world meeting types or cultural nuances (e.g., corporate etiquette, cross-cultural communication).
While specialized domains (e.g., medical conferences) remain outside our scope, our framework can be easily adapted to additional knowledge sources, as evidenced by our tests with short stories and research papers (see \Cref{sec:ablation}).

A small portion of transcripts shows recurring phrases (e.g., `` That's an excellent point'') or overly polite tones that may hint at synthetic origins.
Our multi-stage post-production phase detects and revises these repeated patterns, minimizing mechanical politeness and introducing more diverse expressions.
In practice, only 1 out of 30 transcripts required major edits, suggesting that the remaining artifacts do not substantially undermine \dataset{}’s overall realism.