This appendix contains the definitions of the three quality measures applied in \Cref{sec:data_insights}, i.e., overall authenticity (\Cref{tab:appendix_authenticity_questions}) from \citet{ChenPTK23}, our eighteen questions on behavior authenticity (\Cref{tab:appendix_behavior_questions}) defined from the knowledge, power, conflict, status, trust, support, similarity, and fun \cite{ChoiAVQ20,bales2009interaction}, and the challenges in meeting transcripts defined by \citet{KirsteinWRG24a}.


\begin{table}[ht]
  \centering
  \scriptsize
  \begin{tabularx}{\linewidth}{lX}
    \toprule
    \rowcolor{gray!20}
    \textbf{Item} & \textbf{Description} \\
    \midrule
    Naturalness & How natural the conversation flows, like native English speakers (1-5) \\
    Coherence   & How well the conversation maintains logical flow and connection (1-5) \\
    Interesting & How engaging and content-rich the conversation is (1-5) \\
    Consistency & How consistent each speaker's contributions are (1-5) \\
    \bottomrule
  \end{tabularx}
  \caption{Overall authenticity evaluation following \Cref{sec:data_insights}.}
  \label{tab:appendix_authenticity_questions}
\end{table}



\begin{table*}[ht]
  \centering
  \scriptsize % Adjust the font size if needed
  \begin{tabular}{lll}
    \toprule
    \rowcolor{gray!20}
    \textbf{Short Version} & \textbf{Category} & \textbf{Description} \\
    \midrule
    Q1: Information Exchange    & \cellcolor{pastelBlue}Knowledge  & Participants exchange information or knowledge. \\
    Q2: Knowledge Seeking       & \cellcolor{pastelBlue}Knowledge  & Participants request or seek knowledge. \\
    Q3: Explanation Provision   & \cellcolor{pastelBlue}Knowledge  & Participants clarify previous statements upon request. \\
    Q4: Influence Attempts      & \cellcolor{pastelRed}Power        & Participants attempt to influence another participant’s behavior or decisions. \\
    Q5: Topic Control           & \cellcolor{pastelRed}Power        & Participants take control of a topic or subtopic to steer outcomes in their favor. \\
    Q6: Power Dynamics          & \cellcolor{pastelRed}Power        & A power dynamic exists among participants. \\
    Q7: Response Patterns       & \cellcolor{pastelOrange}Conflict  & Participants fail to engage with others’ suggestions. \\
    Q8: Standpoint Maintenance  & \cellcolor{pastelRed}Power        & Participants insist on their own perspective. \\
    Q9: Recognition Expression  & \cellcolor{pastelGreen}Status     & Participants express recognition, gratitude, or admiration toward others. \\
    Q10: Dependency Expression  & \cellcolor{pastelViolet}Trust     & Participants indicate reliance on another participant’s actions or judgments. \\
    Q11: Support Offering       & \cellcolor{pastelYellow}Support   & Participants offer help or support to others. \\
    Q12: Shared Interests       & \cellcolor{pastelPurple}Similarity& Participants discuss shared interests or motivations. \\
    Q13: View Alignment         & \cellcolor{pastelPurple}Similarity& Participants exhibit aligned views or opinions. \\
    Q14: Mood Management        & \cellcolor{pastelPink}Fun         & Participants attempt to lighten the atmosphere. \\
    Q15: Social Interaction     & \cellcolor{pastelPink}Fun         & Participants discuss leisure activities or enjoyable moments. \\
    Q16: Opinion Divergence     & \cellcolor{pastelOrange}Conflict  & Participants express divergent opinions. \\
    Q17: Conflict Presence      & \cellcolor{pastelOrange}Conflict  & Conflicts or tensions emerge among participants. \\
    Q18: Discussion Dynamics    & \cellcolor{pastelOrange}Conflict  & Participants engage in discussions about disagreements, topics, or decisions. \\
    \bottomrule
  \end{tabular}
  \caption{Psychology-grounded Framework to evaluate participant behavior.}
  \label{tab:appendix_behavior_questions}
\end{table*}



\begin{table*}[ht]
  \centering
  \footnotesize
  % Define a custom column type for better control (optional)
  \newcolumntype{Y}{>{\raggedright\arraybackslash}X}
  \begin{tabularx}{\textwidth}{@{} l Y Y @{}}
    \toprule
    \rowcolor{gray!20}
    \textbf{Category} & \textbf{Definition} & \textbf{Instructions} \\
    \midrule
    Spoken language &
    The extent to which the transcript exhibits spoken-language features---such as colloquialisms, jargon, false starts, or filler words---that make it harder to parse or summarize. &
    1. Are there noticeable filler words, false starts, or informal expressions? \newline
    2. Does domain-specific jargon disrupt straightforward summarization? \newline
    3. How challenging are these elements for generating a coherent summary? \\
    \addlinespace

    Speaker dynamics &
    The challenge of correctly identifying and distinguishing between multiple speakers, tracking who said what, and maintaining awareness of speaker roles if relevant. &
    1. Is it difficult to keep track of speaker identities or roles? \newline
    2. How significantly do these dynamics affect clarity for summarization? \\
    \addlinespace

    Coreference &
    The difficulty in resolving references (e.g., who or what a pronoun refers to) or clarifying references to previous actions or decisions, so the summary remains coherent. &
    1. Are references (e.g., pronouns like “he” or “that”) ambiguous? \newline
    2. Do unclear references to earlier points or events appear? \newline
    3. How difficult is it to track these references for summary generation? \\
    \addlinespace

    Discourse structure &
    The complexity of following the meeting’s high-level flow---especially if it changes topics or has multiple phases (planning, debate, decision). &
    1. Does the transcript jump between topics or phases without clear transitions? \newline
    2. Are meeting phases or topical shifts difficult to delineate? \newline
    3. How challenging is it to maintain an overview for the summary? \\
    \addlinespace

    Contextual turn-taking &
    The challenge of interpreting local context as speakers take turns, including interruptions, redundancies, and how each turn depends on previous utterances. &
    1. Do abrupt speaker turns or interjections complicate continuity? \newline
    2. Are important points lost or repeated inconsistently? \newline
    3. How difficult is it to integrate these nuances into a coherent summary? \\
    \addlinespace

    Implicit context &
    The reliance on unspoken or assumed knowledge, such as organizational history, known issues, or prior decisions, only vaguely referenced in the meeting. &
    1. Do participants refer to hidden topics or internal knowledge without explaining? \newline
    2. Is there essential background or context missing? \newline
    3. How much does summarization rely on understanding this hidden context? \\
    \addlinespace

    Low information density &
    Segments where salient information is sparse, repeated, or only occasionally surfaced---making it hard to find and isolate key points in a sea of less relevant content. &
    1. Are there long stretches with minimal new information? \newline
    2. Is meaningful content buried under trivial or repetitive remarks? \newline
    3. How challenging is it to isolate crucial points for the summary? \\
    \bottomrule
  \end{tabularx}
  \caption{Summary challenges from \citet{KirsteinWRG24a} and their evaluation instructions.}
  \label{tab:summary_challenges}
\end{table*}
