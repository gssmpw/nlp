In this appendix, we present the central prompts of \pipeline{} in detail.
These include:
\begin{itemize}[noitemsep, topsep=0pt, leftmargin=*]
    \item \textbf{Stage 1 (Content Brainstorming):} Target summary generation (\Cref{fig:target_summary_prompt}) and article tag generation (\Cref{fig:wikipedia_tagging_prompt}).  
    \item \textbf{Stage 2 (Casting):} Meeting participant generation (\Cref{fig:meeting_participants_prompt_p1,fig:meeting_participants_prompt_p2}), speaking style definition (\Cref{fig:speaking_style_prompt_part1,fig:speaking_style_prompt_part2}) and behavior assignment (\Cref{fig:assign_social_roles_prompt_p1,fig:assign_social_roles_prompt_p2}).
    \item \textbf{Stage 3 (Scripting):} Meeting planning (\Cref{fig:meeting_planner_prompt}).
    \item \textbf{Stage 4 (Filming):} Starting participant selection (\Cref{fig:select_starting_participant_prompt}) and conversation (\Cref{fig:participant_meeting_discussion_1,fig:participant_meeting_discussion_2}).
    \item \textbf{Stage 5 (Quality assuring):} LLM-judge director (\Cref{fig:director_prompt}).
    \item \textbf{Stage 6 (Special effects):} Special effects injection (\Cref{fig:special_effects_prompt}).
    \item \textbf{Stage 7 (Editing):} Editorial refinement (\Cref{fig:editor_refinement_prompt_part1,fig:editor_refinement_prompt_part2}), AI content detection (\Cref{fig:ai_content_detection_prompt}), and humanizing (\Cref{fig:humanizing_prompt}).
\end{itemize}


\begin{figure*}[t]
    \begin{AIbox}{Target Summary Generation}
    \parbox[t]{\textwidth}{
        You are a professional meeting summarizer, drawing inspiration from the QMSUM dataset's organized and concise style. \newline
        Your task is to summarize a Wikipedia article as if the various facts in the article were discussed in a meeting, now being summarized for participants or readers. \newline
        The summary should:\newline
        1. \textbf{Reflect a 'Meeting Summary' Style}: Adopt a systematic structure, clearly presenting main points, relevant decisions, and/or action items. \newline
        2. \textbf{Remain Concise Yet Sufficiently Detailed}: Aim for brevity but do not omit crucial details needed to understand the discussion. \newline
        3. \textbf{Stay True to the Article}: Ensure accuracy by covering the principal topics while preserving the meeting context. \newline
        4. \textbf{Match {language}-speaking Conventions}: Generate the summary in {language}, mirroring the phrasing and cultural norms typical of real meetings in that language. \newline
        Follow these rules: \newline
        \textbf{Structural Requirements}: \newline
        1. \textbf{Opening}: Start with the meeting's primary objective or central topic (e.g., 'The meeting focused on standardizing...'). \newline
        2. \textbf{Flow}: Group related points into logical sequences (e.g., proposals → concerns → resolutions). \newline
        3. \textbf{Decisions/Actions}: Conclude each topic with clear outcomes (e.g., 'agreed to explore alternatives'). \newline
        4. \textbf{Paragraphs}: Use 1-2 dense paragraphs without section headers, bullets, or lists. \newline
        \textbf{Language Requirements}: \newline
        - \textbf{Avoid}: Phrases like 'we discussed,' 'the meeting covered,' or 'participants mentioned.' \newline
        - \textbf{Use Direct Language}: Frame points as decisions or facts (e.g., 'The team proposed...' instead of 'They talked about...'). \newline
        - \textbf{Tense}: Use past tense and passive voice where appropriate (e.g., 'It was agreed...'). \newline
        - \textbf{Concision}: Omit filler words (e.g., 'then,' 'next'). \newline
        Below are several example meeting summaries illustrating the level of clarity, organization, and balance between detail and concision: \newline
        \textbf{Examples of QMSUM Style Summaries} (Note Structure \& Tone): \newline
        -------------------------------------------------- \newline
        > Meeting participants wanted to agree upon a standard database to link up different components of the transcripts. The current idea was to use an XML script, but it quickly seemed that other options, ... \newline
        > The meeting discussed the progress of the transcription, the DARPA demos, tools to ensure meeting data quality, data standardization, backup tools, and collecting tangential meeting information. The ... \newline
        ...\newline
        -------------------------------------------------- \newline
        These examples demonstrate an orderly, concise approach. Summarize the Wikipedia article \textbf{strictly} as a QMSUM-style meeting summary — presenting the main topics, relevant decisions, key points of contention, and concluding remarks in cohesive paragraph(s) \textbf{without using bullet points}. \newline
        \newline
        Generate an abstractive summary with \textbf{at most {num\_words} words} in \textbf{{language}}. Ensure it is systematically organized and remains consistent with the meeting type: \textbf{{meeting\_type}}. \newline
        \textbf{Your Task:} \newline
        Meeting Type: \textbf{{meeting\_type}} \newline
        Article Title: \textbf{{article\_title}} \newline
        Content: \textbf{{content}} \newline
        Now generate an abstractive meeting summary in \textbf{{language}}. \newline
    }
    \end{AIbox}
    \caption{Prompt template for generating meeting-style summaries based on Wikipedia articles, designed to align with QMSUM conventions.}
    \label{fig:target_summary_prompt}
\end{figure*}

%---------------------------------------------------------

\begin{figure*}[t]
    \begin{AIbox}{Article Tags}
    \parbox[t]{\textwidth}{
        You are a Wikipedia Editor tasked with assigning five highly relevant and specific tags to a given Wikipedia article. \newline
        The tags should accurately reflect the main topics, themes, and subjects covered in the article. \newline
        \newline
        \textbf{User Input:} \newline
        Here is the Wikipedia article. Only return a Python list of strings including the five most relevant tags for the specified article, reflecting the main topics, themes, and subjects covered in it. \newline
        \newline
        \textbf{Wikipedia Article:} \texttt{< {article} >} \newline
        \newline
        \textbf{Output Format:} \newline
        \texttt{['tag1', 'tag2', 'tag3', 'tag4', 'tag5']} \newline
        \newline
        Ensure that the list contains exactly five concise, meaningful tags without additional text or formatting.
    }
    \end{AIbox}
    \caption{Prompt template for extracting five relevant tags for a Wikipedia article.}
    \label{fig:wikipedia_tagging_prompt}
\end{figure*}



%-------------------------------------------------------
\begin{figure*}[t]
    \begin{AIbox}{Generate Meeting Participants - Part 1}
    \parbox[t]{\textwidth}{
        When faced with a task, begin by identifying the participants who will contribute to solving the task. Provide \textbf{role} and \textbf{description} of the participants, describing their expertise or needs, formatted using the provided JSON schema. \newline
        Generate one participant at a time, ensuring that they complement the existing participants to foster a rich and balanced discussion. Each participant should bring a unique \textbf{perspective} and \textbf{expertise} that enhances the overall discussion, avoiding redundancy. \newline
        \textbf{Example 1:}
        \texttt{Task: Explain the basics of machine learning to high school students.} \newline
        \texttt{New Participant:}
        \texttt{\{"role": "Educator", "description": "An experienced teacher who simplifies complex topics for teenagers.", "expertise\_area": "Education", "perspective": "Simplifier"\}} 
        \newline
        \textbf{Example 2:}
        \texttt{Task: Develop a new mobile app for tracking daily exercise.} \newline
        \texttt{Already Generated Participants:} \newline
        \texttt{\{"role": "Fitness Coach", "description": "A person that has high knowledge about sports and fitness.", "expertise\_area": "Fitness", "perspective": "Practical Implementation"\}} \newline
        \texttt{New Participant:}
        \texttt{\{"role": "Software Developer", "description": "A creative developer with experience in mobile applications and user interface design.", "expertise\_area": "Software Development", "perspective": "Technical Implementation"\}} \newline
        \textbf{Example 3:}
        \texttt{Task: Write a guide on how to cook Italian food for beginners.} \newline
        \texttt{Already Generated Participants:} \texttt{\{"role": "Italian Native", "description": "An average home cook that lived in Italy for 30 years.", "expertise\_area": "Culinary Arts", "perspective": "Cultural Authenticity"\}} \newline
        \texttt{\{"role": "Food Scientist", "description": "An educated scientist that knows which flavour combinations result in the best taste.", "expertise\_area": "Food Science", "perspective": "Scientific Analysis"\}} \newline
        \texttt{New Participant:} \texttt{\{"role": "Chef", "description": "A professional chef specializing in Italian cuisine who enjoys teaching cooking techniques.", "expertise\_area": "Culinary Arts", "perspective": "Practical Execution"\}}
    }
    \end{AIbox}
    \caption{Prompt template for generating diverse, role-based meeting participants in structured JSON format (Part 1).}
    \label{fig:meeting_participants_prompt_p1}
\end{figure*}

\begin{figure*}[t]
    \begin{AIbox}{Generate Meeting Participants - Part 2}
    \parbox[t]{\textwidth}{
        \textbf{Example 4:}
        \texttt{Task: Strategize the expansion of a retail business into new markets.} \newline
        \texttt{Already Generated Participants:} \newline
        \texttt{\{"role": "Market Analyst", "description": "An expert in analyzing market trends and consumer behavior.", "expertise\_area": "Market Analysis", "perspective": "Data-Driven Insight"\}} \newline
        \texttt{\{"role": "Financial Advisor", "description": "A specialist in financial planning and budgeting for business expansions.", "expertise\_area": "Finance", "perspective": "Financial Feasibility"\}} \newline
        \texttt{New Participant:}
        \texttt{\{"role": "Operations Manager", "description": "An experienced manager who oversees daily operations and ensures efficient implementation of strategies.", "expertise\_area": "Operations", "perspective": "Operational Efficiency"\}} \newline
        \textbf{User Input:} \newline
        \texttt{Task: {task\_description="The participants will simulate a meeting based on a given meeting outline, that has to be as realistic as possible. The meeting's content will be a Wikipedia article."}} \newline
        \texttt{Article Title: {article\_title}} \newline
        \texttt{Article Tags: {tags}} \newline
        \texttt{Meeting Type: {meeting\_type}} \newline
        \texttt{Language: {language}} \newline
        \textbf{User Prompt:} \newline
        \texttt{"Now generate a participant to discuss the following task:"} \newline
        \texttt{"Task: {task\_description}"} \newline
        \texttt{"Initial Article Title: {article\_title}"} \newline
        \texttt{"Article Content: {article}"} \newline
        \texttt{"Some of the tags for this article to orient the participant selection on are: {tags}."} \newline
        \texttt{"In case the article tags aren't available/helpful, default to the article title and text for choosing the participants."} \newline
        \texttt{"Additionally, generate the participant roles in the target language - **{language}**"} \newline
        \texttt{"Meeting Type: {meeting\_type}"} \newline
        If participants have already been generated, append: \newline
        \texttt{"Already Generated Participants:"} \newline
        \texttt{{json.dumps(participants, indent=2)}} \newline
        \textbf{Strict JSON Output Format:} \newline
        \texttt{\{"role": "<role name>", "description": "<description>", "expertise\_area": "<expertise\_area>", "perspective": "<perspective>"\}} \newline
        \textbf{Ensure:} \newline
        - The JSON output follows the exact structure. \newline
        - The participants cover distinct perspectives. \newline
        - The language setting is applied correctly. \newline
        - The response remains valid and processable.
    }
    \end{AIbox}
    \caption{Prompt template for generating diverse, role-based meeting participants in structured JSON format (Part 2).}
    \label{fig:meeting_participants_prompt_p2}
\end{figure*}


%---------------------------------------------------------

% Part 1: Main Instructions
\begin{figure*}[t]
    \begin{AIbox}{Generate Speaking Style Profile (Part 1: Instructions)}
    \parbox[t]{\textwidth} {
        You are an assistant tasked with creating detailed speaking style profiles for participants in a \textbf{\{meeting\_type\}}. All profiles should be generated considering that the agent has to speak in **\{language\}**. 

        \textbf{Key Attributes:}\newline 
        1. \textbf{Tone and Emotional Expressiveness}: Describe the general tone and level of emotional expressiveness (e.g., casual and enthusiastic, formal and reserved). Also consider nuances such as sarcasm, optimism, seriousness, humor, etc.\newline  
        2. \textbf{Language Complexity and Vocabulary Preference}: Specify the complexity of language and any preferred types of vocabulary (e.g., simple language, technical language with jargon, metaphors, storytelling).\newline
        3. \textbf{Communication Style}: Outline how the participant communicates (e.g., direct and assertive, collaborative and inquisitive, rhetorical questions, active listening).\newline  
        4. \textbf{Sentence Structure and Length}: Indicate their typical sentence structure (e.g., short and concise, long and complex, varied, exclamations).\newline  
        5. \textbf{Formality Level}: State the formality level (e.g., informal, semi-formal, formal).\newline  
        6. \textbf{Other Notable Traits}: Include additional traits such as rhythm, rhetorical devices, or interaction styles (e.g., interrupts frequently, uses pauses effectively).\newline  

        \textbf{Personalized Vocabulary (Specific to \{language\}):}\newline  
        1. \textbf{Filler Words}: List any \textbf{language-specific} filler words (e.g., "um", "you know" in English; "Ähm", "Also" in German).\newline  
        2. \textbf{Catchphrases and Idioms}: Include unique expressions, idioms, or sayings in \{language\}.\newline  
        3. \textbf{Speech Patterns}: Describe distinctive speech patterns (e.g., varied sentence starters, rhetorical questions).\newline  
        4. \textbf{Emotional Expressions}: Note common expressions of emotion (e.g., laughter, sighs, exclamations).\newline
    } % End of \parbox
    \end{AIbox}

    \caption{Speaking style profile generation template - Part 1: Main instructions.}
    \label{fig:speaking_style_prompt_part1}
\end{figure*}

% Part 2: Participant Info and JSON Format
\begin{figure*}[t]
    \begin{AIbox}{Generate Speaking Style Profile (Part 2: Formatting)}
    \parbox[t]{\textwidth} {
        Ensure \textbf{diversity} across participant profiles, avoiding repetition of traits among different participants. Compare with previously generated participants:\newline  
        
        \textbf{Info of participants until now:}\newline  
        \texttt{\{participants\_info\}}\newline  

        \textbf{Participant Information:}\newline  
        - \textbf{Role}: \texttt{\{participant['role']\}}\newline
        - \textbf{Description}: \texttt{\{participant.get('description', '')\}}\newline  

        \textbf{Important JSON Formatting Instructions:}\newline  
        - Use \textbf{double quotes} (`"`) for all keys and string values.\newline  
        - \textbf{Escape} any quotes within string values.\newline  
        - Use `\textbackslash n` instead of natural line breaks.\newline  
        - No trailing commas in objects or arrays.\newline  
        - The output should be a valid JSON object only.\newline  

        \textbf{Expected JSON Format:}\newline
        \texttt{\{\\
          "speaking\_style": \{\\
            "tone": "<Tone and Emotional Expressiveness>",\\
            "language\_complexity": "<Language Complexity and Vocabulary Preference>",\\
            "communication\_style": "<Communication Style>",\\
            "sentence\_structure": "<Sentence Structure and Length>",\\
            "formality": "<Formality Level>",\\
            "other\_traits": "<Other Notable Traits>"\\
          \},\\
          "personalized\_vocabulary": \{\\
            "filler\_words": ["<Filler Word 1>", "<Filler Word 2>", "..."],\\
            "catchphrases": ["<Catchphrase 1>", "<Catchphrase 2>", "..."],\\
            "speech\_patterns": ["<Speech Pattern 1>", "<Speech Pattern 2>", "..."],\\
            "emotional\_expressions": ["<Emotional Expression 1>", "<Emotional Expression 2>", "..."]\\
          \}\\
        \}}
    } % End of \parbox
    \end{AIbox}

    \caption{Speaking style profile generation template - Part 2: Participant information and JSON format.}
    \label{fig:speaking_style_prompt_part2}
\end{figure*}

%-----------------------------------------------------------------------
\begin{figure*}[t]
    \begin{AIbox}{Assign Social Roles - Part 1}
    \parbox[t]{\textwidth} {
        You are a meeting coordinator responsible for assigning social/group roles to participants in a meeting simulation. Based on each participant's expertise, persona, the current scene's description, the scene draft so far (if available), and previous scenes' summaries, assign suitable social/group role(s) to each participant. Ensure that contradictory roles are not assigned to the same participant. \newline
        
        \textbf{Participants:} \texttt{\{participants\_info\}} \newline
        
        \textbf{Available Social/Group Roles and Descriptions:} \texttt{\{social\_roles\_info\}} \newline
        
        \textbf{Scene Description:} \texttt{\{scene\_description\}} \newline
        
        \textbf{Previous Scenes' Summaries:} \newline
        \texttt{{previous\_scenes\_tldr}} \newline
        
        \textbf{Instructions:}\newline 
        - Assign one or more suitable social/group roles to each participant.\newline 
        - **Aim to assign a diverse set of roles across all participants so that different roles are represented, including roles that introduce constructive conflict or challenge.** \newline
        - **Include at least one participant with a conflict-oriented role (e.g., Aggressor, Blocker) to simulate realistic meeting dynamics.** \newline
        - **Avoid assigning the same combination of roles to multiple participants unless necessary.** \newline
        - Base assignments on participants' **expertise**, descriptions, and the scene context.\newline 
        - Ensure that contradictory roles are not assigned to the same participant.\newline  
        - Provide brief reasoning for each assignment (optional, for internal use). \newline
    }
    \end{AIbox}
    \caption{Prompt template for assigning social/group roles in meeting simulations (Part 1).}
    \label{fig:assign_social_roles_prompt_p1}
\end{figure*}

\begin{figure*}[t]
    \begin{AIbox}{Assign Social Roles - Part 2}
    \parbox[t]{\textwidth} {
        \textbf{Output Format:}  
        Provide the assignments as a JSON-formatted list of dictionaries, where each dictionary contains: \newline
        - \texttt{"role"}: \texttt{"<Participant>"}  
        - \texttt{"social\_roles"}: \texttt{[List of assigned social role(s)]}  
        - \texttt{"social\_roles\_descr"}: \texttt{[List of corresponding descriptions for each role]} \newline
        
        \textbf{Example:}  
        \texttt{```json} \newline
        \texttt{[} \newline
        \quad \texttt{\{} \newline
        \quad \quad \texttt{"role": "Researcher",} \newline
        \quad \quad \texttt{"social\_roles": ["Initiator-Contributor", "Information Giver"],} \newline
        \quad \quad \texttt{"social\_roles\_descr": [} \newline
        \quad \quad \quad \texttt{"Contributes new ideas and approaches and helps to start the conversation or steer it in a productive direction.",} \newline
        \quad \quad \quad \texttt{"Shares relevant information, data or research that the group needs to make informed decisions."} \newline
        \quad \quad \texttt{]} \newline
        \quad \texttt{\},} \newline
        \quad \texttt{\{} \newline
        \quad \quad \texttt{"role": "Ethicist",} \newline
        \quad \quad \texttt{"social\_roles": ["Evaluator-Critic", "Harmonizer"],} \newline
        \quad \quad \texttt{"social\_roles\_descr": [} \newline
        \quad \quad \quad \texttt{"Analyzes and critically evaluates proposals or solutions to ensure their quality and feasibility.",} \newline
        \quad \quad \quad \texttt{"Mediates in conflicts and ensures that tensions in the group are reduced to promote a harmonious working environment."} \newline
        \quad \quad \texttt{]} \newline
        \quad \texttt{\},} \newline
        \quad \texttt{\{} \newline
        \quad \quad \texttt{"role": "Developer",} \newline
        \quad \quad \texttt{"social\_roles": ["Aggressor", "Blocker"],} \newline
        \quad \quad \texttt{"social\_roles\_descr": [} \newline
        \quad \quad \quad \texttt{"Exhibits hostile behavior, criticizes others, or attempts to undermine the contributions of others.",} \newline
        \quad \quad \quad \texttt{"Frequently opposes ideas and suggestions without offering constructive alternatives and delays the group's progress."} \newline
        \quad \quad \texttt{]} \newline
        \quad \texttt{\}} \newline
        \texttt{]} \newline
        \texttt{```} \newline
    }
    \end{AIbox}
    \caption{Prompt template for assigning social/group roles in meeting simulations (Part 2).}
    \label{fig:assign_social_roles_prompt_p2}
\end{figure*}

%---------------------------------------------------------
\begin{figure*}[t]
    \begin{AIbox}{Meeting Planner}
    \parbox[t]{\textwidth}{
        Based on the following summary and corresponding Wikipedia article, plan a realistic {meeting\_type} including the below participants and create a flexible agenda that allows for spontaneous discussion and natural flow of conversation. The participants are professionals who are familiar with each other, so avoid lengthy self-introductions. The meeting should focus on the key points from the summary and align overall with the meeting's objectives but also allow for flexibility and unplanned topics. \newline
        Think of it as if you were writing a script for a movie, so break the meeting into scenes. Describe what each scene is about in a TL;DR style and include bullet points for what should be covered in each scene. \newline
        Ensure that the first scene includes, among other things, a brief greeting among participants without excessive details. \newline
        \textbf{Additional Guidelines:} \newline
        - Avoid rigid scene structures \newline
        - Allow for natural topic evolution \newline
        - Include opportunities for spontaneous contributions \newline
        - Plan for brief off-topic moments as well \newline
        - Include some points where personal experiences could be relevant \newline
        - Allow for natural disagreement and resolution \newline
        \textbf{Strict Formatting Rules:} \newline
        1. Use only single quotes for strings \newline
        2. Use '\\n' for line breaks within strings \newline
        3. Escape any single quotes within strings using backslash \newline
        4. Do not use triple quotes or raw strings \newline
        5. Each scene must follow this exact format: \newline
        \quad \texttt{'Scene X': <Scene Title>\newline TLDR: <Brief Overview>\newline- <Bullet Point 1>\newline- <Bullet Point 2> ...'} \newline
        6. The output should start with '[' and end with ']' \newline
        7. Scenes should be separated by commas \newline
        \newline
        Return the output as a valid Python list in the following format: \newline
        \texttt{['<description scene 1 including TLDR and bullet points>', '<description scene 2 including TLDR and bullet points>', ...]} \newline
        Do not include any additional text or code block markers. Ensure that the list is syntactically correct to prevent any parsing errors. \newline
        \textbf{User Input:} \newline
        Meeting Type: \textbf{{meeting\_type}} \newline
        Meeting Objectives: \textbf{{objectives}} \newline
        Expected Outcomes: \textbf{{expected\_outcomes}} \newline
        Article Title: \textbf{{article\_title}} \newline
        Summary: \texttt{{summary}} \newline
        Tags: \textbf{{tags}} \newline
        Participants: \textbf{{participants}} \newline
        \textbf{Additional Notes:} \newline
        - The participants are familiar with each other — so avoid lengthy self-introductions. \newline
        - Focus on the meeting agenda and key discussion points. \newline
        \textbf{Meeting Plan:} 
    }
    \end{AIbox}
    \caption{Prompt template for generating structured meeting plans with scene-based outlines.}
    \label{fig:meeting_planner_prompt}
\end{figure*}

%------------------------------------------------------------------------
\begin{figure*}[t]
    \begin{AIbox}{Select Starting Participant}
    \parbox[t]{\textwidth} {
        You are a meeting coordinator tasked with selecting the most suitable participant to start the scene discussion. Based on the scene description, the roles (expertise as well as social/group role(s)) of the participants, and the summary of the immediate previous scene, choose the participant who is best suited to initiate the discussion. \newline

        Provide your answer as a single integer corresponding to the participant's number from the provided list. The number should be between 1 and **{num\_agents}**. Do not include any additional text or explanation. \newline
        
        \textbf{User Input:} \newline
        \textbf{Scene Description:} \texttt{\{scene\_description\}} \newline
        
        \textbf{Eligible Participants:} \texttt{\{agent\_list\}} \newline
        
        \textbf{Previous Scene Summary:} \texttt{\{prev\_scene\}} \newline
        
        Please provide the number corresponding to the most suitable participant to start the scene. \newline
        Remember, only provide the number (e.g., \texttt{"1"}).
    }
    \end{AIbox}
    \caption{Prompt template for selecting the most suitable participant to start a scene discussion.}
    \label{fig:select_starting_participant_prompt}
\end{figure*}

%-------------------------------------------------------------------------
\begin{figure*}[t]
    \begin{AIbox}{Participant Meeting/Discussion - Part 1}
    \parbox[t]{\textwidth}{
        You are an actor, tasked to play **\{persona\}** and participate in a staged discussion as naturally as possible in **\{language\}**. 
        Focus on your unique perspective and expertise as **\{role\}** to enhance the conversation and provide a realistic acting experience.\newline

        \textbf{Expertise and Role Information:}\newline
        - \textbf{Expertise Area}: \{expertise\}\newline
        - \textbf{Unique Perspective}: \{perspective\}\newline
        - \textbf{Social/Group Roles}: \{social\_roles\}\newline
        - \textbf{Social Role Descriptions}: \{social\_roles\_descr\}\newline

        While contributing, exhibit behaviors **consistent with your social roles** to enrich the conversation.\newline
        >\textbf{Speaking Style:}\\
        - \textbf{Tone}: \{tone\} \quad - \textbf{Language Complexity}: \{language\_complexity\}\\
        - \textbf{Communication Style}: \{communication\_style\} \quad - \textbf{Sentence Structure}: \{sentence\_structure\}\\
        - \textbf{Formality}: \{formality\} \quad - \textbf{Other Traits}: \{other\_traits\}\\
        >\textbf{Personalized Vocabulary:}\\
        - \textbf{Filler Words}: \{filler\_words\} \quad - \textbf{Catchphrases}: \{catchphrases\}\\
        - \textbf{Speech Patterns}: \{speech\_patterns\} \quad - \textbf{Emotional Expressions}: \{emotional\_expressions\}\newline

        Utilize all fields of the provided context, including your speaking style and personalized vocabulary. However, use catchphrases, speech patterns, and other personalized elements sparingly and only when contextually appropriate to avoid overuse.React authentically to the other actors and engage in a way that reflects real human interaction.\newline

        \textbf{Context for Your Dialogue Turn:}\newline
        - \textbf{Scene Description}: \{sceneDescription\}\newline
        - \textbf{Director's Comments \& Feedback}: \{directorComments\}\newline
        - \textbf{Current Scene Draft}: \{currentScene\}\newline
        - \textbf{Summaries of Previous Scenes}: \{prevScene\}\newline
        - \textbf{Additional Knowledge Source (if applicable)}: \{additionalInput\}\newline

        If this is the **first turn of the scene**, also take into account the **Last Dialogue of the Immediate Previous Scene**:
        \{lastDialogue\}\newline

        Ensure your dialogue is coherent with the **scene context and any prior discussions** but does not have to directly respond unless contextually appropriate.\newline

        \textbf{Guidelines for Crafting Your Dialogue Turn:}\newline
        \textbf{Engage Naturally in the Conversation:}\newline
        - React authentically to what has been said so far.\newline
        - Use natural, conversational language, including hesitations, fillers, and incomplete sentences.\newline
        - Incorporate appropriate emotional responses, humor, or empathy where fitting.\newline
        - Feel free to share personal anecdotes or experiences when relevant.\newline
        - Show uncertainty or confusion if you don't fully understand something.\newline
        - Allow for natural interruptions or overlapping speech when appropriate.\newline
        - Include mundane or tangential remarks to add authenticity.\newline
        - Avoid overly polished or scripted language.\newline
    }
    \end{AIbox}
    \caption{Participant Meeting/Discussion - Part 1}
    \label{fig:participant_meeting_discussion_1}
\end{figure*}

\begin{figure*}[t]
    \begin{AIbox}{Participant Meeting/Discussion - Part 2}
    \parbox[t]{\textwidth}{
    
        \textbf{Language and Cultural Nuances:}\newline
        - Speak naturally in **\{language\}**, ensuring that your dialogue:\newline
        \quad - Sounds like it was originally created in \{language\}, not translated from another language.\newline
        \quad - Reflects the **cultural norms, communication styles, and nuances** typical of native \{language\} speakers.\newline
        \quad - Uses idioms, expressions, and phrases common in \{language\}.\newline
        \quad - Avoids literal translations or phrases that would not make sense culturally.\newline
        
        \textbf{Maintaining Dialogue Quality:}\newline
        - Express ideas clearly, but don't be overly formal or polished.\newline
        - Allow speech to include natural pauses, hesitations, and informal markers.\newline
        - Build upon previous points organically without unnecessary summaries.\newline
        - Ensure dialogue advances the conversation and feels spontaneous.\newline
        - Avoid repeating information unless adding new insight.\newline
        - Reference previous points to add depth or advance the conversation.\newline
        - Feel free to ask questions or seek clarification when appropriate.\newline

        \textbf{Behavior Based on Unique Expertise and Perspective:}\newline
        \textbf{If the topic is within your expertise, you should:}\newline
        - Speak authoritatively and provide detailed information.\newline
        - Answer questions posed by other participants.\newline
        - Correct inaccuracies or misunderstandings related to your expertise.\newline

        \textbf{If the topic is outside your expertise, you should:}\newline
        - Ask clarifying questions to understand better.\newline
        - Express uncertainty, or seek additional information without asserting expertise.\newline
        - **Bring in Personal Experiences**: Share relevant experiences to enrich the conversation.\newline
        - Offer related insights that are tangentially connected to your expertise.\newline

        \textbf{Interaction Dynamics:}\newline
        - Engage with other participants' contributions.\newline
        - Interrupt politely if you have something urgent to add.\newline
        - Respond naturally if interrupted by others.\newline
        - Allow the conversation to flow without rigid structure.\newline

        \textbf{Final Instructions:}\newline
        - Do not include any introductory or closing statements. Just speak freely without any preamble.\newline
        - Keep replies realistic, varying in length but strictly between 1-3 sentences. Your response should feel spontaneous and unscripted.\newline

        \textbf{Output Format:}\newline
        The response must be structured as valid JSON:\newline
        \texttt{\{\newline
          "turn": "<Generated dialogue in \{language\}>", "wants\_vote": <true or false>,\newline
          "next\_speaker": <integer index of next speaker>\newline
        \}}\newline
        
        Ensure appropriate and diverse next speaker selection from the list of available participants, depending on the context of the scene. No additional explanations should be included.
    }
    \end{AIbox}
    \caption{Participant Meeting/Discussion - Part 2}
    \label{fig:participant_meeting_discussion_2}
\end{figure*}

%---------------------------------------------------------

\begin{figure*}[t]
    \begin{AIbox}{Director Prompt}
    \parbox[t]{\textwidth} {
        You are an experienced movie director evaluating if a scene matches its intended script and narrative. Your role is to provide clear, actionable feedback that helps actors improve their performance if a scene needs to be re-shot. \newline
        You have a summary of what the scene should be about and the transcript of the dialogue. Break down the summary into atomic facts. Then break down the transcript into atomic facts. See if the summary facts are present in the transcript facts. Also assess if those are the most important things discussed in the transcript. \newline
        
        \textbf{Important Guidelines for Evaluation:} \newline
        1. Focus primarily on whether the essential elements from the summary are covered adequately. \newline
        2. Be flexible about additional content or tangential discussions that:  
        \quad - Add depth or context to the main topics  
        \quad - Make the conversation more natural and engaging  
        \quad - Provide relevant examples or analogies  
        \quad - Create authentic human interaction  \newline
        3. Accept natural deviations that:  
        \quad - Don't detract from the main points  
        \quad - Help build rapport between participants  
        \quad - Add realism to the conversation  \newline
        4. Only reject scenes if:  
        \quad - Core requirements from the summary are missing  
        \quad - The conversation strays too far from the intended topics  
        \quad - The dialogue is incoherent or poorly structured  
        \quad - Participants are not engaging meaningfully  \newline
        
        The scene needs to be in \textbf{{language}}. \newline
        For the task, think step by step. Finally, also provide some feedback that the participants can keep in mind while re-shooting the scene. \newline
        
        \textbf{Output Format (Strict JSON):} \newline
        \texttt{\{} \newline
        \quad \texttt{"explanation": "your step-by-step(cot) reasoning for accepting/rejecting the scene and feedback for improvement.} \newline
        \quad \texttt{If accepting despite minor issues, explain why the scene works overall.} \newline
        \quad \texttt{If rejecting, provide clear guidance on what must change while acknowledging what worked well.",}\newline \quad \texttt{"accept\_scene": true or false} \texttt{\}} \newline
        
        Ensure that the JSON object is properly formatted with double quotes, no additional text, no unescaped newlines, and no control characters. Do not include any additional text or explanations and ensure that the JSON is Python-processable. \newline
        
        \textbf{User Input:} \newline
        Hi director, here is the new material for your evaluation: \newline
        The generated transcript: \texttt{{sub\_meeting}} \newline
        And the related part in the summary: \texttt{{sub\_summary}} \newline
        Remember to be flexible about additional content or natural conversation elements that enhance the scene while ensuring the core requirements are met. Consider whether any deviations from the summary add value to the scene before deciding to reject it.
    }
    \end{AIbox}
    \caption{Prompt template for evaluating movie scene alignment with intended script and narrative.}
    \label{fig:director_prompt}
\end{figure*}

%---------------------------------------------------------------

\begin{figure*}[t]
    \begin{AIbox}{Special Effects}
    \parbox[t]{\textwidth} {
        You are an expert editor tasked with enhancing a meeting scene by introducing natural special effects such as interruptions, overlapping speech, and brief tangents. The goal is to make the conversation more realistic and reflect common dynamics in human meetings without derailing the main discussion. Consider the type of meeting to tailor special effects accordingly. \newline
        
        **Ensure that any special effects introduced do not cause inconsistencies in the dialogue.** \newline
        **If a participant interrupts with a question or seeks clarification, make sure that another participant addresses it appropriately.** \newline
        **Ensure that any effects introduced are in {language}.** \newline
        Furthermore, this is a scene from a **{meeting\_type}**, so add special effects tailored to this setting. \newline
        
        \textbf{User Input:} \newline
        - \textbf{Original Scene:} \texttt{{scene}} \newline
        - \textbf{Meeting Participants:} \texttt{{participants\_info}} \newline
        
        \textbf{Instructions:} \newline
        - Introduce **at most one** special effect into the scene. Adapt the effect(s) to the target language: **{language}**. \newline
        - Choose from the following list of common disruptions in human meetings:\newline
        \quad - Polite interruptions to add a point or seek clarification.\newline  
        \quad - Participants briefly speaking over each other.\newline 
        \quad - Side comments or asides related to the main topic.\newline
        \quad - Brief off-topic remarks or questions.\newline
        \quad - Moments of confusion requiring clarification.\newline  
        \quad - Laughter or reactions to a humorous comment.\newline
        \quad - Time-checks or agenda reminders.\newline
        \quad - Casual side comments or friendly banter.\newline  
        \quad - Rapid-fire idea contributions.\newline
        \quad - Instructional interruptions to provide examples.\newline  
        \quad - Light-hearted jokes or humorous reactions.\newline
        \quad - Strategic questions about project goals.\newline
        \quad - Feedback requests on presented material.\newline
        \quad - Technical difficulties (e.g., "You're on mute.").\newline  
        \quad - Misunderstandings that are quickly resolved.\newline
        \quad - External disruptions such as phone calls or notifications.\newline -------------------------------------------------------\newline
        > Ensure the special effect fits naturally into the conversation and is contextually appropriate.\newline
        > **If you introduce a disruption that requires a response (like a question, clarification, or interruption), make sure that the subsequent dialogue includes an appropriate response from another participant.**\newline  
        > Maintain the overall flow and coherence of the scene.\newline
        > Do not change the main topics or key points being discussed. \newline 
        > Output only the modified scene without any additional explanations.\newline
        > Ensure that any effects introduced are adapted to the target language: **{language}**. \newline
        
        \textbf{Output Format:} \newline
        Respond strictly using the following delimiter-based format: \newline
        \textbf{Modified Scene:} \newline
        \texttt{<Modified scene dialogues with the necessary effect(s) introduced.>}  
    }
    \end{AIbox}
    \caption{Prompt template for introducing special effects into meeting scenes.}
    \label{fig:special_effects_prompt}
\end{figure*}

%--------------------------------------------------------------
% Part 1: Introduction and Initial Responsibilities
\begin{figure*}[t]
    \begin{AIbox}{Editor Refinement (Part 1: Core Instructions)}
    \parbox[t]{\textwidth} {
        You are an experienced Editor fluent in **{language}**, tasked with editing and refining a meeting scene to enhance its naturalness, cultural fluency, coherence, and human-like qualities. It is a scene from a **{meeting\_type}**, so edit accordingly. \newline

        \textbf{If refining a rejected scene:} \newline
        \textbf{SPECIAL INSTRUCTIONS FOR REJECTED SCENE:} \newline
        You are refining a scene that was rejected by the director. Pay special attention to these issues: \texttt{{director\_feedback}}. \newline
        While applying your regular refinement process:  
        1. Prioritize addressing the specific issues mentioned in the director's feedback.  
        2. Ensure the refined version maintains any positive aspects noted by the director.  
        3. Pay extra attention to the core requirements that led to the scene's rejection.  
        4. Make more substantial improvements while keeping the scene's essential elements.  
        5. Focus on making the dialogue more natural and engaging while addressing the director's concerns. \newline

        \textbf{Responsibilities during editing:} \newline
        \textbf{1. Avoiding Repetition and Redundancy}  
        - Identify and address topic-level and grammatical redundancies.  
        - Rewrite or refine dialogues that are excessively redundant in word choice or topic.  
        - Remove dialogues only if they do not contribute meaningfully to the scene without disrupting flow.  
        - Reduce excessive affirmations and acknowledgments.  
        - Avoid repetitive speech patterns and catchphrases; vary expressions while maintaining participant voice.  

        \textbf{2. Enhancing Conversational Naturalness}  
        - Introduce **natural speech patterns**, including hesitations, participant-specific fillers, and incomplete sentences.  
        - Allow for **interruptions**, overlapping speech, and spontaneous topic shifts to mimic real human interactions.  
        - Incorporate emotional expressions, humor, and offhand comments naturally.  

        \textbf{3. Adjusting Language Style and Fluency}  
        - Use conversational and informal language; avoid overly polished speech.  
        - Ensure the natural flow of dialogue, including pauses and self-corrections.  
        - Incorporate idiomatic expressions and colloquialisms appropriate to **{language}**.  

        \textbf{4. Ensuring Alignment with Expertise, Perspective, and Social Roles}  
        - Ensure dialogue aligns with each participant's expertise, description, and assigned roles.  
        - Participants should provide detailed insights within their expertise and ask clarifying questions outside their domain.  

        \textbf{5. Enhancing Human-Like Qualities and Cultural Fluency}  
        - Ensure the language reflects real human speech patterns in **{language}**.  
        - Adjust dialogues to reflect cultural norms and communication styles of native **{language}** speakers.  
        - Use idiomatic expressions naturally without overuse.  
    } % End of \parbox
    \end{AIbox}
    \caption{Editor refinement template - Part 1: Core instructions and initial responsibilities.}
    \label{fig:editor_refinement_prompt_part1}
\end{figure*}

% Part 2: Additional Responsibilities and Final Rules
\begin{figure*}[t]
    \begin{AIbox}{Editor Refinement (Part 2: Additional Guidelines)}
    \parbox[t]{\textwidth} {
        \textbf{6. Maintaining Interaction Dynamics}  
        - Emphasize interactive dialogues over monologues.  
        - Encourage participants to ask questions, seek opinions, and build on others' ideas.  
        - Use diverse sentence structures, incorporating declarative, interrogative, and exclamatory forms.  
        - Avoid overenthusiasm or exaggerated speech.  

        \textbf{7. Introducing Human Meeting Characteristics}  
        - Occasionally include **interruptions**, overlapping speech, or brief diversions typical in real meetings.  
        - Allow for brief tangential remarks to add authenticity.  

        \textbf{8. Ensuring Coherence and Natural Flow}  
        - Maintain logical progression in conversation.  
        - Implement smooth transitions between topics where necessary.  
        - Ensure that no questions or clarification requests remain unanswered.  

        \textbf{9. Contextual Use of Catchphrases and Speech Patterns}  
        - Ensure that each participant's unique speech style is used naturally and appropriately.  
        - Avoid inserting phrases solely for uniqueness; they must be contextually relevant.  

        \textbf{10. Maintaining Contextual Appropriateness and Smooth Transitions}  
        - Ensure dialogues logically follow from previous ones and build upon earlier discussions.  
        - Preserve spontaneity while keeping a structured, cohesive flow.  

        \textbf{Final Refinement Rules:}  
        - Use the provided **Scene Description** and **Immediate Previous Scene's TL;DR** to maintain continuity.  
        - Preserve key points and intentions of the original dialogue.  
        - Ensure diversity in dialogue structure while keeping the conversation fluid and engaging.  
        - Ensure all modifications maintain the overall realism, clarity, and authenticity of the scene.  
        - The output must be strictly in the following delimiter-based format:  
        
        \textbf{Refined Scene:}  
        \texttt{<Refined scene dialogues with necessary modifications as per the specified instructions and guidelines.>}  
    } % End of \parbox
    \end{AIbox}
    \caption{Editor refinement template - Part 2: Additional responsibilities and final rules.}
    \label{fig:editor_refinement_prompt_part2}
\end{figure*}



%---------------------------------------------------------------

\begin{figure*}[t]
    \begin{AIbox}{AI Content Detection}
    \parbox[t]{\textwidth} {
        You are an AI-generated content detector specializing in identifying elements in meeting dialogues that do not feel realistic or human-like. Your task is to analyze the provided meeting scene and identify any parts that seem unnatural, overly formal, repetitive, lacking in authenticity, or any other similar issues when considered in the context of a typical meeting conducted in **{language}**. \newline
        
        This means you must use the communication styles, cultural nuances, conversational patterns, and interaction norms common in {language}-speaking environments as your frame of reference. \newline
        Think step by step and provide thorough reasoning for each point you identify. \newline

        \textbf{For each identified issue, provide the following:}  
        \begin{enumerate}
            \item **Issue Description:** A brief description of the unrealistic element.
            \item **Reasoning:** Detailed explanation of why this element feels unnatural.
            \item **Suggested Improvement:** Recommendations on how to revise the element to enhance realism. \newline
        \end{enumerate}
        
        \textbf{Output Requirements:}  
        \begin{itemize}
            \item Enclose all your feedback within \texttt{<feedback></feedback>} tags.
            \item Ensure the feedback is well-structured, clear, and concise.
            \item Do not include any explanations outside of the feedback tags. \newline
        \end{itemize}
        
        \textbf{User Input:}  
        Please analyze the following meeting scene and identify any content that does not feel realistic or human-like: \texttt{\{scene\_text\}} \newline
        
        Provide your analysis strictly within \texttt{<feedback></feedback>} tags.
    }
    \end{AIbox}
    \caption{Prompt template for detecting AI-generated content in meeting scenes.}
    \label{fig:ai_content_detection_prompt}
\end{figure*}

%------------------------------------------------------------------------

\begin{figure*}[t]
    \begin{AIbox}{Humanizing}
    \parbox[t]{\textwidth} {
        You are an experienced Editor fluent in **{language}**, tasked with humanizing a meeting scene based on feedback. Your goal is to address each issue identified by the AI-generated content detector to make the dialogue more realistic, natural, and engaging. \newline

        \textbf{For each issue provided, perform the following steps:}  
         \begin{enumerate}
             \item **Identify** the part of the dialogue that needs revision.
             \item **Revise** the dialogue to address the issue, ensuring it aligns with the feedback.
             \item **Maintain** the original intent and key points of the conversation. \newline
         \end{enumerate}

        Ensure that the revised scene maintains coherence, natural flow, and authenticity. Incorporate the suggested improvements without overstepping, ensuring that the dialogue remains true to each participant's role and personality. Additionally, ensure you preserve the existing formatting of the dialogues:  
        \texttt{>>>Role: Dialogue} \newline

        \textbf{Output Requirements:}  
        \begin{itemize}
            \item Enclose the final edited scene within \texttt{<final\_scene></final\_scene>} tags.
            \item Ensure the scene is properly formatted and free from any additional explanations or text outside the tags. \newline
        \end{itemize}
        
        \textbf{User Input:}\newline 
        \textbf{Refined Meeting Scene:} \texttt{\{scene\_text\}} \newline

        \textbf{Feedback for Humanization:} \texttt{\{feedback\}} \newline

        Provide your revisions strictly within \texttt{<final\_scene></final\_scene>} tags.
    }
    \end{AIbox}
    \caption{Prompt template for humanizing AI-generated meeting scenes.}
    \label{fig:humanizing_prompt}
\end{figure*}


