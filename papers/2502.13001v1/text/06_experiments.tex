\label{sec:experiments}

\begin{table*}[t]
    \centering
    \renewcommand{\arraystretch}{1.2} % Increase vertical spacing by 20%
    \scriptsize
    \setlength{\tabcolsep}{2.5pt} % Reduced inter-column spacing

    % -- New highlight color for best values --
    \definecolor{highlightGreen}{HTML}{D4F4E9}

    \begin{tabular}{@{}l@{\hspace{2pt}}cccc c cccc c cccc@{}}
        \toprule
        
            & \multicolumn{4}{c}{\textbf{QMSum}} 
            & \multicolumn{1}{c}{}
            & \multicolumn{4}{c}{\textbf{\dataset{} [EN]}}
            & \multicolumn{1}{c}{}
            & \multicolumn{4}{c}{\textbf{\dataset{} [GER]}} \\
        \cmidrule(lr){2-5} \cmidrule(lr){7-10} \cmidrule(lr){12-15}
            & GPT & Gemini & Deepseek & Llama 
            &  & GPT & Gemini & Deepseek & Llama 
            &  & GPT & Gemini & Deepseek & Llama \\
        \midrule
        \rowcolor{gray!20} 
        \multicolumn{15}{c}{\textbf{Meeting Summary Focused Evaluation Metric} (lower is better)} \\
        \midrule
        Coreference
            & \cellcolor{highlightGreen}$0_{\textit{1.22}}$ 
            & $3_{\textit{1.58}}$
            & \cellcolor{highlightGreen}$0_{\textit{1.42}}$
            & $1_{\textit{1.67}}$
            & 
            & \cellcolor{highlightGreen}$0_{\textit{1.45}}$
            & $3_{\textit{1.57}}$
            & $1_{\textit{1.54}}$
            & $2_{\textit{1.60}}$
            &
            & \cellcolor{highlightGreen}$0_{\textit{1.38}}$
            & $3_{\textit{1.57}}$
            & $1.5_{\textit{1.68}}$
            & $2_{\textit{1.52}}$ \\
        Hallucination     
            & $3_{\textit{1.22}}$
            & $4_{\textit{2.04}}$
            & \cellcolor{highlightGreen}$2_{\textit{1.88}}$
            & $3_{\textit{2.08}}$
            & 
            & $4_{\textit{0.98}}$
            & $4_{\textit{1.40}}$
            & $4_{\textit{1.81}}$
            & $4_{\textit{1.02}}$
            &
            & $4_{\textit{1.57}}$
            & $4_{\textit{1.65}}$
            & \cellcolor{highlightGreen}$3_{\textit{1.81}}$
            & $4_{\textit{1.61}}$ \\
        Incoherence     
            & $4_{\textit{1.50}}$
            & $4_{\textit{1.09}}$
            & $4_{\textit{1.85}}$
            & $4_{\textit{1.88}}$
            & 
            & $4_{\textit{0.94}}$
            & $4_{\textit{0.72}}$
            & $4_{\textit{0.94}}$
            & $4_{\textit{1.43}}$
            &
            & $4_{\textit{1.18}}$
            & $4_{\textit{1.39}}$
            & \cellcolor{highlightGreen}$3_{\textit{1.20}}$
            & \cellcolor{highlightGreen}$3_{\textit{1.57}}$ \\
        Irrelevance     
            & \cellcolor{highlightGreen}$2_{\textit{1.70}}$
            & $3_{\textit{1.32}}$
            & \cellcolor{highlightGreen}$2_{\textit{1.48}}$
            & \cellcolor{highlightGreen}$2_{\textit{1.52}}$
            & 
            & $3_{\textit{1.14}}$
            & $3_{\textit{1.07}}$
            & $3_{\textit{1.11}}$
            & \cellcolor{highlightGreen}$2_{\textit{1.42}}$
            &
            & $2_{\textit{1.16}}$
            & $2_{\textit{1.23}}$
            & $2_{\textit{1.29}}$
            & $2_{\textit{1.27}}$ \\
        Language     
            & \cellcolor{highlightGreen}$1_{\textit{1.30}}$
            & $2_{\textit{1.44}}$
            & $2_{\textit{1.44}}$
            & \cellcolor{highlightGreen}$1_{\textit{1.50}}$
            & 
            & $1_{\textit{1.17}}$
            & $1_{\textit{1.20}}$
            & $1_{\textit{1.22}}$
            & $1_{\textit{1.04}}$
            &
            & \cellcolor{highlightGreen}$0_{\textit{1.31}}$
            & \cellcolor{highlightGreen}$0_{\textit{0.97}}$
            & \cellcolor{highlightGreen}$0_{\textit{1.04}}$
            & $1_{\textit{1.31}}$ \\
        Omission     
            & $3_{\textit{0.40}}$
            & $3_{\textit{0.38}}$
            & $3_{\textit{0.36}}$
            & $3_{\textit{0.47}}$
            & 
            & $4_{\textit{0.16}}$
            & $4_{\textit{0.31}}$
            & $4_{\textit{0.48}}$
            & $4_{\textit{0.31}}$
            &
            & $4_{\textit{0.18}}$
            & $4_{\textit{0.00}}$
            & $4_{\textit{0.18}}$
            & $4_{\textit{0.18}}$ \\
        Repetition     
            & $4_{\textit{1.05}}$
            & \cellcolor{highlightGreen}$3_{\textit{0.98}}$
            & \cellcolor{highlightGreen}$3_{\textit{1.01}}$
            & $4_{\textit{1.19}}$
            & 
            & $4_{\textit{0.74}}$
            & $4_{\textit{0.44}}$
            & $4_{\textit{0.39}}$
            & $4_{\textit{0.51}}$
            &
            & $4_{\textit{0.52}}$
            & $4_{\textit{0.79}}$
            & $4_{\textit{0.34}}$
            & $4_{\textit{0.47}}$ \\
        Structure     
            & $4_{\textit{0.90}}$
            & \cellcolor{highlightGreen}$3_{\textit{1.70}}$
            & \cellcolor{highlightGreen}$3_{\textit{1.65}}$
            & $4_{\textit{1.69}}$
            & 
            & $3_{\textit{1.57}}$
            & $3_{\textit{1.53}}$
            & $3_{\textit{1.52}}$
            & $3_{\textit{1.36}}$
            &
            & $3_{\textit{1.66}}$
            & $3_{\textit{1.42}}$
            & $3_{\textit{1.50}}$
            & $3_{\textit{1.54}}$ \\
        \midrule
        \rowcolor{gray!20} 
        \multicolumn{15}{c}{\textbf{General Evaluation Metrics} (higher is better)} \\
        \midrule
        R-1
            & $37.73_{\textit{5.85}}$
            & $39.61_{\textit{7.21}}$
            & $32.37_{\textit{5.18}}$
            & \cellcolor{highlightGreen}$40.91_{\textit{5.21}}$
            & 
            & $39.68_{\textit{5.73}}$
            & $38.82_{\textit{5.79}}$
            & $33.18_{\textit{7.38}}$
            & \cellcolor{highlightGreen}$40.35_{\textit{5.46}}$
            &
            & $33.74_{\textit{4.01}}$
            & $32.03_{\textit{5.14}}$
            & \cellcolor{highlightGreen}$33.90_{\textit{5.37}}$
            & $32.54_{\textit{4.50}}$ \\
        R-2
            & $7.95_{\textit{4.18}}$
            & \cellcolor{highlightGreen}$11.10_{\textit{4.86}}$
            & $5.80_{\textit{2.94}}$
            & $9.79_{\textit{4.08}}$
            & 
            & $8.43_{\textit{3.45}}$
            & \cellcolor{highlightGreen}$8.96_{\textit{3.50}}$
            & $8.07_{\textit{3.33}}$
            & $8.62_{\textit{3.18}}$
            &
            & $7.45_{\textit{2.98}}$
            & $6.98_{\textit{3.72}}$
            & \cellcolor{highlightGreen}$7.93_{\textit{3.67}}$
            & $7.24_{\textit{3.17}}$ \\
        R-L
            & $21.39_{\textit{4.05}}$
            & \cellcolor{highlightGreen}$27.55_{\textit{6.36}}$
            & $18.46_{\textit{3.44}}$
            & $7.95_{\textit{4.18}}$
            & 
            & \cellcolor{highlightGreen}$29.98_{\textit{4.88}}$
            & $27.81_{\textit{4.18}}$
            & $23.16_{\textit{5.98}}$
            & $29.45_{\textit{3.67}}$
            &
            & \cellcolor{highlightGreen}$25.20_{\textit{3.68}}$
            & $23.59_{\textit{4.32}}$
            & $24.60_{\textit{3.29}}$
            & $24.10_{\textit{3.11}}$ \\
        BS (F1)
            & $61.61_{\textit{2.87}}$
            & $60.64_{\textit{3.66}}$
            & $59.44_{\textit{2.78}}$
            & \cellcolor{highlightGreen}$62.28_{\textit{2.76}}$
            & 
            & \cellcolor{highlightGreen}$63.80_{\textit{3.11}}$
            & $63.66_{\textit{2.49}}$
            & $63.16_{\textit{3.37}}$
            & $63.53_{\textit{2.72}}$
            &
            & $64.75_{\textit{1.20}}$
            & $64.58_{\textit{1.74}}$
            & \cellcolor{highlightGreen}$64.88_{\textit{1.77}}$
            & $64.61_{\textit{1.33}}$ \\
        \bottomrule
    \end{tabular}
    \caption{Combined Evaluation Results: Human Evaluation top and Automatic Evaluation bottom. 
    Values are Median$_{Std}$. MESA scores are 1--5 Likert ratings, 
    ROUGE (R-1/R-2/R-L) and BERTScore (BS) are 0--100.}
    \label{tab:summary_performance}
\end{table*}

We evaluate current LLMs on abstractive meeting summarization for real and synthetic transcripts, sampling 30 meetings from each \dataset{} (English, German) and QMSum (90 total).


\subsection{Experimental Setup}
\paragraph{Summarization approaches.} 
We benchmark two closed-weight models (GPT, Gemini) and two open-weight models (Llama, DeepSeek), excluding refinement-based methods \cite{KirsteinLG25a}, as these would produce self-refined GPT summaries.
We use a simple zero-shot prompt requesting an abstract summary of up to 250 tokens, reflecting standard practices in meeting summarization \cite{KirsteinLG25a}.
Full prompt details are provided in \Cref{sec:appendix_A}.


\paragraph{Evaluation metrics.}
We compare system outputs against reference summaries using the established ROUGE (R-1/R-2/R-L) \cite{Lin04} and BERTScore (rescaled F1) \cite{ZhangKWW20} metrics along MESA \cite{KirsteinLG25}, an LLM-based metric for the error types of meeting summarization (e.g., structure, irrelevance).
These metrics enable direct comparisons with prior work and provide detailed insights into model weaknesses.





\subsection{Results}
\label{subsec:results_summ}

\Cref{tab:summary_performance} contains the evaluation scores.

\paragraph{Reasoning boosts performance.}
Although Llama tops the ROUGE/BERTScore metrics on both QMSum and \dataset{}, MESA shows that DeepSeek consistently matches or improves over other models by minimizing common error types ($\sim$1 point lower per category), especially on QMSum.
DeepSeek also outperforms on the German \dataset{} subset, while Gemini generally trails behind.
All models perform comparably on the English subset, with the least language and coreference issues ($\sim$ 1/5).


\paragraph{\dataset{} reveals LLM struggles.} 
MESA scores for categories like incoherence, structure, and repetition remain similar to those on QMSum ($\sim$ 4/5) but with lower deviation for the \dataset{} subsets.
We conclude that the varied topics and meeting formats in \dataset{} add to the overall difficulty and negatively influence summary quality.


\paragraph{Contextualization deficits persist.}
The omission and irrelevance rows show that all models struggle with \dataset{}'s more difficult cross-turn information and information scarcity observed in \Cref{sec:challenges}.
Omission rises from 3/5 on QMSum to 4/5 on \dataset{}, and irrelevance from 2/5 to 3/5 for the English subset (German remains 2/5).
We derive that reliable content understanding \cite{KirsteinRKG24a,KirsteinLG25a} can be tested with our \dataset{} and that current LLMs struggle with this.
