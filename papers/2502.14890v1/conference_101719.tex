% \documentclass[conference]{IEEEtran}
% \IEEEoverridecommandlockouts
% % The preceding line is only needed to identify funding in the first footnote. If that is unneeded, please comment it out.
% \usepackage{cite}
% \usepackage{amsmath,amssymb,amsfonts}
% \usepackage{algorithmic}
% \usepackage{graphicx}
% \usepackage{textcomp}
% \usepackage[ruled,vlined]{algorithm2e}
% \usepackage{booktabs}
% \usepackage{multirow}
% \usepackage{array}
% \usepackage{caption}

% \usepackage{xcolor}
% \def\BibTeX{{\rm B\kern-.05em{\sc i\kern-.025em b}\kern-.08em
%     T\kern-.1667em\lower.7ex\hbox{E}\kern-.125emX}}
% \begin{document}


\documentclass[conference]{IEEEtran}
\IEEEoverridecommandlockouts
% The preceding line is only needed to identify funding in the first footnote. If that is unneeded, please comment it out.
\usepackage{cite}
\usepackage{amsmath,amssymb,amsfonts}
\usepackage{algorithmic}
\usepackage{graphicx}
\usepackage{textcomp}
\usepackage{xcolor}
\usepackage{multirow}


\def\BibTeX{{\rm B\kern-.05em{\sc i\kern-.025em b}\kern-.08em
    T\kern-.1667em\lower.7ex\hbox{E}\kern-.125emX}}
\begin{document}


\title{WeedVision: Multi-Stage Growth and Classification of Weeds using DETR and RetinaNet for Precision Agriculture\\
% {\footnotesize \textsuperscript{*}Note: Sub-titles are not captured in Xplore and
% should not be used}
}

\author{
\IEEEauthorblockN{Taminul Islam}
\IEEEauthorblockA{\textit{School of Computing} \\
\textit{Southern Illinois University}\\
Carbondale, USA\\
taminul.islam@siu.edu}
\and
\IEEEauthorblockN{Toqi Tahamid Sarker}
\IEEEauthorblockA{\textit{School of Computing} \\
\textit{Southern Illinois University}\\
Carbondale, USA\\
toqitahamid.sarker@siu.edu}
\and
\IEEEauthorblockN{Khaled R Ahmed}
\IEEEauthorblockA{\textit{School of Computing} \\
\textit{Southern Illinois University}\\
Carbondale, USA\\
khaled.ahmed@siu.edu}
\and
\IEEEauthorblockN{Cristiana Bernardi Rankrape}
\IEEEauthorblockA{\textit{Department of Plant Soils and Agricultural Systems} \\
\textit{Southern Illinois University}\\
Carbondale, USA\\
cris.rankrape@siu.edu}
\and
\IEEEauthorblockN{Karla Gage}
\IEEEauthorblockA{\textit{School of Agricultural Sciences / School of Biological Sciences} \\
\textit{Southern Illinois University}\\
Carbondale, USA\\
kgage@siu.edu}
}


\maketitle

\begin{abstract}
Weed management remains a critical challenge in agriculture, where weeds compete with crops for essential resources, leading to significant yield losses. Accurate detection of weeds at various growth stages is crucial for effective management yet challenging for farmers, as it requires identifying different species at multiple growth phases. This research addresses these challenges by utilizing advanced object detection models—specifically, the Detection Transformer (DETR) with a ResNet-50 backbone and RetinaNet with a ResNeXt-101 backbone—to identify and classify 16 weed species of economic concern across 174 classes, spanning their 11-week growth stages from seedling to maturity. A robust dataset comprising 203,567 images was developed, meticulously labeled by species and growth stage. The models were rigorously trained and evaluated, with RetinaNet demonstrating superior performance, achieving a mean Average Precision (mAP) of 0.907 on the training set and 0.904 on the test set, compared to DETR's mAP of 0.854 and 0.840, respectively. RetinaNet also outperformed DETR in recall and inference speed of 7.28 FPS, making it more suitable for real-time applications. Both models showed improved accuracy as plants matured. This research provides crucial insights for developing precise, sustainable, and automated weed management strategies, paving the way for real-time species-specific detection systems and advancing AI-assisted agriculture through continued innovation in model development and early detection accuracy.
\end{abstract}

\begin{IEEEkeywords}
Object Detection, Weed Management, DETR, Weed Growth Classification, Weed Detection
\end{IEEEkeywords}

\section{Introduction}
\label{section:introduction}

% redirection is unique and important in VR
Virtual Reality (VR) systems enable users to embody virtual avatars by mirroring their physical movements and aligning their perspective with virtual avatars' in real time. 
As the head-mounted displays (HMDs) block direct visual access to the physical world, users primarily rely on visual feedback from the virtual environment and integrate it with proprioceptive cues to control the avatar’s movements and interact within the VR space.
Since human perception is heavily influenced by visual input~\cite{gibson1933adaptation}, 
VR systems have the unique capability to control users' perception of the virtual environment and avatars by manipulating the visual information presented to them.
Leveraging this, various redirection techniques have been proposed to enable novel VR interactions, 
such as redirecting users' walking paths~\cite{razzaque2005redirected, suma2012impossible, steinicke2009estimation},
modifying reaching movements~\cite{gonzalez2022model, azmandian2016haptic, cheng2017sparse, feick2021visuo},
and conveying haptic information through visual feedback to create pseudo-haptic effects~\cite{samad2019pseudo, dominjon2005influence, lecuyer2009simulating}.
Such redirection techniques enable these interactions by manipulating the alignment between users' physical movements and their virtual avatar's actions.

% % what is hand/arm redirection, motivation of study arm-offset
% \change{\yj{i don't understand the purpose of this paragraph}
% These illusion-based techniques provide users with unique experiences in virtual environments that differ from the physical world yet maintain an immersive experience. 
% A key example is hand redirection, which shifts the virtual hand’s position away from the real hand as the user moves to enhance ergonomics during interaction~\cite{feuchtner2018ownershift, wentzel2020improving} and improve interaction performance~\cite{montano2017erg, poupyrev1996go}. 
% To increase the realism of virtual movements and strengthen the user’s sense of embodiment, hand redirection techniques often incorporate a complete virtual arm or full body alongside the redirected virtual hand, using inverse kinematics~\cite{hartfill2021analysis, ponton2024stretch} or adjustments to the virtual arm's movement as well~\cite{li2022modeling, feick2024impact}.
% }

% noticeability, motivation of predicting a probability, not a classification
However, these redirection techniques are most effective when the manipulation remains undetected~\cite{gonzalez2017model, li2022modeling}. 
If the redirection becomes too large, the user may not mitigate the conflict between the visual sensory input (redirected virtual movement) and their proprioception (actual physical movement), potentially leading to a loss of embodiment with the virtual avatar and making it difficult for the user to accurately control virtual movements to complete interaction tasks~\cite{li2022modeling, wentzel2020improving, feuchtner2018ownershift}. 
While proprioception is not absolute, users only have a general sense of their physical movements and the likelihood that they notice the redirection is probabilistic. 
This probability of detecting the redirection is referred to as \textbf{noticeability}~\cite{li2022modeling, zenner2024beyond, zenner2023detectability} and is typically estimated based on the frequency with which users detect the manipulation across multiple trials.

% version B
% Prior research has explored factors influencing the noticeability of redirected motion, including the redirection's magnitude~\cite{wentzel2020improving, poupyrev1996go}, direction~\cite{li2022modeling, feuchtner2018ownershift}, and the visual characteristics of the virtual avatar~\cite{ogawa2020effect, feick2024impact}.
% While these factors focus on the avatars, the surrounding virtual environment can also influence the users' behavior and in turn affect the noticeability of redirection.
% One such prominent external influence is through the visual channel - the users' visual attention is constantly distracted by complex visual effects and events in practical VR scenarios.
% Although some prior studies have explored how to leverage user blindness caused by visual distractions to redirect users' virtual hand~\cite{zenner2023detectability}, there remains a gap in understanding how to quantify the noticeability of redirection under visual distractions.

% visual stimuli and gaze behavior
Prior research has explored factors influencing the noticeability of redirected motion, including the redirection's magnitude~\cite{wentzel2020improving, poupyrev1996go}, direction~\cite{li2022modeling, feuchtner2018ownershift}, and the visual characteristics of the virtual avatar~\cite{ogawa2020effect, feick2024impact}.
While these factors focus on the avatars, the surrounding virtual environment can also influence the users' behavior and in turn affect the noticeability of redirection.
This, however, remains underexplored.
One such prominent external influence is through the visual channel - the users' visual attention is constantly distracted by complex visual effects and events in practical VR scenarios.
We thus want to investigate how \textbf{visual stimuli in the virtual environment} affect the noticeability of redirection.
With this, we hope to complement existing works that focus on avatars by incorporating environmental visual influences to enable more accurate control over the noticeability of redirected motions in practical VR scenarios.
% However, in realistic VR applications, the virtual environment often contains complex visual effects beyond the virtual avatar itself. 
% We argue that these visual effects can \textbf{distract users’ visual attention and thus affect the noticeability of redirection offsets}, while current research has yet taken into account.
% For instance, in a VR boxing scenario, a user’s visual attention is likely focused on their opponent rather than on their virtual body, leading to a lower noticeability of redirection offsets on their virtual movements. 
% Conversely, when reaching for an object in the center of their field of view, the user’s attention is more concentrated on the virtual hand’s movement and position to ensure successful interaction, resulting in a higher noticeability of offsets.

Since each visual event is a complex choreography of many underlying factors (type of visual effect, location, duration, etc.), it is extremely difficult to quantify or parameterize visual stimuli.
Furthermore, individuals respond differently to even the same visual events.
Prior neuroscience studies revealed that factors like age, gender, and personality can influence how quickly someone reacts to visual events~\cite{gillon2024responses, gale1997human}. 
Therefore, aiming to model visual stimuli in a way that is generalizable and applicable to different stimuli and users, we propose to use users' \textbf{gaze behavior} as an indicator of how they respond to visual stimuli.
In this paper, we used various gaze behaviors, including gaze location, saccades~\cite{krejtz2018eye}, fixations~\cite{perkhofer2019using}, and the Index of Pupil Activity (IPA)~\cite{duchowski2018index}.
These behaviors indicate both where users are looking and their cognitive activity, as looking at something does not necessarily mean they are attending to it.
Our goal is to investigate how these gaze behaviors stimulated by various visual stimuli relate to the noticeability of redirection.
With this, we contribute a model that allows designers and content creators to adjust the redirection in real-time responding to dynamic visual events in VR.

To achieve this, we conducted user studies to collect users' noticeability of redirection under various visual stimuli.
To simulate realistic VR scenarios, we adopted a dual-task design in which the participants performed redirected movements while monitoring the visual stimuli.
Specifically, participants' primary task was to report if they noticed an offset between the avatar's movement and their own, while their secondary task was to monitor and report the visual stimuli.
As realistic virtual environments often contain complex visual effects, we started with simple and controlled visual stimulus to manage the influencing factors.

% first user study, confirmation study
% collect data under no visual stimuli, different basic visual stimuli
We first conducted a confirmation study (N=16) to test whether applying visual stimuli (opacity-based) actually affects their noticeability of redirection. 
The results showed that participants were significantly less likely to detect the redirection when visual stimuli was presented $(F_{(1,15)}=5.90,~p=0.03)$.
Furthermore, by analyzing the collected gaze data, results revealed a correlation between the proposed gaze behaviors and the noticeability results $(r=-0.43)$, confirming that the gaze behaviors could be leveraged to compute the noticeability.

% data collection study
We then conducted a data collection study to obtain more accurate noticeability results through repeated measurements to better model the relationship between visual stimuli-triggered gaze behaviors and noticeability of redirection.
With the collected data, we analyzed various numerical features from the gaze behaviors to identify the most effective ones. 
We tested combinations of these features to determine the most effective one for predicting noticeability under visual stimuli.
Using the selected features, our regression model achieved a mean squared error (MSE) of 0.011 through leave-one-user-out cross-validation. 
Furthermore, we developed both a binary and a three-class classification model to categorize noticeability, which achieved an accuracy of 91.74\% and 85.62\%, respectively.

% evaluation study
To evaluate the generalizability of the regression model, we conducted an evaluation study (N=24) to test whether the model could accurately predict noticeability with new visual stimuli (color- and scale-based animations).
Specifically, we evaluated whether the model's predictions aligned with participants' responses under these unseen stimuli.
The results showed that our model accurately estimated the noticeability, achieving mean squared errors (MSE) of 0.014 and 0.012 for the color- and scale-based visual stimili, respectively, compared to participants' responses.
Since the tested visual stimuli data were not included in the training, the results suggested that the extracted gaze behavior features capture a generalizable pattern and can effectively indicate the corresponding impact on the noticeability of redirection.

% application
Based on our model, we implemented an adaptive redirection technique and demonstrated it through two applications: adaptive VR action game and opportunistic rendering.
We conducted a proof-of-concept user study (N=8) to compare our adaptive redirection technique with a static redirection, evaluating the usability and benefits of our adaptive redirection technique.
The results indicated that participants experienced less physical demand and stronger sense of embodiment and agency when using the adaptive redirection technique. 
These results demonstrated the effectiveness and usability of our model.

In summary, we make the following contributions.
% 
\begin{itemize}
    \item 
    We propose to use users' gaze behavior as a medium to quantify how visual stimuli influences the noticebility of redirection. 
    Through two user studies, we confirm that visual stimuli significantly influences noticeability and identify key gaze behavior features that are closely related to this impact.
    \item 
    We build a regression model that takes the user's gaze behavioral data as input, then computes the noticeability of redirection.
    Through an evaluation study, we verify that our model can estimate the noticeability with new participants under unseen visual stimuli.
    These findings suggest that the extracted gaze behavior features effectively capture the influence of visual stimuli on noticeability and can generalize across different users and visual stimuli.
    \item 
    We develop an adaptive redirection technique based on our regression model and implement two applications with it.
    With a proof-of-concept study, we demonstrate the effectiveness and potential usability of our regression model on real-world use cases.

\end{itemize}

% \delete{
% Virtual Reality (VR) allows the user to embody a virtual avatar by mirroring their physical movements through the avatar.
% As the user's visual access to the physical world is blocked in tasks involving motion control, they heavily rely on the visual representation of the avatar's motions to guide their proprioception.
% Similar to real-world experiences, the user is able to resolve conflicts between different sensory inputs (e.g., vision and motor control) through multisensory integration, which is essential for mitigating the sensory noise that commonly arises.
% However, it also enables unique manipulations in VR, as the system can intentionally modify the avatar's movements in relation to the user's motions to achieve specific functional outcomes,
% for example, 
% % the manipulations on the avatar's movements can 
% enabling novel interaction techniques of redirected walking~\cite{razzaque2005redirected}, redirected reaching~\cite{gonzalez2022model}, and pseudo haptics~\cite{samad2019pseudo}.
% With small adjustments to the avatar's movements, the user can maintain their sense of embodiment, due to their ability to resolve the perceptual differences.
% % However, a large mismatch between the user and avatar's movements can result in the user losing their sense of embodiment, due to an inability to resolve the perceptual differences.
% }

% \delete{
% However, multisensory integration can break when the manipulation is so intense that the user is aware of the existence of the motion offset and no longer maintains the sense of embodiment.
% Prior research studied the intensity threshold of the offset applied on the avatar's hand, beyond which the embodiment will break~\cite{li2022modeling}. 
% Studies also investigated the user's sensitivity to the offsets over time~\cite{kohm2022sensitivity}.
% Based on the findings, we argue that one crucial factor that affects to what extent the user notices the offset (i.e., \textit{noticeability}) that remains under-explored is whether the user directs their visual attention towards or away from the virtual avatar.
% Related work (e.g., Mise-unseen~\cite{marwecki2019mise}) has showcased applications where adjustments in the environment can be made in an unnoticeable manner when they happen in the area out of the user's visual field.
% We hypothesize that directing the user's visual attention away from the avatar's body, while still partially keeping the avatar within the user's field-of-view, can reduce the noticeability of the offset.
% Therefore, we conduct two user studies and implement a regression model to systematically investigate this effect.
% }

% \delete{
% In the first user study (N = 16), we test whether drawing the user's visual attention away from their body impacts the possibility of them noticing an offset that we apply to their arm motion in VR.
% We adopt a dual-task design to enable the alteration of the user's visual attention and a yes/no paradigm to measure the noticeability of motion offset. 
% The primary task for the user is to perform an arm motion and report when they perceive an offset between the avatar's virtual arm and their real arm.
% In the secondary task, we randomly render a visual animation of a ball turning from transparent to red and becoming transparent again and ask them to monitor and report when it appears.
% We control the strength of the visual stimuli by changing the duration and location of the animation.
% % By changing the time duration and location of the visual animation, we control the strengths of attraction to the users.
% As a result, we found significant differences in the noticeability of the offsets $(F_{(1,15)}=5.90,~p=0.03)$ between conditions with and without visual stimuli.
% Based on further analysis, we also identified the behavioral patterns of the user's gaze (including pupil dilation, fixations, and saccades) to be correlated with the noticeability results $(r=-0.43)$ and they may potentially serve as indicators of noticeability.
% }

% \delete{
% To further investigate how visual attention influences the noticeability, we conduct a data collection study (N = 12) and build a regression model based on the data.
% The regression model is able to calculate the noticeability of the offset applied on the user's arm under various visual stimuli based on their gaze behaviors.
% Our leave-one-out cross-validation results show that the proposed method was able to achieve a mean-squared error (MSE) of 0.012 in the probability regression task.
% }

% \delete{
% To verify the feasibility and extendability of the regression model, we conduct an evaluation study where we test new visual animations based on adjustments on scale and color and invite 24 new participants to attend the study.
% Results show that the proposed method can accurately estimate the noticeability with an MSE of 0.014 and 0.012 in the conditions of the color- and scale-based visual effects.
% Since these animations were not included in the dataset that the regression model was built on, the study demonstrates that the gaze behavioral features we extracted from the data capture a generalizable pattern of the user's visual attention and can indicate the corresponding impact on the noticeability of the offset.
% }

% \delete{
% Finally, we demonstrate applications that can benefit from the noticeability prediction model, including adaptive motion offsets and opportunistic rendering, considering the user's visual attention. 
% We conclude with discussions of our work's limitations and future research directions.
% }

% \delete{
% In summary, we make the following contributions.
% }
% % 
% \begin{itemize}
%     \item 
%     \delete{
%     We quantify the effects of the user's visual attention directed away by stimuli on their noticeability of an offset applied to the avatar's arm motion with respect to the user's physical arm. 
%     Through two user studies, we identified gaze behavioral features that are indicative of the changes in noticeability.
%     }
%     \item 
%     \delete{We build a regression model that takes the user's gaze behavioral data and the offset applied to the arm motion as input, then computes the probability of the user noticing the offset.
%     Through an evaluation study, we verified that the model needs no information about the source attracting the user's visual attention and can be generalizable in different scenarios.
%     }
%     \item 
%     \delete{We demonstrate two applications that potentially benefit from the regression model, including adaptive motion offsets and opportunistic rendering.
%     }

% \end{itemize}

\begin{comment}
However, users will lose the sense of embodiment to the virtual avatars if they notice the offset between the virtual and physical movements.
To address this, researchers have been exploring the noticing threshold of offsets with various magnitudes and proposing various redirection techniques that maintain the sense of embodiment~\cite{}.

However, when users embody virtual avatars to explore virtual environments, they encounter various visual effects and content that can attract their attention~\cite{}.
During this, the user may notice an offset when he observes the virtual movement carefully while ignoring it when the virtual contents attract his attention from the movements.
Therefore, static offset thresholds are not appropriate in dynamic scenarios.

Past research has proposed dynamic mapping techniques that adapted to users' state, such as hand moving speed~\cite{frees2007prism} or ergonomically comfortable poses~\cite{montano2017erg}, but not considering the influence of virtual content.
More specifically, PRISM~\cite{frees2007prism} proposed adjusting the C/D ratio with a non-linear mapping according to users' hand moving speed, but it might not be optimal for various virtual scenarios.
While Erg-O~\cite{montano2017erg} redirected users' virtual hands according to the virtual target's relative position to reduce physical fatigue, neglecting the change of virtual environments. 

Therefore, how to design redirection techniques in various scenarios with different visual attractions remains unknown.
To address this, we investigate how visual attention affects the noticing probability of movement offsets.
Based on our experiments, we implement a computational model that automatically computes the noticing probability of offsets under certain visual attractions.
VR application designers and developers can easily leverage our model to design redirection techniques maintaining the sense of embodiment adapt to the user's visual attention.
We implement a dynamic redirection technique with our model and demonstrate that it effectively reduces the target reaching time without reducing the sense of embodiment compared to static redirection techniques.

% Need to be refined
This paper offers the following contributions.
\begin{itemize}
    \item We investigate how visual attractions affect the noticing probability of redirection offsets.
    \item We construct a computational model to predict the noticing probability of an offset with a given visual background.
    \item We implement a dynamic redirection technique adapting to the visual background. We evaluate the technique and develop three applications to demonstrate the benefits. 
\end{itemize}



First, we conducted a controlled experiment to understand how users perceived the movement offset while subjected to various distractions.
Since hand redirection is one of the most frequently used redirections in VR interactions, we focused on the dynamic arm movements and manually added angular offsets to the' elbow joint~\cite{li2022modeling, gonzalez2022model, zenner2019estimating}. 
We employed flashing spheres in the user's field of view as distractions to attract users' visual attention.
Participants were instructed to report the appearing location of the spheres while simultaneously performing the arm movements and reporting if they perceived an offset during the movement. 
(\zhipeng{Add the results of data collection. Analyze the influence of the distance between the gaze map and the offset.}
We measured the visual attraction's magnitude with the gaze distribution on it.
Results showed that stronger distractions made it harder for users to notice the offset.)
\zhipeng{Need to rewrite. Not sure to use gaze distribution or a metric obtained from the visual content.}
Secondly, we constructed a computational model to predict the noticing probability of offsets with given visual content.
We analyzed the data from the user studies to measure the influence of visual attractions on the noticing probability of offsets.
We built a statistical model to predict the offset's noticing probability with a given visual content.
Based on the model, we implement a dynamic redirection technique to adjust the redirection offset adapted to the user's current field of view.
We evaluated the technique in a target selection task compared to no hand redirection and static hand redirection.
\zhipeng{Add the results of the evaluation.}
Results showed that the dynamic hand redirection technique significantly reduced the target selection time with similar accuracy and a comparable sense of embodiment.
Finally, we implemented three applications to demonstrate the potential benefits of the visual attention adapted dynamic redirection technique.
\end{comment}

% This one modifies arm length, not redirection
% \citeauthor{mcintosh2020iteratively} proposed an adaptation method to iteratively change the virtual avatar arm's length based on the primary tasks' performance~\cite{mcintosh2020iteratively}.



% \zhipeng{TO ADD: what is redirection}
% Redirection enables novel interactions in Virtual Reality, including redirected walking, haptic redirection, and pseudo haptics by introducing an offset to users' movement.
% \zhipeng{TO ADD: extend this sentence}
% The price of this is that users' immersiveness and embodiment in VR can be compromised when they notice the offset and perceive the virtual movement not as theirs~\cite{}.
% \zhipeng{TO ADD: extend this sentence, elaborate how the virtual environment attracts users' attention}
% Meanwhile, the visual content in the virtual environment is abundant and consistently captures users' attention, making it harder to notice the offset~\cite{}.
% While previous studies explored the noticing threshold of the offsets and optimized the redirection techniques to maintain the sense of embodiment~\cite{}, the influence of visual content on the probability of perceiving offsets remains unknown.  
% Therefore, we propose to investigate how users perceive the redirection offset when they are facing various visual attractions.


% We conducted a user study to understand how users notice the shift with visual attractions.
% We used a color-changing ball to attract the user's attention while instructing users to perform different poses with their arms and observe it meanwhile.
% \zhipeng{(Which one should be the primary task? Observe the ball should be the primary one, but if the primary task is too simple, users might allocate more attention on the secondary task and this makes the secondary task primary.)}
% \zhipeng{(We need a good and reasonable dual-task design in which users care about both their pose and the visual content, at least in the evaluation study. And we need to be able to control the visual content's magnitude and saliency maybe?)}
% We controlled the shift magnitude and direction, the user's pose, the ball's size, and the color range.
% We set the ball's color-changing interval as the independent factor.
% We collect the user's response to each shift and the color-changing times.
% Based on the collected data, we constructed a statistical model to describe the influence of visual attraction on the noticing probability.
% \zhipeng{(Are we actually controlling the attention allocation? How do we measure the attracting effect? We need uniform metrics, otherwise it is also hard for others to use our knowledge.)}
% \zhipeng{(Try to use eye gaze? The eye gaze distribution in the last five seconds to decide the attention allocation? Basically constructing a model with eye gaze distribution and noticing probability. But the user's head is moving, so the eye gaze distribution is not aligned well with the current view.)}

% \zhipeng{Saliency and EMD}
% \zhipeng{Gaze is more than just a point: Rethinking visual attention
% analysis using peripheral vision-based gaze mapping}

% Evaluation study(ideal case): based on the visual content, adjusting the redirection magnitude dynamically.

% \zhipeng{(The risk is our model's effect is trivial.)}

% Applications:
% Playing Lego while watching demo videos, we can accelerate the reaching process of bricks, and forbid the redirection during the manipulation.

% Beat saber again: but not make a lot of sense? Difficult game has complicated visual effects, while allows larger shift, but do not need large shift with high difficulty



\section{Related Work}
\label{lit_review}

\begin{highlight}
{

Our research builds upon {\em (i)} Assessing Web Accessibility, {\em (ii)} End-User Accessibility Repair, and {\em (iii)} Developer Tools for Accessibility.

\subsection{Assessing Web Accessibility}
From the earliest attempts to set standards and guidelines, web accessibility has been shaped by a complex interplay of technical challenges, legal imperatives, and educational campaigns. Over the past 25 years, stakeholders have sought to improve digital inclusion by establishing foundational standards~\cite{chisholm2001web, caldwell2008web}, enforcing legal obligations~\cite{sierkowski2002achieving, yesilada2012understanding}, and promoting a broader culture of accessibility awareness among developers~\cite{sloan2006contextual, martin2022landscape, pandey2023blending}. 
Despite these longstanding efforts, systemic accessibility issues persist. According to the 2024 WebAIM Million report~\cite{webaim2024}, 95.9\% of the top one million home pages contained detectable WCAG violations, averaging nearly 57 errors per page. 
These errors take many forms: low color contrast makes the interface difficult for individuals with color deficiency or low vision to read text; missing alternative text leaves users relying on screen readers without crucial visual context; and unlabeled form inputs or empty links and buttons hinder people who navigate with assistive technologies from completing basic tasks. 
Together, these accessibility issues not only limit user access to critical online resources such as healthcare, education, and employment but also result in significant legal risks and lost opportunities for businesses to engage diverse audiences. Addressing these pervasive issues requires systematic methods to identify, measure, and prioritize accessibility barriers, which is the first step toward achieving meaningful improvements.

Prior research has introduced methods blending automation and human evaluation to assess web accessibility. Hybrid approaches like SAMBA combine automated tools with expert reviews to measure the severity and impact of barriers, enhancing evaluation reliability~\cite{brajnik2007samba}. Quantitative metrics, such as Failure Rate and Unified Web Evaluation Methodology, support large-scale monitoring and comparative analysis, enabling cost-effective insights~\cite{vigo2007quantitative, martins2024large}. However, automated tools alone often detect less than half of WCAG violations and generate false positives, emphasizing the need for human interpretation~\cite{freire2008evaluation, vigo2013benchmarking}. Recent progress with large pretrained models like Large Language Models (LLMs)~\cite{dubey2024llama,bai2023qwen} and Large Multimodal Models (LMMs)~\cite{liu2024visual, bai2023qwenvl} offers a promising step forward, automating complex checks like non-text content evaluation and link purposes, achieving higher detection rates than traditional tools~\cite{lopez2024turning, delnevo2024interaction}. Yet, these large models face challenges, including dependence on training data, limited contextual judgment, and the inability to simulate real user experiences. These limitations underscore the necessity of combining models with human oversight for reliable, user-centered evaluations~\cite{brajnik2007samba, vigo2013benchmarking, delnevo2024interaction}. 

Our work builds on these prior efforts and recent advancements by leveraging the capabilities of large pretrained models while addressing their limitations through a developer-centric approach. CodeA11y integrates LLM-powered accessibility assessments, tailored accessibility-aware system prompts, and a dedicated accessibility checker directly into GitHub Copilot---one of the most widely used coding assistants. Unlike standalone evaluation tools, CodeA11y actively supports developers throughout the coding process by reinforcing accessibility best practices, prompting critical manual validations, and embedding accessibility considerations into existing workflows.
% This pervasive shortfall reflects the difficulty of scaling traditional approaches---such as manual audits and automated tools---that either demand immense human effort or lack the nuanced understanding needed to capture real-world user experiences. 
%
% In response, a new wave of AI-driven methods, many powered by large language models (LLMs), is emerging to bridge these accessibility detection and assessment gaps. Early explorations, such as those by Morillo et al.~\cite{morillo2020system}, introduced AI-assisted recommendations capable of automatic corrections, illustrating how computational intelligence can tackle the repetitive, common errors that plague large swaths of the web. Building on this foundation, Huang et al.~\cite{huang2024access} proposed ACCESS, a prompt-engineering framework that streamlines the identification and remediation of accessibility violations, while López-Gil et al.~\cite{lopez2024turning} demonstrated how LLMs can help apply WCAG success criteria more consistently---reducing the reliance on manual effort. Beyond these direct interventions, recent work has also begun integrating user experiences more seamlessly into the evaluation process. For example, Huq et al.~\cite{huq2024automated} translate user transcripts and corresponding issues into actionable test reports, ensuring that accessibility improvements align more closely with authentic user needs.
% However, as these AI-driven solutions evolve, researchers caution against uncritical adoption. Othman et al.~\cite{othman2023fostering} highlight that while LLMs can accelerate remediation, they may also introduce biases or encourage over-reliance on automated processes. Similarly, Delnevo et al.~\cite{delnevo2024interaction} emphasize the importance of contextual understanding and adaptability, pointing to the current limitations of LLM-based systems in serving the full spectrum of user needs. 
% In contrast to this backdrop, our work introduces and evaluates CodeA11y, an LLM-augmented extension for GitHub Copilot that not only mitigates these challenges by providing more consistent guidance and manual validation prompts, but also aligns AI-driven assistance with developers’ workflows, ultimately contributing toward more sustainable propulsion for building accessible web.

% Broader implications of inaccessibility—legal compliance, ethical concerns, and user experience
% A Historical Review of Web Accessibility Using WAVE
% "I tend to view ads almost like a pestilence": On the Accessibility Implications of Mobile Ads for Blind Users

% In the research domain, several methods have been developed to assess and enhance web accessibility. These include incorporating feedback into developer tools~\cite{adesigner, takagi2003accessibility, bigham2010accessibility} and automating the creation of accessibility tests and reports for user interfaces~\cite{swearngin2024towards, taeb2024axnav}. 

% Prior work has also studied accessibility scanners as another avenue of AI to improve web development practices~\cite{}.
% However, a persistent challenge is that developers need to be aware of these tools to utilize them effectively. With recent advancements in LLMs, developers might now build accessible websites with less effort using AI assistants. However, the impact of these assistants on the accessibility of their generated code remains unclear. This study aims to investigate these effects.

\subsection{End-user Accessibility Repair}
In addition to detecting accessibility errors and measuring web accessibility, significant research has focused on fixing these problems.
Since end-users are often the first to notice accessibility problems and have a strong incentive to address them, systems have been developed to help them report or fix these problems.

Collaborative, or social accessibility~\cite{takagi2009collaborative,sato2010social}, enabled these end-user contributions to be scaled through crowd-sourcing.
AccessMonkey~\cite{bigham2007accessmonkey} and Accessibility Commons~\cite{kawanaka2008accessibility} were two examples of repositories that store accessibility-related scripts and metadata, respectively.
Other work has developed browser extensions that leverage crowd-sourced databases to automatically correct reading order, alt-text, color contrast, and interaction-related issues~\cite{sato2009s,huang2015can}.

One drawback of collaborative accessibility approaches is that they cannot fix problems for an ``unseen'' web page on-demand, so many projects aim to automatically detect and improve interfaces without the need for an external source of fixes.
A large body of research has focused on making specific web media (e.g., images~\cite{gleason2019making,guinness2018caption, twitterally, gleason2020making, lee2021image}, design~\cite{potluri2019ai,li2019editing, peng2022diffscriber, peng2023slide}, and videos~\cite{pavel2020rescribe,peng2021say,peng2021slidecho,huh2023avscript}) accessible through a combination of machine learning (ML) and user-provided fixes.
Other work has focused on applying more general fixes across all websites.

Opportunity accessibility addressed a common accessibility problem of most websites: by default, content is often hard to see for people with visual impairments, and many users, especially older adults, do not know how to adjust or enable content zooming~\cite{bigham2014making}.
To this end, a browser script (\texttt{oppaccess.js}) was developed that automatically adjusted the browser's content zoom to maximally enlarge content without introducing adverse side-effects (\textit{e.g.,} content overlap).
While \texttt{oppaccess.js} primarily targeted zoom-related accessibility, recent work aimed to enable larger types of changes, by using LLMs to modify the source code of web pages based on user questions or directives~\cite{li2023using}.

Several efforts have been focused on improving access to desktop and mobile applications, which present additional challenges due to the unavailability of app source code (\textit{e.g.,} HTML).
Prefab is an approach that allows graphical UIs to be modified at runtime by detecting existing UI widgets, then replacing them~\cite{dixon2010prefab}.
Interaction Proxies used these runtime modification strategies to ``repair'' Android apps by replacing inaccessible widgets with improved alternatives~\cite{zhang2017interaction, zhang2018robust}.
The widget detection strategies used by these systems previously relied on a combination of heuristics and system metadata (\textit{e.g.,} the view hierarchy), which are incomplete or missing in the accessible apps.
To this end, ML has been employed to better localize~\cite{chen2020object} and repair UI elements~\cite{chen2020unblind,zhang2021screen,wu2023webui,peng2025dreamstruct}.

In general, end-user solutions to repairing application accessibility are limited due to the lack of underlying code and knowledge of the semantics of the intended content.

\subsection{Developer Tools for Accessibility}
Ultimately, the best solution for ensuring an accessible experience lies with front-end developers. Many efforts have focused on building adequate tooling and support to help developers with ensuring that their UI code complies with accessibility standards.

Numerous automated accessibility testing tools have been created to help developers identify accessibility issues in their code: i) static analysis tools, such as IBM Equal Access Accessibility Checker~\cite{ibm2024toolkit} or Microsoft Accessibility Insights~\cite{accessibilityinsights2024}, scan the UI code's compliance with predefined rules derived from accessibility guidelines; and ii) dynamic or runtime accessibility scanners, such as Chrome Devtools~\cite{chromedevtools2024} or axe-Core Accessibility Engine~\cite{deque2024axe}, perform real-time testing on user interfaces to detect interaction issues not identifiable from the code structure. While these tools greatly reduce the manual effort required for accessibility testing, they are often criticized for their limited coverage. Thus, experts often recommend manually testing with assistive technologies to uncover more complex interaction issues. Prior studies have created accessibility crawlers that either assist in developer testing~\cite{swearngin2024towards,taeb2024axnav} or simulate how assistive technologies interact with UIs~\cite{10.1145/3411764.3445455, 10.1145/3551349.3556905, 10.1145/3544548.3580679}.

Similar to end-user accessibility repair, research has focused on generating fixes to remediate accessibility issues in the UI source code. Initial attempts developed heuristic-based algorithms for fixing specific issues, for instance, by replacing text or background color attributes~\cite{10.1145/3611643.3616329}. More recent work has suggested that the code-understanding capabilities of LLMs allow them to suggest more targeted fixes.
For example, a study demonstrated that prompting ChatGPT to fix identified WCAG compliance issues in source code could automatically resolve a significant number of them~\cite{othman2023fostering}. Researchers have sought to leverage this capability by employing a multi-agent LLM architecture to automatically identify and localize issues in source code and suggest potential code fixes~\cite{mehralian2024automated}.

While the approaches mentioned above focus on assessing UI accessibility of already-authored code (\textit{i.e.,} fixing existing code), there is potential for more proactive approaches.
For example, LLMs are often used by developers to generate UI source code from natural language descriptions or tab completions~\cite{chen2021evaluating,GitHubCopilot,lozhkov2024starcoder,hui2024qwen2,roziere2023code,zheng2023codegeex}, but LLMs frequently produce inaccessible code by default~\cite{10.1145/3677846.3677854,mowar2024tab}, leading to inaccessible output when used by developers without sufficient awareness of accessibility knowledge.
The primary focus of this paper is to design a more accessibility-aware coding assistant that both produces more accessible code without manual intervention (\textit{e.g.,} specific user prompting) and gradually enables developers to implement and improve accessibility of automatically-generated code through IDE UI modifications (\textit{e.g.}, reminder notifications).

}
\end{highlight}



% Work related to this paper includes {\em (i)} Web Accessibility and {\em (ii)} Developer Practices in AI-Assisted Programming.

% \ipstart{Web Accessibility: Practice, Evaluation, and Improvements} Substantial efforts have been made to set accessibility standards~\cite{chisholm2001web, caldwell2008web}, establish legal requirements~\cite{sierkowski2002achieving, yesilada2012understanding}, and promote education and advocacy among developers~\cite{sloan2006contextual, martin2022landscape, pandey2023blending}. In the research domain, several methods have been developed to assess and enhance web accessibility. These include incorporating feedback into developer tools~\cite{adesigner, takagi2003accessibility, bigham2010accessibility} and automating the creation of accessibility tests and reports for user interfaces~\cite{swearngin2024towards, taeb2024axnav}. 
% % Prior work has also studied accessibility scanners as another avenue of AI to improve web development practices~\cite{}.
% However, a persistent challenge is that developers need to be aware of these tools to utilize them effectively. With recent advancements in LLMs, developers might now build accessible websites with less effort using AI assistants. However, the impact of these assistants on the accessibility of their generated code remains unclear. This study aims to investigate these effects.

% \ipstart{Developer Practices in AI-Assisted Programming}
% Recent usability research on AI-assisted development has examined the interaction strategies of developers while using AI coding assistants~\cite{barke2023grounded}.
% They observed developers interacted with these assistants in two modes -- 1) \textit{acceleration mode}: associated with shorter completions and 2) \textit{exploration mode}: associated with long completions.
% Liang {\em et al.} \cite{liang2024large} found that developers are driven to use AI assistants to reduce their keystrokes, finish tasks faster, and recall the syntax of programming languages. On the other hand, developers' reason for rejecting autocomplete suggestions was the need for more consideration of appropriate software requirements. This is because primary research on code generation models has mainly focused on functional correctness while often sidelining non-functional requirements such as latency, maintainability, and security~\cite{singhal2024nofuneval}. Consequently, there have been increasing concerns about the security implications of AI-generated code~\cite{sandoval2023lost}. Similarly, this study focuses on the effectiveness and uptake of code suggestions among developers in mitigating accessibility-related vulnerabilities. 


% ============================= additional rw ============================================
% - Paulina Morillo, Diego Chicaiza-Herrera, and Diego Vallejo-Huanga. 2020. System of Recommendation and Automatic Correction of Web Accessibility Using Artificial Intelligence. In Advances in Usability and User Experience, Tareq Ahram and Christianne Falcão (Eds.). Springer International Publishing, Cham, 479–489
% - Juan-Miguel López-Gil and Juanan Pereira. 2024. Turning manual web accessibility success criteria into automatic: an LLM-based approach. Universal Access in the Information Society (2024). https://doi.org/10.1007/s10209-024-01108-z
% - s
% - Calista Huang, Alyssa Ma, Suchir Vyasamudri, Eugenie Puype, Sayem Kamal, Juan Belza Garcia, Salar Cheema, and Michael Lutz. 2024. ACCESS: Prompt Engineering for Automated Web Accessibility Violation Corrections. arXiv:2401.16450 [cs.HC] https://arxiv.org/abs/2401.16450
% - Syed Fatiul Huq, Mahan Tafreshipour, Kate Kalcevich, and Sam Malek. 2025. Automated Generation of Accessibility Test Reports from Recorded User Transcripts. In Proceedings of the 47th International Conference on Software Engineering (ICSE) (Ottawa, Ontario, Canada). IEEE. https://ics.uci.edu/~seal/publications/2025_ICSE_reca11.pdf To appear in IEEE Xplore
% - Achraf Othman, Amira Dhouib, and Aljazi Nasser Al Jabor. 2023. Fostering websites accessibility: A case study on the use of the Large Language Models ChatGPT for automatic remediation. In Proceedings of the 16th International Conference on PErvasive Technologies Related to Assistive Environments (Corfu, Greece) (PETRA ’23). Association for Computing Machinery, New York, NY, USA, 707–713. https://doi.org/10.1145/3594806.3596542
% - Zsuzsanna B. Palmer and Sushil K. Oswal. 0. Constructing Websites with Generative AI Tools: The Accessibility of Their Workflows and Products for Users With Disabilities. Journal of Business and Technical Communication 0, 0 (0), 10506519241280644. https://doi.org/10.1177/10506519241280644
% ============================= additional rw ============================================
\section{Data Description and Preproceccing}

In this research, we conducted a study on 16 weed species at the SIU Horticulture Research Center greenhouse.
% These species were chosen based on their potential for economic damage and are listed in the top 6 most common or most troublesome weeds in broadleaf and grass crops in USA agriculture in a nationwide survey conducted by the Weed Science Society of America \cite{van2017weed}.
We began by preparing soil for seed planting, as shown in  Figure \ref{fig:fig1}(b). Potting soil (Pro-Mix ® BX) was placed into 32 square pots (10.7 cm x 10.7 cm x 9 cm), each labeled by species with white pot stakes. Two seeds from each species were planted per pot. Environmental conditions in the greenhouse, including temperature and lighting, were carefully controlled. Plants were watered as needed and fertilized with all-purpose 20-20-20 nutrient solution every 3 days. Figure \ref{fig:fig2_greenhouse} provides an overview of the greenhouse environment. We monitored the growth stages of each plant on a weekly basis, capturing images from the first week until week 11. Image capture ceased when the weeds entered their flowering stage, which marked the final growth phase in our study. We have captured our images by using an iPhone 15 Pro Max. Table \ref{species_frames} provides a comprehensive overview of our study, detailing the weed species codes, their corresponding scientific and common names, and the number of frames captured for each species on a weekly basis.

\begin{figure}[t]
    % \vspace{-0.2cm}
    \centering
    \includegraphics[width=0.5\textwidth]{figure/fig2_greenhouse.jpg}
    \vspace{-0.4cm}
    \caption{Greenhouse environment with lighting, temperature, and watering setup.}
    \label{fig:fig2_greenhouse}
    \vspace{-0.3cm}
\end{figure}


% ---------------
\begin{table*}[htbp]
\caption{Overview of Weed Species of Economic Concern, Corresponding Codes, and Weekly Frame Counts Captured for Each Species Across 11 Weeks in the Greenhouse}
\vspace{-0.4cm}
\label{species_frames}
\begin{center}
\resizebox{\textwidth}{!}{%
\begin{tabular}{|c|c|c|c|c|c|c|c|c|c|c|c|c|c|c|c|c|}
\hline
\multirow{2}{*}{\textbf{Species Code~\cite{kotleba1994european}}} & \multirow{2}{*}{\textbf{Scientific Name}~\cite{borsch2020world}} & \multirow{2}{*}{\textbf{Common Name}~\cite{wssaCompositeList}} & \multirow{2}{*}{\textbf{Family}} & \multirow{2}{*}{\textbf{Total Frames}} & \multicolumn{11}{|c|}{\textbf{Number of frames/week}} \\
\cline{6-16}
 &  &  &  &  & \textbf{\textit{W\_1}} & \textbf{\textit{W\_2}} & \textbf{\textit{W\_3}} & \textbf{\textit{W\_4}} & \textbf{\textit{W\_5}} & \textbf{\textit{W\_6}} & \textbf{\textit{W\_7}} & \textbf{\textit{W\_8}} & \textbf{\textit{W\_9}} & \textbf{\textit{W\_10}} & \textbf{\textit{W\_11}} \\
\hline
ABUTH & \textit{Abutilon theophrasti} Medik. & Velvetleaf & Malvaceae & 14754 & 1084 & 2451 & 1212 & 1819 & 1414 & 981 & 677 & 1164 & 1084 & 1500 & 1368 \\
AMAPA & \textit{Amaranthus palmeri} S. Watson. & Palmer Amaranth & Amaranthaceae & 17525 & 1441 & 1408 & 2110 & 2014 & 2441 & 1290 & 923 & 1478 & 1393 & 1667 & 1360 \\
AMARE & \textit{Amaranthus retroflexus} L. & Redroot Pigweed & Amaranthaceae & 15380 & 1017 & 1363 & 2110 & 1923 & 1884 & 1150 & 736 & 1237 & 1082 & 1596 & 1282 \\
AMATU & \textit{Amaranthus tuberculatus} (Moq.) Sauer. & Water Hemp & Amaranthaceae & 14852 & 1325 & 1459 & 1565 & 1664 & 1942 & 837 & 730 & 969 & 1638 & 1573 & 1150 \\
AMBEL & \textit{Ambrosia artemisiifolia} L. & Common Ragweed & Asteraceae & 17427 & 1022 & 2215 & 1846 & 1739 & 2162 & 1093 & 1066 & 1432 & 1092 & 2045 & 1715 \\
CHEAL & \textit{Chenopodium album} L. & Common Lambsquarter & Chenopodiaceae & 8015 & 1108 & 954 & 1416 & 661 & 1056 & 305 & 418 & 641 & 453 & 429 & 574 \\
CYPES & \textit{Cyperus esculentus} L. & Yellow Nutsedge & Cyperaceae & 14275 & 909 & 1512 & 1032 & 1499 & 2273 & 978 & 1224 & 1391 & 1182 & 1170 & 1105 \\
DIGSA & \textit{Digitaria sanguinalis} (L.) Scop. & Large Crabgrass & Poaceae & 16962 & 732 & 1312 & 2411 & 2596 & 1649 & 1335 & 1166 & 1261 & 1120 & 1628 & 1692 \\
ECHCG & \textit{Echinochloa crus-galli} (L.) P. Beauv. & Barnyard Grass & Poaceae & 16564 & 1349 & 2067 & 2029 & 1426 & 2221 & 1240 & 929 & 1280 & 1371 & 1332 & 1320 \\
ERICA & \textit{Erigeron canadensis} L. & Horse Weed & Asteraceae & 15134 & 930 & 2183 & 1691 & 1542 & 2715 & 1189 & 609 & 742 & 915 & 1217 & 1401 \\
PANDI & \textit{Panicum dichotomiflorum} Michx. & Full Panicum & Poaceae & 15182 & 1198 & 1400 & 2143 & 1296 & 1979 & 952 & 887 & 1350 & 1425 & 1034 & 1518 \\
SETFA & \textit{Setaria faberi} Herrm. & Gaint Foxtail & Poaceae & 14635 & 1614 & 1195 & 2083 & 1348 & 1944 & 1091 & 715 & 1466 & 843 & 1342 & 994 \\
SETPU & \textit{Setaria pumila} (Poir.) Roem. & Yellow Foxtail & Poaceae & 15211 & 887 & 1390 & 1732 & 1654 & 2040 & 1093 & 747 & 1361 & 1325 & 1348 & 1634 \\
SIDSP & \textit{Sida spinosa} L. & Princkly Sida & Malvaceae & 14452 & 1035 & 1782 & 1583 & 1259 & 2142 & 1373 & 804 & 1059 & 1186 & 1303 & 926 \\
SORHA & \textit{Sorghum halepense} (L.) Pers. & Johnson Grass & Poaceae & 10958 & 0 & 0 & 1444 & 1268 & 1395 & 945 & 749 & 1215 & 1328 & 1116 & 1498 \\
SORVU & \textit{Sorghum bicolor}  (L.) Moench. & Shatter Cane & Poaceae & 9573 & 945 & 1340 & 1959 & 832 & 1065 & 525 & 279 & 748 & 714 & 592 & 574 \\
\hline
\end{tabular}
}
\end{center}
% \vspace{-0.5cm}
\end{table*}


% --------------------





% --------------------

Among the 16 species of weeds studied, SORHA did not emerge in weeks 1 and 2. Consequently, the research encompassed a total of 174 classes. The full dataset initially comprised 2,494,476 frames. After a thorough review process to remove substandard images, 203,567 images were ultimately selected for training. Figure \ref{fig:fig3_plants} presents sample images of four weed species at different growth stages. For ABUTH, images from week 1 (a) and week 11 (b) are shown. Similarly, ERICA is represented by its week 1 (c) and week 11 (d) images. SETFA is depicted in its first week (e) and eleventh week (f) of growth. Lastly, CYPES is illustrated in its initial (g) and final (h) weeks of the study period. Notably, while several species produced flowers in their final growth stages, others did not, reflecting natural growth processes and photoperiod sensitivities.

\begin{figure}[t]
    % \vspace{-0.2cm}
    \centering
    \includegraphics[width=0.5\textwidth]{figure/fig3_plants.png}
    \vspace{-0.4cm}
    \caption{Growth stages example of four weed species. (a,b) ABUTH in week 1 and 11; (c,d) ERICA in week 1 and 11; (e,f) SETFA in week 1 and 11; (g,h) CYPES in week 1 and 11. Images show progression from seedling emergence to mature plants across different species.}
    \label{fig:fig3_plants}
    \vspace{-0.3cm}
\end{figure}


\subsection{Data Preprocessing and Augmentation}
%
Our preprocessing pipeline begins with image normalization, a fundamental step that standardizes the input data. Each image is meticulously scaled to a 0-1 range by dividing all pixel values by 255.0. This normalization process is crucial as it ensures consistency across the dataset and aligns with the input requirements of neural networks, facilitating more efficient and effective training \cite{huang2023normalization}. Following normalization, we perform a color space conversion, transforming the images from the standard RGB (Red, Green, Blue) color space to the HSV (Hue, Saturation, Value) color space. The HSV color space allows us to more precisely isolate plant areas from the background, enhancing the accuracy of subsequent processing steps.

The next step in our pipeline is green area detection. We employ carefully calibrated thresholds for the HSV channels to create a mask that highlights potential plant regions. Specifically, we use hue values ranging from 25/360 to 160/360, a minimum saturation value of 0.20. These thresholds have been empirically determined to effectively isolate green regions corresponding to plant matter while minimizing false positives from non-plant green objects. We apply morphological operations \cite{comer1999morphological} to refine the green mask and improve the continuity of detected plant areas. The refined green areas are then subjected to connected component analysis, which identifies and labels distinct regions within the image. This step is crucial for differentiating individual plants or plant clusters, allowing for more precise analysis and annotation. Fig \ref{fig:dataaug} shows the process of the data augmentation.

\begin{figure}[t]
    % \vspace{-0.2cm}
    \centering
    \includegraphics[width=0.5\textwidth]{figure/data_augmentation_updated.png}
    \vspace{-0.4cm}
    \caption{Data Augmentation process with original image, masked image, and bounding box, respectively, for ERICA (a,b,c) and AMAPA (d,e,f).}
    \label{fig:dataaug}
    \vspace{-0.3cm}
\end{figure}

\subsection{Data Labeling}

Our labeling process creates comprehensive annotations for detected plants, including bounding box coordinates and detailed Pascal VOC XML annotations. We use Python libraries like Pillow \cite{clark2015pillow}, NumPy, and scikit-image for image processing. To ensure accuracy, we implemented a rigorous quality control process, manually refining annotations using LabelImg software \cite{labelImg2024} when necessary. Our labeling convention includes both species code and week number, enhancing the dataset's utility for tracking plant development and species-specific analysis. This meticulous approach results in a high-quality dataset with precise annotations and consistent formatting, suitable for various plant analysis tasks and growth stage tracking. 

Figure \ref{fig:boundingbox} illustrates this process, presenting a side-by-side comparison of an original image and its corresponding labeled version, which we refer to as the ground truth.


\begin{figure}[t]
    \centering
    \includegraphics[width=0.5\textwidth]{figure/fig4_boundingbox.png}
    \vspace{-0.4cm}
    \caption{Illustration of the labeling process for weed detection. The original image (a) shows the weed plant, followed by the selected leaf area (b), highlighted in blue, and the final image (c) with a bounding box and label (AMBEL\_week\_8).}
    \label{fig:boundingbox}
    \vspace{-0.3cm}
\end{figure}

\section{Methodology}

After annotating the dataset, we split the dataset into training, validation, and test sets. We used 184,719 images ($\sim$80\%) to train our object detection models and 23,090 images ($\sim$10\%) to validate the model during training time. The rest of the 23,090 images ($\sim$10\%) are held out to test the trained model’s performance. In this study, we employed two advanced deep-learning models for weed detection and classification: RetinaNet with a ResNeXt-101 backbone and Detection Transformer (DETR) with a ResNet-50 backbone. These models were tasked with classifying weed species and their respective growth stages (in weeks), while simultaneously localizing them within the images via bounding box predictions. We configured and trained these models using PyTorch and mmDetection on an NVIDIA RTX 3090 GPU.

\subsection{Detection Transformer with ResNet-50}

The Detection Transformer (DETR) model is an end-to-end object detection architecture that combines a convolutional backbone with a transformer encoder-decoder \cite{carion2020end}. This approach effectively addresses the complexities of identifying weeds in agricultural images. The backbone of our model ResNet-50 is a convolutional neural network, pre-trained on ImageNet (\texttt{open-mmlab://resnet50}). This 50-layer network, organized into four stages, serves as a powerful feature extractor. We utilize the output from the final stage (out\_indices=(3,)) and freeze the initial stages during training to preserve pre-learned features. The backbone's output can be represented as:
\vspace{-0.2cm}
\begin{equation}
F_{\text{resnet}} = \text{ResNet50}(I)
\end{equation}
where \(I\) is the input image. A Channel Mapper follows the backbone, transforming ResNet-50's 2048-channel output into a 256-channel feature map suitable for the transformer. This dimensionality reduction is achieved through a 1x1 convolution:
\vspace{-0.2cm}
\begin{equation}
F_{\text{neck}} = \text{Conv1x1}(F_{\text{resnet}})
\end{equation}

The core of our DETR model is the transformer module, comprising a 6-layer encoder and decoder. Each encoder layer incorporates a self-attention mechanism with 8 heads, followed by a feed-forward network (FFN) with ReLU activation. The model's bounding box head processes the decoder's output to predict class labels and bounding boxes. We employ cross-entropy loss for classification and a combination of L1 and Generalized IoU losses for bounding box regression. The overall loss function \cite{yin2019context} is defined as:
\vspace{-0.2cm}
\begin{equation}
L = \alpha \cdot L_{\text{cls}} + \beta \cdot L_{\text{bbox}} + \gamma \cdot L_{\text{iou}}
\end{equation}
where \(\alpha\), \(\beta\), and \(\gamma\) are weight coefficients. \( L_{\text{cls}} \) represents the classification loss, which in this case is the cross-entropy loss. \( L_{\text{bbox}} \) represents the bounding box regression loss, which is a combination of L1 loss and Generalized IoU loss, and \( L_{\text{iou}} \) 	represents the IoU loss, which is specifically aimed at improving the localization accuracy by penalizing the model based on the intersection over union between the predicted and ground truth bounding boxes. During training, we utilize the Hungarian algorithm \cite{ye2020cost} for bipartite matching, ensuring a one-to-one correspondence between predicted and ground-truth boxes. This approach optimizes the model's ability to accurately locate and classify weeds within agricultural images. By integrating the robust feature extraction capabilities of ResNet-50 with the DETR architecture's powerful attention mechanisms, our model achieves good performance in weed detection with 174 classes.

\subsection{RetinaNet with ResNeXt-101}

RetinaNet is a single-stage object detection model designed to address the extreme foreground-background class imbalance encountered during training \cite{li2019light}. The architecture comprises three main components: a backbone network for feature extraction, a neck (FPN) for generating multi-scale feature maps, and a detection head for predicting bounding boxes and class probabilities. We utilized ResNeXt-101 as the backbone, a variant of the ResNet architecture that employs grouped convolutions for improved efficiency and performance. The ResNeXt-101 backbone consists of 101 layers organized into four stages, with 32 groups and a base width of 4 channels per group. We initialized the backbone with weights pretrained on ImageNet (\texttt{open-mmlab://resnext101\_32x4d}) to leverage transfer learning. Batch normalization is applied after each convolutional layer to stabilize the learning process.

The Feature Pyramid Network (FPN) enhances the backbone's feature maps by combining high-level semantic features with low-level detailed features, enabling the detection of objects at various scales. The FPN generates multiple feature maps of different resolutions, which are then fed into the detection head. The detection head of RetinaNet comprises two subnetworks: a classification subnetwork for predicting object presence probabilities and a regression subnetwork for predicting bounding box coordinates. Each subnetwork consists of four convolutional layers, followed by a final convolutional layer that produces the desired outputs. To handle class imbalance, we employed the focal loss function \cite{lin2017focal} for training the classification subnetwork:
\vspace{-0.2cm}
\begin{equation}
\text{FL}(p_t) = -\alpha_t (1 - p_t)^\gamma \log(p_t)
\end{equation}

where \(p_t\) is the predicted probability, \(\alpha_t\) is a balancing factor, and \(\gamma\) is the focusing parameter.

We trained our model using an epoch-based training loop with the AdamW optimizer (learning rate \(lr = 0.0001\), weight decay \(wd = 0.0001\)). The learning rate schedule incorporated a linear warmup over the first 1000 iterations. We trained for 12 epochs with a batch size of 16, employing automatic learning rate scaling to accommodate potential batch size changes.


% \begin{algorithm}
% \caption{Model Training Process}\label{alg:training}
% \KwData{Training dataset, validation dataset, initial model parameters}
% \KwResult{Trained model}
% Initialize model parameters $\theta$\;
% \For{epoch = 1 to max\_epochs}{
%     \For{each mini-batch $(X, y)$ in training dataset}{
%         Compute predictions $\hat{y} = f_\theta(X)$\;
%         Compute classification loss $L_{\text{cls}}$ and regression loss $L_{\text{reg}}$\;
%         Compute total loss $L = L_{\text{cls}} + L_{\text{reg}}$\;
%         Backpropagate to compute gradients $\nabla_\theta L$\;
%         Update parameters $\theta$ using AdamW optimizer\;
%     }
%     \If{epoch \% val\_interval == 0}{
%         Evaluate model on validation dataset\;
%         Save model checkpoint if performance improves\;
%     }
% }
% \end{algorithm}

% The algorithm \cite{ilyas2022datamodels} outlines a model training process where the goal is to optimize the model parameters, denoted by \(\theta\). Initially, the model parameters are set to their initial values. The training process runs for a specified number of epochs, iterating over mini-batches of the training dataset in each epoch. For each mini-batch, the model makes predictions \(\hat{y}\), and the classification and regression losses are computed. These losses are summed to obtain the total loss, which is then used to calculate the gradients through backpropagation. The model parameters are updated using the AdamW optimizer. At regular intervals, defined by \texttt{val\_interval}, the model's performance is evaluated on the validation dataset, and the model checkpoint is saved if there is an improvement in performance.


% \vspace{-0.4cm}

\subsection{Evaluation Metrics}

To assess the performance of our weed detection models, we employ a comprehensive set of metrics that capture both the accuracy and robustness of the detections. Our primary metrics are Average Precision (AP), Average Recall (AR), and Mean Average Precision (mAP) evaluated across various Intersection over Union (IoU) thresholds.

AP provides a single-value summary of the precision-recall curve, effectively balancing the trade-off between precision and recall. Precision (P) is defined as the ratio of true positive detections to the sum of true positive and false positive detections: $\text{} P = \frac{TP}{TP + FP}$ and Recall (R) is the ratio of true positive detections to the sum of true positive and false negative detections: $\text{} (R) = \frac{TP}{TP + FN}$.

In this research, a true positive is a detected bounding box that correctly identifies a weed species and has an IoU above a specified threshold (e.g., 0.50) with the ground truth bounding box. A false positive is a detection that either does not sufficiently overlap with any ground truth box or incorrectly identifies the weed species. A false negative occurs when a ground truth weed instance is not detected by the model. AP \cite{robertson2008new} is calculated by integrating the precision over the recall range and it can be defined as:
\vspace{-0.2cm}
\begin{equation}
\text{AP} = \int_0^1 P(R) \, dR
\end{equation}

AR \cite{zhu2004recall} measures the model's ability to detect all relevant objects. It is computed as the average of maximum recalls at specified IoU thresholds:
\vspace{-0.2cm}
\begin{equation}
\text{AR} = \frac{1}{N} \sum_{i=1}^N R_{\text{max}}(IoU_i)
\end{equation}

mAP is the mean of AP values across different classes and is a common metric for evaluating object detection models. It provides a balanced measure of precision and recall across various IoU thresholds. It can be defined as:
\vspace{-0.2cm}
\begin{equation}
\text{mAP} = \frac{1}{C} \sum_{c=1}^C AP_c
\end{equation}

where $AP_c$ is the Average Precision for class $c$, and $C$ is the total number of classes.
We evaluate these metrics at various IoU thresholds. This multi-faceted evaluation approach allows us to comprehensively analyze our models' capabilities in detecting and classifying weeds across various scenarios, providing insights into their precision, recall, and overall detection performance.

\section{Experimental Evaluation}

The evaluation encompasses both training and test datasets, with a detailed analysis across 16 weed species. We employ various metrics, including AP, AR at different IoU thresholds and detection limits, as well as mAP and mean average recall (mAR). Additionally, we compare the inference speed of both models to provide a holistic view of their capabilities.


% \begin{table*}[t]
% \caption{Performance Comparison of DETR and RetinaNet on Training and Test Sets}
% \label{performance_comparison}
% \begin{center}
% \begin{tabular}{|l|cc|cc|}
% \hline
% \multirow{2}{*}{\textbf{Metrics}} & \multicolumn{2}{c|}{\textbf{DETR}} & \multicolumn{2}{c|}{\textbf{RetinaNet}} \\
% \cline{2-5}
% & \textbf{Train} & \textbf{Test} & \textbf{Train} & \textbf{Test} \\
% \hline
% AP@[IoU=0.50:0.95|maxDets=100] & 0.656 & 0.675 & 0.736 & \textbf{0.744} \\
% AP@[IoU=0.50|maxDets=1000] & 0.892 & 0.874 & \textbf{0.934} & 0.918 \\
% AP@[IoU=0.75|maxDets=1000] & 0.727 & 0.709 & 0.806 & \textbf{0.807} \\
% AR@[IoU=0.50:0.95|maxDets=100,300,1000] & 0.862 & \textbf{0.877} & 0.804 & 0.805 \\
% Mean Average Precision (mAP) & 0.854 & 0.840 & \textbf{0.907} & 0.904 \\
% Mean Average Recall (mAR) & 0.941 & 0.936 & \textbf{0.997} & 0.989 \\
% \hline
% Inference Speed (FPS) & \multicolumn{2}{c|}{3.49} & \multicolumn{2}{c|}{\textbf{7.28}} \\
% \hline
% \end{tabular}
% \end{center}
% \end{table*}

\begin{table*}[t]
\vspace{-0.4cm}
\caption{Performance Comparison of DETR and RetinaNet on Training and Test Sets}
\label{performance_comparison}
\begin{center}
\begin{tabular}{|l|c|c|c|c|c|}
\hline
\multirow{2}{*}{\textbf{Model}} & \multicolumn{2}{c|}{\textbf{mAP}} & \multicolumn{2}{c|}{\textbf{mAR}} & \multirow{2}{*}{\textbf{FPS}} \\
\cline{2-5}
& \textbf{\textit{Train}} & \textbf{\textit{Test}} & \textbf{\textit{Train}} & \textbf{\textit{Test}} & \\
\hline
DETR & 0.854 & 0.840 & 0.941 & 0.936 & 3.49 \\
RetinaNet & \textbf{0.907} & 0.904 & \textbf{0.997} & 0.989 & \textbf{7.28} \\
\hline
\end{tabular}
\end{center}
\end{table*}


Table \ref{performance_comparison} compares DETR and RetinaNet performance on training and test sets, highlighting key metrics. RetinaNet consistently outperforms DETR across all presented metrics. In terms of mean Average Precision (mAP), RetinaNet achieves superior scores of 0.907 and 0.904 on training and test sets respectively, compared to DETR's 0.854 and 0.840. This trend continues in mean Average Recall (mAR), where RetinaNet approaches near-perfect scores with 0.997 (training) and 0.989 (test), while DETR achieves 0.941 and 0.936. Notably, RetinaNet's inference speed is significantly faster, operating at 7.28 Frames Per Second (FPS), more than twice the speed of DETR's 3.49 FPS. This substantial difference in processing speed, combined with RetinaNet's superior accuracy metrics, suggests it may be the more efficient choice for real-time or high-volume weed detection tasks.


\begin{table*}[t]
\caption{\footnotesize Performance Comparison of DETR and RetinaNet across Weed Species}
\label{weed_species_performance}
\vspace{-0.6cm}
\begin{center}
\resizebox{\textwidth}{!}{%
\begin{tabular}{|l|c|c|c|c|c|c|c|c|}
\hline
\multirow{2}{*}{\textbf{Species Code}} & \multicolumn{4}{c|}{\textbf{DETR}} & \multicolumn{4}{c|}{\textbf{RetinaNet}} \\
\cline{2-9}
& \textbf{\textit{Average mAP}} & \textbf{\textit{Average mAP\_50}} & \textbf{\textit{Average mAP\_75}} & \textbf{\textit{Average Recall}} & \textbf{\textit{Average mAP}} & \textbf{\textit{Average mAP\_50}} & \textbf{\textit{Average mAP\_75}} & \textbf{\textit{Average Recall}} \\
\hline
ABUTH & 0.683 & 0.907 & 0.719 & 0.973 & 0.720 & 0.924 & 0.779 & 0.993 \\
AMAPA & 0.617 & 0.835 & 0.672 & 0.975 & 0.877 & 0.985 & 0.939 & 0.994 \\
AMARE & 0.575 & 0.807 & 0.598 & 0.957 & 0.617 & 0.941 & 0.684 & 0.987 \\
AMATA & 0.536 & 0.721 & 0.565 & 0.869 & 0.832 & 0.977 & 0.905 & 0.997 \\
AMBEL & 0.817 & 0.978 & 0.898 & 0.993 & 0.663 & 0.926 & 0.740 & 0.994 \\
CHEAL & 0.503 & 0.846 & 0.502 & 0.962 & 0.871 & 0.993 & 0.957 & 0.997 \\
CYPES & 0.643 & 0.861 & 0.680 & 0.986 & 0.781 & 0.971 & 0.853 & 0.995 \\
DIGSA & 0.578 & 0.864 & 0.594 & 0.995 & 0.664 & 0.878 & 0.753 & 0.976 \\
ECHCG & 0.655 & 0.899 & 0.715 & 0.986 & 0.566 & 0.814 & 0.612 & 0.950 \\
ERICA & 0.718 & 0.918 & 0.752 & 0.977 & 0.678 & 0.918 & 0.749 & 0.992 \\
PANDI & 0.670 & 0.929 & 0.723 & 0.979 & 0.724 & 0.934 & 0.799 & 0.993 \\
SETFA & 0.680 & 0.903 & 0.756 & 0.990 & 0.785 & 0.967 & 0.854 & 0.993 \\
SETPU & 0.597 & 0.852 & 0.652 & 0.973 & 0.794 & 0.949 & 0.858 & 0.993 \\
SIDSP & 0.771 & 0.980 & 0.826 & 0.993 & 0.739 & 0.954 & 0.832 & 0.991 \\
SORVU & 0.582 & 0.791 & 0.624 & 0.871 & 0.713 & 0.925 & 0.789 & 0.995 \\
SORHA & 0.527 & 0.715 & 0.544 & 0.892 & 0.693 & 0.858 & 0.780 & 0.894 \\
\hline
\end{tabular}
}
\end{center}
\vspace{-0.5cm}
\end{table*}


Table \ref{weed_species_performance} delves deeper, breaking down performance across all individual weed species. This table shows the average value of all 11 weeks results for 16 species. This view reveals nuances in each model's capabilities. RetinaNet demonstrates more consistent performance across species, with less variation in mAP scores. In contrast, DETR's performance fluctuates more widely, excelling with some species like AMBEL (mAP 0.817) and SIDSP (mAP 0.771), while struggling with others such as CHEAL (mAP 0.503) and SORHA (mAP 0.527). RetinaNet shines particularly bright with species like AMATA (mAP 0.832) and AMAPA (mAP 0.877), though it faces challenges with ECHCG (mAP 0.566).
Across all species, RetinaNet consistently achieves higher recall, often nearing or reaching 1.0, while DETR's recall, though generally high, shows more variability. Both models exhibit the expected decline in mAP as the IoU threshold increases from 0.5 to 0.75, but RetinaNet maintains higher scores more consistently throughout this range.

%----------------

We have selected four species for presenting their growth-wise experimental evaluation in this paper: Palmer amaranth (AMAPA), waterhemp (AMATA), giant foxtail (SETFA), and velvetleaf (ABUTH). These species are considered “driver weeds” or weeds that drive management decisions in USA agriculture due to their aggressive growth habits, herbicide resistance, and significant impact on crop yields \cite{illinois}. AMAPA and AMATA are particularly notorious for their rapid growth and resistance to multiple herbicide modes of action, making them difficult to control and highly competitive with crops.\\

\begin{table}
\caption{Performance Comparison of DETR and RetinaNet for SETFA}
\label{setfa_performance}
\vspace{-0.2cm}
\centering
\resizebox{\columnwidth}{!}{%
\begin{tabular}{|l|cccc|cccc|}
\hline
\multirow{2}{*}{\textbf{Class Name}} & \multicolumn{4}{c|}{\textbf{DETR}} & \multicolumn{4}{c|}{\textbf{RetinaNet}} \\
\cline{2-9}
& \textbf{\textit{mAP}} & \textbf{\textit{mAP\_50}} & \textbf{\textit{mAP\_75}} & \textbf{\textit{Recall}} & \textbf{\textit{mAP}} & \textbf{\textit{mAP\_50}} & \textbf{\textit{mAP\_75}} & \textbf{\textit{Recall}} \\
\hline
SETFA\_Week\_1 & 0.355 & 0.605 & 0.348 & 0.986 & 0.671 & 0.870 & 0.767 & 0.980 \\
SETFA\_Week\_2 & 0.400 & 0.763 & 0.414 & 1.000 & 0.623 & 0.801 & 0.738 & 0.989 \\
SETFA\_Week\_3 & 0.740 & 0.999 & 0.887 & 0.929 & 0.755 & 0.991 & 0.830 & 1.000 \\
SETFA\_Week\_4 & 0.607 & 0.860 & 0.717 & 0.974 & 0.555 & 0.764 & 0.611 & 1.000 \\
SETFA\_Week\_5 & 0.741 & 0.964 & 0.868 & 1.000 & 0.657 & 0.899 & 0.708 & 0.995 \\
SETFA\_Week\_6 & 0.658 & 0.859 & 0.669 & 1.000 & 0.648 & 0.936 & 0.682 & 1.000 \\
SETFA\_Week\_7 & 0.825 & 0.974 & 0.860 & 1.000 & 0.822 & 0.980 & 0.795 & 1.000 \\
SETFA\_Week\_8 & 0.743 & 1.000 & 0.818 & 1.000 & 0.822 & 1.000 & 0.943 & 1.000 \\
SETFA\_Week\_9 & 0.856 & 0.955 & 0.949 & 1.000 & 0.843 & 0.983 & 0.932 & 1.000 \\
SETFA\_Week\_10 & 0.696 & 0.956 & 0.802 & 0.986 & 0.643 & 0.956 & 0.738 & 0.986 \\
SETFA\_Week\_11 & 0.859 & 1.000 & 0.988 & 1.000 & 0.808 & 1.000 & 0.938 & 1.000 \\
\hline
\end{tabular}
}
\end{table}


\begin{table}
\caption{Performance Comparison of DETR and RetinaNet for AMAPA}
\label{amapa_performance}
\vspace{-0.2cm}
\centering
\resizebox{\columnwidth}{!}{%
\begin{tabular}{|l|cccc|cccc|}
\hline
\multirow{2}{*}{\textbf{Class Name}} & \multicolumn{4}{c|}{\textbf{DETR}} & \multicolumn{4}{c|}{\textbf{RetinaNet}} \\
\cline{2-9}
& \textbf{\textit{mAP}} & \textbf{\textit{mAP\_50}} & \textbf{\textit{mAP\_75}} & \textbf{\textit{Recall}} & \textbf{\textit{mAP}} & \textbf{\textit{mAP\_50}} & \textbf{\textit{mAP\_75}} & \textbf{\textit{Recall}} \\
\hline
AMAPA\_Week\_1 & 0.096 & 0.345 & 0.035 & 1.000 & 0.481 & 0.729 & 0.585 & 0.949 \\
AMAPA\_Week\_2 & 0.277 & 0.518 & 0.263 & 1.000 & 0.771 & 0.974 & 0.808 & 1.000 \\
AMAPA\_Week\_3 & 0.354 & 0.718 & 0.354 & 0.925 & 0.636 & 0.933 & 0.657 & 0.995 \\
AMAPA\_Week\_4 & 0.505 & 0.860 & 0.501 & 0.837 & 0.860 & 1.000 & 0.988 & 1.000 \\
AMAPA\_Week\_5 & 0.576 & 0.855 & 0.670 & 0.983 & 0.711 & 0.887 & 0.735 & 1.000 \\
AMAPA\_Week\_6 & 0.839 & 1.000 & 0.930 & 0.991 & 0.860 & 0.986 & 0.917 & 1.000 \\
AMAPA\_Week\_7 & 0.809 & 0.982 & 0.912 & 0.996 & 0.896 & 0.980 & 0.974 & 0.989 \\
AMAPA\_Week\_8 & 0.766 & 0.985 & 0.882 & 1.000 & 0.835 & 1.000 & 0.955 & 1.000 \\
AMAPA\_Week\_9 & 0.796 & 0.934 & 0.865 & 1.000 & 0.836 & 0.945 & 0.865 & 0.994 \\
AMAPA\_Week\_10 & 0.852 & 0.986 & 0.981 & 1.000 & 0.846 & 1.000 & 0.962 & 1.000 \\
AMAPA\_Week\_11 & 0.912 & 1.000 & 1.000 & 1.000 & 0.902 & 1.000 & 1.000 & 1.000 \\
\hline
\end{tabular}
}
\end{table}


%---

\vspace{-0.2cm}

\begin{table}
\caption{Performance Comparison of DETR and RetinaNet for ABUTH}
\label{abuth_performance}
\vspace{-0.2cm}
\centering
\resizebox{\columnwidth}{!}{%
\begin{tabular}{|l|cccc|cccc|}
\hline
\multirow{2}{*}{\textbf{Class Name}} & \multicolumn{4}{c|}{\textbf{DETR}} & \multicolumn{4}{c|}{\textbf{RetinaNet}} \\
\cline{2-9}
& \textbf{\textit{mAP}} & \textbf{\textit{mAP\_50}} & \textbf{\textit{mAP\_75}} & \textbf{\textit{Recall}} & \textbf{\textit{mAP}} & \textbf{\textit{mAP\_50}} & \textbf{\textit{mAP\_75}} & \textbf{\textit{Recall}} \\
\hline
ABUTH\_Week\_1 & 0.418 & 0.723 & 0.471 & 0.994 & 0.605 & 0.899 & 0.689 & 1.000 \\
ABUTH\_Week\_2 & 0.576 & 0.988 & 0.530 & 1.000 & 0.829 & 0.990 & 0.952 & 1.000 \\
ABUTH\_Week\_3 & 0.356 & 0.697 & 0.346 & 1.000 & 0.790 & 0.996 & 0.899 & 1.000 \\
ABUTH\_Week\_4 & 0.408 & 0.771 & 0.396 & 0.996 & 0.725 & 0.973 & 0.844 & 0.995 \\
ABUTH\_Week\_5 & 0.445 & 0.923 & 0.377 & 0.871 & 0.730 & 0.974 & 0.789 & 1.000 \\
ABUTH\_Week\_6 & 0.850 & 1.000 & 1.000 & 0.886 & 0.924 & 0.970 & 0.970 & 0.972 \\
ABUTH\_Week\_7 & 0.885 & 0.932 & 0.931 & 0.993 & 0.966 & 1.000 & 1.000 & 1.000 \\
ABUTH\_Week\_8 & 0.856 & 1.000 & 0.982 & 1.000 & 0.911 & 1.000 & 1.000 & 1.000 \\
ABUTH\_Week\_9 & 0.912 & 0.977 & 0.949 & 0.975 & 0.876 & 0.978 & 0.920 & 1.000 \\
ABUTH\_Week\_10 & 0.880 & 0.967 & 0.923 & 1.000 & 0.868 & 0.971 & 0.893 & 1.000 \\
ABUTH\_Week\_11 & 0.924 & 1.000 & 1.000 & 0.989 & 0.924 & 1.000 & 1.000 & 1.000 \\
\hline
\end{tabular}
}
\end{table}


\begin{table}
\caption{Performance Comparison of DETR and RetinaNet for AMATA}
\label{amata_performance}
\vspace{-0.2cm}
\centering
\resizebox{\columnwidth}{!}{%
\begin{tabular}{|l|cccc|cccc|}
\hline
\multirow{2}{*}{\textbf{Class Name}} & \multicolumn{4}{c|}{\textbf{DETR}} & \multicolumn{4}{c|}{\textbf{RetinaNet}} \\
\cline{2-9}
& \textbf{\textit{mAP}} & \textbf{\textit{mAP\_50}} & \textbf{\textit{mAP\_75}} & \textbf{\textit{Recall}} & \textbf{\textit{mAP}} & \textbf{\textit{mAP\_50}} & \textbf{\textit{mAP\_75}} & \textbf{\textit{Recall}} \\
\hline
AMATA\_Week\_1 & 0.001 & 0.003 & 0.000 & 0.982 & 0.641 & 0.981 & 0.742 & 0.992 \\
AMATA\_Week\_2 & 0.004 & 0.021 & 0.000 & 1.000 & 0.529 & 0.923 & 0.525 & 0.966 \\
AMATA\_Week\_3 & 0.157 & 0.397 & 0.076 & 0.391 & 0.747 & 0.998 & 0.934 & 1.000 \\
AMATA\_Week\_4 & 0.484 & 0.910 & 0.486 & 0.397 & 0.763 & 0.985 & 0.838 & 1.000 \\
AMATA\_Week\_5 & 0.544 & 0.974 & 0.541 & 0.839 & 0.738 & 0.961 & 0.822 & 0.994 \\
AMATA\_Week\_6 & 0.763 & 0.960 & 0.878 & 0.970 & 0.923 & 0.994 & 0.972 & 1.000 \\
AMATA\_Week\_7 & 0.905 & 1.000 & 0.977 & 0.995 & 0.968 & 1.000 & 0.974 & 1.000 \\
AMATA\_Week\_8 & 0.756 & 0.913 & 0.808 & 1.000 & 0.889 & 0.979 & 0.954 & 0.990 \\
AMATA\_Week\_9 & 0.881 & 0.960 & 0.952 & 1.000 & 0.926 & 0.990 & 0.972 & 1.000 \\
AMATA\_Week\_10 & 0.520 & 0.797 & 0.529 & 0.989 & 0.625 & 0.927 & 0.670 & 1.000 \\
AMATA\_Week\_11 & 0.882 & 0.993 & 0.965 & 1.000 & 0.849 & 0.998 & 0.933 & 1.000 \\
\hline
\end{tabular}
}
\end{table}



% \begin{table}
% \caption{Performance Comparison of DETR and RetinaNet for AMBEL}
% \label{ambel_performance}
% \centering
% \resizebox{\columnwidth}{!}{%
% \begin{tabular}{|l|cccc|cccc|}
% \hline
% \multirow{2}{*}{\textbf{Class Name}} & \multicolumn{4}{c|}{\textbf{DETR}} & \multicolumn{4}{c|}{\textbf{RetinaNet}} \\
% \cline{2-9}
% & \textbf{\textit{mAP}} & \textbf{\textit{mAP\_50}} & \textbf{\textit{mAP\_75}} & \textbf{\textit{Recall}} & \textbf{\textit{mAP}} & \textbf{\textit{mAP\_50}} & \textbf{\textit{mAP\_75}} & \textbf{\textit{Recall}} \\
% \hline
% AMBEL\_Week\_1 & 0.676 & 0.970 & 0.788 & 1.000 & 0.842 & 0.981 & 0.946 & 0.990 \\
% AMBEL\_Week\_2 & 0.602 & 0.884 & 0.683 & 1.000 & 0.839 & 0.967 & 0.917 & 0.986 \\
% AMBEL\_Week\_3 & 0.839 & 0.991 & 0.888 & 1.000 & 0.886 & 0.991 & 0.932 & 1.000 \\
% AMBEL\_Week\_4 & 0.826 & 0.980 & 0.980 & 0.988 & 0.903 & 1.000 & 0.989 & 1.000 \\
% AMBEL\_Week\_5 & 0.790 & 1.000 & 0.942 & 1.000 & 0.927 & 1.000 & 0.989 & 1.000 \\
% AMBEL\_Week\_6 & 0.927 & 0.990 & 0.990 & 0.989 & 0.923 & 0.962 & 0.960 & 0.974 \\
% AMBEL\_Week\_7 & 0.910 & 1.000 & 0.977 & 1.000 & 0.944 & 1.000 & 0.980 & 1.000 \\
% AMBEL\_Week\_8 & 0.953 & 1.000 & 1.000 & 0.990 & 0.974 & 1.000 & 1.000 & 1.000 \\
% AMBEL\_Week\_9 & 0.947 & 0.990 & 0.983 & 1.000 & 0.932 & 0.990 & 0.958 & 1.000 \\
% AMBEL\_Week\_10 & 0.708 & 0.958 & 0.787 & 1.000 & 0.729 & 0.965 & 0.828 & 1.000 \\
% AMBEL\_Week\_11 & 0.813 & 0.995 & 0.858 & 0.994 & 0.746 & 0.984 & 0.835 & 0.994 \\
% \hline
% \end{tabular}
% }
% \end{table}




% \begin{table}
% \caption{Performance Comparison of DETR and RetinaNet for AMARE}
% \label{amare_performance}
% \centering
% \resizebox{\columnwidth}{!}{%
% \begin{tabular}{|l|cccc|cccc|}
% \hline
% \multirow{2}{*}{\textbf{Class Name}} & \multicolumn{4}{c|}{\textbf{DETR}} & \multicolumn{4}{c|}{\textbf{RetinaNet}} \\
% \cline{2-9}
% & \textbf{\textit{mAP}} & \textbf{\textit{mAP\_50}} & \textbf{\textit{mAP\_75}} & \textbf{\textit{Recall}} & \textbf{\textit{mAP}} & \textbf{\textit{mAP\_50}} & \textbf{\textit{mAP\_75}} & \textbf{\textit{Recall}} \\
% \hline
% AMARE\_Week\_1 & 0.260 & 0.791 & 0.040 & 1.000 & 0.715 & 0.996 & 0.844 & 1.000 \\
% AMARE\_Week\_2 & 0.102 & 0.272 & 0.019 & 0.977 & 0.208 & 0.341 & 0.215 & 0.796 \\
% AMARE\_Week\_3 & 0.428 & 0.746 & 0.445 & 1.000 & 0.711 & 0.869 & 0.801 & 1.000 \\
% AMARE\_Week\_4 & 0.344 & 0.649 & 0.335 & 0.739 & 0.582 & 0.755 & 0.645 & 0.979 \\
% AMARE\_Week\_5 & 0.564 & 0.843 & 0.671 & 1.000 & 0.701 & 0.906 & 0.777 & 1.000 \\
% AMARE\_Week\_6 & 0.770 & 0.987 & 0.928 & 0.953 & 0.806 & 0.963 & 0.858 & 0.984 \\
% AMARE\_Week\_7 & 0.678 & 0.865 & 0.731 & 0.979 & 0.734 & 0.920 & 0.753 & 0.988 \\
% AMARE\_Week\_8 & 0.719 & 0.920 & 0.776 & 0.992 & 0.706 & 0.962 & 0.709 & 1.000 \\
% AMARE\_Week\_9 & 0.808 & 0.945 & 0.859 & 0.965 & 0.761 & 0.950 & 0.759 & 0.991 \\
% AMARE\_Week\_10 & 0.850 & 0.974 & 0.915 & 1.000 & 0.858 & 0.998 & 0.947 & 1.000 \\
% AMARE\_Week\_11 & 0.804 & 0.890 & 0.862 & 1.000 & 0.831 & 1.000 & 0.948 & 1.000 \\
% \hline
% \end{tabular}
% }
% \end{table}


\begin{figure}[t]
    \centering
    \includegraphics[width=0.5\textwidth]{figure/result_prediction2.png}
    \vspace{-0.4cm}
    \caption{Comparison of object detection results for ABUTH and DIGSA using DETR and RetinaNet models. Row 1 displays predictions for ABUTH, and Row 2 displays predictions for DIGSA, with ground truth and model confidence scores indicated for each detection.}
    \label{fig:detectionresult}
    \vspace{-0.3cm}
\end{figure}


Tables \ref{setfa_performance}, \ref{amapa_performance}, \ref{abuth_performance}, and \ref{amata_performance} present comprehensive performance comparisons between DETR and RetinaNet across four weed species (SETFA, AMAPA, ABUTH, and AMATA) over 11 weeks. Both models demonstrated high performance across various metrics, including mAP, mAP\_50, mAP\_75, and Recall. RetinaNet generally outperformed DETR, showing more consistent and often higher scores across most species and weeks. For instance, RetinaNet achieved peak mAP scores of 0.843 for SETFA, 0.902 for AMAPA, 0.924 for ABUTH, and 0.968 for AMATA. DETR's highest mAP scores were comparable, reaching 0.859 for SETFA, 0.912 for AMAPA, 0.924 for ABUTH, and 0.905 for AMATA. Both models frequently achieved perfect scores of 1.000 in mAP\_50 and Recall metrics across various weeks and species, indicating excellent detection accuracy at lower IoU thresholds and high object detection rates.\\
However, both models exhibited some performance fluctuations, particularly in the early weeks. DETR often struggled more in the initial weeks, with notably low mAP scores such as 0.355 for SETFA in Week 1, 0.096 for AMAPA in Week 1, and 0.001 for AMATA in Week 1. RetinaNet generally showed more stability, with its lowest mAP scores being higher than DETR's in most cases. For example, RetinaNet's lowest mAP for SETFA was 0.555 in Week 4, for AMAPA it was 0.481 in Week 1, and for AMATA it was 0.529 in Week 2. These early-week challenges could be attributed to factors such as less robust feature extraction, difficulty in detecting small objects, or lower-quality images in the initial stages of plant growth. Despite these early challenges, both models demonstrated significant improvement over time, with peak performances often occurring in later weeks (Weeks 8-11). This trend suggests that as plants matured and image quality potentially improved, both DETR and RetinaNet were able to more accurately detect and classify the weed species.


Figure \ref{fig:detectionresult} shows the prediction result of DETR and RetinaNet model. The top row focuses on ABUTH, where the first image (a) shows the original plant without any annotations, followed by the (b) ground truth with a labeled bounding box indicating "ABUTH week 2." The subsequent images display predictions by (c) DETR and (d) RetinaNet models, each bounding box labeled with the species name, the corresponding week, and the model's confidence score, with RetinaNet showing a slightly higher score (98.6) compared to DETR (94.9). The bottom row repeats this structure for DIGSA, showing the (e) original image, (f) the ground truth ("DIGSA week 5"), and the predictions from (g) DETR and (h) RetinaNet. For DIGSA, the confidence scores are close, with DETR predicting 90.1 and RetinaNet predicting 93.9, both models accurately detecting the plant but with varying degrees of confidence.
\section{Conclusion}
This research marks a pivotal advancement in precision agriculture by demonstrating the effectiveness of AI models, particularly RetinaNet, in weed detection and classification across various growth stages and species. Our study, conducted on a comprehensive dataset of 203,567 images spanning 16 weed species over 11 weeks, reveals RetinaNet's superior performance with mAP scores of 0.907 and 0.904 on training and test sets, and an inference speed of 7.28 FPS, significantly outpacing DETR's 0.854 and 0.840 mAP scores and 3.49 FPS speed. Both models exhibit improved accuracy with plant maturation, yet challenges persist during the early growth stages (weeks 1-2) due to poor differentiation between emerging plants and soil. These findings underscore the practical implications for weed management, with RetinaNet recommended for real-time applications due to its accuracy and speed. To integrate these models into existing agricultural practices, farmers should implement mobile-based applications for in-field weed detection using RetinaNet, calibrate the model for specific weed species with their growth stages prevalent in their region, and combine AI-driven detection with GPS-guided precision spraying systems. Despite the controlled greenhouse setting and early-stage detection challenges, this study lays the groundwork for future research aimed at enhancing detection accuracy through custom transformer models and expanding the dataset to include real field conditions. These AI-driven innovations hold the promise of revolutionizing weed management by enabling species-specific, growth-stage-aware detection, potentially reducing herbicide use, cutting costs, and minimizing environmental impact. By following these integration guidelines, farmers can leverage AI models to optimize their weed management strategies, leading to more sustainable and efficient agricultural practices.



\documentclass[conference]{IEEEtran}
\IEEEoverridecommandlockouts
% The preceding line is only needed to identify funding in the first footnote. If that is unneeded, please comment it out.
\usepackage{cite}
\usepackage{amsmath,amssymb,amsfonts}
\usepackage{algorithmic}
\usepackage{graphicx}
\usepackage{textcomp}
\usepackage{xcolor}
\def\BibTeX{{\rm B\kern-.05em{\sc i\kern-.025em b}\kern-.08em
    T\kern-.1667em\lower.7ex\hbox{E}\kern-.125emX}}
\begin{document}

\title{Privacy Preserving Properties \\of \\Vision Classifiers\\
%{\footnotesize \textsuperscript{*}Note: Sub-titles are not captured in %should not be used}
%\thanks{Identify applicable funding agency here. If none, delete this.}
}

%\author{\IEEEauthorblockN{Anonymous Authors}}

\author{\IEEEauthorblockN{1\textsuperscript{st} Pirzada Suhail}
\IEEEauthorblockA{\textit{IIT Bombay} \\
Mumbai, India \\
psuhail@iitb.ac.in}
\and
\IEEEauthorblockN{2\textsuperscript{nd} Amit Sethi}
\IEEEauthorblockA{\textit{IIT Bombay} \\
Mumbai, India \\
asethi@iitb.ac.in}
}

\maketitle

\begin{abstract}

Vision classifiers are often trained on proprietary datasets containing sensitive information, yet the models themselves are frequently shared openly under the privacy-preserving assumption. Although these models are assumed to protect sensitive information in their training data, the extent to which this assumption holds for different architectures remains unexplored. This assumption is challenged by inversion attacks which attempt to reconstruct training data from model weights, exposing significant privacy vulnerabilities. In this study, we systematically evaluate the privacy-preserving properties of vision classifiers across diverse architectures, including Multi-Layer Perceptrons (MLPs), Convolutional Neural Networks (CNNs), and Vision Transformers (ViTs). Using network inversion-based reconstruction techniques, we assess the extent to which these architectures memorize and reveal training data, quantifying the relative ease of reconstruction across models. Our analysis highlights how architectural differences, such as input representation, feature extraction mechanisms, and weight structures, influence privacy risks. By comparing these architectures, we identify which are more resilient to inversion attacks and examine the trade-offs between model performance and privacy preservation, contributing to the development of secure and privacy-respecting machine learning models for sensitive applications. Our findings provide actionable insights into the design of secure and privacy-aware machine learning systems, emphasizing the importance of evaluating architectural decisions in sensitive applications involving proprietary or personal data.

\end{abstract}

\begin{IEEEkeywords}
Safety, Privacy, Network Inversion, Reconstructions
\end{IEEEkeywords}

\section{Introduction}

The advent of modern vision classifiers has revolutionized a wide range of applications, from autonomous vehicles and medical imaging to facial recognition. These models, often trained on proprietary datasets, have become a cornerstone of advancements in artificial intelligence (AI). However, the growing practice of sharing pre-trained models has raised critical concerns about the privacy of the training data. Many assume that the act of sharing a trained model inherently preserves the privacy of the underlying dataset, but this assumption is increasingly being challenged by research demonstrating the potential for data leakage. The question remains: to what extent can different model architectures safeguard sensitive information against reconstruction or inversion attacks?

Model inversion attacks, where an adversary attempts to reconstruct the training data from a model's weights or output, highlight the vulnerability of machine learning systems to privacy breaches. This becomes particularly alarming in domains like healthcare, where proprietary datasets often contain highly sensitive information, or in applications involving biometric data, where privacy is paramount. Vision classifiers, in particular, process inherently personal and identifiable information, such as faces or medical scans, making it imperative to evaluate their privacy-preserving properties comprehensively.

In this study, we systematically assess the privacy-preserving capabilities of three prominent architectures: Multi-Layer Perceptrons (MLPs), Convolutional Neural Networks (CNNs)\cite{dosovitskiy2021imageworth16x16words}, and Vision Transformers (ViTs)\cite{oshea2015introductionconvolutionalneuralnetworks}. These architectures differ significantly in their structure, feature extraction mechanisms, and input processing pipelines, which could influence their tendencies to memorize and inadvertently leak training data. For example, CNNs are designed to capture spatial hierarchies and local patterns, while ViTs, on the other hand, employ self-attention mechanisms that focus on global relationships between input elements, potentially resulting in different privacy implications. By comparing these architectures, we aim to understand their relative vulnerabilities and the factors that contribute to privacy risks.

To evaluate the privacy-preserving properties of these classifiers, we utilize network inversion-based reconstruction techniques as in \cite{suhail2024net}. These methods attempt to recover the training data entirely from the model weights, providing a quantifiable measure of privacy leakage. Such reconstruction attacks exploit the tendency of models to memorize specific details about their training data, particularly when the training set is small or when over-parameterized architectures are used. By systematically applying these techniques, we assess the relative ease of reconstruction across architectures and analyze the impact of architectural differences on privacy.

We conduct our analysis in the most extreme case of privacy risk: at the end of the training process, without any knowledge of the training process itself, without utilizing any unobvious prior information, without any auxiliary datasets and relying on a single trained model. This setup provides a stringent evaluation of privacy risks, focusing on the inherent vulnerabilities of the model architectures. Further the quality of the reconstructed samples is assessed by comparing them to the original training samples using a similarity metric. Reconstructions with higher Structural Similarity Index Measure (SSIM) values indicate a greater privacy risk, as they suggest more effective memorization of the training data by the model. 

Our findings highlight that architectural differences in processing input images, feature extraction, and weight structures contribute to varying degrees of privacy leakage. We apply our evaluation to multiple benchmark datasets, including MNIST, FashionMNIST, CIFAR-10, and SVHN that cover a wide range of complexities and image types, allowing us to study how different architectures behave under varying data conditions.

\section{Related Works}

Privacy concerns in machine learning have led to extensive research in Privacy-Preserving Machine Learning (PPML), particularly in mitigating risks related to membership inference, attribute inference, and model inversion attacks. A foundational study by \cite{8677282} provides an overview of privacy threats in ML, including model inversion, and explores defenses such as differential privacy, homomorphic encryption, and federated learning. Similarly, \cite{xu2021privacypreservingmachinelearningmethods} introduces the Phase, Guarantee, and Utility (PGU) triad, a framework to evaluate PPML techniques across different phases of the ML pipeline. These studies highlight the need for privacy-preserving methods but primarily focus on algorithmic-level defenses rather than evaluating inherent vulnerabilities in different model architectures. Unlike these approaches, our study investigates the privacy risks posed by architectural design choices by analyzing how MLPs, CNNs, and ViTs differ in their susceptibility to model inversion attacks.

Network inversion has emerged as a powerful technique to understand how neural networks encode and manipulate training data. Initially developed for interpretability \cite{KINDERMANN1990277,784232}, it has since been shown to reconstruct sensitive training samples, raising significant privacy concerns \cite{Wong2017NeuralNI,ad}. Early works on network inversion focused on fully connected networks (MLPs) \cite{KINDERMANN1990277,SAAD200778}, demonstrating that they tend to memorize training data, making them vulnerable to inversion attacks. Evolutionary inversion procedures \cite{784232} improved the ability to capture input-output relationships, providing deeper insights into model memorization behavior. More recent studies extended these inversion techniques to CNNs \cite{ad}, showing that hierarchical feature extraction does not necessarily prevent training data leakage. The introduction of ViTs has further complicated this issue, as their global self-attention mechanisms process data differently than CNNs, raising new questions about how they store training information and whether their memorization patterns lead to higher or lower inversion risks.

In adversarial settings, model inversion attacks aim to reconstruct sensitive data by exploiting a model's predictions, gradients, or weights \cite{ad,kumar2019modelinversionnetworksmodelbased}. These attacks have been shown to succeed even without direct access to the training process, as demonstrated by \cite{9833677}, where an adversary with auxiliary knowledge reconstructs sensitive samples. Gradient-based inversion attacks further exacerbate these risks by leaking sensitive training information through shared gradients in federated learning setups \cite{pmlr-v206-wang23g}.

To improve the stability of inversion processes, recent works have explored novel optimization techniques. For example, \cite{liu2022landscapelearningneuralnetwork} proposed learning a loss landscape to make gradient-based inversion faster and more stable. Alternative approaches, such as encoding networks into Conjunctive Normal Form (CNF) and solving them using SAT solvers, offer deterministic solutions for inversion, as introduced by \cite{suhail2024network}. Although computationally expensive, these methods ensure diversity in the reconstructed samples by avoiding shortcuts in the optimization process.

Model Inversion (MI) attacks have also been extended to scenarios involving ensemble techniques, where multiple models trained on shared subjects or entities are used to guide the reconstruction process. The concept of ensemble inversion, as proposed by \cite{wang2021reconstructingtrainingdatadiverse}, enhances the quality of reconstructed data by leveraging the diversity of perspectives provided by multiple models. By incorporating auxiliary datasets similar to the presumed training data, this approach achieves high-quality reconstructions with sharper predictions and higher activations. This work highlights the risks posed by adversaries exploiting shared data entities across models, emphasizing the importance of robust defense mechanisms.

Reconstruction methods for training data have evolved significantly, focusing on improving the efficiency and accuracy of recovering data from models. Traditional optimization-based approaches relied on iteratively refining input data to match a model’s outputs or activations \cite{Wong2017NeuralNI}. More recent advancements have leveraged generative models, such as GANs and autoencoders, to synthesize high-quality reconstructions. These techniques aim to approximate the distribution of training data while maintaining computational efficiency. In the context of privacy risks, works like \cite{haim2022reconstructingtrainingdatatrained} demonstrated that significant portions of training data could be reconstructed from neural network parameters in binary classification settings. This work was later extended to multi-class classification by \cite{buzaglo2023reconstructingtrainingdatamulticlass}, showing that higher-quality reconstructions are possible and revealing the impact of regularization techniques, such as weight decay, on memorization behavior.

The ability to reconstruct training data from model gradients also presents a critical privacy challenge. The study by \cite{wang2023reconstructingtrainingdatamodel} demonstrated that training samples could be fully reconstructed from a single gradient query, even without explicit training or memorization. Recent advancements like \cite{oz2024reconstructingtrainingdatareal}, adapt reconstruction schemes to operate in the embedding space of large pre-trained models, such as DINO-ViT and CLIP. This approach, which introduces clustering-based methods to identify high-quality reconstructions from numerous candidates, represents a significant improvement over earlier techniques that required access to the original dataset. While \cite{pmlr-v162-guo22c} extended differential privacy guarantees to training data reconstruction attacks.

In this paper, we build upon prior work on network inversion and training data reconstruction. Drawing from works like \cite{suhail2024networkcnn} and \cite{suhail2024networkinversionapplications}, we employ network inversion methods to understand the internal representations of neural networks and the patterns they memorize during training. Our study systematically compares the privacy-preserving properties of different vision classifier architectures, including MLPs, CNNs, and ViTs, using network inversion-based reconstruction techniques \cite{suhail2024net}. By evaluating these techniques across datasets such as MNIST, FashionMNIST, CIFAR-10, and SVHN, we explore the impact of architectural differences, input processing mechanisms, and weight structures on the susceptibility of models to inversion attacks.


\section{Methodology}

\subsection{Overview}
In this study, we investigate the ease of reconstruction and the extent of memorization in trained vision classifiers based on different architectures. Our primary focus is on analyzing the most extreme case of training data reconstruction, where the inversion process relies almost entirely on the input-output relationships of the trained model and its learned weights. Unlike prior reconstruction approaches that leverage pre-trained models, auxiliary datasets, gradient information from the training process, or other unobvious priors, our method seeks to reconstruct training data with minimal external dependencies. This approach allows us to systematically evaluate how different architectures—Multi-Layer Perceptrons (MLPs), Convolutional Neural Networks (CNNs), and Vision Transformers (ViTs)—differ in their ability to preserve or expose sensitive training data.

To perform network inversion and data reconstruction, we build upon the methodology introduced in \cite{suhail2024networkcnn, suhail2024networkinversionapplications}, which has primarily focused on CNN-based classifiers. We extend this approach to other architectures, particularly MLPs and ViTs, to assess their relative vulnerability to inversion attacks. Briefly, network inversion techniques aim to generate inputs that align with the learned decision boundaries of a classifier by training a conditioned generator to reconstruct data that maximally activates specific output neurons. In its standard form, this inversion process does not necessarily yield images resembling actual training samples but instead produces arbitrary inputs that satisfy the model’s learned function. However, by modifying the inversion procedure following \cite{suhail2024network, suhail2024net}, we incentivize the generator to reconstruct training-like data by leveraging key properties of the classifier with respect to its training data. These modifications allow us to better quantify the extent of memorization across different architectures and assess the associated privacy risks. The proposed approach to Network Inversion and subsequent training data reconstruction uses a carefully conditioned generator that learns the data distributions in the input space of the trained classifier.

\subsection{Classifier Architectures}
We perform inversion and reconstruction on classifiers based on three distinct architectures:

\begin{itemize}
    \item \textbf{Multi-Layer Perceptrons (MLPs)}: These are fully connected networks where each neuron in one layer is connected to every neuron in the next layer. MLPs process flattened input images, lacking any inherent spatial hierarchy. The inversion and subsequent reconstruction will be performed using the logits and penultimate fully connected layers.
    \item \textbf{Convolutional Neural Networks (CNNs)}: CNNs use convolutional layers to extract hierarchical spatial features from images, enabling them to effectively capture local patterns. In this case the features from the fully connected layers are used after flattening the output of convolutional layers along with the logits in the last layer to perform inversion.
    \item \textbf{Vision Transformers (ViTs)}: ViTs utilize self-attention mechanisms to capture global dependencies across an image. This architecture is particularly effective in modeling long-range interactions within an image. In ViTs we particularly look at the classification token embeddings and use it in the same way as above.
\end{itemize}

These architectures have inherently different memory capacities and generalization properties, affecting their susceptibility to reconstruction attacks.

\subsection{Vector-Matrix Conditioned Generator}
The generator in our approach is conditioned on vectors and matrices to ensure that it learns diverse representations of the data distribution. Unlike simple label conditioning, the vector-matrix conditioning mechanism encodes the label information more intricately, allowing the generator to better capture the input space of the classifier. 

The generator is initially conditioned using $N$-dimensional vectors for an $N$-class classification task. These vectors are derived from a normal distribution and are softmaxed to form a probability distribution. They implicitly encode the labels, promoting diversity in the generated images.

Further, a Hot Conditioning Matrix of size $N \times N$ is used for deeper conditioning. In this matrix, all elements in a specific row or column are set to $1$, corresponding to the encoded label, while the rest are $0$. This conditioning is applied during intermediate stages of the generation process to refine the diversity of the outputs.

\subsection{Training Data Properties}
The classifier exhibits specific properties when interacting with training data, which are exploited to facilitate the reconstruction of training-like samples:
\begin{figure*}[t]
\centering
\includegraphics[width=1\textwidth]{tldr.png} 
\caption{Schematic Approach to Training-Like Data Reconstruction using Network Inversion}
\label{fig:reconstruction}
\end{figure*}

\begin{itemize}
    \item \textbf{Model Confidence:} The classifier is more confident when predicting labels for training samples compared to random samples. Hence, in order to take this into account we condition the generation on one-hot vectors and then minimise its KL Divergence from the classifier's output enforcing generation of samples that are confidently classified buy the classifier. This can be expressed as:
    \begin{equation}
    P(y_{\text{in}} | x_{\text{in}}; \theta) \gg P(y_{\text{ood}} | x_{\text{ood}}; \theta)
    \end{equation}

    \item \textbf{Robustness to Perturbations:} During training the model gets to observe a diverse set of data including different variations of the images in the same class. Due to which the model is relatively robust to perturbations around training data compared to random inverted samples, meaning small changes do not significantly affect predictions. We take this into account in by perturbing the generated images and then ensuring that the perturbed images also produce similar output,
    \begin{equation}
    \frac{\partial f_{\theta}(x_{\text{in}})}{\partial x_{\text{in}}} \ll \frac{\partial f_{\theta}(x_{\text{ood}})}{\partial x_{\text{ood}}}
    \end{equation}

    \item \textbf{Gradient Behavior:} By virtue of training the model on a certain dataset, the gradient of the loss with respect to model weights is expected to be lower for training data compared to random inverted samples, as the model has already been optimized on it, Hence we add a penalty on the gradients that encourages the generation of samples with low gradients as:
    \begin{equation}
    \|\nabla_{\theta} L(f_{\theta}(x_{\text{in}}), y_{\text{in}})\| \ll \|\nabla_{\theta} L(f_{\theta}(x_{\text{ood}}), y_{\text{ood}})\|
    \end{equation}
\end{itemize}

\subsection{Training Data Reconstruction}
The reconstruction process utilizes the generator to produce training-like samples by taking into account the specific properties of the training data with respect to the classifier. The approach is schematically illustrated in Figure \ref{fig:reconstruction}. 


The primary loss function used for reconstruction, 
\begin{align*}
\mathcal{L}_{\text{Recon}} = & \; \alpha \cdot \mathcal{L}_{\text{KL}} 
+ \alpha' \cdot \mathcal{L}_{\text{KL}}^{\text{pert}}
+ \beta \cdot \mathcal{L}_{\text{CE}} 
+ \beta' \cdot \mathcal{L}_{\text{CE}}^{\text{pert}} \\
& + \gamma \cdot \mathcal{L}_{\text{Cosine}} 
+ \delta \cdot \mathcal{L}_{\text{Ortho}} \\
& + \eta_1 \cdot \mathcal{L}_{\text{Var}} 
+ \eta_2 \cdot \mathcal{L}_{\text{Pix}} 
+ \eta_3 \cdot \mathcal{L}_{\text{Grad}}
\end{align*}

where \( \mathcal{L}_{\text{KL}} \) is the KL Divergence loss, \( \mathcal{L}_{\text{CE}} \) is the Cross Entropy loss, \( \mathcal{L}_{\text{Cosine}} \) is the Cosine Similarity loss, and \( \mathcal{L}_{\text{Ortho}} \) is the Feature Orthogonality loss. The hyperparameters \( \alpha, \beta, \gamma, \delta \) control the contribution of each individual loss term defined as:
\[
\mathcal{L}_{\text{KL}} = D_{\text{KL}}(P \| Q) = \sum_{i} P(i) \log \frac{P(i)}{Q(i)}
\]
\[
\mathcal{L}_{\text{CE}} = -\sum_{i} y_{i} \log(\hat{y}_{i})
\]
\[
\mathcal{L}_{\text{Cosine}} = \frac{1}{N(N-1)} \sum_{i \neq j} \cos(\theta_{ij})
\]
\[
\mathcal{L}_{\text{Ortho}} = \frac{1}{N^2} \sum_{i, j} (G_{ij} - \delta_{ij})^2
\]
where \( D_{\text{KL}} \) represents the KL Divergence between the input distribution \( P \) and the output distribution \( Q \), \( y_{i} \) is the set encoded label, \( \hat{y}_{i} \) is the predicted label from the classifier, \( \cos(\theta_{ij}) \) represents the cosine similarity between features of generated images \( i \) and \( j \), \( G_{ij} \) is the element of the Gram matrix, and \( \delta_{ij} \) is the Kronecker delta function. \( N \) is the number of feature vectors in the batch.

Further to take reconstruction into account we also use \(\mathcal{L}_{\text{KL}}^{\text{pert}}\) and \(\mathcal{L}_{\text{CE}}^{\text{pert}}\) that represent the KL divergence and cross-entropy losses applied on perturbed images, weighted by \( \alpha'\) and  \(\beta' \)respectively while \(\mathcal{L}_{\text{Var}}\), \(\mathcal{L}_{\text{Pix}}\) and \(\mathcal{L}_{\text{Grad}}\) represent the variational loss, Pixel Loss and penalty on gradient norm each weighted by \( \eta_1\), \( \eta_2\), and \(\eta_3\) respectively and defined for an Image \(I\) as:
\begin{align*}
\mathcal{L}_{\text{Var}} = \frac{1}{N} \sum_{i=1}^{N} \Bigg( \sum_{h,w} & \Big( ( I_{i, h+1, w} - I_{i, h, w} )^2 \\
& + ( I_{i, h, w+1} - I_{i, h, w} )^2 \Big) \Bigg)
\end{align*}
\[
\mathcal{L}_{\text{Grad}} = \left\| \nabla_{\theta} L(f_{\theta}(I), y) \right\|
\]

\[
\mathcal{L}_{\text{Pix}} = \sum \max(0, -I) + \sum \max(0, I - 1)
\]

By integrating these loss components, the generator is trained to produce samples that closely resemble the training data, thus revealing the extent of memorization within the classifier.


In this section, we present the experimental results obtained by applying the reconstruction technique on the MNIST \cite{deng2012mnist}, FashionMNIST \cite{xiao2017fashionmnistnovelimagedataset}, SVHN, and CIFAR-10 \cite{cf} datasets. Our goal is to evaluate the ease of reconstruction and the extent of memorization in different vision classifier architectures by training a generator to produce images that resemble the training data. The classifier is first trained normally on a given dataset and then held in evaluation mode for the purpose of reconstruction. The conditioned generator, which takes as input latent vectors and conditioning information, is trained to generate images that the classifier maps to specific labels.

We evaluate three vision classifier architectures: Multi-Layer Perceptrons (MLPs), Convolutional Neural Networks (CNNs), and Vision Transformers (ViTs). We implement a 5-layer MLP with Batch Normalization, Leaky ReLU activations, and Dropout layers \cite{JMLR:v15:srivastava14a} to mitigate memorization tendencies. The CNN consists of three convolutional layers followed by batch normalization \cite{pmlr-v37-ioffe15}, dropout layers, and a fully connected layer for classification. The ViT classifier uses a transformer-based architecture with three self-attention layers, each containing four attention heads while the input images are divided into non-overlapping patches of size \(4 \times 4\).

The generator, instead of traditional label embeddings, follows a Vector-Matrix Conditioning approach to ensure diverse and structured image generation. The class labels are encoded into random softmaxed vectors concatenated with the latent vector, followed by multiple layers of transposed convolutions, batch normalization, and dropout layers to encourage diversity in generated images. Once the concatenated latent and conditioning vectors are upsampled to \(N \times N\) spatial dimensions for an \(N\)-class classification task, they are concatenated with a conditioning matrix for further generation up to the required image size (\(28 \times 28\) for MNIST and FashionMNIST, and \(32 \times 32\) for SVHN and CIFAR-10).

To assess the extent of memorization and the relative ease of reconstruction across different architectures, we use the Structural Similarity Index Measure (SSIM). The goal was to evaluate how different architectures handle privacy risks by comparing the quality of reconstructed images. SSIM quantifies the perceptual similarity between two images based on luminance, contrast, and structural similarity components. In this study, SSIM is used to compare the reconstructed images with their closest matches from the training dataset. Higher SSIM values indicate stronger resemblance to training samples, suggesting a greater privacy risk due to increased memorization by the classifier.

Table~\ref{tab:ssim_values} presents the SSIM scores between reconstructed samples and training data, averaged across all classes for each dataset and architecture. Higher SSIM values suggest greater memorization and weaker privacy preservation, as the reconstructed samples strongly resemble real training data.

\begin{table}[h]
\centering
\caption{SSIM values for reconstructed samples across different architectures and datasets.}
\label{tab:ssim_values}
\resizebox{\linewidth}{!}{
\begin{tabular}{|l|c|c|c|}
\hline
\textbf{Dataset}       & \textbf{MLP} & \textbf{ViT} & \textbf{CNN} \\ \hline
\textbf{MNIST}         & 0.83         & 0.78         & 0.73         \\ \hline
\textbf{FashionMNIST}  & 0.74         & 0.64         & 0.63         \\ \hline
\textbf{SVHN}          & 0.71         & 0.68         & 0.69         \\ \hline
\textbf{CIFAR-10}      & 0.65         & 0.62         & 0.58         \\ \hline
\end{tabular}
}
\end{table}

From Table~\ref{tab:ssim_values}, we observe that MLPs exhibit the highest SSIM scores across all datasets, suggesting that they retain more training data details than CNNs and ViTs. This is likely due to their fully connected nature and lack of spatial inductive biases, leading to higher memorization tendencies. Also individual pixels in the images have dedicated weights associated, making memorization easier.

ViTs generally show lower SSIM values than MLPs but remain slightly higher than CNNs, indicating that self-attention mechanisms contribute to retaining finer details in the reconstructed images. Unlike CNNs, ViTs lack pooling layers, which means that more spatial information is preserved, making inversion attacks more feasible. 

CNNs show the lowest SSIM values across all datasets, suggesting that they inherently discard more specific input details due to weight sharing, local receptive fields, and pooling operations. This abstraction reduces direct memorization and makes CNNs relatively more privacy-preserving compared to MLPs and ViTs.

\begin{figure}[ht]
\centering
\includegraphics[width=0.95\linewidth]{riz.png} 
\caption{Comparison of reconstructed samples across MLP, ViT, and CNN architectures. The first column represents actual training samples, while subsequent columns show corresponding reconstructed images.}
\label{fig:sam}
\end{figure}

Figure~\ref{fig:sam} presents a qualitative comparison of reconstructed samples across different classifier architectures. The first column depicts actual training samples, while the subsequent columns display reconstructed images generated from MLPs, ViTs, and CNNs. Notably, MLP-generated reconstructions appear most similar to the original samples, while CNNs yield more abstract representations, indicating lower memorization.

These results highlight the variation in privacy risks across architectures, with MLPs being the most susceptible to memorization, followed by ViTs, and CNNs exhibiting the lowest reconstruction fidelity. The findings emphasize the importance of architectural choices in designing privacy-aware models, where CNNs may be preferable for applications requiring privacy preservation.

\section{Conclusion and Future Work}
In this paper, we systematically evaluated the privacy-preserving properties of vision classifiers by analyzing the extent of memorization and the ease of training data reconstruction across different architectures. Our experimental results, quantified through SSIM scores, indicate that MLPs tend to memorize more information about training samples compared to CNNs and ViTs, making them more prone to reconstruction. These findings highlight the crucial role of architectural choices in mitigating privacy risks, emphasizing the need for privacy-aware model deployment, especially in sensitive applications.

Future research can extend this study by exploring additional factors that influence privacy leakage, such as model depth, dataset complexity, and the impact of regularization techniques. In the case of ViTs it would be also of interest to see how the reconstructions differ with varying patch size. Investigating the impact of differential privacy mechanisms in reducing reconstruction risks could provide valuable insights into enhancing privacy.

\bibliographystyle{IEEEtran}  % Or any other preferred style
\bibliography{references} 

\end{document}

% \bibliographystyle{IEEEtran}
% \bibliography{IEEEabrv,references}
% Add this line to include the .bbl file:

% For arxiv
% \documentclass[conference]{IEEEtran}
\IEEEoverridecommandlockouts
% The preceding line is only needed to identify funding in the first footnote. If that is unneeded, please comment it out.
\usepackage{cite}
\usepackage{amsmath,amssymb,amsfonts}
\usepackage{algorithmic}
\usepackage{graphicx}
\usepackage{textcomp}
\usepackage{xcolor}
\def\BibTeX{{\rm B\kern-.05em{\sc i\kern-.025em b}\kern-.08em
    T\kern-.1667em\lower.7ex\hbox{E}\kern-.125emX}}
\begin{document}

\title{Privacy Preserving Properties \\of \\Vision Classifiers\\
%{\footnotesize \textsuperscript{*}Note: Sub-titles are not captured in %should not be used}
%\thanks{Identify applicable funding agency here. If none, delete this.}
}

%\author{\IEEEauthorblockN{Anonymous Authors}}

\author{\IEEEauthorblockN{1\textsuperscript{st} Pirzada Suhail}
\IEEEauthorblockA{\textit{IIT Bombay} \\
Mumbai, India \\
psuhail@iitb.ac.in}
\and
\IEEEauthorblockN{2\textsuperscript{nd} Amit Sethi}
\IEEEauthorblockA{\textit{IIT Bombay} \\
Mumbai, India \\
asethi@iitb.ac.in}
}

\maketitle

\begin{abstract}

Vision classifiers are often trained on proprietary datasets containing sensitive information, yet the models themselves are frequently shared openly under the privacy-preserving assumption. Although these models are assumed to protect sensitive information in their training data, the extent to which this assumption holds for different architectures remains unexplored. This assumption is challenged by inversion attacks which attempt to reconstruct training data from model weights, exposing significant privacy vulnerabilities. In this study, we systematically evaluate the privacy-preserving properties of vision classifiers across diverse architectures, including Multi-Layer Perceptrons (MLPs), Convolutional Neural Networks (CNNs), and Vision Transformers (ViTs). Using network inversion-based reconstruction techniques, we assess the extent to which these architectures memorize and reveal training data, quantifying the relative ease of reconstruction across models. Our analysis highlights how architectural differences, such as input representation, feature extraction mechanisms, and weight structures, influence privacy risks. By comparing these architectures, we identify which are more resilient to inversion attacks and examine the trade-offs between model performance and privacy preservation, contributing to the development of secure and privacy-respecting machine learning models for sensitive applications. Our findings provide actionable insights into the design of secure and privacy-aware machine learning systems, emphasizing the importance of evaluating architectural decisions in sensitive applications involving proprietary or personal data.

\end{abstract}

\begin{IEEEkeywords}
Safety, Privacy, Network Inversion, Reconstructions
\end{IEEEkeywords}

\section{Introduction}

The advent of modern vision classifiers has revolutionized a wide range of applications, from autonomous vehicles and medical imaging to facial recognition. These models, often trained on proprietary datasets, have become a cornerstone of advancements in artificial intelligence (AI). However, the growing practice of sharing pre-trained models has raised critical concerns about the privacy of the training data. Many assume that the act of sharing a trained model inherently preserves the privacy of the underlying dataset, but this assumption is increasingly being challenged by research demonstrating the potential for data leakage. The question remains: to what extent can different model architectures safeguard sensitive information against reconstruction or inversion attacks?

Model inversion attacks, where an adversary attempts to reconstruct the training data from a model's weights or output, highlight the vulnerability of machine learning systems to privacy breaches. This becomes particularly alarming in domains like healthcare, where proprietary datasets often contain highly sensitive information, or in applications involving biometric data, where privacy is paramount. Vision classifiers, in particular, process inherently personal and identifiable information, such as faces or medical scans, making it imperative to evaluate their privacy-preserving properties comprehensively.

In this study, we systematically assess the privacy-preserving capabilities of three prominent architectures: Multi-Layer Perceptrons (MLPs), Convolutional Neural Networks (CNNs)\cite{dosovitskiy2021imageworth16x16words}, and Vision Transformers (ViTs)\cite{oshea2015introductionconvolutionalneuralnetworks}. These architectures differ significantly in their structure, feature extraction mechanisms, and input processing pipelines, which could influence their tendencies to memorize and inadvertently leak training data. For example, CNNs are designed to capture spatial hierarchies and local patterns, while ViTs, on the other hand, employ self-attention mechanisms that focus on global relationships between input elements, potentially resulting in different privacy implications. By comparing these architectures, we aim to understand their relative vulnerabilities and the factors that contribute to privacy risks.

To evaluate the privacy-preserving properties of these classifiers, we utilize network inversion-based reconstruction techniques as in \cite{suhail2024net}. These methods attempt to recover the training data entirely from the model weights, providing a quantifiable measure of privacy leakage. Such reconstruction attacks exploit the tendency of models to memorize specific details about their training data, particularly when the training set is small or when over-parameterized architectures are used. By systematically applying these techniques, we assess the relative ease of reconstruction across architectures and analyze the impact of architectural differences on privacy.

We conduct our analysis in the most extreme case of privacy risk: at the end of the training process, without any knowledge of the training process itself, without utilizing any unobvious prior information, without any auxiliary datasets and relying on a single trained model. This setup provides a stringent evaluation of privacy risks, focusing on the inherent vulnerabilities of the model architectures. Further the quality of the reconstructed samples is assessed by comparing them to the original training samples using a similarity metric. Reconstructions with higher Structural Similarity Index Measure (SSIM) values indicate a greater privacy risk, as they suggest more effective memorization of the training data by the model. 

Our findings highlight that architectural differences in processing input images, feature extraction, and weight structures contribute to varying degrees of privacy leakage. We apply our evaluation to multiple benchmark datasets, including MNIST, FashionMNIST, CIFAR-10, and SVHN that cover a wide range of complexities and image types, allowing us to study how different architectures behave under varying data conditions.

\section{Related Works}

Privacy concerns in machine learning have led to extensive research in Privacy-Preserving Machine Learning (PPML), particularly in mitigating risks related to membership inference, attribute inference, and model inversion attacks. A foundational study by \cite{8677282} provides an overview of privacy threats in ML, including model inversion, and explores defenses such as differential privacy, homomorphic encryption, and federated learning. Similarly, \cite{xu2021privacypreservingmachinelearningmethods} introduces the Phase, Guarantee, and Utility (PGU) triad, a framework to evaluate PPML techniques across different phases of the ML pipeline. These studies highlight the need for privacy-preserving methods but primarily focus on algorithmic-level defenses rather than evaluating inherent vulnerabilities in different model architectures. Unlike these approaches, our study investigates the privacy risks posed by architectural design choices by analyzing how MLPs, CNNs, and ViTs differ in their susceptibility to model inversion attacks.

Network inversion has emerged as a powerful technique to understand how neural networks encode and manipulate training data. Initially developed for interpretability \cite{KINDERMANN1990277,784232}, it has since been shown to reconstruct sensitive training samples, raising significant privacy concerns \cite{Wong2017NeuralNI,ad}. Early works on network inversion focused on fully connected networks (MLPs) \cite{KINDERMANN1990277,SAAD200778}, demonstrating that they tend to memorize training data, making them vulnerable to inversion attacks. Evolutionary inversion procedures \cite{784232} improved the ability to capture input-output relationships, providing deeper insights into model memorization behavior. More recent studies extended these inversion techniques to CNNs \cite{ad}, showing that hierarchical feature extraction does not necessarily prevent training data leakage. The introduction of ViTs has further complicated this issue, as their global self-attention mechanisms process data differently than CNNs, raising new questions about how they store training information and whether their memorization patterns lead to higher or lower inversion risks.

In adversarial settings, model inversion attacks aim to reconstruct sensitive data by exploiting a model's predictions, gradients, or weights \cite{ad,kumar2019modelinversionnetworksmodelbased}. These attacks have been shown to succeed even without direct access to the training process, as demonstrated by \cite{9833677}, where an adversary with auxiliary knowledge reconstructs sensitive samples. Gradient-based inversion attacks further exacerbate these risks by leaking sensitive training information through shared gradients in federated learning setups \cite{pmlr-v206-wang23g}.

To improve the stability of inversion processes, recent works have explored novel optimization techniques. For example, \cite{liu2022landscapelearningneuralnetwork} proposed learning a loss landscape to make gradient-based inversion faster and more stable. Alternative approaches, such as encoding networks into Conjunctive Normal Form (CNF) and solving them using SAT solvers, offer deterministic solutions for inversion, as introduced by \cite{suhail2024network}. Although computationally expensive, these methods ensure diversity in the reconstructed samples by avoiding shortcuts in the optimization process.

Model Inversion (MI) attacks have also been extended to scenarios involving ensemble techniques, where multiple models trained on shared subjects or entities are used to guide the reconstruction process. The concept of ensemble inversion, as proposed by \cite{wang2021reconstructingtrainingdatadiverse}, enhances the quality of reconstructed data by leveraging the diversity of perspectives provided by multiple models. By incorporating auxiliary datasets similar to the presumed training data, this approach achieves high-quality reconstructions with sharper predictions and higher activations. This work highlights the risks posed by adversaries exploiting shared data entities across models, emphasizing the importance of robust defense mechanisms.

Reconstruction methods for training data have evolved significantly, focusing on improving the efficiency and accuracy of recovering data from models. Traditional optimization-based approaches relied on iteratively refining input data to match a model’s outputs or activations \cite{Wong2017NeuralNI}. More recent advancements have leveraged generative models, such as GANs and autoencoders, to synthesize high-quality reconstructions. These techniques aim to approximate the distribution of training data while maintaining computational efficiency. In the context of privacy risks, works like \cite{haim2022reconstructingtrainingdatatrained} demonstrated that significant portions of training data could be reconstructed from neural network parameters in binary classification settings. This work was later extended to multi-class classification by \cite{buzaglo2023reconstructingtrainingdatamulticlass}, showing that higher-quality reconstructions are possible and revealing the impact of regularization techniques, such as weight decay, on memorization behavior.

The ability to reconstruct training data from model gradients also presents a critical privacy challenge. The study by \cite{wang2023reconstructingtrainingdatamodel} demonstrated that training samples could be fully reconstructed from a single gradient query, even without explicit training or memorization. Recent advancements like \cite{oz2024reconstructingtrainingdatareal}, adapt reconstruction schemes to operate in the embedding space of large pre-trained models, such as DINO-ViT and CLIP. This approach, which introduces clustering-based methods to identify high-quality reconstructions from numerous candidates, represents a significant improvement over earlier techniques that required access to the original dataset. While \cite{pmlr-v162-guo22c} extended differential privacy guarantees to training data reconstruction attacks.

In this paper, we build upon prior work on network inversion and training data reconstruction. Drawing from works like \cite{suhail2024networkcnn} and \cite{suhail2024networkinversionapplications}, we employ network inversion methods to understand the internal representations of neural networks and the patterns they memorize during training. Our study systematically compares the privacy-preserving properties of different vision classifier architectures, including MLPs, CNNs, and ViTs, using network inversion-based reconstruction techniques \cite{suhail2024net}. By evaluating these techniques across datasets such as MNIST, FashionMNIST, CIFAR-10, and SVHN, we explore the impact of architectural differences, input processing mechanisms, and weight structures on the susceptibility of models to inversion attacks.


\section{Methodology}

\subsection{Overview}
In this study, we investigate the ease of reconstruction and the extent of memorization in trained vision classifiers based on different architectures. Our primary focus is on analyzing the most extreme case of training data reconstruction, where the inversion process relies almost entirely on the input-output relationships of the trained model and its learned weights. Unlike prior reconstruction approaches that leverage pre-trained models, auxiliary datasets, gradient information from the training process, or other unobvious priors, our method seeks to reconstruct training data with minimal external dependencies. This approach allows us to systematically evaluate how different architectures—Multi-Layer Perceptrons (MLPs), Convolutional Neural Networks (CNNs), and Vision Transformers (ViTs)—differ in their ability to preserve or expose sensitive training data.

To perform network inversion and data reconstruction, we build upon the methodology introduced in \cite{suhail2024networkcnn, suhail2024networkinversionapplications}, which has primarily focused on CNN-based classifiers. We extend this approach to other architectures, particularly MLPs and ViTs, to assess their relative vulnerability to inversion attacks. Briefly, network inversion techniques aim to generate inputs that align with the learned decision boundaries of a classifier by training a conditioned generator to reconstruct data that maximally activates specific output neurons. In its standard form, this inversion process does not necessarily yield images resembling actual training samples but instead produces arbitrary inputs that satisfy the model’s learned function. However, by modifying the inversion procedure following \cite{suhail2024network, suhail2024net}, we incentivize the generator to reconstruct training-like data by leveraging key properties of the classifier with respect to its training data. These modifications allow us to better quantify the extent of memorization across different architectures and assess the associated privacy risks. The proposed approach to Network Inversion and subsequent training data reconstruction uses a carefully conditioned generator that learns the data distributions in the input space of the trained classifier.

\subsection{Classifier Architectures}
We perform inversion and reconstruction on classifiers based on three distinct architectures:

\begin{itemize}
    \item \textbf{Multi-Layer Perceptrons (MLPs)}: These are fully connected networks where each neuron in one layer is connected to every neuron in the next layer. MLPs process flattened input images, lacking any inherent spatial hierarchy. The inversion and subsequent reconstruction will be performed using the logits and penultimate fully connected layers.
    \item \textbf{Convolutional Neural Networks (CNNs)}: CNNs use convolutional layers to extract hierarchical spatial features from images, enabling them to effectively capture local patterns. In this case the features from the fully connected layers are used after flattening the output of convolutional layers along with the logits in the last layer to perform inversion.
    \item \textbf{Vision Transformers (ViTs)}: ViTs utilize self-attention mechanisms to capture global dependencies across an image. This architecture is particularly effective in modeling long-range interactions within an image. In ViTs we particularly look at the classification token embeddings and use it in the same way as above.
\end{itemize}

These architectures have inherently different memory capacities and generalization properties, affecting their susceptibility to reconstruction attacks.

\subsection{Vector-Matrix Conditioned Generator}
The generator in our approach is conditioned on vectors and matrices to ensure that it learns diverse representations of the data distribution. Unlike simple label conditioning, the vector-matrix conditioning mechanism encodes the label information more intricately, allowing the generator to better capture the input space of the classifier. 

The generator is initially conditioned using $N$-dimensional vectors for an $N$-class classification task. These vectors are derived from a normal distribution and are softmaxed to form a probability distribution. They implicitly encode the labels, promoting diversity in the generated images.

Further, a Hot Conditioning Matrix of size $N \times N$ is used for deeper conditioning. In this matrix, all elements in a specific row or column are set to $1$, corresponding to the encoded label, while the rest are $0$. This conditioning is applied during intermediate stages of the generation process to refine the diversity of the outputs.

\subsection{Training Data Properties}
The classifier exhibits specific properties when interacting with training data, which are exploited to facilitate the reconstruction of training-like samples:
\begin{figure*}[t]
\centering
\includegraphics[width=1\textwidth]{tldr.png} 
\caption{Schematic Approach to Training-Like Data Reconstruction using Network Inversion}
\label{fig:reconstruction}
\end{figure*}

\begin{itemize}
    \item \textbf{Model Confidence:} The classifier is more confident when predicting labels for training samples compared to random samples. Hence, in order to take this into account we condition the generation on one-hot vectors and then minimise its KL Divergence from the classifier's output enforcing generation of samples that are confidently classified buy the classifier. This can be expressed as:
    \begin{equation}
    P(y_{\text{in}} | x_{\text{in}}; \theta) \gg P(y_{\text{ood}} | x_{\text{ood}}; \theta)
    \end{equation}

    \item \textbf{Robustness to Perturbations:} During training the model gets to observe a diverse set of data including different variations of the images in the same class. Due to which the model is relatively robust to perturbations around training data compared to random inverted samples, meaning small changes do not significantly affect predictions. We take this into account in by perturbing the generated images and then ensuring that the perturbed images also produce similar output,
    \begin{equation}
    \frac{\partial f_{\theta}(x_{\text{in}})}{\partial x_{\text{in}}} \ll \frac{\partial f_{\theta}(x_{\text{ood}})}{\partial x_{\text{ood}}}
    \end{equation}

    \item \textbf{Gradient Behavior:} By virtue of training the model on a certain dataset, the gradient of the loss with respect to model weights is expected to be lower for training data compared to random inverted samples, as the model has already been optimized on it, Hence we add a penalty on the gradients that encourages the generation of samples with low gradients as:
    \begin{equation}
    \|\nabla_{\theta} L(f_{\theta}(x_{\text{in}}), y_{\text{in}})\| \ll \|\nabla_{\theta} L(f_{\theta}(x_{\text{ood}}), y_{\text{ood}})\|
    \end{equation}
\end{itemize}

\subsection{Training Data Reconstruction}
The reconstruction process utilizes the generator to produce training-like samples by taking into account the specific properties of the training data with respect to the classifier. The approach is schematically illustrated in Figure \ref{fig:reconstruction}. 


The primary loss function used for reconstruction, 
\begin{align*}
\mathcal{L}_{\text{Recon}} = & \; \alpha \cdot \mathcal{L}_{\text{KL}} 
+ \alpha' \cdot \mathcal{L}_{\text{KL}}^{\text{pert}}
+ \beta \cdot \mathcal{L}_{\text{CE}} 
+ \beta' \cdot \mathcal{L}_{\text{CE}}^{\text{pert}} \\
& + \gamma \cdot \mathcal{L}_{\text{Cosine}} 
+ \delta \cdot \mathcal{L}_{\text{Ortho}} \\
& + \eta_1 \cdot \mathcal{L}_{\text{Var}} 
+ \eta_2 \cdot \mathcal{L}_{\text{Pix}} 
+ \eta_3 \cdot \mathcal{L}_{\text{Grad}}
\end{align*}

where \( \mathcal{L}_{\text{KL}} \) is the KL Divergence loss, \( \mathcal{L}_{\text{CE}} \) is the Cross Entropy loss, \( \mathcal{L}_{\text{Cosine}} \) is the Cosine Similarity loss, and \( \mathcal{L}_{\text{Ortho}} \) is the Feature Orthogonality loss. The hyperparameters \( \alpha, \beta, \gamma, \delta \) control the contribution of each individual loss term defined as:
\[
\mathcal{L}_{\text{KL}} = D_{\text{KL}}(P \| Q) = \sum_{i} P(i) \log \frac{P(i)}{Q(i)}
\]
\[
\mathcal{L}_{\text{CE}} = -\sum_{i} y_{i} \log(\hat{y}_{i})
\]
\[
\mathcal{L}_{\text{Cosine}} = \frac{1}{N(N-1)} \sum_{i \neq j} \cos(\theta_{ij})
\]
\[
\mathcal{L}_{\text{Ortho}} = \frac{1}{N^2} \sum_{i, j} (G_{ij} - \delta_{ij})^2
\]
where \( D_{\text{KL}} \) represents the KL Divergence between the input distribution \( P \) and the output distribution \( Q \), \( y_{i} \) is the set encoded label, \( \hat{y}_{i} \) is the predicted label from the classifier, \( \cos(\theta_{ij}) \) represents the cosine similarity between features of generated images \( i \) and \( j \), \( G_{ij} \) is the element of the Gram matrix, and \( \delta_{ij} \) is the Kronecker delta function. \( N \) is the number of feature vectors in the batch.

Further to take reconstruction into account we also use \(\mathcal{L}_{\text{KL}}^{\text{pert}}\) and \(\mathcal{L}_{\text{CE}}^{\text{pert}}\) that represent the KL divergence and cross-entropy losses applied on perturbed images, weighted by \( \alpha'\) and  \(\beta' \)respectively while \(\mathcal{L}_{\text{Var}}\), \(\mathcal{L}_{\text{Pix}}\) and \(\mathcal{L}_{\text{Grad}}\) represent the variational loss, Pixel Loss and penalty on gradient norm each weighted by \( \eta_1\), \( \eta_2\), and \(\eta_3\) respectively and defined for an Image \(I\) as:
\begin{align*}
\mathcal{L}_{\text{Var}} = \frac{1}{N} \sum_{i=1}^{N} \Bigg( \sum_{h,w} & \Big( ( I_{i, h+1, w} - I_{i, h, w} )^2 \\
& + ( I_{i, h, w+1} - I_{i, h, w} )^2 \Big) \Bigg)
\end{align*}
\[
\mathcal{L}_{\text{Grad}} = \left\| \nabla_{\theta} L(f_{\theta}(I), y) \right\|
\]

\[
\mathcal{L}_{\text{Pix}} = \sum \max(0, -I) + \sum \max(0, I - 1)
\]

By integrating these loss components, the generator is trained to produce samples that closely resemble the training data, thus revealing the extent of memorization within the classifier.


In this section, we present the experimental results obtained by applying the reconstruction technique on the MNIST \cite{deng2012mnist}, FashionMNIST \cite{xiao2017fashionmnistnovelimagedataset}, SVHN, and CIFAR-10 \cite{cf} datasets. Our goal is to evaluate the ease of reconstruction and the extent of memorization in different vision classifier architectures by training a generator to produce images that resemble the training data. The classifier is first trained normally on a given dataset and then held in evaluation mode for the purpose of reconstruction. The conditioned generator, which takes as input latent vectors and conditioning information, is trained to generate images that the classifier maps to specific labels.

We evaluate three vision classifier architectures: Multi-Layer Perceptrons (MLPs), Convolutional Neural Networks (CNNs), and Vision Transformers (ViTs). We implement a 5-layer MLP with Batch Normalization, Leaky ReLU activations, and Dropout layers \cite{JMLR:v15:srivastava14a} to mitigate memorization tendencies. The CNN consists of three convolutional layers followed by batch normalization \cite{pmlr-v37-ioffe15}, dropout layers, and a fully connected layer for classification. The ViT classifier uses a transformer-based architecture with three self-attention layers, each containing four attention heads while the input images are divided into non-overlapping patches of size \(4 \times 4\).

The generator, instead of traditional label embeddings, follows a Vector-Matrix Conditioning approach to ensure diverse and structured image generation. The class labels are encoded into random softmaxed vectors concatenated with the latent vector, followed by multiple layers of transposed convolutions, batch normalization, and dropout layers to encourage diversity in generated images. Once the concatenated latent and conditioning vectors are upsampled to \(N \times N\) spatial dimensions for an \(N\)-class classification task, they are concatenated with a conditioning matrix for further generation up to the required image size (\(28 \times 28\) for MNIST and FashionMNIST, and \(32 \times 32\) for SVHN and CIFAR-10).

To assess the extent of memorization and the relative ease of reconstruction across different architectures, we use the Structural Similarity Index Measure (SSIM). The goal was to evaluate how different architectures handle privacy risks by comparing the quality of reconstructed images. SSIM quantifies the perceptual similarity between two images based on luminance, contrast, and structural similarity components. In this study, SSIM is used to compare the reconstructed images with their closest matches from the training dataset. Higher SSIM values indicate stronger resemblance to training samples, suggesting a greater privacy risk due to increased memorization by the classifier.

Table~\ref{tab:ssim_values} presents the SSIM scores between reconstructed samples and training data, averaged across all classes for each dataset and architecture. Higher SSIM values suggest greater memorization and weaker privacy preservation, as the reconstructed samples strongly resemble real training data.

\begin{table}[h]
\centering
\caption{SSIM values for reconstructed samples across different architectures and datasets.}
\label{tab:ssim_values}
\resizebox{\linewidth}{!}{
\begin{tabular}{|l|c|c|c|}
\hline
\textbf{Dataset}       & \textbf{MLP} & \textbf{ViT} & \textbf{CNN} \\ \hline
\textbf{MNIST}         & 0.83         & 0.78         & 0.73         \\ \hline
\textbf{FashionMNIST}  & 0.74         & 0.64         & 0.63         \\ \hline
\textbf{SVHN}          & 0.71         & 0.68         & 0.69         \\ \hline
\textbf{CIFAR-10}      & 0.65         & 0.62         & 0.58         \\ \hline
\end{tabular}
}
\end{table}

From Table~\ref{tab:ssim_values}, we observe that MLPs exhibit the highest SSIM scores across all datasets, suggesting that they retain more training data details than CNNs and ViTs. This is likely due to their fully connected nature and lack of spatial inductive biases, leading to higher memorization tendencies. Also individual pixels in the images have dedicated weights associated, making memorization easier.

ViTs generally show lower SSIM values than MLPs but remain slightly higher than CNNs, indicating that self-attention mechanisms contribute to retaining finer details in the reconstructed images. Unlike CNNs, ViTs lack pooling layers, which means that more spatial information is preserved, making inversion attacks more feasible. 

CNNs show the lowest SSIM values across all datasets, suggesting that they inherently discard more specific input details due to weight sharing, local receptive fields, and pooling operations. This abstraction reduces direct memorization and makes CNNs relatively more privacy-preserving compared to MLPs and ViTs.

\begin{figure}[ht]
\centering
\includegraphics[width=0.95\linewidth]{riz.png} 
\caption{Comparison of reconstructed samples across MLP, ViT, and CNN architectures. The first column represents actual training samples, while subsequent columns show corresponding reconstructed images.}
\label{fig:sam}
\end{figure}

Figure~\ref{fig:sam} presents a qualitative comparison of reconstructed samples across different classifier architectures. The first column depicts actual training samples, while the subsequent columns display reconstructed images generated from MLPs, ViTs, and CNNs. Notably, MLP-generated reconstructions appear most similar to the original samples, while CNNs yield more abstract representations, indicating lower memorization.

These results highlight the variation in privacy risks across architectures, with MLPs being the most susceptible to memorization, followed by ViTs, and CNNs exhibiting the lowest reconstruction fidelity. The findings emphasize the importance of architectural choices in designing privacy-aware models, where CNNs may be preferable for applications requiring privacy preservation.

\section{Conclusion and Future Work}
In this paper, we systematically evaluated the privacy-preserving properties of vision classifiers by analyzing the extent of memorization and the ease of training data reconstruction across different architectures. Our experimental results, quantified through SSIM scores, indicate that MLPs tend to memorize more information about training samples compared to CNNs and ViTs, making them more prone to reconstruction. These findings highlight the crucial role of architectural choices in mitigating privacy risks, emphasizing the need for privacy-aware model deployment, especially in sensitive applications.

Future research can extend this study by exploring additional factors that influence privacy leakage, such as model depth, dataset complexity, and the impact of regularization techniques. In the case of ViTs it would be also of interest to see how the reconstructions differ with varying patch size. Investigating the impact of differential privacy mechanisms in reducing reconstruction risks could provide valuable insights into enhancing privacy.

\bibliographystyle{IEEEtran}  % Or any other preferred style
\bibliography{references} 

\end{document}


% \vspace{12pt}

\end{document}
