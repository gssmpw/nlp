\section{Related Work}

Recent advancements in deep learning and computer vision have revolutionized weed detection and classification in precision agriculture. Researchers have developed various approaches to address the challenges associated with accurate and efficient weed identification in diverse crop environments.
Object detection models have shown promising results in weed identification. Hasan et al. (2024) \cite{Hasan2024-su} created a dataset of 5,997 images featuring corn and four weed species, demonstrating that YOLOv7 achieved the highest mean average precision (mAP) of 88.50\%, further improved to 89.93\% with data augmentation. Wang et al. (2024) \cite{Wang2024-tt} proposed the CSCW-YOLOv7 model for weed detection in wheat fields, achieving superior precision (97.7\%), recall (98\%), and mAP (94.4\%).
Transfer learning has proven effective for weed species detection. Shackleton et al. (2024) \cite{Shackleton2024-uc} evaluated seven pre-trained CNN models for rangeland weed detection, with EfficientNetV2B1 achieving the highest accuracy of 94.2\%. Ahmad et al. (2021) \cite{Ahmad2021-gs} employed various models for image classification and object detection in corn and soybean systems, with VGG16 achieving 98.9\% accuracy and YOLOv3 reaching 54.3\% mAP.
Traditional machine learning algorithms have also been applied to weed detection. Islam et al. (2021) \cite{Islam2021-iy} compared Random Forest, Support Vector Machine, and k-Nearest Neighbors for weed detection in chilli pepper fields, with RF and SVM achieving 96\% and 94\% accuracy, respectively.
Semantic segmentation approaches have shown promise. Khan et al. (2020) \cite{Khan2020-xb} introduced CED-Net, outperforming traditional models like U-Net and SegNet. Arun et al. (2020) \cite{Arun2020-eu} developed a Reduced U-Net architecture, achieving 95.34\% segmentation accuracy on the CWFID dataset.
Autonomous weeding applications have benefited from deep learning. Adhikari et al. (2019) \cite{Adhikari2019-os} proposed ESNet for autonomous weeding in rice fields, utilizing semantic graphics for data annotation. Teimouri et al. (2018) \cite{Teimouri2018-mx} developed a method to classify weeds into nine growth stages, achieving a maximum accuracy of 78\% for Polygonum spp.
Ensemble learning frameworks have been introduced to improve detection under varied field conditions. Asad et al. (2023) \cite{Asad2023-zv} proposed an approach using diverse models in a teacher-student configuration, significantly outperforming single semantic segmentation models. Moldvai et al. (2024) \cite{Moldvai2024-hb} explored weed detection using multiple features and classifiers, achieving a 94.56\% recall rate with limited data.
\\
Despite these advancements, several limitations persist in existing research. These include the need for larger and more diverse datasets \cite{Gallo2023-zs} \cite{Asad2020-wj}, class imbalance issues \cite{Hasan2024-su}, and computational complexity \cite{Asad2023-zv}. Most studies focus on a limited number of weed species \cite{Moldvai2024-hb} \cite{Ahmad2021-gs} and growth stages, which may not fully represent real-world agricultural settings.
Our research addresses these limitations by creating a comprehensive dataset of 203,567 images featuring 16 common and troublesome weeds in USA agriculture, capturing their full 11-week growth cycle. We implement and adapt state-of-the-art object detection models, DETR and RetinaNet, for weed growth identification. Through a comprehensive comparison of model results, we provide practical insights for weed management in precision agriculture.
This work distinguishes itself by focusing on a large-scale, diverse dataset, considering multiple weed species and growth stages, and offering practical recommendations for farmers on the most effective model for weed detection in real-world scenarios. By addressing the limitations of previous studies, our research contributes significantly to the field of weed identification and management in precision agriculture.