\section{Conclusion}
This research marks a pivotal advancement in precision agriculture by demonstrating the effectiveness of AI models, particularly RetinaNet, in weed detection and classification across various growth stages and species. Our study, conducted on a comprehensive dataset of 203,567 images spanning 16 weed species over 11 weeks, reveals RetinaNet's superior performance with mAP scores of 0.907 and 0.904 on training and test sets, and an inference speed of 7.28 FPS, significantly outpacing DETR's 0.854 and 0.840 mAP scores and 3.49 FPS speed. Both models exhibit improved accuracy with plant maturation, yet challenges persist during the early growth stages (weeks 1-2) due to poor differentiation between emerging plants and soil. These findings underscore the practical implications for weed management, with RetinaNet recommended for real-time applications due to its accuracy and speed. To integrate these models into existing agricultural practices, farmers should implement mobile-based applications for in-field weed detection using RetinaNet, calibrate the model for specific weed species with their growth stages prevalent in their region, and combine AI-driven detection with GPS-guided precision spraying systems. Despite the controlled greenhouse setting and early-stage detection challenges, this study lays the groundwork for future research aimed at enhancing detection accuracy through custom transformer models and expanding the dataset to include real field conditions. These AI-driven innovations hold the promise of revolutionizing weed management by enabling species-specific, growth-stage-aware detection, potentially reducing herbicide use, cutting costs, and minimizing environmental impact. By following these integration guidelines, farmers can leverage AI models to optimize their weed management strategies, leading to more sustainable and efficient agricultural practices.
