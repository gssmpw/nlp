\section{Introduction}
In the vast agricultural landscape of the USA, weed management remains a critical challenge for farmers and agronomists. The diverse climates and fertile soils ideal for crop production also create favorable conditions for a wide variety of weed species \cite{monteiro2022sustainable}. These unwanted plants compete with crops for essential resources such as water, nutrients, and sunlight, potentially leading to significant yield losses and economic setbacks for farmers. Traditional weed control methods often rely on broad-spectrum herbicides or mechanically or labor-intensive removal \cite{ren2024exploring}. However, these approaches can be environmentally harmful, economically inefficient, and increasingly ineffective due to the development of herbicide-resistant weed populations \cite{Gao2024-km}. As such, there is a growing need for more precise, sustainable, and automated weed management strategies.

Recent advancements in computer vision and deep learning have shown promise in addressing this agricultural challenge. Object detection and classification techniques applied to weed identification offer the potential for highly accurate, real-time weed management solutions \cite{Almalky2023-ot}. However, several research gaps persist in this domain, such as (a) limited datasets: most existing studies rely on small datasets or images captured at specific growth stages, failing to capture the dynamic nature of weed development, and (b) lack of diversity: many datasets focus on a limited number of weed species, not reflecting the full range of weeds farmers encounter in real-world scenarios.

The scope of this work addresses these gaps by focusing on 16 weed species of greatest economic concern found commonly across multiple geographies in USA agriculture, tracking their growth from the seedling stage through 11 weeks of development. We created a robust, diverse dataset and implemented advanced object detection models to improve the accuracy and efficiency of weed identification and classification.



\begin{figure}[t]
    % \vspace{-0.2cm}
    \centering
\includegraphics[width=0.5\textwidth,height=5cm]{figure/fig1_soil.jpg}
    \vspace{-0.4cm}
    \caption{Soil preparation and labeling for planting weed seeds in pots inside the greenhouse. (a) shows the prepared pots with soil and pot stakes, (b) displays the close-up of the soil mix used for planting.}
    \label{fig:fig1}
    \vspace{-0.3cm}
\end{figure}

Our research makes several key contributions to the field:
\begin{itemize}
    \item Creation of a unique dataset comprising 203,567 images, capturing the full growth cycle of 16 of the most common and troublesome weed species in USA agriculture.
    \item Meticulous labeling of the dataset, categorized by species and growth stage (week-wise), providing a comprehensive resource for weed identification research.
    \item Implementing the Detection Transformer (DETR) \cite{Carion2020-qn} and RetinaNet \cite{Li2020-lt}, adapting these state-of-the-art object detection architectures for weed identification.
    \item Comprehensive comparison of model results, culminating in evidence-based recommendations for farmers on the most effective model for weed detection in real-world scenarios.
\end{itemize}

This research utilizes DETR and RetinaNet due to their state-of-the-art performance in object detection tasks. DETR introduces a state-of-the-art transformer-based approach, offering end-to-end object detection with potential benefits in handling complex scenes and object relationships. RetinaNet, known for its efficiency and accuracy, employs a focal loss function to address class imbalance issues common in detection tasks. By implementing and comparing these two advanced models, the study aims to evaluate their effectiveness in the specific context of weed detection and classification. This research not only contributes to the growing body of work on AI-assisted agriculture but also provides practical insights for farmers and beyond. By developing more accurate and efficient weed detection systems, we pave the way for precision agriculture techniques that can significantly reduce herbicide use, lower production costs, and minimize environmental impact.

In the following sections, this paper presents related work, followed by a comprehensive outline of the data collection and pre-processing techniques employed. The methodology section describes the steps taken in this research. Subsequently, the models section introduces the implementation and evaluation of DETR and RetinaNet for detecting and classifying 16 weed species at various growth stages. The results section showcases the performance metrics of these models. In conclusion, it summarizes these research findings for the 16 growth stage detection and classification with actionable recommendations for farmers based on the study’s
outcomes.
