 %\documentclass{IEEEcsmag}
  \documentclass[journal,twocolumn]{IEEEtran}
%\documentclass[12pt,draftclsnofoot,onecolumn]{IEEEtran}

\usepackage[colorlinks,urlcolor=blue,linkcolor=blue,citecolor=blue]{hyperref}

\usepackage{upmath}
\usepackage{amsmath}
\usepackage{amssymb}
\usepackage{graphicx}
\usepackage{url}
\usepackage{cite}
\usepackage{mathtools}
\usepackage{cuted}
\usepackage{etoolbox}
\usepackage{subcaption}
\usepackage{tikz}
 \usepackage{diagbox}
\usepackage{multirow}
\let\labelindent\relax
\usepackage{enumitem}
% \setcounter{secnumdepth}{0}
 \newcolumntype{P}[1]{>{\centering\arraybackslash}p{#1}}
\newcommand{\CP}{\color{red}}
\newcommand{\CL}{\color{blue}}
\newcommand{\CB}{\color{black}}
%%%%%%%%%%%  commands for new line at each cell
\usepackage{fourier} 
\usepackage{array}
\usepackage{makecell}
\renewcommand\theadalign{bc}
%%%%%%%%%%%%%%%%%%   for check mark 
\usepackage{pifont}
\newcommand{\cmark}{\ding{51}}
\newcommand{\xmark}{\ding{55}} 
\begin{document}

	%\title{Evolution Toward 6G Wireless Networks: A Resource Management Perspective}
	\title{Towards Vertically Specific Local 6G Networks: State-of-the-Art and Future Directions}
	\author{Mehdi Rasti \thanks{}}
\raggedbottom

	\maketitle	
	\begin{abstract}
	In this article, we first present the vision, key performance indicators, key enabling techniques (KETs), and services of 6G wireless networks. Then, we highlight a series of general resource management (RM) challenges as well as unique RM challenges corresponding to each KET.  The unique RM challenges in 6G necessitate the transformation of existing optimization-based solutions to artificial intelligence/machine learning-empowered solutions. In the sequel, we formulate a joint network selection and subchannel allocation problem for 6G \textit{multi-band} network that provides both further enhanced mobile broadband (FeMBB) and extreme ultra reliable low latency communication (eURLLC) services to the terrestrial and   aerial   users. Our solution highlights the efficacy of multi-band network and demonstrates the robustness of dueling deep Q-learning in obtaining efficient RM solution with faster convergence rate compared to deep-Q network and double deep Q-network algorithms. 
	
	 
	\end{abstract}

\begin{IEEEkeywords}
 6G, resource management, further enhanced mobile broadband (FeMBB), extreme ultra reliable low latency communication (eURLLC), artificial intelligence, machine learning, deep reinforcement learning   
\end{IEEEkeywords} 	
	
	\section{Introduction}
%1- very general about MNO
 %2- specif about 1g to 6g from technology, regulation and bussiness prespectives
 %3-and current question
 %4- intrducing local netwok as a solution to this question and then explain the technology, bussiness models and regulations, for local 6G, 
 %5-and organization of the paper

 A variety of technological, business and regulatory developments have occurred between 1G and 5G mobile communications with varying magnitudes and life-cycles. A new cellular wireless network emerges every ten years relying on new technologies, however disruptive business models and value markets emerge every twenty years, both supported by longer-term life cycle regulations. 
 In mobile communications, the traditional MNO market dominance is challenged by internet giants, and new sharing-based business models emerge where underutilized assets are shared by other stakeholders to make new revenues.  In Europe and globally, major MNOs claim that disruptive competition is already influencing their business, for instance by introduction of MVNOs and OTT application providers. 
 These developments will introduce sharing economy [7] into the mobile communication market and open up new business opportunities to apply mobile connectivity to the various vertical sectors’ specific needs.  
 
 5G beyond and 6G will speed up digitalization coupled with scalable business models extending from connectivity services to deliver various content, context, and commerce platform offerings which will require new regulatory models to govern the mobile business ecosystem [8]. A different viewpoint comes from the massive digitalization of societies and the emergence of new verticals creating new ecosystems and disrupting current business models requiring field-specific regulation changes. 
 %This will open totally new business opportunities and can greatly accelerate digitalization of society. 
 %The evolving next-generation network services are expected to address the need for critical wireless communication for industrial operations, public safety, and critical infrastructure connectivity. Moreover, the global market is primarily driven by the growing need for ultra-reliable low-latency connectivity for Industrial Internet of Things (IIoT) applications, including collaborative robots, industrial cameras, and industrial sensors.
 
 Along this digitization pave and in responding to the vertical sector’s specific needs, it is expected that micro operators  (called nowadays Local or private 5G) plays a disruptive role by  offering reliable wireless connectivity in various indoor environments (such as hospitals, malls, sports arenas, campuses and factories) with locally optimized service and high and heterogeneous data traffic supported by new technology enablers being developed for 5G and beyond. A private or non-public 5G network is a dedicated Local Area Network (LAN) that delivers enhanced internet connectivity to industrial, enterprise, and other customers. 
 % The development of local 6G will introduce innovative sharing economy [7] into the mobile communication market and open up new business opportunities to apply mobile connectivity to the various vertical sectors’ specific needs. Local 5G and 6G will speed up digitalization coupled with scalable business models extending from connectivity services to deliver various content, context and commerce platform offerings which will require new regulatory models to govern the mobile business ecosystem [8].
 Potential companies benefiting from local network solutions are network infrastructure manufacturers, network customization & applications providers, network operators, property/facility owners, and a huge number of verticals (manufacturing, logistics, retail, healthcare, schools, agriculture, etc.) amongst others. The global private 5G network market size was valued at USD 1.38 billion in 2021 and is expected to expand at a compound annual growth rate (CAGR) of 49.0\% from 2022 to 2030.
 
The concept of micro operator opens totally new business opportunities and can greatly accelerate digitalization of society, which needs to be supported by innovative business models, and technological as well as regulatory developments.  
  The current dominant spectrum regulatory approach of auctioning a small number of licenses to deploy mobile communication networks with nationwide coverage obligations is no longer sufficient in 5G beyond and 6G. A radical change to this was already suggested in 2016 to offer a large number of local spectrum access rights in higher frequencies [5] to micro operators.  This is aligned with 5G Advanced and 6G targeting at developing mobile technologies at higher and higher spectrum bands where MmWave bands and even sub-THz bands and VLC are of interest. When systems operate at higher spectrum bands, local networks start to make even more sense as the communication link ranges are drastically dropping. In particular when operating in indoor environments, private networks can achieve total isolation and adjacent buildings networks can utilize the same frequency bands, by customizing and revisiting the current resource allocation and interference management schemes. 
  %This also raises technical challenges including resource allocation and interference management for addressing which the current schemes need to be revisited. 
 
 This paper aims to shed light on the key question of how the current mobile network operator (MNO) dominated market could and should be renewed for smooth entry of new players and fresh investments speeding up digitalization and improving user experience with locally optimized service offerings supported by new key technology enablers being developed for 5G beyond and 6G. To do so, we first review the concept of local 6G and its applications, as well as the state-of-the-art from technology, business, and regulation perspectives. Then we envisions and highlights both the opportunity and the challenges with respect to future local 6G from technology, business, and regulation perspectives, followed by future research directions, architecture for local 6G, and ...

 




 %6G key enabling technologies to future disrupting businesses satisfying UN SDGs and enabled by future regulations 
	



%technology evolution from 1G to 5G
%Business evolution from 1G to 5G
%Regulation evolution from 1G to 5G

%Between 1G and 5G, there has been a technological, business, and regulatory evolution of varying magnitudes and life cycles. Every ten years a new cellular wireless networks emerge relying on new technologies, however, the disruptive business models and value markets are created every twenty years, which both supported by last-longing life cycle regulation (The new regulations and polices (spectrum allocation, market design!, spectrum and infrastructure sharing polices, new bands allocation, MVNO regulation ...) are slower than the revolution of technologies and business models.)

   \section{Concept, Applications, and State-of-the-Art}
   § What is the definition of local 6g?
Local in what sense? Range of wireless coverage, local spectrum access, local city, ….
its difference and overlap with MVNO (is the only difference is with spectrum license which MVNO has no spectrum license, but uOs have although locally), can we imagin that MVNO get  spectrum access from MNOs
General Concept  only for a local environment! 

			§ What is the definition of local 6g?
Local in what sense? Range of wireless coverage, private to a vertical but can be spanned globally, local business, local regulation and local coverage, local city, …
			§ History, 
			§ Terminology, Local, Private networks, ...
   its relation with IoT and cyber physical system 
		§ Its use cases and its Applications
  new KPI

\section{Future Local 6G Networks}
This section envisions and highlights both the opportunities and challenges of local 6G with respect to business, regulation, and technology. 

To meet the requirements for digitization of various industry, 6G will be highly application and vertical-oriented. This has three potential impact on business, that is (i) not only improve the efficiency of industry which impact directly and indirectly on their business,   but also (ii) bring various new  inter and intra sector business  into play, and also (iii) specifically change future MNOs and uOs business models offering them share from interlinked service and network provisioning as well as monetizing the URLLC and mMTC services offering to various APs to meet their specific needs. 
example: designing energy-sector specfic 6G causes that (ii) the current integrated energy systems benifit and handle the uncertainty and .... to meet the sustainability target (such as co2) and new investment of expanding the energy network no longer needed,  (ii) bring energy aggregator, data trading models (data of electrifiefd vehicles with smart grid), (ii) MNOs provide various 6G services, ITS infrastructure and platform provide provide platform software for at the edge....or providing joint communication, control and computation service for it ...> extended business models, not only data provisioning bust also helping essential service provisioning like AI and joint co3

The influencing factors are as follows:
vertical-specific,
softwariziation and virtualization, and x as a service
joint co3,
metaverse,
decentralization, based on web3 and web4, 


The potential efficiency improvement for these B2B applications
brings very different businesses into play.

%A variety of technological, business, and regulatory developments have occurred between 1G and 5G with varying magnitudes and life-cycles. A new cellular wireless network emerges every ten years relying on new technologies, however disruptive business models and value markets emerge every twenty years, both supported by long-term life cycle regulations. New regulations and policies (spectrum allocation, market design!, spectrum and infrastructure sharing policies, new band allocation, MVNO regulation,...) are more slowly developed than the revolution in technology and business models.)

\subsection{Key Enabling Technology}
In the coming years, local and private networks will emerge at a faster pace than ever before due to new key enabling technologies developed for 6G networks. To achieve the aforementioned KPIs, the physical (PHY) and medium access (MAC) layer of local 6G will benefit from the  KETs that include
\textbf{(i)} terahertz (THz) and visible light communications (VLC) ranging from 0.1--10 THz (licensed) and  400--800  THz (unlicensed), respectively,
\textbf{(ii)} ultra-massive spatially modulated MIMO (UM-SM-MIMO),
\textbf{(iii)} reconfigurable intelligent surface (RIS), 
\textbf{(iv)} {U}nderwater-{T}errestrial-{A}ir-{S}pace integrated {Net}works (UTASNet), 
\textbf{(v)} dynamic network slicing, virtualization, and
\textbf{(vi)} variants of non-orthogonal multiple access (NOMA) including delta-OMA (D-OMA)and rate splitting multiple access (RSMA) \cite{6G-Tech}

Among the 6G key enabling technologies, THz, VLC and UM-SM-MIMO are the ones most suited to and closely aligned with private and local connectivity.

Among the 6G key enabling technologies, THz, VLC, and UM-SM-MIMO are the key technologies that are more efficient and well-suited (and possibly specific) to the local connectivity required for local 6G networks. The other key enabling technologies (including dynamic network slicing, virtualization, RIS, NOMA, and RSMA), however, can be used both locally and non-locally. These key technologies, when used in local 6G networks, raise new specific challenges (in addition to general RA problems), in terms of radio resource allocation, PHY, and MAC design, which require to be well addressed. The  challenges specific to local 6G networks include inter and intra cell (or inter and intra local-network) interference management through devicing efficient beamforming, power control, channel allocation, BS assignment schemes. 

The challenges and research directions from these KET asre as follows:
Specific challenges:
General Challenges:
Other challenges: When combined with joint communication, computation and control, or digital twin, or meta verse, ARVR, ....




\subsection{Applications and Business Drivers}
In 1G and 2G, the revenues of MNOs cam from  mobile voice and message-based services in a monopolized market, whereas in 3G and 4G, which offered broadband service, MNO revenues were dominated by data usage increased by OTT applications sharing the market with MNOs. Also, until the advent of 5G, most MVNOs' revenues were derived from B2C, via consumer-driven subscriptions (from consumers or consumer-driven mobile broadband services offered to businesses). By ever-increasing wave of digitalization in various verticals, 5G advanced and 6G are expected to grow into B2B applications, offering a wide range of opportunities for improving efficiency and enhancing existing business processes, some of which can already be solved with the help of 4G without the need for business model innovations. In spite of this, a larger number of businesses will put to use the capabilities and characteristics of key enabling 5G advanced and 6G technologies in order to develop new business models which need to be supported and accelerated by innovative regulations and policies.  


Potential companies benefiting from local network solutions are network infrastructure manufacturers, network customization & applications providers, network operators, property/facility owners and a huge number of verticals (manufacturing, logistics, retail, healthcare, schools, agriculture, etc.) amongst others. Property owners will be part of the value chain with dedicated networks customized for their own needs: hospitals, schools, shopping malls, factories, office building, private households etc. can have different offerings and capabilities of networks in the future. Smart campus concepts will start to evolve very fast once the local networks solutions lay down the platform developing the necessary applications and services.


new use-case and new business models, novel value creation and value capture, new innovation opportunity 


  Over-The-Top services are the applications and services which are accessible
over the internet and ride on top of Telecom Operator’s networks offering
internet access services e.g. search engines, social networks, video and
messaging services etc. The well-known OTT services are WhatsApp, Hike,
Snapchat, Skype, YouTube, Viber, Facebook, E-Commerce sites likeAmazon,
online video games, Taxi aggregation services like Uber, etc.

The traditional income model of
the operators, based on subscriptions and metered services, mainly voice and
messaging is failing.

Current OTT applications are mostly within B2C model, however by future local 6G, not only new novel B2C OTT applications but also a variety of B2B OTT ones significantly emerge and create new markets in health, energy, automobile and ...sectores. 

\subsection{Regulation Perspective}
%Today’s the dominant spectrum regulatory approach of auctioning a small number of licenses to deploy mobile communication networks with nationwide coverage obligations is no longer sufficient in 5G and beyond where higher frequency bands are targeted. A radical change to this was already suggested in 2016 [] via a so-called micro operator (uO) concept to establish local small cell indoor networks for tailored service delivery to disrupt the future 5G mobile communication business ecosystem. 

%The developed uO concept complements the MNO offerings and has its basis in the on-going regulatory and technical developments towards local spectrum access rights in higher frequencies [5] and flexible network implementations for network virtualization [6]. This is aligned with 5G Advanced and 6G targeting at developing mobile technologies at higher and higher spectrum bands where MmWave bands and even sub-THz bands and VLC are of interest. When systems operate at higher spectrum bands, local networks start to make even more sense as the communication link ranges are drastically dropping. In particular when operating in indoor environments, private networks can achieve total isolation and adjacent buildings networks can utilize the same frequency bands. 


The most successful market area so far for local 5G has been Japan where e.g. Nokia opened Local 5G Lab facilities in Tokyo in May 2022 to facilitate deployment of local network solutions. Another prominent area, Germany is also strongly pushing 5G campus networks as new openings for 5G related business. In general, year 2022 is seen as the first year of larger scale local 5G networks deployment, although 4G still has room for local deployments also. Finnish national regulatory authority Traficom has been the pioneer in Europe making licensed spectrum available for local networks. New regulation for local 4G/5G networks in Finland allocated 2300-2320 MHz and 24.25-25.1 GHz for local 4G/5G networks. Secondary use of MNO 3410 – 3800 MHz bands is encouraged via MNO leasing obligation if service is not provided by a MNO in the particular area. In the European level technical conditions for 3800 - 4200 MHz are worked upon for local spectrum use in CEPT (46 member states) for private and public use. Here also dynamic spectrum management may come into play, subject to incumbent protection needs.

status about higher frequencies:

\section{Future Research Directions}
\section{Architecture and Standardization for Local 6G}




\section{Example: joint communication and transportation?}

\section{Conclusions}


%\section{Architecture and Standardization}
%\section{Business Models}
\end{document}