%AL: Section 2; Section 3.

%%%% ijcai25.tex

\typeout{IJCAI--25 Instructions for Authors}

% These are the instructions for authors for IJCAI-25.

\documentclass{article}
\pdfpagewidth=8.5in
\pdfpageheight=11in

% The file ijcai25.sty is a copy from ijcai22.sty
% The file ijcai22.sty is NOT the same as previous years'
\usepackage{ijcai25}

% Use the postscript times font!
\usepackage{times}
\usepackage{soul}
\usepackage{url}
\usepackage[hidelinks]{hyperref}
\usepackage[utf8]{inputenc}
\usepackage[small]{caption}
\usepackage{graphicx}
\usepackage{amsmath}
\usepackage{amsthm}
\usepackage{booktabs}
\usepackage{algorithm}
\usepackage{algorithmic}
\usepackage[switch]{lineno}
\usepackage{wrapfig}
\usepackage{multicol, multirow}
\usepackage{stfloats}
\usepackage{subcaption}
\usepackage{lipsum}

% Comment out this line in the camera-ready submission
% \linenumbers

\urlstyle{same}

\newcommand{\lirpa}{\textsf{\footnotesize auto\_LiRPA} }
\newcommand{\gurobi}{\textsf{\footnotesize Gurobi} }

% the following package is optional:
%\usepackage{latexsym}

% See https://www.overleaf.com/learn/latex/theorems_and_proofs
% for a nice explanation of how to define new theorems, but keep
% in mind that the amsthm package is already included in this
% template and that you must *not* alter the styling.
\newtheorem{example}{Example}
\newtheorem{theorem}{Theorem}

% Following comment is from ijcai97-submit.tex:
% The preparation of these files was supported by Schlumberger Palo Alto
% Research, AT\&T Bell Laboratories, and Morgan Kaufmann Publishers.
% Shirley Jowell, of Morgan Kaufmann Publishers, and Peter F.
% Patel-Schneider, of AT\&T Bell Laboratories collaborated on their
% preparation.

% These instructions can be modified and used in other conferences as long
% as credit to the authors and supporting agencies is retained, this notice
% is not changed, and further modification or reuse is not restricted.
% Neither Shirley Jowell nor Peter F. Patel-Schneider can be listed as
% contacts for providing assistance without their prior permission.

% To use for other confereWarning: Ignoring XDG_SESSION_TYPE=wayland on Gnome. Use QT_QPA_PLATFORM=wayland to run on Wayland anyway.
% actions tensor([0, 1, 0, 2])nces, change references to files and the
% conference appropriate and use other authors, contacts, publishers, and
% organizations.
% Also change the deadline and address for returning papers and the length and
% page charge instructions.
% Put where the files are available in the appropriate places.


% PDF Info Is REQUIRED.

% Please leave this \pdfinfo block untouched both for the submission and
% Camera Ready Copy. Do not include Title and Author information in the pdfinfo section
\pdfinfo{
/TemplateVersion (IJCAI.2025.0)
}

\title{A Scalable Approach to Probabilistic Neuro-Symbolic Verification}


% Multiple author syntax (remove the single-author syntax above and the \iffalse ... \fi here)
\author{
Vasileios Manginas$^1$
\and
Nikolaos Manginas$^1$\and
Edward Stevinson$^2$\and
Sherwin Varghese$^2$\and \\
Nikos Katzouris$^1$\and
Georgios Paliouras$^1$\And
Alessio Lomuscio$^2$\\
\affiliations
$^1$National Centre of Scientific Research "Demokritos"\\
$^2$Imperial College London\\
\emails
\{vmanginas, nmanginas, nkatz, paliourg\}@iit.demokritos.gr, \\
\{e.stevinson22, sherwin.varghese, a.lomuscio\}@imperial.ac.uk,
}

%%%%%%% Math definitions %%%%%%%%%%%%%%%%%
\newcommand{\norm}[1]{\left\lVert#1\right\rVert}
\usepackage{amsthm, amsmath, amssymb}
%%%%%%%%%%%%%%%%%%%%%%%%%%%%%%%%%%%%%%%%%%


\begin{document}

\maketitle
    
\begin{abstract}
    \begin{abstract}  
Test time scaling is currently one of the most active research areas that shows promise after training time scaling has reached its limits.
Deep-thinking (DT) models are a class of recurrent models that can perform easy-to-hard generalization by assigning more compute to harder test samples.
However, due to their inability to determine the complexity of a test sample, DT models have to use a large amount of computation for both easy and hard test samples.
Excessive test time computation is wasteful and can cause the ``overthinking'' problem where more test time computation leads to worse results.
In this paper, we introduce a test time training method for determining the optimal amount of computation needed for each sample during test time.
We also propose Conv-LiGRU, a novel recurrent architecture for efficient and robust visual reasoning. 
Extensive experiments demonstrate that Conv-LiGRU is more stable than DT, effectively mitigates the ``overthinking'' phenomenon, and achieves superior accuracy.
\end{abstract}  
\end{abstract}



\section{Introduction}
\section{Introduction}


\begin{figure}[t]
\centering
\includegraphics[width=0.6\columnwidth]{figures/evaluation_desiderata_V5.pdf}
\vspace{-0.5cm}
\caption{\systemName is a platform for conducting realistic evaluations of code LLMs, collecting human preferences of coding models with real users, real tasks, and in realistic environments, aimed at addressing the limitations of existing evaluations.
}
\label{fig:motivation}
\end{figure}

\begin{figure*}[t]
\centering
\includegraphics[width=\textwidth]{figures/system_design_v2.png}
\caption{We introduce \systemName, a VSCode extension to collect human preferences of code directly in a developer's IDE. \systemName enables developers to use code completions from various models. The system comprises a) the interface in the user's IDE which presents paired completions to users (left), b) a sampling strategy that picks model pairs to reduce latency (right, top), and c) a prompting scheme that allows diverse LLMs to perform code completions with high fidelity.
Users can select between the top completion (green box) using \texttt{tab} or the bottom completion (blue box) using \texttt{shift+tab}.}
\label{fig:overview}
\end{figure*}

As model capabilities improve, large language models (LLMs) are increasingly integrated into user environments and workflows.
For example, software developers code with AI in integrated developer environments (IDEs)~\citep{peng2023impact}, doctors rely on notes generated through ambient listening~\citep{oberst2024science}, and lawyers consider case evidence identified by electronic discovery systems~\citep{yang2024beyond}.
Increasing deployment of models in productivity tools demands evaluation that more closely reflects real-world circumstances~\citep{hutchinson2022evaluation, saxon2024benchmarks, kapoor2024ai}.
While newer benchmarks and live platforms incorporate human feedback to capture real-world usage, they almost exclusively focus on evaluating LLMs in chat conversations~\citep{zheng2023judging,dubois2023alpacafarm,chiang2024chatbot, kirk2024the}.
Model evaluation must move beyond chat-based interactions and into specialized user environments.



 

In this work, we focus on evaluating LLM-based coding assistants. 
Despite the popularity of these tools---millions of developers use Github Copilot~\citep{Copilot}---existing
evaluations of the coding capabilities of new models exhibit multiple limitations (Figure~\ref{fig:motivation}, bottom).
Traditional ML benchmarks evaluate LLM capabilities by measuring how well a model can complete static, interview-style coding tasks~\citep{chen2021evaluating,austin2021program,jain2024livecodebench, white2024livebench} and lack \emph{real users}. 
User studies recruit real users to evaluate the effectiveness of LLMs as coding assistants, but are often limited to simple programming tasks as opposed to \emph{real tasks}~\citep{vaithilingam2022expectation,ross2023programmer, mozannar2024realhumaneval}.
Recent efforts to collect human feedback such as Chatbot Arena~\citep{chiang2024chatbot} are still removed from a \emph{realistic environment}, resulting in users and data that deviate from typical software development processes.
We introduce \systemName to address these limitations (Figure~\ref{fig:motivation}, top), and we describe our three main contributions below.


\textbf{We deploy \systemName in-the-wild to collect human preferences on code.} 
\systemName is a Visual Studio Code extension, collecting preferences directly in a developer's IDE within their actual workflow (Figure~\ref{fig:overview}).
\systemName provides developers with code completions, akin to the type of support provided by Github Copilot~\citep{Copilot}. 
Over the past 3 months, \systemName has served over~\completions suggestions from 10 state-of-the-art LLMs, 
gathering \sampleCount~votes from \userCount~users.
To collect user preferences,
\systemName presents a novel interface that shows users paired code completions from two different LLMs, which are determined based on a sampling strategy that aims to 
mitigate latency while preserving coverage across model comparisons.
Additionally, we devise a prompting scheme that allows a diverse set of models to perform code completions with high fidelity.
See Section~\ref{sec:system} and Section~\ref{sec:deployment} for details about system design and deployment respectively.



\textbf{We construct a leaderboard of user preferences and find notable differences from existing static benchmarks and human preference leaderboards.}
In general, we observe that smaller models seem to overperform in static benchmarks compared to our leaderboard, while performance among larger models is mixed (Section~\ref{sec:leaderboard_calculation}).
We attribute these differences to the fact that \systemName is exposed to users and tasks that differ drastically from code evaluations in the past. 
Our data spans 103 programming languages and 24 natural languages as well as a variety of real-world applications and code structures, while static benchmarks tend to focus on a specific programming and natural language and task (e.g. coding competition problems).
Additionally, while all of \systemName interactions contain code contexts and the majority involve infilling tasks, a much smaller fraction of Chatbot Arena's coding tasks contain code context, with infilling tasks appearing even more rarely. 
We analyze our data in depth in Section~\ref{subsec:comparison}.



\textbf{We derive new insights into user preferences of code by analyzing \systemName's diverse and distinct data distribution.}
We compare user preferences across different stratifications of input data (e.g., common versus rare languages) and observe which affect observed preferences most (Section~\ref{sec:analysis}).
For example, while user preferences stay relatively consistent across various programming languages, they differ drastically between different task categories (e.g. frontend/backend versus algorithm design).
We also observe variations in user preference due to different features related to code structure 
(e.g., context length and completion patterns).
We open-source \systemName and release a curated subset of code contexts.
Altogether, our results highlight the necessity of model evaluation in realistic and domain-specific settings.






\begin{figure*}[bt]
    \centering
    \includegraphics[width=0.9\textwidth]{figures/1_system_overview_short.pdf}
    \caption{A motivating example for probabilistic NeSy verification. In this autonomous driving example we want to verify two logical constraints $\phi$, a safety-oriented one and a common-sense one, on top of two neural networks accepting the same dashcam image as input. The symbolic constraints are compiled into a tractable representation containing only addition, subtraction, and multiplication. During inference, this is used to reason over the NN outputs and calculate the probability that the constraints are satisfied. For verification, we exploit this structure to scalably compute how perturbations in the input affect the probabilistic output of the whole (NNs + reasoning) NeSy system.}
    \label{fig:system}
\end{figure*}

\section{Background} \label{background}
\subsection{Probabilistic NeSy Systems} \label{p-nesy-systems}
Probabilistic NeSy AI aims to combine perception with probabilistic logical reasoning. We provide a brief overview of the operation of such a system based on \cite{marconato2024bears}. Given input $\boldsymbol{x} \in \mathbb{R}^n$, the system utilizes a NN, as well as symbolic knowledge $\mathrm{K}$, to infer a (multi-)label output $\boldsymbol{y} \in \{0, 1\}^m$. In particular, the system computes $p_\theta(\boldsymbol{y} \,\vert\, \boldsymbol{x}; \mathrm{K})$, where $\theta$ refers to the trainable parameters of the NN. This is achieved in a two-step process. First, the system extracts a set of \textit{latent concepts} $\boldsymbol{c} \in \{0, 1\}^k$, through the use of a parameterized neural model $p_\theta(\boldsymbol{c} \,\vert\, \boldsymbol{x})$. These latent concept predictions are then used as input to a reasoning layer, in conjunction with knowledge $\mathrm{K}$, to infer $p(\boldsymbol{y} \,\vert\, \boldsymbol{c}; \mathrm{K})$.

The setting is straightforward to extend to multiple NNs. In that case, the $i^{\text{th}}$ network from a set $\mathrm{E}$ would predict $p_\theta^i(\boldsymbol{c^i} \,\vert\, \boldsymbol{x})$, with $\bigcup_{i \in \mathrm{E}} \boldsymbol{c}^i = \boldsymbol{c}$. Consider the running example of Figure \ref{fig:system}, where two NNs accept the same image as input and output two disjoint sets of latent concepts. These are then combined to form the input to the reasoning layer in order to output the target $\boldsymbol{y}$.

\subsection{Knowledge Compilation} \label{kc}
Probabilistic reasoning in NeSy systems is often performed via reduction to Weighted Model Counting (WMC), which we briefly review next. Consider a propositional logical formula $\phi$ over variables $V$. Each boolean variable $v\in V$ is assigned a weight $p(v)$, which denotes the probability of that variable being true. The Weighted Model Count (WMC) of formula $\phi$ is then defined as:

\begin{equation} \label{eq:wmc}
    \mathrm{WMC (\phi)} = \sum_{\omega \models \phi} \quad \prod_{v \in \omega} p(v) \prod_{v \notin \omega} 1 - p(v).
\end{equation}
In essence, the WMC is the sum of the probability of all worlds $\omega$ that are models of $\phi$. WMC is $\#\mathrm{P}$-hard, since it generalizes a $\#\mathrm{P}$-complete problem, $\#\mathrm{SAT}$, by incorporating weights \cite{chavira2008probabilistic}.

A widely-used approach for solving the WMC problem is \textit{knowledge compilation} (KC) \cite{darwiche2002knowledge,chavira2008probabilistic}. According to this approach, the formula $\phi$ is first compiled into a tractable representation, which is used at inference time to compute a large number of queries - in our case, instances of the WMC problem - in polynomial time. KC techniques push most of the computational effort to the ``off-line'' compilation phase, resulting in computationally cheap ``on-line'' query answering, a concept termed \textit{amortized inference}.

% The idea to shift the majority of the computation onto the off-line compilation phase in order to gain fast on-line queries is termed \textit{amortized inference}, and it is the central motivation for using knowledge compilation. 

The representations obtained via KC take the form of computational graphs, in which the literals, i.e., logical variables and their negations, are found only on leaves of the graph. The nodes are only logical $\mathrm{AND}$ and $\mathrm{OR}$ operations, and the root represents the query. For example, consider the constraints $\phi$ from the autonomous driving example of Figure \ref{fig:system}:
\begin{equation*}
\begin{gathered}
\text{red light} \lor \text{car in front} \implies \text{brake} \\
\text{accelerate} \iff \neg \text{brake}
\end{gathered}
\end{equation*}
\noindent These dictate that (1) if there is a red light or a car in front of the AV, then the AV should brake, and (2) that accelerating and braking should be mutually exclusive and exhaustive, i.e., only one should take place at any given time. Figure \ref{fig:sdd} presents the compiled form of these constraints as a boolean circuit, namely a Sentential Decision Diagram (SDD) \cite{darwiche2011sdd}.
\begin{figure}[!h]
    \centering
    \begin{subfigure}[t]{0.5\linewidth}
        \centering
        \includegraphics[width=0.95\linewidth]{figures/2_sdd.pdf}
        \caption{}
        \label{fig:sdd}
    \end{subfigure}%
    ~ 
    \begin{subfigure}[t]{0.5\linewidth}
        \centering
        \includegraphics[width=0.95\linewidth]{figures/2_ac.pdf}
        \caption{}
        \label{fig:ac}
    \end{subfigure}
    \caption{(a) A Sentential Decision Diagram (SDD) as an example of a computational graph obtained via knowledge compilation and (b) the corresponding arithmetic circuit (AC) derived from the SDD during inference by replacing the AND/OR nodes with multiplication/addition. The SDD has been minimized for conciseness.}
    \label{fig:circuit}
\end{figure}
To perform WMC, the boolean circuit is replaced by an arithmetic one, by replacing the $\mathrm{AND}$ nodes of the graph with multiplication, the $\mathrm{OR}$ nodes with addition, and the negation of literals with subtraction ($1-x$). The resulting structure, shown in Figure \ref{fig:ac}, can compute the WMC of $\phi$ simply by plugging in the literal probabilities at the leaves and traversing the circuit bottom-up. Indeed, one can check that assuming the probabilities:
\begin{alignat*}{2}
p(\text{accelerate}) &= 0.3, \quad p(\text{red light}) &&= 0.6 \\ 
p(\text{brake}) &= 0.7, \quad p(\text{car in front}) &&= 0.8
\end{alignat*}

\noindent this computation correctly calculates the probability of $\phi$ by summing the probability of its 5 different models.


% is an algebraic computational graph containing only subtractions (for negation), additions (for disjunction), and multiplications (for conjunction), that can be used to perform probabilistic logical reasoning.

% In such systems, probabilistic inference is commonly performed by solving the Weighted Model Counting (WMC) problem. 

% The probability of a world is computed as a product over all variables, where positive variables are assigned their probability and negative variables are assigned one minus their probability. 

% rooted directed acyclic graph (DAG)

\subsection{Verification of Neural Networks} \label{verification_concepts}
% Objectives
% \begin{itemize}
%     \item \textbf{1 paragraph} Introduce abstract/deductive and complete/sound verification. Mention that exact NN verification is NP-hard.
%     \item \textbf{1-2 paragraphs} Abstraction-based NN verification techniques: IBP/SIP/RSIP etc. Somewhere here we need to clearly say that these methods allow us to obtain probability \textit{ranges} at the NN outputs rather than probability \textit{values}. Also, we need to note how these bounds are used to calculate robustness.
% \end{itemize}

% 1. What is verification
% 2. Complete
% 3. Incomplete

\paragraph{NN Robustness.}
Verifying the robustness of NN classifiers amounts to proving that the network's correct predictions remain unchanged if the corresponding input is perturbed within a given range $\epsilon$ \cite{Wong+18}. Contrary to empirical machine learning evaluation techniques, NN verification methods reason over infinitely-many inputs to derive certificates for the robustness condition. For a given network $f$, this is formalized as follows: for all inputs $\boldsymbol{x}$, such that $f(\boldsymbol{x})$ is a correct prediction, and for all $\boldsymbol{x}'$, such that $\left\| \boldsymbol{x} - \boldsymbol{x}' \right\| \leq \epsilon$, it holds that $f(\boldsymbol{x}) = f(\boldsymbol{x}')$. 
    
Checking if the robustness condition holds for some $\epsilon$ can be achieved by reasoning over the relations between the network's un-normalized predictions (logits) at the NN's output layer. In particular, it can be seen that if for any $\boldsymbol{x}'$ in an $\epsilon$-ball of $\boldsymbol{x}$, it holds that $\mathit{y_{true} - y_{i}} > 0$, for all $y_i \neq y_{true}$, then the network is robust for $\epsilon$~\cite{gowal2018ibp}. Here $\mathit{y_{true}}$ is the logit corresponding to the correct class and $y_i$ are the logits corresponding to all other labels. This condition can be checked by computing the minimum differences of the predictions for all points in the $\epsilon$-ball. If that minimum is positive, the robustness condition is satisfied. However, finding that minimum is NP-hard \cite{katz2017reluplex}. 

% The area of Verification of Neural Networks~\cite{liu2020algorithms} aims to provide determininstic guarantees of  a network output with respect to a given specification, e.g., robustness to noise, colour changes, etc. Formally, a neural network with parameters $\theta$, denoted by $f_\theta$ is said to be (locally) robust at an input point $x_0$ if no point (or adversarial example) $x_0'$ exists in the neighbourhood causing the prediction of the network to change from its target label \cite{gowal2018ibp-not-the-correct-ref}. For classification problems, this amounts to ensuring that the output class predicted by a network does not change for small input perturbations within a budget denoted by $\epsilon$ \cite{gowal2018ibp}. Input perturbations are usually defined in the $\mathcal{L}_p$-norm function space. Formally, for all inputs set $x$ and perturbed input set $x'$ this is expressed as:
% \begin{equation}
% 	f_\theta(x) = f_\theta(x') \forall x, \norm{x-x'} \leq \epsilon
% \end{equation}
% Robustness verification with respect to input perturbations involves analyzing the unnormalized final layer predictions of the network. Specifically, for all perturbed inputs \(x_i'\) within the perturbation set \(x'\), the network is considered robust if \(y_{\text{true}} - y_j > 0\) holds, where \(y_{\text{true}}\) represents the correct class prediction, and \(y_j\) corresponds to the output for all other classes \cite{gowal2018ibp} and is known for being NP-Hard~\cite{katz2017reluplex}.
%Robustness verification w.r.t.~perturbations over $\mathcal{L}_p$-norm %space within an $\epsilon$ budget entails the inspection of the un-%normalized network's final layer predictions. In other words, for all %perturbed inputs $x_i'$ in the perturbed input set $x'$, if $y_true - %y_j > 0$, where $y_true$ denotes the prediction of the correct class and %$y_j$ denotes the network output for all other classes, then the network %is robust for $\epsilon$ \cite{gowal2018ibp}. This check can be achieved %by computing the minimum differences of the predictions for all points %within the $\epsilon$ region and ensuring that these minimum values are %positive to satisfy the robustness condition. However, finding this %minimum is an NP-hard problem \cite{katz2017reluplex}.

%AL: This requires major re-writing. Alternatively we can just say.
% Robustness verification is known for being NP-Hard~\cite{katz2017reluplex}.


%AL: Imperial MILP references to ad??
%and Semidefinite Programming (SDP), 
%AL: SDP is NOT complete
%AL: MILP is invented at Imperial - shall we use those? Reulplex does not use MILP but SMT.
% Add SMT?


%     Verifying the robustness of NN classifiers amounts to proving that the network's correct predictions remain unchanged if the corresponding input is perturbed within a given range $\epsilon$ \cite{wong2018provable}. Contrary to empirical machine learning evaluation techniques, such as cross-validation, NN verification methods reason over infinitely-many inputs, to derive certificates for the robustness condition, which, for a given network $f$ can be formalized as follows: for all inputs $\boldsymbol{x}$, such that $f(\boldsymbol{x})$ is a correct prediction and for all $\boldsymbol{x}'$ such that $\left\| \boldsymbol{x} - \boldsymbol{x}' \right\| \leq \epsilon$, it holds that $f(\boldsymbol{x}) = f(\boldsymbol{x}')$. 
    
    % Checking if the robustness condition holds for some $\epsilon$ can be achieved by reasoning over the relations between the network's un-normalized predictions (logits) at the NN's output layer. In particular, it can be seen \cite{gowal2018ibp} that if for any $\boldsymbol{x}'$ in an $\epsilon$-ball of $\boldsymbol{x}$, it holds that $\mathit{y_{true} - y_{i}} > 0$, for all $y_i \neq y_{true}$, then the network is robust for $\epsilon$. Here $\mathit{y_{true}}$ is the prediction corresponding to the correct class and the $y_i$'s are the other labels' predictions. In turn, this last condition can be checked by computing the minimum of the predictions' differences for all points in the $\epsilon$-ball and checking if that minimum is positive, in which case the robustness condition is satisfied. However, finding that minimum is NP-hard \cite{Katz+17}. 

% \paragraph{Complete/incomplete Verification.}
% Verification approaches are categorized as either complete or incomplete methods. Complete methods are those that return a definite answer as to whether the property in question is satisfied by reasoning over the exact verification problem. In contrast, incomplete approaches do not reason over an exact formulation of the verification problem and hence are not theoretically guaranteed to solve a problem, and may fail to solve a subset of queries. In return, they can be far less computationally expensive.

% Verification algorithms for neural network robustness are classified into complete and incomplete methods. Complete verification techniques guarantee to find a proof that the specification is true or provide a counterexample if the specification is incorrect. This requires exhaustive search in the worst case \cite{qin2019verification}.

\paragraph{Solver-Based Verification.}
Early verification approaches include Mixed Integer Linear Programming (MILP) \cite{lomuscio2017approach,tjeng2018evaluating,henriksen2020efficient} and Satisfiability Modulo Theories (SMT) \cite{ehlers2017formal,katz2017reluplex}. MILP approaches encode the verification problem as an optimization task over linear constraints, which can be solved by off-the-shelf MILP-solvers. SMT-based verifiers translate the NN operations and the verification query into an SMT formula and use SMT solvers to check for satisfiability. Although these methods are precise and provide exact verification results, they do not scale to large, deep networks, due to their high computational complexity. As such, they are impractical for real-world applications with high-dimensional inputs like images or videos.

\paragraph{Relaxation-Based Verification.}
As the verification problem is NP-hard~\cite{katz2017reluplex}, incomplete techniques that do not reason over an exact formulation of the verification problem, but rather an over-approximating relaxation, are used for efficiency. A salient method that is commonly used is Interval Bound Propagation (IBP), a technique which uses interval arithmetic~\cite{intervalArithmeticSunaga1958} to propagate the input bounds through all the layers of a NN~\cite{gowal2018ibp}. As a non-exact approach to verification, it is not theoretically guaranteed to solve a problem. However, the approach is sound, in that if the lower bound is shown to be positive the network is robust. Therefore, once the bounds of the output layer are obtained, an instance is safe if the lower bound of the logit corresponding to the correct class is greater than the upper bounds of the rest of the logits, since this ensures a correct prediction, even in the worst case.

% In contrast, incomplete verification techniques~\cite{gowal2018ibp} work by approximating the verification queries. They may be able to establish the local robustness of a model, but may be unable to solve the query, returning undefined. Interval Bound Propagation (IBP) is an approximate technique that computes bounds on neuron activations, by propagating intervals through the network layers \cite{gowal2018ibp}. 

% NB EdS: Do we need to mention SDP?
% Semi Definite Programming (SDP) approaches model the verification as an optimisation problem and express the dual representation as a maximum eigenvalue problem \cite{kolter2017provabledefenses,wong2018scalingpa,dvijotham2018ada,bunel2020lagrangiandf}.

% Symbolic Interval Propagation (SIP) refines IBP by using symbolic representations to track dependencies between inputs and intermediate layer activations, improving the tightness of bounds \cite{wang2021beta}. Reverse Symbolic Interval Propagation (RSIP) further improves this by propagating constraints backward from the output to the input, enabling tighter bounds while maintaining efficiency~\cite{gehr2018ai2}. Branch and bound (BaB) based complete verifiers may use an incomplete verifier as a sub-procedure to perform branching on the model input \cite{bunel2018unifiedviewpiecewiselinear} or the ReLU neurons \cite{bunel2018unifiedviewpiecewiselinear,botoeva2020efficient}. IBP is the fastest of the methods above but is the least precise. SIP and RSIP methods enable tighter approximations but, due to their complexity, may be unable to solve verification queries for large models. 

% While complete verifiers provide precise approximations they are computationally expensive to scale and incomplete verifiers scale to a certain extent at the cost of loose approximations.       


\section{Probabilistic Neuro-Symbolic Verification} \label{methodology}
\subsection{Problem Statement} \label{problem_statement}

% We now formally introduce relaxation-based verification in the context of NeSy probabilistic reasoning systems, as a means of solving the robustness verification problem. We provide further intuition by introducing the running example of Figure \ref{fig:system}. We are given a NeSy system which computes $p_{\theta}(\boldsymbol{y} | \boldsymbol{x}, \mathrm{K})$. The goal of our approach is to calculate:

% % The $\epsilon$-robustness of the NeSy system can be then be expressed as:
% $$
% \mathrm{min}_{\mathbf{x'}} \ p_{\theta}(\boldsymbol{y_i} | \boldsymbol{x'}, \mathrm{K}), \quad \mathrm{max}_{\mathbf{x'}} \ p_{\theta}(\boldsymbol{y_i} | \boldsymbol{x'}, \mathrm{K})
% $$
% % p_{\theta}(\boldsymbol{y}_{\text{true}} | \boldsymbol{x'}, 
% % \mathrm{K}) - 
% % p_{\theta}(\boldsymbol{y}_{\text{i}} | \boldsymbol{x'}, 
% % \mathrm{K}) > 0,
% % $$

% for all $\boldsymbol{x'}$ such that 
% $||\boldsymbol{x'} - \boldsymbol{x}|| \leq \epsilon$ and 
% for all $\boldsymbol{y}_{\text{i}} \neq \boldsymbol{y}_{\text{true}}$. 

% That is, we aim to calculate the lower 

% Consider now the NeSy system of the running example of Figure \ref{fig:system}. The neural part of the system is comprised of two neural networks accepting the same dashcam image $\boldsymbol{x}$ as input. The first is an object detection network predicting whether the image contains a red traffic light and whether there is a car in front of the autonomous vehicle (AV). The second is an action selection network which outputs whether to accelerate or brake the AV given the image. The symbolic part of the system is comprised of the conjunction of two constraints, a safety constraint and a common-sense one, as described in Section \ref{kc}. These are compiled into a boolean circuit, and then transformed into an arithmetic one for inference. Overall, given an input image $\boldsymbol{x}$ the system computes the probability $\boldsymbol{y}$ that the specified constraints are satisfied. 

% Our goal is to compute bounds on the probabilistic output of the system, i.e. the root node of the circuit. 

% Similar to Section \ref{verification_concepts}

% In this case, performing robustness verification amounts to evaluating whether the probability $\boldsymbol{y}$ that the constraints are satisfied remains above $0.5$ for any input image $\boldsymbol{x}'$ which is $\epsilon$-close to $\boldsymbol{x}$, if $p_{\theta}(\boldsymbol{y} | \boldsymbol{x}, \mathrm{K}) > 0.5$.

% % is larger than the probability that they are not, i.e 
% % $\boldsymbol{y}_{\text{true}}$ is always $1$.


% % \limits_{\boldsymbol{x'}}
% \begin{equation*}
%     \begin{gathered}
% \mathrm{Range}(\mathit{y_i}) = 
% \Big[
% \min p_{\theta}(\mathit{y_i} | \boldsymbol{x'}), \  
% \max p_{\theta}(\mathit{y_i} | \boldsymbol{x'})
% \Big] \\
% \forall \boldsymbol{x'} \ s.t. \ ||\boldsymbol{x'} - \boldsymbol{x}|| \leq \epsilon
%     \end{gathered}
% \end{equation*}


We now formally define the aim of relaxation-based techniques in the context of NeSy probabilistic reasoning systems. Given a NeSy system, as defined in Section \ref{p-nesy-systems}, our aim is to compute:

\vspace{-0.3cm}
\begin{equation}
\left[
\min\limits_{\boldsymbol{x'}} p(\mathit{y_i} | \boldsymbol{x'}), \  
\max\limits_{\boldsymbol{x'}} p(\mathit{y_i} | \boldsymbol{x'}) 
\right] \quad
\forall \boldsymbol{x'} \ s.t. \ ||\boldsymbol{x'} - \boldsymbol{x}|| \leq \epsilon
\label{eq:goal}
\end{equation}

\noindent for all $\mathit{y_i}$ in $\boldsymbol{y}$. That is, we wish to calculate the minimum and maximum value of each of the probabilistic outputs of the NeSy system, under input perturbations of size $\epsilon$. As described in Section \ref{verification_concepts}, it is then possible to use these bounds to assess the robustness of an instance.

Consider the NeSy system in the running example of Figure \ref{fig:system}. The neural part of the system comprises two NNs, accepting the same dashcam image $\boldsymbol{x}$ as input. The first is an object detector predicting whether the image contains a red traffic light and whether there is a car in front of the autonomous vehicle (AV). The second is an action selector, which outputs whether to accelerate or brake the AV, given the image. The symbolic part of the system is the conjunction of a safety constraint and a common-sense one, as described in Section \ref{kc}. Given an input image $\boldsymbol{x}$, the system computes a single output $\mathit{y}$, denoting the probability that the specified constraints are satisfied. An instance is robust if $\mathrm{min}_{\mathbf{x'}} p(\mathit{y} | \boldsymbol{x'}) > 0.5$, since this means that for all inputs in an $\epsilon$-ball of $\boldsymbol{x}$ the probability of the constraints being satisfied is always greater than 0.5.

% root node of the circuit

\subsection{Exact Solution Complexity} \label{exact_solution}
% Having obtained bounds in the form of a probability range for each output of the NN, we now turn to the task of propagating these bounds through the symbolic component of a NeSy system. Our aim is to find the minimum and maximum probability that the reasoning module can output given the probability range of its inputs, and investigate the complexity of performing this computation exactly.

% Our aim is to find the minimum and maximum probability that the reasoning module can output given the probability range of its inputs. 

% These final bounds can then be used to answer verification queries, such as robustness, as previously outlined.

%  to obtain bounds on the circuit root node,

Let us now assume that via known techniques described in Section \ref{verification_concepts} we have obtained bounds in the form of a probability range for each output of the NN. Next, we turn to the task of propagating these bounds through the symbolic component, in order to obtain maxima/minima on the reasoning output, and investigate the complexity of doing so exactly. First, we show that, in the worst case, to find the solution we have to check all combinations of lower/upper bounds for all NN outputs. To illustrate this, we utilize the circuit representation of the symbolic component obtained via KC. It is known that such circuits represent multi-linear polynomials of the input variables \cite{choi2020probabilistic}. For example, a simple traversal of the SDD of Figure \ref{fig:sdd} yields the polynomial:

\vspace{-0.3cm}
\begin{align*}
    p \; = \; &\big( 1 - p(\text{car in front}) \big) \times \big( 1 - p(\text{red light}) \big) \\
    &\times p(\text{accelerate}) + p(\text{brake})
\end{align*}

Given this formulation, it is possible to obtain bounds on the circuit root node by solving a constrained optimization problem, in which we find the extrema (maximum and minimum) of the polynomial, subject to the bounded domains of the input variables (the NN outputs). We observe that this circuit polynomial is defined on a rectangular domain, since all input variables are defined in a closed interval (e.g. $\text{red light} \in [0.3, 0.4], \text{brake} \in [0.6, 0.9]$). It is known that in this case the extrema lie on the vertices of the domain \cite{laneve2010interval}, i.e., at the extrema each variable is assigned either its lower or upper bound, not something in between. Thus, in the worst case, to find the extrema one needs to search the combinatorial space of $2^n$ possible solutions.

In order to calculate the maximum and minimum output of the symbolic component, for each of the $2^n$ points we have to solve one instance of the WMC problem. Indeed, since each possible solution represents a probability assignment to all input variables, we can use WMC to compute the probabilistic output of the reasoning module under that weight assignment, and then select the maximum and minimum value obtained over all assignments.

Given the two steps above, it can be seen that starting with the formula, i.e., without first compiling it into a circuit, exact bound computation is a $\mathrm{NP}^{\# \mathrm{P}}$-hard problem. Intuitively, we need to search in the combinatorial space of variable configurations (the $\mathrm{NP}$ part), while performing WMC for each configuration (the $\# \mathrm{P}$ part). In this context, performing amortized inference via knowledge compilation entails that instead of solving a $\mathrm{NP}^{\# \mathrm{P}}$-hard problem for every sample, we perform a single $\# \mathrm{P}$-hard compilation at the beginning, and are ``just'' left with an NP problem per sample during runtime. Henceforth, we only consider the latter setting by assuming this initial compilation step.

% To evaluate each possible solution, that is, a probability assignment to each of the input variables, one needs to compute the Weighted Model Count of the formula represented by the circuit. Thus, solving the problem exactly is $\mathrm{NP}^{\# \mathrm{P}}$-hard. Intuitively, we need to search in the combinatorial space of variable configurations (the $\mathrm{NP}$ part), while performing WMC for each configuration (the $\# \mathrm{P}$ part).


\subsection{Relaxation-Based Approach} \label{e2e_abstract}
% As in the pure-NN case, the hardness of exact bound propagation motivates the use of abstraction-based techinques. We now turn to how these techniques can be applied to the neurosymbolic setting.

The NP-hardness of exact bound computation through the compiled symbolic component motivates the use of relaxation-based techniques. We now show how these can be extended to the NeSy setting in order to provide a scalable solution to Equation \ref{eq:goal}. 

Compositional probabilistic NeSy systems can be viewed as a single computational graph, by providing the outputs of the neural network as the inputs of the symbolic probabilistic circuit. In the case of the running example of Figure \ref{fig:system}, the outputs of each of the two networks are concatenated into a single vector and used as input to the arithmetic circuit which represents the constraints. Hence, a NeSy system can be seen as an end-to-end differentiable algebraic computational graph, which accepts an input, an image in this case, and outputs a vector of probabilities. These characteristics allow one to construct the NeSy system as a single module comprising an arbitrary number of neural networks and a single arithmetic circuit. Such a module can be constructed in a machine learning library, such as Pytorch, and subsequently exported as an Open Neural Network Exchange (ONNX) graph \cite{onnxruntime}. Figure \ref{fig:onnx} depicts the ONNX representation of the NeSy system of the running example.

% Importantly, the proposed method is not confined to a particular representation language, such as SDD's, since all tractable representation under a probabilistic semantics can be transformed into arithmetic circuits for inference.


% \begin{wrapfigure}{o}{0.6\linewidth}
%     \includegraphics[width=\linewidth]{figures/3. road_r_onnx_small.pdf}
%   \caption{Unified ONNX representation of the NeSy system from the running example. The input image is passed through the two NNs (left branch is action selection, right branch is object detection) and then through the arithmetic circuit. The NNs are stripped down to 1 convolutional and 1 dense layer for conciseness. The operators in the circuit besides Add, Sub, and Mul, are created by other Python operations such as tensor indexing, concatenation, etc.}
% \end{wrapfigure}

\begin{figure}[!h]
    \centering
    \includegraphics[width=0.6\linewidth]{figures/3_road_r_onnx_small.pdf}
    \caption{Unified ONNX representation of the NeSy system of the running example. The input image is processed by the two NNs (left branch is action selection, right branch is object detection) and then through the arithmetic circuit. The NNs are stripped down to one convolutional layer (Conv + MaxPool + ReLU) and one dense layer (Reshape + Gemm + Softmax/Sigmoid) for conciseness. The operators in the circuit, besides Add, Sub, and Mul, are created by Python operations, such as tensor indexing and concatenation.}
    \vspace{0.1cm}
    % \small\textsuperscript{a} General Matrix Multiplication.
    \label{fig:onnx}
\end{figure}

ONNX is a widespread NN representation, and is the stadard input format for NN verifiers \cite{vnnlib2024}. This includes both solver-based verification tools, such as Marabou \cite{katz2019marabou}, and relaxation-based ones, such as auto\_LiRPA \cite{autolirpaXu2020} and VeriNet \cite{henriksen2020efficient}. Thus, by representing a NeSy system as an end-to-end computational graph and exporting it to this format, it is possible to utilize state-of-the-art tools to perform verification in an almost ``out-of-the-box'' fashion. While our proposed framework is, in principle, compatible with all the aforementioned tools, we focus on relaxation-bsed verifiers, in order to showcase scalable probabilistic NeSy verification. Such verifiers allow us to perturb the input and compute bounds directly on the output of the NeSy system, that is, without computing intermediate bounds on the NN outputs.


% \subsection{Implementation of Softmax Relaxation}
% \input{sections/3. methodology/3.3. softmax approximations}

\section{Experimental Evaluation} \label{experiments}
% In this section we aim to experimentally demonstrate the effectiveness and the applicability of our approach. In the first experiment we evaluate the scalability of the proposed method by constructing a synthetic task where we can control the size of the symbolic component. We compare the runtime of our approach against two baselines, showing that our novel method for end-to-end abstract NeSy verification scales exponentially better. We also show that our approach is comparable in performance with the one baseline which output bounds. 

% In the second experiment we apply our approach to a real-world autonomous driving dataset. In this case, the aforementioned scalability benefits allow us to handle the larger input dimensionalities that are inherent in real-world applications and verify a safety property on top of two 6-layer convolutional NNs.


In this section we empirically evaluate the effectiveness and applicability of our approach. We assess the scalability of the proposed method via a synthetic task based on MNIST addition, a standard benchmark from the NeSy literature \cite{manhaeve2018deepproblog}. Further, we apply our approach to a real-world autonomous driving dataset and verify a safety driving property on top of two 6-layer convolutional NNs. In this case, the scalability of our technique allows us to handle high-dimensional input and larger networks, which are typical of real-world applications. All experiments are run on a machine with 128 AMD EPYC 7543 32-Core processors (3.7GHz) and 400GB of RAM. The code is available online\footnote{https://anonymous.4open.science/r/nesy-veri-6FBD/}.

\subsection{Multi-Digit MNIST Addition} \label{mnist_addition}
In this experiment we evaluate the scalability of our approach as the complexity of the probabilistic reasoning component increases. Specifically, we explore how the approximate nature of our method enhances scalability, while also considering the corresponding trade-off in the quality of verification results. 

\noindent To this end, we compare the following approaches:

\begin{enumerate}
    \item \textbf{End-to-End relaxation-based verification \big(\textsc{E2E-R}\big)} \\
    An implementation of our method in \lirpa, a state-of-the-art relaxation-based verification tool. The input to \lirpa is the NeSy system under verification, which is translated internally into an ONNX graph. The verification method used is IBP, as implemented in \lirpa.
    \item \textbf{Hybrid verification \big(\textsc{R+SLV}\big)} \\
    A hybrid approach consisting of relaxation-based verification for the neural part of the NeSy system and solver-based bound propagation through the symbolic part. The former is implemented in \lirpa using IBP. The latter is achieved by transforming the circuit into a polynomial (see Section \ref{exact_solution}), and solving a constrained optimization problem with the \gurobi solver. The purpose of comparing to this baseline is to assess the trade-off between scalability and quality of results, when using exact vs approximate bound propagation through the symbolic component. 
    \item \textbf{Solver-based verification \big(\textsc{Marabou}\big)} \\
    Exact verification using Marabou, a state-of-the-art SMT-based verification tool, also used as a backend by most NeSy verification works in the literature \cite{xie2022neuro,daggitt2024vehiclebridgingembeddinggap}. Marabou is unable to run on the full NeSy architecture, as the current implementation\footnote{https://github.com/NeuralNetworkVerification/Marabou} does not support several operators, such as Softmax and tensor indexing. To obtain an indication of Marabou's performance, we use it to verify only the neural part of the NeSy system, a subtask of NeSy verification. Specifically, we verify the classification robustness of the CNN performing MNIST digit recognition. 
\end{enumerate}

\paragraph{Dataset.} 
We use a synthetic task, where we can controllably increase the size of the symbolic component, while keeping the neural part constant. In particular, we create a variant of multi-digit MNIST addition \cite{manhaeve2018deepproblog}, where each instance consists of multiple MNIST digit images, and is labelled by the sum of all digits. We can then control the number of MNIST digits per sample, e.g. for 3-digit addition, an instane would be $\big( \, \big( \raisebox{-0.2\height} {\includegraphics[width=0.4cm]{figures/mnist_4.png}}, \raisebox{-0.2\height}{\includegraphics[width=0.4cm]{figures/mnist_7.png}}, \raisebox{-0.2\height}{\includegraphics[width=0.4cm]{figures/mnist_2.png}} \big), \, 13 \big)$. We construct the verification dataset from the $10\mathrm{K}$ samples of the MNIST test set, using each image only once. Thus, for a given $\#\text{digits}$ the verification set contains $10\mathrm{K} / \#\text{digits}$ test instances.

\paragraph{Experimental setting.} 
The NN is a convolutional neural network\footnote{The CNN comprises 2 convolutional layers with max pooling and 2 linear layers, with a final softmax activation.} tasked to recognize single MNIST digits. The CNN is trained in a standard supervised fashion on the MNIST train dataset, consisting of 60K images, and achieves an accuracy of $98\%$ on the test set. The symbolic part consists of the rules of multi-digit addition. It accepts the CNN predictions for the input images and computes a probability for each sum. As the number of summand digits increases, so does the size of the reasoning circuit, since there are more ways to construct a given sum using more digits (e.g. consider the ways in which 2 and 5 digits can sum to 17). We vary the number of digits as well as the size of $\mathcal{L}_{\infty}$-norm perturbations added to the input images. We consider five values for $\#$digits: $\{2, 3, 4, 5, 6\}$ and three values for the perturbation size $\epsilon$: $\{10^{-2}, 10^{-3}, 10^{-4}\}$, resulting in 15 distinct experiments. For each experiment, i.e., combination of $\#$digits and $\epsilon$ values, we use a timeout of 72 hours. \textsc{E2E-R} runs on a single thread, while the Gurobi solver in \textsc{R+SLV} dynamically allocates up to 1024 threads.


% Figure~\ref{fig:mnist-timing} we examine how increasing the number of digits affects the runtime of each method. The metric reported for each method is the time required to verify the robustness of the NeSy system on a single sample, averaged across the entire MNIST test dataset. We run the experiment for three values of epsilon to investigate how the size of the robustness guarantee affects the runtime of each approach. The digit recognition CNN is trained in a standard supervised fashion, achieving an accuracy of $98\%$ on the MNIST test dataset. We use a timeout of 3 days for each verification query, i.e. one number of digits and one epsilon. 

\paragraph{Scalability.} Figure \ref{fig:mnist-timing} presents a scalability comparison between the methods. The figure illustrates the time required to verify the robustness of the NeSy system for a single sample, averaged across the test dataset. All experiments terminate within the timeout limit, with the exception of two configurations for \textsc{R+SLV}. For $\langle \epsilon=10^{-2}, \#\text{digits}=5, 6 \rangle$, \textsc{R+SLV} was not able to verify any instance within the timeout (which is why the lines for $\epsilon=10^{-2}$ stop at 4 digits in Figure \ref{fig:mnist-timing}). For $\langle \epsilon=10^{-3}, \#\text{digits}=6 \rangle$, \textsc{R+SLV} verifies less than 5\% of the examples within the timeout. The reported values in Figure \ref{fig:mnist-timing} are the average runtime for this subset. 

% Explain through bound looseness in a branch-and-bound-based solver.
% This can explain the fact that the orange line for our method doesn't change between 5 \& 6 digits.

\begin{table*}[t]  
    \renewcommand{\arraystretch}{1.1}
    \centering
    \begin{tabular}{cccccc}
         \hline \hline
         \multirow{2}{*}{\textbf{Verification Method}} & \multirow{2}{*}{\textbf{Metric}} & \multicolumn{4}{c}{\textbf{\#MNIST digits}} \\
         % \cline{2-13}
         & & 2 & 3 & 4 & 5 \\
         \hline
         % Average circuit size       & \multicolumn{2}{c|}{s2} & \multicolumn{2}{c|}{s3} & \multicolumn{2}{c|}{s4} & \multicolumn{2}{c|}{s5} \\
         % \hline
         \multirow{2}{*}{\textsc{R+SLV}} & Lower/Upper Bound
                                               & $0.871-0.981$ & $0.815-0.972$ & $0.764-0.962$ & $0.731-0.928$ \\
                                               & Robustness (\%)         & $90.60$ & $86.17$ & $81.33$ & $78.31$ \\\hline
         \multirow{2}{*}{\textsc{E2E-R}}  & Lower/Upper Bound
                                               & $0.871-0.982$ & $0.815-0.974$ & $0.763-0.965$ & $0.716-0.958$ \\
                                               & Robustness (\%)         & $90.60$ & $86.11$ & $81.21$ & $76.67$ \\ \hline \hline
    \end{tabular}
    \caption{Comparison of performance between the proposed approach and the baseline with respect to the size of the symbolic component. We report one metric for bound tightness and one metric for the robustness of the system, according to each method.}
    \label{tab:mnist-perf}
\end{table*}

\begin{figure}[!h]
    \centering
    \includegraphics[width=\linewidth]{figures/4_MNIST_scaling.pdf}
    \caption{Comparison of verification runtime between three methods, with respect to the size of the symbolic component. We report the time required to verify the robustness of the NeSy system on a single sample, averaged across the MNIST test dataset, and repeat the experiment for three values of the perturbation size $\epsilon$.}
    \label{fig:mnist-timing}
\end{figure}

As Figure \ref{fig:mnist-timing} illustrates, \textsc{E2E-R} scales exponentially better than \textsc{R+SLV} -- note that runtimes are in log-scale. This is due to the computational complexity of exact bound propagation through the probabilistic reasoning component, as shown in Section~\ref{exact_solution}. In the surrogate task of verifying the robustness of the CNN only, \textsc{Marabou}'s runtime is 314 seconds per sample, averaged across 100 MNIST test images. It is thus several orders of magnitude slower than our approach, in performing a subtask of NeSy verification. %Consider, for example, that for $\#\text{digits} = 6$, \textsc{E2E-R} verifies 6 copies of the CNN along with a large circuit 3 orders of magnitude faster than \textsc{Marabou} verifies 1 copy of the CNN. 
This indicative performance for Marabou aligns with theoretical~\cite{crownZhang2018} and empirical evidence~\cite{vnnlib2024} on the poor scalability of SMT-based approaches. Our results suggest that this trade-off between completeness and scalability is favourable in the NeSy setting, where the verification task may involve multiple NNs and complex reasoning components.
% While we do not compare Marabou on the actual task, we believe that this preliminary result, along with the well-established view that SMT-based approaches are bad [ref, ref], effectively highlights the strengths of our approach.


% \begin{itemize}
%     \item Our approach scales exponentially better than the abstract+solver baseline. We see that exact bound propagation in the symbolic component is computationally prohibitive, as reinforced by the theoretical results made in Section \ref{exact_solution}.
%     \item All experiments finish within the timeout besides 5 and 6 digits for epsilon $\epsilon = 0.01$ and for 6 digits for epsilon $\epsilon = 0.001$. In the former, the solver-based baseline doesn't finish any example within the timeout (hence the lines stop at 4 digits). Explain through bound looseness in a branch-and-bound-based solver. In the latter, the solver-based baseline runs for under 5\% of the examples. The reported values are the average runtime for the examples that ran. This can explain the fact that the orange line for our method doesn't change between 5 \& 6 digits.
%     \item We run Marabou on 100 MNIST images and takes 314 seconds to verify the CNN, and is thus is several orders of magnitude slower than our approach. Consider for example, that at 6 digits, our approach verifies 6 copies of the CNN along with a large circuit 3 orders of magnitude faster than Marabou verifies 1 copy of the CNN. While we do not manage to compare Marabou on the actual task, we believe that this preliminary result, along with the well-established view that SMT-based approaches are bad [ref, ref], effectively highlights the strengths of our approach.
% \end{itemize}


\paragraph{Quality of verification results.} We next investigate how the complexity of the reasoning component affects the quality of the verification results. In Table \ref{tab:mnist-perf} we report, for $\epsilon=0.001$: (a) the tightness of the output bounds, in the form of lower/upper bound intervals for the probability of the \textit{correct} sum for each sample, averaged across the test set; (b) the robustness of the NeSy system, defined as the the number of robust samples divided by the total samples in the test set.\footnote{We don't report metrics for 6 digits since the full experiment exceeds the timeout.}

As expected, \textsc{R+SLV} outputs strictly tighter bounds than \text{E2E-R} for all configurations. We further observe that the quality of the bounds obtained by \text{E2E-R} degrades as the size of the reasoning circuits increases. This is also expected, since errors compound and accumulate over the larger network. However, the differences between \textsc{R+SLV} and \text{E2E-R} are minimal, especially in terms of robustness.

% the intervals decrease as the number of digits increases because there are more output classes, so the probability of the correct class decreases.

% \begin{itemize}
%     \item In Table \ref{tab:mnist-perf} we see that, as expected, the abstract+solver baseline gives strictly tighter bounds for all digits. Further, we see that the quality of the bounds obtained by our method degrades as the circuit sizes increase. However, the differences with the solver-based baseline are minor, and especially so when comparing the robustness metric. This makes sense.
% \end{itemize}

% Second, we examine how increasing the number of digits affects the performance of each method, which we quantify via two proxy metrics. The first relates to the ``tightness'' of the computed output bounds, i.e. how close the lower/upper bounds are for the output layer of the system. We estimate this by recording the lower and upper bound for the probability of the \textit{correct} sum for each sample, and averaging across the test dataset. This allows us to compare the accuracy of each method with fine granularity. The second metric estimates the robustness of the system under a given method. The soundness of the methods tested, i.e. that they never predict a non-robust system to be robust, implies that reported robustness is an indicator of performance. We estimate this by recording whether the lower bound of the probability of the \textit{correct} sum is greater than 0.5 \footnote{To see that this underapproximates actual robustness, consider that a lower bound $> 0.5$ for the correct class entails that there is no perturbation for which any other class has probability larger than that of the correct class.}. We report a percentage equal to the number of ``robust'' samples divided by the total number of examples. The perturbation size used is $\epsilon=0.001$. The results are shown in Table \ref{tab:mnist-perf} \footnote{the bounds for 6 digits are not shown because the experiment exceeds the timeout.}.

% Naturally, bound tightness is a desirable characteristic for an accurate and reliable verification method.

% Reported robustness can serve as an indicator of performance due to the soundness of the methods tested.

% The soundness of the methods tested implies that higher reported robustness is better. 

% As all methods tested are sound, reported robustness serves as an indicator of performance, since disagreement on a given sample means that one method was unable to prove the robustness of a robust system.

% Since all methods tested are sound, the lack of false positives implies that higher reported robustness is better.

% For a given sample, we record (1) the lower and upper bound for the probability of the \textit{correct} sum and (2) whether that lower bound is greater than 0.5. 

% Regarding the former, we are concerned with the "tightness" of the bounds, i.e. how close the lower/upper bounds are. Naturally, bound tightness is a desirable characteristic for an accurate and reliable verification method. By averaging the lower/upper bound values across the test dataset, we obtain a proxy metric for this type of accuracy. 

% The latter is used as a proxy for the robustness of the system, 

% We report the average value of the lower and upper bounds across the test dataset. This metric thus serves as a proxy for a verification method's accuracy and reliability.

% The former allows us to quantify the "tightness" of the bounds, i.e. how close the lower/upper bounds are.

% Regarding the former, ideally we desire that these bounds are (1) high in magnitude, indicating high classification confidence, and (2) "tight", i.e. the lower/upper bounds are close, indicating a robust system. By averaging these bounds across the test dataset, we obtain a metric serving as a proxy for a method's accuracy of bound computation.

% Further, we record whether the lower bound of the correct sum is greater than 0.5. Note that this is an overapproximation of robustness, since a $> 0.5$ lower bound entails that there is no perturbation for which any other class has probability $> 0.5$. 


\subsection{Autonomous Driving}
In this experiment we apply our proposed approach to a real-world dataset from the autonomous driving domain. The purpose of the experiment is to assess the robustness of a neural autonomous driving system with respect to the safety and common-sense properties of Figure \ref{fig:system}, i.e., to evaluate whether input perturbations cause the neural systems to violate the constraints that they previously satisfied.

\paragraph{Dataset.} To that end, we use the ROad event Awareness Dataset with logical Requirements (ROAD-R) \cite{giunchiglia2023road}. ROAD-R consists of 22 videos of dashcam footage from the point of view of an autonomous vehicle (AV), and is annotated at frame-level with bounding boxes. Each bounding box represents an \textit{agent} (e.g. a pedestrian, vehicles of different types, etc.) performing an \textit{action} (e.g. moving towards the AV, turning, etc.) at a specific \textit{location} (e.g. right pavement, incoming lane, etc.). 


\paragraph{Experimental Setting.} 
We focus on a subset of the dataset that is relevant to the symbolic constraints of Figure \ref{fig:system}. Consequently, we select a subset of frames which adhere to these constraints. Specifically, either the AV is moving forward, there is no red traffic light in the frame, and no car stopped in front of the AV, or the AV is stopped, and there is either a red traffic light or a car stopped in front. By sampling the videos every 2 seconds, we obtain a dataset of 3143 examples, where each example contains a $3 \times 240 \times 320$ image, and four binary labels: red light, car in front, stop, move forward.

% ROAD-R also comes with 243 common-sense logical requirements (e.g. an agent cannot move away and towards the vehicle at the same time).

The neural part of the system comprises two 6-layer CNNs\footnote{The CNNs have 4 convolutional layers with max pooling and 2 linear ones. The object detection network has a sigmoid activation at the output, while the action selection network has a softmax.}, responsible for object detection and action selection respectively. The two networks are trained in a standard supervised fashion using an 80/20 train/test split over the selected frames. The object detection and action selection networks achieve accuracies of $97.2\%$ and $96.3\%$ on the respective test sets. We add $\mathcal{L}_{\infty}$-norm perturbations to the test input images  for five values of perturbation size $\epsilon$: $\{10^{-5}, 5 \cdot 10^{-5}, 10^{-4}, 5 \cdot 10^{-4}, 10^{-3}\}$. 

\begin{table*}[!t]
    \renewcommand{\arraystretch}{1.2}
    \centering
    \begin{tabular}{cccccc}
        \hline \hline
        \multirow{2}{*}{\textbf{Metric}} & \multicolumn{5}{c}{\textbf{Epsilon}} \\ 
        & 1e-5      & 5e-5      & 1e-4      & 5e-4     & 1e-3 \\ \hline
        Robustness (\%) & $96.82\%$ & $92.68\%$ & $82.64\%$ & $6.21\%$ & $0.00\%$ \\
        Runtime per Sample (s) & $0.091$ & $0.092$ & $0.091$ & $0.092$ & $0.092$ \\
        \hline \hline
    \end{tabular}
    \caption{Autonomous driving experiment results, indicating robustness and verification runtime for five values of the $\epsilon$-perturbation.}
    %Standard (unperturbed input) accuracy: $97.2\%$.}
    \label{tab:road-r-robustness}
\end{table*}

Table \ref{tab:road-r-robustness} presents the results. We report robustness, i.e., the fraction of robust instances over the total number of instances in the test set, and verification runtime for \textsc{E2E-R}. Since this task consists of a small arithmetic circuit and a significantly larger neural component, it is the latter that predominantly affects both the computational overhead and the accumulated errors of bound propagation. Therefore, \textsc{E2E-R} and \textsc{R+SLV}, which differ only in the symbolic component, provide nearly identical results that are omitted.

As expected, robust accuracy falls as the perturbation size increases. Regarding the verification runtime, this experiment reinforces our results from Section \ref{mnist_addition}, by demonstrating that the runtime of our approach remains largely unaffected by changes in the value of the perturbation size $\epsilon$.

% Further, given a standard accuracy (unperturbed input) of $97.2\%$, we see that robust accuracy is almost equal to the standard accuracy for the smallest $\epsilon$ and reducing to zero for the largest $\epsilon$.

% We omit \textsc{R+SLV} as it would provide almost identical results with \textsc{E2E-R}. 

% This is because the two approaches differ only in their solution regarding the symbolic component, and thus, given a task where the majority  would be very similar.

% As this NeSy task is characterized by larger neural networks and a small circuit. Thus, the majority of (1) the computational overhead and (2) the accumulated errors of bound propagation lie on the neural component of the system. 

% Thus, the computational overhead in obtaining output bounds lies on the neural side. 

% Contrary to the MNIST addition experiment this task inclu, the majority of the computational overhead lies in propagating bounds through the large neural networks. where the NNs are larger and the circuit. Consequently, both the robustness and the runtime would be almost identical with the results obtained by \textsc{R+SLV}, overheadddddd, errors accumulate etc.

% since the runtimes are almost identical to the those \textsc{E2E-R} -- note that the symbolic part in this case is a relative small circuit, which is easy for Gurobi to handle -- and the robustness of   


\section{Related Work} \label{related-work}
\putsec{related}{Related Work}

\noindent \textbf{Efficient Radiance Field Rendering.}
%
The introduction of Neural Radiance Fields (NeRF)~\cite{mil:sri20} has
generated significant interest in efficient 3D scene representation and
rendering for radiance fields.
%
Over the past years, there has been a large amount of research aimed at
accelerating NeRFs through algorithmic or software
optimizations~\cite{mul:eva22,fri:yu22,che:fun23,sun:sun22}, and the
development of hardware
accelerators~\cite{lee:cho23,li:li23,son:wen23,mub:kan23,fen:liu24}.
%
The state-of-the-art method, 3D Gaussian splatting~\cite{ker:kop23}, has
further fueled interest in accelerating radiance field
rendering~\cite{rad:ste24,lee:lee24,nie:stu24,lee:rho24,ham:mel24} as it
employs rasterization primitives that can be rendered much faster than NeRFs.
%
However, previous research focused on software graphics rendering on
programmable cores or building dedicated hardware accelerators. In contrast,
\name{} investigates the potential of efficient radiance field rendering while
utilizing fixed-function units in graphics hardware.
%
To our knowledge, this is the first work that assesses the performance
implications of rendering Gaussian-based radiance fields on the hardware
graphics pipeline with software and hardware optimizations.

%%%%%%%%%%%%%%%%%%%%%%%%%%%%%%%%%%%%%%%%%%%%%%%%%%%%%%%%%%%%%%%%%%%%%%%%%%
\myparagraph{Enhancing Graphics Rendering Hardware.}
%
The performance advantage of executing graphics rendering on either
programmable shader cores or fixed-function units varies depending on the
rendering methods and hardware designs.
%
Previous studies have explored the performance implication of graphics hardware
design by developing simulation infrastructures for graphics
workloads~\cite{bar:gon06,gub:aam19,tin:sax23,arn:par13}.
%
Additionally, several studies have aimed to improve the performance of
special-purpose hardware such as ray tracing units in graphics
hardware~\cite{cho:now23,liu:cha21} and proposed hardware accelerators for
graphics applications~\cite{lu:hua17,ram:gri09}.
%
In contrast to these works, which primarily evaluate traditional graphics
workloads, our work focuses on improving the performance of volume rendering
workloads, such as Gaussian splatting, which require blending a huge number of
fragments per pixel.

%%%%%%%%%%%%%%%%%%%%%%%%%%%%%%%%%%%%%%%%%%%%%%%%%%%%%%%%%%%%%%%%%%%%%%%%%%
%
In the context of multi-sample anti-aliasing, prior work proposed reducing the
amount of redundant shading by merging fragments from adjacent triangles in a
mesh at the quad granularity~\cite{fat:bou10}.
%
While both our work and quad-fragment merging (QFM)~\cite{fat:bou10} aim to
reduce operations by merging quads, our proposed technique differs from QFM in
many aspects.
%
Our method aims to blend \emph{overlapping primitives} along the depth
direction and applies to quads from any primitive. In contrast, QFM merges quad
fragments from small (e.g., pixel-sized) triangles that \emph{share} an edge
(i.e., \emph{connected}, \emph{non-overlapping} triangles).
%
As such, QFM is not applicable to the scenes consisting of a number of
unconnected transparent triangles, such as those in 3D Gaussian splatting.
%
In addition, our method computes the \emph{exact} color for each pixel by
offloading blending operations from ROPs to shader units, whereas QFM
\emph{approximates} pixel colors by using the color from one triangle when
multiple triangles are merged into a single quad.




\section{Conclusion} \label{conclusion}
\section{Conclusion}
In this work, we propose a simple yet effective approach, called SMILE, for graph few-shot learning with fewer tasks. Specifically, we introduce a novel dual-level mixup strategy, including within-task and across-task mixup, for enriching the diversity of nodes within each task and the diversity of tasks. Also, we incorporate the degree-based prior information to learn expressive node embeddings. Theoretically, we prove that SMILE effectively enhances the model's generalization performance. Empirically, we conduct extensive experiments on multiple benchmarks and the results suggest that SMILE significantly outperforms other baselines, including both in-domain and cross-domain few-shot settings.

% \newpage
% \section{TODOs}
% \begin{itemize}
%     \item Intro: exact approaches sacrifice scalability wrt what?
%     \item What we're doing is actually \textit{harder} than just neurosymbolic verification. We can do quantitative things like what is the \textit{probability} of the NN saying X?
%     \item The effect of reasoning shortcuts on neurosymbolic verification. Why do we care about NeSy verification (and not just neural verification):
    % \begin{itemize}
    %     \item I don't have NN labels (latent concept annotations).
    %     \item I don't care about NN predictions.
    %     \item My NN doesn't have correct outputs (e.g. action selectors via policy network).
    % \end{itemize}
%     \item NeSy training and its effect on robustness.
    % \item Our approach allows us to verify complicated properties on top of large NNs which were previously out of reach.
% \end{itemize}

%% The file named.bst is a bibliography style file for BibTeX 0.99c
\newpage
\bibliographystyle{named}
\bibliography{ijcai25}

\end{document}

 