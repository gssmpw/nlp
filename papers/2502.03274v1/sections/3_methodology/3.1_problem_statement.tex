
% We now formally introduce relaxation-based verification in the context of NeSy probabilistic reasoning systems, as a means of solving the robustness verification problem. We provide further intuition by introducing the running example of Figure \ref{fig:system}. We are given a NeSy system which computes $p_{\theta}(\boldsymbol{y} | \boldsymbol{x}, \mathrm{K})$. The goal of our approach is to calculate:

% % The $\epsilon$-robustness of the NeSy system can be then be expressed as:
% $$
% \mathrm{min}_{\mathbf{x'}} \ p_{\theta}(\boldsymbol{y_i} | \boldsymbol{x'}, \mathrm{K}), \quad \mathrm{max}_{\mathbf{x'}} \ p_{\theta}(\boldsymbol{y_i} | \boldsymbol{x'}, \mathrm{K})
% $$
% % p_{\theta}(\boldsymbol{y}_{\text{true}} | \boldsymbol{x'}, 
% % \mathrm{K}) - 
% % p_{\theta}(\boldsymbol{y}_{\text{i}} | \boldsymbol{x'}, 
% % \mathrm{K}) > 0,
% % $$

% for all $\boldsymbol{x'}$ such that 
% $||\boldsymbol{x'} - \boldsymbol{x}|| \leq \epsilon$ and 
% for all $\boldsymbol{y}_{\text{i}} \neq \boldsymbol{y}_{\text{true}}$. 

% That is, we aim to calculate the lower 

% Consider now the NeSy system of the running example of Figure \ref{fig:system}. The neural part of the system is comprised of two neural networks accepting the same dashcam image $\boldsymbol{x}$ as input. The first is an object detection network predicting whether the image contains a red traffic light and whether there is a car in front of the autonomous vehicle (AV). The second is an action selection network which outputs whether to accelerate or brake the AV given the image. The symbolic part of the system is comprised of the conjunction of two constraints, a safety constraint and a common-sense one, as described in Section \ref{kc}. These are compiled into a boolean circuit, and then transformed into an arithmetic one for inference. Overall, given an input image $\boldsymbol{x}$ the system computes the probability $\boldsymbol{y}$ that the specified constraints are satisfied. 

% Our goal is to compute bounds on the probabilistic output of the system, i.e. the root node of the circuit. 

% Similar to Section \ref{verification_concepts}

% In this case, performing robustness verification amounts to evaluating whether the probability $\boldsymbol{y}$ that the constraints are satisfied remains above $0.5$ for any input image $\boldsymbol{x}'$ which is $\epsilon$-close to $\boldsymbol{x}$, if $p_{\theta}(\boldsymbol{y} | \boldsymbol{x}, \mathrm{K}) > 0.5$.

% % is larger than the probability that they are not, i.e 
% % $\boldsymbol{y}_{\text{true}}$ is always $1$.


% % \limits_{\boldsymbol{x'}}
% \begin{equation*}
%     \begin{gathered}
% \mathrm{Range}(\mathit{y_i}) = 
% \Big[
% \min p_{\theta}(\mathit{y_i} | \boldsymbol{x'}), \  
% \max p_{\theta}(\mathit{y_i} | \boldsymbol{x'})
% \Big] \\
% \forall \boldsymbol{x'} \ s.t. \ ||\boldsymbol{x'} - \boldsymbol{x}|| \leq \epsilon
%     \end{gathered}
% \end{equation*}


We now formally define the aim of relaxation-based techniques in the context of NeSy probabilistic reasoning systems. Given a NeSy system, as defined in Section \ref{p-nesy-systems}, our aim is to compute:

\vspace{-0.3cm}
\begin{equation}
\left[
\min\limits_{\boldsymbol{x'}} p(\mathit{y_i} | \boldsymbol{x'}), \  
\max\limits_{\boldsymbol{x'}} p(\mathit{y_i} | \boldsymbol{x'}) 
\right] \quad
\forall \boldsymbol{x'} \ s.t. \ ||\boldsymbol{x'} - \boldsymbol{x}|| \leq \epsilon
\label{eq:goal}
\end{equation}

\noindent for all $\mathit{y_i}$ in $\boldsymbol{y}$. That is, we wish to calculate the minimum and maximum value of each of the probabilistic outputs of the NeSy system, under input perturbations of size $\epsilon$. As described in Section \ref{verification_concepts}, it is then possible to use these bounds to assess the robustness of an instance.

Consider the NeSy system in the running example of Figure \ref{fig:system}. The neural part of the system comprises two NNs, accepting the same dashcam image $\boldsymbol{x}$ as input. The first is an object detector predicting whether the image contains a red traffic light and whether there is a car in front of the autonomous vehicle (AV). The second is an action selector, which outputs whether to accelerate or brake the AV, given the image. The symbolic part of the system is the conjunction of a safety constraint and a common-sense one, as described in Section \ref{kc}. Given an input image $\boldsymbol{x}$, the system computes a single output $\mathit{y}$, denoting the probability that the specified constraints are satisfied. An instance is robust if $\mathrm{min}_{\mathbf{x'}} p(\mathit{y} | \boldsymbol{x'}) > 0.5$, since this means that for all inputs in an $\epsilon$-ball of $\boldsymbol{x}$ the probability of the constraints being satisfied is always greater than 0.5.

% root node of the circuit