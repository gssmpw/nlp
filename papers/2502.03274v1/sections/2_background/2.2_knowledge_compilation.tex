Probabilistic reasoning in NeSy systems is often performed via reduction to Weighted Model Counting (WMC), which we briefly review next. Consider a propositional logical formula $\phi$ over variables $V$. Each boolean variable $v\in V$ is assigned a weight $p(v)$, which denotes the probability of that variable being true. The Weighted Model Count (WMC) of formula $\phi$ is then defined as:

\begin{equation} \label{eq:wmc}
    \mathrm{WMC (\phi)} = \sum_{\omega \models \phi} \quad \prod_{v \in \omega} p(v) \prod_{v \notin \omega} 1 - p(v).
\end{equation}
In essence, the WMC is the sum of the probability of all worlds $\omega$ that are models of $\phi$. WMC is $\#\mathrm{P}$-hard, since it generalizes a $\#\mathrm{P}$-complete problem, $\#\mathrm{SAT}$, by incorporating weights \cite{chavira2008probabilistic}.

A widely-used approach for solving the WMC problem is \textit{knowledge compilation} (KC) \cite{darwiche2002knowledge,chavira2008probabilistic}. According to this approach, the formula $\phi$ is first compiled into a tractable representation, which is used at inference time to compute a large number of queries - in our case, instances of the WMC problem - in polynomial time. KC techniques push most of the computational effort to the ``off-line'' compilation phase, resulting in computationally cheap ``on-line'' query answering, a concept termed \textit{amortized inference}.

% The idea to shift the majority of the computation onto the off-line compilation phase in order to gain fast on-line queries is termed \textit{amortized inference}, and it is the central motivation for using knowledge compilation. 

The representations obtained via KC take the form of computational graphs, in which the literals, i.e., logical variables and their negations, are found only on leaves of the graph. The nodes are only logical $\mathrm{AND}$ and $\mathrm{OR}$ operations, and the root represents the query. For example, consider the constraints $\phi$ from the autonomous driving example of Figure \ref{fig:system}:
\begin{equation*}
\begin{gathered}
\text{red light} \lor \text{car in front} \implies \text{brake} \\
\text{accelerate} \iff \neg \text{brake}
\end{gathered}
\end{equation*}
\noindent These dictate that (1) if there is a red light or a car in front of the AV, then the AV should brake, and (2) that accelerating and braking should be mutually exclusive and exhaustive, i.e., only one should take place at any given time. Figure \ref{fig:sdd} presents the compiled form of these constraints as a boolean circuit, namely a Sentential Decision Diagram (SDD) \cite{darwiche2011sdd}.
\begin{figure}[!h]
    \centering
    \begin{subfigure}[t]{0.5\linewidth}
        \centering
        \includegraphics[width=0.95\linewidth]{figures/2_sdd.pdf}
        \caption{}
        \label{fig:sdd}
    \end{subfigure}%
    ~ 
    \begin{subfigure}[t]{0.5\linewidth}
        \centering
        \includegraphics[width=0.95\linewidth]{figures/2_ac.pdf}
        \caption{}
        \label{fig:ac}
    \end{subfigure}
    \caption{(a) A Sentential Decision Diagram (SDD) as an example of a computational graph obtained via knowledge compilation and (b) the corresponding arithmetic circuit (AC) derived from the SDD during inference by replacing the AND/OR nodes with multiplication/addition. The SDD has been minimized for conciseness.}
    \label{fig:circuit}
\end{figure}
To perform WMC, the boolean circuit is replaced by an arithmetic one, by replacing the $\mathrm{AND}$ nodes of the graph with multiplication, the $\mathrm{OR}$ nodes with addition, and the negation of literals with subtraction ($1-x$). The resulting structure, shown in Figure \ref{fig:ac}, can compute the WMC of $\phi$ simply by plugging in the literal probabilities at the leaves and traversing the circuit bottom-up. Indeed, one can check that assuming the probabilities:
\begin{alignat*}{2}
p(\text{accelerate}) &= 0.3, \quad p(\text{red light}) &&= 0.6 \\ 
p(\text{brake}) &= 0.7, \quad p(\text{car in front}) &&= 0.8
\end{alignat*}

\noindent this computation correctly calculates the probability of $\phi$ by summing the probability of its 5 different models.


% is an algebraic computational graph containing only subtractions (for negation), additions (for disjunction), and multiplications (for conjunction), that can be used to perform probabilistic logical reasoning.

% In such systems, probabilistic inference is commonly performed by solving the Weighted Model Counting (WMC) problem. 

% The probability of a world is computed as a product over all variables, where positive variables are assigned their probability and negative variables are assigned one minus their probability. 

% rooted directed acyclic graph (DAG)