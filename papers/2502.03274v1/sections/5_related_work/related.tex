Although NN verification and NeSy AI have both seen rapid growth over the last few years, their intersection is under-explored, with some related work existing in the literature. In \cite{akintunde2020verifying}, the authors address the problem of verifying properties accociated with the temporal dynamics of multi-agent systems. The agents of the system combine a neural perception module with a symbolic one, encoding action selection mechanisms via traditional control logic. The verification queries are specified in alternating-time temporal logic, and the corresponding verification problem is cast as a MILP instance, delegated to a custom, Gurobi-based verification tool. The work in \cite{xie2022neuro} goes beyond NN robustness, by verifying more complex properties on top of a NN, or system of networks. The authors introduce a property specification language based on Hoare logic, where variables can be instantiated to NN inputs and outputs. Trained NNs, along with the property under verification, are compiled into an SMT problem, which is delegated to Marabou. \cite{daggitt2024vehiclebridgingembeddinggap} follows a similar approach, in order to verify neurosymbolic programs, i.e.,\ programs containing both neural networks and symbolic code. The authors introduce a property specification language, which allows for NN training and the specification of verification queries. A custom tool then compiles the NNs, the program, and the verification query into an SMT problem, which is again delegated to Marabou.


% purely neural system. The authors follow \cite{albarghouthi2021introductionneuralnetworkverification} and utilize Hoare triplets \cite{10.1145/363235.363259} for writing the verification specification. These are cast as an SMT-formulation via simple wrapper functions for interfacing with Marabou. Finally, \cite{daggitt2024vehiclebridgingembeddinggap} focus on the verification of NeSy programs, which contain both machine learning components and traditional (symbolic) code. This is facilitated by Vehicle, a custom pure functional language, which allows to write verification specifications for the NeSy program. A compiler then translates this specification into an SMT-formulation, and invokes Marabou for solving.

% The aforementioned works utilize logics of limited expressive power (e.g. temporal logic, Hoare logic etc) as a means to specify properties under verification over neural systems. They lack a formal probabilistic semantics and, therefore, cannot be used to verify probabilistic properties. In contrast, we verify the robustness of probabilistic logical reasoning systems that combine general-purpose knowledge with NNs, in order to reason under uncertainty over sub-symbolic input. Moreover, the aforementioned approaches are all based on deductive verification and hence cannot scale to high-dimensional input and large networks, as we show with Marabou's indicative performance in Section \ref{experiments}. This is in contrast to our proposed method, which, being the first to utilize abstraction-based verification in a NeSy setting, is able to scale to input dimensionality, network size and knowledge complexity, that are beyond reach for existing techniques.

The aforementioned approaches cannot verify probabilistic logical reasoning systems. This is because their specification languages (logics of limited expressive power in \cite{akintunde2020verifying,xie2022neuro} and a functional language in \cite{daggitt2024vehiclebridgingembeddinggap}) lack a general-purpose reasoning engine, as well as formal probabilistic semantics. In contrast, our method verifies the robustness of NeSy systems which perform general-purpose reasoning under uncertainty, by combining NNs with probabilistic logical reasoners. Furthermore, all existing approaches are based on solver-based verification and hence cannot scale to high-dimensional input and large networks, as we show with Marabou's indicative performance in Section \ref{experiments}. This is in contrast to our proposed method, which, being the first to utilize relaxation-based verification in a NeSy setting, is able to handle large input dimensionality, network sizes, and knowledge complexity.

% We identify two main points of difference between all existing related work and our proposed approach. First, the highlighted works are unable to verify arbitrary probabilistic reasoning, which is of course at the core of the probabilistic NeSy systems that are the focus of our work. More specifically, \cite{akintunde2020verifying,xie2022neuro} each focus on a specific branch of logic, while \cite{daggitt2024vehiclebridgingembeddinggap} write the specifications in a custom pure functional specification language. While all three works support some kind of logical operators, none have the probabilistic semantics required to perform, and later verify probabilistic reasoning systems. Second, all existing NeSy verification methods utilize deductive techniques. \cite{akintunde2020verifying} formulate the verification query as a MILP problem, and \cite{xie2022neuro,daggitt2024vehiclebridgingembeddinggap} both cast the specification as an SMT-problem and solve it by using the Marabou verifier. In light of the above, the proposed approach is - to the best of our knowledge - the first to allow for the use of abstraction-based techniques in verifying the robustness of probabilistic NeSy systems.

% Proposed approaches have till now focused predominantly on verifying that neural systems satisfy a property. 