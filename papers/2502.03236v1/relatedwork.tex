\section{Related Work}
\paragraph{Dynamic Interacting System \& Graph ODEs}
%Dynamic system modeling has been recognized and extensively studied for decades.
Early practices leverage the recurrent models  to study the sequential pattern of the objects.
Given the correlation among objects, GNNs \cite{nips23ssLRD,www24GAUSS} are then incorporated with the recurrent models \cite{nips19rstgnn,nips19HVGRNN,aaai22Fan}.
%\cite{www21AutoSTG,www21Yuan,kdd20Dai,www22Chen,iclr19Sun,aaai19GuoLFSW}.
In recent years, graph ODEs show encouraging results for modeling the correlation and evolvement collaboratively \cite{kdd21CoupledGraphODE,wsdm23Luo}.
\citet{icml23HOPE,kdd22zhang,PMLR20Xhonneux} introduce second-order ODE on graphs,
 while  \citet{kdd21CoupledGraphODE,wsdm23Luo} consider the structural evolvement to model the dynamic correlation, and
\citet{aaai24ChenWLLsigned} reconsider the graph structure of dynamic systems.
% In parallel, physics-informed neural nets primarily focus on solving PDEs \cite{nips22PNNPDE}, and have not been related to graph dynamics yet.
Recently, \citet{nips23tango} study  time-reversal symmetric systems. \citet{nips21Lee} present a new parameterization of brackets for learning irreversible dynamics. \citet{nips23Lee} further develop a  graph attention mechanism  to enable the irreversibility in deep GNNs.
%So far, existing methods work with the Euclidean space to model system dynamics. 
Different from previous works, we build the differential system upon Riemannian manifold, considering the entropy increasing of dynamic systems.


\begin{figure} 
\centering 
\subfigure[\texttt{Pioneer}]{
\includegraphics[width=0.48\linewidth]{glacier1.pdf}}
%\hspace{-0.05\linewidth}
\subfigure[Euclidean Modeling]{
\includegraphics[width=0.48\linewidth]{glacier2.pdf}}
\vspace{-0.1in}
 \caption{Prediction results of Glacier thickness.}
\label{Fig-glacier}
\end{figure}


\paragraph{Riemannian Deep Learning on Graphs}
%Riemannian manifold is  drawing increasing attentions in recent years for its alignment to graph geometry.
Categorized by the geometry, 
hyperbolic space is suitable to hierarchical structures and show the superiority to the Euclidean counterpart \cite{icml24sun,fu24icml,aaai24YCWei}.
Hyperspherical space achieves remarkable success to embed cyclical structures \cite{icml23sphereFourier}.
Recent studies further investigate the constant curvature spaces \cite{icml20kGCN,nips24sun}, product spaces \cite{iclr19Gu,aaai22sunli,aaai24sunli}, quotient spaces \cite{nips22QGCN}, SPD manifolds \cite{icml23GyroSpace}, etc. 
%Grassmann manifold \cite{nips23GrassFlow}
For dynamic graphs, there exist Riemannian models focusing on link prediction and/or node classification \cite{kdd21yang,aaai21sunli,cikm22sunli,aaai23sunli}.
%, which is essentially different from the regression of dynamic system as ours.
Recently, Ricci curvature is introduced to structure learning \cite{icml23revisitRicci,icdm23sunli} or clustering \cite{ijcai23sunli}, while its physics aspects are not explored.
Also, we notice that \citet{nips20mainfoldode,www24sunli,sigir24sunli,WangSun24cikm,iclr24FlowMatch} studies Riemannian ODEs or SDEs as generative models.

% However, they present as continuous normalizing flows or denoising diffusion models for data generation, and thereby are orthogonal to our study.