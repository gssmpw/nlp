%% Paragraph customisation
\newcommand{\rparagraph}[1]{\vspace{1.2mm}\noindent\textbf{#1.}}
\newcommand{\iparagraph}[1]{\vspace{0.7mm}\noindent\textit{#1.}}
\newcommand{\rrparagraph}[1]{\vspace{0.5mm}\noindent\textit{#1:}}
\newcommand{\sparagraph}[1]{\vspace{0.0mm}\noindent\textbf{#1.}}

\newcommand{\rparagraphnodot}[1]{\vspace{1.2mm}\noindent\textbf{#1}}
\newcommand{\sparagraphnodot}[1]{\vspace{0.0mm}\noindent\textbf{#1}}
\newcommand{\iparagraphnodot}[1]{\vspace{0.0mm}\noindent\textit{#1}}

\usepackage{xspace}
\newcommand{\xlt}{{\textsc{xlt}}\xspace}
\newcommand{\zsxlt}{{\textsc{zs-xlt}}\xspace}
\newcommand{\its}{{\textsc{i2s}}\xspace}
\newcommand{\sti}{{\textsc{s2i}}\xspace}
\newcommand{\tts}{{\textsc{t2s}}\xspace}
\newcommand{\stt}{{\textsc{s2t}}\xspace}


\newcommand{\pp}{\ensuremath{\mathrm{pp}}}

\usepackage{todonotes}
\newcommand{\noindentaftertodo}{\iftodonotes{\noindent}{}}
% defne the fixme command
\newcommand{\fixme}[2][]{\todo[color=yellow,size=\scriptsize,fancyline,caption={},#1]{#2}} % to mark stuff that you know is missing or wrong when you write the text
\newcommand{\Fixme}[2][]{\fixme[inline,#1]{#2}\noindentaftertodo}
% define the note command
\newcommand{\note}[4][]{\todo[author=#2,color=#3,size=\scriptsize,fancyline,caption={},#1]{#4}} % default note settings, used by macros below.
\newcommand{\response}[1]{\vspace{3pt}\hrule\vspace{3pt}\textbf{#1:}}
% one per co-author
\newcommand{\fabian}[2][]{\note[#1]{Fabian}{yellow!40}{#2}}
\newcommand{\Fabian}[2][]{\fabian[inline,#1]{#2}\noindentaftertodo}
