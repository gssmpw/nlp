% This must be in the first 5 lines to tell arXiv to use pdfLaTeX, which is strongly recommended.
\pdfoutput=1
% In particular, the hyperref package requires pdfLaTeX in order to break URLs across lines.

\documentclass[11pt]{article}

% Change "review" to "final" to generate the final (sometimes called camera-ready) version.
% Change to "preprint" to generate a non-anonymous version with page numbers.
\usepackage[preprint]{acl}

% Standard package includes
\usepackage{times}
\usepackage{latexsym}

% For proper rendering and hyphenation of words containing Latin characters (including in bib files)
\usepackage[T1]{fontenc}
% For Vietnamese characters
% \usepackage[T5]{fontenc}
% See https://www.latex-project.org/help/documentation/encguide.pdf for other character sets

% This assumes your files are encoded as UTF8
\usepackage[utf8]{inputenc}

% This is not strictly necessary, and may be commented out,
% but it will improve the layout of the manuscript,
% and will typically save some space.
\usepackage{microtype}

% This is also not strictly necessary, and may be commented out.
% However, it will improve the aesthetics of text in
% the typewriter font.
\usepackage{inconsolata}

%Including images in your LaTeX document requires adding
%additional package(s)
\usepackage{array}
\usepackage{xspace}
\usepackage{booktabs}
\usepackage{amsmath}
\usepackage{amssymb}
\usepackage{graphicx}
\usepackage{adjustbox}
\usepackage{multirow}
\usepackage{comment}
\usepackage{color, colortbl}
\usepackage{arydshln}
\usepackage{hhline}
\usepackage{array}
\usepackage{makecell}
\def\doubleunderline#1{\underline{\underline{#1}}}
\DeclareMathOperator*{\argmax}{arg\,max}
% Packages for colored boxes
\usepackage{tcolorbox}
% Packages for image handling
\usepackage{graphicx}
% Optional: Adjust page margins
% \usepackage[margin=1in]{geometry}
% Packages for lists
\usepackage{enumitem}
\usepackage{adjustbox}
\usepackage{xcolor}
\usepackage{float} 
\usepackage{caption} % Provides \captionof

\newtcolorbox{promptbox}[1]{colback=white!95!black,colframe=black,fonttitle=\bfseries\small,fontupper=\footnotesize,title=#1}


% If the title and author information does not fit in the area allocated, uncomment the following
%
%\setlength\titlebox{<dim>}
%
% and set <dim> to something 5cm or larger.

%% Paragraph customisation
\newcommand{\rparagraph}[1]{\vspace{1.2mm}\noindent\textbf{#1.}}
\newcommand{\iparagraph}[1]{\vspace{0.7mm}\noindent\textit{#1.}}
\newcommand{\rrparagraph}[1]{\vspace{0.5mm}\noindent\textit{#1:}}
\newcommand{\sparagraph}[1]{\vspace{0.0mm}\noindent\textbf{#1.}}

\newcommand{\rparagraphnodot}[1]{\vspace{1.2mm}\noindent\textbf{#1}}
\newcommand{\sparagraphnodot}[1]{\vspace{0.0mm}\noindent\textbf{#1}}
\newcommand{\iparagraphnodot}[1]{\vspace{0.0mm}\noindent\textit{#1}}

\usepackage{xspace}
\newcommand{\xlt}{{\textsc{xlt}}\xspace}
\newcommand{\zsxlt}{{\textsc{zs-xlt}}\xspace}
\newcommand{\its}{{\textsc{i2s}}\xspace}
\newcommand{\sti}{{\textsc{s2i}}\xspace}
\newcommand{\tts}{{\textsc{t2s}}\xspace}
\newcommand{\stt}{{\textsc{s2t}}\xspace}


\newcommand{\pp}{\ensuremath{\mathrm{pp}}}

\usepackage{todonotes}
\newcommand{\noindentaftertodo}{\iftodonotes{\noindent}{}}
% defne the fixme command
\newcommand{\fixme}[2][]{\todo[color=yellow,size=\scriptsize,fancyline,caption={},#1]{#2}} % to mark stuff that you know is missing or wrong when you write the text
\newcommand{\Fixme}[2][]{\fixme[inline,#1]{#2}\noindentaftertodo}
% define the note command
\newcommand{\note}[4][]{\todo[author=#2,color=#3,size=\scriptsize,fancyline,caption={},#1]{#4}} % default note settings, used by macros below.
\newcommand{\response}[1]{\vspace{3pt}\hrule\vspace{3pt}\textbf{#1:}}
% one per co-author
\newcommand{\fabian}[2][]{\note[#1]{Fabian}{yellow!40}{#2}}
\newcommand{\Fabian}[2][]{\fabian[inline,#1]{#2}\noindentaftertodo}


%\title{MVL-SIB: Massively Multilingual Vision Language Benchmarking}
\title{MVL-SIB: A Massively Multilingual Vision-Language Benchmark for Cross-Modal Topical Matching}



  \author{%
  Fabian David Schmidt\textsuperscript{1}\thanks{Equal contribution.}, 
  Florian Schneider\textsuperscript{2}\footnotemark[1], 
  Chris Biemann\textsuperscript{2}, 
  Goran Glavaš\textsuperscript{1} \\
  \textsuperscript{1}Center for Artificial Intelligence and Data Science, University of Würzburg, Germany \\
  \textsuperscript{2}Language Technology Group, University of Hamburg, Germany \\
  \texttt{fabian.schmidt@uni-wuerzburg.de, florian.schneider-1@uni-hamburg.de}\\
  \textbf{Dataset:} \href{https://huggingface.co/datasets/WueNLP/mvl-sib}{MVL-SIB}
}


\begin{document}
\maketitle
%
\begin{abstract}
  In this work, we present a novel technique for GPU-accelerated Boolean satisfiability (SAT) sampling. Unlike conventional sampling algorithms that directly operate on conjunctive normal form (CNF), our method transforms the logical constraints of SAT problems by factoring their CNF representations into simplified multi-level, multi-output Boolean functions. It then leverages gradient-based optimization to guide the search for a diverse set of valid solutions. Our method operates directly on the circuit structure of refactored SAT instances, reinterpreting the SAT problem as a supervised multi-output regression task. This differentiable technique enables independent bit-wise operations on each tensor element, allowing parallel execution of learning processes. As a result, we achieve GPU-accelerated sampling with significant runtime improvements ranging from $33.6\times$ to $523.6\times$ over state-of-the-art heuristic samplers. We demonstrate the superior performance of our sampling method through an extensive evaluation on $60$ instances from a public domain benchmark suite utilized in previous studies. 


  
  % Generating a wide range of diverse solutions to logical constraints is crucial in software and hardware testing, verification, and synthesis. These solutions can serve as inputs to test specific functionalities of a software program or as random stimuli in hardware modules. In software verification, techniques like fuzz testing and symbolic execution use this approach to identify bugs and vulnerabilities. In hardware verification, stimulus generation is particularly vital, forming the basis of constrained-random verification. While generating multiple solutions improves coverage and increases the chances of finding bugs, high-throughput sampling remains challenging, especially with complex constraints and refined coverage criteria. In this work, we present a novel technique that enables GPU-accelerated sampling, resulting in high-throughput generation of satisfying solutions to Boolean satisfiability (SAT) problems. Unlike conventional sampling algorithms that directly operate on conjunctive normal form (CNF), our method refines the logical constraints of SAT problems by transforming their CNF into simplified multi-level Boolean expressions. It then leverages gradient-based optimization to guide the search for a diverse set of valid solutions.
  % Our method specifically takes advantage of the circuit structure of refined SAT instances by using GD to learn valid solutions, reinterpreting the SAT problem as a supervised multi-output regression task. This differentiable technique enables independent bit-wise operations on each tensor element, allowing parallel execution of learning processes. As a result, we achieve GPU-accelerated sampling with significant runtime improvements ranging from $10\times$ to $1000\times$ over state-of-the-art heuristic samplers. Specifically, we demonstrate the superior performance of our sampling method through an extensive evaluation on $60$ instances from a public domain benchmark suite utilized in previous studies.

\end{abstract}

\begin{IEEEkeywords}
Boolean Satisfiability, Gradient Descent, Multi-level Circuits, Verification, and Testing.
\end{IEEEkeywords}
%
\section{Introduction}\label{sec:Intro} 


Novel view synthesis offers a fundamental approach to visualizing complex scenes by generating new perspectives from existing imagery. 
This has many potential applications, including virtual reality, movie production and architectural visualization \cite{Tewari2022NeuRendSTAR}. 
An emerging alternative to the common RGB sensors are event cameras, which are  
 bio-inspired visual sensors recording events, i.e.~asynchronous per-pixel signals of changes in brightness or color intensity. 

Event streams have very high temporal resolution and are inherently sparse, as they only happen when changes in the scene are observed. 
Due to their working principle, event cameras bring several advantages, especially in challenging cases: they excel at handling high-speed motions 
and have a substantially higher dynamic range of the supported signal measurements than conventional RGB cameras. 
Moreover, they have lower power consumption and require varied storage volumes for captured data that are often smaller than those required for synchronous RGB cameras \cite{Millerdurai_3DV2024, Gallego2022}. 

The ability to handle high-speed motions is crucial in static scenes as well,  particularly with handheld moving cameras, as it helps avoid the common problem of motion blur. It is, therefore, not surprising that event-based novel view synthesis has gained attention, although color values are not directly observed.
Notably, because of the substantial difference between the formats, RGB- and event-based approaches require fundamentally different design choices. %

The first solutions to event-based novel view synthesis introduced in the literature demonstrate promising results \cite{eventnerf, enerf} and outperform non-event-based alternatives for novel view synthesis in many challenging scenarios. 
Among them, EventNeRF \cite{eventnerf} enables novel-view synthesis in the RGB space by assuming events associated with three color channels as inputs. 
Due to its NeRF-based architecture \cite{nerf}, it can handle single objects with complete observations from roughly equal distances to the camera. 
It furthermore has limitations in training and rendering speed: 
the MLP used to represent the scene requires long training time and can only handle very limited scene extents or otherwise rendering quality will deteriorate. 
Hence, the quality of synthesized novel views will degrade for larger scenes. %

We present Event-3DGS (E-3DGS), i.e.,~a new method for novel-view synthesis from event streams using 3D Gaussians~\cite{3dgs} 
demonstrating fast reconstruction and rendering as well as handling of unbounded scenes. 
The technical contributions of this paper are as follows: 
\begin{itemize}
\item With E-3DGS, we introduce the first approach for novel view synthesis from a color event camera that combines 3D Gaussians with event-based supervision. 
\item We present frustum-based initialization, adaptive event windows, isotropic 3D Gaussian regularization and 3D camera pose refinement, and demonstrate that high-quality results can be obtained. %

\item Finally, we introduce new synthetic and real event datasets for large scenes to the community to study novel view synthesis in this new problem setting. 
\end{itemize}
Our experiments demonstrate systematically superior results compared to EventNeRF \cite{eventnerf} and other baselines. 
The source code and dataset of E-3DGS are released\footnote{\url{https://4dqv.mpi-inf.mpg.de/E3DGS/}}. 





%
\section{Related Work}
\label{lit_review}

\begin{highlight}
{

Our research builds upon {\em (i)} Assessing Web Accessibility, {\em (ii)} End-User Accessibility Repair, and {\em (iii)} Developer Tools for Accessibility.

\subsection{Assessing Web Accessibility}
From the earliest attempts to set standards and guidelines, web accessibility has been shaped by a complex interplay of technical challenges, legal imperatives, and educational campaigns. Over the past 25 years, stakeholders have sought to improve digital inclusion by establishing foundational standards~\cite{chisholm2001web, caldwell2008web}, enforcing legal obligations~\cite{sierkowski2002achieving, yesilada2012understanding}, and promoting a broader culture of accessibility awareness among developers~\cite{sloan2006contextual, martin2022landscape, pandey2023blending}. 
Despite these longstanding efforts, systemic accessibility issues persist. According to the 2024 WebAIM Million report~\cite{webaim2024}, 95.9\% of the top one million home pages contained detectable WCAG violations, averaging nearly 57 errors per page. 
These errors take many forms: low color contrast makes the interface difficult for individuals with color deficiency or low vision to read text; missing alternative text leaves users relying on screen readers without crucial visual context; and unlabeled form inputs or empty links and buttons hinder people who navigate with assistive technologies from completing basic tasks. 
Together, these accessibility issues not only limit user access to critical online resources such as healthcare, education, and employment but also result in significant legal risks and lost opportunities for businesses to engage diverse audiences. Addressing these pervasive issues requires systematic methods to identify, measure, and prioritize accessibility barriers, which is the first step toward achieving meaningful improvements.

Prior research has introduced methods blending automation and human evaluation to assess web accessibility. Hybrid approaches like SAMBA combine automated tools with expert reviews to measure the severity and impact of barriers, enhancing evaluation reliability~\cite{brajnik2007samba}. Quantitative metrics, such as Failure Rate and Unified Web Evaluation Methodology, support large-scale monitoring and comparative analysis, enabling cost-effective insights~\cite{vigo2007quantitative, martins2024large}. However, automated tools alone often detect less than half of WCAG violations and generate false positives, emphasizing the need for human interpretation~\cite{freire2008evaluation, vigo2013benchmarking}. Recent progress with large pretrained models like Large Language Models (LLMs)~\cite{dubey2024llama,bai2023qwen} and Large Multimodal Models (LMMs)~\cite{liu2024visual, bai2023qwenvl} offers a promising step forward, automating complex checks like non-text content evaluation and link purposes, achieving higher detection rates than traditional tools~\cite{lopez2024turning, delnevo2024interaction}. Yet, these large models face challenges, including dependence on training data, limited contextual judgment, and the inability to simulate real user experiences. These limitations underscore the necessity of combining models with human oversight for reliable, user-centered evaluations~\cite{brajnik2007samba, vigo2013benchmarking, delnevo2024interaction}. 

Our work builds on these prior efforts and recent advancements by leveraging the capabilities of large pretrained models while addressing their limitations through a developer-centric approach. CodeA11y integrates LLM-powered accessibility assessments, tailored accessibility-aware system prompts, and a dedicated accessibility checker directly into GitHub Copilot---one of the most widely used coding assistants. Unlike standalone evaluation tools, CodeA11y actively supports developers throughout the coding process by reinforcing accessibility best practices, prompting critical manual validations, and embedding accessibility considerations into existing workflows.
% This pervasive shortfall reflects the difficulty of scaling traditional approaches---such as manual audits and automated tools---that either demand immense human effort or lack the nuanced understanding needed to capture real-world user experiences. 
%
% In response, a new wave of AI-driven methods, many powered by large language models (LLMs), is emerging to bridge these accessibility detection and assessment gaps. Early explorations, such as those by Morillo et al.~\cite{morillo2020system}, introduced AI-assisted recommendations capable of automatic corrections, illustrating how computational intelligence can tackle the repetitive, common errors that plague large swaths of the web. Building on this foundation, Huang et al.~\cite{huang2024access} proposed ACCESS, a prompt-engineering framework that streamlines the identification and remediation of accessibility violations, while López-Gil et al.~\cite{lopez2024turning} demonstrated how LLMs can help apply WCAG success criteria more consistently---reducing the reliance on manual effort. Beyond these direct interventions, recent work has also begun integrating user experiences more seamlessly into the evaluation process. For example, Huq et al.~\cite{huq2024automated} translate user transcripts and corresponding issues into actionable test reports, ensuring that accessibility improvements align more closely with authentic user needs.
% However, as these AI-driven solutions evolve, researchers caution against uncritical adoption. Othman et al.~\cite{othman2023fostering} highlight that while LLMs can accelerate remediation, they may also introduce biases or encourage over-reliance on automated processes. Similarly, Delnevo et al.~\cite{delnevo2024interaction} emphasize the importance of contextual understanding and adaptability, pointing to the current limitations of LLM-based systems in serving the full spectrum of user needs. 
% In contrast to this backdrop, our work introduces and evaluates CodeA11y, an LLM-augmented extension for GitHub Copilot that not only mitigates these challenges by providing more consistent guidance and manual validation prompts, but also aligns AI-driven assistance with developers’ workflows, ultimately contributing toward more sustainable propulsion for building accessible web.

% Broader implications of inaccessibility—legal compliance, ethical concerns, and user experience
% A Historical Review of Web Accessibility Using WAVE
% "I tend to view ads almost like a pestilence": On the Accessibility Implications of Mobile Ads for Blind Users

% In the research domain, several methods have been developed to assess and enhance web accessibility. These include incorporating feedback into developer tools~\cite{adesigner, takagi2003accessibility, bigham2010accessibility} and automating the creation of accessibility tests and reports for user interfaces~\cite{swearngin2024towards, taeb2024axnav}. 

% Prior work has also studied accessibility scanners as another avenue of AI to improve web development practices~\cite{}.
% However, a persistent challenge is that developers need to be aware of these tools to utilize them effectively. With recent advancements in LLMs, developers might now build accessible websites with less effort using AI assistants. However, the impact of these assistants on the accessibility of their generated code remains unclear. This study aims to investigate these effects.

\subsection{End-user Accessibility Repair}
In addition to detecting accessibility errors and measuring web accessibility, significant research has focused on fixing these problems.
Since end-users are often the first to notice accessibility problems and have a strong incentive to address them, systems have been developed to help them report or fix these problems.

Collaborative, or social accessibility~\cite{takagi2009collaborative,sato2010social}, enabled these end-user contributions to be scaled through crowd-sourcing.
AccessMonkey~\cite{bigham2007accessmonkey} and Accessibility Commons~\cite{kawanaka2008accessibility} were two examples of repositories that store accessibility-related scripts and metadata, respectively.
Other work has developed browser extensions that leverage crowd-sourced databases to automatically correct reading order, alt-text, color contrast, and interaction-related issues~\cite{sato2009s,huang2015can}.

One drawback of collaborative accessibility approaches is that they cannot fix problems for an ``unseen'' web page on-demand, so many projects aim to automatically detect and improve interfaces without the need for an external source of fixes.
A large body of research has focused on making specific web media (e.g., images~\cite{gleason2019making,guinness2018caption, twitterally, gleason2020making, lee2021image}, design~\cite{potluri2019ai,li2019editing, peng2022diffscriber, peng2023slide}, and videos~\cite{pavel2020rescribe,peng2021say,peng2021slidecho,huh2023avscript}) accessible through a combination of machine learning (ML) and user-provided fixes.
Other work has focused on applying more general fixes across all websites.

Opportunity accessibility addressed a common accessibility problem of most websites: by default, content is often hard to see for people with visual impairments, and many users, especially older adults, do not know how to adjust or enable content zooming~\cite{bigham2014making}.
To this end, a browser script (\texttt{oppaccess.js}) was developed that automatically adjusted the browser's content zoom to maximally enlarge content without introducing adverse side-effects (\textit{e.g.,} content overlap).
While \texttt{oppaccess.js} primarily targeted zoom-related accessibility, recent work aimed to enable larger types of changes, by using LLMs to modify the source code of web pages based on user questions or directives~\cite{li2023using}.

Several efforts have been focused on improving access to desktop and mobile applications, which present additional challenges due to the unavailability of app source code (\textit{e.g.,} HTML).
Prefab is an approach that allows graphical UIs to be modified at runtime by detecting existing UI widgets, then replacing them~\cite{dixon2010prefab}.
Interaction Proxies used these runtime modification strategies to ``repair'' Android apps by replacing inaccessible widgets with improved alternatives~\cite{zhang2017interaction, zhang2018robust}.
The widget detection strategies used by these systems previously relied on a combination of heuristics and system metadata (\textit{e.g.,} the view hierarchy), which are incomplete or missing in the accessible apps.
To this end, ML has been employed to better localize~\cite{chen2020object} and repair UI elements~\cite{chen2020unblind,zhang2021screen,wu2023webui,peng2025dreamstruct}.

In general, end-user solutions to repairing application accessibility are limited due to the lack of underlying code and knowledge of the semantics of the intended content.

\subsection{Developer Tools for Accessibility}
Ultimately, the best solution for ensuring an accessible experience lies with front-end developers. Many efforts have focused on building adequate tooling and support to help developers with ensuring that their UI code complies with accessibility standards.

Numerous automated accessibility testing tools have been created to help developers identify accessibility issues in their code: i) static analysis tools, such as IBM Equal Access Accessibility Checker~\cite{ibm2024toolkit} or Microsoft Accessibility Insights~\cite{accessibilityinsights2024}, scan the UI code's compliance with predefined rules derived from accessibility guidelines; and ii) dynamic or runtime accessibility scanners, such as Chrome Devtools~\cite{chromedevtools2024} or axe-Core Accessibility Engine~\cite{deque2024axe}, perform real-time testing on user interfaces to detect interaction issues not identifiable from the code structure. While these tools greatly reduce the manual effort required for accessibility testing, they are often criticized for their limited coverage. Thus, experts often recommend manually testing with assistive technologies to uncover more complex interaction issues. Prior studies have created accessibility crawlers that either assist in developer testing~\cite{swearngin2024towards,taeb2024axnav} or simulate how assistive technologies interact with UIs~\cite{10.1145/3411764.3445455, 10.1145/3551349.3556905, 10.1145/3544548.3580679}.

Similar to end-user accessibility repair, research has focused on generating fixes to remediate accessibility issues in the UI source code. Initial attempts developed heuristic-based algorithms for fixing specific issues, for instance, by replacing text or background color attributes~\cite{10.1145/3611643.3616329}. More recent work has suggested that the code-understanding capabilities of LLMs allow them to suggest more targeted fixes.
For example, a study demonstrated that prompting ChatGPT to fix identified WCAG compliance issues in source code could automatically resolve a significant number of them~\cite{othman2023fostering}. Researchers have sought to leverage this capability by employing a multi-agent LLM architecture to automatically identify and localize issues in source code and suggest potential code fixes~\cite{mehralian2024automated}.

While the approaches mentioned above focus on assessing UI accessibility of already-authored code (\textit{i.e.,} fixing existing code), there is potential for more proactive approaches.
For example, LLMs are often used by developers to generate UI source code from natural language descriptions or tab completions~\cite{chen2021evaluating,GitHubCopilot,lozhkov2024starcoder,hui2024qwen2,roziere2023code,zheng2023codegeex}, but LLMs frequently produce inaccessible code by default~\cite{10.1145/3677846.3677854,mowar2024tab}, leading to inaccessible output when used by developers without sufficient awareness of accessibility knowledge.
The primary focus of this paper is to design a more accessibility-aware coding assistant that both produces more accessible code without manual intervention (\textit{e.g.,} specific user prompting) and gradually enables developers to implement and improve accessibility of automatically-generated code through IDE UI modifications (\textit{e.g.}, reminder notifications).

}
\end{highlight}



% Work related to this paper includes {\em (i)} Web Accessibility and {\em (ii)} Developer Practices in AI-Assisted Programming.

% \ipstart{Web Accessibility: Practice, Evaluation, and Improvements} Substantial efforts have been made to set accessibility standards~\cite{chisholm2001web, caldwell2008web}, establish legal requirements~\cite{sierkowski2002achieving, yesilada2012understanding}, and promote education and advocacy among developers~\cite{sloan2006contextual, martin2022landscape, pandey2023blending}. In the research domain, several methods have been developed to assess and enhance web accessibility. These include incorporating feedback into developer tools~\cite{adesigner, takagi2003accessibility, bigham2010accessibility} and automating the creation of accessibility tests and reports for user interfaces~\cite{swearngin2024towards, taeb2024axnav}. 
% % Prior work has also studied accessibility scanners as another avenue of AI to improve web development practices~\cite{}.
% However, a persistent challenge is that developers need to be aware of these tools to utilize them effectively. With recent advancements in LLMs, developers might now build accessible websites with less effort using AI assistants. However, the impact of these assistants on the accessibility of their generated code remains unclear. This study aims to investigate these effects.

% \ipstart{Developer Practices in AI-Assisted Programming}
% Recent usability research on AI-assisted development has examined the interaction strategies of developers while using AI coding assistants~\cite{barke2023grounded}.
% They observed developers interacted with these assistants in two modes -- 1) \textit{acceleration mode}: associated with shorter completions and 2) \textit{exploration mode}: associated with long completions.
% Liang {\em et al.} \cite{liang2024large} found that developers are driven to use AI assistants to reduce their keystrokes, finish tasks faster, and recall the syntax of programming languages. On the other hand, developers' reason for rejecting autocomplete suggestions was the need for more consideration of appropriate software requirements. This is because primary research on code generation models has mainly focused on functional correctness while often sidelining non-functional requirements such as latency, maintainability, and security~\cite{singhal2024nofuneval}. Consequently, there have been increasing concerns about the security implications of AI-generated code~\cite{sandoval2023lost}. Similarly, this study focuses on the effectiveness and uptake of code suggestions among developers in mitigating accessibility-related vulnerabilities. 


% ============================= additional rw ============================================
% - Paulina Morillo, Diego Chicaiza-Herrera, and Diego Vallejo-Huanga. 2020. System of Recommendation and Automatic Correction of Web Accessibility Using Artificial Intelligence. In Advances in Usability and User Experience, Tareq Ahram and Christianne Falcão (Eds.). Springer International Publishing, Cham, 479–489
% - Juan-Miguel López-Gil and Juanan Pereira. 2024. Turning manual web accessibility success criteria into automatic: an LLM-based approach. Universal Access in the Information Society (2024). https://doi.org/10.1007/s10209-024-01108-z
% - s
% - Calista Huang, Alyssa Ma, Suchir Vyasamudri, Eugenie Puype, Sayem Kamal, Juan Belza Garcia, Salar Cheema, and Michael Lutz. 2024. ACCESS: Prompt Engineering for Automated Web Accessibility Violation Corrections. arXiv:2401.16450 [cs.HC] https://arxiv.org/abs/2401.16450
% - Syed Fatiul Huq, Mahan Tafreshipour, Kate Kalcevich, and Sam Malek. 2025. Automated Generation of Accessibility Test Reports from Recorded User Transcripts. In Proceedings of the 47th International Conference on Software Engineering (ICSE) (Ottawa, Ontario, Canada). IEEE. https://ics.uci.edu/~seal/publications/2025_ICSE_reca11.pdf To appear in IEEE Xplore
% - Achraf Othman, Amira Dhouib, and Aljazi Nasser Al Jabor. 2023. Fostering websites accessibility: A case study on the use of the Large Language Models ChatGPT for automatic remediation. In Proceedings of the 16th International Conference on PErvasive Technologies Related to Assistive Environments (Corfu, Greece) (PETRA ’23). Association for Computing Machinery, New York, NY, USA, 707–713. https://doi.org/10.1145/3594806.3596542
% - Zsuzsanna B. Palmer and Sushil K. Oswal. 0. Constructing Websites with Generative AI Tools: The Accessibility of Their Workflows and Products for Users With Disabilities. Journal of Business and Technical Communication 0, 0 (0), 10506519241280644. https://doi.org/10.1177/10506519241280644
% ============================= additional rw ============================================
%
\section{Paper Review Toxicity Dataset}

\subsection{Paper Review Collection }
To collect paper reviews, we utilized the OpenReview API, starting with a comprehensive list of all venues between 2018 and 2023, including conferences and workshops. 
% The diversity of schemas used by different conferences on OpenReview posed a challenge for data collection. 
% To address this, we focused on the most common schemas across various conferences, enabling us to maximize our data collection 
Consequently, we successfully collected 50,108, spanning 47 unique conferences and workshops 
Among the total collection, we randomly sample 1,495 for the annotations, and 
on average, each review consisted of 1,553 characters. However, due to the heavy human annotations work, we are not using all the processed review. The distribution of the final testing set will be discussed in \S\ref{sec:human_annotation}.

\subsection{Toxicity Review Guideline}
\label{sec:guideline}

Many journals have guidelines and codes of conduct for peer-reviewers,  but the unacceptable behaviour are often difficult to be found on journal websites~\footnote{\url{https://www.nature.com/nature-index/news/linda-beaumont-research-journals-should-take-action-against-toxic-peer-reviews}}. 
ACL Rolling Review is an exception, where a list of lazy review behavior have been mentioned~\footnote{\url{https://aclrollingreview.org/reviewertutorial#6-check-for-lazy-thinking}}. Nevertheless, ``toxic'' behavior have not been clear defined and nevertheless become a norm of peer-review. 
Therefore, from multiple different sources (professional interview, scientific paper, human studies), we have summarized the unacceptable behaviors as follows to define a sentence that can be toxic.  

\paragraph{Emotive Comments.}  
Using emotive or sarcastic language is often the hardest for authors to cope with~\footnote{\url{https://www.nature.com/articles/d41586-020-03394-y}}. 
Such language include the use of the following words or phrases: speaker-oriented adverbs such as surprisingly, obviously and disappointingly, subjective adjectives (e.g. careless), depreciatory modifiers (e.g. even, just), unnecessary expression including narrativizing: (e.g. , ``At this point, I almost stopped reading''), rhetorical questioning (for example, ``Did the authors even read the submission guidelines?''), universalizing (for example, ``As anyone/everyone/any expert knows''), speculating (for example, ``I bet the outlier observations were omitted''), expressive punctuation, such as exclamation marks or scare quotes (for example, ``This is not correct!'').
Similarly, in a professional interview, an experienced journal editor has mentioned that a comment with personal emotion can be toxic, such as using the world ``silly'', or phrase `` they have no idea what they’re doing''. 

\paragraph{Lack of Constructive Feedback.} 
Refers to a situation where feedback provided to someone lacks substance, helpfulness, or guidance for improvement.
Instead of offering specific suggestions, actionable advice, or positive reinforcement, this type of feedback may be vague, overly critical, or unproductive. 
Example1: \textit{``Comparison in experiments looks meaningless.''}
Example2: \textit{``The analysis is shallow.''}
In addition, if the criticism is lack of reference or citations, such sentences are falling into this categories. Example3: \textit{``The paper is not well written and organized.''} 
The reviewer should refer to what problems happens, an alternative sentence with the same criticism can be \textit{``The paper is not well written and organized, examples will be given below.''}

\paragraph{Personal Attack.} 
are directed at the author(s) rather than the nature or quality of the work~\cite{silbiger2019unprofessional}.  Example1: \textit{``The author’s previous work has been unreliable, and this paper is no different. They should find a different career.''} Example2: \textit{``Perhaps the authors are not used to submitting to this conferences as the paper lacks some essential components to it.''}


\paragraph{Excessive Negativity.} A sentence that overly emphasizes flaws of a paper without acknowledging any merits, which can contributes to a toxic review environment. This usually happens in the summary or conclusion of a review.
Example1: \textit{"The paper is not properly written nor well organized; is hard to read with vague contributions and vague positioning with respect to the state of the art."} 
Example2: \textit{``This paper has no significance."}
Example3: \textit{``This paper cannot be revised so as to be suitable for our journal.''}

\subsection{Human Annotation}
\label{sec:human_annotation}
We designed a rigorous annotation process to ensure the reliability of our dataset. Initially, five undergraduate students majoring in computer science identified toxic sentences. These annotations were then verified by three senior researchers. To further enhance agreement among the senior researchers, we conducted a two-phase annotation process: first with 50 sentences, followed by a larger set of 300 sentences.
In each phase, the annotation process was consistent: each annotator independently rated the sentences, after which they discussed any sentences where there was disagreement and could adjust their ratings if necessary. Conducting a smaller set of annotations served two purposes: first, it helped the annotators reach substantial agreement before moving on to the more time-consuming larger set; second, based on Cohen's Kappa score, the two annotators with the highest agreement were selected to annotate the larger set of 300 sentences.
In the initial round of 50 sentences, the Cohen's Kappa scores among the three annotators were 0.74, 0.52, and 0.34, with a Fleiss' Kappa score of 0.55. After discussing the sentences with disagreements, the Fleiss' Kappa score increased to 0.63, and the highest Cohen's Kappa score between two annotators rose to 0.83. The two annotators with the highest Kappa scores were then assigned an additional 300 sentences. In this larger set, the initial Cohen's Kappa score was 0.60, which significantly increased to 0.92 after discussion.
In the experimental section, we will analyze the sentences where the two reviewers failed to reach a final agreement. Ultimately, we combined 36 sentences where all three annotators agreed in the first round and 277 sentences where the two selected annotators agreed in the second round, resulting in a testing set comprising 182 non-toxic and 131 toxic sentences (Table \ref{tab:statistic}). 

\begin{table}[t]
\centering
 \resizebox{0.60\linewidth}{!}{
\begin{tabular}{c|c|c}
    \toprule
     Total & Toxic & Non-Toxic \\
     \midrule
     313 & 131 & 182 \\
    \bottomrule
    \end{tabular}
    }
    \caption{Toxic Review Detection Dataset. The statisic is in sentence level.}
\label{tab:statistic}
\end{table}





\begin{figure}[h]
    \centering
    \includegraphics[width=0.8\linewidth]{images/decision_1.png}
    \caption{Distribution of Six Decision Categories of the Reviews of the \underline{Entire Testing Data}.}
    \label{fig:review_categories_1}
\end{figure}

\begin{figure}[h]
    \centering
    \includegraphics[width=0.8\linewidth]{images/decision_2.png}
    \caption{Distribution of Six Categories of the Reviews of the \underline{Toxic Sentences}.}
    \label{fig:review_categories_2}
\end{figure}

\paragraph{Review Distribution.}
% \begin{figure}
%     \centering
%     \includegraphics[width=0.35\textwidth]{images/decision_1.png}
%     \caption{Distribution of Five Decision Categories of the Reviews the \underline{Entire Testing Data}.}
%     \label{fig:review_categories_1}
%     \vspace{0.8cm} % Adjust this value to reduce the space between figures
%     \includegraphics[width=0.35\textwidth]{images/decision_2.png}
%     \caption{Distribution of Five Categories of the Reviews of the \underline{Toxic Sentences}.}
%     \label{fig:review_categories_2}
% \end{figure}

% The reviews of those sentences included in our finial testing set are grouped in five categories: Strong Reject, Moderate Reject, Weak Reject, Weak Accept, Moderate Accept, Strong Accept. The distribution of each categories for our final testing set is shown in Figure~\ref{fig:review_categories_1}. 
% We also show the distribution of the decision categories of only toxic sentences in Figure~\ref{fig:review_categories_2}. The distribution shift toward more reject decision. 
% This suggests that reviewer who give a reject decision should be more cautious about their decision 

The reviews of the sentences included in our final testing set are categorized into five groups: Strong Reject, Moderate Reject, Weak Reject, Weak Accept, Moderate Accept, and Strong Accept. The distribution of these categories within our final testing set is illustrated in Figure~\ref{fig:review_categories_1}. Additionally, Figure~\ref{fig:review_categories_2} shows the distribution of decision categories specifically for toxic sentences, revealing a shift toward more reject decisions. Even though sentences are analyzed individually, this trend suggests that reviews with overall reject decision receive more toxic comment which is well align with our expectation but could be extremely discouraging for the authors.

\paragraph{Subcategories Distribution.}
We present the distribution of  toxic-subcategories in Figure~\ref{fig:toxic_categories}.
Note that a toxic sentence could belong to multiple categories. 
The most common category is ``lack of constructive feedback'', follows by ``Excessive Negativity'', ``Emotive Comments'', and ``Personal Attack''. 

\begin{figure}[h]
    \centering
    \includegraphics[width=0.8\linewidth]{images/toxic_categories.png}
    \caption{The Distribution of  Toxic-Subcategories.}
    \label{fig:toxic_categories}
\end{figure}

% and the second highest type is weak reject. We then take a closer look at these two subsets regarding the each categories of the toxic review.  

% \begin{figure}
%     \centering
% \includegraphics[width=0.70\linewidth]{images/decision_1.png}
%     \caption{Distribution of Five Decision Categories of the Reviews the Entire Testing Data.}
%     \label{fig:review_categories}
% \end{figure}
% \begin{figure}
%     \centering
% \includegraphics[width=0.70\linewidth]{images/decision_2.png}
%     \caption{Distribution of Five Categories of the Reviews of The Toxic Sentences.}
%     \label{fig:review_categories}
% \end{figure}



% \begin{figure}
%     \centering
%     \begin{minipage}{0.45\textwidth}
%         \centering
%         \includegraphics[width=\textwidth]{images/decision_1.png}
%         \caption{Distribution of Five Decision Categories of the Reviews the \underline{Entire Testing Data}.}
%     \end{minipage}
%     \hspace{3cm} % Adjust this value to reduce the space
%     \begin{minipage}{0.45\textwidth}
%         \centering
%         \includegraphics[width=\textwidth]{images/decision_2.png}
%         \caption{Distribution of Five Categories of the Reviews of the \underline{Toxic Sentences}.}
%     \end{minipage}
% \end{figure}


\paragraph{Discussion Among the Annotators.} 
The main discussion focuses on two key questions: 1) how to differentiate between assertive tones and emotive language, and 2) what constitutes a ``lack of constructive feedback.''

For the first question, some ambiguous sentences include ``Would a superior method here not just be to take this brain information directly?'' and ``The authors lack any clear discussion on how their work is directly relevant to RRL.'' In the first example, one annotator identified "just" as a depreciatory modifier, which falls under the category of \underline{Emotive Comments} (See \S\ref{sec:guideline}). Removing this word does not change the criticism. In the second example, the word ``any'' was discussed, with one annotator noting that removing it would weaken the criticism, categorizing it as an assertive tone. After discussion, the annotators agreed that if a word does not belong to any of the categories outlined in \underline{Emotive Comments} and its removal weakens the criticism, it should be classified as assertive rather than toxic.

For the second question, some ambiguous sentences were ``The proposed copy mechanism is not clear,'' ``The technical contribution of the paper is limited,'' and ``Besides, adding the binary feature in the embedding is not necessary; the LSTM model could learn such sequential correlation.'' The first two examples refer to broad aspects without providing enough detail for the authors to make improvements. In the third example, although the criticism initially seems valid, it is not supported by evidence. Therefore, we concluded that criticisms lacking helpful suggestions for future improvement or not supported by reasons should be considered non-constructive.


\paragraph{Remark.} 
We acknowledge that due to the lack of formal definition of toxicity and few literature are available, our toxicity definition in \S\ref{sec:guideline} can raise different opinion among researchers.  
However, we believe that if a review sentence falls into any categories in our toxicity guideline, it should attract extra attention and the reviewer who writes the sentence should be extra cautious. Furthermore, due to the subject nature of toxicity, the annotations can suffer from inconsistency, therefore, we have designed two stages of annotations and only include those sentences all reviewers are agreed in our final testing set.  


% \subsection{Toxicity Classification Task} 

% \paragraph{Task Formulation.}
% In the classification task, the goal is to classify a sentence as either non-toxic or toxic. Any sentence that belongs to one of the five categories (\S\ref{sec:guideline}) will be classified as toxic. Note that, while we have the five sub-categories of toxicity, we decided to formulate the toxicity classification task as a binary classification for simplicity. Next, we will describe two methods of annotation.


% \paragraph{Data Generation by Human Annotation.}
% We distribute a unique set of sentences to each annotator (five in total). 
% In both the test and training sets, the annotation is completed in two steps: annotate and verify.  
% Specifically, an annotator assigns a label, and then another annotator is called to verify this label.  
% Additionally, when annotators felt unsure about a specific sentence, they had a "consult" category to ask the other annotators to help with the classification. 
% Ultimately, the test set showed a break down of 2839 non toxic sentences to 960 toxic sentences. The annotators marked 26.07\% of sentences toxic. 


% \paragraph{Data Generation by GPT-4 Annotation with Human Verification.} 

% While we mainly depend on human annotation, 
% human annotation is extremely time consuming, and inspired by the success of LLM on annotating data~\cite{he2024if}, we investigate a more efficient way to annotation using GPT-4. 
% We provide an entire review to the GPT-4 API and ask it to output toxic sentences, which are then verified by humans. This process speeds up the annotation process. We use the following prompt along with the review.
% \begin{lstlisting}[linewidth=\columnwidth,breaklines=true,breakautoindent=false,breakindent=0pt]
% Identify sentences which might have a needlessly negative impact on the paper authors and might accordingly benefit from re-wording/re-phrasing. 
% \end{lstlisting}
% \label{fig:prompt1}

% Under both human and machine-human-in-the-loop annotations, we finally construct a toxicity review detection and revision benchmark with the statistic shown in Table~\ref{tab:statistic}. 

% \paragraph{Evaluation Metrics}
% Since the labels are unbalanced, we measured macro F1, precision,  recall for each class. We also use accuracy for standard purpose. 



% \subsection{Detoxify Generation Task} 

% \paragraph{Task Formulation.}
% Given a toxic sentence from a review, the goal is to detoxify the sentence without adding unfaithful content. 

% \paragraph{Data Generation.} We construct a detoxified sentence set for training purposes using a machine-human-in-the-loop method. Specifically, given a toxic sentence, we first prompt ChatGPT to revise it using the prompt below. Then, an annotator reviews the reworded sentence. Only those revisions that improve sentiment while maintaining the original sentence's critical content are approved. Revisions that do not meet this threshold or contain errors or hallucinations are rejected. We provide 2233 toxic sentence revisions, and 1402 human-reviewed revisions remain in the training set.
% For the testing set, we only provide the toxic sentences (i.e. 960) but not the revised version, as there could be many way to revise, simply provide one revision will not be sufficient for the evaluation. 

% \begin{lstlisting}[linewidth=\columnwidth,breaklines=true,breakautoindent=false,breakindent=0pt]
% This text was found in a Paper Review.
% Text: "{sentence}"
% Reword this sentence such that it delivers the critism in a friendly but professional and encouraging manner. 
% Make minimal changes to the original text.
% Reworded Sentence:
% \end{lstlisting}
%     % \caption{Our prompt for generating synthetic data with GPT-4.}
% \label{fig:prompt2}

% \paragraph{Evaluation Metric.}
% We employ human to compare the original toxic sentence and revised sentence and select the sentence by 
% Besides, we also design two automatic judgement.
% Given an original toxic sentence and a detoxified sentence, we ask ChatGPT which is less toxic. The prompt of ChatGPT is given in Appendix. 
% Furthermore, we also use our fine-tuned toxic detection model (see \S\ref{sec:methods}) to obtain the toxic rate of the revised sentence,
% a lower rate indicates less toxic exhibit in the revised sentence.  




%
\section{Experimental Setup}

We evaluate state-of-the-art IARs: VAR-\textit{d}\{16, 20, 24, 30\} (\textit{d} = model depth), RAR-\{B, L, XL, XXL\}, MAR-\{B, L, H\}, trained for class-conditioned generation. The IARs' sizes cover a broad spectrum between 208M for MAR-B, and 2.1B parameters for \varbig. We use IARs shared by the authors of their respective papers in their repositories, 
with details in~\cref{app:model_details}. As these models were trained on ImageNet-1k~\citep{deng2009imagenet} dataset, we use it to perform our privacy attacks. 
For MIA and DI, we take 10000 samples from the training set as members and also 10000 samples from the validation set as non-members. To perform data extraction attack, we use all images from the training data. Additionally, we leverage the known validation set to check for false positives.



%
\section{Results}
We first describe communication patterns within the full chronological context of the game in \textit{League of Legends (LoL)}, separated into four sections based on changing coordination dynamics. Based on this context, we identify core factors players assess to decide when to participate in communication with other teammates. Afterward, we discuss how communication shapes player perceptions toward their teammates, showing player's wariness towards players actively engaging in communication. 

\subsection{Communication Patterns in Context}

We discuss the communication patterns among teammates within the game. We organize the data into chronological phases of the game for a structured analysis of how the context shapes communication patterns. 

\subsubsection{Pre-game stage}
Before gameplay begins, team communication opens with \textit{team drafting}, where players are assigned roles (Top, Mid, Bot, Support, or Jungle) and take turns picking or banning champions. In Solo Ranked mode, roles are pre-assigned based on player preferences selected before queueing. Once teams are set, all players enter \textit{champion select} stage, alternating champion picks and banning up to five champions per team. During this stage, communication is limited to text chat. The usernames are anonymized (i.e., replacing the name with aliases) to prevent queue dodging by checking third-party stats sites such as OP.GG\footnote{https://www.op.gg/}, leaving the chat as the only option to inform individual strengths and preferences. 

Team composition in \textit{LoL} is crucial to the strategy and outcome of the game~\cite{ong2015player}, setting the basis for future interactions. Most participants acknowledged the importance of balanced and synergistic team composition, especially as players move into higher ranks where team coordination outweighs individual excellence. Yet, we observed a distinct lack of verbal communication between the members during this period across all ranks. Participants attributed the lack of willingness to initiate a conversation on the dangers of starting the game on a bad footing. They prioritized ``not creating friction'' during this stage as negative impressions can propagate throughout the game. Some participants attempted communication to reduce such friction, such as P14, who stated,``\textit{If I had the time, I wanted to say that I will be banning [this Champion], just in case a player on my team wanted to play them.}'' However, several participants viewed any communication during the pre-game phase with wariness, as dissatisfaction or conflict at this step portended negative interactions between players in the game (P3, P9, P15). Thus, even when participants expressed doubt about other teammates' unconventional or non-meta champion picks, they refrained from entering into discourse. This contrasts with findings by Kou and Gui~\cite{kou2014}, which showed players attempt to maintain a harmonious and constructive atmosphere through greetings and introductions.

Another emergent code of the reason for not engaging in communication in the pre-game stage stems from different purposes of playing the game (P1, P5, P13, P16, P17). Despite being in ranked mode, which is more prone to increased competitiveness and effort, participants showed differing goals and levels of interest in winning the game. Several players stated that they had previously exerted great mental load in coordinating synergistic plays, but stopped as they gave less importance to winning at all costs (``\textit{I don't really play to win. I play \textit{LoL} to relieve stress, so I don't engage in chat.}'', P5). These players saw verbal communication with the goal of coordination as an unnecessary or even cumbersome component of the pre-game stage.


\subsubsection{Structured phase}
In many MOBAs, including \textit{LoL}, the early stages of the game play out in a formulaic manner: players join their lanes (Top, Mid, and Bot/Support), defeat minions to gain gold, buy items towards certain ``builds'', kill or assist in early objectives (Jungle), and battle counterparts in their respective lanes. Participants at this stage expressed that most players possessed tacit knowledge of what must be done, such as knowing when to aid their Jungle to capture a jungle monster, choosing the opportune moments to leave their lanes, or positioning wards (i.e., a deployable unit which provides a vision of the surrounding area) at the ideal placements. The participants assumed each player knew their ``role'' to fulfill, often comparing it to ``doing their share'' (P1, P3, P7, P19). In line with this belief, players rarely initiated preemptive or proactive verbal communication for strategic or social purposes at the early stage. 

Pings, on the other hand, constantly permeated the game. At this stage, players used ping to provide information relevant to others from their position, such as letting others know if an enemy went missing from their lane. As the players are largely separated and independent from one another, pings (coupled with the minimap and scoreboard) served as the primary channel for maintaining context over the game rather than as warnings or direct guidance to the players. For other non-verbal gestures, while objective votes would occasionally appear, they were rarely answered. Instead, relevant players near the objective would place pings or move toward it to help out their teammates.

Participants viewed the structured phase as a routine, but uncertain period of the game where the pendulum could swing in either team's favor. Players --- especially Jungles who roam the board looking for opportunities to ambush the enemy team in lanes (``gank'') --- sometimes felt hesitant to make calls and demands at this stage since ``\textit{[they] could make a call, but if I fail, they'll start blaming my decisions down the line.}'' (P7) But at this stage, participants believed that they held personal agency over the final game outcome. P1 and P6 stated that they entered the game with the mindset that only they had to succeed regardless of the performance of their teammates. This belief was reflected in their chatting behavior, where players prioritized focusing on their circumstances over the team's (``\textit{I mute the chat so that I don't get swayed by the team, as I can win the game if I do well.}'', P9).


\subsubsection{Group engagement phase}
As the game enters its middle phase, it provides opportunities for more diverse decision-making. Players may swap lanes, seize or trade crucial objectives, and fight in large battles involving multiple champions. At this point, teams typically have a clear outlook on which players and team have the advantage, requiring more team-driven decisions to maintain or overcome their current standing. Thus, players used verbal communication to discuss more complicated tactics that could not be effectively conveyed through pings.

But more often than not, chat messages became judgment-based. As enemy engagement with larger groups occurred more frequently, the availability for chatting would come after death, which led to comments on past actions rather than future choices. Additionally, the respawn timer for deaths becomes longer as the game progresses, providing more time to observe other players than in earlier phases. This gave players more opportunities to express dissatisfaction specifically towards certain plays, such as placing Enemy Missing pings on the map where other teammates are located to bring attention to their questionable play.

This stage also gave much more exposure of each other to the allies as the team would gather at a single point, giving way to greater scrutiny by their teammates. Repeated or critical mistakes put participants on edge, as they braced for criticism from their teammates. They expressed relief or surprise when the chat remained silent or civil, with P8 stating ``\textit{I messed up there. No one is saying anything, thankfully.}''


\subsubsection{Point of no return}
Meanwhile, verbal communication flowed out when the game had a clear trajectory to the end. Previous research has shown that both toxic and non-toxic communication skyrockets near the end of the game~\cite{kwak2015linguistic} when the players have determined the game outcome with certainty. We saw that this phase opened up both positive and negative sides of communication for guaranteed win and loss, respectively. The winning team would compliment and cheer each other through chat messages and emotes, while the losing side devolved into arguments and calling out. The communication at this stage was driven by emotion, showing excitement or venting frustration.


\subsection{Communication Assessment Process}

We describe the factors that users mainly focus on to assess when or when not to involve themselves in communication with their teammates. 

\subsubsection{Calculating communication cost}
One of the most proximate factors behind when communication is performed is the limited action economy of the game. In \textit{LoL} and other MOBAs, players can rarely afford time to type out messages due to the fast-paced nature of the game. In time-sensitive scenarios, the time pressure makes communication particularly costly. It is therefore unsurprising that much of the communication occurs after major events (e.g., battles and objective hunting), as players are given more downtime while waiting for teammates or enemies to respawn or regroup.

For periods where players were still actively involved in gameplay, the players made conscious decisions on choosing which communication media to use based on the perceived action availability and the importance of communicating the message. Players relied on pings for non-critical indications, believing that the mutual understanding of the game would get the message across. However, many players recognized that pings were prone to be missed, ignored, or misinterpreted by their allies (P2, P9, P16, P17, P20). Subsequently, participants typed out information considered to be too important to the situation to be misunderstood or missed by other players even if it caused delays in their gameplay (P10, P11, P14). Simultaneously, the priority of importance constantly shifted --- we observed multiple times participants start to type, but stop to react to an ongoing play, only to never send out their message.

\subsubsection{Evaluating relevance and responsiveness}
When the brief window of communication opportunity is missed, players are unlikely to ever send out that information. In \textit{LoL}, situations can change within seconds and certain communication media cannot keep up with the changing state of the game. For example, almost all study participants did not participate in votes for objectives. Among the tens of objective votes initiated among all the games in this study, no objective vote saw more than three votes, frequently being left with no vote beyond the player who initiated the vote. Some players, when asked why they did not participate, stated that the votes they made often became irrelevant as the game state had changed during the time it took to vote (P2, P11). Other players also discussed how information conveyed through communication can get outdated fast (P1, P8, P9). 

\begin{quote}
I can't always follow through with what I say [in the chat] since the game is really dynamic. My teammates don't understand such situations, so I tend to not chat proactively. - P9
\end{quote}

Thus, some players instead preferred to react through direct action (P8, P10, P11, P16, P20). P10 stated, ``\textit{I think it's enough to show through action rather than [using objective voting]. I can look out for how the player reacts when I request something from them.}''

On the other hand, such action-based responses left the player to assess whether and how the communication was received. P10 stated that they tried to predict whether a player understood their ping direction by how they moved, but it was hard to interpret their intent: ``\textit{members sometimes seem to move towards me but then turn around, and sometimes they even ping back but don't come.}''. P16 discussed how they weren't sure whether the ping was received, but performed it anyway since it felt helpful.

Similarly, participation in surrender votes (or lack thereof) carried different intent by the player. During most of the games that ended in a loss, one or more surrender votes were called by the participant's team. However, only two surrender votes achieved four or more players' participation. However, the reasons why a player chose to not participate varied. Some had decided to wait and see how other teammates voted, which may have paradoxically led many members to not participate in the vote (P4, P9). Meanwhile, others didn't reply as they didn't think the vote was actually calling for a response: P13 stated, ``\textit{I didn't vote because they were just showing their anger. It's just a member venting through a surrender vote that they're not doing well.}''

\subsubsection{Balancing information access and psychological safety}
While recognizing that communication would be useful or even necessary in certain situations, participants also put their psychological safety first over information access. Some players, worn down by the normalization of toxic communication such as flaming, muted the chat (P1, P9).

Many participants expressed the sentiment of ``protecting [their] mentality'', describing how certain communication harmed their psychological well-being. This communication did not always refer to negative communication; P9 often muted players who gave commands as they did not want to be ``swept up'' by others' play-related judgments. This separation even extended to other more widely considered essential communication forms, such as pings. Even after acknowledging that pings were vital and useful to the game, P9 went as far as muting the ping of the support player in the same lane after they sent a barrage of Enemy Missing pings that signified aggression and criticism. 

Additionally, the abundance and high frequency of communication also strained the limited mental capacity of the players. Many players, when asked why they had not replied to an objective vote or other chat messages, stated that they simply did not notice them among other events happening (P1, P2, P3, P9, P12, P15, P18, P20, P21). The information overload caused stress and became distracting to players.

\subsubsection{Reducing potential friction}
As demonstrated in the pre-game stage of the game, players sometimes used communication to minimize friction between their teammates. Some participants sacrificed time to apologize to other players when they believed themselves to be at fault. When asked why, P12 replied, ``\textit{There are too many people who don't come to help gank if I don't apologize.}''. Similarly, P5 sacrificed time typing in an apology after a teammate had died despite still being in the middle of a fight as they didn't wish to give the other player a reason to start an attack.

However, some noted that silence is sometimes the best answer to a negative situation. P4, after dying to the enemy, put into chat ``Fighting!'' (roughly meaning, ``We can do it!''). They stated ``\textit{I don't know why I do it... it probably angers [my teammates] more.
}'' They also stated that ``\textit{for certain people, talking in the chat only spurs them more. You just have to let them be.}'' Other players shared similar sentiments that being quiet and dedicating focus to the game was a better choice (P1, P11, P14).

For female players, the fear of gender-based harassment shaped their communication patterns. While \textit{LoL} does not provide any demographic information of a player to other players, almost all female participants noted experiences of receiving derogatory remarks or doubts about their abilities based on other players' assumptions of their gender, a trend frequently seen in male-dominated online gaming cultures~\cite{fox2016women, norris2004gender, mclean2019female}. They noted that the players were able to correctly guess their gender when the participant's role and champion fit into the preconceived notions of what women ``tended to play'' (i.e., female-identifying support champions, such as Lulu) or their username ``seemed feminine'' (P18, P19, P20, P21, P22). This led to certain players adopting tactics that signaled male-like behavior, such as changing their speech style to be more gender-neutral or male-like (P19, P21) and changing their username to sound more gender-neutral. Cote describes similar instances of ``camouflaging gender'' as one of five main strategies for women coping with harassment~\cite{cote2017coping}. However, some players opted to keep playing their preferred character or maintaining their username even if it signaled their gender, such as P21 who expressed, ``\textit{I cherish and feel attached to my username, so I don’t want to change it just because of [harassment and inappropriate comments].}'' These players valued self-expression and identity even at the risk of increased risk to unpleasant communication experiences.


\subsubsection{Forming performance-based hierachy}
Naturally formed leadership has often been observed in other works on \textit{LoL} teams~\cite{kou2014}. Kim et al. showed that more hierarchy in in-game decision-making led to higher collective intelligence~\cite{kim2017}. While they used ``hierarchy'' to mean varying amounts of communication throughout the game, we observed that the hierarchy extends further to performance-based hierarchy, where teammates in more advantageous positions are given greater weight when communicating with other players. Players actively chose to refrain from suggesting strategic plans when they were ``holding down the team'', recognizing that they held less power and trust among the team members (P8, P10, P12, P14, P22). The player who was losing against the enemy team was viewed as having no ``right'' to lead the team, which was reserved for well-performing players.


\subsubsection{Enforcing norms and habits}
One of the most common answers to why players performed certain communication actions, especially non-verbal actions such as pings and emotes, was ``a force of habit'' (P6, P7, P8, P9, P10, P12, P17). Players formed learned practices of using communication channels at certain points by observing other players exhibit the same behaviors. This promoted, for example, replying to an emote sent by the teammate with their own or pinging readied skills and items to emphasize relevant information for other players throughout the game. 

On the other hand, this meant that players were averse to communication patterns outside of the norm --- participants stated that they had a hard time adapting to new forms of communication, seeing no immediate benefit or impact from using them (P1, P8, P14, P13, P15, P17). Most egregiously, the recently introduced objective pings were largely viewed to be awkward to use and unnecessary (P1, P4, P8, P12).


\subsection{Impact of Communication Assessment}
We describe how the communication patterns and assessment of the players impact the individual players' perspectives on team dynamics.

\subsubsection{Relationship between trust and communication frequency}
Most participants saw value in constant and well-informed communication but with an important distinction: verbal communication with strangers rarely ended well. Players largely recognized frequent verbal communication to burgeon conflict, regardless of the message within. Even when players understood the helpful intent behind positive messages from the players, they compared actively talking players to be possible bad actors who were likely to exhibit toxic behaviors when the game turned against them. (P1, P4, P8, P12, P14)

\begin{quote}
I need to make sure to not disturb Twisted Fate. I saw him start to flame. It's not because I don't want to hear more criticism. I know these types. The more I react and chat with them, the more deviant they will become. - P4  
\end{quote}

Similarly, P19 lamented that players used to socialize more in the chat during the pre-game phase to build a fun and prosocial environment, noting a memorable example of encouraging each other to do well on their academic exams, but noted that such prosocial behavior has become much rarer during the recent seasons. They noted that there are inevitably players ``who take it negatively'' and thus stopped proactively typing non-game related messages in the chat.

Ultimately, players desired assurance and trust of player commitment. The participants trusted actions more than words to prove that the player remained dedicated to the game. Both P10 and P17 pointed out that it was easy to tell who was still ``in the game'' and motivated to try their best and that ``staying on the keyboard'' likely meant that they weren't invested or focused on the game. Players viewed such commitment to be the most important aspect of a ``good'' teammate, sometimes even more than their skill or performance (P9, P14). It is interesting to note that unlike what previous literature may suggest~\cite{marlow2018}, players' averseness to talkative teammates had less to do with the cognitive overload or distraction caused by the frequent communication, but rather due to the threats of future team breakdown. This view in turn also affected how players decided to communicate or not, as they believed that players would not take their suggestions or comments in a positive light. 


\subsubsection{Perception of player commitment and fortitude}

Communication also acted as a mirror of their teammates' mental fortitude. A number of players mentioned how they valued a resilient mindset in their teammates playing the game, referring to players who remained committed to the game until the very end. They saw players who provoked or complained to teammates as ``having a weak mentality'' who had been altered by the bad outcomes of the game to act in an unhelpful manner towards the team through their communication. The communication actions of the teammate informed the participants of how steadfast their teammate remained in disadvantageous situations.  

\begin{quote}
It's not like I constantly reply in the chat or anything, but I pay attention [to the chat] to grasp the overall atmosphere of the team. If the team doesn't collaborate well then we lose, so I try to have a rough understanding of the mentality of the other players. - P13
\end{quote}

There were also instances of communication that helped players maintain a positive view of their teammates. For example, P11 mentioned near the beginning of the game, ``\textit{Looking at the chat, Varus player has strong mentality [for being so positive]. There were lots of points [in his support's] plays that he could have criticized.}'' Unfortunately, this view quickly soured when the Varus player devolved into criticism later in the late game phase where the Varus player started criticizing the support and other players. P11 then noted that the Varus player seemed to merely be ``bearing through the game''.
%
\section{Conclusion}
\label{sec:conclusion}
%

We present the massively multilingual vision-language benchmark MVL-SIB for cross-modal (and text-only topic matching) in 205 languages that offers key advantages over prior multilingual VL benchmarks. Notably, it covers over 100 additional languages without relying on machine translation. MVL-SIB allows for a clear separation between textual language support and vision-language support in LVLMs by comparing performance on mirrored cross-modal and text-only tasks. Moreover, it allows us to study how LVLMs handle single-image versus multi-image formulations of cross-modal topic matching by varying the number of images provided.
In our comparative evaluation of state-of-the-art LVLMs on MVL-SIB, we find that model performance is strongly correlated with both model size and the volume of available pre-training data for each language. However, all LVLMs experience a dramatic performance drop on the lowest-resource languages. Our analysis further reveals that vision-language support deteriorates disproportionately relative to language support, highlighting the need to incorporate low-resource languages into VL training. Moreover, providing multiple images does not benefit open-weight LVLMs in cross-modal topic matching, suggesting that LVLMs are not yet fully effective in multi-image tasks. Lastly, we validate that MVL-SIB correlates well with existing multilingual VL benchmarks, underscoring its reliability as a source of evaluation data for 205 languages.
%
\section{Limitations}
\label{sec:limitations}
%

% more LVLMs
% limited compute, various scales
% more images, western centric
Our work faces three primary limitations. First, although a vast number of LVLMs exist, we selected a representative subset based on key criteria. Specifically, the LVLMs in our study (Qwen2-VL and InternVL, with the exception of Centurio) span a range of parameter counts typical of LLMs. Additionally, we include GPT-4o-mini in the full evaluation and GPT-4o on the subsampled language tiers. Evaluating MVL-SIB across all four tasks \its, \sti, \tts, and \stt (cf. \S\ref{sec:tasks}) at various $k \in \{1,3,5\}$   over 205 languages (i.e., evaluations per model and task, or $2050$, in sum per model) becomes computationally intractable.  This accumulates to $3 \times 205 = 615$ evaluations per model ($205$ for \tts as only $k{=}1$ reference topic exists) or $3 \times 205 + 205 = 2050$ evaluations in total. We therefore both provide subsets of the language tiers to evaluate on and demonstrate that evaluation only requires 1K instances to reliably estimate task performance.
Second, while we strove to choose a diverse set of images to capture the full semantic range of each topic, further diversification is possible by sourcing additional images. However, due to the limited availability of openly licensed images, some topics (e.g., \texttt{politics} and \texttt{entertainment}) are represented predominantly by images that embody the topic in a more Western-centric cultural context. Hand-selecting images by topic for each language or, more broadly, cultural groups would not scale to 205 languages and would hinder the comparability of results. Our results furthermore confirm that models just as well perform on a broad range of languages spanning diverse cultural backgrounds as on English (cf. Figure \ref{fig:intro-img2sent}). At the same time, LVLMs perform best on Western-centric images, mitigating any variation that would originate from using more culture-specific images.
Finally, for the topic \texttt{geography}, we manually selected images that are representative within the context of SIB, as the broader definition of geography is too diffuse to capture visually.
%
\section*{Acknowledgements}
\label{sec:ack}
%

We used AI assistance (chatGPT o3-mini) to polish the writing and the tables of the manuscript as well as to refine the code for our visualizations.
%

% Bibliography entries for the entire Anthology, followed by custom entries
%\bibliography{anthology,custom}
% Custom bibliography entries only
\bibliography{custom}

\appendix

\section{Appendix}
\label{sec:appendix}




\begin{table*}[ht]
    \subsection{Images Per Topic}
    \label{app:images-per-topic}
    \centering
    \begin{tabular}{c}
       \toprule 
            \textbf{Entertainment}  \\ \cmidrule(lr){1-1}
                    \begin{minipage}{0.08\textwidth}
                        \centering
                        \includegraphics[width=\linewidth]{latex/gfx/images_by_topic/entertainment_0_scaled.jpg}
                        % \vspace{0.5em}
                        % \textbf{(a)}
                    \end{minipage}
                    \hfill
                    \begin{minipage}{0.08\textwidth}
                        \centering
                        \includegraphics[width=\linewidth]{latex/gfx/images_by_topic/entertainment_1_scaled.jpg}
                    \end{minipage}
                    \hfill
                    \begin{minipage}{0.08\textwidth}
                        \centering
                        \includegraphics[width=\linewidth]{latex/gfx/images_by_topic/entertainment_2_scaled.jpg}
                    \end{minipage}
                    \hfill
                    \begin{minipage}{0.08\textwidth}
                        \centering
                        \includegraphics[width=\linewidth]{latex/gfx/images_by_topic/entertainment_3_scaled.jpg}
                    \end{minipage}
                    \hfill
                    \begin{minipage}{0.08\textwidth}
                        \centering
                        \includegraphics[width=\linewidth]{latex/gfx/images_by_topic/entertainment_4_scaled.jpg}
                    \end{minipage}
                    \hfill
                    \begin{minipage}{0.08\textwidth}
                        \centering
                        \includegraphics[width=\linewidth]{latex/gfx/images_by_topic/entertainment_5_scaled.jpg}
                    \end{minipage}
                    \hfill
                    \begin{minipage}{0.08\textwidth}
                        \centering
                        \includegraphics[width=\linewidth]{latex/gfx/images_by_topic/entertainment_6_scaled.jpg}
                    \end{minipage}
                    \hfill
                    \begin{minipage}{0.08\textwidth}
                        \centering
                        \includegraphics[width=\linewidth]{latex/gfx/images_by_topic/entertainment_7_scaled.jpg}
                    \end{minipage}
                    \begin{minipage}{0.08\textwidth}
                        \centering
                        \includegraphics[width=\linewidth]{latex/gfx/images_by_topic/entertainment_8_scaled.jpg}
                    \end{minipage}
                    \begin{minipage}{0.08\textwidth}
                        \centering
                        \includegraphics[width=\linewidth]{latex/gfx/images_by_topic/entertainment_9_scaled.jpg}
                    \end{minipage}
                \\
    
            \textbf{Geography}  \\ \cmidrule(lr){1-1}
                    \begin{minipage}{0.08\textwidth}
                        \centering
                        \includegraphics[width=\linewidth]{latex/gfx/images_by_topic/geography_0_scaled.jpg}
                        % \vspace{0.5em}
                        % \textbf{(a)}
                    \end{minipage}
                    \hfill
                    \begin{minipage}{0.08\textwidth}
                        \centering
                        \includegraphics[width=\linewidth]{latex/gfx/images_by_topic/geography_1_scaled.jpg}
                    \end{minipage}
                    \hfill
                    \begin{minipage}{0.08\textwidth}
                        \centering
                        \includegraphics[width=\linewidth]{latex/gfx/images_by_topic/geography_2_scaled.jpg}
                    \end{minipage}
                    \hfill
                    \begin{minipage}{0.08\textwidth}
                        \centering
                        \includegraphics[width=\linewidth]{latex/gfx/images_by_topic/geography_3_scaled.jpg}
                    \end{minipage}
                    \hfill
                    \begin{minipage}{0.08\textwidth}
                        \centering
                        \includegraphics[width=\linewidth]{latex/gfx/images_by_topic/geography_4_scaled.jpg}
                    \end{minipage}
                    \hfill
                    \begin{minipage}{0.08\textwidth}
                        \centering
                        \includegraphics[width=\linewidth]{latex/gfx/images_by_topic/geography_5_scaled.jpg}
                    \end{minipage}
                    \hfill
                    \begin{minipage}{0.08\textwidth}
                        \centering
                        \includegraphics[width=\linewidth]{latex/gfx/images_by_topic/geography_6_scaled.jpg}
                    \end{minipage}
                    \hfill
                    \begin{minipage}{0.08\textwidth}
                        \centering
                        \includegraphics[width=\linewidth]{latex/gfx/images_by_topic/geography_7_scaled.jpg}
                    \end{minipage}
                    \begin{minipage}{0.08\textwidth}
                        \centering
                        \includegraphics[width=\linewidth]{latex/gfx/images_by_topic/geography_8_scaled.jpg}
                    \end{minipage}
                    \begin{minipage}{0.08\textwidth}
                        \centering
                        \includegraphics[width=\linewidth]{latex/gfx/images_by_topic/geography_9_scaled.jpg}
                    \end{minipage}
                \\
    
            \textbf{Health}  \\ \cmidrule(lr){1-1}
                    \begin{minipage}{0.08\textwidth}
                        \centering
                        \includegraphics[width=\linewidth]{latex/gfx/images_by_topic/health_0_scaled.jpg}
                        % \vspace{0.5em}
                        % \textbf{(a)}
                    \end{minipage}
                    \hfill
                    \begin{minipage}{0.08\textwidth}
                        \centering
                        \includegraphics[width=\linewidth]{latex/gfx/images_by_topic/health_1_scaled.jpg}
                    \end{minipage}
                    \hfill
                    \begin{minipage}{0.08\textwidth}
                        \centering
                        \includegraphics[width=\linewidth]{latex/gfx/images_by_topic/health_2_scaled.jpg}
                    \end{minipage}
                    \hfill
                    \begin{minipage}{0.08\textwidth}
                        \centering
                        \includegraphics[width=\linewidth]{latex/gfx/images_by_topic/health_3_scaled.jpg}
                    \end{minipage}
                    \hfill
                    \begin{minipage}{0.08\textwidth}
                        \centering
                        \includegraphics[width=\linewidth]{latex/gfx/images_by_topic/health_4_scaled.jpg}
                    \end{minipage}
                    \hfill
                    \begin{minipage}{0.08\textwidth}
                        \centering
                        \includegraphics[width=\linewidth]{latex/gfx/images_by_topic/health_5_scaled.jpg}
                    \end{minipage}
                    \hfill
                    \begin{minipage}{0.08\textwidth}
                        \centering
                        \includegraphics[width=\linewidth]{latex/gfx/images_by_topic/health_6_scaled.jpg}
                    \end{minipage}
                    \hfill
                    \begin{minipage}{0.08\textwidth}
                        \centering
                        \includegraphics[width=\linewidth]{latex/gfx/images_by_topic/health_7_scaled.jpg}
                    \end{minipage}
                    \begin{minipage}{0.08\textwidth}
                        \centering
                        \includegraphics[width=\linewidth]{latex/gfx/images_by_topic/health_8_scaled.jpg}
                    \end{minipage}
                    \begin{minipage}{0.08\textwidth}
                        \centering
                        \includegraphics[width=\linewidth]{latex/gfx/images_by_topic/health_9_scaled.jpg}
                    \end{minipage}
                \\
                \textbf{Politics}  \\ \cmidrule(lr){1-1}
                    \begin{minipage}{0.08\textwidth}
                        \centering
                        \includegraphics[width=\linewidth]{latex/gfx/images_by_topic/politics_0_scaled.jpg}
                        % \vspace{0.5em}
                        % \textbf{(a)}
                    \end{minipage}
                    \hfill
                    \begin{minipage}{0.08\textwidth}
                        \centering
                        \includegraphics[width=\linewidth]{latex/gfx/images_by_topic/politics_1_scaled.jpg}
                    \end{minipage}
                    \hfill
                    \begin{minipage}{0.08\textwidth}
                        \centering
                        \includegraphics[width=\linewidth]{latex/gfx/images_by_topic/politics_2_scaled.jpg}
                    \end{minipage}
                    \hfill
                    \begin{minipage}{0.08\textwidth}
                        \centering
                        \includegraphics[width=\linewidth]{latex/gfx/images_by_topic/politics_3_scaled.jpg}
                    \end{minipage}
                    \hfill
                    \begin{minipage}{0.08\textwidth}
                        \centering
                        \includegraphics[width=\linewidth]{latex/gfx/images_by_topic/politics_4_scaled.jpg}
                    \end{minipage}
                    \hfill
                    \begin{minipage}{0.08\textwidth}
                        \centering
                        \includegraphics[width=\linewidth]{latex/gfx/images_by_topic/politics_5_scaled.jpg}
                    \end{minipage}
                    \hfill
                    \begin{minipage}{0.08\textwidth}
                        \centering
                        \includegraphics[width=\linewidth]{latex/gfx/images_by_topic/politics_6_scaled.jpg}
                    \end{minipage}
                    \hfill
                    \begin{minipage}{0.08\textwidth}
                        \centering
                        \includegraphics[width=\linewidth]{latex/gfx/images_by_topic/politics_7_scaled.jpg}
                    \end{minipage}
                    \begin{minipage}{0.08\textwidth}
                        \centering
                        \includegraphics[width=\linewidth]{latex/gfx/images_by_topic/politics_8_scaled.jpg}
                    \end{minipage}
                    \begin{minipage}{0.08\textwidth}
                        \centering
                        \includegraphics[width=\linewidth]{latex/gfx/images_by_topic/politics_9_scaled.jpg}
                    \end{minipage}
                \\
                
                
            \textbf{Science \& Technology}  \\ \cmidrule(lr){1-1}
                    \begin{minipage}{0.08\textwidth}
                        \centering
                        \includegraphics[width=\linewidth]{latex/gfx/images_by_topic/science_0_scaled.jpg}
                        % \vspace{0.5em}
                        % \textbf{(a)}
                    \end{minipage}
                    \hfill
                    \begin{minipage}{0.08\textwidth}
                        \centering
                        \includegraphics[width=\linewidth]{latex/gfx/images_by_topic/science_1_scaled.jpg}
                    \end{minipage}
                    \hfill
                    \begin{minipage}{0.08\textwidth}
                        \centering
                        \includegraphics[width=\linewidth]{latex/gfx/images_by_topic/science_2_scaled.jpg}
                    \end{minipage}
                    \hfill
                    \begin{minipage}{0.08\textwidth}
                        \centering
                        \includegraphics[width=\linewidth]{latex/gfx/images_by_topic/science_3_scaled.jpg}
                    \end{minipage}
                    \hfill
                    \begin{minipage}{0.08\textwidth}
                        \centering
                        \includegraphics[width=\linewidth]{latex/gfx/images_by_topic/science_4_scaled.jpg}
                    \end{minipage}
                    \hfill
                    \begin{minipage}{0.08\textwidth}
                        \centering
                        \includegraphics[width=\linewidth]{latex/gfx/images_by_topic/science_5_scaled.jpg}
                    \end{minipage}
                    \hfill
                    \begin{minipage}{0.08\textwidth}
                        \centering
                        \includegraphics[width=\linewidth]{latex/gfx/images_by_topic/science_6_scaled.jpg}
                    \end{minipage}
                    \hfill
                    \begin{minipage}{0.08\textwidth}
                        \centering
                        \includegraphics[width=\linewidth]{latex/gfx/images_by_topic/science_7_scaled.jpg}
                    \end{minipage}
                    \begin{minipage}{0.08\textwidth}
                        \centering
                        \includegraphics[width=\linewidth]{latex/gfx/images_by_topic/science_8_scaled.jpg}
                    \end{minipage}
                    \begin{minipage}{0.08\textwidth}
                        \centering
                        \includegraphics[width=\linewidth]{latex/gfx/images_by_topic/science_9_scaled.jpg}
                    \end{minipage}
                \\
    
            \textbf{Sports}  \\ \cmidrule(lr){1-1}
                    \begin{minipage}{0.08\textwidth}
                        \centering
                        \includegraphics[width=\linewidth]{latex/gfx/images_by_topic/sports_0_scaled.jpg}
                        % \vspace{0.5em}
                        % \textbf{(a)}
                    \end{minipage}
                    \hfill
                    \begin{minipage}{0.08\textwidth}
                        \centering
                        \includegraphics[width=\linewidth]{latex/gfx/images_by_topic/sports_1_scaled.jpg}
                    \end{minipage}
                    \hfill
                    \begin{minipage}{0.08\textwidth}
                        \centering
                        \includegraphics[width=\linewidth]{latex/gfx/images_by_topic/sports_2_scaled.jpg}
                    \end{minipage}
                    \hfill
                    \begin{minipage}{0.08\textwidth}
                        \centering
                        \includegraphics[width=\linewidth]{latex/gfx/images_by_topic/sports_3_scaled.jpg}
                    \end{minipage}
                    \hfill
                    \begin{minipage}{0.08\textwidth}
                        \centering
                        \includegraphics[width=\linewidth]{latex/gfx/images_by_topic/sports_4_scaled.jpg}
                    \end{minipage}
                    \hfill
                    \begin{minipage}{0.08\textwidth}
                        \centering
                        \includegraphics[width=\linewidth]{latex/gfx/images_by_topic/sports_5_scaled.jpg}
                    \end{minipage}
                    \hfill
                    \begin{minipage}{0.08\textwidth}
                        \centering
                        \includegraphics[width=\linewidth]{latex/gfx/images_by_topic/sports_6_scaled.jpg}
                    \end{minipage}
                    \hfill
                    \begin{minipage}{0.08\textwidth}
                        \centering
                        \includegraphics[width=\linewidth]{latex/gfx/images_by_topic/sports_7_scaled.jpg}
                    \end{minipage}
                    \begin{minipage}{0.08\textwidth}
                        \centering
                        \includegraphics[width=\linewidth]{latex/gfx/images_by_topic/sports_8_scaled.jpg}
                    \end{minipage}
                    \begin{minipage}{0.08\textwidth}
                        \centering
                        \includegraphics[width=\linewidth]{latex/gfx/images_by_topic/sports_9_scaled.jpg}
                    \end{minipage}
                \\
    
            \textbf{Travel}  \\ \cmidrule(lr){1-1}
                    \begin{minipage}{0.08\textwidth}
                        \centering
                        \includegraphics[width=\linewidth]{latex/gfx/images_by_topic/travel_0_scaled.jpg}
                        % \vspace{0.5em}
                        % \textbf{(a)}
                    \end{minipage}
                    \hfill
                    \begin{minipage}{0.08\textwidth}
                        \centering
                        \includegraphics[width=\linewidth]{latex/gfx/images_by_topic/travel_1_scaled.jpg}
                    \end{minipage}
                    \hfill
                    \begin{minipage}{0.08\textwidth}
                        \centering
                        \includegraphics[width=\linewidth]{latex/gfx/images_by_topic/travel_2_scaled.jpg}
                    \end{minipage}
                    \hfill
                    \begin{minipage}{0.08\textwidth}
                        \centering
                        \includegraphics[width=\linewidth]{latex/gfx/images_by_topic/travel_3_scaled.jpg}
                    \end{minipage}
                    \hfill
                    \begin{minipage}{0.08\textwidth}
                        \centering
                        \includegraphics[width=\linewidth]{latex/gfx/images_by_topic/travel_4_scaled.jpg}
                    \end{minipage}
                    \hfill
                    \begin{minipage}{0.08\textwidth}
                        \centering
                        \includegraphics[width=\linewidth]{latex/gfx/images_by_topic/travel_5_scaled.jpg}
                    \end{minipage}
                    \hfill
                    \begin{minipage}{0.08\textwidth}
                        \centering
                        \includegraphics[width=\linewidth]{latex/gfx/images_by_topic/travel_6_scaled.jpg}
                    \end{minipage}
                    \hfill
                    \begin{minipage}{0.08\textwidth}
                        \centering
                        \includegraphics[width=\linewidth]{latex/gfx/images_by_topic/travel_7_scaled.jpg}
                    \end{minipage}
                    \begin{minipage}{0.08\textwidth}
                        \centering
                        \includegraphics[width=\linewidth]{latex/gfx/images_by_topic/travel_8_scaled.jpg}
                    \end{minipage}
                    \begin{minipage}{0.08\textwidth}
                        \centering
                        \includegraphics[width=\linewidth]{latex/gfx/images_by_topic/travel_9_scaled.jpg}
                    \end{minipage}
                \\
         \bottomrule   
    \end{tabular}
\end{table*}
\clearpage

\begin{figure*}[ht]
\subsection{Prompts}
\label{appendix:prompts}

\begin{tcolorbox}[
    colback=white!95!black,    % Background color
    colframe=black,            % Frame color
    title=Images-To-Sentences (\its), % Box title
    fonttitle=\bfseries,       % Title font styling
    boxrule=0.5pt,             % Frame thickness
    arc=4pt,                   % Rounded corners
    outer arc=4pt,
    width=\textwidth,          % Box width
    enlarge left by=0mm,
    enlarge right by=0mm,
    before skip=1em,           % Space before the box
    after skip=1em,            % Space after the box
]

Which sentence best matches the topic of the images? The images and the sentences each belong
to one of the following topics: "entertainment", "geography", "health", "politics", "science and technology", "sports", or "travel". Choose one sentence from A, B, C, or D. Output only 
a single letter!

\medskip

\# Images

\begin{center}
    \begin{minipage}{0.18\textwidth}
        \centering
        \includegraphics[width=\linewidth]{latex/gfx/i2s/i2s_image1.jpg}
        % \vspace{0.5em}
        % \textbf{(a)}
    \end{minipage}
    \hfill
    \begin{minipage}{0.18\textwidth}
        \centering
        \includegraphics[width=\linewidth]{latex/gfx/i2s/i2s_image2.jpg}
        % \vspace{0.5em}
        % \textbf{(b)}
    \end{minipage}
    \hfill
    \begin{minipage}{0.18\textwidth}
        \centering
        \includegraphics[width=\linewidth]{latex/gfx/i2s/i2s_image3.jpg}
        % \vspace{0.5em}
        % \textbf{(c)}
    \end{minipage}
    \hfill
    \begin{minipage}{0.18\textwidth}
        \centering
        \includegraphics[width=\linewidth]{latex/gfx/i2s/i2s_image4.jpg}
        % \vspace{0.5em}
        % \textbf{(d)}
    \end{minipage}
    \hfill
    \begin{minipage}{0.18\textwidth}
        \centering
        \includegraphics[width=\linewidth]{latex/gfx/i2s/i2s_image5.jpg}
        % \vspace{0.5em}
        % \textbf{(e)}
    \end{minipage}
\end{center}

\medskip

\# Sentences

\begin{enumerate}[label=\Alph*., itemsep=0pt, topsep=0pt]
    \item \textasciigrave\textasciigrave\textasciigrave Maroochydore führte am Ende die Rangfolge an, mit sechs Punkten Vorsprung vor Noosa als Zweitem.\textasciigrave\textasciigrave\textasciigrave
    \item \textasciigrave\textasciigrave\textasciigrave 
    Es wurden keine schwere Verletzungen gemeldet, jedoch mussten mindestens fünf der zur Zeit der Explosion Anwesenden aufgrund von Schocksymptomen behandelt werden.\textasciigrave\textasciigrave\textasciigrave
    \item \textasciigrave\textasciigrave\textasciigrave 
    Finnland ist ein großartiges Reiseziel für Bootstouren. Das „Land der tausend Seen“ hat auch Tausende von Inseln – in den Seen und in den Küstenarchipelen.\textasciigrave\textasciigrave\textasciigrave
    \item \textasciigrave\textasciigrave\textasciigrave 
    Es ist auch nicht erforderlich, dass Sie eine lokale Nummer von der Gemeinde erhalten, in der Sie leben. Sie können eine Internetverbindung über Satellit in der Wildnis v on Chicken in Alaska erhalten und eine Nummer auswählen, die vorgibt, dass Sie im sonnigen Arizona sind.\textasciigrave\textasciigrave\textasciigrave
\end{enumerate}
Your answer letter: 
\end{tcolorbox}
\end{figure*}

\clearpage

\begin{figure*}[ht]
\begin{tcolorbox}[
    colback=white!95!black,    % Background color
    colframe=black,            % Frame color
    title=Sentences-To-Images (\sti), % Box title
    fonttitle=\bfseries,       % Title font styling
    boxrule=0.5pt,             % Frame thickness
    arc=4pt,                   % Rounded corners
    outer arc=4pt,
    width=\textwidth,          % Box width
    before skip=1em,           % Space before the box
    after skip=1em,            % Space after the box
]

Which image best matches the topic of the sentences?
The sentences and the images each belong to one of the following topics: 
"entertainment", "geography", "health", "politics", "science and technology", "sports", or "travel". 
Choose one image from A, B, C, or D. Output only a single letter!

\medskip

\# Sentences

\begin{itemize}[itemsep=0pt,topsep=2pt]
    \item \textasciigrave\textasciigrave\textasciigrave Maroochydore führte am Ende die Rangfolge an, mit sechs Punkten Vorsprung vor Noosa
 als Zweitem.\textasciigrave\textasciigrave\textasciigrave
    \item \textasciigrave\textasciigrave\textasciigrave 
    Die Schlagmänner der mittleren Reihe, Sachin Tendulkar und Rahul Dravid, zeigten gute Leistungen und erzielten eine Partnerschaft mit 100 Runs.\textasciigrave\textasciigrave\textasciigrave 
    \item \textasciigrave\textasciigrave\textasciigrave 
    Da pro Tag nur achtzehn Medaillen zur Verfügung stehen, hat es ein Anzahl an Ländern nicht auf das Podium geschafft.\textasciigrave\textasciigrave\textasciigrave 
    \item \textasciigrave\textasciigrave\textasciigrave 
    Wintersportarten sind in den nördlichen Regionen am beliebtesten und Italiener nehmen an internationalen Wettkämpfen und olympischen Spielen teil.\textasciigrave\textasciigrave\textasciigrave 
    \item \textasciigrave\textasciigrave\textasciigrave 
    Nach dem Rennen bleibt Keselowski mit 2.250 Punkten Spitzenreiter in der Fahrerwertung.
\end{itemize}

\medskip

\# Images

\begin{center}
    %--- A ---
    \begin{minipage}{0.23\textwidth}
        \centering
        A.
        \includegraphics[width=\linewidth]{latex/gfx/s2i/s2i_image1.jpg}
    \end{minipage}
    \hfill
    %--- B ---
    \begin{minipage}{0.23\textwidth}
        \centering
        B. \vspace{0.1cm}
        \includegraphics[width=\linewidth]{latex/gfx/s2i/s2i_image2.jpg}
    \end{minipage}
    \hfill
    %--- C ---
    \begin{minipage}{0.23\textwidth}
        \centering
        C. \vspace{0.1cm}
        \includegraphics[width=\linewidth]{latex/gfx/s2i/s2i_image3.jpg}
    \end{minipage}
    \hfill
    %--- D ---
    \begin{minipage}{0.23\textwidth}
        \centering
        D. \vspace{0.1cm}
        \includegraphics[width=\linewidth]{latex/gfx/s2i/s2i_image4.jpg}
    \end{minipage}
\end{center}

\medskip

Your answer letter: 

\end{tcolorbox}
\end{figure*}

\clearpage
\begin{figure*}[ht]
\begin{tcolorbox}[
    colback=white!95!black,    % Background color
    colframe=black,            % Frame color
    title=Topic-To-Sentence (\tts), % Box title
    fonttitle=\bfseries,       % Title font styling
    boxrule=0.5pt,             % Frame thickness
    arc=4pt,                   % Rounded corners
    outer arc=4pt,
    width=\textwidth,          % Box width
    before skip=1em,           % Space before the box
    after skip=1em,            % Space after the box
]

Which sentence best matches the topic "sports"? The sentences each belong to one of the following topics: "entertainment", "geography", "health", "politics", "science and technology", "sports", or "travel". Choose one sentence from A, B, C, or D. Output only
 a single letter!

\medskip

\# Sentences

\begin{enumerate}[label=\Alph*., itemsep=0pt, topsep=0pt]
    \item \textasciigrave\textasciigrave\textasciigrave Maroochydore führte am Ende die Rangfolge an, mit sechs Punkten Vorsprung vor Noosa als Zweitem.\textasciigrave\textasciigrave\textasciigrave
    \item \textasciigrave\textasciigrave\textasciigrave 
    Es wurden keine schwere Verletzungen gemeldet, jedoch mussten mindestens fünf der zur Zeit der Explosion Anwesenden aufgrund von Schocksymptomen behandelt werden.\textasciigrave\textasciigrave\textasciigrave
    \item \textasciigrave\textasciigrave\textasciigrave 
    Finnland ist ein großartiges Reiseziel für Bootstouren. Das „Land der tausend Seen“ hat auch Tausende von Inseln – in den Seen und in den Küstenarchipelen.\textasciigrave\textasciigrave\textasciigrave
    \item \textasciigrave\textasciigrave\textasciigrave 
    Es ist auch nicht erforderlich, dass Sie eine lokale Nummer von der Gemeinde erhalten, in der Sie leben. Sie können eine Internetverbindung über Satellit in der Wildnis v on Chicken in Alaska erhalten und eine Nummer auswählen, die vorgibt, dass Sie im sonnigen Arizona sind.\textasciigrave\textasciigrave\textasciigrave
\end{enumerate}

\medskip

Your answer letter: 

\end{tcolorbox}
\end{figure*}

\clearpage
\begin{figure*}[ht]
\begin{tcolorbox}[
    colback=white!95!black,    % Background color
    colframe=black,            % Frame color
    title=Sentences-To-Topics (\stt), % Box title
    fonttitle=\bfseries,       % Title font styling
    boxrule=0.5pt,             % Frame thickness
    arc=4pt,                   % Rounded corners
    outer arc=4pt,
    width=\textwidth,          % Box width
    before skip=1em,           % Space before the box
    after skip=1em,            % Space after the box
]

Which topic best matches the sentences? The sentences belong to one of the following topics: "entertainment", "geography", "health", "politics", "science and technology", "sports", or "travel". Choose one topic from A, B, C, or D. Output only a single letter!

\medskip

\# Sentences

\begin{itemize}[itemsep=0pt,topsep=2pt]
    \item \textasciigrave\textasciigrave\textasciigrave Maroochydore führte am Ende die Rangfolge an, mit sechs Punkten Vorsprung vor Noosa
 als Zweitem.\textasciigrave\textasciigrave\textasciigrave
    \item \textasciigrave\textasciigrave\textasciigrave 
    Die Schlagmänner der mittleren Reihe, Sachin Tendulkar und Rahul Dravid, zeigten gute Leistungen und erzielten eine Partnerschaft mit 100 Runs.
    \item \textasciigrave\textasciigrave\textasciigrave 
    Da pro Tag nur achtzehn Medaillen zur Verfügung stehen, hat es ein Anzahl an Ländern nicht auf das Podium geschafft.\textasciigrave\textasciigrave\textasciigrave 
    \item \textasciigrave\textasciigrave\textasciigrave 
    Wintersportarten sind in den nördlichen Regionen am beliebtesten und Italiener nehmen an internationalen Wettkämpfen und olympischen Spielen teil.\textasciigrave\textasciigrave\textasciigrave 
    \item \textasciigrave\textasciigrave\textasciigrave 
    Nach dem Rennen bleibt Keselowski mit 2.250 Punkten Spitzenreiter in der Fahrerwertung.\textasciigrave\textasciigrave\textasciigrave 
\end{itemize}

\medskip

\# Topics

\begin{enumerate}[label=\Alph*., itemsep=0pt, topsep=0pt]
    \item sports
    \item health
    \item travel
    \item science and technology
\end{enumerate}

\medskip

Your answer letter: 

\end{tcolorbox}
\end{figure*}
\clearpage

\clearpage
\subsection{Further Details}
\label{app:subsec:exp-details}

\rparagraph{Models} We test state-of-the-art instruction fine-tuned LVLMs across various sizes. Smaller models are evaluated on all languages, while larger LVLMs (26B+) are tested on subsets of the MVL-SIB languages (cf. \S\ref{subsec:further-analyses}).

\iparagraph{GPT-4o} We evaluate MVL-SIB on GPT-4o-mini-\textit{2024-07-18} and GPT-4o-\textit{2024-08-06}. We set the image detail in the API to `low', since our tasks require high-level reasoning that does not depend on finer image details.\footnote{We use GPT-4o-mini because evaluating GPT-4o would be too expensive.} Prior works show that GPT-4o is the best-performing multilingual LVLM \cite{schneider-sitaram-2024-m5, vayani2024alm}.

\iparagraph{Qwen2-VL} Qwen2-VL ties a 675M parameter vision-transformer (ViT) into Qwen2 LLMs \cite{wang2024qwen2vl}. An MLP compresses adjacent 2×2 visual tokens embedded by the ViT into one token representation, which is then input to the LLM. 

\iparagraph{InternVL 2.5} Depending on the model size, InternVL uses Qwen2.5 or InternLM as its LLM backbone  \cite{chen2024internvl}. The model embeds images either by a 6B or by a distilled 300M ViT pretrained with CLIP \cite{radford2021clip}. The resulting image patch encodings are downsampled by factor 4 and fed through an MLP to the LLM.

\iparagraph{Centurio} Centurio is the latest massively multilingual LVLM trained on 100 languages \cite{geigle2025centurio}, outperforming alternatives like Parrot \cite{sun2024parrot} or Pangea \cite{yue2024pangea}. It employs Qwen2.5 as its LLM \cite{yang2024qwen2technicalreport} and \texttt{SigLIP SO400/384} as its ViT \cite{zhai2023sigmoid}. The model mixes resolutions by stacking the encodings of the full image and those of 2×2 tiles along the features. The combined embedding is then projected via an MLP to the LLM's input space.

Besides architectures, the LVLMs crucially differ in dataset mixtures on which they were trained. Centurio translates image-caption, VQA, OCR, and a few multi-image datasets to 100 languages with NLLB \cite{nllbteam2022language} to mix 50:50 with the original English data. Qwen2-VL and InternVL, however, were trained on much larger, more diverse datasets that comprise sizable multi-image comparison and video understanding datasets. Moreover, assuming that the LLMs of Qwen2-VL, InternVL, and GPT-4o were pretrained on Flores, their performance would be overly optimistic.


\begin{figure*}[ht]
  \subsubsection{Performance by Model over Languages grouped by Language Tier}
  \label{app:tier-perf-plot}
  \centering
  \adjustbox{max width=\textwidth}{%
    \includegraphics{latex/gfx/tier_img2sent.pdf}%
  }
  \vspace{-0.8cm}
  \caption{\textbf{Images-To-Sentences~@~$k{=}3$.} The English prompt describes the cross-modal  topic matching task, lists all topics, and provides both $k{=}3$ reference images and 4 sentences in the corresponding language \{\texttt{eng\_Latn}, $\dots$, \texttt{nqo\_Nkoo}\}. LVLMs must select the sentence of 4 options that topically fits $k{=}3$ reference images. The sentences spanning 205 languages and 7 topics are drawn from SIB-200 \cite{adelani-etal-2024-sib}, while images for the topics were hand-selected (cf. Appendix \ref{app:images-per-topic}). An example prompt is shown in Appendix \ref{app:images-to-sentences}; further details are in \S\ref{sec:experimental-setup}. \\ \textbf{Plot.} The x-axis orders the languages of the candidate sentences \{\texttt{eng\_Latn}, $\dots$, \texttt{nqo\_Nkoo}\}, respectively, by descending performance (y-axis). The top x-axis indicates the running index of each language $L_i$ ($i \in \{1, \dots, 205\}$). \\
  \textbf{Tiers.} The languages are grouped by tiers derived from \citet{joshi-etal-2020-state} (cf. \S\ref{sec:results}).}
  \vspace{-0.1cm}
  \label{fig:tier-img2sent}
\end{figure*}


\begin{figure*}[ht]
 \subsubsection{Calibration Analysis of Cross-Modal Topic Matching}
 \label{app:fig:calibration}
 \centering
 \adjustbox{max width=\textwidth}{%
   \includegraphics{latex/gfx/calibration.pdf}%
 }
 \caption{ \textbf{Calibration Analysis of Cross-Modal Topic Matching for InternVL 2.5 8B.} \textbf{Analysis:} To assess the reliability of cross-modal topic matching with fewer samples than our full dataset (1004 samples per language), we randomly select 500 trajectories. We then compute performance metrics on cumulative subsets, incrementing by 50 examples at each step. The difference in performance between the full dataset and each subset is calculated to quantify the deviation at each sample size. \textbf{Plot:} The plot displays the average absolute spread in performance (averaged over all languages) along with the standard deviation for InternVL 2.5 8B. We restrict ourselves to a single open-weight LVLM, since the analysis yields identical results across all combinations of LVLMs and language tiers. \textbf{Insights:} Our analysis shows that performance stabilizes rapidly, with deviations of only about 1\% observed at 1,004 instances -- same size as the subset from which the dataset was created. This indicates that reliable evaluation of cross-modal topic matching can be achieved with far fewer than 3,012 samples. }
 \vspace{-0.1cm}
\end{figure*}

\clearpage

\subsection{Overview of Multilingual Vision-Language Benchmarks}
\label{app:summary-mvl-benchmarks}

\begin{table*}[ht]
    \centering
\adjustbox{max width=\textwidth}{%
\large
\begin{tabular}{lcccccc}
\toprule
\textbf{Task} & \textbf{Dataset} & \textbf{Visual Input} & \textbf{Textual Input} & \textbf{Target Output} & \textbf{Metric} & \textbf{\#Lang.} \\
\midrule
\multirow{2}{*}{Captioning} 
    & \multirow{2}{*}{XM3600} & \multirow{2}{*}{Single-Image} & \multirow{2}{*}{Prompt (English)} & \multirow{2}{4cm}{\centering Caption\\(Target Language)} & CIDEr & $36$ \\
   & & & & & &  \\ 
\midrule
\multirow{2}{4cm}{Multiple-Choice Visual Question Answering} 
    & \multirow{2}{*}{BabelImageNet-MC} 
    & \multirow{2}{*}{Single-Image} 
    & \multirow{2}{*}{Question (Target Language)} 
    & \multirow{2}{*}{Letter of the correct Choice} 
    & \multirow{2}{*}{Relaxed Accuracy} 
    & \multirow{2}{*}{$20$} \\
   & & & & & &  \\ 
\midrule
\multirow{3}{4cm}{Text-Heavy Multiple-Choice Visual Question Answering} 
    & M3Exam 
    & \multirow{3}{*}{Single or Multi-Image} 
    & Question (Target Language) 
    & \multirow{3}{*}{Letter of the correct Choice} 
    & \multirow{3}{*}{Relaxed Accuracy} 
    & $7$ \\
    & MMMU     
    &  % Empty cell because of multirow
    & Question (English)        
    &  % Empty cell because of multirow
    &  % Empty cell because of multirow
    & $1$ \\
    & xMMMU    
    &  % Empty cell because of multirow
    & Question (Target Language) 
    &  % Empty cell because of multirow
    &  % Empty cell because of multirow
    & $7$ \\
\midrule
\multirow{2}{4cm}{Text-Heavy Visual\\Question Answering} 
    & MTVQA       
    & \multirow{2}{*}{Single-Image} 
    & \multirow{2}{*}{Question (Target Language)} 
    & \multirow{2}{*}{Word or Phrase (Target Language)} 
    & \multirow{2}{*}{Exact Accuracy} 
    & $9$ \\
    & SMPQA - Name 
    &  % empty cell for Single-Image (multirow)
    &  % empty cell for Question (Target Language) (multirow)
    &  % empty cell for Word or Phrase (Target Language) (multirow)
    &  % empty cell for Exact Accuracy (multirow)
    & $11$ \\
\midrule
\multirow{2}{4cm}{Text-Heavy Visually\\ Grounded Reasoning}  & \multirow{2}{*}{SMPQA - Ground} & \multirow{2}{*}{Single-Image} & \multirow{2}{*}{Question (Target Language)} & \multirow{2}{*}{'yes' / 'no'} & \multirow{2}{*}{Exact Accuracy} & \multirow{2}{*}{$11$} \\
& & & & & &  \\ 
\midrule
\multirow{2}{4cm}{Visio-Linguistic \\ Outlier Detection} 
    & \multirow{2}{*}{M5B-VLOD} & \multirow{2}{*}{Multi-Image} & \multirow{2}{*}{Hypothesis (Target Language)} & \multirow{2}{*}{Letter of the correct Choice} & \multirow{2}{*}{Relaxed Accuracy} & \multirow{2}{*}{$12$} \\
& & & & & &  \\ 
\midrule
\multirow{2}{4cm}{Visual Natural\\ Language Inference} 
    & \multirow{2}{*}{XVNLI} & \multirow{2}{*}{Single-Image} & \multirow{2}{*}{Hypothesis (Target Language)} & \multirow{2}{*}{'yes' / 'no' / 'maybe'} & \multirow{2}{*}{Exact Accuracy} & \multirow{2}{*}{$5$} \\
   & & & & & &  \\ 
\midrule
\multirow{2}{*}{Visual Question Answering} 
    & MaXM 
    & \multirow{2}{*}{Single-Image} 
    & \multirow{2}{*}{Question (Target Language)} 
    & Word or Phrase (Target Language) 
    & \multirow{2}{*}{Exact Accuracy} 
    & $6$ \\
    & xGQA 
    &  % empty cell because of multirow in column 3
    &  % empty cell because of multirow in column 4
    & Word or Phrase (English) 
    &  % empty cell because of multirow in column 6
    & $8$ \\ \midrule
    \multirow{2}{*}{Visually Grounded Reasoning} 
    & M5B-VGR
    & \multirow{2}{*}{Multi-Image} 
    & \multirow{2}{*}{Hypothesis (Target Language)} 
    & \multirow{2}{*}{'yes' / 'no'} 
    & \multirow{2}{*}{Exact Accuracy} 
    & $12$ \\
    & MaRVL   
    &  % empty cell (Multi-Image)
    &  % empty cell (Hypothesis (Target Language))
    &  % empty cell ('yes' / 'no')
    &  % empty cell (Exact Accuracy)
    & $6$ \\

\bottomrule
\end{tabular}    }
    \caption{Summary of multilingual vision-language benchmarks we correlate MVL-SIB against. Relaxed 
 denotes responses that start with the correct option letter (cf. \ref{sec:experimental-setup}).}
    \label{appendix:tab:eval_suite_datasets}
\end{table*}

\rparagraph{xGQA} The xGQA dataset~\citep{pfeiffer-etal-2022-xgqa} is a cross-lingual visual question-answering resource. It extends the well-known English-only GQA dataset~\citep{hudson_gqa_2019} by providing manual translations of the questions in the balanced \textit{test-dev} set. The dataset contains $9666$ questions available in eight languages across five scripts, while the answers remain in English. In addition, it features $300$ unique images from Visual Genome~\citep{krishna_visual_2017}.

%%%%%%%%%%%%%
%
\rparagraph{MaXM}
%
MaXM, introduced by~\citet{changpinyo-etal-2023-maxm}, is a VQA dataset covering seven languages written in five scripts. In this dataset, both the questions and their corresponding answers are presented in the same language. The images are drawn from a subset of the XM3600~\citep{thapliyal_xm3600_2022} dataset and are selected to correspond to regions where the question-answer pair’s language is spoken, ensuring both linguistic and cultural diversity.

%%%%%%%%%%%%%%
%
\rparagraph{XVNLI}
%
The XVNLI dataset~\citep{bugliarello-etal-2022-iglue} introduces the task of Cross-lingual Visual Natural Language Inference, where a model must determine if a textual hypothesis \textit{entails}, \textit{contradicts}, or is \textit{neutral} with respect to a visual premise. This dataset spans five languages across three scripts and includes $357$ unique images from Visual Genome. It is built upon a combination of the text-only SNLI~\citep{bowman_large_2015} dataset and its cross-lingual~\citep{agic_cli_2018} and cross-modal~\citep{xie_visual_entailment_2019} counterparts.

%%%%%%%%%%%%%%%
%
\rparagraph{MaRVL}
%
The MaRVL dataset~\citep{liu-etal-2021-visually} is designed to benchmark models on Multicultural Reasoning over Vision and Language. Each sample consists of two images, a textual statement, and a binary (true/false) answer grounded in the images. Covering five languages across three scripts, MaRVL includes $4914$ culturally diverse images that align with the respective languages. The images in each sample are selected to reflect the culture of the annotator who composed the textual statement in their native language.

%%%%%%%%%%%%%%%
%
\rparagraph{XM3600}
%
The XM3600 dataset~\citep{thapliyal_xm3600_2022} is an extensive multilingual image captioning resource encompassing 36 languages. It contains $261375$ captions across 13 scripts, with 100 unique images per language. The images are chosen to reflect the cultural background of the language, ensuring both cultural and linguistic diversity. All captions were manually produced by professional native speakers rather than being automatically generated. Due to the dataset's large size, we evaluate XM3600 using a randomly selected subset of 500 images per language.

%%%%%%%%%%%%%%
%
\rparagraph{Babel-ImageNet (multiple-choice) (BIN-MC)}
%
Babel-ImageNet~\cite{geigle-etal-2024-babel} translates ImageNet’s~\cite{deng_imagenet_2009} labels into nearly 300 languages, allowing us to assess whether models can recognize and correctly link diverse ImageNet objects to their labels in the target language. Given the computational cost, we focus on languages that appear in only one or two other datasets, in addition to English and a select few high-resource languages, and we use 10 images per class instead of 50. We follow \citet{geigle-etal-2024-african} and frame the task as a multiple-choice problem by mining hard negative options from the complete label pool. This approach avoids the ambiguity inherent in traditional open-ended VQA formats. Negatives are selected based on the English labels, filtering out candidates not translated by Babel-ImageNet into the target language, and ultimately choosing the three most similar negative labels available.

%%%%%%%%%%%%%%
%
\rparagraph{SMPQA}
%
\citet{geigle2025centurio} introduce SMPQA (Synthetic Multilingual Plot QA) as a test dataset for evaluating multilingual OCR capabilities for bar plots and pie charts, covering 11 languages and various scripts and resource levels.

%%%%%%%%%%%%%%%%%%%%
%
\rparagraph{M5B-VGR}
%
The M5B-VGR dataset, presented by~\cite{schneider-sitaram-2024-m5}, is a visually grounded reasoning benchmark akin to MaRVL. Each sample comprises two images, a textual statement, and a binary (true/false) answer based on the images. It spans 12 languages across 7 scripts and features culturally diverse photos from regions where the respective languages are spoken. The images are sampled from the Dollar Street~\cite{gaviria2022dollar} dataset, with 120 samples provided per language.

%%%%%%%%%%%%%%%%%%%%
%
\rparagraph{M5B-VLOD}
%
The M5B-VLOD (Visio-Linguistic Outlier Detection) dataset, also introduced by~\cite{schneider-sitaram-2024-m5}, consists of samples containing five images paired with a textual statement that is true for all but one image. The task is to identify the outlier image that does not match the statement. This dataset covers the same 12 languages as M5B-VGR, with images sampled in a similar manner from the same source, and provides 120 samples per language.

%%%%%%%%%%%%%%%%%%%%%%%
%
\rparagraph{MTVQA}
%
The MTVQA dataset, introduced by~\cite{tang2024mtvqa}, features text-heavy visual question answering tasks. It includes expert human annotations in 9 diverse languages, comprising 6778 question-answer pairs across 2116 images. The images predominantly contain text in the corresponding language, with questions and answers closely tied to that text. These images are sourced from various publicly available datasets.

%%%%%%%%%%%%%%%%%%%%%%
%
\rparagraph{CVQA}
%
The CVQA dataset, introduced by~\cite{romero2024cvqa}, is a multilingual and culturally nuanced VQA benchmark that includes a broad array of languages, many of which are underrepresented in NLP. It consists of 10000 questions spanning 30 countries and 31 languages, forming 39 distinct country-language pairs (for instance, Spanish appears in 7 different splits corresponding to 7 Spanish-speaking countries). The images were manually collected by human annotators to accurately depict the culture associated with each country-language pair. Each sample includes one image and a question in the respective language. Although the test set is not publicly available, the authors permit up to 5 daily leaderboard submissions for evaluation.

%%%%%%%%%%%%%%%%%%%%%%
%
\rparagraph{M3Exam}
%
The M3Exam dataset, presented by~\cite{zhang_m3exam_2023}, contains real-world exam questions in 9 languages, available as either text-only or multimodal samples. For our evaluation, we only include samples that require at least one image. Due to the limited number of samples for Swhalili and Javanese, we focus on the remaining 7 languages. The selected samples consist of multiple-choice questions in the target language, accompanied by up to 8 images that may appear in both the question and the answer options, with the number of choices ranging from 4 to 8 per sample.

%%%%%%%%%%%%%%%%%%%%%%
%
\rparagraph{xMMMU}
%
xMMMU, introduced by~\cite{yue2024pangea}, comprises college-level multiple-choice VQA samples in seven languages. It was automatically translated using GPT4o from a randomly selected subset of 300 questions from the MMMU~\cite{yue_mmmu_2023} validation split.

\subsection{Prompts}
\label{appendix:sec:evaluation:prompts}
%
We list the prompts for each dataset in our test suite used for all models in Figure~\ref{appendix:fig:evaluation:prompts}.
%
\begin{figure*}[ht!]\tiny
    \centering
    %
    \begin{promptbox}{SMPQA}
    <IMG>\{QUESTION\}\textbackslash{n}Answer the question using a single word or phrase.
    \end{promptbox}
    %
    %
    \begin{promptbox}{CVQA}
    <IMG>\{QUESTION\}\textbackslash{n}There are several options:\textbackslash{n}A. \{OPTION A\}\textbackslash{n}B. \{OPTION B\}\textbackslash{n}C. \{OPTION C\}\textbackslash{n}D. \{OPTION D\}\textbackslash{n}Answer with the option's letter from the given choices directly.
    \end{promptbox}
    %
    %
    \begin{promptbox}{xMMMU}
    \{QUESTION\}\textbackslash{n}There are several options:\textbackslash{n}A. \{OPTION A\}\textbackslash{n}B. \{OPTION B\}\textbackslash{n}C. \{OPTION C\}\textbackslash{n}D. \{OPTION D\}\textbackslash{n}Answer with the option's letter from the given choices directly.
    \end{promptbox}
    %
    %
    \begin{promptbox}{MTVQA}
    <IMG>\{QUESTION\}\textbackslash{n}Answer the question using a single word or phrase.\textbackslash{n}Answer in \{LANGUAGE\}.
    \end{promptbox}
    %
    %
    \begin{promptbox}{M3Exam}
    \{QUESTION\}\textbackslash{n}Options:\textbackslash{n}A. \{OPTION A\}\textbackslash{n}B. \{OPTION B\}\textbackslash{n}C. \{OPTION C\}\textbackslash{n}D. \{OPTION D\}\textbackslash{n} Answer with the option's letter from the given choices directly.
    \end{promptbox}
    %
    %
    \begin{promptbox}{BIN-MC}
    <IMG>Which of these choices (in English) is shown in the image?\textbackslash{n} Choices:\textbackslash{n}A. \{CHOICE A\}\textbackslash{n}B. \{CHOICE B\}\textbackslash{n}C. \{CHOICE C\}\textbackslash{n}D. \{CHOICE D\}\textbackslash{n} Answer with the letter from the given choices directly.
    \end{promptbox}
    %
    \begin{promptbox}{xGQA}
    <IMG>\{QUESTION\}?\textbackslash{n}Answer the question using a single word or phrase.\textbackslash{n}Answer in English.
    \end{promptbox}
    %
    \begin{promptbox}{MaXM}
    <IMG>\{QUESTION\}?\textbackslash{n}Answer the question using a single word or phrase.\textbackslash{n}Answer in \{LANGUAGE\}.
    \end{promptbox}
    %
    \begin{promptbox}{MaRVL}
    <IMG>Given the two images <IMG><IMG>, is it correct to say ``\{HYPOTHESIS\}''? Answer yes or no.'
    \end{promptbox}
    %
    \begin{promptbox}{XVNLI}
    <IMG>Is it guaranteed true that ``\{HYPOTHESIS\}''? Yes, no, or maybe? Answer in English.
    \end{promptbox}
    %
    \begin{promptbox}{M5-VGR}
    Given the two images <IMG><IMG>, is it correct to say ``\{HYPOTHESIS\}''? Answer yes or no.'
    \end{promptbox}
    %
    \begin{promptbox}{M5-VLOD}
    Based on the 5 images <IMG><IMG><IMG><IMG><IMG> ordered from top-left to bottom-right, which image does not match the hypothesis ``\{HYPOTHESIS\}''? Choose one from [A, B, C, D, E] and only output a single letter:
    \end{promptbox}
    %
    \begin{promptbox}{XM3600}
    Briefly describe the image in \{LANGUAGE\} in one sentence.
    \end{promptbox}
    %
    %
    \caption{Prompts used for the different datasets of our test suite. For M3Exam and xMMMU, the questions contain images at individual positions, and also the options can consist of images. In total, a sample of M3Exam can contain up to 8 images and 8 options, and a sample of xMMMU can contain up to 4 images and 4 options.}
    \label{appendix:fig:evaluation:prompts}
\end{figure*}
\begin{table*}[ht]
\subsection{Full Results For Subsets by Task, Model, and Language}
\label{app:subsets-full-results}
    \rowcolors{2}{gray!15}{white} % Alternate row coloring: gray on every second row
    \begin{adjustbox}{max width=\textwidth}
    \centering
    \small
    \begin{tabular}{l c | c c c | c c c c c | c | c c}
    \toprule
    & & \multicolumn{3}{c|}{\textbf{Qwen2-VL}} & \multicolumn{5}{c|}{\textbf{InternVL 2.5}} & \multicolumn{1}{c|}{\textbf{Centurio Qwen}} & \multicolumn{2}{c}{\textbf{GPT-4o}} \\
    \cmidrule(lr){3-5} \cmidrule(lr){6-10}  \cmidrule(lr){11-11} \cmidrule(lr){12-13}
    \textbf{Lang} & \textbf{Tier} & Qwen2-VL 2B & Qwen2-VL 7B & Qwen2-VL 72B & InternVL 2.5 4B & InternVL 2.5 8B & InternVL 2.5 26B & InternVL 2.5 38B & InternVL 2.5 78B & Centurio Qwen 8B & GPT-4o-mini & GPT-4o \\
    \midrule
\texttt{acm\_Arab} & 3 & 34.9 & 55.4 & 80.8 & 50.0 & 61.0 & 74.1 & 80.2 & 82.2 & 54.4 & 72.0 & N/A \\
\texttt{aka\_Latn} & 2 & 28.7 & 38.0 & 57.2 & 34.6 & 46.3 & 51.3 & 58.3 & 57.0 & 42.4 & 61.4 & N/A \\
\texttt{apc\_Arab} & 3 & 34.9 & 54.8 & 80.0 & 49.1 & 61.3 & 74.2 & 78.8 & 82.1 & 54.4 & 71.5 & N/A \\
\texttt{arb\_Arab} & 4 & 35.2 & 57.1 & 81.4 & 51.7 & 62.2 & 75.8 & 80.8 & 82.8 & 55.2 & 73.2 & N/A \\
\texttt{azj\_Latn} & 2 & 34.6 & 53.8 & 80.5 & 49.5 & 58.6 & 70.9 & 78.3 & 80.0 & 51.8 & 72.8 & N/A \\
\texttt{bul\_Cyrl} & 3 & 35.0 & 56.6 & 79.3 & 50.3 & 65.3 & 77.1 & 79.8 & 81.6 & 53.9 & 71.5 & N/A \\
\texttt{ces\_Latn} & 4 & 36.2 & 58.2 & 79.5 & 52.0 & 65.6 & 78.5 & 79.5 & 81.5 & 53.2 & 72.7 & N/A \\
\texttt{eng\_Latn} & 4 & 36.3 & 65.8 & 79.8 & 52.5 & 67.7 & N/A & 79.7 & 81.7 & 54.8 & 68.3 & N/A \\
\texttt{fin\_Latn} & 4 & 33.0 & 57.1 & 79.1 & 49.8 & 64.1 & 76.5 & 79.2 & 81.1 & 54.1 & 70.2 & N/A \\
\texttt{hat\_Latn} & 1 & 32.1 & 50.4 & 77.8 & 43.0 & 54.7 & 66.1 & 74.1 & 74.2 & 54.9 & 69.3 & N/A \\
\texttt{hau\_Latn} & 1 & 29.1 & 39.2 & 56.3 & 32.4 & 41.8 & 45.2 & 58.0 & 57.5 & 43.9 & 70.2 & N/A \\
\texttt{min\_Arab} & 2 & 27.0 & 30.4 & 58.4 & 27.8 & 33.6 & 36.3 & 53.4 & 47.8 & 37.1 & 44.8 & N/A \\
\texttt{umb\_Latn} & 1 & 29.1 & 34.8 & 52.5 & 31.1 & 42.5 & 45.2 & 50.8 & 52.0 & 35.4 & 47.4 & N/A \\
    \bottomrule
    \end{tabular}
    \end{adjustbox}
    \caption{\textbf{Subsets of language tiers for \its @ $k{=}1$}.}
    \label{tab:subsets-full-results-i2s-k1}
\end{table*}

\begin{table*}[ht]
    \rowcolors{2}{gray!15}{white} % Alternate row coloring: gray on every second row
    \begin{adjustbox}{max width=\textwidth}
    \centering
    \small
    \begin{tabular}{l c | c c c | c c c c c | c | c c}
    \toprule
    & & \multicolumn{3}{c|}{\textbf{Qwen2-VL}} & \multicolumn{5}{c|}{\textbf{InternVL 2.5}} & \multicolumn{1}{c|}{\textbf{Centurio Qwen}} & \multicolumn{2}{c}{\textbf{GPT-4o}} \\
    \cmidrule(lr){3-5} \cmidrule(lr){6-10}  \cmidrule(lr){11-11} \cmidrule(lr){12-13}
    \textbf{Lang} & \textbf{Tier} & Qwen2-VL 2B & Qwen2-VL 7B & Qwen2-VL 72B & InternVL 2.5 4B & InternVL 2.5 8B & InternVL 2.5 26B & InternVL 2.5 38B & InternVL 2.5 78B & Centurio Qwen 8B & GPT-4o-mini & GPT-4o \\
    \midrule
    \texttt{acm\_Arab} & 3 & 35.2 & 55.1 & 82.4 & 45.6 & 60.6 & 74.0 & 85.5 & 86.7 & 59.7 & 79.3 & 86.0 \\
    \texttt{aka\_Latn} & 2 & 27.9 & 38.3 & 63.2 & 34.2 & 46.3 & 52.9 & 61.1 & 60.6 & 42.6 & 63.3 & 79.3 \\
    \texttt{apc\_Arab} & 3 & 34.7 & 54.3 & 82.2 & 45.1 & 60.5 & 73.9 & 84.1 & 85.7 & 59.4 & 78.5 & 85.8 \\
    \texttt{arb\_Arab} & 4 & 35.2 & 56.4 & 81.8 & 46.9 & 60.9 & 76.0 & 85.0 & 87.1 & 58.7 & 79.7 & 86.7 \\
    \texttt{azj\_Latn} & 2 & 33.1 & 53.4 & 82.3 & 46.2 & 60.9 & 71.4 & 82.7 & 83.5 & 57.9 & 78.6 & 86.5 \\
    \texttt{bul\_Cyrl} & 3 & 34.7 & 54.6 & 82.7 & 45.9 & 64.4 & 78.8 & 84.8 & 84.5 & 58.8 & 79.5 & 86.4 \\
    \texttt{ces\_Latn} & 4 & 35.6 & 56.8 & 82.2 & 48.9 & 65.6 & 78.4 & 83.6 & 85.3 & 57.3 & 78.6 & 86.3 \\
    \texttt{eng\_Latn} & 4 & 34.8 & 63.1 & 82.6 & 49.2 & 67.9 & 80.1 & 84.9 & 85.5 & 60.0 & 78.1 & 87.1 \\
    \texttt{fin\_Latn} & 4 & 33.5 & 56.4 & 81.9 & 45.8 & 65.6 & 77.2 & 83.2 & 84.4 & 59.7 & 78.3 & 86.5 \\
    \texttt{hat\_Latn} & 1 & 30.7 & 52.7 & 78.9 & 42.1 & 58.1 & 67.9 & 77.6 & 79.3 & 58.0 & 77.1 & 85.1 \\
    \texttt{hau\_Latn} & 1 & 28.7 & 39.0 & 61.6 & 30.2 & 42.2 & 46.8 & 59.8 & 59.8 & 44.5 & 75.6 & 85.1 \\
    \texttt{min\_Arab} & 2 & 27.4 & 32.6 & 61.0 & 28.9 & 32.1 & 36.4 & 55.9 & 50.2 & 35.9 & 46.3 & 76.1 \\
    \texttt{umb\_Latn} & 1 & 28.9 & 34.8 & 58.7 & 30.5 & 42.2 & 45.8 & 52.6 & 53.5 & 35.8 & 46.8 & 61.1 \\
    \bottomrule
    \end{tabular}
    \end{adjustbox}
    \caption{\textbf{Subsets of language tiers for \its @ $k{=}3$}.}
    \label{tab:subsets-full-results-i2s-k3}
\end{table*}


\begin{table*}[ht]
    \rowcolors{2}{gray!15}{white} % Alternate row coloring: gray on every second row
    \begin{adjustbox}{max width=\textwidth}
    \centering
    \small
    \begin{tabular}{l c | c c c | c c c c c | c | c c}
    \toprule
    & & \multicolumn{3}{c|}{\textbf{Qwen2-VL}} & \multicolumn{5}{c|}{\textbf{InternVL 2.5}} & \multicolumn{1}{c|}{\textbf{Centurio Qwen}} & \multicolumn{2}{c}{\textbf{GPT-4o}} \\
    \cmidrule(lr){3-5} \cmidrule(lr){6-10}  \cmidrule(lr){11-11} \cmidrule(lr){12-13}
    \textbf{Lang} & \textbf{Tier} & Qwen2-VL 2B & Qwen2-VL 7B & Qwen2-VL 72B & InternVL 2.5 4B & InternVL 2.5 8B & InternVL 2.5 26B & InternVL 2.5 38B & InternVL 2.5 78B & Centurio Qwen 8B & GPT-4o-mini & GPT-4o \\
    \midrule
    \texttt{acm\_Arab} & 3 & 33.7 & 48.7 & 83.2 & 47.6 & 60.6 & 73.9 & 85.8 & 87.1 & 61.1 & 77.3 & N/A \\
    \texttt{aka\_Latn} & 2 & 27.0 & 34.0 & 64.8 & 34.5 & 47.0 & 52.6 & 60.6 & 60.3 & 43.2 & 60.7 & N/A \\
    \texttt{apc\_Arab} & 3 & 32.7 & 48.9 & 83.9 & 46.2 & 61.2 & 73.6 & 84.7 & 87.3 & 60.3 & 77.4 & N/A \\
    \texttt{arb\_Arab} & 4 & 33.3 & 50.6 & 83.2 & 49.3 & 62.3 & 75.6 & 85.9 & 87.7 & 60.2 & 78.2 & N/A \\
    \texttt{azj\_Latn} & 2 & 32.2 & 47.9 & 82.2 & 45.5 & 63.0 & 72.8 & 83.6 & 85.3 & 59.4 & 78.1 & N/A \\
    \texttt{bul\_Cyrl} & 3 & 33.2 & 50.1 & 83.1 & 48.5 & 66.0 & 77.6 & 85.2 & 86.5 & 60.2 & 78.3 & N/A \\
    \texttt{ces\_Latn} & 4 & 34.9 & 52.2 & 82.2 & 48.6 & 66.0 & 78.8 & 84.7 & 86.4 & 58.9 & 79.3 & N/A \\
    \texttt{eng\_Latn} & 4 & 34.9 & 58.9 & 83.9 & 48.1 & 68.7 & N/A & 84.5 & 87.0 & 62.4 & 77.4 & N/A \\
    \texttt{fin\_Latn} & 4 & 32.6 & 51.1 & 82.2 & 46.8 & 66.6 & 76.3 & 83.7 & 86.3 & 60.7 & 77.0 & N/A \\
    \texttt{hat\_Latn} & 1 & 31.1 & 49.5 & 80.0 & 41.3 & 59.4 & 68.7 & 77.0 & 80.4 & 58.5 & 75.6 & N/A \\
    \texttt{hau\_Latn} & 1 & 28.4 & 35.9 & 63.3 & 31.4 & 42.8 & 46.2 & 60.5 & 60.4 & 44.7 & 73.9 & N/A \\
    \texttt{min\_Arab} & 2 & 27.4 & 30.1 & 62.0 & 28.5 & 32.3 & 36.3 & 55.7 & 49.8 & 36.4 & 43.9 & N/A \\
    \texttt{umb\_Latn} & 1 & 28.0 & 31.7 & 59.1 & 31.6 & 42.7 & 46.6 & 52.3 & 53.3 & 35.7 & 45.4 & N/A \\

    \bottomrule
    \end{tabular}
    \end{adjustbox}
    \caption{\textbf{Subsets of language tiers for \its @ $k{=}5$}.}
    \label{tab:subsets-full-results-i2s-k5}
\end{table*}

\begin{table*}[ht]
    \rowcolors{2}{gray!15}{white} % Alternate row coloring: gray on every second row
    \begin{adjustbox}{max width=\textwidth}
    \centering
    \small
    \begin{tabular}{l c | c c c | c c c c c | c | c c}
    \toprule
    & & \multicolumn{3}{c|}{\textbf{Qwen2-VL}} & \multicolumn{5}{c|}{\textbf{InternVL 2.5}} & \multicolumn{1}{c|}{\textbf{Centurio Qwen}} & \multicolumn{2}{c}{\textbf{GPT-4o}} \\
    \cmidrule(lr){3-5} \cmidrule(lr){6-10}  \cmidrule(lr){11-11} \cmidrule(lr){12-13}
    \textbf{Lang} & \textbf{Tier} & Qwen2-VL 2B & Qwen2-VL 7B & Qwen2-VL 72B & InternVL 2.5 4B & InternVL 2.5 8B & InternVL 2.5 26B & InternVL 2.5 38B & InternVL 2.5 78B & Centurio Qwen 8B & GPT-4o-mini & GPT-4o \\
    \midrule
    \texttt{acm\_Arab} & 3 & 46.6 & 81.1 & 91.1 & 71.5 & 80.5 & 84.9 & 91.3 & 89.6 & 83.2 & 89.9 & 92.3 \\
    \texttt{aka\_Latn} & 2 & 32.3 & 53.2 & 68.1 & 44.4 & 59.6 & 64.4 & 68.6 & 66.5 & 63.5 & 79.9 & 87.7 \\
    \texttt{apc\_Arab} & 3 & 46.0 & 80.9 & 91.3 & 70.0 & 80.4 & 84.6 & 90.9 & 89.1 & 82.7 & 89.4 & 92.5 \\
    \texttt{arb\_Arab} & 4 & 48.1 & 82.7 & 91.4 & 73.1 & 81.3 & 86.5 & 91.7 & 90.2 & 84.9 & 89.9 & 92.8 \\
    \texttt{azj\_Latn} & 2 & 42.6 & 80.2 & 90.2 & 65.5 & 76.9 & 82.0 & 89.1 & 87.3 & 82.0 & 89.1 & 91.9 \\
    \texttt{bul\_Cyrl} & 3 & 47.1 & 83.3 & 91.2 & 73.4 & 83.5 & 88.2 & 90.6 & 89.2 & 83.9 & 89.3 & 92.5 \\
    \texttt{ces\_Latn} & 4 & 50.8 & 82.8 & 90.6 & 75.3 & 85.3 & 89.8 & 90.8 & 90.0 & 84.6 & 90.1 & 92.2 \\
    \texttt{eng\_Latn} & 4 & 56.7 & 85.7 & 91.8 & 81.4 & 87.0 & 91.3 & 91.8 & 91.5 & 85.4 & 88.5 & 92.0 \\
    \texttt{fin\_Latn} & 4 & 42.3 & 82.0 & 91.4 & 70.5 & 82.3 & 87.1 & 90.2 & 88.1 & 81.8 & 89.2 & 92.3 \\
    \texttt{hat\_Latn} & 1 & 40.2 & 70.8 & 87.3 & 57.9 & 74.1 & 79.2 & 84.9 & 83.7 & 82.0 & 87.3 & 91.6 \\
    \texttt{hau\_Latn} & 1 & 32.6 & 52.1 & 66.5 & 40.0 & 53.8 & 57.5 & 70.7 & 66.2 & 66.5 & 87.3 & 91.9 \\
    \texttt{min\_Arab} & 2 & 28.3 & 42.3 & 68.0 & 32.8 & 41.5 & 41.4 & 63.7 & 56.4 & 49.8 & 62.8 & 86.4 \\
    \texttt{umb\_Latn} & 1 & 30.3 & 47.1 & 61.4 & 39.3 & 52.4 & 57.2 & N/A & 59.6 & 53.4 & 62.9 & 71.7 \\
    \bottomrule
    \end{tabular}
    \end{adjustbox}
    \caption{\textbf{Subsets of language tiers for \tts @ $k{=}1$}.}
    \label{tab:subsets-full-results-t2s-k1}
\end{table*}

\begin{table*}[ht]
    \rowcolors{2}{gray!15}{white} % Alternate row coloring: gray on every second row
    \begin{adjustbox}{max width=\textwidth}
    \centering
    \small
    \begin{tabular}{l c | c c c | c c c c c | c | c c}
    \toprule
    & & \multicolumn{3}{c|}{\textbf{Qwen2-VL}} & \multicolumn{5}{c|}{\textbf{InternVL 2.5}} & \multicolumn{1}{c|}{\textbf{Centurio Qwen}} & \multicolumn{2}{c}{\textbf{GPT-4o}} \\
    \cmidrule(lr){3-5} \cmidrule(lr){6-10}  \cmidrule(lr){11-11} \cmidrule(lr){12-13}
    \textbf{Lang} & \textbf{Tier} & Qwen2-VL 2B & Qwen2-VL 7B & Qwen2-VL 72B & InternVL 2.5 4B & InternVL 2.5 8B & InternVL 2.5 26B & InternVL 2.5 38B & InternVL 2.5 78B & Centurio Qwen 8B & GPT-4o-mini & GPT-4o \\
    \midrule
    \texttt{aeb\_Arab} & 3 & 43.1 & 62.9 & 80.1 & 36.8 & 53.2 & 68.7 & 79.3 & 77.6 & 29.8 & 76.3 & N/A \\
    \texttt{arb\_Arab} & 4 & 44.3 & 66.5 & 80.2 & 37.7 & 55.2 & 71.9 & 81.4 & 79.9 & 30.8 & 77.7 & N/A \\
    \texttt{ben\_Beng} & 3 & 39.8 & 64.4 & 81.2 & 41.2 & 50.9 & 69.0 & 79.1 & 77.2 & 33.0 & 76.4 & N/A \\
    \texttt{eng\_Latn} & 4 & 41.9 & 71.7 & N/A & 47.7 & 66.2 & N/A & 81.7 & 81.3 & 35.3 & 77.5 & N/A \\
    \texttt{fao\_Latn} & 2 & 33.6 & 52.2 & 75.1 & 30.9 & 43.9 & 56.8 & 73.6 & 69.0 & 29.2 & 75.8 & N/A \\
    \texttt{kac\_Latn} & 1 & 28.8 & 35.4 & 51.5 & 26.4 & 37.8 & 44.7 & 51.6 & 46.9 & 26.7 & 46.6 & N/A \\
    \texttt{kas\_Deva} & 2 & 33.3 & 47.1 & 72.4 & 30.2 & 42.4 & 53.0 & 69.0 & 59.3 & 27.7 & 66.5 & N/A \\
    \texttt{lit\_Latn} & 3 & 38.1 & 63.6 & 78.9 & 34.7 & 50.2 & 68.8 & 78.4 & 75.1 & 30.1 & 77.4 & N/A \\
    \texttt{lua\_Latn} & 1 & 31.0 & 43.8 & 57.6 & 29.3 & 41.7 & 48.8 & 57.4 & 55.9 & 27.7 & 54.9 & N/A \\
    \texttt{mal\_Mlym} & 2 & 32.1 & 64.3 & 80.1 & 34.1 & 47.0 & 62.3 & 77.0 & 73.2 & 31.0 & 76.2 & N/A \\
    \texttt{srp\_Cyrl} & 4 & 42.7 & 66.2 & 79.5 & 37.0 & 53.0 & 71.3 & 80.1 & 78.2 & 31.7 & 77.9 & N/A \\
    \texttt{tur\_Latn} & 4 & 42.7 & 66.6 & 78.5 & 38.7 & 56.4 & 74.4 & 80.5 & 77.6 & 31.0 & 77.4 & N/A \\
    \texttt{wol\_Latn} & 1 & 31.5 & 42.1 & 56.8 & 28.4 & 40.0 & 49.8 & 57.0 & 54.0 & 27.2 & 56.5 & N/A \\
    \bottomrule
    \end{tabular}
    \end{adjustbox}
    \caption{\textbf{Subsets of language tiers for \sti @ $k{=}1$}.}
    \label{tab:subsets-full-results-s2i-k1}
\end{table*}
\begin{table*}[ht]
    \rowcolors{2}{gray!15}{white} % Alternate row coloring: gray on every second row
    \begin{adjustbox}{max width=\textwidth}
    \centering
    \small
    \begin{tabular}{l c | c c c | c c c c c | c | c c}
    \toprule
    & & \multicolumn{3}{c|}{\textbf{Qwen2-VL}} & \multicolumn{5}{c|}{\textbf{InternVL 2.5}} & \multicolumn{1}{c|}{\textbf{Centurio Qwen}} & \multicolumn{2}{c}{\textbf{GPT-4o}} \\
    \cmidrule(lr){3-5} \cmidrule(lr){6-10}  \cmidrule(lr){11-11} \cmidrule(lr){12-13}
    \textbf{Lang} & \textbf{Tier} & Qwen2-VL 2B & Qwen2-VL 7B & Qwen2-VL 72B & InternVL 2.5 4B & InternVL 2.5 8B & InternVL 2.5 26B & InternVL 2.5 38B & InternVL 2.5 78B & Centurio Qwen 8B & GPT-4o-mini & GPT-4o \\
    \midrule
    \texttt{aeb\_Arab} & 3 & 44.1 & 65.3 & 89.6 & 38.5 & 58.6 & 76.0 & 92.0 & 89.0 & 31.7 & 85.4 & 88.3 \\
    \texttt{arb\_Arab} & 4 & 44.8 & 67.6 & 89.1 & 39.2 & 59.8 & 78.9 & 92.4 & 90.1 & 32.8 & 86.8 & 89.1 \\
    \texttt{ben\_Beng} & 3 & 40.4 & 67.0 & 88.9 & 43.4 & 58.1 & 74.5 & 90.8 & 89.2 & 33.9 & 85.7 & 87.9 \\
    \texttt{eng\_Latn} & 4 & 43.1 & 70.4 & 88.5 & 44.5 & 69.0 & 84.0 & 92.1 & 90.0 & 36.1 & 86.4 & 89.1 \\
    \texttt{fao\_Latn} & 2 & 35.0 & 55.2 & 87.5 & 34.9 & 50.6 & 68.9 & 89.5 & 85.9 & 30.5 & 86.0 & 89.1 \\
    \texttt{kac\_Latn} & 1 & 31.6 & 41.0 & 69.7 & 29.2 & 46.3 & 57.5 & 72.3 & 68.0 & 27.4 & 64.2 & 70.5 \\
    \texttt{kas\_Deva} & 2 & 35.1 & 50.1 & 87.0 & 31.9 & 48.9 & 60.6 & 86.5 & 78.0 & 28.9 & 80.4 & 86.1 \\
    \texttt{lit\_Latn} & 3 & 42.1 & 64.7 & 89.1 & 39.2 & 58.5 & 77.4 & 91.0 & 89.4 & 32.2 & 86.2 & 89.0 \\
    \texttt{lua\_Latn} & 1 & 33.7 & 46.2 & 74.6 & 33.0 & 50.8 & 61.5 & 78.5 & 77.1 & 27.8 & 70.4 & 79.3 \\
    \texttt{mal\_Mlym} & 2 & 33.7 & 65.3 & 88.0 & 38.1 & 50.8 & 69.7 & 90.7 & 88.2 & 31.0 & 85.1 & 87.8 \\
    \texttt{srp\_Cyrl} & 4 & 44.1 & 65.1 & 88.7 & 39.7 & 59.4 & 78.3 & 91.8 & 89.8 & 32.6 & 87.1 & 89.5 \\
    \texttt{tur\_Latn} & 4 & 44.6 & 66.7 & 88.3 & 41.3 & 61.0 & 81.7 & 91.8 & 88.8 & 34.3 & 87.0 & 88.5 \\
    \texttt{wol\_Latn} & 1 & 34.8 & 45.8 & 75.2 & 32.9 & 49.7 & 62.3 & 79.2 & 77.0 & 28.1 & 72.7 & 83.1 \\
    \bottomrule
    \end{tabular}
    \end{adjustbox}
    \caption{\textbf{Subsets of language tiers for \sti @ $k{=}3$}.}
    \label{tab:subsets-full-results-s2i-k3}
\end{table*}

\begin{table*}[ht]
    \rowcolors{2}{gray!15}{white} % Alternate row coloring: gray on every second row
    \begin{adjustbox}{max width=\textwidth}
    \centering
    \small
    \begin{tabular}{l c | c c c | c c c c c | c | c c}
    \toprule
    & & \multicolumn{3}{c|}{\textbf{Qwen2-VL}} & \multicolumn{5}{c|}{\textbf{InternVL 2.5}} & \multicolumn{1}{c|}{\textbf{Centurio Qwen}} & \multicolumn{2}{c}{\textbf{GPT-4o}} \\
    \cmidrule(lr){3-5} \cmidrule(lr){6-10}  \cmidrule(lr){11-11} \cmidrule(lr){12-13}
    \textbf{Lang} & \textbf{Tier} & Qwen2-VL 2B & Qwen2-VL 7B & Qwen2-VL 72B & InternVL 2.5 4B & InternVL 2.5 8B & InternVL 2.5 26B & InternVL 2.5 38B & InternVL 2.5 78B & Centurio Qwen 8B & GPT-4o-mini & GPT-4o \\
    \midrule
    \texttt{aeb\_Arab} & 3 & 44.0 & 65.4 & 91.6 & 39.0 & 57.6 & 77.0 & 93.6 & 91.8 & 32.8 & 88.9 & N/A \\
    \texttt{arb\_Arab} & 4 & 44.3 & 65.7 & 90.8 & 40.0 & 59.9 & 79.4 & 94.0 & 91.9 & 33.5 & 88.7 & N/A \\
    \texttt{ben\_Beng} & 3 & 41.1 & 63.2 & 89.3 & 45.9 & 59.4 & 76.4 & 93.0 & 91.5 & 34.8 & 88.4 & N/A \\
    \texttt{eng\_Latn} & 4 & 43.4 & 68.6 & 89.6 & 43.0 & 68.7 & N/A & 93.7 & 91.7 & 35.6 & 89.1 & N/A \\
    \texttt{fao\_Latn} & 2 & 35.3 & 56.6 & 90.4 & 36.0 & 50.8 & 71.8 & 92.8 & 89.5 & 30.7 & 87.8 & N/A \\
    \texttt{kac\_Latn} & 1 & 31.8 & 42.3 & 74.8 & 31.6 & 46.2 & 60.6 & 79.2 & 73.9 & 27.0 & 68.4 & N/A \\
    \texttt{kas\_Deva} & 2 & 35.3 & 48.8 & 89.0 & 34.1 & 50.0 & 64.7 & 91.3 & 82.9 & 29.4 & 83.2 & N/A \\
    \texttt{lit\_Latn} & 3 & 42.1 & 63.5 & 91.1 & 39.5 & 57.7 & 78.1 & 93.8 & 92.1 & 32.8 & 88.2 & N/A \\
    \texttt{lua\_Latn} & 1 & 34.6 & 48.4 & 80.2 & 34.6 & 49.7 & 63.1 & 85.1 & 82.8 & 28.1 & 75.9 & N/A \\
    \texttt{mal\_Mlym} & 2 & 31.6 & 62.7 & 88.9 & 39.0 & 51.8 & 69.3 & 93.2 & 91.7 & 31.5 & 87.1 & N/A \\
    \texttt{srp\_Cyrl} & 4 & 44.5 & 63.5 & 90.7 & 41.2 & 58.4 & 79.4 & 93.6 & 91.4 & 33.3 & 89.1 & N/A \\
    \texttt{tur\_Latn} & 4 & 45.6 & 64.8 & 90.3 & 40.9 & 59.9 & 81.4 & 93.3 & 91.3 & 34.8 & 88.5 & N/A \\
    \texttt{wol\_Latn} & 1 & 34.4 & 46.3 & 82.1 & 34.8 & 50.9 & 64.3 & 86.1 & 82.2 & 27.9 & 77.9 & N/A \\
    \bottomrule
    \end{tabular}
    \end{adjustbox}
    \caption{\textbf{Subsets of language tiers for \sti @ $k{=}5$}.}
    \label{tab:subsets-full-results-s2i-k5}
\end{table*}

\begin{table*}[ht]
    \rowcolors{2}{gray!15}{white} % Alternate row coloring: gray on every second row
    \begin{adjustbox}{max width=\textwidth}
    \centering
    \small
    \begin{tabular}{l c | c c c | c c c c c | c | c c}
    \toprule
    & & \multicolumn{3}{c|}{\textbf{Qwen2-VL}} & \multicolumn{5}{c|}{\textbf{InternVL 2.5}} & \multicolumn{1}{c|}{\textbf{Centurio Qwen}} & \multicolumn{2}{c}{\textbf{GPT-4o}} \\
    \cmidrule(lr){3-5} \cmidrule(lr){6-10}  \cmidrule(lr){11-11} \cmidrule(lr){12-13}
    \textbf{Lang} & \textbf{Tier} & Qwen2-VL 2B & Qwen2-VL 7B & Qwen2-VL 72B & InternVL 2.5 4B & InternVL 2.5 8B & InternVL 2.5 26B & InternVL 2.5 38B & InternVL 2.5 78B & Centurio Qwen 8B & GPT-4o-mini & GPT-4o \\
    \midrule
    \texttt{aeb\_Arab} & 3 & 77.9 & 85.2 & 91.0 & 87.5 & 83.5 & 83.9 & 91.5 & 91.9 & 88.8 & 91.5 & N/A \\
    \texttt{arb\_Arab} & 4 & 81.5 & 86.5 & 91.7 & 88.9 & 86.1 & 88.0 & 92.8 & 93.4 & 89.8 & 92.7 & N/A \\
    \texttt{ben\_Beng} & 3 & 68.8 & 85.2 & 91.2 & 83.7 & 79.5 & 81.6 & 91.2 & 90.8 & 86.4 & 91.1 & N/A \\
    \texttt{eng\_Latn} & 4 & 86.1 & 89.1 & 91.0 & 90.6 & 91.7 & 92.0 & 92.4 & 93.5 & 89.9 & 92.4 & N/A \\
    \texttt{fao\_Latn} & 2 & 52.1 & 72.1 & 86.3 & 71.0 & 72.8 & 76.9 & 86.5 & 83.9 & 80.3 & 90.0 & N/A \\
    \texttt{kac\_Latn} & 1 & 41.3 & 51.5 & 58.8 & 55.2 & 54.6 & 55.7 & 57.4 & 58.0 & 54.7 & 58.2 & N/A \\
    \texttt{kas\_Deva} & 2 & 50.1 & 72.4 & 83.5 & 71.7 & 65.1 & 63.4 & 80.4 & 74.2 & 75.0 & 82.9 & N/A \\
    \texttt{lit\_Latn} & 3 & 67.2 & 81.7 & 90.1 & 76.6 & 77.7 & 84.6 & 89.6 & 89.6 & 84.2 & 91.1 & N/A \\
    \texttt{lua\_Latn} & 1 & 49.4 & 58.5 & 65.2 & 61.3 & 61.0 & 59.0 & 64.8 & 68.5 & 63.3 & 65.8 & N/A \\
    \texttt{mal\_Mlym} & 2 & 49.0 & 83.0 & 89.7 & 76.9 & 69.1 & 74.3 & 88.7 & 87.2 & 80.6 & 90.5 & N/A \\
    \texttt{srp\_Cyrl} & 4 & 77.3 & 86.1 & 91.0 & 84.8 & 85.6 & 87.0 & 91.5 & 92.3 & 88.2 & 91.7 & N/A \\
    \texttt{tur\_Latn} & 4 & 76.4 & 85.0 & 91.3 & 85.1 & 87.5 & 89.4 & 91.7 & 91.6 & 88.2 & 91.5 & N/A \\
    \texttt{wol\_Latn} & 1 & 50.4 & 59.1 & 66.8 & 63.3 & 64.0 & 62.0 & 66.6 & 67.6 & 65.5 & 69.5 & N/A \\
    \bottomrule
    \end{tabular}
    \end{adjustbox}
    \caption{\textbf{Subsets of language tiers for \stt @ $k{=}1$}.}
    \label{tab:subsets-full-results-s2t-k1}
\end{table*}

\begin{table*}[ht]
    \rowcolors{2}{gray!15}{white} % Alternate row coloring: gray on every second row
    \begin{adjustbox}{max width=\textwidth}
    \centering
    \small
    \begin{tabular}{l c | c c c | c c c c c | c | c c}
    \toprule
    & & \multicolumn{3}{c|}{\textbf{Qwen2-VL}} & \multicolumn{5}{c|}{\textbf{InternVL 2.5}} & \multicolumn{1}{c|}{\textbf{Centurio Qwen}} & \multicolumn{2}{c}{\textbf{GPT-4o}} \\
    \cmidrule(lr){3-5} \cmidrule(lr){6-10}  \cmidrule(lr){11-11} \cmidrule(lr){12-13}
    \textbf{Lang} & \textbf{Tier} & Qwen2-VL 2B & Qwen2-VL 7B & Qwen2-VL 72B & InternVL 2.5 4B & InternVL 2.5 8B & InternVL 2.5 26B & InternVL 2.5 38B & InternVL 2.5 78B & Centurio Qwen 8B & GPT-4o-mini & GPT-4o \\
    \midrule
    \texttt{aeb\_Arab} & 3 & 94.1 & 93.9 & 98.7 & 97.2 & 95.3 & 96.3 & 98.8 & 99.0 & 96.6 & 98.4 & 98.4 \\
    \texttt{arb\_Arab} & 4 & 95.7 & 94.4 & 98.7 & 97.3 & 96.7 & 98.0 & 98.7 & 99.1 & 95.7 & 98.6 & 98.4 \\
    \texttt{ben\_Beng} & 3 & 86.9 & 93.9 & 98.7 & 95.3 & 93.0 & 94.7 & 98.6 & 98.4 & 96.5 & 98.5 & 98.3 \\
    \texttt{eng\_Latn} & 4 & 96.5 & 95.3 & 98.3 & 98.0 & 98.1 & 98.7 & 98.7 & 98.9 & 96.7 & 98.1 & 98.0 \\
    \texttt{fao\_Latn} & 2 & 69.4 & 86.6 & 97.6 & 88.7 & 87.1 & 92.6 & 97.2 & 97.2 & 94.1 & 98.2 & 98.3 \\
    \texttt{kac\_Latn} & 1 & 57.6 & 68.0 & 80.1 & 75.0 & 77.7 & 75.5 & 79.4 & 79.9 & 75.3 & 78.6 & 79.1 \\
    \texttt{kas\_Deva} & 2 & 68.6 & 88.4 & 96.5 & 86.9 & 81.2 & 82.3 & 95.4 & 91.4 & 90.5 & 95.9 & 96.7 \\
    \texttt{lit\_Latn} & 3 & 87.1 & 92.4 & 98.6 & 92.2 & 94.9 & 97.6 & 98.2 & 98.3 & 94.9 & 98.6 & 98.2 \\
    \texttt{lua\_Latn} & 1 & 66.7 & 76.9 & 86.2 & 82.1 & 83.0 & 82.1 & 86.5 & 88.3 & 81.4 & 87.4 & 92.2 \\
    \texttt{mal\_Mlym} & 2 & 61.1 & 93.3 & 98.7 & 91.2 & 85.5 & 90.7 & 98.3 & 97.7 & 94.0 & 98.3 & 98.1 \\
    \texttt{srp\_Cyrl} & 4 & 93.2 & 94.4 & 98.7 & 96.2 & 96.5 & 97.6 & 98.7 & 98.9 & 96.5 & 98.3 & 98.4 \\
    \texttt{tur\_Latn} & 4 & 92.5 & 94.9 & 98.8 & 96.6 & 97.2 & 98.6 & 98.6 & 99.0 & 96.0 & 98.6 & 98.1 \\
    \texttt{wol\_Latn} & 1 & 66.5 & 77.1 & 88.9 & 84.5 & 84.9 & 86.0 & 88.8 & 89.6 & 85.0 & 89.3 & 95.1 \\
    \bottomrule
    \end{tabular}
    \end{adjustbox}
    \caption{\textbf{Subsets of language tiers for \stt @ $k{=}3$}.}
    \label{tab:subsets-full-results-s2t-k3}
\end{table*}

\begin{table*}[ht]
    \rowcolors{2}{gray!15}{white} % Alternate row coloring: gray on every second row
    \begin{adjustbox}{max width=\textwidth}
    \centering
    \small
    \begin{tabular}{l c | c c c | c c c c c | c | c c}
    \toprule
    & & \multicolumn{3}{c|}{\textbf{Qwen2-VL}} & \multicolumn{5}{c|}{\textbf{InternVL 2.5}} & \multicolumn{1}{c|}{\textbf{Centurio Qwen}} & \multicolumn{2}{c}{\textbf{GPT-4o}} \\
    \cmidrule(lr){3-5} \cmidrule(lr){6-10}  \cmidrule(lr){11-11} \cmidrule(lr){12-13}
    \textbf{Lang} & \textbf{Tier} & Qwen2-VL 2B & Qwen2-VL 7B & Qwen2-VL 72B & InternVL 2.5 4B & InternVL 2.5 8B & InternVL 2.5 26B & InternVL 2.5 38B & InternVL 2.5 78B & Centurio Qwen 8B & GPT-4o-mini & GPT-4o \\
    \midrule
    \texttt{aeb\_Arab} & 3 & 97.2 & 97.4 & 99.7 & 98.7 & 97.7 & 98.8 & 99.7 & 99.7 & 98.3 & 99.5 & N/A \\
    \texttt{arb\_Arab} & 4 & 97.6 & 97.4 & 99.6 & 98.6 & 98.5 & 99.3 & 99.7 & 99.7 & 97.6 & 99.3 & N/A \\
    \texttt{ben\_Beng} & 3 & 92.6 & 96.9 & 99.5 & 97.6 & 96.8 & 97.9 & 99.6 & 99.5 & 97.7 & 99.2 & N/A \\
    \texttt{eng\_Latn} & 4 & 98.2 & 97.5 & 99.2 & 99.1 & 99.0 & 99.5 & 99.4 & 99.6 & 97.7 & 99.1 & N/A \\
    \texttt{fao\_Latn} & 2 & 79.9 & 91.8 & 99.4 & 92.7 & 92.9 & 96.3 & 99.2 & 99.5 & 96.9 & 99.0 & N/A \\
    \texttt{kac\_Latn} & 1 & 68.0 & 75.0 & 87.4 & 81.5 & 86.0 & 83.9 & 87.1 & 87.3 & 80.9 & 86.3 & N/A \\
    \texttt{kas\_Deva} & 2 & 78.4 & 92.4 & 99.0 & 92.3 & 87.9 & 87.5 & 99.0 & 96.6 & 93.8 & 98.7 & N/A \\
    \texttt{lit\_Latn} & 3 & 93.8 & 96.0 & 99.4 & 95.9 & 98.2 & 99.3 & 99.4 & 99.7 & 97.2 & 99.4 & N/A \\
    \texttt{lua\_Latn} & 1 & 75.9 & 83.3 & 91.9 & 88.1 & 90.1 & 89.1 & 92.4 & 93.5 & 85.2 & 93.1 & N/A \\
    \texttt{mal\_Mlym} & 2 & 67.6 & 96.7 & 99.6 & 95.5 & 90.2 & 95.7 & 99.5 & 99.1 & 97.4 & 99.2 & N/A \\
    \texttt{srp\_Cyrl} & 4 & 97.2 & 97.4 & 99.4 & 98.0 & 99.0 & 99.4 & 99.6 & 99.6 & 98.2 & 99.4 & N/A \\
    \texttt{tur\_Latn} & 4 & 95.6 & 97.2 & 99.4 & 98.4 & 98.8 & 99.5 & 99.4 & 99.7 & 97.6 & 99.4 & N/A \\
    \texttt{wol\_Latn} & 1 & 77.5 & 84.3 & 94.5 & 90.4 & 92.7 & 92.9 & 95.2 & 95.7 & 91.1 & 94.9 & N/A \\

    \bottomrule
    \end{tabular}
    \end{adjustbox}
    \caption{\textbf{Subsets of language tiers for \stt @ $k{=}5$}.}
    \label{tab:subsets-full-results-s2t-k5}
\end{table*}

\begin{table*}[ht]
\subsection{Full Performance by Task, Model, and Language}
\label{app:per-language-results}
\subsubsection{Images-To-Topics}
\label{app:images-to-topics}
    \begin{adjustbox}{max width=\textwidth}

    \begin{tabular}{l *{7}{c}}

    & \multicolumn{7}{c}{Topics}\\
    \cmidrule{2-8}
Model & 
 Entertainment & 
 Geography & 
 Health &
 Politics &
 Science \& Tech. &
 Sports &
 Travel \\ \midrule
QwenVL-2.5-2B   &  99.4 & 83.1 & 100.0 &  99.9  & 100.0  & 100.0 &  95.7 \\
QwenVL-2.5-7B   & 100.0 & 92.9 & 100.0 & 100.0  &  99.8  & 100.0 & 100.0 \\
InternVL-2.5-4B &  99.5 & 84.5 & 100.0 & 100.0  &  99.7  & 100.0 & 100.0 \\
InternVL-2.5-8B & 100.0 & 90.0 & 100.0 & 100.0  &  97.5  & 100.0 & 100.0 \\
Centurio-Qwen   &  99.6 & 90.6 & 100.0 & 100.0  &  99.2  & 100.0 & 100.0 \\ \bottomrule
    \end{tabular} 
    \end{adjustbox}
    \caption{\textbf{Image-To-Topics.} For a reference image, the model must pick the correct topic out of 4 choices.}
    \label{tab:images-to-topics}
\end{table*}



\begin{table*}[ht]
    \subsubsection{Images-To-Sentences}
    \label{app:images-to-sentences}
    \begin{minipage}{0.48\textwidth}
        \centering
        \renewcommand{\arraystretch}{0.8} % Reduce row height 
        \rowcolors{2}{gray!15}{white} % Alternate row coloring: gray on every second row
        \begin{adjustbox}{max height=0.6\textheight, max width=\columnwidth}
            \begin{tabular}{l *{18}{c}}
                \toprule
                Lang.  & 
                \multicolumn{3}{c}{QwenVL-2 2B} & 
                \multicolumn{3}{c}{QwenVL-2 7B} & 
                \multicolumn{3}{c}{InternVL-2.5 4B} & 
                \multicolumn{3}{c}{InternVL-2.5 8B} & 
                \multicolumn{3}{c}{Centurio-Qwen} &
                \multicolumn{3}{c}{4o-mini} \\
                \cmidrule(lr){2-4} \cmidrule(lr){5-7} \cmidrule(lr){8-10} \cmidrule(lr){11-13} \cmidrule(lr){14-16} \cmidrule(lr){17-19}
                \textit{k} & 
                1 & 3 & 5 & 
                1 & 3 & 5 & 
                1 & 3 & 5 & 
                1 & 3 & 5 & 
                1 & 3 & 5 & 
                1 & 3 & 5 \\
                \midrule
                % Add more rows as needed
                \texttt{Avg}       & 31.8 & 31.4 & 30.7 & 46.6 & 46.4 & 42.4 & 40.9 & 39.0 & 39.3 & 53.1 & 53.3 & 53.7 & 47.8 & 50.5 & 51.1 & 64.2 & 69.2 & 67.7 \\
                \texttt{eng\_Latn} & 36.3 & 34.8 & 34.9 & 65.8 & 63.1 & 58.9 & 52.5 & 49.2 & 48.1 & 67.7 & 67.9 & 68.7 & 54.8 & 60.0 & 62.4 & 68.3 & 78.1 & 77.4 \\
                \texttt{ace\_Arab} & 27.7 & 27.9 & 27.4 & 33.4 & 34.3 & 31.7 & 29.5 & 28.9 & 29.4 & 34.6 & 34.1 & 34.0 & 38.9 & 37.5 & 38.3 & 48.1 & 49.4 & 46.6 \\
                \texttt{ace\_Latn} & 32.4 & 33.1 & 32.0 & 45.5 & 45.7 & 42.0 & 39.9 & 38.7 & 38.4 & 55.7 & 57.1 & 58.0 & 46.1 & 51.7 & 51.5 & 62.4 & 67.0 & 65.3 \\
                \texttt{acm\_Arab} & 34.9 & 35.2 & 33.7 & 55.4 & 55.1 & 48.7 & 50.0 & 45.6 & 47.6 & 61.0 & 60.6 & 60.6 & 54.4 & 59.7 & 61.1 & 72.0 & 79.3 & 77.3 \\
                \texttt{acq\_Arab} & 35.2 & 35.0 & 32.8 & 56.4 & 55.2 & 50.0 & 50.8 & 45.8 & 48.3 & 61.8 & 60.8 & 61.0 & 54.1 & 59.1 & 60.9 & 72.6 & 79.5 & 78.0 \\
                \texttt{aeb\_Arab} & 34.6 & 33.5 & 32.2 & 53.6 & 52.0 & 46.7 & 48.2 & 44.2 & 45.1 & 59.2 & 58.2 & 59.3 & 55.3 & 59.4 & 60.3 & 71.8 & 78.6 & 76.3 \\
                \texttt{afr\_Latn} & 35.4 & 35.0 & 34.7 & 56.8 & 57.4 & 52.8 & 48.4 & 45.2 & 46.4 & 61.0 & 64.3 & 64.7 & 55.4 & 60.8 & 61.8 & 71.0 & 78.5 & 77.7 \\
                \texttt{ajp\_Arab} & 35.2 & 34.1 & 33.1 & 55.9 & 55.4 & 49.2 & 49.8 & 46.2 & 48.0 & 60.4 & 58.7 & 60.8 & 53.9 & 59.2 & 60.5 & 72.6 & 78.6 & 76.8 \\
                \texttt{aka\_Latn} & 28.7 & 27.9 & 27.0 & 38.0 & 38.3 & 34.0 & 34.6 & 34.2 & 34.5 & 46.3 & 46.3 & 47.0 & 42.4 & 42.6 & 43.2 & 61.4 & 63.3 & 60.7 \\
                \texttt{als\_Latn} & 33.4 & 33.5 & 32.4 & 55.4 & 54.7 & 51.0 & 43.9 & 42.3 & 42.9 & 59.1 & 61.0 & 62.8 & 51.1 & 56.6 & 58.3 & 72.6 & 78.5 & 77.4 \\
                \texttt{amh\_Ethi} & 24.9 & 25.3 & 24.9 & 30.1 & 31.5 & 29.5 & 33.8 & 33.5 & 32.4 & 29.7 & 28.3 & 27.4 & 44.3 & 46.8 & 47.1 & 57.2 & 58.1 & 54.4 \\
                \texttt{apc\_Arab} & 34.9 & 34.7 & 32.7 & 54.8 & 54.3 & 48.9 & 49.1 & 45.1 & 46.2 & 61.3 & 60.5 & 61.2 & 54.4 & 59.4 & 60.3 & 71.6 & 78.5 & 77.3 \\
                \texttt{arb\_Arab} & 35.2 & 35.2 & 33.3 & 57.1 & 56.4 & 50.6 & 51.7 & 46.9 & 49.3 & 62.2 & 60.9 & 62.3 & 55.2 & 58.7 & 60.2 & 73.2 & 79.7 & 78.2 \\
                \texttt{arb\_Latn} & 28.7 & 28.7 & 28.9 & 39.2 & 39.5 & 37.5 & 35.4 & 34.0 & 35.2 & 43.5 & 44.7 & 46.3 & 46.3 & 46.2 & 46.9 & 67.8 & 72.9 & 71.6 \\
                \texttt{ars\_Arab} & 34.7 & 34.6 & 33.1 & 56.4 & 56.4 & 50.7 & 51.3 & 46.6 & 49.2 & 62.0 & 60.7 & 61.5 & 54.7 & 59.1 & 59.7 & 72.7 & 80.2 & 78.4 \\
                \texttt{ary\_Arab} & 33.8 & 32.4 & 31.6 & 52.2 & 50.7 & 45.7 & 45.8 & 43.1 & 43.5 & 56.8 & 56.5 & 57.5 & 54.0 & 57.4 & 58.8 & 70.3 & 76.5 & 74.9 \\
                \texttt{arz\_Arab} & 35.2 & 33.9 & 33.2 & 56.1 & 55.5 & 49.6 & 50.9 & 46.5 & 47.9 & 61.4 & 58.8 & 61.6 & 55.1 & 60.0 & 61.3 & 72.9 & 79.0 & 77.9 \\
                \texttt{asm\_Beng} & 30.7 & 30.2 & 30.0 & 46.1 & 45.5 & 41.0 & 44.6 & 42.6 & 40.4 & 56.4 & 52.8 & 51.6 & 51.5 & 53.1 & 51.8 & 70.8 & 75.7 & 74.1 \\
                \texttt{ast\_Latn} & 35.9 & 34.9 & 34.1 & 58.7 & 58.0 & 54.2 & 46.8 & 43.6 & 45.1 & 61.9 & 63.8 & 64.2 & 54.9 & 61.6 & 61.7 & 69.3 & 77.5 & 76.3 \\
                \texttt{awa\_Deva} & 34.4 & 34.5 & 33.0 & 48.1 & 46.5 & 41.8 & 43.4 & 40.9 & 39.9 & 63.0 & 59.5 & 57.4 & 53.8 & 58.7 & 58.7 & 70.7 & 77.6 & 76.5 \\
                \texttt{ayr\_Latn} & 29.0 & 28.0 & 28.2 & 33.8 & 33.5 & 31.1 & 30.5 & 30.0 & 29.7 & 40.6 & 41.3 & 41.9 & 35.4 & 34.5 & 35.7 & 44.0 & 47.3 & 46.1 \\
                \texttt{azb\_Arab} & 31.9 & 31.3 & 31.0 & 44.3 & 44.2 & 40.1 & 33.3 & 31.8 & 32.7 & 49.8 & 49.2 & 51.3 & 52.9 & 54.2 & 55.5 & 63.1 & 68.7 & 66.6 \\
                \texttt{azj\_Latn} & 34.6 & 33.1 & 32.2 & 53.8 & 53.4 & 47.9 & 49.5 & 46.2 & 45.5 & 58.6 & 60.9 & 63.0 & 51.8 & 57.9 & 59.4 & 72.8 & 78.6 & 78.1 \\
                \texttt{bak\_Cyrl} & 32.5 & 33.8 & 32.2 & 49.3 & 48.8 & 42.6 & 41.5 & 40.1 & 39.3 & 54.9 & 55.8 & 57.0 & 49.4 & 51.2 & 53.1 & 71.0 & 76.2 & 75.0 \\
                \texttt{bam\_Latn} & 28.0 & 27.7 & 26.9 & 34.7 & 34.3 & 32.4 & 32.3 & 31.2 & 31.5 & 41.9 & 42.0 & 43.6 & 40.5 & 40.2 & 41.4 & 43.4 & 45.2 & 43.9 \\
                \texttt{ban\_Latn} & 35.5 & 36.3 & 35.1 & 51.9 & 54.3 & 49.1 & 41.2 & 39.2 & 39.4 & 56.4 & 61.5 & 61.1 & 51.6 & 56.8 & 57.6 & 67.3 & 72.3 & 70.3 \\
                \texttt{bel\_Cyrl} & 34.1 & 35.6 & 33.7 & 52.8 & 51.9 & 47.2 & 45.9 & 43.6 & 43.8 & 57.7 & 57.0 & 58.5 & 51.9 & 55.7 & 56.2 & 73.0 & 78.6 & 78.2 \\
                \texttt{bem\_Latn} & 28.7 & 28.9 & 28.6 & 39.5 & 40.2 & 36.6 & 35.4 & 34.7 & 34.9 & 46.4 & 47.7 & 47.3 & 39.9 & 40.3 & 40.9 & 56.0 & 58.0 & 55.4 \\
                \texttt{ben\_Beng} & 32.8 & 33.6 & 32.2 & 48.8 & 46.5 & 41.6 & 47.0 & 44.3 & 43.3 & 62.2 & 58.1 & 55.7 & 51.7 & 55.7 & 54.3 & 72.4 & 78.9 & 77.1 \\
                \texttt{bho\_Deva} & 33.8 & 33.5 & 32.5 & 45.3 & 44.1 & 39.7 & 42.3 & 38.6 & 38.2 & 61.2 & 56.2 & 56.5 & 52.3 & 56.3 & 57.2 & 70.5 & 76.1 & 75.2 \\
                \texttt{bjn\_Arab} & 28.0 & 27.2 & 27.3 & 31.6 & 33.8 & 31.2 & 27.7 & 27.5 & 28.6 & 33.5 & 32.0 & 32.5 & 35.5 & 36.4 & 35.4 & 47.1 & 47.9 & 45.1 \\
                \texttt{bjn\_Latn} & 33.3 & 31.9 & 32.1 & 50.6 & 51.9 & 48.0 & 41.8 & 39.0 & 39.4 & 54.2 & 57.3 & 57.8 & 53.6 & 57.3 & 59.2 & 66.4 & 73.3 & 71.2 \\
                \texttt{bod\_Tibt} & 25.3 & 24.5 & 24.4 & 25.6 & 26.2 & 25.0 & 25.5 & 25.1 & 24.7 & 42.7 & 41.7 & 40.7 & 38.0 & 37.6 & 36.5 & 32.5 & 32.7 & 30.5 \\
                \texttt{bos\_Latn} & 35.9 & 35.4 & 34.1 & 58.6 & 58.0 & 53.8 & 51.7 & 49.3 & 50.3 & 65.9 & 67.1 & 67.6 & 52.9 & 58.0 & 59.8 & 71.4 & 79.2 & 78.2 \\
                \texttt{bug\_Latn} & 33.2 & 32.6 & 32.5 & 44.5 & 44.5 & 40.7 & 37.1 & 36.2 & 36.8 & 51.5 & 54.0 & 54.2 & 45.9 & 48.6 & 48.5 & 59.5 & 62.0 & 60.3 \\
                \texttt{bul\_Cyrl} & 35.0 & 34.7 & 33.2 & 56.6 & 54.6 & 50.1 & 50.3 & 45.9 & 48.5 & 65.3 & 64.4 & 66.0 & 53.9 & 58.8 & 60.2 & 71.5 & 79.5 & 78.3 \\
                \texttt{cat\_Latn} & 36.6 & 36.0 & 34.6 & 59.4 & 58.5 & 54.0 & 51.1 & 46.0 & 48.3 & 64.9 & 66.4 & 66.4 & 55.1 & 59.0 & 60.8 & 71.4 & 78.6 & 78.4 \\
                \texttt{ceb\_Latn} & 33.4 & 33.3 & 32.7 & 55.5 & 56.1 & 51.2 & 47.4 & 44.2 & 45.0 & 60.4 & 62.4 & 62.7 & 53.6 & 57.7 & 59.4 & 71.4 & 77.2 & 76.1 \\
                \texttt{ces\_Latn} & 36.2 & 35.6 & 34.9 & 58.2 & 56.8 & 52.2 & 52.0 & 48.9 & 48.6 & 65.6 & 65.6 & 66.0 & 53.2 & 57.3 & 58.9 & 72.7 & 78.6 & 79.3 \\
                \texttt{cjk\_Latn} & 29.0 & 29.2 & 29.0 & 36.9 & 36.8 & 34.5 & 32.7 & 32.4 & 33.2 & 43.3 & 45.6 & 46.6 & 38.2 & 38.1 & 38.0 & 43.3 & 46.2 & 45.3 \\
                \texttt{ckb\_Arab} & 29.1 & 29.4 & 29.2 & 41.7 & 39.4 & 36.1 & 35.7 & 35.2 & 35.4 & 39.9 & 39.2 & 39.0 & 40.3 & 40.7 & 41.5 & 64.7 & 66.4 & 64.3 \\
                \texttt{crh\_Latn} & 33.6 & 31.3 & 30.7 & 50.6 & 50.1 & 45.3 & 43.1 & 40.9 & 40.9 & 58.7 & 60.8 & 61.8 & 53.4 & 57.5 & 59.1 & 69.4 & 74.5 & 72.9 \\
                \texttt{cym\_Latn} & 30.5 & 30.1 & 30.2 & 53.3 & 53.1 & 48.6 & 40.9 & 39.5 & 39.9 & 57.7 & 59.7 & 61.7 & 50.9 & 52.8 & 52.5 & 71.6 & 77.8 & 77.0 \\
                \texttt{dan\_Latn} & 36.7 & 35.9 & 35.0 & 59.8 & 58.4 & 53.2 & 53.3 & 48.1 & 49.3 & 64.7 & 66.4 & 66.9 & 55.2 & 61.1 & 63.6 & 70.8 & 78.7 & 77.5 \\
                \texttt{deu\_Latn} & 36.5 & 36.7 & 35.4 & 61.0 & 59.2 & 53.8 & 51.7 & 46.6 & 47.8 & 67.2 & 68.2 & 67.7 & 54.9 & 60.6 & 63.0 & 70.7 & 79.0 & 78.5 \\
                \texttt{dik\_Latn} & 29.3 & 28.4 & 27.7 & 35.7 & 35.9 & 33.5 & 33.1 & 32.4 & 33.5 & 45.5 & 45.8 & 46.6 & 37.9 & 38.1 & 38.3 & 44.1 & 45.8 & 44.3 \\
                \texttt{dyu\_Latn} & 29.0 & 28.1 & 28.0 & 36.3 & 37.3 & 34.0 & 33.0 & 31.4 & 32.4 & 44.1 & 45.6 & 47.6 & 41.3 & 42.7 & 42.6 & 45.6 & 48.8 & 46.4 \\
                \texttt{dzo\_Tibt} & 23.9 & 23.8 & 24.4 & 24.4 & 25.5 & 24.3 & 24.5 & 25.1 & 24.8 & 39.5 & 37.5 & 36.1 & 38.4 & 37.5 & 36.4 & 30.9 & 31.5 & 29.5 \\
                \texttt{ell\_Grek} & 34.0 & 35.0 & 34.6 & 52.5 & 50.7 & 46.2 & 45.5 & 43.9 & 41.8 & 62.7 & 62.0 & 63.3 & 50.5 & 53.0 & 54.0 & 73.3 & 78.7 & 77.8 \\
                \texttt{epo\_Latn} & 34.1 & 33.8 & 33.2 & 58.0 & 57.9 & 53.3 & 50.3 & 46.7 & 47.4 & 61.7 & 63.8 & 64.8 & 52.6 & 58.7 & 58.1 & 70.0 & 77.1 & 77.0 \\
                \texttt{est\_Latn} & 34.4 & 34.0 & 32.9 & 55.7 & 56.6 & 51.3 & 48.5 & 44.4 & 43.9 & 61.8 & 63.0 & 64.2 & 53.4 & 57.8 & 57.7 & 70.0 & 78.3 & 77.1 \\
                \texttt{eus\_Latn} & 33.5 & 34.2 & 33.4 & 50.3 & 50.4 & 45.8 & 44.0 & 41.3 & 42.6 & 60.1 & 61.1 & 61.8 & 52.2 & 57.2 & 58.6 & 69.2 & 76.7 & 75.1 \\
                \texttt{ewe\_Latn} & 26.5 & 27.0 & 26.3 & 33.8 & 34.1 & 30.6 & 30.3 & 30.6 & 30.0 & 41.3 & 42.1 & 42.1 & 34.8 & 34.3 & 34.7 & 40.7 & 42.3 & 41.1 \\
                \texttt{fao\_Latn} & 29.9 & 29.6 & 29.7 & 49.3 & 48.5 & 44.7 & 40.3 & 37.2 & 38.8 & 54.4 & 56.8 & 58.8 & 51.8 & 53.7 & 54.4 & 68.9 & 74.7 & 72.4 \\
                \texttt{fij\_Latn} & 28.5 & 27.4 & 27.5 & 37.4 & 37.6 & 35.0 & 32.0 & 31.3 & 32.4 & 43.1 & 43.9 & 45.4 & 39.0 & 40.6 & 40.2 & 60.4 & 62.8 & 61.2 \\
                \texttt{fin\_Latn} & 33.0 & 33.5 & 32.6 & 57.1 & 56.4 & 51.1 & 49.8 & 45.8 & 46.8 & 64.1 & 65.6 & 66.6 & 54.1 & 59.7 & 60.7 & 70.2 & 78.3 & 77.0 \\
                \texttt{fon\_Latn} & 27.4 & 26.1 & 26.2 & 31.9 & 31.5 & 28.8 & 29.8 & 29.9 & 30.0 & 39.9 & 40.8 & 40.9 & 34.9 & 34.3 & 36.1 & 36.9 & 38.0 & 37.6 \\
                \texttt{fra\_Latn} & 35.9 & 35.5 & 34.1 & 62.7 & 61.3 & 55.3 & 52.2 & 48.1 & 49.4 & 67.0 & 67.7 & 68.3 & 54.2 & 59.5 & 61.2 & 71.7 & 79.8 & 78.7 \\
                \texttt{fur\_Latn} & 34.2 & 32.4 & 32.2 & 50.7 & 51.4 & 47.8 & 43.2 & 40.7 & 41.6 & 56.8 & 60.0 & 60.1 & 52.6 & 55.5 & 56.1 & 66.6 & 73.7 & 71.9 \\
                \texttt{fuv\_Latn} & 29.2 & 28.4 & 28.2 & 38.0 & 38.3 & 34.5 & 33.9 & 32.8 & 31.7 & 43.1 & 44.3 & 45.0 & 39.0 & 39.2 & 39.7 & 50.5 & 54.0 & 52.2 \\
                \texttt{gaz\_Latn} & 29.0 & 27.7 & 27.4 & 34.2 & 35.3 & 32.4 & 31.9 & 31.1 & 31.6 & 38.7 & 39.3 & 41.1 & 39.3 & 39.5 & 39.2 & 65.4 & 68.2 & 65.8 \\
                \texttt{gla\_Latn} & 28.8 & 28.6 & 28.6 & 42.5 & 43.8 & 40.7 & 36.5 & 35.5 & 35.0 & 46.4 & 47.4 & 49.3 & 47.5 & 49.3 & 49.0 & 67.4 & 71.4 & 69.3 \\
                \texttt{gle\_Latn} & 29.0 & 29.4 & 29.1 & 46.6 & 46.9 & 43.3 & 37.5 & 37.7 & 38.2 & 49.5 & 51.4 & 52.8 & 49.1 & 50.6 & 50.8 & 70.1 & 73.9 & 72.2 \\
                \texttt{glg\_Latn} & 35.0 & 35.3 & 34.0 & 60.2 & 58.8 & 54.2 & 50.7 & 47.8 & 47.3 & 64.0 & 64.0 & 64.9 & 54.8 & 61.3 & 62.6 & 70.1 & 78.9 & 77.8 \\
                \texttt{grn\_Latn} & 33.4 & 32.9 & 32.4 & 44.1 & 44.4 & 41.2 & 37.6 & 37.6 & 38.1 & 52.5 & 53.7 & 55.1 & 45.9 & 48.6 & 48.8 & 64.1 & 68.6 & 66.6 \\
                \texttt{guj\_Gujr} & 30.7 & 31.6 & 30.2 & 46.1 & 44.8 & 39.4 & 44.8 & 42.9 & 40.9 & 60.6 & 56.0 & 54.6 & 48.4 & 50.7 & 50.4 & 70.9 & 77.5 & 76.4 \\
                \texttt{hat\_Latn} & 32.1 & 30.7 & 31.1 & 50.4 & 52.7 & 49.5 & 43.0 & 42.1 & 41.3 & 54.7 & 58.1 & 59.4 & 54.9 & 58.0 & 58.5 & 69.3 & 77.1 & 75.6 \\
                \texttt{hau\_Latn} & 29.1 & 28.7 & 28.4 & 39.2 & 39.0 & 35.9 & 32.4 & 30.2 & 31.4 & 41.8 & 42.2 & 42.8 & 43.9 & 44.5 & 44.7 & 70.2 & 75.6 & 73.9 \\
                \texttt{heb\_Hebr} & 35.4 & 34.9 & 34.3 & 57.0 & 56.8 & 51.8 & 47.9 & 44.9 & 46.8 & 58.5 & 57.2 & 59.1 & 54.5 & 60.3 & 60.3 & 71.7 & 80.4 & 78.7 \\
                \texttt{hin\_Deva} & 34.2 & 34.2 & 33.0 & 50.7 & 48.4 & 43.6 & 47.0 & 44.8 & 43.7 & 64.9 & 60.3 & 60.7 & 52.2 & 57.8 & 57.6 & 72.3 & 79.7 & 78.1 \\
                \texttt{hne\_Deva} & 34.8 & 35.4 & 34.4 & 45.6 & 44.7 & 40.1 & 40.7 & 38.8 & 37.0 & 61.7 & 57.5 & 56.8 & 53.1 & 56.1 & 57.4 & 71.1 & 76.1 & 74.5 \\
                \texttt{hrv\_Latn} & 35.5 & 35.8 & 34.8 & 59.4 & 59.2 & 54.7 & 51.8 & 48.6 & 49.7 & 65.9 & 67.9 & 67.9 & 54.4 & 59.0 & 60.3 & 72.7 & 80.5 & 78.7 \\
                \texttt{hun\_Latn} & 35.1 & 33.7 & 32.4 & 56.2 & 54.4 & 49.2 & 48.0 & 45.2 & 46.4 & 64.6 & 65.4 & 66.7 & 56.6 & 61.8 & 62.2 & 71.4 & 76.9 & 76.2 \\
                \texttt{hye\_Armn} & 27.8 & 28.2 & 27.3 & 48.5 & 47.7 & 43.2 & 33.0 & 33.2 & 33.0 & 35.0 & 33.1 & 33.3 & 42.3 & 43.7 & 43.4 & 72.2 & 77.2 & 76.3 \\
                \texttt{ibo\_Latn} & 27.8 & 27.1 & 26.8 & 37.2 & 36.9 & 34.2 & 36.0 & 35.0 & 35.0 & 46.1 & 44.6 & 43.7 & 46.6 & 47.1 & 47.1 & 69.8 & 74.3 & 72.6 \\
                \texttt{ilo\_Latn} & 33.5 & 33.5 & 33.0 & 50.7 & 52.7 & 48.1 & 43.0 & 42.1 & 41.9 & 57.9 & 61.1 & 61.7 & 49.9 & 53.6 & 54.0 & 69.1 & 75.2 & 73.4 \\
                \texttt{ind\_Latn} & 37.8 & 37.0 & 35.8 & 60.2 & 59.6 & 54.4 & 50.6 & 47.1 & 48.3 & 64.1 & 65.9 & 65.8 & 54.6 & 60.7 & 61.3 & 71.9 & 79.5 & 79.5 \\
                \texttt{isl\_Latn} & 28.7 & 28.0 & 27.8 & 50.2 & 49.5 & 45.7 & 40.4 & 38.6 & 38.5 & 53.4 & 55.0 & 56.5 & 50.9 & 54.3 & 56.5 & 71.1 & 77.6 & 76.4 \\
                \texttt{ita\_Latn} & 35.4 & 35.1 & 33.9 & 62.1 & 60.3 & 55.7 & 51.8 & 47.9 & 49.6 & 67.6 & 67.1 & 67.7 & 53.3 & 59.8 & 61.8 & 71.0 & 80.3 & 79.7 \\
                \texttt{jav\_Latn} & 34.9 & 33.7 & 33.8 & 54.1 & 56.0 & 51.1 & 43.2 & 41.1 & 41.4 & 55.6 & 59.2 & 59.8 & 52.4 & 57.7 & 58.2 & 70.2 & 77.6 & 75.7 \\
                \texttt{jpn\_Jpan} & 37.1 & 37.6 & 36.5 & 56.7 & 55.0 & 50.0 & 51.7 & 47.3 & 47.7 & 65.8 & 66.6 & 67.3 & 53.6 & 60.2 & 60.4 & 72.6 & 79.8 & 79.2 \\
                \texttt{kab\_Latn} & 26.0 & 25.9 & 25.4 & 28.1 & 29.1 & 27.4 & 27.4 & 27.1 & 26.8 & 33.5 & 33.3 & 34.6 & 32.8 & 31.5 & 30.6 & 35.0 & 34.0 & 34.2 \\
                \texttt{kac\_Latn} & 28.6 & 26.8 & 27.2 & 33.7 & 34.1 & 31.1 & 30.1 & 29.6 & 29.8 & 43.5 & 43.5 & 43.7 & 35.8 & 34.5 & 35.2 & 41.1 & 43.4 & 41.6 \\
                \texttt{kam\_Latn} & 28.3 & 27.7 & 28.0 & 36.0 & 36.9 & 33.7 & 33.7 & 32.5 & 33.5 & 44.6 & 45.2 & 46.2 & 38.5 & 40.2 & 40.2 & 51.4 & 53.8 & 52.0 \\
                \texttt{kan\_Knda} & 28.8 & 28.6 & 29.1 & 44.5 & 42.9 & 38.4 & 42.8 & 40.5 & 38.6 & 56.4 & 51.5 & 48.6 & 45.9 & 49.3 & 46.4 & 71.6 & 77.0 & 76.0 \\
                \texttt{kas\_Arab} & 33.5 & 33.1 & 32.0 & 41.4 & 41.0 & 37.0 & 35.4 & 35.4 & 34.3 & 51.0 & 50.4 & 50.1 & 47.7 & 48.9 & 50.0 & 65.3 & 67.9 & 63.8 \\
                \texttt{kas\_Deva} & 30.4 & 29.6 & 28.9 & 37.5 & 37.3 & 34.4 & 31.9 & 29.3 & 31.3 & 51.1 & 48.3 & 48.7 & 46.3 & 47.0 & 48.3 & 60.8 & 64.6 & 62.2 \\
                \texttt{kat\_Geor} & 29.7 & 30.3 & 30.1 & 47.5 & 47.0 & 41.0 & 39.8 & 38.9 & 37.5 & 37.4 & 34.6 & 33.3 & 49.3 & 52.6 & 53.2 & 71.8 & 79.5 & 77.7 \\
                \texttt{kaz\_Cyrl} & 33.5 & 34.3 & 32.9 & 49.9 & 48.3 & 43.8 & 45.1 & 41.5 & 42.1 & 56.0 & 56.0 & 57.7 & 53.9 & 57.9 & 58.2 & 70.8 & 79.2 & 78.3 \\
                \texttt{kbp\_Latn} & 28.5 & 27.5 & 27.1 & 34.2 & 34.4 & 32.4 & 31.8 & 30.5 & 31.5 & 42.5 & 41.3 & 41.8 & 34.6 & 34.0 & 34.8 & 37.9 & 39.7 & 38.8 \\
                \texttt{kea\_Latn} & 34.5 & 33.2 & 32.6 & 51.6 & 52.5 & 48.4 & 46.5 & 44.0 & 44.5 & 60.3 & 62.1 & 63.1 & 54.0 & 57.6 & 57.4 & 68.3 & 73.7 & 72.1 \\
                \texttt{khk\_Cyrl} & 31.0 & 31.2 & 30.6 & 46.9 & 45.5 & 41.0 & 37.5 & 35.7 & 36.3 & 45.7 & 44.9 & 46.5 & 42.4 & 43.4 & 42.5 & 70.5 & 76.3 & 75.5 \\
                \texttt{khm\_Khmr} & 27.5 & 28.3 & 27.6 & 44.6 & 44.1 & 40.6 & 40.5 & 39.6 & 37.5 & 47.9 & 44.2 & 41.5 & 48.9 & 49.3 & 48.4 & 69.0 & 72.5 & 70.3 \\
                \texttt{kik\_Latn} & 29.8 & 29.4 & 28.8 & 38.3 & 38.6 & 35.7 & 34.8 & 34.3 & 34.3 & 47.4 & 46.7 & 48.2 & 41.9 & 42.8 & 41.7 & 52.4 & 53.3 & 51.6 \\
                \texttt{kin\_Latn} & 28.2 & 27.4 & 28.2 & 37.6 & 37.1 & 35.3 & 31.9 & 31.8 & 32.0 & 44.6 & 44.5 & 44.7 & 36.3 & 36.0 & 36.4 & 70.7 & 75.3 & 73.9 \\
                \texttt{kir\_Cyrl} & 32.9 & 33.2 & 31.7 & 47.6 & 46.6 & 41.1 & 41.4 & 40.1 & 40.2 & 51.9 & 51.4 & 52.1 & 51.4 & 55.2 & 55.4 & 71.4 & 77.8 & 76.8 \\
                \texttt{kmb\_Latn} & 28.1 & 28.7 & 27.9 & 35.1 & 35.1 & 32.0 & 33.0 & 31.7 & 31.5 & 42.2 & 44.3 & 44.7 & 36.7 & 36.1 & 37.4 & 42.5 & 44.9 & 42.5 \\
                \texttt{kmr\_Latn} & 31.0 & 30.4 & 29.2 & 42.8 & 42.9 & 38.1 & 38.9 & 39.0 & 38.0 & 51.1 & 51.1 & 52.4 & 43.0 & 44.1 & 44.5 & 65.3 & 68.1 & 66.9 \\
                \texttt{knc\_Arab} & 25.9 & 24.9 & 25.4 & 28.0 & 29.1 & 27.8 & 25.6 & 26.4 & 25.6 & 31.4 & 29.7 & 30.4 & 31.6 & 28.5 & 29.0 & 32.4 & 31.8 & 30.4 \\
                \texttt{knc\_Latn} & 29.3 & 28.5 & 28.8 & 37.5 & 37.8 & 34.2 & 34.0 & 33.0 & 33.0 & 44.4 & 45.0 & 46.7 & 37.1 & 38.5 & 37.7 & 46.0 & 47.6 & 46.4 \\
                \texttt{kon\_Latn} & 30.4 & 30.2 & 29.5 & 41.8 & 43.1 & 39.3 & 36.4 & 35.9 & 36.5 & 49.2 & 50.6 & 50.9 & 43.0 & 43.4 & 44.3 & 57.1 & 59.2 & 57.0 \\
                \texttt{kor\_Hang} & 36.6 & 35.4 & 33.6 & 56.6 & 54.5 & 48.6 & 52.5 & 47.3 & 48.0 & 66.9 & 65.9 & 68.2 & 54.3 & 59.5 & 60.5 & 70.8 & 78.8 & 77.7 \\
                \bottomrule
            \end{tabular}
        \end{adjustbox}
    \end{minipage}
    \hfill
    \begin{minipage}{0.48\textwidth}
        \centering
        \renewcommand{\arraystretch}{0.8} % Reduce row height 
        \rowcolors{2}{gray!15}{white} % Alternate row coloring: gray on every second row
        \begin{adjustbox}{max height=0.6\textheight, max width=\columnwidth}
            \begin{tabular}{l *{18}{c}}
                \toprule
                Lang.  & 
                \multicolumn{3}{c}{QwenVL-2 2B} & 
                \multicolumn{3}{c}{QwenVL-2 7B} & 
                \multicolumn{3}{c}{InternVL-2.5 4B} & 
                \multicolumn{3}{c}{InternVL-2.5 8B} & 
                \multicolumn{3}{c}{Centurio-Qwen} &
                \multicolumn{3}{c}{4o-mini} \\
                \cmidrule(lr){2-4} \cmidrule(lr){5-7} \cmidrule(lr){8-10} \cmidrule(lr){11-13} \cmidrule(lr){14-16} \cmidrule(lr){17-19}
                \textit{k} & 
                1 & 3 & 5 & 
                1 & 3 & 5 & 
                1 & 3 & 5 & 
                1 & 3 & 5 & 
                1 & 3 & 5 & 
                1 & 3 & 5 \\
                \midrule
                \texttt{lao\_Laoo} & 26.6 & 27.0 & 26.0 & 36.8 & 36.4 & 34.0 & 39.6 & 38.0 & 37.5 & 46.5 & 43.7 & 41.1 & 46.8 & 46.2 & 47.0 & 53.4 & 54.8 & 50.6 \\
                \texttt{lij\_Latn} & 33.2 & 31.8 & 31.5 & 49.8 & 50.1 & 46.7 & 43.1 & 41.3 & 42.7 & 56.4 & 58.6 & 59.1 & 51.1 & 55.8 & 55.2 & 66.3 & 71.1 & 69.7 \\
                \texttt{lim\_Latn} & 33.1 & 31.9 & 31.7 & 50.3 & 50.7 & 47.0 & 42.0 & 38.8 & 39.7 & 56.6 & 58.0 & 60.1 & 51.5 & 55.9 & 57.9 & 66.8 & 71.9 & 69.6 \\
                \texttt{lin\_Latn} & 30.5 & 30.4 & 30.9 & 40.7 & 42.2 & 39.5 & 38.0 & 36.7 & 36.8 & 49.7 & 50.9 & 52.1 & 48.8 & 52.0 & 52.8 & 65.1 & 68.1 & 66.8 \\
                \texttt{lit\_Latn} & 34.9 & 34.4 & 33.5 & 56.9 & 56.6 & 52.3 & 49.9 & 45.1 & 46.6 & 61.2 & 61.5 & 62.8 & 52.2 & 57.5 & 57.9 & 72.0 & 77.2 & 77.6 \\
                \texttt{lmo\_Latn} & 33.9 & 32.7 & 31.3 & 52.2 & 52.7 & 48.1 & 43.2 & 41.6 & 41.4 & 57.0 & 60.8 & 62.0 & 51.7 & 56.9 & 57.3 & 68.2 & 73.5 & 70.9 \\
                \texttt{ltg\_Latn} & 32.5 & 31.5 & 30.4 & 45.7 & 47.4 & 42.6 & 42.3 & 39.9 & 41.0 & 58.7 & 61.0 & 62.8 & 50.3 & 52.9 & 52.6 & 64.8 & 68.6 & 67.0 \\
                \texttt{ltz\_Latn} & 34.5 & 34.3 & 33.3 & 49.7 & 51.5 & 46.8 & 41.6 & 39.8 & 40.7 & 57.2 & 60.0 & 61.5 & 53.1 & 56.9 & 58.8 & 71.7 & 77.7 & 77.0 \\
                \texttt{lua\_Latn} & 30.2 & 30.3 & 29.5 & 40.2 & 39.9 & 36.7 & 36.0 & 35.3 & 35.7 & 48.3 & 50.5 & 50.3 & 39.8 & 40.8 & 40.8 & 51.9 & 53.1 & 53.0 \\
                \texttt{lug\_Latn} & 28.2 & 27.3 & 27.2 & 36.3 & 36.2 & 33.4 & 32.5 & 31.0 & 32.3 & 43.2 & 43.7 & 44.5 & 37.2 & 37.8 & 36.8 & 59.8 & 63.2 & 60.8 \\
                \texttt{luo\_Latn} & 29.6 & 28.3 & 28.0 & 36.9 & 37.5 & 34.5 & 32.3 & 31.4 & 31.3 & 43.2 & 45.1 & 44.9 & 38.1 & 41.0 & 40.3 & 47.4 & 49.8 & 48.2 \\
                \texttt{lus\_Latn} & 32.7 & 31.3 & 30.8 & 42.5 & 42.8 & 38.5 & 36.3 & 36.2 & 35.5 & 54.0 & 54.1 & 55.6 & 42.5 & 43.7 & 43.5 & 56.9 & 61.1 & 60.2 \\
                \texttt{lvs\_Latn} & 33.5 & 33.5 & 32.9 & 57.5 & 57.0 & 51.9 & 49.6 & 47.2 & 47.4 & 64.6 & 64.7 & 65.0 & 51.8 & 55.3 & 56.1 & 71.8 & 78.5 & 78.0 \\
                \texttt{mag\_Deva} & 34.2 & 34.3 & 32.1 & 46.3 & 45.2 & 40.5 & 41.6 & 39.5 & 38.2 & 62.4 & 58.1 & 57.0 & 53.1 & 57.5 & 58.1 & 70.2 & 76.3 & 75.0 \\
                \texttt{mai\_Deva} & 34.8 & 35.2 & 34.0 & 45.2 & 44.3 & 39.8 & 39.3 & 37.4 & 37.1 & 63.5 & 58.3 & 57.0 & 54.3 & 57.2 & 57.3 & 71.7 & 78.4 & 77.1 \\
                \texttt{mal\_Mlym} & 30.8 & 30.8 & 29.9 & 46.2 & 44.3 & 38.3 & 42.1 & 40.4 & 37.7 & 55.7 & 51.1 & 48.5 & 45.5 & 47.9 & 48.4 & 70.3 & 77.1 & 74.8 \\
                \texttt{mar\_Deva} & 32.0 & 32.2 & 31.3 & 48.2 & 46.4 & 42.3 & 42.7 & 40.6 & 39.9 & 57.9 & 53.5 & 54.4 & 51.5 & 55.2 & 55.5 & 72.0 & 78.4 & 77.5 \\
                \texttt{min\_Arab} & 27.0 & 27.4 & 27.4 & 30.4 & 32.6 & 30.1 & 27.8 & 28.9 & 28.5 & 33.6 & 32.1 & 32.3 & 37.1 & 35.9 & 36.4 & 44.8 & 46.3 & 43.9 \\
                \texttt{min\_Latn} & 34.4 & 33.9 & 33.4 & 50.5 & 51.3 & 46.4 & 41.5 & 39.0 & 39.4 & 55.0 & 58.8 & 59.0 & 52.7 & 56.7 & 57.4 & 66.7 & 71.9 & 70.7 \\
                \texttt{mkd\_Cyrl} & 35.2 & 34.6 & 33.4 & 56.1 & 55.6 & 51.3 & 48.0 & 45.0 & 46.0 & 61.1 & 59.8 & 61.7 & 53.7 & 58.2 & 60.5 & 72.0 & 79.0 & 78.1 \\
                \texttt{mlt\_Latn} & 33.5 & 32.7 & 32.9 & 53.0 & 53.5 & 48.4 & 42.9 & 41.6 & 42.5 & 58.9 & 60.1 & 63.0 & 51.1 & 54.8 & 54.6 & 71.3 & 77.6 & 76.3 \\
                \texttt{mni\_Beng} & 29.7 & 27.7 & 27.7 & 32.2 & 33.4 & 31.3 & 30.2 & 30.2 & 29.4 & 37.8 & 35.1 & 34.2 & 37.6 & 36.7 & 38.4 & 50.3 & 51.4 & 49.9 \\
                \texttt{mos\_Latn} & 29.5 & 28.3 & 28.0 & 34.1 & 34.9 & 31.4 & 31.3 & 30.9 & 31.0 & 41.9 & 43.6 & 44.0 & 36.6 & 36.6 & 37.4 & 42.3 & 43.8 & 42.1 \\
                \texttt{mri\_Latn} & 28.2 & 27.0 & 26.6 & 39.2 & 39.3 & 36.4 & 33.9 & 33.0 & 33.5 & 46.8 & 47.6 & 49.2 & 47.7 & 49.2 & 49.4 & 65.6 & 68.7 & 67.9 \\
                \texttt{mya\_Mymr} & 26.9 & 26.1 & 25.5 & 41.6 & 41.1 & 37.3 & 37.1 & 34.1 & 32.5 & 35.2 & 33.4 & 31.1 & 47.0 & 46.8 & 45.6 & 69.9 & 72.8 & 71.0 \\
                \texttt{nld\_Latn} & 37.5 & 36.9 & 35.4 & 60.6 & 59.9 & 54.8 & 50.3 & 47.8 & 49.1 & 65.5 & 66.3 & 67.2 & 55.2 & 62.0 & 63.5 & 70.4 & 79.0 & 77.7 \\
                \texttt{nno\_Latn} & 36.0 & 34.7 & 33.7 & 58.3 & 58.0 & 52.7 & 50.0 & 46.7 & 46.4 & 62.5 & 65.3 & 65.1 & 55.6 & 60.8 & 62.4 & 70.5 & 78.4 & 77.2 \\
                \texttt{nob\_Latn} & 36.9 & 35.8 & 34.7 & 59.3 & 58.5 & 53.8 & 51.4 & 47.9 & 48.5 & 64.3 & 66.6 & 67.0 & 55.7 & 61.6 & 63.9 & 70.1 & 77.9 & 77.4 \\
                \texttt{npi\_Deva} & 34.6 & 35.5 & 33.8 & 49.3 & 48.2 & 42.8 & 42.9 & 40.6 & 40.9 & 63.2 & 58.4 & 57.8 & 53.7 & 57.2 & 58.3 & 71.0 & 78.1 & 77.5 \\
                \texttt{nqo\_Nkoo} & 24.3 & 24.2 & 24.8 & 25.0 & 25.4 & 24.9 & 25.1 & 25.3 & 24.5 & 26.5 & 25.9 & 25.3 & 26.2 & 24.3 & 25.6 & 24.7 & 24.1 & 24.0 \\
                \texttt{nso\_Latn} & 29.2 & 28.6 & 28.4 & 38.7 & 39.2 & 35.7 & 33.5 & 31.4 & 32.7 & 45.2 & 46.8 & 48.1 & 40.6 & 40.9 & 41.4 & 62.9 & 67.5 & 65.9 \\
                \texttt{nus\_Latn} & 28.0 & 26.7 & 26.4 & 29.8 & 30.9 & 28.8 & 30.3 & 31.0 & 30.3 & 37.2 & 38.3 & 39.2 & 35.1 & 35.5 & 35.9 & 36.3 & 37.5 & 36.1 \\
                \texttt{nya\_Latn} & 30.3 & 28.8 & 28.5 & 40.3 & 40.6 & 37.5 & 34.8 & 34.8 & 35.2 & 47.2 & 48.6 & 49.1 & 40.6 & 41.0 & 41.0 & 68.1 & 72.7 & 71.9 \\
                \texttt{oci\_Latn} & 34.3 & 34.1 & 32.4 & 56.6 & 56.1 & 51.2 & 47.9 & 44.2 & 46.4 & 60.5 & 63.4 & 64.7 & 52.5 & 57.1 & 57.5 & 72.0 & 76.9 & 77.0 \\
                \texttt{ory\_Orya} & 26.7 & 25.9 & 26.3 & 41.5 & 39.5 & 33.6 & 43.8 & 40.8 & 38.1 & 57.5 & 52.2 & 50.3 & 44.5 & 48.9 & 48.5 & 65.4 & 67.2 & 63.9 \\
                \texttt{pag\_Latn} & 35.2 & 34.6 & 33.4 & 50.5 & 53.3 & 48.9 & 45.1 & 42.6 & 43.1 & 58.6 & 62.0 & 62.5 & 51.8 & 54.9 & 56.1 & 65.3 & 71.0 & 70.0 \\
                \texttt{pan\_Guru} & 30.5 & 31.6 & 30.4 & 45.5 & 43.7 & 38.3 & 44.5 & 42.4 & 39.1 & 59.5 & 56.5 & 54.1 & 49.6 & 50.7 & 49.0 & 70.2 & 74.8 & 73.2 \\
                \texttt{pap\_Latn} & 34.4 & 34.1 & 33.1 & 52.2 & 52.5 & 47.6 & 46.5 & 42.9 & 44.7 & 56.8 & 60.5 & 61.2 & 51.0 & 55.1 & 55.9 & 67.2 & 73.9 & 72.0 \\
                \texttt{pbt\_Arab} & 32.7 & 32.7 & 30.1 & 48.8 & 47.3 & 42.6 & 37.9 & 36.9 & 37.5 & 49.7 & 48.5 & 49.2 & 48.4 & 51.3 & 50.4 & 69.9 & 75.2 & 73.7 \\
                \texttt{pes\_Arab} & 35.5 & 35.0 & 33.7 & 55.4 & 52.6 & 47.2 & 48.5 & 45.8 & 48.3 & 63.1 & 63.5 & 64.3 & 55.7 & 59.3 & 61.2 & 72.7 & 79.6 & 79.0 \\
                \texttt{plt\_Latn} & 29.4 & 29.7 & 29.8 & 40.2 & 41.4 & 38.1 & 34.2 & 33.9 & 34.7 & 46.0 & 48.0 & 47.9 & 40.5 & 41.5 & 42.2 & 69.0 & 74.6 & 72.0 \\
                \texttt{pol\_Latn} & 37.1 & 36.4 & 35.0 & 60.1 & 58.7 & 53.1 & 50.7 & 47.0 & 49.0 & 65.9 & 67.5 & 67.6 & 54.3 & 59.6 & 60.9 & 72.3 & 79.2 & 77.8 \\
                \texttt{por\_Latn} & 35.2 & 35.2 & 33.9 & 61.8 & 60.6 & 54.6 & 52.5 & 47.7 & 49.1 & 66.2 & 66.5 & 66.5 & 54.5 & 60.3 & 61.3 & 71.2 & 79.0 & 78.5 \\
                \texttt{prs\_Arab} & 33.8 & 34.3 & 32.5 & 53.9 & 51.4 & 46.1 & 46.8 & 44.3 & 45.2 & 62.5 & 62.1 & 62.3 & 54.4 & 59.1 & 60.1 & 71.6 & 77.8 & 76.5 \\
                \texttt{quy\_Latn} & 30.8 & 30.5 & 30.0 & 38.7 & 38.9 & 35.5 & 34.2 & 33.3 & 34.0 & 47.0 & 48.4 & 49.1 & 43.3 & 44.1 & 44.2 & 50.7 & 54.3 & 53.0 \\
                \texttt{ron\_Latn} & 35.8 & 35.0 & 33.3 & 59.9 & 58.9 & 54.2 & 50.7 & 47.2 & 49.3 & 67.5 & 68.1 & 69.3 & 54.9 & 59.7 & 61.2 & 73.1 & 79.1 & 78.4 \\
                \texttt{run\_Latn} & 28.1 & 27.3 & 27.4 & 36.4 & 37.1 & 34.1 & 31.1 & 31.0 & 31.8 & 43.5 & 43.8 & 43.6 & 37.6 & 36.6 & 36.0 & 66.7 & 72.0 & 69.8 \\
                \texttt{rus\_Cyrl} & 35.5 & 35.8 & 34.1 & 59.2 & 57.1 & 51.9 & 51.1 & 46.3 & 47.3 & 68.6 & 66.0 & 67.5 & 54.1 & 58.4 & 60.4 & 71.3 & 80.9 & 79.3 \\
                \texttt{sag\_Latn} & 29.4 & 28.7 & 27.6 & 37.7 & 37.6 & 34.9 & 32.8 & 32.1 & 32.9 & 44.4 & 46.3 & 45.8 & 43.1 & 44.9 & 43.5 & 45.8 & 47.3 & 44.9 \\
                \texttt{san\_Deva} & 31.2 & 29.6 & 29.2 & 39.7 & 40.1 & 37.4 & 36.9 & 36.0 & 34.7 & 55.6 & 52.9 & 51.8 & 51.8 & 54.2 & 54.7 & 68.0 & 70.1 & 69.5 \\
                \texttt{sat\_Olck} & 24.7 & 24.5 & 24.9 & 24.8 & 25.1 & 24.2 & 24.7 & 24.9 & 24.1 & 26.3 & 25.1 & 24.6 & 26.3 & 25.2 & 25.8 & 24.9 & 26.0 & 24.5 \\
                \texttt{scn\_Latn} & 33.4 & 32.1 & 31.5 & 51.5 & 52.9 & 47.5 & 43.2 & 42.1 & 42.9 & 56.2 & 59.1 & 60.0 & 51.8 & 55.5 & 57.0 & 68.9 & 75.1 & 73.4 \\
                \texttt{shn\_Mymr} & 27.2 & 25.7 & 25.1 & 31.1 & 30.5 & 27.8 & 32.3 & 30.2 & 28.3 & 42.3 & 39.1 & 36.9 & 31.7 & 29.5 & 30.3 & 34.3 & 35.1 & 32.2 \\
                \texttt{sin\_Sinh} & 24.3 & 24.7 & 25.0 & 40.0 & 38.9 & 35.1 & 37.6 & 36.4 & 34.9 & 31.7 & 29.6 & 28.7 & 44.0 & 44.8 & 43.7 & 68.3 & 72.4 & 70.4 \\
                \texttt{slk\_Latn} & 35.3 & 35.0 & 34.0 & 57.4 & 55.9 & 51.5 & 51.3 & 48.3 & 49.6 & 64.9 & 65.0 & 66.7 & 52.0 & 57.3 & 59.1 & 72.4 & 79.3 & 78.4 \\
                \texttt{slv\_Latn} & 34.8 & 34.5 & 32.9 & 57.1 & 57.0 & 52.9 & 51.0 & 47.3 & 48.1 & 63.6 & 64.4 & 65.9 & 53.5 & 57.4 & 58.4 & 71.0 & 78.6 & 77.7 \\
                \texttt{smo\_Latn} & 28.1 & 28.4 & 28.7 & 38.5 & 38.9 & 35.3 & 33.3 & 32.4 & 32.6 & 44.6 & 45.8 & 46.5 & 47.9 & 48.6 & 49.4 & 67.9 & 71.2 & 71.1 \\
                \texttt{sna\_Latn} & 28.3 & 28.3 & 27.8 & 37.5 & 37.5 & 34.5 & 32.5 & 33.1 & 32.5 & 44.6 & 44.8 & 44.7 & 39.2 & 38.5 & 38.6 & 68.5 & 74.9 & 72.8 \\
                \texttt{snd\_Arab} & 31.2 & 30.5 & 29.5 & 44.2 & 44.4 & 38.8 & 41.1 & 38.8 & 39.2 & 50.1 & 48.1 & 48.8 & 52.1 & 52.9 & 52.8 & 71.6 & 77.8 & 76.3 \\
                \texttt{som\_Latn} & 28.3 & 28.2 & 27.7 & 41.1 & 41.1 & 38.6 & 35.7 & 34.2 & 34.2 & 42.0 & 43.6 & 44.2 & 45.5 & 46.2 & 48.1 & 69.0 & 74.2 & 73.1 \\
                \texttt{sot\_Latn} & 28.5 & 27.4 & 27.2 & 38.8 & 38.4 & 35.9 & 34.5 & 33.0 & 33.0 & 46.7 & 47.1 & 47.4 & 41.6 & 41.6 & 42.4 & 65.5 & 70.1 & 68.4 \\
                \texttt{spa\_Latn} & 35.4 & 35.1 & 33.9 & 61.4 & 60.3 & 55.4 & 49.8 & 46.1 & 47.8 & 67.0 & 65.8 & 66.3 & 54.9 & 59.4 & 61.9 & 70.3 & 79.3 & 78.3 \\
                \texttt{srd\_Latn} & 33.8 & 32.1 & 30.9 & 50.5 & 50.4 & 46.4 & 41.0 & 39.4 & 39.7 & 58.1 & 60.4 & 62.3 & 52.4 & 55.3 & 56.2 & 66.9 & 72.0 & 70.1 \\
                \texttt{srp\_Cyrl} & 35.9 & 34.7 & 32.8 & 54.5 & 53.7 & 49.4 & 48.7 & 46.7 & 47.2 & 60.6 & 61.2 & 63.3 & 54.7 & 57.4 & 60.1 & 73.4 & 80.1 & 78.4 \\
                \texttt{ssw\_Latn} & 28.6 & 27.9 & 27.8 & 38.1 & 38.3 & 35.0 & 33.8 & 33.3 & 33.3 & 45.0 & 44.3 & 44.5 & 47.4 & 48.3 & 48.7 & 65.5 & 68.8 & 67.7 \\
                \texttt{sun\_Latn} & 35.3 & 34.6 & 34.6 & 54.0 & 54.0 & 50.0 & 45.0 & 43.6 & 44.0 & 57.3 & 59.9 & 60.5 & 53.4 & 58.2 & 58.9 & 69.6 & 77.6 & 75.5 \\
                \texttt{swe\_Latn} & 36.4 & 35.5 & 35.0 & 59.2 & 59.5 & 55.3 & 53.1 & 48.3 & 49.5 & 66.0 & 67.5 & 68.0 & 54.9 & 61.5 & 62.4 & 72.8 & 79.3 & 78.8 \\
                \texttt{swh\_Latn} & 30.1 & 28.9 & 29.0 & 50.8 & 52.1 & 47.9 & 40.0 & 39.6 & 39.5 & 54.0 & 55.6 & 56.2 & 51.1 & 52.8 & 52.1 & 71.5 & 77.1 & 76.5 \\
                \texttt{szl\_Latn} & 32.9 & 31.9 & 31.8 & 48.5 & 48.7 & 44.9 & 40.0 & 37.9 & 40.2 & 57.1 & 59.4 & 60.2 & 49.2 & 53.4 & 55.1 & 67.5 & 71.5 & 69.2 \\
                \texttt{tam\_Taml} & 27.0 & 26.7 & 26.5 & 44.9 & 43.6 & 38.5 & 41.8 & 39.2 & 36.9 & 54.7 & 50.8 & 49.1 & 49.1 & 51.2 & 49.8 & 70.8 & 76.9 & 74.6 \\
                \texttt{taq\_Latn} & 29.3 & 29.0 & 28.7 & 37.0 & 38.0 & 34.7 & 32.5 & 33.2 & 33.3 & 44.5 & 46.4 & 46.0 & 37.9 & 38.9 & 39.6 & 45.1 & 47.4 & 45.0 \\
                \texttt{taq\_Tfng} & 24.2 & 23.6 & 24.4 & 26.4 & 26.8 & 25.9 & 26.5 & 26.2 & 25.6 & 27.6 & 25.5 & 25.8 & 27.6 & 25.1 & 26.0 & 26.6 & 25.5 & 26.5 \\
                \texttt{tat\_Cyrl} & 32.1 & 33.5 & 32.2 & 48.2 & 47.5 & 42.6 & 42.5 & 40.0 & 41.6 & 53.6 & 55.0 & 55.9 & 49.2 & 52.0 & 53.2 & 70.7 & 77.8 & 77.4 \\
                \texttt{tel\_Telu} & 29.4 & 31.1 & 29.2 & 45.9 & 44.9 & 38.7 & 43.2 & 40.1 & 39.0 & 59.0 & 54.8 & 53.2 & 47.0 & 49.2 & 48.6 & 71.2 & 77.4 & 75.4 \\
                \texttt{tgk\_Cyrl} & 31.8 & 30.8 & 29.4 & 46.6 & 45.7 & 41.5 & 38.7 & 37.9 & 39.7 & 47.2 & 47.6 & 48.7 & 47.6 & 49.7 & 50.3 & 71.9 & 77.8 & 76.5 \\
                \texttt{tgl\_Latn} & 34.0 & 33.9 & 33.0 & 56.5 & 56.6 & 53.1 & 49.0 & 45.0 & 46.3 & 65.0 & 65.2 & 65.9 & 53.7 & 57.8 & 59.4 & 72.6 & 79.2 & 77.5 \\
                \texttt{tha\_Thai} & 35.6 & 36.6 & 35.0 & 55.7 & 53.4 & 48.2 & 48.4 & 46.5 & 48.3 & 60.9 & 59.0 & 60.0 & 55.2 & 59.6 & 61.0 & 74.6 & 78.8 & 78.3 \\
                \texttt{tir\_Ethi} & 24.6 & 24.4 & 25.1 & 27.9 & 28.5 & 27.6 & 30.6 & 30.0 & 29.7 & 27.3 & 26.7 & 26.3 & 42.0 & 43.5 & 43.2 & 46.4 & 46.0 & 42.1 \\
                \texttt{tpi\_Latn} & 33.8 & 32.3 & 31.8 & 52.2 & 53.0 & 48.5 & 43.4 & 41.9 & 43.3 & 60.1 & 62.5 & 63.0 & 53.1 & 56.1 & 56.5 & 67.8 & 74.0 & 70.5 \\
                \texttt{tsn\_Latn} & 28.7 & 28.3 & 27.8 & 37.2 & 37.1 & 34.2 & 32.9 & 31.5 & 32.3 & 44.3 & 43.4 & 45.6 & 42.7 & 43.8 & 42.8 & 63.5 & 66.2 & 63.8 \\
                \texttt{tso\_Latn} & 28.0 & 28.2 & 27.6 & 37.1 & 37.1 & 34.0 & 32.2 & 31.8 & 31.8 & 44.2 & 44.8 & 45.1 & 46.4 & 47.4 & 47.6 & 61.0 & 64.8 & 62.9 \\
                \texttt{tuk\_Latn} & 32.0 & 31.9 & 31.2 & 47.9 & 47.8 & 43.6 & 40.3 & 38.6 & 40.2 & 55.1 & 56.3 & 57.8 & 46.4 & 50.7 & 51.6 & 70.6 & 76.3 & 75.0 \\
                \texttt{tum\_Latn} & 28.4 & 28.5 & 27.7 & 38.0 & 37.8 & 34.7 & 34.5 & 34.0 & 34.5 & 45.5 & 47.3 & 47.7 & 39.8 & 41.4 & 41.9 & 63.2 & 66.0 & 63.8 \\
                \texttt{tur\_Latn} & 36.5 & 36.1 & 35.0 & 59.2 & 58.5 & 53.5 & 52.8 & 47.5 & 48.7 & 66.2 & 67.3 & 68.8 & 55.9 & 62.2 & 63.3 & 72.5 & 80.3 & 79.2 \\
                \texttt{twi\_Latn} & 29.1 & 28.1 & 27.9 & 38.5 & 38.5 & 34.4 & 34.8 & 33.9 & 34.0 & 48.5 & 49.5 & 48.6 & 43.4 & 43.7 & 43.9 & 60.4 & 63.9 & 61.5 \\
                \texttt{tzm\_Tfng} & 24.4 & 24.3 & 24.1 & 25.9 & 26.7 & 25.0 & 25.8 & 26.0 & 25.3 & 25.9 & 25.9 & 24.8 & 27.9 & 25.7 & 26.0 & 25.9 & 26.6 & 25.4 \\
                \texttt{uig\_Arab} & 28.2 & 28.0 & 26.9 & 42.2 & 41.6 & 37.0 & 34.8 & 34.4 & 33.8 & 55.3 & 53.5 & 53.4 & 46.0 & 45.4 & 46.1 & 68.9 & 73.0 & 70.6 \\
                \texttt{ukr\_Cyrl} & 34.2 & 34.8 & 33.5 & 55.6 & 53.6 & 49.0 & 49.0 & 45.8 & 46.7 & 65.3 & 64.9 & 66.4 & 53.1 & 57.9 & 58.5 & 72.5 & 79.9 & 79.1 \\
                \texttt{umb\_Latn} & 29.1 & 28.9 & 28.0 & 34.8 & 34.8 & 31.7 & 31.1 & 30.5 & 31.6 & 42.5 & 42.2 & 42.7 & 35.4 & 35.8 & 35.7 & 47.5 & 46.7 & 45.3 \\
                \texttt{urd\_Arab} & 33.6 & 33.8 & 33.3 & 50.1 & 48.2 & 44.0 & 44.4 & 42.3 & 42.9 & 59.3 & 58.2 & 58.0 & 52.6 & 57.6 & 56.8 & 72.6 & 78.6 & 77.7 \\
                \texttt{uzn\_Latn} & 32.1 & 32.1 & 31.0 & 50.1 & 50.4 & 45.6 & 46.6 & 43.1 & 44.7 & 56.2 & 57.4 & 59.6 & 51.6 & 56.8 & 56.5 & 71.4 & 78.0 & 77.6 \\
                \texttt{vec\_Latn} & 34.5 & 32.7 & 32.3 & 54.0 & 54.3 & 49.9 & 43.8 & 41.1 & 42.3 & 58.4 & 61.4 & 62.1 & 52.4 & 57.4 & 59.6 & 68.8 & 74.8 & 73.1 \\
                \texttt{vie\_Latn} & 35.2 & 35.2 & 33.6 & 60.0 & 59.4 & 54.1 & 49.8 & 47.5 & 48.7 & 65.9 & 63.8 & 66.4 & 56.6 & 60.9 & 61.5 & 73.5 & 79.3 & 79.1 \\
                \texttt{war\_Latn} & 34.2 & 34.5 & 33.6 & 54.3 & 55.0 & 50.5 & 46.3 & 43.8 & 45.1 & 61.3 & 63.3 & 64.1 & 51.7 & 55.1 & 55.4 & 70.6 & 75.8 & 74.8 \\
                \texttt{wol\_Latn} & 30.3 & 29.8 & 28.9 & 39.6 & 40.3 & 36.2 & 35.1 & 33.7 & 35.0 & 48.3 & 49.9 & 50.6 & 44.0 & 45.3 & 45.9 & 53.9 & 57.6 & 55.8 \\
                \texttt{xho\_Latn} & 29.7 & 29.1 & 29.2 & 40.9 & 40.9 & 38.6 & 37.2 & 36.1 & 36.3 & 49.5 & 49.4 & 48.9 & 46.9 & 49.7 & 49.8 & 69.7 & 73.6 & 73.0 \\
                \texttt{ydd\_Hebr} & 27.9 & 27.4 & 27.5 & 37.4 & 39.8 & 36.2 & 30.2 & 30.8 & 31.5 & 32.8 & 32.4 & 31.1 & 41.2 & 41.9 & 42.5 & 67.1 & 72.8 & 71.0 \\
                \texttt{yor\_Latn} & 27.6 & 26.6 & 27.0 & 36.8 & 37.5 & 33.4 & 34.8 & 33.9 & 33.8 & 44.9 & 43.8 & 45.2 & 42.9 & 44.7 & 44.9 & 64.4 & 66.9 & 65.7 \\
                \texttt{yue\_Hant} & 37.0 & 38.0 & 37.1 & 60.1 & 57.5 & 53.9 & 53.4 & 49.7 & 50.4 & 66.7 & 68.2 & 68.5 & 56.1 & 60.8 & 61.8 & 72.4 & 79.8 & 78.6 \\
                \texttt{zho\_Hans} & 37.0 & 36.4 & 36.1 & 61.3 & 58.2 & 54.8 & 55.0 & 51.6 & 51.5 & 66.7 & 68.6 & 68.4 & 55.6 & 61.6 & 64.2 & 72.8 & 79.7 & 79.5 \\
                \texttt{zho\_Hant} & 37.0 & 37.5 & 36.8 & 59.9 & 57.1 & 53.7 & 52.8 & 48.6 & 50.4 & 66.5 & 68.1 & 68.2 & 55.4 & 60.5 & 62.1 & 72.5 & 80.2 & 78.7 \\
                \texttt{zsm\_Latn} & 36.6 & 35.7 & 34.6 & 59.1 & 58.7 & 53.6 & 48.0 & 45.3 & 46.2 & 62.9 & 64.2 & 64.8 & 54.7 & 59.7 & 61.1 & 71.9 & 80.0 & 77.3 \\
                \texttt{zul\_Latn} & 28.1 & 27.9 & 27.4 & 38.0 & 38.1 & 36.5 & 32.8 & 33.3 & 32.4 & 44.9 & 44.2 & 43.8 & 47.0 & 50.0 & 50.9 & 68.6 & 74.4 & 73.4 \\
                \bottomrule
            \end{tabular}
        \end{adjustbox}
    \end{minipage}
\end{table*}

\begin{table*}[ht]
    \subsubsection{Topic-To-Sentence}
    \label{app:topic-to-sentences}
    \begin{minipage}{0.48\textwidth}
        \centering
        \renewcommand{\arraystretch}{0.8} % Reduce row height 
        \rowcolors{2}{gray!15}{white} % Alternate row coloring: gray on every second row
        \begin{adjustbox}{max height=0.6\textheight, max width=\columnwidth}
            \begin{tabular}{l *{6}{c}}
                \toprule
                Lang.  & 
                \multicolumn{1}{c}{QwenVL-2 2B} & 
                \multicolumn{1}{c}{QwenVL-2 7B} & 
                \multicolumn{1}{c}{InternVL-2.5 4B} & 
                \multicolumn{1}{c}{InternVL-2.5 8B} & 
                \multicolumn{1}{c}{Centurio-Qwen} &
                \multicolumn{1}{c}{4o-mini} \\
                \midrule
                % Add more rows as needed
\texttt{Avg}       & 38.6 & 66.5 & 54.5 & 68.0 & 72.4 & 81.5 \\
\texttt{eng\_Latn} & 56.7 & 85.7 & 81.4 & 87.0 & 85.4 & 88.5 \\
\texttt{ace\_Arab} & 29.9 & 46.4 & 34.9 & 42.1 & 54.1 & 66.3 \\
\texttt{ace\_Latn} & 39.6 & 68.7 & 52.0 & 71.8 & 72.9 & 80.6 \\
\texttt{acm\_Arab} & 46.6 & 81.1 & 71.5 & 80.5 & 83.2 & 89.9 \\
\texttt{acq\_Arab} & 47.5 & 81.6 & 71.3 & 80.6 & 84.0 & 90.0 \\
\texttt{aeb\_Arab} & 45.7 & 79.6 & 68.4 & 78.2 & 82.1 & 88.9 \\
\texttt{afr\_Latn} & 49.9 & 82.5 & 70.6 & 82.3 & 83.4 & 89.3 \\
\texttt{ajp\_Arab} & 46.3 & 81.1 & 70.5 & 80.2 & 83.6 & 89.3 \\
\texttt{aka\_Latn} & 32.3 & 53.2 & 44.4 & 59.6 & 63.5 & 79.9 \\
\texttt{als\_Latn} & 41.2 & 77.6 & 57.9 & 76.1 & 80.0 & 89.4 \\
\texttt{amh\_Ethi} & 26.0 & 38.1 & 37.5 & 31.2 & 68.5 & 81.6 \\
\texttt{apc\_Arab} & 46.0 & 80.9 & 70.0 & 80.4 & 82.7 & 89.4 \\
\texttt{arb\_Arab} & 48.1 & 82.7 & 73.1 & 81.3 & 84.9 & 89.9 \\
\texttt{arb\_Latn} & 31.2 & 56.1 & 44.9 & 56.6 & 69.8 & 85.1 \\
\texttt{ars\_Arab} & 48.0 & 82.4 & 72.7 & 81.2 & 84.3 & 90.0 \\
\texttt{ary\_Arab} & 43.3 & 77.0 & 64.1 & 75.9 & 81.4 & 88.2 \\
\texttt{arz\_Arab} & 46.6 & 81.2 & 71.5 & 80.7 & 83.1 & 89.3 \\
\texttt{asm\_Beng} & 37.0 & 69.6 & 57.8 & 72.0 & 79.2 & 88.2 \\
\texttt{ast\_Latn} & 50.1 & 82.1 & 71.6 & 82.2 & 83.5 & 89.4 \\
\texttt{awa\_Deva} & 41.5 & 74.5 & 59.4 & 78.2 & 81.7 & 88.6 \\
\texttt{ayr\_Latn} & 32.2 & 46.3 & 37.0 & 52.3 & 54.3 & 62.0 \\
\texttt{azb\_Arab} & 38.7 & 69.5 & 46.0 & 68.0 & 77.8 & 83.2 \\
\texttt{azj\_Latn} & 42.6 & 80.2 & 65.5 & 76.9 & 82.0 & 89.1 \\
\texttt{bak\_Cyrl} & 38.1 & 74.3 & 53.1 & 69.6 & 74.7 & 88.2 \\
\texttt{bam\_Latn} & 30.9 & 46.6 & 39.4 & 53.3 & 58.8 & 58.5 \\
\texttt{ban\_Latn} & 46.2 & 75.3 & 56.5 & 74.8 & 79.0 & 85.4 \\
\texttt{bel\_Cyrl} & 44.2 & 80.4 & 63.7 & 77.7 & 83.9 & 89.4 \\
\texttt{bem\_Latn} & 31.1 & 52.9 & 42.3 & 59.3 & 59.9 & 72.6 \\
\texttt{ben\_Beng} & 38.9 & 77.2 & 64.5 & 76.9 & 81.9 & 88.6 \\
\texttt{bho\_Deva} & 39.1 & 69.8 & 56.4 & 76.8 & 80.3 & 88.0 \\
\texttt{bjn\_Arab} & 28.8 & 44.3 & 33.0 & 41.2 & 52.8 & 64.8 \\
\texttt{bjn\_Latn} & 43.3 & 73.6 & 57.2 & 72.8 & 78.6 & 84.9 \\
\texttt{bod\_Tibt} & 26.3 & 28.4 & 26.9 & 52.5 & 48.9 & 49.9 \\
\texttt{bos\_Latn} & 50.8 & 83.6 & 73.5 & 83.6 & 82.9 & 88.8 \\
\texttt{bug\_Latn} & 39.9 & 64.8 & 50.4 & 68.8 & 69.3 & 77.2 \\
\texttt{bul\_Cyrl} & 47.1 & 83.3 & 73.4 & 83.5 & 83.9 & 89.3 \\
\texttt{cat\_Latn} & 51.9 & 83.8 & 75.2 & 84.0 & 84.2 & 89.9 \\
\texttt{ceb\_Latn} & 41.8 & 79.7 & 63.6 & 79.9 & 82.1 & 88.6 \\
\texttt{ces\_Latn} & 50.8 & 82.8 & 75.3 & 85.3 & 84.6 & 90.1 \\
\texttt{cjk\_Latn} & 31.0 & 50.6 & 40.6 & 55.2 & 57.0 & 60.8 \\
\texttt{ckb\_Arab} & 30.3 & 61.3 & 43.1 & 49.2 & 58.2 & 82.3 \\
\texttt{crh\_Latn} & 41.1 & 74.8 & 59.4 & 75.7 & 80.0 & 86.4 \\
\texttt{cym\_Latn} & 33.6 & 72.8 & 49.1 & 72.8 & 75.9 & 88.5 \\
\texttt{dan\_Latn} & 52.6 & 83.6 & 74.8 & 84.5 & 83.3 & 88.5 \\
\texttt{deu\_Latn} & 52.9 & 85.2 & 77.4 & 85.5 & 84.7 & 89.7 \\
\texttt{dik\_Latn} & 32.0 & 48.6 & 41.0 & 59.3 & 55.2 & 61.9 \\
\texttt{dyu\_Latn} & 33.4 & 51.3 & 40.4 & 58.6 & 59.0 & 63.0 \\
\texttt{dzo\_Tibt} & 24.9 & 24.8 & 24.2 & 46.9 & 46.8 & 45.1 \\
\texttt{ell\_Grek} & 40.4 & 77.7 & 60.3 & 79.2 & 81.6 & 90.1 \\
\texttt{epo\_Latn} & 46.2 & 83.1 & 69.3 & 81.1 & 83.2 & 89.2 \\
\texttt{est\_Latn} & 42.3 & 79.0 & 66.3 & 77.5 & 80.2 & 89.3 \\
\texttt{eus\_Latn} & 40.2 & 71.2 & 55.6 & 74.5 & 77.5 & 88.4 \\
\texttt{ewe\_Latn} & 29.3 & 45.1 & 37.2 & 53.5 & 50.5 & 57.2 \\
\texttt{fao\_Latn} & 36.0 & 74.0 & 54.3 & 73.5 & 76.0 & 87.4 \\
\texttt{fij\_Latn} & 31.6 & 51.8 & 38.8 & 57.5 & 56.7 & 77.6 \\
\texttt{fin\_Latn} & 42.3 & 82.0 & 70.5 & 82.3 & 81.8 & 89.2 \\
\texttt{fon\_Latn} & 29.6 & 42.6 & 36.5 & 50.9 & 49.2 & 55.2 \\
\texttt{fra\_Latn} & 53.3 & 85.2 & 77.7 & 86.4 & 85.9 & 89.9 \\
\texttt{fur\_Latn} & 43.4 & 76.4 & 61.4 & 77.5 & 79.9 & 86.6 \\
\texttt{fuv\_Latn} & 33.0 & 52.0 & 42.6 & 56.7 & 56.9 & 68.5 \\
\texttt{gaz\_Latn} & 30.3 & 45.8 & 38.9 & 48.4 & 53.1 & 81.7 \\
\texttt{gla\_Latn} & 32.3 & 59.9 & 44.7 & 59.3 & 71.6 & 85.8 \\
\texttt{gle\_Latn} & 32.1 & 66.5 & 47.6 & 62.2 & 73.8 & 87.7 \\
\texttt{glg\_Latn} & 51.2 & 83.7 & 75.8 & 85.0 & 85.2 & 88.8 \\
\texttt{grn\_Latn} & 40.2 & 64.1 & 51.7 & 69.6 & 75.4 & 81.3 \\
\texttt{guj\_Gujr} & 34.0 & 72.8 & 56.5 & 74.3 & 79.0 & 88.7 \\
\texttt{hat\_Latn} & 40.2 & 70.8 & 57.9 & 74.1 & 82.0 & 87.3 \\
\texttt{hau\_Latn} & 32.6 & 52.1 & 40.0 & 53.8 & 66.5 & 87.3 \\
\texttt{heb\_Hebr} & 46.0 & 82.7 & 69.1 & 76.9 & 82.7 & 88.8 \\
\texttt{hin\_Deva} & 39.9 & 78.9 & 65.2 & 80.3 & 84.1 & 89.1 \\
\texttt{hne\_Deva} & 41.1 & 70.6 & 55.6 & 76.5 & 80.3 & 88.3 \\
\texttt{hrv\_Latn} & 51.5 & 84.0 & 73.2 & 84.0 & 83.1 & 89.7 \\
\texttt{hun\_Latn} & 44.3 & 80.9 & 66.8 & 81.3 & 81.2 & 89.9 \\
\texttt{hye\_Armn} & 28.0 & 69.7 & 40.5 & 38.4 & 64.2 & 88.4 \\
\texttt{ibo\_Latn} & 29.6 & 49.4 & 41.1 & 57.0 & 71.1 & 85.8 \\
\texttt{ilo\_Latn} & 39.8 & 73.7 & 56.6 & 76.0 & 76.3 & 87.8 \\
\texttt{ind\_Latn} & 53.4 & 84.2 & 75.0 & 84.7 & 84.4 & 89.3 \\
\texttt{isl\_Latn} & 34.1 & 74.1 & 55.2 & 70.6 & 77.2 & 88.7 \\
\texttt{ita\_Latn} & 53.6 & 84.5 & 75.5 & 86.6 & 86.0 & 89.3 \\
\texttt{jav\_Latn} & 46.0 & 77.2 & 59.9 & 74.4 & 80.9 & 87.5 \\
\texttt{jpn\_Jpan} & 52.0 & 81.9 & 76.0 & 86.8 & 84.7 & 90.0 \\
\texttt{kab\_Latn} & 27.4 & 35.7 & 29.7 & 40.3 & 41.3 & 46.3 \\
\texttt{kac\_Latn} & 30.8 & 47.0 & 37.7 & 56.3 & 52.6 & 57.3 \\
\texttt{kam\_Latn} & 31.6 & 51.6 & 40.4 & 56.2 & 57.9 & 69.7 \\
\texttt{kan\_Knda} & 30.3 & 69.3 & 52.7 & 68.1 & 77.5 & 87.9 \\
\texttt{kas\_Arab} & 36.1 & 62.5 & 46.3 & 67.9 & 71.9 & 84.9 \\
\texttt{kas\_Deva} & 34.6 & 55.2 & 43.5 & 63.3 & 67.4 & 80.8 \\
\texttt{kat\_Geor} & 32.8 & 69.0 & 51.9 & 44.9 & 76.8 & 88.9 \\
\texttt{kaz\_Cyrl} & 39.2 & 76.3 & 58.9 & 71.7 & 81.1 & 89.0 \\
\texttt{kbp\_Latn} & 30.4 & 45.3 & 38.1 & 54.2 & 51.4 & 57.1 \\
\texttt{kea\_Latn} & 45.9 & 76.3 & 63.8 & 78.2 & 81.1 & 86.2 \\
\texttt{khk\_Cyrl} & 33.3 & 69.1 & 44.6 & 57.6 & 61.4 & 88.5 \\
\texttt{khm\_Khmr} & 29.9 & 63.6 & 52.7 & 57.9 & 73.0 & 87.4 \\
\texttt{kik\_Latn} & 32.4 & 52.5 & 42.6 & 60.0 & 64.9 & 70.8 \\
\texttt{kin\_Latn} & 31.2 & 49.9 & 39.8 & 54.4 & 54.8 & 86.8 \\
\texttt{kir\_Cyrl} & 37.9 & 72.1 & 53.6 & 67.7 & 77.9 & 88.0 \\
\texttt{kmb\_Latn} & 29.3 & 46.6 & 38.6 & 53.3 & 53.0 & 59.2 \\
\texttt{kmr\_Latn} & 34.7 & 60.1 & 50.5 & 65.1 & 65.9 & 83.4 \\
\texttt{knc\_Arab} & 28.0 & 34.2 & 29.0 & 37.3 & 36.9 & 46.1 \\
\texttt{knc\_Latn} & 33.1 & 52.0 & 42.8 & 59.8 & 60.5 & 64.2 \\
\texttt{kon\_Latn} & 33.6 & 59.5 & 46.0 & 63.5 & 65.4 & 73.9 \\
\texttt{kor\_Hang} & 48.6 & 82.6 & 76.0 & 86.0 & 83.3 & 88.9 \\

                \bottomrule
            \end{tabular}
        \end{adjustbox}
    \end{minipage}
    \hfill
    \begin{minipage}{0.48\textwidth}
        \centering
        \renewcommand{\arraystretch}{0.8} % Reduce row height 
        \rowcolors{2}{gray!15}{white} % Alternate row coloring: gray on every second row
        \begin{adjustbox}{max height=0.6\textheight, max width=\columnwidth}
            \begin{tabular}{l *{6}{c}}
                \toprule
                Lang.  & 
                \multicolumn{1}{c}{QwenVL-2 2B} & 
                \multicolumn{1}{c}{QwenVL-2 7B} & 
                \multicolumn{1}{c}{InternVL-2.5 4B} & 
                \multicolumn{1}{c}{InternVL-2.5 8B} & 
                \multicolumn{1}{c}{Centurio-Qwen} &
                \multicolumn{1}{c}{4o-mini} \\
                \midrule
\texttt{lao\_Laoo} & 28.2 & 53.1 & 47.6 & 57.8 & 74.6 & 80.1 \\
\texttt{lij\_Latn} & 42.2 & 75.8 & 60.3 & 76.7 & 79.2 & 86.4 \\
\texttt{lim\_Latn} & 42.5 & 77.8 & 61.9 & 75.7 & 80.0 & 85.1 \\
\texttt{lin\_Latn} & 34.6 & 59.4 & 46.5 & 62.5 & 75.6 & 81.2 \\
\texttt{lit\_Latn} & 44.6 & 81.3 & 63.7 & 77.3 & 80.3 & 89.6 \\
\texttt{lmo\_Latn} & 44.1 & 77.2 & 61.3 & 77.4 & 79.9 & 85.9 \\
\texttt{ltg\_Latn} & 38.2 & 71.0 & 54.6 & 72.6 & 75.4 & 83.4 \\
\texttt{ltz\_Latn} & 44.8 & 75.6 & 59.7 & 75.7 & 82.8 & 89.0 \\
\texttt{lua\_Latn} & 34.3 & 56.0 & 45.2 & 61.9 & 62.1 & 69.1 \\
\texttt{lug\_Latn} & 30.5 & 48.6 & 39.9 & 53.4 & 53.6 & 76.5 \\
\texttt{luo\_Latn} & 31.7 & 50.0 & 39.9 & 55.3 & 56.7 & 63.1 \\
\texttt{lus\_Latn} & 38.0 & 62.6 & 49.2 & 69.2 & 68.3 & 74.7 \\
\texttt{lvs\_Latn} & 42.6 & 81.4 & 66.1 & 78.0 & 82.2 & 89.5 \\
\texttt{mag\_Deva} & 40.6 & 71.7 & 56.5 & 77.0 & 80.7 & 87.9 \\
\texttt{mai\_Deva} & 42.5 & 70.5 & 53.8 & 77.7 & 80.7 & 89.3 \\
\texttt{mal\_Mlym} & 32.5 & 70.9 & 52.9 & 67.5 & 77.7 & 88.0 \\
\texttt{mar\_Deva} & 37.1 & 72.8 & 56.4 & 73.2 & 79.9 & 88.6 \\
\texttt{min\_Arab} & 28.3 & 42.3 & 32.8 & 41.5 & 49.8 & 62.8 \\
\texttt{min\_Latn} & 43.4 & 73.1 & 56.5 & 73.4 & 77.5 & 84.4 \\
\texttt{mkd\_Cyrl} & 45.3 & 82.8 & 69.3 & 79.5 & 82.3 & 89.2 \\
\texttt{mlt\_Latn} & 41.5 & 76.4 & 57.5 & 74.2 & 82.2 & 89.5 \\
\texttt{mni\_Beng} & 29.3 & 46.1 & 36.2 & 48.3 & 57.1 & 68.2 \\
\texttt{mos\_Latn} & 32.1 & 48.0 & 40.5 & 56.7 & 53.1 & 58.9 \\
\texttt{mri\_Latn} & 30.2 & 53.4 & 42.1 & 60.4 & 69.3 & 83.1 \\
\texttt{mya\_Mymr} & 27.8 & 54.7 & 44.2 & 41.0 & 67.2 & 86.5 \\
\texttt{nld\_Latn} & 54.1 & 84.0 & 76.9 & 85.3 & 85.1 & 89.0 \\
\texttt{nno\_Latn} & 50.5 & 83.3 & 72.8 & 83.7 & 83.5 & 88.4 \\
\texttt{nob\_Latn} & 52.3 & 83.8 & 75.2 & 84.0 & 84.2 & 88.7 \\
\texttt{npi\_Deva} & 42.4 & 73.8 & 59.0 & 76.3 & 80.4 & 88.3 \\
\texttt{nqo\_Nkoo} & 25.6 & 24.8 & 25.3 & 26.4 & 27.0 & 26.6 \\
\texttt{nso\_Latn} & 31.9 & 51.8 & 41.0 & 56.9 & 62.6 & 80.3 \\
\texttt{nus\_Latn} & 28.4 & 38.7 & 34.8 & 46.1 & 45.2 & 51.5 \\
\texttt{nya\_Latn} & 31.8 & 56.0 & 43.8 & 61.3 & 62.4 & 85.2 \\
\texttt{oci\_Latn} & 47.9 & 81.5 & 70.3 & 81.4 & 83.8 & 89.0 \\
\texttt{ory\_Orya} & 27.5 & 62.4 & 54.2 & 71.5 & 75.9 & 86.4 \\
\texttt{pag\_Latn} & 42.8 & 75.3 & 59.5 & 77.5 & 78.6 & 84.3 \\
\texttt{pan\_Guru} & 34.0 & 68.4 & 56.3 & 73.8 & 79.1 & 88.4 \\
\texttt{pap\_Latn} & 43.3 & 75.1 & 62.7 & 76.8 & 78.8 & 85.4 \\
\texttt{pbt\_Arab} & 37.7 & 70.1 & 48.6 & 65.8 & 72.8 & 86.6 \\
\texttt{pes\_Arab} & 44.5 & 82.0 & 67.5 & 82.5 & 82.7 & 90.0 \\
\texttt{plt\_Latn} & 33.6 & 53.8 & 42.0 & 57.6 & 59.9 & 87.0 \\
\texttt{pol\_Latn} & 51.1 & 83.4 & 73.9 & 85.1 & 84.7 & 89.3 \\
\texttt{por\_Latn} & 53.9 & 85.5 & 77.5 & 86.4 & 85.4 & 89.1 \\
\texttt{prs\_Arab} & 42.7 & 80.0 & 65.7 & 80.2 & 82.0 & 88.1 \\
\texttt{quy\_Latn} & 34.9 & 57.4 & 42.9 & 61.8 & 67.4 & 68.8 \\
\texttt{ron\_Latn} & 50.0 & 83.9 & 74.3 & 85.3 & 84.2 & 89.5 \\
\texttt{run\_Latn} & 30.3 & 48.3 & 37.7 & 53.5 & 54.2 & 84.1 \\
\texttt{rus\_Cyrl} & 50.0 & 84.4 & 76.3 & 85.4 & 85.3 & 89.2 \\
\texttt{sag\_Latn} & 32.2 & 54.0 & 41.5 & 59.2 & 65.9 & 63.1 \\
\texttt{san\_Deva} & 36.1 & 59.2 & 50.1 & 69.8 & 78.6 & 84.8 \\
\texttt{sat\_Olck} & 24.9 & 24.6 & 24.6 & 26.1 & 26.3 & 27.6 \\
\texttt{scn\_Latn} & 43.5 & 76.4 & 59.6 & 76.2 & 80.2 & 87.7 \\
\texttt{shn\_Mymr} & 27.3 & 41.2 & 36.7 & 54.0 & 44.5 & 52.2 \\
\texttt{sin\_Sinh} & 26.2 & 54.2 & 43.9 & 35.6 & 65.7 & 86.6 \\
\texttt{slk\_Latn} & 48.5 & 83.5 & 73.7 & 83.6 & 83.7 & 89.8 \\
\texttt{slv\_Latn} & 47.6 & 82.2 & 70.9 & 82.0 & 82.0 & 89.0 \\
\texttt{smo\_Latn} & 31.8 & 53.1 & 40.7 & 58.9 & 71.5 & 86.1 \\
\texttt{sna\_Latn} & 30.4 & 51.7 & 41.1 & 56.5 & 57.6 & 85.4 \\
\texttt{snd\_Arab} & 35.4 & 64.7 & 50.3 & 65.8 & 79.9 & 88.7 \\
\texttt{som\_Latn} & 33.6 & 55.2 & 43.3 & 54.9 & 64.7 & 85.0 \\
\texttt{sot\_Latn} & 31.7 & 52.7 & 42.0 & 58.3 & 61.4 & 82.7 \\
\texttt{spa\_Latn} & 52.7 & 84.8 & 76.6 & 85.9 & 85.3 & 89.2 \\
\texttt{srd\_Latn} & 42.3 & 74.3 & 56.9 & 76.9 & 80.9 & 84.7 \\
\texttt{srp\_Cyrl} & 46.2 & 82.9 & 71.5 & 80.9 & 83.8 & 89.3 \\
\texttt{ssw\_Latn} & 30.7 & 48.6 & 39.7 & 53.0 & 69.1 & 82.6 \\
\texttt{sun\_Latn} & 45.2 & 77.0 & 63.7 & 75.8 & 79.9 & 88.1 \\
\texttt{swe\_Latn} & 50.8 & 83.9 & 76.1 & 84.8 & 84.0 & 89.6 \\
\texttt{swh\_Latn} & 33.5 & 68.2 & 48.6 & 68.3 & 75.6 & 88.7 \\
\texttt{szl\_Latn} & 42.0 & 76.1 & 57.1 & 76.6 & 78.7 & 87.4 \\
\texttt{tam\_Taml} & 28.3 & 65.9 & 50.4 & 67.4 & 77.2 & 87.5 \\
\texttt{taq\_Latn} & 31.9 & 52.7 & 41.8 & 57.7 & 58.2 & 66.0 \\
\texttt{taq\_Tfng} & 25.8 & 29.4 & 27.3 & 29.1 & 28.6 & 31.7 \\
\texttt{tat\_Cyrl} & 38.1 & 73.5 & 53.9 & 69.4 & 75.0 & 88.8 \\
\texttt{tel\_Telu} & 31.5 & 68.7 & 54.6 & 72.0 & 78.8 & 87.7 \\
\texttt{tgk\_Cyrl} & 33.2 & 68.3 & 50.6 & 61.8 & 74.4 & 88.0 \\
\texttt{tgl\_Latn} & 43.2 & 80.4 & 69.1 & 83.8 & 82.1 & 89.4 \\
\texttt{tha\_Thai} & 48.0 & 84.0 & 74.2 & 79.4 & 84.8 & 90.3 \\
\texttt{tir\_Ethi} & 25.2 & 32.8 & 33.1 & 30.3 & 62.0 & 69.6 \\
\texttt{tpi\_Latn} & 41.7 & 75.4 & 57.5 & 79.2 & 82.6 & 86.4 \\
\texttt{tsn\_Latn} & 31.3 & 48.1 & 39.6 & 54.5 & 63.9 & 79.7 \\
\texttt{tso\_Latn} & 30.4 & 49.4 & 38.3 & 55.2 & 66.5 & 77.4 \\
\texttt{tuk\_Latn} & 38.0 & 70.3 & 52.3 & 71.1 & 70.9 & 87.6 \\
\texttt{tum\_Latn} & 30.5 & 51.7 & 40.5 & 57.6 & 59.0 & 81.8 \\
\texttt{tur\_Latn} & 49.4 & 82.1 & 74.3 & 84.1 & 84.1 & 89.4 \\
\texttt{twi\_Latn} & 32.6 & 54.2 & 44.5 & 62.0 & 63.9 & 78.7 \\
\texttt{tzm\_Tfng} & 25.2 & 27.3 & 25.7 & 27.8 & 29.5 & 29.3 \\
\texttt{uig\_Arab} & 29.2 & 61.3 & 42.7 & 69.2 & 67.8 & 87.1 \\
\texttt{ukr\_Cyrl} & 46.6 & 83.7 & 71.3 & 83.3 & 84.8 & 89.7 \\
\texttt{umb\_Latn} & 30.3 & 47.1 & 39.3 & 52.4 & 53.4 & 62.9 \\
\texttt{urd\_Arab} & 39.7 & 76.7 & 58.6 & 76.5 & 80.5 & 88.9 \\
\texttt{uzn\_Latn} & 37.9 & 72.2 & 58.2 & 69.6 & 80.7 & 88.7 \\
\texttt{vec\_Latn} & 45.7 & 79.9 & 65.3 & 80.0 & 81.6 & 87.7 \\
\texttt{vie\_Latn} & 53.9 & 84.8 & 74.7 & 84.0 & 85.7 & 89.7 \\
\texttt{war\_Latn} & 42.8 & 78.4 & 61.5 & 79.2 & 82.3 & 88.4 \\
\texttt{wol\_Latn} & 34.8 & 57.8 & 45.4 & 63.0 & 67.5 & 70.2 \\
\texttt{xho\_Latn} & 32.1 & 55.1 & 44.6 & 61.7 & 73.0 & 86.1 \\
\texttt{ydd\_Hebr} & 28.7 & 56.6 & 37.2 & 36.2 & 64.9 & 87.5 \\
\texttt{yor\_Latn} & 29.0 & 48.0 & 41.4 & 54.7 & 60.1 & 80.7 \\
\texttt{yue\_Hant} & 53.8 & 84.2 & 77.4 & 85.1 & 85.9 & 89.5 \\
\texttt{zho\_Hans} & 51.6 & 84.0 & 78.5 & 85.8 & 84.6 & 88.9 \\
\texttt{zho\_Hant} & 53.8 & 84.2 & 77.6 & 86.2 & 85.7 & 89.8 \\
\texttt{zsm\_Latn} & 51.5 & 82.5 & 72.7 & 82.8 & 83.4 & 89.3 \\
\texttt{zul\_Latn} & 29.4 & 50.0 & 39.9 & 55.5 & 70.7 & 85.8 \\

                \bottomrule
            \end{tabular}
        \end{adjustbox}
    \end{minipage}
\end{table*}

\begin{table*}[ht]
    \subsubsection{Sentences-To-Images}
    \label{app:sentences-to-images}
    \begin{minipage}{0.48\textwidth}
        \centering
        \renewcommand{\arraystretch}{0.8} % Reduce row height 
        \rowcolors{2}{gray!15}{white} % Alternate row coloring: gray on every second row
        \begin{adjustbox}{max height=0.6\textheight, max width=\columnwidth}
            \begin{tabular}{l *{18}{c}}
                \toprule
                Lang.  & 
                \multicolumn{3}{c}{QwenVL-2 2B} & 
                \multicolumn{3}{c}{QwenVL-2 7B} & 
                \multicolumn{3}{c}{InternVL-2.5 4B} & 
                \multicolumn{3}{c}{InternVL-2.5 8B} & 
                \multicolumn{3}{c}{Centurio-Qwen} &
                \multicolumn{3}{c}{4o-mini} \\
                \cmidrule(lr){2-4} \cmidrule(lr){5-7} \cmidrule(lr){8-10} \cmidrule(lr){11-13} \cmidrule(lr){14-16} \cmidrule(lr){17-19}
                \textit{k} & 
                1 & 3 & 5 & 
                1 & 3 & 5 & 
                1 & 3 & 5 & 
                1 & 3 & 5 & 
                1 & 3 & 5 & 
                1 & 3 & 5 \\
                \midrule
                % Add more rows as needed
                \texttt{Avg}       & 35.1 & 36.8 & 36.8 & 52.8 & 54.5 & 54.4 & 32.2 & 35.4 & 36.6 & 45.8 & 51.8 & 51.7 & 29.2 & 30.3 & 30.5 & 69.2 & 80.2 & 83.0 \\
                \texttt{eng\_Latn} & 41.9 & 43.1 & 43.4 & 71.7 & 70.4 & 68.6 & 47.7 & 44.5 & 43.0 & 66.2 & 69.0 & 68.7 & 35.3 & 36.1 & 35.6 & 77.5 & 86.4 & 89.1 \\
                \texttt{ace\_Arab} & 29.2 & 30.4 & 30.6 & 36.7 & 38.8 & 41.8 & 26.3 & 28.2 & 30.9 & 29.7 & 31.8 & 30.7 & 26.8 & 26.5 & 26.4 & 53.7 & 70.0 & 73.6 \\
                \texttt{ace\_Latn} & 36.5 & 38.0 & 38.0 & 52.5 & 55.8 & 56.9 & 31.2 & 36.9 & 38.2 & 45.5 & 54.1 & 54.6 & 28.5 & 29.4 & 29.4 & 66.5 & 80.9 & 84.9 \\
                \texttt{acm\_Arab} & 44.1 & 45.3 & 43.9 & 63.8 & 66.6 & 65.2 & 37.9 & 39.0 & 39.9 & 54.1 & 59.6 & 59.7 & 30.5 & 33.2 & 33.5 & 77.5 & 86.6 & 88.8 \\
                \texttt{acq\_Arab} & 44.6 & 45.6 & 44.4 & 64.7 & 66.1 & 65.0 & 37.6 & 38.7 & 39.9 & 54.7 & 60.4 & 59.9 & 30.1 & 32.6 & 33.5 & 77.2 & 86.6 & 88.6 \\
                \texttt{aeb\_Arab} & 43.1 & 44.1 & 44.0 & 62.9 & 65.3 & 65.4 & 36.8 & 38.5 & 39.0 & 53.2 & 58.6 & 57.6 & 29.8 & 31.7 & 32.8 & 76.3 & 85.4 & 88.9 \\
                \texttt{afr\_Latn} & 40.3 & 41.6 & 42.7 & 64.4 & 64.2 & 63.2 & 36.2 & 40.3 & 40.2 & 54.3 & 60.8 & 60.6 & 31.0 & 32.4 & 32.0 & 76.6 & 86.3 & 88.6 \\
                \texttt{ajp\_Arab} & 43.2 & 44.6 & 44.4 & 63.5 & 65.0 & 63.8 & 36.0 & 38.7 & 39.5 & 53.3 & 60.3 & 59.8 & 30.2 & 33.5 & 34.2 & 77.8 & 85.7 & 88.0 \\
                \texttt{aka\_Latn} & 30.2 & 31.7 & 32.4 & 43.1 & 45.2 & 46.6 & 28.2 & 32.3 & 33.6 & 40.8 & 48.9 & 48.7 & 28.4 & 27.9 & 28.3 & 66.4 & 79.5 & 82.9 \\
                \texttt{als\_Latn} & 36.8 & 39.7 & 39.6 & 60.1 & 62.3 & 64.5 & 33.4 & 36.8 & 38.8 & 47.9 & 55.8 & 55.9 & 30.4 & 32.3 & 32.1 & 76.8 & 87.4 & 89.3 \\
                \texttt{amh\_Ethi} & 24.8 & 25.0 & 25.5 & 28.7 & 30.4 & 31.8 & 26.0 & 30.1 & 32.4 & 26.1 & 27.8 & 28.4 & 28.6 & 27.7 & 27.7 & 70.1 & 80.6 & 82.9 \\
                \texttt{apc\_Arab} & 43.1 & 44.6 & 45.0 & 64.2 & 64.8 & 63.9 & 36.5 & 37.5 & 38.9 & 53.2 & 59.8 & 59.0 & 30.0 & 33.3 & 33.6 & 76.9 & 86.1 & 89.0 \\
                \texttt{arb\_Arab} & 44.3 & 44.8 & 44.3 & 66.5 & 67.6 & 65.7 & 37.7 & 39.2 & 40.0 & 55.2 & 59.8 & 59.9 & 30.8 & 32.8 & 33.5 & 77.7 & 86.8 & 88.7 \\
                \texttt{arb\_Latn} & 29.6 & 31.5 & 31.0 & 43.3 & 48.0 & 51.6 & 28.5 & 31.5 & 33.2 & 37.5 & 44.4 & 44.4 & 27.5 & 28.2 & 28.1 & 73.4 & 84.8 & 87.4 \\
                \texttt{ars\_Arab} & 43.9 & 44.5 & 43.8 & 66.4 & 66.9 & 65.4 & 37.9 & 39.1 & 40.1 & 54.8 & 59.6 & 59.8 & 30.9 & 32.7 & 33.5 & 77.9 & 86.4 & 88.7 \\
                \texttt{ary\_Arab} & 42.8 & 43.7 & 43.4 & 60.9 & 63.7 & 63.3 & 35.9 & 37.4 & 38.0 & 49.9 & 56.1 & 55.6 & 29.2 & 32.3 & 33.1 & 76.0 & 85.9 & 88.4 \\
                \texttt{arz\_Arab} & 43.3 & 44.9 & 44.0 & 64.0 & 66.1 & 65.0 & 37.5 & 39.2 & 40.2 & 53.9 & 60.8 & 59.8 & 30.0 & 33.3 & 34.3 & 77.6 & 86.6 & 88.8 \\
                \texttt{asm\_Beng} & 35.2 & 36.9 & 36.5 & 59.6 & 63.9 & 59.9 & 36.6 & 41.9 & 43.8 & 45.9 & 51.4 & 52.4 & 32.3 & 33.2 & 33.9 & 75.8 & 85.8 & 88.1 \\
                \texttt{ast\_Latn} & 41.8 & 43.0 & 42.5 & 63.7 & 64.6 & 63.6 & 37.9 & 40.8 & 40.4 & 57.7 & 63.8 & 63.6 & 30.2 & 32.7 & 34.4 & 76.5 & 86.3 & 88.8 \\
                \texttt{awa\_Deva} & 40.5 & 42.4 & 43.3 & 61.1 & 64.1 & 61.7 & 35.1 & 38.4 & 41.8 & 54.3 & 62.3 & 62.3 & 31.6 & 33.5 & 32.8 & 75.4 & 86.3 & 88.2 \\
                \texttt{ayr\_Latn} & 28.4 & 30.8 & 31.1 & 35.2 & 38.4 & 39.6 & 26.3 & 29.7 & 31.5 & 36.8 & 43.1 & 43.5 & 26.7 & 27.1 & 26.9 & 46.7 & 63.0 & 67.8 \\
                \texttt{azb\_Arab} & 37.6 & 40.0 & 38.7 & 53.4 & 58.1 & 60.2 & 31.1 & 32.6 & 35.7 & 43.8 & 49.3 & 48.9 & 29.9 & 30.7 & 30.9 & 71.0 & 83.8 & 86.6 \\
                \texttt{azj\_Latn} & 39.7 & 42.3 & 42.6 & 64.8 & 64.4 & 64.8 & 35.3 & 39.1 & 41.3 & 49.6 & 55.7 & 54.8 & 31.0 & 32.5 & 32.5 & 77.5 & 86.4 & 88.8 \\
                \texttt{bak\_Cyrl} & 34.6 & 39.2 & 38.6 & 57.4 & 60.1 & 61.2 & 30.6 & 36.4 & 38.2 & 42.9 & 48.8 & 48.2 & 29.2 & 29.5 & 29.9 & 76.0 & 86.1 & 88.7 \\
                \texttt{bam\_Latn} & 27.7 & 30.3 & 29.7 & 34.2 & 37.3 & 37.6 & 25.8 & 29.4 & 31.2 & 34.1 & 40.8 & 40.3 & 26.1 & 26.8 & 27.0 & 46.3 & 62.4 & 63.3 \\
                \texttt{ban\_Latn} & 39.9 & 41.6 & 41.3 & 59.5 & 62.9 & 62.5 & 33.6 & 38.5 & 39.4 & 49.7 & 58.2 & 58.3 & 29.4 & 31.6 & 32.3 & 72.6 & 84.2 & 87.1 \\
                \texttt{bel\_Cyrl} & 40.6 & 42.8 & 42.2 & 63.2 & 63.3 & 63.4 & 33.1 & 38.1 & 40.1 & 50.2 & 54.4 & 54.4 & 31.0 & 32.2 & 31.6 & 77.1 & 85.8 & 88.8 \\
                \texttt{bem\_Latn} & 31.5 & 33.3 & 33.2 & 41.5 & 45.5 & 46.5 & 28.4 & 31.8 & 34.2 & 41.6 & 48.2 & 47.9 & 27.6 & 28.0 & 27.5 & 58.7 & 75.3 & 80.2 \\
                \texttt{ben\_Beng} & 39.8 & 40.4 & 41.1 & 64.4 & 67.0 & 63.2 & 41.2 & 43.4 & 45.9 & 50.9 & 58.1 & 59.4 & 33.0 & 33.9 & 34.8 & 76.4 & 85.7 & 88.4 \\
                \texttt{bho\_Deva} & 39.8 & 41.8 & 42.7 & 58.9 & 61.4 & 61.3 & 34.9 & 38.6 & 41.2 & 53.5 & 60.1 & 60.2 & 30.1 & 32.5 & 32.1 & 76.2 & 86.5 & 88.5 \\
                \texttt{bjn\_Arab} & 28.4 & 29.5 & 28.3 & 34.0 & 35.6 & 37.5 & 25.3 & 26.7 & 29.7 & 29.2 & 30.9 & 30.0 & 26.2 & 26.4 & 27.0 & 51.8 & 68.0 & 71.7 \\
                \texttt{bjn\_Latn} & 38.2 & 41.3 & 41.4 & 58.2 & 62.8 & 63.3 & 32.8 & 36.8 & 36.3 & 49.3 & 54.9 & 55.8 & 28.9 & 31.7 & 32.7 & 71.7 & 84.1 & 87.5 \\
                \texttt{bod\_Tibt} & 23.6 & 24.7 & 24.7 & 23.9 & 24.0 & 24.4 & 22.8 & 23.4 & 24.9 & 33.3 & 35.6 & 36.4 & 25.7 & 25.5 & 25.9 & 39.2 & 41.7 & 41.5 \\
                \texttt{bos\_Latn} & 42.8 & 44.7 & 44.5 & 66.5 & 64.9 & 63.3 & 37.2 & 39.4 & 39.2 & 57.5 & 63.4 & 63.3 & 30.7 & 33.2 & 33.7 & 76.9 & 87.1 & 89.2 \\
                \texttt{bug\_Latn} & 35.8 & 38.1 & 38.5 & 51.4 & 53.1 & 53.8 & 31.1 & 35.4 & 36.7 & 44.5 & 52.5 & 53.4 & 28.2 & 29.2 & 29.8 & 63.6 & 79.2 & 82.8 \\
                \texttt{bul\_Cyrl} & 43.2 & 44.3 & 45.8 & 66.0 & 66.7 & 64.9 & 37.8 & 40.2 & 42.2 & 56.2 & 62.4 & 61.2 & 32.3 & 33.8 & 34.1 & 77.1 & 86.8 & 88.4 \\
                \texttt{cat\_Latn} & 42.3 & 43.9 & 43.3 & 65.7 & 64.4 & 62.1 & 38.7 & 40.9 & 40.3 & 57.9 & 63.2 & 62.6 & 31.6 & 34.1 & 33.6 & 77.8 & 87.0 & 89.3 \\
                \texttt{ceb\_Latn} & 38.6 & 42.7 & 42.0 & 62.4 & 63.5 & 65.5 & 35.1 & 39.5 & 39.1 & 54.3 & 59.6 & 59.9 & 31.0 & 33.7 & 34.1 & 77.0 & 86.3 & 87.9 \\
                \texttt{ces\_Latn} & 42.7 & 43.2 & 42.7 & 65.5 & 64.8 & 63.0 & 37.3 & 39.7 & 41.1 & 60.4 & 65.4 & 64.6 & 31.1 & 32.8 & 34.2 & 77.9 & 86.6 & 88.6 \\
                \texttt{cjk\_Latn} & 30.4 & 33.4 & 32.9 & 38.6 & 40.9 & 42.4 & 27.8 & 30.1 & 31.5 & 36.4 & 43.5 & 43.2 & 27.4 & 27.6 & 27.3 & 48.7 & 64.4 & 69.9 \\
                \texttt{ckb\_Arab} & 30.7 & 31.9 & 31.6 & 45.5 & 47.5 & 48.0 & 28.7 & 31.7 & 34.4 & 32.9 & 36.7 & 36.7 & 27.5 & 26.7 & 26.9 & 70.4 & 83.2 & 85.3 \\
                \texttt{crh\_Latn} & 37.8 & 40.2 & 41.3 & 58.9 & 62.6 & 62.3 & 33.6 & 38.3 & 38.5 & 47.6 & 55.2 & 53.1 & 29.3 & 32.0 & 33.4 & 74.9 & 86.1 & 88.5 \\
                \texttt{cym\_Latn} & 31.9 & 34.6 & 33.8 & 55.4 & 58.1 & 59.3 & 29.2 & 34.0 & 35.2 & 46.3 & 53.2 & 53.7 & 28.5 & 30.5 & 30.5 & 76.3 & 86.7 & 88.8 \\
                \texttt{dan\_Latn} & 41.9 & 43.1 & 43.2 & 66.2 & 66.5 & 64.7 & 38.3 & 40.6 & 40.5 & 58.1 & 63.1 & 62.3 & 30.7 & 32.9 & 32.8 & 77.1 & 86.2 & 88.4 \\
                \texttt{deu\_Latn} & 42.0 & 42.7 & 43.0 & 68.4 & 67.0 & 64.5 & 41.0 & 41.7 & 41.5 & 60.8 & 64.6 & 63.1 & 32.3 & 32.9 & 33.4 & 77.0 & 86.9 & 88.8 \\
                \texttt{dik\_Latn} & 29.6 & 32.9 & 33.1 & 39.3 & 43.3 & 45.6 & 28.2 & 31.9 & 32.9 & 38.4 & 48.0 & 47.0 & 26.7 & 27.8 & 27.1 & 49.0 & 66.8 & 72.6 \\
                \texttt{dyu\_Latn} & 29.7 & 32.3 & 33.1 & 37.6 & 39.6 & 42.3 & 27.3 & 30.5 & 32.2 & 37.4 & 45.1 & 44.7 & 26.9 & 27.7 & 27.1 & 49.4 & 65.9 & 71.5 \\
                \texttt{dzo\_Tibt} & 23.6 & 23.9 & 24.6 & 21.3 & 20.9 & 21.8 & 21.8 & 23.1 & 23.3 & 30.3 & 31.6 & 32.3 & 26.0 & 26.1 & 25.7 & 37.4 & 38.1 & 36.4 \\
                \texttt{ell\_Grek} & 38.1 & 40.2 & 40.9 & 63.1 & 63.4 & 61.7 & 36.1 & 40.3 & 41.1 & 51.9 & 55.5 & 53.9 & 32.9 & 33.0 & 33.6 & 76.7 & 86.5 & 88.5 \\
                \texttt{epo\_Latn} & 39.6 & 41.2 & 41.2 & 64.7 & 66.0 & 64.2 & 36.4 & 39.8 & 40.3 & 54.0 & 60.9 & 59.5 & 31.2 & 33.4 & 34.3 & 76.8 & 85.9 & 89.3 \\
                \texttt{est\_Latn} & 36.8 & 38.5 & 38.6 & 62.0 & 64.0 & 63.2 & 33.3 & 39.0 & 39.7 & 49.4 & 59.0 & 58.4 & 31.0 & 31.8 & 32.3 & 77.8 & 85.9 & 87.8 \\
                \texttt{eus\_Latn} & 34.5 & 37.4 & 36.7 & 53.8 & 57.2 & 57.5 & 30.7 & 37.2 & 37.0 & 47.0 & 54.8 & 53.5 & 29.4 & 31.9 & 31.7 & 76.7 & 85.8 & 87.6 \\
                \texttt{ewe\_Latn} & 28.3 & 29.0 & 30.4 & 35.5 & 36.9 & 38.0 & 25.8 & 27.7 & 30.4 & 35.3 & 41.1 & 41.0 & 26.6 & 26.4 & 26.5 & 44.9 & 61.0 & 65.0 \\
                \texttt{fao\_Latn} & 33.6 & 35.0 & 35.3 & 52.2 & 55.2 & 56.6 & 30.9 & 34.9 & 36.0 & 43.9 & 50.6 & 50.8 & 29.2 & 30.5 & 30.7 & 75.9 & 86.0 & 87.8 \\
                \texttt{fij\_Latn} & 29.9 & 32.8 & 32.5 & 39.7 & 44.0 & 46.3 & 26.3 & 30.1 & 32.2 & 37.3 & 46.0 & 47.3 & 26.7 & 27.2 & 27.6 & 63.8 & 78.1 & 82.3 \\
                \texttt{fin\_Latn} & 37.6 & 39.7 & 40.9 & 64.2 & 64.9 & 64.3 & 34.7 & 39.8 & 39.6 & 55.7 & 61.2 & 60.6 & 30.4 & 32.3 & 33.3 & 78.0 & 86.5 & 88.4 \\
                \texttt{fon\_Latn} & 27.0 & 28.7 & 29.2 & 35.0 & 35.2 & 35.1 & 25.2 & 27.7 & 30.1 & 35.2 & 40.7 & 40.3 & 26.2 & 26.3 & 26.5 & 43.8 & 57.6 & 61.9 \\
                \texttt{fra\_Latn} & 43.2 & 41.7 & 42.6 & 67.8 & 67.6 & 64.5 & 41.1 & 41.9 & 42.8 & 62.6 & 65.7 & 65.7 & 32.8 & 34.0 & 34.4 & 77.1 & 86.6 & 88.8 \\
                \texttt{fur\_Latn} & 39.1 & 41.3 & 42.0 & 59.9 & 61.5 & 61.5 & 33.9 & 37.4 & 38.7 & 50.8 & 58.8 & 58.8 & 28.8 & 31.1 & 31.4 & 73.8 & 85.6 & 88.0 \\
                \texttt{fuv\_Latn} & 29.6 & 32.2 & 32.7 & 39.2 & 41.3 & 42.3 & 27.8 & 30.7 & 31.9 & 37.5 & 45.6 & 46.0 & 26.8 & 27.1 & 27.7 & 52.3 & 70.8 & 75.8 \\
                \texttt{gaz\_Latn} & 27.7 & 30.0 & 29.7 & 33.8 & 36.0 & 37.3 & 25.6 & 28.9 & 30.7 & 31.9 & 39.3 & 41.1 & 26.1 & 26.6 & 26.8 & 67.0 & 82.3 & 85.5 \\
                \texttt{gla\_Latn} & 29.6 & 31.5 & 30.3 & 44.0 & 46.2 & 47.6 & 28.3 & 31.6 & 32.8 & 37.7 & 42.5 & 43.3 & 28.1 & 29.1 & 29.2 & 72.6 & 83.5 & 86.2 \\
                \texttt{gle\_Latn} & 30.1 & 30.8 & 31.3 & 47.1 & 49.0 & 50.3 & 29.1 & 31.3 & 32.6 & 39.1 & 44.6 & 44.5 & 29.5 & 29.8 & 29.4 & 73.7 & 84.1 & 87.5 \\
                \texttt{glg\_Latn} & 42.3 & 42.3 & 42.9 & 64.1 & 65.7 & 64.4 & 39.8 & 40.8 & 40.8 & 60.1 & 64.8 & 63.9 & 30.8 & 33.9 & 34.0 & 77.6 & 86.4 & 88.8 \\
                \texttt{grn\_Latn} & 36.4 & 38.6 & 39.5 & 51.5 & 55.5 & 58.7 & 31.4 & 36.6 & 38.9 & 44.6 & 55.3 & 55.8 & 28.2 & 29.6 & 29.5 & 68.8 & 81.9 & 84.9 \\
                \texttt{guj\_Gujr} & 34.5 & 35.8 & 34.3 & 63.7 & 63.8 & 60.5 & 36.3 & 41.1 & 41.4 & 51.7 & 58.1 & 58.5 & 32.5 & 32.2 & 30.8 & 76.5 & 86.2 & 88.7 \\
                \texttt{hat\_Latn} & 37.2 & 38.5 & 39.2 & 54.6 & 57.8 & 58.5 & 32.2 & 37.7 & 38.7 & 46.3 & 56.4 & 57.0 & 30.8 & 32.7 & 32.5 & 75.8 & 85.2 & 87.9 \\
                \texttt{hau\_Latn} & 29.2 & 31.5 & 30.8 & 38.0 & 39.7 & 40.6 & 26.0 & 28.4 & 29.6 & 35.6 & 43.3 & 43.5 & 27.1 & 27.8 & 27.7 & 75.2 & 84.1 & 86.7 \\
                \texttt{heb\_Hebr} & 41.6 & 43.6 & 43.4 & 65.6 & 65.9 & 63.3 & 37.1 & 39.8 & 39.5 & 50.4 & 56.8 & 56.1 & 30.9 & 33.5 & 34.0 & 77.0 & 85.6 & 88.2 \\
                \texttt{hin\_Deva} & 41.9 & 41.9 & 42.5 & 65.4 & 66.0 & 63.0 & 37.8 & 41.4 & 43.8 & 56.6 & 61.6 & 62.1 & 32.3 & 33.8 & 32.9 & 78.1 & 86.2 & 88.5 \\
                \texttt{hne\_Deva} & 38.9 & 41.3 & 41.4 & 58.0 & 62.0 & 60.9 & 34.0 & 37.3 & 39.8 & 52.8 & 60.7 & 61.8 & 30.3 & 32.2 & 33.0 & 76.1 & 85.7 & 87.7 \\
                \texttt{hrv\_Latn} & 42.8 & 43.2 & 44.2 & 66.9 & 65.7 & 63.6 & 37.5 & 39.9 & 39.6 & 58.0 & 62.5 & 63.3 & 30.6 & 33.2 & 33.9 & 76.9 & 87.5 & 89.3 \\
                \texttt{hun\_Latn} & 38.1 & 40.0 & 41.8 & 64.7 & 64.9 & 64.0 & 35.2 & 38.0 & 38.3 & 51.9 & 57.4 & 56.5 & 30.7 & 32.8 & 33.5 & 77.2 & 86.7 & 88.5 \\
                \texttt{hye\_Armn} & 30.2 & 31.6 & 31.3 & 55.9 & 60.5 & 56.8 & 29.4 & 30.9 & 33.2 & 28.3 & 28.2 & 26.8 & 27.5 & 28.4 & 28.5 & 76.6 & 85.6 & 88.5 \\
                \texttt{ibo\_Latn} & 29.2 & 30.7 & 30.0 & 37.7 & 41.0 & 42.9 & 27.7 & 30.3 & 32.5 & 39.0 & 45.9 & 46.0 & 27.9 & 28.5 & 28.2 & 72.4 & 83.8 & 87.2 \\
                \texttt{ilo\_Latn} & 36.5 & 39.0 & 38.1 & 57.6 & 59.2 & 59.9 & 33.7 & 39.2 & 39.9 & 50.2 & 59.4 & 59.7 & 29.4 & 31.3 & 31.5 & 75.3 & 85.1 & 88.3 \\
                \texttt{ind\_Latn} & 43.8 & 43.2 & 43.5 & 68.4 & 67.6 & 65.9 & 39.0 & 41.2 & 39.8 & 59.7 & 62.7 & 62.2 & 32.3 & 33.2 & 34.1 & 77.9 & 86.4 & 89.0 \\
                \texttt{isl\_Latn} & 30.1 & 32.2 & 32.4 & 53.9 & 56.4 & 57.6 & 30.4 & 34.1 & 35.4 & 41.3 & 48.3 & 48.0 & 30.2 & 29.7 & 30.1 & 77.4 & 86.2 & 88.2 \\
                \texttt{ita\_Latn} & 43.1 & 43.8 & 43.8 & 67.2 & 66.8 & 63.7 & 39.4 & 41.6 & 41.5 & 60.4 & 64.7 & 62.7 & 31.6 & 32.9 & 34.0 & 77.5 & 86.9 & 88.9 \\
                \texttt{jav\_Latn} & 38.8 & 41.4 & 40.7 & 60.1 & 62.3 & 62.4 & 33.8 & 38.1 & 37.5 & 48.7 & 58.0 & 57.5 & 30.7 & 32.8 & 33.4 & 75.5 & 85.9 & 88.0 \\
                \texttt{jpn\_Jpan} & 43.4 & 44.4 & 43.9 & 68.3 & 68.6 & 66.0 & 40.2 & 42.5 & 41.7 & 59.0 & 65.4 & 63.3 & 30.0 & 31.4 & 32.9 & 78.0 & 86.2 & 88.8 \\
                \texttt{kab\_Latn} & 26.9 & 27.6 & 27.7 & 27.5 & 28.4 & 30.1 & 23.3 & 25.5 & 26.3 & 29.4 & 32.9 & 32.5 & 26.3 & 25.9 & 25.7 & 35.9 & 46.0 & 49.1 \\
                \texttt{kac\_Latn} & 28.8 & 31.6 & 31.8 & 35.4 & 41.0 & 42.3 & 26.4 & 29.2 & 31.6 & 37.8 & 46.3 & 46.2 & 26.7 & 27.4 & 27.0 & 46.6 & 64.2 & 68.4 \\
                \texttt{kam\_Latn} & 29.2 & 31.0 & 31.5 & 38.7 & 41.7 & 43.1 & 27.1 & 30.1 & 31.0 & 38.2 & 45.2 & 45.2 & 26.9 & 27.8 & 27.3 & 53.2 & 70.0 & 73.9 \\
                \texttt{kan\_Knda} & 28.3 & 29.7 & 28.9 & 58.7 & 61.2 & 58.1 & 33.3 & 38.1 & 39.6 & 44.5 & 48.1 & 49.4 & 30.8 & 31.7 & 29.9 & 76.0 & 84.9 & 87.5 \\
                \texttt{kas\_Arab} & 35.9 & 36.2 & 37.4 & 49.4 & 53.4 & 54.3 & 30.9 & 33.5 & 35.7 & 42.9 & 48.2 & 47.6 & 27.8 & 27.7 & 28.9 & 70.9 & 83.3 & 86.3 \\
                \texttt{kas\_Deva} & 33.3 & 35.1 & 35.3 & 47.1 & 50.1 & 48.8 & 30.2 & 31.9 & 34.1 & 42.4 & 48.9 & 50.0 & 27.7 & 28.9 & 29.4 & 66.5 & 80.4 & 83.2 \\
                \texttt{kat\_Geor} & 33.4 & 35.2 & 34.8 & 58.5 & 63.7 & 61.6 & 30.7 & 35.7 & 37.2 & 31.1 & 32.5 & 33.3 & 30.2 & 30.4 & 29.7 & 76.7 & 85.8 & 88.4 \\
                \texttt{kaz\_Cyrl} & 36.9 & 39.9 & 40.5 & 60.5 & 63.0 & 64.8 & 32.2 & 38.0 & 40.1 & 42.6 & 49.2 & 47.7 & 30.6 & 31.9 & 30.7 & 77.7 & 85.6 & 88.2 \\
                \texttt{kbp\_Latn} & 26.5 & 28.0 & 29.4 & 36.2 & 38.0 & 39.2 & 25.2 & 29.0 & 30.9 & 36.4 & 41.8 & 40.5 & 26.8 & 26.6 & 26.9 & 45.0 & 57.8 & 62.8 \\
                \texttt{kea\_Latn} & 38.2 & 41.1 & 40.5 & 59.2 & 61.4 & 61.0 & 34.6 & 39.5 & 38.9 & 50.5 & 59.2 & 59.2 & 30.2 & 31.1 & 30.5 & 73.3 & 84.5 & 86.5 \\
                \texttt{khk\_Cyrl} & 31.9 & 33.8 & 34.5 & 50.5 & 52.3 & 54.1 & 28.2 & 33.2 & 35.8 & 36.0 & 40.5 & 39.3 & 28.0 & 27.4 & 26.9 & 76.3 & 85.6 & 87.2 \\
                \texttt{khm\_Khmr} & 27.6 & 29.2 & 27.9 & 51.7 & 55.4 & 52.2 & 31.8 & 35.9 & 37.4 & 41.2 & 46.6 & 46.8 & 29.4 & 30.5 & 30.4 & 75.7 & 84.7 & 86.5 \\
                \texttt{kik\_Latn} & 30.4 & 32.8 & 32.5 & 42.1 & 45.8 & 48.1 & 28.5 & 31.6 & 33.6 & 42.0 & 49.9 & 48.7 & 28.3 & 28.1 & 28.1 & 55.9 & 71.6 & 74.8 \\
                \texttt{kin\_Latn} & 28.5 & 29.9 & 30.2 & 37.7 & 39.8 & 39.6 & 26.2 & 28.7 & 31.4 & 37.3 & 43.6 & 44.0 & 26.9 & 27.9 & 27.0 & 73.6 & 84.4 & 87.2 \\
                \texttt{kir\_Cyrl} & 35.8 & 39.3 & 39.5 & 54.0 & 56.9 & 59.5 & 30.4 & 36.5 & 38.5 & 40.9 & 47.5 & 45.9 & 29.5 & 30.0 & 30.3 & 77.0 & 86.1 & 88.2 \\
                \texttt{kmb\_Latn} & 28.4 & 31.0 & 30.9 & 36.2 & 38.8 & 40.6 & 25.6 & 29.7 & 31.6 & 36.1 & 43.2 & 42.7 & 26.6 & 27.0 & 26.5 & 45.0 & 62.8 & 66.4 \\
                \texttt{kmr\_Latn} & 31.5 & 35.2 & 34.6 & 45.3 & 47.9 & 48.1 & 30.2 & 35.4 & 36.4 & 39.4 & 47.8 & 48.3 & 27.2 & 26.7 & 27.1 & 68.8 & 82.6 & 85.4 \\
                \texttt{knc\_Arab} & 26.5 & 27.2 & 26.9 & 26.7 & 28.6 & 29.1 & 23.7 & 25.1 & 26.8 & 27.8 & 28.5 & 28.4 & 25.1 & 25.4 & 25.7 & 36.1 & 44.5 & 47.2 \\
                \texttt{knc\_Latn} & 29.9 & 31.6 & 32.3 & 41.0 & 43.0 & 43.8 & 27.2 & 30.9 & 32.3 & 39.8 & 47.1 & 47.5 & 27.1 & 27.3 & 27.6 & 50.1 & 67.1 & 72.8 \\
                \texttt{kon\_Latn} & 32.0 & 34.9 & 34.2 & 45.3 & 49.5 & 51.9 & 29.1 & 33.2 & 35.6 & 41.5 & 50.7 & 51.0 & 27.3 & 28.6 & 28.3 & 58.7 & 75.3 & 80.3 \\
                \texttt{kor\_Hang} & 40.5 & 43.3 & 43.3 & 68.1 & 68.4 & 65.4 & 39.9 & 40.7 & 40.7 & 60.5 & 65.1 & 62.9 & 30.5 & 31.9 & 32.9 & 77.3 & 85.9 & 87.9 \\
                \bottomrule
            \end{tabular}
        \end{adjustbox}
    \end{minipage}
    \hfill
    \begin{minipage}{0.48\textwidth}
        \centering
        \renewcommand{\arraystretch}{0.8} % Reduce row height 
        \rowcolors{2}{gray!15}{white} % Alternate row coloring: gray on every second row
        \begin{adjustbox}{max height=0.6\textheight, max width=\columnwidth}
            \begin{tabular}{l *{18}{c}}
                \toprule
                Lang.  & 
                \multicolumn{3}{c}{InternVL-2.5-4B} & 
                \multicolumn{3}{c}{InternVL-2.5-4B} & 
                \multicolumn{3}{c}{QwenVL-2.5-2B} & 
                \multicolumn{3}{c}{QwenVL-2.5-2B} & 
                \multicolumn{3}{c}{Centurio-Qwen} &
                \multicolumn{3}{c}{4o-mini} \\
                \cmidrule(lr){2-4} \cmidrule(lr){5-7} \cmidrule(lr){8-10} \cmidrule(lr){11-13} \cmidrule(lr){14-16} \cmidrule(lr){17-19}
                \textit{k} & 
                1 & 3 & 5 & 
                1 & 3 & 5 & 
                1 & 3 & 5 & 
                1 & 3 & 5 & 
                1 & 3 & 5 & 
                1 & 3 & 5 \\
                \midrule
                \texttt{lao\_Laoo} & 28.1 & 28.7 & 28.0 & 40.8 & 42.6 & 40.3 & 30.2 & 33.4 & 34.3 & 39.4 & 44.1 & 44.6 & 29.9 & 28.9 & 29.2 & 67.7 & 77.8 & 79.5 \\
                \texttt{lij\_Latn} & 39.5 & 42.7 & 43.3 & 61.6 & 61.9 & 63.1 & 34.8 & 37.6 & 38.5 & 49.6 & 56.9 & 57.3 & 29.7 & 31.3 & 31.2 & 73.9 & 85.0 & 86.9 \\
                \texttt{lim\_Latn} & 39.1 & 42.3 & 41.4 & 60.5 & 62.6 & 62.8 & 33.5 & 38.6 & 38.7 & 51.0 & 59.0 & 58.4 & 29.4 & 31.0 & 31.0 & 73.3 & 84.7 & 87.3 \\
                \texttt{lin\_Latn} & 31.8 & 34.7 & 34.8 & 46.1 & 49.2 & 51.0 & 28.8 & 34.0 & 34.3 & 41.8 & 51.1 & 50.5 & 29.0 & 30.3 & 30.0 & 67.2 & 81.6 & 85.5 \\
                \texttt{lit\_Latn} & 38.1 & 42.1 & 42.1 & 63.6 & 64.7 & 63.5 & 34.7 & 39.2 & 39.5 & 50.2 & 58.5 & 57.7 & 30.1 & 32.2 & 32.8 & 77.4 & 86.2 & 88.2 \\
                \texttt{lmo\_Latn} & 39.8 & 41.4 & 41.4 & 61.1 & 62.9 & 63.2 & 34.2 & 38.4 & 39.6 & 50.4 & 58.6 & 58.0 & 29.4 & 31.3 & 31.8 & 74.2 & 84.8 & 87.9 \\
                \texttt{ltg\_Latn} & 37.0 & 38.1 & 38.0 & 55.5 & 58.0 & 58.7 & 31.3 & 35.8 & 37.2 & 46.3 & 56.0 & 55.5 & 28.3 & 30.0 & 30.1 & 71.8 & 84.0 & 87.0 \\
                \texttt{ltz\_Latn} & 39.9 & 41.7 & 41.5 & 60.3 & 61.3 & 60.3 & 33.5 & 37.5 & 38.3 & 49.7 & 56.9 & 55.9 & 29.9 & 31.3 & 31.3 & 76.3 & 86.5 & 88.3 \\
                \texttt{lua\_Latn} & 31.0 & 33.7 & 34.6 & 43.8 & 46.2 & 48.4 & 29.3 & 33.0 & 34.6 & 41.7 & 50.8 & 49.7 & 27.7 & 27.8 & 28.1 & 54.9 & 70.4 & 75.9 \\
                \texttt{lug\_Latn} & 29.4 & 30.6 & 30.5 & 37.1 & 39.0 & 40.0 & 26.2 & 29.4 & 31.5 & 36.9 & 44.5 & 43.6 & 26.7 & 27.3 & 27.1 & 60.1 & 76.8 & 80.2 \\
                \texttt{luo\_Latn} & 31.1 & 32.8 & 33.2 & 38.1 & 40.8 & 42.6 & 26.4 & 30.7 & 32.2 & 37.7 & 46.2 & 47.7 & 26.9 & 27.7 & 26.9 & 49.0 & 66.9 & 73.6 \\
                \texttt{lus\_Latn} & 33.2 & 36.0 & 36.4 & 48.0 & 52.4 & 53.2 & 29.9 & 34.6 & 35.6 & 44.8 & 56.7 & 56.7 & 27.1 & 28.5 & 29.2 & 61.0 & 76.5 & 80.0 \\
                \texttt{lvs\_Latn} & 40.1 & 41.9 & 42.0 & 66.8 & 64.4 & 64.4 & 35.4 & 39.7 & 40.0 & 51.6 & 59.1 & 59.2 & 30.9 & 33.0 & 33.5 & 77.0 & 85.4 & 87.6 \\
                \texttt{mag\_Deva} & 39.9 & 41.7 & 42.7 & 59.0 & 63.7 & 62.2 & 33.5 & 37.6 & 40.9 & 54.9 & 60.5 & 62.5 & 30.4 & 33.0 & 33.3 & 76.5 & 86.2 & 88.3 \\
                \texttt{mai\_Deva} & 39.4 & 41.7 & 40.9 & 58.5 & 62.4 & 61.2 & 33.7 & 36.9 & 39.5 & 54.5 & 61.6 & 62.6 & 30.3 & 32.8 & 33.3 & 76.7 & 86.2 & 87.9 \\
                \texttt{mal\_Mlym} & 32.1 & 33.7 & 31.6 & 64.3 & 65.3 & 62.7 & 34.1 & 38.1 & 39.0 & 47.0 & 50.8 & 51.8 & 31.0 & 31.0 & 31.5 & 76.2 & 85.1 & 87.1 \\
                \texttt{mar\_Deva} & 36.6 & 37.9 & 39.0 & 61.5 & 62.0 & 59.3 & 35.1 & 38.2 & 40.4 & 49.2 & 55.8 & 57.5 & 31.3 & 32.9 & 32.3 & 76.5 & 86.0 & 88.2 \\
                \texttt{min\_Arab} & 28.1 & 28.6 & 27.7 & 32.2 & 34.2 & 36.4 & 24.3 & 27.0 & 28.9 & 27.8 & 30.1 & 29.5 & 26.2 & 26.0 & 26.0 & 49.5 & 65.3 & 68.8 \\
                \texttt{min\_Latn} & 38.1 & 40.8 & 40.9 & 58.2 & 60.9 & 61.1 & 32.9 & 37.4 & 37.8 & 48.9 & 55.5 & 55.6 & 29.2 & 31.0 & 32.0 & 71.9 & 83.5 & 86.5 \\
                \texttt{mkd\_Cyrl} & 42.4 & 43.0 & 44.1 & 64.9 & 65.1 & 64.2 & 36.0 & 38.7 & 39.7 & 52.6 & 58.5 & 57.2 & 31.4 & 32.9 & 33.5 & 77.0 & 86.4 & 88.2 \\
                \texttt{mlt\_Latn} & 37.8 & 39.0 & 38.2 & 57.4 & 60.8 & 62.4 & 32.8 & 36.2 & 37.8 & 50.0 & 56.9 & 56.7 & 30.6 & 32.1 & 32.9 & 76.1 & 85.5 & 88.0 \\
                \texttt{mni\_Beng} & 28.4 & 30.0 & 30.0 & 41.2 & 47.4 & 47.0 & 29.3 & 31.0 & 32.6 & 35.0 & 37.5 & 40.6 & 27.1 & 28.0 & 28.0 & 54.9 & 72.0 & 77.7 \\
                \texttt{mos\_Latn} & 28.2 & 31.6 & 32.0 & 37.1 & 39.6 & 42.6 & 26.5 & 29.9 & 31.2 & 35.9 & 42.9 & 43.0 & 27.5 & 27.2 & 27.1 & 47.6 & 64.0 & 69.0 \\
                \texttt{mri\_Latn} & 28.5 & 30.5 & 29.8 & 40.0 & 42.1 & 43.1 & 27.1 & 29.4 & 31.6 & 36.3 & 42.1 & 42.0 & 28.2 & 29.2 & 28.3 & 70.0 & 82.8 & 86.0 \\
                \texttt{mya\_Mymr} & 27.6 & 27.7 & 27.0 & 52.1 & 56.1 & 56.1 & 30.7 & 33.6 & 33.9 & 30.8 & 33.8 & 34.2 & 27.9 & 27.9 & 28.0 & 74.6 & 84.1 & 87.5 \\
                \texttt{nld\_Latn} & 42.1 & 43.1 & 43.7 & 67.0 & 65.8 & 64.4 & 38.8 & 41.8 & 41.2 & 59.4 & 63.4 & 62.3 & 31.9 & 33.1 & 32.5 & 77.6 & 85.5 & 87.8 \\
                \texttt{nno\_Latn} & 41.1 & 42.4 & 42.7 & 64.6 & 64.3 & 62.8 & 37.2 & 40.7 & 40.5 & 56.1 & 62.3 & 61.4 & 31.1 & 33.1 & 32.8 & 76.5 & 86.3 & 88.9 \\
                \texttt{nob\_Latn} & 41.4 & 43.2 & 43.2 & 66.5 & 66.3 & 65.0 & 38.3 & 40.6 & 40.7 & 57.8 & 63.1 & 62.2 & 30.5 & 33.3 & 33.2 & 76.9 & 86.4 & 89.2 \\
                \texttt{npi\_Deva} & 38.8 & 41.8 & 40.6 & 60.3 & 64.6 & 61.0 & 34.7 & 39.3 & 40.9 & 50.8 & 58.3 & 59.1 & 31.1 & 32.5 & 33.2 & 77.1 & 85.9 & 88.2 \\
                \texttt{nqo\_Nkoo} & 23.9 & 24.8 & 24.7 & 23.0 & 23.3 & 24.7 & 22.8 & 24.4 & 24.6 & 25.3 & 25.7 & 26.0 & 24.8 & 25.4 & 25.1 & 25.6 & 27.6 & 28.8 \\
                \texttt{nso\_Latn} & 29.4 & 31.6 & 31.5 & 40.7 & 41.7 & 42.7 & 27.6 & 29.4 & 31.7 & 38.7 & 47.2 & 46.2 & 26.8 & 27.6 & 27.0 & 64.6 & 79.1 & 83.6 \\
                \texttt{nus\_Latn} & 27.2 & 28.5 & 29.2 & 31.7 & 33.8 & 35.4 & 25.4 & 27.9 & 28.7 & 33.2 & 38.4 & 38.6 & 26.9 & 25.8 & 26.5 & 41.2 & 53.3 & 57.8 \\
                \texttt{nya\_Latn} & 30.1 & 32.7 & 33.3 & 43.0 & 44.9 & 46.6 & 27.5 & 31.1 & 32.5 & 41.2 & 48.5 & 49.1 & 27.3 & 27.9 & 28.1 & 72.4 & 83.2 & 86.5 \\
                \texttt{oci\_Latn} & 42.7 & 42.7 & 42.8 & 63.0 & 63.7 & 63.3 & 37.7 & 40.1 & 40.3 & 55.7 & 62.0 & 60.8 & 30.9 & 32.4 & 33.0 & 76.9 & 86.4 & 88.2 \\
                \texttt{ory\_Orya} & 25.7 & 26.0 & 26.0 & 59.8 & 60.5 & 61.4 & 34.8 & 39.6 & 40.0 & 47.0 & 52.8 & 55.0 & 31.6 & 29.7 & 29.8 & 75.5 & 85.0 & 86.6 \\
                \texttt{pag\_Latn} & 38.4 & 40.7 & 40.8 & 58.9 & 62.2 & 61.9 & 34.4 & 38.9 & 39.8 & 52.0 & 62.3 & 61.7 & 29.1 & 31.0 & 31.0 & 71.1 & 84.1 & 87.3 \\
                \texttt{pan\_Guru} & 31.7 & 33.7 & 32.6 & 60.6 & 63.0 & 60.2 & 35.6 & 42.2 & 42.7 & 51.3 & 58.9 & 59.5 & 31.0 & 31.4 & 30.5 & 76.1 & 85.7 & 87.7 \\
                \texttt{pap\_Latn} & 38.6 & 40.5 & 40.1 & 58.2 & 60.4 & 59.3 & 33.9 & 39.2 & 38.7 & 51.2 & 58.7 & 58.4 & 28.8 & 31.6 & 30.7 & 73.6 & 83.6 & 86.8 \\
                \texttt{pbt\_Arab} & 36.2 & 38.8 & 37.9 & 54.2 & 56.4 & 56.1 & 30.6 & 33.7 & 37.6 & 42.1 & 44.0 & 42.8 & 28.3 & 29.2 & 29.5 & 76.0 & 84.8 & 87.9 \\
                \texttt{pes\_Arab} & 42.7 & 43.3 & 42.8 & 66.8 & 66.6 & 64.7 & 37.7 & 39.8 & 41.8 & 55.5 & 60.7 & 59.3 & 31.4 & 33.9 & 34.0 & 77.3 & 86.7 & 88.8 \\
                \texttt{plt\_Latn} & 30.4 & 32.5 & 32.3 & 40.8 & 42.1 & 43.2 & 26.9 & 30.9 & 32.2 & 38.1 & 45.9 & 46.4 & 28.1 & 28.4 & 27.4 & 72.5 & 84.5 & 87.3 \\
                \texttt{pol\_Latn} & 44.0 & 44.4 & 44.3 & 66.8 & 66.4 & 63.2 & 38.0 & 40.1 & 39.6 & 60.6 & 65.1 & 64.6 & 30.7 & 33.6 & 34.2 & 76.1 & 86.5 & 88.6 \\
                \texttt{por\_Latn} & 42.1 & 42.6 & 42.9 & 68.1 & 68.2 & 65.8 & 40.3 & 42.1 & 42.3 & 61.8 & 66.7 & 65.7 & 32.4 & 33.5 & 34.2 & 77.1 & 86.6 & 88.9 \\
                \texttt{prs\_Arab} & 42.1 & 42.7 & 42.6 & 65.3 & 65.5 & 63.9 & 37.8 & 38.3 & 40.8 & 54.7 & 59.4 & 59.4 & 31.7 & 33.2 & 33.4 & 75.9 & 86.4 & 88.3 \\
                \texttt{quy\_Latn} & 31.8 & 35.2 & 35.1 & 43.3 & 47.2 & 48.0 & 28.3 & 32.4 & 34.4 & 41.2 & 49.3 & 50.2 & 27.4 & 28.3 & 28.2 & 55.9 & 73.1 & 78.5 \\
                \texttt{ron\_Latn} & 42.5 & 44.9 & 45.4 & 67.8 & 66.3 & 63.3 & 38.9 & 41.0 & 41.0 & 58.8 & 62.1 & 61.5 & 31.5 & 33.0 & 34.2 & 77.1 & 85.9 & 88.3 \\
                \texttt{run\_Latn} & 28.4 & 30.4 & 30.0 & 36.6 & 36.9 & 37.6 & 26.0 & 29.2 & 30.2 & 36.0 & 41.5 & 41.8 & 26.3 & 27.8 & 27.2 & 70.6 & 83.2 & 87.3 \\
                \texttt{rus\_Cyrl} & 43.2 & 43.4 & 43.8 & 68.7 & 69.0 & 64.6 & 40.6 & 42.3 & 42.2 & 61.7 & 66.2 & 64.7 & 31.9 & 33.3 & 33.3 & 76.8 & 86.2 & 88.6 \\
                \texttt{sag\_Latn} & 31.1 & 33.5 & 34.3 & 39.4 & 43.1 & 45.7 & 27.6 & 31.2 & 33.0 & 38.5 & 48.1 & 47.2 & 27.9 & 27.8 & 27.4 & 51.6 & 67.3 & 72.9 \\
                \texttt{san\_Deva} & 32.6 & 33.8 & 33.4 & 54.5 & 57.6 & 57.3 & 33.6 & 36.8 & 39.2 & 46.6 & 53.3 & 54.9 & 29.6 & 30.4 & 30.5 & 72.5 & 82.5 & 86.1 \\
                \texttt{sat\_Olck} & 24.2 & 23.9 & 24.4 & 22.2 & 22.8 & 23.5 & 22.8 & 23.9 & 24.5 & 24.1 & 24.6 & 24.7 & 24.3 & 25.3 & 25.3 & 25.0 & 25.9 & 25.7 \\
                \texttt{scn\_Latn} & 39.4 & 42.3 & 43.1 & 61.3 & 63.9 & 64.3 & 33.5 & 38.1 & 39.6 & 49.8 & 58.1 & 58.1 & 29.2 & 31.5 & 32.0 & 75.1 & 85.7 & 87.8 \\
                \texttt{shn\_Mymr} & 26.8 & 27.9 & 25.9 & 39.4 & 43.1 & 44.5 & 26.6 & 28.3 & 29.1 & 40.5 & 48.6 & 49.5 & 25.9 & 25.8 & 26.0 & 44.3 & 54.8 & 56.6 \\
                \texttt{sin\_Sinh} & 24.4 & 26.0 & 25.0 & 46.2 & 47.3 & 45.7 & 31.5 & 35.4 & 36.2 & 28.5 & 29.7 & 30.7 & 29.5 & 28.5 & 28.6 & 74.9 & 85.0 & 87.3 \\
                \texttt{slk\_Latn} & 43.3 & 43.0 & 43.1 & 65.6 & 63.6 & 62.1 & 37.5 & 39.8 & 40.2 & 56.2 & 62.0 & 62.1 & 30.6 & 32.5 & 33.3 & 77.3 & 86.8 & 88.4 \\
                \texttt{slv\_Latn} & 41.3 & 43.7 & 43.9 & 63.8 & 64.3 & 62.8 & 36.3 & 39.9 & 40.1 & 53.6 & 60.4 & 60.6 & 29.8 & 31.8 & 32.6 & 77.2 & 86.5 & 88.4 \\
                \texttt{smo\_Latn} & 30.4 & 32.3 & 32.2 & 41.9 & 45.3 & 48.0 & 26.7 & 30.8 & 32.8 & 38.2 & 46.3 & 47.3 & 28.6 & 29.5 & 29.1 & 72.5 & 84.2 & 87.3 \\
                \texttt{sna\_Latn} & 29.6 & 31.0 & 30.6 & 38.4 & 39.6 & 40.5 & 26.5 & 30.2 & 31.9 & 38.2 & 46.0 & 45.9 & 27.9 & 27.5 & 27.3 & 73.2 & 84.4 & 86.9 \\
                \texttt{snd\_Arab} & 32.4 & 33.5 & 33.0 & 48.2 & 52.9 & 54.8 & 30.9 & 35.2 & 36.8 & 40.0 & 46.1 & 44.5 & 29.8 & 31.6 & 31.4 & 75.8 & 86.1 & 88.2 \\
                \texttt{som\_Latn} & 28.2 & 30.2 & 30.5 & 37.6 & 38.3 & 38.4 & 27.3 & 29.5 & 31.4 & 35.4 & 40.2 & 40.8 & 27.2 & 26.6 & 27.3 & 71.8 & 83.2 & 86.8 \\
                \texttt{sot\_Latn} & 28.8 & 31.4 & 31.1 & 39.8 & 41.7 & 42.5 & 26.9 & 30.5 & 32.0 & 38.4 & 45.4 & 46.0 & 27.0 & 27.5 & 27.0 & 68.5 & 81.4 & 85.6 \\
                \texttt{spa\_Latn} & 42.6 & 43.0 & 43.7 & 67.7 & 66.6 & 63.6 & 41.4 & 41.9 & 42.1 & 62.0 & 65.8 & 64.4 & 31.7 & 34.8 & 34.2 & 77.0 & 87.1 & 89.2 \\
                \texttt{srd\_Latn} & 40.1 & 41.5 & 41.8 & 60.5 & 61.6 & 62.1 & 33.2 & 38.7 & 39.3 & 50.5 & 57.9 & 58.3 & 30.5 & 32.2 & 33.2 & 72.3 & 84.8 & 87.4 \\
                \texttt{srp\_Cyrl} & 42.7 & 44.1 & 44.5 & 66.2 & 65.1 & 63.5 & 37.0 & 39.7 & 41.2 & 53.0 & 59.4 & 58.4 & 31.7 & 32.6 & 33.3 & 77.9 & 87.1 & 89.1 \\
                \texttt{ssw\_Latn} & 28.5 & 31.2 & 30.6 & 37.0 & 40.2 & 40.0 & 26.3 & 29.8 & 31.2 & 37.1 & 44.4 & 43.8 & 28.2 & 28.2 & 28.0 & 67.5 & 81.0 & 84.7 \\
                \texttt{sun\_Latn} & 38.8 & 40.7 & 41.1 & 60.3 & 61.9 & 62.2 & 34.7 & 38.8 & 37.9 & 49.0 & 56.2 & 56.6 & 29.6 & 31.6 & 32.5 & 75.1 & 85.6 & 88.1 \\
                \texttt{swe\_Latn} & 41.3 & 43.0 & 42.7 & 66.3 & 66.1 & 63.6 & 38.5 & 41.4 & 41.4 & 59.6 & 63.4 & 62.2 & 31.7 & 33.4 & 33.7 & 77.8 & 86.2 & 88.4 \\
                \texttt{swh\_Latn} & 30.5 & 33.1 & 33.6 & 48.6 & 52.6 & 54.7 & 28.9 & 34.0 & 35.1 & 42.6 & 49.0 & 48.6 & 29.5 & 29.4 & 30.3 & 76.3 & 86.9 & 89.7 \\
                \texttt{szl\_Latn} & 40.4 & 41.2 & 40.3 & 60.5 & 63.0 & 62.1 & 33.5 & 35.6 & 37.8 & 50.9 & 57.7 & 57.3 & 29.0 & 30.9 & 31.3 & 73.9 & 85.0 & 87.4 \\
                \texttt{tam\_Taml} & 27.8 & 28.7 & 27.6 & 57.8 & 61.4 & 58.9 & 33.9 & 37.9 & 40.2 & 42.4 & 47.8 & 48.2 & 31.5 & 32.9 & 31.2 & 75.9 & 85.1 & 87.6 \\
                \texttt{taq\_Latn} & 30.4 & 32.5 & 32.6 & 41.2 & 44.5 & 44.7 & 27.2 & 31.7 & 33.5 & 38.9 & 47.4 & 47.9 & 27.2 & 27.0 & 26.7 & 51.8 & 68.6 & 72.6 \\
                \texttt{taq\_Tfng} & 24.0 & 24.2 & 24.4 & 24.9 & 25.4 & 27.0 & 23.4 & 24.9 & 25.9 & 27.4 & 28.9 & 29.7 & 24.8 & 25.2 & 25.1 & 28.1 & 32.5 & 33.0 \\
                \texttt{tat\_Cyrl} & 36.3 & 39.3 & 38.4 & 55.5 & 58.8 & 60.2 & 31.0 & 35.8 & 37.8 & 42.5 & 48.9 & 46.8 & 29.1 & 30.0 & 29.5 & 76.2 & 85.4 & 88.4 \\
                \texttt{tel\_Telu} & 31.7 & 32.2 & 31.0 & 59.1 & 60.0 & 56.1 & 35.9 & 39.6 & 39.7 & 49.0 & 56.2 & 56.8 & 32.3 & 31.8 & 31.2 & 75.9 & 85.6 & 87.9 \\
                \texttt{tgk\_Cyrl} & 32.8 & 35.2 & 34.6 & 50.8 & 52.4 & 55.6 & 29.4 & 33.6 & 35.8 & 38.0 & 41.1 & 41.7 & 28.4 & 29.0 & 28.6 & 76.0 & 85.8 & 88.0 \\
                \texttt{tgl\_Latn} & 38.3 & 41.9 & 42.6 & 65.3 & 67.5 & 67.4 & 38.5 & 41.6 & 41.3 & 58.7 & 63.1 & 62.2 & 31.2 & 33.7 & 34.2 & 76.3 & 86.1 & 88.3 \\
                \texttt{tha\_Thai} & 44.7 & 44.6 & 44.4 & 69.9 & 70.5 & 69.6 & 41.4 & 42.9 & 43.1 & 53.5 & 56.1 & 57.9 & 30.5 & 31.2 & 31.8 & 76.6 & 86.1 & 88.0 \\
                \texttt{tir\_Ethi} & 24.5 & 24.0 & 24.1 & 27.0 & 27.8 & 28.2 & 25.1 & 27.9 & 30.0 & 25.6 & 25.9 & 26.5 & 27.8 & 26.8 & 27.4 & 57.2 & 69.3 & 71.4 \\
                \texttt{tpi\_Latn} & 37.9 & 40.0 & 39.5 & 60.8 & 63.7 & 64.3 & 35.4 & 40.9 & 41.9 & 54.2 & 63.5 & 64.6 & 31.1 & 32.2 & 32.9 & 75.5 & 86.2 & 88.5 \\
                \texttt{tsn\_Latn} & 28.7 & 32.4 & 31.9 & 37.5 & 40.2 & 41.1 & 26.8 & 29.2 & 31.1 & 38.5 & 44.8 & 45.6 & 27.4 & 28.0 & 27.6 & 64.7 & 78.4 & 83.2 \\
                \texttt{tso\_Latn} & 28.5 & 30.0 & 30.1 & 38.0 & 39.7 & 40.8 & 26.3 & 29.3 & 30.7 & 38.6 & 44.9 & 45.0 & 27.9 & 29.2 & 28.4 & 61.7 & 76.9 & 81.9 \\
                \texttt{tuk\_Latn} & 34.7 & 37.0 & 38.2 & 53.7 & 57.8 & 61.0 & 30.4 & 35.5 & 37.5 & 42.9 & 51.5 & 50.9 & 28.4 & 29.5 & 30.2 & 75.2 & 85.7 & 88.9 \\
                \texttt{tum\_Latn} & 29.3 & 32.1 & 31.3 & 39.7 & 43.3 & 45.3 & 26.8 & 31.2 & 32.9 & 38.9 & 45.5 & 44.9 & 27.1 & 27.6 & 27.4 & 65.1 & 80.8 & 85.1 \\
                \texttt{tur\_Latn} & 42.7 & 44.6 & 45.6 & 66.6 & 66.7 & 64.8 & 38.7 & 41.3 & 40.9 & 56.4 & 61.0 & 59.9 & 31.0 & 34.3 & 34.8 & 77.3 & 87.1 & 88.7 \\
                \texttt{twi\_Latn} & 30.7 & 32.1 & 33.2 & 43.6 & 45.8 & 47.3 & 28.1 & 32.4 & 33.2 & 40.9 & 49.1 & 48.6 & 27.7 & 28.0 & 27.5 & 66.3 & 79.8 & 84.1 \\
                \texttt{tzm\_Tfng} & 24.2 & 24.4 & 24.1 & 23.6 & 22.7 & 23.3 & 22.7 & 24.3 & 24.3 & 25.0 & 26.4 & 27.1 & 25.1 & 25.1 & 25.4 & 26.9 & 28.7 & 28.9 \\
                \texttt{uig\_Arab} & 29.9 & 30.9 & 30.8 & 50.5 & 52.8 & 55.1 & 28.7 & 31.7 & 34.1 & 43.6 & 47.5 & 48.3 & 28.2 & 27.5 & 28.2 & 75.6 & 85.3 & 87.5 \\
                \texttt{ukr\_Cyrl} & 42.2 & 43.1 & 43.7 & 66.4 & 66.3 & 65.0 & 38.5 & 41.4 & 43.4 & 56.3 & 62.0 & 60.9 & 32.3 & 33.8 & 33.0 & 77.8 & 86.3 & 88.2 \\
                \texttt{umb\_Latn} & 29.4 & 31.9 & 32.4 & 35.7 & 39.4 & 41.0 & 26.1 & 29.3 & 31.3 & 36.2 & 43.8 & 44.8 & 27.5 & 27.2 & 26.9 & 47.8 & 65.6 & 71.4 \\
                \texttt{urd\_Arab} & 40.3 & 42.5 & 42.7 & 62.0 & 64.0 & 63.8 & 35.9 & 37.4 & 40.2 & 52.3 & 57.4 & 57.7 & 31.7 & 33.1 & 32.6 & 77.2 & 86.5 & 88.3 \\
                \texttt{uzn\_Latn} & 34.7 & 36.5 & 35.9 & 59.2 & 63.0 & 63.4 & 32.8 & 38.0 & 39.5 & 45.3 & 52.1 & 51.3 & 30.1 & 31.7 & 32.9 & 76.1 & 85.5 & 88.1 \\
                \texttt{vec\_Latn} & 41.1 & 42.9 & 43.2 & 63.4 & 63.5 & 62.7 & 35.6 & 39.5 & 39.2 & 54.4 & 60.8 & 61.1 & 29.5 & 32.0 & 32.0 & 75.5 & 86.0 & 88.4 \\
                \texttt{vie\_Latn} & 42.5 & 42.7 & 44.2 & 68.2 & 67.6 & 64.6 & 39.9 & 40.3 & 40.2 & 59.5 & 64.1 & 63.5 & 31.7 & 34.1 & 35.5 & 77.1 & 86.5 & 88.8 \\
                \texttt{war\_Latn} & 40.0 & 42.1 & 42.3 & 63.3 & 64.2 & 64.3 & 35.6 & 40.4 & 39.6 & 54.7 & 62.2 & 60.6 & 31.7 & 33.0 & 33.4 & 76.0 & 86.2 & 87.7 \\
                \texttt{wol\_Latn} & 31.5 & 34.8 & 34.4 & 42.1 & 45.8 & 46.3 & 28.4 & 32.9 & 34.8 & 40.0 & 49.7 & 50.9 & 27.2 & 28.1 & 27.9 & 56.5 & 72.6 & 77.8 \\
                \texttt{xho\_Latn} & 31.1 & 31.9 & 32.1 & 41.4 & 44.6 & 45.0 & 28.6 & 32.5 & 33.6 & 42.0 & 49.5 & 48.9 & 28.2 & 28.6 & 28.3 & 72.5 & 84.3 & 86.2 \\
                \texttt{ydd\_Hebr} & 27.5 & 28.5 & 28.1 & 39.8 & 43.0 & 45.5 & 25.0 & 29.0 & 31.5 & 28.0 & 29.3 & 29.8 & 26.5 & 28.1 & 28.2 & 72.7 & 84.8 & 86.9 \\
                \texttt{yor\_Latn} & 27.1 & 28.6 & 29.1 & 36.6 & 36.7 & 37.9 & 26.0 & 29.3 & 31.3 & 36.6 & 42.0 & 40.7 & 27.7 & 27.2 & 27.4 & 68.7 & 80.8 & 83.7 \\
                \texttt{yue\_Hant} & 45.0 & 45.0 & 45.2 & 67.3 & 66.6 & 65.0 & 39.1 & 40.5 & 39.8 & 60.1 & 65.6 & 64.1 & 30.0 & 31.4 & 31.3 & 78.7 & 87.1 & 89.0 \\
                \texttt{zho\_Hans} & 44.9 & 45.5 & 43.7 & 68.4 & 67.6 & 65.1 & 38.4 & 39.8 & 39.1 & 61.1 & 66.1 & 65.1 & 30.3 & 32.2 & 32.5 & 78.0 & 86.7 & 88.4 \\
                \texttt{zho\_Hant} & 45.2 & 45.5 & 44.0 & 67.1 & 67.6 & 65.7 & 38.4 & 39.5 & 39.9 & 60.0 & 65.6 & 64.3 & 29.8 & 32.5 & 31.9 & 77.4 & 86.5 & 88.8 \\
                \texttt{zsm\_Latn} & 41.7 & 43.4 & 43.9 & 67.1 & 66.5 & 64.9 & 36.7 & 39.8 & 38.6 & 56.6 & 61.3 & 60.3 & 32.0 & 33.3 & 33.4 & 77.2 & 85.7 & 89.0 \\
                \texttt{zul\_Latn} & 28.8 & 30.3 & 31.0 & 37.6 & 38.8 & 38.8 & 25.7 & 28.9 & 30.6 & 37.4 & 43.4 & 44.0 & 28.9 & 27.8 & 27.6 & 74.1 & 84.1 & 87.7 \\
                \bottomrule
            \end{tabular}
        \end{adjustbox}
    \end{minipage}
\end{table*}




\begin{table*}[ht]
    \subsubsection{Sentences-To-Topics}
    \label{app:sentences-to-topics}
    \begin{minipage}{0.48\textwidth}
        \centering
        \renewcommand{\arraystretch}{0.8} % Reduce row height 
        \rowcolors{2}{gray!15}{white} % Alternate row coloring: gray on every second row
        \begin{adjustbox}{max height=0.6\textheight, max width=\columnwidth}
            \begin{tabular}{l *{18}{c}}
                \toprule
                Lang.  & 
                \multicolumn{3}{c}{InternVL-2.5-4B} & 
                \multicolumn{3}{c}{InternVL-2.5-4B} & 
                \multicolumn{3}{c}{QwenVL-2.5-2B} & 
                \multicolumn{3}{c}{QwenVL-2.5-2B} & 
                \multicolumn{3}{c}{Centurio-Qwen} &
                \multicolumn{3}{c}{4o-mini} \\
                \cmidrule(lr){2-4} \cmidrule(lr){5-7} \cmidrule(lr){8-10} \cmidrule(lr){11-13} \cmidrule(lr){14-16} \cmidrule(lr){17-19}
                \textit{k} & 
                1 & 3 & 5 & 
                1 & 3 & 5 & 
                1 & 3 & 5 & 
                1 & 3 & 5 & 
                1 & 3 & 5 & 
                1 & 3 & 5 \\
                \midrule
                % Add more rows as needed
                \texttt{Avg} & 58.2 & 73.2 & 79.5 & 70.3 & 82.8 & 87.7 & 70.8 & 85.4 & 89.5 & 69.6 & 83.8 & 88.4 & 75.5 & 88.0 & 91.5 & 82.9 & 93.5 & 95.8 \\
\texttt{eng\_Latn} & 86.1 & 96.5 & 98.2 & 89.1 & 95.3 & 97.5 & 90.6 & 98.0 & 99.1 & 91.7 & 98.1 & 99.0 & 89.9 & 96.7 & 97.7 & 92.4 & 98.1 & 99.1 \\
\texttt{ace\_Arab} & 36.9 & 48.1 & 54.6 & 55.2 & 70.0 & 77.9 & 52.5 & 69.0 & 73.6 & 42.3 & 50.9 & 54.3 & 58.1 & 75.5 & 82.1 & 66.8 & 85.4 & 91.6 \\
\texttt{ace\_Latn} & 59.8 & 79.6 & 87.2 & 71.8 & 86.6 & 92.6 & 74.6 & 91.4 & 96.2 & 74.4 & 92.6 & 96.9 & 76.3 & 91.5 & 95.7 & 80.5 & 94.9 & 98.2 \\
\texttt{acm\_Arab} & 80.8 & 95.2 & 97.4 & 85.7 & 94.2 & 97.4 & 88.0 & 97.1 & 98.6 & 84.9 & 95.9 & 98.5 & 90.6 & 96.2 & 98.0 & 92.2 & 98.5 & 99.4 \\
\texttt{acq\_Arab} & 81.3 & 95.3 & 97.4 & 86.2 & 94.0 & 96.9 & 88.6 & 97.2 & 98.7 & 85.5 & 96.5 & 98.5 & 89.9 & 95.9 & 97.8 & 92.2 & 98.7 & 99.4 \\
\texttt{aeb\_Arab} & 77.9 & 94.1 & 97.2 & 85.2 & 93.9 & 97.4 & 87.5 & 97.2 & 98.7 & 83.5 & 95.3 & 97.7 & 88.8 & 96.6 & 98.3 & 91.5 & 98.4 & 99.5 \\
\texttt{afr\_Latn} & 77.0 & 92.9 & 96.1 & 82.9 & 93.6 & 96.4 & 84.3 & 96.0 & 97.9 & 85.8 & 96.6 & 98.9 & 88.4 & 96.1 & 97.8 & 91.6 & 98.3 & 99.3 \\
\texttt{ajp\_Arab} & 79.2 & 94.4 & 96.8 & 85.4 & 94.1 & 97.3 & 87.6 & 96.9 & 98.3 & 83.2 & 95.7 & 98.2 & 90.1 & 96.9 & 98.2 & 92.0 & 98.0 & 99.5 \\
\texttt{aka\_Latn} & 49.1 & 66.8 & 76.6 & 57.5 & 73.9 & 80.9 & 61.5 & 79.9 & 86.8 & 61.4 & 82.2 & 89.3 & 65.6 & 81.3 & 85.5 & 80.2 & 94.5 & 98.2 \\
\texttt{als\_Latn} & 61.8 & 84.4 & 91.7 & 79.9 & 91.6 & 96.2 & 74.8 & 91.3 & 95.3 & 77.2 & 94.5 & 98.6 & 83.4 & 95.1 & 97.5 & 91.4 & 98.3 & 99.3 \\
\texttt{amh\_Ethi} & 24.9 & 29.6 & 33.6 & 37.0 & 48.0 & 54.0 & 49.5 & 64.3 & 71.0 & 28.2 & 33.8 & 35.5 & 68.0 & 83.0 & 88.5 & 85.2 & 96.8 & 98.4 \\
\texttt{apc\_Arab} & 78.3 & 95.0 & 97.8 & 86.7 & 94.9 & 97.7 & 87.9 & 96.7 & 98.1 & 84.8 & 96.0 & 98.1 & 89.5 & 97.1 & 98.5 & 92.0 & 98.4 & 99.5 \\
\texttt{arb\_Arab} & 81.5 & 95.7 & 97.6 & 86.5 & 94.4 & 97.4 & 88.9 & 97.3 & 98.6 & 86.1 & 96.7 & 98.5 & 89.8 & 95.7 & 97.6 & 92.7 & 98.6 & 99.3 \\
\texttt{arb\_Latn} & 39.2 & 51.7 & 60.4 & 58.5 & 76.3 & 83.9 & 59.8 & 79.2 & 85.4 & 56.6 & 76.1 & 84.0 & 69.0 & 87.0 & 92.6 & 87.7 & 98.2 & 99.1 \\
\texttt{ars\_Arab} & 81.3 & 95.7 & 97.5 & 86.2 & 94.2 & 97.2 & 89.0 & 97.4 & 98.6 & 85.8 & 96.3 & 98.4 & 90.0 & 96.0 & 97.7 & 92.3 & 98.5 & 99.2 \\
\texttt{ary\_Arab} & 76.6 & 92.6 & 96.9 & 82.7 & 93.2 & 97.2 & 84.7 & 96.1 & 98.0 & 80.1 & 93.8 & 97.1 & 87.4 & 96.1 & 98.5 & 90.6 & 98.0 & 99.5 \\
\texttt{arz\_Arab} & 79.2 & 95.3 & 97.4 & 85.5 & 94.4 & 97.2 & 88.2 & 97.4 & 98.9 & 83.7 & 95.7 & 98.2 & 89.6 & 96.8 & 98.2 & 91.9 & 98.6 & 99.4 \\
\texttt{asm\_Beng} & 54.8 & 77.5 & 84.5 & 80.4 & 92.0 & 95.2 & 79.6 & 92.6 & 96.3 & 71.9 & 85.2 & 91.0 & 84.6 & 95.6 & 97.7 & 91.2 & 98.3 & 99.2 \\
\texttt{ast\_Latn} & 79.1 & 95.0 & 97.2 & 83.9 & 94.1 & 96.6 & 86.8 & 96.9 & 98.7 & 87.9 & 97.9 & 99.3 & 88.9 & 97.0 & 98.8 & 90.5 & 98.2 & 99.5 \\
\texttt{awa\_Deva} & 68.6 & 88.1 & 93.1 & 83.7 & 94.0 & 96.9 & 82.7 & 94.9 & 98.1 & 82.3 & 94.6 & 97.9 & 86.1 & 95.9 & 97.8 & 91.7 & 98.4 & 99.1 \\
\texttt{ayr\_Latn} & 41.9 & 57.5 & 68.8 & 49.3 & 66.6 & 74.9 & 51.1 & 71.0 & 79.0 & 50.8 & 70.0 & 78.2 & 54.1 & 73.5 & 81.1 & 57.6 & 78.9 & 87.3 \\
\texttt{azb\_Arab} & 59.5 & 81.3 & 89.4 & 76.4 & 91.1 & 95.4 & 72.3 & 90.2 & 94.1 & 71.2 & 86.5 & 91.6 & 80.1 & 93.2 & 96.6 & 85.8 & 97.2 & 99.0 \\
\texttt{azj\_Latn} & 69.2 & 85.1 & 92.2 & 82.1 & 92.9 & 96.7 & 77.6 & 94.2 & 96.6 & 79.0 & 94.3 & 97.9 & 84.7 & 95.3 & 97.9 & 91.6 & 98.8 & 99.5 \\
\texttt{bak\_Cyrl} & 58.5 & 79.2 & 86.9 & 78.9 & 91.5 & 96.5 & 73.4 & 91.1 & 95.3 & 72.2 & 89.7 & 95.9 & 78.2 & 93.2 & 96.4 & 90.6 & 98.6 & 99.5 \\
\texttt{bam\_Latn} & 40.7 & 52.7 & 63.1 & 47.6 & 65.5 & 73.6 & 51.4 & 71.1 & 78.7 & 52.9 & 72.5 & 82.1 & 57.7 & 77.2 & 83.2 & 55.4 & 78.8 & 87.4 \\
\texttt{ban\_Latn} & 68.3 & 87.8 & 93.3 & 78.7 & 92.5 & 96.1 & 79.8 & 94.6 & 97.6 & 79.5 & 95.9 & 98.6 & 82.7 & 95.0 & 98.1 & 86.7 & 98.0 & 99.3 \\
\texttt{bel\_Cyrl} & 73.3 & 91.8 & 95.5 & 84.5 & 93.7 & 96.7 & 82.4 & 96.4 & 97.9 & 83.4 & 94.9 & 97.6 & 89.1 & 97.2 & 98.6 & 91.6 & 98.6 & 99.4 \\
\texttt{bem\_Latn} & 46.5 & 63.3 & 73.0 & 54.9 & 74.2 & 80.9 & 59.4 & 79.5 & 85.3 & 60.0 & 81.3 & 89.2 & 63.6 & 79.7 & 83.0 & 71.0 & 90.6 & 96.5 \\
\texttt{ben\_Beng} & 68.8 & 86.9 & 92.6 & 85.2 & 93.9 & 96.9 & 83.7 & 95.3 & 97.6 & 79.5 & 93.0 & 96.8 & 86.4 & 96.5 & 97.7 & 91.1 & 98.5 & 99.2 \\
\texttt{bho\_Deva} & 65.6 & 85.9 & 91.8 & 83.5 & 93.6 & 96.9 & 81.9 & 95.4 & 98.1 & 80.5 & 94.6 & 97.9 & 85.5 & 96.4 & 98.3 & 90.7 & 98.5 & 99.2 \\
\texttt{bjn\_Arab} & 36.6 & 45.8 & 51.2 & 49.8 & 64.6 & 72.4 & 47.9 & 62.5 & 67.3 & 39.6 & 44.5 & 45.6 & 53.1 & 68.3 & 76.6 & 63.4 & 85.5 & 92.1 \\
\texttt{bjn\_Latn} & 65.1 & 87.3 & 93.7 & 77.4 & 90.1 & 94.5 & 76.9 & 92.8 & 96.5 & 77.0 & 92.0 & 97.1 & 80.4 & 93.9 & 97.6 & 87.7 & 97.8 & 99.6 \\
\texttt{bod\_Tibt} & 26.9 & 30.1 & 31.8 & 31.7 & 35.3 & 41.0 & 33.4 & 39.5 & 44.1 & 48.6 & 53.1 & 56.9 & 50.0 & 62.5 & 67.6 & 56.8 & 64.4 & 66.4 \\
\texttt{bos\_Latn} & 79.5 & 94.2 & 97.1 & 85.0 & 93.9 & 96.1 & 86.2 & 96.6 & 98.3 & 87.7 & 97.8 & 99.2 & 88.9 & 96.6 & 98.4 & 91.6 & 98.5 & 99.4 \\
\texttt{bug\_Latn} & 58.8 & 77.5 & 85.4 & 69.2 & 85.4 & 91.4 & 71.9 & 89.2 & 94.9 & 71.7 & 89.7 & 95.4 & 71.5 & 89.0 & 93.7 & 77.4 & 93.8 & 97.9 \\
\texttt{bul\_Cyrl} & 79.5 & 94.3 & 97.4 & 85.8 & 94.2 & 96.8 & 85.8 & 96.7 & 98.3 & 87.7 & 97.4 & 99.2 & 89.3 & 96.2 & 98.2 & 92.1 & 98.7 & 99.4 \\
\texttt{cat\_Latn} & 82.0 & 95.6 & 97.8 & 85.1 & 94.1 & 96.8 & 87.7 & 96.9 & 98.5 & 88.7 & 97.6 & 99.2 & 89.5 & 96.6 & 98.4 & 92.0 & 98.4 & 99.1 \\
\texttt{ceb\_Latn} & 65.8 & 85.2 & 91.3 & 81.5 & 92.5 & 96.2 & 80.0 & 93.1 & 96.7 & 83.3 & 96.1 & 98.9 & 86.2 & 95.7 & 98.0 & 90.9 & 98.4 & 99.2 \\
\texttt{ces\_Latn} & 82.9 & 95.1 & 97.6 & 84.4 & 92.7 & 96.1 & 87.3 & 97.3 & 98.7 & 89.5 & 97.9 & 99.2 & 89.4 & 95.6 & 97.7 & 91.8 & 98.4 & 99.2 \\
\texttt{cjk\_Latn} & 44.4 & 59.5 & 69.7 & 53.0 & 69.8 & 78.0 & 55.1 & 73.3 & 81.0 & 54.0 & 74.7 & 84.0 & 59.2 & 76.3 & 82.5 & 59.4 & 80.8 & 88.8 \\
\texttt{ckb\_Arab} & 42.7 & 57.4 & 61.5 & 67.4 & 84.2 & 90.3 & 61.5 & 80.1 & 86.2 & 47.1 & 57.2 & 61.4 & 61.7 & 77.4 & 82.1 & 85.0 & 97.4 & 98.9 \\
\texttt{crh\_Latn} & 64.2 & 82.2 & 90.2 & 78.8 & 91.6 & 96.3 & 77.4 & 93.7 & 97.0 & 78.0 & 93.4 & 97.4 & 83.2 & 94.9 & 97.7 & 88.9 & 98.1 & 99.4 \\
\texttt{cym\_Latn} & 48.7 & 63.2 & 73.3 & 74.2 & 86.6 & 93.0 & 64.7 & 82.2 & 88.2 & 74.2 & 91.5 & 97.0 & 77.4 & 93.1 & 96.7 & 90.5 & 98.2 & 99.4 \\
\texttt{dan\_Latn} & 80.7 & 94.4 & 97.2 & 84.0 & 93.3 & 95.9 & 85.7 & 96.8 & 98.6 & 88.3 & 97.3 & 98.8 & 89.0 & 96.2 & 97.8 & 92.1 & 98.3 & 99.2 \\
\texttt{deu\_Latn} & 84.5 & 95.9 & 97.9 & 85.8 & 94.4 & 96.7 & 86.8 & 97.3 & 99.0 & 88.5 & 97.6 & 98.8 & 90.0 & 96.1 & 97.7 & 91.6 & 98.4 & 99.3 \\
\texttt{dik\_Latn} & 45.2 & 61.8 & 72.1 & 54.2 & 70.8 & 77.2 & 58.4 & 77.7 & 83.3 & 59.9 & 79.8 & 86.7 & 58.0 & 78.9 & 85.9 & 58.5 & 81.3 & 88.4 \\
\texttt{dyu\_Latn} & 43.4 & 59.0 & 68.6 & 54.0 & 72.5 & 79.8 & 57.3 & 76.8 & 84.6 & 58.2 & 78.1 & 87.3 & 60.3 & 80.4 & 87.5 & 60.0 & 83.1 & 89.9 \\
\texttt{dzo\_Tibt} & 24.2 & 27.1 & 26.6 & 24.9 & 22.7 & 24.1 & 28.7 & 27.8 & 27.0 & 41.2 & 41.7 & 43.3 & 44.5 & 55.7 & 59.5 & 54.4 & 57.8 & 57.6 \\
\texttt{ell\_Grek} & 72.9 & 89.5 & 92.7 & 83.6 & 91.9 & 95.6 & 81.2 & 95.0 & 97.8 & 82.8 & 95.6 & 98.7 & 87.1 & 96.8 & 98.3 & 92.3 & 98.6 & 99.4 \\
\texttt{epo\_Latn} & 70.4 & 90.4 & 96.0 & 83.5 & 93.3 & 96.8 & 83.1 & 96.6 & 98.2 & 86.0 & 96.9 & 99.2 & 88.2 & 96.3 & 98.0 & 91.5 & 98.6 & 99.3 \\
\texttt{est\_Latn} & 63.6 & 84.0 & 91.4 & 79.9 & 91.4 & 95.7 & 76.5 & 92.7 & 96.6 & 78.5 & 94.8 & 98.1 & 84.2 & 95.2 & 97.5 & 91.5 & 98.4 & 99.4 \\
\texttt{eus\_Latn} & 57.9 & 80.5 & 87.5 & 71.8 & 85.8 & 90.6 & 71.7 & 88.1 & 92.6 & 73.8 & 92.7 & 97.3 & 79.4 & 92.6 & 96.0 & 90.3 & 98.3 & 99.5 \\
\texttt{ewe\_Latn} & 40.2 & 52.9 & 63.9 & 47.7 & 62.7 & 71.3 & 50.9 & 67.6 & 76.6 & 51.5 & 71.8 & 80.9 & 50.3 & 69.8 & 77.3 & 53.7 & 75.4 & 82.6 \\
\texttt{fao\_Latn} & 52.1 & 69.4 & 79.9 & 72.1 & 86.6 & 91.8 & 71.0 & 88.7 & 92.7 & 72.8 & 87.1 & 92.9 & 80.3 & 94.1 & 96.9 & 90.0 & 98.2 & 99.0 \\
\texttt{fij\_Latn} & 46.7 & 59.2 & 68.5 & 53.6 & 71.0 & 77.6 & 57.0 & 77.7 & 83.0 & 56.6 & 73.6 & 81.8 & 61.1 & 80.5 & 86.3 & 78.5 & 92.2 & 96.6 \\
\texttt{fin\_Latn} & 64.6 & 82.8 & 91.3 & 83.5 & 92.4 & 96.0 & 80.0 & 95.1 & 98.5 & 85.2 & 96.7 & 99.2 & 85.7 & 94.8 & 97.6 & 91.5 & 98.4 & 99.2 \\
\texttt{fon\_Latn} & 39.0 & 51.2 & 62.3 & 47.9 & 63.1 & 71.0 & 52.3 & 69.5 & 77.4 & 49.3 & 70.2 & 80.2 & 51.5 & 70.8 & 77.7 & 54.1 & 73.8 & 82.2 \\
\texttt{fra\_Latn} & 85.8 & 95.8 & 97.9 & 85.9 & 94.8 & 96.7 & 88.8 & 97.3 & 98.7 & 89.4 & 97.9 & 98.9 & 90.9 & 96.6 & 98.4 & 92.1 & 98.2 & 99.0 \\
\texttt{fur\_Latn} & 70.7 & 90.8 & 95.5 & 79.6 & 91.7 & 96.1 & 81.0 & 95.1 & 97.8 & 83.2 & 97.2 & 98.8 & 84.1 & 95.6 & 98.3 & 87.7 & 97.7 & 99.1 \\
\texttt{fuv\_Latn} & 44.7 & 60.3 & 71.2 & 53.5 & 70.8 & 77.7 & 57.5 & 78.5 & 84.5 & 57.0 & 78.3 & 86.4 & 58.2 & 79.3 & 84.6 & 65.2 & 87.4 & 93.3 \\
\texttt{gaz\_Latn} & 36.9 & 48.6 & 58.5 & 45.4 & 61.6 & 70.3 & 50.4 & 70.3 & 78.9 & 45.2 & 63.7 & 72.9 & 55.5 & 74.7 & 81.5 & 82.2 & 96.9 & 99.3 \\
\texttt{gla\_Latn} & 43.8 & 55.4 & 65.7 & 60.3 & 76.6 & 85.0 & 56.6 & 75.0 & 80.9 & 57.5 & 76.0 & 85.1 & 73.7 & 88.7 & 94.2 & 87.6 & 97.4 & 99.0 \\
\texttt{gle\_Latn} & 43.9 & 54.5 & 64.6 & 66.4 & 83.3 & 90.0 & 59.7 & 80.2 & 85.0 & 62.0 & 81.0 & 88.7 & 76.9 & 91.7 & 96.0 & 89.9 & 98.1 & 99.0 \\
\texttt{glg\_Latn} & 83.0 & 96.0 & 97.6 & 84.4 & 93.8 & 96.6 & 88.2 & 97.2 & 98.7 & 89.3 & 97.9 & 99.0 & 89.3 & 96.4 & 98.1 & 91.4 & 98.3 & 99.1 \\
\texttt{grn\_Latn} & 61.4 & 80.5 & 88.8 & 70.0 & 87.0 & 92.3 & 74.7 & 91.7 & 95.5 & 73.7 & 91.2 & 96.7 & 77.3 & 92.2 & 95.5 & 82.6 & 95.8 & 98.2 \\
\texttt{guj\_Gujr} & 52.0 & 68.8 & 74.2 & 82.2 & 93.6 & 96.3 & 80.7 & 94.4 & 96.9 & 77.4 & 92.4 & 96.9 & 83.7 & 95.8 & 98.1 & 90.9 & 98.4 & 99.4 \\
\texttt{hat\_Latn} & 58.0 & 79.9 & 87.9 & 74.3 & 90.2 & 94.4 & 73.9 & 91.4 & 96.2 & 75.0 & 92.8 & 97.9 & 83.0 & 95.4 & 98.1 & 88.7 & 98.2 & 99.3 \\
\texttt{hau\_Latn} & 44.0 & 57.2 & 67.9 & 50.4 & 67.3 & 74.9 & 52.0 & 71.9 & 77.9 & 53.4 & 72.1 & 81.8 & 66.3 & 81.2 & 86.7 & 87.8 & 97.5 & 99.0 \\
\texttt{heb\_Hebr} & 76.4 & 93.1 & 96.1 & 84.5 & 93.7 & 96.4 & 86.0 & 95.5 & 98.3 & 78.8 & 93.3 & 97.2 & 87.7 & 96.2 & 98.2 & 91.5 & 98.4 & 99.2 \\
\texttt{hin\_Deva} & 72.2 & 88.9 & 93.5 & 85.5 & 95.0 & 97.7 & 85.4 & 96.4 & 98.2 & 83.9 & 95.6 & 98.4 & 87.6 & 96.6 & 98.2 & 91.8 & 98.4 & 99.4 \\
\texttt{hne\_Deva} & 64.5 & 85.5 & 91.3 & 83.7 & 94.7 & 97.3 & 82.0 & 95.7 & 98.2 & 80.3 & 94.7 & 98.5 & 84.7 & 95.7 & 98.0 & 90.6 & 98.5 & 99.2 \\
\texttt{hrv\_Latn} & 79.8 & 94.5 & 97.2 & 85.1 & 93.9 & 96.7 & 86.1 & 96.8 & 98.4 & 87.5 & 97.7 & 99.3 & 88.3 & 96.8 & 98.5 & 92.0 & 98.7 & 99.5 \\
\texttt{hun\_Latn} & 69.8 & 87.2 & 93.4 & 83.6 & 92.9 & 96.4 & 81.3 & 94.8 & 97.7 & 83.4 & 95.7 & 98.5 & 85.3 & 95.9 & 97.6 & 91.5 & 98.6 & 99.3 \\
\texttt{hye\_Armn} & 41.7 & 53.5 & 59.4 & 78.3 & 88.9 & 93.6 & 59.1 & 74.4 & 80.5 & 38.7 & 42.0 & 42.6 & 69.3 & 82.9 & 87.4 & 91.7 & 98.4 & 99.5 \\
\texttt{ibo\_Latn} & 40.6 & 54.7 & 65.9 & 53.7 & 69.6 & 77.5 & 56.7 & 73.1 & 80.0 & 54.8 & 74.4 & 82.5 & 71.7 & 86.2 & 91.4 & 87.3 & 97.0 & 98.6 \\
\texttt{ilo\_Latn} & 63.0 & 83.1 & 90.7 & 77.1 & 89.3 & 93.9 & 78.5 & 93.8 & 96.7 & 79.2 & 94.8 & 98.6 & 79.4 & 93.5 & 96.3 & 88.9 & 97.5 & 99.1 \\
\texttt{ind\_Latn} & 84.2 & 95.6 & 97.7 & 86.1 & 94.4 & 97.1 & 88.1 & 96.5 & 98.6 & 88.4 & 97.6 & 99.0 & 89.7 & 96.3 & 97.5 & 92.0 & 98.2 & 99.2 \\
\texttt{isl\_Latn} & 46.3 & 59.7 & 69.6 & 72.1 & 85.0 & 90.7 & 67.6 & 87.3 & 91.8 & 69.7 & 83.4 & 89.5 & 81.0 & 94.0 & 96.9 & 91.1 & 98.1 & 99.2 \\
\texttt{ita\_Latn} & 84.4 & 96.1 & 97.8 & 86.9 & 94.7 & 96.9 & 88.0 & 97.2 & 98.7 & 89.6 & 97.8 & 99.0 & 90.3 & 96.7 & 98.2 & 91.6 & 98.4 & 99.3 \\
\texttt{jav\_Latn} & 66.8 & 88.0 & 94.3 & 79.1 & 91.8 & 95.7 & 78.8 & 93.6 & 96.8 & 78.3 & 93.4 & 97.8 & 82.9 & 95.3 & 97.7 & 89.9 & 98.1 & 99.4 \\
\texttt{jpn\_Jpan} & 83.8 & 95.9 & 97.8 & 86.2 & 95.4 & 97.3 & 87.6 & 97.0 & 98.7 & 89.4 & 97.7 & 98.9 & 89.8 & 96.7 & 97.8 & 92.1 & 98.2 & 99.1 \\
\texttt{kab\_Latn} & 31.8 & 37.2 & 43.9 & 35.6 & 41.3 & 45.0 & 37.5 & 46.9 & 53.4 & 40.8 & 49.3 & 53.8 & 43.5 & 52.8 & 57.7 & 42.8 & 58.2 & 65.3 \\
\texttt{kac\_Latn} & 41.3 & 57.6 & 68.0 & 51.5 & 68.0 & 75.0 & 55.2 & 75.0 & 81.5 & 54.6 & 77.7 & 86.0 & 54.7 & 75.3 & 80.9 & 58.2 & 78.6 & 86.3 \\
\texttt{kam\_Latn} & 44.5 & 57.4 & 66.8 & 52.2 & 68.1 & 75.8 & 53.6 & 71.8 & 79.1 & 55.6 & 75.9 & 83.7 & 59.8 & 77.8 & 82.7 & 65.7 & 86.7 & 92.6 \\
\texttt{kan\_Knda} & 39.9 & 48.0 & 51.5 & 78.7 & 90.8 & 94.9 & 74.3 & 90.6 & 94.4 & 70.4 & 83.5 & 89.1 & 80.5 & 93.1 & 96.5 & 89.9 & 98.5 & 99.3 \\
\texttt{kas\_Arab} & 50.3 & 67.9 & 78.7 & 74.7 & 89.9 & 94.5 & 72.9 & 89.9 & 94.1 & 70.4 & 83.2 & 89.9 & 74.7 & 90.2 & 95.1 & 86.7 & 97.1 & 99.4 \\
\texttt{kas\_Deva} & 50.1 & 68.6 & 78.4 & 72.4 & 88.4 & 92.4 & 71.7 & 86.9 & 92.3 & 65.1 & 81.2 & 87.9 & 75.0 & 90.5 & 93.8 & 82.9 & 95.9 & 98.7 \\
\texttt{kat\_Geor} & 54.1 & 72.4 & 80.3 & 80.6 & 91.2 & 95.5 & 69.0 & 88.3 & 93.2 & 45.8 & 52.9 & 56.6 & 81.7 & 94.7 & 97.8 & 91.7 & 98.5 & 99.3 \\
\texttt{kaz\_Cyrl} & 61.1 & 80.4 & 87.8 & 81.8 & 92.7 & 97.3 & 74.8 & 92.6 & 96.2 & 73.3 & 90.7 & 96.7 & 84.7 & 95.6 & 97.5 & 91.8 & 98.5 & 99.5 \\
\texttt{kbp\_Latn} & 37.4 & 52.2 & 62.5 & 51.7 & 69.2 & 77.5 & 52.2 & 70.3 & 77.5 & 55.7 & 74.7 & 82.7 & 53.8 & 71.2 & 76.7 & 53.7 & 74.6 & 82.4 \\
\texttt{kea\_Latn} & 69.9 & 88.7 & 93.5 & 79.0 & 92.1 & 96.3 & 81.7 & 95.2 & 97.8 & 83.0 & 96.1 & 99.1 & 84.0 & 96.4 & 98.7 & 87.5 & 97.4 & 99.1 \\
\texttt{khk\_Cyrl} & 43.9 & 61.6 & 69.5 & 71.4 & 84.4 & 91.1 & 61.6 & 79.5 & 84.8 & 55.0 & 70.7 & 77.5 & 63.1 & 78.0 & 83.1 & 90.5 & 98.4 & 99.5 \\
\texttt{khm\_Khmr} & 41.9 & 57.7 & 63.1 & 73.4 & 85.1 & 90.6 & 70.0 & 83.0 & 87.3 & 60.1 & 76.9 & 83.1 & 77.1 & 90.4 & 94.9 & 90.3 & 97.9 & 99.5 \\
\texttt{kik\_Latn} & 47.0 & 63.7 & 73.3 & 56.7 & 73.2 & 79.4 & 60.9 & 79.1 & 85.3 & 58.9 & 78.9 & 84.6 & 66.1 & 80.2 & 85.0 & 67.8 & 87.3 & 92.8 \\
\texttt{kin\_Latn} & 42.2 & 56.1 & 67.0 & 51.8 & 68.9 & 77.4 & 55.1 & 74.0 & 81.4 & 53.8 & 73.6 & 83.0 & 57.3 & 74.3 & 79.2 & 88.0 & 98.1 & 99.3 \\
\texttt{kir\_Cyrl} & 57.0 & 76.9 & 85.2 & 75.5 & 89.2 & 95.0 & 72.1 & 90.1 & 95.4 & 68.2 & 85.2 & 91.7 & 80.2 & 93.3 & 96.9 & 91.5 & 98.6 & 99.5 \\
\texttt{kmb\_Latn} & 41.8 & 53.5 & 62.8 & 48.7 & 66.0 & 72.7 & 52.0 & 70.5 & 77.8 & 52.2 & 72.4 & 79.8 & 54.5 & 72.9 & 77.1 & 55.1 & 77.2 & 84.8 \\
\texttt{kmr\_Latn} & 49.1 & 67.6 & 79.0 & 63.8 & 80.8 & 86.5 & 66.8 & 86.2 & 90.9 & 63.8 & 84.8 & 91.4 & 66.3 & 85.8 & 90.1 & 82.5 & 96.4 & 98.9 \\
\texttt{knc\_Arab} & 30.9 & 37.6 & 42.9 & 36.9 & 49.2 & 57.5 & 36.2 & 46.3 & 51.7 & 33.0 & 40.7 & 41.4 & 35.9 & 48.7 & 57.0 & 41.6 & 59.0 & 68.3 \\
\texttt{knc\_Latn} & 46.1 & 61.8 & 72.8 & 55.1 & 71.6 & 79.2 & 58.8 & 79.8 & 86.4 & 59.2 & 79.4 & 87.3 & 60.7 & 79.9 & 86.1 & 61.7 & 83.2 & 91.2 \\
\texttt{kon\_Latn} & 49.7 & 66.2 & 74.9 & 59.8 & 79.3 & 86.3 & 62.8 & 83.6 & 89.8 & 62.7 & 83.2 & 91.9 & 65.6 & 82.3 & 87.8 & 70.1 & 91.0 & 95.8 \\
\texttt{kor\_Hang} & 81.8 & 93.8 & 96.7 & 86.8 & 94.6 & 96.8 & 87.5 & 96.8 & 98.9 & 88.3 & 97.7 & 99.1 & 88.7 & 95.8 & 97.7 & 91.1 & 98.3 & 99.2 \\

                \bottomrule
            \end{tabular}
        \end{adjustbox}
    \end{minipage}
    \hfill
    \begin{minipage}{0.48\textwidth}
        \centering
        \renewcommand{\arraystretch}{0.8} % Reduce row height 
        \rowcolors{2}{gray!15}{white} % Alternate row coloring: gray on every second row
        \begin{adjustbox}{max height=0.6\textheight, max width=\columnwidth}
            \begin{tabular}{l *{18}{c}}
                \toprule
                Lang.  & 
                \multicolumn{3}{c}{InternVL-2.5-4B} & 
                \multicolumn{3}{c}{InternVL-2.5-4B} & 
                \multicolumn{3}{c}{QwenVL-2.5-2B} & 
                \multicolumn{3}{c}{QwenVL-2.5-2B} & 
                \multicolumn{3}{c}{Centurio-Qwen} &
                \multicolumn{3}{c}{4o-mini} \\
                \cmidrule(lr){2-4} \cmidrule(lr){5-7} \cmidrule(lr){8-10} \cmidrule(lr){11-13} \cmidrule(lr){14-16} \cmidrule(lr){17-19}
                \textit{k} & 
                1 & 3 & 5 & 
                1 & 3 & 5 & 
                1 & 3 & 5 & 
                1 & 3 & 5 & 
                1 & 3 & 5 & 
                1 & 3 & 5 \\
                \midrule
\texttt{lao\_Laoo} & 38.6 & 52.4 & 59.0 & 61.2 & 75.9 & 82.9 & 67.8 & 82.4 & 87.9 & 55.3 & 72.5 & 80.7 & 77.5 & 91.4 & 94.8 & 83.8 & 94.4 & 97.3 \\
\texttt{lij\_Latn} & 72.5 & 89.3 & 94.3 & 80.4 & 91.7 & 96.2 & 82.6 & 95.3 & 98.1 & 83.0 & 95.5 & 98.7 & 83.8 & 95.7 & 98.0 & 88.4 & 97.3 & 99.0 \\
\texttt{lim\_Latn} & 70.2 & 89.4 & 94.8 & 78.5 & 92.0 & 95.9 & 82.1 & 95.8 & 97.8 & 80.8 & 95.2 & 98.4 & 82.7 & 96.1 & 98.2 & 87.5 & 97.5 & 99.1 \\
\texttt{lin\_Latn} & 50.9 & 67.4 & 77.3 & 61.4 & 78.3 & 84.1 & 65.4 & 84.0 & 89.6 & 62.7 & 83.8 & 91.7 & 75.4 & 89.8 & 93.3 & 80.3 & 95.7 & 98.5 \\
\texttt{lit\_Latn} & 67.2 & 87.1 & 93.8 & 81.7 & 92.4 & 96.0 & 76.6 & 92.2 & 95.9 & 77.7 & 94.9 & 98.2 & 84.2 & 94.9 & 97.2 & 91.1 & 98.6 & 99.4 \\
\texttt{lmo\_Latn} & 72.1 & 90.6 & 95.2 & 81.2 & 93.0 & 96.4 & 83.7 & 96.0 & 98.6 & 83.3 & 95.8 & 98.7 & 84.3 & 96.0 & 98.0 & 89.1 & 98.0 & 99.2 \\
\texttt{ltg\_Latn} & 60.4 & 80.0 & 89.7 & 76.4 & 90.6 & 95.2 & 73.6 & 90.8 & 95.9 & 74.0 & 93.5 & 97.9 & 79.5 & 94.3 & 97.4 & 86.3 & 96.8 & 99.1 \\
\texttt{ltz\_Latn} & 71.0 & 89.3 & 94.6 & 79.3 & 91.2 & 95.4 & 79.5 & 94.8 & 97.5 & 81.6 & 96.2 & 98.8 & 86.3 & 96.7 & 98.5 & 90.6 & 98.6 & 99.3 \\
\texttt{lua\_Latn} & 49.4 & 66.7 & 75.9 & 58.5 & 76.9 & 83.3 & 61.3 & 82.1 & 88.1 & 61.0 & 83.0 & 90.1 & 63.3 & 81.4 & 85.2 & 65.8 & 87.4 & 93.1 \\
\texttt{lug\_Latn} & 41.9 & 54.6 & 63.5 & 49.3 & 67.1 & 74.8 & 52.0 & 70.8 & 78.3 & 52.7 & 73.4 & 82.3 & 54.7 & 73.3 & 79.1 & 72.4 & 91.3 & 96.3 \\
\texttt{luo\_Latn} & 45.0 & 60.8 & 69.6 & 52.9 & 69.0 & 77.7 & 57.9 & 76.5 & 84.5 & 56.2 & 78.0 & 87.4 & 57.9 & 76.4 & 82.8 & 60.5 & 82.2 & 90.2 \\
\texttt{lus\_Latn} & 53.5 & 74.1 & 82.5 & 65.8 & 82.8 & 88.4 & 69.7 & 87.4 & 93.1 & 70.6 & 89.5 & 95.5 & 70.0 & 88.0 & 92.8 & 73.9 & 90.9 & 95.8 \\
\texttt{lvs\_Latn} & 67.8 & 86.4 & 94.2 & 83.8 & 93.0 & 96.4 & 80.5 & 94.0 & 97.1 & 80.7 & 95.2 & 98.1 & 87.2 & 95.3 & 97.8 & 91.3 & 98.4 & 99.3 \\
\texttt{mag\_Deva} & 67.2 & 86.6 & 92.9 & 84.5 & 94.7 & 97.3 & 83.0 & 95.4 & 98.1 & 81.7 & 94.9 & 98.4 & 85.4 & 96.6 & 99.0 & 91.5 & 98.4 & 99.5 \\
\texttt{mai\_Deva} & 66.4 & 86.2 & 92.0 & 83.3 & 94.5 & 97.6 & 82.8 & 95.7 & 98.2 & 82.3 & 95.1 & 98.6 & 85.4 & 96.5 & 98.7 & 91.7 & 98.4 & 99.4 \\
\texttt{mal\_Mlym} & 49.0 & 61.1 & 67.6 & 83.0 & 93.3 & 96.7 & 76.9 & 91.2 & 95.5 & 69.1 & 85.5 & 90.2 & 80.6 & 94.0 & 97.4 & 90.5 & 98.3 & 99.2 \\
\texttt{mar\_Deva} & 60.6 & 80.2 & 86.7 & 82.0 & 92.9 & 96.2 & 77.9 & 92.6 & 96.2 & 76.7 & 92.2 & 95.9 & 84.0 & 95.1 & 97.7 & 90.8 & 98.6 & 99.5 \\
\texttt{min\_Arab} & 34.5 & 42.8 & 48.7 & 47.9 & 62.6 & 69.7 & 46.0 & 59.7 & 64.3 & 36.2 & 39.2 & 38.4 & 49.6 & 64.3 & 69.9 & 62.9 & 82.2 & 89.9 \\
\texttt{min\_Latn} & 65.7 & 86.8 & 92.3 & 76.2 & 89.3 & 93.6 & 77.8 & 92.3 & 96.3 & 76.6 & 92.9 & 96.8 & 80.4 & 93.8 & 97.0 & 85.7 & 97.2 & 98.9 \\
\texttt{mkd\_Cyrl} & 76.2 & 92.7 & 96.6 & 85.2 & 93.9 & 96.8 & 83.8 & 95.8 & 98.2 & 84.7 & 96.5 & 98.9 & 87.8 & 96.2 & 98.2 & 91.7 & 98.5 & 99.4 \\
\texttt{mlt\_Latn} & 63.0 & 83.9 & 90.2 & 78.3 & 91.7 & 96.0 & 76.5 & 92.4 & 96.2 & 76.4 & 92.4 & 96.9 & 85.9 & 95.8 & 98.4 & 91.4 & 98.4 & 99.3 \\
\texttt{mni\_Beng} & 36.7 & 51.8 & 58.8 & 62.6 & 79.7 & 85.8 & 57.5 & 72.5 & 79.9 & 46.5 & 57.1 & 63.8 & 60.5 & 79.2 & 87.1 & 68.6 & 88.9 & 95.0 \\
\texttt{mos\_Latn} & 43.4 & 57.4 & 69.1 & 54.2 & 69.8 & 77.4 & 56.0 & 75.7 & 82.5 & 55.1 & 76.4 & 85.8 & 55.4 & 76.5 & 83.4 & 57.2 & 80.0 & 89.3 \\
\texttt{mri\_Latn} & 40.6 & 51.7 & 63.1 & 50.7 & 67.2 & 74.9 & 53.9 & 73.4 & 79.4 & 52.4 & 68.3 & 73.5 & 72.2 & 87.6 & 93.0 & 84.1 & 96.6 & 98.7 \\
\texttt{mya\_Mymr} & 36.4 & 44.1 & 46.7 & 70.9 & 85.2 & 91.0 & 59.1 & 72.7 & 76.9 & 40.6 & 51.7 & 56.8 & 71.2 & 84.7 & 89.7 & 90.0 & 97.9 & 99.2 \\
\texttt{nld\_Latn} & 83.0 & 95.8 & 97.9 & 85.0 & 93.9 & 96.3 & 87.5 & 97.2 & 98.7 & 88.3 & 97.4 & 98.8 & 89.4 & 96.3 & 97.9 & 91.6 & 98.4 & 99.3 \\
\texttt{nno\_Latn} & 78.5 & 93.2 & 96.0 & 83.2 & 92.4 & 95.3 & 84.5 & 96.8 & 98.7 & 86.6 & 96.9 & 98.9 & 88.0 & 97.0 & 98.7 & 91.3 & 98.5 & 99.3 \\
\texttt{nob\_Latn} & 80.1 & 94.2 & 96.6 & 83.8 & 93.2 & 96.0 & 85.8 & 96.9 & 98.3 & 87.9 & 97.2 & 98.9 & 88.9 & 96.7 & 98.4 & 91.5 & 98.5 & 99.3 \\
\texttt{npi\_Deva} & 66.1 & 87.0 & 91.5 & 83.5 & 93.6 & 97.3 & 80.8 & 95.0 & 97.9 & 78.4 & 92.7 & 96.6 & 84.3 & 96.3 & 98.1 & 91.2 & 98.3 & 99.2 \\
\texttt{nqo\_Nkoo} & 25.2 & 28.0 & 28.7 & 30.2 & 33.3 & 37.9 & 27.1 & 30.2 & 32.6 & 27.5 & 30.1 & 31.7 & 24.9 & 30.3 & 34.1 & 27.3 & 33.6 & 38.2 \\
\texttt{nso\_Latn} & 45.8 & 60.7 & 69.9 & 50.7 & 69.8 & 77.9 & 55.6 & 76.3 & 83.6 & 55.0 & 77.5 & 86.2 & 61.3 & 79.1 & 83.9 & 79.5 & 94.9 & 98.2 \\
\texttt{nus\_Latn} & 35.8 & 46.5 & 54.0 & 44.0 & 55.1 & 62.9 & 47.9 & 63.3 & 69.6 & 47.0 & 61.9 & 69.2 & 48.7 & 61.4 & 66.7 & 49.4 & 66.2 & 75.5 \\
\texttt{nya\_Latn} & 46.5 & 61.9 & 72.5 & 56.2 & 75.2 & 82.8 & 57.0 & 77.2 & 84.9 & 60.6 & 83.1 & 90.3 & 64.1 & 79.1 & 83.8 & 85.5 & 96.9 & 99.1 \\
\texttt{oci\_Latn} & 79.4 & 94.0 & 97.4 & 83.0 & 93.8 & 96.9 & 86.8 & 97.2 & 98.3 & 87.6 & 97.6 & 99.1 & 89.0 & 96.9 & 98.6 & 91.3 & 98.4 & 99.3 \\
\texttt{ory\_Orya} & 32.0 & 38.2 & 40.5 & 78.5 & 90.9 & 95.4 & 79.4 & 93.0 & 97.1 & 72.5 & 85.8 & 91.3 & 81.4 & 94.1 & 97.2 & 88.6 & 97.6 & 99.1 \\
\texttt{pag\_Latn} & 64.2 & 85.0 & 90.8 & 77.8 & 92.3 & 95.8 & 80.5 & 94.7 & 98.1 & 81.5 & 96.4 & 99.0 & 81.4 & 94.6 & 97.5 & 85.1 & 96.9 & 99.1 \\
\texttt{pan\_Guru} & 47.3 & 60.9 & 64.7 & 80.5 & 92.7 & 96.6 & 80.8 & 95.0 & 97.6 & 76.5 & 90.7 & 95.3 & 83.1 & 95.0 & 97.3 & 91.1 & 98.4 & 99.5 \\
\texttt{pap\_Latn} & 70.7 & 89.4 & 94.4 & 78.6 & 92.1 & 95.9 & 81.9 & 94.5 & 97.6 & 82.2 & 96.1 & 98.9 & 81.7 & 94.8 & 97.7 & 87.7 & 97.4 & 99.3 \\
\texttt{pbt\_Arab} & 57.2 & 78.0 & 85.7 & 76.2 & 89.6 & 94.3 & 70.1 & 87.7 & 92.2 & 66.9 & 82.8 & 88.6 & 77.0 & 91.6 & 95.4 & 90.3 & 98.0 & 99.1 \\
\texttt{pes\_Arab} & 77.8 & 94.5 & 97.3 & 86.0 & 95.3 & 97.6 & 85.8 & 97.0 & 98.5 & 84.7 & 95.7 & 98.1 & 86.9 & 96.2 & 98.3 & 91.9 & 98.6 & 99.4 \\
\texttt{plt\_Latn} & 47.0 & 61.7 & 72.3 & 55.0 & 71.8 & 78.9 & 58.4 & 78.1 & 83.8 & 59.1 & 78.3 & 86.6 & 60.6 & 78.4 & 84.8 & 88.7 & 98.2 & 99.2 \\
\texttt{pol\_Latn} & 83.8 & 95.4 & 97.4 & 86.0 & 94.6 & 96.6 & 87.9 & 96.9 & 98.3 & 89.0 & 97.7 & 99.1 & 90.3 & 96.4 & 97.8 & 91.4 & 98.3 & 99.2 \\
\texttt{por\_Latn} & 84.2 & 96.1 & 98.0 & 86.9 & 94.9 & 97.3 & 88.2 & 97.4 & 98.6 & 89.6 & 97.6 & 99.3 & 89.9 & 96.4 & 98.0 & 92.2 & 98.3 & 99.3 \\
\texttt{prs\_Arab} & 76.3 & 92.4 & 96.1 & 85.5 & 94.8 & 97.3 & 83.6 & 95.6 & 97.6 & 83.6 & 95.3 & 97.7 & 86.8 & 94.9 & 97.1 & 91.6 & 98.1 & 99.1 \\
\texttt{quy\_Latn} & 48.7 & 65.5 & 76.2 & 58.6 & 76.1 & 83.6 & 60.7 & 80.5 & 87.6 & 59.4 & 79.2 & 86.5 & 63.6 & 82.5 & 88.6 & 67.7 & 87.4 & 92.9 \\
\texttt{ron\_Latn} & 79.8 & 95.3 & 97.8 & 86.6 & 94.0 & 96.9 & 86.0 & 97.1 & 98.6 & 88.5 & 98.1 & 99.0 & 88.2 & 96.3 & 98.2 & 91.6 & 98.3 & 99.4 \\
\texttt{run\_Latn} & 39.9 & 53.0 & 63.5 & 49.3 & 65.9 & 73.6 & 51.6 & 69.3 & 77.9 & 51.6 & 71.0 & 80.5 & 55.2 & 73.8 & 79.4 & 85.7 & 97.2 & 99.4 \\
\texttt{rus\_Cyrl} & 84.8 & 96.1 & 98.3 & 86.2 & 94.6 & 96.4 & 87.7 & 97.7 & 98.8 & 90.2 & 98.2 & 99.4 & 90.2 & 97.1 & 98.4 & 91.8 & 98.5 & 99.4 \\
\texttt{sag\_Latn} & 45.1 & 60.4 & 70.5 & 56.1 & 72.4 & 79.4 & 57.5 & 79.4 & 85.5 & 57.2 & 77.3 & 85.5 & 65.3 & 83.2 & 88.4 & 61.1 & 83.6 & 90.6 \\
\texttt{san\_Deva} & 53.5 & 72.1 & 79.2 & 75.0 & 87.0 & 91.8 & 75.1 & 90.1 & 94.0 & 74.5 & 88.3 & 92.4 & 81.5 & 94.0 & 96.7 & 86.5 & 97.5 & 99.2 \\
\texttt{sat\_Olck} & 25.5 & 26.1 & 28.1 & 29.3 & 29.8 & 33.0 & 26.0 & 27.0 & 27.9 & 31.9 & 27.4 & 29.3 & 29.1 & 28.7 & 31.6 & 25.5 & 29.1 & 29.9 \\
\texttt{scn\_Latn} & 69.0 & 88.9 & 94.0 & 79.3 & 92.6 & 96.8 & 82.1 & 96.1 & 98.3 & 80.9 & 95.1 & 98.5 & 85.6 & 95.8 & 98.5 & 89.2 & 98.0 & 99.5 \\
\texttt{shn\_Mymr} & 40.5 & 55.1 & 60.8 & 52.6 & 70.5 & 79.2 & 53.7 & 73.2 & 81.8 & 51.9 & 75.8 & 84.0 & 50.6 & 72.1 & 82.7 & 52.7 & 76.8 & 85.3 \\
\texttt{sin\_Sinh} & 28.3 & 33.5 & 35.0 & 64.1 & 78.4 & 85.1 & 61.8 & 75.5 & 80.5 & 33.8 & 35.8 & 38.3 & 68.2 & 82.7 & 87.3 & 89.2 & 97.8 & 99.3 \\
\texttt{slk\_Latn} & 79.9 & 94.1 & 97.4 & 84.0 & 93.2 & 95.9 & 86.6 & 97.1 & 98.5 & 87.2 & 97.5 & 99.1 & 88.4 & 95.5 & 97.4 & 91.1 & 98.3 & 99.2 \\
\texttt{slv\_Latn} & 75.9 & 93.0 & 97.2 & 83.1 & 93.0 & 96.2 & 83.7 & 96.6 & 98.5 & 85.9 & 97.1 & 99.2 & 86.2 & 95.7 & 98.1 & 91.1 & 98.4 & 99.3 \\
\texttt{smo\_Latn} & 46.6 & 61.3 & 71.3 & 55.7 & 73.0 & 82.0 & 58.3 & 77.7 & 84.5 & 57.7 & 77.2 & 86.9 & 72.1 & 88.2 & 92.5 & 86.7 & 97.1 & 99.3 \\
\texttt{sna\_Latn} & 44.5 & 59.0 & 69.6 & 51.0 & 69.0 & 77.1 & 54.7 & 73.2 & 80.7 & 54.1 & 76.3 & 85.1 & 60.6 & 76.5 & 81.4 & 88.2 & 97.5 & 99.2 \\
\texttt{snd\_Arab} & 45.2 & 62.9 & 71.4 & 71.1 & 87.2 & 94.2 & 70.5 & 87.3 & 92.2 & 69.5 & 82.9 & 88.5 & 81.7 & 93.7 & 96.4 & 89.7 & 98.4 & 99.3 \\
\texttt{som\_Latn} & 42.9 & 54.2 & 66.3 & 50.3 & 66.5 & 74.1 & 54.0 & 72.6 & 81.0 & 51.5 & 68.4 & 78.4 & 65.4 & 82.0 & 87.6 & 84.7 & 97.0 & 99.2 \\
\texttt{sot\_Latn} & 45.5 & 59.3 & 69.5 & 52.6 & 70.1 & 77.9 & 56.0 & 76.5 & 83.7 & 57.2 & 79.2 & 87.3 & 62.6 & 79.5 & 84.2 & 83.4 & 96.2 & 98.9 \\
\texttt{spa\_Latn} & 85.4 & 96.5 & 98.3 & 86.0 & 94.3 & 96.5 & 89.2 & 97.5 & 98.8 & 90.1 & 97.8 & 99.1 & 89.5 & 96.5 & 97.7 & 92.0 & 98.5 & 99.2 \\
\texttt{srd\_Latn} & 70.5 & 89.9 & 95.2 & 79.5 & 91.4 & 95.5 & 81.3 & 94.7 & 97.3 & 81.9 & 96.3 & 98.8 & 84.8 & 95.7 & 98.5 & 86.7 & 97.2 & 99.0 \\
\texttt{srp\_Cyrl} & 77.3 & 93.2 & 97.2 & 86.1 & 94.4 & 97.4 & 84.8 & 96.2 & 98.0 & 85.6 & 96.5 & 99.0 & 88.2 & 96.5 & 98.2 & 91.7 & 98.3 & 99.4 \\
\texttt{ssw\_Latn} & 41.1 & 53.1 & 62.6 & 47.2 & 67.2 & 74.1 & 52.8 & 72.4 & 80.4 & 53.3 & 74.6 & 83.4 & 68.8 & 85.7 & 91.3 & 80.9 & 96.1 & 98.7 \\
\texttt{sun\_Latn} & 68.0 & 87.8 & 93.5 & 79.6 & 91.3 & 95.3 & 80.6 & 94.3 & 97.2 & 80.1 & 94.7 & 98.3 & 82.6 & 95.2 & 98.0 & 90.3 & 98.6 & 99.3 \\
\texttt{swe\_Latn} & 80.0 & 94.4 & 97.2 & 84.2 & 93.8 & 96.2 & 86.2 & 96.7 & 99.0 & 87.9 & 97.3 & 98.7 & 89.5 & 96.7 & 98.3 & 91.6 & 98.2 & 99.1 \\
\texttt{swh\_Latn} & 46.1 & 60.0 & 70.5 & 68.3 & 84.4 & 90.4 & 62.6 & 82.5 & 89.3 & 66.8 & 86.2 & 94.0 & 75.7 & 91.5 & 95.6 & 90.8 & 98.5 & 99.2 \\
\texttt{szl\_Latn} & 72.6 & 90.0 & 94.7 & 80.0 & 91.8 & 95.4 & 81.8 & 95.4 & 97.4 & 81.9 & 95.8 & 98.5 & 84.4 & 95.5 & 98.5 & 89.2 & 98.1 & 99.3 \\
\texttt{tam\_Taml} & 35.0 & 45.1 & 50.1 & 77.2 & 88.6 & 93.1 & 72.8 & 88.5 & 92.1 & 67.8 & 81.9 & 86.7 & 80.9 & 92.1 & 95.3 & 89.7 & 98.4 & 99.4 \\
\texttt{taq\_Latn} & 47.1 & 62.5 & 72.9 & 58.2 & 72.3 & 80.1 & 59.6 & 77.2 & 84.8 & 58.6 & 76.7 & 85.7 & 60.6 & 79.4 & 85.0 & 64.1 & 83.6 & 91.0 \\
\texttt{taq\_Tfng} & 25.9 & 32.4 & 34.5 & 28.6 & 34.0 & 37.0 & 33.7 & 42.0 & 47.7 & 28.3 & 36.2 & 42.6 & 27.8 & 35.6 & 39.7 & 31.9 & 42.3 & 49.5 \\
\texttt{tat\_Cyrl} & 58.4 & 76.3 & 84.9 & 79.4 & 90.7 & 95.9 & 73.6 & 90.2 & 94.4 & 71.4 & 88.5 & 93.5 & 78.9 & 93.5 & 96.5 & 90.5 & 98.5 & 99.6 \\
\texttt{tel\_Telu} & 43.3 & 54.4 & 59.1 & 80.2 & 92.3 & 95.8 & 79.4 & 94.1 & 96.5 & 74.5 & 91.5 & 96.3 & 84.1 & 95.5 & 97.7 & 90.5 & 98.0 & 99.4 \\
\texttt{tgk\_Cyrl} & 49.3 & 68.5 & 76.4 & 72.0 & 88.1 & 94.0 & 67.9 & 86.6 & 92.3 & 62.8 & 78.2 & 84.1 & 75.2 & 90.7 & 95.1 & 90.2 & 98.1 & 99.5 \\
\texttt{tgl\_Latn} & 65.3 & 84.0 & 90.6 & 82.8 & 94.3 & 97.2 & 83.3 & 95.2 & 97.7 & 86.8 & 96.6 & 98.7 & 86.0 & 96.3 & 98.3 & 91.1 & 98.2 & 99.2 \\
\texttt{tha\_Thai} & 81.9 & 94.7 & 96.9 & 87.6 & 95.8 & 98.1 & 88.0 & 97.3 & 98.4 & 83.5 & 95.2 & 97.8 & 90.9 & 97.3 & 98.6 & 92.0 & 98.1 & 99.2 \\
\texttt{tir\_Ethi} & 24.2 & 27.7 & 30.5 & 34.8 & 43.0 & 50.0 & 44.9 & 59.2 & 66.3 & 26.5 & 31.2 & 31.7 & 64.0 & 81.1 & 86.4 & 73.0 & 90.1 & 94.5 \\
\texttt{tpi\_Latn} & 67.6 & 83.9 & 90.3 & 78.4 & 90.8 & 94.7 & 79.8 & 95.0 & 98.1 & 81.7 & 95.4 & 99.1 & 86.1 & 96.6 & 99.1 & 88.6 & 98.1 & 99.4 \\
\texttt{tsn\_Latn} & 43.0 & 58.7 & 69.1 & 49.0 & 67.9 & 75.4 & 53.9 & 73.9 & 80.3 & 55.2 & 76.8 & 85.0 & 63.1 & 80.7 & 85.8 & 78.0 & 94.2 & 97.7 \\
\texttt{tso\_Latn} & 42.6 & 54.7 & 65.4 & 49.1 & 67.6 & 76.2 & 53.3 & 72.7 & 79.8 & 53.8 & 74.5 & 83.3 & 68.9 & 84.8 & 89.7 & 78.4 & 92.8 & 97.6 \\
\texttt{tuk\_Latn} & 54.7 & 72.9 & 82.7 & 73.2 & 88.0 & 94.6 & 70.1 & 88.9 & 94.8 & 71.2 & 87.8 & 93.9 & 73.4 & 89.9 & 94.1 & 89.1 & 98.3 & 99.5 \\
\texttt{tum\_Latn} & 44.1 & 58.9 & 69.1 & 55.1 & 73.8 & 81.0 & 57.0 & 76.6 & 83.3 & 58.6 & 78.8 & 86.9 & 62.7 & 78.8 & 83.9 & 80.7 & 95.0 & 98.5 \\
\texttt{tur\_Latn} & 76.4 & 92.5 & 95.6 & 85.0 & 94.9 & 97.2 & 85.1 & 96.6 & 98.4 & 87.5 & 97.2 & 98.8 & 88.2 & 96.0 & 97.6 & 91.5 & 98.6 & 99.4 \\
\texttt{twi\_Latn} & 48.7 & 65.5 & 75.6 & 57.8 & 73.1 & 79.7 & 61.1 & 79.2 & 84.9 & 62.0 & 81.2 & 88.9 & 64.1 & 79.8 & 84.8 & 80.8 & 94.0 & 97.7 \\
\texttt{tzm\_Tfng} & 25.9 & 30.1 & 31.5 & 24.8 & 27.2 & 28.8 & 30.4 & 34.4 & 37.2 & 24.9 & 29.8 & 32.4 & 27.9 & 33.7 & 35.7 & 29.0 & 35.2 & 39.4 \\
\texttt{uig\_Arab} & 40.3 & 51.9 & 55.8 & 70.7 & 88.4 & 93.4 & 61.6 & 81.4 & 88.2 & 72.6 & 88.6 & 94.5 & 73.1 & 87.0 & 91.7 & 89.5 & 98.6 & 99.5 \\
\texttt{ukr\_Cyrl} & 80.3 & 94.5 & 97.3 & 85.9 & 94.2 & 97.1 & 86.4 & 97.4 & 98.4 & 87.7 & 96.7 & 98.6 & 89.6 & 96.0 & 98.2 & 91.5 & 98.2 & 99.4 \\
\texttt{umb\_Latn} & 41.5 & 55.0 & 65.7 & 47.4 & 66.7 & 74.6 & 53.4 & 73.5 & 79.9 & 54.3 & 75.1 & 83.2 & 55.4 & 75.1 & 81.0 & 58.6 & 81.6 & 89.7 \\
\texttt{urd\_Arab} & 65.9 & 82.2 & 89.0 & 83.3 & 93.4 & 97.5 & 81.5 & 94.3 & 97.1 & 79.7 & 92.3 & 96.4 & 85.6 & 95.1 & 96.9 & 91.8 & 98.3 & 99.3 \\
\texttt{uzn\_Latn} & 55.7 & 72.7 & 81.0 & 75.7 & 90.1 & 94.4 & 73.5 & 89.7 & 94.2 & 70.2 & 89.4 & 95.6 & 82.4 & 94.4 & 97.3 & 91.2 & 98.7 & 99.5 \\
\texttt{vec\_Latn} & 75.3 & 92.6 & 96.4 & 81.8 & 92.9 & 96.3 & 85.9 & 96.5 & 98.5 & 87.1 & 96.9 & 98.6 & 87.7 & 96.4 & 98.7 & 89.8 & 97.9 & 99.2 \\
\texttt{vie\_Latn} & 83.0 & 95.9 & 97.9 & 86.3 & 95.3 & 97.6 & 88.2 & 97.1 & 99.0 & 87.8 & 97.2 & 99.1 & 90.0 & 96.5 & 98.3 & 92.6 & 98.5 & 99.3 \\
\texttt{war\_Latn} & 68.7 & 87.1 & 93.1 & 80.5 & 92.9 & 96.7 & 81.6 & 94.9 & 97.9 & 83.5 & 96.4 & 98.8 & 86.5 & 96.0 & 98.5 & 89.9 & 98.2 & 99.0 \\
\texttt{wol\_Latn} & 50.4 & 66.5 & 77.5 & 59.1 & 77.1 & 84.3 & 63.3 & 84.5 & 90.4 & 64.0 & 84.9 & 92.7 & 65.5 & 85.0 & 91.1 & 69.5 & 89.3 & 94.9 \\
\texttt{xho\_Latn} & 45.8 & 60.6 & 70.3 & 54.0 & 71.2 & 79.1 & 57.2 & 77.4 & 84.0 & 60.5 & 80.6 & 88.3 & 72.7 & 87.6 & 92.3 & 87.5 & 98.0 & 99.2 \\
\texttt{ydd\_Hebr} & 32.7 & 44.4 & 51.8 & 59.9 & 77.4 & 84.9 & 53.6 & 73.4 & 82.1 & 33.9 & 39.9 & 41.5 & 66.1 & 84.8 & 90.1 & 87.4 & 97.9 & 99.1 \\
\texttt{yor\_Latn} & 38.7 & 51.4 & 61.3 & 48.2 & 62.6 & 70.1 & 53.9 & 70.1 & 78.6 & 53.6 & 72.2 & 81.1 & 61.4 & 78.2 & 82.4 & 83.0 & 95.6 & 97.9 \\
\texttt{yue\_Hant} & 85.1 & 96.6 & 98.0 & 87.0 & 95.1 & 97.3 & 88.9 & 97.3 & 98.8 & 90.0 & 98.0 & 98.5 & 89.8 & 96.1 & 97.1 & 91.7 & 98.5 & 99.3 \\
\texttt{zho\_Hans} & 86.5 & 97.1 & 98.5 & 87.9 & 95.1 & 97.4 & 88.7 & 97.3 & 98.8 & 90.9 & 97.7 & 98.3 & 90.3 & 96.3 & 98.0 & 91.9 & 98.3 & 99.1 \\
\texttt{zho\_Hant} & 86.3 & 96.8 & 98.1 & 87.7 & 94.9 & 97.4 & 88.9 & 97.3 & 99.0 & 90.7 & 98.1 & 98.9 & 90.4 & 96.0 & 97.5 & 92.2 & 98.3 & 99.2 \\
\texttt{zsm\_Latn} & 79.8 & 95.3 & 97.4 & 85.1 & 94.9 & 97.4 & 85.0 & 96.2 & 97.9 & 86.7 & 97.4 & 98.8 & 87.8 & 96.5 & 98.0 & 91.8 & 98.0 & 99.3 \\
\texttt{zul\_Latn} & 40.3 & 52.8 & 63.4 & 49.1 & 66.8 & 74.9 & 52.6 & 72.1 & 78.5 & 54.0 & 76.2 & 84.2 & 71.1 & 86.0 & 91.2 & 88.6 & 97.6 & 99.1 \\

                \bottomrule
            \end{tabular}
        \end{adjustbox}
    \end{minipage}
\end{table*}


\end{document}
