\documentclass[letterpaper, 10 pt, conference]{ieeeconf}  % Comment this line out if you need a4paper

% \documentclass[a4paper, 10pt, conference]{ieeeconf}      % Use this line for a4 paper

\IEEEoverridecommandlockouts                              % This command is only needed if 
                                                          % you want to use the \thanks command

\overrideIEEEmargins                                      % Needed to meet printer requirements.

%In case you encounter the following error:
%Error 1010 The PDF file may be corrupt (unable to open PDF file) OR
%Error 1000 An error occurred while parsing a contents stream. Unable to analyze the PDF file.
%This is a known problem with pdfLaTeX conversion filter. The file cannot be opened with acrobat reader
%Please use one of the alternatives below to circumvent this error by uncommenting one or the other
%\pdfobjcompresslevel=0
%\pdfminorversion=4

% See the \addtolength command later in the file to balance the column lengths
% on the last page of the document
% The following packages can be found on http:\\www.ctan.org
\usepackage{listings}
\usepackage{amssymb}
\usepackage{caption}
\usepackage{multirow}
\usepackage{wrapfig} % To wrap text around figures
\usepackage{graphicx} % To include graphics
\usepackage{lipsum} % For placeholder text
% \usepackage[numbers]{natbib}
% \usepackage[margin=0.75in,top=1in]{geometry}
\usepackage[product-units=repeat]{siunitx}
 \usepackage{xcolor}
\usepackage{graphicx}
\usepackage{amsmath}
\usepackage{cite}
\usepackage{subfig}
%\usepackage{epsfig} % for postscript graphics files
%\usepackage{mathptmx} % assumes new font selection scheme installed
%\usepackage{times} % assumes new font selection scheme installed
%\usepackage{amsmath} % assumes amsmath package installed
%\usepackage{amssymb}  % assumes amsmath package installed
\usepackage{hyperref}

\title{\LARGE \bf
%Integrating Human-Like Teleoperation for Robotic Arms
% suggested Title below I think is more descriptive and suggestive of your
% novel approach to a challenging problem
%Learning Human-Like Motion and End Effector Control for Teleoperated Redundant Degrees-of-Freedom Robotic Arms

%Crafting Human-Like Motion in Teleoperated Robotic Arms"**

%Art of Mimicry: Redundant Manipulators Learning from Human Motion for Teleoperation

From Human Hands to Robotic Limbs: A Study in Motor Skill Embodiment for Telemanipulation 
}



\author{Haoyi~Shi$^{1}$, Mingxi~Su$^{1}$, Ted~Morris$^{1}$,  Vassilios~Morellas$^{1}$, and Nikolaos~Papanikolopoulos$^{1}$%
\thanks{$^{1}$Haoyi~Shi, Mingxi~Su, Ted~Morris, Vassilios~Morellas, and Nikolaos~Papanikolopoulos are with the Minnesota Robotics Institute (MNRI), University of Minnesota
{\tt\footnotesize $\{$shi00317 $|$  su000111 $|$ tmorris $|$ morellas $|$ papan001$\}$@umn.edu}}%
\thanks{\url{https://sites.google.com/umn.edu/teleop/home}}%
}

\begin{document}


\maketitle
\thispagestyle{empty}
\pagestyle{empty}
%%%%%%%%%%%%%%%%%%%%%%%%%%%%%%%%%%%%%%%%%%%%%%%%%%%%%%%%%%%%%%%%%%%%%%%%%%%%%%%%
\begin{abstract}

This paper presents a teleoperation system for controlling a redundant degree-of-freedom (DOF) robot manipulator using human arm gestures. We propose a GRU-based Variational Autoencoder (VAE) to learn a latent representation of the manipulator’s configuration space, capturing its complex joint kinematics. A fully-connected neural network maps human arm configurations into this latent space, allowing the system to mimic and generate corresponding manipulator trajectories in real-time through the VAE decoder. The proposed method shows promising results in teleoperating the manipulator, enabling the generation of novel manipulator configurations from human gestures that were not present during training. Experimental evaluation highlights the effectiveness of this approach, although limitations remain concerning training data diversity.

\end{abstract}


%%%%%%%%%%%%%%%%%%%%%%%%%%%%%%%%%%%%%%%%%%%%%%%%%%%%%%%%%%%%%%%%%%%%%%%%%%%%%%%%
\vspace{-4.5mm}
\section{Introduction}
\section{Introduction}

Video generation has garnered significant attention owing to its transformative potential across a wide range of applications, such media content creation~\citep{polyak2024movie}, advertising~\citep{zhang2024virbo,bacher2021advert}, video games~\citep{yang2024playable,valevski2024diffusion, oasis2024}, and world model simulators~\citep{ha2018world, videoworldsimulators2024, agarwal2025cosmos}. Benefiting from advanced generative algorithms~\citep{goodfellow2014generative, ho2020denoising, liu2023flow, lipman2023flow}, scalable model architectures~\citep{vaswani2017attention, peebles2023scalable}, vast amounts of internet-sourced data~\citep{chen2024panda, nan2024openvid, ju2024miradata}, and ongoing expansion of computing capabilities~\citep{nvidia2022h100, nvidia2023dgxgh200, nvidia2024h200nvl}, remarkable advancements have been achieved in the field of video generation~\citep{ho2022video, ho2022imagen, singer2023makeavideo, blattmann2023align, videoworldsimulators2024, kuaishou2024klingai, yang2024cogvideox, jin2024pyramidal, polyak2024movie, kong2024hunyuanvideo, ji2024prompt}.


In this work, we present \textbf{\ours}, a family of rectified flow~\citep{lipman2023flow, liu2023flow} transformer models designed for joint image and video generation, establishing a pathway toward industry-grade performance. This report centers on four key components: data curation, model architecture design, flow formulation, and training infrastructure optimization—each rigorously refined to meet the demands of high-quality, large-scale video generation.


\begin{figure}[ht]
    \centering
    \begin{subfigure}[b]{0.82\linewidth}
        \centering
        \includegraphics[width=\linewidth]{figures/t2i_1024.pdf}
        \caption{Text-to-Image Samples}\label{fig:main-demo-t2i}
    \end{subfigure}
    \vfill
    \begin{subfigure}[b]{0.82\linewidth}
        \centering
        \includegraphics[width=\linewidth]{figures/t2v_samples.pdf}
        \caption{Text-to-Video Samples}\label{fig:main-demo-t2v}
    \end{subfigure}
\caption{\textbf{Generated samples from \ours.} Key components are highlighted in \textcolor{red}{\textbf{RED}}.}\label{fig:main-demo}
\end{figure}


First, we present a comprehensive data processing pipeline designed to construct large-scale, high-quality image and video-text datasets. The pipeline integrates multiple advanced techniques, including video and image filtering based on aesthetic scores, OCR-driven content analysis, and subjective evaluations, to ensure exceptional visual and contextual quality. Furthermore, we employ multimodal large language models~(MLLMs)~\citep{yuan2025tarsier2} to generate dense and contextually aligned captions, which are subsequently refined using an additional large language model~(LLM)~\citep{yang2024qwen2} to enhance their accuracy, fluency, and descriptive richness. As a result, we have curated a robust training dataset comprising approximately 36M video-text pairs and 160M image-text pairs, which are proven sufficient for training industry-level generative models.

Secondly, we take a pioneering step by applying rectified flow formulation~\citep{lipman2023flow} for joint image and video generation, implemented through the \ours model family, which comprises Transformer architectures with 2B and 8B parameters. At its core, the \ours framework employs a 3D joint image-video variational autoencoder (VAE) to compress image and video inputs into a shared latent space, facilitating unified representation. This shared latent space is coupled with a full-attention~\citep{vaswani2017attention} mechanism, enabling seamless joint training of image and video. This architecture delivers high-quality, coherent outputs across both images and videos, establishing a unified framework for visual generation tasks.


Furthermore, to support the training of \ours at scale, we have developed a robust infrastructure tailored for large-scale model training. Our approach incorporates advanced parallelism strategies~\citep{jacobs2023deepspeed, pytorch_fsdp} to manage memory efficiently during long-context training. Additionally, we employ ByteCheckpoint~\citep{wan2024bytecheckpoint} for high-performance checkpointing and integrate fault-tolerant mechanisms from MegaScale~\citep{jiang2024megascale} to ensure stability and scalability across large GPU clusters. These optimizations enable \ours to handle the computational and data challenges of generative modeling with exceptional efficiency and reliability.


We evaluate \ours on both text-to-image and text-to-video benchmarks to highlight its competitive advantages. For text-to-image generation, \ours-T2I demonstrates strong performance across multiple benchmarks, including T2I-CompBench~\citep{huang2023t2i-compbench}, GenEval~\citep{ghosh2024geneval}, and DPG-Bench~\citep{hu2024ella_dbgbench}, excelling in both visual quality and text-image alignment. In text-to-video benchmarks, \ours-T2V achieves state-of-the-art performance on the UCF-101~\citep{ucf101} zero-shot generation task. Additionally, \ours-T2V attains an impressive score of \textbf{84.85} on VBench~\citep{huang2024vbench}, securing the top position on the leaderboard (as of 2025-01-25) and surpassing several leading commercial text-to-video models. Qualitative results, illustrated in \Cref{fig:main-demo}, further demonstrate the superior quality of the generated media samples. These findings underscore \ours's effectiveness in multi-modal generation and its potential as a high-performing solution for both research and commercial applications.
\vspace{-3.5mm}

\section{related work}
\vspace{-2.5mm}
One approach to achieve high-DoF manipulator teleoperation is Master-Slave or Twin-Master, where the operator manually controls another manipulator with identical kinematics to the target manipulator~\cite{singh_haptic-guided_2020,su_heterogeneous_2021}. However, this method requires an additional manipulator with the same kinematic structure as the remotely controlled device. This requires either two identical manipulators or building a custom one. Both options are costly and limit the manipulator system's adaptability to other types of manipulators. 

A Human arm is defined as a $7$-DOF kinematic structure~\cite{prokopenko_assessment_2001}. Hence, teleoperating a high-DOF manipulator by mapping the human arm joints to the target manipulator is another approach. Previous work~\cite{penco_robust_2018}, achieved humanoid robot ($7$-DOF arm) teleoperation by attaching IMU sensors to the human operator and mapping whole-body joints with dynamic filters between the operator and robot. However, a limitation of this method is that the required robot kinematic structure must be similar to the human arm kinematic structure. Other strategies have been explored to simplify the kinematic representation of the human arm for teleoperation~\cite{ajoudani_reduced-complexity_2018,su_deep_2019,arduengo_human_2021}. For example,~\cite{su_deep_2019} introduced elbow elevation angle as a constraint for a human arm mapped on the robot as a swivel motion. 
% using captured operator's elbow elevation angle by RGBD camera sensor and target hand 6-Dof pose as input for neural network module to generate mapped 7-Dof manipulator joints configuration.
~\cite{arduengo_human_2021} proposes a method that separates a redundant $7$-DOF manipulator as a $3$-DOF manipulator attached to a $4$-DOF end-effector. Using IMU sensors to couple the operator's hand with $3$-DOF end-effector and elbow with $4$-DOF end-effector position changes, with IK calculation for both sub-manipulators' joints in real-time to achieve teleoperation.

Our work draws inspiration from the concept of modifying kinematic representations. While most manipulators use a combination of single DOF revolute or prismatic joints, human arm articulation kinematics are represented by a combination of ball-and-socket and condyloid joints to achieve complex joint movements. Finding robot combinatorial joint kinematics that can mimic complex human articulated arm motion is an essential step in our method. Specifically, and instead of imposing explicit constraints to redundant DOF robots, we propose a generative deep learning (DL) framework to find a low-dimensional robust implicit, kinematic representation that describes complex redundant DOF robotic motion. Our framework devises a GRU-based variational autoencoder deep neural network (DNN) that generates robot trajectories which faithfully mimic human gestural intent for completing tasks.

% \subsection{Generative Model}
A Variational Autoencoder (VAE) is a neural network architecture with an encoder, a latent space, and decoder module\cite{kingma_auto-encoding_2022}. It was first introduced as an image-generative model that efficiently uses a neural network to approximate the likelihood function for the latent space distribution, which is modeled as a mixture of multiple Gaussian distributions derived from the training dataset. By sampling latent space features from Gaussian distributions, we can generate a new, unforeseen image by the decoder, which endows the characteristic features of an actual image. Such an architecture can also be used to prescribe $3$D physical motions. ~\cite{hamalainen_affordance_2019} introduced a method that integrates two Variational Autoencoder (VAE) modules: one for processing visual sensor data and the other for the manipulator's trajectory. By utilizing the latent representation from the vision module, the decoder of trajectory-VAE can generate movement paths for object-reaching tasks in the $2$D Cartesian plane.

On the other hand, the Recurrent Neural Networks (RNNs) are the most commonly used methods for sequential data prediction and feature extraction. For example, \cite{weigend_anytime_2023} utilizes Long Short-Term Memory (LSTM) as a neural network module to predict the human elbows and wrists position in order to implement an intuitive control for manipulator based on the time series data from an IMU sensor in a smartwatch.

 VAEs and RNNs can be integrated together for inference tasks. For example \cite{bowman_generating_2016}, proposed an LSTM-based VAE for missing word-imputing tasks. The LSTM allows the VAE model to consider global concepts of a sentence and generate more diverse and well-formed sentences compared with the standard RNN language model. Furthermore, in \cite{ozdemir_embodied_2021}, they also present an LSTM-based VAE framework for robot-embodied Language Learning. This allows the system to generate the correct action description by executing the action. 

The remainder of the paper is as follows. Section $3$ discusses the proposed teleoperation system, including the training dataset, the proposed GRU-based VAE model for discovering the manipulator's latent distribution space, and the fully-connected neural network module used to map human arm configurations to the manipulator's latent space, along with the model training results. Section $4$ outlines the experimental design and presents the results used to evaluate the proposed teleoperation system. Section $5$ addresses the limitations of the current method, while Section $6$ provides the conclusion. This study is under the IRB: STUDY$00009131$.

% VAE
% The entire framework have three parts. Encoder and Decoder which
% decoder: P(z) *P(X|Z) = P(X)
% encoder: P(x) * q(z|x) = q(z)


\vspace{-3.5mm}
\section{methodology}
\label{sec:methodology}
\vspace{-2mm}
\section{Methodology}

\subsection{Problem Definition}

Given a multivariate time series input $X \in \mathbb{R}^{C  \times T}$, multivariate time series forecasting tasks are designed to predict its future $F$ time steps $\hat{Y}\in \mathbb{R}^{C \times F}$ using past $T$ steps. $C $ is the number of variates or channels.

\subsection{Preliminary Analysis}

This section presents why RevIN~\citep{Kim_revin,liu2022non}, High-pass, and Low-pass filters fail to address the Mid-Frequency Spectrum Gap. Let the input univariate time series be $ x(t) $ with length $ T $ and target $ y(t) $ with length $ F $. 

\begin{definition}[Frequency Spectral Energy]\label{def:energy}
The Fourier transform of $x(t)$, $X(f)$, and its spectral energy $E_X(f)$ is given by:
\vspace{-0.2cm}
\begin{align}
X(f) = \sum_{t=0}^{T-1} x(t) e^{-i 2 \pi f t / {T-1}}, \quad &f = 0, 1, \dots, T-1\notag\\
E_X(f) = |X(f)|^2.
\end{align}
\vspace{-0.2cm}
\end{definition}

\textbf{Impact of RevIN on Frequency Spectrum \quad}
\begin{definition}[Reversible Instance Normalization]\label{def:RevIN}
Given a \textbf{forecast model} $ f: \mathbb{R}^T \rightarrow \mathbb{R}^F $ that generates a forecast $ \hat{y}(t) $ from a given input $x(t)$, RevIN is defined as:
\vspace{-0.2cm}
\begin{align}
&\hat{x}(t) = \frac{x(t) - \mu}{\sigma},\quad t = 0, 1, \dots, T-1\notag\\
&\hat{y}(t) = f(\hat{x}(t)), \quad \hat{y}(t)_{rev}= \hat{y}(t) \cdot \sigma + \mu,\notag\\
&\mu = \frac{1}{T} \sum_{t=0}^{T-1} x(t), \quad \sigma = \sqrt{\frac{1}{T} \sum_{t=0}^{T-1} (x(t) - \mu)^2}.
\end{align}
\vspace{-0.2cm}
\end{definition}

\begin{theorem} [Frequency Spectrum after RevIN] \label{theorem:RevIN}
\vspace{-0.2cm}
The spectral energy of $\hat{x}(t)$ (transformed using RevIN):
\begin{align}
E_{\hat{X}}(0)=0,& \quad f=0, \notag\\
E_{\hat{X}}(f) = \left( \frac{1}{\sigma} \right)^2 |X(f)|^2,&\quad f = 1,2,\dots, T-1 . 
\end{align}
\vspace{-0.2cm}
\end{theorem}
The proof is in Appendix~\ref{app:RevIN}. Theorem~\ref{theorem:RevIN} suggests that RevIN scales the absolute spectral energy by $ \sigma^2 $ but does not affect its relative distribution except $E_{\hat{X}}(0)=0$. Thus, RevIN preserves the relative spectral energy distribution and leaves the Mid-Frequency Spectrum Gap unresolved. \textit{However, our experiments still employ RevIN to ensure a fair comparison with other baselines.}
\begin{figure*}[h]
  \centering
  \includegraphics[width=1.\linewidth]{Faker/source/assets/jpg/ReFocus.jpg}
  \caption{General structure of \textbf{ReFocus}. `Adaptive Mid-Frequency Energy Optimizer (AMEO)' enhances mid-frequency components modeling, and `Energy-based Key-Frequency Picking Block' (EKPB) effectively captures shared Key-Frequency across channels}
  \label{fig:refocus}
\end{figure*}

\begin{figure*}[h]
  \centering
  \includegraphics[width=0.7\linewidth]{Faker/source/assets/jpg/ket.jpg}
  \caption{General process of the \textbf{Key-Frequency Enhanced Training strategy (KET)}, where spectral information from other channels is randomly introduced into each channel, to enhance the extraction of the shared Key-Frequency.}
  \label{fig:reshuffle}
\end{figure*}
\textbf{Impact of High- and Low-pass filter \quad}
We still define $\hat{x}(t)$ to be the filtered (processed) signal, obtained by applying a filter $H(f)$ (High/Low-pass filter). The filter $ H(f) $ is 1 in the passband (High/Low frequency) and 0 in the stopband (Middle frequency). So $E_{\hat{X}}(f)=0,\quad E_{\hat{X}}\leq E_X(f)$ for middle frequencies, which creates even larger gap.

\subsection{Overall Structure of The Proposed ReFocus}

In this section, we elucidate the overall architecture of \textbf{ReFocus}, depicted in Figure \ref{fig:refocus}. We define frequency domain projection as $D1\rightarrow D2$ representing a projection from dimension $D1$ to $D2$ in the frequency domain~\citep{xu2024fits}. Initially, we apply \textbf{AMEO} to the input $X \in \mathbb{R}^{C \times T}$, yielding the processed spectrum $ X_{am} \in \mathbb{R}^{C  \times T} $. Next, we use a projection $T\rightarrow D$ to transform $ X_{am}$ into the Variate Embedding $ X_{em} \in \mathbb{R}^{C  \times D}$~\citep{LiuiTransformer}. Then, $X_{em}$ go through $N$ \textbf{EKPB} to generate representation $H_{N+1}$, which is projected to obtain final prediction $\hat{Y}$. 

\textbf{Adaptive Mid-Frequency Energy Optimizer \quad}
Building upon the \textbf{Preliminary Analysis}, we propose a convolution- and residual learning-based solution to address the Mid-Frequency Spectrum Gap, which we denoted as AMEO. 
\begin{definition}[Adaptive Mid-Frequency Energy Optimizer]\label{def:AMEO}
AMEO is defined as:
\begin{align}
&\hat{x}(t) = x(t)-\frac{\beta}{K}\sum_{k=0}^{K-1} \tilde{x}(t+K-1-k),\notag\\
&\tilde{x}(t) =\notag\\
&\begin{cases}
x(t-(\frac{K}{2}+1)), \quad \text{if } \frac{K}{2}+1 \leq t < T+\frac{K}{2}+1, \\
0,  \quad\text{if } 0 \leq t < \frac{K}{2}+1 \text{ or } T+\frac{K}{2}+1 \leq t < T+K.
\end{cases}
\end{align}
\vspace{-0.2cm}
\end{definition}

It is equivalent to $x=x-\beta \cdot Conv(x)$. $Conv$ is a 1D convolution (Zero-padding at both ends, stride $s=1$, kernel size $K$, with values initialized as $ \frac{1}{K} $). $\beta \in \mathbb{R}^{1}$ is a hyperparameter.

\begin{theorem} [Frequency Spectrum after AMEO] \label{theorem:AMEO}
The spectral energy of $\hat{x}(t)$ obtained using AMEO:
\begin{align}
E_{\hat{X}}(f) =|X(f)|^2 \left\{1 - \beta \cdot \underbrace{\frac{1}{K} \sum_{k=0}^{K-1} e^{i 2 \pi f (\frac{3K}{2}-k -2) / {T-1}}}_{G(f)}\right\}^2
\end{align}
\vspace{-0.2cm}
\end{theorem}

The proof is in Appendix~\ref{app:AMEO}. We have $E_{\hat{X}}(f) =|X(f)|^2(1-\beta  \cdot G(f))^2$. Generally, $ G(f) $ behaves as a decay function, gradually reducing its value from \textbf{One} to \textbf{Zero}. Such \textbf{decay behavior} makes AMEO relatively enhances mid-frequency components, thus addressing the Mid-Frequency Spectrum Gap.

\textbf{Energy-based Key-Frequency Picking Block \quad} In each \textbf{EKPB}, the input $ H_i \in \mathbb{R}^{C  \times D} (H_1=X_{em}) $ is first processed through an MLP to generate $ H_i^k \in \mathbb{R}^{C  \times Q}$. Then, FFT is applied to get $ H_i^f \in \mathbb{R}^{C  \times (Q/2+1)}$. For $ H_i^f$, we calculate its energy, denoted as $ H_i^e \in \mathbb{R}^{C  \times (Q/2+1)}$. A cross-channel softmax is then applied to $ H_i^e$ per frequency to obtain a probability distribution $ H_i^{soft} \in \mathbb{R}^{C  \times (Q/2+1)}$. Using $H_i^{soft}$, we select values from $ H_i^f$ across channels for each frequency, resulting in $K^f_i \in \mathbb{R}^{1  \times (Q/2+1)}$, which represents the Shared Key-Frequency across all channels. Then iFFT is performed on $K^f_i$ to get $K_i\in \mathbb{R}^{1  \times Q}$, followed by projection $Q\rightarrow D$ and repeating (C times) to get $\hat{K}_i \in \mathbb{R}^{C  \times D}$. This $\hat{K}_i$ is point-wisely added to $\hat{H_i}\in \mathbb{R}^{C  \times D}$ , which is the projection of $ H_i$ using projection $D\rightarrow D$. Then, an MLP and $Add\&Norm$ is applied to the result $HK\in \mathbb{R}^{C  \times D}$ to fuse inter-series dependencies information, and another MLP and $Add\&Norm$ is used to capture intra-series variations~\citep{LiuiTransformer}. The output of each \textbf{EKPB} is $\hat{O_i} \in \mathbb{R}^{C  \times D}$, where $H_{i+1}=\hat{O_i}$.

\subsection{Key-Frequency Enhanced Training strategy}

In real-world time series, certain channels often exhibit spectral dependencies, which may not be fully captured in the training set, and the specific channels with such dependencies are also unknown~\citep{geweke1984freqchannel,Zhao2024freqchannel}. So this work borrows insight from recent advancement of mix-up in time series~\citep{zhou2023mixup,ansari2024mixup}, randomly introducing spectral information from other channels into each channel, to enhance the extraction of the shared Key-Frequency, as in Figure~\ref{fig:reshuffle}. Given a multivariate time series input $X \in \mathbb{R}^{C \times T}$ and its ground-truth $Y \in \mathbb{R}^{C \times F}$, we generate a pseudo sample pair: 

\begin{align}
X' = iFFT(FFT(X) +\alpha \cdot FFT(X[\text{perm},:]))&,  \notag\\ 
Y' = iFFT(FFT(Y) +\alpha \cdot FFT(Y[\text{perm},:]))&.
\end{align}

$\alpha \in \mathbb{R}^{C \times 1}$ is a weight vector sampled from a normal distribution, $\text{perm}$ is a reshuffled channel index. Since $FFT$ and $iFFT$ are linear operations, this mix-up process can be equivalently simplified in the \textbf{Time Domain}:
\begin{align}
X' = X +\alpha \cdot X[\text{perm},:]&,  \notag\\
Y' = Y +\alpha \cdot Y[\text{perm},:]&
 \end{align}
We alternate training between real and synthetic data to preserve the spectral dependencies in real samples. This combines the advantages of data augmentation, such as improved generalization, while mitigating potential drawbacks like over-smoothing and training instability~\citep{ryu2024tf,alkhalifah2022tf}.













\vspace{-1mm}
\section{experiment and results}
\label{sec:experiments}

The experimental setup is shown in Fig~\ref{fig:experiment}. It shows three predefined target poses and the corresponding target area, which is the gray bounding box, and the gray dot shows the center of the target area, providing visual information for participants. Each participant is required to teleoperate the redundant robot manipulator for the target pose-reaching task, which uses the end-effector's tip to reach the gray dot for each target area while making the orientation of the end-effector as perpendicular as possible to the target area surface. The manipulator begins with the fixed starting configuration. This configuration aligns with the operator’s arm being extended straight forward. The operator is required to sequentially reach each target location in numerical order, following the specified task requirements. The objective of the experiment is to test the accuracy of teleoperation and the generality of the proposed system under different operator anthropometric upper body measurements. Four participants engage in this experiment and have a mean height of $169$±$1.24$ \textit{cm} and a mean arm span of $173.9$±$3.0$ \textit{cm}. Note that participant $1$ engages in training data collection but has not been practiced for the experiment. We hold a preparation phase for each participant, including setting up the Xsens Awinda body tracking system described in~\ref{awinda_data} and measuring each participant's arm and span and full-body height for fine-tuning the body representation profile in the Xsens Awinda software. Then, we provide each participant ten minutes to move their arm to become familiar with the relation between their arm joint angles and the Kinova joint angles. In the real test procedure, each participant is required to finish the task five times. Tasks finishing time, end-effector pose, and right arm joints angle trajectory are recorded for each participant.
Table~\ref{experiment_result} presents a summary of the results. The Euclidean distance quantifies the average distance between the end-effector's tip and the target center position across trials in Cartesian space when it touches the target surface. A smaller Euclidean distance indicates higher positional accuracy. Furthermore, the orientation difference between the target pose and the instant pose when the end-effector's tip touches the target surface is represented using the cosine similarity. A value closer to $1$ indicates that the two orientations are aligned in the same direction. In all trials, participants successfully teleoperated the manipulator to reach the desired target area, with a mean absolute error $2.51$±$0.75$ \textit{cm}, and the mean cosine similarity for all target poses is $0.97$±$0.01$. This supports the generality and accuracy of the proposed system.

\begin{table}[]
\scriptsize % Reduce font size
\begin{tabular}{cllll}
\multirow{2}{*}{\textbf{Participant}} &
  \multirow{2}{*}{\textbf{Time (s)}} &
  \multicolumn{3}{l}{\textbf{\begin{tabular}[c]{@{}l@{}}Euclidean Dist. (cm)/\\ Orientation Diff. (radian)\end{tabular}}} \\ \cline{3-5} 
\noalign{\vskip 2pt}
             &              & \textbf{Target 1} & \textbf{Target 2} & \textbf{Target 3} \\[2pt] \hline
\noalign{\vskip 2pt}
\textbf{1} &
  57.39±1.85 &
  \begin{tabular}[c]{@{}l@{}}1.91±0.39/\\ 0.98±0.00\end{tabular} &
  \begin{tabular}[c]{@{}l@{}}4.09±0.51/\\ 0.97±0.00\end{tabular} &
  \begin{tabular}[c]{@{}l@{}}1.12±0.40/\\ 0.99±0.00\end{tabular} \\[5pt] \hline
\noalign{\vskip 2pt} % Space above row 2
\textbf{2} &
  55.96±5.46 &
  \begin{tabular}[c]{@{}l@{}}2.07±0.29/\\ 0.97±0.01\end{tabular} &
  \begin{tabular}[c]{@{}l@{}}1.50±0.27/\\ 0.96±0.00\end{tabular} &
  \begin{tabular}[c]{@{}l@{}}2.13±0.13/\\ 0.99±0.00\end{tabular} \\[5pt] \hline
\noalign{\vskip 2pt} % Space above row 3
\textbf{3} &
  186.66±5.98 &
  \begin{tabular}[c]{@{}l@{}}2.51±0.75/\\ 0.97±0.01\end{tabular} &
  \begin{tabular}[c]{@{}l@{}}3.53±0.55/\\ 0.99±0.00\end{tabular} &
  \begin{tabular}[c]{@{}l@{}}1.92±0.57/\\ 0.99±0.00\end{tabular} \\[5pt] \hline
\noalign{\vskip 2pt} % Space above row 4
\textbf{4} &
  138.58±27.34 &
  \begin{tabular}[c]{@{}l@{}}2.42±0.34/\\ 0.97±0.00\end{tabular} &
  \begin{tabular}[c]{@{}l@{}}3.67±0.67/\\ 0.98±0.00\end{tabular} &
  \begin{tabular}[c]{@{}l@{}}4.99±1.17/\\ 0.99±0.00\end{tabular} \\[5pt] \hline
\noalign{\vskip 2pt} % Space above row 4
\textbf{All} & 116.17±13.91 & \multicolumn{3}{c}{2.73±0.95 / 0.98±0.00}      
\end{tabular}
\captionsetup{font=footnotesize}
\caption{The standard error of the cosine similarity for orientation reported as $0.00$ is rounding and does not represent an actual zero.}
\label{experiment_result}
\vspace{-8mm}
\end{table}






\vspace{-2.8mm}
\section{limitations}
\label{sec:limitations}
\vspace{-1.9mm}

\section{Limitations}
\label{sec:limitations}

Although the proposed convolution operation is local equivariant via the restricted receptive field, when multiple layers are combined in a deep network, the full model does not become local equivariant and remains global equivariant.
However, from the experiments presented in Sec.~\ref{sub-sec:segmentation}, where the network aims to perform local predictions, our model shows robustness to such scenarios which indicates that the model relies on local features to perform the predictions. 

\vspace{-2.5mm}

\section{conclusion}
\label{sec:conclusion}
\paragraph{Summary}
Our findings provide significant insights into the influence of correctness, explanations, and refinement on evaluation accuracy and user trust in AI-based planners. 
In particular, the findings are three-fold: 
(1) The \textbf{correctness} of the generated plans is the most significant factor that impacts the evaluation accuracy and user trust in the planners. As the PDDL solver is more capable of generating correct plans, it achieves the highest evaluation accuracy and trust. 
(2) The \textbf{explanation} component of the LLM planner improves evaluation accuracy, as LLM+Expl achieves higher accuracy than LLM alone. Despite this improvement, LLM+Expl minimally impacts user trust. However, alternative explanation methods may influence user trust differently from the manually generated explanations used in our approach.
% On the other hand, explanations may help refine the trust of the planner to a more appropriate level by indicating planner shortcomings.
(3) The \textbf{refinement} procedure in the LLM planner does not lead to a significant improvement in evaluation accuracy; however, it exhibits a positive influence on user trust that may indicate an overtrust in some situations.
% This finding is aligned with prior works showing that iterative refinements based on user feedback would increase user trust~\cite{kunkel2019let, sebo2019don}.
Finally, the propensity-to-trust analysis identifies correctness as the primary determinant of user trust, whereas explanations provided limited improvement in scenarios where the planner's accuracy is diminished.

% In conclusion, our results indicate that the planner's correctness is the dominant factor for both evaluation accuracy and user trust. Therefore, selecting high-quality training data and optimizing the training procedure of AI-based planners to improve planning correctness is the top priority. Once the AI planner achieves a similar correctness level to traditional graph-search planners, strengthening its capability to explain and refine plans will further improve user trust compared to traditional planners.

\paragraph{Future Research} Future steps in this research include expanding user studies with larger sample sizes to improve generalizability and including additional planning problems per session for a more comprehensive evaluation. Next, we will explore alternative methods for generating plan explanations beyond manual creation to identify approaches that more effectively enhance user trust. 
Additionally, we will examine user trust by employing multiple LLM-based planners with varying levels of planning accuracy to better understand the interplay between planning correctness and user trust. 
Furthermore, we aim to enable real-time user-planner interaction, allowing users to provide feedback and refine plans collaboratively, thereby fostering a more dynamic and user-centric planning process.


\vspace{-4.5mm}
\section*{ACKNOWLEDGMENT}
The authors would like to thank all the members of the Center for Distributed Robotics Laboratory for their help. This work is supported by the Minnesota Robotics Institute
(MnRI) and the National Science Foundation through
grants  \#CNS-1531330, \#CNS-1919631, and \#CNS-1939033.
USDA/NIFA has also supported this work through the grants
2020-67021-30755 and 2023-67021-39829.



%%%%%%%%%%%%%%%%%%%%%%%%%%%%%%%%%%%%%%%%%%%%%%%%%%%%%%%%%%%%%%%%%%%%%%%%%%%%%%%%


\clearpage
% \newpage

\bibliographystyle{ieeetr}
\bibliography{kinova_teleop}




\end{document}
