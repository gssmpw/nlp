\documentclass[letterpaper, 10 pt, conference]{ieeeconf}  % Comment this line out if you need a4paper

% \documentclass[a4paper, 10pt, conference]{ieeeconf}      % Use this line for a4 paper

\IEEEoverridecommandlockouts                              % This command is only needed if 
                                                          % you want to use the \thanks command

\overrideIEEEmargins                                      % Needed to meet printer requirements.

%In case you encounter the following error:
%Error 1010 The PDF file may be corrupt (unable to open PDF file) OR
%Error 1000 An error occurred while parsing a contents stream. Unable to analyze the PDF file.
%This is a known problem with pdfLaTeX conversion filter. The file cannot be opened with acrobat reader
%Please use one of the alternatives below to circumvent this error by uncommenting one or the other
%\pdfobjcompresslevel=0
%\pdfminorversion=4

% See the \addtolength command later in the file to balance the column lengths
% on the last page of the document
% The following packages can be found on http:\\www.ctan.org
\usepackage{listings}
\usepackage{amssymb}
\usepackage{caption}
\usepackage{multirow}
\usepackage{wrapfig} % To wrap text around figures
\usepackage{graphicx} % To include graphics
\usepackage{lipsum} % For placeholder text
% \usepackage[numbers]{natbib}
% \usepackage[margin=0.75in,top=1in]{geometry}
\usepackage[product-units=repeat]{siunitx}
 \usepackage{xcolor}
\usepackage{graphicx}
\usepackage{amsmath}
\usepackage{cite}
\usepackage{subfig}
%\usepackage{epsfig} % for postscript graphics files
%\usepackage{mathptmx} % assumes new font selection scheme installed
%\usepackage{times} % assumes new font selection scheme installed
%\usepackage{amsmath} % assumes amsmath package installed
%\usepackage{amssymb}  % assumes amsmath package installed
\usepackage{hyperref}

\title{\LARGE \bf
%Integrating Human-Like Teleoperation for Robotic Arms
% suggested Title below I think is more descriptive and suggestive of your
% novel approach to a challenging problem
%Learning Human-Like Motion and End Effector Control for Teleoperated Redundant Degrees-of-Freedom Robotic Arms

%Crafting Human-Like Motion in Teleoperated Robotic Arms"**

%Art of Mimicry: Redundant Manipulators Learning from Human Motion for Teleoperation

From Human Hands to Robotic Limbs: A Study in Motor Skill Embodiment for Telemanipulation 
}



\author{Haoyi~Shi$^{1}$, Mingxi~Su$^{1}$, Ted~Morris$^{1}$,  Vassilios~Morellas$^{1}$, and Nikolaos~Papanikolopoulos$^{1}$%
\thanks{$^{1}$Haoyi~Shi, Mingxi~Su, Ted~Morris, Vassilios~Morellas, and Nikolaos~Papanikolopoulos are with the Minnesota Robotics Institute (MNRI), University of Minnesota
{\tt\footnotesize $\{$shi00317 $|$  su000111 $|$ tmorris $|$ morellas $|$ papan001$\}$@umn.edu}}%
\thanks{\url{https://sites.google.com/umn.edu/teleop/home}}%
}

\begin{document}


\maketitle
\thispagestyle{empty}
\pagestyle{empty}
%%%%%%%%%%%%%%%%%%%%%%%%%%%%%%%%%%%%%%%%%%%%%%%%%%%%%%%%%%%%%%%%%%%%%%%%%%%%%%%%
\begin{abstract}

This paper presents a teleoperation system for controlling a redundant degree-of-freedom (DOF) robot manipulator using human arm gestures. We propose a GRU-based Variational Autoencoder (VAE) to learn a latent representation of the manipulator’s configuration space, capturing its complex joint kinematics. A fully-connected neural network maps human arm configurations into this latent space, allowing the system to mimic and generate corresponding manipulator trajectories in real-time through the VAE decoder. The proposed method shows promising results in teleoperating the manipulator, enabling the generation of novel manipulator configurations from human gestures that were not present during training. Experimental evaluation highlights the effectiveness of this approach, although limitations remain concerning training data diversity.

\end{abstract}


%%%%%%%%%%%%%%%%%%%%%%%%%%%%%%%%%%%%%%%%%%%%%%%%%%%%%%%%%%%%%%%%%%%%%%%%%%%%%%%%
\vspace{-4.5mm}
\section{Introduction}
\section{Introduction}\label{sec:intro}

In computational finance, Monte Carlo simulations are used extensively to estimate the expected value of financial payoffs based on the solution of stochastic differential equations (SDEs) which model the evolution of stock prices, interest rates, exchange rates and other quantities \cite{glasserman04}.  Monte Carlo methods are very general and flexible, but for high accuracy it requires generating a large number of costly SDE path approximations, which has motivated research into a number of variance reduction or, equivalently, cost reduction techniques. One such method is
Multilevel Monte Carlo (MLMC), which was proposed in \cite{GILES2008} and was adapted for various applications that are summarised in \cite{Giles_overview17} and successfully combined with other methods such as quasi-Monte Carlo methods. The main idea of MLMC is to approximate the payoff using different time stepping resolutions when numerically solving the underlying SDE and to generate an optimal number of samples on each level, such that the overall computational cost is minimised subject to the desired bound on the variance. %, such that the total computational cost is minimised. 
The computational savings come from the fact that most samples are computed on the coarser levels and hence are less expensive while only a few samples from the finest levels are required \cite{GILES2008}.


Among the directions in which the computational cost 
of MLMC methods could further be reduced, an important avenue is the use of lower precision calculations, especially for the first Monte Carlo levels where the targeted accuracy is relatively low. 
 An overview of the research on mixed precision for the standard Monte Carlo (MC) framework is provided in \cite{ChowMixedPrecisionStandardMC} but only a few references study the potential of low precision computation in the MLMC framework \cite{Rounding_error_oliver}. To the best of our knowledge, the only MLMC framework with customised precision in the literature is \cite{brugger2014mixed}, but they use a uniform precision for all operations on each Monte Carlo level instead of optimising 
 the precision of each intermediary variable to reduce as much as possible the cost of path generation.
 
An important motivation for an MLMC framework with variable precision would be performing the low precision computations on reconfigurable hardware devices such as Field Programmable Gate Arrays (FPGAs). FPGAs contain customizable logic blocks and connectors that make it easy to adapt the digital circuit architecture for a specific application, leading to a highly parallel and optimised implementation. Therefore they are successfully exploited in applications that require high speed and have high computational workload, such as signal processing \cite{woods2008fpga}, and real time applications like high frequency trading \cite{HFT1,HFT2}. That is why a number of previous works in hardware architecture design implemented the MLMC algorithm to price financial options using FPGAs as accelerators, which resulted in improved speed and power efficiency compared to full CPU architectures \cite{Schryver2013AMM}. The paper \cite{lindsey2016domain} also proposed 
a Domain Specific Language to automate the configuration of FPGAs for this specific application. However, only \cite{brugger2014mixed} proposed a heuristic to reduce the precision in calculations.

In addition, all aforementioned works considered that the random number generation (RNG) is performed in single or double precision. Yet in most cases an important portion of the workload in the overall MLMC simulation comes from the RNG and in \cite{brugger2014mixed} this limited the total computational savings.
To reduce the cost of MLMC simulations in particular those based on the Geometric Brownian Motion (GBM), \cite{approximateICDF_Oliver, NestedOliver} have proposed to use approximate random numbers that are generated by applying an approximation of the inverse CDF to uniform random numbers. In \cite{NestedOliver}, the authors proposed a way to integrate these lower precision random variables into a \textit{nested} MLMC framework and completed a numerical analysis to bound the resulting error at each MC level by a product of the time step and the error in the random number approximation. The same authors show in \cite{approximateICDF_Oliver} that using approximate random variables reduces the cost of path generation by a factor 7.


In this paper we propose a nested MLMC framework that combines the use of approximate random normal variables and lower precision calculations to reduce the computational cost of MLMC even further than \cite{brugger2014mixed,NestedOliver}. We illustrate the efficiency of our framework in Matlab, after making several assumptions on the cost of operations and size of the errors that we carefully justify. We focus on the case of GBM and use the approximate RNG methods presented in \cite{approximateICDF_Oliver} as well as a new slightly modified method that combines CDF inversion and the central limit theorem. To choose the precision of the variables in the low precision path generation, we introduce a novel method to optimise the bit-widths. This optimisation is performed before the main path generation loop is executed and is based on a linear model of the payoff error  
due to rounding when computing in low precision. The error model relies on algorithmic differentiation in a similar manner to \cite{unifying-bwoptim,bitwidth-AD,ADAPT}. The bit-width optimisation procedure can be performed off-line, so this stage can be excluded from the on-line time complexity of our framework. The user specified desired accuracy is then enforced by calculating on-line the number of samples that need to be generated.

In terms of hardware design, we suggest implementing the low precision path generation on FPGAs and the full-precision ones on a CPU or GPU. 
The FPGA offers enough flexibility to define a separate bit-width for every variable in the low precision path generation, and can be reconfigured periodically to update the bit-widths when the market parameters have changed considerably. 


The paper is organized as follows : \Cref{sec:MLMC} introduces MLMC and nested MLMC to make clear the estimator that is implemented in our framework. Then in \Cref{sec:RNG} we detail the methods that could be used to obtain approximate random normally distributed numbers very cheaply for the low precision path generation. In \Cref{sec:error_model} and \Cref{sec:costModel} we propose an error model and a cost model (resp.) that we then use to formulate the optimisation problem that is solved to obtain the optimal bit-widths of fixed point variables in \Cref{sec:optimisation}. Finally we summarise our results and future directions in \Cref{sec:conclusion}.



\vspace{-3.5mm}

\section{related work}
\vspace{-2.5mm}
One approach to achieve high-DoF manipulator teleoperation is Master-Slave or Twin-Master, where the operator manually controls another manipulator with identical kinematics to the target manipulator~\cite{singh_haptic-guided_2020,su_heterogeneous_2021}. However, this method requires an additional manipulator with the same kinematic structure as the remotely controlled device. This requires either two identical manipulators or building a custom one. Both options are costly and limit the manipulator system's adaptability to other types of manipulators. 

A Human arm is defined as a $7$-DOF kinematic structure~\cite{prokopenko_assessment_2001}. Hence, teleoperating a high-DOF manipulator by mapping the human arm joints to the target manipulator is another approach. Previous work~\cite{penco_robust_2018}, achieved humanoid robot ($7$-DOF arm) teleoperation by attaching IMU sensors to the human operator and mapping whole-body joints with dynamic filters between the operator and robot. However, a limitation of this method is that the required robot kinematic structure must be similar to the human arm kinematic structure. Other strategies have been explored to simplify the kinematic representation of the human arm for teleoperation~\cite{ajoudani_reduced-complexity_2018,su_deep_2019,arduengo_human_2021}. For example,~\cite{su_deep_2019} introduced elbow elevation angle as a constraint for a human arm mapped on the robot as a swivel motion. 
% using captured operator's elbow elevation angle by RGBD camera sensor and target hand 6-Dof pose as input for neural network module to generate mapped 7-Dof manipulator joints configuration.
~\cite{arduengo_human_2021} proposes a method that separates a redundant $7$-DOF manipulator as a $3$-DOF manipulator attached to a $4$-DOF end-effector. Using IMU sensors to couple the operator's hand with $3$-DOF end-effector and elbow with $4$-DOF end-effector position changes, with IK calculation for both sub-manipulators' joints in real-time to achieve teleoperation.

Our work draws inspiration from the concept of modifying kinematic representations. While most manipulators use a combination of single DOF revolute or prismatic joints, human arm articulation kinematics are represented by a combination of ball-and-socket and condyloid joints to achieve complex joint movements. Finding robot combinatorial joint kinematics that can mimic complex human articulated arm motion is an essential step in our method. Specifically, and instead of imposing explicit constraints to redundant DOF robots, we propose a generative deep learning (DL) framework to find a low-dimensional robust implicit, kinematic representation that describes complex redundant DOF robotic motion. Our framework devises a GRU-based variational autoencoder deep neural network (DNN) that generates robot trajectories which faithfully mimic human gestural intent for completing tasks.

% \subsection{Generative Model}
A Variational Autoencoder (VAE) is a neural network architecture with an encoder, a latent space, and decoder module\cite{kingma_auto-encoding_2022}. It was first introduced as an image-generative model that efficiently uses a neural network to approximate the likelihood function for the latent space distribution, which is modeled as a mixture of multiple Gaussian distributions derived from the training dataset. By sampling latent space features from Gaussian distributions, we can generate a new, unforeseen image by the decoder, which endows the characteristic features of an actual image. Such an architecture can also be used to prescribe $3$D physical motions. ~\cite{hamalainen_affordance_2019} introduced a method that integrates two Variational Autoencoder (VAE) modules: one for processing visual sensor data and the other for the manipulator's trajectory. By utilizing the latent representation from the vision module, the decoder of trajectory-VAE can generate movement paths for object-reaching tasks in the $2$D Cartesian plane.

On the other hand, the Recurrent Neural Networks (RNNs) are the most commonly used methods for sequential data prediction and feature extraction. For example, \cite{weigend_anytime_2023} utilizes Long Short-Term Memory (LSTM) as a neural network module to predict the human elbows and wrists position in order to implement an intuitive control for manipulator based on the time series data from an IMU sensor in a smartwatch.

 VAEs and RNNs can be integrated together for inference tasks. For example \cite{bowman_generating_2016}, proposed an LSTM-based VAE for missing word-imputing tasks. The LSTM allows the VAE model to consider global concepts of a sentence and generate more diverse and well-formed sentences compared with the standard RNN language model. Furthermore, in \cite{ozdemir_embodied_2021}, they also present an LSTM-based VAE framework for robot-embodied Language Learning. This allows the system to generate the correct action description by executing the action. 

The remainder of the paper is as follows. Section $3$ discusses the proposed teleoperation system, including the training dataset, the proposed GRU-based VAE model for discovering the manipulator's latent distribution space, and the fully-connected neural network module used to map human arm configurations to the manipulator's latent space, along with the model training results. Section $4$ outlines the experimental design and presents the results used to evaluate the proposed teleoperation system. Section $5$ addresses the limitations of the current method, while Section $6$ provides the conclusion. This study is under the IRB: STUDY$00009131$.

% VAE
% The entire framework have three parts. Encoder and Decoder which
% decoder: P(z) *P(X|Z) = P(X)
% encoder: P(x) * q(z|x) = q(z)


\vspace{-3.5mm}
\section{methodology}
\label{sec:methodology}
\vspace{-2mm}
% introduce PDDL domains
% why Gripper env as testing context
% motivation: comparing classical vs LLM planners
% - classical: PDDL solver fast-downward
% - LLM: gpt-4o
% explanation and refinement are two distinguishing features of LLM planners
% - how we demonstrate explanation and refinement in the study
We evaluate user trust in two planners over a set of planning problems and study the potential factors influencing user trust in the planners. In particular, we compare a language-model-based planner, denoted as an \emph{LLM Planner}, with a traditional graph-search-based planner, denoted as a \emph{PDDL Solver}. The PDDL Solver uses Fast Downwards \cite{fastdownward} as its underlying model, processing planning problems described in PDDL to generate an optimal solution. In comparison, the LLM Planner employs GPT-4o to interpret the planning problem and extract a solution generated by the language model. Unlike the PDDL Solver, the LLM Planner can reason through the planning problem, explain its proposed solution, and iteratively refine the solution based on external feedback. This study investigates how the correctness of solutions, the quality of explanations, and the refinement process influence user trust.

\subsection{Planning Problem}
% \begin{wrapfigure}{r}{0.4\textwidth}
% % \begin{figure}[t]
%     \centering
%     \includegraphics[width=\linewidth]{figures/problem-example.pdf}
%     \caption{A running example of a planning problem in our study.}
%     \Description{Planning Problem Example}
%     \label{fig: problem-example}
% % \end{figure}
% \end{wrapfigure}

We describe each planning problem in the \emph{Planning Domain Definition Language (PDDL)} and propose two planners to generate plans that solve the problem. We select the \emph{gripper} planning problems from the International Planning Competition \cite{IPC} for plan generation and evaluation. In a gripper planning problem, a robot moves balls between a set of rooms using two grippers. The objective is to create a plan for the robot to move the balls to the target rooms we defined. We present a few running examples of the gripper problem in Figure \ref{fig: correctness}.

A planning problem consists of a \emph{planning domain} and a \emph{problem description}, expressed in PDDL. 

\paragraph{Planning Domain}
A planning domain refers to the universal aspects of a problem that remains consistent across different instances of the problem. In particular, it defines the types of objects, predicates, and actions that exist in the planning problem. We present an example of the gripper problem in Appendix \ref{app: grippers}.

\paragraph{Problem Description} A problem description specifies the particular instance of a planning task within a given domain. It includes the planning domain to which it pertains, a set of objects, the initial state of these objects, and the goal state to be achieved.

\paragraph{Plan}
A plan is a sequence of actions with specific input parameters. Recall that an action corresponds to a state transition. If a plan (a sequence of actions) transits from the initial state to the goal state defined by a problem, then we consider the plan to be \emph{correct}. If a plan does not transit to the goal state or there exists an action violating its precondition, then the plan is \emph{wrong}.

\begin{figure}[t]
    \centering
    \includegraphics[width=0.8\linewidth]{figures/correct.jpeg}
    \caption{Examples where LLM Planner correctly generates a plan for the gripper planning problem.}
    \Description{Planning Problem Correctness}
    \label{fig: correct}
\end{figure}

\subsection{PDDL Solver}
The PDDL Solver takes the planning domain and the problem description as inputs and then generates a plan described in PDDL. 
% It generates a plan in the following format:
% \vspace{4pt}
% \begin{lstlisting}[language=completion]
% (move robot1 room1 room3)
% (pick robot1 ball2 room3 rgripper1)
% (move robot1 room3 room2) ......
% \end{lstlisting}
Next, we convert the generated plan into natural language for user studies following the procedure in \cite{seipp-et-al-zenodo2022} and display it to users. We present an example in Figure \ref{fig: correct}.

The PDDL Solver applies a graph search algorithm to find a path (i.e., a list of transitions) from the initial state to the goal state. It either generates a \emph{correct} plan---defined as the shortest path between the initial and goal states---or returns a signal indicating that no solution exists for the given problem.

\subsection{LLM Planner}

The LLM Planner addresses planning problems by querying a large language model. In particular, it transmits the planning domain and problem description to the language model using a structured prompt format. The planner then retrieves a natural language plan from the language model. We use GPT-4o as the language model for the planner. To ensure the output adheres to the desired format, we include a few in-context examples within the prompts.

A language model solves a planning problem by interpreting the domain and problem descriptions, simulating state transitions, and generating a sequence of actions to achieve the goal. While effective for reasoning and plan generation, language models may struggle with large state spaces. Unlike the PDDL Solver, the LLM Planner may generate \emph{incorrect} plans that violate the problem specifications (e.g., preconditions of actions) or fail to achieve the goal.

\subsection{Explanation and Refinement}
Alongside the generated plans, we offer detailed explanations of all the plans and revisions of any incorrect plans. This study examines how these explanations and refinements influence human trust in the two planners.

\paragraph{LLM Planner with Explanation (LLM+Expl)}
For each generated plan, we manually provide a natural language explanation. This explanation includes an assessment of the plan’s correctness, identification of any violations of action preconditions, and an analysis of inconsistencies between the final state achieved and the intended goal state. We present examples of explanations in Figure \ref{fig: explain} in Appendix.

In particular, if a plan is correct, the explanation is simply ``the plan successfully satisfies the goal conditions.'' 
If a plan is incorrect, we identify the underlying cause as either a violation of action preconditions or a failure to achieve the goal state. In cases involving precondition violations, we specify the action responsible for the issue. For example, consider the action ``robot moves from room 1 to room 2,'' but the robot is initially located in room 3. This scenario constitutes a violation of the precondition for the ``move'' action. In the latter case, we describe the differences between the final state achieved and the intended goal state, e.g., ``fail to move ball 2 to room 2.''

% \begin{wrapfigure}{r}{0.5\textwidth}
%     \centering
%     \includegraphics[width=0.98\linewidth]{figures/refine.jpeg}
%     \includegraphics[width=0.98\linewidth]{figures/refine-correct.jpeg}
%     \includegraphics[width=0.98\linewidth]{figures/refine-wrong.jpeg}
%     \caption{Plan refinement by the LLM Planner. The top row presents two choices of plan refinement (where the refinement starts). The second and third row shows the refinement outcomes of the two choices, where the second row shows a correctly refined plan and the third row shows an incorrect plan.}
%     \Description{Refinement}
%     \label{fig: refine}
% \end{wrapfigure}

\paragraph{LLM Planner with Refinement (LLM+Refine)}
Note that a plan generated by the LLM Planner could be incorrect. Therefore, we offer a prompting mechanism for the LLM Planner to refine the generated plan according to the user feedback. The mechanism works as follows:

1. Request the user to indicate the step number of the first action in the plan that is incorrect, such as the step where an action’s precondition is violated. We present a sample user interface on the left of Figure \ref{fig: refine} in Appendix.

2. Send the planning domain, problem description, and the original plan to the language model. Then, query the model to rewrite the subsequent steps starting from the user-specified step number. We present a sample input prompt in Figure \ref{fig: refine-prompt} in the Appendix.

3. Replace the original plan with the newly refined plan and display it to the user.

This mechanism allows users to interact with the language model to refine the plan. It enables the language model to focus on a subset of steps, facilitating a deeper interpretation of the incorrect component. However, the correctness of the refined plan is not guaranteed. Figure \ref{fig: refine} in the Appendix shows an example of a correctly refined plan and an incorrectly refined plan.


\vspace{-1mm}
\section{experiment and results}
\label{sec:experiments}

The experimental setup is shown in Fig~\ref{fig:experiment}. It shows three predefined target poses and the corresponding target area, which is the gray bounding box, and the gray dot shows the center of the target area, providing visual information for participants. Each participant is required to teleoperate the redundant robot manipulator for the target pose-reaching task, which uses the end-effector's tip to reach the gray dot for each target area while making the orientation of the end-effector as perpendicular as possible to the target area surface. The manipulator begins with the fixed starting configuration. This configuration aligns with the operator’s arm being extended straight forward. The operator is required to sequentially reach each target location in numerical order, following the specified task requirements. The objective of the experiment is to test the accuracy of teleoperation and the generality of the proposed system under different operator anthropometric upper body measurements. Four participants engage in this experiment and have a mean height of $169$±$1.24$ \textit{cm} and a mean arm span of $173.9$±$3.0$ \textit{cm}. Note that participant $1$ engages in training data collection but has not been practiced for the experiment. We hold a preparation phase for each participant, including setting up the Xsens Awinda body tracking system described in~\ref{awinda_data} and measuring each participant's arm and span and full-body height for fine-tuning the body representation profile in the Xsens Awinda software. Then, we provide each participant ten minutes to move their arm to become familiar with the relation between their arm joint angles and the Kinova joint angles. In the real test procedure, each participant is required to finish the task five times. Tasks finishing time, end-effector pose, and right arm joints angle trajectory are recorded for each participant.
Table~\ref{experiment_result} presents a summary of the results. The Euclidean distance quantifies the average distance between the end-effector's tip and the target center position across trials in Cartesian space when it touches the target surface. A smaller Euclidean distance indicates higher positional accuracy. Furthermore, the orientation difference between the target pose and the instant pose when the end-effector's tip touches the target surface is represented using the cosine similarity. A value closer to $1$ indicates that the two orientations are aligned in the same direction. In all trials, participants successfully teleoperated the manipulator to reach the desired target area, with a mean absolute error $2.51$±$0.75$ \textit{cm}, and the mean cosine similarity for all target poses is $0.97$±$0.01$. This supports the generality and accuracy of the proposed system.

\begin{table}[]
\scriptsize % Reduce font size
\begin{tabular}{cllll}
\multirow{2}{*}{\textbf{Participant}} &
  \multirow{2}{*}{\textbf{Time (s)}} &
  \multicolumn{3}{l}{\textbf{\begin{tabular}[c]{@{}l@{}}Euclidean Dist. (cm)/\\ Orientation Diff. (radian)\end{tabular}}} \\ \cline{3-5} 
\noalign{\vskip 2pt}
             &              & \textbf{Target 1} & \textbf{Target 2} & \textbf{Target 3} \\[2pt] \hline
\noalign{\vskip 2pt}
\textbf{1} &
  57.39±1.85 &
  \begin{tabular}[c]{@{}l@{}}1.91±0.39/\\ 0.98±0.00\end{tabular} &
  \begin{tabular}[c]{@{}l@{}}4.09±0.51/\\ 0.97±0.00\end{tabular} &
  \begin{tabular}[c]{@{}l@{}}1.12±0.40/\\ 0.99±0.00\end{tabular} \\[5pt] \hline
\noalign{\vskip 2pt} % Space above row 2
\textbf{2} &
  55.96±5.46 &
  \begin{tabular}[c]{@{}l@{}}2.07±0.29/\\ 0.97±0.01\end{tabular} &
  \begin{tabular}[c]{@{}l@{}}1.50±0.27/\\ 0.96±0.00\end{tabular} &
  \begin{tabular}[c]{@{}l@{}}2.13±0.13/\\ 0.99±0.00\end{tabular} \\[5pt] \hline
\noalign{\vskip 2pt} % Space above row 3
\textbf{3} &
  186.66±5.98 &
  \begin{tabular}[c]{@{}l@{}}2.51±0.75/\\ 0.97±0.01\end{tabular} &
  \begin{tabular}[c]{@{}l@{}}3.53±0.55/\\ 0.99±0.00\end{tabular} &
  \begin{tabular}[c]{@{}l@{}}1.92±0.57/\\ 0.99±0.00\end{tabular} \\[5pt] \hline
\noalign{\vskip 2pt} % Space above row 4
\textbf{4} &
  138.58±27.34 &
  \begin{tabular}[c]{@{}l@{}}2.42±0.34/\\ 0.97±0.00\end{tabular} &
  \begin{tabular}[c]{@{}l@{}}3.67±0.67/\\ 0.98±0.00\end{tabular} &
  \begin{tabular}[c]{@{}l@{}}4.99±1.17/\\ 0.99±0.00\end{tabular} \\[5pt] \hline
\noalign{\vskip 2pt} % Space above row 4
\textbf{All} & 116.17±13.91 & \multicolumn{3}{c}{2.73±0.95 / 0.98±0.00}      
\end{tabular}
\captionsetup{font=footnotesize}
\caption{The standard error of the cosine similarity for orientation reported as $0.00$ is rounding and does not represent an actual zero.}
\label{experiment_result}
\vspace{-8mm}
\end{table}






\vspace{-2.8mm}
\section{limitations}
\label{sec:limitations}
\vspace{-1.9mm}
\section*{Limitations}
Although our approach enhances personality trait control in LLMs, it comes with additional computational costs compared to prompt-based techniques. This is because it requires both generating appropriate adjustment queries and making direct modifications to the model’s internal representations, rather than relying solely on inference-time adjustments. However, this trade-off is justified, as our method ensures more stable and interpretable personality shifts, effectively addressing intrinsic biases and providing greater reliability in personality expression.

\vspace{-2.5mm}

\section{conclusion}
\label{sec:conclusion}
\section{Concluding Remarks}
In this paper, we proposed a novel approach utilizing multimodal LLMs to generate gesture-aware speech recognition transcripts for patients with language disorders. Our framework integrates verbal speech and iconic gestures, enabling the generation of enriched transcripts that capture the latent meaning conveyed through both modalities. Through extensive experimentation, we demonstrated that the proposed method effectively contextualizes incomplete or disfluent speech by incorporating gesture information, leading to more accurate and meaningful representations of the speaker's intent. These findings highlight the potential of our approach to significantly contribute to the field of speech and language therapy, offering innovative tools that can enhance the quality of life for individuals with language disorders by facilitating better communication and assessment methods.

\subsection{Ethical Statement} 
Our dataset was obtained from AphasiaBank with the approval of the Institutional Review Board (IRB) and adheres to the data sharing guidelines set by TalkBank\footnote{https://talkbank.org/share/ethics.html}. This includes complying with the Ground Rules for all TalkBank databases, which are based on the American Psychological Association Code of Ethics~\cite{american2002ethical}.

\subsection{Limitation \& Future Work} 
%This study represents a preliminary investigation into using multimodal LLMs to generate gesture-aware speech recognition transcripts. 
While the results are promising, we recognize several limitations and outline our plans to extend this work further.

One primary limitation is the absence of a definitive ground truth for quantitative evaluation. Since our model generates transcripts by synthesizing speech and gesture data from scratch, traditional benchmarks, such as comparisons with standard speech recognition outputs, are insufficient. Moreover, existing original transcripts lack gesture annotations, making direct comparisons challenging. In future work, we aim to address this gap by collaborating with certified pathologists to conduct qualitative assessments, such as A-B preference tests, to evaluate the effectiveness of gesture-enriched transcripts in accurately conveying the speaker's intentions.

To support quantitative evaluations, we plan to develop novel metrics that assess transcript quality, including grammar accuracy, semantic consistency, and the integration of multimodal information. Such metrics will provide a more objective basis for assessing our model's performance and facilitate comparisons with other multimodal and unimodal approaches.

Another limitation of this study is its focus on structured gestures from a specific task, the Peanut Butter Sandwich Task. While this task offers a controlled context for testing our approach, it does not encompass the diversity of gestures and communication patterns seen in everyday scenarios. As part of our future work, we plan to expand the scope of our model to include tasks such as the Cinderella Story Recall Task~\cite{bird1996cinderella}, which involves unstructured and complex narrative gestures. This expansion will allow us to evaluate the adaptability and robustness of our model in handling varied linguistic and gestural contexts.

In summary, while this study establishes a strong foundation for gesture-aware speech recognition, we aim to refine and extend our methods through collaborative qualitative evaluations, the development of robust quantitative metrics, and broader task applications. These efforts will ensure that our approach continues to evolve, ultimately contributing to more effective communication tools and interventions for individuals with language disorders.





\vspace{-4.5mm}
\section*{ACKNOWLEDGMENT}
The authors would like to thank all the members of the Center for Distributed Robotics Laboratory for their help. This work is supported by the Minnesota Robotics Institute
(MnRI) and the National Science Foundation through
grants  \#CNS-1531330, \#CNS-1919631, and \#CNS-1939033.
USDA/NIFA has also supported this work through the grants
2020-67021-30755 and 2023-67021-39829.



%%%%%%%%%%%%%%%%%%%%%%%%%%%%%%%%%%%%%%%%%%%%%%%%%%%%%%%%%%%%%%%%%%%%%%%%%%%%%%%%


\clearpage
% \newpage

\bibliographystyle{ieeetr}
\bibliography{kinova_teleop}




\end{document}
