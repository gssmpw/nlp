The initial training for mapping human arm gestures to the manipulator’s latent distribution space showed promising performance for redundant robot manipulator teleoperation. However, the initial training data is collected from only one individual. This introduces potential biases, in other words, different individuals can have distinct conceptions of a robot manipulator joint configuration corresponding to their arm configuration. For example, in the experimental results, participant $3$ took significantly longer to teleoperate the robot manipulator to reach the target pose. We observe that this participant attempted to use a lower back bending motion to control the Joint $0$ of the manipulator, which should ideally correlate with the shoulder joint's abduction/adduction. This variance highlights the need for more diverse data collection across different individuals to capture a broader range of natural human motions to correspond with the robot manipulator joint configuration. A comparable example is the MNIST dataset, where different individuals can interpret handwritten numbers differently, leading to variations in classification outcomes. Incorporating articulated thoracic and pelvic motion as additional features in human gesture mapping to the manipulator’s latent distribution space can be a promising opportunity for future work.
\begin{figure}[t]
    \centering
    \includegraphics[width=0.7\linewidth]{images/experiment.jpg}
    \captionsetup{font=footnotesize}
    \caption{Experiment setup. The operator needs to teleoperate the Kinova Gen$3$ $7$-DOF manipulator to reach three designated target poses. }
    \label{fig:experiment}
    \vspace{-7mm}
\end{figure}
The proposed GRU-based VAE model focuses on approximating the robot's configuration space and generating joint angle trajectories. However, it currently lacks information about joint velocities, which could be crucial for improving the smoothness and responsiveness of the teleoperation. Future work could address this by extending the data collection pipeline to include joint velocity trajectories, as discussed in Section~\ref{kinova_data}. By incorporating velocity information into the latent distribution space, the model could generate smoother and more precise robot manipulator movements, enhancing teleoperation performance.
