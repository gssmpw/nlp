
The experimental setup is shown in Fig~\ref{fig:experiment}. It shows three predefined target poses and the corresponding target area, which is the gray bounding box, and the gray dot shows the center of the target area, providing visual information for participants. Each participant is required to teleoperate the redundant robot manipulator for the target pose-reaching task, which uses the end-effector's tip to reach the gray dot for each target area while making the orientation of the end-effector as perpendicular as possible to the target area surface. The manipulator begins with the fixed starting configuration. This configuration aligns with the operator’s arm being extended straight forward. The operator is required to sequentially reach each target location in numerical order, following the specified task requirements. The objective of the experiment is to test the accuracy of teleoperation and the generality of the proposed system under different operator anthropometric upper body measurements. Four participants engage in this experiment and have a mean height of $169$±$1.24$ \textit{cm} and a mean arm span of $173.9$±$3.0$ \textit{cm}. Note that participant $1$ engages in training data collection but has not been practiced for the experiment. We hold a preparation phase for each participant, including setting up the Xsens Awinda body tracking system described in~\ref{awinda_data} and measuring each participant's arm and span and full-body height for fine-tuning the body representation profile in the Xsens Awinda software. Then, we provide each participant ten minutes to move their arm to become familiar with the relation between their arm joint angles and the Kinova joint angles. In the real test procedure, each participant is required to finish the task five times. Tasks finishing time, end-effector pose, and right arm joints angle trajectory are recorded for each participant.
Table~\ref{experiment_result} presents a summary of the results. The Euclidean distance quantifies the average distance between the end-effector's tip and the target center position across trials in Cartesian space when it touches the target surface. A smaller Euclidean distance indicates higher positional accuracy. Furthermore, the orientation difference between the target pose and the instant pose when the end-effector's tip touches the target surface is represented using the cosine similarity. A value closer to $1$ indicates that the two orientations are aligned in the same direction. In all trials, participants successfully teleoperated the manipulator to reach the desired target area, with a mean absolute error $2.51$±$0.75$ \textit{cm}, and the mean cosine similarity for all target poses is $0.97$±$0.01$. This supports the generality and accuracy of the proposed system.

\begin{table}[]
\scriptsize % Reduce font size
\begin{tabular}{cllll}
\multirow{2}{*}{\textbf{Participant}} &
  \multirow{2}{*}{\textbf{Time (s)}} &
  \multicolumn{3}{l}{\textbf{\begin{tabular}[c]{@{}l@{}}Euclidean Dist. (cm)/\\ Orientation Diff. (radian)\end{tabular}}} \\ \cline{3-5} 
\noalign{\vskip 2pt}
             &              & \textbf{Target 1} & \textbf{Target 2} & \textbf{Target 3} \\[2pt] \hline
\noalign{\vskip 2pt}
\textbf{1} &
  57.39±1.85 &
  \begin{tabular}[c]{@{}l@{}}1.91±0.39/\\ 0.98±0.00\end{tabular} &
  \begin{tabular}[c]{@{}l@{}}4.09±0.51/\\ 0.97±0.00\end{tabular} &
  \begin{tabular}[c]{@{}l@{}}1.12±0.40/\\ 0.99±0.00\end{tabular} \\[5pt] \hline
\noalign{\vskip 2pt} % Space above row 2
\textbf{2} &
  55.96±5.46 &
  \begin{tabular}[c]{@{}l@{}}2.07±0.29/\\ 0.97±0.01\end{tabular} &
  \begin{tabular}[c]{@{}l@{}}1.50±0.27/\\ 0.96±0.00\end{tabular} &
  \begin{tabular}[c]{@{}l@{}}2.13±0.13/\\ 0.99±0.00\end{tabular} \\[5pt] \hline
\noalign{\vskip 2pt} % Space above row 3
\textbf{3} &
  186.66±5.98 &
  \begin{tabular}[c]{@{}l@{}}2.51±0.75/\\ 0.97±0.01\end{tabular} &
  \begin{tabular}[c]{@{}l@{}}3.53±0.55/\\ 0.99±0.00\end{tabular} &
  \begin{tabular}[c]{@{}l@{}}1.92±0.57/\\ 0.99±0.00\end{tabular} \\[5pt] \hline
\noalign{\vskip 2pt} % Space above row 4
\textbf{4} &
  138.58±27.34 &
  \begin{tabular}[c]{@{}l@{}}2.42±0.34/\\ 0.97±0.00\end{tabular} &
  \begin{tabular}[c]{@{}l@{}}3.67±0.67/\\ 0.98±0.00\end{tabular} &
  \begin{tabular}[c]{@{}l@{}}4.99±1.17/\\ 0.99±0.00\end{tabular} \\[5pt] \hline
\noalign{\vskip 2pt} % Space above row 4
\textbf{All} & 116.17±13.91 & \multicolumn{3}{c}{2.73±0.95 / 0.98±0.00}      
\end{tabular}
\captionsetup{font=footnotesize}
\caption{The standard error of the cosine similarity for orientation reported as $0.00$ is rounding and does not represent an actual zero.}
\label{experiment_result}
\vspace{-8mm}
\end{table}




