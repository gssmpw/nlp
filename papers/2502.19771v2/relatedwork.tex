\section{Related work}
The ethical discussions about AI have been mainly anthropocentric, often neglecting the impact of these technologies on non-human animals \cite{Singer2023, Coghlan2023, Ziesche2021}. However, recent work has begun to address this gap. Previous research has revealed systematic biases in computer vision training datasets (e.g., ImageNet), which predominantly depict livestock freely roaming on pasture rather than in modern farming environments indoors \cite{Hagendorff2023}. Their analysis of five prominent computer vision models (e.g., InceptionV3 and VGG16) demonstrated significantly lower accuracy in classifying animals in indoor housing systems compared to outdoor settings, indicating poor out-of-distribution generalization capabilities. The authors hypothesized that future generative models trained on these dataset will further generate images that misrepresent livestock farming, such as images showing animals freely roaming outdoors. 

Previous research on AI's impact on non-human animals has mainly focused on philosophical investigations of speciesism bias in AI systems \cite{Bossert2021, Singer2023}. Philosophers who oppose speciesism believe that any being capable of suffering deserves equal consideration of interests, and raising livestock in factory farms for human consumption violates their interests \cite{Singer1975}. Many philosophers noted that AI systems are normalizing speciesism practices, such as livestock farming, killing, and eating animals \cite{Ziesche2021, Hagendorff2022, Singer2023}. Analysis of word embeddings from models like GloVe revealed that terms referring to farmed animals are more strongly associated with negative attributes (e.g., “ugly”, “primitive”) than positive qualities (e.g., “intelligent”, “brave”) \cite{Hagendorff2023}. They argue that incorporating animal interests into AI development is not just ethically imperative but also practically important given the interconnected nature of human and animal welfare.

To our knowledge, no research has examined how AI-generated images may misrepresent the reality of livestock farming, which could alter the future path towards aligning farming practices with societal values.