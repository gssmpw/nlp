\section{Related Work}
\subsection{Emotion Recognition and Engagement Estimation}
\subsubsection{Technical Foundations of Emotion Recognition}
Emotion recognition relies on artificial intelligence (AI) systems to analyze multimodal data and classify emotional states. These systems process raw inputs—such as images, audio, or physiological signals—by extracting features and mapping them to discrete categories (e.g., happiness, anger) or continuous dimensions (e.g., valence, arousal). Early methods employed handcrafted features, such as Gabor filters for facial expressions____ and MFCCs for audio signals____, but struggled with the complexity of real-world data. 
Modern systems leverage machine learning and deep learning to learn hierarchical representations automatically. Convolutional neural networks (CNNs) excel at spatial patterns in image data____, while recurrent neural networks (RNNs) and transformers effectively handle sequential data, capturing subtle cues like tonal variations or micro-expressions____. These advancements have significantly enhanced the ability of AI systems to detect and interpret emotional signals.

\subsubsection{Engagement as a Focus in Emotion Recognition}
Engagement is a dynamic and context-sensitive emotional state that reflects a user’s attention or involvement. It provides actionable insights for adaptive systems, such as e-learning platforms and human-computer interaction (HCI) frameworks____. 
Estimating engagement involves challenges such as the ambiguity of cues (e.g., gaze, posture, or micro-expressions) and their dependence on context____. Multimodal approaches integrating modalities like facial expressions, voice, and physiological data improve accuracy but complicate data synchronization, especially with noisy or incomplete inputs____.
Engagement also evolves, requiring temporal modeling techniques such as RNNs or transformers to capture patterns and intervene effectively. For instance, in e-learning, systems can detect waning focus and re-engage students through interactive quizzes or tailored content____. However, the subtlety of engagement cues, such as slight shifts in head position or blink rates, remains a significant challenge, particularly in low-resolution or noisy data____.

\subsection{Human-in-the-Loop (HITL) Systems}
\subsubsection{Concept and Role of HITL Systems}
HITL systems combine the efficiency of automated models with the contextual understanding of human expertise. Machines process large-scale data but struggle with edge cases or ambiguity, while humans provide intuition and domain knowledge____. HITL frameworks enable human intervention during annotation, validation, or refinement stages. For instance, human annotators can correct model misclassifications in emotion recognition by incorporating broader context____. This iterative process improves immediate accuracy and enhances model performance through feedback in subsequent training cycles.

\subsubsection{HITL in Emotion Recognition}
The complexity of emotion recognition, particularly for nuanced states like engagement, makes it an ideal application for HITL frameworks. Automated models often overemphasize specific features, such as gaze, which may be misleading in specific contexts. HITL systems mitigate these issues by enabling human annotators to refine predictions, ensuring accurate and contextually appropriate annotations. This approach is especially valuable for noisy or biased data, such as cross-cultural emotion recognition____. HITL frameworks also enhance generalization by incorporating diverse human corrections, enabling models to adapt to various expressions and contexts. This hybrid methodology ensures robustness in real-world applications.

\subsubsection{Advantages of HITL for Engagement Estimation}
HITL frameworks offer significant advantages for engagement estimation by leveraging human expertise to address challenges that automated models often face. Humans can interpret subtle and ambiguous cues, especially in context-dependent scenarios where behavioral signals vary based on individual, cultural, or environmental factors____. By validating and refining model predictions, HITL systems improve annotation accuracy and reliability, ensuring higher-quality data for training and evaluation____. Additionally, human intervention is critical in identifying and mitigating biases in model outputs, resulting in more balanced and equitable predictions____. 

HITL systems are particularly beneficial for real-time applications such as adaptive learning and virtual reality (VR) environments. For instance, IoT-based learning platforms leverage human feedback to adjust content dynamically, sustaining user engagement and improving learning outcomes____. Likewise, immersive VR interfaces can adapt to user interactions in real-time, enhancing engagement and overall satisfaction____.