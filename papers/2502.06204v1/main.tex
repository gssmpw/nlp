% 
% Annual Cognitive Science Conference
% Sample LaTeX Paper -- Proceedings Format
% 

% Original : Ashwin Ram (ashwin@cc.gatech.edu)       04/01/1994
% Modified : Johanna Moore (jmoore@cs.pitt.edu)      03/17/1995
% Modified : David Noelle (noelle@ucsd.edu)          03/15/1996
% Modified : Pat Langley (langley@cs.stanford.edu)   01/26/1997
% Latex2e corrections by Ramin Charles Nakisa        01/28/1997 
% Modified : Tina Eliassi-Rad (eliassi@cs.wisc.edu)  01/31/1998
% Modified : Trisha Yannuzzi (trisha@ircs.upenn.edu) 12/28/1999 (in process)
% Modified : Mary Ellen Foster (M.E.Foster@ed.ac.uk) 12/11/2000
% Modified : Ken Forbus                              01/23/2004
% Modified : Eli M. Silk (esilk@pitt.edu)            05/24/2005
% Modified : Niels Taatgen (taatgen@cmu.edu)         10/24/2006
% Modified : David Noelle (dnoelle@ucmerced.edu)     11/19/2014
% Modified : Roger Levy (rplevy@mit.edu)     12/31/2018



%% Change "letterpaper" in the following line to "a4paper" if you must.

\documentclass[10pt,letterpaper]{article}

\usepackage{cogsci}

\cogscifinalcopy % Uncomment this line for the final submission 

\usepackage{graphicx}
\usepackage{pslatex}
\usepackage{apacite}
\usepackage{float} % Roger Levy added this and changed figure/table
                   % placement to [H] for conformity to Word template,
                   % though floating tables and figures to top is
                   % still generally recommended!

%\usepackage[none]{hyphenat} % Sometimes it can be useful to turn off
%hyphenation for purposes such as spell checking of the resulting
%PDF.  Uncomment this block to turn off hyphenation.
\usepackage{natbib}
\usepackage{hyperref}

\usepackage{algpseudocode}
\usepackage{algorithm}
\usepackage[dvipsnames]{xcolor}
\usepackage{booktabs}
\usepackage{array}
\usepackage{tabularx}
\usepackage{linguex}
\usepackage{multirow}
\usepackage{amsmath}
\usepackage{amssymb}
\usepackage{listings}
\usepackage{makecell}
\usepackage{microtype}      % microtypography
\usepackage{xcolor}         % colors
\usepackage{graphicx}       % figures (add this?)
\usepackage{tikz}
\usepackage[many]{tcolorbox}
\usepackage{textcomp}
\usepackage{multirow}
\usepackage{wrapfig}
\usepackage{caption}
\usepackage{soul}
\usepackage{adjustbox}
\usepackage{enumitem}
\usepackage{caption}
\usepackage{tabularx}
%%%% our macros %%%%

% === Rose's prompt template for appendix ===
\definecolor{CB_gray}{gray}{0.5}
\newcommand{\clarification}{text to print}
\tcbset{prompt/.style={
    enhanced,
    size=fbox,
    boxrule=3pt,
    arc=2mm,
    auto outer arc,
    left=10pt,
    right=10pt,
    top=10pt,
    bottom=10pt,
    fontupper=\fontfamily{cmtt}\selectfont, 
    colback=CB_gray!10,
    colframe=CB_gray!25,
    coltitle=CB_gray!100, 
}}

\tcbset{vignette/.style={
    enhanced,
    size=fbox,
    boxrule=3pt,
    arc=2mm,
    auto outer arc,
    left=10pt,
    right=10pt,
    top=10pt,
    bottom=10pt,
    fontupper=\fontfamily{Avenir}\selectfont, 
    colback=CB_gray!10,
    colframe=CB_gray!25,
    coltitle=CB_gray!100, 
}}

\definecolor{codegreen}{rgb}{0,0.6,0}
\definecolor{codegray}{rgb}{0.5,0.5,0.5}
\definecolor{codepurple}{rgb}{0.58,0,0.82}
\definecolor{backcolour}{rgb}{0.95,0.95,0.92}



%Code listing style named "mystyle"
\lstdefinestyle{mystyle}{
  commentstyle=\color{codegreen},
  keywordstyle=\color{magenta},
  numberstyle=\tiny\color{codegray},
  basicstyle=\ttfamily\footnotesize,
  stringstyle=\color{codepurple},
  breakatwhitespace=false,         
  breaklines=true,                 
  captionpos=b,                    
  keepspaces=true,                 
  numbers=left,                    
  numbersep=5pt,                  
  showspaces=false,                
  showstringspaces=false,
  showtabs=false,                  
  tabsize=2
}
\lstset{style=mystyle}
%
\newcommand{\etal}{\textit{et~al}.~\ }
\newcommand{\ie}{\textit{i.e.,}~}
\newcommand{\cf}{\textit{c.f.,}~}

\newcommand{\ndg}[1]{\textcolor{green}{[NDG: #1]}}
\newcommand{\kg}[1]{\textcolor{blue}{[KG: #1]}} 
\newcommand{\mf}[1]{\textcolor{brown}{[MF: #1]}}
\newcommand{\pt}[1]{\textcolor{orange}{[PT: #1]}}

\def\algorithmautorefname{Algorithm}
\renewcommand{\sectionautorefname}{\S}
\def\Snospace~{\S{}}
\renewcommand*\sectionautorefname{\Snospace}
\renewcommand{\subsectionautorefname}{\sectionautorefname}
\renewcommand{\subsubsectionautorefname}{\sectionautorefname}
\renewcommand*{\figureautorefname}{Fig.}
\renewcommand*{\tableautorefname}{Tab.}
\renewcommand*{\appendixautorefname}{App.}
\renewcommand*{\algorithmautorefname}{Alg.}

% === COCO Signature ===
% Color palette: https://www.canva.com/colors/color-palettes/ice-infinity/
\definecolor{coco1}{HTML}{D9E4EC}
\definecolor{coco2}{HTML}{B7CFDC}
\definecolor{coco3}{HTML}{6AABD2}
\definecolor{coco4}{HTML}{385E72}
\hypersetup{
    colorlinks=true,
    linkcolor=coco3,
    filecolor=coco4,      
    urlcolor=coco3,
    citecolor=coco3,
}


% \usepackage{minted}
% \usepackage[finalizecache,cachedir=minted-cache]{minted}
% uncomment for arxiv
\usepackage[frozencache,cachedir=minted-cache]{minted}


\usemintedstyle{friendly} % a light color style


%%%% end our macros %%%%
%\setlength\titlebox{4.5cm}
% You can expand the titlebox if you need extra space
% to show all the authors. Please do not make the titlebox
% smaller than 4.5cm (the original size).
%%If you do, we reserve the right to require you to change it back in
%%the camera-ready version, which could interfere with the timely
%%appearance of your paper in the Proceedings.



\title{Non-literal Understanding of Number Words by Language Models}
 
\author{
  {\large \bf Polina Tsvilodub$^\dagger$$^*$} \\
  Department of Linguistics, \\
  University of T\"ubingen, Germany
  \And {\large \bf Kanishk Gandhi$^\dagger$} \\
  Department of Computer Science \\
  Stanford University
\And{\large \bf Haoran Zhao$^\dagger$} \\
  University of Washington
  \AND{\large \bf Jan-Philipp Fr\"anken} \\
  Department of Psychology, \\
  Stanford University
  \And {\large \bf Michael Franke} \\
  Department of Linguistics, \\
  University of T\"ubingen, Germany
  \And {\large \bf Noah D. Goodman} \\
  Departments of Psychology \& \\Computer Science, \\
  Stanford University
  }
 
\begin{document}
\maketitle
\begingroup
\renewcommand\thefootnote{}\footnotetext{$\dagger$ These authors contributed equally to this work.\\ \indent \indent $^*$polina.tsvilodub@uni-tuebingen.de}
\endgroup
\begin{abstract}
Humans naturally interpret numbers non-literally, effortlessly combining context, world knowledge, and speaker intent. We investigate whether large language models (LLMs)  interpret numbers similarly, focusing on hyperbole and pragmatic halo effects. Through systematic comparison with human data and computational models of pragmatic reasoning, we find that LLMs diverge from human interpretation in striking ways.
By decomposing pragmatic reasoning into testable components, grounded in the Rational Speech Act framework, we pinpoint where LLM processing diverges from human cognition --- not in prior knowledge, but in reasoning with it. 
This insight leads us to develop a targeted solution --- chain-of-thought prompting inspired by an RSA model makes LLMs' interpretations more human-like. Our work demonstrates how computational cognitive models can both diagnose AI-human differences and guide development of more human-like language understanding capabilities.

\textbf{Keywords:} 
hyperbole; pragmatic halo; large language models; pragmatics; Rational Speech Act
\end{abstract}


\section{Introduction}
\label{sec:intro}
\section{Introduction}

Tutoring has long been recognized as one of the most effective methods for enhancing human learning outcomes and addressing educational disparities~\citep{hill2005effects}. 
By providing personalized guidance to students, intelligent tutoring systems (ITS) have proven to be nearly as effective as human tutors in fostering deep understanding and skill acquisition, with research showing comparable learning gains~\citep{vanlehn2011relative,rus2013recent}.
More recently, the advancement of large language models (LLMs) has offered unprecedented opportunities to replicate these benefits in tutoring agents~\citep{dan2023educhat,jin2024teach,chen2024empowering}, unlocking the enormous potential to solve knowledge-intensive tasks such as answering complex questions or clarifying concepts.


\begin{figure}[t!]
\centering
\includegraphics[width=1.0\linewidth]{Figs/Fig.intro.pdf}
\caption{An illustration of coding tutoring, where a tutor aims to proactively guide students toward completing a target coding task while adapting to students' varying levels of background knowledge. \vspace{-5pt}}
\label{fig:example}
\end{figure}

\begin{figure}[t!]
\centering
\includegraphics[width=1.0\linewidth]{Figs/Fig.scaling.pdf}
\caption{\textsc{Traver} with the trained verifier shows inference-time scaling for coding tutoring (detailed in \S\ref{sec:scaling_analysis}). \textbf{Left}: Performance vs. sampled candidate utterances per turn. \textbf{Right}: Performance vs. total tokens consumed per tutoring session. \vspace{-15pt}}
\label{fig:scale}
\end{figure}


Previous research has extensively explored tutoring in educational fields, including language learning~\cite{swartz2012intelligent,stasaski-etal-2020-cima}, math reasoning~\cite{demszky-hill-2023-ncte,macina-etal-2023-mathdial}, and scientific concept education~\cite{yuan-etal-2024-boosting,yang2024leveraging}. 
Most aim to enhance students' understanding of target knowledge by employing pedagogical strategies such as recommending exercises~\cite{deng2023towards} or selecting teaching examples~\cite{ross-andreas-2024-toward}. 
However, these approaches fall short in broader situations requiring both understanding and practical application of specific pieces of knowledge to solve real-world, goal-driven problems. 
Such scenarios demand tutors to proactively guide people toward completing targeted tasks (e.g., coding).
Furthermore, the tutoring outcomes are challenging to assess since targeted tasks can often be completed by open-ended solutions.



To bridge this gap, we introduce \textbf{coding tutoring}, a promising yet underexplored task for LLM agents.
As illustrated in Figure~\ref{fig:example}, the tutor is provided with a target coding task and task-specific knowledge (e.g., cross-file dependencies and reference solutions), while the student is given only the coding task. The tutor does not know the student's prior knowledge about the task.
Coding tutoring requires the tutor to proactively guide the student toward completing the target task through dialogue.
This is inherently a goal-oriented process where tutors guide students using task-specific knowledge to achieve predefined objectives. 
Effective tutoring requires personalization, as tutors must adapt their guidance and communication style to students with varying levels of prior knowledge. 


Developing effective tutoring agents is challenging because off-the-shelf LLMs lack grounding to task-specific knowledge and interaction context.
Specifically, tutoring requires \textit{epistemic grounding}~\citep{tsai2016concept}, where domain expertise and assessment can vary significantly, and \textit{communicative grounding}~\citep{chai2018language}, necessary for proactively adapting communications to students' current knowledge.
To address these challenges, we propose the \textbf{Tra}ce-and-\textbf{Ver}ify (\textbf{\model}) agent workflow for building effective LLM-powered coding tutors. 
Leveraging knowledge tracing (KT)~\citep{corbett1994knowledge,scarlatos2024exploring}, \model explicitly estimates a student's knowledge state at each turn, which drives the tutor agents to adapt their language to fill the gaps in task-specific knowledge during utterance generation. 
Drawing inspiration from value-guided search mechanisms~\citep{lightman2023let,wang2024math,zhang2024rest}, \model incorporates a turn-by-turn reward model as a verifier to rank candidate utterances. 
By sampling more candidate tutor utterances during inference (see Figure~\ref{fig:scale}), \model ensures the selection of optimal utterances that prioritize goal-driven guidance and advance the tutoring progression effectively. 
Furthermore, we present \textbf{Di}alogue for \textbf{C}oding \textbf{T}utoring (\textbf{\eval}), an automatic protocol designed to assess the performance of tutoring agents. 
\eval employs code generation tests and simulated students with varying levels of programming expertise for evaluation. While human evaluation remains the gold standard for assessing tutoring agents, its reliance on time-intensive and costly processes often hinders rapid iteration during development. 
By leveraging simulated students, \eval serves as an efficient and scalable proxy, enabling reproducible assessments and accelerated agent improvement prior to final human validation. 



Through extensive experiments, we show that agents developed by \model consistently demonstrate higher success rates in guiding students to complete target coding tasks compared to baseline methods. We present detailed ablation studies, human evaluations, and an inference time scaling analysis, highlighting the transferability and scalability of our tutoring agent workflow.


\section{Pragmatic Number Interpretation in Humans}
\label{sec:related-work}
\section{Background and related work}
% 重点看Artistic data visualization: Beyond visual analytics 和Visualization criticism-the missing link between information visualization and art 的被引


This section reviews the background on artistic data visualization and previous research related to this topic.

\subsection{Artistic Data Visualization in Art History Context}
\label{ssec:contemporary}

Art history has been marked by transformative periods characterized by different aesthetic pursuits, materials, and methods. Since the time of Plato, imitation (or \textit{mimesis}, which views art as a mirror to the world around us) has been an important pursuit~\cite{pooke2021art}. Successful artworks, such as Greek sculptures and the convincing illusions of depth and space in Renaissance paintings, exemplify this tradition.
The advent of modern society and new technology, especially photography, posed a significant challenge to the notion of art as imitation~\cite{perry2004themes}. By the 1850s, modern art began to emerge with the core goal of transcending traditional forms and conventions. Movements like Post Impressionism, Expressionism, and Cubism revolutionized art through innovative uses of form (\eg color, texture, composition), moving art towards abstraction and experimentation. 
After World War II, the Cold War and the surge of various social problems heightened skepticism about the progress narrative of modernity and the superiority of the capitalist system, leading to the rise of postmodernism and the birth of contemporary art~\cite{hopkins2000after,harrison1992art}. One prominent feature of contemporary art is the absence of fixed standards or genres historically defined by the church or the academy. Postmodern design neither defines a cohesive set of aesthetic values nor restricts the range of media used~\cite{pooke2021art}, sparking novel practices such as installations, performances, lens-based media, videos, and land-based art~\cite{hopkins2000after}.
Meanwhile, artists have increasingly drawn energy from various philosophical and critical theories such as gender studies, psychoanalysis, Marxism, and post-structuralism~\cite{pooke2021art}. As a result, as described by Foster~\cite{foster1999recodings}, artists have increasingly become ``manipulators of signs and symbols... and the viewer an active reader of messages rather than a passive contemplator of the aesthetic''. Hopkins~\cite{hopkins2000after} described this shift as the ``death of the object'' and ``the move to conceptualism''. 
% Joseph Kosuth, one of the most important representatives of conceptual artists, also once said that “all art (after Duchamp) is conceptual (in nature) because art only exists conceptually”
% As argued by Danto~\cite{danto2015after}, traditional notions of aesthetics can no longer apply to contemporary art. ``“All there is at the end,” Danto wrote, “is theory, art having finally become vaporized in a dazzle of pure thought about itself, and remaining, as it were, solely as the object of its own theoretical consciousness.''
% The Anti-aesthetic (1983) has described these as ‘anti-aesthetic’ strategies – typified, as we have seen, by a conceptually driven approach to the art object and to the process of its production.

Emerging within the contemporary art historical context, data art has been significantly propelled by the advent of big data. An early milestone was Kynaston McShine's 1970 exhibition \textit{Information} at the Museum of Modern Art (MoMA). 
% In the exhibition catalogue, McShine wrote~\cite{information_moma}: ``Increasingly artists use mail, telegrams, telex machines, etc., for transmission of works themselves—photographs, films, documents—or of information about their activity.'' 
% The millennium era has witnessed substantial growth in this area.
In 2008, Google’s Data Arts Team was founded to explore the advancement of what creativity and technology can do together~\cite{google}.
% data artist Aaron Koblin
In 2012, Viégas and Wattenberg's \textit{Wind Map}, an artwork that visualizes real-time air movement, became the first web-based artwork to be included in MoMA's permanent collection~\cite{wind}.
Since 2013, the academic conference IEEE VIS has included an Arts Program (IEEE VISAP), showcasing artistic data visualizations through accepted papers and curated exhibitions. 
As noted by Barabási~\cite{dataism} (a Fellow of the American Physical Society and the head of a data art lab), data has become a vital medium for artists to deal with the complexities of our society: ``Humanity is facing a complexity explosion. We are confronted with too much data for any of us to make sense of...The traditional tools and mediums of art, be they canvas or chisel, are woefully inadequate for this task...today’s and tomorrow’s artists can embrace new tools and mediums that scale to the challenge, ensuring that their practice can continue to reflect our changing epistemology.''
% a physicist and head of a data art lab, has noted, 

% Artists are now exploring new mediums and methods, incorporating datasets, computer technology, and algorithms into their work.



\subsection{Research on Artistic Data Visualization}
\label{ssec:artisticvis}

Artistic data visualization is also referred to as artistic visualization, data art, or information art~\cite{holmquist2003informative,rodgers2011exploring,few,viegas2007artistic}. Despite the varying terminologies, there is a basic consensus that artistic data visualization must be art practice grounded in real data~\cite{viegas2007artistic}.
Since the early 2000s, a series of papers introduced innovative artistic systems for visualizing everyday data, such as museum visit routes and bus schedule information~\cite{skog2003between,holmquist2003informative,viegas2004artifacts}.
In 2007, Viégas and Wattenberg~\cite{viegas2007artistic} explicitly proposed the concept of \textit{artistic data visualization} and viewed it as a promising domain beyond visual analytics.
% and defined it as ``visualization of data done by artists with the intent of making art''. 
Kosara~\cite{kosara2007visualization} drew a spectrum of visualization design, positioning artistic visualization and pragmatic visualization at opposite ends of this spectrum to demonstrate that the goals of these two types of design often diverge. 
% advocating that analytical visualizations prioritize practicality, while artistic data visualizations emphasize sublime quality, that is, the capacity to inspire awe and grandeur and elicit profound emotional or intellectual responses. 
% In 2011, Rodgers and Bartram~\cite{rodgers2011exploring} utilized artistic data visualization to enhance residential energy use feedback. 
However, overall, research on this subject has been sparse. Among those relevant papers, most have focused on specific applications of artistic data visualization. 
%~\cite{rodgers2011exploring,schroeder2015visualization,perovich2020chemicals}
For instance, Rodgers and Bartram~\cite{rodgers2011exploring} utilized ambient artistic data visualization to enhance residential energy use feedback. Samsel~\etal~\cite{samsel2018art} transferred artistic styles from paintings into scientific visualization.
Artistic practice has also been observed in fields such as data physicalization~\cite{hornecker2023design,perovich2020chemicals,offenhuber2019data} and sonification~\cite{enge2024open}. For example, Hornecker~\etal~\cite{hornecker2023design} found that many artists are practicing transforming data into tangible artifacts or installations. Enge~\etal~\cite{enge2024open} discussed a set of representative artistic cases that combine sonification and visualization.
% dragicevic2020data
% Offenhuber~\cite{offenhuber2019data} created a set of artworks in urban settings that transform air quality data into situated displays, allowing people to encounter visualizations in their daily lives.

% But in contrast, empirical studies that describe the characteristics of artistic visualization and how they are created are extremely scarce. This scarcity forms a stark contrast to the increasingly rich and diverse practices by artists in the field.
% As for the difference between artistic data visualization and traditional visualizations for analytics, Vi{\'e}gas and Wattenberg~\cite{viegas2007artistic} thought that the most salient feature of artistic data visualizations is their forceful expression of viewpoints.
% In Ramirez~\cite{ramirez2008information}'s opinion, functional information visualizations are concerned with usability and performance while aesthetic information visualizations are concerned with visually attractive forms of representation design.
% Donath~\etal~\cite{donath2010data} presented a series of tools developed to integrate artistic expressions in generating unique and customized visualizations based on users' personal data, such as health monitoring data, album records, and e-mail records. 

On the other hand, some studies, while not focusing on artistic data visualization, have explored a key art-related concept: aesthetics. 
% ~\cite{moere2012evaluating,cawthon2007effect,lau2007towards} explored the aesthetics of visualization design in their research. They
For example, Moere~\etal~\cite{moere2012evaluating} compared analytical, magazine, and artistic visualization styles, noting that analytical styles enhance the discovery of analytical insights, while the other two induce more meaning-based insights. Cawthon~\etal~\cite{cawthon2007effect} asked participants to rank eleven visualization types on an aesthetic scale from ``ugly'' to ``beautiful'', finding that some visualizations (\eg sunburst) were perceived as more beautiful than others (\eg beam trees).
% Moere~\etal~\cite{moere2012evaluating} displayed data in three different styles (analytical style, magazine style, artistic style) and found that these styles led to different perceptions of usability and types of insights.
% More importantly, the authors found that the sunburst chart ranks the highest in aesthetics and is one of the top-performing visualizations in both efficiency and effectiveness, thus exemplifying the notion that "beautiful is indeed usable".
Factors such as embellishment~\cite{bateman2010useful}, colorfulness~\cite{harrison2015infographic}, and interaction~\cite{stoll2024investigating} have also been found to influence aesthetics. 
% borkin2013makes,tanahashi2012design
However, most of these studies have simplified aesthetics to hedonic features (\eg beauty), without delving into the nuanced connotations of aesthetics.
% most of these studies have simplified aesthetics to concepts like 'beauty,' 'preference,' or 'pleasing,' without exploring the deeper essence of aesthetics as the core of art.

The value of artistic data visualization to the visualization community is still in controversy. For instance, Few~\cite{few} argued for a clearer distinction between data art and data visualization; he highlighted the negative consequences when data art ``masquerades as data visualization'', such as making visualizations difficult to perceive. Willers~\cite{willers2014show} thought the artistic approach is ``unlikely be appreciated if the intention was for rapid decision making.''
% In an interview, American artist and technologist Harris commented that ``data can be made pretty by design, but this is a superficial prettiness, like a boring woman wearing too much makeup.''~\cite{harris2015beauty} 
To address these gaps, more empirical research needs to be conducted to explore how artistic data visualizations are designed, their underlying pursuits, and how they might inspire our community.




% Examining this field not only helps us understand the latest application of data visualization in various domains but also extends our understanding of the aesthetic and humanistic aspects of data visualization.
% there should be more theoretical investigation into artistic data visualization. 

% These three concepts emphasize placing or installing visualizations at physical places that people will encounter in their daily lives. 

% ~\cite{wang2019emotional}


% gap between art and science~\cite{judelman2004aesthetics}
% constructive visualization~\cite{huron2014constructive}
% data feminism~\cite{d2020data}
% critical infovis~\cite{dork2013critical}
% citizen data and participation~\cite{valkanova2015public}

% \x{Lee~\etal~\cite{lee2013sketchstory}, give users artistic freedom to create their own visualizations.}


% Aesthetics refers to the study of beauty, taste, and sensory perception and is deeply intertwined with art.
% a particular taste for or approach to what is pleasing to the senses and especially sight

% why shouldn't all charts be scatter plot~\cite{bertini2020shouldn}
% aesthetic model~\cite{lau2007towards}
% Aesthetics for Communicative Visualization : a Brief Review
% Stacked graphs--geometry \& aesthetics~\cite{byron2008stacked}
% storyline optimization~\cite{tanahashi2012design}
% graphic designers rate the attractiveness of non-standard and pictorial visualizations higher than standard and abstract ones, whereas the opposite is true for laypeople.~\cite{quispel2014would}
% evaluate aesthetics - wordcloud
% An Evaluation of Semantically Grouped Word Cloud Designs, tag cloud, wordle

% On the other hand, empirical studies conducted with designers have shown that functionality is not the only design goal of visualization. For example, Bigelow~\etal~\cite{bigelow2014reflections} found that designers would frequently fine-tune the non-data elements in their designs, and data encoding was even "a later consideration with respect to other visual elements of the infographic".
% Moere~\cite{moere2011role} - design
% Quispel~\etal~\cite{quispel2018aesthetics} found that for information designers, clarity and aesthetics are both important for making a design attractive.

\section{Experiments}
\label{sec:experiments}
%This section introduces the benchmark with three datasets and the evaluation metrics. We further provide details on the selected models, the hyperparameters, and the baseline systems.
%
\setlength{\tabcolsep}{10.7pt}
\begin{table*}[h]
    \centering
    \footnotesize
\begin{tabular}{lccccc}
\toprule
\textbf{Dataset}   & \textbf{Domain}  & \textbf{\# Cases}    & \textbf{\# Diseases} & \textbf{Synthetic}   & \textbf{License\dag}    \\
\midrule
DDxPlus~\citep{fansi2022ddxplus}   & \textit{respiratory} & 1.3M    & 49           & \checkmark & CC-BY      \\
iCraft-MD~\citep{li2024mediq}   & \textit{skin}    & 140    & 394         &   \checkmark       & MIT        \\
RareBench~\citep{chen2024rarebench} & \textit{rare}    & 2,185    & 421         &     $\times$ 
& Apache-2.0 \\ 
\bottomrule
\end{tabular}
\caption{Overview of the selected sources for constructing DDx benchmark. We consider three domains (i.e., disease categories) (\textit{respiratory}, \textit{skin}, \textit{rare}) over different sizes of diagnosis options. All selected sources are applicable for \textit{commercial} usage. \dag License: Creative Commons Attribution International License (CC-BY). }
    \label{tab:datasets_overview}
\end{table*}


% Israa: I changed the columns order
\iffalse
\begin{tabular}{lccccc}
\toprule
\textbf{Dataset}   & \textbf{\# Cases}    & \textbf{Domain}  & \textbf{Synthetic}   & \textbf{\# Diagnosis} & \textbf{License\dag}    \\
\midrule
DDxPlus~\citep{fansi2022ddxplus}   & 1.3M & \textit{general} & \checkmark & 49           & CC-BY      \\
CraftMD~\citep{johri2024craft,johri2025craftmd}   & 140         & \textit{skin}    &   $\times$       & 393          & MIT        \\
RareBench~\citep{chen2024rarebench} &     2,185      & \textit{rare}    &    $\times$        &     421         & Apache-2.0 \\ \bottomrule
\end{tabular}
\fi
\subsection{DDx Benchmark}
\label{subsec:ddx_benchmark}

We introduce a comprehensive DDx benchmark integrating three datasets -- DDxPlus, iCraft-MD, and RareBench, covering \textit{respiratory}, \textit{skin}, and \textit{rare} diseases for a robust assessment of diagnostic performance. This addresses limitations of prior work, which often relies on a single dataset and single-turn evaluation for differential diagnosis. DDxPlus~\citep{fansi2022ddxplus} provides a large-scale structured dataset with 1.3 million synthetic respiratory patient cases across 49 respiratory-related pathologies. iCraft-MD ~\citep{li2024mediq} includes 394 skin diseases, adapting static dermatological clinical vignettes (from original Craft-MD dataset~\citep{johri2024craft, johri2025craftmd}) into an interactive setting%\cl{We use the ineractive setting by doing XYZ with our history taking simulator.} 
\footnote{Interactive DDx is a more complex information-seeking setup, since in the real world the full patient profile might not be accessible initially.} -- the system is only provided with partial patient information and is expected to proactively ask questions and gather information. %It consists of 140 dermatology cases, with 100 sourced from an online medical question bank and 40 designed by expert clinicians. In the interactive setting, it requires iterative history-taking and diagnostic refinement. 
RareBench~\citep{chen2024rarebench} expands DDxPlus with 421 rare diseases. We select three subsets from RareBench -- RAMEDIS (Europe), MME (Canada), and PUMCH (China) -- to ensure diversity in regional representation.

To enable a consistent evaluation across datasets, we standardize each dataset into a structured format: (i) optional initial patient information (e.g., age, sex, chief complaint); (ii) full patient profile (complete list of symptoms and antecedents); and (iii) full set of possible diseases for differential diagnosis. %A key challenge in iCraft-MD and RareBench is the lack of predefined diagnostic options and redundant disease labels; we address this by leveraging GPT-4o to generate unique, non-redundant differential diagnosis sets, yielding 394 distinct dermatological conditions in iCraft-MD and 102 rare diseases in RareBench. 
This refinement enhances diagnostic consistency and supports the evaluation of interactive DDx. We sample 100 patients from each dataset at a fixed random seed, due to the cost of experiments and excessive time for reasoning steps. Detailed dataset statistics are in~\autoref{app:ddx_benchmark_details}.


\subsection{Evaluation Metrics}
%\danny{Evaluation for each module, and the tracking progress states}
To evaluate diagnostic performance, we employ three metrics. First, we compute the \textit{average rank} of the correct disease, which represents the model’s ability to position the correct diagnosis closer to the top. If the diagnosis does not appear in the top-10 position, we assign a rank of 11. Second, we use \textit{GTPA@k} (Ground Truth Pathology Accuracy)~\citep{fansi2022ddxplus}, which measures whether the ground truth diagnosis appears within the top-\textit{k} predicted diagnoses.
Third, we introduce a new metric suitable for the iterative setting: \textit{average progress rate} ($\Delta$ Progress). Inspired by AgentQuest \cite{gioacchini-etal-2024-agentquest}, it tracks changes in rank $r$ of the ground truth pathology in the differential diagnosis. For each patient case i, we average the progress in rank ($r_{i,t}-r_{i,t+1}$) over N iterations of differential diagnosis, then aggregate over M patients. This metric quantifies how effectively the system refines and converges on the correct diagnosis over successive iterations:

\small{
\begin{equation*}
\Delta \text{Progress} = \frac{1}{M} \sum_{i=1}^{M} \left( \frac{1}{N_i - 1} \sum_{t=1}^{N_i-1} \Bigl( r_{i,t} - r_{i,t+1} \Bigr) \right)
\end{equation*}}
\normalsize
\subsection{Models and Tasks}

We evaluate on GPT-4o (version: \textsc{2024-11-20})~\citep{hurst2024gpt}, Llama3.1-70B and Llama3.1-8B~\citep{dubey2024llama} across all tasks, ensuring a comparison of LLMs at varying scales.  Our experiments are conducted in two setups: (1) optimizing individual agents; and (2) interactive differential diagnosis. In the first task, we evaluate the two agents (knowledge retrieval, diagnosis strategy) in a single-turn setting. This allows us to isolate the effectiveness of the reasoning mechanisms without the confounding factor of incomplete information. In the second task, we assess MEDDxAgent's performance at interactive DDx, comparing it against the single-turn diagnostic agents and history taking simulator. Interactive differential diagnosis, as suggested by~\citet{li2024mediq}, is a challenging yet realistic scenario, where only initial patient information is available -- without a complete list of symptoms and antecedents. This setup highlights how limited information constrains the single-turn setting (i.e., no iteration), compared to MEDDxAgent's iterative interactions, which refine and enhance the diagnostic process.

\subsection{Hyperparameters and Optimization}
\label{subsec:hyperaparemeters_optimization}

\setlength{\tabcolsep}{2.3pt}
\begin{table*}[h]
\centering
\scriptsize
\begin{tabular}{rccccccccc}
\toprule
                               & \multicolumn{3}{c}{\textbf{DDxPlus}} & \multicolumn{3}{c}{\textbf{iCraft-MD}} & \multicolumn{3}{c}{\textbf{RareBench}} \\ \cmidrule(lr){2-4} \cmidrule(lr){5-7} \cmidrule(lr){8-10}
                               & \textbf{GTPA@1 $\uparrow$}          & \textbf{GTPA@5 $\uparrow$}   & \textbf{Avg Rank $\downarrow$} & \textbf{GTPA@1 $\uparrow$}       & \textbf{GTPA@5 $\uparrow$}     & \textbf{Avg Rank $\downarrow$}   & \textbf{GTPA@1 $\uparrow$}        & \textbf{GTPA@5 $\uparrow$}   & \textbf{Avg Rank $\downarrow$}     \\\midrule
                               %& \multicolumn{9}{c}{\textbf{GPT-4o}}                                                                 \\\midrule
Retrieval (PubMed)                   & 0.69           &  0.90   &  2.27  & 0.68        &     \textbf{0.79}    & 3.23  & 0.45         &   0.72  &    3.92   \\
Retrieval (Wiki)                   &    0.69        &   0.90 &  2.24  & \textbf{0.69}         &     \textbf{0.79}    & \textbf{3.22}  &   0.45       &  0.74  &  4.00    \\ 
\cmidrule(lr){2-10}
Zero-shot (Standard)                     &     0.69           &     0.90       &      2.21      &       0.68         &      0.77        &         3.37       &       0.46       &      0.72       &   3.99             \\
Zero-shot (CoT)                    &     0.71          &     0.92       &      2.10      &       0.68         &     0.77         &         3.35       &       0.47       &    0.69         &   4.02              \\ %\cmidrule(lr){2-10}
%Few-shot (Standard, Dyn\_BERT) &       0.96        &  1.00           &     1.06           &     0.67         &     0.77           &      3.31        &     0.78        &       0.90     & 2.19     \\
Few-shot (Standard, Dyn\_BAII)$\ddag$ &      0.96          &      \textbf{1.00}      &    1.06         &        0.62        &        0.72      &     3.85           &     0.79         &   \textbf{0.91}          &      \textbf{2.03}           \\
Few-shot (CoT, Dyn\_BERT)      &       0.96         &    \textbf{1.00}        &  1.05        &     0.64           &      0.73        &         3.68      &     0.81         &  \textbf{0.91}            &          2.04      \\
Few-shot (CoT, Dyn\_BAII)      &       \textbf{0.97}         &     \textbf{1.00}       &       \textbf{1.03}      &         0.60       &      0.70        &        4.00      &         \textbf{0.82}     &      0.88       &         2.11      \\

%History (n=5)         & 0.45           &      & 4.13 & 0.40        & 5.58     &      & 0.11         &      &  7.84   \\
%History (n=10)        & 0.59           & 3.16    & -  & 0.45        & 5.35        & -  & 0.24         & 6.67      &  -   \\
%History (n=15)        & 0.69           &      & 2.47 & 0.46        &       &  5.23   & 0.36         &       &    5.49 \\
%\midrule
%                                & \multicolumn{9}{c}{\textbf{Llama3.1-70B}}                                                           \\ \midrule
% Zero-shot (Standard)                     &      0.54          &    0.78        &       3.53     &     0.40           &     0.64         &        4.87      &      0.39        &       0.77      &         4.05         \\
% Zero-shot (CoT)                      &     0.45          &    0.78      &       3.69     &       \textbf{0.48}         &      \textbf{0.66}        &         \textbf{4.50}       &       0.49       &       0.75      &   3.91              \\\cmidrule(lr){2-10}
% Retrieval (PubMed)                 & 0.56           &  0.79     & 3.42 & 0.44        &    0.63     &  4.72 & 0.38         &   0.75       & 3.96 \\
% Retrieval (Wiki)                    &   0.49         & 0.77   &  3.60  &       0.44  &  \textbf{0.66}       &  4.71 &     0.39     & 0.75   &  4.09   \\\cmidrule(lr){2-10}
% %Few-shot (Standard, Dyn\_BERT) &      0.84         &  0.94          &                 1.68           &      0.40        &         0.62       &    4.96          &       0.72      &    0.87       & 2.44      \\
% Few-shot (Standard, Dyn\_BAII)$\ddag$ &        0.86        &    \textbf{0.95}        &    1.59         &     0.40           &      0.63        &         5.02       &     0.73         &       0.87      &      2.44           \\
% Few-shot (CoT, Dyn\_BERT)      &      \textbf{0.91}          &     \textbf{0.95}       &   \textbf{1.55}         &      0.45          &   0.61           &        4.93     &     \textbf{0.75}         &   \textbf{0.88}          &        \textbf{2.35}      \\
% Few-shot (CoT, Dyn\_BAII)      &     0.89           &     0.93       &       1.71     &      0.45          &    0.63          &       4.90      &         0.71     &     0.87        &         2.62     \\
%History (n=5)         & 0.45           &     &  4.15 & 0.29        &    &     6.48   & 0.30         &       &   6.04  \\
%History (n=10)        & 0.58           &      & 3.12 & 0.33        &     &    5.82   & 0.36         &     &    4.51   \\
%History (n=15)        & 0.56           &       & 3.50 & 0.36        &       &  5.36   & 0.31         &     &    4.80   \\
% \midrule
%                                & \multicolumn{9}{c}{\textbf{Llama3.1-8B}}                                                            \\\midrule
% Zero-shot (Standard)                      &    0.45            &      0.68      &   9.00          &    0.27             &    0.43          &      7.02         &    0.33           &    0.57         &             5.45     \\
% Zero-shot (CoT)                     &    0.45            &      0.70      &   4.51          &    0.27             &      0.40        &      7.25         &    0.24           &     0.55        &             5.65     \\\cmidrule(lr){2-10}
% Retrieval (PubMed)                   & 0.42           &  0.67   & 4.50  & 0.29        &   \textbf{0.44}   &   6.93   & 0.35         &   0.59   &    5.33  \\
% Retrieval (Wiki)                    &    0.43        &  0.67   &  4.56  &  0.29       &   \textbf{0.44}      & 6.93  &     0.36     &   0.67 &    4.80   \\\cmidrule(lr){2-10}
% %Few-shot (Standard, Dyn\_BERT)  &     0.97           &   1.00         &  1.04           &      0.22          &    0.40          &         7.32       &     0.70         &     0.81        &   2.94              \\
% Few-shot (Standard, Dyn\_BAII)$\ddag$  &     \textbf{0.97}           &    \textbf{1.00}        &   \textbf{1.03}          &    0.21            &      0.42        &    6.93            &  \textbf{0.71}            &     0.83        &     \textbf{2.80}            \\
% Few-shot (CoT, Dyn\_BERT)     &     0.95           &     0.99       &       1.19      &     0.29           &      0.36        &        7.28     &  0.64            &    \textbf{0.84}         &        2.96      \\
% Few-shot (CoT, Dyn\_BAII)       &      \textbf{0.97}          &    \textbf{1.00}        &       \textbf{1.03}     &      \textbf{0.30}          &   \textbf{0.44}           &         \textbf{6.66}     &  0.65            &      0.82       &        2.95      \\
%History (n=5)         & 0.23           & 6.85    & -  & 0.10        & 8.78        &  - & 0.05         & 8.38       &   - \\
%History (n=10)        & 0.35           & 5.46    &  - & 0.12        & 8.39     &   -   & \textbf{0.13}         & 8.25   &     -   \\
%History (n=15)        & 0.40           & 5.44   &  -  & 0.11        & 8.30       &  -  & \textbf{0.11}        & 8.95       &  -  \\

\bottomrule
    \end{tabular}
    \vspace{-0.8em}
    \caption{Results in the non-interactive setting for the knowledge retrieval agent (\textit{upper}) and the diagnosis strategy agent (\textit{bottom}). %, assuming that there are existing \emph{full} patient profiles. 
    $\ddag$ Only Few-shot (Standard, Dyn\_BAII) results are recorded, since the method is consistently better than Dyn\_BERT. All models exhibit similar trends. To give a more concise overview, we only report GPT-4o here. The full set of results can be found in~\autoref{tab:with_patient_profile_full} in Appendix.}
    %Since we discover similar trends for other models, for brevity, we recorded the results of GPT-4o, where the full results are noted in 
    %Appendix~\ref{app:additional_experiments}.}%\cc{could decide which model to present}}
    \label{tab:with_patient_profile}
    \vspace{-1.0em}
\end{table*}

%\subsection{Hyperparameters and Optimization} 
For the knowledge retrieval agent, we limit searches to a maximum of three medical keywords per query. Wikipedia is used as an open-access resource, while PubMed retrieval is restricted to full-text articles from commercially licensed sources,\footnote{We use MediaWiki API: \url{https://en.wikipedia.org/w/api.php} and \textsc{biopython}~\url{https://biopython.org/}.} ensuring that retrieved information is clinically validated and relevant to the diagnostic task. For the diagnosis strategy agent, we take 5 examples for few-shot learning. For dynamic few-shot, we use BioClinicalBERT (BERT)~\citep{alsentzer-etal-2019-publicly} and \textsc{bge-base-en-v1.5} (BAII)~\citep{xiao-etal-2024-bge} embeddings, based on the structure proposed by~\citet{wu2024streambench}. Specifically, it uses L2 distance on normalized embeddings, a similar setting to cosine similarity. With the history taking simulator, we create an iterative environment, which we evaluate at 5, 10, and 15 maximum questions. This is based on prior clinical studies that indicate physicians typically ask fewer than 15 questions per consultation~\citep{ely1999analysis}. This ensures that our model operates within a realistic range, capturing essential patient details without excessive interaction. 
%\textbf{DDxDriver:} 
For MEDDxAgent's iterative learning, we select the optimized history taking simulator and diagnostic agents and experiment on interactive DDx. Our setup is inspired by previous work~\citep{johri2025craftmd}, which demonstrates that updating the patient profile with new history-taking dialogue significantly enhances performance. We experiment with 1 to 3 iterations, with 5 questions per iteration. This aligns with the history-taking simulator setting (5 questions per iteration, max 15 for 3 iterations). Additionally, we set the DDxDriver's instruction for each agent and simulator to a list of length 10.
\vspace{-0.5em}



\section{Discussion}
\label{sec:discussion}
 We compared human interpretation data for numerical utterances to LLMs' interpretations, finding substantial differences.
This manifests in LLMs' tendency toward literal interpretations, reversed halo effects (preferring exact interpretations for round rather than sharp numbers; Exp.~1), and inconsistent affect attribution between literal and hyperbolic utterances (Exp.~2), despite human-like prior representations (Exp.~3).
These findings point to a disconnect in LLM pragmatic reasoning --- despite possessing accurate prior knowledge about prices, affect and utterance probabilities --- and despite this knowledge being structured in a way that could support human-like inference when processed through an RSA framework --- LLMs fail to consistently leverage this information when directly prompted to make pragmatic interpretations. 

Our findings highlight an important methodological contribution for understanding LLM behaviors: by systematically decomposing pragmatic reasoning into testable components (priors, affect mappings, utterance likelihoods, and interpretations), we can precisely locate differences between human and AI reasoning. 
This approach extends beyond traditional behavioral comparisons, allowing us to identify whether differences stem from knowledge gaps or reasoning mechanisms. Such detailed cognitive modeling approaches may prove valuable for understanding other aspects of LLM behavior, particularly in cases where surface-level performance masks deeper processing differences from human cognition.
Importantly, our results demonstrate that cognitively-inspired chain-of-thought prompting can help bridge this gap between knowledge and application. We achieved improved correlations with human judgments by decomposing the RSA model's computational steps into natural language reasoning chains. This success suggests that while LLMs may not naturally develop human-like pragmatic reasoning through training alone, they can successfully implement such reasoning when given appropriate computational frameworks that mirror human cognitive processes.

Based on our results, future research could address several important follow-up questions. For instance, potential training modifications to help LLMs better integrate their prior knowledge and context when interpreting hyperbole could be analyzed. Identifying factors that influence how LLMs apply this knowledge in context is also an open question. We report some exploratory analyses (see supplementary materials) that begin to probe these questions through variations in prompting of the models.

Ultimately, our work demonstrates that evaluating LLMs through the lens of cognitive modeling provides a nuanced understanding of how these models deviate from human understanding. By integrating LLMs with cognitive models of pragmatic language use, we can both critically assess the models' internal consistency and provide a framework for improving their performance in interpreting non-literal language.


\section{Acknowledgments}
PT and MF acknowledge support by the state of Baden-W\"urttemberg through bwHPC and the German Research Foundation (DFG) through grant INST 35/1597-1 FUGG. 
MF is a member of the Machine Learning Cluster of Excellence at University of T\"ubingen, EXC number 2064/1 – Project number 39072764.
\bibliographystyle{apacite}

\setlength{\bibleftmargin}{.125in}
\setlength{\bibindent}{-\bibleftmargin}

\bibliography{CogSci_Template}
\clearpage
\section*{Supplementary Materials}
\begin{figure*}[h!]
    \centering
    \includegraphics[width=0.9\textwidth]{figs/zero-shot-lms.png} 
    \caption{Probabilities of each pair of $(u, s)$, predicted by different LLMs under \textbf{zero-shot} prompting (with $\tau=1$) (facets, x-axis), plotted against human results (y-axis), coded for each type of interpretation (color) and item (dot shape).}
    \label{fig:expt1}
\end{figure*}

\begin{table}[h]
\centering
\begin{tabular}{llll}
\toprule
 \textbf{LLM} & GPT & Claude & Gemini \\ \midrule
 1-shot price prior CoT& 0.7  & 0.596 & 0.774 \\
 1-shot speaker goals CoT & 0.472 & 0.552 & 0.501 \\
 full LM-based RSA & 0.783  & 0.80 & 0.76 \\
 \bottomrule
\end{tabular}
\caption{Correlation between human predictions of $(s,u)$ probabilities, and results from LLMs under different RSA-inspired approaches. Results for ablations of the CoT prompt are presented where only reasoning about the prior (1-shot price prior CoT) or the speaker communicative goal (1-shot speaker goal CoT) are included. LLM-based RSA refers to results of the full RSA model with both LLM priors and LLM speaker likelihoods. \label{tab:gpt-prompting-comparison}}
\end{table}

\subsection{LLM Zero-Shot performance}
\label{sec:app:zero-shot}
To test whether LLMs arrive at non-literal meanings of
numbers when people do, we closely follow the procedure
and the scenarios presented in \citet{kao2014nonliteral}. To this end, we construct zero-shot prompts to sample LLMs' judgments of probabilities of different true prices $s \in S$, given a speaker's utterance mentioning a price $u \in U$. An example prompt is presented in~\autoref{tab:prompts}.
Results of LLM predictions for all items and all $(u,s)$ pairs are shown against human results in~\autoref{fig:expt1}. 
Under zero-shot prompting, LLMs did not show high correlation with human results, instead showing a tendency towards literal interpretations. 
Furthermore, different models exhibited distinct distributional patterns: GPT-4o-mini tended to assign inflated probabilities to individual utterance-meaning pairs, while Gemini-1.5-pro generally exhibited a bimodal distribution of ratings at the ends of the scale.

\subsection{Guiding LLM Interpretation with the RSA model}
\label{sec:prompting}

To test if the computational steps formalized by the RSA model
can be used to guide LLMs’ interpretation of hyperbole and pragmatic halo in a more human-like way, we compare two possible approaches to integrating the RSA model with LLMs. 

First, we construct a one-shot chain-of-thought (CoT) prompt that verbally describes critical components within the RSA model: reasoning about possible speaker goals and priors of prices for an example every-day item (a toaster).
The full prompt is shown in~\autoref{prompt:rsa}.
\begin{figure}[htpb]
\centering
\begin{tcolorbox}[
width=1\linewidth,
title={One-Shot RSA Prompt}]
\fontsize{5pt}{5pt}\selectfont
\ttfamily
\begin{lstlisting}[language={}]
EXAMPLE:
Anne bought a new toaster. A friend asked her, "Was it expensive?" Anne said, "It cost \$1000."
Please provide the probability that Anne thinks that the toaster is expensive.
Let's think step by step and consider Anne's goals. To answer her friend's question, Anne might want to tell her friend the price, so that her friend can judge whether the toaster is expensive or not. 
She could have the goal to communicate the exact price, or to communicate her attitude about the price or both.
Anne said "\$1000", but given general world knowledge, it is unlikely that a toaster costs literally \$1000. Therefore, it is unlikely that Anne wants to communicate the exact price. A toaster that costs \$1000 would be considered expensive, which would be upsetting. Therefore, it is more likely that Anne wants to communicate that she is upset and felt that the toaster was too expensive, using a hyperbole to talk about the price.
Therefore, it is likely that Anne thinks that the toaster is expensive. The answer is: 0.9
A: 0.9
\end{lstlisting}
\end{tcolorbox}
\caption{\textbf{One-Shot RSA Prompt}
The system prompt and one-shot chain-of-thought prompt for teaching a model to simulate an RSA-model.}
\label{prompt:rsa}
\end{figure}

Results reported in~\autoref{tab:cor_table} (1-shot RSA CoT) indicate that the RSA-couched prompting effectively helped to guide LLMs towards more human-like interpretation, improving the correlation between LLM predictions and human data from \citet{kao2014nonliteral}.

To critically assess the robustness of the prompting and which aspects of the prompt really drive the performance improvements, we ablate parts of the prompt corresponding to different computational components of the RSA model.
Specifically, we construct a speaker-goals prompt which only exemplifies reasoning about different speaker goals (see~\autoref{prompt:rsa-qud}), and a priors prompt which only exemplifies reasoning about price priors (see~\autoref{prompt:rsa-priors}).
\begin{figure}[htpb]
\centering
\begin{tcolorbox}[
width=1\linewidth,
title={Ablated QUD-only One-Shot Prompt}]
\fontsize{5pt}{5pt}\selectfont
\ttfamily
\begin{lstlisting}[language={}]
In each scenario, two friends are talking about the price of an item.
Please read the scenarios carefully and provide the probability that the item has the desribed price.
Provide the estimates on a continuous scale between 0 and 1, where 0 stands for "impossible" and 1 stands for "extremely likely".
Write ONLY your final answer as 'A:<rating>'.

EXAMPLE:
Anne bought a new toaster. A friend asked her, "Was it expensive?" Anne said, "It cost $1000."
Please provide the probability that the toaster cost $50.
Let's think step by step and consider the possible communicative goals of Anne.
Anne might want to communicate about the price, about her attitude towards the price, or both.
For communicating the price, she would choose to be precise, ignoring other possible meaning dimesnions. For communicating her attitude, she would choose a an expression that signal attitude, where other possible dimensions like being precise don't matter. For communicating both, she might choose an utterance that trades off both goals. 
Thr utterance seems to fit the goals attitude communication and both. Therefore, the answer is: 0.75
A: 0.75

YOUR TURN:

\end{lstlisting}
\end{tcolorbox}
\caption{\textbf{Ablated QUD-only One-Shot Prompt}
The system prompt and one-shot chain-of-thought prompt for teaching a model to reason about the communicative goals, as suggested by the RSA-model.}
\label{prompt:rsa-qud}
\end{figure}
\begin{figure}[htpb]
\centering
\begin{tcolorbox}[
width=1\linewidth,
title={Ablated Priors-only One-Shot Prompt}]
\fontsize{5pt}{5pt}\selectfont
\ttfamily
\begin{lstlisting}[language={}]
In each scenario, two friends are talking about the price of an item.
Please read the scenarios carefully and provide the probability that the item has the desribed price.
Provide the estimates on a continuous scale between 0 and 1, where 0 stands for "impossible" and 1 stands for "extremely likely".
Write ONLY your final answer as 'A:<rating>'.

EXAMPLE:
Anne bought a new toaster. A friend asked her, "Was it expensive?" Anne said, "It cost $1000."
Please provide the probability that the toaster cost $50.
Let's think step by step and consider the prior probability of toaster prices.
Given general world knowledge, it is unlikely that a toaster costs literally $1000. Rather, a price around $50 would be considered a normal price for a toaster. Therefore, a toaster that costs $1000 would be considered expensive. 
Since Anne stated an unlikely price for the toaster, it is likely that the true price of the toaster was not what would normally be expected a priori. Therefore, the answer is: 0.75
A: 0.75

YOUR TURN:


\end{lstlisting}
\end{tcolorbox}
\caption{\textbf{Ablated Priors-only One-Shot Prompt}
The system prompt and one-shot chain-of-thought prompt for teaching a model to reason about the priors of prices of an item, as suggested by the RSA-model.}
\label{prompt:rsa-priors}
\end{figure}
Results of these ablations as measured by the correlation with human data are presented in~\autoref{tab:gpt-prompting-comparison}.
Compared to the full one-shot CoT prompt, the speaker goals only prompt led to lower correlation between LLM and human data for all LLMs.
The priors only prompt, on the other hand, increased the correlation between LLM and human results more strongly than the full one-shot CoT prompt (see~Table~\ref{tab:cor_table} in main text). 
These ablation results suggest
that LLM performance can be supported through RSA model
inspired prompting, but the prompt components required for substantially increasing LLM performance may not necessarily have to fully replicate all the computational components
needed for explaining human performance.

\subsection{LM-RSA Simulations}
Second, we used the RSA model to quantify the LLMs' internal consistency between its own predicted priors and zero-shot prompting based predictions.
To this end, we used the priors of prices for different items and for priors for affect, given a price, predicted by LLMs, elicited in Experiment~3.
We then fit the RSA model proposed by \citet{kao2014nonliteral} using the priors from each LLM, resulting in the \textit{LM-RSA model}.
The RSA model includes two hyperparameters that were fit to human behavioral data. 
To adjust for biases against using the extreme probability ratings for the $(u,s)$ pairs, a power-law transformation was performed: we
multiplied the predicted probability for each $(u,s)$ pair by a free parameter $\lambda$, and renormalized the probabilities to sum up to 1 for each utterance $u$. 
The $\lambda$ was jointly fit with the model’s cost ratio $C$. 
$C(u) = 1$ was used when $u$ was a round number (divisible by 10) and the cost for sharp utterances was fit to human data.
We tune the cost and $\lambda$ hyperparameters individually for each LM-RSA. 
The optimal $\lambda$ was chosen via search over $[0, 1)$ with steps of 0.01. The optimal $C$ was chosen via search over $[1, 4)$ with steps of 
0.1. The best hyperparameters which were used to produce results reported in the main text are shown in~\autoref{tab:rsa-hyperparams}.
\begin{table}[h]
\centering
\begin{tabular}{lll}
\toprule
                   & $\lambda$ & $C$ \\ 
\midrule
Human priors       & 0.44   & 1.2  \\ 
GPT-4o-mini priors & 0.41   & 1.4   \\ 
Claude-3.5-sonnet priors & 0.39   & 2.0   \\ 
Gemini-1.5-pro priors & 0.38 & 1.1 \\ 
\bottomrule
\end{tabular}
\caption{Best hyperparameters of the RSA model, fit separately for each LLM-RSA model. \label{tab:rsa-hyperparams}}
\end{table}


\subsection{Eliciting LLMs' Utterance Likelihoods}
In Experiment~3, we focused on assessing key aspects that might be the root cause of LLMs' deficiencies in non-lieral number interpretation, informed by the RSA model: 
(1) the price priors that LLMs assign to for each item, (2) the priors of affective responses conditional on item prices, and (3) the LLMs' representations of speaker likelihoods of uttering different prices $u$, given different true price states $s$ and speaker goals. 
The low correlation of LM-RSA, zero-shot LLM and human predictions (Experiment 3) revealed that the challenge in achieving human-like pragmatic interpretation of hyperbole and halo was not due to lack in the models’ price and affect priors.
Therefore, we investigated whether the LLM might lack the third computational component that constitutes the pragmatic interpreter in the RSA model: the pragmatic speaker $S_1$.  
Specifically, we tested whether the LLMs captured the likelihoods of different utterances $u \in U$ conveying the speaker's communicative goal and intended meaning about an item with sufficient accuracy. 

We collected likelihoods of uttering a price $u \in U$, given different states and goals of the speaker based on \textit{zero-shot prompting}, as shown in~\autoref{prompt:affect}.
\begin{figure}[htpb]
\centering
\begin{tcolorbox}[
width=1\linewidth,
title={Pragmatic Speaker Prompt}]
\fontsize{5pt}{5pt}\selectfont
\ttfamily
\begin{lstlisting}[language={}]
In each scenario, two friends are talking about the price of an item.
Please read the scenarios carefully and provide the probability that the speaker would say the following utterance, given their communicative goal and the true price of the item.
Provide the estimates on a continuous scale between 0 and 1, where 0 stands for "impossible" and 1 stands for "extremely likely".
Write ONLY your final answer as 'A:rating'.
Bob bought a laptop. The laptop cost \$100. A friend asked Bob if the laptop was expensive. 
Bob wants to communicate both their attitude towards the price of the laptop they bought and the price of the laptop. 
Bob wants to precisely communicate the price of the laptop they bought.
Bob thinks the laptop is too expensive.
How likely is it that Bob will say: 'The laptop cost \$1000.'?
\end{lstlisting}
\end{tcolorbox}
\caption{\textbf{Pragmatic Speaker Prompt}
The system prompt and one-shot prompt for asking a model to predict the probability of a specific utterance, when communicating \textit{both affect and the exact state}. The sentence about the affect and the preciseness of the price are removed, if the communicative goal only contains one dimension, respectively.}
\label{prompt:affect}
\end{figure}
We used these raw LLM predictions to calculate the speaker likelihood $P_{LLM}(u \mid s, a, g)$, where $g \in G$ is the speaker goal (communicating exact or round price, and communicating the price, the affect, or both), $a$ represented affect and could take on the values 0 or 1, resulting in 12 condition, some of which are identical.
We calculated the final $P_{LLM}(u \mid s, a, g)$ by aggregating over the raw LLM predictions over the irrelevant meaning dimensions as suggested in the RSA model (repeated from main text): 
$$S_1 (u \mid s, a, g) \propto \sum_{s', a'} \delta_{g(s, a) = g(s', a')}L_0(s', a' \mid u) \cdot e^{-c(u)}$$
That is, for instance, when the goal $g$ is to convey affect only, the scores are aggregated and renormalized across the values of $S$.  

We investigated the accuracy of the utterance likelihood representations of the LLMs by fitting a \textit{full LM RSA model}.
The $P_{LLM}$ was used together with LLM priors elicited from the same LLM in Experiment 3 to fit this RSA model (via enumeration) and predict posteriors of different meanings, given utterances. 
 
We report the correlation of the full LM RSA posteriors of all $(u, s)$ with human data in \autoref{tab:gpt-prompting-comparison} (``full LM-based RSA''). 
We found that the full LM-RSA model showed notably higher correlation with human data than the zero-shot predictions of the same LLM models under zero-shot prompting (see~\autoref{tab:cor_table} in the main text).
The correlations are also higher than for the one-shot RSA CoT based results (\autoref{tab:cor_table}), suggesting that more explicit RSA-based task decomposition might guide LLMs towards more human-like interpretation in a more stringent way.
In other words, the computational components of the RSA model provide a structure which allows to build a fully LLM-based system that rationalizes an observed utterance in a more human-like way, based  only on assuming the space of possible speaker goals.
At the same time, the correlations of the full LM-RSA models are lower than for the LM-RSA model (based only on LLM priors), suggesting that LLMs' utterance likelihoods might be less human-like than LLMs' priors. 
Additioanlly, the correlations of full LM-RSA models are ordinally the same as the zero-shot predictions (from Gemini to Claude). 
Taken together, these comparisons suggest that the lack in representing human-like utterance likelihoods might contribute to the LLMs' interpretation difficulties in a zero-shot setting

These detailed analyses open up interesting avenues for investigating why LLM utterance likelihoods differ compared to humans, and whether, e.g., particular training or dataset aspects lead to the discrepancy between humans and LLMs.
We turn to some explorations in the next section.

\subsection{Free Generation of Prices}
Based on the main results, we report another exploratory investigation as to \textit{why} the LLMs may have deficient performance on hyperbole and halo, despite correctly representing component information suggested by then RSA model. 

One potential reason could be that the materials from \citet{kao2014nonliteral} used in Experiment 1 as the state space $S$ were out-of-distribution for the LLMs.
Focusing on GPT-4o-mini, we ran an exploratory free generation study, identifying which prices the LLM would generate under $\tau=1$ when prompted to complete the speaker's utterance about the price of an item, given the speaker's goal (e.g., to convey the state, the affect, or both; being precise or imprecise about the price). 
We then qualitatively assessed whether the produced numbers differed from $U$ = \{50 + k, 500 + k, 1000 + k, 5000 + k, 10000 + k\}, with $k \in \{0, 1, 2, 3\}$.
Additionally, we compared whether LLMs freely generate higher prices when prompted to convey affect (i.e., that the item was expensive), than when prompted without affect . The full prompt can be found in~\autoref{prompt:free-generation}. 
\begin{figure}[htpb]
\centering
\begin{tcolorbox}[
width=1\linewidth,
title={Free Generation Prompt}]
\fontsize{5pt}{5pt}\selectfont
\ttfamily
\begin{lstlisting}[language={}]
In each scenario, two friends are talking about the price of an item.
Please read the scenarios carefully and complete the speaker's utterance with your best guess, given their communicative goal and the true price of the item.
Write ONLY your numerical completion for the utterance as 'A:<completion>'.
\end{lstlisting}
\end{tcolorbox}
\caption{\textbf{Free Generation Prompt}
The system prompt for asking a model to generate the likely utterance mentioning a price, when communicating a particular goal.}
\label{prompt:free-generation}
\end{figure}

Based on one exploratory simulation for all items and prices, we found that LLM produced different prices than in $U$. For instance, the predicted utterances for the electric kettle ranged from \$30 to \$150. The LLM produced higher prices when prompted to communicate that the speaker think the item is expensive (under the goal to communicate affect or both meaning dimensions), than when the affect was not present (mean predicted prices across items: \$342~vs.~\$590). These results suggests that the LLM are, in principle, able to generate hyperbolic utterances, but may be sensitive to the exact price numbers when prompted to interpret utterances or assess the likelihood of particular utterances.

\end{document}


