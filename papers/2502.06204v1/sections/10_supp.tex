\begin{figure*}[h!]
    \centering
    \includegraphics[width=0.9\textwidth]{figs/zero-shot-lms.png} 
    \caption{Probabilities of each pair of $(u, s)$, predicted by different LLMs under \textbf{zero-shot} prompting (with $\tau=1$) (facets, x-axis), plotted against human results (y-axis), coded for each type of interpretation (color) and item (dot shape).}
    \label{fig:expt1}
\end{figure*}

\begin{table}[h]
\centering
\begin{tabular}{llll}
\toprule
 \textbf{LLM} & GPT & Claude & Gemini \\ \midrule
 1-shot price prior CoT& 0.7  & 0.596 & 0.774 \\
 1-shot speaker goals CoT & 0.472 & 0.552 & 0.501 \\
 full LM-based RSA & 0.783  & 0.80 & 0.76 \\
 \bottomrule
\end{tabular}
\caption{Correlation between human predictions of $(s,u)$ probabilities, and results from LLMs under different RSA-inspired approaches. Results for ablations of the CoT prompt are presented where only reasoning about the prior (1-shot price prior CoT) or the speaker communicative goal (1-shot speaker goal CoT) are included. LLM-based RSA refers to results of the full RSA model with both LLM priors and LLM speaker likelihoods. \label{tab:gpt-prompting-comparison}}
\end{table}

\subsection{LLM Zero-Shot performance}
\label{sec:app:zero-shot}
To test whether LLMs arrive at non-literal meanings of
numbers when people do, we closely follow the procedure
and the scenarios presented in \citet{kao2014nonliteral}. To this end, we construct zero-shot prompts to sample LLMs' judgments of probabilities of different true prices $s \in S$, given a speaker's utterance mentioning a price $u \in U$. An example prompt is presented in~\autoref{tab:prompts}.
Results of LLM predictions for all items and all $(u,s)$ pairs are shown against human results in~\autoref{fig:expt1}. 
Under zero-shot prompting, LLMs did not show high correlation with human results, instead showing a tendency towards literal interpretations. 
Furthermore, different models exhibited distinct distributional patterns: GPT-4o-mini tended to assign inflated probabilities to individual utterance-meaning pairs, while Gemini-1.5-pro generally exhibited a bimodal distribution of ratings at the ends of the scale.

\subsection{Guiding LLM Interpretation with the RSA model}
\label{sec:prompting}

To test if the computational steps formalized by the RSA model
can be used to guide LLMs’ interpretation of hyperbole and pragmatic halo in a more human-like way, we compare two possible approaches to integrating the RSA model with LLMs. 

First, we construct a one-shot chain-of-thought (CoT) prompt that verbally describes critical components within the RSA model: reasoning about possible speaker goals and priors of prices for an example every-day item (a toaster).
The full prompt is shown in~\autoref{prompt:rsa}.
\begin{figure}[htpb]
\centering
\begin{tcolorbox}[
width=1\linewidth,
title={One-Shot RSA Prompt}]
\fontsize{5pt}{5pt}\selectfont
\ttfamily
\begin{lstlisting}[language={}]
EXAMPLE:
Anne bought a new toaster. A friend asked her, "Was it expensive?" Anne said, "It cost \$1000."
Please provide the probability that Anne thinks that the toaster is expensive.
Let's think step by step and consider Anne's goals. To answer her friend's question, Anne might want to tell her friend the price, so that her friend can judge whether the toaster is expensive or not. 
She could have the goal to communicate the exact price, or to communicate her attitude about the price or both.
Anne said "\$1000", but given general world knowledge, it is unlikely that a toaster costs literally \$1000. Therefore, it is unlikely that Anne wants to communicate the exact price. A toaster that costs \$1000 would be considered expensive, which would be upsetting. Therefore, it is more likely that Anne wants to communicate that she is upset and felt that the toaster was too expensive, using a hyperbole to talk about the price.
Therefore, it is likely that Anne thinks that the toaster is expensive. The answer is: 0.9
A: 0.9
\end{lstlisting}
\end{tcolorbox}
\caption{\textbf{One-Shot RSA Prompt}
The system prompt and one-shot chain-of-thought prompt for teaching a model to simulate an RSA-model.}
\label{prompt:rsa}
\end{figure}

Results reported in~\autoref{tab:cor_table} (1-shot RSA CoT) indicate that the RSA-couched prompting effectively helped to guide LLMs towards more human-like interpretation, improving the correlation between LLM predictions and human data from \citet{kao2014nonliteral}.

To critically assess the robustness of the prompting and which aspects of the prompt really drive the performance improvements, we ablate parts of the prompt corresponding to different computational components of the RSA model.
Specifically, we construct a speaker-goals prompt which only exemplifies reasoning about different speaker goals (see~\autoref{prompt:rsa-qud}), and a priors prompt which only exemplifies reasoning about price priors (see~\autoref{prompt:rsa-priors}).
\begin{figure}[htpb]
\centering
\begin{tcolorbox}[
width=1\linewidth,
title={Ablated QUD-only One-Shot Prompt}]
\fontsize{5pt}{5pt}\selectfont
\ttfamily
\begin{lstlisting}[language={}]
In each scenario, two friends are talking about the price of an item.
Please read the scenarios carefully and provide the probability that the item has the desribed price.
Provide the estimates on a continuous scale between 0 and 1, where 0 stands for "impossible" and 1 stands for "extremely likely".
Write ONLY your final answer as 'A:<rating>'.

EXAMPLE:
Anne bought a new toaster. A friend asked her, "Was it expensive?" Anne said, "It cost $1000."
Please provide the probability that the toaster cost $50.
Let's think step by step and consider the possible communicative goals of Anne.
Anne might want to communicate about the price, about her attitude towards the price, or both.
For communicating the price, she would choose to be precise, ignoring other possible meaning dimesnions. For communicating her attitude, she would choose a an expression that signal attitude, where other possible dimensions like being precise don't matter. For communicating both, she might choose an utterance that trades off both goals. 
Thr utterance seems to fit the goals attitude communication and both. Therefore, the answer is: 0.75
A: 0.75

YOUR TURN:

\end{lstlisting}
\end{tcolorbox}
\caption{\textbf{Ablated QUD-only One-Shot Prompt}
The system prompt and one-shot chain-of-thought prompt for teaching a model to reason about the communicative goals, as suggested by the RSA-model.}
\label{prompt:rsa-qud}
\end{figure}
\begin{figure}[htpb]
\centering
\begin{tcolorbox}[
width=1\linewidth,
title={Ablated Priors-only One-Shot Prompt}]
\fontsize{5pt}{5pt}\selectfont
\ttfamily
\begin{lstlisting}[language={}]
In each scenario, two friends are talking about the price of an item.
Please read the scenarios carefully and provide the probability that the item has the desribed price.
Provide the estimates on a continuous scale between 0 and 1, where 0 stands for "impossible" and 1 stands for "extremely likely".
Write ONLY your final answer as 'A:<rating>'.

EXAMPLE:
Anne bought a new toaster. A friend asked her, "Was it expensive?" Anne said, "It cost $1000."
Please provide the probability that the toaster cost $50.
Let's think step by step and consider the prior probability of toaster prices.
Given general world knowledge, it is unlikely that a toaster costs literally $1000. Rather, a price around $50 would be considered a normal price for a toaster. Therefore, a toaster that costs $1000 would be considered expensive. 
Since Anne stated an unlikely price for the toaster, it is likely that the true price of the toaster was not what would normally be expected a priori. Therefore, the answer is: 0.75
A: 0.75

YOUR TURN:


\end{lstlisting}
\end{tcolorbox}
\caption{\textbf{Ablated Priors-only One-Shot Prompt}
The system prompt and one-shot chain-of-thought prompt for teaching a model to reason about the priors of prices of an item, as suggested by the RSA-model.}
\label{prompt:rsa-priors}
\end{figure}
Results of these ablations as measured by the correlation with human data are presented in~\autoref{tab:gpt-prompting-comparison}.
Compared to the full one-shot CoT prompt, the speaker goals only prompt led to lower correlation between LLM and human data for all LLMs.
The priors only prompt, on the other hand, increased the correlation between LLM and human results more strongly than the full one-shot CoT prompt (see~Table~\ref{tab:cor_table} in main text). 
These ablation results suggest
that LLM performance can be supported through RSA model
inspired prompting, but the prompt components required for substantially increasing LLM performance may not necessarily have to fully replicate all the computational components
needed for explaining human performance.

\subsection{LM-RSA Simulations}
Second, we used the RSA model to quantify the LLMs' internal consistency between its own predicted priors and zero-shot prompting based predictions.
To this end, we used the priors of prices for different items and for priors for affect, given a price, predicted by LLMs, elicited in Experiment~3.
We then fit the RSA model proposed by \citet{kao2014nonliteral} using the priors from each LLM, resulting in the \textit{LM-RSA model}.
The RSA model includes two hyperparameters that were fit to human behavioral data. 
To adjust for biases against using the extreme probability ratings for the $(u,s)$ pairs, a power-law transformation was performed: we
multiplied the predicted probability for each $(u,s)$ pair by a free parameter $\lambda$, and renormalized the probabilities to sum up to 1 for each utterance $u$. 
The $\lambda$ was jointly fit with the model’s cost ratio $C$. 
$C(u) = 1$ was used when $u$ was a round number (divisible by 10) and the cost for sharp utterances was fit to human data.
We tune the cost and $\lambda$ hyperparameters individually for each LM-RSA. 
The optimal $\lambda$ was chosen via search over $[0, 1)$ with steps of 0.01. The optimal $C$ was chosen via search over $[1, 4)$ with steps of 
0.1. The best hyperparameters which were used to produce results reported in the main text are shown in~\autoref{tab:rsa-hyperparams}.
\begin{table}[h]
\centering
\begin{tabular}{lll}
\toprule
                   & $\lambda$ & $C$ \\ 
\midrule
Human priors       & 0.44   & 1.2  \\ 
GPT-4o-mini priors & 0.41   & 1.4   \\ 
Claude-3.5-sonnet priors & 0.39   & 2.0   \\ 
Gemini-1.5-pro priors & 0.38 & 1.1 \\ 
\bottomrule
\end{tabular}
\caption{Best hyperparameters of the RSA model, fit separately for each LLM-RSA model. \label{tab:rsa-hyperparams}}
\end{table}

