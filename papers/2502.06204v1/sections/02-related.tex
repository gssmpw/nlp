
Our work builds on the computational cognitive model developed by \citet{kao2014nonliteral} in the Rational Speech Act (RSA) framework, which explains human interpretation of hyperbolic numerical expressions in terms of reasoning about the speaker's  communicative intent and prior world knowledge.
Specifically, the RSA framework models pragmatic communication as recursive rational reasoning between speakers and listeners \citep{goodman2016pragmatic, degen2023rational}. In the basic RSA model, a pragmatic speaker $S_1$ chooses utterances $u$ to inform a literal listener $L_0$ of a meaning $m$, minimizing the listener's surprisal: 
$$S_1(u \mid m) = \frac{\exp(\log(P(m \mid [\![u]\!])-C(u)))}{\sum_{u'} \exp(\log( P(m \mid [\![u']\!] ) - C(u')))} $$
where $C(u)$ is the cost of the utterance and $[\![u]\!]$ is the set of meanings compatible with $u$. A pragmatic listener $L_1$ then performs Bayesian inference over possible meanings by reasoning about this speaker:
$$ L_1(m \mid u)\propto S_1(u \mid m) P(m)$$
where $P(m)$ is the prior probability of a meaning.

To model hyperbolic interpretations like in our coffee example, \citet{kao2014nonliteral} extend this framework to capture how a single utterance can convey multiple meanings. Their extended model represents a multi-dimensional meaning space where an utterance about price conveys both the actual price state $s$ (e.g., the literal cost of the coffee) and the speaker's affect $a$ (e.g., that it was surprisingly expensive). The model also incorporates different communicative goals $g$, allowing the speaker to emphasize either or both of these dimensions: %\mf{best replace $L_0$ here as well if you followed my suggestion before for vanilla RSA}
$$S_1 (u \mid s, a, g) \propto \sum_{s', a'} \delta_{g(s, a) = g(s', a')}P(s', a' \mid [\![u]\!]) \cdot e^{-c(u)}$$
The pragmatic listener then interprets the utterance through joint inference over the speaker's goal and intended meaning:
$$L_1(s, a \mid u) \propto \sum_{g} S_1(u \mid s, a, g) P_{S}(s) P_{A}(a\mid s) P_{G}(g) $$
where $P_{S}$ represents prior beliefs about prices (e.g., how much coffee typically costs), $P_{A}$ captures the relationship between prices and affect (e.g., when a coffee price would be considered exasperating), and $P_{G}$ represents the prior over different communicative goals, assumed to be uniform. \citet{kao2014nonliteral} showed that this model successfully captures how humans interpret both hyperbolic expressions and the pragmatic differences between round and precise numbers, with model predictions strongly correlating with human judgments.

To explore whether the RSA model can guide LLMs toward more human-like interpretations, we develop a chain-of-thought (CoT) prompt that explicitly walks through key reasoning steps: considering possible speaker intentions, evaluating prior price expectations, and interpreting the utterance accordingly. We demonstrate this reasoning process with an example item (see supplementary) before eliciting the model's interpretation.

