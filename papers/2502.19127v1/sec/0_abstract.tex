\begin{abstract}
Large Language Models (LLMs) often struggle to align their responses with objective facts, resulting in the issue of \textbf{factual hallucinations}, which can be difficult to detect and mislead users without relevant knowledge. While post-training techniques have been employed to mitigate the issue, existing methods usually suffer from poor generalization and trade-offs in different capabilities. In this paper, we propose to address it by directly augmenting LLM's fundamental ability to precisely leverage its existing \emph{memory}--the knowledge acquired from pre-training data. We introduce \textbf{self-memory alignment (SMA)}, which fine-tunes the model on self-generated responses to precise and simple factual questions through preference optimization. Furthermore, we construct \textbf{FactualBench}, a comprehensive and precise factual QA dataset containing 181k Chinese data spanning 21 domains, to facilitate both evaluation and training. Extensive experiments show that SMA significantly improves LLMs' overall performance, with consistent enhancement across various benchmarks concerning factuality, as well as helpfulness and comprehensive skills.
\end{abstract}