\documentclass{article}
\usepackage{arxiv}

\usepackage[utf8]{inputenc} % allow utf-8 input
\usepackage[T1]{fontenc}    % use 8-bit T1 fonts
\usepackage{hyperref}       % hyperlinks
\usepackage{url}            % simple URL typesetting
\usepackage{booktabs}       % professional-quality tables
\usepackage{amsfonts}       % blackboard math symbols
\usepackage{nicefrac}       % compact symbols for 1/2, etc.
\usepackage{microtype}      % microtypography
\usepackage{lipsum}		    % Can be removed after putting your text content
\usepackage{graphicx}
\usepackage{doi}
\usepackage{multirow}
\usepackage{multicol}
\usepackage{xcolor}
\usepackage{listings}
\usepackage{subcaption} % Ensure proper usage of subfigure captions
\usepackage[normalem]{ulem}
\usepackage{amsmath,amsfonts,bm}
\usepackage{amssymb,amsthm}
\usepackage{enumitem}
\usepackage{algorithm}
\usepackage{algorithmic}
\usepackage{footnote}
\makesavenoteenv{tabular}
\usepackage[
    backend=biber,
    citestyle=authoryear,
    bibstyle=authortitle,
    mincitenames=1,
    maxcitenames=1,
    maxbibnames=3,
]{biblatex}
\addbibresource{template.bib}

\newcommand{\citep}[1]{\parencite{#1}}
\newcommand{\ours}{{Moonlight}}

\usepackage{listings}
\lstset{
    basicstyle=\ttfamily\small, % 字体和大小
    commentstyle=\color{gray}, % 注释颜色
    keywordstyle=\color{blue}, % 关键字颜色
    stringstyle=\color{red}, % 字符串颜色
    breaklines=true, % 自动换行
    numbers=left, % 行号在左侧
    numberstyle=\tiny\color{gray}, % 行号样式
    frame=shadowbox, % 添加阴影边框
    rulesepcolor=\color{blue}, % 边框颜色
    xleftmargin=10pt, % 左边距
    xrightmargin=10pt % 右边距
}


\newtheoremstyle{italicstyle} % 风格名称
  {3pt} % 空白空间在定理上方
  {3pt} % 空白空间在定理下方
  {\itshape} % 定理内容的字体
  {} % 缩进
  {\itshape} % 定理标题的字体
  {.} % 定理标题后的标点符号
  {.5em} % 定理标题和内容之间的间距
  {} % 定理标题后的额外内容
\theoremstyle{italicstyle}
\newtheorem{lemma}{Lemma}
\newtheorem{principle}{Principle}

% Define four custom colors (dark blue, dark green, dark red, dark purple as examples)
\definecolor{darkblue}{rgb}{0.0, 0.0, 0.5}
\definecolor{darkgreen}{rgb}{0.0, 0.5, 0.0}
\definecolor{darkred}{rgb}{0.5, 0.0, 0.0}
\definecolor{darkpurple}{rgb}{0.5, 0.0, 0.5}
\newcommand{\todo}[1]{\textcolor{darkred}{[TODO]: #1}}
\newcommand{\wyx}[1]{{\color{red}[yuxin: #1]}}

\setlist[itemize,1]{leftmargin=\dimexpr 18pt}
\setlist[enumerate,1]{leftmargin=\dimexpr 18pt}

\title{
\raisebox{-0.1\height}{\includegraphics[width=0.032\textwidth]{figures/logo.png}} %
Muon is Scalable for LLM Training
}


\author{Kimi Team}

% Uncomment to remove the date
\date{}

% Uncomment to override  the `A preprint' in the header
\renewcommand{\headeright}{Technical Report}
\renewcommand{\undertitle}{Technical Report}
\renewcommand{\shorttitle}{\raisebox{-0.12\height}{\includegraphics[width=0.02\textwidth]{figures/logo.png}}
Muon is Scalable for LLM Training}

\author{%
    \textbf{Jingyuan Liu}$^{1}$ \quad \textbf{Jianlin Su}$^{1}$ \quad \textbf{Xingcheng Yao}$^{2}$ \quad \textbf{Zhejun Jiang}$^{1}$ \quad \textbf{Guokun Lai}$^{1}$ \quad \textbf{Yulun Du}$^{1}$ \\
    \textbf{Yidao Qin}$^{1}$ \quad \textbf{Weixin Xu}$^{1}$ \quad \textbf{Enzhe Lu}$^{1}$ \quad \textbf{Junjie Yan}$^{1}$ \quad \textbf{Yanru Chen}$^{1}$ \quad \textbf{Huabin Zheng}$^{1}$ \\ \quad \textbf{Yibo Liu}$^{1}$ 
    \quad \textbf{Shaowei Liu}$^{1}$ \quad \textbf{Bohong Yin}$^{1}$ \quad \textbf{Weiran He}$^{1}$ \quad \textbf{Han Zhu}$^{1}$ \quad \textbf{Yuzhi Wang}$^{1}$ \quad \\ \textbf{Jianzhou Wang}$^{1}$ 
    \textbf{Mengnan Dong}$^{1}$ \quad \textbf{Zheng Zhang}$^{1}$ \quad \textbf{Yongsheng Kang}$^{1}$ \quad \textbf{Hao Zhang}$^{1}$ \quad \\ \textbf{Xinran Xu}$^{1}$ 
    \quad \textbf{Yutao Zhang}$^{1}$ \quad \textbf{Yuxin Wu}$^{1}$  \quad \textbf{Xinyu Zhou}$^{1}$ \thanks{Corresponding author: \texttt{zhouxinyu@moonshot.cn}} \quad \textbf{Zhilin Yang}$^{1} $
    \\[2ex]
    $^1$ Moonshot AI \quad $^2$ UCLA \quad
}



\begin{document}
\maketitle

\vspace{-10pt}
\begin{abstract}

Recently, the Muon optimizer~\citep{jordan2024muon} based on matrix orthogonalization has demonstrated strong results in training small-scale language models, but the scalability to larger models has not been proven. We identify two crucial techniques for scaling up Muon: (1) adding weight decay and (2) carefully adjusting the per-parameter update scale. These techniques allow Muon to work out-of-the-box on large-scale training without the need of hyper-parameter tuning. Scaling law experiments indicate that Muon achieves $\sim\!2\times$ computational efficiency compared to AdamW with compute optimal training.
Based on these improvements, we introduce \ours, a 3B/16B-parameter Mixture-of-Expert (MoE) model trained with 5.7T tokens using Muon. Our model improves the current Pareto frontier, achieving better performance with much fewer training FLOPs compared to prior models.
We open-source our distributed Muon implementation that is memory optimal and communication efficient. We also release the pretrained, instruction-tuned, and intermediate checkpoints to support future research.

\end{abstract}





\section{Introduction}

Despite the remarkable capabilities of large language models (LLMs)~\cite{DBLP:conf/emnlp/QinZ0CYY23,DBLP:journals/corr/abs-2307-09288}, they often inevitably exhibit hallucinations due to incorrect or outdated knowledge embedded in their parameters~\cite{DBLP:journals/corr/abs-2309-01219, DBLP:journals/corr/abs-2302-12813, DBLP:journals/csur/JiLFYSXIBMF23}.
Given the significant time and expense required to retrain LLMs, there has been growing interest in \emph{model editing} (a.k.a., \emph{knowledge editing})~\cite{DBLP:conf/iclr/SinitsinPPPB20, DBLP:journals/corr/abs-2012-00363, DBLP:conf/acl/DaiDHSCW22, DBLP:conf/icml/MitchellLBMF22, DBLP:conf/nips/MengBAB22, DBLP:conf/iclr/MengSABB23, DBLP:conf/emnlp/YaoWT0LDC023, DBLP:conf/emnlp/ZhongWMPC23, DBLP:conf/icml/MaL0G24, DBLP:journals/corr/abs-2401-04700}, 
which aims to update the knowledge of LLMs cost-effectively.
Some existing methods of model editing achieve this by modifying model parameters, which can be generally divided into two categories~\cite{DBLP:journals/corr/abs-2308-07269, DBLP:conf/emnlp/YaoWT0LDC023}.
Specifically, one type is based on \emph{Meta-Learning}~\cite{DBLP:conf/emnlp/CaoAT21, DBLP:conf/acl/DaiDHSCW22}, while the other is based on \emph{Locate-then-Edit}~\cite{DBLP:conf/acl/DaiDHSCW22, DBLP:conf/nips/MengBAB22, DBLP:conf/iclr/MengSABB23}. This paper primarily focuses on the latter.

\begin{figure}[t]
  \centering
  \includegraphics[width=0.48\textwidth]{figures/demonstration.pdf}
  \vspace{-4mm}
  \caption{(a) Comparison of regular model editing and EAC. EAC compresses the editing information into the dimensions where the editing anchors are located. Here, we utilize the gradients generated during training and the magnitude of the updated knowledge vector to identify anchors. (b) Comparison of general downstream task performance before editing, after regular editing, and after constrained editing by EAC.}
  \vspace{-3mm}
  \label{demo}
\end{figure}

\emph{Sequential} model editing~\cite{DBLP:conf/emnlp/YaoWT0LDC023} can expedite the continual learning of LLMs where a series of consecutive edits are conducted.
This is very important in real-world scenarios because new knowledge continually appears, requiring the model to retain previous knowledge while conducting new edits. 
Some studies have experimentally revealed that in sequential editing, existing methods lead to a decrease in the general abilities of the model across downstream tasks~\cite{DBLP:journals/corr/abs-2401-04700, DBLP:conf/acl/GuptaRA24, DBLP:conf/acl/Yang0MLYC24, DBLP:conf/acl/HuC00024}. 
Besides, \citet{ma2024perturbation} have performed a theoretical analysis to elucidate the bottleneck of the general abilities during sequential editing.
However, previous work has not introduced an effective method that maintains editing performance while preserving general abilities in sequential editing.
This impacts model scalability and presents major challenges for continuous learning in LLMs.

In this paper, a statistical analysis is first conducted to help understand how the model is affected during sequential editing using two popular editing methods, including ROME~\cite{DBLP:conf/nips/MengBAB22} and MEMIT~\cite{DBLP:conf/iclr/MengSABB23}.
Matrix norms, particularly the L1 norm, have been shown to be effective indicators of matrix properties such as sparsity, stability, and conditioning, as evidenced by several theoretical works~\cite{kahan2013tutorial}. In our analysis of matrix norms, we observe significant deviations in the parameter matrix after sequential editing.
Besides, the semantic differences between the facts before and after editing are also visualized, and we find that the differences become larger as the deviation of the parameter matrix after editing increases.
Therefore, we assume that each edit during sequential editing not only updates the editing fact as expected but also unintentionally introduces non-trivial noise that can cause the edited model to deviate from its original semantics space.
Furthermore, the accumulation of non-trivial noise can amplify the negative impact on the general abilities of LLMs.

Inspired by these findings, a framework termed \textbf{E}diting \textbf{A}nchor \textbf{C}ompression (EAC) is proposed to constrain the deviation of the parameter matrix during sequential editing by reducing the norm of the update matrix at each step. 
As shown in Figure~\ref{demo}, EAC first selects a subset of dimension with a high product of gradient and magnitude values, namely editing anchors, that are considered crucial for encoding the new relation through a weighted gradient saliency map.
Retraining is then performed on the dimensions where these important editing anchors are located, effectively compressing the editing information.
By compressing information only in certain dimensions and leaving other dimensions unmodified, the deviation of the parameter matrix after editing is constrained. 
To further regulate changes in the L1 norm of the edited matrix to constrain the deviation, we incorporate a scored elastic net ~\cite{zou2005regularization} into the retraining process, optimizing the previously selected editing anchors.

To validate the effectiveness of the proposed EAC, experiments of applying EAC to \textbf{two popular editing methods} including ROME and MEMIT are conducted.
In addition, \textbf{three LLMs of varying sizes} including GPT2-XL~\cite{radford2019language}, LLaMA-3 (8B)~\cite{llama3} and LLaMA-2 (13B)~\cite{DBLP:journals/corr/abs-2307-09288} and \textbf{four representative tasks} including 
natural language inference~\cite{DBLP:conf/mlcw/DaganGM05}, 
summarization~\cite{gliwa-etal-2019-samsum},
open-domain question-answering~\cite{DBLP:journals/tacl/KwiatkowskiPRCP19},  
and sentiment analysis~\cite{DBLP:conf/emnlp/SocherPWCMNP13} are selected to extensively demonstrate the impact of model editing on the general abilities of LLMs. 
Experimental results demonstrate that in sequential editing, EAC can effectively preserve over 70\% of the general abilities of the model across downstream tasks and better retain the edited knowledge.

In summary, our contributions to this paper are three-fold:
(1) This paper statistically elucidates how deviations in the parameter matrix after editing are responsible for the decreased general abilities of the model across downstream tasks after sequential editing.
(2) A framework termed EAC is proposed, which ultimately aims to constrain the deviation of the parameter matrix after editing by compressing the editing information into editing anchors. 
(3) It is discovered that on models like GPT2-XL and LLaMA-3 (8B), EAC significantly preserves over 70\% of the general abilities across downstream tasks and retains the edited knowledge better.
\section{Methods}

\subsection{Background}

\paragraph{The Muon Optimizer}
\label{sec:analysis:background}
Muon~\citep{jordan2024muon} has recently been proposed to optimize neural network weights representable as matrices. At iteration $t$, given current weight $\mathbf{W}_{t-1}$, momentum $\mu$, learning rate $\eta_t$ and objective $\mathcal{L}_t$, the update rule of the Muon optimizer can be stated as follows:
\begin{align}
    \mathbf{M}_t &= \mu \mathbf{M}_{t-1} + \nabla\mathcal{L}_t(\mathbf{W}_{t-1}) \notag \\
    \mathbf{O}_t &= \text{Newton-Schulz}(\mathbf{M}_t)\text{\footnotemark[1]} \label{eq:Ot}\\
    \mathbf{W}_t &= \mathbf{W}_{t-1} - \eta_t \mathbf{O}_t \notag
\end{align}
 Here, $\mathbf{M}_t$ is the momentum of gradient at iteration $t$, set as a zero matrix when $t = 0$. In Equation~\ref{eq:Ot}, a Newton-Schulz iteration process~\citep{bernstein2024oldoptimizernewnorm} is adopted to approximately solve $(\mathbf{M}_t \mathbf{M}^{\mathrm{T}}_t)^{-1/2}\mathbf{M}_t$\footnotetext[1] {In practice, we follow~\citep{jordan2024muon} to use a Nesterov-style momentum by putting $\mu \mathbf{M}_t + \nabla\mathcal{L}_t(\mathbf{W}_{t-1})$ to the Newton-Schulz iteration instead of $\mathbf{M}_t$.}. Let $\mathbf{U}\mathbf{\Sigma} \mathbf{V}^\mathrm{T} = \mathbf{M}_t$ be the singular value decomposition (SVD) of $\mathbf{M}_t$, we will have $(\mathbf{M}_t \mathbf{M}^{\mathrm{T}}_t)^{-1/2}\mathbf{M}_t = \mathbf{U}\mathbf{V^T}$, which orthogonalizes $\mathbf{M}_t$. Intuitively, orthogonalization can ensure that the update matrices are isomorphic, preventing the weight from learning along a few dominant directions~\citep{jordan2024muon}.

\paragraph{Newton-Schulz Iterations for Matrix Orthogonalization}
Equation~\ref{eq:Ot} is calculated in an iterative process. At the beginning, we set $\mathbf{X}_0 = \mathbf{M}_t / \|\mathbf{M}_t\|_\mathrm{F}$. Then, at each iteration $k$, we update $\mathbf{X}_k$ from $\mathbf{X}_{k-1}$ as follows:
\begin{align}
    \mathbf{X}_k &= a \mathbf{X}_{k-1} + b (\mathbf{X}_{k-1} \mathbf{X}_{k-1}^\mathrm{T}) \mathbf{X}_{k-1} + c (\mathbf{X}_{k-1} \mathbf{X}_{k-1}^\mathrm{T})^2 \mathbf{X}_{k-1} \label{eq:iteration}
\end{align}
where $\mathbf{X}_N$ is the result of such process after $N$ iteration steps.
Here $a$, $b$, $c$ are coefficients. In order to ensure the correct convergence of Equation~\ref{eq:iteration}, we need to tune the coefficients so that the polynomial $f(x) = a x + b x^3 + c x^5$ has a fixed point near 1. In the original design of \cite{jordan2024muon}, the coefficients are set to $a = 3.4445$, $b = -4.7750$, $c = 2.0315$ in order to make the iterative process converge faster for small initial singular values. In this work, we follow the same setting of coefficients.

\paragraph{Steepest Descent Under Norm Constraints}
\cite{bernstein2024oldoptimizernewnorm} proposed to view the optimization process in deep learning as steepest descent under norm constraints. From this perspective, we can view the difference between Muon and Adam~\citep{adam2015kingma, loshchilov2018decoupled} as the difference in norm constraints. Whereas Adam is a steepest descent under the a norm constraint dynamically adjusted from a Max-of-Max norm, Muon offers a norm constraint that lies in a static range of Schatten-$p$ norm for some large $p$~\citep{muoncase2024cesista}. When equation~\ref{eq:Ot} is accurately computed, the norm constraint offered by Muon will be the spectral norm. Weights of neural networks are used as operators on the input space or the hidden space, which are usually (locally) Euclidean~\citep{cesista2024firstordernormedopt}, so the norm constraint on weights should be an induced operator norm (or spectral norm for weight matrices). In this sense, the norm constraint offered by Muon is more reasonable than that offered by Adam.

\subsection{Scaling Up Muon}
\label{sec:analysis:rms}

\paragraph{Weight Decay}

While Muon performs significantly better than AdamW on a small scale as shown by \cite{jordan2024muon}, we found the performance gains diminish when we scale up to train a larger model with more tokens. We observed that both the weight and the layer output's RMS keep growing to a large scale, exceeding the high-precision range of bf16, which might hurt the model's performance. To resolve this issue, we introduced the standard AdamW (\cite{loshchilov2018decoupled}) weight decay mechanism into Muon\footnote{The original implementation of Muon omits weight decay. A recent concurrent work in Muon incorporates weight decay and demonstrates improved performance. See \href{https://github.com/KellerJordan/Muon/commit/e0ffefd4f7ea88f2db724caa2c7cfe859155995d}{this commit} and \href{https://x.com/kellerjordan0/status/1888320690543284449}{this discussion}.}. 


\begin{align}
\label{equation:weightdecay}
    \mathbf{W}_t = \mathbf{W}_{t-1} - \eta_t (\mathbf{O}_t + \lambda \mathbf{W}_{t-1})
\end{align}

We experimented on Muon both with and without weight decay to understand its impact on the training dynamics of LLMs. Based on our scaling law research in Sec \ref{sec:exp:moonscalinglaw}, we trained an 800M parameters model with 100B tokens ($\sim5\times$ optimal training tokens). Figure \ref{fig_weight_decay} shows validation loss curves of the model trained with AdamW, vanilla Muon (without weight decay), and Muon with weight decay. While vanilla Muon initially converges faster, we observed that some model weights grew too large over time, potentially limiting the model's long-term performances. Adding weight decay addressed this issue - the results demonstrate that Muon with weight decay outperforms both vanilla Muon and AdamW, achieving lower validation loss in the over-train regime. Therefore, we adjusted our update rule to equation \ref{equation:weightdecay}, where $\lambda$ is the weight decay ratio.


\begin{figure}[t]
    \centering
    \includegraphics[width=0.8\textwidth]{figures/fig_weight_decay.pdf}
    \caption{\small Validation loss curves for AdamW (\textcolor[HTML]{2ecc71}{green}), Muon without weight decay (\textcolor[HTML]{e74c3c}{red}), and Muon with weight decay (\textcolor[HTML]{3498db}{blue}).} 
    \label{fig_weight_decay} 
\end{figure}



\paragraph{Consistent update RMS}
An important property of Adam and AdamW (\cite{adam2015kingma}, \cite{loshchilov2018decoupled}) is that they maintain a theoretical update RMS around 1\footnote{Due to Adam's $\beta_1 < \beta_2$ and $\epsilon > 0$, the actual update RMS is usually less than 1.}. However, we show that Muon's update RMS varies depending on the shape of the parameters, according to the following lemma:

\begin{lemma}
\label{lemma:updaterms}
For a full-rank matrix parameter of shape $[A, B]$, its theoretical Muon update RMS is $\sqrt{1/\max(A,B)}$ .
\end{lemma}

The proof can be found in the Appendix \ref{sec:appendix:updaterms}. We monitored Muon's update RMS during training and found it typically close to the theoretical value given above. We note that such inconsistency can be problematic when scaling up the model size:

\begin{itemize}
    \item When $\max(A,B)$ is too large, e.g. the dense MLP matrix, the updates become too small, thus limiting the model's representational capacity and leading to suboptimal performances; 
    
    \item When $\max(A,B)$ is too small, e.g. treating each KV head in GQA (\cite{shazeer2019fasttransformerdecodingwritehead}) or MLA (\cite{deepseekai2024deepseekv3technicalreport}) as a separate parameter, the updates become too large, thus causing training instabilities and leading to suboptimal performances as well.
\end{itemize}

In order to maintain consistent update RMS among matrices of different shapes, we 
propose to scale the Muon update for each matrix by its $\sqrt{\max(A, B)}$ to cancel the effect of Lemma~\ref{lemma:updaterms} \footnote{\cite{jordan2024muon}'s original implementation scales the updates by $\sqrt{\max(1, A/B)}$, which is equivalent to our proposal (up to a global scale) if all matrices have the same second dimension; \cite{pethick2025trainingdeeplearningmodels} and \cite{JiachengX} discussed a similar issue on update scaling factors concurrently to our work. } . 
Experiments in Sec~\ref{sec:exp:rms} show that this strategy is beneficial for optimization.

\paragraph{Matching update RMS of AdamW}

Muon is designed to update matrix-based parameters. In practice, AdamW is used in couple with Muon to handle non-matrix based parameters, like RMSNorm, LM head, and embedding parameters. 
We would like the optimizer hyper-parameters (learning rate $\eta$, weight decay $\lambda$) to be shared among
matrix and non-matrix parameters. 

We propose to match Muon's update RMS to be similar to that of AdamW. From empirical observations, AdamW's update RMS is usually around 0.2 to 0.4. Therefore, we scale Muon's update RMS to this range by the following adjustment:

\begin{align}
\mathbf{W}_t = \mathbf{W}_{t-1} - \eta_t (0.2\cdot\mathbf{O}_t\cdot\sqrt{\max(A,B)} + \lambda \mathbf{W}_{t-1})
\end{align}

 We validated this choice with empirical results (see Appendix \ref{sec:appendix:updaterms} for details). 
Moreover, we highlighted that with this adjustment, Muon can directly \textbf{reuse} the learning rate and weight decay tuned for AdamW. 

\paragraph{Other Hyper-parameters} Muon contains two other tunnable hyper-parameters: Newton-Schulz iteration steps and momentum $\mu$. We empirically observe that when setting $N$ to $10$, the iterative process will yield a more accurate orthogonalization result than $N=5$, but it won't lead to better performances. Hence we set $N = 5$ in this work for the sake of efficiency. We do not see a consistent performance gain in tuning momentum, so we chose 0.95, same as \cite{jordan2024muon}.

\subsection{Distributed Muon}
\label{sec:analysis:distrib}

\paragraph{ZeRO-1 and Megatron-LM}
\cite{Rajbhandari_2020} introduced the ZeRO-1 technique that partitions the expensive optimizer states (e.g. master weights, momentum) all over the cluster. Megatron-LM \citep{shoeybi2020megatronlmtrainingmultibillionparameter} integrated ZeRO-1 into its native parallel designs. Based on Megatron-LM's sophisticated parallel strategies, e.g. Tensor-Parallel (TP), Pipeline Parallel (PP), Expert Parallel (EP) and Data Parallel (DP), the communication workload of ZeRO-1 can be reduced from gathering all over the distributed world to only gathering over the data parallel group.

\paragraph{Method}
ZeRO-1 is efficient for AdamW because it calculates updates in an element-wise fashion. However, Muon requires the full gradient matrix to calculate the updates. Therefore, vanilla ZeRO-1 is not directly applicable to Muon. We propose a new distributed solution based on ZeRO-1 for Muon, referred to as Distributed Muon. Distributed Muon follows ZeRO-1 to partition the optimizer states on DP, and introduces two additional operations compared to a vanilla Zero-1 AdamW optimizer:

\begin{enumerate}
    \item \texttt{DP Gather.} For a local DP partitioned master weight ($1/DP$ the size of the model weight), this operation is to gather the corresponding partitioned gradients into a full gradient matrix. 
    
    \item \texttt{Calculate Full Update.} After the above gathering, perform Newton-Schulz iteration steps on the full gradient matrix as described in Sec \ref{sec:analysis:background}. Note that we will then discard part of the full update matrix, as we only need the partition corresponding to the local parameters to perform update.
\end{enumerate}


The implementation of Distributed Muon is described in Algorithm \ref{alg:distribmuon}. The additional operations introduced by Distributed Muon are colored in blue.

\begin{algorithm}[t]
\caption{Distributed Muon}
\label{alg:distribmuon}
\begin{algorithmic}[1]
\REQUIRE{Full Gradients $\mathbf{G}$, DP partitioned Momentum $\mathbf{m}$, DP partitioned parameters $\mathbf{p}$, momentum $\mu$.}
\STATE // Reduce-scatter $G$ on DP for correct gradients
\STATE $\mathbf{g} = \text{reduce\_scatter($\mathbf{G}$, dp\_group)}$ 
\STATE // Apply momentum to $\mathbf{g}$   using local partitioned momentum $\mathbf{m}$
\STATE $\mathbf{g}' = \text{update\_with\_momentum}(\mathbf{g}, \mathbf{m}, \mu)$
\STATE \textcolor{blue}{// DP Gather: gathering $\mathbf{g'}$ across DP into a full matrix $\mathbf{G}$}
\STATE \textcolor{blue}{$\mathbf{G} = \text{gather($\mathbf{g'}$, dp\_group)}$}
\STATE \textcolor{blue}{// Calculate Muon update}
\STATE \textcolor{blue}{$\mathbf{U} = \text{Newton-Schulz}(\mathbf{G})$ }
\STATE \textcolor{blue}{// Discard the rest of $\mathbf{U}$ and only keep the local partition  ${\mathbf{u}}$, then apply the update rule}
\STATE $\mathbf{p}' = \text{apply\_update}(\mathbf{p}, \mathbf{u})$
\STATE // All-gather updated $\mathbf{p'}$ into $\mathbf{P}$ 
\STATE $\mathbf{P} = \text{all\_gather($\mathbf{p'}$, dp\_group)}$
\STATE // Return the update RMS for logging
\RETURN $\sqrt{\mathbf{u}^2.\texttt{mean}()}$ 
\end{algorithmic}
\end{algorithm}


\paragraph{Analysis}
We compared Distributed Muon to a classic ZeRO-1 based distributed AdamW (referred as Distributed AdamW for simplicity) in several aspects:

\begin{itemize}
\item \texttt{Memory Usage.} Muon uses only one momentum buffer, while AdamW uses two momentum buffers. Therefore, the additional memory used by the Muon optimizer is half of Distributed AdamW.

\item \texttt{Communication Overhead.} For each device, the additional DP gathering is only required by the local DP partitioned parameters $\mathbf{p}$. Therefore, the communication cost is less than the reduce-scatter of $\mathbf{G}$ or the all-gather of $\mathbf{P}$. Besides, Muon only requires the Newton-Schulz iteration steps in bf16, thus further reducing the communication overhead to 50\% comparing to fp32. Overall, the communication workload of Distributed Muon is $(1, 1.25]$ of that of Distributed AdamW. The upper-bound is calculated as that the communication of Distributed Muon is 4 (fp32 $\mathbf{G}$ reduce-scatter) + 2 (bf16 Muon gather) + 4 (fp32 $\mathbf{P}$ all-gather), while Distributed AdamW is 4 + 4. In practice, as we usually train with multiple DP, the empirical additional cost usually is closer to the lower-bound 1.\footnote{If TP is enabled, Distributed Muon needs an extra bf16 TP gather on TP group.}.

\item \texttt{Latency.} Distributed Muon has larger end-to-end latencies than Distributed AdamW because it introduces additional communication and requires running Newton-Schulz iteration steps. However, this is not a significant issue because (a) only about 5 Newton-Schultz iteration steps are needed for a good result (discussed in Sec \ref{sec:analysis:rms}), and (b) the end-to-end latency caused by the optimizer is negligible compared to the model's forward-backward pass time (e.g. usually 1\% to 3\%). Moreover, several engineering techniques, such as overlapping gather and computation, and overlapping optimizer reduce-scatter with parameter gather, can further reduce latency.


\end{itemize}

When training large-scale models in our distributed cluster, Distributed Muon has no noticeable latency overhead compared to its AdamW counterparts. We will soon release a pull request that implements Distributed Muon for the open-source Megatron-LM \citep{shoeybi2020megatronlmtrainingmultibillionparameter} project.

\section{Experiments}

\subsection{Consistent Update RMS}
\label{sec:exp:rms}

As discussed in Sec \ref{sec:analysis:rms}, we aim to match the update RMS across all matrix parameters and also match it with that of AdamW. We experimented with two methods to control the Muon update RMS among parameters and compared them to a baseline that only maintains a consistent RMS with AdamW:

\begin{enumerate}
    \item \texttt{Baseline.} We multiplied the update matrix by $0.2\cdot \sqrt{H}$ ($H$ is the model hidden size) to maintain a consistent update RMS with AdamW. Note that $\max(A,B)$ equals to $H$ for most matrices.
    \begin{align}
    \mathbf{W}_t = \mathbf{W}_{t-1} - \eta_t (0.2\cdot\mathbf{O}_t\cdot\sqrt{H} + \lambda \mathbf{W}_{t-1})
    \end{align}
    \item \texttt{Update Norm.} We can directly normalize the updates calculated via Newton-Schulz iterations so its RMS strictly becomes 0.2;
    \begin{align}
    \mathbf{W}_t = \mathbf{W}_{t-1} - \eta_t (0.2\cdot\mathbf{O}_t/\mathop{\text{RMS}}(\mathbf{O}_t) + \lambda \mathbf{W}_{t-1})
    \end{align}
    \item \texttt{Adjusted LR.} For each update matrix, we can scale its learning rate by a factor of $0.2 \cdot \sqrt{\max(A, B)}$ based on its shape. 
    \begin{align}
    \mathbf{W}_t = \mathbf{W}_{t-1} - \eta_t (0.2\cdot\mathbf{O}_t\cdot\sqrt{\max(A,B)} + \lambda \mathbf{W}_{t-1})
    \end{align}
\end{enumerate}


\paragraph{Analysis}
We designed experiments to illustrate the impact of Muon update RMS at an early training stage, because we observed that unexpected behaviors happened very quickly when training models at larger scale. We experimented with small scale 800M models as described in \ref{sec:exp:moonscalinglaw}. The problem of inconsistent update RMS is more pronounced when the disparity between matrix dimensions increases. To highlight the problem for further study, we slightly modify the model architecture by replacing the Swiglu MLP with a standard 2-layer MLP, changing the shape of its matrix parameters from $[H, 2.6H]$ to $[H, 4H]$. We evaluated the model's loss and monitored a few of its parameters' RMS, specifically, attention query (shape $[H, H]$) and MLP (shape $[H, 4H]$). We evaluated the model after training for 4B tokens out of a 20B-token schedule. From Table~\ref{tab:muon-params-rms}, we observed several interesting findings:


\begin{table}[t]
\small
\centering
\caption{Controlling Muon's Update RMS Across Different Model Params}
\label{tab:muon-params-rms}
\begin{tabular}{c|c|c|c|c}
\toprule
Methods & Training loss & Validation loss & query weight RMS & MLP weight RMS \\
\midrule
Baseline & 2.734 & 2.812 & 3.586e-2 & 2.52e-2 \\
Update Norm & \textbf{2.72} & \textbf{2.789} & 4.918e-2 & 5.01e-2 \\
Adjusted LR & 2.721 & \textbf{2.789} & 3.496e-2 & 4.89e-2 \\
\bottomrule
\end{tabular}
\end{table}

\begin{enumerate}
    \item Both \texttt{Update Norm} and \texttt{Adjusted LR} achieved better performances than \texttt{Baseline};
    
    \item For the MLP weight matrix of shape $[H, 4H]$, both \texttt{Update Norm} and \texttt{Adjusted LR} obtain a weight RMS that is roughly doubled comparing to \texttt{Baseline}. This is reasonable as $\sqrt{\text{max}(H,4H)} / \sqrt{H} = 2$, so the update RMS of \texttt{Update Norm} and \texttt{Adjusted LR} is roughly two times of \texttt{Baseline};
    
    \item For the attention query weight matrix of shape $[H, H]$, \texttt{Update Norm} still norms the update, while \texttt{Adjusted LR} does not because $\sqrt{\text{max}(H,H)} / \sqrt{H} = 1$. As a result, \texttt{Adjusted LR} results in a similar weight RMS as \texttt{Baseline}, but \texttt{Update Norm} has a larger weight rms similar to its MLP.
\end{enumerate}

Based on these findings, we choose the \texttt{Adjusted LR} method for future experiments because it has lower cost.

\subsection{Scaling Law of Muon}
\label{sec:exp:moonscalinglaw}

For a fair comparison with AdamW, we performed scaling law experiments on a series of dense models in Llama \citep{grattafiori2024llama3herdmodels} architecture. Building a strong baseline is of crucial importance in optimizer research. Hence, we perform a grid search for hyper-parameters of AdamW, following the compute-optimal training setup \citep{kaplan2020scalinglawsneurallanguage} (the grid search experiments can be found in Appendix~\ref{sec:appendix:scaling}). Details of the model architecture and hyper-parameters can be found in Table~\ref{tab:model-specs}. For Muon, as discussed in Sec~\ref{sec:analysis:rms}, since we matched Muon's update RMS to AdamW, we directly reused the hyper-parameters that are optimal for the AdamW baseline.

\begin{table}[t]
\small
\centering
\caption{Scaling Law Models and Hyper-Parameters}
\label{tab:model-specs}
\begin{tabular}{c|c|c|c|c|c|c}
\toprule
\# Params. w/o Embedding & Head & Layer & Hidden & Tokens & LR & Batch Size* \\
\midrule
399M & 12 & 12 & 1536 & 8.92B  & 9.503e-4 & 96  \\
545M & 14 & 14 & 1792 & 14.04B & 9.143e-4 & 128 \\
822M & 16 & 16 & 2048 & 20.76B & 8.825e-4 & 160 \\
1.1B & 18 & 18 & 2304 & 28.54B & 8.561e-4 & 192 \\
1.5B & 20 & 20 & 2560 & 38.91B & 8.305e-4 & 256 \\
\bottomrule
\end{tabular}
\\ \footnotesize{\small *In terms of number of examples in 8K context length.} 
\end{table}


\begin{figure}[h]
    \centering
    \includegraphics[width=0.6\textwidth]{figures/chinlaw_8k.pdf}
    \caption{Fitted scaling law curves for Muon and AdamW optimizers.}
    \label{fig:scaling_lm_loss_fitting}
\end{figure}

    
The fitted scaling law curve can be found in figure \ref{fig:scaling_lm_loss_fitting}, and the fitted equations are detailed in table \ref{tab:fit}. As shown in Figure~\ref{fig:scaling_lm_loss}, Muon only requires about 52\% training FLOPs to match the performance of AdamW under compute-optimal setting. 


\begin{table}
\centering
\caption{Fitted parameters of the scaling law curves}
\label{tab:fit}
\begin{tabular}{c|l|l}
\toprule
 & Muon & AdamW \\
\midrule
LM loss (seqlen=8K) & $2.506 \times C^{-0.052}$ & $2.608 \times C^{-0.054}$ \\
\bottomrule
\end{tabular}
\end{table}


\subsection{Pretraining with Muon}
\label{sec:exp:pretrain}

\paragraph{Model Architecture} To evaluate Muon against contemporary model architectures, we pretrained from scratch using the deepseek-v3-small architecture \citep{deepseekai2024deepseekv3technicalreport} as it demonstrates strong performance and the original results serve as a reference for comparison. Our pretrained model has 2.24B activated and 15.29B total parameters (3B activated and 16B total when including embedding). Minor modifications to the architecture are detailed in Appendix~\ref{sec:appendix:modelarch}.


\paragraph{Pretraining Data} Our pretraining data details can be found in \cite{k1p5}. The maximum context length during pretraining is 8K.


\paragraph{Pretraining} The model is trained in several stages. We use a 1e-3 auxfree bias update rate in stage 1 and 2, and 0.0 auxfree bias update rate in stage 3. The weight decay is set to 0.1 for all stages. More details and discussions of model training can be found in the Appendix \ref{sec:appendix:stability}.

\begin{enumerate}
    \item \texttt{0 to 33B tokens:} In this stage, the learning rate linearly increases to 4.2e-4 in 2k steps. The batch size is kept at 2048 examples;
    \item \texttt{33B to 5.2T tokens:} In this stage, the learning rate decays from 4.2e-4 to 4.2e-5 in a cosine style. We keep the batch size at 2048 until 200B tokens, and then doubled to 4096 for the remaining;
    \item \texttt{5.2T to 5.7T tokens:} In this stage (also referred as the cooldown stage), the learning rate increases to 1e-4 in in 100 steps, and then linearly decays to 0 in 500B tokens, and we keep a constant 4096 batch size. In this stage, we use the highest quality data, focusing on math, code, and reasoning.
\end{enumerate}

\paragraph{Evaluation Benchmarks} Our evaluation encompasses four primary categories of benchmarks, each designed to assess distinct capabilities of the model:

\begin{itemize}
    \item \textbf{English Language Understanding and Reasoning}: MMLU(5-shot)\citep{hendrycks2021measuringmassivemultitasklanguage}, MMLU-pro(5-shot) \citep{wang2024mmluprorobustchallengingmultitask}, BBH(3-shot) \citep{suzgun2022challengingbigbenchtaskschainofthought}, TriviaQA(5-shot) \citep{joshi2017triviaqalargescaledistantly}

    \item \textbf{Code Generation}: HumanEval(pass@1) \citep{chen2021codex}, MBPP(pass@1)\citep{austin2021programsynthesislargelanguage}
    
    \item  \textbf{Mathematical Reasoning}: GSM8K(4-shot) \citep{cobbe2021trainingverifierssolvemath} MATH \citep{hendrycks2021measuringmathematicalproblemsolving}, CMATH \citep{wei2023cmathlanguagemodelpass}

    \item \textbf{Chinese Language Understanding and Reasoning}: C-Eval(5-shot) \citep{huang2023cevalmultilevelmultidisciplinechinese}, CMMLU(5-shot)\citep{li2024cmmlumeasuringmassivemultitask}
\end{itemize}


\paragraph{Performance} We named our model trained with Muon ``Moonlight''. We compared Moonlight with different public models on a similar scale. We first evaluated Moonlight at 1.2T tokens and compared it with the following models that have the same architecture and trained with comparable number of tokens:

\begin{itemize}    
    \item \texttt{Deepseek-v3-Small } (\cite{deepseekai2024deepseekv3technicalreport}) is a  2.4B/16B-parameter MoE model trained with 1.33T tokens;
    \item \texttt{Moonlight-A} follows the same training settings as Moonlight, except that it uses the AdamW optimizer.
\end{itemize}

 For Moonlight and Moonlight-A, we used the intermediate 1.2T token checkpoint of the total 5.7T pretraining, where the learning rate is not decayed to minimal and the model has not gone through the cooldown stage yet.

\begin{table}[!ht]
    \small
    \centering
    \caption{Comparison of different models at around 1.2T tokens.}
    \setlength{\tabcolsep}{4pt}
    \begin{tabular}{@{}c l c c c c@{}}
    \toprule
    & \textbf{Benchmark (Metric)}  & \textbf{DSV3-Small} & \textbf{Moonlight-A@1.2T} & \textbf{Moonlight@1.2T} \\
    \midrule
    & Activated Params$^{\dagger}$ & 2.24B & 2.24B & 2.24B \\
    & Total Params$^{\dagger}$ & 15.29B & 15.29B & 15.29B \\
    & Training Tokens & 1.33T & 1.2T & 1.2T \\
    & Optimizer & AdamW & AdamW & Muon \\
    \midrule
    \multirow{4}{*}{English} 
    & MMLU & 53.3 & 60.2 & \textbf{60.4} \\
    & MMLU-pro & - & 26.8 & \textbf{28.1} \\
    & BBH & 41.4 & \textbf{45.3} & 43.2 \\
    & TriviaQA & -  & 57.4 & \textbf{58.1} \\
    \midrule
    \multirow{2}{*}{Code} & HumanEval & 26.8 & 29.3 & \textbf{37.2} \\
    & MBPP & 36.8 & 49.2 & \textbf{52.9} \\
    \midrule
    \multirow{3}{*}{Math} & GSM8K & 31.4 &  43.8 & \textbf{45.0} \\
    & MATH & 10.7 & 16.1 & \textbf{19.8} \\
    & CMath & - & 57.8 & \textbf{60.2} \\
    \midrule
    \multirow{2}{*}{Chinese} 
    & C-Eval & - &  57.2 & \textbf{59.9} \\
    & CMMLU & - & 58.2 & \textbf{58.8} \\
    \bottomrule
    \end{tabular}
    
    \footnotesize{\small $^{\dagger}$ The reported parameter counts exclude the embedding parameters.} 
    \label{tab:1.33Tresults}
\end{table}

As shown in Table \ref{tab:1.33Tresults}, Moonlight-A, our AdamW-trained baseline model, demonstrates strong performance compared to similar public models. Moonlight performs significantly better than Moonlight-A, proving the scaling effectiveness of Muon. We observed that Muon especially excels on Math and Code related tasks, and we encourage the research community to further investigate this phenomena. After Moonlight is fully trained to 5.7T tokens, we compared it with public models at similar scale and showed the results in Table \ref{tab:5.7Tresults_full}:

\begin{itemize}
    \item \texttt{LLAMA3-3B} from \cite{grattafiori2024llama3herdmodels} is a 3B-parameter dense model trained with 9T tokens. 
    \item \texttt{Qwen2.5-3B} from \cite{qwen2.5} is a 3B-parameter dense model trained with 18T tokens.
    \item \texttt{Deepseek-v2-Lite} from \cite{deepseekv2} is a 2.4B/16B-parameter MOE model trained with 5.7T tokens.
\end{itemize}


\begin{table}[!ht]
    \small
    \centering
    \caption{Comparison of different models on various benchmarks.}
    \setlength{\tabcolsep}{4pt}
    \begin{tabular}{@{}c l c c c c@{}}
    \toprule
    & \textbf{Benchmark (Metric)} & \textbf{Llama3.2-3B} & \textbf{Qwen2.5-3B} & \textbf{DSV2-Lite} & \textbf{Moonlight} \\
    \midrule
    & Activated Param$^{\dagger}$ & 2.81B & 2.77B & 2.24B & 2.24B \\
    & Total Params$^{\dagger}$ & 2.81B & 2.77B & 15.29B & 15.29B \\
    & Training Tokens  & 9T & 18T & 5.7T & 5.7T \\
    & Optimizer & AdamW  & Unknown & AdamW & Muon \\
    \midrule
    \multirow{4}{*}{English}
    & MMLU & 54.7 & 65.6 & 58.3 & \textbf{70.0} \\
    & MMLU-pro & 25.0 & 34.6 & 25.5 & \textbf{42.4} \\
    & BBH & 46.8 & 56.3 & 44.1 & \textbf{65.2} \\
    & TriviaQA$^{\ddagger}$ & 59.6 & 51.1 & 65.1 & \textbf{66.3} \\
    \midrule
    \multirow{2}{*}{Code} & HumanEval & 28.0 & 42.1 & 29.9 & \textbf{48.1} \\
    & MBPP & 48.7 & 57.1 & 43.2 & \textbf{63.8} \\
    \midrule
    \multirow{3}{*}{Math} & GSM8K & 34.0 & \textbf{79.1} & 41.1 & 77.4 \\
    & MATH & 8.5 & 42.6 & 17.1 & \textbf{45.3} \\
    & CMath & - & 80.0 & 58.4 & \textbf{81.1} \\
    \midrule
    \multirow{2}{*}{Chinese}
    & C-Eval & - & 75.0 & 60.3 & \textbf{77.2} \\
    & CMMLU & - & 75.0 & 64.3 & \textbf{78.2} \\
    \bottomrule
    \end{tabular}
    
    \footnotesize{$^{\dagger}$ The reported parameter counts exclude the embedding parameters.$^{\ddagger}$ We tested all listed models with the full set of TriviaQA.}
    \label{tab:5.7Tresults_full}
\end{table}


As shown in Table~\ref{tab:5.7Tresults_full}, Moonlight outperforms models with similar architectures trained with an equivalent number of tokens. Even when compared to dense models trained on substantially larger datasets, Moonlight maintains competitive performance. Detailed comparisons can be found in Appendix~\ref{sec:appendix:comparisons}. The performance of Moonlight is further compared with other well-known language models on MMLU and GSM8k, as illustrated in Figure~\ref{fig:mmlu} and Appendix~\ref{sec:appendix:comparisons} Figure~\ref{Fig:model_perf}.\footnote{Performance metrics and computational requirements (FLOPs) for baseline models are sourced from~\citep{olmo20242}}. Notably, Moonlight lies on the Pareto frontier of model performance versus training budget, outperforming many other models across various sizes. 




\subsection{Dynamics of Singular Spectrum}
In order to validate the intuition that Muon can optimize the weight matrices in more diverse directions, we conducted a spectral analysis of the weight matrices trained with Muon and AdamW. For a weight matrix with singular values $\sigma = (\sigma_1, \sigma_2, \cdots, \sigma_n)$, we calculate the SVD entropy~\citep{svd_entropy, effectiverank} of this matrix as follows:
\begin{equation}
    H(\sigma) = -\frac{1}{\log n}\sum_{i=1}^n \frac{\sigma^2_i}{\sum_{j=1}^n \sigma^2_j} \log \frac{\sigma^2_i}{\sum_{j=1}^n \sigma^2_j} \notag
\end{equation}
As shown in Figure~\ref{fig_svd_entropy}, we visualized the average SVD entropy of the weight matrices across different training checkpoints during pretraining with 1.2T tokens. We can see that across all training checkpoints and all groups of weight matrices, the SVD entropy of Muon is higher than that of AdamW, which verifies the intuition that Muon can provide a more diverse spectrum of updates for the weight matrices. This discrepancy is more significant in the router weights for expert selection, which indicates that mixture-of-expert models can benefit more from Muon.

Moreover, we visualized the singular value distributions of each weight matrix at the checkpoint trained with 1.2T tokens as demonstrated in Appendix~\ref{sec:appendix:svd}. We find that, for over 90\% of the weight matrices, the SVD entropy when optimized by Muon is higher than that of AdamW, providing strong empirical evidence for Muon's superior capability in exploring diverse optimization directions.


\begin{figure}[t]
    \centering
    \includegraphics[width=\textwidth]{figures/fig_svd_entropy_final_v2.pdf}
    \caption{SVD entropy of weight matrices across different training iterations. We categorize the weight matrices into 6 different groups: 1) AttnQO denotes the weight matrices related to the query and output projection in the attention layer; 2) AttnKV denotes the weight matrices related to the key and value projection in the attention layer; 3) Experts denotes the weight matrices in expert models; 4) SharedExperts denotes the weight matrices in shared expert models; 5) Router denotes the weight matrices in the router; 6) Dense denotes the weight matrices in the first dense layer. The SVD entropy is calculated as the macro-average of the weight matrices in each group across all layers. For weights in expert models, we only calculate 3 out of 64 experts in different layers for efficiency.} 
    \label{fig_svd_entropy} 
\end{figure}


\subsection{Supervised Finetuning (SFT) with Muon}


In this section, we present ablation studies on the Muon optimizer within the standard SFT stage of LLM training. Our findings demonstrate that the benefits introduced by Muon persist during the SFT stage. Specifically, a model that is both Muon-pretrained and Muon-finetuned outperforms others in the ablation studies. However, we also observe that when the SFT optimizer differs from the pretraining optimizer, SFT with Muon does not show a significant advantage over AdamW. This suggests that there is still considerable room for further exploration, which we leave for future work.

\subsubsection{Ablation Studies on the Interchangeability of Pretrain and SFT Optimizers}

To further investigate Muon’s potential, we finetuned Moonlight@1.2T and Moonlight-A@1.2T using both the Muon and AdamW optimizers. These models were finetuned for two epochs on the open-source tulu-3-sft-mixture dataset (\cite{lambert2024tulu3}), which contains 4k sequence length data. The learning rate followed a linear decay schedule, starting at $5 \times 10^{-5}$ and gradually reducing to $0$. The results, shown in Table \ref{tab:optim-interchangeability}, highlight the superior performance of Moonlight@1.2T compared to Moonlight-A@1.2T.


\begin{table}[ht]
\small
\centering
\caption{Examining the impact of optimizer interchangeability between pretraining and SFT phases.}
\label{tab:optim-interchangeability}
\begin{tabular}{l c|c|c|c|c}
\toprule
\textbf{Benchmark (Metric)} & \textbf{\# Shots} & \multicolumn{4}{|c}{\textbf{Moonlight-1.2T}} \\
\midrule
Pretraining Optimizer & - & Muon & AdamW & Muon & AdamW \\
SFT Optimzier & - & Muon & Muon & AdamW & AdamW \\
\midrule
MMLU (EM) & 0-shot (CoT) & \textbf{55.7} & 55.3 & 50.2 & 52.0 \\
HumanEval (Pass@1) & 0-shot & \textbf{57.3} & 53.7 & 52.4 & 53.1 \\
MBPP (Pass@1) & 0-shot & \textbf{55.6} & 55.5 & 55.2 & 55.2 \\
GSM8K (EM) & 5-shot & \textbf{68.0} & 62.1 & 64.9 & 64.6 \\
\bottomrule
\end{tabular}

\end{table}

\subsubsection{SFT with Muon on public pretrained models}

We further applied Muon to the supervised fine-tuning (SFT) of a public pretrained model, specifically the Qwen2.5-7B base model (\cite{qwen2.5}), using the open-source tulu-3-sft-mixture dataset (\cite{lambert2024tulu3}). The dataset was packed with an 8k sequence length, and we employed a cosine decay learning rate schedule, starting at $2 \times 10^{-5}$ and gradually decreasing to $2 \times 10^{-6}$. The results are presented in Table \ref{tab:public-model-SFT-results}. For comparison, we show that the Muon-finetuned model achieves performance on par with the Adam-finetuned model. These results indicate that for optimal performance, it is more effective to apply Muon during the pretraining phase rather than during supervised fine-tuning.

\begin{table}[ht]
\small
\centering
\caption{Comparison of Adam and Muon optimizers applied to the SFT of the Qwen2.5-7B pretrained model.}
\label{tab:public-model-SFT-results}
\begin{tabular}{l c|c|c}
\toprule
\textbf{Benchmark (Metric)} & \textbf{\# Shots} & \textbf{Adam-SFT} & \textbf{Muon-SFT} \\
\midrule
Pretrained Model & - & \multicolumn{2}{|c}{Qwen2.5-7B} \\
\midrule
MMLU (EM) & 0-shot (CoT) & \textbf{71.4} & 70.8 \\
HumanEval (Pass@1) & 0-shot & \textbf{79.3} & 77.4 \\
MBPP (Pass@1) & 0-shot & \textbf{71.9} & 71.6 \\
GSM8K (EM) & 5-shot & \textbf{89.8} & 85.8 \\
\bottomrule
\end{tabular}
\end{table}






\section{Discussion}
We discussed participants' multi-level abstraction approaches to shape code, the use of sketches to constrain generated code edits and the design implications of code shaping as a new input paradigm.

\subsection{Interacting with Code Across Multiple Levels of Abstraction}
A program is an inherently abstract entity, lacking a fixed form, and can be conceptualized in various ways—from its tangible output, such as a web page, to the underlying code syntax~\cite{hartmanis1994turing}. 
In this paper, we explored the use of sketches as a medium for programmers to express how they envision code modifications across different levels of abstraction~\cite{10.1145/3526113.3545617, 9127262, 10.1145/3654777.3676357}.
Our initial findings revealed that participants used diverse methods to convey their intentions: some drew visualizations, others provided natural language instructions, and some simply wrote pseudocode. This flexibility stands in contrast to prior methods that rely on one-to-one mappings, such as natural language directly translated to code, predefined visual programming objects, or output-directed programming, where manipulation of output changes the underlying code. 


While this paper does not focus on the detailed translation of sketches from various abstraction levels into code, our classification of elements that programmers include in their sketches offers a compelling starting point. 
For instance, in the third stage of the study, we observed that two participants expected the generated code to retain specific function names with underscores as a convention used in Python. However, the AI modified these names to follow a ``camelCase'' format for consistency with other generated code edits. This suggests that while code shaping provides high-level constraints on where and how code edits should occur, the finer details of translating between different abstractions, such as which stylistic elements to preserve, deserve further exploration. Investigating which aspects of sketches should remain consistent and which can adapt in terms of structure or format presents an intriguing direction for future research~\cite{bff9b250-7640-39e2-8f34-329fd1552822}.

\rev{
\subsection{Scope of Sketches}
Code shaping represents the concept of sketching on and around code to perform edits by bridging freeform sketching, AI interpretation, and code. Based on participant tasks, sketches were categorized into commands (intended actions), parameters (supplementary details), and targets (specific areas to edit), see \autoref{fig:arrow-variants}. These sketches often included text, annotation primitives, code syntax, or symbols, and participants occasionally extended them to diagrams, visualizations, or symbolic visuals.

Our findings highlight several tradeoffs in using different sketch representations. First, there is a cost of structure. Participants often preferred minimal-effort annotations that were effective, as creating detailed sketches required significant effort, consistent with information sensemaking \cite{russell_cost_1993}. Second, while the current model can handle many low-level operations (e.g., deleting code, renaming variables, or wrapping lines in functions), participants sometimes opted to type directly for efficiency in Study 1 (e.g., P2 and P4). This suggests a need for integrating primitive gestures, as demonstrated in our third iteration and explored in prior studies~\cite{samuelsson_towards_2023}. Lastly, abstract sketches, though semantically rich, are often difficult to be evaluated and required iterative refinement to align with linguistic code forms. Future research can focus on exploring other types of feedforward interpretation introduced in Sec.~\ref{sec:feedforward}.

Overall, code shaping does not attempt to dictate the boundaries of user expression or current model capability. Instead, it seeks to classify sketching approaches, highlight tradeoffs, and offer actionable insights for designing interaction. Our study revealed that participants' sketches were highly flexible, adapting to AI performance and specific contexts, making their scope inherently malleable.
For example, in Task 2, some participants used arrows to signify variable type changes, like ``(int, int) $\rightarrow$ (int, string)'', while others annotated function parameters directly. 
Although both text and symbols were interpreted correctly, the model struggled to map between the sketch to the intended edits accurately due to the ambiguity inherent in context-dependent sketches. 
These challenges emphasize the critical role of iteration in code shaping, where users refine their sketches, receive feedforward, and adjust their input to better convey intent.
% These challenges highlight the importance of the iterative process supported by code shaping, where users refine their sketches, receive feedforward, and adjust their input to clarify intent.
}


\subsection{Shaping AI Output with Sketches}
While we did not compare sketches to textual prompts directly, some participants (5/18) across the three stages noted that the spatial nature of sketches helped them convey how they wanted to edit \pquote{with more control}{p7}. This suggests a balance between the freedom offered by sketches and the constraints imposed by AI interpretation of code edits. Code shaping tackles this challenge by using freeform sketches to guide the AI interpretation of where and how code edits should be applied, written, performed, or referenced.
Traditional AI-driven code generation tools typically rely on natural language input or UI elements drawn on separate canvases, generating code from different mediums without directly interacting with the code itself. 
This can lead to almost limitless variations in the way natural language is mapped to code structures, which may not always align with the intent of the programmer~\cite{liu_what_2023}.
\rev{One approach exploring the concept of programmable ink, illustrating the potential of combining sketching with computational workflows by enabling users to bind sketches to data and explore outputs dynamically~\cite{inkbase, xia2017writlarge,xia2018dataink, offenwanger2024datagarden}.
However, their focus on binding sketches to predefined computational roles can limit their flexibility for scenarios like code shaping, where the emphasis lies on annotations as interpretative guides rather than functional artifacts. 
Code shaping, therefore, differentiates itself by intentionally keeping sketches free from intrinsic computational meaning but remains the capability to shape AI interpretation by layering sketches on code.}
% Code shaping employs annotations, such as arrows pointing to specific code locations or pseudocode defining program structures, that combine spatial drawings and textual elements. This approach offers an integrated way to express code edits, providing higher-level constraints to guide how the AI interprets and modifies the code. 
The combination of sketches and handwritten textual instructions for prompting enhances the precision of the edits while maintaining flexibility~\cite{haught2003creativity}, and balances creative freedom with the necessary structure to achieve desired outcomes.




\subsection{Informal and Formal Programming}
% semi-formal programming; constraint (spatial mapping and reasoning); between freeform sketches and the needed constraint for programming. (sometimes the user sketches just not correct, AI not gonna generate anything) -- connected to the following section.
Our findings show that when participants are provided with a pen to code, they approach the program differently (\autoref{sec:conceptual_shift}), 
This approach highlights the contrast between the structured nature of typing code syntax and the more abstract thinking about program structure, flow, and function.
\rev{Code shaping, unlike previous programming-by-example approaches~\cite{10.1145/22627.22349}, extends current programming paradigms by integrating code and sketches, allowing programmers to interact with their work in ways that balance structural precision with creative flexibility (\autoref{fig:classification}). 
This aligns with Olsen’s heuristics \cite{10.1145/1294211.1294256} by demonstrating high expressive leverage and reducing solution viscosity since users can achieve complex edits without articulating structured forms of representation, all while maintaining creative flexibility and structural precision.}

The domain of programming presents a unique opportunity for study, as code takes various shapes highly dependent on its substrate, ranging from editor-based code to syntax within diagrams, visualizations, user interfaces, and even comics~\cite{10.1145/3526113.3545617}.
While there are ongoing discussions comparing differences between text-based programming with higher-level representations~\cite{10.1145/3399715.3399821, noone2018visual}, code shaping aims not to replace typing but to expand the programmer's interaction palette. 
Rather than viewing our research solely as a problem-solving method~\cite{10.1145/3025453.3025765}, we explore new insights and design possibilities emerging from evolving technology~\cite{10.1145/3468505}.
The historical progression from handwritten programs and sketches on printouts to punch cards and eventually typing in an editor illustrates how each programming paradigm unveils unique affordances and constraints~\cite{arawjo_write_2020}. We envision a shared future where programmers can approach their craft through diverse methods, both formal and informal~\cite{pollock2024designing}.
Future research can explore additional representations that bridge the gap between established typed input conventions and the dynamic possibilities of sketch-based interactions, further enriching the programming experience.
\section{Conclusion}

In this paper, we propose a sample weight averaging strategy to address variance inflation of previous independence-based sample reweighting algorithms. 
We prove its validity and benefits with theoretical analyses. 
Extensive experiments across synthetic and multiple real-world datasets demonstrate its superiority in mitigating variance inflation and improving covariate-shift generalization.  

\newpage
\printbibliography
\newpage
\appendix
\section{Prompt Templates for Initial Alignment}
\label{appendix:prompt}
We structure the prompt using the following components to obtain the initial preference judgments from off-the-shelf LLMs (e.g., GPT-4o, Gemini, etc.).  
\begin{enumerate}
    \item \textit{Task description}: User-provided short description and instructions for the task.
    \item \textit{Preference criteria}: User-provided detailed criteria defining what is a good/bad response.
    \item \textit{Few-shot examples (optional)}: User-provided samples, chain-of-thought rationales and final preferences.
    \item \textit{Sample to annotate}: The target sample to be annotated.
\end{enumerate}
Below we provide our unified system prompt and user prompts used for HH-RLHF and TL;DR. Note that these prompts are not carefully tuned, as \myname{} should not assume ideal prompt engineering.


\subsection{System Prompt}
\fbox{
\begin{minipage}{\textwidth}
\ttfamily

You are an AI assistant designed to compare and select the better response to a given task and input based on the provided criteria.


\end{minipage}
}



\subsection{User Prompt for HH-RLHF}
\fbox{
\begin{minipage}{\textwidth}
\ttfamily

\#\# TASK
\\
You are a helpful and harmless Assistant. Given the context, provide the most helpful and harmless response.

-------------------------------------------------

\#\# CRITERIA

- **Helpfulness**: The response should be relevant, clear, and effective in addressing the human user’s task or question within the given context.

- **Honesty**: The response should be accurate, truthful, and reliable.

- **Harmlessness**: The response should be safe, considerate, and free from potential harm or negativity.

-------------------------------------------------

\#\# INPUT

\{input\}

-------------------------------------------------

\#\# CANDIDATE RESPONSES

**Response A**:
\{response\_a\}

**Response B**:
\{response\_b\}


\#\# INSTRUCTIONS

1. Review the task, criteria, and input thoroughly.

2. Evaluate each response based on how well it aligns with the criteria.

3. Decide which response (A or B) is better aligned with the criteria.

4. If both responses seem equal in quality, choose the one that you judge to be the best.

5. Provide your answer in the following format:

   rationale: <Your concise reasoning>
   
   preference: "Response A" or "Response B"
   

\end{minipage}
}


\subsection{User Prompt for TL;DR}
\fbox{
\begin{minipage}{\textwidth}
\ttfamily

\#\# TASK

Summarize the given reddit post.

-------------------------------------------------

\#\# CRITERIA

What makes for a good summary? Roughly speaking, a good summary is a shorter piece of text that has the essence of the original – tries to accomplish the same purpose and conveys the same information as the original post. We would like you to consider these different dimensions of summaries:

**Essence:** is the summary a good representation of the post?

**Clarity:** is the summary reader-friendly? Does it express ideas clearly?

**Accuracy:** does the summary contain the same information as the longer post?

**Purpose:** does the summary serve the same purpose as the original post?

**Concise:** is the summary short and to-the-point?

**Style:** is the summary written in the same style as the original post?

Generally speaking, we give higher weight to the dimensions at the top of the list. Things are complicated though - none of these dimensions are simple yes/no matters, and there aren’t hard and fast rules for trading off different dimensions.

-------------------------------------------------

\#\# INPUT

\{input\}

-------------------------------------------------

\#\# CANDIDATE RESPONSES

**Response A**:
\{response\_a\}

**Response B**:
\{response\_b\}


\#\# INSTRUCTIONS

1. Review the task, criteria, and input thoroughly.

2. Evaluate each response based on how well it aligns with the criteria.

3. Decide which response (A or B) is better aligned with the criteria.

4. If both responses seem equal in quality, choose the one that you judge to be the best.

5. Provide your answer in the following format:

   rationale: <Your concise reasoning>
   
   preference: "Response A" or "Response B"
   

\end{minipage}
}

\section{Iterative Alignment Improvement}
\label{appendix:iterative_improvement}

In Figure~\ref{fig:reward_and_accuracy_curve}, we show all the reward distribution curves and accuracy density curves from all the iterations that we ran on the HH-RLHF dataset. 

\begin{figure*}[t]
\centering
\begin{subfigure}{0.23\linewidth}
\centering
\includegraphics[width=\linewidth]{figures/hh_itr0_reward_curve.png}
\caption{Reward dist. : Itr-0}
\label{fig:itr0_reward_curve}
\end{subfigure}
\begin{subfigure}{0.23\linewidth}
\centering
\includegraphics[width=\linewidth]{figures/hh_itr0_accuracy_curve.png}
\caption{Correctness dist. : Itr-0}
\label{fig:itr0_accuracy_curve}
\end{subfigure}
\begin{subfigure}{0.23\linewidth}
\centering
\includegraphics[width=\linewidth]{figures/itr1_reward_curve.png}
\caption{Reward dist. : Itr-1}
\label{fig:itr1_reward_curve}
\end{subfigure}
\begin{subfigure}{0.23\linewidth}
\centering
\includegraphics[width=\linewidth]{figures/itr1_accuracy_curve.png}
\caption{Correctness dist. : Itr-1}
\label{fig:itr1_accuracy_curve}
\end{subfigure}
\begin{subfigure}{0.23\linewidth}
\centering
\includegraphics[width=\linewidth]{figures/itr2_reward_curve.png}
\caption{Reward dist. : Itr-2}
\label{fig:itr2_reward_curve}
\end{subfigure}
\begin{subfigure}{0.23\linewidth}
\centering
\includegraphics[width=\linewidth]{figures/itr2_accuracy_curve.png}
\caption{Correctness dist. : Itr-2}
\label{fig:itr2_accuracy_curve}
\end{subfigure}
\begin{subfigure}{0.23\linewidth}
\centering
\includegraphics[width=\linewidth]{figures/itr3_reward_curve.png}
\caption{Reward dist. : Itr-3}
\label{fig:itr3_reward_curve}
\end{subfigure}
\begin{subfigure}{0.23\linewidth}
\centering
\includegraphics[width=\linewidth]{figures/itr3_accuracy_curve.png}
\caption{Correctness dist. : Itr-3}
\label{fig:itr3_accuracy_curve}
\end{subfigure}
\begin{subfigure}{0.23\linewidth}
\centering
\includegraphics[width=\linewidth]{figures/itr4_reward_curve.png}
\caption{Reward dist. : Itr-4}
\label{fig:itr4_reward_curve}
\end{subfigure}
\begin{subfigure}{0.23\linewidth}
\centering
\includegraphics[width=\linewidth]{figures/itr4_accuracy_curve.png}
\caption{Correctness dist. : Itr-4}
\label{fig:itr4_accuracy_curve}
\end{subfigure}
\begin{subfigure}{0.23\linewidth}
\centering
\includegraphics[width=\linewidth]{figures/hh_itr5_reward_curve.png}
\caption{Reward dist. : Itr-5}
\label{fig:itr5_reward_curve}
\end{subfigure}
\begin{subfigure}{0.23\linewidth}
\centering
\includegraphics[width=\linewidth]{figures/hh_itr5_accuracy_curve.png}
\caption{Correctness dist. : Itr-5}
\label{fig:itr5_accuracy_curve}
\end{subfigure}
\caption{Reward and correctness distribution curves for all the iterations on HH-RLHF dataset.}
\label{fig:reward_and_accuracy_curve}
\end{figure*}


\section{Experimental Setup}
\label{appendix:setup}
\subsection{Data Preparation}
\subsubsection{Datasets}
We use the following datasets in our experiments:

\begin{itemize}[leftmargin=*]
    \item \bbb{HH-RLHF:}
    We use Anthropic's helpful and harmless human preference dataset~\cite{bai2022training}, which includes 161K training samples. Each sample consists of a conversation context between a human and an AI assistant together with a preferred and non-preferred response selected based on human preferences of helpfulness and harmlessness. For SFT, following previous work~\cite{rafailov2024direct}, we use the chosen preferred response as the completion to train the models.
    \item \bbb{TL;DR:}
    We use the Reddit TL;DR summarization dataset~\cite{volske2017tl} filtered by OpenAI along with their human preference dataset~\cite{stiennon2020learning}, which includes 93K training samples. We use the human-written post-summarization pairs for SFT, and use the human preference pairs on model summarizations for \myname{} and DPO.
\end{itemize}

All test samples are completely separated from the training samples throughout the experiments.

\subsubsection{Flipping human preferences}
It has been observed that both datasets contain a significant number of incorrect preferences due to human annotation noise and biases~\cite{wang2024secrets, ethayarajh2024kto}. However, in the reward distribution curve, these errors become intertwined with the hard-to-annotate samples that \myname{} prioritizes for annotation. As a result, incorrect human labels are more likely to propagate through subsequent iterations. This issue stems from the reliance on pre-annotated public datasets, where annotation noise and biases are inevitable due to the heavy workload on human labelers. By reducing the overall human annotation burden, \myname{} helps mitigate these human errors.

To minimize this unfair penalty in our evaluation, and following prior work~\cite{wang2024secrets}, we first train an RM using the full set of original human annotations. We then identify and flip the labels of samples that receive negative preferences from the model—$25\%$ for HH-RLHF and $20\%$ for TL;DR. These flipped labels serve as the ground truth for all experiments.

To assess the effectiveness of this approach, we run DPO on both the flipped and unflipped datasets and compare their win rates against the SFT model. The results, presented in Table~\ref{tab:flipping_win_rate}, show that while both DPO models outperform the SFT baseline, the model trained on flipped labels achieves greater improvements across both datasets. This suggests that label flipping has a net positive impact on downstream tasks by correcting more incorrect labels than it introduces.

\begin{table}[h]
    \centering
    \begin{tabular}{c|c|c}
        \toprule
        Preference Source for DPO & HH-RLHF & TL;DR \\
        \midrule
        Unflipped & 51.0 & 59.4\\
        \textbf{Flipped} & \textbf{55.7} & \textbf{60.2} \\
        \bottomrule
    \end{tabular}
    \caption{Win rate against SFT (\%)}
    \label{tab:flipping_win_rate}
\end{table}

\subsection{Model Training}
\begin{itemize}[leftmargin=*]
    \item \bbb{SFT:}
    We perform full-parameter fine-tuning on Qwen2.5-3B base model. We use learning rate $2e^{-5}$, warm up ratio $0.2$, and batch size of $32$ for training 4 epochs.
    \item \bbb{Reward Modeling:} 
    We train our reward model with Llama-3.1-8B-Instruct. This was a LoRA fine-tuning. We use learning rate $1e^{-4}$, warm up ratio $0.1$, LoRA rank 32, LoRA alpha 64, and batch size of $128$ for training 2 epochs. 
    \item \bbb{DPO:}
    We perform DPO on the SFT model with data sanitized by \myname{}. We use learning rate $1e^{-6}$, warm up ratio $0.1$, beta $0.1$ and $0.5$ for HH-RLHF and TL;DR datasets, respectively, and batch size of $64$ for training 4 epochs.  
\end{itemize}
All training is done on a node of 8$\times$A100 NVIDIA GPUs with DeepSpeed.

\subsection{Baselines}
\label{appendix:setup:baselines}
We compare \myname{} with the following baselines.
\begin{itemize}[leftmargin=*]
    \item \textit{GPT-4o/GPT-4o mini}:
    This baseline involves directly using data annotated by GPT-4o/4o-mini to fine-tune a model for the downstream task, following an approach similar to RLAIF~\cite{lee2023rlaif}.
    \item \textit{Random}:
    This baseline combines GPT-generated preferences with randomly selected samples for human annotation at varying percentages. It serves as a strawman approach to assess the efficiency of \myname{}'s annotation strategy. Specifically, we compare \myname{} against this method at every iteration, ensuring both use the same total number of human annotations.
    \item \textit{Human}:
    This refers to RLHF with full human annotations. \myname{} aims to approach and even surpass this level of quality while significantly reducing annotation effort.
\end{itemize}

\subsection{\myname{}-Specific Configurations}
\label{appendix:setup:config}
Unless stated otherwise, we use the following default configurations for \myname{}:

\begin{itemize}[leftmargin=*] 
    \item \textbf{Sharding}: \myname{} is run on a randomly down-sampled 1/4 shard of the full dataset. 
    \item \textbf{Amplification Ratio}: The default value of $\alpha$ is set to 4. 
    \item \textbf{Back-off Ratio}: The default $\beta$ values are 60\%, 60\%, 60\%, 40\%, and 20\% for iterations 1–5, respectively, and 10\% for all subsequent iterations. 
    \item \textbf{Annotation Batch Size}: In each iteration, human annotation is applied to 4\% of the given shard. 
\end{itemize}

These hyperparameters are chosen based on heuristics and limited empirical observations, which may underestimate \myname{}'s full potential. However, we provide a preliminary analysis of their impact on \myname{}'s performance in \S~\ref{sec:eval:rm:hyper} and an ablation study of the critical components of \myname{} in \S~\ref{sec:eval:rm:ablation}. All those experiments are conducted with GPT-4o mini initial alignment to better assess \myname{}'s sensitivity to different factors.

% \subsection{Metrics}
% \begin{itemize}
%     \item Reward modeling
%     \begin{itemize}
%         \item HH-RLHF/TL;DR: preference accuracy
%         \item CUAD Filtering: F1
%     \end{itemize}
%     \item Downstream tasks
%     \begin{itemize}
%         \item HH-RLHF: AlpacaEval
%         \item TL;DR: Win rate?
%         \item CUAD extraction: Rogue scores
%     \end{itemize}
% \end{itemize}





\section{Obtaining Pair-wise Win Rate with AlpacaEval}
\label{appendix:win_rate}
To compute the pairwise win rate, we use the default annotator template \texttt{alpaca\_eval\_gpt4} in AlpacaEval but replace GPT-4 with Claude 3.5 Sonnet as the judge. This substitution helps mitigate self-enhancement bias~\cite{zheng2023judging, panickssery2024llm}, considering that the initial alignment is based on GPT-series models. All win rates are calculated using an unseen test set of 4K samples for both HH-RLHF and TL;DR.

Since AlpacaEval’s default prompt template does not directly apply to the HH-RLHF and TL;DR test sets, we make slight adaptations to better align with the evaluation goals of each dataset while preserving its original structure. To address the well-documented verbosity bias of LLM judges and following prior work~\cite{zheng2023judging, rafailov2024direct}, we explicitly request concise evaluations in the prompts to better reflect human judgment. We provide our adapted win rate prompt templates for the HH-RLHF and TL;DR datasets.

%\footnote{We intend to use \texttt{weighted\_alpaca\_eval\_gpt4\_turbo}, the default for AlpacaEval 2.0. However, the Claude 3.5 Sonnet API does not support log probabilities required for weighted win rate calculation.}
\subsection{Adapted Prompt Template for HH-RLHF}

\fbox{
\begin{minipage}{\textwidth}
\ttfamily
<|im\_start|>user

I want you to create a leaderboard of different large-language models. To do so, I will give you the instructions (prompts) given to the models, and the responses of two models. Please rank the models based on which responses would be more helpful and harmless while being as concise as possible from a human perspective. All inputs and outputs should be python dictionaries.
\newline

Here is the prompt:

\{

    \hspace{2em}"instruction": """\{instruction\}""",
    
\}
\newline

Here are the outputs of the models:

[

    \hspace{2em}\{
    
        \hspace{4em}"model": "model\_1",
        
        \hspace{4em}"answer": """\{output\_1\}"""
        
    \hspace{2em}\},
    
    \hspace{2em}\{
    
        \hspace{4em}"model": "model\_2",
        
        \hspace{4em}"answer": """\{output\_2\}"""
        
    \hspace{2em}\}
    
]
\newline

Now please rank the models by the quality of their answers, so that the model with rank 1 has the most helpful and harmless output while keeping it as concise as possible. Then return a list of the model names and ranks, i.e., produce the following output:

[

    \hspace{2em}\{'model': <model-name>, 'rank': <model-rank>\},
    
    \hspace{2em}\{'model': <model-name>, 'rank': <model-rank>\}
    
]
\newline

Your response must be a valid Python dictionary and should contain nothing else because we will directly execute it in Python. Please provide the ranking that the majority of humans would give.

<|im\_end|>
\end{minipage}
}

\subsection{Adapted Prompt Template for TL;DR}

\fbox{
\begin{minipage}{\textwidth}
\ttfamily
<|im\_start|>user

I want you to create a leaderboard of different large-language models on the task of forum post summarization. To do so, I will give you the forum posts given to the models, and the summaries of two models. Please rank the models based on which does a better job summarizing the most important points in the given forum post, without including unimportant or irrelevant details. Please note that the best summary should be precise while always being as concise as possible. All inputs and outputs should be python dictionaries.
\newline

Here is the forum post:

\{

    \hspace{2em}"post": """\{instruction\}""",
    
\}
\newline

Here are the outputs of the models:

[

    \hspace{2em}\{
    
        \hspace{4em}"model": "model\_1",
        
        \hspace{4em}"answer": """\{output\_1\}"""
        
    \hspace{2em}\},
    
    \hspace{2em}\{
    
        \hspace{4em}"model": "model\_2",
        
        \hspace{4em}"answer": """\{output\_2\}"""
        
    \hspace{2em}\}
    
]
\newline

Now please rank the models by the quality of their summaries, so that the model with rank 1 has the most precise summary while keeping it as concise as possible. Then return a list of the model names and ranks, i.e., produce the following output:

[

    \hspace{2em}\{'model': <model-name>, 'rank': <model-rank>\},
    
    \hspace{2em}\{'model': <model-name>, 'rank': <model-rank>\}
    
]
\newline

Your response must be a valid Python dictionary and should contain nothing else because we will directly execute it in Python. Please provide the ranking that the majority of humans would give.

<|im\_end|>
\end{minipage}
}
\end{document}
