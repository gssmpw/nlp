\documentclass{article}
\usepackage{arxiv}

\usepackage[utf8]{inputenc} % allow utf-8 input
\usepackage[T1]{fontenc}    % use 8-bit T1 fonts
\usepackage{hyperref}       % hyperlinks
\usepackage{url}            % simple URL typesetting
\usepackage{booktabs}       % professional-quality tables
\usepackage{amsfonts}       % blackboard math symbols
\usepackage{nicefrac}       % compact symbols for 1/2, etc.
\usepackage{microtype}      % microtypography
\usepackage{lipsum}		    % Can be removed after putting your text content
\usepackage{graphicx}
\usepackage{doi}
\usepackage{multirow}
\usepackage{multicol}
\usepackage{xcolor}
\usepackage{listings}
\usepackage{subcaption} % Ensure proper usage of subfigure captions
\usepackage[normalem]{ulem}
\usepackage{amsmath,amsfonts,bm}
\usepackage{amssymb,amsthm}
\usepackage{enumitem}
\usepackage{algorithm}
\usepackage{algorithmic}
\usepackage{footnote}
\makesavenoteenv{tabular}
\usepackage[
    backend=biber,
    citestyle=authoryear,
    bibstyle=authortitle,
    mincitenames=1,
    maxcitenames=1,
    maxbibnames=3,
]{biblatex}
\addbibresource{template.bib}

\newcommand{\citep}[1]{\parencite{#1}}
\newcommand{\ours}{{Moonlight}}

\usepackage{listings}
\lstset{
    basicstyle=\ttfamily\small, % 字体和大小
    commentstyle=\color{gray}, % 注释颜色
    keywordstyle=\color{blue}, % 关键字颜色
    stringstyle=\color{red}, % 字符串颜色
    breaklines=true, % 自动换行
    numbers=left, % 行号在左侧
    numberstyle=\tiny\color{gray}, % 行号样式
    frame=shadowbox, % 添加阴影边框
    rulesepcolor=\color{blue}, % 边框颜色
    xleftmargin=10pt, % 左边距
    xrightmargin=10pt % 右边距
}


\newtheoremstyle{italicstyle} % 风格名称
  {3pt} % 空白空间在定理上方
  {3pt} % 空白空间在定理下方
  {\itshape} % 定理内容的字体
  {} % 缩进
  {\itshape} % 定理标题的字体
  {.} % 定理标题后的标点符号
  {.5em} % 定理标题和内容之间的间距
  {} % 定理标题后的额外内容
\theoremstyle{italicstyle}
\newtheorem{lemma}{Lemma}
\newtheorem{principle}{Principle}

% Define four custom colors (dark blue, dark green, dark red, dark purple as examples)
\definecolor{darkblue}{rgb}{0.0, 0.0, 0.5}
\definecolor{darkgreen}{rgb}{0.0, 0.5, 0.0}
\definecolor{darkred}{rgb}{0.5, 0.0, 0.0}
\definecolor{darkpurple}{rgb}{0.5, 0.0, 0.5}
\newcommand{\todo}[1]{\textcolor{darkred}{[TODO]: #1}}
\newcommand{\wyx}[1]{{\color{red}[yuxin: #1]}}

\setlist[itemize,1]{leftmargin=\dimexpr 18pt}
\setlist[enumerate,1]{leftmargin=\dimexpr 18pt}

\title{
\raisebox{-0.1\height}{\includegraphics[width=0.032\textwidth]{figures/logo.png}} %
Muon is Scalable for LLM Training
}


\author{Kimi Team}

% Uncomment to remove the date
\date{}

% Uncomment to override  the `A preprint' in the header
\renewcommand{\headeright}{Technical Report}
\renewcommand{\undertitle}{Technical Report}
\renewcommand{\shorttitle}{\raisebox{-0.12\height}{\includegraphics[width=0.02\textwidth]{figures/logo.png}}
Muon is Scalable for LLM Training}

\author{%
    \textbf{Jingyuan Liu}$^{1}$ \quad \textbf{Jianlin Su}$^{1}$ \quad \textbf{Xingcheng Yao}$^{2}$ \quad \textbf{Zhejun Jiang}$^{1}$ \quad \textbf{Guokun Lai}$^{1}$ \quad \textbf{Yulun Du}$^{1}$ \\
    \textbf{Yidao Qin}$^{1}$ \quad \textbf{Weixin Xu}$^{1}$ \quad \textbf{Enzhe Lu}$^{1}$ \quad \textbf{Junjie Yan}$^{1}$ \quad \textbf{Yanru Chen}$^{1}$ \quad \textbf{Huabin Zheng}$^{1}$ \\ \quad \textbf{Yibo Liu}$^{1}$ 
    \quad \textbf{Shaowei Liu}$^{1}$ \quad \textbf{Bohong Yin}$^{1}$ \quad \textbf{Weiran He}$^{1}$ \quad \textbf{Han Zhu}$^{1}$ \quad \textbf{Yuzhi Wang}$^{1}$ \quad \\ \textbf{Jianzhou Wang}$^{1}$ 
    \textbf{Mengnan Dong}$^{1}$ \quad \textbf{Zheng Zhang}$^{1}$ \quad \textbf{Yongsheng Kang}$^{1}$ \quad \textbf{Hao Zhang}$^{1}$ \quad \\ \textbf{Xinran Xu}$^{1}$ 
    \quad \textbf{Yutao Zhang}$^{1}$ \quad \textbf{Yuxin Wu}$^{1}$  \quad \textbf{Xinyu Zhou}$^{1}$ \thanks{Corresponding author: \texttt{zhouxinyu@moonshot.cn}} \quad \textbf{Zhilin Yang}$^{1} $
    \\[2ex]
    $^1$ Moonshot AI \quad $^2$ UCLA \quad
}



\begin{document}
\maketitle

\vspace{-10pt}
\begin{abstract}

Recently, the Muon optimizer~\citep{jordan2024muon} based on matrix orthogonalization has demonstrated strong results in training small-scale language models, but the scalability to larger models has not been proven. We identify two crucial techniques for scaling up Muon: (1) adding weight decay and (2) carefully adjusting the per-parameter update scale. These techniques allow Muon to work out-of-the-box on large-scale training without the need of hyper-parameter tuning. Scaling law experiments indicate that Muon achieves $\sim\!2\times$ computational efficiency compared to AdamW with compute optimal training.
Based on these improvements, we introduce \ours, a 3B/16B-parameter Mixture-of-Expert (MoE) model trained with 5.7T tokens using Muon. Our model improves the current Pareto frontier, achieving better performance with much fewer training FLOPs compared to prior models.
We open-source our distributed Muon implementation that is memory optimal and communication efficient. We also release the pretrained, instruction-tuned, and intermediate checkpoints to support future research.

\end{abstract}





\section{Introduction}

% State of the world (robots for creative activites)
The term ``robot,'' originally signifying `forced labor,' has long been associated with labor and work. Robots have demonstrated their utility in various automated productive and social contexts, where the primary goals are improving productivity, safety, and fostering social interactions with humans~\cite{simoes2022designing, weidemann2021role, honig2018understanding}. However, an increasing number of cases feature using of robots in creative settings. Unlike productive contexts, where the focus is on efficiency and task completion~\cite{arents2022smart}, or social contexts, where communication and trust are prioritized~\cite{nam2020trust, saunderson2019robots}, creative environments prioritize artistic innovation and expression~\cite{hsueh2024counts}. This shift fundamentally alters the dynamics of human-robot interaction, redefining the roles and expectations for both humans and robots.

For instance, robots’ social behaviors are leveraged to support the generation and expression of creative ideas~\cite{hu2021exploring, sandoval2022human, alves2020creativity}, and programmable robotic movements and trajectories are employed to inspire artistic activities such as sketching~\cite{lin2020your}. These studies often engage participants from creative fields who possess limited prior experience with robotics, and are typically conducted in short-term, experimental settings. Consequently, the findings from these studies remain constrained since much can be learned from professional practitioners' experiences to inform system design such as digital fabrication~\cite{hirsch2023nothing}. There is a notable gap in research examining the long-term, active, and practical experience of integrating robotic systems into the creative processes. As a result, the deeper insights into how robots facilitate and shape creative processes, beyond simply augmenting human creativity, remain underexplored. In this study, we aim to better understand the impacts of robots on creative processes and outcomes.

As early as Leonardo da Vinci's 16th century ``Automaton,'' artists have explored the creative affordances of robotic systems~\cite{shanken2002cybernetics, pagliarini2009development, jeon2017robotic}. The artistic creation process typically encompasses various stages, including the exploration of materials and techniques, ongoing experimentation and iteration, and the continual refinement of the artists' insights into their creative subjects~\cite{lewis2023art, sturdee2022state}. Therefore, investigating the artistic process involving robots offers an opportunity to gain deeper insights into robots' creative potential. Robotic art, in particular, provides a compelling case for this exploration.

We define robotic art as artworks that utilize robotic or automated machines to create artistic experiences and tangible artifacts. One example is robotic installation art, in which robots are programmed to follow specific rules that embody the artist’s expression (\autoref{fig:teaser} (a)). Another example is responsive art, in which robots react to their environment, with behaviors that change over time or in response to spectators (\autoref{fig:teaser} (b)). Additionally, there are robotic creators, which possess a degree of agency, allowing them to collaborate with human artists and produce works that extend beyond mere replication of human-created art (\autoref{fig:teaser} (c) and (d)). As such, robotic art becomes a rich case for exploring human-machine interactions in creative contexts. Gaining a deeper understanding of how robots facilitate artistic expression can provide insights for designing computing systems to support creative activities~\cite{gomez2021robot}.

% Therefore, we did...
We draw on semi-structured, in-depth interviews with renowned professional robotic artists to investigate the use of robots in artistic practice. Specifically, our goal is to understand how artistic exploration of robotic systems challenges conventional assumptions about the functions of robots, such as their roles in automating repetitive tasks or serving human needs. We also explore the implications of robots in the artistic process and examine how creativity may emerge within robotic art. To address these interrelated inquiries, our study focuses on the practice of robotic art, posing the research question: \textit{How do robotic artists utilize robots in their artistic practice?} We approach this inquiry through the perspectives and experiences of robotic artists, who creatively design, modify, and repurpose robotic systems for artistic expression and exploration.

% The key findings are...
Our findings highlight the social, material, and temporal dimensions of artists' practices that shape their creativity and artistic outcomes. The creation of robotic art is largely a social process, as artists receive both explicit and implicit feedback through the audience's reactions and reception of their work. Simultaneously, the embodiment and malfunctions inherent to robotic systems drive artistic experimentation. The temporal processes of creation and exhibition, beyond just the final product, further enhance the creative value. Our empirical analysis presents how creativity emerges through the interplay of social, material, and temporal interactions among artists, robots, audiences, and the environment.

% The contributions of this work are...
We make two main contributions to HCI in this study. 
First, we elucidate the interactive mechanisms among key actors---human creators, machines, audiences, and environments---within the practice of robotic art, a topic that remains underexplored in HCI. Our findings reveal the significance of sociality (e.g., interactions between artists and audiences), materiality (e.g., the embodiment and malfunctions of robots), and temporality (e.g., the processes of creation and exhibition) in shaping creative values. We propose that these three facets are central to the creative process and facilitate the emergence of creativity in robotic art.
Second, drawing from the findings, we offer implications for \textit{socially informed}, \textit{material-attentive}, and \textit{process-oriented} creation with computing systems. We suggest leveraging these three aspects to enhance creativity and the creative experience. Specifically, we discuss the value of incorporating implicit audience feedback, designing with technical malfunctions, and focusing on the post-creation process to foster alternative creative experiences with machines~\cite{alter2010designing, juarez2022glitch}.



\section{Methods}

\subsection{Background}

\paragraph{The Muon Optimizer}
\label{sec:analysis:background}
Muon~\citep{jordan2024muon} has recently been proposed to optimize neural network weights representable as matrices. At iteration $t$, given current weight $\mathbf{W}_{t-1}$, momentum $\mu$, learning rate $\eta_t$ and objective $\mathcal{L}_t$, the update rule of the Muon optimizer can be stated as follows:
\begin{align}
    \mathbf{M}_t &= \mu \mathbf{M}_{t-1} + \nabla\mathcal{L}_t(\mathbf{W}_{t-1}) \notag \\
    \mathbf{O}_t &= \text{Newton-Schulz}(\mathbf{M}_t)\text{\footnotemark[1]} \label{eq:Ot}\\
    \mathbf{W}_t &= \mathbf{W}_{t-1} - \eta_t \mathbf{O}_t \notag
\end{align}
 Here, $\mathbf{M}_t$ is the momentum of gradient at iteration $t$, set as a zero matrix when $t = 0$. In Equation~\ref{eq:Ot}, a Newton-Schulz iteration process~\citep{bernstein2024oldoptimizernewnorm} is adopted to approximately solve $(\mathbf{M}_t \mathbf{M}^{\mathrm{T}}_t)^{-1/2}\mathbf{M}_t$\footnotetext[1] {In practice, we follow~\citep{jordan2024muon} to use a Nesterov-style momentum by putting $\mu \mathbf{M}_t + \nabla\mathcal{L}_t(\mathbf{W}_{t-1})$ to the Newton-Schulz iteration instead of $\mathbf{M}_t$.}. Let $\mathbf{U}\mathbf{\Sigma} \mathbf{V}^\mathrm{T} = \mathbf{M}_t$ be the singular value decomposition (SVD) of $\mathbf{M}_t$, we will have $(\mathbf{M}_t \mathbf{M}^{\mathrm{T}}_t)^{-1/2}\mathbf{M}_t = \mathbf{U}\mathbf{V^T}$, which orthogonalizes $\mathbf{M}_t$. Intuitively, orthogonalization can ensure that the update matrices are isomorphic, preventing the weight from learning along a few dominant directions~\citep{jordan2024muon}.

\paragraph{Newton-Schulz Iterations for Matrix Orthogonalization}
Equation~\ref{eq:Ot} is calculated in an iterative process. At the beginning, we set $\mathbf{X}_0 = \mathbf{M}_t / \|\mathbf{M}_t\|_\mathrm{F}$. Then, at each iteration $k$, we update $\mathbf{X}_k$ from $\mathbf{X}_{k-1}$ as follows:
\begin{align}
    \mathbf{X}_k &= a \mathbf{X}_{k-1} + b (\mathbf{X}_{k-1} \mathbf{X}_{k-1}^\mathrm{T}) \mathbf{X}_{k-1} + c (\mathbf{X}_{k-1} \mathbf{X}_{k-1}^\mathrm{T})^2 \mathbf{X}_{k-1} \label{eq:iteration}
\end{align}
where $\mathbf{X}_N$ is the result of such process after $N$ iteration steps.
Here $a$, $b$, $c$ are coefficients. In order to ensure the correct convergence of Equation~\ref{eq:iteration}, we need to tune the coefficients so that the polynomial $f(x) = a x + b x^3 + c x^5$ has a fixed point near 1. In the original design of \cite{jordan2024muon}, the coefficients are set to $a = 3.4445$, $b = -4.7750$, $c = 2.0315$ in order to make the iterative process converge faster for small initial singular values. In this work, we follow the same setting of coefficients.

\paragraph{Steepest Descent Under Norm Constraints}
\cite{bernstein2024oldoptimizernewnorm} proposed to view the optimization process in deep learning as steepest descent under norm constraints. From this perspective, we can view the difference between Muon and Adam~\citep{adam2015kingma, loshchilov2018decoupled} as the difference in norm constraints. Whereas Adam is a steepest descent under the a norm constraint dynamically adjusted from a Max-of-Max norm, Muon offers a norm constraint that lies in a static range of Schatten-$p$ norm for some large $p$~\citep{muoncase2024cesista}. When equation~\ref{eq:Ot} is accurately computed, the norm constraint offered by Muon will be the spectral norm. Weights of neural networks are used as operators on the input space or the hidden space, which are usually (locally) Euclidean~\citep{cesista2024firstordernormedopt}, so the norm constraint on weights should be an induced operator norm (or spectral norm for weight matrices). In this sense, the norm constraint offered by Muon is more reasonable than that offered by Adam.

\subsection{Scaling Up Muon}
\label{sec:analysis:rms}

\paragraph{Weight Decay}

While Muon performs significantly better than AdamW on a small scale as shown by \cite{jordan2024muon}, we found the performance gains diminish when we scale up to train a larger model with more tokens. We observed that both the weight and the layer output's RMS keep growing to a large scale, exceeding the high-precision range of bf16, which might hurt the model's performance. To resolve this issue, we introduced the standard AdamW (\cite{loshchilov2018decoupled}) weight decay mechanism into Muon\footnote{The original implementation of Muon omits weight decay. A recent concurrent work in Muon incorporates weight decay and demonstrates improved performance. See \href{https://github.com/KellerJordan/Muon/commit/e0ffefd4f7ea88f2db724caa2c7cfe859155995d}{this commit} and \href{https://x.com/kellerjordan0/status/1888320690543284449}{this discussion}.}. 


\begin{align}
\label{equation:weightdecay}
    \mathbf{W}_t = \mathbf{W}_{t-1} - \eta_t (\mathbf{O}_t + \lambda \mathbf{W}_{t-1})
\end{align}

We experimented on Muon both with and without weight decay to understand its impact on the training dynamics of LLMs. Based on our scaling law research in Sec \ref{sec:exp:moonscalinglaw}, we trained an 800M parameters model with 100B tokens ($\sim5\times$ optimal training tokens). Figure \ref{fig_weight_decay} shows validation loss curves of the model trained with AdamW, vanilla Muon (without weight decay), and Muon with weight decay. While vanilla Muon initially converges faster, we observed that some model weights grew too large over time, potentially limiting the model's long-term performances. Adding weight decay addressed this issue - the results demonstrate that Muon with weight decay outperforms both vanilla Muon and AdamW, achieving lower validation loss in the over-train regime. Therefore, we adjusted our update rule to equation \ref{equation:weightdecay}, where $\lambda$ is the weight decay ratio.


\begin{figure}[t]
    \centering
    \includegraphics[width=0.8\textwidth]{figures/fig_weight_decay.pdf}
    \caption{\small Validation loss curves for AdamW (\textcolor[HTML]{2ecc71}{green}), Muon without weight decay (\textcolor[HTML]{e74c3c}{red}), and Muon with weight decay (\textcolor[HTML]{3498db}{blue}).} 
    \label{fig_weight_decay} 
\end{figure}



\paragraph{Consistent update RMS}
An important property of Adam and AdamW (\cite{adam2015kingma}, \cite{loshchilov2018decoupled}) is that they maintain a theoretical update RMS around 1\footnote{Due to Adam's $\beta_1 < \beta_2$ and $\epsilon > 0$, the actual update RMS is usually less than 1.}. However, we show that Muon's update RMS varies depending on the shape of the parameters, according to the following lemma:

\begin{lemma}
\label{lemma:updaterms}
For a full-rank matrix parameter of shape $[A, B]$, its theoretical Muon update RMS is $\sqrt{1/\max(A,B)}$ .
\end{lemma}

The proof can be found in the Appendix \ref{sec:appendix:updaterms}. We monitored Muon's update RMS during training and found it typically close to the theoretical value given above. We note that such inconsistency can be problematic when scaling up the model size:

\begin{itemize}
    \item When $\max(A,B)$ is too large, e.g. the dense MLP matrix, the updates become too small, thus limiting the model's representational capacity and leading to suboptimal performances; 
    
    \item When $\max(A,B)$ is too small, e.g. treating each KV head in GQA (\cite{shazeer2019fasttransformerdecodingwritehead}) or MLA (\cite{deepseekai2024deepseekv3technicalreport}) as a separate parameter, the updates become too large, thus causing training instabilities and leading to suboptimal performances as well.
\end{itemize}

In order to maintain consistent update RMS among matrices of different shapes, we 
propose to scale the Muon update for each matrix by its $\sqrt{\max(A, B)}$ to cancel the effect of Lemma~\ref{lemma:updaterms} \footnote{\cite{jordan2024muon}'s original implementation scales the updates by $\sqrt{\max(1, A/B)}$, which is equivalent to our proposal (up to a global scale) if all matrices have the same second dimension; \cite{pethick2025trainingdeeplearningmodels} and \cite{JiachengX} discussed a similar issue on update scaling factors concurrently to our work. } . 
Experiments in Sec~\ref{sec:exp:rms} show that this strategy is beneficial for optimization.

\paragraph{Matching update RMS of AdamW}

Muon is designed to update matrix-based parameters. In practice, AdamW is used in couple with Muon to handle non-matrix based parameters, like RMSNorm, LM head, and embedding parameters. 
We would like the optimizer hyper-parameters (learning rate $\eta$, weight decay $\lambda$) to be shared among
matrix and non-matrix parameters. 

We propose to match Muon's update RMS to be similar to that of AdamW. From empirical observations, AdamW's update RMS is usually around 0.2 to 0.4. Therefore, we scale Muon's update RMS to this range by the following adjustment:

\begin{align}
\mathbf{W}_t = \mathbf{W}_{t-1} - \eta_t (0.2\cdot\mathbf{O}_t\cdot\sqrt{\max(A,B)} + \lambda \mathbf{W}_{t-1})
\end{align}

 We validated this choice with empirical results (see Appendix \ref{sec:appendix:updaterms} for details). 
Moreover, we highlighted that with this adjustment, Muon can directly \textbf{reuse} the learning rate and weight decay tuned for AdamW. 

\paragraph{Other Hyper-parameters} Muon contains two other tunnable hyper-parameters: Newton-Schulz iteration steps and momentum $\mu$. We empirically observe that when setting $N$ to $10$, the iterative process will yield a more accurate orthogonalization result than $N=5$, but it won't lead to better performances. Hence we set $N = 5$ in this work for the sake of efficiency. We do not see a consistent performance gain in tuning momentum, so we chose 0.95, same as \cite{jordan2024muon}.

\subsection{Distributed Muon}
\label{sec:analysis:distrib}

\paragraph{ZeRO-1 and Megatron-LM}
\cite{Rajbhandari_2020} introduced the ZeRO-1 technique that partitions the expensive optimizer states (e.g. master weights, momentum) all over the cluster. Megatron-LM \citep{shoeybi2020megatronlmtrainingmultibillionparameter} integrated ZeRO-1 into its native parallel designs. Based on Megatron-LM's sophisticated parallel strategies, e.g. Tensor-Parallel (TP), Pipeline Parallel (PP), Expert Parallel (EP) and Data Parallel (DP), the communication workload of ZeRO-1 can be reduced from gathering all over the distributed world to only gathering over the data parallel group.

\paragraph{Method}
ZeRO-1 is efficient for AdamW because it calculates updates in an element-wise fashion. However, Muon requires the full gradient matrix to calculate the updates. Therefore, vanilla ZeRO-1 is not directly applicable to Muon. We propose a new distributed solution based on ZeRO-1 for Muon, referred to as Distributed Muon. Distributed Muon follows ZeRO-1 to partition the optimizer states on DP, and introduces two additional operations compared to a vanilla Zero-1 AdamW optimizer:

\begin{enumerate}
    \item \texttt{DP Gather.} For a local DP partitioned master weight ($1/DP$ the size of the model weight), this operation is to gather the corresponding partitioned gradients into a full gradient matrix. 
    
    \item \texttt{Calculate Full Update.} After the above gathering, perform Newton-Schulz iteration steps on the full gradient matrix as described in Sec \ref{sec:analysis:background}. Note that we will then discard part of the full update matrix, as we only need the partition corresponding to the local parameters to perform update.
\end{enumerate}


The implementation of Distributed Muon is described in Algorithm \ref{alg:distribmuon}. The additional operations introduced by Distributed Muon are colored in blue.

\begin{algorithm}[t]
\caption{Distributed Muon}
\label{alg:distribmuon}
\begin{algorithmic}[1]
\REQUIRE{Full Gradients $\mathbf{G}$, DP partitioned Momentum $\mathbf{m}$, DP partitioned parameters $\mathbf{p}$, momentum $\mu$.}
\STATE // Reduce-scatter $G$ on DP for correct gradients
\STATE $\mathbf{g} = \text{reduce\_scatter($\mathbf{G}$, dp\_group)}$ 
\STATE // Apply momentum to $\mathbf{g}$   using local partitioned momentum $\mathbf{m}$
\STATE $\mathbf{g}' = \text{update\_with\_momentum}(\mathbf{g}, \mathbf{m}, \mu)$
\STATE \textcolor{blue}{// DP Gather: gathering $\mathbf{g'}$ across DP into a full matrix $\mathbf{G}$}
\STATE \textcolor{blue}{$\mathbf{G} = \text{gather($\mathbf{g'}$, dp\_group)}$}
\STATE \textcolor{blue}{// Calculate Muon update}
\STATE \textcolor{blue}{$\mathbf{U} = \text{Newton-Schulz}(\mathbf{G})$ }
\STATE \textcolor{blue}{// Discard the rest of $\mathbf{U}$ and only keep the local partition  ${\mathbf{u}}$, then apply the update rule}
\STATE $\mathbf{p}' = \text{apply\_update}(\mathbf{p}, \mathbf{u})$
\STATE // All-gather updated $\mathbf{p'}$ into $\mathbf{P}$ 
\STATE $\mathbf{P} = \text{all\_gather($\mathbf{p'}$, dp\_group)}$
\STATE // Return the update RMS for logging
\RETURN $\sqrt{\mathbf{u}^2.\texttt{mean}()}$ 
\end{algorithmic}
\end{algorithm}


\paragraph{Analysis}
We compared Distributed Muon to a classic ZeRO-1 based distributed AdamW (referred as Distributed AdamW for simplicity) in several aspects:

\begin{itemize}
\item \texttt{Memory Usage.} Muon uses only one momentum buffer, while AdamW uses two momentum buffers. Therefore, the additional memory used by the Muon optimizer is half of Distributed AdamW.

\item \texttt{Communication Overhead.} For each device, the additional DP gathering is only required by the local DP partitioned parameters $\mathbf{p}$. Therefore, the communication cost is less than the reduce-scatter of $\mathbf{G}$ or the all-gather of $\mathbf{P}$. Besides, Muon only requires the Newton-Schulz iteration steps in bf16, thus further reducing the communication overhead to 50\% comparing to fp32. Overall, the communication workload of Distributed Muon is $(1, 1.25]$ of that of Distributed AdamW. The upper-bound is calculated as that the communication of Distributed Muon is 4 (fp32 $\mathbf{G}$ reduce-scatter) + 2 (bf16 Muon gather) + 4 (fp32 $\mathbf{P}$ all-gather), while Distributed AdamW is 4 + 4. In practice, as we usually train with multiple DP, the empirical additional cost usually is closer to the lower-bound 1.\footnote{If TP is enabled, Distributed Muon needs an extra bf16 TP gather on TP group.}.

\item \texttt{Latency.} Distributed Muon has larger end-to-end latencies than Distributed AdamW because it introduces additional communication and requires running Newton-Schulz iteration steps. However, this is not a significant issue because (a) only about 5 Newton-Schultz iteration steps are needed for a good result (discussed in Sec \ref{sec:analysis:rms}), and (b) the end-to-end latency caused by the optimizer is negligible compared to the model's forward-backward pass time (e.g. usually 1\% to 3\%). Moreover, several engineering techniques, such as overlapping gather and computation, and overlapping optimizer reduce-scatter with parameter gather, can further reduce latency.


\end{itemize}

When training large-scale models in our distributed cluster, Distributed Muon has no noticeable latency overhead compared to its AdamW counterparts. We will soon release a pull request that implements Distributed Muon for the open-source Megatron-LM \citep{shoeybi2020megatronlmtrainingmultibillionparameter} project.

\section{Experiments}
\label{sec:sup_esperiments}
In this section, we will provide a more detailed overview of the experimental setups, present additional visualization results and runtime comparisons, carry out further ablation studies, and address the limitations of our proposed method.

\begin{figure}[!t]
  \centering
    \includegraphics[width=1.0\linewidth]{Images/MHRNID.pdf}
  \caption{Examples of the MHRNID dataset.} 
  \label{fig:mhrnid}
\end{figure}

\subsection{Experimental Setups}
We implement all the experiments using PyTorch Lightning on multiple NVIDIA A40 GPUs. All experiments were conducted once after setting the seed to the same values as~\cite{yu2022towards} and~\cite{undem}.

\subsubsection{Implementation Details of other Comparison Methods}
For Shooting, we migrated their implementation code from Opencv to PyTorch based on the implementation idea provided by~\cite{shooting}. Note that the Shooting method produces a distorted composite image after random projective transformation. We maintain the transformation parameter and adjust the clean image accordingly to ensure that the moiré image aligns with the clean image during the subsequent demoiréing stage.
For UnDeM~\cite{undem}, we directly use their 384$\times$384 moiré image synthesis network trained on UHDM~\cite{yu2022towards} and FHDMi~\cite{he2020fhde} and also train their synthesis network on TIP~\cite{sun2018moire} in their code framework~\cite{undem}.
For MoireSpace~\cite{yang2023doing}, we utilize the moiré patterns provided by their dataset to obtain the synthesis result by deploying their multiply blending strategy. We resize their moiré patterns to 384$\times$384 for a fair comparison.

\input{Tables/tbl_runtime}

\subsubsection{Mixed High-Resolution Natural Image Dataset}
In the Zero-Shot experiments, we collected a comprehensive Mixed High-Resolution Natural Image Dataset (MHRNID) to avoid data overlap between the training and test sets. The MHRNID dataset consists of the super-resolution datasets DF2K-OST~\cite{wang2021real}, the natural image datasets UHD-LOL4K~\cite{wang2023uhdlol4k}, and UHD-IQA~\cite{hosu2024uhdiqa} collated and incorporated, which contains 26,000 high-definition images. We also provide several visual examples of MHRNID, as shown in Figure~\ref{fig:mhrnid}.

\subsubsection{Implementation Details of Demoiréing Models}
For MBCNN~\cite{zheng2020image} and ESDNet-L~\cite{yu2022towards}, we followed the experimental settings from~\cite{yu2022towards} and~\cite{undem}. We trained for 150 epochs on UHDM~\cite{yu2022towards} and FHDMi~\cite{he2020fhde} and 70 epochs on TIP~\cite{sun2018moire}. Additionally, we trained for 50 epochs on the MHRNID dataset.

% \subsection{More Qualitative Comparisons}


\subsection{More Qualitative Comparisons}

\subsubsection{Moiré Image Synthesis}
The visualization results of synthesis moiré images on the MHRNID dataset using Shooting~\cite{shooting}, UnDeM~\cite{undem}, and our UniDemoiré are shown in Figure~\ref{fig:synthesis_compare}. 
The moiré image produced by our UniDemoiré is notably superior to other synthesis methods in terms of diversity and realism. In comparison, the moiré image generated by the Shooting~\cite{shooting} method is excessively distorted, UnDeM's network~\cite{undem} is susceptible to anomalies during image generation, and the moiré pattern dataset provided by MoireSpace~\cite{yang2023doing} is of subpar quality. Additionally, the multiplication strategy results in a darker synthesized image.


\subsubsection{Demoiréing}
Figure~\ref{fig:zero-shot} shows the visualization results of zero-shot demoiréing on UHDM~\cite{yu2022towards}. Additionally, Figures~\ref{fig:cd_fhdmi} and~\ref{fig:cd_tip} illustrate the demoiréing results on FHDMi~\cite{he2020fhde} and TIP~\cite{sun2018moire} using ESDNet-L~\cite{yu2022towards} trained on UHDM~\cite{yu2022towards}. Our method's model effectively removes moiré artifacts and retains high-frequency details, indicating the strong generalization ability of our proposed UniDemoiré.

% \subsubsection{Zero-Shot Demoiréing}
% The visualization results of zero-shot demoiréing results on UHDM and FHDMi are shown in Figure XX, YY. 

% \subsubsection{Cross-Datasets Demoiréing}
% The visualization results of demoiréing results on FHDMi~\cite{he2020fhde} and TIP~\cite{sun2018moire} using ESDNet-L~\cite{yu2022towards} trained on UHDM~\cite{yu2022towards} are shown in Figure ~\ref{fig:cd_fhdmi},~\ref{fig:cd_tip}. The model trained by our method removes moiré artifacts more cleanly and preserves high-frequency details better. This implies that our proposed UniDemoiré has good generalization ability.



\begin{figure}[t]
% \vspace{-3ex}
  \centering
    \includegraphics[width=1.0\linewidth]{Images/limitations.pdf}
\vspace{-1ex}
 \caption{Failure Examples.} 
  \label{fig:limitations}
% \vspace{-3ex}
\end{figure}






\subsection{Runtime Comparisons}
\label{sec:runtime}
Table~\ref{tab:Runtime} shows the comparison of the parameters and the running time of our synthesis module with other methods. 
To ensure fair comparisons, our method and UnDeM use torchinfo for parameter counting, with all methods utilizing 256x256 input images.
Our experimental results indicate that our method, slightly exceeding UnDeM in parameters, achieves a runtime comparable to non-learning algorithms like Shooting and MoireSpace, demonstrating the efficiency of our MIB and TRN.
Furthermore, our model's FLOPs are 5.6266G, significantly lower than UnDeM's 26.7576G, indicating high performance and reduced computational cost.


\subsection{Additional Ablation Study}
\label{sec:sup_ablation}
The results of the additional ablation experiments are in Table~\ref{tab:Exp_ablation_sup}. 
where ``$\mathcal{L}_{per} \rightarrow \mathcal{L}_1$'' denotes replacing the perception loss $\mathcal{L}_{per}$ in the synthesis network with the L1 loss $\mathcal{L}_1$. ``Uformer $\rightarrow$ UNet" denotes switching the entire backbone network of the TRN from Uformer to UNet~\cite{ronneberger2015u}. For a fair comparison, we kept the number of upsampling/downsampling blocks and the base channel in UNet consistent with Uformer, while removing the attention block.



% \begin{table}[t]
% \caption{Ablation experiments with our method on the target dataset FHDMi. Note that both blending and demoireing training processes here are performed on the UHDM dataset. The ``B'' represents the dataset used in the blending process, and the ``DT'' represents the dataset used in the demoireing training process. The ``None'' indicates only paired moiré data are used to train ESDNet-L~\cite{yu2022towards}.}
% \centering
% \renewcommand\tabcolsep{5.0pt}
% \vspace{-8pt}
% % \resizebox{10cm}{!}       % \textwidth, 10cm, 这里用 10cm 好像效果好于 \textwidth
% \scalebox{0.79}
% {       

% \begin{tabular}{ccc|ccc}
% \toprule
% \multirow{2}{*}{Model}    & \multirow{2}{*}{Datasets(B\&DT)} & \multirow{2}{*}{Components} & \multicolumn{3}{c}{Metric}            \\ 
% \cmidrule{4-6} 
%                           &                                  &                             & \ua{PSNR } & \ua{SSIM } & \da{LPIPS } \\ 
% \midrule
% \multirow{5}{*}{ESDNet-L~\cite{yu2022towards}} & \multirow{5}{*}{UHDM}            & ALL                         & \textbf{20.7563}    & \textbf{0.7771}     & \textbf{0.2425}      \\
%                           &                                  & $w/o$ LDM                   & 20.6868    & 0.7535     & 0.2502      \\
%                           &                                  & $w/o$ TRN                   & 20.5869    & 0.7538     & 0.2515      \\
%                           &                                  & $w/o$ fusion block          & 20.5005    & 0.7510     & 0.2511      \\ 
%                           &                                  & $None$                      & 20.3422    & 0.7699     & 0.2525      \\ 
% \bottomrule
% \end{tabular}

% }
% \label{tab:Exp_ablation}
% \end{table}

% \begin{table}[h]
% \caption{Additional ablation studies. UHDM is the source dataset for training, while FHDMi is the target dataset for testing.
% The ``loss (pixel space)'' denotes 
% }
% \centering
% % \renewcommand\tabcolsep{5.0pt}
% % \vspace{-8pt}
% % \resizebox{10cm}{!}       % \textwidth, 10cm, 这里用 10cm 好像效果好于 \textwidth
% \scalebox{0.95}
% {       

% \begin{tabular}{cc|ccc}
% \toprule
% \multirow{2}{*}{Model} & \multirow{2}{*}{Components} & \multicolumn{3}{c}{Metric}                     \\ 
% \cmidrule{3-5} 
%                           &                    & \ua{PSNR }       & \ua{SSIM }      & \da{LPIPS }     \\ 
% \midrule
% \multirow{5}{*}{ESDNet-L} & ALL                & \textbf{22.0638} & \textbf{0.8021} & \textbf{0.1707} \\
%                           & $\mathcal{L}^{(pixel)}$ & 20.6696          & 0.7701          & 0.2389          \\
%                           & $w/o$ Grain Merge  & 20.6599          & 0.7658          & 0.2451          \\
%                           & $w/o$ Multiply     & 20.5479          & 0.7670          & 0.2412          \\ 
%                           % &             $None$                      & 20.3422    & 0.7699     & 0.2525      \\ 
% \bottomrule
% \end{tabular}
% }
% \label{tab:Exp_ablation_sup}
%  % \vspace{-2ex}
% \end{table}



\begin{table}[h]
\centering
\setlength{\tabcolsep}{1.9mm}
\scalebox{1.0}{
\begin{tabular}{lccc}
%\scalebox{0.9}{
%\begin{tabular}{lccc}
\toprule
% \multirow{2.5}{*}{Components}  & \multicolumn{3}{c}{Test set: FHDMi}  \\ \cmidrule(l){2-4} 
Components                     &\ua{PSNR}   &\ua{SSIM} &\da{LPIPS} \\ 
% \midrule
% \multicolumn{4}{l}{\textit{Using:   \textbf{ESDNet-L}, trained with \textbf{UHDM}, 50 Epochs}} \\ 
\midrule
ALL                                                      & \textbf{20.7543} & \textbf{0.7653} & \textbf{0.2136} \\
MIB ($w/o$ Multiply)                                     & 20.3158          & 0.7598          & 0.2328          \\
MIB ($w/o$ Grain Merge)                                  & 20.3930          & 0.7587          & 0.2414          \\
TRN ($w/o$ CARAFE)                                       & 20.4414          & 0.7408          & 0.2256          \\
TRN ($\mathcal{L}_{per}$ $\rightarrow$ $\mathcal{L}_1$)  & 20.1404          & 0.7447          & 0.2495          \\
TRN (Uformer $\rightarrow$ UNet)                         & 20.3899          & 0.7476          & 0.2413          \\
\bottomrule
\end{tabular}
}
\caption{Additional ablation studies. Source: UHDM, Target: FHDMi.}

\label{tab:Exp_ablation_sup}
\end{table}


		
		
		
		
		


The results of two sets of ablation experiments on layer blending strategies also show that using only one of them leads to distortion of the synthesis results, which in turn affects the model's generalization ability.
The results of the ``$\mathcal{L}_{per}\rightarrow\mathcal{L}_1$'' show that computing the loss function in this way leads to a degradation of the model performance because moiré patterns can disrupt image structures by generating strip-shaped artifacts. 
The results of the ``$w/o$ CARAFE'' indicate that using the CARAFE upsampling operator~\cite{wang2019carafe} yields better fusion performance than the transposed convolution originally employed by Uformer~\cite{Wang2022Uformer}.
Furthermore, the results from the “Uformer $\rightarrow$ UNet” demonstrate that the LeWin Transformer Block within Uformer is more effective at extracting color features from moiré patterns compared to the original UNet architecture.

% \begin{figure*}[t]
% % \vspace{-3ex}
%   \centering
%     \includegraphics[width=0.7\linewidth]{Images/limitations_3.pdf}
% \vspace{-1ex}
%  \caption{Failure Examples.} 
%   \label{fig:limitations}
% % \vspace{-3ex}
% \end{figure*}

\subsection{Limitations}

In some cases, particularly when the moiré artifacts in the target domain significantly differ from those in the source domain, our solution may struggle to completely remove all artifacts, as Figure~\ref{fig:limitations} shows. However, even in these challenging scenarios, our method tends to perform better at artifact removal compared to the baselines. Our performance can be further refined by generating more diverse moiré patterns and synthesized training data.
In Figure~\ref{fig:limitations}, we show a failure case. When the moiré artifacts in the target domain are too different from those in the source domain, our solution still struggles to produce a completely moiré-free result. However, we still remove the artifacts comparatively better than baselines.
% \yj{Will update based on your fig.}

\begin{figure*}[!p]
  \centering
    \includegraphics[width=1\linewidth]{Images/Synthesis_Result_Compare.pdf}
  \caption{
  Qualitative comparisons of synthesized moire images were obtained using the shooting method, UnDeM, MoireSpace, and our UniDemoiré.
  % Qualitative comparisons of our models with other state-of-the-art methods on the UHDM dataset using ESDNet-L.
  }
  \label{fig:synthesis_compare}
\end{figure*}

\begin{figure*}[!p]
  \centering
    \includegraphics[width=1.0\linewidth]{Images/Zero_Shot_Result.pdf}
  \caption{ Qualitative comparisons of zero-shot evaluation on the UHDM dataset.} 
  \label{fig:zero-shot}
\end{figure*}

\begin{figure*}[!p]
  \centering
    \includegraphics[width=1\linewidth]{Images/FHDMi_Result.pdf}
  \caption{
  Qualitative comparisons of our models with other state-of-the-art methods on the FHDMi dataset.
  % Qualitative comparisons of our models with other state-of-the-art methods on the UHDM dataset using ESDNet-L.
  }
  \label{fig:cd_fhdmi}
\end{figure*}

\begin{figure*}[!p]
  \centering
    \includegraphics[width=1\linewidth]{Images/TIP_Result.pdf}
  \caption{
  Qualitative comparisons of our models with other state-of-the-art methods on the TIP dataset.
  % Qualitative comparisons of our models with other state-of-the-art methods on the TIP dataset using ESDNet-L.
  }
  \label{fig:cd_tip}
\end{figure*}


\newpage
\clearpage

\section{Discussion}
\label{Summary}
The study of opinion dynamics has been conventionally done on networks with strictly positive edges. In the real world however, networks often contain negative social connections, which can spread negative or opposing influence, thus creating a need to understand how these edges affect influence maximisation efforts in networks. 
To address this concern, we present a model for competitive spread of opinions in signed networks under voter dynamics. For comparison, we propose a complementary approach where controllers only observe the absolute weights of all edges i.e. they consider all edges to be positive. In both instances we present gradient ascent algorithms to numerically solve the problem in large-scale arbitrary networks. We test the robustness of our results in networks of varied structures under diverse budget conditions and adversarial allocations. 
We find that in networks where 20\% of edges are negative, controllers gain maximally (nearly 18\%) from awareness of negative edges, under conditions of scarce resources, and against competitors who deliberately avoids nodes with negative connections. 

We also propose a supporting theoretical approach to verify the accuracy of our algorithms. We present closed-form solutions in simplified network structures that provide further insights to the problem. We observe that in networks with highly concentrated positive links, allocations on nodes are driven by their negative degrees and the competitor's allocation on these nodes. Finally, we examine the problem under game-theoretic settings, where we highlight conditions under which a controller could lose vote-shares by implementing strategies that use the knowledge of negative ties in the network. Specifically, we show that when controllers have considerably less resources (or in some cases, excess budget), their prioritisation of nodes to target, may inadvertently disclose knowledge of negative ties to a competitor who was otherwise unaware, thus compromising their position of advantage.

The results in this paper present compelling evidence for considering negative ties in any influence maximisation exercise and thus contributes to the literature on competitive opinion dynamics in signed networks. Possible extensions to this work could include studying the problem under different constraint functions. For instance, the effect of modified budget constraints that explore the implications of an additional cost to retrieve information about the presence of negative ties, on influence maximisation efforts.
Additionally, this problem could be further studied in other realistic opinion models (e.g. Deffaunt model). 
% Going forward, this work could also serve as a foundation to guide empirical investigation on maximising opinion spread in the presence of negative edges.    



% We now propose an analytical framework in support of our numerical results. Note that, obtaining closed-form analytical solution for \cref{optimisation} on networks with inherent complexities can be challenging. We therefore simplify the problem first by adopting a degree-based mean-field approach that approximates system dynamics and helps us obtain analytical expressions for optimal al

% In this work, we propose WildLong, a novel framework for synthesizing diverse, scalable, and realistic instruction-response datasets designed for long-context tasks. Our approach addresses key challenges in dataset creation by leveraging meta-information extraction from real-world user queries, graph-based modeling of co-occurrence relationships, and adaptive instruction-response generation.
% WildLong is built on the principles of diversity, scalability, and realism, enabling it to support complex reasoning tasks such as cross-document comparison, and aggregation, which are essential for real-world applications. By integrating meta-information into the data generation process and systematically exploring new combinations through graph-based modeling, WildLong generates diverse datasets that reflect the complexity of extended contexts.
% Experimental results demonstrate that WildLong significantly improves long-context task performance, surpassing other open-source long-context-optimized models across multiple benchmarks. Importantly, this improvement is achieved without requiring supplementary short-context instruction tuning, highlighting the robustness and generalizability of our approach.
% The success of WildLong highlights the potential of structured, meta-information-driven data synthesis to enhance the capabilities of LLMs for complex, real-world tasks. By addressing the critical gaps in long-context dataset diversity and quality, WildLong sets a new standard for long-context instruction tuning and paves the way for further advancements in equipping LLMs to tackle the challenges of extended-context reasoning.
% We propose WildLong, a framework for synthesizing diverse, scalable, and realistic instruction-response datasets for long-context tasks. By leveraging meta-information extraction, graph-based modeling, and adaptive instruction generation, WildLong generates long-context instruction-tuning data with real-world complexity.
% Experiments show improved long-context task performance while retaining short-context performance without additional short-context fine-tuning, demonstrating its robustness and generalizability. We hope WildLong provides insights into generalizing instruction tuning and inspires further advancements in long-context reasoning for LLMs.
We propose WildLong, a framework for synthesizing diverse, scalable, and realistic instruction-response datasets for long-context tasks. 
It integrates meta-information extraction to ensure realistic complexity, graph-based modeling for systematic instruction expansion, and adaptive instruction generation for enhanced contextual relevance.
Our fine-tuned models consistently outperform baselines and maintain short-context performance without mixing short-context data. Notably, our finetuned Llama-3.1-8B model surpasses most open-source long-context models on Longbench-Chat and demonstrates competitive performances with even larger models across benchmarks.
WildLong enables the synthesis of instruction-tuning data that produces robust models capable of handling diverse long-context tasks. Extending beyond synthetic QA and summarization, it bridges the gap to more complex, realistic challenges, advancing the effectiveness of long-context LLMs.
We hope WildLong provides insights into generalizing synthetic data and inspires further progress in long-context reasoning for LLMs.
\newpage
\printbibliography
\newpage
\appendix
\bibliographystyle{ACM-Reference-Format}
\bibliography{reference}

\appendix

\newpage

\section{Appendix A: Researchers Prompt Examples}
\label{appendix:a}

Below we provided researchers prompts examples, age and research experience (Exp), grouped by education level.

\begin{longtable}{|p{1cm}|p{2cm}|p{10cm}|}
\hline
\textbf{PID} & \textbf{Description} & \textbf{Prompt Example} \\ \hline
\multicolumn{3}{|c|}{\textbf{Master Students}} \\ \hline
P3 & Yr-1 Master \newline Age: 23 \newline Exp: 0.5 Yrs & I'm [anonymized] with 2 years of design experience, I fuse data, user insights, and business objectives to craft empowering user experiences, one interaction at a time. \newline I am interested in VR/AR and accessibility. I want to build a portfolio website focused on research projects. This website's color is based on orange and blue. \\ \hline
P8 & Yr-1 Master \newline Age: 23 \newline Exp: 1 Yrs & My name is [anonymized], I'm a first-year Human-Computer Interaction Master student. I studied Computer Science and Psychology, also really like cognitive science and drawing. I especially like comics, so I want my website to have some American comics styles. I want a personal website to display my front-end projects, my drawing works, and my research. Each project category will be a book (Book1, Book2, Book3...), when I click on the book, it will deliver me to the specific project page. \\ \hline
P11 & Yr-1 Master \newline Age: 23 \newline Exp: 0 Yrs & This is my homepage of my personal website. It includes the navigation bar, introduction of myself, and other basic things of the website page. The background color will be black, and the style of the website will be simple but also creative. My name is [anonymized], I used to study computer science, but now I changed into a design major. I think my website can show both of the knowledge about coding and design. The website will include pages for my research, my design, and my coding work, as well as a page for my personal life because I want to show my personality to the interviewer. \\ \hline
P4 & Yr-2 Master \newline Age: 23 \newline Exp: 3 Yrs & I'm [anonymized], a second-year master student of the human-computer interaction program. I studied psychology before and have some research projects related to that. I worked as a UX design intern in a few places. I'm graduating in May 2025 and want to apply for UX designer jobs. \\ \hline
\multicolumn{3}{|c|}{\textbf{Yr-1 PhD Students}} \\ \hline
P1 & Yr-1 PhD \newline Age: 25 \newline Exp: 3 Yrs & My name is [anonymized], and I'm a first-year PhD student at [anonymized] and a practicing speech-language pathologist. My research interests include AI integration when developing communication tools for AAC users. \\ \hline
P2 & Yr-1 PhD \newline Age: 25 \newline Exp: 4 Yrs & My name is [anonymized]. I'm now a first-year PhD student in [anonymized], working with Dr. [anonymized]. I have an education background in both electrical engineering and design. My research now focuses on AI-based assistive technologies, especially personalization systems for blind users. \\ \hline
P7 & Yr-1 PhD \newline Age: 29 \newline Exp: 4 Yrs & My website should showcase my current affiliation and my research publications. I wish it to be of vibrant color but with simplistic design. There should be different tabs to hold various content. \\ \hline
P10 & Yr-1 PhD \newline Age: 24 \newline Exp: 3 Yrs & Name: [anonymized] \newline Academic background: PhD student [anonymized] \newline Research interest: Data science for education using natural language processing tools \newline Personal hobbies: drawing, piano, cooking (pictures of dishes I cooked) \newline Style: minimalism \newline Base color: white \\ \hline
P13 & Yr-1 PhD \newline Age: 27 \newline Exp: 3.5 Yrs & My name is [anonymized], and I am an accessibility and UX researcher. I use he/him pronouns. I want to create an accessibility and data visualization portfolio website. I want to have a dark background with white text. The font size of the text should have high contrast and be very readable. \\ \hline
\multicolumn{3}{|c|}{\textbf{Yr-3/4 PhD Students}} \\ \hline
P5 & Yr-3 PhD \newline Age: 28 \newline Exp: 5.5 Yrs & I am [anonymized], a PhD student starting my third year in [anonymized]. My work is at the intersection of Human-Computer Interaction, Aging and Accessibility, and Personal and Health Informatics. My research focuses on investigating technologies for collecting and sharing personal health information among underrepresented populations, including older adults and people with mild cognitive impairment and dementia. Recently I have been working on supporting older adults in the data labeling process for training their personalized activity trackers. My work informs strategies that engage older adults as end-users in machine learning. \\ \hline
P6 & Yr-3 PhD \newline Age: 31 \newline Exp: 10 Yrs & This is my homepage for a website that I can use to showcase my credentials, blogging, and consulting work. I am [anonymized] and would like to introduce myself as a broadly trained social/behavioral scientist now working at the intersection of metascience and human-computer interaction. \\ \hline
P12 & Yr-3 PhD \newline Age: 29 \newline Exp: 4 Yrs & I want to build a research website showcasing my interests and publications. I am a 3rd year PhD student named [anonymized], my pronouns are [anonymized], and my research interests are broadly in multilingual NLP, human-centered NLP, authorship analysis, and explainability. \\ \hline
P9 & Yr-4 PhD \newline Age: 31 \newline Exp: 8 Yrs & I want the circles to be interlinked like a network and when you click on one I want it to expand and highlight more information and the rest to pull back to the sides. I would want to group them thematically with an overarching team science page. Each bubble you click on opens. The top right about corner would be static. \\ \hline
\end{longtable}

\newpage

\section{Appendix B: Prompts for Agentic Pipeline}
\label{appendix:b}

\subsection{Prompt for PRD generation}

\begin{lstlisting}
Please generate a Product Requirements Document (PRD) targeting the creation of a modern and user-friendly personal website for Junior Researchers based on the following user's sketch (the picture I sent you) and prompt.
User's prompt: ${userPrompt}
In the PRD, specify what images are needed and where they should be placed (e.g., hero image, profile image, etc.) using the format: [term(size)], please use concrete keywords like [(profile-picture)medium] instead of vague descriptions like [image1(small)].
There are 3 keywords for the size (small, medium, large, landscape, or portrait). Remember this only applies to images; for icons, you can just define them without the expected format.
Example: [portfolio-preview(landscape)]`
\end{lstlisting}


\subsection{Prompt for website code generation}

\begin{lstlisting}

You are a design engineer tasked with creating a user interface for junior researcher based on a user's wireframe sketch. Prioritize the user's considerations as design preferences while ensuring the design adheres to these principles:
1. Apply shadows judiciously enough to create depth but not overly done.
2. Use the Gestalt principles (proximity, similarity, continuity, closure, and connectedness) to enhance visual perception and organization.
3. Ensure accessibility, particularly in color choices; use contrasting colors for text, such as white text on suitable background colors, to ensure readability. Feel free to use gradients if they enhance the design's aesthetics and functionality.
4. Maintain consistency across the design.
5. Establish a clear hierarchy to guide the user's eye through the interface.
Additional considerations:
2. Utilize a CSS icon library Font Awesome in your <head> tag to include vector glyph icons.
3. Ensure all elements that can be rounded, such as buttons and containers, have consistent rounded corners to maintain a cohesive and modern visual style.
Based on the following Product Requirements Document (PRD) and User Prompt.
Product Requirements Document (PRD): ${storedPRD}
User's prompt: ${userPrompt}
Please incorporate the following images as specified:
${imageInsertionInstructions}
Please provide your output in HTML, CSS, and JavaScript without any explanations and natural languages(only code),with an emphasis on JavaScript for dynamic user interactions such as clicks and hovers.`;
      
\end{lstlisting}

\subsection{Prompt for code iteration idea}

\begin{lstlisting}

Based on the previously generated code, generate 3-5 ideas to improve the website design:
Previously Generated Code:
${previousCode}
Based on the previous design, please provide optimizations and enhancements focusing on:
1. Visual Consistency: Ensure a cohesive look and feel across the entire interface.
2. Unique Imagery: Suggest diverse and non-repetitive images that align with the theme of each section.
3. Component Refinement: Enhance the details of each UI component, considering:
- Button designs (hover states, shadows, etc.)
- Input field styles and interactions
- Card layouts and information hierarchy
4. Layout Improvements: Propose better ways to organize content for improved readability and user flow.
5. Color Scheme: Refine the color palette to improve contrast and visual appeal.
6. Typography: Suggest improvements in font choices, sizes, and text formatting for better readability.
7. Responsive Design: Ensure the layout adapts well to different screen sizes.
8. Interaction Design: Add subtle animations or transitions to improve user experience.
9. Accessibility: Suggest improvements to make the design more inclusive and easier to use for all users.
10. Performance Optimization: If applicable, propose ways to optimize the code for faster loading and rendering.
Please provide concise, innovative ideas that could enhance the user experience, visual appeal, or functionality of the website. Consider the existing code and suggest improvements or new features.`

\end{lstlisting}
\end{document}
