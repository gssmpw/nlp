\begin{table*}[htbp]
\scriptsize
\centering
\begin{tabular}{@{}p{\linewidth}@{}}
\toprule
\textbf{Task Instruction for Base Method in Noun Phrase Completion}\\
\midrule
Instructions:\\
1. You are given a head noun and emergent property. Your task is to generate a conceptual combination by adding one modifier.\\
2. You can use function word without any constraint.\\
3. Modifier should not have the given emergent property on its own, but the combination exhibits the emergent property.\\
4. Use the previous examples to learn the task.\\
5. Answer in dictionary format: \{\{"combination": "\{\{generated\_combination\}\}", "modifier": "\{\{generated\_modifier\}\}"\}\}. Do not include other formatting.\\
\\
<Example 1>\\
- Head noun: apple\\
- Emergent property: unappetizing\\
- Correct answer: \{\{"combination": "brown apple", "modifier": "brown"\}\}\\
\\
Above answer is correct because each component "brown" and "apple" do not possess "unappetizing" but "brown apple" does.\\
\\
<Example 2>\\
- Head noun: banknote\\
- Emergent property: useless\\
- Wrong answer: \{\{"combination": "burned banknote", "modifier": "burned"\}\}\\
\\
Above answer is wrong because modifier "burned" directly elicit property "useless". Avoid modifier which has given property in itself.\\
\\
\\
Then let's begin:\\
\bottomrule
\end{tabular}
    \caption{Task instruction for Base Method in \textsc{Noun Phrase Completion}.}
\label{tab:npc_prompt_base}
\end{table*}