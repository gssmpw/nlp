\begin{table*}[htbp]
\scriptsize
\centering
\begin{tabular}{@{}p{\linewidth}@{}}
\toprule
\textbf{Task Instruction for Spread Activation Method in Noun Phrase Completion}\\
\midrule
Instructions:\\
1. You are given a head noun, emergent property and a set of relevant concepts to solve a task. Your task is to generate a conceptual combination by adding one modifier.\\
2. You can use function word without any constraint.\\
3. Modifier should not have the given emergent property on its own, but the combination exhibits the emergent property.\\
4. Come up with your reasoning process before giving your final answer.\\
5. Final answer should follow dictionary format: \{\{"combination": "\{\{generated\_combination\}\}", "modifier": "\{\{generated\_modifier\}\}"\}\}. Do not use other formatting.\\
    \\
<Example 1>\\
- Head noun: apple\\
- Emergent property: unappetizing\\
- Relevant concepts: ['bland', 'bitter', 'inedible', 'insipid', 'unappealing', 'unpalatable']\\
- Correct answer: The goal is to find a modifier that does not inherently have the emergent property "unappetizing," but do when combined with "apple". Related concepts such as bitter, inedible or unpalatable make apple unappetizing. To represent bitter apple, "yellow" can be used as a modifier. But yellow is somewhat related to bitter because of the color of lemons. To represent inedible or unpalatable apple, "plastic" or "brown" can be used as a modifier. However plastic is directly related to inedible. "Brown" as a modifier doesn't imply inedible on its own, but when paired with "apple," it suggest an inedible state. So the answer is \{\{"combination": "brown apple", "modifier": "brown"\}\}\\
\\
Above answer is correct because each component "brown" and "apple" do not possess "unappetizing" but "brown apple" does.\\
\\
<Example 2>\\
- Head noun: banknote\\
- Emergent property: useless\\
- Relevant concepts: ['counterfeit', 'worthless', 'ineffective', 'futile', 'meaningless', 'pointless']\\
- Wrong answer: The goal is to find a modifier that does not inherently have the emergent property "useless," but do when combined with "banknote". Related concepts such as counterfeit or worthless make banknote useless. To represent counterfeit banknote, "fake" can be used as a modifier. But fake is somewhat related to useless because of the meaning. To represent worthless banknote, "burned" can be used as a modifier. So the answer is \{\{"combination": "burned banknote", "modifier": "burned"\}\}\\
\\
Above answer is wrong because modifier "burned" directly elicit property "useless".  Avoid modifier which elicit given property in itself.\\
\\
\\
Then let's begin:\\
\bottomrule
\end{tabular}
    \caption{Task instruction for Spread Activation Method in \textsc{Noun Phrase Completion}.}
    \label{tab:npc_prompt_sa}
\end{table*}