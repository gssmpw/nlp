\begin{table*}[htbp]
\scriptsize
\centering
\begin{tabular}{@{}p{\linewidth}@{}}
\toprule
\textbf{Task Instruction for Chain-of-Thought Method in Property Induction (Emergent)}\\
\midrule
Instructions:\\
1. You are given a combination of concepts. Your task is to generate emergent property of a combination.\\
2. Find a property that does not belong to any of the individual component in the combination but emerges when the words are combined.\\
3. Use the previous examples to learn how to do this.\\
4. Come up with your reasoning process before giving your final answer.\\
5. Answer in dictionary format: \{\{"property": "\{\{generated\_property\}\}"\}\}. Do not include other formatting.\\
    \\
<Example 1>\\
- Combination: Brown apple\\
- Correct answer: Let's think step-by-step. "Brown" signifies a color change due to aging, oxidation, or decay, indicating the apple is no longer fresh. "Apple" is a fruit that is typically appealing when fresh, but browning suggests overripeness or spoilage. Concepts like spectrum (color change) and growth (life cycle of the fruit) reinforce this transition. So the answer is \{\{"property": "unappetizing"\}\}\\
\\
Above answer is correct because property "unappetizing" does not belong to either "brown" and "apple", but belong to brown apple\\
\\
<Example 2>\\
- Combination: burned banknote\\
- Wrong answer: Let's think step-by-step. Individually, "burned" does not mean something is useless. A "banknote" by itself is valuable and serves as a medium of exchange. However, when combined, a "burned banknote" implies that the note is damaged beyond recognition, making it invalid for transactions and effectively useless as currency. So the answer is  \{\{"property": "useless"\}\}\\
\\
Above answer is wrong because "burned" something can directly belong to property "useless". Ensure that the emergent property you generate does not directly describe any of the individual words but is a characteristic of the combination as a whole.\\
\\
Then let's begin:\\
\bottomrule
\end{tabular}
    \caption{Task instruction for CoT Method in \textsc{Property Induction}.}
    \label{tab:pi_prompt_cot}
\end{table*}