\section{Related Work}
\subsection{Conceptual Combination}
Conceptual combination is a mental process of combining concepts. Unique features often arise from the combination, especially when the combinations of elements are unusual~\cite{estes2002emergence, ward2007creative}.
People come up with novel ideas to merge conflicts; thus, it is a prominent part of creativity~\cite{hampton1997emergent}.
In scientific discovery, conceptual combination creates new scientific concepts~\cite{thagard1984conceptual}, such as ``light wave.'' Even if nobody could observe light waves, the properties shared with waves, such as reflection, led to a theory that light is a wave, leading to much progress in science.

\subsection{Conceptual Combination in NLP} 
In NLP, understanding conceptual combination has been studied in two different lines. For instance, interpreting ``pet bird'' as ``a bird kept as a pet'' exemplifies \textit{noun compound interpretation}. Alternatively, imagining ``pet bird''s attributes like ``probably lives in a cage'' illustrates \textit{property verification}.

Noun Compound Interpretation focuses on generating plausible meanings for noun-noun compounds, such as interpreting ``dog house'' as ``a house for a dog''. The goal is to resolve ambiguity and provide clear interpretations of compound nouns. While related to combining nouns, it mainly emphasizes understanding the relationships between the words~\cite{hendrickx2013semeval,shwartz2018olive,shwartz2019still, coil2023chocolate, rambelli2024can}.

The second line of research focuses on the properties of conceptual combinations. \citet{srivastava2023bigbench} introduced a benchmark for identifying emergent properties in conceptual combinations. However, their approach has limitations: (1) it only identifies emergent properties, (2) the benchmark is limited to multiple-choice selection tasks without generative tasks, and (3) it does not involve creating conceptual combinations. In contrast, our approach (1) uses property types as part of the constraints and (2) includes both generative tasks and multiple-choice selection tasks. Concurrent work~\cite{ross2024artificial} deeply explores LLM's understanding of modifier adjective-noun conceptual combination with a membership inference question such as `Is a fake reef still a reef?'. The work shares interest with us about property cancellation. However, there's still a gap in evaluating LLMs for conceptual combination in generative way.