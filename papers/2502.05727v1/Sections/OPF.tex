 % The OPF problem is an optimization problem critical for power grid operation. This problem focuses on minimizing the cost of power grid operation by setting generator limits, while making sure to adhere to power grid physics, like transmission line and generator limits, and balancing supply and demand. In this work, we focus on the Direct Current (DC) version of the optimal power flow problem, as many methods are focused on application to DCOPF, and it is a fundamental problem. The formulation for the OPF problem is shown below.





 The OPF is a fundamental optimization problem in power grid operations, aimed at finding the most cost-effective generation profiles while adhering to the physical constraints of the power grid. This work specifically focuses on the simplified version of the OPF problem, i.e., DC-OPF, where its formulation is presented below.


\small
\begin{subequations}
\begin{gather}
    \texttt{min} \sum_{i \in \mathcal{G}} C_i(P_{\texttt{G}}^{i}) \label{min} \\
    \texttt{subject to:} 
    % Power balance
    \sum_{i \in \mathcal{G}_{k}} P_{\texttt{G}}^{i} - \sum_{j \in \mathcal{L}_{k}} P_{\texttt{D}}^{j} = \sum_{l \in \mathcal{B}_{k}} P_{l,k}, \quad \forall k \in \mathcal{N} \label{PB} \\
    % Line flow constraints
    P_{l,k} = \frac{\theta_{l} - \theta_{k}}{X_{l,k}}, \quad \forall (l, k) \in \mathcal{L} \label{LF} \\
    % Generation limits
    P_{\texttt{G}}^{\text{min},i} \leq P_{\texttt{G}}^{i} \leq P_{\texttt{G}}^{\text{max},i}, \quad \forall i \in \mathcal{G} \label{GL} \\
    % Phase angle limits
    \theta_{k}^{\text{min}} \leq \theta_{k} \leq \theta_{k}^{\text{max}}, \quad \forall k \in \mathcal{N} \label{PA} \\
    % Line capacity limits
    -P_{l,k}^{\text{max}} \leq P_{l,k} \leq P_{l,k}^{\text{max}}, \quad \forall (l, k) \in \mathcal{L} \label{LC}
\end{gather}
\end{subequations}
\normalsize


% \begin{equation}
% \min \sum_{i \in \mathcal{G}} C_i(P_{Gi}) \label{min}
% \end{equation}
% \text{subject to:} \\
% \text{Power balance:} \\ 
% \begin{equation}
% \sum_{i \in \mathcal{G}_k} P_{Gi} - \sum_{j \in \mathcal{L}_k} P_{Dj} = \sum_{l \in \mathcal{B}_k} P_{lk}, \forall k \in \mathcal{N} \quad\label{PB}
% \end{equation}
%  \text{Line flow constraints:} \\
% \begin{equation}
% P_{lk} = \frac{\theta_l - \theta_k}{X_{lk}}, \quad \forall (l, k) \in \mathcal{L}\label{LF}
% \end{equation}
% \text{Generation limits:} \\
% \begin{equation}
% P_{Gi}^{\text{min}} \leq P_{Gi} \leq P_{Gi}^{\text{max}}, \quad \forall i \in \mathcal{G}\label{GL}
% \end{equation}
% \text{Phase angle limits:} \\
% \begin{equation}
% \theta_k^{\text{min}} \leq \theta_k \leq \theta_k^{\text{max}}, \quad \forall k \in \mathcal{N}\label{PA}
% \end{equation}
% \text{Line capacity limits:} \\
% \begin{equation}
% -P_{lk}^{\text{max}} \leq P_{lk} \leq P_{lk}^{\text{max}}, \quad \forall (l, k) \in \mathcal{L}\label{LC}
% \end{equation}

% The OPF problem aims to minimize the total generation cost, expressed by the cost function \( \sum_{i \in \mathcal{G}} C_i(P_{Gi}) \), where \( C_i(P_{Gi}) \) represents the cost associated with generating power at generator \( i \). The active power generation at generator \( i \) is denoted by \( P_{Gi} \), and the set of all generators is given by \( \mathcal{G} \). The power balance constraint ensures that, for each bus \( k \), the total power generation from the generators connected to the bus, represented by the set \( \mathcal{G}_k \), minus the total power demand from the loads connected to the bus, represented by the set \( \mathcal{L}_k \), equals the net power flow on the lines connected to the bus, represented by \( \sum_{l \in \mathcal{B}_k} P_{lk} \), where \( \mathcal{B}_k \) is the set of lines connected to bus \( k \). The active power flow on line \( (l, k) \) is given by \( P_{lk} \), and it is defined by \( P_{lk} = \frac{\theta_l - \theta_k}{X_{lk}} \), where \( \theta_k \) and \( \theta_l \) are the voltage phase angles at buses \( k \) and \( l \), respectively, and \( X_{lk} \) is the reactance of the line. The generation limits constrain \( P_{Gi} \) to lie within its minimum and maximum bounds, \( P_{Gi}^{\text{min}} \) and \( P_{Gi}^{\text{max}} \), respectively. Similarly, phase angle limits are enforced by \( \theta_k^{\text{min}} \leq \theta_k \leq \theta_k^{\text{max}} \) for all buses. Finally, line capacity limits ensure that the power flow on line \( (l, k) \) stays within \( -P_{lk}^{\text{max}} \leq P_{lk} \leq P_{lk}^{\text{max}} \). The sets \( \mathcal{N} \) and \( \mathcal{L} \) represent the buses and loads in the network, respectively.

This problem aims to minimize the total generation cost, i.e., \( \sum_{i \in \mathcal{G}} C_i(P_{\texttt{G}}^{i}) \). In \eqref{min}, \( C_i(P_{\texttt{G}}^{i}) \) represents the generation cost of plant \( i \). The active power output of generator \( i \) is denoted by \( P_{\texttt{G}}^{i} \), and the set of all generators is given by \( \mathcal{G} \). The load at bus \( j \) is denoted by \( P_{\texttt{D}}^{j} \), and the set of loads is given by \( \mathcal{L} \).

The power balance constraint is captured by \eqref{PB} and ensures supply matches demand. Note, the power exchange of bus $i$ with neighboring buses is represented by \( \sum_{l \in \mathcal{B}_{k}} P_{l,k} \). The set of lines connected to bus \( k \) is shown by \( \mathcal{B}_{k} \).
%The power balance constraint ensures that, for each bus \( k \), the total power generation from the generators connected to the bus, represented by the set \( \mathcal{G}_{k} \), minus the total power demand from the loads connected to the bus, represented by the set \( \mathcal{L}_{k} \), equals the net power flow on the lines connected to the bus, represented by \( \sum_{l \in \mathcal{B}_{k}} P_{l,k} \), where \( \mathcal{B}_{k} \) is the set of lines connected to bus \( k \). 
%
The active power flow on line \( (l, k) \) is given by \( P_{l,k} \), and it is defined by \( P_{l,k} = \frac{\theta_{l} - \theta_{k}}{X_{l,k}} \), where \( \theta_{k} \) and \( \theta_{l} \) are the voltage phase angles at buses \( k \) and \( l \), respectively, and \( X_{l,k} \) is line's reactance.

Generators' output \( P_{\texttt{G}}^{i} \) is limited to lower and upper bounds, \( P_{\texttt{G}}^{\text{min},i} \) and \( P_{\texttt{G}}^{\text{max},i} \), respectively. Similarly, phase angle limits are enforced by (1e) for all buses. Finally, line capacity limits ensure that the power flow on line \( (l, k) \) stays within its limits.
%\( -P_{l,k}^{\text{max}} \leq P_{l,k} \leq P_{l,k}^{\text{max}} \).
%
The sets \( \mathcal{N} \) and \( \mathcal{L} \) represent the buses and loads in the network.

% The OPF problem, shown in \eqref{min}-\eqref{LC}, is typically resolved repeatedly every 15 minutes in accordance with the updated load demand ($P_D$). Instead of solving the optimization problem using an iterative solver, ML-based optimization proxies can be used to replace this repeated process. Optimization proxies are trained with historical data to develop a mapping between the load demand and the optimal generation profile. This mapping allows for faster, real-time decision-making.

% Existing OPF optimization proxies usually rely on MATPOWER
% data transferred from MATLAB to Python for training,
% creating a point of vulnerability where attacks could corrupt the data. In the subsequent section, we will present how data poisoning attacks are complete for three different optimization proxies, specifically, penalty method \cite{Liu2022Pen}, DC3 method \cite{donti2021dc3} (a post-repair method) and \LOOPLC~ \cite{Li_2023} method(a direct mapping method).