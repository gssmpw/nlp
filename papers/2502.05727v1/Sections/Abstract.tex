%As renewable energy penetration on the grid increases along with the number of extreme weather events, power grid operation requires more decisions to be made with prompt responses to external factors. 
The increased integration of clean yet stochastic energy resources and the growing number of extreme weather events are narrowing the decision-making window of power grid operators. This time constraint is fueling a plethora of research on Machine Learning (ML)-based optimization proxies. While finding a fast solution is appealing, the inherent vulnerabilities of the learning-based methods are hindering their adoption. One of these vulnerabilities is data poisoning attacks, which adds perturbations to ML training data, leading to incorrect decisions. The impact of poisoning attacks on learning-based power system optimizers have not been thoroughly studied, which creates a critical vulnerability. In this paper, we examine the impact of data poisoning attacks on ML-based optimization proxies that are used to solve the DC Optimal Power Flow problem. Specifically, we compare the resilience of three different methods—a penalty-based method, a post-repair approach, and a direct mapping approach—against the adverse effects of poisoning attacks. We will use the optimality and feasibility of these proxies as performance metrics. The insights of this work will establish a foundation for enhancing the resilience of neural power system optimizers.
%While poisoning attacks have been studied, there remains a gap in our understanding of them compared to other types of attacks on power systems. 
% The impact of poisoning attacks on learning-based power system optimizers has not been well studied which creates a critical vulnerability.
% In this paper, we explore the impact of data poisoning attacks on ML-based optimization proxies that are used to solve the DC Optimal Power Flow problem. To this end, we compare how the architecture of three different methods - a penalty-based method, projection-based approaches, and single-shot solutions - tolerate the adverse impacts of poisoning attacks on the optimality and feasibility of the optimization proxies. The resulting insight of this work will serve as the foundation of increasing the resilience of these methods.