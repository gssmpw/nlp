As discussed in Section \ref{FormPoisAttack}, to perturb the values, we add a perturbation amount, $\delta$, in the direction that maximizes loss. Once all the input points reach their bounded values, the error stays the same, regardless of the change in perturbation amount. This is seen in Fig. \ref{fig:all_methods},  where the error for both optimality and feasibility stops increasing once the perturbation added reaches the bound for the input points. Due to this, we focused on our choice of bound in our implementation and selected $\delta$ to reach the bound, as we wanted to demonstrate the extent of the damage that could be done by the attack. 

To select the bound, we considered the varying knowledge of multiple grid operators looking at the data. We assumed that the operators had a good intuition of the range of values to expect from the demand and generation. If values were set to zero or doubled, which would happen with a bound of 1, that would be apparent to operators. However, we wanted to show the extent of the attack's impact by suggesting a bound that was close enough to 1. The bound value of 0.75 was deemed reasonable, as the input values were formatted such that most were very small (much less than 1), and the bound of 0.75 was difficult to detect given our data set. 



% \begin{figure*}[b!]
% \begin{subfigure}{0.45\linewidth}
% \includegraphics[width=.95\linewidth]{penalty_opt_eps.png}
% \caption{Penalty (Optimality) Error}
% \end{subfigure}\hfill
% \begin{subfigure}{0.45\linewidth}
% \includegraphics[width=.95\linewidth]{penalty_feas_eps.png}
% \caption{Penalty (Feasibility) Error}
% \end{subfigure}\hfill
% \caption{Plots of the error of the Penalty method results for (a) optimality and (b) feasibility  versus perturbation amount, Delta, with a bound of 0.75. The values above the green feasibility line are infeasible. The red "x" represents the unperturbed case. The error reaches a max when the input values have reached the perturbation bound for both optimality and feasibility of the Penalty Method. All results other than the unperturbed ones are infeasible.}
% \label{penalty}
% \end{figure*}

% \begin{figure*}
% \begin{subfigure}{0.45\linewidth}
% \includegraphics[width=.95\linewidth]{LOOP_opt_eps.png}
% \caption{LOOP-LC (Optimality) Error}
% \end{subfigure}\hfill
% \begin{subfigure}{0.45\linewidth}
% \includegraphics[width=.95\linewidth]{LOOP_feas_eps.png}
% \caption{LOOP-LC (Feasibility) Error}
% \end{subfigure}\hfill
% \vspace{-.1cm}
% \caption{Plots of the error of the \LOOPLC method results for (a) optimality and (b) feasibility  versus perturbation amount, Delta, with a bound of 0.75. The values above the green feasibility line are infeasible. The red "x" represents the unperturbed case. The error reaches a max when the input values have reached the perturbation bound for both optimality and feasibility of the Penalty Method. All values, including the results of perturbed inputs, are feasible.}
% \label{LOOP}
% \vspace{-.4cm}
% \end{figure*}

% \begin{figure*}
% \begin{subfigure}{0.45\linewidth}
% \includegraphics[width=.95\linewidth]{DC3_opt_eps.png}
% \caption{DC3 (Optimality) Error}
% \end{subfigure}\hfill
% \begin{subfigure}{0.45\linewidth}
% \includegraphics[width=.95\linewidth]{DC3_feas_eps.png}
% \caption{DC3 (Feasibility) Error}
% \end{subfigure}\hfill
% \vspace{-.1cm}
% \caption{
% Plots of the error of the DC3 method results for (a) optimality and (b) feasibility  versus perturbation amount, Delta, with a bound of 0.75. The values above the green feasibility line are infeasible. The red "x" represents the unperturbed case. The error reaches a max when the input values have reached the perturbation bound for both optimality and feasibility of the Penalty Method. All results other than the unperturbed ones are infeasible.}
% \label{DC3}
% \vspace{-.4cm}
% \end{figure*}


% \begin{figure*}

%     % Part (a): Penalty
%     \begin{subfigure}{\columnwidth}
%         \includegraphics[width=\columnwidth]{penalty_opt_eps.png}
%         \caption{Penalty (Optimality)}
%         \centering
%     \end{subfigure}
    
%     \hfill
    
%     \begin{subfigure}{\columnwidth}
%         \includegraphics[width=\columnwidth]{penalty_feas_eps.png}
%         \caption{Penalty (Feasibility) Error}
%         \centering
%     \end{subfigure}

%     \hfill
    
%     % Part (b): LOOP
%     \begin{subfigure}{\columnwidth}
%         \includegraphics[width=\columnwidth]{LOOP_opt_eps.png}
%         \caption{LOOP (Optimality) Error}
%         \centering
%     \end{subfigure}

%     \newpage
    
%     % Part (c): DC3
%     \begin{subfigure}[b]{\columnwidth}
%         \includegraphics[width=\columnwidth]{DC3_opt_eps.png}
%         \caption{DC3 (Optimality) Error}
%         \centering
%     \end{subfigure}
    
%     \hfill
    
%     \begin{subfigure}[b]{\columnwidth}
%         \includegraphics[width=\columnwidth]{DC3_feas_eps.png}
%         \caption{DC3 (Feasibility) Error}
%         \centering
%     \end{subfigure}

%     \hfill
    
%     \caption{Demonstration of the error values reaching a maximum when the input values have reached the bound for both optimality and feasibility. (a) Penalty method, (b) LOOP method, (c) DC3 method.}
% \label{eps}
% \end{figure*}


