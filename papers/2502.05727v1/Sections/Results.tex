Table \ref{tab1} summarizes the results of the poisoning attack on the different methods. As presented in the table, the poisoning attack has a significant impact on the optimality value for all three methods. For the two methods in which feasibility is not guaranteed (i.e., the penalty and DC3 methods) there is also a considerable impact on the solution feasibility.

It is insightful to compare the architectural differences and how they affect the results. For instance, DC3 shows considerable feasibility violations, with an average violation of about 33 per output point. With more steps walking to the feasible region, this may improve, but the time would increase.  Attacking the penalty-based method has a slightly smaller impact on its feasibility, an average violation of about 30 per output point, which is to be expected due to the penalty term not being strictly enforced. These feasibility violations are problematic since if these methods were implemented in the grid operation, these violations could have potentially adverse consequences, such as causing an imbalance in supply and demand. The \LOOPLC~shows no feasibility violations, as it prioritizes feasibility over optimality. 

Interestingly, \LOOPLC~shows a strong performance in the optimality context. This is most likely due to the back propagation of the error in order to correct the optimality that takes place after applying the feasibility module. However, even the optimality gap of \LOOPLC~was three times worse when the attack was implemented. This directly affects what the cost of electricity generation would be if an attack were to happen to grid data.



Regardless of the structure, data poisoning attacks had a negative effect on the outputs of the optimization proxies. It would appear that the most resilient optimization proxies need hard enforcement of the feasibility, in addition to taking errors into consideration during the training process. It is likely that factoring in other methods of adversarial robustness in addition to this type of structure would help ensure resilience in the face of malicious actors who intend to harm grid operations.

For example, the penalty has a smaller impact on its feasibility than the DC3 method, but with more steps DC3, could have a feasible result. This is a trade-off that increases the time and has a risk of moving the output further away from the optimal, as is seen by the greater time taken and greater optimality error in DC3 than the penalty method.




% \textbf{Penalty \cite{Liu2022Pen}} & 0.01693 & 0.96144 & 0.09982 & 45.71349 & 0.06629 & 0.00044\\
% \hline
% \textbf{LOOP-LC \cite{Li_2023} } & 0.02651 & 0.10788 & 0 & 0 & 0.06668 & 0.07800\\
% \hline 
% \textbf{DC3 \cite{donti2021dc3}} & 0.01649 & 0.64685 &  0.09313 & 11.03552 & 0.07848 & 0.07800\\
% \hline 
% \multicolumn{7}{l}{$^{\mathrm{*}}$These values are based on a perturbation bound of 0.75*original value.} \\
% \end{tabular}
% \label{tab1}
% \end{center}
% \end{table*}

% \begin{table*}[htbp]
% \caption{Summary of Adversarial Attack Results*}
% \begin{center}
% \begin{tabular}{|c|>{\columncolor{green!20}}c|>{\columncolor{red!20}}c|>{\columncolor{green!20}}c|>{\columncolor{red!20}}c|>{\columncolor{green!20}}c|>{\columncolor{red!20}}c|}
% \hline
%  & \multicolumn{2}{|c|}{\textbf{Optimality Gap}} & \multicolumn{2}{|c|}{\textbf{Feasibility Gap}} & \multicolumn{2}{|c|}{\textbf{Time on CPU (sec)}} \\
% \hline 
% \textbf{Optimizer} & \textbf{Original} & \textbf{Attack} & \textbf{Original} & \textbf{Attack} & \textbf{Original} & \textbf{Attack} \\
% \hline 

% \textbf{Penalty \cite{Liu2022Pen}} & 0.01693 & 0.96144 & 0.09982 & 45.71349 & 0.06629 & 0.00044\\
% \hline
% \textbf{LOOP-LC \cite{Li_2023} } & 0.02651 & 0.10788 & 0 & 0 & 0.06668 & 0.07800\\
% \hline 
% \textbf{DC3 \cite{donti2021dc3}} & 0.01649 & 0.64685 &  0.09313 & 11.03552 & 0.07848 & 0.07800\\
% \hline 
% \multicolumn{7}{l}{$^{\mathrm{*}}$These values are based on a perturbation bound of 1*original value.} \\
% \end{tabular}
% \label{tab1}
% \end{center}
% \end{table*}

