

Today's power grid is unprepared to face its operational challenges ranging from an ever--increasing integration of renewables to an unprecedented number of extreme weather events. These factors increase the power system's variability, necessitating rapid dispatch decision-making. Put differently, these complexities can derail the balance of supply and demand and make solving power dispatch problems, like Optimal Power Flow (OPF), even more challenging.

Power grid operators often need to solve the OPF problem every 15 minutes and they rely on traditional solvers, that utilize time--consuming iterative algorithms like \cite{matpow2011, wachter2006interior}, to accomplish this task. %\cite{Zam2020Learn}.
The traditional iterative processes, while reliable, struggle to keep up with the rapid changes and variability introduced by renewable integration and extreme weather events. Moreover, the OPF problem's significance in grid operation makes it a prime target for cyber-intrusions, which often aim to disrupt the delicate balance between supply and demand \cite{Duan2018DataInteg, Rahman2014OPFAtt, Yang2017DataInteg}. As the traditional methods become inadequate under these evolving challenges, the potential of Machine Learning (ML) becomes increasingly apparent. ML-based optimization proxies utilize historical data in an offline training phase to develop models (i.e., neural networks (NN)) that achieve rapid online computation. This approach offers a much faster alternative to traditional solvers, significantly speeding up the decision-making process in grid operations.

% In grid operation, grid operators are required to resolve the OPF problem every 15 minutes with the aid of traditional solvers \cite{Zam2020Learn}. OPF is a very important problem for power grid operation, which makes it a common target for attacks\cite{Duan2018DataInteg, Rahman2014OPFAtt, Yang2017DataInteg}. A fundamental way to attack grid operation is creating an imbalance between supply and demand that traditional solvers are used to help balance. Traditional solvers, which utilize iterative algorithms, are inherently time-consuming. This iterative nature becomes limiting in real-time decision-making, especially with the increasing variability on the grid. Consequently, these traditional solvers may become inadequate for managing grid operations effectively.
% In this context, Machine Learning (ML) emerges as a crucial tool, offering the ability to make faster operational decisions. Moreover, ML-based optimization proxies can leverage extensive historical data to enhance their accuracy, a capability beyond the reach of traditional solvers.
% These advantages underscore the growing body of research focused on employing ML in power system optimization.

While ML-based optimization proxies present a promising tool for power system optimization, they introduce the risk of generating non-feasible solutions. Infeasibilities result in system failures and grid outages.  Traditional solvers, though slower, consistently provide optimal solutions that adhere to the physical constraints of the grid. To apply ML-based optimization proxies effectively, it is necessary to assess the risks (including poor performance) thoroughly. A significant concern is the vulnerability of optimization proxies to adversarial attacks \cite{szegedy2014intrig}, where minor perturbations to the input data can cause the neural network to generate decisions that deviate significantly from optimality and violate system constraints. For instance, optimization proxies often rely on data generated by traditional solvers (such as a MATPOWER solver\cite{matpow2011}) for training. Hence, any attack on the training data provided by these solvers would jeopardize the performance of the trained models. The result of the data corruption can be severe, including financial losses due to generators producing excess power, damage to transmission lines from unworkable network solutions, and even blackouts resulting from insufficient power production. Therefore, it is crucial to investigate the impacts of such attacks on the optimality and feasibility of decisions made by ML-based optimization proxies.

% could be inaccurate or infeasible predictions by the NN, which could lead to wasted money if the generators overproduce, blackouts if the generators under produce, and even transmission lines blowing if the network outputs an infeasible solution[]. It is important to explore the effect of attacks on optimality and feasibility of power systems 


%In this paper we will explore the vulnerability of power grid optimization methods to a specific type of adversarial attack, called data poisoning, where the training input data is perturbed [].






% These methods have shown great promise in using ML to generate optimal feasible solutions. One thing that these methods have in common, however, is that many of them use MATPOWER in order to generate training data for the Neural Networks(NN). This consistent use of MATPOWER leaves a vulnerability, in that if this data is corrupted, it will negatively affect these methods. The result of this corrupt data would be inaccurate or infeasible predictions by the network which could lead to wasted money if the generators overproduce, blackouts if the generators under produce, and even transmission lines blowing if the network outputs an infeasible solution.
% The popularity of machine learning has lead to a lot of research in the potential negative impacts of modifying training data with small perturbations, known as data poisoning attacks. They have been extensively studied as a threat to the security of NN's, however, are understudied in the applications of OPF solving ML networks. In this paper, we seek to explore the impact of data poisoning attacks on optimization methods using ML with an emphasis on obeying feasibility. 





% One way ML is being used is in solving the Optimal Power Flow problem. The Optimal Power Flow (OPF) problem is an optimization problem critical for power grid operation. This problem focuses on minimizing the cost of power grid operation by setting generator limits, while making sure to adhere to power grid physics, like transmission line and generator limits, and balancing supply and demand[]. The OPF problem needs to be re-solved frequently and quickly throughout the day, typically every 15 minutes, to keep up with changing load and generation values []. This makes it a very useful place for the application of machine learning which is able to solve problems like this quickly. However, one of the challenges with ML is difficulty enforcing physical system constraints in the problem solution for example, transmission line limits in OPF. 
