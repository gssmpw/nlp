%  This method removed the iterations involved in optimization problems. The general idea is that they are considering minimizing a function $f (x, u)$, where
% $x$ is the set of inputs to the distribution, and $u$ is the set of
% parameters while obeying constraints. There are two ways in
% which approximating this is completed. The first way is by using a solver to generate $u$ and use it as the ground truth, then
% treating is as a supervised learning problem to train the neural
% network to output the optimal $v$, then mapping the solution to the feasible region to output $u$. For the version without the
% solver the method tries to minimize the expected objective
% value over the set of input distributions, then map to the feasible region. The authors achieve
% the mapping to the feasible region by putting the problems in terms of independent variables,
% moving the feasible region to be centered around the origin by
% moving by an interior point, mapping the feasible region to the
% l-infinite norm unit ball, having a neural approximator find a
% solution on the l-infinite norm unit ball, then using a gauge
% map to map that solution to the feasible region.

% In this paper we focus on the method using the solver to generate a ground truth. In \LOOPLC~, feasibility of the solution is guaranteed. This means that the poisoning attack will take advantage of this and shift the outputted solution further away from the true optimal. We will perturb the data being used to train the NN, labeled as the "Optimization module" in Fig. \ref{loopfig}.

The \LOOPLC~method employs a neural approximator to transform input data directly into a near-optimal feasible solution. This approach ensures hard feasibility without relying on penalty terms and operates without iterations, utilizing a straightforward feed-forward process. The technique begins by training a neural network to map inputs to a virtual optimal point within the $\ell_\infty$-norm unit ball. Following this, a feasibility module is used to project this virtual prediction onto actual feasible solutions.

The method implements inequalities by using a gauge mapping technique to project the virtual optimal point from the \(\ell_\infty\)-norm unit ball onto the feasible space. For equality constraints, it determines the values of dependent variables based on the independent variables, effectively reconstructing the solution to the original problem. This integrated approach allows the \LOOPLC~method to achieve reliable and efficient optimization outcomes, free from the iterative adjustments typical of other methods.

 In the \LOOPLC~method, solution feasibility is guaranteed. Consequently, our poisoning attack strategy focuses on undermining the optimality of the solutions. We aim to shift the neural network's outputted solution further from the true optimal. This targeted manipulation occurs specifically within the ``Optimization module," as illustrated in Fig. \ref{loopfig}.

\begin{figure}[h!]
\centerline{\includegraphics[scale=0.067]{LOOP_Attack.png}}
\vspace{-.2cm}
\caption{Poisoning the workflow of the \LOOPLC~method.}
\label{loopfig}
\end{figure}



\captionsetup{skip=5pt} % Adjust caption spacing to make it closer to each subfigure

\begin{figure*}[h!]
\centering
\begin{subfigure}{0.45\linewidth}
    \includegraphics[width=\linewidth]{penalty_opt.png}
    \caption{Penalty (Optimality) Error}
\end{subfigure}\hfill
\begin{subfigure}{0.45\linewidth}
    \includegraphics[width=\linewidth]{penalty_feas.png}
    \caption{Penalty (Feasibility) Error}
\end{subfigure}

\vspace{10pt} % Add space between rows

\begin{subfigure}{0.45\linewidth}
    \includegraphics[width=\linewidth]{LOOP_opt.png}
    \caption{\LOOPLC~(Optimality) Error}
\end{subfigure}\hfill
\begin{subfigure}{0.45\linewidth}
    \includegraphics[width=\linewidth]{LOOP_feas.png}
    \caption{\LOOPLC~(Feasibility) Error}
\end{subfigure}

\vspace{10pt} % Add space between rows

\begin{subfigure}{0.45\linewidth}
    \includegraphics[width=\linewidth]{DC3_opt.png}
    \caption{DC3 (Optimality) Error}
\end{subfigure}\hfill
\begin{subfigure}{0.45\linewidth}
    \includegraphics[width=\linewidth]{DC3_feas.png}
    \caption{DC3 (Feasibility) Error}
\end{subfigure}

\caption{Plots of the error for each method (Penalty, \LOOPLC, and DC3) for both (left) optimality and (right) feasibility, versus perturbation amount, with a bound of 0.75. The values above the green feasibility line are infeasible. The red ``x" represents the unperturbed case. In the Penalty and DC3 methods, all results other than the unperturbed case are infeasible, while \LOOPLC~ maintains feasibility across all cases.}
\label{fig:all_methods}
\vspace{-0.3cm}
\end{figure*}


\begin{table*}[h!]
\caption{Summary of Adversarial Attack Results (Used on a perturbation bound of 0.75 $\times$ original value.)}
\vspace{-.2cm}
\begin{center}
\begin{tabular}{|c|>{\columncolor{green!20}}c|>{\columncolor{red!20}}c|>{\columncolor{green!20}}c|>{\columncolor{red!20}}c|>{\columncolor{green!20}}c|>{\columncolor{red!20}}c|}
\hline
 & \multicolumn{2}{c|}{\textbf{Optimality Gap}} & \multicolumn{2}{c|}{\textbf{Feasibility Gap}} & \multicolumn{2}{c|}{\textbf{Time on CPU (sec)}} \\

\hline 
\textbf{Optimizer} & \textbf{Unattacked} & \textbf{Attacked} & \textbf{Unattacked} & \textbf{Attacked} & \textbf{Unattacked} & \textbf{Attacked} \\
\hline 
\textbf{Penalty \cite{Liu2022Pen}} & 0.11412 & 2.20422 & 0.06166 & 29.95326 & 0.00017 & 0.00027\\
\hline
\textbf{LOOP-LC \cite{Li_2023} } & 0.03786 & 0.10964 & 0 & 0 & 0.06986 & 0.06801\\
\hline 
\textbf{DC3 \cite{donti2021dc3}} & 0.01654 & 2.77317 & 0.10314 & 33.25656 & 0.08053 & 0.10336\\
\hline 
%\multicolumn{7}{l}{$^{\mathrm{*}}$These values are based on a perturbation bound of 0.75*original value.} \\
\end{tabular}
\label{tab1}
\end{center}
\end{table*}

