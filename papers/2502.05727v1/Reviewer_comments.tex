\begin{document}
----------------------- REVIEW 1 ---------------------
SUBMISSION: 69
TITLE: Impact of Data Poisoning Attacks on Feasibility and Optimality of Neural Power System Optimizers
AUTHORS: Nora Agah, Meiyi Li and Javad Mohammadi

----------- Overall evaluation -----------
SCORE: 1 (weak accept)
----- TEXT:
The paper is in the scope of a practical field, however, the following considerations made:

[Presentation and Writing]

1- The introduction should be systematic which means context to the field, importance of the field, a mini literature review of the field, then gap, and finally, your contributions, and paper structure.
2- Figures should be in Times New Roman font, and be inserted in vector (SVG) to the paper, avoiding lack of quality.
3- The equations 1a to 1f should be 1.1 to 1.6
4- There are undefined abbreviations. For example, in the abstract, you have used Machine Learning, and ML. So you have not defined ML as Machine Learning (ML). Recheck for all undefined ones.
5- The functionality of the clip function is not ordered. It should be equation (4).
6- Instead of using <we> pronoun and active voice sentences, use passive voice sentences. It is the professional way.

[Technical]

1- Considering the computation complexity (CC), just time is considered in Table 1. The authors should consider the other CC important factors. Then, re-conduct the comparison.
2- What the authors did, in this paper, is mainly about data-driven simulations. So, it cannot be considered as experimental setup. Consequently, <Simulation> is a better candidate instead of <Experimental>. Regarding this, update the words.
3- From 21 references, just 3 of them are 2024. the authors should use more recent works instead.
4- The upper and lower boundaries numerical values of the terms are not presented in the paper. Should be clarified.



----------------------- REVIEW 2 ---------------------
SUBMISSION: 69
TITLE: Impact of Data Poisoning Attacks on Feasibility and Optimality of Neural Power System Optimizers
AUTHORS: Nora Agah, Meiyi Li and Javad Mohammadi

----------- Overall evaluation -----------
SCORE: 2 (accept)
----- TEXT:
The paper focuses on a critical yet under-researched vulnerability in ML-based power system optimizers-data poisoning attacks. It is a highly relevant study considering the increasing reliance on ML for real-time decision-making in power grids.

The comparison of the three approaches—Penalty-based, DC3, and LOOP-LC—is clear and systematic. The focus on both feasibility and optimality metrics strengthens the rigor of the study.

The experimental results are presented in detail with clear visualizations, such as that in Fig. 4, and a concise summary in Table I. The metrics such as optimality gap, feasibility gap, and CPU time effectively provide a holistic view of the performance.

These include extension proposals for the development of poisoning attacks for AC-OPF and the incorporation of adversarial defenses.

Include a clearer motivation for choosing data poisoning over evasion attacks, emphasizing why this threat is particularly critical for grid optimization compared to operational phase attacks.

Add a few recent works (2023–2024) on adversarial robustness in ML-based power systems to highlight state-of-the-art methods that deal with vulnerabilities.

Discuss mitigation techniques like adversarial training or robust optimization briefly to contextualize the importance of this study further.

Improve the labels and legends of the graphs in Fig. 4 for readability purposes. For example, placing units where appropriate and bolding titles to the axes will enhance this. Ensure that legends appear the same way in all plots of Fig. 4 for consistency.

State your reason for choosing a perturbation bound of 0.75 instead of any other. Present a short justification.

Make practical implications more explicit, as in how grid operators might apply this work toward making their system more resilient against adversarial attacks.

Use uniform citation style for all references. For instance, the DOIs must be formatted properly.


\end{document}