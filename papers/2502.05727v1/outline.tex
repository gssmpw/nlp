Motiviation
1. Introduce grid. Grid changes-more variables (renewables), more extreme weather events, more decisions to be made - grid is facing these challenges and it is difficult to balance supply and demand and power system operations, for example in OPF?
2.Traditional iterative solvers are slow. Need to resolve every 15 mins. We have a lot of historical data that can be used to train. AI solvers is needed in the grid. 
3. Operators are hesitant also to use AI methods because of uncertainty of the risk. The risks that come with AI solvers, important to explore things like poisoning to understand and improve robustness

Literature Review
1. Existing AI optimization proxies
2. Poisoning literature review in application to power systems 
3. Emphasize gap in poisoning of power systems methods

Contribution
1. Determining the impact of poisoning attacks on new power grid AI solvers
2. Giving guide to increase robustness of future solvers and ways to modify current solvers to be more robust


Problem Formulation
1. DCOPF
This is about the formultion. Define the mathematical formulation. 
2. Explain how AI Solvers for DCOPF work.
3. Poisoning explanation and formulation

AI Method Overview
1. Penalty
2. LOOP-LC
3. DC3

Results
Explain experiment setup-like dataset 
Give the metrics-how we calc feasibility and optimality gap

Provide tables and figures-give one showing the trenf of eps and error?

Conclusion
1.Our observations from the results. There is an impact! This can cause many problems 
2. Future directions - ACOPF, training or modifying for robustness, more methods


