\documentclass[conference]{IEEEtran}
\IEEEoverridecommandlockouts
% The preceding line is only needed to identify funding in the first footnote. If that is unneeded, please comment it out.
\usepackage{cite}
\usepackage{amsmath,amssymb,amsfonts}
\usepackage{colortbl}
\usepackage{xcolor}
\usepackage{algorithmic}
\usepackage{graphicx}
\usepackage{textcomp}
\usepackage{xcolor}
\usepackage{caption}
\usepackage{subcaption}

\def\BibTeX{{\rm B\kern-.05em{\sc i\kern-.025em b}\kern-.08em
    T\kern-.1667em\lower.7ex\hbox{E}\kern-.125emX}}

\makeatletter
\def\ps@IEEEtitlepagestyle{%
\def\@oddfoot{\mycopyrightnotice}%
\def\@evenfoot{}%
}

\def\mycopyrightnotice{%
{\footnotesize 979-8-3315-4112-5/25/\$31.00~\copyright~2025 IEEE\hfill} % Revise this line accordingly!
\gdef\mycopyrightnotice{}
}

\begin{document}


\newcommand{\LOOPLC}{$\mathcal{LOOP-LC}$}


\title{ 
%Threat of Data Poisoning: Impact on Feasibility and Optimality in Machine Learning-based Optimization Proxies\\ 
%Data Poisoning of Data-Driven Optimization Methods Applied to  Optimal Power Flow\\
Impact of Data Poisoning Attacks on Feasibility and Optimality of Neural Power System Optimizers

\thanks{This material is based upon work supported by the National Science Foundation (NSF) Graduate Research Fellowship under Grant No. DGE 2137420. Any opinion, findings, and conclusions or recommendations expressed in this material are those of the authors(s) and do not necessarily reflect NSF views.}
}

\author{\IEEEauthorblockN{Nora Agah}
\IEEEauthorblockA{
\textit{The University of Texas at Austin}\\
Austin, Texas, USA \\
Norakagah@utexas.edu}
\and
\IEEEauthorblockN{Meiyi Li}
\IEEEauthorblockA{
\textit{The University of Texas at Austin}\\
Austin, Texas, USA \\
 meiyil@utexas.edu}
\and
\IEEEauthorblockN{Javad Mohammadi}
\IEEEauthorblockA{\textit{The University of Texas at Austin}\\
Austin, Texas, USA \\
javadm@utexas.edu}
}

\maketitle

\begin{abstract} \begin{abstract}

To develop generalizable models in multi-agent reinforcement learning, recent approaches have been devoted to discovering task-independent skills for each agent, which generalize across tasks and facilitate agents' cooperation. However, particularly in partially observed settings, such approaches struggle with sample efficiency and generalization capabilities due to two primary challenges: (a) How to incorporate global states into coordinating the skills of different agents? (b) How to learn generalizable and consistent skill semantics when each agent only receives partial observations? To address these challenges, we propose a framework called \textbf{M}asked \textbf{A}utoencoders for \textbf{M}ulti-\textbf{A}gent \textbf{R}einforcement \textbf{L}earning (MA2RL), which encourages agents to infer unobserved entities by reconstructing entity-states from the entity perspective. The entity perspective helps MA2RL generalize to diverse tasks with varying agent numbers and action spaces. Specifically, we treat local entity-observations as masked contexts of the global entity-states, and MA2RL can infer the latent representation of dynamically masked entities, facilitating the assignment of task-independent skills and the learning of skill semantics. Extensive experiments demonstrate that MA2RL achieves significant improvements relative to state-of-the-art approaches, demonstrating extraordinary performance, remarkable zero-shot generalization capabilities and advantageous transferability.

 % Additional rewards transform the original MTRL problem into a multi-objective MTRL problem, and the coupling relationship between the outputs of SP and ACP further complicates the optimization process. To solve this challenge, TSAC assigns a virtual expected budget to convert the multi-objective MTRL into a constrained single-objective formulation and then employs the Lagrangian method to transform a constrained single-objective optimization into an unconstrained one. The multiplier in the Lagrangian method automatically adjusts the weights during the training process, promoting cooperation between SP and ACP.
\end{abstract}
\begin{IEEEImpStatement}
The Current policies trained by Multi-Agent Reinforcement Learning (MARL) predominantly rely on meticulously designed structured environments, which considerably constrain the agents' generalization capabilities across multitasking and cross-task skill reuse. In this paper, we design a novel masked autoencoders for MARL to coordinate the skills of different agents and learn generalizable and consistent skill semantics when each agent only receives partial observations. Experimental results demonstrate that our proposed MA2RL framework significantly enhances both the asymptotic performance and generalization capabilities of the generalizable models. Specifically, MA2RL introduces masked autoencoders tailored for MARL, aimed at enhancing generalizable models. The framework holds promise for inspiring further explorations into the generalization of multi-agent reinforcement learning.
\end{IEEEImpStatement}


% Note that keywords are not normally used for peerreview papers.
\begin{IEEEkeywords}
Multi-Agent reinforcement learning, generalization, self-supervised learning.
\end{IEEEkeywords}


\IEEEpeerreviewmaketitle

\end{abstract}

\begin{IEEEkeywords}
Poisoning Attack, Machine Learning, Optimal Power Flow
\end{IEEEkeywords}

\vspace{-.2cm}
\section{Introduction}
\vspace{-.2cm}
\subsection{Motivation} 

Today's power grid is unprepared to face its operational challenges ranging from an ever--increasing integration of renewables to an unprecedented number of extreme weather events. These factors increase the power system's variability, necessitating rapid dispatch decision-making. Put differently, these complexities can derail the balance of supply and demand and make solving power dispatch problems, like Optimal Power Flow (OPF), even more challenging.

Power grid operators often need to solve the OPF problem every 15 minutes and they rely on traditional solvers, that utilize time--consuming iterative algorithms like \cite{matpow2011, wachter2006interior}, to accomplish this task. %\cite{Zam2020Learn}.
The traditional iterative processes, while reliable, struggle to keep up with the rapid changes and variability introduced by renewable integration and extreme weather events. Moreover, the OPF problem's significance in grid operation makes it a prime target for cyber-intrusions, which often aim to disrupt the delicate balance between supply and demand \cite{Duan2018DataInteg, Rahman2014OPFAtt, Yang2017DataInteg}. As the traditional methods become inadequate under these evolving challenges, the potential of Machine Learning (ML) becomes increasingly apparent. ML-based optimization proxies utilize historical data in an offline training phase to develop models (i.e., neural networks (NN)) that achieve rapid online computation. This approach offers a much faster alternative to traditional solvers, significantly speeding up the decision-making process in grid operations.

% In grid operation, grid operators are required to resolve the OPF problem every 15 minutes with the aid of traditional solvers \cite{Zam2020Learn}. OPF is a very important problem for power grid operation, which makes it a common target for attacks\cite{Duan2018DataInteg, Rahman2014OPFAtt, Yang2017DataInteg}. A fundamental way to attack grid operation is creating an imbalance between supply and demand that traditional solvers are used to help balance. Traditional solvers, which utilize iterative algorithms, are inherently time-consuming. This iterative nature becomes limiting in real-time decision-making, especially with the increasing variability on the grid. Consequently, these traditional solvers may become inadequate for managing grid operations effectively.
% In this context, Machine Learning (ML) emerges as a crucial tool, offering the ability to make faster operational decisions. Moreover, ML-based optimization proxies can leverage extensive historical data to enhance their accuracy, a capability beyond the reach of traditional solvers.
% These advantages underscore the growing body of research focused on employing ML in power system optimization.

While ML-based optimization proxies present a promising tool for power system optimization, they introduce the risk of generating non-feasible solutions. Infeasibilities result in system failures and grid outages.  Traditional solvers, though slower, consistently provide optimal solutions that adhere to the physical constraints of the grid. To apply ML-based optimization proxies effectively, it is necessary to assess the risks (including poor performance) thoroughly. A significant concern is the vulnerability of optimization proxies to adversarial attacks \cite{szegedy2014intrig}, where minor perturbations to the input data can cause the neural network to generate decisions that deviate significantly from optimality and violate system constraints. For instance, optimization proxies often rely on data generated by traditional solvers (such as a MATPOWER solver\cite{matpow2011}) for training. Hence, any attack on the training data provided by these solvers would jeopardize the performance of the trained models. The result of the data corruption can be severe, including financial losses due to generators producing excess power, damage to transmission lines from unworkable network solutions, and even blackouts resulting from insufficient power production. Therefore, it is crucial to investigate the impacts of such attacks on the optimality and feasibility of decisions made by ML-based optimization proxies.

% could be inaccurate or infeasible predictions by the NN, which could lead to wasted money if the generators overproduce, blackouts if the generators under produce, and even transmission lines blowing if the network outputs an infeasible solution[]. It is important to explore the effect of attacks on optimality and feasibility of power systems 


%In this paper we will explore the vulnerability of power grid optimization methods to a specific type of adversarial attack, called data poisoning, where the training input data is perturbed [].






% These methods have shown great promise in using ML to generate optimal feasible solutions. One thing that these methods have in common, however, is that many of them use MATPOWER in order to generate training data for the Neural Networks(NN). This consistent use of MATPOWER leaves a vulnerability, in that if this data is corrupted, it will negatively affect these methods. The result of this corrupt data would be inaccurate or infeasible predictions by the network which could lead to wasted money if the generators overproduce, blackouts if the generators under produce, and even transmission lines blowing if the network outputs an infeasible solution.
% The popularity of machine learning has lead to a lot of research in the potential negative impacts of modifying training data with small perturbations, known as data poisoning attacks. They have been extensively studied as a threat to the security of NN's, however, are understudied in the applications of OPF solving ML networks. In this paper, we seek to explore the impact of data poisoning attacks on optimization methods using ML with an emphasis on obeying feasibility. 





% One way ML is being used is in solving the Optimal Power Flow problem. The Optimal Power Flow (OPF) problem is an optimization problem critical for power grid operation. This problem focuses on minimizing the cost of power grid operation by setting generator limits, while making sure to adhere to power grid physics, like transmission line and generator limits, and balancing supply and demand[]. The OPF problem needs to be re-solved frequently and quickly throughout the day, typically every 15 minutes, to keep up with changing load and generation values []. This makes it a very useful place for the application of machine learning which is able to solve problems like this quickly. However, one of the challenges with ML is difficulty enforcing physical system constraints in the problem solution for example, transmission line limits in OPF. 

\subsection{Literature Review} 

Extensive research has been devoted to developing optimization proxies that expedite the resolution of the OPF problem while ensuring the feasibility of decisions. A commonly employed method is the incorporation of penalty terms aimed at minimizing constraint violations \cite{Liu2022Pen, pan2022deepopf, liu2024teaching}. Although this approach is applicable across a variety of power system optimization problems, it merely penalizes rather than eliminating violations. Alternatively, some methods introduce a repair module that utilizes iterative algorithms to adjust the solutions. These methods range from using commercial solvers to identify the closest feasible solution \cite{zhao2020deepopf}, to optimizing a function that minimizes violations \cite{donti2021dc3}.  Further advancements involve differentiable mapping functions that directly enforce constraint satisfaction \cite{ Li_2023, chen2023end}. For instance, in our previous work \cite{li2023learning}, we introduced an iteration-free optimization proxy that ensures the hard feasibility of solutions through a single feed-forward process. However, research into assessing these methods' performance, such as optimality and feasibility, particularly under adversarial conditions, remains sparse.


% For power systems optimization proxies, a focus on enforcing feasibility constraints is even more pervasive in the literature \cite{Liu2022Pen, donti2021dc3, Li_2023, Gao2024Physics}. This includes adding a penalty in the ML laws functions for violating these constraints or mixing together ML with traditional mathematical solvers. These methods have shown great promise in using ML to generate optimal feasible solutions. However, there is little work done in showing these methods' resilience to adversarial attacks.

% Adversarial attacks, which involve introducing small perturbations to input data to maximize disruption in the output of optimization proxies, specifically within the power system context. Adversarial attacks are categorized into two main types: evasion and poisoning attacks.  Evasion attacks occur during the run/test time of the neural network, where the perturbation is added to input data, compromising the output by fitting the wrong network to the data. These attacks have been extensively studied in their application to the power system. In contrast, poisoning occurs during the training phase, where perturbations are added to the training data, thereby affecting the neural networks' weights. Although less studied, due in part to the assumption that training data is more secure than data used in runtime\cite{agah2024datapois}, they represent a significant threat. Evasion attacks only affect one round of outputs, the outputs coming directly from inputting the perturbed data. Poisoning attacks skew the network structure so that, until there is retraining, every round of outputs are affected. This is a very large impact, and needs to be studied. There are vulnerabilities in the power system that can lead to poisoning attacks. Notably, many OPF optimization proxies rely on MATPOWER data transferred from MATLAB to Python for training \cite{zhao2020deepopf, donti2021dc3, Li_2023}, creating a point of vulnerability where attacks could corrupt the data. Therefore, investigating data poisoning is vital for evaluating the reliability of ML-based optimization proxies.

Adversarial attacks in the power system context typically involve introducing limited perturbations to input data, aiming to maximize disruption in the output of optimization proxies. These attacks are categorized into two primary types: evasion and poisoning attacks. Evasion attacks occur during the operational or test phase of the NN, where perturbations in the input data compromise the output by misleading the neural network. Such attacks have been extensively studied within power systems \cite{Wang2024Evasion}. In contrast, poisoning attacks happen during the training phase, where perturbations to the training data alter the NN's weights, potentially causing long-term disruptions. Poisoning attacks are less explored, partly due to the assumption that training data is more secure than operational data \cite{agah2024datapois}. However, poisoning attacks, unlike evasion attacks that affect only a single output session, can distort the structure of the neural network until retraining occurs, thus impacting all subsequent outputs significantly \cite{abbasi2023brainwash}.

The extensive impact of data poisoning attacks has also led to research on a variety of methods to strengthen the robustness of NNs against these attacks \cite{CompSurvey}. This research includes methods for detecting poisoned data \cite{DetectPois}, or training the network to be more resilient to poisoned data \cite{inbook}. Understanding data poisoning attacks in relation to the power system will allow these robustness strategies to be applied effectively, helping strengthen security of ML methods used in grid operation.

In this paper, we focus on today's common training practices and shed light on their data poisoning vulnerabilities.
Specifically given that many OPF optimization proxies utilize MATPOWER data transferred from MATLAB to Python for training \cite{zhao2020deepopf, donti2021dc3, Li_2023}, there exists a notable intrusion vulnerability. 
%to poisoning at this data transfer point. 
Thus, a thorough study of data poisoning is crucial for assessing the reliability of ML-based optimization proxies.





% {\color{blue} the flow will be easier to follow if: 1. what is Adversarial attacks(your sentence"Adversarial attacks involve adding a small perturbation to input data in order to maximize the loss of the neural network"), 2. Adversarial attacks categories(your sentence "Adversarial attacks fall under two main categories: Evasion attacks and Poisoning attacks"). 3. explain Evasion attacks(your sentence "Evasion attacks are when that perturbation is added to the input data during the run/test time of the NN. This affects the output by exploiting the fact that networks are not perfectly fitted to the data." "Evasion attacks are heavily studied in their application to the power system", 4. explain Poisoning attacks(its difference from  Evasion attacks, its features: your sentence "assumption that a networks data is more secure during training then during runtime \cite{agah2024datapois}.", its importance and research gap: your sentence"Poisoning on the other hand has been studied,", "many of the OPF optimization proxies use MATPOWER data for training, then send that data to matlab from python creating a vulnerability point for attacks to exploit and corrupt the MATPOWER data. Therefore, data poisoning is an important attack method to explore in relation to power grid ML based optimizer reliability.")}

% Adversarial attacks fall under two main categories: Evasion attacks and Poisoning attacks[]. Adversarial attacks involve adding a small perturbation to input data in order to maximize the loss of the neural network. Evasion attacks are when that perturbation is added to the input data during the run/test time of the NN. This affects the output by exploiting the fact that networks are not perfectly fitted to the data. Poisoning attacks are when the perturbation is added to the input data when the network is training affecting the weights of the network during the training process. Evasion attacks are heavily studied in their application to the power system[]. Poisoning on the other hand has been studied, but slightly less so in part due to the assumption that a networks data is more secure during training then during runtime \cite{agah2024datapois}. However, many of the OPF optimization proxies use MATPOWER data for training, then send that data to matlab from python creating a vulnerability point for attacks to exploit and corrupt the MATPOWER data. Therefore, data poisoning is an important attack method to explore in relation to power grid ML based optimizer reliability.


\subsection{Contribution} In this paper, we investigate the impact of data poisoning attacks on the optimality and feasibility of ML-based optimization proxies applied to solving the DC-OPF problem.
%, an area that is not sufficiently explored. 
We conduct tests on three distinct types of ML-based optimization proxies: the penalty method \cite{Liu2022Pen}, the DC3 method \cite{Li_2023} (a post-repair approach), and the \LOOPLC~ method \cite{donti2021dc3} (a direct mapping approach). Our research aims to develop more robust ML methods that can withstand adversarial threats, thereby enhancing both the reliability and security of power system operations. In addition, our findings provide valuable insights for power system operators and can facilitate the adoption of neural-based optimizers. This paper is organized as follows: Section \ref{PF} describes the problem formulation for the DC-OPF problem, Section \ref{PA} details our implementation of the data poisoning attack, Section \ref{Res} discusses the results of the attack simulation, and finally Section \ref{conc} provides our conclusion and potential extensions of this work.


% In this paper, we analyze the impact of data poisoning attacks on ML-based optimization proxies in optimality and feasibility applied to ACOPF, which we feel is currently understudied. We test the data poisoning attack implementation on three types of ML-based optimization proxies: penalty method \cite{Liu2022Pen}, DC3 method \cite{Li_2023} (a post-repair method) and \LOOPLC~ method \cite{donti2021dc3} (a direct mapping method). We aim to provide potential solutions for more robust ML methods and safeguard these ML models against adversarial threats, ultimately enhancing the reliability and security of power system operations.
% This paper is organized as follows: Setion II will discuss OPF and the ML-based solvers we used, Section III will give our data poisoning attack implementation method, Section IV will give our experiment set up and results, and Section V will give our concluding thoughts and our suggestions for future work such as work to improve robustness.



% There has been difficulty with ML enforcing the feasibility constraints. This includes adding a penalty in the ML laws functions for violating these constraints or mixing together ML with traditional mathematical solvers. These methods have shown great promise in using ML to generate optimal feasible solutions. One thing that these methods have in common, however, is that many of them use MATPOWER in order to generate training data for the Neural Networks(NN). This consistent use of MATPOWER leaves a vulnerability, in that if this data is corrupted, it will negatively affect these methods. The result of this corrupt data would be inaccurate or infeasible predictions by the network which could lead to wasted money if the generators overproduce, blackouts if the generators under produce, and even transmission lines blowing if the network outputs an infeasible solution.
% The popularity of machine learning has lead to a lot of research in the potential negative impacts of modifying training data with small perturbations, known as data poisoning attacks. They have been extensively studied as a threat to the security of NN's, however, are understudied in the applications of OPF solving ML networks. In this paper, we seek to explore the impact of data poisoning attacks on optimization methods using ML with an emphasis on obeying feasibility. 
  

\section{Problem formulation}\label{PF}
\subsection{Optimal Power Flow}  % The OPF problem is an optimization problem critical for power grid operation. This problem focuses on minimizing the cost of power grid operation by setting generator limits, while making sure to adhere to power grid physics, like transmission line and generator limits, and balancing supply and demand. In this work, we focus on the Direct Current (DC) version of the optimal power flow problem, as many methods are focused on application to DCOPF, and it is a fundamental problem. The formulation for the OPF problem is shown below.





 The OPF is a fundamental optimization problem in power grid operations, aimed at finding the most cost-effective generation profiles while adhering to the physical constraints of the power grid. This work specifically focuses on the simplified version of the OPF problem, i.e., DC-OPF, where its formulation is presented below.


\small
\begin{subequations}
\begin{gather}
    \texttt{min} \sum_{i \in \mathcal{G}} C_i(P_{\texttt{G}}^{i}) \label{min} \\
    \texttt{subject to:} 
    % Power balance
    \sum_{i \in \mathcal{G}_{k}} P_{\texttt{G}}^{i} - \sum_{j \in \mathcal{L}_{k}} P_{\texttt{D}}^{j} = \sum_{l \in \mathcal{B}_{k}} P_{l,k}, \quad \forall k \in \mathcal{N} \label{PB} \\
    % Line flow constraints
    P_{l,k} = \frac{\theta_{l} - \theta_{k}}{X_{l,k}}, \quad \forall (l, k) \in \mathcal{L} \label{LF} \\
    % Generation limits
    P_{\texttt{G}}^{\text{min},i} \leq P_{\texttt{G}}^{i} \leq P_{\texttt{G}}^{\text{max},i}, \quad \forall i \in \mathcal{G} \label{GL} \\
    % Phase angle limits
    \theta_{k}^{\text{min}} \leq \theta_{k} \leq \theta_{k}^{\text{max}}, \quad \forall k \in \mathcal{N} \label{PA} \\
    % Line capacity limits
    -P_{l,k}^{\text{max}} \leq P_{l,k} \leq P_{l,k}^{\text{max}}, \quad \forall (l, k) \in \mathcal{L} \label{LC}
\end{gather}
\end{subequations}
\normalsize


% \begin{equation}
% \min \sum_{i \in \mathcal{G}} C_i(P_{Gi}) \label{min}
% \end{equation}
% \text{subject to:} \\
% \text{Power balance:} \\ 
% \begin{equation}
% \sum_{i \in \mathcal{G}_k} P_{Gi} - \sum_{j \in \mathcal{L}_k} P_{Dj} = \sum_{l \in \mathcal{B}_k} P_{lk}, \forall k \in \mathcal{N} \quad\label{PB}
% \end{equation}
%  \text{Line flow constraints:} \\
% \begin{equation}
% P_{lk} = \frac{\theta_l - \theta_k}{X_{lk}}, \quad \forall (l, k) \in \mathcal{L}\label{LF}
% \end{equation}
% \text{Generation limits:} \\
% \begin{equation}
% P_{Gi}^{\text{min}} \leq P_{Gi} \leq P_{Gi}^{\text{max}}, \quad \forall i \in \mathcal{G}\label{GL}
% \end{equation}
% \text{Phase angle limits:} \\
% \begin{equation}
% \theta_k^{\text{min}} \leq \theta_k \leq \theta_k^{\text{max}}, \quad \forall k \in \mathcal{N}\label{PA}
% \end{equation}
% \text{Line capacity limits:} \\
% \begin{equation}
% -P_{lk}^{\text{max}} \leq P_{lk} \leq P_{lk}^{\text{max}}, \quad \forall (l, k) \in \mathcal{L}\label{LC}
% \end{equation}

% The OPF problem aims to minimize the total generation cost, expressed by the cost function \( \sum_{i \in \mathcal{G}} C_i(P_{Gi}) \), where \( C_i(P_{Gi}) \) represents the cost associated with generating power at generator \( i \). The active power generation at generator \( i \) is denoted by \( P_{Gi} \), and the set of all generators is given by \( \mathcal{G} \). The power balance constraint ensures that, for each bus \( k \), the total power generation from the generators connected to the bus, represented by the set \( \mathcal{G}_k \), minus the total power demand from the loads connected to the bus, represented by the set \( \mathcal{L}_k \), equals the net power flow on the lines connected to the bus, represented by \( \sum_{l \in \mathcal{B}_k} P_{lk} \), where \( \mathcal{B}_k \) is the set of lines connected to bus \( k \). The active power flow on line \( (l, k) \) is given by \( P_{lk} \), and it is defined by \( P_{lk} = \frac{\theta_l - \theta_k}{X_{lk}} \), where \( \theta_k \) and \( \theta_l \) are the voltage phase angles at buses \( k \) and \( l \), respectively, and \( X_{lk} \) is the reactance of the line. The generation limits constrain \( P_{Gi} \) to lie within its minimum and maximum bounds, \( P_{Gi}^{\text{min}} \) and \( P_{Gi}^{\text{max}} \), respectively. Similarly, phase angle limits are enforced by \( \theta_k^{\text{min}} \leq \theta_k \leq \theta_k^{\text{max}} \) for all buses. Finally, line capacity limits ensure that the power flow on line \( (l, k) \) stays within \( -P_{lk}^{\text{max}} \leq P_{lk} \leq P_{lk}^{\text{max}} \). The sets \( \mathcal{N} \) and \( \mathcal{L} \) represent the buses and loads in the network, respectively.

This problem aims to minimize the total generation cost, i.e., \( \sum_{i \in \mathcal{G}} C_i(P_{\texttt{G}}^{i}) \). In \eqref{min}, \( C_i(P_{\texttt{G}}^{i}) \) represents the generation cost of plant \( i \). The active power output of generator \( i \) is denoted by \( P_{\texttt{G}}^{i} \), and the set of all generators is given by \( \mathcal{G} \). The load at bus \( j \) is denoted by \( P_{\texttt{D}}^{j} \), and the set of loads is given by \( \mathcal{L} \).

The power balance constraint is captured by \eqref{PB} and ensures supply matches demand. Note, the power exchange of bus $i$ with neighboring buses is represented by \( \sum_{l \in \mathcal{B}_{k}} P_{l,k} \). The set of lines connected to bus \( k \) is shown by \( \mathcal{B}_{k} \).
%The power balance constraint ensures that, for each bus \( k \), the total power generation from the generators connected to the bus, represented by the set \( \mathcal{G}_{k} \), minus the total power demand from the loads connected to the bus, represented by the set \( \mathcal{L}_{k} \), equals the net power flow on the lines connected to the bus, represented by \( \sum_{l \in \mathcal{B}_{k}} P_{l,k} \), where \( \mathcal{B}_{k} \) is the set of lines connected to bus \( k \). 
%
The active power flow on line \( (l, k) \) is given by \( P_{l,k} \), and it is defined by \( P_{l,k} = \frac{\theta_{l} - \theta_{k}}{X_{l,k}} \), where \( \theta_{k} \) and \( \theta_{l} \) are the voltage phase angles at buses \( k \) and \( l \), respectively, and \( X_{l,k} \) is line's reactance.

Generators' output \( P_{\texttt{G}}^{i} \) is limited to lower and upper bounds, \( P_{\texttt{G}}^{\text{min},i} \) and \( P_{\texttt{G}}^{\text{max},i} \), respectively. Similarly, phase angle limits are enforced by (1e) for all buses. Finally, line capacity limits ensure that the power flow on line \( (l, k) \) stays within its limits.
%\( -P_{l,k}^{\text{max}} \leq P_{l,k} \leq P_{l,k}^{\text{max}} \).
%
The sets \( \mathcal{N} \) and \( \mathcal{L} \) represent the buses and loads in the network.

% The OPF problem, shown in \eqref{min}-\eqref{LC}, is typically resolved repeatedly every 15 minutes in accordance with the updated load demand ($P_D$). Instead of solving the optimization problem using an iterative solver, ML-based optimization proxies can be used to replace this repeated process. Optimization proxies are trained with historical data to develop a mapping between the load demand and the optimal generation profile. This mapping allows for faster, real-time decision-making.

% Existing OPF optimization proxies usually rely on MATPOWER
% data transferred from MATLAB to Python for training,
% creating a point of vulnerability where attacks could corrupt the data. In the subsequent section, we will present how data poisoning attacks are complete for three different optimization proxies, specifically, penalty method \cite{Liu2022Pen}, DC3 method \cite{donti2021dc3} (a post-repair method) and \LOOPLC~ \cite{Li_2023} method(a direct mapping method).

\subsection{ML-based Optimization Proxies}The OPF problem, as outlined in \eqref{min}-\eqref{LC}, is typically resolved every 15 minutes to accommodate updated load demands ($P_D$). Traditionally handled using iterative solvers, this repeated process can alternatively be managed using ML-based optimization proxies. These proxies leverage historical data to develop a mapping between demand and the optimal generation profile, enabling faster, real-time decision-making.

Most existing OPF optimization proxies, including those we examine in this paper, utilize MATPOWER data that is transferred from MATLAB to Python for training. This data transfer introduces a potential vulnerability, as it could be targeted by data corruption attacks. In the following section, we discuss how data poisoning attacks are implemented against three distinct types of optimization proxies.


\section{Poisoning Attack Implementation}\label{PA}

Poisoning attacks generally adopt one of two strategies. In a black box attack, the attacker has limited knowledge about NN and its data, relying mostly on estimations to inflict damage. Conversely, a white box attack assumes that an attacker has access to considerable details of the grid network and its data. In our study, we employ a white box attack approach, where the attacker is well-informed about the dataset and has a rough idea of the network's architecture. This enables us to demonstrate the significant impact on the optimal solutions produced by the method. Our approach modifies the strategy outlined by Chen et al. in their study of evasion attacks on demand prediction \cite{evasion2019demand}. We specifically engineer an attack to maximize the loss, misleading the neural network into predicting higher generation needs than required. This leads to an unnecessary increase in energy production, thus disrupting the supply-demand equilibrium.

In what follows, we will outline our methods of poisoning attacks.
%and explain our development of the attack function, {\color{red}which demonstrates our pre-existing knowledge of grid operations}. 
Then, we will detail the implementation of poisoning attacks on three different OPF proxies: the penalty method \cite{Liu2022Pen}, the DC3 method \cite{Li_2023}, and the \LOOPLC~ method \cite{donti2021dc3}.

% Poisoning attacks have two general strategies. A black box attack is where the attacker has very little knowledge of the NN and data and estimates how to damage it properly. In a white box attack, the attacker has some amount of knowledge of the network/data. In our method, we went with a white box attack where the attacker has knowledge of the data set as well as an estimate of what the network looks like in order to show the impact to the optimal solution generated by the method. Our method was modified from Chen et al.'s work on evasion attacks on demand prediction in \cite{evasion2019demand}. We focused on an attack that maximized the loss by causing the neural network to predict that there needs to be more generation than is actually true. This will cause an imbalance of supply and demand and cause for there to be wasted generation when generators are told to produce more energy than needed. 


% To implement the attack, we used the neural network architecture and the loss function to add perturbations to the training data by calculating the gradient of the loss function, with respect to the input data by using the direction to move each input point to maximize loss. We then add or subtract a set perturbation amount to each data point in order to maximize loss while ensuring that the amount the input points are moved is within a certain bound. This is done in order to avoid detection from those working to protect themselves from adversaries. For our loss function, we use mean squared error (MSE). We find the loss by finding the MSE between a very large valued target and the unattacked optimal generation prediction, in order to achieve the generation maximization attack. The adversarial maximization problem is shown in \eqref{pois}, which is modified from \cite{bai2021recent}. 




\subsection{Formulating the Poisoning Attack Strategy}Data poisoning attacks are meticulously crafted to manipulate the training data of ML-based optimization proxies. These proxies are designed to map load demands, the inputs, to their respective optimal generation profiles, the outputs. By subtly introducing calculated perturbations into the load demand data, attackers aim to skew the proxy's ability to predict the most efficient (optimal and feasible) power output setting.

The perturbations are guided by the gradient of a Mean Squared Error (MSE) loss function, which measures the deviation between the perturbed (attacked) generation targets and the proxy's predictions. The objective is to manipulate the model into recommending higher than necessary energy outputs, maximizing the loss \cite{bai2021recent}:

\small
\begin{equation}
\texttt{max}_\delta \; L(\theta, x + \delta, y)\label{pois}
\end{equation}
\normalsize

In \eqref{pois}, $L$ is the loss function, $\delta$ is the perturbation, $\theta$ is the model parameters, $x$ is the model input, and $y$ is the output.

A bounding mechanism is applied to increase the likelihood that these perturbations evade detection by protective measures. This mechanism restricts the degree of perturbation applied to each data point while showing {\color{black}pre-existing knowledge of
grid operation (shown by $x_{\text{orig}}$)}:

% The bound is used to avoid detection from defenders against attacks. The bounding clips an adversarial data if the perturbation added to the original input point varies a certain proportional amount from the original input point. This clipping is seen in \eqref{clip}, where $x_{\text{new}}$ is the adversarial data point and $x_{\text{orig}}$, is the original, unperturbed data.

\vspace{-.4cm}

\small
\begin{equation}
x_{\text{new, bounded}} = \texttt{clip}\left(x_{\text{new}}, \, x_{\text{orig}} - \text{L} \cdot |x_{\text{orig}}|, \, x_{\text{orig}} + \text{L} \cdot |x_{\text{orig}}|\right) \label{clip}
\end{equation}
\normalsize

\vspace{.2cm}

The functionality of the clip function is summarized below:
\small
\begin{equation}
\texttt{clip}(x, a, b) = 
\begin{cases}
x = a, & \text{if } x < a \\
x = x, & \text{if } a \leq x \leq b \\
x = b, & \text{if } x > b
\end{cases}
\end{equation}
\normalsize

\noindent where $L$ is the bound that we select to avoid detection, $x_\text{new}$ is the new, perturbed input data that has been attacked. Also, $x_\text{orig}$ is the unattacked, original input data, and $x_\text{new, bounded}$ is the new, attacked data once the bound has been applied. This approach strategically enforces limits on the alterations to ensure the changes do not attract undue attention, while still achieving the intended disruptions to model training. This subtle yet strategic manipulation leads to optimality and feasibility gaps in the grid operation once the model is deployed, demonstrating the critical need for robust defenses against such adversarial threats in the training of ML-based optimization proxies. In the following proxy methods, we perturb the input data at the same location in the data pipeline, however, the way in which that perturbed data will affect the algorithm is different for each case.  \label{FormPoisAttack}

\subsection{Poisoning Optimization Proxies}In the following, we
discuss how data poisoning attacks are implemented against
three distinct types of optimization proxies:  the penalty method \cite{Liu2022Pen}, the DC3 method \cite{Li_2023} (a post--repair approach), and the \LOOPLC~ method \cite{donti2021dc3} (a direct mapping approach).

\subsubsection{Penalty Method}% The penalty method is one of the classical ways to address feasibility violations in ML training. It involves bringing in constraints into the loss function and creating a loss that increases as the neural network results approach violation of the constraints. This incentivizes the network to keep solutions within the feasible region. 

% For the penalty method, since it is entirely NN based, we simply perturbed the input training data to this NN. This implementation is shown in Fig. \ref{penfig}. Adversarial perturbations will exploit the fact that feasibility constraints are not absolutely required to be enforced potentially causing constraint violations in addition to moving the value away from the true optima.

The penalty method is a well-established approach used to manage feasibility violations in ML training. It incorporates constraints directly into the loss function, which is designed to increase as the NN outputs approach the boundaries of these constraints. This design incentivizes the network to produce solutions that remain within the feasible region.

Our approach for introducing perturbations into the penalty method framework is shown in Fig. \ref{penfig}. By exploiting the inherent flexibility of penalty-based methods in enforcing feasibility constraints, adversarial perturbations can cause constraint violations and push solutions further from the optima.

\begin{figure}[h!]
\centerline{\includegraphics[scale=0.085]{Penalty_Attack.png}}
\vspace{-.2cm}
\caption{Poisoning the workflow of Penalty methods. 
%diagram showing where in the pipeline the poisoning attack takes place.
}
\label{penfig}
\end{figure}

\vspace{-.2cm}

\subsubsection{Deep Constraint Completion and Correction (DC3) Method} We also focused on poisoning the DC3 method from Donti et al in \cite{donti2021dc3}. This paper proposes a method to incorporate
feasibility constraints in deep learning in order to make
the process more usable for problems with real physical constraints like OPF, called Deep Constraint Completion and Correction (DC3). In \cite{donti2021dc3}, authors first solve the necessary equality constraints, then infer the remaining ones. To this end, they will
use deep NN to find a partial solution and later infer the remaining solutions using equality constraints. The resulting solution is then moved to the region that also
satisfies the inequality constraints (the feasible region) by taking gradient descent steps along the equality-satisfying region,
and the loss is back-propagated and training continues. As this
paper mentions, convergence to the optima is not guaranteed
in the gradient descent portion, but during test, the proposed
solution should be close to the actual solution. 

We expect a trade off between being close to the optimal value and being within the feasible region due to the walk into the feasible region potentially taking the solution further away from the optimal. The poisoning attack would exploit this inherent weakness by either causing a feasibility violation or keeping the solution far away from the true optima (see Fig. \ref{dc3fig}). In \cite{donti2021dc3}, the number of correction steps for convex problems is selected as ten to balance time, optimality, and feasibility. 

\begin{figure}[ht]
\centerline{\includegraphics[scale=0.059]{DC3_Attack.png}}
\vspace{-.2cm}
\caption{Poisoning the workflow of the DC3 method.}
\label{dc3fig}
\end{figure}


\subsubsection{Learning to Optimize the Optimization Process with Linear Constraints (LOOP--LC) Method} %  This method removed the iterations involved in optimization problems. The general idea is that they are considering minimizing a function $f (x, u)$, where
% $x$ is the set of inputs to the distribution, and $u$ is the set of
% parameters while obeying constraints. There are two ways in
% which approximating this is completed. The first way is by using a solver to generate $u$ and use it as the ground truth, then
% treating is as a supervised learning problem to train the neural
% network to output the optimal $v$, then mapping the solution to the feasible region to output $u$. For the version without the
% solver the method tries to minimize the expected objective
% value over the set of input distributions, then map to the feasible region. The authors achieve
% the mapping to the feasible region by putting the problems in terms of independent variables,
% moving the feasible region to be centered around the origin by
% moving by an interior point, mapping the feasible region to the
% l-infinite norm unit ball, having a neural approximator find a
% solution on the l-infinite norm unit ball, then using a gauge
% map to map that solution to the feasible region.

% In this paper we focus on the method using the solver to generate a ground truth. In \LOOPLC~, feasibility of the solution is guaranteed. This means that the poisoning attack will take advantage of this and shift the outputted solution further away from the true optimal. We will perturb the data being used to train the NN, labeled as the "Optimization module" in Fig. \ref{loopfig}.

The \LOOPLC~method employs a neural approximator to transform input data directly into a near-optimal feasible solution. This approach ensures hard feasibility without relying on penalty terms and operates without iterations, utilizing a straightforward feed-forward process. The technique begins by training a neural network to map inputs to a virtual optimal point within the $\ell_\infty$-norm unit ball. Following this, a feasibility module is used to project this virtual prediction onto actual feasible solutions.

The method implements inequalities by using a gauge mapping technique to project the virtual optimal point from the \(\ell_\infty\)-norm unit ball onto the feasible space. For equality constraints, it determines the values of dependent variables based on the independent variables, effectively reconstructing the solution to the original problem. This integrated approach allows the \LOOPLC~method to achieve reliable and efficient optimization outcomes, free from the iterative adjustments typical of other methods.

 In the \LOOPLC~method, solution feasibility is guaranteed. Consequently, our poisoning attack strategy focuses on undermining the optimality of the solutions. We aim to shift the neural network's outputted solution further from the true optimal. This targeted manipulation occurs specifically within the ``Optimization module," as illustrated in Fig. \ref{loopfig}.

\begin{figure}[h!]
\centerline{\includegraphics[scale=0.067]{LOOP_Attack.png}}
\vspace{-.2cm}
\caption{Poisoning the workflow of the \LOOPLC~method.}
\label{loopfig}
\end{figure}



\captionsetup{skip=5pt} % Adjust caption spacing to make it closer to each subfigure

\begin{figure*}[h!]
\centering
\begin{subfigure}{0.45\linewidth}
    \includegraphics[width=\linewidth]{penalty_opt.png}
    \caption{Penalty (Optimality) Error}
\end{subfigure}\hfill
\begin{subfigure}{0.45\linewidth}
    \includegraphics[width=\linewidth]{penalty_feas.png}
    \caption{Penalty (Feasibility) Error}
\end{subfigure}

\vspace{10pt} % Add space between rows

\begin{subfigure}{0.45\linewidth}
    \includegraphics[width=\linewidth]{LOOP_opt.png}
    \caption{\LOOPLC~(Optimality) Error}
\end{subfigure}\hfill
\begin{subfigure}{0.45\linewidth}
    \includegraphics[width=\linewidth]{LOOP_feas.png}
    \caption{\LOOPLC~(Feasibility) Error}
\end{subfigure}

\vspace{10pt} % Add space between rows

\begin{subfigure}{0.45\linewidth}
    \includegraphics[width=\linewidth]{DC3_opt.png}
    \caption{DC3 (Optimality) Error}
\end{subfigure}\hfill
\begin{subfigure}{0.45\linewidth}
    \includegraphics[width=\linewidth]{DC3_feas.png}
    \caption{DC3 (Feasibility) Error}
\end{subfigure}

\caption{Plots of the error for each method (Penalty, \LOOPLC, and DC3) for both (left) optimality and (right) feasibility, versus perturbation amount, with a bound of 0.75. The values above the green feasibility line are infeasible. The red ``x" represents the unperturbed case. In the Penalty and DC3 methods, all results other than the unperturbed case are infeasible, while \LOOPLC~ maintains feasibility across all cases.}
\label{fig:all_methods}
\vspace{-0.3cm}
\end{figure*}


\begin{table*}[h!]
\caption{Summary of Adversarial Attack Results (Used on a perturbation bound of 0.75 $\times$ original value.)}
\vspace{-.2cm}
\begin{center}
\begin{tabular}{|c|>{\columncolor{green!20}}c|>{\columncolor{red!20}}c|>{\columncolor{green!20}}c|>{\columncolor{red!20}}c|>{\columncolor{green!20}}c|>{\columncolor{red!20}}c|}
\hline
 & \multicolumn{2}{c|}{\textbf{Optimality Gap}} & \multicolumn{2}{c|}{\textbf{Feasibility Gap}} & \multicolumn{2}{c|}{\textbf{Time on CPU (sec)}} \\

\hline 
\textbf{Optimizer} & \textbf{Unattacked} & \textbf{Attacked} & \textbf{Unattacked} & \textbf{Attacked} & \textbf{Unattacked} & \textbf{Attacked} \\
\hline 
\textbf{Penalty \cite{Liu2022Pen}} & 0.11412 & 2.20422 & 0.06166 & 29.95326 & 0.00017 & 0.00027\\
\hline
\textbf{LOOP-LC \cite{Li_2023} } & 0.03786 & 0.10964 & 0 & 0 & 0.06986 & 0.06801\\
\hline 
\textbf{DC3 \cite{donti2021dc3}} & 0.01654 & 2.77317 & 0.10314 & 33.25656 & 0.08053 & 0.10336\\
\hline 
%\multicolumn{7}{l}{$^{\mathrm{*}}$These values are based on a perturbation bound of 0.75*original value.} \\
\end{tabular}
\label{tab1}
\end{center}
\end{table*}



\vspace{-.2cm}
\section{Results} \label{Res}
\vspace{-.2cm}
\subsection{Simulation Setup}
\subsubsection{Dataset}\begin{table*}
\label{datasetdescription}
\centering
\caption{Overview of the evaluation dataset, where M denotes malware and B denotes benign applications.}
\begin{tabular}{c|c|c|c|c|c|c|c|c} 
\hline
           & \textbf{Time Interval} & \textbf{Sample Size} & \begin{tabular}[c]{@{}c@{}}\textbf{The number of}\\\textbf{Existing family}\end{tabular} & \begin{tabular}[c]{@{}c@{}}\textbf{The number of}\\\textbf{New family}\end{tabular} & \textbf{Packed} & \textbf{Malicious} & \textbf{Benign} & \textbf{M/(M+B)\%}  \\ 
\hline
Test set 1 & 2020.05 - 2021.01 & 3015  & 21                                                               & 24                                                          & 18     & 284       & 2731   & 9.42                       \\ 
\hline
Test set 2 & 2021.01 - 2021.12 & 3015      & 28                                                               & 32                                                          & 30     & 298       & 2717   & 9.88                      \\ 
\hline
Test set 3 & 2021.12 - 2023.12 & 3016      & 34                                                               & 36                                                          & 40     & 302       & 2714   & 10.01                       \\
\hline                  
\end{tabular}
\end{table*}
\subsubsection{Metrics}\begin{table}
%\begin{minipage}{\columnwidth} 
  \centering
	 \caption{\textcolor{black}{Structural code quality metrics used in this study.}}
	 \label{Table:Quality Metrics Used in This Study.}
  \begin{threeparttable}
%\begin{sideways}
\begin{adjustbox}{width=1.0\textwidth,center}
%\begin{adjustbox}{width=\textheight,totalheight=\textwidth,keepaspectratio}
\begin{tabular}{llll}\hline
\toprule
\bfseries Quality Attribute & \bfseries Study &   \bfseries Metric & \bfseries Description \\
\midrule
%\multicolumn{2}{l}{\textbf{\textit{Internal Quality Attribute }}}\\
%\midrule
Cohesion & \cite{pantiuchina2018improving,chavez2017does} &↓ LCOM& Lack of Cohesion of Methods   \\ 
Coupling &  \cellcolor{gray!30}\cite{chavez2017does,pantiuchina2018improving} & \cellcolor{gray!30}↓ CBO&\cellcolor{gray!30}Coupling Between Objects    \\
         & \cite{pantiuchina2018improving} & ↓ RFC & Response For Class   \\
         & \cellcolor{gray!30}\cite{islam2018characteristics} & \cellcolor{gray!30}↓ NII &\cellcolor{gray!30}Number of Incoming Invocations  \\
         & \cite{islam2018characteristics} & ↓ NOI &Number of Outgoing Invocations \\
Complexity & \cellcolor{gray!30}\cite{chavez2017does} & \cellcolor{gray!30}↓ CC & \cellcolor{gray!30}Cyclomatic Complexity 
           \\
           & \cite{chavez2017does,pantiuchina2018improving,singh2012evaluation} & ↓ WMC& Weighted Method Count  \\
           & \cellcolor{gray!30}\cite{islam2018characteristics} & \cellcolor{gray!30}↓ NL & \cellcolor{gray!30}Nesting Level  \\
           & \cite{islam2018characteristics} & ↓ NLE &Nesting Level Else-if  \\
           & \cellcolor{gray!30}\cite{islam2018characteristics} & \cellcolor{gray!30}↓ HCPL & \cellcolor{gray!30}Hal. Calculated Program Length \\
            & \cite{islam2018characteristics} & ↓ HDIF & Hal. Difficulty  \\
             & \cellcolor{gray!30}\cite{islam2018characteristics} & \cellcolor{gray!30}↓ HEFF & \cellcolor{gray!30}Hal. Effort  \\
              & \cite{islam2018characteristics} & ↓ HNDB & Hal. Number of Delivered Bugs   \\
               & \cellcolor{gray!30}\cite{islam2018characteristics} &\cellcolor{gray!30}↓ HPL & \cellcolor{gray!30}Hal. Program Length   \\
                & \cite{islam2018characteristics} & ↓ HPV & Hal. Program Vocabulary  \\
                 & \cellcolor{gray!30}\cite{islam2018characteristics} & \cellcolor{gray!30}↓ HTRP &\cellcolor{gray!30}Hal. Time Required to Program  \\
                  & \cite{islam2018characteristics} & ↓ HVOL &Hal. Volume \\
                   & \cellcolor{gray!30}\cite{islam2018characteristics} &\cellcolor{gray!30}↑ MIMS & \cellcolor{gray!30}Maintainability Index (MS) \\
                    & \cite{islam2018characteristics} &↑ MI& Maintainability Index (OV) \\
                     & \cellcolor{gray!30}\cite{islam2018characteristics} &\cellcolor{gray!30}↑ MISEI& \cellcolor{gray!30}Maintainability Index (SEIV) \\
                      & \cite{islam2018characteristics} &↑ MISM&  Maintainability Index (SV)\\
Inheritance & \cellcolor{gray!30}\cite{chavez2017does,singh2012evaluation} & \cellcolor{gray!30}↓ DIT &\cellcolor{gray!30}Depth of Inheritance Tree 
  \\
   & \cite{chavez2017does,singh2012evaluation} & ↓ NOC &Number of Children   \\
 & \cellcolor{gray!30}\cite{bavota2015experimental} & \cellcolor{gray!30}↓ NOA & \cellcolor{gray!30}Number of Operations Added by Subclass   \\
        
Design Size & \cite{chavez2017does} & ↓ LOC & Lines of Code \\
& \cellcolor{gray!30}\cite{islam2018characteristics} &\cellcolor{gray!30}↓ TLOC &\cellcolor{gray!30}Total Lines of Code   \\
& \cite{chavez2017does} & ↓ LLOC &Logical Lines of Code   \\
& \cellcolor{gray!30}\cite{islam2018characteristics} & \cellcolor{gray!30}↓ TLLOC&\cellcolor{gray!30}Total Logical Lines of Code  \\
            & \cite{chavez2017does} & ↑  CLOC&Lines with Comments  \\
            & \cellcolor{gray!30}\cite{stroggylos2007refactoring} & \cellcolor{gray!30}↓ NPM &\cellcolor{gray!30}Number of Public Methods  \\
           % & & Total Number of Methods (TNM) \\
            &\cite{islam2018characteristics} &↓ NOS& Number of Statements  \\
            &\cellcolor{gray!30}\cite{islam2018characteristics} & \cellcolor{gray!30}↓ TNOS&\cellcolor{gray!30}Total Number of Statements  \\
           % & & Total Number of Accessor Methods (TNG) \\
           % & & Total Number of Attributes (TNA) \\
\bottomrule
\multicolumn{4}{l}{\tiny 
↑ by a metric indicates the higher the better for that metric; 
↓ by a metric indicates the lower the better for that metric.}
%Hal.=Halstead; MS= Microsoft version; OV=Original version; SEIV=SEI version; SV=SourceMeter version.}
% \begin{tablenotes}
 %    The first note
 %   \end{tablenotes}
\end{tabular}
\end{adjustbox}
%\end{sideways}
%\footnote{xxx}
%\end{minipage}
\end{threeparttable}
\end{table}
%\footnotesize{$^a$ The smallest spatial unit is county, $^b$ more details in appendix A}

\subsubsection{Motivation for The Choice of Bound}As discussed in Section \ref{FormPoisAttack}, to perturb the values, we add a perturbation amount, $\delta$, in the direction that maximizes loss. Once all the input points reach their bounded values, the error stays the same, regardless of the change in perturbation amount. This is seen in Fig. \ref{fig:all_methods},  where the error for both optimality and feasibility stops increasing once the perturbation added reaches the bound for the input points. Due to this, we focused on our choice of bound in our implementation and selected $\delta$ to reach the bound, as we wanted to demonstrate the extent of the damage that could be done by the attack. 

To select the bound, we considered the varying knowledge of multiple grid operators looking at the data. We assumed that the operators had a good intuition of the range of values to expect from the demand and generation. If values were set to zero or doubled, which would happen with a bound of 1, that would be apparent to operators. However, we wanted to show the extent of the attack's impact by suggesting a bound that was close enough to 1. The bound value of 0.75 was deemed reasonable, as the input values were formatted such that most were very small (much less than 1), and the bound of 0.75 was difficult to detect given our data set. 



% \begin{figure*}[b!]
% \begin{subfigure}{0.45\linewidth}
% \includegraphics[width=.95\linewidth]{penalty_opt_eps.png}
% \caption{Penalty (Optimality) Error}
% \end{subfigure}\hfill
% \begin{subfigure}{0.45\linewidth}
% \includegraphics[width=.95\linewidth]{penalty_feas_eps.png}
% \caption{Penalty (Feasibility) Error}
% \end{subfigure}\hfill
% \caption{Plots of the error of the Penalty method results for (a) optimality and (b) feasibility  versus perturbation amount, Delta, with a bound of 0.75. The values above the green feasibility line are infeasible. The red "x" represents the unperturbed case. The error reaches a max when the input values have reached the perturbation bound for both optimality and feasibility of the Penalty Method. All results other than the unperturbed ones are infeasible.}
% \label{penalty}
% \end{figure*}

% \begin{figure*}
% \begin{subfigure}{0.45\linewidth}
% \includegraphics[width=.95\linewidth]{LOOP_opt_eps.png}
% \caption{LOOP-LC (Optimality) Error}
% \end{subfigure}\hfill
% \begin{subfigure}{0.45\linewidth}
% \includegraphics[width=.95\linewidth]{LOOP_feas_eps.png}
% \caption{LOOP-LC (Feasibility) Error}
% \end{subfigure}\hfill
% \vspace{-.1cm}
% \caption{Plots of the error of the \LOOPLC method results for (a) optimality and (b) feasibility  versus perturbation amount, Delta, with a bound of 0.75. The values above the green feasibility line are infeasible. The red "x" represents the unperturbed case. The error reaches a max when the input values have reached the perturbation bound for both optimality and feasibility of the Penalty Method. All values, including the results of perturbed inputs, are feasible.}
% \label{LOOP}
% \vspace{-.4cm}
% \end{figure*}

% \begin{figure*}
% \begin{subfigure}{0.45\linewidth}
% \includegraphics[width=.95\linewidth]{DC3_opt_eps.png}
% \caption{DC3 (Optimality) Error}
% \end{subfigure}\hfill
% \begin{subfigure}{0.45\linewidth}
% \includegraphics[width=.95\linewidth]{DC3_feas_eps.png}
% \caption{DC3 (Feasibility) Error}
% \end{subfigure}\hfill
% \vspace{-.1cm}
% \caption{
% Plots of the error of the DC3 method results for (a) optimality and (b) feasibility  versus perturbation amount, Delta, with a bound of 0.75. The values above the green feasibility line are infeasible. The red "x" represents the unperturbed case. The error reaches a max when the input values have reached the perturbation bound for both optimality and feasibility of the Penalty Method. All results other than the unperturbed ones are infeasible.}
% \label{DC3}
% \vspace{-.4cm}
% \end{figure*}


% \begin{figure*}

%     % Part (a): Penalty
%     \begin{subfigure}{\columnwidth}
%         \includegraphics[width=\columnwidth]{penalty_opt_eps.png}
%         \caption{Penalty (Optimality)}
%         \centering
%     \end{subfigure}
    
%     \hfill
    
%     \begin{subfigure}{\columnwidth}
%         \includegraphics[width=\columnwidth]{penalty_feas_eps.png}
%         \caption{Penalty (Feasibility) Error}
%         \centering
%     \end{subfigure}

%     \hfill
    
%     % Part (b): LOOP
%     \begin{subfigure}{\columnwidth}
%         \includegraphics[width=\columnwidth]{LOOP_opt_eps.png}
%         \caption{LOOP (Optimality) Error}
%         \centering
%     \end{subfigure}

%     \newpage
    
%     % Part (c): DC3
%     \begin{subfigure}[b]{\columnwidth}
%         \includegraphics[width=\columnwidth]{DC3_opt_eps.png}
%         \caption{DC3 (Optimality) Error}
%         \centering
%     \end{subfigure}
    
%     \hfill
    
%     \begin{subfigure}[b]{\columnwidth}
%         \includegraphics[width=\columnwidth]{DC3_feas_eps.png}
%         \caption{DC3 (Feasibility) Error}
%         \centering
%     \end{subfigure}

%     \hfill
    
%     \caption{Demonstration of the error values reaching a maximum when the input values have reached the bound for both optimality and feasibility. (a) Penalty method, (b) LOOP method, (c) DC3 method.}
% \label{eps}
% \end{figure*}




\subsection{Simulation Results and Discussions}
% \begin{figure*}[htpb!]
% \label{}
% \centering

%     {{\label{ROCIowaCedar} \includegraphics[width=\textwidth/3]{figures/IowaCedar_roc.png}}}%
%     \qquad
%     {{\label{ROCIowaDesMoines} \includegraphics[width=\textwidth/3]{figures/IowaDesMoines_roc.png} }%
%   \captionsetup{justification=centering}
%   \caption{\Acf{ROC} curves for \acf{RW} Iowa (CR) and  \acf{RW} Iowa (DM) dataset. Dummy model here represents a model whose output is solely a ``no Flood'' for all pixels.}
%   \label{fig:RW_ROC_Curves}%
% \end{figure*}



\section{Results and Discussions}
\label{sec:Results}

In this section, we aim to answer three main questions. First, we want to validate our hypothesis that \ac{SYN} data is a viable proxy for \ac{RW} data when training ML models for downscaling. Secondly, we seek to assess how much more skillful ML-based downscaling is compared to classical, non-data-driven techniques, such as our baseline methods, \textit{i.e.}, thresholded bicubic and Lanczos interpolation. Finally, we would like to appraise the extent to which data-driven models like ours are transferable (in terms of usefulness) to other regions without major performance degradations.  
To assess the quality of the models, we conduct a multiple comparison test --namely the Holm-Bonferroni procedure \cite{HolmBonferroni1979} -- that is designed to control the \ac{FWER}. We notice that, with a \ac{FWER} of $10^{-3}$, all the differences in model performance are significant. The only exception to this trend was observed in \ac{RW}-GH for whom the pairwise differences between \ac{RCAN} and \ac{ESRT}, Lanczos and Bicubic were not significant with the aforementioned \ac{FWER}. 

%Finally, we aim to find out the factors influencing the transferability of our models from one region to another.

\subsection{Potential of using SYN Data for RW downscaling}

In order to evaluate the utility of synthetic data for training, we compare performances of our candidate models on both \ac{SYN} and \ac{RW} Iowa data whose results are presented in Table \ref{tab:IowaResults}. We notice that 
\textbf{(i)} For the Iowa datasets, there is a drop in performance of all the models when going from \ac{SYN} to \ac{RW} datasets, 
\textbf{(ii)} for the \ac{RW}-IA (CR) as well as \ac{RW}-IA (DM) datasets, both bicubic and Lanczos interpolation have accuracies and MCC up to 70.89\% and 0.42 respectively while the deep learning models have accuracies and MCC up to 73.34\% and 0.46 respectively, 
\textbf{(iii)} There is a roughly 6\% accuracy improvement for the \ac{SYN} data for the deep learning models compared to the bicubic and lanczos models and this improvement drops to about 3\% for \ac{RW} data,  
\textbf{(iv)} the performance of all the models remain consistent across both \ac{RW}-IA datasets and \textbf{(v)} in \figref{fig:RW_ROC_Curves}, we observe that there is a high degree of overlap among the \ac{ROC} curves for the data-driven models.

From (i) and (iv) we can conclude that \ac{SYN} data is more intricate than \ac{RW} data. This implies that the benefits yielded by training with \ac{SYN} dataset, while significant, is not as prominent in the \ac{RW} Iowa datasets. 
% This may be due to sensor noise prevalent in the \ac{RW} Landsat-8 data that can be harder to reproduce in the synthetically generated examples. 
(i), (iii) and (v) implies that while \ac{SYN} data is not an exact replacement for \ac{RW} data, it provides a rather significant edge, which is all the more important when there is insufficient \ac{RW} for training. From (ii) we can conclude that the three proposed data driven models outperform classical super-resolution techniques such as bicubic and lanczos, conclusion supported by the \ac{ROC} curves in Figure \ref{fig:RW_ROC_Curves} for whom the data-driven models, in general, lie above the non-data-driven alternatives. Observation (iv) shows that  for the climatically similar \ac{RW}-Iowa(CR) and \ac{RW}-Iowa(DM) regions, training on \ac{SYN} Iowa data does indeed provide an edge. 

% have similar climate. 

\begin{figure*}[t!]
    \centering
    \begin{subfigure}[t]{0.5\textwidth}
        \centering
        \includegraphics[width=\textwidth/2]{figures/IowaCedar_roc.png}
        \caption{}
    \end{subfigure}%
    ~ 
    \begin{subfigure}[t]{0.5\textwidth}
        \centering
        \includegraphics[width=\textwidth/2]{figures/IowaDesMoines_roc.png}
        \caption{}
    \end{subfigure}
    \vspace*{0.5cm}
    \caption{    \label{fig:RW_ROC_Curves} \Acf{ROC} curves for (a) RW-IA (CR) and (b) RW-IA (DM) dataset. Na\"ive model here represents a model whose output is solely a ``no Flood'' for all pixels. Star here represents the pixel-wise classifier with a threshold of 0.5.}
\end{figure*}


\subsection{Effectiveness of data-driven approaches}

In order to evaluate the effectiveness of ML models in the downscaling task, we compare performances of our candidate models to Lanczos and bicubic interpolation methods by looking at figures of some sample predictions from Iowa (Figure \ref{fig:RWIowaDesMoines}), performance comparison in the region of Iowa in Table \ref{tab:IowaResults} and the ROC curves in Figure \ref{fig:RW_ROC_Curves} for \ac{RW} data. We notice that 
\textbf{(vi)} For RW-IA (DM) samples, the deep learning models maintain a higher degree of spatial continuity in the predicted \ac{FIM}, 
\textbf{(vii)} We observe that  bicubic and Lanczos interpolation produces over-smoothed \ac{FIM} reconstructions, while the plain \ac{RDN}, \ac{RCAN} and \ac{ESRT} models are more detail-inclusive. Similar conclusions can be drawn upon inspecting the \ac{ROC} curves in Figure \ref{fig:RW_ROC_Curves} and 
\textbf{(viii)} For RW-IA (CR), the ML models show a performance improvement of 3.06\% when comparing the best ML model and non-data-driven method and, while for RW-IA (DM) there is a performance improvement of 2.45\%.


Figures \ref{fig:EUSamples} and \ref{fig:RWIowaDesMoines} show the spatial disparity among the models whose details are often obscured in aggregated metrics such as accuracy. (vi) This implies that these data-driven models are better are recognizing an underlying stream network geometry than the classical methods. However, when it comes to narrow river streams, all the models struggle capturing the nuances of the \ac{FIM} resultant from localized high elevation features such as small islands within rivers or man-made structures. (vii) shows a clear advantage of our data-driven approaches over the non-data-driven alternatives. (viii) indicates the benefits of the data-driven models when evaluated over Iowa. 



\subsection{Applicability of our models to external regions}

To evaluate how transferable our models are, we draw conclusions from figures of the sample predictions from Western Europe (Figure \ref{fig:EUSamples}) and Ghana (Figure \ref{fig:GhanaSamples}) as well as the performance comparison in Table \ref{tab:ExternalResults}. We notice that 
\textbf{(ix)} for Ghana all of the models fail to adequately inundate the pixels over separated areas on account of several disconnected regions of inundation in the chosen area,
\textbf{(x)} the ML models outperform non-data driven methods for RW-EU, 
\textbf{(xi)} for the RW-EU dataset, there is an improvement of 4.89\% when comparing the accuracy of the best data- and non-data-driven methods, 
\textbf{(xii)} For RW-RR and RW-GH, there is marginal improvement (up to 0.77\% in accuracy) of the ML methods over the non-data driven methods and 
\textbf{(xiii)} For RW-EU, we notice that the ML models produce more connected streams over the non-data-driven models. 

(x) and (xi) implies that the models are transferable when considering hydroclimaticalogically similar regions since Iowa and the Meuse river in Europe lie within mid temperate zones. Similar to the observation (vi) for RW-IA (DM), (xiii) implies that the benefits of the ML model in identifying underlying network streams is also transferable to hydroclimatologically similar regions. In contrast, (xii) and (ix) both imply that the trained ML models struggle to generalize to RW-RR \& RW-GH. We speculate that this may be due to the significant differences in geography and climate when compared to Iowa. 

% More specifically, we notice that Ghana has a lot of disconnected regions when compared to Iowa and Western Europe, possibly indicating a geomorphological dissimilarity. Additionally, in the case of Red River and Ghana, we also speculate that they include drivers to flood inundation that are different from Iowa and Western Europe, which lie within mild temperate zones. Ghana on the other hand has a tropical (dry and hot) climate.

Our study directly implies that good quality synthetic data can be useful surrogates for downscaling low-resolution \acp{WFM} to high-resolution \acp{FIM} in regions, where such data are hard to come by, even when downscaling by a factor of 10. We noticed that such models were readily transferable to climatically similar regions as the region of training. However, Such derived ML models did not feature significantly different transferability when evaluated over hydroclimatologically dissimilar regions, which we attribute to different flood inundation characteristics, primarily at finer scales. A possible avenue to circumvent such issues is to explore additional training approaches that fall under the general area of domain adaptation. Nevertheless, data-driven models are still advantageous (and, hence, preferable) over non-data-driven alternatives in transfer scenarios like the one we considered here. 


%%%%%%%%%%%%%%%%%%%%%%%%%%%%%%% unused text %%%%%%%%%%%%%%%%%%%%%%%%%%%%%%%%%%%%%%%



% \tabref{tab:AccuracyResults} depicts test accuracies obtained by our models on both \ac{SYN} and \ac{RW} data. For Iowan floods, a comparison of \ac{SYN} and \ac{RW} results shows \textbf{(i)} bicubic and Lanczos interpolations remarkably gaining about $3\%$ in accuracy, as well as \textbf{(ii)} \ac{RDN} and \ac{RCAN} remaining relatively stable, while \textbf{(iii)} topography-aware models loosing $2.7\%$ in performance. From (i) one can conclude that \ac{SYN} data are morphologically slightly more intricate than \ac{RW} data. Also, (i) and (ii) likely imply that \ac{SYN} data, excluding topography, can serve as satisfactory surrogates of \ac{RW} data. However, as implied by (iii), our topography-dependent models seems to be particularly sensitive to distributional shifts of their combined inputs (\acp{WFM} and topographic features). More specifically, the topography-informed models' performance edge, while still statistically significant, is extremely marginal, even when compared to our non-data-driven approaches. Next, when comparing results between the cases of Iowan and Ghanaian \ac{RW} data, one observes that \textbf{(iv)} the accuracy of bicubic and Lanczos interpolations drops by almost $5\%$ due to over-smoothing. This may imply that Ghanaian \acp{FIM} bare a more complex morphology, when compared to Iowan \acp{FIM}. Also, \textbf{(v)} our topography-agnostic, data-driven models' performance degrades more gracefully (by about $2\%$), while \textbf{(vi)} our topography-aware models perform, virtually, as bad as our non-data-driven approaches. Hence, the differences in the data populations of the two regions we considered are significant enough to render our topography-dependent models noncompetitive. 





\section{Conclusion and Future Works}\label{conc} 
The impact of poisoning attacks on neural power system optimizers has been understudied compared to that of evasion attacks. Through this work, we have focused on this gap by implementing poisoning attacks on the latest ML--based optimization proxies focused on the Optimal Power Flow problem. We measured the adverse impact of data poisoning attacks on the optimality and feasibility of the solutions output by ML-based OPF solvers. We have also identified the impacts of different ML-based solver structures and their resilience to poisoning attacks.

We aim to expand this work on multiple fronts. First, we plan to implement poisoning attacks for AC versions of the different ML--based methods. It would also be beneficial to determine the impact of the Poisoning Attack on other methods with different structures or even with some adversarial samples used in training. Additionally, we will implement classical methods to improve the robustness of NN against adversarial attacks, such as adversarial training. %This can help explore ways to make these methods more secure. The understanding from this work and potential future work make it easier to build resilient ML in the future for use in power systems operation. 


%\section*{Acknowledgment}



\bibliographystyle{IEEEtran}
\bibliography{TPEC22}

\end{document}
