\documentclass[conference]{IEEEtran}
\usepackage{multirow}
\usepackage[table]{xcolor}
\usepackage[numbers]{natbib}
\usepackage{multicol}
\usepackage[bookmarks=true]{hyperref}
\usepackage{amsmath,amsfonts,amsthm,amssymb}
% \usepackage{bbm}
\usepackage{dsfont}
\allowdisplaybreaks
\usepackage[caption=false,font=footnotesize,labelfont=sf,textfont=sf]{subfig}
\usepackage[T1]{fontenc}
% T1 fonts will be used to generate the final print and online PDFs,
% so please use T1 fonts in your manuscript whenever possible.
% Other font encondings may result in incorrect characters.
%
%\usepackage{mathtools}
\usepackage{graphicx}
% Used for displaying a sample figure. If possible, figure files should
% be included in EPS format.


%\usepackage[table]{xcolor}
%\usepackage{cancel}
%\usepackage{textcomp}
%\usepackage{gensymb}
\usepackage{nicematrix}
%\usepackage{mathtools} % for '\smashoperator' macro
\usepackage{float} 
\graphicspath{{Images/}} %configuring the graphicx package
\usepackage{calc} % To reset the counter in the document after title page
\let\labelindent\relax
\usepackage{enumitem} % Includes lists
\usepackage{xspace}
%\usepackage{xfrac}
%\usepackage{subcaption}
\usepackage{cleveref}
\usepackage{bm}
%\usepackage{bbold}
\usepackage{tabularx}
\usepackage{multirow}
% \usepackage[ruled,linesnumbered]{algorithm2e}\RestyleAlgo{ruled}
% \newcommand\mycommfont[1]{\footnotesize\ttfamily\textcolor{blue}{#1}}
% \SetCommentSty{mycommfont}% TEST
% \SetKw{Return}{return}
% \SetKw{Continue}{continue}

% \usepackage{etoolbox}
% \makeatletter
% \patchcmd{\@makecaption}
%   {\scshape}
%   {}
%   {}
%   {}
% \makeatletter
% \patchcmd{\@makecaption}
%   {\\}
%   {.\ }
%   {}
%   {}
% \makeatother
% \def\tablename{Table}

\begin{document}
%
\title{Effective Sampling for Robot Motion Planning 
Through the Lens of Lattices}

% You will get a Paper-ID when submitting a pdf file to the conference system

\author{\authorblockN{Itai Panasoff and Kiril Solovey}
\authorblockA{Viterbi Faculty of Electrical and Computer Engineering\\
Technion--Israel Institute of Technology, Haifa, Israel\\
itaip@campus.technion.ac.il, kirilsol@technion.ac.il}
}


\maketitle


%
\setlength\unitlength{1mm}
\newcommand{\twodots}{\mathinner {\ldotp \ldotp}}
% bb font symbols
\newcommand{\Rho}{\mathrm{P}}
\newcommand{\Tau}{\mathrm{T}}

\newfont{\bbb}{msbm10 scaled 700}
\newcommand{\CCC}{\mbox{\bbb C}}

\newfont{\bb}{msbm10 scaled 1100}
\newcommand{\CC}{\mbox{\bb C}}
\newcommand{\PP}{\mbox{\bb P}}
\newcommand{\RR}{\mbox{\bb R}}
\newcommand{\QQ}{\mbox{\bb Q}}
\newcommand{\ZZ}{\mbox{\bb Z}}
\newcommand{\FF}{\mbox{\bb F}}
\newcommand{\GG}{\mbox{\bb G}}
\newcommand{\EE}{\mbox{\bb E}}
\newcommand{\NN}{\mbox{\bb N}}
\newcommand{\KK}{\mbox{\bb K}}
\newcommand{\HH}{\mbox{\bb H}}
\newcommand{\SSS}{\mbox{\bb S}}
\newcommand{\UU}{\mbox{\bb U}}
\newcommand{\VV}{\mbox{\bb V}}


\newcommand{\yy}{\mathbbm{y}}
\newcommand{\xx}{\mathbbm{x}}
\newcommand{\zz}{\mathbbm{z}}
\newcommand{\sss}{\mathbbm{s}}
\newcommand{\rr}{\mathbbm{r}}
\newcommand{\pp}{\mathbbm{p}}
\newcommand{\qq}{\mathbbm{q}}
\newcommand{\ww}{\mathbbm{w}}
\newcommand{\hh}{\mathbbm{h}}
\newcommand{\vvv}{\mathbbm{v}}

% Vectors

\newcommand{\av}{{\bf a}}
\newcommand{\bv}{{\bf b}}
\newcommand{\cv}{{\bf c}}
\newcommand{\dv}{{\bf d}}
\newcommand{\ev}{{\bf e}}
\newcommand{\fv}{{\bf f}}
\newcommand{\gv}{{\bf g}}
\newcommand{\hv}{{\bf h}}
\newcommand{\iv}{{\bf i}}
\newcommand{\jv}{{\bf j}}
\newcommand{\kv}{{\bf k}}
\newcommand{\lv}{{\bf l}}
\newcommand{\mv}{{\bf m}}
\newcommand{\nv}{{\bf n}}
\newcommand{\ov}{{\bf o}}
\newcommand{\pv}{{\bf p}}
\newcommand{\qv}{{\bf q}}
\newcommand{\rv}{{\bf r}}
\newcommand{\sv}{{\bf s}}
\newcommand{\tv}{{\bf t}}
\newcommand{\uv}{{\bf u}}
\newcommand{\wv}{{\bf w}}
\newcommand{\vv}{{\bf v}}
\newcommand{\xv}{{\bf x}}
\newcommand{\yv}{{\bf y}}
\newcommand{\zv}{{\bf z}}
\newcommand{\zerov}{{\bf 0}}
\newcommand{\onev}{{\bf 1}}

% Matrices

\newcommand{\Am}{{\bf A}}
\newcommand{\Bm}{{\bf B}}
\newcommand{\Cm}{{\bf C}}
\newcommand{\Dm}{{\bf D}}
\newcommand{\Em}{{\bf E}}
\newcommand{\Fm}{{\bf F}}
\newcommand{\Gm}{{\bf G}}
\newcommand{\Hm}{{\bf H}}
\newcommand{\Id}{{\bf I}}
\newcommand{\Jm}{{\bf J}}
\newcommand{\Km}{{\bf K}}
\newcommand{\Lm}{{\bf L}}
\newcommand{\Mm}{{\bf M}}
\newcommand{\Nm}{{\bf N}}
\newcommand{\Om}{{\bf O}}
\newcommand{\Pm}{{\bf P}}
\newcommand{\Qm}{{\bf Q}}
\newcommand{\Rm}{{\bf R}}
\newcommand{\Sm}{{\bf S}}
\newcommand{\Tm}{{\bf T}}
\newcommand{\Um}{{\bf U}}
\newcommand{\Wm}{{\bf W}}
\newcommand{\Vm}{{\bf V}}
\newcommand{\Xm}{{\bf X}}
\newcommand{\Ym}{{\bf Y}}
\newcommand{\Zm}{{\bf Z}}

% Calligraphic

\newcommand{\Ac}{{\cal A}}
\newcommand{\Bc}{{\cal B}}
\newcommand{\Cc}{{\cal C}}
\newcommand{\Dc}{{\cal D}}
\newcommand{\Ec}{{\cal E}}
\newcommand{\Fc}{{\cal F}}
\newcommand{\Gc}{{\cal G}}
\newcommand{\Hc}{{\cal H}}
\newcommand{\Ic}{{\cal I}}
\newcommand{\Jc}{{\cal J}}
\newcommand{\Kc}{{\cal K}}
\newcommand{\Lc}{{\cal L}}
\newcommand{\Mc}{{\cal M}}
\newcommand{\Nc}{{\cal N}}
\newcommand{\nc}{{\cal n}}
\newcommand{\Oc}{{\cal O}}
\newcommand{\Pc}{{\cal P}}
\newcommand{\Qc}{{\cal Q}}
\newcommand{\Rc}{{\cal R}}
\newcommand{\Sc}{{\cal S}}
\newcommand{\Tc}{{\cal T}}
\newcommand{\Uc}{{\cal U}}
\newcommand{\Wc}{{\cal W}}
\newcommand{\Vc}{{\cal V}}
\newcommand{\Xc}{{\cal X}}
\newcommand{\Yc}{{\cal Y}}
\newcommand{\Zc}{{\cal Z}}

% Bold greek letters

\newcommand{\alphav}{\hbox{\boldmath$\alpha$}}
\newcommand{\betav}{\hbox{\boldmath$\beta$}}
\newcommand{\gammav}{\hbox{\boldmath$\gamma$}}
\newcommand{\deltav}{\hbox{\boldmath$\delta$}}
\newcommand{\etav}{\hbox{\boldmath$\eta$}}
\newcommand{\lambdav}{\hbox{\boldmath$\lambda$}}
\newcommand{\epsilonv}{\hbox{\boldmath$\epsilon$}}
\newcommand{\nuv}{\hbox{\boldmath$\nu$}}
\newcommand{\muv}{\hbox{\boldmath$\mu$}}
\newcommand{\zetav}{\hbox{\boldmath$\zeta$}}
\newcommand{\phiv}{\hbox{\boldmath$\phi$}}
\newcommand{\psiv}{\hbox{\boldmath$\psi$}}
\newcommand{\thetav}{\hbox{\boldmath$\theta$}}
\newcommand{\tauv}{\hbox{\boldmath$\tau$}}
\newcommand{\omegav}{\hbox{\boldmath$\omega$}}
\newcommand{\xiv}{\hbox{\boldmath$\xi$}}
\newcommand{\sigmav}{\hbox{\boldmath$\sigma$}}
\newcommand{\piv}{\hbox{\boldmath$\pi$}}
\newcommand{\rhov}{\hbox{\boldmath$\rho$}}
\newcommand{\upsilonv}{\hbox{\boldmath$\upsilon$}}

\newcommand{\Gammam}{\hbox{\boldmath$\Gamma$}}
\newcommand{\Lambdam}{\hbox{\boldmath$\Lambda$}}
\newcommand{\Deltam}{\hbox{\boldmath$\Delta$}}
\newcommand{\Sigmam}{\hbox{\boldmath$\Sigma$}}
\newcommand{\Phim}{\hbox{\boldmath$\Phi$}}
\newcommand{\Pim}{\hbox{\boldmath$\Pi$}}
\newcommand{\Psim}{\hbox{\boldmath$\Psi$}}
\newcommand{\Thetam}{\hbox{\boldmath$\Theta$}}
\newcommand{\Omegam}{\hbox{\boldmath$\Omega$}}
\newcommand{\Xim}{\hbox{\boldmath$\Xi$}}


% Sans Serif small case

\newcommand{\Gsf}{{\sf G}}

\newcommand{\asf}{{\sf a}}
\newcommand{\bsf}{{\sf b}}
\newcommand{\csf}{{\sf c}}
\newcommand{\dsf}{{\sf d}}
\newcommand{\esf}{{\sf e}}
\newcommand{\fsf}{{\sf f}}
\newcommand{\gsf}{{\sf g}}
\newcommand{\hsf}{{\sf h}}
\newcommand{\isf}{{\sf i}}
\newcommand{\jsf}{{\sf j}}
\newcommand{\ksf}{{\sf k}}
\newcommand{\lsf}{{\sf l}}
\newcommand{\msf}{{\sf m}}
\newcommand{\nsf}{{\sf n}}
\newcommand{\osf}{{\sf o}}
\newcommand{\psf}{{\sf p}}
\newcommand{\qsf}{{\sf q}}
\newcommand{\rsf}{{\sf r}}
\newcommand{\ssf}{{\sf s}}
\newcommand{\tsf}{{\sf t}}
\newcommand{\usf}{{\sf u}}
\newcommand{\wsf}{{\sf w}}
\newcommand{\vsf}{{\sf v}}
\newcommand{\xsf}{{\sf x}}
\newcommand{\ysf}{{\sf y}}
\newcommand{\zsf}{{\sf z}}


% mixed symbols

\newcommand{\sinc}{{\hbox{sinc}}}
\newcommand{\diag}{{\hbox{diag}}}
\renewcommand{\det}{{\hbox{det}}}
\newcommand{\trace}{{\hbox{tr}}}
\newcommand{\sign}{{\hbox{sign}}}
\renewcommand{\arg}{{\hbox{arg}}}
\newcommand{\var}{{\hbox{var}}}
\newcommand{\cov}{{\hbox{cov}}}
\newcommand{\Ei}{{\rm E}_{\rm i}}
\renewcommand{\Re}{{\rm Re}}
\renewcommand{\Im}{{\rm Im}}
\newcommand{\eqdef}{\stackrel{\Delta}{=}}
\newcommand{\defines}{{\,\,\stackrel{\scriptscriptstyle \bigtriangleup}{=}\,\,}}
\newcommand{\<}{\left\langle}
\renewcommand{\>}{\right\rangle}
\newcommand{\herm}{{\sf H}}
\newcommand{\trasp}{{\sf T}}
\newcommand{\transp}{{\sf T}}
\renewcommand{\vec}{{\rm vec}}
\newcommand{\Psf}{{\sf P}}
\newcommand{\SINR}{{\sf SINR}}
\newcommand{\SNR}{{\sf SNR}}
\newcommand{\MMSE}{{\sf MMSE}}
\newcommand{\REF}{{\RED [REF]}}

% Markov chain
\usepackage{stmaryrd} % for \mkv 
\newcommand{\mkv}{-\!\!\!\!\minuso\!\!\!\!-}

% Colors

\newcommand{\RED}{\color[rgb]{1.00,0.10,0.10}}
\newcommand{\BLUE}{\color[rgb]{0,0,0.90}}
\newcommand{\GREEN}{\color[rgb]{0,0.80,0.20}}

%%%%%%%%%%%%%%%%%%%%%%%%%%%%%%%%%%%%%%%%%%
\usepackage{hyperref}
\hypersetup{
    bookmarks=true,         % show bookmarks bar?
    unicode=false,          % non-Latin characters in AcrobatÕs bookmarks
    pdftoolbar=true,        % show AcrobatÕs toolbar?
    pdfmenubar=true,        % show AcrobatÕs menu?
    pdffitwindow=false,     % window fit to page when opened
    pdfstartview={FitH},    % fits the width of the page to the window
%    pdftitle={My title},    % title
%    pdfauthor={Author},     % author
%    pdfsubject={Subject},   % subject of the document
%    pdfcreator={Creator},   % creator of the document
%    pdfproducer={Producer}, % producer of the document
%    pdfkeywords={keyword1} {key2} {key3}, % list of keywords
    pdfnewwindow=true,      % links in new window
    colorlinks=true,       % false: boxed links; true: colored links
    linkcolor=red,          % color of internal links (change box color with linkbordercolor)
    citecolor=green,        % color of links to bibliography
    filecolor=blue,      % color of file links
    urlcolor=blue           % color of external links
}
%%%%%%%%%%%%%%%%%%%%%%%%%%%%%%%%%%%%%%%%%%%



\begin{abstract}
Sampling-based methods for motion planning, which capture the structure of the robot's free space via (typically random) sampling, have gained popularity due to their scalability, simplicity, and for offering global guarantees, such as probabilistic completeness and asymptotic optimality. Unfortunately, the practicality of those guarantees remains limited as they do not provide insights into the behavior of motion planners for a finite number of samples (i.e., a finite running time). In this work, we harness lattice theory and the concept of $\bm{(\delta,\eps)}$-completeness by Tsao et al.~(2020) to construct deterministic sample sets that endow their planners with strong finite-time guarantees while minimizing running time. In particular, we introduce a highly-efficient deterministic sampling approach based on the $\bm{A_d^*}$ lattice, which 
is the best-known geometric covering in dimensions $\bm{\leq 21}$. Using our new sampling approach, we obtain at least an order-of-magnitude speedup over existing deterministic and uniform random sampling methods for complex motion-planning problems. Overall, our work provides deep mathematical insights while advancing the practical applicability of sampling-based motion planning.
\end{abstract}
%

\IEEEpeerreviewmaketitle

\begin{figure}
    \centering
    \begin{tikzpicture}[font=\footnotesize]
        \node (img) {\includegraphics[width=0.7\columnwidth]{figpaper/nfe_vs_fvd_vs_ep_ffs_teaser.pdf}};
            \node[anchor=north west, xshift=25pt, yshift=-5pt] at (img.north west) {
                \begin{tabular}{ll}
                \scriptsize
                    \textcolor[HTML]{A0A0A0}{\rule{6pt}{6pt}} &Rolling Diffusion \cite{ruhe2024rollingdiffusionmodels} \\
                    \textcolor[HTML]{e9cbc4}{\rule{6pt}{6pt}} &Diffusion Forcing \cite{chen2024diffusionforcing} \\
                    \textcolor[HTML]{F4A700}{\rule{6pt}{6pt}} &MaskFlow (\textit{Ours})
                \end{tabular}
            };
    \end{tikzpicture}
    \vspace{-7pt}
    \caption{\textbf{Our method (MaskFlow) improves video quality compared to baselines while simultaneously requiring fewer function evaluations (NFE)} when generating videos $2\times$, $5\times$, and $10\times$ longer than the training window.
}
    \label{fig:teaser}
    \vspace{-10pt}
\end{figure}

\section{Introduction}

Due to the high computational demands of both training and sampling processes, long video generation remains a challenging task in computer vision. Many recent state-of-the-art video generation approaches train on fixed sequence lengths \cite{blattmann2023stable,blattmann2023align_videoldm,ho2022video} and thus struggle to scale to longer sampling horizons. Many use cases not only require long video generation, but also require the ability to generate videos with varying length. A common way to address this is by adopting an autoregressive diffusion approach similar to LLMs \cite{gao2024vid}, where videos are generated frame by frame. This has other downsides, since it requires traversing the entire denoising chain for every frame individually, which is computationally expensive. Since autoregressive models condition the generative process recursively on previously generated frames, error accumulation, specifically when rolling out to videos longer than the training videos, is another challenge.
\par
Several recent works \cite{ruhe2024rollingdiffusionmodels, chen2024diffusionforcing} have attempted to unify the flexibility of autoregressive generation approaches with the advantages of full sequence generation. These approaches are built on the intuition that the data corruption process in diffusion models can serve as an intermediary for injecting temporal inductive bias. Progressively increasing noise schedules \cite{xie2024progressive,ruhe2024rollingdiffusionmodels} are an example of a sampling schedule enabled by this paradigm. These works impose monotonically increasing noise schedules w.r.t. frame position in the window during training, limiting their flexibility in interpolating between fully autoregressive, frame-by-frame generation and full-sequence generation. This is alleviated in \cite{chen2024diffusionforcing}, where independent, uniformly sampled noise levels are applied to frames during training, and the diffusion model is trained to denoise arbitrary sequences of noisy frames. All of these works use continuous representations.
\par
We transfer this idea to a discrete token space for two main reasons: First, it allows us to use a masking-based data corruption process, which enables confidence-based heuristic sampling that drastically speeds up the generative process. This becomes especially relevant when considering frame-by-frame autoregressive generation. Second, it allows us to use discrete flow matching dynamics, which provide a more flexible design space and the ability to further increase our sampling speed. Specifically, we adopt a \emph{frame-level masking} scheme in training (versus a \emph{constant-level masking} baseline, see Figure~\ref{fig:training}), which allows us to condition on an arbitrary number of previously generated frames while still being consistent with the training task. This makes our method inherently versatile, allowing us to generate videos using both full-sequence and autoregressive frame-by-frame generation, and use different sampling modes. We show that confidence-based masked generative model (MGM) style sampling is uniquely suited to this setting, generating high-quality results with a low number of function evaluations (NFE), and does not degrade quality compared to diffusion-like flow matching (FM)-style sampling that uses larger NFE. 
Combining frame-level masking during training with MGM-style sampling enables highly efficient long-horizon rollouts of our video generation models beyond $10 \times$ training frame lengths without degradation. We also demonstrate that this sampling method can be applied in a timestep-\emph{independent} setting that omits explicit timestep conditioning, even when models were trained in a timestep-dependent manner, which further underlines the flexibility of our approach. In summary, our contributions are the following:

\begin{itemize}
    \item To the best of our knowledge, we are the first to unify the paradigms of discrete representations in video flow matching with rolling out generative models to generate arbitrary-length videos. 
    \item We introduce MaskFlow, a frame-level masking approach that supports highly flexible sampling methods in a single unified model architecture.
    \item We demonstrate that MaskFlow with MGM-style sampling generates long videos faster while simultaneously preserving high visual quality (as shown in Figure~\ref{fig:teaser}).
    \item Additionally, we demonstrate an additional increase in quality when using full autoregressive generation or partial context guidance combined with MaskFlow for very long sampling horizons.
    \item We show that we can apply MaskFlow to both timestep-dependent and timestep-independent model backbones without re-training.
\end{itemize}

\begin{figure}
    \centering
    \includegraphics[width=0.75\linewidth]{figpaper/training.pdf}
    \caption{\textbf{MaskFlow Training:} For each video, Baseline training applies a single masking ratios to all frames, whereas our method samples masking ratios independently for each frame.}
    \vspace{-10pt}
    \label{fig:training}
\end{figure}

















\section{Preliminaries}\label{sec:preliminaries}
%We provide basic definitions regarding motion planning, sampling-based planning, and $(\delta,\epsilon)$-completeness. 
The motion-planning problem concerns computing a collision-free path for a robot in an environment cluttered with obstacles. We concider a holonomic robot with a configuration space $\C = \dR^d$. %(we assume that $\C$ is of full dimensions, i.e., not a manifold). For simplicity, we set $\C=\dR^d$ throughout the paper, as our lattice-based sample sets are infinite. 
The dimension $d\geq 2$ represents the DoF and is finite. A motion planning problem is a tuple $\M:= (\C_f, q_{s},  q_{g})$, where $\C_f\subseteq \C$ is the free space (the set of collision-free configurations), and $q_s,q_g\in \C_f$ are the start and goal configurations, respectively. A solution for $\M$ is a continuous collision-free path $\pi:[0,1]\to\C_f$ that begins at $\pi(0) = q_{s}$ and ends at $\pi(1) = q_g$.

Two critical properties of a given path $\pi$ for a problem $\M= (\C_f, q_{s},  q_{g})$, are its length $\ell(\pi)\geq 0$, and its clearance. For a given value $\delta\geq 0$, we say that the path $\pi$ is $\delta$-clear if $\bigcup_{0 \leq t \leq 1}\B_{\delta}(\pi(t)) \subseteq \C_f$, where $\B_\rho(p)$ is the $d$-dimensional closed Euclidean ball with radius $\rho>0$ centered at $p\in\mathbb{R}^d$. We denote $\B_\rho:=\B_\rho(o)$, where $o$ is the origin of $\dR^d$.

\subsection{Probabilistic roadmaps and completeness}
We present a formal definition of the Probabilistic Roadmap (PRM) method \cite{kavraki1996probabilistic}, which constructs a discrete graph capturing the connectivity of $\C_f$ through sampling. Albeit sampling usually refers to a randomized process, here we consider deterministic sampling, as was recently done in~\cite{tsao2020sample,dayan2023near}.\footnote{With deterministic sampling, the term ``probabilistic'' in PRM can seem misleading. Nevertheless, we choose to stick to PRM considering its popularity and the underlying graph structure it represents, which is a key to our analysis.} We emphasize that our analysis below is not confined to PRMs, and applies to various PRM-based planners, as mentioned above. 

For a given motion planning problem $\M = (\C_f,q_s,q_g)$, a sample (point) set $\X\subset \C$, and a connection radius $r>0$, PRM generates a graph denoted by $G_{\mathcal{M}(\X, r)} = (V,E)$. The vertex set $V$ consists of all collision-free configurations in $\X \cup \{ q_s, q_g\}$. The set of undirected edges, $E$, consists of all the vertex pairs $u, v\in V$ such that the Euclidean distance between them is at most $r$, and the straight-line segment $\overline{uv}$ between them is collision-free. That is, 
\begin{align*}
V := & (\X\cup \{ q_s, q_g\}) \cap \C_f, \\
E := & \left\{ \{v, u\} \in V\times V : \lVert v-u\rVert\leq r, \overline{uv} \subset \C_f \right\}.
\end{align*}

In this work, we are interested in obtaining sample sets and connection radii for PRM that achieve a desired solution quality in terms of path length. Unlike most theoretical results for SBP, which consider asymptotic guarantees, here we rely on a stronger deterministic notion.

\begin{definition}[($\delta,\epsilon$)-completeness~\cite{tsao2020sample}] Given a sample set $\X\subset \C$ and connection radius $r>0$, the pair $(\X, r)$ is ($\delta,\epsilon$)-complete for a clearance parameter $\delta>0$ and stretch factor $\epsilon>0$, if for \emph{every} $\delta$-clear problem $\mathcal{M}=(\C_f, q_s, q_g)$, the  graph $G_{\mathcal{M}(\X, r)}$ contains a path from $q_s$ to $q_g$ with length at most $(1+\eps)$ times the optimal $\delta$-clear length, denoted by $\text{OPT}_\delta$. That is, it holds that
\[
    \ell(G_{\mathcal{M}(\X, r)},q_s, q_g)\leq (1+\epsilon)\text{OPT}_\delta,
\]
where $\ell(G_{\mathcal{M}(\X, r)},q_s, q_g)$ denotes the length of the shortest path from $q_s$ to $q_g$ in the graph $G_{\mathcal{M}(\X, r)}$.
\end{definition}

The property of ($\delta,\epsilon$)-completeness has several key advantages over asymptotic notions, such as PC and AO. First, there exists (with probability $1$) a \emph{finite} sample set $\X$ and radius $r\in (0,\infty)$  that jointly guarantee ($\delta,\epsilon$)-completeness. Second, if a solution is not found using a \decomp pair $(\X,r)$, then no $\delta$-clear solution exists. That is, ($\delta,\epsilon$)-completeness can be used for deterministic infeasibility proofs~\cite{li2023sampling}. Third, the computational complexity of constructing a PRM graph can be tuned according to the desired values of $\delta$ and $\epsilon$. %In contrast, PC and AO properties provide no meaningful guarantees for a finite number of samples. Thus, if no solution is found using randomized samples, it can be difficult to determine whether no solution exists. Lastly, it is usually more difficult to tune randomized samples according to user specifications of desired solution quality and clearance. 



%An important thing to note, is that we run into a problem many algorithms run into: will this algorithm end with a result? if so, in how much time? General MP problems are able to solve this problem using a recently developed infeasibility-checking parallel-running algorithm (see Li and Dantam, 2023~\cite{li2023sampling}). Using learning and topological methods they are able to get an answer to the question: "Will we ever get a path from $q_s$ to $q_g$?".


%The problem is, that this answer comes at an asymptotically-assured time, not in a finite time. In our $(\delta,\epsilon)$-complete setting, if we build a sample set in a way that is guaranteed to possess this property---we are guaranteed to get a $\delta$-clear path, if one exists, in finite time. If we didn't get a path, it means that such a path doesn't exist. This serves as a sort of "infeasibility proof" for sample sets in the $(\delta,\epsilon)$-complete setting.


\subsection{Problem definition}
A $(\delta,\epsilon)$-complete sample set $\X$ and radius $r>0$ can be obtained by constructing a so-called $\beta$-cover~\cite{dayan2023near}. 

\begin{definition}
For a given $\beta>0$, a sample set $\X\subset \dR^d$ is a $\beta$-cover if for every point $p\in \C=\dR^d$ there exists a sample $x \in \X$ such that $\|p-x\|\leq \beta$.
\end{definition}

Note that the coverage property above is defined with respect to the whole configuration space, rather than a specific free space. The connection between $(\delta,\epsilon)$-completeness and $\beta$-cover is established in the following lemma.
\begin{lemma}[Completeness-cover relation~\cite{tsao2020sample}]\label{lem:cover}
  Fix $\delta >0$ and $\epsilon>0$. Suppose that a sample set $\X$ is a ${\beta^*}$-cover, where
  \begin{equation}
    {\beta^*}={\beta^*}(\delta,\epsilon):=\frac{\delta\epsilon}{\sqrt{1+\epsilon^2}}.
  \end{equation}
  Then $(\X,{r^*})$ is $(\delta,\epsilon)$-complete, where
  \begin{equation}
      {r^*}={r^*}(\delta,\epsilon):=\frac{2\delta(1+\epsilon)}{\sqrt{1+\epsilon^2}}.
  \end{equation}
\end{lemma}

%In the above lemma, the ${\beta^*}$-cover is with respect to the full configuration space $\C$, rather than a specific free space $\C_f$. This is because the $(\delta,\epsilon)$-completeness needs to be satisfied for any free space. 
The lemma prescribes an approach for constructing sample sets and connection radii that satisfy the $(\delta,\epsilon)$-completeness requirement. %When $\epsilon\rightarrow\infty$ (i.e. "any solution"), ${\beta^*}=\delta$ and $r=2\delta$---which makes sense, we want a $\delta$-clear path and a $2\delta$ connection radius ensures the PC property.
Considering that we will fix the radius $r^*:=r^*(\delta,\epsilon)$ throughout this work, we will say that a sample set $\X$ is $(\delta,\epsilon)$-complete if the pair $(\X,r^*)$ is $(\delta,\epsilon)$-complete.

\Cref{lem:cover} still leaves a critical unresolved question: How do we find a \decomp sample $\X$ minimizing the computation time of the PRM graph and subsequent methods? In this work, we are interested in the following two related problems. The first deals with finding a sample set of minimal size, which can be used as a proxy for computation time. 

\begin{problem}[Sample complexity]\label{problem:sample}
  For a given $\delta>0,\epsilon>0$, find a $(\delta,\epsilon)$-complete sample set $\X$ of  minimal \emph{sample complexity}, i.e., $|\X\cap \B_{r^*}|$. 
\end{problem}

Previous work~\cite{tsao2020sample,dayan2023near} has considered a slightly different notion for sample complexity aiming to minimize the global expression $|\X\cap \C|$ with a bounded $\C$. We believe that our local notion, within an ${r^*}$-ball, is more informative and optimistic, as typical problems have obstacles, and the solution lies in a small subset of the search space. Furthermore, it would allow us to obtain tighter analyses by exploiting methods from discrete geometry that reason about ball structures.

Previous work has introduced several sample sets and analyzed their sample complexity~\cite{tsao2020sample,dayan2023near}. In this work, we consider more compact sets. Moreover, we introduce a new notion that better captures the computational complexity of constructing a PRM graph. In particular, we leverage the observation that the computational complexity of sampling-based planning is typically dominated by the amount of collision checks performed~\cite{KleinbortSH16}. Furthermore, nearest-neighbor search, which is another key contributor to the algorithm's computational complexity, 
can be eliminated for deterministic samples, assuming that they have a regular structure (as for lattice-based samples, which we describe below).

Collision checks are run both on the PRM vertices and edges, where edge checks are usually performed via dense sampling of configurations along the edges and individually validating each configuration. Thus, the total number of collision checks is proportional to the total edge length of the graph. We use this observation to develop a more accurate proxy for computational complexity. In particular, we will estimate the length of edges adjacent to the origin point $o \in \dR^d$, which is a vertex in all the sample sets introduced below. Moreover, due to the regularity of the sets, the attribute below is equal across all vertices (in the absence of obstacles).

\begin{problem}[Collision-check complexity]\label{problem:collision}
  For given \mbox{$\delta>0,\epsilon>0$}, find a $(\delta,\epsilon)$-complete sample set $\X$ of  minimal \emph{collision-check complexity}, i.e., minimizing the expression
  \[
      CC_\X:=\sum_{x\in \X\cap \B{r^*}} \|x\|.
  \]
\end{problem}

%To our knowledge, our work is the first to consider this notion to estimate the computational cost of sampling-based algorithms.

% Our main goal in this paper will be to research the sample and collision complexity of sample sets. Formally:
% \begin{definition}[Sample and Collision Complexity]
%     For a set $\chi\subset\mathbb{R}^d$ which is $(\delta, \epsilon)$-complete, we define:
%     \begin{enumerate}
%         \item \textbf{Sample Complexity}: a function $f=f(n,\delta,\epsilon)$ for which:
%         \[
%             \|\chi_A\|\triangleq\|\chi\cap A\|\leq k\cdot f(n,\delta,\epsilon)
%         \]
%         for some $A\subset\mathbb{R}^d,k>0,n>n_0,\delta<\delta_0,\epsilon<\epsilon_0$.
%         \item \textbf{Collision Complexity}: a function $g=g(n,\delta,\epsilon)$ for which
%         \[
%             I_{\chi,A}(r)\triangleq\sum_{p\in \chi_A}\|p\|^2\leq k\cdot g(n,\delta,\epsilon) 
%         \]
%         for some $A\subset\mathbb{R}^d,r>r_0,k>0,n>n_0,\delta<\delta_0,\epsilon<\epsilon_0$.
%     \end{enumerate}
% \end{definition}


% Let us start by defining the motion planning problem. In terms of the \emph{workspace}, the physical 3D space the robot resides in, the problem deals with navigating a set of points (together referred to as the robot) around a space filled with obstacles. 
% Now, because of how difficult it is to make sure every part of the robot is in obstacle-free space at any point, it is beneficial to consider what is called the \emph{configuration space} (denoted by the letter $C$)---a parameterization of the robot's position in space. A basic example would be a two-link manipulator arm. This manipulator has two degrees of freedom, and can be described using each link's relative angle: $(\theta_1,\theta_2)$. Thus, the configuration space of this robot is a torus $T^2=S^1\times S^1$.

% Using the configuration space allows us to consider a \emph{single point} $c\in C$ as the robot, thus a planning problem becomes conceptually simpler:

% \begin{definition}[Free Space]  

% \end{definition}

% \begin{definition}[Planning Phase]

% \end{definition}

% \noindent Using these definitions, the motion planning problem is, formally:
% \begin{definition}[Motion planning problem]
%     The MP problem is a tuple $M=(C_f, x_s, x_g)$, where $C_f$ is the free space and $\{x_s,x_g\}$ are the start and goal configurations. The problem $M$ is then solved at the aforementioned planning-phase.
% \end{definition}

% \textit{The prominent algorithmic framework} for solving motion planning problems for more than 20 years is \emph{Probability Roadmap} (or PRM)~\cite{kavraki1996probabilistic}. PRM (and, generally, PRM-based algorithms) build a graph using sample points and perform search algorithms on it to get a path. It follows these general steps: 
% \begin{enumerate}
%   \item \underline{Points:} Sample the configuration space, using some sampling scheme (e.g., randomly).
%   \item Perform \emph{Collision Detection} (CD) and remove configurations that are in collision.
%   \item \underline{Edges:} Connect sampled configurations. For example, using \emph{Nearest-Neighbor} methods (NN): $k$-NN (choosing the $k>0$ nearest neighbors) or ${r^*}$-NN (choosing all neighbors within a radius $r>0$).  
%   \item Perform CD on the connected configurations: for example, by checking dense intervals on the connected edge for collision of specific configurations. Remove edges that have collision.
%   \item \underline{Graph Search:} Utilize a search algorithm to obtain a path from the graph induced by the aforementioned points and edges. \emph{A*} is a classic search algorithm that takes the sample points, and “estimates” the distance from each sample point to the goal. It searches the graph, keeping track of a “score”: $current(v) = so\_far(v) + estimate(v)$, this way receiving an optimal path at the end---with the main difficulty being calculating a good estimate() function (this was expanded to the multiple-robot setting as M* \cite{wagner2015subdimensional}).
% \end{enumerate}
% With steps (1) to (4), we end up with the following formal definition of a PRM graph, as notated by Dayan et al.~\cite{dayan2023near}:
% \begin{definition}[PRM-Graph]
%     Let $\chi\subset C_f$ a sample set, $\{x_s, x_g\} \subset C_f$ start and goal samples and $r>0$ connection radius. Define the PRM graph to be $G_{M(\chi,r)}=(V,E)$ for:
%     \begin{align*}
%         V\triangleq & (\chi\cup\{x_s,x_g\})\cap C_f \\
%         E\triangleq & \{u,v\in V | \|u-v\|\leq r, [u,v]\subset C_f\}
%     \end{align*}
% \end{definition}
% PRM was proven~\cite{ladd2002generalizing} to be \emph{probabilistically-complete} (PC) --- meaning that with a number of samples tending to infinity, it would find some solution path with a probability tending to 1. It was later expanded to PRM*~\cite{karaman2011sampling} - an \emph{asymptotically-optimal} (AO) version of PRM, meaning that this time with a number of samples tending to infinity, it would find an \textbf{optimal} (length-wise) solution path with a probability tending to 1.

% Our paper will deal specific sample sets on PRM graphs, ones that posses a unique feature that is important in robotics (definition taken from~\cite{dayan2023near}):
% \begin{definition}[$(\delta, \epsilon)$-complete sets]
%     Given a sample set $\chi$ and a connection radius $r>0$, we say that $(\chi,r)$ is $(\delta,\epsilon)$-complete for some stretch factor $\epsilon>0$ and clearance factor $\delta>0$ if for every $\delta$-clear $M=(C_f,x_s,x_g)$ it holds that:
%     \begin{align*}
%         shortest\_path(G_{M(\chi,r)},x_s,x_g)\leq (1+\epsilon)OPT_{\delta}
%     \end{align*}    
% \end{definition}
% From the same paper, let us define the following concept:
% \begin{definition}[${\beta^*}$-covers]
%     Let $\delta>0$ be the clearance factor, and $\epsilon>0$ be the stretch factor. Then ${\beta^*}=\frac{\delta\epsilon}{\sqrt{1+\epsilon^2}}$ is the radius taken around samples, used to create a $(\delta, \epsilon)$-complete set. We define the balls that participate in the cover as \emph{${\beta^*}$-balls}.
% \end{definition}
%%% Local Variables:
%%% mode: latex
%%% TeX-master: "../main"
%%% End:

\section{Lattice-based sample sets}\label{sec:lattices}
We derive sample sets optimizing sample complexity (Problem~\ref{problem:sample}) and collision-check complexity (Problem~\ref{problem:collision}) both in theory and experiments. We focus on sample sets induced by lattices. %In this section, we review basic definitions of lattices, describe three families of lattices that are of interest, and derive \decomp sample sets using them.

A lattice is a point set in Euclidean space with a regular structure~\cite{conway2013sphere}.

\begin{definition}[Lattice]\label{definition:lattice}
  A lattice $\Lambda$ is defined as all the linear combinations (with integer coefficients) of a basis\footnote{A basis can be of full rank ($m=N$) or subdimensional ($m<N$). A basis can be non-unique.} $E_\Lambda=\{e_i\in \dR^N\}_{i=1}^m$ of rank $1\leq m\leq N$, i.e.,
  \[\Lambda:=\left\{\sum_{i=1}^m a_i e_i\middle| a_i\in\mathbb{Z},e_i\in E_\Lambda\right\}.\]
\end{definition}

It would be convenient to view lattices through their generator matrices. 
\begin{definition}[Lattice generator]\label{definition:generator}
    The generator matrix $G_\Lambda$ of a lattice $\Lambda$ with basis $E_\Lambda=\{e_i\in \dR^N\}_{i=1}^m$ is an $m\times N$ matrix such that for every  $1\leq i\leq m$, the row $i$ is equal to $e_i$. Note that 
\(\Lambda=\left\{a\cdot G_\Lambda\middle| a\in\dZ^{1\times m}\right\}.\)
Additionally, define $\det(\Lambda):=\det(G_{\Lambda}G_{\Lambda}^t)$.
\end{definition}

\subsection{Useful lattices}
We describe three lattices, visualized in~\Cref{fig:2d_lattices,fig:3d_lattices}. The first lattice is a simple rectangular grid, which is provided to benchmark more complicated and efficient lattices. Below, we fix the dimension $d\geq 2$.

\begin{definition}[$\dZ^d$ lattice]
   The $\dZ^d$ lattice is defined by the identity generator matrix $I\in \dR^{d\times d}$, with $\det(\ZN)=1$~\cite[p.~106]{conway2013sphere}.
\end{definition}
  


% \begin{figure}[H]
%     \centering
%     \begin{subfigure}[b]{0.49\textwidth}
%         \includegraphics[width=\textwidth]{Images/ZN_2D.png}
%         %\caption{$Z^2$: A normal grid.}
%         %\label{fig:sub1}
%     \end{subfigure}
%     \hfill
%     \begin{subfigure}[b]{0.49\textwidth}
%         \includegraphics[width=\textwidth]{Images/ZN_3D.png}
%         %\caption{$Z^3$: A normal grid.}
%         %\label{fig:sub2}
%     \end{subfigure}
%     \caption{The lattice $\ZN$ in 2,3 dimensions, respectively. A standard "cube" ($[0,w]^d,d=2,3$ for some $w>0$) can be seen tessellating the space. \kiril{Are we referring to this figure when describing the cube? In any case, please explain here what you mean by the cube. Some readers would read the caption of this figure before arriving at the definition of a cube later on.}\itai{added explanation for what I meant}}
%     \label{fig:q2s3zeropointthree}
% \end{figure}

More efficient sample sets can be generated via the $D_d^*$ lattice~\cite[p120]{conway2013sphere}. This lattice was also presented in~\cite{dayan2023near}, where it was called a ``staggered grid''. In this work, we provide improved sample complexity bounds for lattice-based sample sets (including for the $D_d^*$ lattice %, which was analyzed in~\cite{dayan2023near}, 
and the $A_d^*$ lattice defined later on), following~\cite{conway2013sphere}, and develop theoretical bounds for collision-check complexity.

\begin{definition}[$D_d^*$ lattice]
   The $D_d^*$ lattice is defined by the  generator matrix
   \begin{align*}
     G_{\DN}=
     \begin{pmatrix}
         1 & 0 & 0 & \dots & 0 & 0 \\
         0 & 1 & 0 & \dots & 0 & 0 \\
         \vdots & \vdots & \vdots & \ddots & \vdots & 0 \\
         0 & 0 & 0 & \dots & 1 & 0 \\
         \frac{1}{2} & \frac{1}{2} & \frac{1}{2} & \dots & \frac{1}{2} & \frac{1}{2}
     \end{pmatrix}\in \dR^{d\times d}, 
 \end{align*}
 with $\det(\DN)=\frac{1}{4}$~\cite[p.~120]{conway2013sphere}.
 \end{definition}
 
% \begin{figure}[H]
%     \centering
%     \begin{subfigure}[b]{0.49\textwidth}
%         \includegraphics[width=\textwidth]{Images/DN_3D.png}
%        % \caption{$D_2^*$: A.K.A the \emph{staggered grid}.}
%         %\label{fig:sub1}
%     \end{subfigure}
%     \hfill
%     \begin{subfigure}[b]{0.49\textwidth}
%         \includegraphics[width=\textwidth]{Images/DN_3D.png}
%         %\caption{$D_3^*$: the \emph{BCC} structure.}
%         %\label{fig:sub2}
%     \end{subfigure}
%     \caption{The $\DN$ lattice in 2,3 dimensions, respectively. Looks similar to $\ZN$, with A "cube" tessellating the space. Here, though, another sample is added in the middle - covering the space more efficiently. \kiril{same comment as earlier. What are you trying to illustrate with those cubes here? This is a good place to mention that in 2d the this lattice can be viewed as a rotated $\dZ^d$ lattice, but in 3d they diverge, and refer to the Dayan text.}\itai{clarified what I meant also here}}
%     \label{fig:q2s3zeropointthree}
% \end{figure}

\begin{figure}[t]
  \centering
  \subfloat[$\X_{\dZ_3}^{\delta,\epsilon}$ sample set.]{
    \includegraphics[width=0.46\columnwidth,clip]{Images/ZN_3D.png}
    %\label{fig:3d_lattices:z}
    }
  \hfill
  \subfloat[$\X_{D_3^*}^{\delta,\epsilon}=\X_{A_3^*}^{\delta,\epsilon}$ sample sets.]{
    \includegraphics[width=0.46\columnwidth,clip]{Images/AN_3D.png}
    %\label{fig:3d_lattices:da}
    }
  \caption{\decomp sample sets in $\dR^3$ derived from the lattices $\dZ^3, D_3^*$ and $A^*_3$. Note the sets $\X_{D_d^*}^{\delta,\epsilon},\X_{A_d^*}^{\delta,\epsilon}$ coincide for $d=3$, and diverge for $d\geq 4$.
Note that the density of  $\X_{D^*_3}^{\delta,\eps}$ and $\X_{A^*_3}^{\delta,\eps}$ (also known as the Body-Centered Cubic structure in crystallography), and is lower than the density of $\X_{\dZ^3}^{\delta,\eps}$.}
  \label{fig:3d_lattices}
\end{figure}

The following $A^*_d$ lattice~\cite[p115]{conway2013sphere} leads to even more efficient sample sets. This lattice is also called a "hexagonal grid", and was previously used for 2D path planning~\cite{BAILEY2021103560,TengEA17}. This work is the first to consider its application in dimensions $d\geq 3$, and moreover, in the context of \decomps guarantees.

%\yaniv{Was used in low dimensions e.g. https://www.mdpi.com/2220-9964/13/5/166 https://www.mdpi.com/2220-9964/11/4/231 https://www.mdpi.com/2072-4292/13/21/4216 as well as "Path-length analysis for grid-based path planning" and "hexagonal grid-based sampling planner for aquatic environmental monitoring using unmanned surface vehicles" } 

\begin{definition}[$A^*_d$ lattice]
  The $A^*_d$ lattice is defined through the generator matrix \begin{align*}
    G_{\AN}=
    \begin{pmatrix}
        1 & -1 & 0 & 0 & \dots & 0 & 0 \\
        1 & 0 & -1 & 0 & \dots & 0 & 0 \\
        \vdots & \vdots & \vdots & \vdots & \ddots & \vdots & 0 \\
        1 & 0 & 0 & 0 & \dots & -1 & 0 \\
        \frac{-d}{d+1} & \frac{1}{d+1} & \frac{1}{d+1} & \frac{1}{d+1} & \dots & \frac{1}{d+1} & \frac{1}{d+1}
    \end{pmatrix}_{d\times (d+1)}\!\!\!,
\end{align*}
with $\det(\AN)=\frac{1}{\sqrt{d+1}}$~\cite[p.~115]{conway2013sphere}.
\end{definition}

Note that $A^*_d$ is contained in $\mathbb{R}^{d+1}$ (due to the number of rows of the generator matrix), but the lattice itself is $d$-dimensional as it lies in a $d$-dimensional hyperplane (for any lattice point $(x_1,\ldots x_{d+1})\in A^*_d$ it holds that  $\sum_{i=1}^{d+1} x_i=0$). 
Our motivation for considering $\AN$ is its low \emph{density}, defined as the average number of spheres (centered on lattice points)  containing a point of the space~\cite{conway2013sphere}. In particular, $\AN$ is the best lattice covering (and best covering in general) in terms of density for dimension $d\leq 5$ (see~\cite{ryshkov1975solution}) and overall the best \emph{known} covering for $d\leq 21$. %\itai{here is the full quote: "An* is the best covering of Rn for n=2, the best lattice covering for n<=5, and the best covering known for n<=21". This is what the book says. from here, its up to us to interpret it, honestly. I just take it at face value: they proved mathematically it is the best lattice covering for n<=5, and for n<=21 they just aren't aware (at the time of writing the book) of anything better)}

\subsection{From lattices to \decomp sample sets}
We derive sample sets from the lattices above by transforming the lattices such that the resulting point sets lie in $\dR^d$ and form $\beta^*$-covers (Lemma~\ref{lem:cover}). To achieve that, we will leverage the geometry of the lattices and their covering radius, which is defined below.

\begin{definition} (Covering radius~\cite{conway2013sphere})
   For a point set $\X\subset \dR^d$, a covering radius is defined to be
    \[
        f_{\X} = \sup_{y\in\dR^d}\inf_{x\in\X} \|x-y\|.
    \]
\end{definition}
When considering the covering radius of $\AN$, which lies in $\dR^{d+1}$, we will abuse the above definition to refer to its covering radius in the $d$-dimensional plane $\sum_{i=1}^{d+1}x_{d+1}=0$. Note that in order to cover $\dR^d$ with balls of radius $\rho>0$ centered at the points of a set $\X$, it must hold that $\rho\geq f_{\X}$.

%In the next theorem, we will be using the covering radii of the lattices defined in the previous section to derive \decomp sample sets. In particular, we rescale $\dZ^d$ and $\DN$ to yield the required sample sets. The third sample set is obtained from $\AN$ after applying transformations, that are elaborated in the proof. 
% \begin{lemma}[Optimal covering radius]
%     Let $\Lambda$ be a lattice in $\dR^d$. Then $Cover\left(\Lambda\right)$ is the optimal \left(in terms of density\right) covering radius for this $d$-dimensional lattice. Relying on Conway~\cite{conway2013sphere}, we have:
%     \itai{I'm correcting this. the optimal radius is related to the original lattice, not the rescaled one.}
%     \begin{enumerate}[topsep=1pt,itemsep=1ex,partopsep=1ex,parsep=1ex]
%         \item $Cover\left(\AN\right) = \sqrt{\frac{d}{2}}$
%         \item $Cover\left(\DN\right) = \frac{\sqrt{d}}{2}$
%         \item $Cover\left(\ZN\right) = \sqrt{\frac{d\left(d+2\right)}{12\left(d+1}}$
%     \end{enumerate}
% \end{lemma}
% \itai{changes here.}
% Using the concept of covering radii for lattices, we will explicitly find sample sets defined by our three lattices:
% \begin{definition}[Lattice Sample Set]
%     Let $\Lambda$ be a lattice, and $\delta,\epsilon$ be the clearance and stretching parameters. We define $\XL$ to be the set $\left(c\cdot\Lambda\right)$ for some $c>0$ such that it's a \decomp set.
% \end{definition}
% Specifically for \Lattices we get the following theorem:

\begin{thm} \label{thm:decomp_lattices}
  Fix $\delta>0,\epsilon>0$, and take $\beta^*$ as defined in \Cref{lem:cover}. Then the following sample sets are \decomp:
  \begin{enumerate}
      \item $\XZ:=\left\{\frac{2\beta^*}{\sqrt{d}}\cdot v,v\in\mathbb{Z}^d\right \}=\frac{2\beta^*}{\sqrt{d}}\cdot \ZN$.
      \item $\XD:=\\
      \begin{cases}
        d\text{ is odd: } \left\{\frac{4\beta^*}{\sqrt{2d-1}}G_{\DN}^t\cdot v,v\in\mathbb{Z}^d\right \}=\frac{4\beta^*}{\sqrt{2d-1}}\cdot \DN,\\
        d\text{ is even: }
        \left\{\sqrt{\frac{8}{d}}\beta^* G_{\DN}^t\cdot v,v\in\mathbb{Z}^d\right\}=\sqrt{\frac{8}{d}}\beta^*\cdot \DN.
      \end{cases}$
      \item $\XA:=\left\{\sqrt{\frac{12\left(d+1\right)}{d\left(d+2\right)}}\beta^* T\cdot v,v\in\mathbb{Z}^d\right\}$, %=\sqrt{\frac{12\left(d+1\right)}{d\left(d+2\right)}}\beta T(\ZN)$    
       where   \\$T:=\begin{pmatrix}
                    1 &  1  & \dots & 1 & a - 1\\
                    -1 & 0  & \dots & 0 & a \\
                    0 & -1  & \dots & 0 & a \\
                    \vdots & \vdots  &  \ddots & \vdots & \vdots \\
                    0 & 0  &  \dots & -1 & a \\
                \end{pmatrix}\in \dR^{d \times d}$ and $a=\frac{1}{d+1 - \sqrt{d+1}}$. % and  $T(\ZN):=\{Tv,v\in\dZ^d\}$.
  \end{enumerate}
 % \kiril{To make this notation consistent with Definition~\ref{definition:generator}, shouldn't we write $aG_\Lambda$, instead of $a G_{\Lambda}$?}
\end{thm}

Each of the new sample sets can be viewed as a lattice in $\dR^d$, according to Definition~\ref{definition:lattice}. For instance, $\XA$ is a lattice with the generator matrix $\sqrt{\frac{12\left(d+1\right)}{d\left(d+2\right)}}\beta^* T^t$. Also, note that the result above for $\XD$ is a tightening 
of the result in~\cite{dayan2023near} for odd dimensions.

\begin{proof}
    %We will rescale the lattices \Lattices to obtain \decomp sample sets.
    The lattices \Lattices require a transformation to achieve $(\de)$-completeness for a given value of $\delta$ and $\epsilon$. For a given lattice $\Lambda$, we compute a rescaling factor $w_\Lambda>0$ such that the covering radius of the lattice $w_\Lambda \Lambda$ is not bigger than $\beta^*$. This would imply that $w_\Lambda \Lambda$ is \decomp according to Lemma~\ref{lem:cover}. In particular, $w_\Lambda = \beta^*f_\Lambda^{-1}$ where $f_\Lambda$ is the covering radius of $\Lambda$. Next, we consider each of the three lattices individually.     
% in the context of using unscaled distances---i.e., in a cube $[0,1]^d$ we have exactly one cube for the lattice $\mathbb{Z}^d$, in contrast with many more cubes if we start rescaling the grid. Since we want the samples to be sufficiently dense as to form a \decomp set, we will need to rescale the space. This is why in each of the following sections, given our lattice $\Lambda$, we use a \emph{rescale coefficient} $f_\Lambda:=f_\Lambda\left(d\right)$, together with a \emph{rescaling factor} $w_\Lambda>0$, defined by:
%    \[
%        \beta = f_\Lambda w_\Lambda\Rightarrow w_\Lambda = \frac{\beta}{f_\Lambda}.
%    \]
%    Meaning: we start with the covering radius (i.e. the rescale coefficient $f_\Lambda$), and we rescale it (i.e. multiply it with the rescaling factor) to "the smallest size possible" (i.e. the $\beta$-ball that covers the space), which is sufficient for $\left(\delta,\epsilon\right)$-completeness due to Lemma~\ref{lem:cover}.
    
    % One thing to note is, that using our scaling equation, our space went from some implicit "originating" space of radius $R_{orig}$ (where, as we said, the lattices are still unscaled)---which becomes $R$ with the relation:
    % \begin{align*}
    %     R_{\text{orig}}\cdot w_\Lambda = R \Rightarrow \frac{R_{\text{orig}}\cdot\beta}{f_\Lambda}=R\Rightarrow R_{\text{orig}}=\frac{R}{\beta}\cdot f_\Lambda.
    % \end{align*}
    % We will find $f_\Lambda$ for each of our three lattices, which will immediately give us our \decomp sets. \kiril{Note to self: rewrite this paragraph.}

\begin{figure}[!h]
  \centering
\includegraphics[width=0.7\columnwidth]{Images/EPGt_visualization.png}
  \caption{Visualization of embedding the lattice  $A_2^*$ originally defined in $\dR^3$ onto $\dR^2$ via the mapping $T$. The blue rectangle represents the plane $H$, where the corresponding $A_2^*$ lattice points are drawn in red. The points are generated by taking integer vectors in $\dR^d$ and applying the mapping $G^t$.  $H$ and $A_2^*$ is reflected onto the plane $H_0=\{x_3=0\}$ using the mapping $PG^t$ (denoted by the green rectangle). The third dimension is removed via the mapping $E$ to yield the embedding of $A_2^*$ in $\dR^2$.}
  \label{fig:egpt_visual}
\end{figure}

\vspace{5pt}
\noindent\emph{The lattice $\dZ^d$.} This lattice can be viewed as a set of axis-aligned unit hypercubes whose vertices are the lattice points. The center point of a cube is located at a distance of $\sqrt{d}/2$ from the cube's vertices, which yields the covering radius $f_{\ZN} =\sqrt{d}/2$, and the rescaling factor $w_{\ZN} = {\beta^*}/f_{\ZN}=2{\beta^*}/\sqrt{d}$. As a result, the sample set $\XZ:=\frac{2{\beta^*}}{\sqrt{d}}\ZN$, is a ${\beta^*}$-cover, and \decomp due to Lemma~\ref{lem:cover}. 

% Seeing as the cube has a side of length $w_{\ZN}$, this gives us the scaling equation:
%     \[
%         \beta = \frac{\sqrt{d}}{2}w_{\ZN} \Rightarrow f_{\ZN}\left(d\right)=\frac{\sqrt{d}}{2},
%     \]
%     from which we get our set definition:
%     \begin{align*}
%         \XZ = \{x\in\dR^d|x=w_{\ZN} v\cdot G_{\ZN},v\in\mathbb{Z}^{1\times d}\} = \\
%         \{x\in\dR^d|x=\frac{\beta}{f_{\ZN}\left(d\right)}I_{d\times d}\cdot v,v\in\mathbb{Z}^d\} = \\
%         \{x\in\dR^d|x=\frac{2\beta}{\sqrt{d}} v,v\in\mathbb{Z}^d\}:=\frac{2\beta}{\sqrt{d}}\ZN.
%     \end{align*}
    
\vspace{5pt}
\noindent\emph{The lattice $\DN$.}
We use the covering radius of $\DN$, which depends on whether the dimension $d$ is odd or even~\cite[page 120]{conway2013sphere}. In particular, 
$f_{D_{d\text{-odd}}^*}=\frac{\sqrt{2d-1}}{4}$, and $f_{D_{d\text{-even}}^*}=\frac{\sqrt{2d}}{4}$. This immediately implies the definition of $\XD$.
 %      \begin{cases}
%         \frac{4\beta}{\sqrt{2d-1}}\cdot \DN,\quad  \textup{if } d \textup{ is odd,}\\
%         \sqrt{\frac{8}{d}}\beta\cdot \DN, \quad
%         \textup{otherwise.}
%       \end{cases}
% \]

% Using this radius as a baseline for what we want our $\beta$-ball to be, we get the following rescaling equations:
%     \begin{align*}
%                 \beta = \frac{\sqrt{2d-1}}{4}w \Rightarrow f_{D_{d\text{-odd}}^*}\left(d\right) = \frac{\sqrt{2d-1}}{4}, \\
%                 \beta = \sqrt{\frac{d}{8}}w \Rightarrow f_{D_{d\text{-even}}^*}\left(d\right) = \sqrt{\frac{d}{8}},
%     \end{align*}
%     from which we get our set definitions:
%     \begin{align*}
%         D_{d\text{-odd}} = \{x\in\dR^d|x=wv\cdot G_{\DN},v\in\mathbb{Z}^{1\times d}\} = \\
%         \{x\in\dR^d|x=\frac{4\beta}{\sqrt{2d-1}}G_{\DN}^t\cdot v,v\in\mathbb{Z}^d\}:=\\
%         \frac{4\beta}{\sqrt{2d-1}}\DN,\\
%         D_{d\text{-even}} = \{x\in\dR^d|x=wv\cdot G_{\DN},v\in\mathbb{Z}^{1\times d}\} = \\
%         \{x\in\dR^d|x=\sqrt{\frac{8}{d}}\beta G_{\DN}^t\cdot v,v\in\mathbb{Z}^d\}:=\\
%         \sqrt{\frac{8}{d}}\beta\DN.
%     \end{align*}
%     Finally, we analyze the lattice $\AN$.
    % Using a lattice's generator, we can define the lattice points generated from it as:
    % \[
    %     \Lambda=\{x \in \dR^{d} \mid \exists g\in \mathbb{Z}^{d} \text{ s.t. } x=g\cdot G\}
    % \]

\vspace{5pt}
\noindent\emph{The lattice $\AN$.} 
    Recall that $A_d^*$ has the generator matrix $G:=G_{\AN}\in \dR^{d\times (d+1)}$.
    As this is a mapping from $d$ to $d+1$, we start with the process of embedding $A_d^*$ in $\dR^d$ (see Figure~\ref{fig:egpt_visual}). Afterward, we show that the embedding in $\dR^d$ shares the same covering radius as the original set in $\dR^{d+1}$.

    Any row $i$ of $G$, denoted by  $G_i:=(g_{i1},g_{i2},\ldots,g_{i(d+1)})$, lies on the (hyper)plane $H: \{\sum_{j=1}^{d+1} g_{ij} = 0\}$. Thus, $A_d^*$ itself is contained in that $d$-dimensional plane. It remains to find a transformation of $\AN$ such that the dimension is reduced to $d$ while maintaining the structure of the points in $\AN$.

In the first step, we reflect $\AN$ lattice points onto the plane $H_0:=\{x_{d+1}=0\}$ using the Householder matrix 
\begin{align}\label{eq:reflection}
 P:=\begin{pNiceArray}{cw{c}{1cm}c|c}[margin]
            \Block{3-3}<\Large>{I_d - \frac{1}{D-\sqrt{D}}\mathds{1}}
            & & & \tfrac{1}{\sqrt{D}} \\
            & & & \Vdots \\
            & & & \tfrac{1}{\sqrt{D}} \\
            \hline
            \tfrac{1}{\sqrt{D}} & \dots& \tfrac{1}{\sqrt{D}} & \tfrac{1}{\sqrt{D}}
        \end{pNiceArray},
    \end{align}
where $D:=d+1$, $I_d$ is an $d\times d$ identity matrix, and $\mathds{1}$ is the $d\times d$ matrix with $1$s in all its entries.  That is, we reflect a lattice point $p\in \dR^{d+1}$ by computing the value ${v}=P\cdot p$. (See the derivation of $P$ in \conditionaltext{\Cref{app:decomp_lattice_proof}}{ the supplementary material}.)

It remains to eliminate the $(d+1)$th dimension of the points $v:=P\cdot p$. This is accomplished by the mapping
\begin{align*}
        E=
        \begin{pmatrix}
            1 & 0 & \dots & 0 & 0 \\
            0 & 1 & \dots & 0 & 0 \\
            \vdots & \vdots & \ddots & \vdots & 0 \\
            0 & 0 & \dots & 1 & 0
        \end{pmatrix}_{d\times(d+1)}.
    \end{align*}
    
We finish by computing the an explicit mapping %$\mathbb{Z}^{d}\rightarrow\dR^d$ 
that yields the embedding of $A_d^*$ to $\dR^d$. In particular, we have the embedding $T(g):=EPG^t(g)$, for $g\in \dZ^d$, where $T$ is as specified in the statement of this theorem. (See the derivation of $T$ in  \conditionaltext{\Cref{app:decomp_lattice_proof}}{ in the supplementary material}.)

    %      \begin{figure}[H]
    %     \centering
    %     \begin{subfigure}[b]{0.3\textwidth}
    %         \includegraphics[width=\textwidth]{Images/EPGt_visual_explanation1.png}
    %         %\caption{Sample points in $H_0$}
    %         %\label{fig:epgt_visual1}
    %     \end{subfigure}
    %     \hfill
    %     \begin{subfigure}[b]{0.3\textwidth}
    %         \includegraphics[width=\textwidth]{Images/EPGt_visual_explanation2.png}
    %         %\caption{$H_0$ rotated to the "floor"}
    %         %\label{fig:epgt_visual2}
    %     \end{subfigure}
    %     \hfill
    %     \begin{subfigure}[b]{0.3\textwidth}
    %         \includegraphics[width=\textwidth]{Images/EPGt_visual_explanation3.png}
    %         %\caption{The samples as they look in $\dR^2$}
    %         %\label{fig:epgt_visual3}
    %     \end{subfigure}
    %     \caption{Visualization of embedding the lattice  $A_2^*$ originally defined in $\dR^3$ onto $\dR^2$ via the mapping $T$. [Left] The blue rectangle represents the plane $H$, where the corresponding $A_2^*$ lattice points are drawn in red. The points are generated by taking integer vectors in $\dR^d$ and applying the mapping $G^t$.  [Center] $H$ and $A_2^*$ is reflected onto the plane $H_0=\{x_3=0\}$ using the mapping $PG^t$. [Right] The third dimension is removed, via the mapping $E$, to yield the embedding of $A_2^*$ in $\dR^2$.}
    %     \label{fig:egpt_visual}
    % \end{figure}


For the final part, we wish to derive an \decomp sample set $\XA$. Here, we first recall that the covering radius $f_{\AN}$ is equal to $\sqrt{\frac{d\left(d+2\right)}{12\left(d+1\right)}}$~\cite[page 115]{conway2013sphere}. Considering that reflections and embeddings are isometries, i.e.,  they preserve distances between pairs of points, we can use the same covering radius after mapping the points of $\AN$ using $T$. Thus, we obtain the rescaling coefficient $w_{\AN}:={\beta^*} f_{\AN}^{-1}$, which implies that $\XA:=
         \{w_{\AN}T\cdot v,v\in\mathbb{Z}^d\}$ is \decomp.
% \begin{align*}\XA:&=
%          \left\{w_{\AN}T\cdot v,v\in\mathbb{Z}^d\right\}\\
%          &= 
%         \left\{\sqrt{\frac{12\left(d+1\right)}{d\left(d+2\right)}}\beta T\cdot v,v\in\mathbb{Z}^d\right\}. %=\sqrt{\frac{12\left(d+1\right)}{d\left(d+2\right)}}\beta T(\ZN).
%     \end{align*}    
    % This is what the transformation does: first (Fig.~\ref{fig:epgt_visual1}), $G^t$ takes an integer vector in $\dR^d$ and turns it to a vector in $\AN\subset\dR^{d+1}$. Next (Fig.~\ref{fig:epgt_visual2}), $P$ takes the element in $\AN$ and reflects it into $[X_1,\dots,X_d]$---effectively lowering its dimension. Lastly (Fig.~\ref{fig:epgt_visual3}), because we are still in $\dR^{d+1}$, we use $E$ to embed it into $\dR^d$.

    %     \beta = \sqrt{\frac{d\left(d+2\right)}{12\left(d+1\right)}}w_{\AN} \Rightarrow f_{\AN}\left(d\right)=\sqrt{\frac{d\left(d+2\right)}{12\left(d+1\right)}},
    % \]
    % from which we get our set definition:
\end{proof}

%%% Local Variables:
%%% mode: latex
%%% TeX-master: "../main"
%%% End:
% \section{Sample Complexity}\label{sec:sample_comlpexity}
We present our main theoretical results on sample complexity. First, we adapt Verger-Gaugry's~\cite{verger2005covering} lower bound result to obtain a new lower bound on the sample complexity of \decomp sets. Notice that this bound applies to any sample set, not necessarily a lattice-based one.

\begin{thm}[Sample-complexity lower bound~\protect{\cite[Theorem~3.10]{verger2005covering}}]
    Consider the configuration space $\C$ being a $d$-dimensional R-ball\itai{change here}, $\B_R\subset \dR^d$. Without loss of generality, assume\footnote{These radius ranges are assumed so that they fit into the setting of a theorem we use later.} $R=\sqrt{d}\beta$ or $R\geq d\cdot\beta$, where $\beta:=\beta\left(\delta,\epsilon\right)$ as defined in Lemma 1. Suppose that $\X$ is a sample set that is a $\beta$-cover for $\C$. Then there exists $c>0$ such that:
    \[
        |\X|\geq cd\left(\frac{R}{\beta\left(\de\right)}\right)^d =: n_{LB}^{\de}.
    \]
    % \triangleq \|\X_{LB}\|
\end{thm}
\begin{proof}
Theorem 3.10 in~\cite{verger2005covering} states that if a sample set $\X^*$ is a $\frac{1}{2}$-cover for $\B_{R^*}$, where $R^*=\frac{\sqrt{d}}{2}$ or $R^*\geq\frac{d}{2}$, then there exists $c>0$ such that $|\X^*|\geq cd\cdot \left(2R^*\right)^d$.


Next, we use a rescaling argument to apply those finding to our setting. In particular, we wish to cover $\B_R$ with $\beta$-balls. To convert $\beta$-balls into $\frac{1}{2}$-balls we need to rescale by $\frac{1}{2\beta}$, which makes it so that $R$ becomes $R^*=\frac{R}{2\beta}$ in the rescaled space. In this rescaled space, $\B_{R^*}$ is covered by $\frac{1}{2}$-balls. 


For the radius $R^*$ to fit Verger-Gaugry's requirements we need to ensure  that:
\begin{equation*}
    \frac{R}{2\beta} = R^* = \frac{\sqrt{d}}{2},
\end{equation*}
which implies that:
\[
    R=\sqrt{d}\beta,
\]
and similarly if \footnote{Notice that this lower bound applies to any ball $\B_r$, as long as it is $\sqrt{d}$ times or more than $d$-times the "minimal radius" ball of $\beta$.} $R\geq d\beta$. I want to demonstrate next why we can safely assume that our C-Space's radius is $\sqrt{d}$-times the radius of the $\beta$-ball. \itai{new segment here}


Taking $d=10$ for example, if our space's radius is $\sqrt{10}$ times bigger than a $\beta$-ball then the volume is $\sqrt{10}^{10}=10^5$ bigger than a $\beta$-ball. This means we expect to find about $10^5$ smaller $\beta$-balls in it. With a normal cubic $[0,1]^d$ world, we expect to find about $\frac{1}{\beta^d}=\left(\frac{1}{\beta}\right)^d=\left(\frac{1}{\beta^2}\right)^{\frac{d}{2}}$ balls. Taking the same $d=10$, we get that for our spherical space to have less $\beta$-balls than a cubic space, we will need $\frac{1}{\beta^2}\geq 10 \Rightarrow \beta \leq 0.32$. We get:
\[
    \beta=\delta\sqrt{\frac{\epsilon^2}{1+\epsilon^2}}\leq0.32 \iff \frac{\epsilon^2}{1+\epsilon^2}\leq \left(\frac{0.32}{\delta}\right)^2
\]
But notice that:
\[
    \frac{0.32}{\delta}<1 \iff \delta>0.32
\]
So for $\delta\leq 0.32$, we get $\frac{0.32}{\delta}\geq1$. But $\frac{\epsilon^2}{1+\epsilon^2}<1$ for all $\epsilon>0$, which makes the above inequality true for all $\epsilon>0$. All this is to say the following: the requirements on $\delta,\epsilon$ aren't harsh and would probably be fulfilled in most use cases, allowing us to rely on this lower bound. Returning to Verger-Gaugry's result, we get the lower bound:
\begin{equation*}
    |\X|\ \geq cd\cdot \left(2R^*\right)^d = cd\cdot \left(\frac{R}{\beta}\right)^d,
\end{equation*}
which concludes the proof. \qedsymbol
\end{proof}
\itai{Note to self: either remove the following part or find a suitable comparison between square and spherical CSpaces.}


\itai{I moved this section HERE, Im not sure why it was down below.}
Let us compare this lower-bound to a more naive bound, asserting the minimal number of points in a set $\X_{base}$ should be at least the volume of $\B_R$ divided by the volume of a $\beta$-ball:
\begin{equation}
        |\X_{base}|\geq \frac{Vol\left(\B_R\right)}{Vol\left(\B_\beta\right)}=\frac{c_d R^d}{c_d\beta^d}=\left(\frac{R}{\beta}\right)^d.
\end{equation}
It is immediately clear that the new lower bound is $d$ times better than the classic one. This is another reason to support spherical configuration spaces.

Next, we will define what is a "lattice sample set" and show that we can give explicit representations of them such that they'll be \decomp sets. As we will be talking about covering the space throughout the paper, let us first consider the following definition from Conway and Sloane~\cite{conway2013sphere}:
\begin{definition} (Covering Radius)
    \itai{changes here} For a collection of points $\X$, a covering radius is defined to be:
    \[
        R_{cover}^{\X} = \sup_{y\in\mathbb{R}^d}\inf_{x\in\X} \|x-y\|.
    \]
\end{definition}
\itai{added some clarification here}
This definition makes it more clear that one can not cover $\mathbb{R}^d$ with a radius smaller than $R_{cover}^{\X}$ for a given point set $\X$. In the next theorem, we will be using the covering radii's optimality to construct $\beta$-balls, paving the way to explicitly construct \decomp sample sets.
% \begin{lemma}[Optimal covering radius]
%     Let $\Lambda$ be a lattice in $\mathbb{R}^d$. Then $Cover\left(\Lambda\right)$ is the optimal \left(in terms of density\right) covering radius for this $d$-dimensional lattice. Relying on Conway~\cite{conway2013sphere}, we have:
%     \itai{I'm correcting this. the optimal radius is related to the original lattice, not the rescaled one.}
%     \begin{enumerate}[topsep=1pt,itemsep=1ex,partopsep=1ex,parsep=1ex]
%         \item $Cover\left(\AN\right) = \sqrt{\frac{d}{2}}$
%         \item $Cover\left(\DN\right) = \frac{\sqrt{d}}{2}$
%         \item $Cover\left(\ZN\right) = \sqrt{\frac{d\left(d+2\right)}{12\left(d+1}}$
%     \end{enumerate}
% \end{lemma}
% \itai{changes here.}
% Using the concept of covering radii for lattices, we will explicitly find sample sets defined by our three lattices:
% \begin{definition}[Lattice Sample Set]
%     Let $\Lambda$ be a lattice, and $\delta,\epsilon$ be the clearance and stretching parameters. We define $\XL$ to be the set $\left(c\cdot\Lambda\right)$ for some $c>0$ such that it's a \decomp set.
% \end{definition}
% Specifically for \Lattices we get the following theorem:
\begin{thm}
  Fix $\delta>0,\epsilon>0$. Then the following sets are \decomp:
  \itai{added a new case for Dn (even/odd)}
  \begin{enumerate}
      \item $\XZ=\frac{2\beta\left(\delta,\epsilon\right)}{\sqrt{d}}\ZN$,
      \item $\XD=
      \begin{cases}
        d\text{ is odd: }
        \{x\in\mathbb{R}^d|x=\frac{4\beta\left(\delta,\epsilon\right)}{\sqrt{2d-1}}G_{\DN}^t\cdot v,v\in\mathbb{Z}^d\}:=\frac{4\beta\left(\delta,\epsilon\right)}{\sqrt{2d-1}}\DN,\\
        d\text{ is even: }
        \{x\in\mathbb{R}^d|x=\sqrt{\frac{8}{d}}\beta G_{\DN}^t\cdot v,v\in\mathbb{Z}^d\}:=\sqrt{\frac{8}{d}}\beta\DN,
      \end{cases}$
      \item $\XA=\{\mathbb{R}^d|x=\sqrt{\frac{12\left(d+1\right)}{d\left(d+2\right)}}\beta\left(\delta,\epsilon\right) T\cdot v,v\in\mathbb{Z}^d\}:=\sqrt{\frac{12\left(d+1\right)}{d\left(d+2\right)}}\beta\left(\delta,\epsilon\right) T(\AN)$,
      
      
      with:
      $T=\begin{pmatrix}
                    1 &  1  & \dots & 1 & x - 1\\
                    -1 & 0  & \dots & 0 & x \\
                    0 & -1  & \dots & 0 & x \\
                    \vdots & \vdots  &  \ddots & \vdots & \vdots \\
                    0 & 0  &  \dots & -1 & x \\
                \end{pmatrix}\in \mathbb{R}^{d \times d}, x=\frac{1}{(d+1) - \sqrt{d+1}}$.
  \end{enumerate}
\end{thm}
\begin{proof}
    \itai{this section onwards has changes} We need to show that we can rescale the lattices \Lattices by some $c>0$ such that they become \decomp. We remember that the lattices were introduced in the first section in the context of using unscaled distances---i.e., in a cube $[0,1]^d$ we have exactly one cube for the lattice $\mathbb{Z}^d$, in contrast with many more cubes if we start rescaling the grid. Since we want the samples to be sufficiently dense as to form a \decomp set, we will need to rescale the space. This is why in each of the following sections, given our lattice $\Lambda$, we use a \emph{rescale coefficient} $f_\Lambda:=f_\Lambda\left(d\right)$, together with a \emph{rescaling factor} $w_\Lambda>0$, defined by:
    \[
        \beta = f_\Lambda w_\Lambda\Rightarrow w_\Lambda = \frac{\beta}{f_\Lambda}.
    \]
    Meaning: we start with the covering radius (i.e. the rescale coefficient $f_\Lambda$), and we rescale it (i.e. multiply it with the rescaling factor) to "the smallest size possible" (i.e. the $\beta$-ball that covers the space), which is sufficient for $\left(\delta,\epsilon\right)$-completeness due to Lemma~\ref{lem:cover}.
    One thing to note is, that using our scaling equation, our space went from some implicit "originating" space of radius $R_{orig}$ (where, as we said, the lattices are still unscaled)---which becomes $R$ with the relation:
    \begin{align*}
        R_{orig}\cdot w_\Lambda = R \Rightarrow \frac{R_{orig}\cdot\beta}{f_\Lambda}=R\Rightarrow R_{orig}=\frac{R}{\beta}\cdot f_\Lambda.
    \end{align*}
    We will find $f_\Lambda$ for each of our three lattices, which will immediately give us our \decomp sets.


    \itai{This whole section was overhauled (no more enumerate, rewritten, etc.)} We first look at $\ZN$. Consider the cube $[0,w_{\ZN}]^d$ which tessellates the space. To find the rescale coefficient, remember that we need to completely cover each of these lattice cubes. Thus, the radius needs to reach the center of the cube. Seeing as the cube has a side of length $w_{\ZN}$, this gives us the scaling equation:
    \[
        \beta = \frac{\sqrt{d}}{2}w_{\ZN} \Rightarrow f_{\ZN}\left(d\right)=\frac{\sqrt{d}}{2},
    \]
    from which we get our set definition:
    \begin{align*}
        \XZ = \{x\in\mathbb{R}^d|x=w_{\ZN} v\cdot G_{\ZN},v\in\mathbb{Z}^{1\times d}\} = \\
        \{x\in\mathbb{R}^d|x=\frac{\beta}{f_{\ZN}\left(d\right)}I_{d\times d}\cdot v,v\in\mathbb{Z}^d\} = \\
        \{x\in\mathbb{R}^d|x=\frac{2\beta}{\sqrt{d}} v,v\in\mathbb{Z}^d\}:=\frac{2\beta}{\sqrt{d}}\ZN.
    \end{align*}
    Next, we look at $\DN$. Here we use the fact that the covering radius for $D_{d\text{-odd}}^*$ is \(\frac{\sqrt{2d-1}}{4}\), and for $D_{d\text{-even}}^*$ is \(\frac{\sqrt{2d}}{4}=\sqrt{\frac{d}{8}}\) (~\cite[page 120]{conway2013sphere}). Using this radius as a baseline for what we want our $\beta$-ball to be, we get the following rescaling equations:
    \begin{align*}
                \beta = \frac{\sqrt{2d-1}}{4}w \Rightarrow f_{D_{d\text{-odd}}^*}\left(d\right) = \frac{\sqrt{2d-1}}{4}, \\
                \beta = \sqrt{\frac{d}{8}}w \Rightarrow f_{D_{d\text{-even}}^*}\left(d\right) = \sqrt{\frac{d}{8}},
    \end{align*}
    from which we get our set definitions:
    \begin{align*}
        D_{d\text{-odd}} = \{x\in\mathbb{R}^d|x=wv\cdot G_{\DN},v\in\mathbb{Z}^{1\times d}\} = \\
        \{x\in\mathbb{R}^d|x=\frac{4\beta}{\sqrt{2d-1}}G_{\DN}^t\cdot v,v\in\mathbb{Z}^d\}:=\\
        \frac{4\beta}{\sqrt{2d-1}}\DN,\\
        D_{d\text{-even}} = \{x\in\mathbb{R}^d|x=wv\cdot G_{\DN},v\in\mathbb{Z}^{1\times d}\} = \\
        \{x\in\mathbb{R}^d|x=\sqrt{\frac{8}{d}}\beta G_{\DN}^t\cdot v,v\in\mathbb{Z}^d\}:=\\
        \sqrt{\frac{8}{d}}\beta\DN.
    \end{align*}
    Finally, we analyze the lattice $\AN$.
    % Using a lattice's generator, we can define the lattice points generated from it as:
    % \[
    %     \Lambda=\{x \in \mathbb{R}^{d} \mid \exists g\in \mathbb{Z}^{d} \text{ s.t. } x=g\cdot G\}
    % \]
    Recall that $A_d^*$ is defined using this \emph{generator matrix}:
    \begin{align} \label{eq:generaror}
        G:=G_{\AN}=
        \begin{pmatrix}
            1 & -1 & 0 & 0 & \dots & 0 & 0 \\
            1 & 0 & -1 & 0 & \dots & 0 & 0 \\
            . & . & . & . & \dots & . & 0 \\
            1 & 0 & 0 & 0 & \dots & -1 & 0 \\
            \frac{-d}{d+1} & \frac{1}{d+1} & \frac{1}{d+1} & \frac{1}{d+1} & \dots & \frac{1}{d+1} & \frac{1}{d+1} \\
        \end{pmatrix}\in \mathbb{R}^{d\times \left(d+1\right)}.
    \end{align}
    As this is a mapping from $d$ to $d+1$, we start with the process of embedding $A_d^*$ in $\mathbb{R}^d$. Afterwards we show that the embedding we created in $\mathbb{R}^d$ shares the same covering radius as the original set in $\mathbb{R}^{d+1}$.

    
    Notice that the rows of $G$ are all in the plane $H_0: \sum_i x_i = 0$, so $A_d^*$ itself is contained in that $d$-dimensional plane. It remains to find a transformation of $\AN$ such that the dimension is reduced to $d$, while maintaining the structure of the points in $\AN$.
    
    
    First, we will rotate $H_0$ so that we zero out the last coordinate. To do that, we will use a \emph{Householder matrix}~\cite{householder1958unitary}, which gives us a way to reflect a point set with a plane. Defining the plane through its normal vector $\hat{n}\in \mathbb{R}^{d+1}$, the Householder matrix is defined as:
    \begin{align}\label{eq:reflection}
        P=I-2\hat{n}\hat{n}^t.
    \end{align}
    We need to find the normal $\hat{n}$ to a plane that is exactly between (angle-wise) the plane $H_0$ and the $X_{d+1}=0$ plane, because reflecting against this plane will give us points that lie in $X_{d+1}=0$. \itai{fixed many typos and rewritten this section. The angle thing wasnt accurate enough IMO, so I proved it another way.}
     Finding an angle between planes is equivalent to finding the angle between their normal vectors, which leads us to the following \emph{intuition} in two dimensions.
     \begin{figure}[H]
        \centering
        \includegraphics[width=0.45\linewidth]{Images/normal_inuition.jpg}
        \caption{Illustraion of finding the reflection normal in $d=2$}
        \label{fig:reflection}
    \end{figure}
     the plane $H_0$ has a normal vector $u=\left(\frac{1}{\sqrt{2}},\frac{1}{\sqrt{2}}\right)$. We want to find the vector splitting the angle between u and $(0,-1)$. We can look for it on the vertical line descending from $u$, and because $\|u\|=\|(0,-1)\|=1$ then to make sure it splits the angle we just need that 
     \begin{align*}
         n\cdot u = n\cdot (-e_2) \iff \\
         \left(\frac{1}{\sqrt{2}},\frac{1}{\sqrt{2}}-\alpha\right)\cdot\left(\frac{1}{\sqrt{2}},\frac{1}{\sqrt{2}}\right)=\left(\frac{1}{\sqrt{2}},\frac{1}{\sqrt{2}}-\alpha\right)\cdot(0,-1) \Rightarrow \\
         \frac{1}{2}+\frac{1}{2}-\alpha\frac{1}{\sqrt{2}}=\alpha-\frac{1}{\sqrt{2}} \Rightarrow \\
         1+\frac{1}{\sqrt{2}}=\alpha(1+\frac{1}{\sqrt{2}})\Rightarrow\alpha=1,
     \end{align*}
    getting the normal $n=\left(\frac{1}{\sqrt{2}},\frac{1}{\sqrt{2}}-1\right)$. It can be shown that the same idea repeats in $d=3$ with $n=\left(\frac{1}{\sqrt{3}},\frac{1}{\sqrt{3}},\frac{1}{\sqrt{3}}-1\right)$, leading us to the following generalization.
    Let $e_{d+1}=\left(0,\dots,0,1\right)\in\mathbb{R}^{d+1},u=\left(\frac{1}{\sqrt{d+1}},\dots,\frac{1}{\sqrt{d+1}}\right)\in~\mathbb{R}^{d+1}$. Define $n:=\left(-\frac{1}{\sqrt{d+1}},\dots,-\frac{1}{\sqrt{d+1}},1-\frac{1}{\sqrt{d+1}}\right)$ (this is the same as our $d=2,3$ cases, only with an opposite sign). We are now going to show that this vector, when used with the Householder matrix, will rotate $H_0$ to "the floor"---i.e. reflection onto the $[X_1,\dots,X_d]$ plane. Using $\hat{n}=\frac{n}{\|n\|}$ to get a normalized vector, we can calculate the Householder matrix, getting
    \begin{align}
        P=
        \begin{bNiceArray}{cw{c}{1cm}c|c}[margin]
            \Block{3-3}<\Large>{I_d - \frac{1}{D-\sqrt{D}}\mathds{1}} 
            & & & \dfrac{1}{\sqrt{D}} \\
            & & & \Vdots \\
            & & & \dfrac{1}{\sqrt{D}} \\
            \hline
            \dfrac{1}{\sqrt{D}} & \dots& \dfrac{1}{\sqrt{D}} & \dfrac{1}{\sqrt{D}}
        \end{bNiceArray}.
    \end{align}
    We need to show that for all $i\leq d+1$, the last element of $v=PG_{\AN}^t\cdot e_i$ is always zero. First, for all $i<d+1$, we easily get
    \begin{align*}
        PG_{\AN}^t\cdot e_i=P\cdot
        \begin{pmatrix}
            1 \\
            0 \\
            \vdots \\
            0 \\
            -1 \\
            0 \\
            \vdots \\
            0
        \end{pmatrix}.
    \end{align*}
    But, in the matrix $P$, the last elements of all $i<d+1$ columns are all equal. This means that taking $+1$ of one and $-1$ of the other will give us 0 there. We are left with $e_{d+1}$, which gives us
        \begin{align*}
        PG_{\AN}^t\cdot e_{d+1}=P\cdot
        \begin{pmatrix}
            -\frac{D-1}{D} \\
            \frac{1}{D} \\
            \vdots \\
            \frac{1}{D}
        \end{pmatrix}.
    \end{align*}
    Looking specifically at the last element, we get
    \begin{align*}
        \frac{1}{\sqrt{D}}\cdot\frac{1-D}{D} + (D-1)\frac{1}{D}\frac{1}{\sqrt{D}}=\frac{1-D+D-1}{D\sqrt{D}}=0,
    \end{align*}
    showing what we needed.
    % Let $\angle\left(u,v\right)$ denote the angle between the vectors $u,v$. Then we get that $\cos\left({\angle\left(u,n\right)}\right) = \cos\left({\angle\left(-e_{d+1},n\right)}\right)$:
    % \begin{align*}
    %     \cos\left({\angle\left(u,n\right)}\right) = \frac{u\cdot n}{\|u\|\|n\|} = \frac{\frac{-1}{d+1}\cdot d + \frac{1}{\sqrt{d+1}} - \frac{1}{d + 1}}{\|n\|} = \frac{\frac{1}{\sqrt{d+1}} - 1}{\|n\|}, \\
    %     \cos{\angle\left(\left(-e_{d+1}\right),n\right)} = \frac{\left(-e_{d+1}\right)\cdot n}{\|-e_{d+1}\|\|n\|} = \frac{\frac{1}{\sqrt{d+1}} - 1}{\|n\|},
    % \end{align*}
    % In $d=2$ the three vectors $-e_{d+1},u,n$ are on the same 3D-plane, as we have
    % \[
    %     u\times n = n \times (-e_{d+1}) = (\frac{1}{\sqrt{D}}, \frac{1}{\sqrt{D}}, 0).
    % \]
    % For three vectors $-e_{d+1},u,n$ on the same plane to define two angles that have the same cosine value, they need to have angles $\pi+x,\pi-x$. But these three unique vectors define three unique angles, and $\pi+x +\pi-x=2\pi$, which is a contradiction. This means that their angles are equal, finishing our claim for $d=2$. This serves as a motivation for defining this $n$ vector for all $d$ dimensions. I
    
    
    Using this, all we have left to do is use an embedding that ignores the last element, meaning $E:\mathbb{R}^{n+1}\Rightarrow\mathbb{R}^d$ defined by $E\left(x_1,\dots,x_d,x_{d+1}\right)=\left(x_1,\dots,x_d\right)$. We finish with the following definition that helps us explicitly calculate points in $A_d^*$ (using (\ref{eq:generaror}), (\ref{eq:reflection})):
    \begin{align}
        T: \mathbb{Z}^d\Rightarrow\mathbb{R}^d \nonumber \\
        T\left(g\right) = EPG^t\left(g\right).
    \end{align}
    \itai{a new explanation about T. }
    \begin{figure}[H]
        \centering
        \begin{subfigure}[b]{0.49\textwidth}
            \includegraphics[width=\textwidth]{Images/EPGt_visual_explanation1.png}
            \caption{Sample points in $H_0$}
            \label{fig:epgt_visual1}
        \end{subfigure}
        \hfill
        \begin{subfigure}[b]{0.49\textwidth}
            \includegraphics[width=\textwidth]{Images/EPGt_visual_explanation2.png}
            \caption{$H_0$ rotated to the "floor"}
            \label{fig:epgt_visual2}
        \end{subfigure}
        \hfill
        \begin{subfigure}[b]{0.49\textwidth}
            \includegraphics[width=\textwidth]{Images/EPGt_visual_explanation3.png}
            \caption{The samples as they look in $\mathbb{R}^2$}
            \label{fig:epgt_visual3}
        \end{subfigure}
        \caption{Lowering dimensions in $A_2^*$}
        \label{fig:egpt_visual}
    \end{figure}
    This is what the transformation does: first (Fig.~\ref{fig:epgt_visual1}), $G^t$ takes an integer vector in $\mathbb{R}^d$ and turns it to a vector in $\AN\subset\mathbb{R}^{d+1}$. Next (Fig.~\ref{fig:epgt_visual2}), $P$ takes the element in $\AN$ and reflects it into $[X_1,\dots,X_d]$---effectively lowering its dimension. Lastly (Fig.~\ref{fig:epgt_visual3}), because we are still in $\mathbb{R}^{d+1}$, we use $E$ to embed it into $\mathbb{R}^d$.
    
    \itai{changed all N to D in the following sections and changed a bit the order of things}
    Formally, we start by defining the embedding $E$. Let $D:=d+1$, then
    \begin{align*}
        E=
        \begin{pmatrix}
            1 & 0 & \dots & 0 & 0 \\
            0 & 1 & \dots & 0 & 0 \\
            . & . & \dots & . & 0 \\
            0 & 0 & \dots & 1 & 0
        \end{pmatrix},
    \end{align*}
    which, together with $P$ (in (6)), we use to calculate $EP$ 
    \begin{align*}
        \left(EP\right)^t=
        \begin{bNiceArray}{cw{c}{1cm}c}[margin]
            \Block{3-3}<\Large>{I_d - \frac{1}{D-\sqrt{D}}} 
            & &  \\
            & &  \\
            & &  \\
            \hline
            \dfrac{1}{\sqrt{D}} & \Cdots& \dfrac{1}{\sqrt{D}}
        \end{bNiceArray}\in\mathbb{R}^{d\times(d+1)}.
    \end{align*}
    From here it can be shown that
    \begin{align}
        T^t=G\left(EP\right)^t=
        \begin{pmatrix}
            1 & -1 &  0  & \dots & 0 \\
            1 & 0  &  -1 & \dots & 0 \\
            . & .  &  .  & \dots & . \\
            1 & 0  &  0  & \dots & -1 \\
            \frac{1}{D - \sqrt{D}} - 1 & \frac{1}{D - \sqrt{D}} & \frac{1}{D - \sqrt{D}} & \dots & \frac{1}{D - \sqrt{D}}
        \end{pmatrix}\in\mathbb{R}^{d\times d}
    \end{align}
    Being that reflections are isometries, and so are embeddings, we can use the same covering radius after using $T$. So, we are given the rescaling coefficient through the covering radius (~\cite[page 115]{conway2013sphere}):
    \[    
        \beta = \sqrt{\frac{d\left(d+2\right)}{12\left(d+1\right)}}w_{\AN} \Rightarrow f_{\AN}\left(d\right)=\sqrt{\frac{d\left(d+2\right)}{12\left(d+1\right)}},
    \]
    from which we get our set definition:
    \begin{align*}
        \XA = \{x\in\mathbb{R}^d|x=w_{\AN}T\cdot v,v\in\mathbb{Z}^d\} = \\
        \{x\in\mathbb{R}^d|x=\sqrt{\frac{12\left(d+1\right)}{d\left(d+2\right)}}\beta T\cdot v,v\in\mathbb{Z}^d\}\, \qedsymbol
    \end{align*} 
\end{proof}
% The above result directly follows from~\cite{verger2005covering} which provides a lower bound on the number of $\tfrac{1}{2}$-balls necessary to cover a ball of radius $R \geq \frac{d}{2}$. A simple rescaling argument is used to make it applicable to our setting where $\beta\left(\de\right)$-balls are used for covering the configuration space. \kiril{Does this description give justice to what you derived or is there something more complicated that I'm missing here?}

% This is a fair upper bound on $\beta=\delta\alpha$, as $\alpha \leq 1$ and we already take $\delta$ as small value, usually. For example, for a 10 degree problem, we would need $\beta\leq\frac{1}{6}\approx 0.17$---then for $\delta = 0.1$ we would need $\epsilon < 1.7$. \kiril{I don't understand this discussion. Please clarify.}
% NOTE \itai{I commented out all this part \left(see tex file\right). Its irrelant as I rewrote the theorem.}
% Let us compare this bound with the previous bound~\cite[Theorem~1]{tsao2020sample}, which employed volume arguments. \kiril{Is this the bound you had in mind?} In particular, the previous bound is at least 
% \kiril{please fill in.}
% Thus, the new upper bound is bigger by a factor of \kiril{fill} than the old one. 

% \kiril{The following text needs to be clarified as in the first equation the new bound is stated and I substituted it with the text above.}
% {\color{blue} To compare this LB to the "classic" volume-derived lower bound, let us return to a configuration space defined in a ball of general radius $r\geq \beta d$. \kiril{return from where? let's keep everything consistent and keep the same setting throughout the paper unless absolutely necessary.}
% We get the following lower bound:
% \begin{align}
%     n \geq k d\cdot\left(2r_0\right)^d =k\cdot d \left(\frac{r}{\beta}\right)^d 
% \end{align}
% NOTE \itai{I commented out all this part also \left(see tex file\right). To clarify: this is a comparison to the classic "volume divided by volume" lower bound, I didn't compare this lower bound to the one in tsao. you dealt with a square C-Space there and I found it hard to compare your approximation to this one.}
% \begin{proof}
%   We rely on , which states that if a sample set $\X$ is a for  then
% \begin{align}
%    |\X| \geq k\cdot d \cdot\left(2R\right)^{d},\,k>0. 
% \end{align}
% \kiril{What is the purpose of $k$? Does it hold for any $k$? If so, can we substitute $\geq$ with $>$ and get rid of $k$?}
% In our setting, we wish to construct a $\beta\left(\de\right)$-cover of our configuration space. 
% We want $\frac{1}{2}$-balls in~\cite{verger2005covering}  to be mapped to balls of radius $\beta\left(\de\right)$-balls, so we need a rescaling factor of $2\beta$. This means that if we want to cover a general ball of radius $r$ by $\beta$-balls, then we need a ball of initial radius $R$ \left(before the rescaling\right) such that:
% \[
%     R\cdot 2\beta=rarrow R=\frac{r}{2\beta}
% \]
% Taking $R\geq\frac{d}{2}$ to fit Verger's result, we reach the following bound on $r>0$:
% \begin{align*}
%     \frac{d}{2} \leq R = \frac{r}{2\beta} arrow r \geq \beta d
% \end{align*}
% To get a feeling for this bound, let us choose our configuration space as a ball of radius $r=\frac{\sqrt{d}}{2}$. This is logical because it is the circumsphere of the regular $[0,1]^d$ cube. With this we need to have:
% \[
%     \frac{\sqrt{d}}{2} \geq \beta d arrow \beta \leq \frac{1}{2\sqrt{d}}
% \]
% \end{proof}
\itai{rewritten section}
% We are now going to find new upper bounds for our newly defined sample sets \Lattices. We name an upper bound for a sample set by the name \emph{sample complexity}. 
% Throughout the analysis, we assume that the configuration space is the Euclidean $d$-dimensional $R$-ball denoted by $\B_R$. This assumption allows us to obtain tighter bounds, than for, say, unit hypercube configuration spaces, which were assumed in previous sample-complexity analysis. 
% To this end, we first introduce the \emph{Gram matrix} of a lattice, relevant to the following Theorem:
%   \begin{definition}[Gram matrix and quadratic form]
%       The \emph{Gram matrix} of a lattice $\Lambda$ with generator matrix $G_\Lambda$ is defined to be $Gr_\Lambda:=G_\Lambda G^t_\Lambda$. Additionally the quadratic form of $\Lambda$ is defined as $q_\Lambda\left(a\right):=aGr_\Lambda a^t$.
%   \end{definition}
In this next theorem, we derive analytical upper bounds for \Lattices, a property we call \emph{sample complexity}. To achieve that, we first derive an upper bound for general lattices. Later on, in our experimental results (Section~\ref{sec:experiments}), we compare the sample complexity of those sets for varying values of $\de$, and $d$. 

\begin{thm}[General lattice sample complexity]
\label{general_sample_complexity}
    Consider a lattice $\Lambda$ that defines the \decomp set $\XL$ in a C-Space $\B_R$. Recall that $\beta=\beta\left(\delta,\epsilon\right)$ is explicitly given in equation (1). Then we get the sample complexity:
    \begin{align*}
        |\XL\cap \B_R| = O\left(\frac{1}{d}\left(\left(\frac{R}{\beta}\right)^2\frac{2\pi e}{d}\cdot f_\Lambda^2\left(d\right)\right)^{\frac{d}{2}} + \left(\frac{R}{\beta}\cdot f_\Lambda\left(d\right)\right)^{d-2}\right).
    \end{align*}
\end{thm}
\begin{proof}
  We wish to estimate the size of sample set $\X_\Lambda$ induced by the lattice $\Lambda$ and the scaling factor $f_\Lambda\left(d\right)$. We first show that this is equivalent to estimating the expression $|\B_R\cap \Lambda|$. Let $v=aG_\Lambda,\,a\in\mathbb{Z}^d$ be a lattice point, then by definition of a \emph{Gram matrix} as being $Gr_\Lambda:=G_\Lambda G_\Lambda^t$, we obtain 
  \[
    aGr_\Lambda a^t:=aG_\Lambda G^t_\Lambda a^t=\|aG_\Lambda\|^2=\|v\|^2.
  \]
  This leads to the relation
  \begin{align*}
      \B_R\cap \Lambda &= \{v\in\Lambda |\,\|v\| \leq R\} =\{a\in\dZ^d|a Gr_\Lambda a^t\leq R^2\} \\ & =E_{R^2}\left(Gr_\Lambda\right)\cap \dZ^d,
  \end{align*}
  where $E_{s}\left(A\right):=\{x\in\dR^d|x A x^t\leq s\}$. 
  
  
  Next, observe that the Gram matrix $Gr_\Lambda$ of the lattices \Lattices contains only rational numbers, and hence defines what is called a \emph{"rational quadratic form"}. This allows us to use Landau's \emph{rational ellipsoid bound theorem}~\cite{ivic2004lattice} which asserts that
  \begin{equation*}
      |E_{r^2}\left(Gr_\Lambda\right)\cap \dZ^d|=\frac{\vol\left(\B_{s}\right)}{\sqrt{d}}+\Theta\left(s^{d-2}\right),
  \end{equation*}
hence,
\begin{align}\label{volume_delta}
    |\B_R\cap \Lambda|=\frac{\vol\left(\B_{R}\right)}{\sqrt{d}}+\Theta\left(R^{d-2}\right).
\end{align}
As~(\ref{volume_delta}) deals with the unscaled lattice, we now use the rescaling actions performed in Theorem 1 to move "back" to the "unscaled" version of our spherical C-Space, in order to reach the sample complexity evaluation:
\begin{align}
    \|\XL\cap \B_R\|\underset{[T2]}{=}\|\Lambda\cap\B_{R_{orig}}\|\underset{\left(\ref{volume_delta}\right)}{\leq}\frac{1}{\sqrt{\pi}d}\left(\frac{2\pi e}{d}\right)^{\frac{d}{2}}R_{orig}^d+c\left(R_{orig}^{d-2}\right) \nonumber\\
    \underset{[T2]}{=}\frac{1}{\sqrt{\pi}d}\left(\frac{R}{\beta}\right)^d\left(\frac{2\pi e}{d}\cdot f_\Lambda^2\left(d\right)\right)^{\frac{d}{2}} +  c\left(\frac{R}{\beta}\right)^{d-2}\left(f_\Lambda\left(d\right)\right)^{d-2} \nonumber\\
    \Rightarrow \|\XL\cap \B_R\| \leq \frac{1}{\sqrt{\pi}d}\left(\frac{R}{\beta}\right)^d\left(\frac{2\pi e}{d}\cdot f_\Lambda^2\left(d\right)\right)^{\frac{d}{2}} +  c\left(\frac{R}{\beta}\right)^{d-2}\left(f_\Lambda\left(d\right)\right)^{d-2},
\end{align}
with some $c>0$ from~(\ref{volume_delta}). This finishes the proof.  
\end{proof}

The following corollary leverages the last theorem and the specific structure of \Lattices to derive specific upper bounds.

\begin{cor}[$\XZ,\XD,\XA$ sample complexity]
    Fix $\beta=\beta\left(\delta,\epsilon\right)$ and let $\theta=~\theta\left(\delta,\epsilon\right):=~\frac{R}{\beta}$. Consider the configuration space  $\C=\B_R$ for $R=\sqrt{d}\beta$ or $R \geq d\beta$. Then the following bounds hold:
    \begin{enumerate}[topsep=1pt,itemsep=1ex,partopsep=1ex,parsep=1ex]
        \item 
            $|\XZ\cap\B_R| = O\left(\frac{1}{d}\left[\left(2.07\theta\right)^d+\left(\frac{d}{4}\theta\right)^\frac{d-2}{2}\right]\right)$
        \item 
            $|\XD\cap\B_R| = O\left(\frac{1}{d}\left[\left(1.46\theta\right)^d + \left(\frac{d}{8}\theta - \frac{1}{16}\theta\right)^\frac{d-2}{2}\right]\right)$
        \item 
        $|\XD\cap\B_R| = O\left(\frac{1}{d}\left[\left(1.19\theta\right)^d+\left(\frac{d}{12}\theta\right)^\frac{d-2}{2}\right]\right)$
    \end{enumerate}
\end{cor}
\begin{proof}
    First, we look at $\ZN$. From Theorem~\ref{general_sample_complexity}, we get
    \begin{align*}
        |\XZ\cap \B_R| = O\left(\frac{1}{d}\left(\theta^2\frac{2\pi e}{d}\cdot f_{\ZN}^2\left(d\right)\right)^{\frac{d}{2}} + \left(\theta\cdot f_{\ZN}\left(d\right)\right)^{d-2}\right) \\
        = O\left(\frac{1}{d}\left(\theta\sqrt{\frac{\pi e}{2}}\right)^d+\left(\theta\frac{\sqrt{d}}{2}\right)^{d-2}\right)\\
        = O\left(\frac{1}{d}\left[\left(2.07\theta\right)^d+\left(\frac{d}{4}\theta\right)^\frac{d-2}{2}\right]\right).
    \end{align*}
    Next, we look at $\DN$. From Theorem~\ref{general_sample_complexity}, we get
    \begin{align*}
        |\XD\cap \B_R| = O\left(\frac{1}{d}\left(\theta^2\frac{2\pi e}{d}\cdot f_{\DN}^2\left(d\right)\right)^{\frac{d}{2}} + \left(\theta\cdot f_{\DN}\left(d\right)\right)^{d-2}\right) \\
        = O\left(\frac{1}{d}\left(\theta^2\frac{2\pi e}{d}\cdot \frac{2d-1}{16}\right)^{\frac{d}{2}}+\left(\theta\frac{\sqrt{2d-1}}{4}\right)^{d-2}\right)\\
        =O\left(\frac{1}{d}\left(\theta^2\frac{2\pi e}{8}\cdot \frac{2d-1}{2d}\right)^{\frac{d}{2}}+\left(\theta\frac{2d-1}{16}\right)^{\frac{d-2}{2}}\right).
    \end{align*}
    But, since
    \begin{align*}
        \left(\frac{2d-1}{2d}\right)^\frac{d}{2}=\left(\left(1+\frac{\left(-1\right)}{2d}\right)^{2d}\right)^\frac{\frac{d}{2}}{2d}\overset{n\rightarrow\infty}{\longrightarrow} e^{\frac{1}{4}},
    \end{align*}
    we obtain:
    \begin{align*}
         \|\XD\cap \B_R\|=O\left(\frac{1}{d}\left(\theta\sqrt{\frac{\pi e}{4}}\right)^d+\left(\theta\frac{2d-1}{16}\right)^{\frac{d-2}{2}}\right)\\
         = O\left(\frac{1}{d}\left[\left(1.46\theta\right)^d + \left(\frac{d}{8}\theta - \frac{1}{16}\theta\right)^\frac{d-2}{2}\right]\right).
    \end{align*}
    Lastly, we look at $\AN$. From equation (7) above it can be seen that
    \[
        T^tT=G\left(EP\right)^t\left(EP\right)G^t=
        \begin{pmatrix}
            2 & 1 &  1  & \dots & 1 & -1 \\
            1 & 2  &  1 & \dots & 1 & -1 \\
            . & .  &  . & \dots & . & .  \\
            1 & 1  &  1 & \dots & 2 & -1 \\
            -1 & -1 & -1 & \dots & -1 & x
        \end{pmatrix}\in\mathbb{R}^{d\times d}, x=\frac{D-1}{D}.
    \]
    Furthermore, it can be shown that $T^tT=GG^t=Gr_\Lambda$. This means that we can use Theorem~\ref{general_sample_complexity} to approximate the number of lattice points in the embedded lattice, as they both share the gram matrix $Gr_\Lambda$---which intuitively makes sense, as they both represent the lattice (only that one of them is in a higher dimension).
    From Theorem~\ref{general_sample_complexity}, we get:
    \begin{align*}
        |\XD\cap \B_R| = O\left(\frac{1}{d}\left(\theta^2\frac{2\pi e}{d}\cdot f_{\AN}^2\left(d\right)\right)^{\frac{d}{2}} + \left(\theta\cdot f_{\AN}\left(d\right)\right)^{d-2}\right) \\
        = O\left(\frac{1}{d}\left(\theta^2\frac{2\pi e}{d}\cdot \frac{d\left(d+2\right)}{12\left(d+1\right)}\right)^{\frac{d}{2}}+\left(\theta\sqrt{\frac{d\left(d+2\right)}{12\left(d+1\right)}}\right)^{d-2}\right)\\
        =O\left(\theta^2\frac{1}{d^2}\left(\frac{2\pi e}{12}\cdot \frac{d+2}{d+1}\right)^{\frac{d}{2}}+\left(\frac{d}{12}\theta\right)^{\frac{d-2}{2}}\left(\frac{d+2}{d+1}\right)^{\frac{d-2}{2}}\right).
    \end{align*}
    But, since
    \begin{align*}
        \left(\frac{d+2}{d+1}\right)^\frac{d}{2}=\left(\left(1+\frac{1}{d+1}\right)^{d+1}\right)^\frac{\frac{d}{2}}{d+1}\overset{n\rightarrow\infty}{\longrightarrow} e^{\frac{1}{2}},\\
    \end{align*}
    and similarly for the right term, we obtain
    \begin{align*}
         |\XA\cap \B_R| =O\left(\frac{1}{d}\left(\sqrt{\theta\frac{\pi e}{6}})^d+\left(\frac{d}{12}\theta\right)^{\frac{d-2}{2}}\right)\right)\\
         = O\left(\frac{1}{d}\left[\left(1.19\theta\right)^d+\left(\frac{d}{12}\theta\right)^\frac{d-2}{2}\right]\right).
    \end{align*}
\end{proof}
    \itai{made changes here}
\subsection*{Comparing sample complexity for \Lattices}
    What is quickly clear from these results, is that in terms of sample complexity we get increasingly better evaluations going for the lattices: $\mathbb{Z}^d\rightarrow D_d^*\rightarrow A_d^*$. Both the first term and the error term of the sample complexity expression get better: 
    \begin{enumerate}
        \item The first term gets exponentially better, going from $2.07^d$ to $1.46^d$ and $1.19^d$.
        \item The error term also gets better, with its base going from $\frac{d}{4}$ to $\frac{d}{8}$ and $\frac{d}{12}$. 
    \end{enumerate}
    In both comparisons we disregarded the common $\theta$ term.
    Combining these two improvements, we see that $A_n^*$ is the superior lattice here, in terms of lattice sample density.


    Next, we show that $\XA$ leads to even better performance in terms \emph{CC-complexity}.

% We complete this section with an analysis of the CC complexity of our sample sets. 
% \begin{thm}[$\XZ,\XD,\XA$ CC complexity]
%   Consider the configuration space $\B_{r_c}$ for (rc=?). Then the following upper bounds on CC complexity hold: \kiril{Why is this missing information?}
%     \begin{enumerate}
%         \item 
%         \begin{align*}
%             \frac{\|\chi_{D_d^*,C-Space\|}}{\|\chi_{LB}\|}= ?? \\
%             \|\chi_{D_d^*,\beta-ball}\|= ??
%         \end{align*}        
%         \item 
%         \begin{align*}
%             \frac{\|\chi_{A_d^*,C-Space\|}}{\|\chi_{LB}}\|\ = ?? \\
%             \|\chi_{A_d^*,\beta-ball}\|= ??
%         \end{align*}
%     \end{enumerate}
% \end{thm}

%%% Local Variables:
%%% mode: latex
%%% TeX-master: "../main"
%%% End:

\section{Sample complexity}\label{sec:sample_complexity}
We derive the following lower and upper bounds on the sample complexity of the sets $\XZ,\XD$, and $\XA$. %We start with a general lower bound that applies to any sample set (i.e., not necessarily one derived from a lattice) that is a ${\beta^*}$-cover. We then proceed to lattice-based lower and upper bounds. %To simplify the analysis, for the remainder of the paper, we assume that the configuration space is a $d$-dimensional $R$-ball for some $R\in (0,\infty)$, i.e., $\C:=\B_R$. 


%\subsection{Lattice-based bounds}
% The above result directly follows from~\cite{verger2005covering} which provides a lower bound on the number of $\tfrac{1}{2}$-balls necessary to cover a ball of radius $R \geq \frac{d}{2}$. A simple rescaling argument is used to make it applicable to our setting where ${\beta^*}\left(\de\right)$-balls are used for covering the configuration space. \kiril{Does this description give justice to what you derived or is there something more complicated that I'm missing here?}

% This is a fair upper bound on ${\beta^*}=\delta\alpha$, as $\alpha \leq 1$ and we already take $\delta$ as small value, usually. For example, for a 10 degree problem, we would need ${\beta^*}\leq\frac{1}{6}\approx 0.17$---then for $\delta = 0.1$ we would need $\epsilon < 1.7$. \kiril{I don't understand this discussion. Please clarify.}
% NOTE \itai{I commented out all this part \left(see tex file\right). Its irrelant as I rewrote the theorem.}
% Let us compare this bound with the previous bound~\cite[Theorem~1]{tsao2020sample}, which employed volume arguments. \kiril{Is this the bound you had in mind?} In particular, the previous bound is at least 
% \kiril{please fill in.}
% Thus, the new upper bound is bigger by a factor of \kiril{fill} than the old one. 

% \kiril{The following text needs to be clarified as in the first equation the new bound is stated and I substituted it with the text above.}
% {\color{blue} To compare this LB to the "classic" volume-derived lower bound, let us return to a configuration space defined in a ball of general radius $r\geq {\beta^*} d$. \kiril{return from where? let's keep everything consistent and keep the same setting throughout the paper unless absolutely necessary.}
% We get the following lower bound:
% \begin{align}
%     n \geq k d\cdot\left(2r_0\right)^d =k\cdot d \left(\frac{r}{{\beta^*}}\right)^d 
% \end{align}
% NOTE \itai{I commented out all this part also \left(see tex file\right). To clarify: this is a comparison to the classic "volume divided by volume" lower bound, I didn't compare this lower bound to the one in tsao. you dealt with a square C-Space there and I found it hard to compare your approximation to this one.}
% \begin{proof}
%   We rely on , which states that if a sample set $\X$ is a for  then
% \begin{align}
%    |\X| \geq k\cdot d \cdot\left(2R\right)^{d},\,k>0. 
% \end{align}
% \kiril{What is the purpose of $k$? Does it hold for any $k$? If so, can we substitute $\geq$ with $>$ and get rid of $k$?}
% In our setting, we wish to construct a ${\beta^*}\left(\de\right)$-cover of our configuration space. 
% We want $\frac{1}{2}$-balls in~\cite{verger2005covering}  to be mapped to balls of radius ${\beta^*}\left(\de\right)$-balls, so we need a covering radius of $2{\beta^*}$. This means that if we want to cover a general ball of radius $r$ by ${\beta^*}$-balls, then we need a ball of initial radius $R$ \left(before the rescaling\right) such that:
% \[
%     R\cdot 2{\beta^*}=rarrow R=\frac{r}{2{\beta^*}}
% \]
% Taking $R\geq\frac{d}{2}$ to fit Verger's result, we reach the following bound on $r>0$:
% \begin{align*}
%     \frac{d}{2} \leq R = \frac{r}{2{\beta^*}} arrow r \geq {\beta^*} d
% \end{align*}
% To get a feeling for this bound, let us choose our configuration space as a ball of radius $r=\frac{\sqrt{d}}{2}$. This is logical because it is the circumsphere of the regular $[0,1]^d$ cube. With this we need to have:
% \[
%     \frac{\sqrt{d}}{2} \geq {\beta^*} d arrow {\beta^*} \leq \frac{1}{2\sqrt{d}}
% \]
% \end{proof}

% We are now going to find new upper bounds for our newly defined sample sets \Lattices. We name an upper bound for a sample set by the name \emph{sample complexity}. 
% Throughout the analysis, we assume that the configuration space is the Euclidean $d$-dimensional $R$-ball denoted by $\B_R$. This assumption allows us to obtain tighter bounds, than for, say, unit hypercube configuration spaces, which were assumed in previous sample-complexity analysis. 
% To this end, we first introduce the \emph{Gram matrix} of a lattice, relevant to the following Theorem:
%   \begin{definition}[Gram matrix and quadratic form]
%       The \emph{Gram matrix} of a lattice $\Lambda$ with generator matrix $G_\Lambda$ is defined to be $Gr_\Lambda:=G_\Lambda G^t_\Lambda$. Additionally the quadratic form of $\Lambda$ is defined as $q_\Lambda\left(a\right):=aGr_\Lambda a^t$.
%   \end{definition}
% Next, we will show that those properties lead to superior sample and collision check complexity, among the three lattice types we consider in this work. We mention that in dimension $2$, the lattice $A_2^*$ corresponds to the known "Hexagonal Grid", and in dimension $3$, the lattices $A_3^*$ and $D_3^*$ coincide as the "Body Centered Cubic" (or "Staggered Grid" in~\cite{dayan2023near}).

%Next, we derive analytical lower and upper bounds that exploit the underlying lattice structure.  %Later on, in our experimental results (Section~\ref{sec:experiments}), we compare the sample complexity of those sets for varying values of $\de$, and $d$. 

% previous long version
% \begin{thm}[Sample-complexity bounds]
% \label{thm:general_sample_complexity}
%     Consider a lattice $\Lambda\in \{\dZ^d,D_d^*,A_d^*\}$ with a covering radius $f_\Lambda$, which yields the \decomp set $\XL$ for some $\delta>0,\epsilon>0$, according to Theorem~\ref{thm:decomp_lattices}. Fix a radius $R>0$ and denote $\theta:=\frac{R}{{\beta^*}}$ and $b_d:=\partial(\B_1)$. Then, 
% \begin{align}\label{eq:sample bounds}
%         |\XL\cap \B_R| = \frac{b_d}{\sqrt{\det(\Lambda)}}\theta^df^d_\Lambda + P_d(\theta f_\Lambda), %:=n_{\textup{lattice}}^{\de},
%     \end{align}
%     where for a value $\alpha>0$, $P_d(\alpha)=\Omega(\alpha^{d-2})$ for $d\geq 3$. \kiril{Aren't we missing some upper bounds here?} Additionally, specific results exist for $P_3(\alpha)=O\left(\alpha^{\frac{21}{32}+\epsilon}\right)$ %(see Heath et al.~\cite{heath1999lattice})
%     and  $P_4(\alpha)=O\left(\alpha\log^{2/3}\alpha\right)$ %(see Walfisz et al.~\cite{walfisz1959gitterpunkte})
% .\footnote{For two univariate functions $f,g$, the notation $f(\alpha)=O(g(\alpha))$ (or $f(\alpha)=\Omega(g(\alpha))$)  means that there exists a constant $m_u>0$ (or $m_l>0$) such that for a large enough $\alpha$ it holds that $f(\alpha)\leq m_u(g(\alpha))$ (or $f(\alpha)\geq m_l(g(\alpha))$). Since in our context, $\alpha=\frac{R}{{\beta^*}}f_\Lambda$, where $f_\Lambda$ is fixed, the bounds hold for a sufficiently large ratio $R/{\beta^*}$.}
%     Specifically, for $R=r(\de)$ as in ~\Cref{lem:cover} we can write explicitly that \begin{align}\label{eq:sample bounds}
%         n_{\textup{lattice}}^{\de}= 
%         \frac{b_d}{\sqrt{\det(\Lambda)}}\left(2f_\Lambda\left(1+\frac{1}{\epsilon}\right)\right)^d + P_d\left(f_\Lambda\left(1+\frac{1}{\epsilon}\right)\right),
%     \end{align}
%     that is, the upper limit on the number of samples in the PRM connection-radius ball depends only on $d,\epsilon$.
% \end{thm}

\begin{thm}[Sample-complexity bounds]
\label{thm:general_sample_complexity}
    Consider a lattice $\Lambda\in \{\dZ^d,D_d^*,A_d^*\}$ with a covering radius $f_\Lambda$, which yields the \decomp set $\XL$ for some $\delta>0,\epsilon>0$. %, according to Theorem~\ref{thm:decomp_lattices}.
    Then,
\begin{align}\label{eq:sample bounds}
        |\XL\cap \B_{r^*}|= 
        \frac{\partial(\B_1)}{\sqrt{\det(\Lambda)}}\btheta_{r^*}^d + P_d(\btheta_{r^*}),
    \end{align}
where $\btheta_{r^*}:=2f_\Lambda\left(1+\frac{1}{\epsilon}\right)$, and $P_d(\alpha)\in \dR$ is the discrepancy function~\cite{ivic2004lattice}.\footnote{For a value $\alpha>0$, $|P_3(\alpha)|=\Omega_+(\sqrt{\alpha\log(\alpha)})$ and $|P_3(\alpha)|=O\left(\alpha^{\frac{21}{32}+\epsilon}\right)$, $|P_4(\alpha)|=\Omega(\alpha^{2})$, $|P_4(\alpha)|=O\left(\alpha\log^{2/3}\alpha\right)$, and $|P_d(\alpha)|=\Theta(\alpha^{d-2})$ for $d>4$. Notice that $P_d$ can be negative.  For two functions $f,g$, the notation $f(\alpha)=O(g(\alpha))$ (or $f(\alpha)=\Omega(g(\alpha))$)  means that there exists a constant $m_u>0$ (or $m_l>0$) such that for a large enough $\alpha$ it holds that $f(\alpha)\leq m_u(g(\alpha))$ (or $f(\alpha)\geq m_l(g(\alpha))$). %Since in our context, $\alpha=\frac{R}{{\beta^*}}f_\Lambda$, where $f_\Lambda$ is fixed, the bounds hold for a sufficiently large ratio $R/{\beta^*}$.
}
\end{thm}

\begin{proof}
  We estimate the size of the sample set $\XL$ induced by the lattice $\Lambda$ and the scaling factor $f_\Lambda$ by exploiting the relation between $\XL$ and the grid lattice~$\dZ^d$. In particular, a ball with respect to $\XL$ can be viewed as a rescaled ball for the  $\dZ^d$ lattice. This allows the use of bounds on the number of $\dZ^d$ points within an ellipse. 

  Fix a ball radius $R>0$. 
  Due to the rescaling performed in Theorem~\ref{thm:decomp_lattices}  we transition from the lattice $\Lambda$, which is a $f_\Lambda$-cover for $\dR^d$, into the set $\XL$, which is a ${\beta^*}$-cover for $\dR^d$. In particular, the rescaling factor is $w_\Lambda:={\beta^*}/f_\Lambda$, which is multiplied by $\Lambda$ to obtain $\XL$ (for $\AN$ we also applied an isometric transformation, but this does not change the scale reasoning). Thus, the ball $\B_R$ with respect to $\XL$ can be viewed as the ball $\B_{\btheta}$, where $\btheta_R=R w_\Lambda=\frac{R}{{\beta^*}}f_\Lambda$. Thus, $|\XL\cap \B_R|=|\Lambda\cap\B_{\btheta_R}|$. For the remainder of the proof, we wish to bound the expression $|\Lambda\cap\B_{\btheta_R}|$.

  Let $v=aG_\Lambda$ be a lattice $\Lambda$ point, where $a\in\mathbb{Z}^d$. By definition of the \emph{Gram matrix} $\text{Gr}_\Lambda:=G_\Lambda G_\Lambda^t$, we obtain 
  \[
    a\text{Gr}_\Lambda a^t:=aG_\Lambda G^t_\Lambda a^t=\|aG_\Lambda\|^2=\|v\|^2.
  \]
This leads to the relation
  \begin{align*}
      \B_{\btheta_R}\cap \Lambda &= \{v\in\Lambda |\,\|v\| \leq \btheta_R\} =\{a\in\dZ^d|a \text{Gr}_\Lambda a^t\leq \btheta_R^2\} \\ & =E_{\btheta_R^2}\left(\text{Gr}_\Lambda\right)\cap \dZ^d,
  \end{align*}
  where $E_{s}\left(A\right):=\{x\in\dR^d|x A x^t\leq s\}$ for a matrix $A$ and $s>0$. 
  
  Next, observe that the Gram matrix $Gr_\Lambda$ of the lattices \Lattices contains only rational numbers, and hence defines a \emph{rational quadratic form}, which allows us employ \emph{rational ellipsoid bounds}~\cite{ivic2004lattice} for $|E_{\btheta_R}\left(\text{Gr}_\Lambda\right)\cap \dZ^d|$. Hence,
\begin{align}
    |\XL\cap \B_R
|&=|\B_{\btheta_R}\cap \Lambda|=\left|E_{\btheta_R}\left(\text{Gr}_\Lambda\right)\cap \dZ^d\right|\nonumber\\
&=\frac{\partial(\B_{\btheta_R})}{\sqrt{\det(\Lambda)}}+P_d(\btheta_R)=\frac{\partial(\B_1)}{\sqrt{\det(\Lambda)}}\btheta_R^d+P_d(\btheta_R). \label{eq:sample_complexity}
\end{align}
%where $P_d(\btheta_R)$ is as defined in the statement of Theorem~\ref{thm:general_sample_complexity}.
%For the last transition, see the definition of $A(x,E_p)$ in~\cite[Page 16]{ivic2004lattice}. \kiril{What do mean here? The transition only uses the definition of the ball volume.} 
The expression in Equation~\eqref{eq:sample bounds} immediately follows by plugging
\[\btheta_{r^*}=\frac{r^*}{{\beta^*}}f_\Lambda = \frac{2\delta(1+\epsilon)}{\sqrt{1+\epsilon^2}}\cdot \frac{\sqrt{1+\epsilon^2}}{\delta\epsilon} f_\Lambda = \frac{2(1+\eps)}{\eps}f_\Lambda.\]
\end{proof}
% Hence,
% \begin{align}\label{volume_delta}
%     |\B_{\btheta}\cap \Lambda|\leq \frac{\vol\left(\B_{\btheta}\right)}{\sqrt{d}}+b\btheta^{d-2}.
% As Equation~(\ref{volume_delta}) deals with the unscaled lattice $\Lambda$, we now use the rescaling performed in Theorem~\ref{thm:decomp_lattices} to extend the analysis above to $\XL$. In particular, recall that from the proof in Theorem~\ref{thm:decomp_lattices}, we have that  $\|\XL\cap \B_R\|=\|\Lambda\cap\B_{\bar{\theta}}\|$. Thus, using Equation~(\ref{volume_delta}) we have that 
% \begin{align*}
%     \|\XL\cap \B_R\|&\leq \frac{1}{\sqrt{\pi}d}\left(\frac{2\pi e}{d}\right)^{\frac{d}{2}}\bar{\theta}^d+b\bar{\theta}^{d-2}, %\\
%   %&= \frac{1}{\sqrt{d}}
%    % \left(\frac{2\pi e}{d}\right)^{\frac{d}{2}}\left(\frac{Rf_\Lambda}{{\beta^*}}\right)^d+c\bar{\theta}^{d-2} %\\
%    % &=\frac{1}{\sqrt{\pi}d}\left(\frac{R}{{\beta^*}}\right)^d\left(\frac{2\pi e}{d}\cdot f_\Lambda^2\left(d\right)\right)^{\frac{d}{2}} +  c\left(\frac{R}{{\beta^*}}\right)^{d-2}\left(f_\Lambda\left(d\right)\right)^{d-2}, \nonumber %\\
%    %& \leq \frac{1}{\sqrt{\pi}d}\left(\frac{R}{{\beta^*}}\right)^d\left(\frac{2\pi e}{d}\cdot f_\Lambda^2\left(d\right)\right)^{\frac{d}{2}} +  c\left(\frac{R}{{\beta^*}}\right)^{d-2}\left(f_\Lambda\left(d\right)\right)^{d-2},
% \end{align*}
% with some $c>0$ from~(\ref{volume_delta}). This finishes the proof. 

% The following corollary leverages the last theorem and the specific structure of \Lattices to derive specific upper bounds.
% \begin{cor}[$\XZ,\XD,\XA$ sample complexity]\label{cor:specific_sample_complexity}
%     Fix ${\beta^*}={\beta^*}\left(\delta,\epsilon\right)$ and let $\theta:=\frac{R}{{\beta^*}}$. Consider the configuration space  $\C=\B_R$ for $R=\sqrt{d}{\beta^*}$ or $R \geq d{\beta^*}$ \kiril{Why do we have those conditions on $R$ here? They were only needed for the lower bound, no?}. Then the following bounds hold:
%     \begin{enumerate}[topsep=1pt,itemsep=1ex,partopsep=1ex,parsep=1ex]
%         \item 
%             $|\XZ\cap\B_R| = O\left(\frac{1}{d}\left(2.07\theta\right)^d+\left(\theta\sqrt{\frac{d}{4}}\right)^{d-2}\right)$;
%         \item 
%             $|\XD\cap\B_R| = O\left(\frac{1}{d}\left(1.46\theta\right)^d + \left(\theta\sqrt{\frac{d}{8}-\frac{1}{16}}\right)^{d-2}\right)$; \kiril{Add cases for even and odd dimensions.}
%         \item 
%         $|\XD\cap\B_R| = O\left(\frac{1}{d}\left(1.19\theta\right)^d+\left(\theta\sqrt{\frac{d}{12}}\right)^{d-2}\right)$.
%     \end{enumerate}
%     \kiril{Considering the approximations we use, we need to substitute the equality signs here with $~$.}
% \end{cor}
% \begin{proof}
%     Relying on the rescaling coefficients $f_\Lambda$ we got in Theorem~\ref{thm:decomp_lattices}, and the general lattice sample complexity reached in Theorem~\ref{general_sample_complexity}, we work on analyzing the sample complexity of our lattices. First, we look at $\ZN$. From Theorem~\ref{general_sample_complexity}, we get
%     \begin{align*}
%         |\XZ\cap \B_R| &= O\left(\frac{1}{d}\left(\theta^2\frac{2\pi e}{d}\cdot f_{\ZN}^2\right)^{\frac{d}{2}} + \left(\theta\cdot f_{\ZN}\right)^{d-2}\right) \\
%         &= O\left(\frac{1}{d}\left(\theta\sqrt{\frac{\pi e}{2}}\right)^d+\left(\theta\frac{\sqrt{d}}{2}\right)^{d-2}\right)\\
%        & = O\left(\frac{1}{d}\left(2.07\theta\right)^d+\left(\theta\sqrt{\frac{d}{4}}\right)^{d-2}\right).
%     \end{align*}

%    Next, we look at $\DN$:
%     \begin{align*}
%         |\XD\cap \B_R| &= O\left(\frac{1}{d}\left(\theta^2\frac{2\pi e}{d}\cdot f_{\DN}^2\right)^{\frac{d}{2}} + \left(\theta\cdot f_{\DN}\right)^{d-2}\right) \\
%         &= O\left(\frac{1}{d}\left(\theta^2\frac{2\pi e}{d}\cdot \frac{2d-1}{16}\right)^{\frac{d}{2}}+\left(\theta\frac{\sqrt{2d-1}}{4}\right)^{d-2}\right)\\
%         &=O\left(\frac{1}{d}\left(\theta^2\frac{2\pi e}{8}\cdot \frac{2d-1}{2d}\right)^{\frac{d}{2}}+\left(\theta\sqrt{\frac{d}{8}-\frac{1}{16}}\right)^{d-2}\right).
%     \end{align*}
%     Due to
%     \begin{align*}
%         \left(\frac{2d-1}{2d}\right)^\frac{d}{2}=\left(\left(1+\frac{\left(-1\right)}{2d}\right)^{2d}\right)^\frac{\frac{d}{2}}{2d}\overset{d\rightarrow\infty}{\longrightarrow} e^{\frac{1}{4}},
%     \end{align*}
%     we obtain:
%     \begin{align*}
%          \|\XD\cap \B_R\|&=O\left(\frac{1}{d}\left(\theta\sqrt{\frac{\pi e}{4}}\right)^d+\left(\theta\sqrt{\frac{d}{8}-\frac{1}{16}}\right)^{d-2}\right)\\
%          & = O\left(\frac{1}{d}\left(1.46\theta\right)^d + \left(\theta\sqrt{\frac{d}{8}-\frac{1}{16}}\right)^{d-2}\right).
%     \end{align*}
    
%     Lastly, we look at $\AN$. From equation (7) above it can be seen that
%     \[
%         T^tT=G\left(EP\right)^t\left(EP\right)G^t=
%         \begin{pmatrix}
%             2 & 1 &  1  & \dots & 1 & -1 \\
%             1 & 2  &  1 & \dots & 1 & -1 \\
%             \vdots & \vdots  &  \vdots & \ddots & \vdots & \vdots  \\
%             1 & 1  &  1 & \dots & 2 & -1 \\
%             -1 & -1 & -1 & \dots & -1 & \frac{d}{d+1}
%         \end{pmatrix}\in\dR^{d\times d}.
%     \]
%     Furthermore, it can be shown that $T^tT=GG^t=Gr_{\AN}$. Thus, we can use Theorem~\ref{general_sample_complexity} to approximate the number of lattice points in the embedded lattice, as they both share the Gram matrix $Gr_{\AN}$:%---which intuitively makes sense, as they both represent the lattice (only that one of them is in a higher dimension).
%     %From Theorem~\ref{general_sample_complexity}, we get:
%     \begin{align*}
%         |\XD\cap \B_R|& = O\left(\frac{1}{d}\left(\theta^2\frac{2\pi e}{d}\cdot f_{\AN}^2\right)^{\frac{d}{2}} + \left(\theta\cdot f_{\AN}\right)^{d-2}\right) \\
%         &= O\left(\frac{1}{d}\left(\theta^2\frac{2\pi e}{d}\cdot \frac{d\left(d+2\right)}{12\left(d+1\right)}\right)^{\frac{d}{2}}+\left(\theta\sqrt{\frac{d\left(d+2\right)}{12\left(d+1\right)}}\right)^{d-2}\right)\\
%         &=O\left(\frac{\theta^d}{d}\left(\frac{2\pi e}{12}\cdot \frac{d+2}{d+1}\right)^{\frac{d}{2}}+\left(\theta\sqrt{\frac{d}{12}}\right)^{d-2}\left(\frac{d+2}{d+1}\right)^{\frac{d-2}{2}}\right).
%     \end{align*}
%     Due to
%     \begin{align*}
%         \left(\frac{d+2}{d+1}\right)^\frac{d}{2}=\left(\left(1+\frac{1}{d+1}\right)^{d+1}\right)^\frac{\frac{d}{2}}{d+1}\overset{d\rightarrow\infty}{\longrightarrow} e^{\frac{1}{2}},\\
%     \end{align*}
%     and similarly for the right term, we obtain
%     \begin{align*}
%          |\XA\cap \B_R|& =O\left(\frac{\theta^d}{d}\left(\sqrt{\frac{\pi e}{6}}\right)^d+\left(\theta\sqrt{\frac{d}{12}}\right)^{d-2}\right)\\
%          & = O\left(\frac{1}{d}\left(1.19\theta\right)^d+\left(\theta\sqrt{\frac{d}{12}}\right)^{d-2}\right).
%     \end{align*}
%     \kiril{Note to self: I need to revisit this proof after the details of the approximation are made clear.}
% \end{proof}

\niceparagraph{Discussion.} 
The expression in Equation~\eqref{eq:sample bounds} is identical for our three sample sets, except for the value of the covering radius $f_\Lambda$. This highlights the fact that a smaller covering radii leads to a lower sample complexity. Also, notice that the value $f_\Lambda$ is raised to the power of $d$ in Equation~\eqref{eq:sample bounds}, which emphasizes the difference between the sets in terms of sample complexity. 
See a plot of the sample complexity in theory and practice in~\Cref{fig:limit_graph_upper}, wherein $\XA$ has the lowest sample complexity (except in dimensions $3$ where it coincides with $\XD$). Notice that the theoretical bounds are well-aligned with the practical values. 

For example, consider $d=12$, where the $\XA$ sample set is  $\approx 4$ times smaller than $\XD$.  Even when this difference is not as big, e.g., in dimension $6$ where the size of $\XA$ is  $\approx 1.63$ times smaller than $\XD$, we observe tremendous impact in terms of the running of the motion-planning algorithm in experiments. This follows from the fact that the sample complexity studied here corresponds to the branching factor of the underlying search algorithm, which is known to have a significant impact on the running time. In the next session, we show that the superiority of $\XA$ is maintained also for the collision-check complexity metric.

\begin{figure}[thb]
% \centering  
\includegraphics[width=\columnwidth]{Images/sample_complexity.pdf}
\caption{A sample-complexity  plot for the sample sets $\XZ,\XD$, and $\XA$ with $\delta=1$ and $\eps=2$. The dashed line represents the theoretical approximation (\Cref{eq:sample bounds}), where the asymptotic error term $P_d$ is excluded. The solid line depicts the practical value, i.e., the  number of lattice points within the $r^*$-ball in practice. Missing values are due to memory limitations.}
\label{fig:limit_graph_upper}
\end{figure}


    %Our upper bounds suggest that the sample complexity improves as we move from $\XZ$ to $\XD$, and $\XA$. Both the first term and the error term of the sample complexity expression get better: (i) The first term gets exponentially better, going from $2.07^d$ to $1.46^d$ and $1.19^d$. (ii) The second term also gets better, with its base going from $\sqrt{\frac{d}{4}}$ to $\sqrt{\frac{d}{8}}$ and $\sqrt{\frac{d}{12}}$. (In both comparisons we disregarded the common $\theta$ term.)

%Our experimental results in Section~\ref{sec:experiments} indicate that our theoretical analysis is quite tight in terms of the size of the actual size of $\XL$ as compared to the bounds and correspondingly in terms of the rankings of the sample sets. That is, the sample set $\XA$ is superior in theory and practice. 

    %Finally, we compare the general lower-bound (Equation~\eqref{eq:general_LB}) with the lattice-based lower-bound (Equation~\eqref{eq:sample bounds}). In particular, note that $n_{VG}^{\de}=\Omega(d\theta^d)$, whereas $n_{\textup{lattice}}^{\de}=\frac{b_d}{\sqrt{\det(\Lambda)}}\theta^df^d_\Lambda + \Omega(\theta^{d-2}f_\Lambda^{d-2})$, which implies that the latter is always bigger, and so sharper. \kiril{Is this correct? The lattice bound doesn't have a $d$ multiplied with $\theta^d$.} \kiril{Can we say about it something more concrete? Can we plot the general lower bound in the experiments or is it difficult?} \itai{lets discuss this} \kiril{Remind me, what is the conclusion we ended up with?}

% We complete this section with an analysis of the CC complexity of our sample sets. 
% \begin{thm}[$\XZ,\XD,\XA$ CC complexity]
%   Consider the configuration space $\B_{r_c}$ for (rc=?). Then the following upper bounds on CC complexity hold: \kiril{Why is this missing information?}
%     \begin{enumerate}
%         \item 
%         \begin{align*}
%             \frac{\|\chi_{D_d^*,C-Space\|}}{\|\chi_{LB}\|}= ?? \\
%             \|\chi_{D_d^*,{\beta^*}-ball}\|= ??
%         \end{align*}        
%         \item 
%         \begin{align*}
%             \frac{\|\chi_{A_d^*,C-Space\|}}{\|\chi_{LB}}\|\ = ?? \\
%             \|\chi_{A_d^*,{\beta^*}-ball}\|= ??
%         \end{align*}
%     \end{enumerate}
% \end{thm}

%%% Local Variables:
%%% mode: latex
%%% TeX-master: "../main"
%%% End:

\section{Collision-check complexity}\label{sec:collision_complexity}
In this section, we seek to analyze the collision-check complexity of our sample sets $\XZ,\XD,\XA$, as the length of the graph edges is a better proxy for the computational complexity of constructing the resulting PRM than sample complexity. 

%The biggest effect on the runtime of a PRM algorithm are the collision checks. Specifically, running neighborhood checks (like r-NN) forces you to check along all the edges connecting the center of a ball with other vertices for the purpose of finding a possible collision.


%For this reason, we would like to have some estimate on the overall size of the sum of lengths from the ball center to the other vertices.
%
% To make calculations simpler, we'll use $\sum\|\cdot\|^2$ instead. \kiril{Note to self: Need to revise the definition of CC-complexity to account for this squared value and provide a motivation about it.}


%\itai{Some thoughts: this sum can also be seen as the Moment of inertia of a collection of particles of similar mass, scattered on a lattice inside a sphere. Is this useful to note? if anything, the connection could be interesting. In the beginning I tried to find some way to connect the two, as I thought maybe physicists already have estimates for this kind of thing. For example, the Dn* lattice in d=2 is known to be a crystal lattice for some materials. is our calculation getting the approximate moment of inertia for a pure crystal ball? checking online, iron for example is a BCC crystal. so a crystal iron ball has a moment of $0.4Mr^2$, and you'll see later we reach O($r^2$) approximately. I just wonder if we should point out the connection at any point, even as a comparison.} \kiril{This is interesting. We could point out this connection, in a more concise manner, after Problem 2. This could definitely go into the thesis.}


% Next, introduce a few more definitions that are closely related to the  collision-check complexity bounds.
Recall that the collision-check complexity of a lattice-based sample set $\X_\Lambda$  is 
$CC_{\X_\Lambda}=\sum_{x\in \X_\Lambda\cap \B_{r^*}}\|x\|$.
A naive approach to upper-bound $CC_{\X_\Lambda}$ is to multiply the number of points in a $r^*$-ball (which corresponds to the sample complexity of $\X_\Lambda$) with the radius $r^*$, i.e.,
\begin{align}
    CC_{\X_\Lambda}\leq r^*\cdot |X_\Lambda \cap \B_{r^*}|=r^*\cdot \frac{\partial(B_1)}{\sqrt{\det(\Lambda)}}\btheta^d_{r^*} + r^*\cdot P_d(\btheta_{r^*}). \label{eq:cc_naive}
\end{align}
The following is a tighter bound, which reduces the coefficient of the dominant factor of the naive bound.  

%A tighter bound can be obtained by partitioning the $r$-ball into annuli and separately counting the distances to the points in a given annulus. %In the proof below, for simplicity, we use the bound  $P_d(\btheta)=\Theta(\btheta^{d-2})$, which is applicable to any dimension $d\geq 3$, even though tighter bounds are available for $d\in \{3,4\}$.

%That is, an annulus is defined as  $A^{r'}_{r''}:=\B_{r'}(o)\setminus \B_
%{r''}(o)$ for some $0\le r''<r'\le r$, and  the distances corresponding to this annulus are $r' \cdot |\X\cap A^{r'}_{r''}|$. Note that the expression $|\X\cap A^{r'}_{r''}|$ can be approximated via Equation~\ref{eq:sample bounds} by subtracting from the upper bound of $|\X\cap \B_{r'}(o)|$ the lower bound of $|\X\cap \B_{r''}(o)|$. However, this should be done carefully since this method can yield a strictly positive number of points for an annulus of zero width due to the slackness in Equation~\ref{eq:sample bounds}. We have the following theorem.

%\begin{definition}[Ball collision complexity]\label{def:lattice_sums}
%  Let \(P_{\Lambda,r}:= \{\Lambda\cap \B_{r^*}\left(0\right)\}\) be the lattice $\Lambda$ points inside the $r$-ball centered at $o$, for some $r\geq 0$. The \emph{collision complexity} of $P_{r^*}$ is defined to be
%   $$\text{CC}_{\Lambda}\left(r\right):=\sum_{x\in P_{\Lambda,r}} \|x\|.$$
% Also, let \(P_{F_{\Lambda,p}}:=F_{\Lambda,p}\cap \Lambda\) be the lattice samples that belong to an FR $F_{\Lambda,p}$, and consider another FR $F_{\Lambda,p'}$ where both FRs are defined with the same basis of $\Lambda$, where $p,p'\in \dR^d$. \kiril{I made some clarifications. Are they correct?}
% We define the relative sum to be the squared distance of the points in $P_{F_{\Lambda,p}}$ from the center $c_{\Lambda,p'}$ of $F_{\Lambda,p'}$, i.e.,
%         \[
%            I_{F,p'}:=\sum_{x\in P_{F}} \|c_{\Lambda,p'}-x\|^2.
%           %I_{F_{\Lambda,p}}=
%         \]
%\end{definition}

% Using these definitions, we work towards bounding \(I\left(r\right)\), which would then lead to collision-check complexity bounds. 

\begin{thm}[CC complexity bound]\label{thm:CC}
  Consider a lattice $\Lambda\in \{\dZ^d,D_d^*,A_d^*\}$ with a covering radius $f_\Lambda$, which yields the \decomp set $\X:=\XL$ for some $\delta>0,\epsilon>0$ and dimension $d\geq 2$. %Fix a radius $R>0$.\footnote{Notice that since the connection radius we use is $r:=r(\delta,\epsilon)$, we could potentially limit $R$ to be at most $r$. However, we keep $R>0$ for the sake of generality.}
    % Then we have that 
    % \begin{align}
    %     CC_\xi(R)&\leq \left(1 - \frac{x^{d+2} - x}{ dx - (d+1))}\right)\cdot R \frac{\partial(B_1)}{\sqrt{\det(\Lambda)}}\btheta^d_{R}  \nonumber\\&+ R\cdot P_d(\btheta_R).
    % \end{align}
  Then,
    \begin{align}
        CC_\X\leq \zeta \cdot r^* \cdot \frac{\partial(B_1)}{\sqrt{\det(\Lambda)}}  \btheta_{r^*}^d + r^* \cdot P_d(\btheta_{r^*}), \label{eq:CC_thm}
    \end{align}
    where $\btheta_{r^*}:=2f_\Lambda\left(1+\frac{1}{\epsilon}\right),\zeta:=\left(1 - \frac{\xi^{d+2} - \xi}{ d\xi - (d+1)}\right)<1$ and     
    $\xi:=\left(\frac{d}{d+1}\right)^d$. 
\end{thm}
\begin{proof}
  We improve the naive bound in Equation~\eqref{eq:cc_naive} by partitioning the $r$-ball into annuli. Consider a sequence of ${k+1\geq 2}$  radii ${0<r_k<\ldots<r_0=r^*}$, where $r_i:=\td^i r$ and $\td:=\frac{d}{d+1}$. This leads to the bound
\begin{align}
\label{eq:cc_eval}
CC_\X&\leq \sum_{i=0}^{k-1}r_i |\X\cap (\B_{r_i}\setminus \B_{r_{i+1}})| + r_k|\X\cap \B_{r_k}|\nonumber\\
  &= \sum_{i=0}^{k-1}r_i\left(|\X\cap \B_{r_i}|- |\X\cap \B_{r_{i+1}}|\right)+ r_k|\X\cap \B_{r_k}| \nonumber\\
  &= r^*|\X\cap \B_{r}| + \sum_{i=1}^{k}(r_i - r_{i-1})|\X\cap \B_{r_i}| \nonumber\\
  & =r^*\left(\frac{\partial(B_1)}{\sqrt{\det(\Lambda)}}\btheta^d_{r} +P_d(\btheta^{d-2}_{r})\right) \nonumber\\&+ \sum_{i=1}^{k}(r_i - r_{i-1}) \left(\frac{\partial(B_1)}{\sqrt{\det(\Lambda)}}\btheta^d_{r_i} +P_d(\btheta^{d-2}_{r_i})\right) \nonumber\\
  &= \frac{\partial(B_1)}{\sqrt{\det(\Lambda)}} \left(\underbrace{r^* \btheta^d_{r^*} + \sum_{i=1}^k(r_i-r_{i-1}) \btheta^d_{r_i}}_{:=\gamma}\right) + r^* P_d(\btheta_{r^*}).
\end{align}

Next, it can be shown that  
\[\gamma= r^* \btheta^d_{r^*} \left(1 - \frac{\xi^{d+2} - \xi}{ d\xi - (d+1)}\right)=r^* \btheta^d_{r^*} \zeta,\]
where 
$\btheta_{r_i}= r_i\frac{\btheta_{r^*}}{r^*}, k=d,$ and $\xi:=\td^d=\left(\frac{d}{d+1}\right)^d$. See \conditionaltext{\Cref{app:CC}}{ the supplementary material} for the full details of this derivation, including the motivation behind the definitions of $r_i$ and $k$.

By plugging the value of $\gamma$ in Equation~\eqref{eq:cc_eval}, Equation~\eqref{eq:CC_thm} follows. 
Notice that $\td < 1$. Therefore, $\td^{d(d+1)}-1 < 0$ and $\td^{d+1} - 1 < 0$, so our expression in the parenthesis is less than 1, which implies an improvement over the trivial bound (Equation~\eqref{eq:cc_naive}). 
\end{proof}

\niceparagraph{Discussion.}
Theorem~\ref{thm:CC} leads to a constant-factor improvement over the naive bound concerning the coefficient of the main bound term.  The value of $\gamma$  ranges from $\approx 0.751$ for $d=2$ and monotonically increases towards $1$ as $d\rightarrow\infty$ (see a plot in \conditionaltext{\Cref{fig:annuli_bound:app}}{ the supplementary material}).  Although our result currently leads to modest improvement, we hope it will pave the way for tighter bounds in the future. Nevertheless, our analysis emphasizes the impact of the coverage quality of lattices on the running time, which is more substantial than what the sample-complexity bound suggests. 

See a plot of the theoretical bound, along with the practical values in Figure~\ref{fig:limit_graph_upper}, where both values show similar trends. With that said, the discrepancy between the leading value in \Cref{eq:CC_thm} and the practical value is bigger than in the sample-complexity case, which suggests that the result in Theorem~\ref{thm:CC} could be tightened. Finally, note that collision-check complexity is roughly an order of magnitude larger than the sample complexity, particularly in higher dimensions. 
 
\begin{figure}[thb]
% \centering  
% \includegraphics[width=\columnwidth,trim={3cm 8cm 3cm 8cm},clip]{Images/collision_complexity.eps}
\includegraphics[width=\columnwidth]{Images/collision_complexity.pdf}
\caption{A collision-check complexity  plot for the sample sets $\XZ,\XD$, and $\XA$ with $\delta=1$ and $\eps=2$. The dashed line represents the theoretical approximation (\Cref{eq:CC_thm}), where the asymptotic error term $P_d$ is excluded. } 
\label{fig:limit_graph_upper}
\end{figure}


%the improvement term of the main term as a function of the dimension: notice that the improvement rate diminishes as $d$ gets larger.%Comparatively, the trivial bound would've given us:
%\[
%    CC_\X(r)\leq r \frac{\partial(B_1)}{\sqrt{d}}\btheta^d_{r} + \Theta\left(r \btheta^{d-2}_{r} \right)
%\]
% \begin{figure*}[!h]
%   \centering
%   \subfloat["H" map]{
%     \includegraphics[width=0.3\textwidth]{Images/Hmap_basic.png}
%     \label{fig:exp_3body:hmap}}
%   \hfil
%   \subfloat["Random polygons" map]{
%     \includegraphics[width=0.3\textwidth]{Images/rndpoly_basic.png}
%     \label{fig:exp_3body:rndpoly}}
%   \caption{The scenarios we used to test our sample set, in each map we wanted each disk-robot to switch places with the next neighbor. \kiril{Add more dimensions to the left and right plots.}}
%   \label{fig:exp_3body}
% \end{figure*}

% \kiril{Note to self: revise this paragraph.} To compare these values with a theoretical "better" one, a simulation was run with the following algorithm: 
% \begin{enumerate}
%     \item Assume $d=6,r_0=1, r_D=\left(\frac{D}{D+1}\right)^{k}$.
%     \item In addition to these two radii, add $D-2$ more concentric circles of random radius values between the values $r_0, r_D$.
%     \item For this random division, calculate the fitting $r^{d+1},r^{d-1}$ improvement parameters (we get them due to setting $r=1$).
% \end{enumerate}
% After running this process for 1000 iterations, the best parameters we got were $0.881, 0.821$, for $r^{d+1},r^{d-1}$ (respectively). This suggests that while there is room for improvement in this method, it is close enough to the best improvement we can get in the method of "layered approximation."
% \subsection*{A bit of intuition on the achieved upper bound}
% Notice that asymptotically, as $d$ gets very large, the term $\left(1+\frac{1-e^{-d}}{d}\right)$ goes very quickly to $1$. So at higher $d$ values, we are effectively left with the original sample sizes, only now times the rescaled radius again, getting $O(radius\cdot \#samples)$.
% % \[
% %     I\left(r\right)\approx O\left(\left(\sum_i\|e_i\|^2\right)\btheta^{d-2}\left(\frac{\btheta}{r \left(d-1\right)}+1\right)\right).
% % \]
% % Seeing as $\btheta>1$ and $d$ is high, we can ignore the $+1$ term, and get
% % \[
% %     I\left(r\right)\approx O\left(\frac{\sum_i\|e_i\|^2}{r\left(d-1\right)}\btheta^{d-1}\right),
% % \]
% % and because, as we will see below, we deal with lattices such that ${\sum_i\|e_i\|^2=O\left(d\right)}$, we end up with
% % \[
% %     I\left(r\right)\approx O\left(\frac{d}{r\left(d-1\right)}\btheta^{d-1}\right)= O\left(\frac{1}{r}\btheta^{d-1}\right).
% % \]
% % Recall that in theorem~\ref{general_sample_complexity} our dominant term for evaluating the number of samples is $\left(\btheta\right)^{d-2}$. Expecting an average distance of $c\btheta$ for points in the sphere (
% Looking at random points, we have an expected distance from the center of $\frac{n}{n+1}r$, so also with random points we "expect" to get more or less $O(\frac{n}{n+1}\cdot radius\cdot \#samples)=O(radius\cdot \#samples)$, which is similar to our term, suggesting that our upper bound is at least of the optimal order.
% \itai{Next part is irrelevant with this new computation. with this, it just becomes a direct extension of the sample-complexity section calculations.}
% % our total distance for a group of points is 
% % \[
% %     I\left(r\right)\approx c\btheta \cdot \left(\btheta\right)^{d-2}=O\left(\btheta\right)^{r-1},
% % \]
% % which is very similar to the type of term we managed to get. The specific complexity values for the lattices are not significantly different from this intuition, but it is important to show it rigorously.

%Next, we perform experiments to check the effectiveness of our lattices, compared to each other and to other sample sets.

%%% Local Variables:
%%% mode: latex
%%% TeX-master: "../main"
%%% End:

\section{Experimental results}\label{sec:experiments}
%In the previous two sections, we derived theoretical bounds for the sample complexity and collision-check complexity of lattice-based sample sets. We then tested the tightness of those bounds by comparing them with their actual values in simulations. 
We study practical aspects of our theoretical findings in motion planning problems. We report comparisons between the three lattice-based sets, as well as uniform random sampling. An additional experiment studying the effect of the parameters~$\delta$ and~$\eps$ is reported in\conditionaltext{ the appendix}{ the supplementary material}. 

\begin{figure*}[]
\hspace*{-0.8cm}
  \centering
\subfloat[Kenny ($\uparrow$), Narrow ($\downarrow$)]{
\includegraphics[width=0.364\columnwidth,clip]{Images/Scenarios/K_N_Vert_scenarios.png}
    %\label{fig:3d_lattices:z}
    }
\subfloat[Zigzag]{
\includegraphics[width=0.3\columnwidth,clip]{Images/Scenarios/ZZB1_scenario.png}
    %\label{fig:3d_lattices:z}
    }
\subfloat[Unique Maze]{
\includegraphics[width=0.725\columnwidth,clip]{Images/Scenarios/UM_scenarios.png}
    %\label{fig:3d_lattices:da}
    }
\subfloat[Bugtrap]{\includegraphics[width=0.725\columnwidth,trim={1.1cm 1.1cm 1.1cm 1.1cm},clip]{Images/Scenarios/BT_scenarios.png}
    %\label{fig:3d_lattices:da}
    }

  \caption{A subset of the scenarios used in the experimental results. Some of the figures depict several scenarios, where each scenario consists of a workspace environment, along with an initial configuration specifying the number of robots, their initial positions, and a permutation of their target positions. Each configuration is drawn in a different color, where the scenario name is indicated by the first letters of the workspace name along with a number. (a) (Top) For the "Kenny" workspace, there is a single scenario K1, where robot $1$ starts in the bottom position, robot $2$ in the top position, and robot $3$ in the remaining position, where the disc size corresponds to the robot geometry. In this map, the robots are forced to perform a simultaneous movement. (a) (Bottom)   
  For the "Narrow Room" workspace, we illustrate the scenarios N3,N4, and N5. Due to overlap, the discs in N4 are slightly shrunk for better visualizations. (b) The Zigzag scenario contains narrow, winding passages in which one of the robots can hide in a pocket to let the other robot pass. Swaps can also occur in the top or bottom corners, albeit with greater care. (c),(d) Those workspaces are taken from OMPL~\cite{sucan2012the-open-motion-planning-library}.}
  \label{fig:scenarios}
\end{figure*}

% \begin{figure}[H]
%   \centering
%   
%   \caption{words}
%   \label{fig:3d_lattices}
% \end{figure}

\subsection{Implementation details and planners}
The experiments were performed on an ASUS Vivobook 16x laptop equipped with an Intel Core i9-13900H CPU, 32GB DDR4 RAM, and SSD storage, running Ubuntu 22.04.5 LTS OS.
The planners were implemented in C\texttt{++} within OMPL~\cite{sucan2012the-open-motion-planning-library}, with FCL~\cite{Pan2012FCL} for collision detection, and GNAT~\cite{gipson2013resolution} for nearest-neighbor search (where applicable).

We use a single-query planner where an implicitly-represented PRM graph is explored using a search heuristic similar to FMT*~\cite{JSCP15}, BIT*~\cite{GammellBS20}, and GLS~\cite{MandalikaCSS19}. Those state-of-the-art approaches are well-suited for settings where samples are generated in large batches, as lattice-based sampling facilitates. We focus on the single-query setting, as it allows us to experiment with more complex problem scenarios (e.g., in terms of dimensions and tightness) than in a multi-query setting where the entire configuration space needs to be explored, which requires additional memory and compute time. 

The planner we use, which is termed for simplicity implicit A* (iA*), can be viewed as a simplified version of BIT* with a single sample batch searched using the A* algorithm. iA* generates a sample set $\X$ from a given sample distribution  ($\XZ,\XD,\XA$ or uniform random sampling). Instead of constructing the entire PRM graph $G:=G_{\M(\X,r)}$ resulting from $\X$ and a given radius parameter $r$, iA* constructs a partial graph $G'\subset G$ in an implicit manner, where the construction is guided by the underlying A* search. That is, when a vertex $v$ of $G$ is expanded, its neighbors $N_v$ within an $r$-neighborhood are retrieved from $\X$, and the edges between $v$ and every $u\in N_v$ are collision-checked and added to the explored portion of the graph $G'$. For lattice samples, we set $r:=r^*$. The radius for random samples is described below. 

We consider two flavors of iA*. In the first flavors, 
denoted by \glo, vertex neighbors (as $N_v$ above) are retrieved by calling a \emph{global} nearest-neighbor (NN) data structure. Before starting the A* search, this data structure is initialized with the set $\X$. %The use of NN is standard in sampling-based planners in general and in implicit search planners such as~\cite{JSCP15,GammellBS20,MandalikaCSS19}, in particular. 

Although the benefits of lattice-based sampling over randomized sampling are already apparent for the \glo flavor (especially $\XA$), the performance of iA* can be further improved by exploiting the \emph{local} regular structure of lattices. In the second flavor of iA*, denoted by \loc,  which only applies to lattice-based sets, vertex neighbors are efficiently retrieved without NN data structures. Given a lattice-based sample set $\X_\Lambda$ and a connection radius $r>0$, denote by $N_x:=\B_r(x)\cap \X_\Lambda$ the $r$ nearest-neighbors of a vertex $x\in \X_\Lambda$. Then, for another sample $x'\in \X_\Lambda$ it holds that $N_{x'}:=\B_r(x')\cap \X_\Lambda=(x'-x)+N_x$, i.e., $N_{x'}$ is a translation of $N_x$. Hence, the computation of the neighbor set $N_0$ is only performed from scratch once\footnote{We compute the neighbor set $N_0$ by performing a breadth-first search from the origin vertex, and traversing integer vectors $v\in \dZ^d$ in an increasing radius around the origin, which are then multiplied by the generator matrix to obtain lattice points. The computational cost of this operation is negligible compared to the  A* search itself and hence omitted from the running time results below.} at the beginning of the run of the search algorithm, where the $N_v$ is easily obtained from vector operations.
We associate every sample $x\in \X_\Lambda$ with the integer vector $v\in \dZ^d$ such that $x:=vG_\Lambda$, which allows us to keep track of explored vertices efficiently. 

\subsection{Scenarios}
We test the planners on a variety of motion-planning problems where $d\in \{4,6,8,10,12\}$. Each scenario consists of a multi-robot system of $m$ labeled planar disc robots that need to (simultaneously) exchange positions (i.e., robot $i\in \{0,\ldots,m-1\}$ moves to the start position of robot $i+1 \mod m$) while avoiding collisions with each other and static obstacles. This setting is considered for two reasons. First, such multi-robot systems can be viewed as Euclidean systems, which allows applying our theoretical results (see discussion in Section~\ref{sec:future} of extensions to more general systems). (Specifically, the configuration space of an $m$-robots system is $\dR^{2m}$.) Second, this setting allows us to test systems of various dimensions while still being able to visualize the problem setting (which is in 2D).  Third, it provides a simple approach for determining the value $\delta$ (see below).

A subset of the tested scenarios is found in Figure~\ref{fig:scenarios} (additional scenarios are found in\conditionaltext{~\Cref{fig:scenarios:app}}{ the supplementary material} along with a detailed description). The scenarios present various difficulty levels for the planners, where the most difficult scenarios consist of narrow passages for the individual robots and a significant amount of coordination between the robots, giving rise to narrow passages in the full configuration space. 

Unless stated otherwise, the parameters $\delta$ and $\eps$ were specified in the following manner. We assigned $\eps:=10$ to focus on running time scalability rather than solution quality. The parameter $\delta$ was initially set to be the clearance of the start configuration (capturing both distances from obstacles and between robots), which was decreased until a solution using $\XA$ was found. We discuss the automatic tuning of $\delta$ and $\eps$  in Section~\ref{sec:future}.

\subsection{Comparison between lattice-based sample sets}
In the first set of experiments, we study the running time and solution quality (in terms of path length) of the three lattice-based sample sets $\XZ,\XD$, and $\XA$ within the \loc flavor of iA* for a selected set of scenarios. Although in some scenarios, the performance between the sample sets can be comparable, in terms of running time, especially for $\XD$ and $\XA$, here we highlight situations where large gaps occur. 

The results are reported in Table~\ref{tbl:lattice_comparison}. (Results for additional scenarios are provided in\conditionaltext{~\Cref{tbl:lattice_comparison:app} in the appendix}{ the supplementary material}.) In terms of running time, $\XA$ outperforms the other sample sets, as predicted by our theoretical results with respect to sample complexity and collision-check complexity. $\XZ$ results in at least one order of magnitude (up to 2 orders) slower running times than the other two sample sets. $\XA$ outperforms $\XD$, being at least $3\times$, and sometimes as much as $10\times$, faster. The notations "dnf" and "-" indicate a running time threshold of 1000 seconds was exceeded and failure to find a solution, respectively. 

Regarding solution quality, $\XZ$ outperforms the other sample sets, except for the last scenario (unless it does not manage to find a solution) due to a denser graph. The difference in the solution length suggests that the completeness-cover relation Lemma~\ref{lem:cover} can be tightened by, e.g., reducing the value of ${\beta^*}$, albeit we emphasize this is a worst-case bound. From a practical perspective, the difference in the path length can be reduced via post-processing techniques with negligible computational cost. To summarize, $\XA$-sampling can drastically reduce computational cost while achieving comparable solution quality to the other sample distribution. For this reason, we omit comparisons with $\XZ$ and $\XD$ in the remainder of this section. 

\begin{table}[]
\caption{Comparison of running time and solution length using lattices-based sample sets (where the underlying lattice is denoted in the table) in the iA*-\loc algorithm. Solution length is normalized with respect to the length obtained using $\XA$. } \label{tbl:lattice_comparison}
\centering
\begin{tabular}{|c||ccc|cc|}
\hline
 & \multicolumn{3}{c|}{\cellcolor[HTML]{EFEFEF} Time (s)} & \multicolumn{2}{c|}{\cellcolor[HTML]{EFEFEF} Length (r)} \\ \cline{2-6} 
\multirow{-2}{*}{\begin{tabular}[c]{@{}c@{}}Scenario\\ (robot \#)\end{tabular}} & \multicolumn{1}{c|}{\cellcolor[HTML]{FFFFC7}$\ZN$} & \multicolumn{1}{c|}{\cellcolor[HTML]{FFFFC7}$\DN$} & \cellcolor[HTML]{FFFFC7}$\AN$ & \multicolumn{1}{c|}{\cellcolor[HTML]{FFFFC7}$\ZN$} & \cellcolor[HTML]{FFFFC7}$\DN$ \\ \hline \hline
\cellcolor[HTML]{ECF4FF}N1(5) & \multicolumn{1}{c|}{165.35} & \multicolumn{1}{c|}{4.59} & 0.36 & \multicolumn{1}{c|}{0.65} & 0.79 \\
\cellcolor[HTML]{ECF4FF}N1B(6) & \multicolumn{1}{c|}{dnf} & \multicolumn{1}{c|}{328.30} & 15.08 & \multicolumn{1}{c|}{dnf} & 0.89 \\ \hline
\cellcolor[HTML]{ECF4FF}BT10(2) & \multicolumn{1}{c|}{-} & \multicolumn{1}{c|}{1.20} & 0.30 & \multicolumn{1}{c|}{-} & 0.92 \\
\cellcolor[HTML]{ECF4FF}BT5(3) & \multicolumn{1}{c|}{0.54} & \multicolumn{1}{c|}{0.14} & 0.06 & \multicolumn{1}{c|}{0.38} & 0.51 \\
\cellcolor[HTML]{ECF4FF}BT1(4) & \multicolumn{1}{c|}{146.69} & \multicolumn{1}{c|}{50.81} & 3.51 & \multicolumn{1}{c|}{0.95} & 1.03 \\
\hline
\cellcolor[HTML]{ECF4FF}K1(3) & \multicolumn{1}{c|}{32.31} & \multicolumn{1}{c|}{4.97} & 1.37 & \multicolumn{1}{c|}{0.82} & 0.89 \\ \hline
\cellcolor[HTML]{ECF4FF}UM4(2) & \multicolumn{1}{c|}{-} & \multicolumn{1}{c|}{8.47} & 2.43 & \multicolumn{1}{c|}{-} & 0.90 \\
\cellcolor[HTML]{ECF4FF}UM2(3) & \multicolumn{1}{c|}{13.35} & \multicolumn{1}{c|}{1.22} & 0.04 & \multicolumn{1}{c|}{1.04} & 1.52 \\
\hline
\end{tabular}
\end{table}



% Please add the following required packages to your document preamble:
% \usepackage{multirow}
\begin{table}[]
\caption{Comparison of running time and solution length between $\XA$ (using \loc and \glo) and uniform random sampling. For \rnd we report the average values in terms of running and solution length (the latter is given as normalized value with  respect to the solution length with $\XA$).}
\label{tbl:lattice_vs_random}
\centering
\begin{tabular}{|c||ccc|c|c|}
\hline
 & \multicolumn{3}{c|}{\cellcolor[HTML]{EFEFEF} Time (s)} & \cellcolor[HTML]{EFEFEF}Length (r) & \cellcolor[HTML]{EFEFEF}Success (\%) \\ \cline{2-6} 
\multirow{-2}{*}{\begin{tabular}[c]{@{}c@{}}Scenario\\ (robot \#)\end{tabular}} & \multicolumn{1}{c|}{\cellcolor[HTML]{FFFFC7}\begin{tabular}[c]{@{}c@{}}$\AN$\\ \loc \end{tabular}} & \multicolumn{1}{c|}{\cellcolor[HTML]{FFFFC7}\begin{tabular}[c]{@{}c@{}}$\AN$\\ \glo \end{tabular}} & \cellcolor[HTML]{FFFFC7}\begin{tabular}[c]{@{}c@{}}\rnd\\ \glo\end{tabular} & \cellcolor[HTML]{FFFFC7}\begin{tabular}[c]{@{}c@{}}\rnd\\ \glo\end{tabular} & \cellcolor[HTML]{FFFFC7}\begin{tabular}[c]{@{}c@{}}\rnd\\ \glo\end{tabular} \\ \hline\cellcolor[HTML]{ECF4FF}N1(5) & \multicolumn{1}{c|}{0.36} & \multicolumn{1}{c|}{3.05} & 4.16 & 1.48 & 80 \\
\cellcolor[HTML]{ECF4FF}N1(5) & \multicolumn{1}{c|}{0.36} & \multicolumn{1}{c|}{3.05} & 4.16 & 1.48 & 80 \\
\cellcolor[HTML]{ECF4FF}N2(5) & \multicolumn{1}{c|}{0.41} & \multicolumn{1}{c|}{2.67} & 2.74 & 2.43 & 65 \\
\cellcolor[HTML]{ECF4FF}N3(5) & \multicolumn{1}{c|}{0.59} & \multicolumn{1}{c|}{3.83} & 5.44 & 2.02 & 85 \\ \hline
\cellcolor[HTML]{ECF4FF}BT3(3) & \multicolumn{1}{c|}{5.38} & \multicolumn{1}{c|}{14.15} & 62.22 & 1.05 & 100 \\
\cellcolor[HTML]{ECF4FF}BT8(3) & \multicolumn{1}{c|}{12.17} & \multicolumn{1}{c|}{19.31} & 169.32 & 1.00 & 100 \\
\cellcolor[HTML]{ECF4FF}BT8B(3) & \multicolumn{1}{c|}{3.17} & \multicolumn{1}{c|}{3.60} & 41.63 & 1.16 & 100 \\ \hline
\cellcolor[HTML]{ECF4FF}UM4(2) & \multicolumn{1}{c|}{2.43} & \multicolumn{1}{c|}{2.93} & 12.71 & 0.96 & 70 \\
\cellcolor[HTML]{ECF4FF}UM1(3) & \multicolumn{1}{c|}{6.68} & \multicolumn{1}{c|}{58.62} & 49.35 & 0.98 & 100 \\
\cellcolor[HTML]{ECF4FF}UM2(3) & \multicolumn{1}{c|}{0.04} & \multicolumn{1}{c|}{2.94} & 4.49 & 1.95 & 75 \\ \hline
\cellcolor[HTML]{ECF4FF}ZZB1(2) & \multicolumn{1}{c|}{0.44} & \multicolumn{1}{c|}{0.49} & 10.44 & 0.89 & 100 \\
\cellcolor[HTML]{ECF4FF}ZZB2(2) & \multicolumn{1}{c|}{0.71} & \multicolumn{1}{c|}{2.51} & 272.70 & 0.89 & 100 \\
\cellcolor[HTML]{ECF4FF}ZZB3(2) & \multicolumn{1}{c|}{0.47} & \multicolumn{1}{c|}{7.69} & 341.84 & 0.88 & 100 \\ \hline
\end{tabular}
\end{table}

\subsection{Comparison with randomized sampling}
We compare the performance of iA* when using $\XA$-sampling and uniform random sampling (denoted by \rnd). We show that the advantage of $\XA$ stems not only from algorithmic speedups due to the regular lattice structure (as in \loc) but also from structural properties leading to a more efficient coverage of space. Hence, we run both \glo and \loc using $\XA$, where \rnd is tested only with \glo. 

For a given scenario, we fix the parameters $\delta$ and $\eps$ used to derive $\XA$. To derive a set of random samples $\XR$, we compute the number of $\XA$ samples within the scenario (while ignoring collision with obstacles and between robots) and produce the same number of points via uniform random sampling. Importantly, when running \glo with $\XR$ we specify the standard connection radius  $r_\text{rnd}(n)=\psi \left(\frac{\log n}{n}\right)^{1/d}$, where $n:=|\XR|$ and $\psi$ is a constant, such that asymptotic optimality is guaranteed~\cite{karaman2011sampling}. The results are reported in  Table~\ref{tbl:lattice_vs_random}, where \rnd is averaged over 20 repetitions. (Results for additional scenarios are provided in\conditionaltext{~\Cref{tbl:lattice_vs_random:app} in the appendix}{ the supplementary material}.)
%For \loc using $\XA$ we report the total running, and solution length. For \glo using $\XA$ we report the running time, which consists of the construction of the NN data structure and the search algorithm. The running time of \glo using uniform random sampling (\rnd) is reported in a similar manner, where we provide both the mean and average values across 50 repetitions. We also report the solution length of \rnd as a ratio between it and the length using $\XA$. Finally, we report the success rate of \rnd. \kiril{How do we determine timeouts for \rnd, and what do we report for runtime in such a case?}\itai{I took the largest possible time value I could without the computer crashing (e.g. 10-20 times the runtime of An* allowed, up to 1000 times in scenarios where An*'s time was VERY short), resulting in just a few timeouts in most runs. they are reported as fails and their runtime ignored (I average over successes). }


Note that \rnd achieves a perfect success rate only in some of the scenarios. Furthermore, its running time is typically at least $5\times$ slower than \loc, and in some scenarios, up to three orders of magnitude slower. This time gap can be partially attributed to \rnd constructing and maintaining a nearest-neighbor data structure. However, notice that in some scenarios, $\AN$-\glo significantly outperforms \rnd. One explanation of that is $\AN$ requires a smaller connection radius than \rnd, which leads to a lower sample complexity and collision check complexity. Another reason is that \rnd, due to poor coverage of space, is forced to make detours and so considers additional vertices and collision checks. We provide further evidence for those points in\conditionaltext{~\Cref{tbl:lattice_vs_random:app} in the appendix}{ the supplementary material}. We also report results for \rnd where the connection radius $r_\text{rnd}(n)$ is substituted with $r^*$, which further reduces success rates and emphasizes the efficiency of space coverage using $\XA$.

%In terms of solution quality, $\XA$ consistently obtains shorter paths (with the exception of one scenario where \rnd obtained a better solution, albeit, with a success rate of only $0.42$), at times twice as short, than \rnd. 

%\yaniv{How is RND different exactly from PRM and from your algo? I would expect that using the same number of vertices results in similar runtimes but lower success rate for RND, maybe try to explain why that isn't the case }

Overall, we conclude that locality can substantially speed up performance from an algorithmic perspective, while the structural properties of $\XA$ also improve performance in terms of success rate, running time, and solution quality. %Finally, we mention that before settling on the scenarios reported in this paper, we considered an extended collection, which also includes easier problems than those reported here (requiring a small number of samples due to the high clearance of the solution). We observed on those simpler problems comparable performance between \loc and \rnd, as expected. Overall, we have not encountered problems where \rnd had any benefit compared with \loc in terms of running time, solution quality, or success rate. 





\section{Limitations and future work}\label{sec:future}
We leveraged foundational results in lattice theory to develop a theoretical framework for generating efficient sample sets for motion planning that endow their planners with finite-time guarantees, which are vastly superior to previous asymptotic properties. We demonstrated the practical potential of the framework in challenging motion-planning scenarios wherein our $A_d^*$-based sampling procedures lead to substantial speeds over previous methods (deterministic and random).
Below, we discuss several limitations of our work, which motivate further research directions.

\niceparagraph{Multi-resolution search.} The choice of the sampling parameters $\delta$ and $\epsilon$ can significantly impact the sample and collision-check complexity of the sampling distribution, and hence the planner's performance. In this work, we selected those values according to a rule of thumb, which might not be generally applicable to broader problem settings, or lead to feasible solutions.  We plan on exploring algorithmic extension, which would allow the automatic selection of those parameters during runtime. One promising direction is exploiting multi-resolution search~\cite{saxena2022amra,FuSSA23}, which allows for adjusting sampling densities according to the space's structure. This has the potential of not only simplifying the usage of our sample sets but also significantly improving the planner's performance.  

\niceparagraph{Non-Euclidean systems.} Our current work focuses on geometric systems (whose state space is Euclidean)  and path-length cost functions, which limits its applicability. In the future, we plan to extend our work to more general settings, which entails at least two challenges. First, the adaptation of our sampling distributions to non-Euclidean spaces is non-trivial due to the heterogeneity of the space coordinates, e.g., position and rotation components in SE(3)~\cite{yershova2004deterministic}, or state derivatives in dynamical systems~\cite{janson2018deterministic}. Second, those settings call for more general cost functions, which requires revising the concept of \decomps and its relation to geometric coverage (particularly, Lemma~\ref{lem:cover}). We plan to address those challenges incrementally. As a first step, we plan on generalizing our theory to linear systems with quadratic costs, where we believe that the structure of LQR controllers~\cite{liberzon2011calculus} can be useful. More general approaches could be obtained from exploiting the locally-linear structure of nonlinear systems using differential geometry~\cite{BulloLewis04}.

\niceparagraph{Incremental sampling.} Uniform random sampling is naturally amenable for densification, i.e., the incremental addition of new sampling points, making them applicable for anytime planner like RRT or RRT*~\cite{LaVKuf01,karaman2011sampling}. In contrast, lattice-based sampling requires the introduction of large batches, which currently limits their applicability. One way to bridge this gap could be generalizing Halton sequences~\cite{kuipers2012uniform}, a popular method for deterministic incremental low-discrepancy sampling, to the $A^*_d$ lattice. Halton sequences ensue from transforming points from the standard grid lattice $\dZ^d$, and it would be interesting to understand the structure emerging from feeding those points through the generator matrix of the lattice $A^*_d$. Another interesting direction is incrementally introducing the elements of lattice-based samples according to a random permutation.

 
%\itai{I also considered at a few points using a dRRT-like "oracle" system to create an RRT that runs on a lattice, somehow.} \kiril{Let's discuss this. }


% incremental

% multi-resolution

% improved CC bounds


% \subsection{Better bounds and square bounds}
% Better upper bounds, using the lattice structure more carefully, may be developed---maybe somehow utilizing the lattice structure itself to get a bound.


% In addition to that, we limited ourselves to looking at the sample and collision complexity inside balls, but it could be possible to adept our results to the case of general cubic C-Spaces. For this, the study of $\ZN$-points in $d$-dimensional parallelotopes has to be conducted , as this translates to the number of $\LN$-points in $d$-dimensional cubes.


% In general, this is a difficult subject with much greater mathematical depths and complexity outside the scope of this paper.

% \subsection{A lattice version of the \decomp theory}
% In this paper we used the theoretical results from Dayan et al.~\cite{dayan2023near}, which fit the circumstance well enough. Still, it is possible that a theorem could be fitted ad-hoc to the case of lattices, resulting in a much clearer statement about the resolution needed to solve a problem efficiently.

% \subsection{Dynamic resolution}
% Across testing, a clear problem became choosing the clearance. In our case, wisely choosing the start locations and the maps assisted us in getting a good clearance. In the general case, though, it isn't clear if it is at all possible to get a good clearance. Even if the optimal clearance for a problem is found, it is possible that the resulting lattice sample set is too dense, resulting in an infeasible runtime. One good direction to explore, then, would be to develop an algorithm similar to what Fu et al.~\cite{fu2021toward} did. In this paper, you attempt to develop a tree-like structure in the kynodynamic settings, using a variable set of scaling factors. Borrowing from that, using the $\AN$ sample set with a changing scaling factor to fit the longest edge you can could prove beneficial. It would also serve as a way to bypass the need to "intelligently" guess the correct $\delta$ value for the map.

% \subsection{Lattices in kinodynamic algorithms}
% In some kinodynamic algorithms, like KinoRRT (ref?) and steerable-needles MP~\cite{fu2021toward}, progressing from where the robot is to the next point uses either randomly sampled controls (like in KinoRRT, in which the time and control inputs are randomly sampled) or deterministicly sampled controls (like in the steerable-needle case, which uses motion primitives). It is perhaps interesting, then, to study the influence of specifically using $\AN$ with these algorithms, perhaps to see if some improvements in some aspects of them could be made.

%%% Local Variables:
%%% mode: latex
%%% TeX-master: "../main"
%%% End:

%\section{LEFTOVERS}

\subsection{General lower bound}
We adapt a previous result to obtain a new lower bound on the sample complexity of \decomp sets. The bound is asymptotic with respect to the dimension, and should be contrasted with our lattice-specific bounds presented later on, which are nonasymptotic. % To simplify the analysis, for the following lemma we assume that the configuration space is a $d$-dimensional $R$-ball for some $R\in (0,\infty)$, i.e., $\C:=\B_R$.

\begin{lemma}[Verger-Gaugry's~\cite{verger2005covering} lower bound]\label{thm:sample_lower}
  Suppose that $\X$ is a  $\beta$-cover. Fix a radius $R>0$ such that  $R=\sqrt{d}\beta$ or $R\geq d\beta$, $\beta:=\beta\left(\delta,\epsilon\right)$. Then for $d>0$ large enough there exists $c>0$ such that 
    \begin{equation}\label{eq:general_LB}
        |\X| \geq   cd\left(\frac{R}{\beta}\right)^d =: n_{VG}^{\de}.
    \end{equation}
    % \triangleq \|\X_{LB}\|
\end{lemma}

\begin{proof}
Theorem 3.10 in~\cite{verger2005covering} states that if a sample set $\X^*$ is a $\frac{1}{2}$-cover for $\B_{R^*}$, where $R^*=\frac{\sqrt{d}}{2}$ or $R^*\geq\frac{d}{2}$, then there exists $c>0$ such that for $d>0$ large enough, $|\X^*|\geq cd\cdot \left(2R^*\right)^d$. Next, we use a rescaling argument to apply those findings to our setting. In particular, we wish to cover $\B_R$ with $\beta$-balls. To convert $\beta$-balls into $\frac{1}{2}$-balls we rescale $\C$ by $\frac{1}{2\beta}$, which yields a ball of radius $R^*=\frac{R}{2\beta}$. In this rescaled space, $\B_{R^*}$ is covered by $\frac{1}{2}$-balls. 

For the radius $R^*$ to fit Verger-Gaugry's requirements, we need to ensure that
\[
    \frac{R}{2\beta} = R^* = \frac{\sqrt{d}}{2},
\]
which implies that $R=\sqrt{d}\beta$, 
and similarly  %\footnote{Notice that this lower bound applies to any ball $\B_r$, as long as it is $\sqrt{d}$ times or more than $d$-times the "minimal radius" ball of $\beta$.}
$R\geq d\beta$ for the other case. Returning to Verger-Gaugry's result, we get the lower bound
\[
    |\X|=|\X^*| \geq cd\cdot \left(2R^*\right)^d = cd\cdot \left(\frac{R}{\beta}\right)^d,
\]
which concludes the proof. %\qedsymbol
\end{proof}

%\kiril{Instead of the following discussion, can you generate a plot for the two conditions $R=\delta\beta,R\geq d\cdot \beta$ in dimensions $3,6,9$, where on $x$-axis you fix $\delta$ and on the $y$-axis you plot the allowed $\eps$-range?} \kiril{Alternatively, move into appendix.} \itai{TODO: discuss this. is this part nessecary?}

%\kiril{Start of text to be commented out.}

We discuss the applicability of the above lemma in our setting and show that the requirement $R\geq d\beta$ is not overly restrictive. \kiril{Update the discussion below to $d=5$ or $6$, which is more relevant. Feel free to change the numbers, but try to stick to our setting, i.e., a ball Cspace. No need to include here all the calculations, just the outcome.}\itai{I added some computations, tell me what you think of this version} Using the requirement above, ~\Cref{lem:cover}---in which we defined $\beta(\delta,\epsilon)$, and using $d=6,R=1$, one can reach the following connection:
\begin{align*}
    \epsilon \leq \frac{1}{\sqrt{36\delta^2-1}}
\end{align*}
First, this implies a minimal clearance of ${\delta>\frac{1}{6}\approx0.17}$. Second, this shows that for low enough $\delta$ values we have an unlimited maximal $\epsilon\rightarrow\infty$ (when $\delta\downarrow0.17$). It does pose a limit on the clearance, but it still enables a good approximation in many situations. See \Cref{fig:limit_graph_lower} for the exact figure showing the relationship between $\delta$ and $\epsilon$. \itai{we may need to remove this part, as we now know that this is asymptotic. I added a note that it is asymptotic.} \kiril{let's keep it. It strengthens our lattice bounds.}\itai{done + graph}
\begin{figure}[thb]
\centering  
\includegraphics[width=0.9\columnwidth]{Images/eps_delta_lower_limit.png}
\caption{Dependence of $\epsilon$ on $\delta$ in order to get a valid lower limit (\Cref{thm:sample_lower}).}
\label{:limit_graph_lower}
\end{figure}


%  if our space's radius is $\sqrt{10}$ times bigger than a $\beta$-ball then the volume is $\sqrt{10}^{10}=10^{\frac{10}{2}}$ bigger than a $\beta$-ball. This means we expect to find about $10^5$ smaller $\beta$-balls in it. With a normal cubic $[0,1]^d$ world, we expect to find about $\frac{1}{\beta^d}=\left(\frac{1}{\beta}\right)^d=\left(\frac{1}{\beta^2}\right)^{\frac{d}{2}}$ balls. Taking the same $d=10$, we get that for our spherical space to have less $\beta$-balls than a cubic space, we will need $\frac{1}{\beta^2}\geq 10 \Rightarrow \beta \leq 0.32$. We get:
% \[
%     \beta=\delta\sqrt{\frac{\epsilon^2}{1+\epsilon^2}}\leq0.32 \iff \frac{\epsilon^2}{1+\epsilon^2}\leq \left(\frac{0.32}{\delta}\right)^2
% \]
% But notice that:
% \[
%     \frac{0.32}{\delta}<1 \iff \delta>0.32
% \]
% So for $\delta\leq 0.32$, we get $\frac{0.32}{\delta}\geq1$. But $\frac{\epsilon^2}{1+\epsilon^2}<1$ for all $\epsilon>0$, which makes the above inequality true for all $\epsilon>0$. All this is to say the following: the requirements on $\delta,\epsilon$ aren't harsh and would probably be fulfilled in most use cases, allowing us to rely on this lower bound.

We conclude this part with a comparison of the bound $n_{VG}^{\de}$ with a naive lower-bound, which asserts that the minimal number of points in a set $\X$ that is $\beta$-cover should be at least the volume of $\B_R$ divided by the volume of a $\beta$-ball, i.e., 
\begin{equation}
        |\X|\geq \frac{\partial(\B_R)}{\partial(\B_\beta)}=\frac{\partial(\B_1) R^d}{\partial(\B_ 1)\beta^d}=\left(\frac{R}{\beta}\right)^d:=n_{\text{naive}}^{\de}, 
\end{equation}
where $\partial(A)$ is the volume, or Lebesgue measure, of a set $A\subset \dR^d$. 
Notice that $n_{VG}^{\de}$ is $d$ times tighter than $n_{\text{naive}}^{\de}$, although this result is asymptotic. 


\subsection{a complicated lettice definition (not needed?)}
A lattice $\Lambda$ with a basis $E$ is defined to be:
\[
    \Lambda\triangleq\smashoperator[r]{\bigcup_{\{p=k\cdot e_i | k\in \mathbb{N},e_i\in E\}}}\{v|v\text{ is a vertex of } F_{\Lambda,p}\}
\]

\subsection{something from the lower bound section Im not sure of anymore}
with $D\subset \mathbb{R}^d$ some set. This calculation is equivalent to a sample set taken uniformly on the set $D$, spaced out by balls of radius $\epsilon>0$.  This means we can extrapolate the number of samples this lower bound would've gotten on the same $\frac{\sqrt{d}}{2}$-radius ball. We know the $[0,1]^n$-cube is of volume 1, so the new number of samples is:
\begin{align*}
    \|\chi_{new-LB}\|=\|\chi_{LB}\|*Vol(B_2(0,\frac{\sqrt{d}}{2})) \\
    =\sqrt{\frac{e}{2}}\Big(1-\frac{2\delta}{1-2\delta}\Big)^2\Big(\sqrt{\frac{d-1}{2\pi e}}\cdot \frac{1-2\delta}{\delta}\Big)^d \cdot c_d(\frac{\sqrt{d}}{2})^d \\
    =c_2c_d\cdot (d(d-1))^{\frac{n}{2}}\Big(\frac{1-2\delta}{2\delta}\Big)^d
\end{align*}
for some $c_2>0$. Comparing the two values:
\begin{align*}
    \frac{\|\chi\|}{\|\chi_{new-LB}\|} = \frac{4c_1\delta d\cdot\frac{d^{\frac{d}{2}}}{(2\delta)^d}}{kc_d\cdot (d(d-1))^{\frac{n}{2}}\Big(\frac{1-2\delta}{2\delta}\Big)^d} \\
    =\frac{c_3\delta c_d^{-1}d}{(d-1)^{\frac{n}{2}}}(1-2\delta)^d
\end{align*}
Using the same approximation for $c_d^{-1}\approx\sqrt{\pi d}(\frac{d}{2\pi e})^{\frac{d}{2}}$:
\begin{align*}
    \approx\frac{c_3\delta \sqrt{\pi d}(\frac{d}{2\pi e})^{\frac{d}{2}}d}{(d-1)^{\frac{n}{2}}}(1-2\delta)^d \\
    =c_4\delta d^{1.5}\Big(\frac{d}{d-1}\Big)^{\frac{d}{2}}\Big(\frac{1-2\delta}{\sqrt{2\pi e}}\Big)^d \\
\end{align*}
Again, using the similar calculations as the previous section, we can use $\Big(\frac{d}{d-1}\Big)^\frac{d}{2}\rightarrow \sqrt{e}$:
\begin{align*}
    \leq c_5\delta d^{1.5}\cdot\Big(\frac{1}{\sqrt{2\pi e}}\Big)^d\rightarrow 0
\end{align*}

\subsection{anstar definitions we dont need}
In general, the $A_d^*$ set , is defined through the $A_d$ set:
\begin{align}
    A_d=\{(x_i)_1^{d+1} \in \mathbb{Z}^{d+1} \mid \sum_i x_i = 0\}
\end{align}
He then continues to define the "dual" to $A_d$ as:
\begin{align}
    A_d^*=\bigcup_{i=0}^d ([i]+A_d)
\end{align}
For some vector $[i]$. 

\subsection{anstar upper bound leftovers}

Following this definition, we can use a well known result: if $T:\mathbb{R}^d\rightarrow\mathbb{R}^d$ then \\ ${Vol(T(S))=\det(T)\cdot Vol(S)}$. Let us first calculate $\det(T)$:
\begin{align}
    \det(T) = \sqrt{\det(T)\det(T)} = \sqrt{\det(T^t)\det(T)} =\sqrt{\det(T^tT)} \nonumber \\ \label{det_T}
    = \det((EPG^t)^t(EPG^t)) = \det(G(EP)^t(EP)G^t)
\end{align}
 In ~\cite{conway2013sphere}, the determinant of the lattice is defined as $\det(GG^t)$ and Conway cites it~\cite{conway2013sphere} to be $\frac{1}{d+1}$. Combine these facts with~(\ref{det_T}) and you get that $\det(T)=\sqrt{n+1}$. 


Let us use this fact to compute two things: first, let us find the size of the whole sample set. Conway~\cite{conway2013sphere} gives us the radius of the most optimal ("thinnest" in the book's terms) ball radius, which is said to be:
\begin{align}
    R=\sqrt{\frac{n(n+2)}{12(n+1)}}
\end{align}
So, let us assume we live in a cube of size $[0,w]^n$ with optimal balls of the above radius $R$. We would like the cube size to be such that if we rescale $[0,w]^n$ to $[\delta,1-\delta]^n$ we would also get $\beta$-Balls from $R$-Balls, to fit with the setting in~\cite{dayan2023near}. So:
\begin{equation}
    \begin{cases}
      wx=1-2\delta\\
      Rx=\beta
    \end{cases}
    \Rightarrow 
    \begin{cases}
         x=\frac{\beta}{R} \\
         w=\frac{(1-2\delta)R}{\beta} = \frac{1-2\delta}{\delta}\frac{R}{\alpha}
    \end{cases}
\end{equation}
So using these facts, we need a set $S\subset \mathbb{R}^d$ such that $T(S)=[0,w]^d$, getting:
\begin{align*}
    Vol(S)=\sqrt{d+1}\cdot Vol([0,w]^d)=\sqrt{d+1}\cdot(\frac{1-2\delta}{\delta}\frac{R}{\alpha})^d\\
    =\sqrt{d+1}(\frac{1-2\delta}{\delta})^d\cdot(\sqrt{\frac{d(d+2)}{12(d+1)}})^d(\frac{\sqrt{1+\epsilon^2}}{\epsilon})^d \\
    =\sqrt{d+1}(\frac{1-2\delta}{\delta})^d\cdot(\frac{d(d+2)}{12(d+1)}(1+\frac{1}{\epsilon^2}))^{\frac{d}{2}}\\
    = \sqrt{d+1}(\frac{1}{\delta} - 2)^d\cdot(\frac{d(d+2)}{12(d+1)}(1+\frac{1}{\epsilon^2}))^{\frac{d}{2}}\\
    \leq \sqrt{d+1}(\frac{1-2\delta}{\delta})^d\cdot(\frac{d(d+2)}{12(d+1)})^{\frac{d}{2}}
\end{align*}
Using similar methods, we will compute the number of samples in a ball of radius $\beta$. We want a set $\Tilde{B}\subset \mathbb{R}^d$ such that $T(\Tilde{B})=B(\delta,c)$ for some center point $c\in\mathbb{R}^d$. Remember from the last paragraph that this comes from a ball of radius $R$, so:
\begin{align*}
    \|B_\beta\cap\Lambda\|\leq \frac{Vol(B_\beta)}{\sqrt{d}}+k(\beta)^{d-2}\leq \frac{\beta^d}{\sqrt{d}c_d^{-1}}+k(\beta)^{d-2}\\
    \approx\beta^{d-2}[\frac{\beta^2}{\pi\cdot d}(\frac{d}{2\pi e})^{\frac{d}{2}} + k]\\
    \|\chi_{Ball}\|\leq Vol(\Tilde{B})=\sqrt{d+1}Vol(B(\beta,c))=\sqrt{d+1}c_dR^d\\
    =\sqrt{d+1}(\frac{d(d+2)}{12(d+1)})^{\frac{d}{2}}\frac{1}{\sqrt{\pi d}(\frac{d}{2\pi e})^{\frac{d}{2}}}\\
    =\sqrt{\frac{d+1}{\pi d}}(\frac{2\pi e \cdot (d+2)}{d+1})^{\frac{d}{2}}\rightarrow \frac{(\sqrt{\frac{2\pi e}{12}})^d}{\sqrt{\pi}}\approx\frac{1.19^d}{\sqrt{\pi}}
\end{align*}

Remembering that we want the integer vectors in this space, and that small cubes in that grid are of volume 1, we get that:
\begin{align}
    \|\chi_{A_d^*}\|\leq Vol(S)=\sqrt{d+1}(\frac{1-2\delta}{\delta})^d\cdot(\frac{d(d+2)}{12(d+1)})^{\frac{d}{2}}
\end{align}
And Comparing it now to $\|\chi_{SG^*}\|$, we get:
\begin{align*}
    \frac{\|\chi_{A_d^*}\|}{\|\chi_{SG^*}\|}\leq\frac{\sqrt{d+1}(\frac{1-2\delta}{\delta})^d\cdot(\frac{d(d+2)}{12(d+1)})^{\frac{d}{2}}}{2(\frac{1-2\delta}{\delta})^n(\frac{2d-1}{16})^{\frac{d}{2}}} \\
    =\frac{\sqrt{d+1}}{2}(\frac{16d(d+2)}{12(d+1)(2d-1)})^{\frac{d}{2}}\leq\frac{\sqrt{d+1}}{2}(\frac{16}{12\cdot2})^{\frac{d}{2}} \\
    = \frac{\sqrt{d+1}}{2}(\frac{2}{3})^{\frac{d}{2}}\rightarrow 0
\end{align*}
This is already good. It is exponentially better than the $SG^*$ set. Let us compare it to the lower bound:
\begin{align*}
    \frac{\|\chi_{A_d^*}\|}{\|\chi_{LB}\|}=\frac{\sqrt{d+1}(\frac{1-2\delta}{\delta})^d\cdot(\frac{d(d+2)}{12(d+1)})^{\frac{d}{2}}}{\sqrt{\frac{e}{2}}\Big(1-\frac{2\delta}{1-2\delta}\Big)^2\Big(\sqrt{\frac{d-1}{2\pi e}}\cdot \frac{1-2\delta}{\delta}\Big)^d} \\
    \leq \sqrt{\frac{2(d+1)}{e}}(\frac{2\pi e\cdot d(d+2)}{12(d^2-1)}))^{\frac{d}{2}} \\
    \rightarrow (\sqrt{\frac{\pi e}{6}})^d\approx1.19^d
\end{align*}
So $\|\chi_{A_d^*}\|\leq1.19^d\|\chi_{LB}\|$, compared to $1.43^d$ for the $SG^*$ set---exponentially better!
\subsection{some remains from the onion-set section}
We need to compare it to the SG*-grid coverage of the original ball. In that, we get:
\begin{align}
    \|\chi_{SG*}\|\approx2\frac{Vol(B_{\beta}(0))}{(2\frac{2}{\sqrt{2n-1}}\delta)^n}=\frac{2c_n(\frac{\sqrt{n}}{2})^n}{(2\frac{2}{\sqrt{2n-1}}\delta)^n}=2c_n(n(2n-1))^\frac{n}{2}(\frac{1}{8\delta})^n
\end{align}
In comparison:
\begin{align*}
    \frac{\|\chi\|}{\|\chi_{SG*}\|}\approx\frac{cn^{\frac{n}{2}+1}ln(n)(\frac{1}{2\delta})^n}{2c_n(n(2n-1))^\frac{n}{2}(\frac{1}{8\delta})^n}=2c\cdot c_n^{-1} \cdot nln(n)(\frac{1}{2n-1})^{\frac{n}{2}}4^n
\end{align*}
Using~\cite{tsao2020sample}, we insert an approximation for $c_n^{-1}\approx\sqrt{\pi n}(\frac{n}{2\pi e})^{\frac{n}{2}}$:
\begin{align*}
    \frac{\|\chi\|}{\|\chi_{SG*}\|}\approx 2c \sqrt{\pi n}(\frac{n}{2\pi e})^{\frac{n}{2}}\cdot nln(n)(\frac{1}{2n-1})^{\frac{n}{2}}4^n \\ 
    =2c \sqrt{\pi} n^{1.5}ln(n)(\frac{n}{2n-1})^{\frac{n}{2}}(\frac{8}{\pi e})^{\frac{n}{2}} \\
    =2c \sqrt{\pi} n^{1.5}ln(n)(\frac{2n}{2n-1})^{\frac{n}{2}}(\frac{4}{\pi e})^{\frac{n}{2}} \\
\end{align*}
On the side, we can see that:
\begin{align}
    (\frac{2n}{2n-1})^{\frac{n}{2}}=(1+\frac{1}{2n-1})^{\frac{n}{2}}=((1+\frac{1}{2n-1})^{2n-1})^{\frac{\frac{n}{2}}{2n-1}}\rightarrow e^{\frac{1}{4}}
\end{align}
So finally:
\begin{align*}
    \frac{\|\chi\|}{\|\chi_{SG*}\|}\approx 2c\sqrt{\pi}e^{\frac{1}{4}}n^{1.5}ln(n)\cdot (0.684)^n\rightarrow0
\end{align*}
So for a high enough $n\in\mathbb{N}$, we get $\|\chi\|<\|\chi_{SG*}\|$. In general, for a high enough dimension, it behaves as $\|\chi\|\approx\|\chi_{SG*}\|\cdot 0.685^n$. It can be checked that we get $\|\chi\|<\|\chi_{SG*}\|$ starting at $n=c\cdot 19$: the $c>0$ is not specified in any of the referenced material, but assuming $c>1$, this makes the onion set impractical for motion planning problems, as we rarely reach robots with $n\geq 19$.

\subsection{ONION set leftovers: taken out of the paper for now}

\kiril{The following hasn't been defined. Why it's here?}
\begin{align*}
    \frac{\|\chi_{Onion, C-Space}\|}{\|\chi_{LB}\|} = O(\ln{d})
\end{align*}
\subsection*{Complexity of sample sizes: the non-constructive "Onion" set}

To show what's theoretically possible, using a non-constructive method, we'll be using in this section a covering set from a paper in~\cite{verger2005covering} that used an "onion" like construction to cover $\B_{\beta}(0)\subset\mathbb{R}^d$ (instead of a cube). Intuitively, this is what he does:
\begin{enumerate}
    \item Cover a sphere $S_{R}(0)$ for some $R>0$, using balls of radius $r=\frac{1}{2}$. 
    \item Show that this collection of spheres actually covers an annulus from $R$ to $R-\frac{1}{2R}$.
    \item Recursively cover smaller and smaller annuluses, until the smaller radius is less than $\frac{1}{2}$, allowing you to cover it with one extra ball.
\end{enumerate}
It can be noted that (1) is done using probabilistic methods, making the whole set non-constructive.
Using this method, he manages to cover a ball of radius $R\geq \frac{\sqrt{d}}{2}$ with $n$ balls of radius \(\frac{1}{2}\), with:
\[
    n \leq c\cdot dln(d)(2R)^d
\]
We need to cover the spherical C-Space $\B_\frac{\sqrt{d}}{2}$ with balls of radius $\beta$, so let us resize the sphere: turn balls of radius $\beta$ into balls of radius $\frac{1}{2}$, so resizing the whole space by a factor of $\frac{1}{2\beta}$. This turns the ball of radius $\frac{\sqrt{d}}{2}$ to a ball of radius $\frac{\sqrt{d}}{4\beta}$.


Our problem is, then: cover a ball of radius $\beta=\frac{\sqrt{d}}{4\delta}$ with balls of radius $\frac{1}{2}$, which falls under~\cite{verger2005covering} if:
\[
    \frac{\sqrt{d}}{4\beta} \geq \frac{\sqrt{d}}{2} \Rightarrow \beta \leq \frac{1}{2}
\]
Which is reasonable. This gives us:
\begin{align}
    \|\chi\|\leq cdln(d)(2\frac{\sqrt{d}}{4\beta})^d
\end{align}
Comparing it to the same lower bound all other lattices were compared to, we get:
\begin{align*}
    \frac{\|\chi_{Onion, C-Space}\|}{\|\chi_{LB}\|} \leq \frac{cdln(d)(2\frac{\sqrt{d}}{4\beta})^d}{k\cdot d (2\frac{\sqrt{d}}{4\beta})^d} =\frac{c}{d}\ln{d}
    \Rightarrow \frac{\|\chi_{Onion, C-Space}\|}{\|\chi_{LB}\|} = O(\ln{d})
\end{align*}
This means that this semi-random set achieves a covering that is only $O(\ln{d})$ times above the lower bound!

\subsection{calcing anstar $\rho$}

 From pages LXII and 10 of Conway~\cite{conway2013sphere}, we know 
    \[
        \rho = (\delta\sqrt{\det{\Lambda}})^{\frac{1}{d}}
    \]
    where $\delta$ is a parameter given in page 115 which is independent of choice of basis. The only thing left to calculate, then, is $\det{\Lambda}=det(G_\Lambda G_\Lambda^t)$. Let us calculate it:
    \begin{align*}
        \det{\Lambda}&=det(G_\Lambda G_\Lambda^t)\\
        &=\det\begin{pmatrix}
            1 & -1 &  0  & \dots & 0 & 0 & 0 \\
            0 & 1  &  -1 & \dots & 0 & 0 & 0 \\
            . & .  &  . & \dots & . & .  & .\\
            0 & 0  &  0 & \dots & 1 & -1 & 0\\
            \frac{-d}{d+1} & \frac{1}{d+1} & \frac{1}{d+1} & \dots & \frac{1}{d+1} & \frac{1}{d+1} & \frac{1}{d+1}
        \end{pmatrix}
        \begin{pmatrix}
            1 & 0 & \dots & 0 & 0 & \frac{-d}{d+1} \\
            -1 & 1 & \dots & 0 & 0 & \frac{1}{d+1} \\
            0 & -1 & \dots & 0 & 0 & \frac{1}{d+1} \\
            . & . & \dots & . & .  & .\\
            0 & 0 & \dots & 0 & -1 & \frac{1}{d+1}\\
            0 & 0 & \dots & 0 & 0 & \frac{1}{d+1}
        \end{pmatrix}\\
        &=\det\begin{pmatrix}
            2 & -1 &  0  & 0 & 0 &\dots & 0 & 0 & -1\\
            -1 & 2  &  -1 & 0 & 0 &\dots & 0 & 0 & 0\\
            0 & -1  &  2 & -1 & 0 & \dots & 0 & 0 & 0\\
            . & .  &  . & . & . & \dots & . & . & .\\
            0 & 0  &  0 & 0 & 0 & \dots & -1 & 2 & 0\\
            -1 & 0  &  0 & 0 & 0 &\dots & 0 & 0 & \frac{d}{d+1}\\
        \end{pmatrix},\\
    \end{align*}
    now add all $d-1$ rows to the last row:
    \begin{align*}
        &=\det\begin{pmatrix}
            2 & -1 &  0  & 0 & 0 &\dots & 0 & 0 & -1\\
            -1 & 2  &  -1 & 0 & 0 &\dots & 0 & 0 & 0\\
            0 & -1  &  2 & -1 & 0 & \dots & 0 & 0 & 0\\
            . & .  &  . & . & . & \dots & . & . & .\\
            0 & 0  &  0 & 0 & 0 & \dots & -1 & 2 & 0\\
            0 & 0  &  0 & 0 & 0 &\dots & 0 & 1 & \frac{d}{d+1}-1\\
        \end{pmatrix}\\
        &=\det\begin{pmatrix}
            2 & -1 &  0  & 0 & 0 &\dots & 0 & 0 & -1\\
            0 & 1.5  &  -1 & 0 & 0 &\dots & 0 & 0 & 0.5\\
            0 & -1  &  2 & -1 & 0 & \dots & 0 & 0 & 0\\
            . & .  &  . & . & . & \dots & . & . & .\\
            0 & 0  &  0 & 0 & 0 & \dots & -1 & 2 & 0\\
            0 & 0  &  0 & 0 & 0 &\dots & 0 & 1 & \frac{d}{d+1}-1\\
        \end{pmatrix}\\
        &=\frac{1}{2}\det\begin{pmatrix}
            2 & -1 &  0  & 0 & 0 &\dots & 0 & 0 & -1\\
            0 & 3  &  -2 & 0 & 0 &\dots & 0 & 0 & -1\\
            0 & -1  &  2 & -1 & 0 & \dots & 0 & 0 & 0\\
            . & .  &  . & . & . & \dots & . & . & .\\
            0 & 0  &  0 & 0 & 0 & \dots & -1 & 2 & 0\\
            0 & 0  &  0 & 0 & 0 &\dots & 0 & 1 & \frac{d}{d+1}-1\\
        \end{pmatrix}\\
            &=\frac{1}{2}\det\begin{pmatrix}
            2 & -1 &  0  & 0 & 0 &\dots & 0 & 0 & -1\\
            0 & 3  &  -2 & 0 & 0 &\dots & 0 & 0 & -1\\
            0 & 0  &  \frac{4}{3} & -1 & 0 & \dots & 0 & 0 & \frac{-1}{3}\\
            . & .  &  . & . & . & \dots & . & . & .\\
            0 & 0  &  0 & 0 & 0 & \dots & -1 & 2 & 0\\
            0 & 0  &  0 & 0 & 0 &\dots & 0 & 1 & \frac{d}{d+1}-1\\
        \end{pmatrix}.
    \end{align*}
    we keep going like this until the last row, getting
    \begin{align*}
        &=\frac{1}{(d-1)!}\det\begin{pmatrix}
            2 & -1 &  0  & 0 & 0 &\dots & 0 & 0 & -1\\
            0 & 3  &  -2 & 0 & 0 &\dots & 0 & 0 & -1\\
            0 & 0  &  4 & -3 & 0 & \dots & 0 & 0 & -1\\
            . & .  &  . & . & . & \dots & . & . & .\\
            0 & 0  &  0 & 0 & 0 & \dots & 0 & d & -1\\
            0 & 0  &  0 & 0 & 0 &\dots & 0 & 1 & \frac{d}{d+1}-1\\
        \end{pmatrix}\\
        &=\frac{1}{(d-1)!}\det\begin{pmatrix}
            2 & -1 &  0  & 0 & 0 &\dots & 0 & 0 & -1\\
            0 & 3  &  -2 & 0 & 0 &\dots & 0 & 0 & -1\\
            0 & 0  &  4 & -3 & 0 & \dots & 0 & 0 & -1\\
            . & .  &  . & . & . & \dots & . & . & .\\
            0 & 0  &  0 & 0 & 0 & \dots & 0 & d & -1\\
            0 & 0  &  0 & 0 & 0 &\dots & 0 & 0 & \frac{d}{d+1}-1+\frac{1}{d}\\
        \end{pmatrix}\\
        &=\frac{1}{(d-1)!}\cdot d!(\frac{1}{d(d+1)})
        =\frac{1}{d+1}.
    \end{align*}

    The following short lemma will help us get the diameter.
\begin{lemma}\label{lemma:diameter}
    Let $\{e_i\}$ be vectors such that $\angle\left(e_i,e_j\right)\leq \frac{\pi}{2}$ for all $i\neq j$. Then, the diameter of the FR created by these vectors is $\D=\|\sum_i e_i\|$.
\end{lemma}
\begin{proof}
    \begin{align*}
        \|e_i+e_j\|^2&= \|e_i\|^2+\|e_j\|^2+2\langle e_i,e_j \rangle \\
        &= \|e_i\|^2+\|e_j\|^2+2\cos{\left(\angle\left(e_i,e_j\right)\right)}\|e_i\|\|e_j\|,
    \end{align*}
    and if $\angle\left(e_i,e_j\right)\leq \frac{\pi}{2}$ then $\cos{\left(\angle\left(e_i,e_j\right)\right)}\geq 0$, so \(\|e_i+e_j\|^2 \geq \|e_i\|^2\), giving us \(\|e_i+e_j\| \geq \|e_i\|\).
    
    
    Overall, this means that we can keep adding base FR elements such that the norm keeps getting larger, until we can't add more elements. At the end, the distance between $v=0,w=\sum_i e_i$ is the largest it can be. Seeing as how the vertices of the FR are also its extreme points, we can not make it larger, meaning we found the diameter.
\subsection{LEFTOVERS from the old collision complexity proof}
We start with introducing a few basic definitions that would be used throughout this section. First, we define a \emph{fundamental region} (FR) of a lattice, which is the closed set contained inside the linear combination of that lattice's basis vectors. The geometric shape of a given FR can be viewed as ``lattice tile'', as interior-disjoint translations of it can be used to cover the space. \kiril{Are the following definitions taken from somewhere or brand new? If the former then a citation would be good.}

%This creates something that can be viewed as the "lattice tile" (see Figure~\ref{fig:reflection)), which is because the union of all the FRs covers (tile) the space. 

 \begin{figure}
    \centering
    \includegraphics[width=0.66\linewidth]{Images/FR_2D.png}
    \caption{Examples of FR tiles in $d=2$. \kiril{Explain what we see here. E.g., what are the basis and FRs?} }
    \label{fig:reflection}
\end{figure}

%From this illustration, it can be also seen that FRs are not unique, some even having a completely different basis.

% Also called a \emph{Parallelotope} sometimes. 
\begin{definition}[Fundamental region]
    For a given lattice $\Lambda$ with a basis ${E_\Lambda=\{e_i\in \dR^d\}_{i=1}^m}$ (where $m\leq d$), and a point $p\in \dR^d$, the FR is defined as 
    \begin{align*}
        F_{\Lambda,p}:=\left\{p+\sum_{i=1}^m a_i e_i\middle|\,a\in[0,1]^m\right\}. 
    \end{align*}
    The center of $F_{\Lambda,p}$ is defined to be \[c_{\Lambda,p}=p+\frac{1}{2^{m}}\sum_{a\in \{0,1\}^m}\sum_{i=1}^ma_ie_i.\]
\end{definition}
\kiril{Way may consider removing $p$ from the definition of the FR.}
We mention that the geometric shape of a FR can is not unique, which follows from non-uniqueness of the lattice basis (see Figure~\ref{fig:reflection}). 

\kiril{Ideally, figures should be saved as pdfs (or other vector formats). I usually used \url{http://ipe.otfried.org/}.}.


% A for a given FR, its center is the average of all binary linear combinations of basis vectors, which means $2^{|E_\Lambda|}$ of them. \kiril{add it to the FR figure}
% \begin{definition}[FR center]\label{def:lattice_center}
%     Given a lattice $\Lambda$ with basis ${E_\Lambda=\{e_i\in \dR^d\}_{i=1}^m}$, the center of the FR $F_{\Lambda,p}$ in relation to some point $p\in \mathbb{R}^d$, is
% \end{definition}

\begin{lemma}\label{Lattice-properties}
    An FR center \(c_{\Lambda,p}\) for a  lattice \(\Lambda\) with basis ${E_\Lambda=\{e_i\in \dR^d\}_{i=1}^m}$ (where $m\leq d$) has the following properties: 
    \begin{enumerate}
        \item \(c_{\Lambda,p}=p+\frac{\sum_{i=1}^m e_i}{2}\), i.e., the center of the FR is the average of the base vectors; \kiril{Is average the right word here if we divide by $2$?}
        \item \(\{c_{\Lambda,p_i}|p_i\in\Lambda\}=\Lambda+c_{\Lambda,o}\), i.e., the collection of centers of FR regions whose point offset is a lattice point, is equal to the lattice translated with the center of the FR $F_{\Lambda,o}$, where $o$ is the origin of $\dR^d$.
    \end{enumerate}
\end{lemma}
\begin{proof}
    For the first property, we rely on the definition of $c_{\Lambda,p}$ and expand the summation expression within it. In particular, we decompose it by individually choosing the $a_i$ values, and separating the inner sum according to the dependencies on $a_i$'s. That is,
%    use a combinatorial argument. Fix a value for $a_1$. If the rest of the $a_j$ values for $j\neq i$  are chosen, then we get two "copies" of the series: one where $a_i=0$ and one where $a_i=1$. So, the relative weight of each vector is $\frac{1}{2}$. More formally, notice that
 \kiril{Please check that it makes sense.}    \begin{align*}
  \sum_{a\in \{0,1\}^m}\sum_{i=1}^ma_ie_i &= \sum_{a_1=0}^1\ldots \sum_{a_m=0}^1\sum_{i=1}^ma_ie_i \\ &=
\sum_{a_1=0}^1\ldots \sum_{a_m=0}^1a_1e_1+\sum_{a_1=0}^1\ldots \sum_{a_m=0}^1\sum_{i=\bm{2}}^ma_ie_i \\ &=
\sum_{a_{\bm{2}}=0}^1\ldots \sum_{a_m=0}^1e_1+\sum_{a_1=0}^1\ldots \sum_{a_m=0}^1\sum_{i=\bm{2}}^ma_ie_i                             \\
                                      &=
2^{m-1}e_1+2\sum_{a_{\bm{2}}=0}^1\ldots \sum_{a_m=0}^1\sum_{i=\bm{2}}^ma_ie_i                             \\
                                        &=
2^{m-1}e_1+2\cdot 2^{m-2}e_2+ 2\cdot 2\sum_{a_{\bm{3}}=0}^1\ldots \sum_{a_m=0}^1\sum_{i=\bm{3}}^ma_ie_i                             \\  & = \ldots = 2^{m-1}\sum_{i=1}^m e_i. 
    \end{align*}
    Thus, $c_{\Lambda,p}=p+\frac{\sum_{i=1}^m e_i}{2}$.

    Now consider the second property. From the previous paragraph it follows that $c_{\Lambda,o}=\frac{\sum_ie_i}{2}$. For some $p_i\in \Lambda$ it holds that
    \[c_{\Lambda,p_i}=p_i+\frac{\sum_{i=1}^m e_i}{2}= p_i+o+\frac{\sum_{i=1}^m e_i}{2}=p_i + c_{\Lambda,o},\]
    which completes the proof. 
\end{proof}

\begin{lemma}\label{fr_shifted_sum} Consider an FR $F:=F_{\Lambda,p}$ for a lattice $\Lambda$. Then, 
    \[
        I_{F,p} = I_{F,o} - |P_F|\cdot \|c_{\Lambda,p}\|^2.
    \]
\kiril{The notations here are still rather confusing. In the equation, is $p$ the same as in $F_{\Lambda,p}$, or should be $p'$? There's also ambiguity with respect to $c_{\Lambda,p}$ since it doesn't explicitly reflect a specific FR. I suggest to change it to $c(F_{\Lambda,p})$ for clarity.}
\end{lemma}
\begin{proof}
  The statement follows from expanding the definition of $I_{F,p}$:
    \begin{align}
        I_{F,p} &= \sum_{x\in P_F} \|x-c_{\Lambda,p}\|^2 = \sum_{x\in P_F} \sum_{j=1}^d \left(x(j)-c_{\Lambda,p}(j)\right)^2 \nonumber\\
        &= \sum_{x\in P_F} \sum_{j=1}^d \left(x(j)^2-2x(j)c_{\Lambda,p}(j)+c_{\Lambda,p}(j)^2\right) \nonumber\\
        &=\sum_{x\in P_F} \sum_{j=1}^d x(j)^2 + \sum_{x\in P_F} \sum_{j=1}^d c_{\Lambda,p}(j)^2 - 2\sum_{x\in P_F} \sum_{j=1}^d x(j)c_{\Lambda,p}(j) \nonumber\\
        &= I_{F,o} + |P_F|\cdot\|c_{\Lambda,p}\|^2 - 2 \sum_{j=1}^d \sum_{x\in P_F} x(j)c_{\Lambda,p}(j) \nonumber \\
        &= I_{F,o} + |P_F|\cdot\|c_{\Lambda,p}\|^2 - 2 \sum_{j=1}^d c_{\Lambda,p}(j) \sum_{x\in P_F} x(j) \nonumber \\
        &= I_{F,o} + |P_F|\cdot\|c_{\Lambda,p}\|^2 - 2 \sum_{j=1}^d c_{\Lambda,p}(j) \underbrace{\left(\sum_{x\in P_F} \left(x(j) - c_{\Lambda,p}(j)\right)+\sum_{x\in P_F} c_{\Lambda,p}(j)\right)}_\text{(*)}. \label{eq:IFo}
    \end{align}
    Next, we show that the expression in ($*$) is equal to $|P_F|c_{\Lambda,p}(j)$. In particular, we rely on the fact that for any given $x\in P_F$ it holds that $x=p+\sum_1^m a_ie_i$ for some $a\in\{0,1\}^m$ (and vice versa).
    Thus, 
    \begin{align*}     c_{\Lambda,p}&=p+\frac{1}{2^{m}}\sum_{a\in \{0,1\}^m}\sum_{j=1}^ma_ie_i \\ & =\frac{1}{2^m}\sum_{a\in \{0,1\}^m}\left(p+\sum_{j=1}^m a_ie_i\right)\\
      & =\frac{1}{2^m}\sum_{x\in P_F}x.%\\
%     &=\frac{1}{2^{|E_\Lambda|}}\sum_{j=1}^{|E_\Lambda|} x(j),\\
%        &\Rightarrow 2^{|E_\Lambda|}c_{\Lambda,p}=\sum_{j=1}^{|E_\Lambda|} x(j)\Rightarrow 0=\sum_{j=1}^{|E_\Lambda|} \left(x(j)-c_{\Lambda,p}\right),
    \end{align*}
    \kiril{I fixed some equations, which I believe were incorrect. Please check me.}
    Hence, $c_{\Lambda,p}(j)=\frac{1}{2^m} \sum_{x\in P_F}x(j)$, for $1\leq j\leq d$, 
    which implies that the vertices of the FR are symmetric around the center, intuitively speaking. From here we obtain
    \[(*)=\sum_{x\in P_F} c_{\Lambda,p}(j)=|P_F|c_{\Lambda,p}(j),\]
which we plug into \eqref{eq:IFo} to yield
    \begin{align*}
        I_{F,p}&=  I_{F,o} + |P_F|\|c_{\Lambda,p}\|^2 - 2 |P_F|\sum_{j=1}^d c_{\Lambda,p}(j)^2 \\
        &= I_{F,o} + |P_F|\|c_{\Lambda,p}\|^2 - 2|P_F|\|c_{\Lambda,p}\|^2 = I_{F,o} - |P_F|\|c_{\Lambda,p}\|^2.
    \end{align*}
\end{proof}

Notice that the value of $I_{F,p}$ is independent  of the particular point $p$: this notation talks about the sum-of-edges distance from the center of some cell that is located at an unknown location. The $p$ just denotes where the corner of the cell is, and this property \emph{does not effect} the distance from the center. \kiril{I still don't see why this is true. More care should be taken where explaining this claim.} \itai{I tried a different explanation. let's discuss it if you think it's not enough.} \kiril{Also, we already have $I_\Lambda(r)$, which looks similar.} \kiril{I want to revisit this statement after we improve the notation.} Hence, from now we denote $I_{\Lambda}:=I_{F,p}$ for some FR $F$ and point $p$. Thus, we can simplify the previous lemma statement to 
%\equiv CONST\) for any $F$ you choose, so we will just call it \(I_{\Lambda}\) from now on. Also, in our case since we are using a lattice, it has $|P_F|=2^d$ vertices, so we will just use that from now on. So we rephrase our lemma to appear as such:
\begin{align}
            I_{F,o} = I_\Lambda + 2^d\cdot \|p\|^2,
\end{align}
which holds in our case since  $|P_F|=2^d$ for our three lattices. $\ZN,\DN$ have $d$-dim bases in $\mathbb{R}^d$ and $\AN$ has a $d$-dim base in $\mathbb{R}^{d+1}$.
% \begin{lemma}
%     Let \(D_{i,r}\) be the $D_i$ such that \(\{p_i\in D_i\}\cap \B_r(0)\neq \phi\) (all the cells that contribute a vertex in the ball). Then:
%     \[
%         \lim_{n\rightarrow \infty}I(r)=\lim_{n\rightarrow \infty}{\frac{1}{N}\sum_{D_{i,r}} I_{0,D_i}}
%     \]
% \end{lemma}
% \begin{proof}
%     I know that for most of the ball, each vertex is counted N times (once for each neighbor). The trick here is to show that for the vertices that are on the edge of the ball, that are covered less than N times, the contribution to the sum goes to 0 asymptotically.
% \end{proof}
We now move to first major contribution of this section, which is a general upper-bound on collision complexity.
\begin{thm}[Collision-complexity upper bound]\label{complexity_upper_bound}
    Let $\Lambda$ be a lattice and let $F$ be some FR for the basic $E_\Lambda$. Denote by $\D$ the diameter of $F$, which is the diameter of the smallest sphere that contains $F$. Let $\rho$ be the inradius of $F$, which is the radius of the biggest  sphere that fits within $F$.
For any $r>0$ it holds that
\begin{align*}
        I\left(r\right) \leq \frac{1}{4}\left(\sum_{e\in E_\Lambda}\|e\|^2\right) \btheta^{d-1}\left(\frac{\btheta}{\rho d}+1\right)\left(\frac{\vol(\B_1)}{\sqrt{d}}\btheta+ \alpha\right), 
    \end{align*}
    where $\alpha$ is some positive constant, and
    \[
        \btheta:=\frac{r+\D-\rho}{\beta}f_\Lambda.
    \]
\kiril{We already used $\btheta$ for a different purpose.}\kiril{It would be good to visualize this theorem.} \kiril{Remove the parts related to rescaling.}
    %Then for some radius $r>0$, if we use $\theta\left(\beta,R\right):=\frac{R}{\beta}$ from corollary~\ref{cor:specific_sample_complexity} and define the rescaled radius to be
    %\[
    %    \btheta=\theta\left(\beta,r+\D-\rho\right)f_\Lambda,
    %\]
    %we get
    % \[
    %     g:=\frac{r+D-\rho}{\beta}f_\Lambda
    % \]
    % \begin{align*}
    %     g(r)&:=\frac{r-\rho}{\beta}f_\Lambda,\\
    %     % a_n(r)&:=\frac{I_\Lambda}{2^n}(\frac{Vol(\B_1)}{\sqrt{n}}g(r)^2 + k_1),
    % \end{align*}
    % we get
\end{thm}
\begin{proof}
    %First note that for now, we will use the unscaled space. Later, we will be using $\btheta$ instead of just $r+\D-\rho$, because it performs a rescaling. This is done to enable us to use the lattice point bound in equation (9) above.\kiril{Those statements are quite vague and I don't follow them.}
  We start by showing that
    \begin{align}
        I\left(r\right)\leq \sum_{i=1}^{[\frac{r+\D}{\rho}]-1} \frac{I_\Lambda}{2^d}(r+\D-i\rho)^{d-2}\left(\frac{ \vol\left(\B_1\right)}{2^d}\left(r+\D-i\rho\right)^2+\alpha\right).
    \end{align}
    Let
    \begin{align*}
        J\left(r\right):=\frac{1}{2^d}\smashoperator[r]{\sum_{F\in \F_r}} I_{F,o},
    \end{align*}
     be the partial sum-of-squares approximation, only using FRs fully contained in the $r$-ball (denoted by the set $\F_r$) \kiril{This definition is vague, and should be defined more formally.}).
     For a given FR $F\in \F_r$, denote by $c_F$ its center.
    %$$C:=\{c_F| F\subset \B_r\}.$$
     Then using \Cref{fr_shifted_sum} we obtain   \begin{align*}
        J\left(r\right)&= \frac{1}{2^d}\smashoperator[r]{\sum_{F\in \F_r}} I_{F,o} \underset{[L\ref{fr_shifted_sum}]}{=} \frac{1}{2^d}\smashoperator[r]{\sum_{F\in \F_r}} \left(I_\Lambda + 2^d\|c_F\|^2\right)\\
        &=\smashoperator[r]{\sum_{F\in \F_r}}\frac{I_\Lambda}{2^d} + \smashoperator[r]{\sum_{F\in \F_r}}\|c_F\|^2. %\leq \frac{I_\Lambda}{2^d}|B_{r-\rho}\cap\Lambda|+\smashoperator[r]{\sum_{F\in \F_r}}\|c_F\|^2. 
    \end{align*}
   Next, we observe that $|\F_r|\leq |\B_{r-\rho}\cap \Lambda|$: every FR fully included  which  follows from \kiril{be more explicit here w.r.t to the definition of $I_\Lambda$ and why this property follows. }. We also notice that $\sum_{F\in \F_r}\|c_F\|^2\leq I(r-\rho)$, which follows the second property in~\Cref{Lattice-properties}. In particular, \kiril{be more formal and include all the necessary notations and definitions.} the lemma shows that the collection of centers of is itself is a "smaller copy" of the main lattice. More than that, from the symmetry of the FR we can say that the insphere which defines the inradius is centered around the centroid (\itai{I hope this is right, otherwise I need to make some changes to this lemma}). Seeing as the insphere is included in the FR, and that we picked only the FRs that are included in the ball, we can safely reduce the radius of the ball by the inradius and still have all the centroids we started with. Therefor, using the bound on $|\B_r\cap\Lambda|$ we got at theorem~\ref{general_sample_complexity}, we get
    \begin{align}
        &\overset{\leq}{[L\ref{Lattice-properties}]} \frac{I_\Lambda}{2^d}\cdot|B_{r-\rho}\cap\Lambda| + I\left(r-\rho\right) \nonumber\\
        &\overset{\leq}{[T\ref{general_sample_complexity}]} \frac{I_\Lambda}{2^d}\cdot\left(\frac{Vol\left(\B_{r-\rho}\right)}{\sqrt{d}}+a\left(r-\rho\right)^{d-2}\right) + I\left(r-\rho\right) \nonumber\\
        &=\frac{I_\Lambda}{2^d}\cdot\left(\frac{Vol\left(\B_1\right)}{\sqrt{d}}\cdot \left(r-\rho\right)^d+a\left(r-\rho\right)^{d-2}\right) + I\left(r-\rho\right)\nonumber\\
        &=\frac{I_\Lambda}{2^d}\left(\frac{Vol\left(\B_1\right)}{\sqrt{d}}\left(r-\rho\right)^2 + a\right)\left(r-\rho\right)^{d-2} + I\left(r-\rho\right) \nonumber\\
        &=b_d\left(r\right)\left(r-\rho\right)^{d-2} + I\left(r-\rho\right)
    \end{align}
    For $b_d\left(r\right):=\frac{I_\Lambda}{2^d}\left(\frac{Vol\left(\B_1\right)}{\sqrt{d}}\left(r-\rho\right)^2 + a\right)$. Notice that $J\left(r\right)$ counts some edges too few times: edges near the border of the sphere may appear in less than $2^d$ neighboring cells, due to some cells not being included in $J\left(r\right)$, so dividing by $2^d$ lowers their contribution too much. To solve this, we will notice that for a given FR we can take the diameter $\D$ and expand $B_r$ with it. Doing this, every relevant point of $B_r$ will now contain all $2^d$ neighboring cells in the $B_{r+\D}$ ball. We do have extra points now, but the normalizer $\frac{1}{2^d}$ makes sure their contribution decays to 0. We get
    \[
        I\left(r\right)\leq J\left(r+\D\right),
    \]
    and overall
    \begin{align*}
        I\left(r\right) &\leq J\left(r+\D\right) \overset{\left(13\right)}{\leq} b_d\left(r+\D\right)\left(r+\D-\rho\right)^{d-2} + I\left(r+\D-\rho\right)\\
        &\leq b_d\left(r+\D\right)\left(r+\D-\rho\right)^{d-2} + b_d\left(r+\D-\rho\right)\left(r+\D-2\rho\right)^{d-2} + I\left(r-2\rho\right) \\
        &\leq \dots,
    \end{align*}
    which we continue until
    \[
        r+\D-m\rho < \rho \Rightarrow \frac{r+\D}{\rho} - 1 < m.
    \]
    This means we stop at \(m=[\frac{r+D}{\rho}]-1\), getting
    \begin{align*}
        I\left(r\right) &\leq \sum_{i=1}^{[\frac{r+\D}{\rho}]-1} b_d\left(r+\D-\left(i-1\right)\rho\right)\left(r+\D-i\rho\right)^{d-2}\\
        &= \sum_{i=1}^{[\frac{r+\D}{\rho}]-1} \frac{I_\Lambda}{2^d}\left(\frac{Vol\left(\B_1\right)}{\sqrt{d}}\left(r+\D-i\rho\right)^2 + k_1\right)\left(r+\D-i\rho\right)^{d-2}\\
        &=\sum_{i=1}^{[\frac{r+\D}{\rho}]-1} \left(\frac{I_\Lambda Vol\left(\B_1\right)}{2^d\sqrt{d}}\left(r+\D-i\rho\right)^d+k_1\frac{I_\Lambda}{2^d}\left(r+\D-i\rho\right)^{d-2}\right)\\
    \end{align*}
    which is because the last $I\left(r+\D-m\rho\right)=0$, due to capturing either nothing or a single point. 
    
    
    Now that we finished our base upper bound, we can now finish the theorem by further developing it. Using $k=[\frac{r+\D}{\rho}]-1$, and utilizing the known comparison between sums and integrals, we develop the following
    % Notice that $J(r)$ counts some edges too few times: edges near the border of the sphere may appear in less than $2^d$ neighboring cells, so dividing by $2^d$ lowers their contribution too much. To solve this, we will notice that for a FR we can take the diameter $D:=\text{Diam}(FR)$ and expand $B_r$ with it. Doing this, every relevant point of $B_r$ will now contain all $2^d$ neighboring cells in the $B_{r+D}$ ball. We do have extra points now, but the normalizer $\frac{1}{2^d}$ makes sure their contribution decays to 0. We get
    % \[
    %     I(r)\leq J(r+D)
    % \]
    % Which is because v is the longest vector out of all the others in the cell (as he is a sum of all of them, and they are positive (<- show this)). Due to this, the larger radius now contains all cells we didn't count before with a few more - but they are counted less, as they are weighted with $\frac{1}{2^n}$.
    \begin{align}
        &A\sum_1^k \left(r+\D-i\cdot \rho\right)^{d-B} \nonumber\\
        &\leq A\Big[\left(r+\D - 1\cdot \rho\right)^{d-B} + \int_1^k \left(r+\D-x\cdot \rho\right)^{d-B} dx\Big] \nonumber\\
        &= A\Big[\left(r+\D-\rho\right)^{d-B} + [\frac{\left(r+\D-k\cdot \rho\right)^{d-B+1}}{-\rho\left(d-B+1\right)}-\frac{\left(r+\D-\rho\right)^{d-B+1}}{-\rho\left(d-B+1\right)}]\Big] \nonumber\\
        &=A\Big[\left(r+\D-\rho\right)^{d-B} + \frac{1}{\rho\left(d-B+1\right)}[\left(r+\D-\rho\right)^{d-B+1} - \left(r+\D-k\cdot \rho\right)^{d-B+1}]\Big].
        %=_{(2)} (r+\|v\|)^n + \frac{1}{\rho(n+1)}[(r+\|v\|)^{n+1} -((n+1)k\rho(r+\|v\|)^n+O(smaller?)+(r+\|v\|)^{n+1})] \\
    \end{align}
    We also know that 
    % \begin{align*}
    %         k=[\frac{r+\|v\|}{\rho}]-1\geq \frac{r+\|v\|}{\rho}-2 \\
    %         \Rightarrow k\rho \geq r+\|v\|-2\rho \\
    %         \Rightarrow 1-k\rho \leq 1 - (r+\|v\|) + 2\rho
    % \end{align*}
    \begin{align*}
            k&=[\frac{r+\D}{\rho}]-1\leq \frac{r+\D}{\rho}-1 \\
            &\Rightarrow k\rho \leq r+\D-\rho \Rightarrow -k\rho \geq -\left(r+\D\right)+\rho \\
            &\Rightarrow r+\D-k\rho \geq \rho,
    \end{align*}
    which means that
    \begin{align*}
        % &=A\Big[(r+D-\rho)^{n-B} + \frac{1}{\rho(n-B+1)}[(r+D-\rho)^{n-B+1} - (r+D-k\cdot \rho)^{n-B+1}]\Big]\\
        \left(13\right)&\leq A\Big[\left(r+\D-\rho\right)^{d-B} + \frac{\left(\left(r+\D-\rho\right)^{d-B+1} - \rho^{d-B+1}\right)}{\rho\left(d-B+1\right)}\Big].
    \end{align*}
    Returning to our original expression and setting $\Tilde{r}:=r+D-\rho$, we get
    \begin{align*}
        &\sum_{i=1}^{[\frac{r+\D}{\rho}]-1} \left(\frac{I_\Lambda Vol\left(\B_1\right)}{2^d\sqrt{d}}\left(r+\D-i\rho\right)^d+a\frac{I_\Lambda}{2^d}\left(r+\D-i\rho\right)^{d-2}\right)\\
        &\leq \frac{I_\Lambda Vol\left(\B_1\right)}{2^d\sqrt{d}}\Big[\Tilde{r}^d\left(1+\frac{\Tilde{r}}{\rho \left(d+1\right)}\right) - \frac{\rho^d}{\left(d+1\right)}\Big] + \\
        &\qquad a\frac{I_\Lambda}{2^d}\Big[\Tilde{r}^{d-2}\left(1+\frac{\Tilde{r}}{\rho\left(d-1\right)}\right) - \frac{\rho^{d-2}}{d-1}\Big] \\
        &\leq \frac{I_\Lambda Vol\left(\B_1\right)}{2^d\sqrt{d}}\Big[\Tilde{r}^d\left(1+\frac{\Tilde{r}}{\rho \left(d+1\right)}\right)\Big] +
        a\frac{I_\Lambda}{2^d}\Big[\Tilde{r}^{d-2}\left(1+\frac{\Tilde{r}}{\rho \left(d-1\right)}\right)\Big].
    \end{align*}
    Because this refers to the unscaled space, we now want to move back to our rescaled space, using $\btheta$ and getting
    \begin{align*}
        &\frac{I_\Lambda Vol\left(\B_1\right)}{2^d\sqrt{d}}\Big[\btheta^d\left(1+\frac{\btheta}{\rho \left(d+1\right)}\right)\Big] +
        a\frac{I_\Lambda}{2^d}\Big[\btheta^{d-2}\left(1+\frac{\btheta}{\rho \left(d-1\right)}\right)\Big] \\
        &\leq \frac{I_\Lambda}{2^d}\btheta^{d-2}\left(1+\frac{\btheta}{\rho \left(d-1\right)}\right)\left(\frac{Vol\left(\B_1\right)}{\sqrt{d}}\btheta^2 + a\right).
    \end{align*}
    \itai{Here is another way, that reaches a DIFFERENT result for some reason}
    SECOND proof: 
    \begin{align*}
        &A\sum_1^k \left(r+\D-i\cdot \rho\right)^{d-B} \\
        = &A(r+\D-\rho)^{d-B}\sum_0^k \left(1-\frac{i\cdot \rho}{r+\D-\rho}\right)^{d-B} \\
    \end{align*}
    Let $x=\frac{i\cdot \rho}{r+\D-\rho}$. Then, $x>0$ (obvious, bc $\D >\rho\Rightarrow r+\D>\rho$). but also:
    \[
        x<1 \iff \frac{i\cdot \rho}{r+\D-\rho} < 1 \iff i<\frac{r+\D}{\rho}-1
    \]
    which is exactly what $k=[\frac{r+\D}{\rho}]-1$ holds, and $i\leq k$. From here, we use \({(1-x)^N<e^{-xN}}\) for $x\in(0,1)$:
    \begin{align*}
        \leq &A(r+\D-\rho)^{d-B}\sum_0^k e^{-\frac{i(d-B)\cdot \rho}{r+\D-\rho}} \\
        \leq &A(r+\D-\rho)^{d-B}\sum_0^{\infty} e^{-\frac{i(d-B)\cdot \rho}{r+\D-\rho}} \\
        = &A(r+\D-\rho)^{d-B} \left(1+\frac{1}{e^{\frac{(d-B)\cdot \rho}{r+\D-\rho}}}\right)\\
        \leq &2A(r+\D-\rho)^{d-B}
    \end{align*}
    At $r\rightarrow\infty$ we get $A(r+\D-\rho)^{d-B}$ instead of $2A\cdot(\dots)$.
    \itai{And just to contrast, the integral approximation got:}
    \begin{align*}
        \left(13\right)&\leq A\Big[\left(r+\D-\rho\right)^{d-B} + \frac{\left(\left(r+\D-\rho\right)^{d-B+1} - \rho^{d-B+1}\right)}{\rho\left(d-B+1\right)}\Big].
    \end{align*}
    \itai{this is much shorter than the other term, but also much better asymptotically. but I dont logically understand this. this means I can just ignore the sum alltogether. and then I get that the sum of length is pretty much the same order of magnitude as the number of samples, which makes no sense to me.}
    \itai{OK, I made a graph and appearently the number of samples IS proportional to the sum of edges! this is a test for a unit ball, I need to make sure its not a special case}

     \begin{figure}[H]
    \centering
    \includegraphics[width=0.66\linewidth]{Images/new_plot.png}
    \caption{proportional! }
    \label{fig:reflection}
\end{figure}

    Returning to the lattice basis $\{e_i\}$, we will now show that
    \[
        I_\Lambda=2^{d-2}\sum_i\|e_i\|^2.
    \]
    This is because
    \begin{align*}
        I_\Lambda &= \sum_{x\in P_{F_{\Lambda,p}}}\|x - c_{\Lambda,p}\|^2=_{[\ref{Lattice-properties}]} \sum_{x\in P_{F_{\Lambda,p}}}\|x - \frac{\sum e_i}{2}\|^2\\
        &=\sum_{a_i\in\{0,1\}}\|\sum_1^d a_i e_i -\frac{\sum e_i}{2}\|^2=\sum_{a_i\in\{0,1\}}\langle\sum_1^d\left(a_i-\frac{1}{2}\right)e_i,\sum_1^d\left(a_i-\frac{1}{2}\right)e_i\rangle\\
        &=\sum_{b_i\in \{\pm\frac{1}{2}\}} \sum_{i,j} b_ib_j\langle e_i,e_j\rangle= \sum_{b_i\in \{\pm\frac{1}{2}\}}b_i^2\sum_i\|e_i\|^2 + \sum_{b_i\in \{\pm\frac{1}{2}\}}\sum_{i\neq j}\langle e_i,e_j\rangle.
    \end{align*}
    We now notice two things: first, that $b_i^2=\frac{1}{4} \forall i$. Second, that for each $i,j$ there are 4 options: $\frac{1}{2},\frac{1}{2}$ and $-\frac{1}{2},-\frac{1}{2}$ which both lead to $\frac{1}{4}$, and $-\frac{1}{2},\frac{1}{2}$ and $\frac{1}{2},-\frac{1}{2}$ which both lead to $-\frac{1}{4}$. For that reason, all the terms of the second inner-product cancel each other out! We are left with:
    \begin{align*}
        \sum_{b_i\in \{\pm\frac{1}{2}\}}\frac{1}{4}\sum_i\|e_i\|^2=\frac{2^d}{4}\sum_i\|e_i\|^2=2^{d-2}\sum_i\|e_i\|^2,
    \end{align*}
    inserting this into the expression, we get our final version:
    \begin{align*}
        I\left(r\right)&\leq \frac{I_\Lambda}{2^d}\btheta^{d-2}\left(1+\frac{\btheta}{\rho \left(d-1\right)}\right)\left(\frac{Vol\left(\B_1\right)}{\sqrt{d}}\btheta^2 + a\right)\\
        &= \frac{1}{4}\left(\sum_i\|e_i\|^2\right) \btheta^{d-2}\left(\frac{\btheta}{\rho \left(d-1\right)}+1\right)\left(\frac{Vol\left(\B_1\right)}{\sqrt{d}}\btheta^2+ a\right).
    \end{align*}
\end{proof}
\end{proof}

    First, notice that the first two terms form a telescopic sum, which leaves us with
    \begin{align}
        =r^{N+1} - (r-\frac{[(r-r_0)n]}{rn})^{N+1} - \frac{1}{n}\sum_{i=0}^{[rn]-1}(r-\frac{i+1}{n})^N\nonumber\\        
        \leq r^{N+1} - r_0^{N+1} -\frac{1}{n}\sum_{i=0}^{[(r-r_0)n]-1}(r-\frac{i+1}{n})^N.
    \end{align}
    Looking at the second term, we can say that
    \begin{align*}
        \frac{1}{n}\sum_{i=0}^{[(r-r_0)n]-1}(r-\frac{i+1}{n})^N=\frac{r^N}{n}\sum_{i=0}^{[(r-r_0)n]-1}(1-\frac{i+1}{rn})^N=\frac{r^N}{n}\sum_{i=1}^{[(r-r_0)n]}(1-\frac{i}{rn})^N.
    \end{align*}
    Letting \(x=\frac{i}{rn}\), we trivially know that $x>0$, but we also know that
    \[
        x<1 \iff \frac{i}{rn} < 1 \iff i < rn,
    \]
    and surely enough \(i<[(r-r_0)n]<(r-r_0)n<rn\). This means we can use the bound of \({(1-x)^N>e^{\frac{-Nx}{1-x}}}\), getting
    \begin{align*}
        &\frac{r^N}{n}\sum_{i=1}^{[(r-r_0)n]}(1-\frac{i}{rn})^N\geq\frac{r^N}{n}\sum_{i=1}^{[(r-r_0)n]}e^{-N\frac{\frac{i}{rn}}{1-\frac{i}{rn}}}\\
        =&\frac{r^N}{n}\sum_{i=1}^{[(r-r_0)n]}e^{-N\frac{i}{rn-i}}.
    \end{align*}
    But we know that $i<(r-r_0)n\Rightarrow r_0n<rn-i$, which means that ${e^{-N\frac{i}{rn-i}}>e^{-N\frac{i}{r_0n}}}$. Using this with the formula for a geometric sum, we get
    \begin{align*}
        >&\frac{r^N}{n}\sum_{i=1}^{[(r-r_0)n]}e^{-N\frac{i}{r_0n}}= \frac{r^N}{n}\sum_{i=1}^{[(r-r_0)n]}\left(e^{\frac{-N}{r_0n}}\right)^i\\
        =&\frac{r^N}{n}\frac{e^{-\frac{[(r-r_0)n]N}{rn}} - e^{-\frac{N}{r_0n}}}{e^{-\frac{N}{r_0n}} - 1}\\
    \end{align*}
    and since $[(r-r_0)n]<(r-r_0)n$, we get
    \begin{align*}
        >& \frac{r^N}{n}\frac{e^{-\frac{(r-r_0)nN}{rn}} - e^{-\frac{N}{r_0n}}}{e^{-\frac{N}{r_0n}} - 1}
        = \frac{r^N}{n}\frac{e^{-\frac{(r-r_0)N}{r}} - e^{-\frac{N}{r_0n}}}{e^{-\frac{N}{r_0n}} - 1}\\
        =& r^N\frac{\frac{1}{n}}{e^{-\frac{N}{r_0n}} - 1}\left(e^{-\frac{(r-r_0)N}{r}} - e^{-\frac{N}{r_0n}}\right).
    \end{align*}
    First, we can quickly say that
    \[
        \lim_{n\rightarrow\infty}\left(e^{-\frac{(r-r_0)N}{r}} - e^{-\frac{N}{r_0n}}\right)=e^{-\frac{(r-r_0)N}{r}} - 1.
    \]
    Second, notice that
    \[
        \lim_{n\rightarrow\infty}\frac{\frac{1}{n}}{e^{-\frac{N}{r_0n}} - 1}=\frac{0^+}{0^-},
    \]
    meaning we can use the L'H\^ospital rule, getting
    \[
    \lim_{n\rightarrow\infty}\frac{\frac{1}{n}} {e^{\frac{-N}{r_0n}} - 1}
    =\lim_{n\rightarrow\infty}\frac{\frac{-1}{n^2}}{\frac{N}{r_0n^2}e^{\frac{-N}{r_0n}}}=\frac{-r_0}{N}\lim_{n\rightarrow\infty}e^{\frac{N}{r_0n}}=\frac{-r_0}{N}.
    \]
    Summarizing, using (10) with the above we get
    \begin{align}
        \sum_{i=0}^{[(r-r_0)n]-1} \left(r_i^{N+1} - r_ir_{i+1}^N\right)\leq& r^{N+1} - r_0^{N+1} -\frac{1}{n}\sum_{i=0}^{[(r-r_0)n]-1}(r-\frac{i+1}{n})^N\nonumber\\
        \leq&r^{N+1} - r_0^{N+1} - \left(r^N\cdot\frac{-r_0}{N}\left(e^{-\frac{(r-r_0)N}{r}} - 1\right)\right)\nonumber\\
        =& r^{N+1} - r_0^{N+1} - r^N\cdot\frac{r_0}{N}\left(1 - e^{-\frac{(r-r_0)N}{r}}\right).
    \end{align}
    Let $f(r_0):=r_0\left(1 - e^{-\frac{(r-r_0)N}{r}}\right)=r_0\left(1 - e^{\frac{N}{r}r_0-N}\right)$. We need to choose $r_0$ so that $f(r_0)$ gets a maximal value, causing the whole expression get the lowest value - resulting in the tightest bound. Let us solve it:
    \begin{align*}
        0&=\frac{d}{dr_0}f(r_0)=\left(1 - e^{\frac{N}{r}r_0-N}\right) - r_0\cdot e^{\frac{N}{r}r_0-N}\cdot \frac{N}{r}\\
        1&=e^{\frac{N}{r}r_0-N}(\frac{N}{r}r_0+1) \Rightarrow e^{N+1}=e^{\frac{N}{r}r_0+1}\left(\frac{N}{r}r_0+1\right).
    \end{align*}
    This type of equation, i.e.
    \[
        x=ye^y,x>0,
    \]
    has a solution $y=W_0(x)$ called a \emph{Lambert W function}. So, we get
    \begin{align*}
        \frac{N}{r}r_0+1=W_0(e^{N+1})\Rightarrow 
        r_0=\frac{r}{N}\left(W_0(e^{N+1})-1\right)
    \end{align*}
    as the value for the maximum. \itai{add check that it is indeed a maximum..} Let us use this equation again to get
    \begin{align*}
        \frac{N}{r}r_0+1=W_0(e^{N+1}) 
        \Rightarrow& \frac{N}{r}r_0-N=W_0(e^{N+1})-N-1\\
        \Rightarrow& e^{\frac{N}{r}r_0-N}=e^{W_0(e^{N+1})}e^{-(N+1)},
    \end{align*}
    but by definition
    \[
        W_0(e^{N+1})e^{W_0(e^{N+1})}=e^{N+1}\Rightarrow W_0(e^{N+1})=\frac{e^{N+1}}{W_0(e^{N+1})},
    \]
    which means that
    \begin{align*}
        e^{\frac{N}{r}r_0-N}=e^{W_0(e^{N+1})}e^{-(N+1)}=\frac{e^{N+1}}{W_0(e^{N+1})}e^{-(N+1)}=\frac{1}{W_0(e^{N+1})}.
    \end{align*}
    From here we can calculate the maximal $f$ value, getting
    \begin{align*}
        f(r_0) =& r_0\left(1 - e^{\frac{N}{r}r_0-N}\right) \\
        \leq& \frac{r}{N}\left(W_0(e^{N+1})-1\right)(1-\frac{1}{W_0(e^{-(N+1)})})\\
        =&\frac{r}{N}(W_0(e^{N+1})-1 - 1+\frac{1}{W_0(e^{-(N+1)})})\\
        =&\frac{r}{N}(W_0(e^{N+1})+\frac{1}{W_0(e^{-(N+1)})}-2)\\
        =&\frac{r}{N}(a_N+\frac{1}{a_N}-2), a_N:=W_0(e^{N+1}).
    \end{align*}
    Returning to (11), we get the expression
    \[
        (11) \leq r^{N+1} - r_0^{N+1} - \frac{r^N}{N}\left[\frac{r}{N}(a_N+\frac{1}{a_N}-2)\right].
    \]
    Remember that after summing the annulus we add the inner $r_0$-ball, which causes our final expression to be
    \[
        \leq r^{N+1} - \frac{r^{N+1}}{N^2}(a_N+\frac{1}{a_N}-2)=r^{N+1}\left(1-\frac{a_N+\frac{1}{a_N}-2}{N^2}\right)
    \]

In order that we do get better evaluations for each lattice, we will need to calculate $\D, \rho, \sum_i \|e_i\|^2$. Notice that, so far, this was done for a general lattice. Each lattice will differentiate by these values.
Using this theorem, we move on to our second contribution for this section.
\begin{thm}[\Lattices collision complexity]\label{collision_complexity_results}
    Fix $\beta=\beta\left(\delta,\epsilon\right)$ and $\theta:=\theta\left(\beta,R\right)$ as used previously. Consider the configuration space  $\C=\B_R$ and some $r$-ball $\B_r\subseteq\C$. Assume $\beta\leq1$ and $R\leq\sqrt{d}$. Then the following bounds hold:
    \begin{enumerate}[topsep=1pt,itemsep=1ex,partopsep=1ex,parsep=1ex]
        \item 
            $\text{CC}\left(\XZ\cap\B_r\right) \approx O\left(\frac{\beta}{\sqrt{d}}\left(\theta\left(\beta,r\right)\sqrt{\frac{d}{4}}+\sqrt{d}-\frac{1}{2}\right)^{d-1}\right)$
        \item 
            $\text{CC}\left(\XD\cap\B_R\right) \approx O\left(\frac{\beta}{\sqrt{d}}\left(\frac{r}{\beta}\sqrt{\frac{d}{8}-\frac{1}{16}}+\sqrt{\frac{9}{4}d}-\frac{1}{2}\right)^{d-1}\right)$
        \item 
            $\text{CC}\left(\XD\cap\B_R\right) \approx O\left(\frac{\beta}{\sqrt{d}}\left(\frac{r}{\beta}\sqrt{\frac{d}{12}}+\sqrt{d}-\frac{1}{2}\right)^{d-1}\right)$
    \end{enumerate}
\end{thm}
\begin{proof}
    First note that assuming our C-Space has $R\leq \sqrt{d}$ is reasonable, because
        \[
        \sqrt{d}\geq R\geq d\beta\Rightarrow\beta\leq\frac{1}{\sqrt{d}},
    \]
    and taking $d=10$ (similarly to our example in theorem~\ref{sample_lower}) requires us to have $\beta\leq 0.32$---similar to the situation in theorem~\ref{sample_lower}, which we explained makes sense.

    
    As mentioned before, we need to determine $\D, \rho, \sum_i \|e_i\|^2$ for each lattice. We get $I_\Lambda$ using all our previous knowledge. Regarding the inradius, Burchard et al.~\cite{burchard2015perimeter} gives us a lower bound for the inradius:
        \[
        \rho \geq Det\left(e_1,\dots,e_n\right) = \sqrt{\det\left(\Lambda\right)}
    \]
    The first term is the determinant of the lattice's base vectors. This is the volume of $F_\Lambda$, which is also a size given to us by Conway~\cite{conway2013sphere} as $\sqrt{\delta\left(\Lambda\right)}$. Finally, the diameter of each lattice will be determined using \cref{lemma:diameter}. Let us move on to our analysis of the lattices. Note that we will be using the shortened version of the approximation
    \[
        I\left(r\right)\approx O\left(\frac{\sum_i\|e_i\|^2}{\rho\left(d-1\right)}\btheta^{d-1}\right).
    \]
    because, as explained above, $Vol\left(\B_1\right)$ goes to 0 very quickly and becomes irrelevant.


    First, we look at $\ZN$. Using theorem \ref{thm:decomp_lattices} we know that our basis is 
    \[
        c\cdot e_i=c\cdot\left(0,\dots,1,\dots,0\right),
    \]
    for $c=\frac{2\beta}{\sqrt{d}}$. Being that the FRs are squares, the inradius is naturally exactly $\rho=\frac{c}{2}=\frac{\beta}{\sqrt{d}}$. 
    For the diameter, in a cube the longest distance is between the two "corners" (i.e. $(0,\dots,0)$ and $(1,\dots,1)$), so our diameter is
    \begin{align*}
        D&=c\|\left(1,\dots,1\right)\|\\
        &=c\sqrt{d}=\frac{2\beta}{\sqrt{d}}\sqrt{d}=2\beta.
    \end{align*}
    Lastly, we can see that 
    \[
        \sum_i \|e_i\|^2 =c^2d=\frac{4\beta^2}{d}d=4\beta^2,
    \]
    which gives us
    \begin{align*}
        &\frac{\sum_i\|e_i\|^2}{\rho\left(d-1\right)}=
        \frac{4\beta^2}{\frac{\beta}{\sqrt{d}}\left(d-1\right)}=\frac{4\beta\sqrt{d}}{d-1}.
    \end{align*}
    We now calculate $\btheta$ using the previous calculations and theorem~\ref{thm:decomp_lattices}, getting
    \begin{align*}
        \btheta &= \frac{r+D-\rho}{\beta}f_{\ZN}\\
        &= \frac{r+2\beta - \frac{\beta}{\sqrt{d}}}{\beta}\frac{\sqrt{d}}{2}\\
        & = \frac{r}{\beta}\frac{\sqrt{d}}{2}+\sqrt{d}-\frac{1}{2}.
    \end{align*}
    With these two calculations we get the final complexity
    \begin{align*}
        I\left(r\right)&\approx O\left(\frac{\sum_i\|e_i\|^2}{\rho\left(d-1\right)}\btheta^{d-1}\right)\\
        &=O\left(\frac{4\beta\sqrt{d}}{d-1}\left(\frac{r}{\beta}\frac{\sqrt{d}}{2}+\sqrt{d}-\frac{1}{2}\right)^{d-1}\right)\\
        &=O\left(\frac{\beta}{\sqrt{d}}\left(\frac{r}{\beta}\frac{\sqrt{d}}{2}+\sqrt{d}-\frac{1}{2}\right)^{d-1}\right)\\
        &=O\left(\frac{\beta}{\sqrt{d}}\left(\theta\left(\beta,r\right)\sqrt{\frac{d}{4}}+\sqrt{d}-\frac{1}{2}\right)^{d-1}\right).
    \end{align*}
    Next we look at $\DN$, using the base we got from theorem~\ref{thm:decomp_lattices}---the base vectors are $c\left(1,0,\dots,0\right)$ to $c\left(0,\dots,0,1,0\right)$ and $c\left(\frac{1}{2},\dots,\frac{1}{2}\right)$, with $c=\frac{4\beta}{\sqrt{2d-1}}$ (WLOG, $d$ is odd). We know from Conway~\cite{conway2013sphere} (page 120) that $\rho=\frac{1}{2}$, so after rescaling the space, we get $\frac{2\beta}{\sqrt{2d-1}}$. Next, our base vectors are all in the $[\mathbb{R}_+,\dots,\mathbb{R}_+]$ quadrant, so naturally (similar to the "cube" case!) the longest distance between any two points in the FR has to be between $\overset{\rightarrow}{0}$ and $\sum_i e_i$, getting that our diameter is
    \begin{align*}
        D&=c\|\left(1,0,\dots,0\right) + \dots + \left(0,\dots,1,0\right)+\left(\frac{1}{2},\dots,\frac{1}{2}\right)\|\\
        &=\frac{c}{2}\|\left(3,3,\dots,3,1\right)\|=\frac{c}{2}\sqrt{9\left(d-1\right)+1}\\
        &=\frac{c}{2}\sqrt{9d-8}=2\beta\sqrt{\frac{9d-8}{2d-1}}.
    \end{align*}
    Lastly, we can see that 
    \begin{align*}
        &\sum_i \|e_i\|^2 =c^2\left(d-1\right)+c^2\frac{d}{4}=\frac{c^2}{4}\left(5d-4\right)\\
        &=\frac{4\beta^2}{2d-1}\left(5d-4\right)=
        4\beta^2\frac{\left(5d-4\right)}{2d-1},
    \end{align*}
    which gives us
    \begin{align*}
        &\frac{\sum_i\|e_i\|^2}{\rho\left(d-1\right)}=
        \frac{4\beta^2\frac{\left(5d-4\right)}{2d-1}}{\frac{2\beta}{\sqrt{2d-1}}\left(d-1\right)}\\
        &=2\beta\frac{\sqrt{2d-1}\left(5d-4\right)}{\left(d-1\right)\left(2d-1\right)}.
    \end{align*}
    We now calculate $\btheta$ using the previous calculations and theorem~\ref{thm:decomp_lattices}, getting
    \begin{align*}
        \btheta &= \frac{r+D-\rho}{\beta}f_{\DN}\\
        &= \frac{r+2\beta\sqrt{\frac{9d-8}{2d-1}}-\frac{2\beta}{\sqrt{2d-1}}}{\beta}\frac{\sqrt{2d-1}}{4}\\
        &= \frac{r}{\beta}\frac{\sqrt{2d-1}}{4}+\sqrt{\frac{9}{4}d-2}-\frac{1}{2}.
    \end{align*}
    With these two calculations we get the final complexity
    \begin{align*}
        I\left(r\right)&\approx O\left(\frac{\sum_i\|e_i\|^2}{\rho\left(d-1\right)}\btheta^{d-1}\right)\\
        &=O\left(2\beta\frac{\sqrt{2d-1}\left(5d-4\right)}{\left(d-1\right)\left(2d-1\right)}\left(\frac{r}{\beta}\frac{\sqrt{2d-1}}{4}+\sqrt{\frac{9}{4}d-2}-\frac{1}{2}\right)^{d-1}\right)\\
        &=O\left(\frac{\beta\sqrt{d-\frac{1}{2}}}{d-1}\left(\frac{r}{\beta}\sqrt{\frac{d}{8}-\frac{1}{16}}+\sqrt{\frac{9}{4}d-2}-\frac{1}{2}\right)^{d-1}\right)\\
        &=O\left(\frac{\beta}{\sqrt{d}}\left(\frac{r}{\beta}\sqrt{\frac{d}{8}-\frac{1}{16}}+\sqrt{\frac{9}{4}d}-\frac{1}{2}\right)^{d-1}\right)\\
    \end{align*}
    Finally, we look at $\AN$. The base we got from theorem~\ref{thm:decomp_lattices} uses the vectors $\left(c,-c,0\dots,0\right)$ to $\left(c,0,\dots,0,-c,0\right)$ and $\left(\frac{-cd}{d+1},\frac{c}{d+1}\dots,\frac{c}{d+1}\right)$ for $c=\sqrt{\frac{12\left(d+1\right)}{d\left(d+2\right)}}\beta$. As our method relies on the diameter of the FR, and this FR's diameter is roughly $O\left(d\right)$ (due to the first coordinate being filled on all vectors) in contrast to $O\left(\sqrt{d}\right)$ in the other lattices, we will use a different FR: we will use
    \begin{align*}
        v_1&=\left(c,-c,0,\dots,0\right)\\
        v_2&=\left(0,c,-c,\dots,0\right)\\
        v_d&=\left(0,\dots,0,c,-c,0\right)\\
        w&=\left(\frac{-cd}{d+1},\frac{c}{d+1}\dots,\frac{c}{d+1}\right).
    \end{align*}
    This spans the same lattice, as we can recreate the original basis from this (by adding $i$ vectors in a row).  To calculate the inradius, we refer to pages LXII and 10 of Conway~\cite{conway2013sphere}, citing that 
    \[
        \rho = \left(\delta\sqrt{\det{\Lambda}}\right)^{\frac{1}{d}},
    \]
    where $\delta$ is a parameter given in page 115 which is independent of choice of basis. The only thing left to calculate, then, is $\det{\Lambda}=det\left(G_\Lambda G_\Lambda^t\right)$. Let us calculate it:
    \begin{align*}
        \det{\Lambda}&=det\left(G_\Lambda G_\Lambda^t\right)\\
        &=\det\begin{pmatrix}
            1 & -1 &  0  & \dots & 0 & 0 & 0 \\
            0 & 1  &  -1 & \dots & 0 & 0 & 0 \\
            . & .  &  . & \dots & . & .  & .\\
            0 & 0  &  0 & \dots & 1 & -1 & 0\\
            \frac{-d}{d+1} & \frac{1}{d+1} & \frac{1}{d+1} & \dots & \frac{1}{d+1} & \frac{1}{d+1} & \frac{1}{d+1}
        \end{pmatrix}
        \begin{pmatrix}
            1 & 0 & \dots & 0 & 0 & \frac{-d}{d+1} \\
            -1 & 1 & \dots & 0 & 0 & \frac{1}{d+1} \\
            0 & -1 & \dots & 0 & 0 & \frac{1}{d+1} \\
            . & . & \dots & . & .  & .\\
            0 & 0 & \dots & 0 & -1 & \frac{1}{d+1}\\
            0 & 0 & \dots & 0 & 0 & \frac{1}{d+1}
        \end{pmatrix}\\
        &=\det\begin{pmatrix}
            2 & -1 &  0  & 0 & 0 &\dots & 0 & 0 & -1\\
            -1 & 2  &  -1 & 0 & 0 &\dots & 0 & 0 & 0\\
            0 & -1  &  2 & -1 & 0 & \dots & 0 & 0 & 0\\
            . & .  &  . & . & . & \dots & . & . & .\\
            0 & 0  &  0 & 0 & 0 & \dots & -1 & 2 & 0\\
            -1 & 0  &  0 & 0 & 0 &\dots & 0 & 0 & \frac{d}{d+1}\\
        \end{pmatrix},\\
    \end{align*}
    now add all $d-1$ rows to the last row:
    \begin{align*}
        &=\det\begin{pmatrix}
            2 & -1 &  0  & 0 & 0 &\dots & 0 & 0 & -1\\
            -1 & 2  &  -1 & 0 & 0 &\dots & 0 & 0 & 0\\
            0 & -1  &  2 & -1 & 0 & \dots & 0 & 0 & 0\\
            . & .  &  . & . & . & \dots & . & . & .\\
            0 & 0  &  0 & 0 & 0 & \dots & -1 & 2 & 0\\
            0 & 0  &  0 & 0 & 0 &\dots & 0 & 1 & \frac{d}{d+1}-1\\
        \end{pmatrix}.
    \end{align*}
    We now cancel out the first non-zero coordinate in the second row using the first row, and then multiply it by an integer so it becomes an integer row, getting
    \begin{align*}
        &=\det\begin{pmatrix}
            2 & -1 &  0  & 0 & 0 &\dots & 0 & 0 & -1\\
            0 & 1.5  &  -1 & 0 & 0 &\dots & 0 & 0 & 0.5\\
            0 & -1  &  2 & -1 & 0 & \dots & 0 & 0 & 0\\
            . & .  &  . & . & . & \dots & . & . & .\\
            0 & 0  &  0 & 0 & 0 & \dots & -1 & 2 & 0\\
            0 & 0  &  0 & 0 & 0 &\dots & 0 & 1 & \frac{d}{d+1}-1\\
        \end{pmatrix}\\
        &=\frac{1}{2}\det\begin{pmatrix}
            2 & -1 &  0  & 0 & 0 &\dots & 0 & 0 & -1\\
            0 & 3  &  -2 & 0 & 0 &\dots & 0 & 0 & -1\\
            0 & -1  &  2 & -1 & 0 & \dots & 0 & 0 & 0\\
            . & .  &  . & . & . & \dots & . & . & .\\
            0 & 0  &  0 & 0 & 0 & \dots & -1 & 2 & 0\\
            0 & 0  &  0 & 0 & 0 &\dots & 0 & 1 & \frac{d}{d+1}-1\\
        \end{pmatrix}.
    \end{align*}
    we keep going like this recursively
    \begin{align*}
            &=\frac{1}{2}\det\begin{pmatrix}
            2 & -1 &  0  & 0 & 0 &\dots & 0 & 0 & -1\\
            0 & 3  &  -2 & 0 & 0 &\dots & 0 & 0 & -1\\
            0 & 0  &  \frac{4}{3} & -1 & 0 & \dots & 0 & 0 & \frac{-1}{3}\\
            . & .  &  . & . & . & \dots & . & . & .\\
            0 & 0  &  0 & 0 & 0 & \dots & -1 & 2 & 0\\
            0 & 0  &  0 & 0 & 0 &\dots & 0 & 1 & \frac{d}{d+1}-1\\
        \end{pmatrix},
    \end{align*}
    until ultimately, we reach the last row, getting
    \begin{align*}
        &=\frac{1}{\left(d-1\right)!}\det\begin{pmatrix}
            2 & -1 &  0  & 0 & 0 &\dots & 0 & 0 & -1\\
            0 & 3  &  -2 & 0 & 0 &\dots & 0 & 0 & -1\\
            0 & 0  &  4 & -3 & 0 & \dots & 0 & 0 & -1\\
            . & .  &  . & . & . & \dots & . & . & .\\
            0 & 0  &  0 & 0 & 0 & \dots & 0 & d & -1\\
            0 & 0  &  0 & 0 & 0 &\dots & 0 & 1 & \frac{d}{d+1}-1\\
        \end{pmatrix}\\
        &=\frac{1}{\left(d-1\right)!}\det\begin{pmatrix}
            2 & -1 &  0  & 0 & 0 &\dots & 0 & 0 & -1\\
            0 & 3  &  -2 & 0 & 0 &\dots & 0 & 0 & -1\\
            0 & 0  &  4 & -3 & 0 & \dots & 0 & 0 & -1\\
            . & .  &  . & . & . & \dots & . & . & .\\
            0 & 0  &  0 & 0 & 0 & \dots & 0 & d & -1\\
            0 & 0  &  0 & 0 & 0 &\dots & 0 & 0 & \frac{d}{d+1}-1+\frac{1}{d}\\
        \end{pmatrix}\\
        &=\frac{1}{\left(d-1\right)!}\cdot d!\left(\frac{1}{d\left(d+1\right)}\right)
        =\frac{1}{d+1}.
    \end{align*}    
    Looking at page 115 of~\cite{conway2013sphere}, we see that the determinant of the classic basis is similar to our FR's, which means we can just use the inradius for the classic basis, which is quoted to be
    \[
        \rho = \frac{1}{2}\sqrt{\frac{d}{d+1}},
    \]
    and after rescaling the space we get
    \[
        \rho=\frac{1}{2}\sqrt{\frac{d}{d+1}}\sqrt{\frac{12\left(d+1\right)}{d\left(d+2\right)}}\beta
        =\frac{1}{2}\sqrt{\frac{12}{d+2}}\beta.
    \]
    Next, to calculate the diameter we need to consider distances between all binary sums ("in the sum or not") of base vectors. If we add $v_i,v_{i+1}$ then the $i+1$-coordinate cancels out and we get $\left(\dots,1,0,-1,\dots\right)$. This lowers the possible contribution of this coordinate, as it only appears in these vectors and in $w$, and in $w$ it is $\frac{-1}{d+1}$---resulting in a total of $\frac{1}{d+1}$ (in absolute value). This is in contrast to adding $v_i,v_j$ where $j\geq i+2$, so we get $\left(\dots,1,-1,\dots,1,-1,\dots\right)$. When adding it to $w$ now, we get $1-\frac{1}{d+1}>\frac{1}{d+1}$ (in absolute value). So we see that in general, the longest difference between FR points we can get is when adding $u=w+v_1+v_3+v_5+\dots$, getting
    \begin{align*}
        D^2&=\|u\|^2=c^2\|\left(1,-1,1,-1,\dots,-1,0\right)+\left(\frac{-d}{d+1},\frac{1}{d+1},\dots,\frac{1}{d+1}\right)\|^2\\
        &=c^2\|\left(1-\frac{d}{d+1},-\left(1-\frac{1}{d+1}\right),1-\frac{1}{d+1},\dots,-\left(1-\frac{1}{d-1}\right),\frac{1}{d+1}\right)\|^2\\
        &=c^2\|\left(\frac{1}{d+1},-\left(1-\frac{1}{d+1}\right),1-\frac{1}{d+1},\dots,-\left(1-\frac{1}{d+1}\right),\frac{1}{d+1}\right)\|^2\\
        &=c^2\left(\frac{2}{\left(d+1\right)^2}+\frac{d}{2}\left(1-\frac{1}{d+1}\right)^2\right)\\
        &=\frac{1}{2}c^2\frac{4+d^3}{\left(d+1\right)^2},\\
        \Rightarrow D&=\frac{c}{\sqrt{2}}\frac{d}{d+1}\sqrt{d+\frac{4}{d^2}}\leq\frac{c}{\sqrt{2}}\sqrt{d+2}\\
        &=\frac{1}{\sqrt{2}}\sqrt{\frac{12\left(d+1\right)}{d\left(d+2\right)}}\beta\sqrt{d+1}=\sqrt{\frac{6\left(d+1\right)}{d}}\beta\leq\sqrt{12}\beta.
    \end{align*}
    Lastly, we can see that 
    \begin{align*}
        &\sum_i \|e_i\|^2 =c^2\left(d\cdot\left(2\right)+1\cdot\left(\frac{d^2}{\left(d+1\right)^2}+d\frac{1}{\left(d+1\right)^2}\right)\right)\\
        &=c^2\left(2d+\frac{d}{d+1}\right)
        = \frac{12\left(d+1\right)}{d\left(d+2\right)}\beta^2\left(2d+\frac{d}{d+1}\right)\\
        &\leq\frac{12}{d}\beta^2\left(2d+1\right)\leq 36\beta^2,
    \end{align*}
    which gives us
    \begin{align*}
        &\frac{\sum_i\|e_i\|^2}{\rho\left(d-1\right)}\leq
        \frac{36\beta^2}{\frac{1}{2}\sqrt{\frac{12}{d+2}}\beta\left(d-1\right)}\\
        &=\frac{36\beta\sqrt{d+2}}{\sqrt{3}\left(d-1\right)}.
    \end{align*}
    We now calculate $\btheta$ using the previous calculations and theorem~\ref{thm:decomp_lattices}, getting
    \begin{align*}
        \btheta &= \frac{r+D-\rho}{\beta}f_{\DN}\\
        &\leq \frac{r+\sqrt{12}\beta-\frac{1}{2}\sqrt{\frac{12}{d+2}}\beta}{\beta}\sqrt{\frac{d\left(d+2\right)}{12\left(d+1\right)}}\\
        &= \frac{r}{\beta}\sqrt{\frac{d\left(d+2\right)}{12\left(d+1\right)}}+\sqrt{\frac{d\left(d+2\right)}{\left(d+1\right)}}-\frac{1}{2}\sqrt{\frac{d}{\left(d+1\right)}}\\
        &=\sqrt{\frac{d+2}{d+1}}\left(\frac{r}{\beta}\sqrt{\frac{d}{12}}+\sqrt{d}-\frac{1}{2}\sqrt{\frac{d}{d+2}}\right).
    \end{align*}
    Remember that we have seen in corollary~\ref{cor:specific_sample_complexity} that $\left(\sqrt{\frac{d+2}{d+1}}\right)^d\rightarrow\sqrt{e}$. With  the calculations above, we get the final complexity
    \begin{align*}
        I\left(r\right)&\approx O\left(\frac{\sum_i\|e_i\|^2}{\rho\left(d-1\right)}\btheta^{d-1}\right)\\
        &=O\left(\frac{36\beta\sqrt{d+2}}{\sqrt{3}\left(d-1\right)}\left(\sqrt{\frac{d+2}{d+1}}\left(\frac{r}{\beta}\sqrt{\frac{d}{12}}+\sqrt{d}-\frac{1}{2}\sqrt{\frac{d}{d+2}}\right)\right)^{d-1}\right)\\
        &=O\left(\frac{\beta}{\sqrt{d}}\left(\sqrt{\frac{d+2}{d+1}}\right)^{d-1}\left(\frac{r}{\beta}\sqrt{\frac{d}{12}}+\sqrt{d}-\frac{1}{2}\sqrt{\frac{d}{d+2}}\right)^{d-1}\right)\\
        &=O\left(\frac{\beta}{\sqrt{d}}\left(\frac{r}{\beta}\sqrt{\frac{d}{12}}+\sqrt{d}-\frac{1}{2}\sqrt{\frac{d}{d+2}}\right)^{d-1}\right)\\
        &=O\left(\frac{\beta}{\sqrt{d}}\left(\frac{r}{\beta}\sqrt{\frac{d}{12}}+\sqrt{d}-\frac{1}{2}\right)^{d-1}\right)\\
    \end{align*}
\end{proof}
\subsection*{Comparing collision 
complexity for \Lattices}
The collision complexity of the lattices seems to be naturally "carried over" from the sample complexity. In this aspect, all conclusions we drew previously apply here: $\AN$ still rises as having the least total edge length, resulting in a lower amount of collision checks. The higher degree of the power in comparison to sample complexity ($d-1$ against $d-2$) means these differences are even more accentuated. 

\subsection*{Comparing $\AN$ to random sample sets}
 
 First thing we compared was the classic $PRM$ algorithm to our setting, with the purpose of seeing how $\AN$ stands against random sample sets, which are known to generally give good enough results. This test was conducted with a regular PRM algorithm, with the "H" map, at $d=6$ using 3 moving disks, with some augments that we will soon list. Just to note, we used a much higher clearance of $\delta=25$, which was found to empirically work with $\AN$. This high clearance value allows us also to see another facet of the LPRM issue: clearance rates are pivotal to having feasible runtimes in MPs. The following method was used in this test section:
 \begin{enumerate}
     \item Each of the three robots has a set location in a corner of the "H" map. 
     \item We create a test set this way: shift every robot a bit, into a legal position, by $N$ steps. This creates $N^3$ sets. Permute the three robots (we plan in a $d=3\cdot2=6$, so these are different tests), getting 6 test. Overall: $6N^3$ tests. We chose $N=3$ for 162 tests.
     \item On each of these tests, run two problems: first, a run of $\AN$ with LPRM. On finishing, save the length and solution time and get the number of samples used. Second, a run of PRM using a fixed amount of samples---the number we got from the LPRM run.
     \item The second test is then run in two different modes. First, we use LPRM's radius, the one used from Dayan et al.~\cite{dayan2023near}. Then we use PRM*'s radius. 
     \item What we look for are relative qualities (our data divided by their data) and success rates, all averaged over the 162 tests.
 \end{enumerate}
 Results: (all done at $\delta=25$)
 \begin{enumerate}
     \item \underline{Both with same radius:} 
     \begin{enumerate}
        \item Success rate: $\AN$ with LPRM: 100\%, random with PRM: $\epsilon=10$ with success=42\%, $\epsilon=5$ with success=53\%, $\epsilon=2$ with success=100\%.
        \item Total time quality, $\frac{\text{LPRM time}}{\text{rnd PRM time}}$: $\epsilon=10$ with quality=7.77, $\epsilon=5$ with quality=10.87, $\epsilon=2$ with quality=11.93.
        \item $A^*$ \textbf{run only} time quality, $\frac{A^*-\text{LPRM time}}{A^*-\text{rnd PRM time}}$: $\epsilon=10$ with quality=1.96, $\epsilon=5$ with quality=2.39, $\epsilon=2$ with quality=4.55.
        \item Length quality, $\frac{\text{LPRM length}}{\text{rnd PRM time}}$: $\epsilon=10$ with quality=1.25, $\epsilon=5$ with quality=1.22, $\epsilon=2$ with quality=1.13.
     \end{enumerate}
     \item \underline{Each with its own radius:}
          \begin{enumerate}
         \item Success rate: all 100\% at all values.
         \item Total time quality, $\frac{\text{LPRM time}}{\text{rnd PRM time}}$: $\epsilon=10$ with quality=1.12, $\epsilon=5$ with quality=1.3, $\epsilon=2$ with quality=3.96.
        \item $A^*$ \textbf{run only} time quality, $\frac{A^*-\text{LPRM time}}{A^*-\text{rnd PRM time}}$: $\epsilon=10$ with quality=1.2, $\epsilon=5$ with quality=1.64, $\epsilon=2$ with quality=4.13.
         \item Length quality, $\frac{\text{LPRM length}}{\text{rnd PRM time}}$: $\epsilon=10$ with quality=1.25, $\epsilon=5$ with quality=1.22, $\epsilon=2$ with quality=1.13.
     \end{enumerate}
 \end{enumerate}
%  \begin{figure*}[!h]
%   \centering
%   \subfloat[this image is a placeholder]{
%     \includegraphics[width=0.3\textwidth]{Images/CC_relative_by_e.png}
%     \label{fig:exp_3body:hmap}}
%   \hfil
%   \subfloat[this image is a placeholder]{
%     \includegraphics[width=0.3\textwidth]{Images/CC_relative_by_e.png}
%     \label{fig:exp_3body:rndpoly}}
%   \caption{The scenarios we used to test our sample set, in each map we wanted each disk-robot to switch places with the next neighbor.}
%   \label{fig:exp_3body}
% \end{figure*}

\emph{Second section conclusions:} there are several conclusions we can draw. 
\begin{enumerate}
    \item First, that if PRM isn't given a good enough sampling radius, then it will not find a solution a meaningful percentage of the time. This is in contrast to LPRM, which is deterministically guaranteed to get a result, after you pass certain sample density "threshold".
    \item Second, we can see that while rnd-PRM had a better average path quality, it was not substantial (at worse 2x longer).
    \item Lastly, it is clear that rnd-PRM is quicker than LPRM, which is to be expected: we have much more edges in LPRM than in rnd-PRM. This is true also in case we only count the $A^*$ runtime only (without the time to build the graph). In $\epsilon=10$ the runtimes do get very close, but in realistically lower $\epsilon$ values rnd-PRM is much quicker again.
\end{enumerate}

Overall, it is notable that LPRM has not demonstrated a significant improvement over rnd-PRM, at least in our setting. Its strengths lay in its theoretical promise to achieve a good result in finite time, in contrast to rnd-PRM. We will demonstrate in the next section the promise of the $\AN$ lattice over the others.

\subsection*{Comparing runtimes between the lattices}
 
 The second set of tests compared the lattices themselves: \Lattices. This version of LPRM saw two important improvements to the runtime, that greatly improved it.
 \begin{enumerate}
     \item The first one was to move to a fully "lazy" implementation, on a graph that already implicitly exists: we know in advance that each point in the lattice has $2d$ neighbors, and we know their location. Therefor we don't have to build the graph at all, and just consider the implicit neighbors. That is in contrast to a graph with random sampling, in which the graph is incrementally built as you add samples, which makes comparing to an implicit graph impossible.
     \item The second was to recognize that we can calculate a points neighbor's in a general ball, in advance, and then just have a point's neighbors be these same points - only shifted! This is due to the regularity of the lattice structure.
 \end{enumerate}
The results, each averaged over 6 different case, are as follows:

\begin{table*}[t]
\centering
\caption{LPRM vs RPRM, Both with the same radius (from Dayan~\cite{dayan2023near})}
\begin{tabularx}{\linewidth}{|X|X|X|X||X|X|X|X|X|X|X|X|X|} \hline
 Map & ID & Scenario & NN &
 \multicolumn{4}{|c|}{Construction times [S]} &
 \multicolumn{4}{|c|}{$A^*$ times [S]} &
  Success
 \\
 \hline
 & & & & $\ZN$ & $\DN$ & $\AN$ & Random & $\ZN$ & $\DN$ & $\AN$ & Random & \\
 \hline
 \multirow{4}{*}{K}
  & 1 & 2-Swap & No  & $0.0001$    &$9E(-5)$&   $9E(-5)$ & $2.308$ & $5.505$ & $2.212$ & \cellcolor{green} $1.214$ & \cellcolor{pink} $1.569$ & 100\%\\
  \cline{2-13}
  & 1 & 2-Swap & Yes  & $7.643$    &$3.897$&   $3.087$ & $2.308$ & $6.140$ & $2.468$ & \cellcolor{green}$1.363$ & \cellcolor{pink}$1.569$ & 100\%\\
  \cline{2-13}
  & 2& 3-Swap   & No & $0.0042$    &$0.0014$&   $0.0032$ & NA & $78.493$ & $10.391$ &\cellcolor{green} $3.786$ & \cellcolor{pink}NA & 0\% \\
    \cline{2-13}
  & 2& 3-Swap   & Yes & $0.0042$    &$0.0014$&   $0.0032$ & NA & $78.493$ & $10.391$ &\cellcolor{green} $3.786$ & \cellcolor{pink}NA & 0\% \\
 \hline
  \multirow{4}{*}{UM}
  & 3 & 2-Swap  & No & $0.00028$    &$0.00013$&   $0.00014$ & $1.061$ & $54.638$ & $17.463$ & \cellcolor{green}$7.316$ & \cellcolor{pink}$8.650$ & 100\% \\
  \cline{2-13}
    & 3 & 2-Swap  & Yes & $3.277$    &$1.799$&   $1.278$ & $1.061$ & $64.355$ & $19.701$ &\cellcolor{green} $8.346$ & \cellcolor{pink}$8.650$ & 100\% \\
  \cline{2-13}
  & 4& 3-Swap  & No & $0.00637$    &$0.001851$&   $0.000691$ & $40.972$ & $506.254$ & $25.898$ &\cellcolor{green} $7.029$ & \cellcolor{pink}$21.380$ & 95\% \\
    \cline{2-13}
    & 4& 3-Swap  & Yes & $437.237$    &$124.184$&   $68.451$ & $40.972$ & $912.857$ & $56.335$ & \cellcolor{pink}$21.774$ & \cellcolor{green}$21.380$ & 95\% \\
 \hline
 \multirow{4}{*}{RP}
  & 5& 2-Swap  & No & $0.000179$    &$0.000172$&   $0.450$ & $0.465$ & $5.532$ & $1.954$ & \cellcolor{green}$1.612$ & \cellcolor{pink}$2.323$ & 100\%\\
  \cline{2-13}
    & 5& 2-Swap  & Yes & $1.472$    &$0.770$&   $0.554$ & $0.465$ & $6.335$ & $2.185$ & \cellcolor{green}$1.672$ & \cellcolor{pink}$2.323$ & 100\%\\
  \cline{2-13}
  & 6 & 3-Swap  & No & $0.0062$    &$0.0020$&   $0.0007$ & $21.227$ & $812.614$ & $7.207$ & \cellcolor{pink}$7.347$ & \cellcolor{green}$5.833$ & 100\%\\
    \cline{2-13}
  & 6 & 3-Swap  & Yes & $225.829$    &$61.552$&   $33.688$ & $21.227$ & $1127.65$ & $9.771$ & \cellcolor{pink}$10.898$ & \cellcolor{green}$5.833$ & 100\%\\
 \hline
\end{tabularx}
\end{table*}

\begin{enumerate}
    \item \textbf{Test \#1}: d=6, $\delta=9, \epsilon=10$, "H" map, 3 disk robots $(x,y)$. Results:
    \begin{enumerate}
        \item Zn average total runtime: 1141 seconds.
        \item Dn* average total runtime: 55 seconds.
        \item An* average total runtime: 23 seconds.
    \end{enumerate}
    \underline{Conclusion:} $\AN$ runs 50x faster than $\ZN$.
    \item \textbf{Test \#2}: d=6, $\delta=??, \epsilon=10$, "random polygons" map, 3 disk robots $(x,y)$. Results:
    \begin{enumerate}
        \item Zn average total runtime: ?? seconds.
        \item Dn* average total runtime: ?? seconds.
        \item An* average total runtime: ?? seconds.
    \end{enumerate}
    \underline{Conclusion:} $\AN$ runs ?? faster than $\ZN$.
\end{enumerate}
\emph{First section conclusions:} it is clear from these comparisons that $\AN$ performs much better than $\ZN$. Another point to notice is the differences between the lattices: if we wanted an even lower clearance or stretch factor, $\ZN$'s runtime would become infeasible to run. We believe that if you are considering using a deterministic sample set, for any purpose, the set $\AN$ has demonstrated a substantial superiority over other lattices---making it a good choice to use.

\ifincludeappendix
\appendix
  \newpage
\appendix
\appendices

\section{Robot Setups}\label{app:robot_setup}

\subsection{Simulation Robot Setups}
To ensure fairness, we utilize the same Franka Panda arm for evaluations in both the LIBERO~\cite{LIBERO23} and our Open6DOR V2 benchmarks. For SIMPLER~\cite{simplerenv24}, we use the Google Robot exclusively to conduct the baseline experiments, adhering to all configurations outlined in SIMPLER, as presented in Table~\ref{tab:simpler_env}. 

\subsection{Real World Robot Setups}
As for manipulation tasks, in \cref{fig:robots}, we perform 6-DoF rearrangement tasks using the Franka Panda equipped with a gripper and the UR robot arm with a LeapHand, while articulated object manipulation is conducted using the Flexiv arm equipped with a suction tool. All the robot arms mount a Realsense D415 camera to its end for image capturing.
\begin{figure}[h!]
\centering
\includegraphics[width=0.96\linewidth]{figs/src/robots.pdf}
\vspace{-5pt}
\captionof{figure}{\textbf{The robots used in our real-world experiments.}}
\vspace{-5pt}
\label{fig:robots}
\end{figure}


In \cref{fig:franka_setup}, we present the workspace and robotic arm for real-world 6-DoF rearrangement. Unlike Rekep~\cite{ReKep24}, CoPa~\cite{CoPa24} et al., we utilize only a single RealSense D415 camera. This setup significantly reduces the additional overhead associated with environmental setup and multi-camera calibration, and it is more readily reproducible.
\begin{figure}[h!]
\centering
\includegraphics[width=1.0\linewidth]{figs/src/franka_setup.pdf}
\captionof{figure}{\textbf{6-DoF rearrangement robot setup.}}
\vspace{-10pt}
\label{fig:franka_setup}
\end{figure}


As for navigation tasks, we provide a visualization of our robotic dog in~\cref{fig:dog_setup}. Following Uni-Navid~\cite{uninavid24}, our robotic dog is Unitree GO2 and we mount a RealSense D455 camera on the head of the robotic dog. Here, we only use the RGB frames with a resolution of $640\times480$ in the setting of  $90^\circ$ HFOV. We also mount a portable Wi-Fi at the back of the robot dog, which is used to communicate with the remote server (send captured images and receive commands). Unitree GO2 is integrated with a LiDAR-L1, which is only used for local motion planning. 
\begin{figure}
\begin{center}
  \includegraphics[width=0.7\linewidth]{figs/src/robotdog.pdf}
\end{center}
   \caption{\textbf{Navigation robot setup.} We use Unitree GO2 as our embodiment, and we mount RealSense D455, a portable Wi-Fi and a LiDAR-L1. Note that, our model only takes RGB frames as input. The portable Wi-Fi is used for communication with the remote server and the Lidar is used for the local controller API of Unitree Dog.}
   \vspace{-10pt}
\label{fig:dog_setup}
\end{figure}


\section{Additional Experiments}\label{app:add_exp}

\subsection{Articulated Objects Manipulation Evaluation}
We further integrate \ours~with articulated object manipulation, as illustrated in \cref{tab:manip}, and evaluate its practicality in robotic manipulation tasks using the PartNet-Mobility Dataset within the SAPIEN~\cite{SAPIEN20} simulator. Our experimental setup follows ManipLLM~\cite{ManipLLM24}, employing the same evaluation metrics. Specifically, we directly utilize the segmentation centers provided by SAM as contact points, leverage PointSO to generate contact directions, and use VLM to determine subsequent motion directions. The results demonstrate significant improvements over the baseline. Notably, our model achieves this performance without dividing the data into training and testing sets, operating instead in a fully zero-shot across most tasks. This underscores the robustness and generalization of our approach.
% !TEX root = ../../top.tex
% !TEX spellcheck = en-US

% Temporary modification of the table column separation width
\setlength\mytabcolsep{\tabcolsep}
\setlength\tabcolsep{2pt}


\renewcommand{\manipimgcar}[1]{\includegraphics[width=0.13\linewidth]{#1}}
\renewcommand{\manipimgmix}[1]{\includegraphics[width=0.07\linewidth]{#1}}
\renewcommand{\manipimg}[1]{\includegraphics[width=0.08\linewidth]{#1}}


\begin{figure*}[t]
	\centering
	\small
%	\vspace{-3mm}
	\begin{tabular}{c|c|c}
		% Cars 
		\begin{tabular}{c|c}
			\manipimgcar{fig/manip/car_olivier_lat_489_277/init} & \manipimgcar{fig/manip/car_olivier_lat_34_308/init} \\
			\manipimgcar{fig/manip/car_olivier_lat_489_277/final} & \manipimgcar{fig/manip/car_olivier_lat_34_308/final} \\
		\end{tabular}
		&
		% Mixers 
		\begin{tabular}{cc|cc}
			\manipimgmix{fig/manip/mixer_lat_1424_171/init} & \manipimgmix{fig/manip/mixer_lat_1424_171/final} & \manipimgmix{fig/manip/mixer_lat_957_629/init} & \manipimgmix{fig/manip/mixer_lat_957_629/final} \\
		\end{tabular}
		&
		% Chairs
		\begin{tabular}{cc|cc}
			\manipimg{fig/manip/chair_sepreg_lat_550_531/init} & \manipimg{fig/manip/chair_sepreg_lat_550_531/final} & \manipimg{fig/manip/chair_sepreg_lat_745_952/init} & \manipimg{fig/manip/chair_sepreg_lat_745_952/final} \\
		\end{tabular} \\
		\midrule
		% Cars 2
		\begin{tabular}{c|c}
		\manipimgcar{fig/manip/car_olivier_pose_290/init} & \manipimgcar{fig/manip/car_olivier_pose_691/init} \\
		\manipimgcar{fig/manip/car_olivier_pose_290/final} & \manipimgcar{fig/manip/car_olivier_pose_691/final} \\
		\end{tabular}
		&
		% Mixers 2
		\begin{tabular}{cc|cc}
		\manipimgmix{fig/manip/mixer_pose_872/init} & \manipimgmix{fig/manip/mixer_pose_872/final} & \manipimgmix{fig/manip/mixer_pose_183/init} & \manipimgmix{fig/manip/mixer_pose_183/final} \\
		\end{tabular}
		&
		% Chairs 2
		\begin{tabular}{cc|cc}
		\manipimg{fig/manip/chair_sepreg_pose_422/init} & \manipimg{fig/manip/chair_sepreg_pose_422/final} & \manipimg{fig/manip/chair_sepreg_pose_859/init} & \manipimg{fig/manip/chair_sepreg_pose_859/final} \\
		\end{tabular} \\
	\end{tabular}
	\caption{\textbf{Additional shape manipulation.} We manipulate four shapes per dataset: (\textit{top}) by changing the latent of specific parts (car body, mixer helix, and all chair parts) and (\textit{bottom}) by editing part poses (car wheels, mixer width, chair width and height). In all cases, the parts adapts to the modifications and to each other, maintaining a coherent whole.}
	\label{fig:supp-manip}
\end{figure*}


% Restore table column separation width
\setlength{\tabcolsep}{\mytabcolsep}


\subsection{Spatial VQA on EmbSpatial-Bench~\cite{embspatial24} \& SpatialBot-Bench~\cite{SpatialBot24}}
To further demonstrate \sofar's spatial reasoning capabilities, we conducted Spatial VQA tests within the EmbSpatial-Bench~\cite{embspatial24} and SpatialBot-Bench~\cite{SpatialBot24}. As shown in \cref{tab:embspatial,tab:spatialbot}, \sofar~significantly outperformed all baselines, achieving more than a 20\% performance improvement in EmbSpatial-Bench.
\begin{table}[h!]
\centering
\setlength{\tabcolsep}{8.5pt}
\caption{Zero-shot performance of LVLMs in EmbSpatial-Bench~\cite{embspatial24}. \textbf{Bold} indicates the best results.}
\resizebox{0.97\linewidth}{!}{
\begin{tabular}{lcc}
\toprule
Model & Generation & Likelihood \\
\midrule
BLIP-2~\cite{BLIP223} & 37.99 & 35.71 \\
InstructBLIP~\cite{InstructBLIP23} & 38.85 & 33.41 \\
MiniGPT4~\cite{MiniGPT4_23} & 23.54 & 31.70 \\
LLaVA-1.6~\cite{LLaVA23} & 35.19 & 38.84 \\
\midrule
GPT-4V~\cite{GPT4Vision23} & 36.07 & - \\
\rowcolor{linecolor1}Qwen-VL-Max~\cite{qwenvl23} & 49.11 & - \\
\rowcolor{linecolor2}\textbf{\ours} & \textbf{70.88} & - \\
\bottomrule
\end{tabular}
}
\label{tab:embspatial}
\end{table}
\begin{table}[h!]
\centering
\setlength{\tabcolsep}{3pt}
\caption{Zero-shot performance of LVLMs in SpatialBot-Bench~\cite{SpatialBot24}.
SpatialBot-3B: SpatialBot-Phi2-3B-RGB, SpatialBot-8B: SpatialBot-Llama3-8B-RGB.
}
\resizebox{1.00\linewidth}{!}{
\begin{tabular}{lcccccc}
\toprule
Model & Pos & Exist & Count & Reach & Size & Avg \\
\midrule
ChatGPT-4o~\cite{GPT4o24} & 70.6 & 85.0 & 84.5 & 51.7 & \textbf{43.3} & 67.0\\
SpatialBot-3B~\cite{SpatialBot24} & 64.7 & 80.0 & 88.0 & \textbf{61.7} & 28.3 & 64.5\\
SpatialBot-8B~\cite{SpatialBot24} & 55.9 & 80.0 & \textbf{91.2} & 40.0 & 20.0 & 57.4\\
\midrule
\rowcolor{linecolor2}\textbf{\ours} & \textbf{76.5} & \textbf{87.5} & 80.0 & 57.5 & 40.0 & \textbf{68.3}\\
\bottomrule
\end{tabular}
}
\label{tab:spatialbot}
\end{table}


\subsection{Close-Loop Execution Experiment}\label{app:close_loop}
We demonstrate the closed-loop replan capabilities of \sofar~within Simpler-Env~\cite{simplerenv24} in \cref{fig:close_loop}. The instruction for both tasks is ``pick the coke can'' In \cref{fig:close_loop} (a), the model initially misidentified the coke can as a Fanta can. After correction by the VLM, the model re-identified and located the correct object. In \cref{fig:close_loop} (b), the model accidentally knocks over the Coke can during motion due to erroneous motion planning. Subsequently, the model re-plans and successfully achieves the grasp.

\subsection{Long Horizon Object Manipulation Experiment}\label{app:long_horizon}
\cref{fig:long_horizon} illustrates the execution performance of our model on long-horizon tasks. Through the VLM~\cite{GPT4o24,gemini23}, complex instructions such as ``making breakfast'' and ``cleaning up the desktop'' can be decomposed into sub-tasks. In the second example, we deliberately chose complex and uncommon objects as assets, such as ``Aladdin's lamp'' and ``puppets'', but \sofar~is able to successfully complete all tasks.
\begin{figure}[h!]
\centering
\includegraphics[width=1.0\linewidth]{figs/src/close_loop_execution.pdf}
\vspace{-15pt}
\captionof{figure}{\textbf{Close-loop execution of our \sofar.}}
\label{fig:close_loop}
\end{figure}


\subsection{In the Wild Evaluation of Semantic Orientation}
We provide a qualitative demonstration of the accuracy of PointSO under in-the-wild conditions, as shown in \cref{fig:in_the_wild}, where the predicted Semantic Orientation is marked in the images. We obtained single-sided point clouds by segmenting objects using Florence-2~\cite{florence2} and SAM~\cite{SAM23} and fed them into PointSO. It can be observed that our model achieves good performance across different views, objects, and instructions, which proves the effectiveness and generalization of PointSO.
\begin{figure*}[h!]
\centering
\includegraphics[width=1.0\linewidth]{figs/src/long_horizon.pdf}
\captionof{figure}{\textbf{Long-horizon object manipulation experiment of our \sofar.}}
\label{fig:long_horizon}
\end{figure*}

\begin{figure}[h!]
\centering
\includegraphics[width=1.0\linewidth]{figs/src/in_the_wild.pdf}
\vspace{-15pt}
\captionof{figure}{\textbf{In the wild evaluation of PointSO.}}
\label{fig:in_the_wild}
\end{figure}

\begin{figure}[t!]
\centering
\includegraphics[width=1.0\linewidth]{figs/src/cross_view.pdf}
\captionof{figure}{\textbf{Cross view generalization of our \sofar.}}
\label{fig:cross_view}
\end{figure}


\subsection{Cross-View Generalization}
\sofar~gets point clouds in the world coordinate system using an RGB-D camera to obtain grasping poses, and it is not limited to a fixed camera perspective. In addition, PointSO generates partial point clouds from different perspectives through random camera views to serve as data augmentation for training data, which also generalizes to camera perspectives in the real world. \cref{fig:cross_view} illustrates \sofar's generalization capability for 6-DoF object manipulation across different camera poses. It can be observed that whether it's a front view, side view, or ego view, \sofar~can successfully execute the ``upright the bottle'' instruction.

\subsection{Failure Case Distribution Analysis}
Based on the failure cases from real-world experiments, we conducted a quantitative analysis of the failure case distribution for \sofar, with the results shown in \cref{fig:failure_case}. It can be observed that 31\% of the failures originated from grasping issues, including objects being too small, inability to generate reasonable grasping poses, and instability after grasping leading to sliding or dropping. Next, 23\% were due to incorrect or inaccurate Semantic Orientation prediction. For tasks such as upright or upside - down, highly precise angle estimation (<5°) is required for smooth execution. Object analysis and detection accounted for approximately 20\% of the errors. The instability of open-vocabulary detection modules like Florence2~\cite{florence2} and Grounding DINO~\cite{groundingdino23} often led to incorrect detection of out-of-distribution objects or object parts. In addition, since our Motion Planning did not take into account the working space range of the robotic arm and potential collisions of the manipulated object, occasional deadlocks and collisions occurred during motion. Finally, there were issues with the Task Planning of the VLM~\cite{GPT4o24,gemini23}. For some complex Orientations, the VLM occasionally failed to infer the required angles and directions to complete the task. Employing a more powerful, thought-enabled VLM~\cite{gpt_o1} might alleviate such errors.

\begin{figure*}[t]
\centering
\includegraphics[width=\linewidth]{CameraReady/Figures/failure_case_image.pdf}
\caption{Failure cases of our approach.}
\label{fig:failure_case}
\end{figure*}

\subsection{Ablation Study}\label{app:ablation}
\subsubsection{Scaling Law}
The scaling capability of models and data is one of the most critical attributes today and a core feature of foundation models~\cite{FoundationModel21}. We investigate the performance of PointSO across different data scales, as illustrated in \cref{tab:scaling_law}. 
We obtain the subset for OrienText300K from Objaverse-LVIS, which consists of approximately 46,000 3D objects with category annotations. The selection was based on the seven criteria mentioned in the main text. Objects meeting all seven criteria formed the strict subset, comprising around 15k objects. When including objects without textures and those of lower quality, the total increases to approximately 26k objects.
It can be seen that the increase in data volume is the most significant factor driving the performance improvement of PointSO. It can be anticipated that with further data expansion, such as Objaverse-XL~\cite{ObjaverseXL23}, PointSO will achieve better performance.
\begin{table}[t!]
\setlength{\tabcolsep}{7pt}
\caption{\textbf{Data scaling property} of semantic orientation with different training data scales evaluated on OrienText300K test split. All experiments are conducted with PointSO-Base.
}
\label{tab:scaling_law}
\centering
\resizebox{1.0\linewidth}{!}{
\begin{tabular}{lccccc}
\toprule[0.95pt]
    Data Scale & \texttt{45°} & \texttt{30°} & \texttt{15°} & \texttt{5°} & Average \\ 
    \midrule[0.6pt]
    5\% & 57.03 & 46.09 & 39.84 & 27.34 & 42.58 \\
    10\% & 61.72 & 53.13 & 43.75 & 30.47 & 47.27 \\
    50\% & 76.56 & 72.66 & 66.41 & 56.25 & 67.97 \\
    \rowcolor{linecolor2}100\% & \textbf{79.69} & \textbf{77.34} & \textbf{70.31} & \textbf{62.50} & \textbf{72.46} \\
    \bottomrule[0.95pt]
\end{tabular}
}
\end{table}

\subsubsection{Cross-Modal Fusion Choices}\label{app:fusion}
We further conduct an ablation study on the multi-modal fusion methods in PointSO, testing commonly used feature fusion techniques such as cross-attention, multiplication, addition, and concatenation, as shown in \cref{tab:fusion}. The results indicate that simple addition achieves the best performance. This may be attributed to the fact that instructions in the semantic domain are typically composed of short phrases or sentences, and the text CLS token already encodes sufficiently high-level semantic information.
% \usepackage{multirow}
% \usepackage{booktabs}


\begin{table}[h]
	\centering
	\caption{Model performance under two coefficient fusion methods.}
	\begin{tabular}{c|ccc} 
		\toprule
		\multirow{2}{*}{Fusion mode} & \multicolumn{3}{c}{R40}                           \\
		& Easy           & Moderate       & Hard            \\ 
		\hline
		straight                     & 90.92          & 82.84          & 80.29           \\
		our                          & \textbf{91.96} & \textbf{83.31} & \textbf{80.59}  \\
		\bottomrule
	\end{tabular}
\label{tabel6}
\end{table}

\begin{table*}[h!]
\setlength{\tabcolsep}{3pt}
\caption{\textbf{Ablation study of open vocabulary detection modules} on Open6DOR~\cite{Open6DOR24} perception tasks.
}
\label{tab:detection_ab}
\centering
\resizebox{0.98\linewidth}{!}{
\begin{tabular}{lccccccccccc}
\toprule[0.95pt]
\multirow{2}{*}[-0.5ex]{Method} & \multicolumn{3}{c}{\textbf{Position Track}} & \multicolumn{4}{c}{\textbf{Rotation Track}} & \multicolumn{3}{c}{\textbf{6-DoF Track}} & \multirow{2}{*}[-0.5ex]{Time Cost (s)}\\
\cmidrule(lr){2-4} \cmidrule(lr){5-8} \cmidrule(lr){9-11}
& Level 0 & Level 1 & Overall & Level 0 & Level 1 & Level 2 & Overall & Position & Rotation & Overall \\ 
\midrule[0.6pt]
YOLO-World~\cite{yoloworld24} & 59.0 & 37.7 & 53.3 & 48.3 & 36.1 & 62.0 & 44.9 & 53.4 & 44.6 & 27.8 & \textbf{7.4s}\\
\rowcolor{linecolor1}Grounding DINO~\cite{groundingdino23} & 92.2 & 71.5 & 86.7 & 64.7 & 41.1 & 69.8 & 55.5 & 87.2 & 51.6 & 44.6 & 9.2s\\
\rowcolor{linecolor2}Florence-2~\cite{florence2} & \textbf{96.0} & \textbf{81.5} & \textbf{93.0} & \textbf{68.6} & \textbf{42.2} & \textbf{70.1} & \textbf{57.0} & \textbf{92.7} & \textbf{52.7} & \textbf{48.7} & \textbf{8.5s}\\
\bottomrule[0.95pt]
\end{tabular}
}
\end{table*}

\subsubsection{Open Vocabulary Object Detection Module}
\sofar~utilize a detection foundation model to localize the interacted objects or parts, then generate masks with SAM~\cite{SAM23}. Although not the SOTA performance on the COCO benchmark, Florence-2~\cite{florence2} exhibits remarkable generalization in in-the-wild detection tasks, even in simulator scenarios. \cref{tab:detection_ab} illustrates the performance of various detection modules in Open6DOR~\cite{Open6DOR24} Perception, where Florence-2 achieves the best results and outperforms Grounding DINO~\cite{groundingdino23} and YOLO-World~\cite{yoloworld24}.

\vspace{3pt}
\section{Additional Implementation Details}\label{app:implementation_details}

\subsection{Detail Real World Experiment Results}\label{app:detail_realworld}
To fully demonstrate the generalization of \sofar~rather than cherry-picking, we carefully design 60 different real-world experimental tasks, covering more than 100 different and diverse objects. Similar to the Open6DOR~\cite{Open6DOR24} benchmark in the simulator, we divide these 60 tasks into three parts: position-track, orientation-track, and the most challenging comprehensive \& 6-DoF-track. Each track is further divided into simple and hard levels. The position-simple track includes tasks related to front \& back \& left \& right spatial relationships, while the position-hard track includes tasks related to between, center, and customized. The orientation-simple track includes tasks related to the orientation of object parts, and the orientation-hard track includes tasks related to whether the object is upright or flipped (with very strict requirements for angles in both upright and flipped cases). Comprehensive tasks involve complex instruction understanding and long-horizon tasks; 6-DoF tasks simultaneously include requirements for both object position and orientation instructions. In \cref{tab:detailed_realworld}, we present the complete task instructions, as well as the performance metrics of \sofar~and the baseline. Due to the large number of tasks, we performed each task three times. It can be seen that \sofar~achieves the best performance in all tracks, especially in the orientation-track and comprehensive \& 6-DoF-track. We also show all the objects used in the real-world experiments in \cref{fig:real_obj}, covering a wide range of commonly and uncommonly used objects in daily life.

\vspace{-10pt}
\begin{table*}[t!]
\setlength{\tabcolsep}{6pt}
\caption{\textbf{Detailed zero-shot real-world 6-DoF rearrangement results}.}
\label{tab:detailed_realworld}
\centering
\resizebox{0.96\linewidth}{!}{
\begin{tabular}{lcccc}
\toprule[0.95pt]
    Task & CoPa~\cite{CoPa24} & ReKep-Auto~\cite{ReKep24} & \sofar-LLaVA~(Ours) & \sofar~(Ours) \\ 
    \midrule[0.6pt]
    \multicolumn{5}{c}{\textit{Positional Object Manipulation}}\\
    \midrule
    Move the soccer ball to the right of the bread. & 2/3 & 3/3 & 3/3 & \textbf{3/3} \\
    Place the doll to the right of the lemon. & 3/3 & 3/3 & 3/3 & \textbf{3/3} \\
    Put the pliers on the right side of the soccer ball. & 1/3 & 1/3 & 3/3 & \textbf{2/3} \\
    Move the pen to the right of the doll. & 3/3 & 2/3 & 3/3 & \textbf{3/3} \\
    Place the carrot on the left of the croissant. & 2/3 & 3/3 & 2/3 & \textbf{2/3} \\
    Move the avocado to the left of the baseball. & 3/3 & 2/3 & 2/3 & \textbf{3/3} \\
    Pick the pepper and place it to the left of the charger. & 1/3 & 2/3 & 2/3 & \textbf{2/3} \\
    Place the baseball on the left side of the mug. & 3/3 & 2/3 & 2/3 & \textbf{3/3} \\
    Arrange the flower in front of the potato. & 2/3 & 3/3 & 2/3 & \textbf{3/3} \\
    Put the volleyball in front of the knife. & 3/3 & 3/3 & 3/3 & \textbf{3/3} \\
    Place the ice cream cone in front of the potato. & 2/3 & 3/3 & 2/3 & \textbf{3/3} \\
    Move the bitter melon to the front of the forklift. & 2/3 & 1/3 & 2/3 & \textbf{2/3} \\
    Place the orange at the back of the stapler. & 3/3 & 2/3 & 3/3 & \textbf{3/3} \\
    Move the panda toy to the back of the shampoo bottle. & 2/3 & 3/3 & 3/3 & \textbf{2/3} \\
    pick the pumpkin and place it behind the pomegranate. & \textbf{3/3} & 2/3 & 1/3 & 2/3 \\
    Place the basketball at the back of the board wipe. & 2/3 & 2/3 & 3/3 & \textbf{2/3} \\
    Put the apple inside the box. & 3/3 & 2/3 & 3/3 & \textbf{3/3} \\
    Place the waffles on the center of the plate. & 3/3 & 2/3 & 3/3 & \textbf{3/3} \\
    Move the hamburger into the bowl.& 2/3 & 2/3 & 2/3 & \textbf{3/3} \\
    Pick the puppet and put it into the basket. & 1/3 & 2/3 & 2/3 & \textbf{2/3} \\
    Drop the grape into the box. & 2/3 & 3/3 & 3/3 & \textbf{2/3} \\
    Put the doll between the lemon and the USB. & 2/3 & 2/3 & 2/3 & \textbf{3/3} \\
    Set the duck toy in the center of the cart, bowl, and camera. & 2/3 & 1/3 & 2/3 & \textbf{2/3} \\
    Place the strawberry between the Coke bottle and the glue. & 2/3 & 2/3 & 3/3 & \textbf{3/3} \\
    Put the pen behind the basketball and in front of the vase. & 2/3 & 1/3 & 2/3 & \textbf{2/3} \\
    Total success rate& 74.7\% & 72.0\% & 81.3\% & \textbf{85.3\%} \\
    \midrule
    \multicolumn{5}{c}{\textit{Orientational Object Manipulation}}\\
    \midrule
    Turn the yellow head of the toy car to the right. & 2/3 & 2/3 & 1/3 & \textbf{2/3} \\
    Adjust the knife handle so it points to the right. & 2/3 & 1/3 & 2/3 & \textbf{2/3} \\
    Rotate the cap of the bottle towards the right. & 2/3 & 2/3 & 2/3 & \textbf{2/3}\\
    Rotate the tip of the screwdriver to face the right. & 0/3 & 0/3 & 1/3 & \textbf{1/3}\\
    Rotate the stem of the apple to the right. & 0/3 & 1/3 & 1/3 & \textbf{2/3}\\
    Turn the front of the toy car to the left. & 0/3 & 0/3 & 2/3 & \textbf{2/3} \\
    Rotate the cap of the bottle towards the left. & 2/3 & 1/3 & 1/3 & \textbf{2/3}\\
    Adjust the pear's stem to the right. & 1/3 & 1/3 & 1/3 & \textbf{1/3}\\
    Turn the mug handle to the right. & 1/3 & 1/3 & 2/3 & \textbf{2/3}\\
    Rotate the handle of the mug to towards right.& 2/3 & 1/3 &\textbf{2/3} & 1/3\\
    Rotate the box so the text side faces forward. & 0/3 & 1/3 & 0/3 & \textbf{1/3}\\
    Adjust the USB port to point forward. & 0/3 & 0/3 & 1/3 & \textbf{1/3}\\
    Set the bottle upright. & 0/3 & 1/3 & 0/3 & \textbf{1/3}\\
    Place the coffee cup in an upright position. & 1/3 & 1/3 & 2/3 & \textbf{2/3}\\
    Upright the statue of liberty& 0/3 & 0/3 & \textbf{1/3} & 0/3\\
    Stand the doll upright. & 0/3 & 1/3 & 0/3 & \textbf{1/3}\\
    Right the Coke can. & 0/3 & 0/3 & 1/3 & \textbf{1/3}\\
    Flip the bottle upside down. & 0/3 & 0/3 & 0/3 & \textbf{1/3}\\
    Turn the coffee cup upside down. & 0/3 & 0/3 & 1/3 & \textbf{1/3}\\
    Invert the shampoo bottle upside down. & 0/3 & 0/3 & 0/3 & \textbf{0/3}\\
    Total success rate& 21.7\% & 23.3\% & 35.0\% & \textbf{43.3\%} \\
    \midrule
    \multicolumn{5}{c}{\textit{Comprehensive 6-DoF Object Manipulation}}\\
    \midrule
    Pull out a tissue.& 3/3 & 3/3 & 2/3 & \textbf{3/3}\\
    Place the right bottle into the
    box and arrange it in a 3×3 pattern. & 0/3 & 0/3 & 0/3 & \textbf{1/3}\\
    Take the tallest box and position it on the right side. & 1/3 & 1/3 & 3/3 & \textbf{3/3}\\
    Grasp the error bottle and put it on the right side. & 1/3 & 2/3 & 1/3 & \textbf{2/3} \\
    Take out the green test tube and place it between the two bottles. & 2/3 & 2/3 & 3/3 & \textbf{3/3}\\
    Pack the objects on the table into the box one by one. & 1/3 & 1/3 & 0/3 & \textbf{1/3}\\
    Rotate the loopy doll to face the yellow dragon doll & 0/3 & 1/3 & 1/3 & \textbf{1/3}\\
    Right the fallen wine glass and arrange it neatly in a row. & 0/3 & 0/3 & 0/3 & \textbf{0/3}\\
    Grasp the handle of the knife and cut the bread.& 0/3 & 0/3 & 0/3 & \textbf{1/3}\\
    Pick the baseball into the cart and turn the cart to facing right. & 0/3 & 0/3 & 1/3 & \textbf{2/3}\\
    Place the mug on the left of the ball and the handle turn right. & 0/3 & 0/3 & 1/3 & \textbf{1/3}\\
    Aim the camera at the toy truck. & 1/3 & 0/3 & 1/3 & \textbf{1/3}\\
    Rotate the flashlight to illuminate the loopy. & 0/3 & 0/3 & 1/3 & \textbf{1/3}\\
    Put the pen into the pen container. & 0/3 & 1/3 & 0/3 & \textbf{1/3} \\
    Pour out chips from the chips cylinder to the plate. & 0/3 & 1/3 & 1/3 & \textbf{1/3} \\
    Total success rate& 20.0\% & 26.7\% & 33.3\% & \textbf{48.9\%} \\
    \bottomrule[0.95pt]
\end{tabular}
}
\end{table*}
\vspace{-10pt}
\begin{figure}[t!]
\centering
\includegraphics[width=1.0\linewidth]{figs/src/real_obj.pdf}
\captionof{figure}{\textbf{The real-world assets used in our real-world experiments.} More than 100 diverse objects are used in our 6-DoF rearrangement experiments.}
\label{fig:real_obj}
\vspace{-10pt}
\end{figure}


\input{tabs/PointSO_configurations}
\usepackage{graphicx}
\usepackage{url}
\usepackage{color, soul}
\usepackage{bm}
\usepackage{amsmath} % for the equation* environment
\usepackage{lineno}

\subsection{PointSO Model Details}\label{app:pointso_details}
For PointSO, we utilize FPS + KNN to perform patchify and employ a small PointNet~\cite{PointNet} as the patch encoder. Subsequently, a standard Transformer encoder is adopted as the backbone, followed by a single linear layer to map the output to a three-dimensional vector space. All parameter configurations follow prior work on point cloud representation learning~\cite{ACT23,ReCon23,ShapeLLM24}. Detailed hyperparameter and model configurations are provided in \cref{tab:hyper_params,tab:PointSO_configs}.

\subsection{SoFar-LLaVA Model Details}\label{app:model_details}
\begin{figure*}[t!]
\begin{center}
\includegraphics[width=0.89\linewidth]{figs/src/sofar_llava.pdf}
\caption{\textbf{Pipeline of \sofar-LLaVA}, a fine-tuned vision language model based on visual instruction tuning.
}
\label{fig:sofar_llava}
\vspace{-5pt}
\end{center}
\end{figure*}

In addition to leveraging the extensive knowledge and strong generalization capabilities of closed-source/open-source pretrained VLMs~\cite{ChatGPT22,gemini23,qwenvl23} for zero-shot or in-context learning, \sofar~can also enhance the planning performance of open-source models through visual instruction tuning for rapid fine-tuning. The pipeline of the model is illustrated in \cref{fig:sofar_llava}. A JSON-formatted 6-DoF scene graph, processed through a text tokenizer, along with the image refined by SoM~\cite{SoM23}, is fed into an LLM (\eg, LLaMA~\cite{LLaMA23,LLaMA2_23}) for supervised fine-tuning~\cite{LLaVA23}.
In the Open6DOR~\cite{Open6DOR24} task, we supplement the training dataset with additional samples retrieved and manually annotated from Objaverse~\cite{objaverse23}, ensuring alignment with the object categories in the original benchmark. This dataset includes approximately 3,000 6-DoF object manipulation instructions. Using this data, we construct dialogue-style training data based on ChatGPT and train the \sofar-LLaVA model. The training hyperparameters are detailed in \cref{tab:hyper_params}. Similarly, we finetune PointSO on this training dataset and achieve superior performance on the Open6DOR task.

\subsection{ChatGPT API Costs}
The knowledge of OrienText300K is derived from the annotations of 3D modelers on Sketchfab, combined with ChatGPT's filtering and comprehension capabilities. To generate semantic direction annotations, we filter the 800K dataset of Objaverse~\cite{objaverse23} and apply ChatGPT to approximately 350K of the filtered data to generate semantic text-view index pairs. The OpenAI official API was used for these calls, with the GPT-4o version set to 2024-08-06 and the output format configured as JSON. The total cost for debugging and execution amounted to approximately \$10K.


\section{Additional Benchmark Statistic Analysis}
\subsection{6-DoF SpatialBench Analysis}
We conduct a statistical analysis of the manually constructed 6-DoF SpatialBench, with category comparisons and word cloud visualizations shown in \cref{fig:spatialvqa_statistic}. We collect diverse image data from the internet, encompassing scenes such as indoor, outdoor, and natural landscapes. The questions may involve one or multiple objects, with varying levels of uncertainty in image resolution. Most importantly, we are the first to propose a VQA benchmark for orientation understanding, focusing on both quantitative and qualitative evaluation of orientation.


\subsection{Open6DOR V2 Analysis}
Open6DOR V2 builds upon Open6DOR V1 by removing some incorrectly labeled data and integrating assets and metrics into Libero, enabling closed-loop policy evaluation. The detailed number of tasks is presented in \cref{tab:open6dorv2_statistic}, comprising over 4,500 tasks in total. Notably, we remove level 2 of the position track in Open6DOR V1~\cite{Open6DOR24} because it requires manual inspection, which is not conducive to open-source use and replication by the community. Besides, due to the randomness of object drops in the scene, approximately 8\% of the samples already satisfy the evaluation metrics in their initial state.

\vspace{3pt}
\section{Additional Related Works}\label{app:related_work}
\subsection{3D Representation Learning}
Research on 3D Representation Learning encompasses various methods, including point-based~\cite{PointNet,PointNet++}, voxel-based~\cite{voxelnet15}, and multiview-based approaches~\cite{MVCNN3D15,MVTN}. 
Point-based methods~\cite{PointNext,PointTrans21} have gained prominence in object classification~\cite{ModelNet15,ScanObjectNN19} due to their sparsity yet geometry-informative representation. On the other hand, voxel-based methods~\cite{voxelrcnn21,SyncSpecCNN17,VPP23} offer dense representation and translation invariance, leading to a remarkable performance in object detection~\cite{ScanNet17} and segmentation~\cite{ShapeNetPart16, S3DIS16}.
The evolution of attention mechanisms~\cite{AttentionIsAllYouNeed,ReKo23} has also contributed to the development of effective representations for downstream tasks, as exemplified by the emergence of 3D Transformers~\cite{PointTrans21,groupfree21, voxeltransformer21}. Notably, 3D self-supervised representation learning has garnered significant attention in recent studies. PointContrast~\cite{PointContrast20} utilizes contrastive learning across different views to acquire discriminative 3D scene representations. Innovations such as Point-BERT~\cite{PointBERT} and Point-MAE~\cite{PointMAE} introduce masked modeling~\cite{MAE,BERT} pretraining into the 3D domain. 
ACT~\cite{ACT23} pioneers cross-modal geometry understanding through 2D or language foundation models such as CLIP~\cite{CLIP} or BERT~\cite{BERT}. 
Following ACT, {\scshape ReCon}~\cite{ReCon23} further proposes a learning paradigm that unifies generative and contrastive learning. PPT~\cite{ppt24} highlights the significance of positional encoding in 3D representation learning
Additionally, leveraging foundation vision-language models like CLIP~\cite{ACT23,CLIP} has spurred the exploration of a new direction in open-world 3D representation learning. This line of work seeks to extend the applicability and adaptability of 3D representations in diverse and open-world/vocabulary scenarios~\cite{OpenScene23,CLIPFO3D23,PLA23,Lowis3D23,OVIR3D23,PointGCC23}.

\section{Additional Discussions}
\subsection{Relation to Affordance \& 6-DoF Pose Estimation}
Conceptually, this semantic orientation is a counterpart of \textit{affordance}~\citep{Affordance77,AffordanceHRI16,MoveWithAffordanceMaps20,HandsAsAffordancesProbes22} but beyond,
as SO and affordance all present potential actions and interactions with objects.
However, SO also contains the spatial understanding of intra-object part-level attributes more than affordance learning.
Compared to vanilla 6-DoF pose estimation, our proposed SO combined with the 3-DoF translation understanding has the same DoF completeness.
The difference is, our proposed SO is grounded by languages, making it useful for open-world manipulation requiring complicated spatial reasoning~\cite{RobotsThatUseLanguage20,SayCan22,Open6DOR24}. 
In addition, our Semantic Orientation can be auto-labeled from Internet 3D data that achieves higher scalability, introduced in the next section.
\begin{figure}[t!]
\begin{center}
% \includegraphics[width=0.85\linewidth]{figs/src/spatialvqa_statistic.pdf}
\includegraphics[width=\linewidth]{figs/src/spatialvqa_statistic.pdf}
\vspace{-15pt}
\caption{\textbf{6-DoF SpatialBench statistics}. (a) Statistical analysis of the task type, question type, and object relation. (b) Word cloud visualization.}
\label{fig:spatialvqa_statistic}
\end{center}
\end{figure}



\subsection{Comparison to Concurrent Works}
\begin{figure*}[t!]
\includegraphics[width=\linewidth]{figs/src/simpler_visual.pdf}
\vspace{-15pt}
\caption{An example of \ours~how to finish ``move near'' task in SIMPLER~\cite{simplerenv24}.}
\label{fig:simpler_visual}
\end{figure*}

\subsubsection{Comparison with ReKep~\cite{ReKep24}}
Recently, ReKep has succeeded in executing complex robotic tasks, such as long-horizon manipulation, based on the relationships and constraints between spatial key points. 
Its structural design offers many insights that \sofar~can draw upon, yet it also presents several issues: 
(1) Overly customized prompt engineering. ReKep requires manually designed complex system prompts for each task during inference. 
While this approach may be described as ``no training'', it cannot be considered a true zero-shot transfer. In contrast, \sofar~achieves genuine zero-shot transfer by eliminating the need for any human involvement during inference; (2) Using constraints based solely on key points fails to capture the full 6-DoF pose integrity of objects. For example, in the ``pouring water'' task, merely bringing the spout of the kettle close to the cup may lead to incorrect solutions, such as the kettle overturning; (3) ReKep requires all key points to be present in the first frame, and each step of the process—from mask extraction to feature dimensionality reduction, clustering, and filtering—introduces additional hyperparameters.

\subsubsection{Comparison with Orient Anything~\cite{orient_anything24}}
Recently, Orient Anything also highlighted the importance of orientation in spatial perception and adopted a training data construction approach similar to Our PointSO. Our primary distinction lies in semantic orientation, which is language-conditioned orientation. In contrast, Orient Anything is limited to learning basic directions such as ``front'' and ``top''. By aligning with textual information, semantic orientation better enhances spatial perception, understanding, and robotic manipulation.

\subsection{Future Works}
Future work includes further expanding the OrienText300K with larger datasets like Objaverse-XL~\cite{ObjaverseXL23}, enhancing the performance of semantic orientation through self-supervised learning and pretraining methods~\cite{MAE,CLIP,ACT23,ReCon23}, and demonstrating its effectiveness in a broader range of robotic scenarios, such as navigation~\cite{GOAT24}, mobile manipulation~\cite{homerobot23}, lifelong learning~\cite{LIBERO23}, spatio-temporal reasoning~\cite{ReKep24,LeaFLF23,CrossVideoSC24,thinking24}, humanoid~\cite{OmniH2O24,SmoothHumanoidLCP24,Exbody24,humanup25}, and human-robot interaction~\cite{HOI4D22,InteractiveHO23}.

\begin{table*}[t!]
\centering
\caption{\textbf{Statistics of Open6DOR V2 Benchmark.} The entire benchmark comprises three independent tracks, each featuring diverse tasks with careful annotations. The tasks are divided into different levels based on instruction categories, with statistics demonstrated above.}
\label{tab:open6dorv2_statistic}
\resizebox{\textwidth}{!}{
\setlength{\tabcolsep}{3.5pt}
    \begin{tabular}{c|ccccc|cc|c|c|c|c|c}
    \toprule
        Track & \multicolumn{7}{c|}{Position-track} & \multicolumn{3}{c|}{Rotation-track} & 6-DoF-track & Totel\\
        \midrule
        Level  & \multicolumn{5}{c|}{Level 0} & \multicolumn{2}{c|}{Level 1}  & Level 0 & Level 1 & Level 2 & - & -\\
        \midrule
        Task Catog. &  Left & Right  & Top &  Behind &  Front & Between  & Center & Geometric & Directional & Semantic & - & - \\
        \midrule
        Task Stat. & 296 & 266 & 209 & 297 & 278 & 193 & 159 & 318 & 367 & 134 & 1810 & 4535\\
        
        \midrule
        Benchmark Stat. &\multicolumn{7}{c|}{1698} & \multicolumn{3}{c|}{1027} & 1810 & 4535\\
    \bottomrule 
    \end{tabular}
}
\end{table*}

\section{Additional Visualizations}\label{app:visualization}

\subsection{Robotic Manipulation}
As shown in \cref{fig:simpler_visual}, we present a visualization of executing a task named ``move near''.
According to the input image and task instruction - ``\textit{move blue plastic bottle near pepsi can}'', \ours~can predict the center coordinate of the target object (bottle) and relative target (pepsi can), and it would infer the place coordinate and produce a series of grasp pose.

\subsection{6-DoF SpatialBench}
To further evaluate 6-DoF spatial understanding, we construct a 6-DoF SpatialBench.
We present examples of question-answer pairs from the 6-DoF SpatialBench, with quantitative and qualitative questions shown in \cref{fig:spatialbench_show1,fig:spatialbench_show2}, respectively. 
The benchmark we constructed is both challenging and practical, potentially involving calculations based on the laws of motion, such as ``\textit{Assuming a moving speed of 0.5 m/s, how many seconds would it take to walk from here to the white flower?}'' Moreover, it covers a wide range of spatially relevant scenarios across both indoor and outdoor environments.


\subsection{System Prompts}
Prompt engineering significantly enhances ChatGPT's capabilities. The model's understanding and reasoning abilities can be greatly improved by leveraging techniques such as Chain-of-Thought~\cite{CoT22} and In-Context Learning~\cite{GPT3_20}. \cref{fig:filter_prompt,fig:instruction_prompt} illustrate the system prompt we used in constructing OrienText300K.
\cref{fig:open6dor_prompt}, \cref{fig:manip_prompt}, and \cref{fig:vqa_prompt} illustrate the system prompt we used when evaluating \sofar on Open6DOR (simulation), object manipulation (both simulation and real worlds), and VQA, respectively.
Note that different from previous methods~\cite{VoxPoser23,ReKep24}, \sofar does not require complicated in-context examples.


\begin{figure*}[h!]
\centering
\includegraphics[width=0.97\linewidth]{figs/src/show1.pdf}
\captionof{figure}{\textbf{Visualization example of 6-DoF SpatialBench data samples.}
% 6-DoF SpatialBench includes complex spatial reasoning of absolute numbers.
}
\label{fig:spatialbench_show1}
\end{figure*}

\begin{figure*}[h!]
\centering
\includegraphics[width=0.97\linewidth]{figs/src/show2.pdf}
\captionof{figure}{\textbf{Visualization example of 6-DoF SpatialBench data samples.}}
\label{fig:spatialbench_show2}
\end{figure*}

\begin{figure*}[h!]
\centering
\includegraphics[width=0.97\linewidth]{figs/src/filter_prompt.pdf}
\captionof{figure}{\textbf{The system prompt of ChatGPT-4o used for filtering Objaverse data.}}
\label{fig:filter_prompt}
\end{figure*}

\begin{figure*}[h!]
\centering
\includegraphics[width=0.97\linewidth]{figs/src/instruction_prompt.pdf}
\captionof{figure}{\textbf{The system prompt of ChatGPT-4o used for generating Semantic Direction-Index pairs.}}
\label{fig:instruction_prompt}
\end{figure*}

\begin{figure*}[h!]
\centering
\includegraphics[width=0.97\linewidth]{figs/src/open6dor_prompt.pdf}
\captionof{figure}{\textbf{The system prompt of Open6DOR tasks.}}
\label{fig:open6dor_prompt}
\end{figure*}

\begin{figure*}[h!]
\centering
\includegraphics[width=0.97\linewidth]{figs/src/manip_prompt.pdf}
\captionof{figure}{\textbf{The system prompt of general manipulation tasks.}}
\label{fig:manip_prompt}
\end{figure*}

\begin{figure*}[h!]
\centering
\includegraphics[width=0.97\linewidth]{figs/src/vqa_prompt.pdf}
\captionof{figure}{\textbf{The system prompt of visual-question-answering tasks.}}
\label{fig:vqa_prompt}
\end{figure*}


\fi

\section*{Acknowledgments}
We thank Rom Pinchasi, Roy Meshulam, Dan Halperin, and Oren Salzman for fruitful discussions. We also thank Ido Jacobi, Roy Steinberg and Yaniv Hasidoff for comments on the manuscript. 

\bibliographystyle{plainnat}
\bibliography{references}

\end{document}
%%% Local Variables:
%%% mode: latex
%%% TeX-master: t
%%% End:
