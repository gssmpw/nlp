\section{Lattice-based sample sets}\label{sec:lattices}
We derive sample sets optimizing sample complexity (Problem~\ref{problem:sample}) and collision-check complexity (Problem~\ref{problem:collision}) both in theory and experiments. We focus on sample sets induced by lattices. %In this section, we review basic definitions of lattices, describe three families of lattices that are of interest, and derive \decomp sample sets using them.

A lattice is a point set in Euclidean space with a regular structure~\cite{conway2013sphere}.

\begin{definition}[Lattice]\label{definition:lattice}
  A lattice $\Lambda$ is defined as all the linear combinations (with integer coefficients) of a basis\footnote{A basis can be of full rank ($m=N$) or subdimensional ($m<N$). A basis can be non-unique.} $E_\Lambda=\{e_i\in \dR^N\}_{i=1}^m$ of rank $1\leq m\leq N$, i.e.,
  \[\Lambda:=\left\{\sum_{i=1}^m a_i e_i\middle| a_i\in\mathbb{Z},e_i\in E_\Lambda\right\}.\]
\end{definition}

It would be convenient to view lattices through their generator matrices. 
\begin{definition}[Lattice generator]\label{definition:generator}
    The generator matrix $G_\Lambda$ of a lattice $\Lambda$ with basis $E_\Lambda=\{e_i\in \dR^N\}_{i=1}^m$ is an $m\times N$ matrix such that for every  $1\leq i\leq m$, the row $i$ is equal to $e_i$. Note that 
\(\Lambda=\left\{a\cdot G_\Lambda\middle| a\in\dZ^{1\times m}\right\}.\)
Additionally, define $\det(\Lambda):=\det(G_{\Lambda}G_{\Lambda}^t)$.
\end{definition}

\subsection{Useful lattices}
We describe three lattices, visualized in~\Cref{fig:2d_lattices,fig:3d_lattices}. The first lattice is a simple rectangular grid, which is provided to benchmark more complicated and efficient lattices. Below, we fix the dimension $d\geq 2$.

\begin{definition}[$\dZ^d$ lattice]
   The $\dZ^d$ lattice is defined by the identity generator matrix $I\in \dR^{d\times d}$, with $\det(\ZN)=1$~\cite[p.~106]{conway2013sphere}.
\end{definition}
  


% \begin{figure}[H]
%     \centering
%     \begin{subfigure}[b]{0.49\textwidth}
%         \includegraphics[width=\textwidth]{Images/ZN_2D.png}
%         %\caption{$Z^2$: A normal grid.}
%         %\label{fig:sub1}
%     \end{subfigure}
%     \hfill
%     \begin{subfigure}[b]{0.49\textwidth}
%         \includegraphics[width=\textwidth]{Images/ZN_3D.png}
%         %\caption{$Z^3$: A normal grid.}
%         %\label{fig:sub2}
%     \end{subfigure}
%     \caption{The lattice $\ZN$ in 2,3 dimensions, respectively. A standard "cube" ($[0,w]^d,d=2,3$ for some $w>0$) can be seen tessellating the space. \kiril{Are we referring to this figure when describing the cube? In any case, please explain here what you mean by the cube. Some readers would read the caption of this figure before arriving at the definition of a cube later on.}\itai{added explanation for what I meant}}
%     \label{fig:q2s3zeropointthree}
% \end{figure}

More efficient sample sets can be generated via the $D_d^*$ lattice~\cite[p120]{conway2013sphere}. This lattice was also presented in~\cite{dayan2023near}, where it was called a ``staggered grid''. In this work, we provide improved sample complexity bounds for lattice-based sample sets (including for the $D_d^*$ lattice %, which was analyzed in~\cite{dayan2023near}, 
and the $A_d^*$ lattice defined later on), following~\cite{conway2013sphere}, and develop theoretical bounds for collision-check complexity.

\begin{definition}[$D_d^*$ lattice]
   The $D_d^*$ lattice is defined by the  generator matrix
   \begin{align*}
     G_{\DN}=
     \begin{pmatrix}
         1 & 0 & 0 & \dots & 0 & 0 \\
         0 & 1 & 0 & \dots & 0 & 0 \\
         \vdots & \vdots & \vdots & \ddots & \vdots & 0 \\
         0 & 0 & 0 & \dots & 1 & 0 \\
         \frac{1}{2} & \frac{1}{2} & \frac{1}{2} & \dots & \frac{1}{2} & \frac{1}{2}
     \end{pmatrix}\in \dR^{d\times d}, 
 \end{align*}
 with $\det(\DN)=\frac{1}{4}$~\cite[p.~120]{conway2013sphere}.
 \end{definition}
 
% \begin{figure}[H]
%     \centering
%     \begin{subfigure}[b]{0.49\textwidth}
%         \includegraphics[width=\textwidth]{Images/DN_3D.png}
%        % \caption{$D_2^*$: A.K.A the \emph{staggered grid}.}
%         %\label{fig:sub1}
%     \end{subfigure}
%     \hfill
%     \begin{subfigure}[b]{0.49\textwidth}
%         \includegraphics[width=\textwidth]{Images/DN_3D.png}
%         %\caption{$D_3^*$: the \emph{BCC} structure.}
%         %\label{fig:sub2}
%     \end{subfigure}
%     \caption{The $\DN$ lattice in 2,3 dimensions, respectively. Looks similar to $\ZN$, with A "cube" tessellating the space. Here, though, another sample is added in the middle - covering the space more efficiently. \kiril{same comment as earlier. What are you trying to illustrate with those cubes here? This is a good place to mention that in 2d the this lattice can be viewed as a rotated $\dZ^d$ lattice, but in 3d they diverge, and refer to the Dayan text.}\itai{clarified what I meant also here}}
%     \label{fig:q2s3zeropointthree}
% \end{figure}

\begin{figure}[t]
  \centering
  \subfloat[$\X_{\dZ_3}^{\delta,\epsilon}$ sample set.]{
    \includegraphics[width=0.46\columnwidth,clip]{Images/ZN_3D.png}
    %\label{fig:3d_lattices:z}
    }
  \hfill
  \subfloat[$\X_{D_3^*}^{\delta,\epsilon}=\X_{A_3^*}^{\delta,\epsilon}$ sample sets.]{
    \includegraphics[width=0.46\columnwidth,clip]{Images/AN_3D.png}
    %\label{fig:3d_lattices:da}
    }
  \caption{\decomp sample sets in $\dR^3$ derived from the lattices $\dZ^3, D_3^*$ and $A^*_3$. Note the sets $\X_{D_d^*}^{\delta,\epsilon},\X_{A_d^*}^{\delta,\epsilon}$ coincide for $d=3$, and diverge for $d\geq 4$.
Note that the density of  $\X_{D^*_3}^{\delta,\eps}$ and $\X_{A^*_3}^{\delta,\eps}$ (also known as the Body-Centered Cubic structure in crystallography), and is lower than the density of $\X_{\dZ^3}^{\delta,\eps}$.}
  \label{fig:3d_lattices}
\end{figure}

The following $A^*_d$ lattice~\cite[p115]{conway2013sphere} leads to even more efficient sample sets. This lattice is also called a "hexagonal grid", and was previously used for 2D path planning~\cite{BAILEY2021103560,TengEA17}. This work is the first to consider its application in dimensions $d\geq 3$, and moreover, in the context of \decomps guarantees.

%\yaniv{Was used in low dimensions e.g. https://www.mdpi.com/2220-9964/13/5/166 https://www.mdpi.com/2220-9964/11/4/231 https://www.mdpi.com/2072-4292/13/21/4216 as well as "Path-length analysis for grid-based path planning" and "hexagonal grid-based sampling planner for aquatic environmental monitoring using unmanned surface vehicles" } 

\begin{definition}[$A^*_d$ lattice]
  The $A^*_d$ lattice is defined through the generator matrix \begin{align*}
    G_{\AN}=
    \begin{pmatrix}
        1 & -1 & 0 & 0 & \dots & 0 & 0 \\
        1 & 0 & -1 & 0 & \dots & 0 & 0 \\
        \vdots & \vdots & \vdots & \vdots & \ddots & \vdots & 0 \\
        1 & 0 & 0 & 0 & \dots & -1 & 0 \\
        \frac{-d}{d+1} & \frac{1}{d+1} & \frac{1}{d+1} & \frac{1}{d+1} & \dots & \frac{1}{d+1} & \frac{1}{d+1}
    \end{pmatrix}_{d\times (d+1)}\!\!\!,
\end{align*}
with $\det(\AN)=\frac{1}{\sqrt{d+1}}$~\cite[p.~115]{conway2013sphere}.
\end{definition}

Note that $A^*_d$ is contained in $\mathbb{R}^{d+1}$ (due to the number of rows of the generator matrix), but the lattice itself is $d$-dimensional as it lies in a $d$-dimensional hyperplane (for any lattice point $(x_1,\ldots x_{d+1})\in A^*_d$ it holds that  $\sum_{i=1}^{d+1} x_i=0$). 
Our motivation for considering $\AN$ is its low \emph{density}, defined as the average number of spheres (centered on lattice points)  containing a point of the space~\cite{conway2013sphere}. In particular, $\AN$ is the best lattice covering (and best covering in general) in terms of density for dimension $d\leq 5$ (see~\cite{ryshkov1975solution}) and overall the best \emph{known} covering for $d\leq 21$. %\itai{here is the full quote: "An* is the best covering of Rn for n=2, the best lattice covering for n<=5, and the best covering known for n<=21". This is what the book says. from here, its up to us to interpret it, honestly. I just take it at face value: they proved mathematically it is the best lattice covering for n<=5, and for n<=21 they just aren't aware (at the time of writing the book) of anything better)}

\subsection{From lattices to \decomp sample sets}
We derive sample sets from the lattices above by transforming the lattices such that the resulting point sets lie in $\dR^d$ and form $\beta^*$-covers (Lemma~\ref{lem:cover}). To achieve that, we will leverage the geometry of the lattices and their covering radius, which is defined below.

\begin{definition} (Covering radius~\cite{conway2013sphere})
   For a point set $\X\subset \dR^d$, a covering radius is defined to be
    \[
        f_{\X} = \sup_{y\in\dR^d}\inf_{x\in\X} \|x-y\|.
    \]
\end{definition}
When considering the covering radius of $\AN$, which lies in $\dR^{d+1}$, we will abuse the above definition to refer to its covering radius in the $d$-dimensional plane $\sum_{i=1}^{d+1}x_{d+1}=0$. Note that in order to cover $\dR^d$ with balls of radius $\rho>0$ centered at the points of a set $\X$, it must hold that $\rho\geq f_{\X}$.

%In the next theorem, we will be using the covering radii of the lattices defined in the previous section to derive \decomp sample sets. In particular, we rescale $\dZ^d$ and $\DN$ to yield the required sample sets. The third sample set is obtained from $\AN$ after applying transformations, that are elaborated in the proof. 
% \begin{lemma}[Optimal covering radius]
%     Let $\Lambda$ be a lattice in $\dR^d$. Then $Cover\left(\Lambda\right)$ is the optimal \left(in terms of density\right) covering radius for this $d$-dimensional lattice. Relying on Conway~\cite{conway2013sphere}, we have:
%     \itai{I'm correcting this. the optimal radius is related to the original lattice, not the rescaled one.}
%     \begin{enumerate}[topsep=1pt,itemsep=1ex,partopsep=1ex,parsep=1ex]
%         \item $Cover\left(\AN\right) = \sqrt{\frac{d}{2}}$
%         \item $Cover\left(\DN\right) = \frac{\sqrt{d}}{2}$
%         \item $Cover\left(\ZN\right) = \sqrt{\frac{d\left(d+2\right)}{12\left(d+1}}$
%     \end{enumerate}
% \end{lemma}
% \itai{changes here.}
% Using the concept of covering radii for lattices, we will explicitly find sample sets defined by our three lattices:
% \begin{definition}[Lattice Sample Set]
%     Let $\Lambda$ be a lattice, and $\delta,\epsilon$ be the clearance and stretching parameters. We define $\XL$ to be the set $\left(c\cdot\Lambda\right)$ for some $c>0$ such that it's a \decomp set.
% \end{definition}
% Specifically for \Lattices we get the following theorem:

\begin{thm} \label{thm:decomp_lattices}
  Fix $\delta>0,\epsilon>0$, and take $\beta^*$ as defined in \Cref{lem:cover}. Then the following sample sets are \decomp:
  \begin{enumerate}
      \item $\XZ:=\left\{\frac{2\beta^*}{\sqrt{d}}\cdot v,v\in\mathbb{Z}^d\right \}=\frac{2\beta^*}{\sqrt{d}}\cdot \ZN$.
      \item $\XD:=\\
      \begin{cases}
        d\text{ is odd: } \left\{\frac{4\beta^*}{\sqrt{2d-1}}G_{\DN}^t\cdot v,v\in\mathbb{Z}^d\right \}=\frac{4\beta^*}{\sqrt{2d-1}}\cdot \DN,\\
        d\text{ is even: }
        \left\{\sqrt{\frac{8}{d}}\beta^* G_{\DN}^t\cdot v,v\in\mathbb{Z}^d\right\}=\sqrt{\frac{8}{d}}\beta^*\cdot \DN.
      \end{cases}$
      \item $\XA:=\left\{\sqrt{\frac{12\left(d+1\right)}{d\left(d+2\right)}}\beta^* T\cdot v,v\in\mathbb{Z}^d\right\}$, %=\sqrt{\frac{12\left(d+1\right)}{d\left(d+2\right)}}\beta T(\ZN)$    
       where   \\$T:=\begin{pmatrix}
                    1 &  1  & \dots & 1 & a - 1\\
                    -1 & 0  & \dots & 0 & a \\
                    0 & -1  & \dots & 0 & a \\
                    \vdots & \vdots  &  \ddots & \vdots & \vdots \\
                    0 & 0  &  \dots & -1 & a \\
                \end{pmatrix}\in \dR^{d \times d}$ and $a=\frac{1}{d+1 - \sqrt{d+1}}$. % and  $T(\ZN):=\{Tv,v\in\dZ^d\}$.
  \end{enumerate}
 % \kiril{To make this notation consistent with Definition~\ref{definition:generator}, shouldn't we write $aG_\Lambda$, instead of $a G_{\Lambda}$?}
\end{thm}

Each of the new sample sets can be viewed as a lattice in $\dR^d$, according to Definition~\ref{definition:lattice}. For instance, $\XA$ is a lattice with the generator matrix $\sqrt{\frac{12\left(d+1\right)}{d\left(d+2\right)}}\beta^* T^t$. Also, note that the result above for $\XD$ is a tightening 
of the result in~\cite{dayan2023near} for odd dimensions.

\begin{proof}
    %We will rescale the lattices \Lattices to obtain \decomp sample sets.
    The lattices \Lattices require a transformation to achieve $(\de)$-completeness for a given value of $\delta$ and $\epsilon$. For a given lattice $\Lambda$, we compute a rescaling factor $w_\Lambda>0$ such that the covering radius of the lattice $w_\Lambda \Lambda$ is not bigger than $\beta^*$. This would imply that $w_\Lambda \Lambda$ is \decomp according to Lemma~\ref{lem:cover}. In particular, $w_\Lambda = \beta^*f_\Lambda^{-1}$ where $f_\Lambda$ is the covering radius of $\Lambda$. Next, we consider each of the three lattices individually.     
% in the context of using unscaled distances---i.e., in a cube $[0,1]^d$ we have exactly one cube for the lattice $\mathbb{Z}^d$, in contrast with many more cubes if we start rescaling the grid. Since we want the samples to be sufficiently dense as to form a \decomp set, we will need to rescale the space. This is why in each of the following sections, given our lattice $\Lambda$, we use a \emph{rescale coefficient} $f_\Lambda:=f_\Lambda\left(d\right)$, together with a \emph{rescaling factor} $w_\Lambda>0$, defined by:
%    \[
%        \beta = f_\Lambda w_\Lambda\Rightarrow w_\Lambda = \frac{\beta}{f_\Lambda}.
%    \]
%    Meaning: we start with the covering radius (i.e. the rescale coefficient $f_\Lambda$), and we rescale it (i.e. multiply it with the rescaling factor) to "the smallest size possible" (i.e. the $\beta$-ball that covers the space), which is sufficient for $\left(\delta,\epsilon\right)$-completeness due to Lemma~\ref{lem:cover}.
    
    % One thing to note is, that using our scaling equation, our space went from some implicit "originating" space of radius $R_{orig}$ (where, as we said, the lattices are still unscaled)---which becomes $R$ with the relation:
    % \begin{align*}
    %     R_{\text{orig}}\cdot w_\Lambda = R \Rightarrow \frac{R_{\text{orig}}\cdot\beta}{f_\Lambda}=R\Rightarrow R_{\text{orig}}=\frac{R}{\beta}\cdot f_\Lambda.
    % \end{align*}
    % We will find $f_\Lambda$ for each of our three lattices, which will immediately give us our \decomp sets. \kiril{Note to self: rewrite this paragraph.}

\begin{figure}[!h]
  \centering
\includegraphics[width=0.7\columnwidth]{Images/EPGt_visualization.png}
  \caption{Visualization of embedding the lattice  $A_2^*$ originally defined in $\dR^3$ onto $\dR^2$ via the mapping $T$. The blue rectangle represents the plane $H$, where the corresponding $A_2^*$ lattice points are drawn in red. The points are generated by taking integer vectors in $\dR^d$ and applying the mapping $G^t$.  $H$ and $A_2^*$ is reflected onto the plane $H_0=\{x_3=0\}$ using the mapping $PG^t$ (denoted by the green rectangle). The third dimension is removed via the mapping $E$ to yield the embedding of $A_2^*$ in $\dR^2$.}
  \label{fig:egpt_visual}
\end{figure}

\vspace{5pt}
\noindent\emph{The lattice $\dZ^d$.} This lattice can be viewed as a set of axis-aligned unit hypercubes whose vertices are the lattice points. The center point of a cube is located at a distance of $\sqrt{d}/2$ from the cube's vertices, which yields the covering radius $f_{\ZN} =\sqrt{d}/2$, and the rescaling factor $w_{\ZN} = {\beta^*}/f_{\ZN}=2{\beta^*}/\sqrt{d}$. As a result, the sample set $\XZ:=\frac{2{\beta^*}}{\sqrt{d}}\ZN$, is a ${\beta^*}$-cover, and \decomp due to Lemma~\ref{lem:cover}. 

% Seeing as the cube has a side of length $w_{\ZN}$, this gives us the scaling equation:
%     \[
%         \beta = \frac{\sqrt{d}}{2}w_{\ZN} \Rightarrow f_{\ZN}\left(d\right)=\frac{\sqrt{d}}{2},
%     \]
%     from which we get our set definition:
%     \begin{align*}
%         \XZ = \{x\in\dR^d|x=w_{\ZN} v\cdot G_{\ZN},v\in\mathbb{Z}^{1\times d}\} = \\
%         \{x\in\dR^d|x=\frac{\beta}{f_{\ZN}\left(d\right)}I_{d\times d}\cdot v,v\in\mathbb{Z}^d\} = \\
%         \{x\in\dR^d|x=\frac{2\beta}{\sqrt{d}} v,v\in\mathbb{Z}^d\}:=\frac{2\beta}{\sqrt{d}}\ZN.
%     \end{align*}
    
\vspace{5pt}
\noindent\emph{The lattice $\DN$.}
We use the covering radius of $\DN$, which depends on whether the dimension $d$ is odd or even~\cite[page 120]{conway2013sphere}. In particular, 
$f_{D_{d\text{-odd}}^*}=\frac{\sqrt{2d-1}}{4}$, and $f_{D_{d\text{-even}}^*}=\frac{\sqrt{2d}}{4}$. This immediately implies the definition of $\XD$.
 %      \begin{cases}
%         \frac{4\beta}{\sqrt{2d-1}}\cdot \DN,\quad  \textup{if } d \textup{ is odd,}\\
%         \sqrt{\frac{8}{d}}\beta\cdot \DN, \quad
%         \textup{otherwise.}
%       \end{cases}
% \]

% Using this radius as a baseline for what we want our $\beta$-ball to be, we get the following rescaling equations:
%     \begin{align*}
%                 \beta = \frac{\sqrt{2d-1}}{4}w \Rightarrow f_{D_{d\text{-odd}}^*}\left(d\right) = \frac{\sqrt{2d-1}}{4}, \\
%                 \beta = \sqrt{\frac{d}{8}}w \Rightarrow f_{D_{d\text{-even}}^*}\left(d\right) = \sqrt{\frac{d}{8}},
%     \end{align*}
%     from which we get our set definitions:
%     \begin{align*}
%         D_{d\text{-odd}} = \{x\in\dR^d|x=wv\cdot G_{\DN},v\in\mathbb{Z}^{1\times d}\} = \\
%         \{x\in\dR^d|x=\frac{4\beta}{\sqrt{2d-1}}G_{\DN}^t\cdot v,v\in\mathbb{Z}^d\}:=\\
%         \frac{4\beta}{\sqrt{2d-1}}\DN,\\
%         D_{d\text{-even}} = \{x\in\dR^d|x=wv\cdot G_{\DN},v\in\mathbb{Z}^{1\times d}\} = \\
%         \{x\in\dR^d|x=\sqrt{\frac{8}{d}}\beta G_{\DN}^t\cdot v,v\in\mathbb{Z}^d\}:=\\
%         \sqrt{\frac{8}{d}}\beta\DN.
%     \end{align*}
%     Finally, we analyze the lattice $\AN$.
    % Using a lattice's generator, we can define the lattice points generated from it as:
    % \[
    %     \Lambda=\{x \in \dR^{d} \mid \exists g\in \mathbb{Z}^{d} \text{ s.t. } x=g\cdot G\}
    % \]

\vspace{5pt}
\noindent\emph{The lattice $\AN$.} 
    Recall that $A_d^*$ has the generator matrix $G:=G_{\AN}\in \dR^{d\times (d+1)}$.
    As this is a mapping from $d$ to $d+1$, we start with the process of embedding $A_d^*$ in $\dR^d$ (see Figure~\ref{fig:egpt_visual}). Afterward, we show that the embedding in $\dR^d$ shares the same covering radius as the original set in $\dR^{d+1}$.

    Any row $i$ of $G$, denoted by  $G_i:=(g_{i1},g_{i2},\ldots,g_{i(d+1)})$, lies on the (hyper)plane $H: \{\sum_{j=1}^{d+1} g_{ij} = 0\}$. Thus, $A_d^*$ itself is contained in that $d$-dimensional plane. It remains to find a transformation of $\AN$ such that the dimension is reduced to $d$ while maintaining the structure of the points in $\AN$.

In the first step, we reflect $\AN$ lattice points onto the plane $H_0:=\{x_{d+1}=0\}$ using the Householder matrix 
\begin{align}\label{eq:reflection}
 P:=\begin{pNiceArray}{cw{c}{1cm}c|c}[margin]
            \Block{3-3}<\Large>{I_d - \frac{1}{D-\sqrt{D}}\mathds{1}}
            & & & \tfrac{1}{\sqrt{D}} \\
            & & & \Vdots \\
            & & & \tfrac{1}{\sqrt{D}} \\
            \hline
            \tfrac{1}{\sqrt{D}} & \dots& \tfrac{1}{\sqrt{D}} & \tfrac{1}{\sqrt{D}}
        \end{pNiceArray},
    \end{align}
where $D:=d+1$, $I_d$ is an $d\times d$ identity matrix, and $\mathds{1}$ is the $d\times d$ matrix with $1$s in all its entries.  That is, we reflect a lattice point $p\in \dR^{d+1}$ by computing the value ${v}=P\cdot p$. (See the derivation of $P$ in \conditionaltext{\Cref{app:decomp_lattice_proof}}{ the supplementary material}.)

It remains to eliminate the $(d+1)$th dimension of the points $v:=P\cdot p$. This is accomplished by the mapping
\begin{align*}
        E=
        \begin{pmatrix}
            1 & 0 & \dots & 0 & 0 \\
            0 & 1 & \dots & 0 & 0 \\
            \vdots & \vdots & \ddots & \vdots & 0 \\
            0 & 0 & \dots & 1 & 0
        \end{pmatrix}_{d\times(d+1)}.
    \end{align*}
    
We finish by computing the an explicit mapping %$\mathbb{Z}^{d}\rightarrow\dR^d$ 
that yields the embedding of $A_d^*$ to $\dR^d$. In particular, we have the embedding $T(g):=EPG^t(g)$, for $g\in \dZ^d$, where $T$ is as specified in the statement of this theorem. (See the derivation of $T$ in  \conditionaltext{\Cref{app:decomp_lattice_proof}}{ in the supplementary material}.)

    %      \begin{figure}[H]
    %     \centering
    %     \begin{subfigure}[b]{0.3\textwidth}
    %         \includegraphics[width=\textwidth]{Images/EPGt_visual_explanation1.png}
    %         %\caption{Sample points in $H_0$}
    %         %\label{fig:epgt_visual1}
    %     \end{subfigure}
    %     \hfill
    %     \begin{subfigure}[b]{0.3\textwidth}
    %         \includegraphics[width=\textwidth]{Images/EPGt_visual_explanation2.png}
    %         %\caption{$H_0$ rotated to the "floor"}
    %         %\label{fig:epgt_visual2}
    %     \end{subfigure}
    %     \hfill
    %     \begin{subfigure}[b]{0.3\textwidth}
    %         \includegraphics[width=\textwidth]{Images/EPGt_visual_explanation3.png}
    %         %\caption{The samples as they look in $\dR^2$}
    %         %\label{fig:epgt_visual3}
    %     \end{subfigure}
    %     \caption{Visualization of embedding the lattice  $A_2^*$ originally defined in $\dR^3$ onto $\dR^2$ via the mapping $T$. [Left] The blue rectangle represents the plane $H$, where the corresponding $A_2^*$ lattice points are drawn in red. The points are generated by taking integer vectors in $\dR^d$ and applying the mapping $G^t$.  [Center] $H$ and $A_2^*$ is reflected onto the plane $H_0=\{x_3=0\}$ using the mapping $PG^t$. [Right] The third dimension is removed, via the mapping $E$, to yield the embedding of $A_2^*$ in $\dR^2$.}
    %     \label{fig:egpt_visual}
    % \end{figure}


For the final part, we wish to derive an \decomp sample set $\XA$. Here, we first recall that the covering radius $f_{\AN}$ is equal to $\sqrt{\frac{d\left(d+2\right)}{12\left(d+1\right)}}$~\cite[page 115]{conway2013sphere}. Considering that reflections and embeddings are isometries, i.e.,  they preserve distances between pairs of points, we can use the same covering radius after mapping the points of $\AN$ using $T$. Thus, we obtain the rescaling coefficient $w_{\AN}:={\beta^*} f_{\AN}^{-1}$, which implies that $\XA:=
         \{w_{\AN}T\cdot v,v\in\mathbb{Z}^d\}$ is \decomp.
% \begin{align*}\XA:&=
%          \left\{w_{\AN}T\cdot v,v\in\mathbb{Z}^d\right\}\\
%          &= 
%         \left\{\sqrt{\frac{12\left(d+1\right)}{d\left(d+2\right)}}\beta T\cdot v,v\in\mathbb{Z}^d\right\}. %=\sqrt{\frac{12\left(d+1\right)}{d\left(d+2\right)}}\beta T(\ZN).
%     \end{align*}    
    % This is what the transformation does: first (Fig.~\ref{fig:epgt_visual1}), $G^t$ takes an integer vector in $\dR^d$ and turns it to a vector in $\AN\subset\dR^{d+1}$. Next (Fig.~\ref{fig:epgt_visual2}), $P$ takes the element in $\AN$ and reflects it into $[X_1,\dots,X_d]$---effectively lowering its dimension. Lastly (Fig.~\ref{fig:epgt_visual3}), because we are still in $\dR^{d+1}$, we use $E$ to embed it into $\dR^d$.

    %     \beta = \sqrt{\frac{d\left(d+2\right)}{12\left(d+1\right)}}w_{\AN} \Rightarrow f_{\AN}\left(d\right)=\sqrt{\frac{d\left(d+2\right)}{12\left(d+1\right)}},
    % \]
    % from which we get our set definition:
\end{proof}

%%% Local Variables:
%%% mode: latex
%%% TeX-master: "../main"
%%% End: