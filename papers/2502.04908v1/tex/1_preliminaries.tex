\section{Preliminaries}\label{sec:preliminaries}
%We provide basic definitions regarding motion planning, sampling-based planning, and $(\delta,\epsilon)$-completeness. 
The motion-planning problem concerns computing a collision-free path for a robot in an environment cluttered with obstacles. We concider a holonomic robot with a configuration space $\C = \dR^d$. %(we assume that $\C$ is of full dimensions, i.e., not a manifold). For simplicity, we set $\C=\dR^d$ throughout the paper, as our lattice-based sample sets are infinite. 
The dimension $d\geq 2$ represents the DoF and is finite. A motion planning problem is a tuple $\M:= (\C_f, q_{s},  q_{g})$, where $\C_f\subseteq \C$ is the free space (the set of collision-free configurations), and $q_s,q_g\in \C_f$ are the start and goal configurations, respectively. A solution for $\M$ is a continuous collision-free path $\pi:[0,1]\to\C_f$ that begins at $\pi(0) = q_{s}$ and ends at $\pi(1) = q_g$.

Two critical properties of a given path $\pi$ for a problem $\M= (\C_f, q_{s},  q_{g})$, are its length $\ell(\pi)\geq 0$, and its clearance. For a given value $\delta\geq 0$, we say that the path $\pi$ is $\delta$-clear if $\bigcup_{0 \leq t \leq 1}\B_{\delta}(\pi(t)) \subseteq \C_f$, where $\B_\rho(p)$ is the $d$-dimensional closed Euclidean ball with radius $\rho>0$ centered at $p\in\mathbb{R}^d$. We denote $\B_\rho:=\B_\rho(o)$, where $o$ is the origin of $\dR^d$.

\subsection{Probabilistic roadmaps and completeness}
We present a formal definition of the Probabilistic Roadmap (PRM) method \cite{kavraki1996probabilistic}, which constructs a discrete graph capturing the connectivity of $\C_f$ through sampling. Albeit sampling usually refers to a randomized process, here we consider deterministic sampling, as was recently done in~\cite{tsao2020sample,dayan2023near}.\footnote{With deterministic sampling, the term ``probabilistic'' in PRM can seem misleading. Nevertheless, we choose to stick to PRM considering its popularity and the underlying graph structure it represents, which is a key to our analysis.} We emphasize that our analysis below is not confined to PRMs, and applies to various PRM-based planners, as mentioned above. 

For a given motion planning problem $\M = (\C_f,q_s,q_g)$, a sample (point) set $\X\subset \C$, and a connection radius $r>0$, PRM generates a graph denoted by $G_{\mathcal{M}(\X, r)} = (V,E)$. The vertex set $V$ consists of all collision-free configurations in $\X \cup \{ q_s, q_g\}$. The set of undirected edges, $E$, consists of all the vertex pairs $u, v\in V$ such that the Euclidean distance between them is at most $r$, and the straight-line segment $\overline{uv}$ between them is collision-free. That is, 
\begin{align*}
V := & (\X\cup \{ q_s, q_g\}) \cap \C_f, \\
E := & \left\{ \{v, u\} \in V\times V : \lVert v-u\rVert\leq r, \overline{uv} \subset \C_f \right\}.
\end{align*}

In this work, we are interested in obtaining sample sets and connection radii for PRM that achieve a desired solution quality in terms of path length. Unlike most theoretical results for SBP, which consider asymptotic guarantees, here we rely on a stronger deterministic notion.

\begin{definition}[($\delta,\epsilon$)-completeness~\cite{tsao2020sample}] Given a sample set $\X\subset \C$ and connection radius $r>0$, the pair $(\X, r)$ is ($\delta,\epsilon$)-complete for a clearance parameter $\delta>0$ and stretch factor $\epsilon>0$, if for \emph{every} $\delta$-clear problem $\mathcal{M}=(\C_f, q_s, q_g)$, the  graph $G_{\mathcal{M}(\X, r)}$ contains a path from $q_s$ to $q_g$ with length at most $(1+\eps)$ times the optimal $\delta$-clear length, denoted by $\text{OPT}_\delta$. That is, it holds that
\[
    \ell(G_{\mathcal{M}(\X, r)},q_s, q_g)\leq (1+\epsilon)\text{OPT}_\delta,
\]
where $\ell(G_{\mathcal{M}(\X, r)},q_s, q_g)$ denotes the length of the shortest path from $q_s$ to $q_g$ in the graph $G_{\mathcal{M}(\X, r)}$.
\end{definition}

The property of ($\delta,\epsilon$)-completeness has several key advantages over asymptotic notions, such as PC and AO. First, there exists (with probability $1$) a \emph{finite} sample set $\X$ and radius $r\in (0,\infty)$  that jointly guarantee ($\delta,\epsilon$)-completeness. Second, if a solution is not found using a \decomp pair $(\X,r)$, then no $\delta$-clear solution exists. That is, ($\delta,\epsilon$)-completeness can be used for deterministic infeasibility proofs~\cite{li2023sampling}. Third, the computational complexity of constructing a PRM graph can be tuned according to the desired values of $\delta$ and $\epsilon$. %In contrast, PC and AO properties provide no meaningful guarantees for a finite number of samples. Thus, if no solution is found using randomized samples, it can be difficult to determine whether no solution exists. Lastly, it is usually more difficult to tune randomized samples according to user specifications of desired solution quality and clearance. 



%An important thing to note, is that we run into a problem many algorithms run into: will this algorithm end with a result? if so, in how much time? General MP problems are able to solve this problem using a recently developed infeasibility-checking parallel-running algorithm (see Li and Dantam, 2023~\cite{li2023sampling}). Using learning and topological methods they are able to get an answer to the question: "Will we ever get a path from $q_s$ to $q_g$?".


%The problem is, that this answer comes at an asymptotically-assured time, not in a finite time. In our $(\delta,\epsilon)$-complete setting, if we build a sample set in a way that is guaranteed to possess this property---we are guaranteed to get a $\delta$-clear path, if one exists, in finite time. If we didn't get a path, it means that such a path doesn't exist. This serves as a sort of "infeasibility proof" for sample sets in the $(\delta,\epsilon)$-complete setting.


\subsection{Problem definition}
A $(\delta,\epsilon)$-complete sample set $\X$ and radius $r>0$ can be obtained by constructing a so-called $\beta$-cover~\cite{dayan2023near}. 

\begin{definition}
For a given $\beta>0$, a sample set $\X\subset \dR^d$ is a $\beta$-cover if for every point $p\in \C=\dR^d$ there exists a sample $x \in \X$ such that $\|p-x\|\leq \beta$.
\end{definition}

Note that the coverage property above is defined with respect to the whole configuration space, rather than a specific free space. The connection between $(\delta,\epsilon)$-completeness and $\beta$-cover is established in the following lemma.
\begin{lemma}[Completeness-cover relation~\cite{tsao2020sample}]\label{lem:cover}
  Fix $\delta >0$ and $\epsilon>0$. Suppose that a sample set $\X$ is a ${\beta^*}$-cover, where
  \begin{equation}
    {\beta^*}={\beta^*}(\delta,\epsilon):=\frac{\delta\epsilon}{\sqrt{1+\epsilon^2}}.
  \end{equation}
  Then $(\X,{r^*})$ is $(\delta,\epsilon)$-complete, where
  \begin{equation}
      {r^*}={r^*}(\delta,\epsilon):=\frac{2\delta(1+\epsilon)}{\sqrt{1+\epsilon^2}}.
  \end{equation}
\end{lemma}

%In the above lemma, the ${\beta^*}$-cover is with respect to the full configuration space $\C$, rather than a specific free space $\C_f$. This is because the $(\delta,\epsilon)$-completeness needs to be satisfied for any free space. 
The lemma prescribes an approach for constructing sample sets and connection radii that satisfy the $(\delta,\epsilon)$-completeness requirement. %When $\epsilon\rightarrow\infty$ (i.e. "any solution"), ${\beta^*}=\delta$ and $r=2\delta$---which makes sense, we want a $\delta$-clear path and a $2\delta$ connection radius ensures the PC property.
Considering that we will fix the radius $r^*:=r^*(\delta,\epsilon)$ throughout this work, we will say that a sample set $\X$ is $(\delta,\epsilon)$-complete if the pair $(\X,r^*)$ is $(\delta,\epsilon)$-complete.

\Cref{lem:cover} still leaves a critical unresolved question: How do we find a \decomp sample $\X$ minimizing the computation time of the PRM graph and subsequent methods? In this work, we are interested in the following two related problems. The first deals with finding a sample set of minimal size, which can be used as a proxy for computation time. 

\begin{problem}[Sample complexity]\label{problem:sample}
  For a given $\delta>0,\epsilon>0$, find a $(\delta,\epsilon)$-complete sample set $\X$ of  minimal \emph{sample complexity}, i.e., $|\X\cap \B_{r^*}|$. 
\end{problem}

Previous work~\cite{tsao2020sample,dayan2023near} has considered a slightly different notion for sample complexity aiming to minimize the global expression $|\X\cap \C|$ with a bounded $\C$. We believe that our local notion, within an ${r^*}$-ball, is more informative and optimistic, as typical problems have obstacles, and the solution lies in a small subset of the search space. Furthermore, it would allow us to obtain tighter analyses by exploiting methods from discrete geometry that reason about ball structures.

Previous work has introduced several sample sets and analyzed their sample complexity~\cite{tsao2020sample,dayan2023near}. In this work, we consider more compact sets. Moreover, we introduce a new notion that better captures the computational complexity of constructing a PRM graph. In particular, we leverage the observation that the computational complexity of sampling-based planning is typically dominated by the amount of collision checks performed~\cite{KleinbortSH16}. Furthermore, nearest-neighbor search, which is another key contributor to the algorithm's computational complexity, 
can be eliminated for deterministic samples, assuming that they have a regular structure (as for lattice-based samples, which we describe below).

Collision checks are run both on the PRM vertices and edges, where edge checks are usually performed via dense sampling of configurations along the edges and individually validating each configuration. Thus, the total number of collision checks is proportional to the total edge length of the graph. We use this observation to develop a more accurate proxy for computational complexity. In particular, we will estimate the length of edges adjacent to the origin point $o \in \dR^d$, which is a vertex in all the sample sets introduced below. Moreover, due to the regularity of the sets, the attribute below is equal across all vertices (in the absence of obstacles).

\begin{problem}[Collision-check complexity]\label{problem:collision}
  For given \mbox{$\delta>0,\epsilon>0$}, find a $(\delta,\epsilon)$-complete sample set $\X$ of  minimal \emph{collision-check complexity}, i.e., minimizing the expression
  \[
      CC_\X:=\sum_{x\in \X\cap \B{r^*}} \|x\|.
  \]
\end{problem}

%To our knowledge, our work is the first to consider this notion to estimate the computational cost of sampling-based algorithms.

% Our main goal in this paper will be to research the sample and collision complexity of sample sets. Formally:
% \begin{definition}[Sample and Collision Complexity]
%     For a set $\chi\subset\mathbb{R}^d$ which is $(\delta, \epsilon)$-complete, we define:
%     \begin{enumerate}
%         \item \textbf{Sample Complexity}: a function $f=f(n,\delta,\epsilon)$ for which:
%         \[
%             \|\chi_A\|\triangleq\|\chi\cap A\|\leq k\cdot f(n,\delta,\epsilon)
%         \]
%         for some $A\subset\mathbb{R}^d,k>0,n>n_0,\delta<\delta_0,\epsilon<\epsilon_0$.
%         \item \textbf{Collision Complexity}: a function $g=g(n,\delta,\epsilon)$ for which
%         \[
%             I_{\chi,A}(r)\triangleq\sum_{p\in \chi_A}\|p\|^2\leq k\cdot g(n,\delta,\epsilon) 
%         \]
%         for some $A\subset\mathbb{R}^d,r>r_0,k>0,n>n_0,\delta<\delta_0,\epsilon<\epsilon_0$.
%     \end{enumerate}
% \end{definition}


% Let us start by defining the motion planning problem. In terms of the \emph{workspace}, the physical 3D space the robot resides in, the problem deals with navigating a set of points (together referred to as the robot) around a space filled with obstacles. 
% Now, because of how difficult it is to make sure every part of the robot is in obstacle-free space at any point, it is beneficial to consider what is called the \emph{configuration space} (denoted by the letter $C$)---a parameterization of the robot's position in space. A basic example would be a two-link manipulator arm. This manipulator has two degrees of freedom, and can be described using each link's relative angle: $(\theta_1,\theta_2)$. Thus, the configuration space of this robot is a torus $T^2=S^1\times S^1$.

% Using the configuration space allows us to consider a \emph{single point} $c\in C$ as the robot, thus a planning problem becomes conceptually simpler:

% \begin{definition}[Free Space]  

% \end{definition}

% \begin{definition}[Planning Phase]

% \end{definition}

% \noindent Using these definitions, the motion planning problem is, formally:
% \begin{definition}[Motion planning problem]
%     The MP problem is a tuple $M=(C_f, x_s, x_g)$, where $C_f$ is the free space and $\{x_s,x_g\}$ are the start and goal configurations. The problem $M$ is then solved at the aforementioned planning-phase.
% \end{definition}

% \textit{The prominent algorithmic framework} for solving motion planning problems for more than 20 years is \emph{Probability Roadmap} (or PRM)~\cite{kavraki1996probabilistic}. PRM (and, generally, PRM-based algorithms) build a graph using sample points and perform search algorithms on it to get a path. It follows these general steps: 
% \begin{enumerate}
%   \item \underline{Points:} Sample the configuration space, using some sampling scheme (e.g., randomly).
%   \item Perform \emph{Collision Detection} (CD) and remove configurations that are in collision.
%   \item \underline{Edges:} Connect sampled configurations. For example, using \emph{Nearest-Neighbor} methods (NN): $k$-NN (choosing the $k>0$ nearest neighbors) or ${r^*}$-NN (choosing all neighbors within a radius $r>0$).  
%   \item Perform CD on the connected configurations: for example, by checking dense intervals on the connected edge for collision of specific configurations. Remove edges that have collision.
%   \item \underline{Graph Search:} Utilize a search algorithm to obtain a path from the graph induced by the aforementioned points and edges. \emph{A*} is a classic search algorithm that takes the sample points, and “estimates” the distance from each sample point to the goal. It searches the graph, keeping track of a “score”: $current(v) = so\_far(v) + estimate(v)$, this way receiving an optimal path at the end---with the main difficulty being calculating a good estimate() function (this was expanded to the multiple-robot setting as M* \cite{wagner2015subdimensional}).
% \end{enumerate}
% With steps (1) to (4), we end up with the following formal definition of a PRM graph, as notated by Dayan et al.~\cite{dayan2023near}:
% \begin{definition}[PRM-Graph]
%     Let $\chi\subset C_f$ a sample set, $\{x_s, x_g\} \subset C_f$ start and goal samples and $r>0$ connection radius. Define the PRM graph to be $G_{M(\chi,r)}=(V,E)$ for:
%     \begin{align*}
%         V\triangleq & (\chi\cup\{x_s,x_g\})\cap C_f \\
%         E\triangleq & \{u,v\in V | \|u-v\|\leq r, [u,v]\subset C_f\}
%     \end{align*}
% \end{definition}
% PRM was proven~\cite{ladd2002generalizing} to be \emph{probabilistically-complete} (PC) --- meaning that with a number of samples tending to infinity, it would find some solution path with a probability tending to 1. It was later expanded to PRM*~\cite{karaman2011sampling} - an \emph{asymptotically-optimal} (AO) version of PRM, meaning that this time with a number of samples tending to infinity, it would find an \textbf{optimal} (length-wise) solution path with a probability tending to 1.

% Our paper will deal specific sample sets on PRM graphs, ones that posses a unique feature that is important in robotics (definition taken from~\cite{dayan2023near}):
% \begin{definition}[$(\delta, \epsilon)$-complete sets]
%     Given a sample set $\chi$ and a connection radius $r>0$, we say that $(\chi,r)$ is $(\delta,\epsilon)$-complete for some stretch factor $\epsilon>0$ and clearance factor $\delta>0$ if for every $\delta$-clear $M=(C_f,x_s,x_g)$ it holds that:
%     \begin{align*}
%         shortest\_path(G_{M(\chi,r)},x_s,x_g)\leq (1+\epsilon)OPT_{\delta}
%     \end{align*}    
% \end{definition}
% From the same paper, let us define the following concept:
% \begin{definition}[${\beta^*}$-covers]
%     Let $\delta>0$ be the clearance factor, and $\epsilon>0$ be the stretch factor. Then ${\beta^*}=\frac{\delta\epsilon}{\sqrt{1+\epsilon^2}}$ is the radius taken around samples, used to create a $(\delta, \epsilon)$-complete set. We define the balls that participate in the cover as \emph{${\beta^*}$-balls}.
% \end{definition}
%%% Local Variables:
%%% mode: latex
%%% TeX-master: "../main"
%%% End:
