\subsection{Additional details for Theorem~1}\label{app:decomp_lattice_proof}
We provide details that were omitted in the proof of Theorem~1. First, we derive the matrix $P$.
Given the planes $H$ and  $H_0:=\{x_{d+1}=0\}$, we wish  to find a plane $H_{ref}$ that is half-way (angle-wise) between $H$ and $H_0$. This would allow to reflect points in $H$ onto $H_0$ through $H_{ref}$ where the reflection is achieved using the Householder matrix $P:=I-2\hat{n}_{ref}\hat{n}^t_{ref}$, where $\hat{n}_{ref}\in \dR^{d+1}$ is the normal of $H_{ref}$~\cite{householder1958unitary}. That is, we reflect a lattice point $p\in \dR^{d+1}$ by computing the value \({p_{\text{reflected}}=P\cdot p}\).

Next, we show that the normal 
\begin{equation*}
    \hat{n}_{ref}:=\frac{1}{\sqrt{2-\frac{2}{\sqrt{d+1}}}}\cdot \left(-\tfrac{1}{\sqrt{d+1}},\dots,-\tfrac{1}{\sqrt{d+1}},1-\tfrac{1}{\sqrt{d+1}}\right)
\end{equation*}
satisfies those requirements.\footnote{We obtained the expression for $\hat{n}_{ref}$ by first considering $d=2$, where the task is more tangible, and then generalizing to higher dimensions.}  Consider the Householder matrix
\begin{align}\label{eq:reflection}
 P&=% I-2\hat{n}\hat{n}^t= 
 I-2\hat{n}_{ref}\hat{n}^t_{ref}\nonumber\\
 &=\begin{pNiceArray}{cw{c}{1cm}c|c}[margin]
            \Block{3-3}<\Large>{I_d - \frac{1}{D-\sqrt{D}}\mathds{1}} 
            & & & \dfrac{1}{\sqrt{D}} \\
            & & & \Vdots \\
            & & & \dfrac{1}{\sqrt{D}} \\
            \hline
            \dfrac{1}{\sqrt{D}} & \dots& \dfrac{1}{\sqrt{D}} & \dfrac{1}{\sqrt{D}}
        \end{pNiceArray},
    \end{align}
where $D:=d+1$, $I_d$ is an $d\times d$ identity matrix, and $\mathds{1}$ is the $d\times d$ matrix with $1$s in all its entries. 

Consider a point $p\in A^*_d$. Next, we show that it is reflected onto the plane  $H_0$, i.e., for $v=P\cdot p$, we get $v_{d+1}=0$. To do that, we move to the basis of the integer vector space, and show that for all $1\leq i\leq d$, taking the base element $e_i=(0,\dots,1,\dots,0)$, the $(d+1)$th element of $v=PG^t\cdot e_i$ (i.e., using the generator and then the reflector) is zero. First, for all $i<d$ it holds that 
    \begin{align*}
        PG^t\cdot e_i=P\cdot
        \begin{pmatrix}
        1 &  0&  \dots&  0& -1& 0 &\dots& 0
            % 1 \\
            % 0 \\
            % \vdots \\
            % 0 \\
            % -1 \\
            % 0 \\
            % \vdots \\
            % 0
        \end{pmatrix}^t.
    \end{align*}
Now, considering that the elements of the final row of $P$ are all equal to $1/\sqrt{D}$, we obtain a zero in the $(d+1)$th dimension. It remains to calculate the expression resulting from multiplying with $e_d$:
        \begin{align*}
        PG^t\cdot e_d=P\cdot
        \begin{pmatrix}
        -\frac{D-1}{D} &  \frac{1}{D}&  \dots&  \frac{1}{D}
        \end{pmatrix}^t.
    \end{align*}
    Looking specifically at the last element, we see that it is equal to     \begin{align*}
        \frac{1}{\sqrt{D}}\cdot\frac{1-D}{D} + (D-1)\frac{1}{D}\frac{1}{\sqrt{D}}=\frac{1-D+D-1}{D\sqrt{D}}=0.
    \end{align*}

    That is, by applying the transformation $P$ on the lattice points, we reflect them onto the $x_{d+1}=0$ plane. It remains to get rid of the $(d+1)$th dimension. This is accomplished by the mapping
\begin{align*}
        E=
        \begin{pmatrix}
            1 & 0 & \dots & 0 & 0 \\
            0 & 1 & \dots & 0 & 0 \\
            \vdots & \vdots & \ddots & \vdots & 0 \\
            0 & 0 & \dots & 1 & 0
        \end{pmatrix}_{d\times(d+1)}.
    \end{align*}
    
It remains to compute the  explicit embedding $T(g):=EPG^t(g)$, for $g\in \dZ^d$. We first calculate 
    \begin{align*}
        \left(EP\right)^t=
        \begin{pNiceArray}{cw{c}{1cm}c}[margin]
            \Block{3-3}<\Large>{I_d - \frac{1}{D-\sqrt{D}}\mathds{1}} 
            & &  \\
            & &  \\
            & &  \\
            \hline
            \dfrac{1}{\sqrt{D}} & \dots & \dfrac{1}{\sqrt{D}}
        \end{pNiceArray}_{d\times(d+1)}.
    \end{align*}
 Next, it can be shown that
    \begin{align}
        T^t&=G\left(EP\right)^t\nonumber\\
        &=\begin{pmatrix}
            1 & -1 &  0  & \dots & 0 \\
            1 & 0  &  -1 & \dots & 0 \\
            \vdots & \vdots  &  \vdots  & \ddots & \vdots \\
            1 & 0  &  0  & \dots & -1 \\
            \frac{1}{D - \sqrt{D}} - 1 & \frac{1}{D - \sqrt{D}} & \frac{1}{D - \sqrt{D}} & \dots & \frac{1}{D - \sqrt{D}}
        \end{pmatrix}_{d\times(d+1)}\!\!\!\!\!\!.
    \end{align}

    %      \begin{figure}[H]
    %     \centering
    %     \begin{subfigure}[b]{0.3\textwidth}
    %         \includegraphics[width=\textwidth]{Images/EPGt_visual_explanation1.png}
    %         %\caption{Sample points in $H_0$}
    %         %\label{fig:epgt_visual1}
    %     \end{subfigure}
    %     \hfill
    %     \begin{subfigure}[b]{0.3\textwidth}
    %         \includegraphics[width=\textwidth]{Images/EPGt_visual_explanation2.png}
    %         %\caption{$H_0$ rotated to the "floor"}
    %         %\label{fig:epgt_visual2}
    %     \end{subfigure}
    %     \hfill
    %     \begin{subfigure}[b]{0.3\textwidth}
    %         \includegraphics[width=\textwidth]{Images/EPGt_visual_explanation3.png}
    %         %\caption{The samples as they look in $\dR^2$}
    %         %\label{fig:epgt_visual3}
    %     \end{subfigure}
    %     \caption{Visualization of embedding the lattice  $A_2^*$ originally defined in $\dR^3$ onto $\dR^2$ via the mapping $T$. [Left] The blue rectangle represents the plane $H$, where the corresponding $A_2^*$ lattice points are drawn in red. The points are generated by taking integer vectors in $\dR^d$ and applying the mapping $G^t$.  [Center] $H$ and $A_2^*$ is reflected onto the plane $H_0=\{x_3=0\}$ using the mapping $PG^t$. [Right] The third dimension is removed, via the mapping $E$, to yield the embedding of $A_2^*$ in $\dR^2$.}
    %     \label{fig:egpt_visual}
    % \end{figure}


\subsection{Additional details for Theorem~3}\label{app:CC}
We provide details omitted from the main body of the text. 
We start with a simplified derivation of a single annulus, which would inform the more advanced construction. Fix ${0<r_1<
  r^*}$ forced it to a single line, and define $\btheta_{r'} := \frac{r'}{{\beta^*}}f_\Lambda$, and observe that 
\begin{align}
  CC_\X&\leq  r^*\cdot
\left|\X\cap (\B_{r^*}\setminus \B_{r_1})\right| + r_1\cdot
\left|\X\cap \B_{r_1}\right|\nonumber \\ & = r^*\left(|\X\cap \B_{r^*}|-|\X \cap \B_{r_1}|\right) + r_1 \left|\X\cap \B_{r_1}\right| \nonumber\\
& = r^*|\X\cap \B_{r^*}|+ (r_1-r^*) |\X\cap \B_{r_1}| \nonumber \\  
  & = r^*\frac{\partial(B_1)}{\sqrt{\det(\Lambda)}}\btheta^d_{r^*} +r^* P_d(\btheta_{r^*}) + (r_1-r^*) \frac{\partial(B_1)}{\sqrt{\det(\Lambda)}}\btheta^d_{r_1}\nonumber\\& + (r_1-r^*) P_d(\btheta_{r_1}) \nonumber
\\
& = \frac{\partial(B_1)}{\sqrt{\det(\Lambda)}}\theta^d\left({r^*}^{d+1}+{r_1}^{d+1}-r^*{r_1}^{d}\right)\nonumber\\&+ r P_d(\btheta_{r^*}) + (r_1-r^*) P_d(\btheta_{r_1})\nonumber \\ & = \frac{\partial(B_1)}{\sqrt{\det(\Lambda)}}\theta^d\left({r^*}^{d+1}+{r_1}^{d+1}-r^*{r_1}^{d}\right)+ r^* P_d(\btheta_{r^*}), \label{eq:CC1}
\end{align}
where the sample complexity bound in Equation~(5) is used. For simplicity, we bound throughout the error term with $r^* P_d(\btheta_{r^*})$.
Next, we optimize the value $r_1$ to minimize the expression in Equation~\eqref{eq:CC1}.

Consider the function $f(r_1)={r^*}^{d+1}-{r^*} r^d_1 + {r^*}^{d+1}_1$. We look for the minimum of $f(r_1)$ by requiring that
\begin{align*}
            f'(r_1)=-{r^*} dr_1^{d-1}+(d+1)r_1^d=0,
\end{align*}
which yields the value $r'_1:=\frac{d}{d+1}{r^*}$. This value is  a minimum since
\begin{align*}
 f^{(2)}(r_1)|_{r'_1}=&\left(-{r^*}(d-1)r_1^{d-2}+d(d+1)r_1^{d-1}\right)|_{r_1'}\\
        =&{r^*}^{d-1}\left(\frac{d^d}{(d+1)^{d-2}}-\frac{d^{d-2}(d-1)}{(d+1)^{d-2}}\right)\\
        =&{r^*}^{d-1}d^{d-2}\frac{d^2-d+1}{(d+1)^{d-2}},
    \end{align*}
    and we know that $d^2-d+1>0$ for all $d\geq 2$.%, then $r_1=\frac{d}{d+1}r$ indeed minimizes $f(r_1)$.
    
Now, we apply the above line of reasoning in a recursive manner by considering a sequence of $k+1\geq 2$ radii ${0<r_k<\ldots<r_0={r^*}}$ where $r_i:=\td^i r^*$, where $\td:=\frac{d}{d+1}$. This leads to the bound
\begin{align}
\label{eq:cc_eval_app}
CC_\X&\leq \sum_{i=0}^{k-1}r_i |\X\cap (\B_{r_i}\setminus \B_{r_{i+1}})| + r_k|\X\cap \B_{r_k}|\nonumber\\
  &= \frac{\partial(B_1)}{\sqrt{\det(\Lambda)}} \left(\underbrace{{r^*} \btheta^d_{r^*} + \sum_{i=1}^k(r_i-r_{i-1}) \btheta^d_{r_i}}_{:=\gamma}\right) + {r^*} P_d(\btheta_{r^*}).
\end{align}

We show that 
\[\gamma:=r \btheta^d_{r^*} + \sum_{i=1}^k(r_i-r_{i-1}) \btheta^d_{r_i}= {r^*} \btheta^d_{r^*} \left(1 - \frac{\xi^{d+2} - \xi}{ d\xi - (d+1)}\right),\]
where $r_i=\td^i {r^*},\td:=\frac{d}{d+1}, \btheta_{r_i}= r_i\frac{\btheta_{r^*}}{r^*}, k=d,$ and $\xi:=\td^d=\left(\frac{d}{d+1}\right)^d$. In particular,
\begin{align}
  \gamma &={r^*} \btheta^d_{r^*} + \sum_{i=1}^k(r_i-r_{i-1}) r_i^d\frac{\btheta^d_{r^*}}{{r^*}^d} \nonumber\\
  & = {r^*} \btheta^d_{r^*} + \sum_{i=1}^k{r^*}\td^{i-1}(\td-1) \td^{di} {r^*}^d\frac{\btheta^d_{r^*}}{{r^*}^d} \nonumber\\ 
  &=  {r^*} \btheta^d_{r^*} + \sum_{i=1}^k{r^*} (\td-1) \td^{di+ i -1} \btheta^d_{r^*} \nonumber\\ 
  &=   {r^*} \btheta^d_{r^*} \left(1 + \sum_{i=1}^k (\td-1) \td^{di+ i -1} \right)\nonumber\\
  &= {r^*} \btheta^d_{r^*} \left(1 + \frac{\td-1}{\td}\sum_{i=1}^k \td^{(d+1)i} \right)\nonumber\\
  &= {r^*} \btheta^d_{r^*} \left(1 + \frac{\td-1}{\td}\frac{\left(\td^{d+1}\right)^{k+1} - \td^{d+1}}{\td^{d+1} - 1} \right)\nonumber\\
  &= {r^*} \btheta^d_{r^*} \left(1 + \td^d(\td-1)\frac{\left(\td^{d+1}\right)^k - 1}{\td^{d+1} - 1} \right).\nonumber\\
  % & = r \btheta^d_{r} + \sum_{i=1}^kr\td^{i-1}(\td-1) \td^{di} r^d\frac{\btheta^d_{r}}{r^d} \\ &  =  r \btheta^d_{r} + \sum_{i=1}^kr (\td-1) \td^{di+ i -1} \btheta^d_{r} \\ & =   r \btheta^d_{r} \left(1 + \sum_{i=1}^k (\td-1) \td^{di+ i -1} \right)\\
\end{align}

Taking $k=d$ results in $r_k=(\frac{d}{d+1})^d {r^*}\approx\frac{1}{e}{r^*}$. 
To use the sample set analysis, we need a large enough $r$ value, so assuming the original $r$ is large enough, we can deduce safely that $\frac{r}{e}$ is also large enough. 
Notice also that $\td - 1 = \frac{-1}{d+1}$, and thus $(d+1)\td=d$, so returning to our expression, and substituting $\xi:=\td^d=\left(\frac{d}{d+1}\right)^d$, we obtain 
\begin{align*}
    \gamma&={r^*} \btheta^d_{r^*} \left(1 - \frac{\td^d\left(\td^{d(d+1)} - 1\right)}{(d+1)(\td^{d+1} - 1)}\right)\\
    &= {r^*} \btheta^d_{r^*} \left(1 - \frac{\xi\left(\xi^{d+1} - 1\right)}{(d+1)(\td \xi - 1)}\right) 
    \\&= {r^*} \btheta^d_{r^*} \left(1 - \frac{\xi^{d+2} - \xi}{ d\xi - (d+1)}\right)
    :={r^*} \btheta^d_{r^*}\zeta.
\end{align*}

We finish this section with a plot of the value $\gamma$ in Figure~\ref{fig:annuli_bound:app}.

\begin{figure}[thb]
\centering  
\includegraphics[width=0.9\columnwidth]{Images/annuli_bound.pdf}
\caption{Plot of the improvement factor $\gamma$.}
\label{fig:annuli_bound:app}
\end{figure}

\subsection{Additional experimental results}
Additional scenarios, which were omitted from the main paper, are given in Figure~\ref{fig:scenarios:app}. Extended results comparing lattice-based samples using the \loc algorithm are provided in Table~\ref{tbl:lattice_comparison:app}.

\begin{figure*}[tbh]
  \centering
%     \hspace*{-0.66cm}
% \subfloat[Zigzag-bypass (long)]{\includegraphics[width=2.18\columnwidth,clip]{Images/Scenarios/ZZB3H_scenario.png}
%     %\label{fig:3d_lattices:da}
%     }
%     \newline
\subfloat[Zigzag-bypass (short)]{\includegraphics[width=1.15\columnwidth,clip]{Images/Scenarios/ZZB2H_scenario.png}
    %\label{fig:3d_lattices:da}
    }
\subfloat[Narrow (more scenarios)]{\includegraphics[width=0.465\columnwidth,clip]{Images/Scenarios/N1_scenarios.png}
    %\label{fig:3d_lattices:da}
    }
  \caption{Additional scenarios used in the experiments. The scenario ZZB3, which is not illustrated here, is similar to ZZB2, only that the horizontal hallways are twice as long.}
  \label{fig:scenarios:app}
\end{figure*}

\begin{table}[tbh]
\caption{Extended comparison of running time and solution length using lattices-based sample sets (where the underlying lattice is denoted in the table) in the iA*-\loc algorithm. Solution length is normalized with respect to the length obtained using $\XA$. }
\centering
\label{tbl:lattice_comparison:app}
\begin{tabular}{|c||ccc|cc|}
\hline
 & \multicolumn{3}{c|}{\cellcolor[HTML]{EFEFEF} Time (s)} & \multicolumn{2}{c|}{\cellcolor[HTML]{EFEFEF} Length (r)} \\ \cline{2-6} 
\multirow{-2}{*}{\begin{tabular}[c]{@{}c@{}}Scenario\\ (robot \#)\end{tabular}} & \multicolumn{1}{c|}{\cellcolor[HTML]{FFFFC7}$\ZN$} & \multicolumn{1}{c|}{\cellcolor[HTML]{FFFFC7}$\DN$} & \cellcolor[HTML]{FFFFC7}$\AN$ & \multicolumn{1}{c|}{\cellcolor[HTML]{FFFFC7}$\ZN$} & \cellcolor[HTML]{FFFFC7}$\DN$ \\ \hline \hline
\cellcolor[HTML]{ECF4FF}N4(2) & \multicolumn{1}{c|}{0.00} & \multicolumn{1}{c|}{0.00} & 0.00 & \multicolumn{1}{c|}{0.62} & 0.74 \\
\cellcolor[HTML]{ECF4FF}N1(5) & \multicolumn{1}{c|}{165.35} & \multicolumn{1}{c|}{4.59} & 0.36 & \multicolumn{1}{c|}{0.65} & 0.79 \\
\cellcolor[HTML]{ECF4FF}N2(5) & \multicolumn{1}{c|}{62.68} & \multicolumn{1}{c|}{1.81} & 0.41 & \multicolumn{1}{c|}{0.85} & 0.95 \\
\cellcolor[HTML]{ECF4FF}N3(5) & \multicolumn{1}{c|}{142.27} & \multicolumn{1}{c|}{2.91} & 0.59 & \multicolumn{1}{c|}{0.65} & 0.87 \\
\cellcolor[HTML]{ECF4FF}N5(5) & \multicolumn{1}{c|}{dnf} & \multicolumn{1}{c|}{4.82} & 3.32 & \multicolumn{1}{c|}{dnf} & 0.82 \\
\cellcolor[HTML]{ECF4FF}N1B(6) & \multicolumn{1}{c|}{dnf} & \multicolumn{1}{c|}{328.30} & 15.08 & \multicolumn{1}{c|}{dnf} & 0.89 \\ \hline
\cellcolor[HTML]{ECF4FF}BT4(2) & \multicolumn{1}{c|}{0.04} & \multicolumn{1}{c|}{0.01} & 0.01 & \multicolumn{1}{c|}{0.69} & 0.85 \\
\cellcolor[HTML]{ECF4FF}BT10(2) & \multicolumn{1}{c|}{-} & \multicolumn{1}{c|}{1.20} & 0.30 & \multicolumn{1}{c|}{-} & 0.92 \\
\cellcolor[HTML]{ECF4FF}BT5(3) & \multicolumn{1}{c|}{0.54} & \multicolumn{1}{c|}{0.14} & 0.06 & \multicolumn{1}{c|}{0.38} & 0.51 \\
\cellcolor[HTML]{ECF4FF}BT1(4) & \multicolumn{1}{c|}{146.69} & \multicolumn{1}{c|}{50.81} & 3.51 & \multicolumn{1}{c|}{0.95} & 1.03 \\
\cellcolor[HTML]{ECF4FF}BT6(4) & \multicolumn{1}{c|}{dnf} & \multicolumn{1}{c|}{153.40} & 12.36 & \multicolumn{1}{c|}{dnf} & 1.04 \\
\cellcolor[HTML]{ECF4FF}BT7(4) & \multicolumn{1}{c|}{240.88} & \multicolumn{1}{c|}{5.38} & 4.36 & \multicolumn{1}{c|}{0.95} & 0.96 \\ \hline
\cellcolor[HTML]{ECF4FF}K1(3) & \multicolumn{1}{c|}{32.31} & \multicolumn{1}{c|}{4.97} & 1.37 & \multicolumn{1}{c|}{0.82} & 0.89 \\ \hline
\cellcolor[HTML]{ECF4FF}UM4(2) & \multicolumn{1}{c|}{-} & \multicolumn{1}{c|}{8.47} & 2.43 & \multicolumn{1}{c|}{-} & 0.90 \\
\cellcolor[HTML]{ECF4FF}UM1(3) & \multicolumn{1}{c|}{482.17} & \multicolumn{1}{c|}{25.15} & 6.68 & \multicolumn{1}{c|}{0.84} & 1.16 \\
\cellcolor[HTML]{ECF4FF}UM2(3) & \multicolumn{1}{c|}{13.35} & \multicolumn{1}{c|}{1.22} & 0.04 & \multicolumn{1}{c|}{1.04} & 1.52 \\
\cellcolor[HTML]{ECF4FF}UM4B3(3) & \multicolumn{1}{c|}{99.35} & \multicolumn{1}{c|}{1.03} & 0.66 & \multicolumn{1}{c|}{1.59} & 0.89 \\
\cellcolor[HTML]{ECF4FF}UM3(4) & \multicolumn{1}{c|}{236.31} & \multicolumn{1}{c|}{223.87} & 64.57 & \multicolumn{1}{c|}{0.63} & 0.97 \\ \hline
\cellcolor[HTML]{ECF4FF}ZZB1(2) & \multicolumn{1}{c|}{1.93} & \multicolumn{1}{c|}{1.01} & 0.44 & \multicolumn{1}{c|}{0.94} & 0.94 \\
\cellcolor[HTML]{ECF4FF}ZZB2(2) & \multicolumn{1}{c|}{2.91} & \multicolumn{1}{c|}{0.93} & 0.71 & \multicolumn{1}{c|}{0.94} & 0.94 \\
\cellcolor[HTML]{ECF4FF}ZZB3(2) & \multicolumn{1}{c|}{2.26} & \multicolumn{1}{c|}{0.84} & 0.47 & \multicolumn{1}{c|}{0.95} & 0.95 \\ \hline\end{tabular}
\end{table}

\begin{table*}[tbh]
\centering
\begin{tabular}{|c|cccl|ccl|cl|cl|}
\hline
 & \multicolumn{4}{c|}{\cellcolor[HTML]{EFEFEF} Total time (s)} & \multicolumn{3}{c|}{\cellcolor[HTML]{EFEFEF} Search time (s)} & \multicolumn{2}{c|}{\cellcolor[HTML]{EFEFEF}Length (r)} & \multicolumn{2}{c|}{\cellcolor[HTML]{EFEFEF}Success (\%)} \\ \cline{2-12} 
\multirow{-2}{*}{\begin{tabular}[c]{@{}c@{}}Scenario\\ (Robot \#)\end{tabular}} & \multicolumn{1}{c|}{\cellcolor[HTML]{FFFFC7}\begin{tabular}[c]{@{}c@{}}$\AN$\\ \loc\end{tabular}} & \multicolumn{1}{c|}{\cellcolor[HTML]{FFFFC7}\begin{tabular}[c]{@{}c@{}}$\AN$\\ \glo\end{tabular}} & \multicolumn{1}{c|}{\cellcolor[HTML]{FFFFC7}\begin{tabular}[c]{@{}c@{}}\rnd\\ \glo\end{tabular}} & \multicolumn{1}{c|}{\cellcolor[HTML]{FFFFC7}\begin{tabular}[c]{@{}c@{}}\rndm\\ \glo\end{tabular}} & \multicolumn{1}{c|}{\cellcolor[HTML]{FFFFC7}\begin{tabular}[c]{@{}c@{}}$\AN$\\ \glo\end{tabular}} & \multicolumn{1}{c|}{\cellcolor[HTML]{FFFFC7}\begin{tabular}[c]{@{}c@{}}\rnd\\ \glo\end{tabular}} & \multicolumn{1}{c|}{\cellcolor[HTML]{FFFFC7}\begin{tabular}[c]{@{}c@{}}\rndm\\ \glo\end{tabular}} & \multicolumn{1}{c|}{\cellcolor[HTML]{FFFFC7}\begin{tabular}[c]{@{}c@{}}\rnd\\ \glo\end{tabular}} & \multicolumn{1}{c|}{\cellcolor[HTML]{FFFFC7}\begin{tabular}[c]{@{}c@{}}\rndm\\ \glo\end{tabular}} & \multicolumn{1}{c|}{\cellcolor[HTML]{FFFFC7}\begin{tabular}[c]{@{}c@{}}\rnd\\ \glo\end{tabular}} & \multicolumn{1}{c|}{\cellcolor[HTML]{FFFFC7}\begin{tabular}[c]{@{}c@{}}\rndm\\ \glo\end{tabular}} \\ \hline
\cellcolor[HTML]{ECF4FF}N1(5) & \multicolumn{1}{c|}{0.36} & \multicolumn{1}{c|}{3.05} & \multicolumn{1}{c|}{4.16} & 3.40 & \multicolumn{1}{c|}{0.84} & \multicolumn{1}{c|}{3.37} & 2.59 & \multicolumn{1}{c|}{1.48} & 1.46 & \multicolumn{1}{c|}{80.00} & 90 \\
\cellcolor[HTML]{ECF4FF}N2(5) & \multicolumn{1}{c|}{0.41} & \multicolumn{1}{c|}{2.67} & \multicolumn{1}{c|}{2.74} & 4.28 & \multicolumn{1}{c|}{0.82} & \multicolumn{1}{c|}{2.11} & 3.62 & \multicolumn{1}{c|}{2.43} & 3.31 & \multicolumn{1}{c|}{65.00} & 95 \\
\cellcolor[HTML]{ECF4FF}N3(5) & \multicolumn{1}{c|}{0.59} & \multicolumn{1}{c|}{3.83} & \multicolumn{1}{c|}{5.44} & 4.22 & \multicolumn{1}{c|}{1.72} & \multicolumn{1}{c|}{4.65} & 3.39 & \multicolumn{1}{c|}{2.02} & 1.56 & \multicolumn{1}{c|}{85.00} & 85 \\
\cellcolor[HTML]{ECF4FF}N5(5) & \multicolumn{1}{c|}{3.32} & \multicolumn{1}{c|}{31.48} & \multicolumn{1}{c|}{23.42} & 26.19 & \multicolumn{1}{c|}{20.02} & \multicolumn{1}{c|}{18.62} & 21.14 & \multicolumn{1}{c|}{0.89} & 0.88 & \multicolumn{1}{c|}{100.00} & 100 \\ \hline
\cellcolor[HTML]{ECF4FF}BT9(2) & \multicolumn{1}{c|}{0.13} & \multicolumn{1}{c|}{0.13} & \multicolumn{1}{c|}{0.77} & 0.42 & \multicolumn{1}{c|}{0.13} & \multicolumn{1}{c|}{0.77} & 0.42 & \multicolumn{1}{c|}{1.10} & 1.41 & \multicolumn{1}{c|}{95.00} & 40 \\
\cellcolor[HTML]{ECF4FF}BT10(2) & \multicolumn{1}{c|}{0.30} & \multicolumn{1}{c|}{0.31} & \multicolumn{1}{c|}{1.16} & 0.46 & \multicolumn{1}{c|}{0.31} & \multicolumn{1}{c|}{1.16} & 0.46 & \multicolumn{1}{c|}{1.13} & 1.27 & \multicolumn{1}{c|}{95.00} & 75 \\
\cellcolor[HTML]{ECF4FF}BT1B(3) & \multicolumn{1}{c|}{34.83} & \multicolumn{1}{c|}{47.58} & \multicolumn{1}{c|}{118.27} & 62.88 & \multicolumn{1}{c|}{47.27} & \multicolumn{1}{c|}{118.11} & 62.70 & \multicolumn{1}{c|}{0.93} & 0.99 & \multicolumn{1}{c|}{100.00} & 100 \\
\cellcolor[HTML]{ECF4FF}BT2(3) & \multicolumn{1}{c|}{5.62} & \multicolumn{1}{c|}{7.08} & \multicolumn{1}{c|}{22.67} & 28.17 & \multicolumn{1}{c|}{6.97} & \multicolumn{1}{c|}{22.61} & 28.11 & \multicolumn{1}{c|}{0.93} & 1.12 & \multicolumn{1}{c|}{100.00} & 95 \\
\cellcolor[HTML]{ECF4FF}BT2B(3) & \multicolumn{1}{c|}{9.58} & \multicolumn{1}{c|}{14.40} & \multicolumn{1}{c|}{41.67} & 21.23 & \multicolumn{1}{c|}{14.13} & \multicolumn{1}{c|}{4.36} & 21.09 & \multicolumn{1}{c|}{1.00} & 1.05 & \multicolumn{1}{c|}{100.00} & 95 \\
\cellcolor[HTML]{ECF4FF}BT3(3) & \multicolumn{1}{c|}{5.38} & \multicolumn{1}{c|}{14.15} & \multicolumn{1}{c|}{62.22} & 32.10 & \multicolumn{1}{c|}{12.80} & \multicolumn{1}{c|}{61.54} & 31.39 & \multicolumn{1}{c|}{1.05} & 1.11 & \multicolumn{1}{c|}{100.00} & 100 \\
\cellcolor[HTML]{ECF4FF}BT5(3) & \multicolumn{1}{c|}{0.06} & \multicolumn{1}{c|}{0.38} & \multicolumn{1}{c|}{0.27} & 0.19 & \multicolumn{1}{c|}{0.12} & \multicolumn{1}{c|}{0.14} & 0.05 & \multicolumn{1}{c|}{0.57} & 0.57 & \multicolumn{1}{c|}{100.00} & 85 \\
\cellcolor[HTML]{ECF4FF}BT8(3) & \multicolumn{1}{c|}{12.17} & \multicolumn{1}{c|}{19.31} & \multicolumn{1}{c|}{169.32} & 79.52 & \multicolumn{1}{c|}{18.87} & \multicolumn{1}{c|}{169.12} & 79.31 & \multicolumn{1}{c|}{1.00} & 1.05 & \multicolumn{1}{c|}{100.00} & 100 \\
\cellcolor[HTML]{ECF4FF}BT8B(3) & \multicolumn{1}{c|}{3.17} & \multicolumn{1}{c|}{3.60} & \multicolumn{1}{c|}{41.63} & 24.23 & \multicolumn{1}{c|}{3.55} & \multicolumn{1}{c|}{41.60} & 24.20 & \multicolumn{1}{c|}{1.16} & 1.20 & \multicolumn{1}{c|}{100.00} & 100 \\
\cellcolor[HTML]{ECF4FF}BT11(3) & \multicolumn{1}{c|}{17.21} & \multicolumn{1}{c|}{35.21} & \multicolumn{1}{c|}{51.19} & 31.23 & \multicolumn{1}{c|}{34.34} & \multicolumn{1}{c|}{50.77} & 30.80 & \multicolumn{1}{c|}{0.88} & 0.95 & \multicolumn{1}{c|}{100.00} & 100 \\
\cellcolor[HTML]{ECF4FF}BT1(4) & \multicolumn{1}{c|}{3.51} & \multicolumn{1}{c|}{97.33} & \multicolumn{1}{c|}{63.69} & 68.39 & \multicolumn{1}{c|}{13.88} & \multicolumn{1}{c|}{18.83} & 22.18 & \multicolumn{1}{c|}{1.02} & 1.04 & \multicolumn{1}{c|}{100.00} & 100 \\
\cellcolor[HTML]{ECF4FF}BT6(4) & \multicolumn{1}{c|}{12.36} & \multicolumn{1}{c|}{124.16} & \multicolumn{1}{c|}{106.73} & 95.75 & \multicolumn{1}{c|}{43.87} & \multicolumn{1}{c|}{61.96} & 50.18 & \multicolumn{1}{c|}{1.01} & 1.03 & \multicolumn{1}{c|}{100.00} & 100 \\
\cellcolor[HTML]{ECF4FF}BT7(4) & \multicolumn{1}{c|}{4.36} & \multicolumn{1}{c|}{95.89} & \multicolumn{1}{c|}{60.65} & 56.06 & \multicolumn{1}{c|}{15.06} & \multicolumn{1}{c|}{15.24} & 9.29 & \multicolumn{1}{c|}{1.00} & 1.01 & \multicolumn{1}{c|}{100.00} & 100 \\ \hline
\cellcolor[HTML]{ECF4FF}UM4(2) & \multicolumn{1}{c|}{2.43} & \multicolumn{1}{c|}{2.93} & \multicolumn{1}{c|}{12.71} & 2.26 & \multicolumn{1}{c|}{2.90} & \multicolumn{1}{c|}{12.69} & 2.24 & \multicolumn{1}{c|}{0.96} & 1.41 & \multicolumn{1}{c|}{70.00} & 5 \\
\cellcolor[HTML]{ECF4FF}UM4B1(2) & \multicolumn{1}{c|}{4.81} & \multicolumn{1}{c|}{5.68} & \multicolumn{1}{c|}{17.38} & 5.03 & \multicolumn{1}{c|}{5.64} & \multicolumn{1}{c|}{17.35} & 5.01 & \multicolumn{1}{c|}{0.86} & 1.12 & \multicolumn{1}{c|}{90.00} & 45 \\
\cellcolor[HTML]{ECF4FF}UM1(3) & \multicolumn{1}{c|}{6.68} & \multicolumn{1}{c|}{58.62} & \multicolumn{1}{c|}{49.35} & 31.92 & \multicolumn{1}{c|}{47.14} & \multicolumn{1}{c|}{42.58} & 24.86 & \multicolumn{1}{c|}{0.98} & 1.09 & \multicolumn{1}{c|}{100.00} & 100 \\
\cellcolor[HTML]{ECF4FF}UM2(3) & \multicolumn{1}{c|}{0.04} & \multicolumn{1}{c|}{2.94} & \multicolumn{1}{c|}{4.49} & 3.44 & \multicolumn{1}{c|}{0.21} & \multicolumn{1}{c|}{2.97} & 1.83 & \multicolumn{1}{c|}{1.95} & 2.23 & \multicolumn{1}{c|}{75.00} & 40 \\
\cellcolor[HTML]{ECF4FF}UM5(3) & \multicolumn{1}{c|}{2.87} & \multicolumn{1}{c|}{31.79} & \multicolumn{1}{c|}{29.71} & 21.31 & \multicolumn{1}{c|}{19.01} & \multicolumn{1}{c|}{22.18} & 13.63 & \multicolumn{1}{c|}{1.05} & 1.26 & \multicolumn{1}{c|}{95.00} & 85 \\ \hline
\cellcolor[HTML]{ECF4FF}ZZB1(2) & \multicolumn{1}{c|}{0.44} & \multicolumn{1}{c|}{0.49} & \multicolumn{1}{c|}{10.44} & 0.47 & \multicolumn{1}{c|}{0.48} & \multicolumn{1}{c|}{10.43} & 0.45 & \multicolumn{1}{c|}{0.89} & 1.01 & \multicolumn{1}{c|}{100.00} & 5 \\
\cellcolor[HTML]{ECF4FF}ZZB2(2) & \multicolumn{1}{c|}{0.71} & \multicolumn{1}{c|}{2.51} & \multicolumn{1}{c|}{272.70} & 2.66 & \multicolumn{1}{c|}{1.22} & \multicolumn{1}{c|}{271.96} & 1.83 & \multicolumn{1}{c|}{0.89} & 2.62 & \multicolumn{1}{c|}{100.00} & 65 \\
\cellcolor[HTML]{ECF4FF}ZZB3(2) & \multicolumn{1}{c|}{0.47} & \multicolumn{1}{c|}{7.69} & \multicolumn{1}{c|}{341.84} & 5.33 & \multicolumn{1}{c|}{1.18} & \multicolumn{1}{c|}{338.15} & 1.45 & \multicolumn{1}{c|}{0.88} & 2.58 & \multicolumn{1}{c|}{100.00} & 65 \\ \hline
\end{tabular}
\caption{Comparison of running time and solution length between $\XA$ (using \loc and \glo) and uniform random sampling. For random sampling we report the average values in terms of running and solution length (the latter is given as normalized value with  respect to the solution length with $\XA$). }
\label{tbl:lattice_vs_random:app}
\end{table*}

\begin{figure*}[th]
  \centering
%     \hspace*{-0.66cm}
% \subfloat[Zigzag-bypass (long)]{\includegraphics[width=2.18\columnwidth,clip]{Images/Scenarios/ZZB3H_scenario.png}
%     %\label{fig:3d_lattices:da}
%     }
%     \newline
\subfloat{\includegraphics[width=\columnwidth,clip]{Images/tuning1.pdf}
    }
    \subfloat{\includegraphics[width=\columnwidth,clip]{Images/tuning2.pdf}
    }
    \newline
\subfloat{\includegraphics[width=\columnwidth,clip]{Images/tuning3.pdf}
    }
\subfloat{\includegraphics[width=\columnwidth,clip]{Images/tuning4.pdf}
    }
      \caption{Effect of the parameters $\delta,\epsilon$ on the performance of \loc with $\XA$ for $\delta=2.5$ (left) and $\delta=4$ (right). We report the running time (top) and solution length (bottom). The absence of data points for the parameters $\delta=4, \eps\in \{2,4,5\}$ indicates a solution failure.  
  }
  \label{fig:parameters:app}
\end{figure*}

\subsection{Comparison with Random Sampling}
Extended results where $\XA$-samples are compared with \rnd are given in Table~\ref{tbl:lattice_vs_random:app}. Here, we consider two versions of random sampling. The first version, denoted by \rnd, which is identical to the one considered in the main paper, uses random sampling together with the asymptotically optimal connection radius $r_{\textup{rnd}}(n)$, which is commonly used in practice. The second version, denoted by \rndm uses the radius as ${r^*}$ used for lattice-based sampling. The latter is used to further emphasize the inferiority of uniform random sampling as compared to $\XA$ due to identical parameters between $\XA$-\glo and \rndm-\glo (except for the sampling distribution). In particular, the move to \rndm  severely reduces the success rates in some of the scenarios.

Another addition in Table~\ref{tbl:lattice_vs_random:app} is the running time of the search algorithm (under "search time"). Recall that the total running time for \glo consists of the (i) construction of the sample set and the nearest-neighbor data structure and the (ii) running the search algorithm. Although both $\XA$-\glo and \rnd use the same number of samples, the construction time is usually larger in the former due to an additional step of constructing the lattice samples over the entire configuration space, which is currently implemented in a naive and unoptimized manner. In this sense, the comparison between $\XA$-\glo and \rnd is not entirely fair. Thus, we also report the running time of the search algorithm, which can be the computational bottleneck, especially for more complicated robot geometries where the collision-check operation is more expensive~\cite{KleinbortSH16}. Although the search time for $\XA$-\glo is usually lower for most scenarios, we argue that with more expensive collision checks, the advantage of lattice-based sample sets would be even more prominent.

\subsection{Effect of parameter choice}
We report the effect of the choice of the $\delta$ and $\eps$ parameters on solution length and running time for the \loc algorithm using $\XA$ sampling. We specifically focus on the ZZB3 scenario due to the availability of several homotopy classes for the solution, where each class has a different length and level of difficulty. For instance, in one class, the robots use the rightmost part of the workspace, which consists of a winding path, and exchange positions halfway between---leading to a relatively short solution length. In a second class, the robots use the long passage to the left, which consists of long straight-line motions and yields a significantly longer solution length.


We set $\delta\in \{2.75,4\}$ and report the solution length and running time in Figure~\ref{fig:parameters:app} for $\eps\in \{0.5,0.6,\ldots,1,2,\ldots,10\}$. Observe that for $\delta=2.75$ the planner obtains a low-length solution already for high $\eps$ values, whereas $\delta=4$ initially uncovers an inefficient solution length-wise but eventually settles on the better homotopy class when $\eps$ is reduced. From values of $\eps\leq 1$ the length relatively stabilizes, while the runtime jumps at several orders of magnitude, which highlights the exponential dependence of sample and collision-check complexity on the value $\eps$. Finding the middle-ground $\eps$ value is an important goal, which we leave for future work. 

Notice that the planner fails to find a solution for $\delta=4$ and $\eps\in\{2,4,5\}$. Due to our \decomps result, this implies that no $4$-clear solution exists. Despite this, the planner does succeed for some values of $\eps$, which suggests that our sufficient conditions for \decomps are not necessary. The success could also be explained by the specific arrangement of the points in $\XA$, which coincidentally induces a connected component via the second homotopy class for this specific scenario. It should also be noted that the sample set $\X_{\AN}^{4,\eps}$ can be viewed (via Lemma~1) as the sample set $\X_{\AN}^{2.5,\eps'}$ for $\eps$ small enough, which explains the success of the planner with  $\delta=4$ and smaller $\eps$ values. 

% Please add the following required packages to your document preamble:
% \usepackage{multirow}
% \usepackage[table,xcdraw]{xcolor}
% Beamer presentation requires \usepackage{colortbl} instead of \usepackage[table,xcdraw]{xcolor}


