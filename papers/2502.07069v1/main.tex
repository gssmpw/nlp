\documentclass[lettersize,journal]{IEEEtran}
%\IEEEoverridecommandlockouts
\usepackage{cite}
\usepackage{graphicx}
\usepackage{amsmath,amssymb,amsfonts}
\usepackage[center]{caption}
\usepackage[font=footnotesize]{caption}
\usepackage{xcolor}
\usepackage{algorithm,algorithmicx,algpseudocode}
\usepackage{bm}
\usepackage{mathtools}
\usepackage{nccmath}
\usepackage{enumitem}
\usepackage{array}
\definecolor{Green}{rgb}{0.13, 0.55, 0.13}

\usepackage{amsthm}
\newtheorem{theorem}{Theorem}
\newtheorem{proposition}{Proposition}
\newtheorem{lemma}{Lemma}
\newtheorem{definition}{Definition}

\DeclareMathOperator*{\argmin}{arg\,min}

\title{\huge Semantics-Aware Updates from Remote IoT Devices to Interconnected LEO Satellites}

\author{Erfan Delfani, and Nikolaos Pappas, 
		\IEEEmembership{Senior Member, IEEE}.
		\thanks{The authors are with the Department of Computer and Information Science at Linköping University, Sweden, email: \{\texttt{erfan.delfani, nikolaos.pappas\}@liu.se}. This work has been supported in part by the Swedish Research Council (VR), ELLIIT, and the EU (ETHER, 101096526, ELIXIRION, 101120135, and SOVEREIGN, 101131481).}}
        

\begin{document}

    %\setlength{\abovecaptionskip}{1pt}
    %\setlength{\belowcaptionskip}{3pt} 
    %\setlength{\textfloatsep}{-2pt}
	
	\maketitle
	
	\begin{abstract}
       Providing timely and informative data in Integrated Terrestrial and Non-Terrestrial Networks (T-NTNs) is critical as data volume continues to grow while the resources available on devices remain limited. To address this, we adopt a semantics-aware approach to optimize the Version Age of Information (VAoI) in a status update system in which a remote Energy Harvesting (EH) Internet of Things (IoT) device samples data and transmits it to a network of interconnected Low Earth Orbit (LEO) satellites for dissemination and utilization. The optimal update policy is derived through stochastic modeling and optimization of the VAoI across the network. The results indicate that this policy reduces the frequency of updates by skipping stale or irrelevant data, significantly improving energy efficiency.
    \end{abstract}

    \begin{IEEEkeywords}
    Semantics-aware communication, Status update, Version AoI, IoT, Energy harvesting, LEO, Satellite networks.
    \end{IEEEkeywords}

    %\vspace{-12pt}
    
    
	\section{Introduction}

    The integration of Terrestrial and Non-Terrestrial (T-NT) communication networks has unlocked new opportunities, enabling seamless extended coverage and enhanced remote communication and processing capabilities. Incorporating aerial and space platforms—such as Unmanned Aerial Vehicles (UAVs), High Altitude Platforms (HAPs), and satellites—provides critical infrastructure for connecting remote nodes in rural areas, islands, ships, and airplanes, facilitating global services\cite{ntontinether}. Among NT networks, LEO satellites are experiencing rapid growth, forming a dense web of interconnected nodes around the Earth. These satellites can connect directly to ground devices and gateways, serving as end-users or relays to support global caching, processing, and information management. However, the hardware, software, and energy constraints of satellite nodes make efficient resource management essential for sustaining network performance, particularly when handling large volumes of data. This challenge is further amplified when dealing with time-sensitive data and real-time decision-making from remote IoT devices. In such scenarios, \emph{the communication of timely and informative data plays a crucial role in the performance of the network.}  

    The \emph{semantics-aware communication} paradigm is a novel approach that optimizes the generation, transmission, and utilization of fresh and informative data \cite{kountouris2021semantics}. In this framework, semantic attributes such as \emph{timeliness}, \emph{relevance}, and \emph{value} are quantified using metrics like Age of Information (AoI)\cite{kaul2012real}, Age of Incorrect Information (AoII)\cite{maatouk2020age}, and VAoI \cite{yates2021Vage}. AoI measures data freshness but disregards its content. In contrast, AoII and VAoI account for content, with VAoI requiring only minimal knowledge of it. By optimizing these metrics, integrated T-NT networks can ensure the timely delivery of relevant information while minimizing the transmission of outdated or uninformative data. This approach not only enhances network efficiency but also significantly reduces energy consumption.

    Several studies have examined semantics-aware communication in LEO satellite networks, where the data may be utilized within the LEO network---for example, for distributed processing---or ultimately delivered to a ground destination node via dissemination in the LEO network.
    The works \cite{soret2020latency, chiariotti2022age} investigate AoI in multihop satellite networks under packet erasure channels and queuing policies, deriving closed-form bounds for both average and Peak AoI (PAoI). Path selection in dynamic LEO constellations with Inter-Satellite Link (ISL) interruptions is explored in \cite{li2022age}, where AoI is optimized under path and arrival rate constraints. Access control mechanisms for minimizing PAoI in GEO/LEO heterogeneous networks for IoT gateways are examined in \cite{cai2022age}. 
    %The study in \cite{lu2023analysis} models LEO satellite-ground service using an on-off process, providing a closed-form expression for average AoI. 
    Optimization of average AoI in LEO satellite-terrestrial integrated networks is addressed in \cite{gao2022non} through a Non-Orthogonal Multiple Access (NOMA)-based two-user scheme, with satellites offering orthogonal access for other users. Protocols for timely dual-hop status updates in Satellite IoT (SIoT) systems, leveraging relaying LEO satellites to enhance PAoI over fading channels, are developed in \cite{huang2023age, jiao2022age}. The work in \cite{xu2022age} introduces an age-optimal delivery protocol for two-hop SIoT links, incorporating data compression, transmission scheduling, and spatial-temporal correlations. In \cite{hong2023age}, AoII is employed as a content-based metric, formulating an AoII minimization problem under power, network stability, and freshness constraints in a downlink NOMA-based status update system. Lastly, \cite{liao2024information} proposes an LEO IoT architecture with edge intelligence for collaborative task processing, optimizing PAoI while considering terminal energy constraints.

    The aforementioned studies have predominantly emphasized the timeliness of information, particularly AoI and PAoI, while largely disregarding content-based metrics such as VAoI in LEO satellite networks. Additionally, they have primarily considered status updates from individual LEO satellites, failing to account for their interconnected topology.
    This work addresses these gaps with a semantics-aware approach to handling information from an energy-constrained IoT device to a network of interconnected LEO satellites arranged in a ring topology, a typical configuration enabled by permanent ISLs \cite{chaudhry2021laser}. We analyze VAoI at LEO nodes and optimize its average across the network by adopting a transmission policy at the IoT device, subject to energy constraints in an EH scenario. 

    \section{System Model}

    We consider a system model, as depicted in Fig. \ref{fig_SysModel}, where a remote EH IoT device measures and transmits status updates from an information source to a network of $(N+1)$ LEO satellites in an orbital plane.
    During a visibility window, the IoT device connects to a satellite, referred to as the Connected Satellite (CS), and transmits updates according to an \emph{update policy}, while adhering to the constraints imposed by the harvested energy stored in the device's battery. The update policy, denoted by $\pi$, decides whether the device transmits a fresh update to the CS, thereby consuming energy, or remains idle to conserve energy for future use. This decision, or \emph{action}, is made sequentially in each time slot along a slotted time axis.
    
    The device harvests energy from ambient sources and stores it in a battery with capacity $B$. The energy harvesting process follows a Bernoulli distribution with an arrival probability of $\beta$, which is commonly used as a general stochastic model\cite{dong2015near,delfani2024version}. Each transmission to the CS consumes one energy unit and occupies one time slot.
    Furthermore, we assume the updates are propagated among destination satellites using a bi-directional ring topology, where updates received at the CS are disseminated to all network nodes. 
    The ultimate objective is to derive an update policy that optimizes network performance, delivering timely and informative data while efficiently using energy, with the VAoI as the semantic performance metric.
    
    The VAoI is a semantic metric in status update systems that measures both the timeliness and relevance of information, reflecting how many versions the receiver lags behind the source as new content or \emph{versions} are generated \cite{yates2021Vage}. By labeling the new content at the source with version numbers, the VAoI at a destination node $D$ can be defined as follows:
    \begin{align}
        \Delta(t) \overset{\text{def}}{=} V_S(t) - V_D(t),
    \end{align}
    where $V_S(t)$ is the version stored at the source, and $V_D(t)$ denotes the version stored at node $D$ at time $t$. We assume that a new version at the source is generated with probability $p_g$ in each time slot, following a Bernoulli distribution.

    Each node retains only the freshest version of the data, discarding previous versions. When the update policy determines that an update should be transmitted, the CS collects the latest data. These updates are propagated to neighboring nodes, which subsequently relay them to their respective neighbors, continuing the dissemination process. We assume that transmissions between neighboring satellites via ISLs are deterministic, occurring error-free and at regular intervals. Each transmission is completed within a single time slot.

    \begin{figure}[t!]
		\centering
		\includegraphics[width=2.95in]{RingGossipingSat.eps}
		\caption{Status updates from an IoT device to a network of $(N+1)$ LEO satellites in an orbital plane.}
        \vspace{-8pt}
		\label{fig_SysModel}
    \end{figure}

    \section{Average VAoI within the Satellite Network}
    \label{Sec_SysSetup}

    Our objective is to model the average VAoI in the satellite network and optimize it by deriving an optimal update policy. Considering a time horizon $T$, the time-average VAoI of the $n$-th satellite and the average VAoI across the entire network, for a given update policy $\pi$, are defined as follows:
    \begin{align}
        \label{eqn_FiniteHorizonAverageVAoIn}
        \bar{\Delta}_{n,T}^\pi & \overset{\text{def}}{=} \frac{1}{T} \sum_{t=0}^{T-1} \mathbb{E} \left[ \Delta_n^\pi(t) \right] \\
        \label{eqn_FiniteHorizonAverageVAoI}
        \bar{\Delta}_T^\pi &\overset{\text{def}}{=} \frac{1}{N+1} \sum_{n=-\frac{N}{2}}^{\frac{N}{2}} \bar{\Delta}_{n,T}^\pi
    \end{align}
    where $\Delta_n^\pi(t)$ denotes the VAoI at the $n$-th satellite at time $t$ under the policy $\pi$.

    \begin{proposition}
    \label{Prop_VAoIn}
    The VAoI at the $n$-th satellite, which is $|n|$ hops away from the CS, is given by:
    \begin{align}
        \label{eqn_VAoIn}
        \Delta_{n}(t) = \zeta_{|n|}(t) +& \Delta_0(t-|n|),\\ &n \!\in\! \left\{-\frac{N}{2},-\frac{N}{2}\!+\!1,\cdots,\frac{N}{2}\!-\!1,\frac{N}{2}\right\}, \notag
    \end{align}
    where $\Delta_0(t)$ denotes the VAoI at the Connected Satellite, and $\zeta_{m}(t)$ represents a Binomial process with parameters $m$ and $p_g$, i.e., $\zeta_{m}(t) \sim Bin(m, p_g), \ \ m \in \{0,1,2, \cdots \}$,
    and $p_g$ is the version generation probability at the source.
    \end{proposition}

    \begin{proof}
    The VAoI at node $n$ in Fig. \ref{fig_SysModel}, which is $|n|$ hops away from CS, is given by $\Delta_{n}(t)\!=\!V_S(t) \!-\! V_{n}(t)$, where $V_S(t)$ is the stored version at the source at time $t$ and $V_{n}(t)$ is the stored version at node $n$ at time $t$. The current version at node $n$ is equal to the stored version at the CS in the $|n|$-th slot prior: $V_{n}(t)=V_0(t-|n|).$ Therefore, we can rewrite the VAoI at node $n$ as follows:
    \begin{align}
        \label{eqn_ProofVAoIn}
        \Delta_{n}(t) &= V_S(t) - V_0(t-|n|) \notag \\
        & = \underbrace{V_S(t) - V_S(t-|n|)}_{\zeta_{|n|}(t)} + \underbrace{V_S(t-|n|) - V_0(t-|n|)}_{\Delta_0(t-|n|)} \notag
    \end{align}
    where we have defined $\zeta_{|n|}(t) \overset{\text{def}}{=} V_S(t) - V_S(t-|n|)$. This process measures the number of version generations at the source during the past $|n|$ time slots. The version generation at the source in each time slot follows a Bernoulli distribution with parameter $p_g$. Consequently, the number of version generations at the source during $|n|$ time slots follows a Binomial distribution, i.e., $\zeta_{|n|}(t) \sim Bin(|n|, p_g).$
    \end{proof}

    Equation \eqref{eqn_VAoIn} in Proposition \ref{Prop_VAoIn} states that the VAoI at node $n$ at time $t$ is equal to the VAoI at the CS at $|n|$ time slots earlier (i.e., at $t - |n|$), plus the number of version changes that occurred during the last $|n|$ time slots.

    \begin{lemma}
    \label{Lemma_AvgVAoI}
    Under the update policy $\pi$, the average VAoI at the $n$-th node and across the entire network of LEO satellites over $T$ time slots are, respectively, given by:
    \begin{gather}
        \bar{\Delta}_{n,T}^\pi \!=\! |n|p_g \!+ \! \mathbb{E} \left[ \frac{1}{T} \! \sum_{t=0}^{T\!-\!1}  \Delta_0^\pi(t\!-\!|n|) \right], \\
        \label{eqn_FirstSimpAvgVAoI}
        \bar{\Delta}_T^\pi \!=\!  \frac{N(N\!+\!2)}{4(N\!+\!1)}p_g \!+\! \frac{1}{N\!+\!1} \!\!\sum_{n=-\frac{N}{2}}^{\frac{N}{2}} \! \mathbb{E} \left[ \frac{1}{T} \! \sum_{t=0}^{T\!-\!1}  \Delta_0^\pi(t\!-\!|n|) \right]\!.
    \end{gather}
    \end{lemma}

    \begin{proof}
    We can simplify \eqref{eqn_FiniteHorizonAverageVAoIn} using \eqref{eqn_VAoIn}:

    \begin{align}
        \bar{\Delta}_{n,T}^\pi &\!=\! \frac{1}{T} \sum_{t=0}^{T\!-\!1} \mathbb{E} \left[ \zeta_{|n|}(t) \!+\! \Delta_0^\pi \left(t\!-\!|n|\right) \right].
    \end{align}

    By substituting the expected value of the Binomial process $\zeta_{|n|}(t)$, i.e., $\mathbb{E} \left[ \zeta_{|n|}(t) \right] = |n|p_g$, the average VAoI at the $n$-th node can be further simplified:

    \begin{align}
        \label{eqn_ProofSimpAvgVAoIn}
        \bar{\Delta}_{n,T}^\pi &\!=\! |n| p_g \!+\! \frac{1}{T} \!\sum_{t=0}^{T\!-\!1} \mathbb{E} \left[ \Delta_0^\pi(t\!-\!|n|) \right]
    \end{align}
    and by substituting \eqref{eqn_ProofSimpAvgVAoIn} into \eqref{eqn_FiniteHorizonAverageVAoI}, we obtain the average VAoI for the entire network:
    \begin{align}
        \bar{\Delta}_T^\pi &\!=\!  \frac{1}{N\!+\!1} \!\!\!\sum_{n=-\frac{N}{2}}^{\frac{N}{2}} \!\!\left\{ |n| p_g \!+\! \frac{1}{T} \!\sum_{t=0}^{T\!-\!1} \mathbb{E} \left[ \Delta_0^\pi(t\!-\!|n|) \right] \right\} \\
        & \!=\!  \frac{N(N\!+\!2)}{4(N\!+\!1)}p_g \!+\! \frac{1}{T} \frac{1}{N\!+\!1} \!\sum_{n\!=\!-\!\frac{N}{2}}^{\frac{N}{2}} \! \sum_{t=0}^{T\!-\!1} \mathbb{E} \left[ \Delta_0^\pi(t\!-\!|n|) \right]\!. \notag \qedhere
    \end{align} 
    \end{proof}

    Lemma \ref{Lemma_AvgVAoI} shows that the average VAoI at a single node or across the network depends on the VAoI at the CS and the system parameters, $N$ and $p_g$. Thus, optimizing them through the optimal policy $\pi^\ast$ reduces to optimizing the VAoI at the CS. We proceed with the optimization of the Version VAoI across the entire network\footnote{We omit the optimization of the average VAoI at the $n$-th node, as it is directly obtained in a similar manner.} in the following section. 

    \begin{figure}[t!]
		\centering
		\includegraphics[width=2.9in]{ConnectedIoTmodel.eps}
        \vspace{6pt}
		\caption{System setup between the IoT device and the Connected Satellite.}
        \vspace{-8pt}
		\label{fig_SysModel_Single}
    \end{figure}

    \section{Optimization Problem}
    \label{sec_ProbFormulation}
    
    Our aim is to optimize the average VAoI in the network, as represented by \eqref{eqn_FirstSimpAvgVAoI}. For a large time horizon, i.e., when $T \gg \frac{N}{2}$, the time average of the VAoI process with small shifts remains constant; in other words:

    \begin{align}
        \label{eqn_InfHorizinApprox}
        \frac{1}{T} \sum_{t=0}^{T-1}  \Delta_0^\pi(t-|n|) \approx \frac{1}{T} \sum_{t=0}^{T-1}  \Delta_0^\pi(t), \quad |n| \leq \frac{N}{2}.
    \end{align}

   Thus, for large values of $T$, specifically for an infinite horizon, we can summarize the average VAoI of the network by substituting \eqref{eqn_InfHorizinApprox} into \eqref{eqn_FirstSimpAvgVAoI} as follows:

    
    \begin{align}
        \label{eqn_FirstSimpAvgVAoIapprox}
        \bar{\Delta}_T^\pi = \frac{N(N+2)}{4(N+1)}p_g + \mathbb{E} \left[ \frac{1}{T} \sum_{t=0}^{T-1}  \Delta_0^\pi(t) \right], \quad T \gg \frac{N}{2}.
    \end{align}

    Therefore, the optimization of the average VAoI in the network can be formulated as follows:

    \begin{align}
        \bar{\Delta}^\ast &\overset{\text{def}}{=} \min_{\pi \in \Pi} \lim_{T \rightarrow \infty} \bar{\Delta}_T^\pi, \\
        \bar{\Delta}^\ast &= \frac{N(N+2)}{4(N+1)}p_g + \underbrace{\min_{\pi \in \Pi} \lim_{T \rightarrow \infty} \mathbb{E} \left[ \frac{1}{T} \sum_{t=0}^{T-1}  \Delta_0^\pi(t) \right]}_{\bar{\Delta}_0^\ast},
    \end{align}
    where a standard infinite horizon average cost Markov Decision Process (MDP) problem is obtained:

    \begin{align}
        \mathcal{P}_1: \quad \bar{\Delta}_0^\ast = \min_{\pi \in \Pi} \lim_{T \rightarrow \infty} \frac{1}{T} \mathbb{E} \left[  \sum_{t=0}^{T-1}  \Delta_0^\pi(t) \! \mid \! s(0) \right],
    \end{align}
    where $s(0)$ denotes the initial state of the system variables, and $\Pi$ represents the set of all feasible policies. This problem involves a status update system in which the EH IoT device measures the information source's status and transmits it to the CS node over a wireless channel, which is typically error-prone, as shown in Fig. \ref{fig_SysModel_Single}. The success probability of each update packet in a time slot is denoted by $p_s$. The update policy $\pi$ is a sequence of actions over time, represented by $a(t)$, where $a(t)=1$ indicates a transmission at time $t$, and $a(t)=0$ denotes an idle action, conserving battery for later optimal use. An optimal policy, denoted by $\pi^\ast$, minimizes the average VAoI in problem $\mathcal{P}_1$. This optimal policy can be derived using dynamic programming algorithms such as the Relative Value Iteration Algorithm (RVIA)\cite{puterman2014markov} for the MDP problem $\mathcal{P}_1$, characterized by a tuple $<\mathcal{S},\mathcal{A},P,C>$, where $\mathcal{S}$ represents the state space, $\mathcal{A}$ denotes the action space, $P$ is the transition probability function, and $C$ is the cost function:

    \begin{itemize}[leftmargin=0.15in]
		\item \textit{States}: We define the state vector $s(t)\overset{\text{def}}{=}\left[b(t),\Delta(t)\right]^T \in \mathcal{S}$, where $b(t) \in \{0,1,2,\ldots,B\}$ represents the state of the device's battery, and $\Delta(t) \in \{0,1,2,\cdots,\Delta_{\text{max}}\}$ denotes the VAoI at the CS at time $t$. Here, we have omitted the subscript and superscript of $\Delta_0^\pi(t)$ for simplicity. We have also truncated the high values of VAoI because exceedingly stale data in the network becomes useless and is skipped by the system; thus, counting these higher values is unnecessary. The resulting state space, $\mathcal{S}$, is a finite set:
        \begin{align*}
            \mathcal{S} \overset{\text{def}}{=} \big\{(b,\Delta)\!: b \in \{0,1,\cdots,B\} , \Delta \in \{0,1,\cdots,\Delta_{\text{max}}\} \big\}.
        \end{align*}
        
		\item \textit{Actions}: At time $t$, the action $a(t)=0$ indicates remaining idle, while $a(t)=1$ corresponds to transmitting an update. The action $a(t)$ is restricted to $0$ whenever $b(t)=0$.
        
		\item \textit{Transition probabilities}: Given the following equation: 
		\begin{align}
			\label{TransProb_Eqn}
			P\left[s(t\!+\!1)| s(t),a(t)\right]\!&=\!P\left[b(t\!+\!1)|b(t),a(t)\right]  \\
			&\times  P\left[\Delta(t\!+\!1)|b(t),\Delta(t),a(t)\right], \notag
		\end{align}
		the transition probabilities are presented in Section \ref{sec_TransProb}.
        
		\item \textit{Cost function}: The transition cost function is equal to the VAoI, i.e., $C\big( s(t), a(t),s(t+1)\big)\overset{\text{def}}{=}\Delta(t+1)$.
	\end{itemize}

    \subsection{Transition Probabilities}
    \label{sec_TransProb}
    The transition probabilities between the system states are presented by introducing the following Bernoulli processes: the energy arrival process, $\mathit{\textit{e}}(t)$, the channel success process, $\mathit{\textit{c}}(t)$, and the version generation process, $\mathit{\textit{z}}(t)$, given by:
	\begin{align}
		\label{BernoulliProcesses}
		\begin{matrix}
			\mathit{\textit{e}}(t) \!=\!
			\begin{cases}
				1 & \text{w.p. } \beta, \\
				0 & \text{w.p. } \bar{\beta}, \\
			\end{cases} & 
			\mathit{\textit{c}}(t) \!=\! 
			\begin{cases}
				1 & \text{w.p. } p_s, \\
				0 & \text{w.p. } \bar{p}_s, \\
			\end{cases} \\
			\mathit{\textit{z}}(t) \!=\! 
			\begin{cases}
				1 & \text{w.p. } p_g, \\
				0 & \text{w.p. } \bar{p}_g. \\
			\end{cases}
		\end{matrix}
	\end{align}
    where $\bar{\beta}\overset{\text{def}}{=}1-\beta$, $\bar{p}_s\overset{\text{def}}{=}1-p_s$, and $\bar{p}_g\overset{\text{def}}{=}1-p_g$. 
	We can now describe the evolution of the states based on the explanation provided in Sections \ref{Sec_SysSetup} and \ref{sec_ProbFormulation}:
	\begin{align}
		b(t\!+\!1) &\!=\! \min \left\{ b(t) \!+\! \mathit{\textit{e}}(t) \!-\! a(t) ,B \right\}. \\
		\Delta(t\!+\!1) 
		& \!=\! \begin{cases}
			\mathit{\textit{z}}(t), \quad  a(t)\!=\!1 \text{ and } \mathit{\textit{c}}(t)\!=\!1, \\
			\min \! \left\{ \Delta(t) \!+\! \mathit{\textit{z}}(t) ,\Delta_{\text{max}} \right\}\!, \ \text{otherwise.}
		\end{cases}
	\end{align}
	
	The transition probabilities are subsequently determined using \eqref{TransProb_Eqn} and the following equations:
	\begin{align}
		P&\left[b(t\!+\!1)\big|b(t),a(t)\right] \\
        &\!=\! 
		\begin{cases}
			\beta & a(t)\!=\!0,\!\  b(t\!+\!1)\!=\!\min \!\left\{b(t)\!+\!1,\!B \right\}\!,\\
			\bar{\beta} & a(t)\!=\!0,\!\  b(t\!+\!1)\!=\!b(t),\\
			\beta & a(t)\!=\!1,\!\  b(t\!+\!1)\!=\!b(t),\\
			\bar{\beta} & a(t)\!=\!1,\!\  b(t\!+\!1)\!=\!b(t)\!-\!1.
		\end{cases} \notag
	\end{align}
	\begin{align}
		P&\left[\Delta(t\!+\!1) \big|b(t),\Delta(t),a(t)\right] \\
		&\!=\! 
		\begin{cases}
			p_g & a(t)\!=\!0,\  \Delta(t\!+\!1)\!=\!\min \{\Delta(t)\!+\!1,\Delta_{\text{max}} \},\\
			\bar{p}_g & a(t)\!=\!0,\  \Delta(t\!+\!1)\!=\!\Delta(t),\\
			p_g\bar{p}_s & a(t)\!=\!1,\  \Delta(t\!+\!1)\!=\!\min \{\Delta(t)\!+\!1,\Delta_{\text{max}} \},\\
			\bar{p}_g\bar{p}_s & a(t)\!=\!1,\  \Delta(t\!+\!1)\!=\!\Delta(t),\\
			p_gp_s & a(t)\!=\!1,\  \Delta(t\!+\!1)\!=\!1,\\
			\bar{p}_gp_s & a(t)\!=\!1,\  \Delta(t\!+\!1)\!=\!0.
		\end{cases} \notag
	\end{align}

    \begin{figure}[tb!]
		\centering
		\includegraphics[width=2.6in,trim={0cm 0cm 0cm 0.2cm}]{Actions1.eps}
		\caption{The structure of the optimal update policy.}
        \vspace{-8pt}
		\label{fig_OptimalActions}
    \end{figure}

    \section{Numerical Results}
    The optimal policy for the MDP problem $\mathcal{P}_1$ is derived using the RVIA. We compare the performance of this policy with two baselines: the Greedy policy, which transmits an update whenever energy arrives and the battery is not empty, and the Randomized Stationary (RS) policy, which transmits with probability $\alpha$ in each time slot, provided the battery is not empty. In all simulations, the system parameters are fixed at $N=64$, $p_g = 0.3$, $p_s = 0.5$, $B = 20$, $\Delta_{\text{max}} = 30$, and $T=3000$, unless stated otherwise. The expected values are averaged over $8000$ Monte Carlo iterations.

    \subsection{The Structure of the Optimal Policy}
    The derived optimal policy is depicted in Fig. \ref{fig_OptimalActions} for the energy arrival probability $\beta=0.1$. The red circles represent the idle action $a=0$, while the blue asterisks indicate the update action $a=1$. The optimal policy exhibits a threshold-based structure, where for each battery state $b$, the device remains idle until the VAoI at the CS reaches a certain threshold, after which it begins transmitting updates. This highlights an important aspect of semantics-aware information handling schemes: \emph{conserving energy for later usage when it is most needed.} This is crucial in severe energy-limited scenarios, as early battery depletion at low VAoI causes long update gaps, leading to high VAoI. Conversely, delaying depletion until VAoI is reasonably high mitigates excessive growth.

    \begin{figure}[tb!]
		\centering
		\includegraphics[width=2.6in,trim={0cm 0cm 0cm 0.2cm}]{AverageVAoIvsBeta.eps}
		\caption{The average VAoI for various policies vs. $\beta$.}
        \vspace{-8pt}
		\label{fig_VAoIvsBeta}
    \end{figure}

    \subsection{The Impact of Energy Arrival Probability $(\beta)$}

    Fig. \ref{fig_VAoIvsBeta} shows the average VAoI as a function of energy arrival probability $\beta$ for the optimal, Greedy, and RS policies, considering $\alpha$ values of $0.1$, $0.2$, and $0.3$. As $\beta$ increases, the average VAoI decreases, with the optimal policy performing best by using optimal update thresholds. The performance gap between the optimal and Greedy policies narrows as energy availability increases, converging toward the \emph{always update} policy. However, for energy-constrained devices with low energy arrival rates, this gap widens, making optimal actions crucial for a fresher, more informative system.
    
    In this simulation, the RS policy performs worse than the Greedy policy. Notably, the Greedy policy is a special case of the RS policy with $\alpha = 1$, meaning that increasing $\alpha$ causes the RS policy to converge toward the Greedy policy, as shown in Fig. \ref{fig_VAoIvsBeta}. When the energy arrival probability is high, energy is more often available. In such cases, the Greedy policy consumes all available energy, while the RS policy with $\alpha < 1$ sometimes skips updates, allowing the battery to store more energy. Consequently, the Greedy policy results in more updates, whereas the RS policy results in fewer. This explains why the RS policy achieves a worse average VAoI. In contrast, when the energy arrival probability is low, long intervals occur between consecutive energy arrivals. During these intervals, both the Greedy policy (transmitting at the first opportunity) and the RS policy (transmitting at a random time slot) deplete the battery, resulting in the same number of updates and similar performance.

    \textit{Remark:} An important result here is that to maintain a specific average VAoI in the network, for instance at a level of $8$, the optimal policy requires an energy arrival of $0.1$, compared to $0.2$ for the Greedy policy. This indicates that \emph{by employing a semantics-aware update policy, the energy consumption of the remote IoT device can be reduced to $50\%$}. The reduction arises from a lower number of updates, which consequently decreases the number of disseminations among satellites. This results in a substantial improvement in energy efficiency, thereby supporting more sustainable energy use and extending the system's lifetime.

    \begin{figure}[tb!]
		\centering
		\includegraphics[width=2.6in,trim={0cm 0cm 0cm 0.2cm}]{ApproximationErrorVsT1.eps}
		\caption{The impact of $T$ on the precise and approximate VAoI.}
        \vspace{-8pt}
		\label{fig_ApproxError}
	\end{figure}

    \subsection{The Impact of the Time Horizon $(T)$}
    In Section \ref{sec_ProbFormulation}, the optimization problem was formulated under the assumption that $T \gg \frac{N}{2}$. In this section, we analyze the impact of $T$ when it does not satisfy this assumption and, consequently, an approximation of VAoI is optimized. The curves corresponding to the precise and approximated average VAoI, given by Equations \eqref{eqn_FirstSimpAvgVAoI} and \eqref{eqn_FirstSimpAvgVAoIapprox}, respectively, are depicted in Fig. \ref{fig_ApproxError}. The resulting error is also illustrated as a percentage in red. The curves confirm that when $T \gg \frac{N}{2}$ the error becomes negligible. For $T \geq 2N$, the error decreases to below $10\%$, and for $T \geq 10N$, it decreases to below $2\%$.

    \section{Conclusion}
    We developed an optimal update policy to ensure timely and efficient transmission of informative updates from a ground IoT device to a network of interconnected LEO satellites. Within this network, updates are disseminated among nodes arranged in a ring topology. By formulating and optimizing the VAoI across the LEO network, the proposed policy reduces the transmission of stale and irrelevant updates, thereby enhancing energy efficiency.

    \bibliographystyle{IEEEtran}
    \bibliography{Refs}
    
\end{document}
