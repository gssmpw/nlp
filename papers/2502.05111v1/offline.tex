\section{Offline Token Preprocessing}
\label{sec:offline}
% \timothy{I think that ideally the section should start with the high-level flow of how the algorithm works. I'm not actually very clear on the details---I assume the transducers are being composed with a PDA? Is the PDA deterministic (I think an earlier version said it was)? If not, how do we actually run it during decoding? Shouldn't the composition involve a lot of preprocessing time?}

% In this section, we present how our approach efficiently precomputes the mapping from LLM tokens to sets of possible terminal sequences.
% 
Our approach starts by preprocessing the lexer to efficiently construct a lookup table that relates LLM tokens to terminal sequences (\autoref{sec:lexing}) and vice versa (\autoref{sec:realizable}). 
The preprocessed lexer is then used to analyze the parser to determine what terminal sequences are actually possible sequences in the grammar (\autoref{sec:parsing}).

% \loris{mini outline here with sec num}

\subsection{Lexer Preprocessing }
\label{sec:lexing}

% \loris{not sure if this is defined in sec 2, but I'll def here for now. Don't we need to also define that each terminal type is defined by a regex?}
Let $\alphabet$ denote the set of string characters, $\alphabet^\ast$ denote the set of strings over this alphabet, and $\terms$ denote the set of terminals (i.e., grammar-level tokens).
A \emph{lexer} is a function $\lexer$ that given an input string $w \in \alphabet^\ast$, returns a pair $(T_1\ldots T_k, w_r)$, where $T_1 \ldots T_k \in \terms^\ast$ is a sequence of terminals and $w_r \in \alphabet^\ast$ is the suffix of $w$ that has not been lexed (i.e., mapped to language terminals) yet. 

Typically lexers resolve ambiguity by making some simplifying assumptions that also help improve efficiency and avoid backtracking.
We use the same assumptions and describe them next.


\paragraph{Maximal Munch Principle and Lookahead}

Consider a language that contains two different tokens for the increment operator \texttt{++} and the addition operator \text{+}. 
% 
Although the input string \texttt{++} could be tokenized as two separate \texttt{+} addition tokens, in practice lexers prioritize the longer valid token to resolve ambiguity (and which usually captures the intended syntax of the programming language).
% 
This behavior is called the \emph{maximal munch principle}: the lexer matches the longest possible substring starting at the current position that aligns with a defined token pattern.

Under a strict interpretation of the maximal munch principle, if the lexer reaches the end of the input stream while processing a partial triple-quoted Python string \texttt{"""a"}, the lexer should tokenize the input as two strings \texttt{""} and \texttt{"a"}.
% 
However, supporting such cases would require either waiting until the end of the input string to produce any tokens or allowing backtracking.
% 
As such, in practice many lexers (including Python's) will raise an error and stop lexing if the scanned prefix cannot be tokenized as a single terminal.
%To avoid backtracking, many lexers would raise an error when tokenizing the string \texttt{"""a"} rather than backtracking to tokenize it as two strings. \loris{when is the error raised, and what is the error?}
% 
This greedy behavior disallows some strings, but guarantees that the lexer can resolve all tokenization ambiguities by inspecting only the next character at each step.
% \timothy{Should maybe add a sentence stating that this is not too far from how actual lexers work.}

\begin{definition}[1-lookahead]
A lexer $\lexer$ is \emph{1-lookahead} if for every string $w \in \alphabet^\ast$ and valid continuation $c \in \alphabet$ of $\sent$, whenever $\lexer(\sent) = (\term_1 \ldots \term_k, r)$ then $\lexer(\sent c)$ is either $(\term_1 \ldots \term_k \term_{k+1}, c)$ for some $T_{k+1} \in \terms$ or $(\term_1 \ldots \term_k, r c)$.
\end{definition}

Terminals are specified as a set of regular expressions. It is oftentimes convenient to work with a \textit{lexing automaton}, which is the finite state automaton (FSA) that accepts strings matching any terminal definition \cite{mcnaughton1960regular}.
We refer the reader to \autoref{app:fsa-definition} for formal definitions of the semantics of FSA, but recall that an FSA is a tuple $\automaton = (\Sigma, Q, q_0, \delta, F)$ where $\Sigma$ is the alphabet, $Q$ is the set of states with initial state $q_0\in Q$, $\delta$ contains transitions of the form $q \xrightarrow{c} q'$, and $F\subseteq Q$ is the set of final states.

% \loris{you need to end with fact that one can build FSA when you have 1-lookahead and point to example figure, or start next section with that. Next para just assumes you magically have an FSA}

\paragraph{Lexing Transducer}

\begin{figure}
\centering
{\footnotesize
\begin{tikzpicture}[
    shorten >=1pt,
    node distance=2.5cm and 2.5cm,
    on grid,
    auto,
    scale=1.0,
    every state/.style={minimum size=0.8cm}
] 
   \node[state,initial,accepting,initial text={}] (q_0)   {$q_0$}; 
   \node[state] (q_1) [right=of q_0] {$q_1$}; 
   \node[state] (q_2) [right=of q_1, yshift=30pt] {$q^{\texttt{B}}_2$}; 
   \node[state] (q_3) [right=of q_1, yshift=-30pt] {$q^{\texttt{C}}_3$};

    \path[->] 
    (q_0) edge node {\texttt{a}:$\epsilon$} (q_1)

    (q_1) edge [bend right=10] node[swap, pos=0.5] {\texttt{b}:$\epsilon$} (q_2) 
          edge [bend left=10] node[pos=0.5] {\texttt{c}:$\epsilon$} (q_3) 

    (q_2) edge [loop right] node[pos=0.15, above] {
            \texttt{b}:$\epsilon$
            } ()
          edge[bend right=10] node[swap, above=2pt] {
            \texttt{a}:\texttt{B}
          } (q_1)

    (q_3) edge [loop right] node[pos=0.15, above] {\texttt{c}:$\epsilon$} ()
          edge[bend left=10] node[swap, below=2pt] {\texttt{a}:\texttt{C}} (q_1)

;
\end{tikzpicture}
}
\caption{A lexing transducer $\transducer_\automaton$ derived from FSA $\automaton$ in \autoref{fig:overview}. }
\label{fig:char-transducer}
\end{figure}

A 1-lookahead maximal munch lexer can be defined from a lexing automaton as follows:
% Given an FSA that recognizes a given set of language tokens, the maximal munch principle with 1-lookahead operates as follows:
The input is processed character-by-character by transitioning through the FSA's states.
When no valid transition exists for the next character $c$, the lexer checks whether the current state corresponds to a valid language token. 
% 
If it does and the tokenizer has at this point produced a pair $(T_1\ldots T_k, w_r)$, the not-yet tokenized string $w_r$ is tokenized with token $T_{k+1}$ corresponding to the reached state, the FSA is reset to its initial state $q_0$ (and the tokenizer state $(T_1\ldots T_k T_{k+1}, \varepsilon)$ with the empty string $\varepsilon$), and the character $c$ is consumed as the starting character of a new token $q_0$. 
If the current state does not correspond to a valid token or if $c$ cannot be consumed at $q_0$, then $c$ is invalid. 
% 
Invalid characters inform what tokens should be masked during constrained decoding.

% A finite-state transducer (FST) extends an FSA with outputs, formally defined as a tuple $\transducer = (\alphabet, \terms, Q, q_0, \delta, F)$ where $\Sigma, Q, q_0$ and $F$ are as in FSA; $\Gamma$ is the output alphabet, and $\delta \subseteq Q \times (\Sigma \cup \{\epsilon\}) \times \Gamma^\ast \times Q$ is the set of transitions. 
%
% Each transition $(q, c, \term_1 \ldots \term_k, q') \in \delta$ means that $q$, $c$, $q'$ are as in FSA and the transition outputs the sequence $\term_1 \ldots \term_k$.

This process can be formalized as a finite-state transducer (FST), an extension of a finite-state automaton that can produce output terminals when reading characters. Given the original lexing automaton $\automaton$ representing valid tokens, we write $\transducer_\automaton$ to denote the \emph{lexing transducer}, the FST corresponding to $\automaton$.
% 
The construction of $\transducer_\automaton$ from $\automaton$ is formalized in \autoref{alg:fst-construction} in Appendix A, but at a high level the process simply adds transitions to handle cases where no valid transition exists for the current character $c$, outputting terminals and exiting final states. 
\autoref{fig:char-transducer} shows the lexing transducer derived from the FSA in \autoref{fig:overview}.

% \loris{next sence is broken, also you never point at figures}
% When the FSA cannot process $c$ from its current state, the transducer the token recognized up to that point, resets to the initial state $q_0$, and processes $c$ from $q_0$. 

% \begin{lemma}
% If $L_\automaton$ is 1-lookahead, then $\transducer_\automaton(\sent) = T_1 \ldots T_k$ if and only if $L_\automaton(\sent) = (T_1 \ldots T_k, r)$ for some suffix $r$.
% \end{lemma}

% \kh{TODO: Proof, by induction on $w$. }

% \loris{this para name should be more about LLM.}
\paragraph{LLM Token to Terminals}

\begin{figure}
\centering
{\footnotesize
\begin{tikzpicture}[
    shorten >=1pt,
    node distance=2.5cm and 3.0cm,
    on grid,
    auto,
    scale=1.0,
    every state/.style={minimum size=0.8cm}
] 
   \node[state,initial,accepting,initial text={}] (q_0)   {$q_\epsilon$}; 
   \node[state] (q_1) [right=of q_0] {$q_{\texttt{a}}$}; 
   \node[state] (q_2) [right=of q_1] {$q_{\texttt{ab}}$}; 
   % \node[state] (q_4) [below=of q_3] {$q^{(}_4$};
   % \node[state] (q_5) [below=of q_4] {$q^{)}_5$};
    \path[->] 
    (q_0) edge [bend left=10] node {$\epsilon$:\texttt{a}} (q_1)
          edge [loop above] node {\texttt{a}:\texttt{a}, \texttt{b}:\texttt{b}, \texttt{c}:\texttt{c}} ()
          % edge [bend right=30]  node[pos=0.78] {( / $\epsilon$} (q_4)
          % edge [bend right=30]  node[pos=0.81] {) / $\epsilon$} (q_5)
    (q_1) edge [bend left=10] node {\texttt{ab}:\texttt{b}, \texttt{ac}:\texttt{c}} (q_0)
          edge node {$\epsilon$:\texttt{b}} (q_2) 
    (q_2) edge [bend right=40] node[pos=0.4, swap] {\texttt{aba}:\texttt{a}} (q_0)
;
\end{tikzpicture}
}
\caption{Detokenizing transducer for vocabulary $\vocab=\{\texttt{a}, \texttt{b}, \texttt{c}, \texttt{ab}, \texttt{ac}, \texttt{aba}\}$.}
\label{fig:detokenizing}
\end{figure}

\begin{figure}
\centering
{\footnotesize
\begin{tikzpicture}[
    shorten >=1pt,
    node distance=2.5cm and 2.7cm,
    on grid,
    auto,
    scale=1.0,
    every state/.style={minimum size=0.8cm}
] 
   \node[state,initial,accepting,initial text={}] (q_0)   {$q_0$}; 
   \node[state] (q_1) [right=of q_0] {$q_1$}; 
   \node[state] (q_2) [right=of q_1, yshift=40pt] {$q^{\texttt{B}}_2$}; 
   \node[state] (q_3) [right=of q_1, yshift=-40pt] {$q^{\texttt{C}}_3$};
   % \node[state] (q_4) [below=of q_3] {$q^{(}_4$};
   % \node[state] (q_5) [below=of q_4] {$q^{)}_5$};
    \path[->] 
    (q_0) edge node {\texttt{a}:$\epsilon$, \texttt{aba}:\texttt{B}} (q_1)
          edge [bend left=40]  node[pos=0.4, swap] {\texttt{ab}:$\epsilon$} (q_2)
          edge [bend right=40]  node[pos=0.4] {\texttt{ac}:$\epsilon$} (q_3)
          % edge [bend right=60]  node[pos=0.3, right=-5pt] {
          %   \begin{tabular}{c} \texttt{+}/$\epsilon$, \\ \texttt{\{0+,00+\}}\\/\texttt{NUM} \end{tabular}
          % } (q_3)
          % edge [bend right=30]  node[pos=0.78] {( / $\epsilon$} (q_4)
          % edge [bend right=30]  node[pos=0.81] {) / $\epsilon$} (q_5)
    (q_1) edge [bend right=10] node[swap, pos=0.3] {\texttt{b}:$\epsilon$} (q_2) 
          edge [bend left=10] node[pos=0.3] {\texttt{c}:$\epsilon$} (q_3) 
    %       edge [loop above] node[pos=0.8, right] {\texttt{\{a,0,a0\}}/$\epsilon$} ()
    %       edge [bend left=80] node[swap, pos=0.65, left=-8pt] {
    %         \begin{tabular}{c} \texttt{\{0+,00+\}}\\/\texttt{NAME} \end{tabular}
    %       } (q_3)
    (q_2) edge [loop right] node[pos=0.15, above] {
            \begin{tabular}{c} \texttt{b}:$\epsilon$, \\ \texttt{ab}:\texttt{B} \end{tabular}
            } ()
          edge[bend right=10] node[swap, above=2pt] {
            \begin{tabular}{c} \texttt{a}:\texttt{B}, \\ \texttt{aba}:\texttt{BB} \end{tabular}
          } (q_1)
          edge [bend right=10] node[swap, pos=0.4] {\texttt{ac}:\texttt{B}} (q_3)
    %       edge [bend left=40] node[pos=0.2, below=2pt] {\texttt{EOS}/\texttt{NUM};$\$$} (q_0)
    (q_3) edge [loop right] node[pos=0.15, above] {
            \begin{tabular}{c} \texttt{c}:$\epsilon$, \\ \texttt{ac}:\texttt{C} \end{tabular}
            } ()
          edge [bend right=10] node[swap, pos=0.6] {\texttt{ab}:\texttt{C}} (q_2)
          edge[bend left=10] node[swap, below=2pt] {
            \begin{tabular}{c} \texttt{a}:\texttt{C}, \\ \texttt{aba}:\texttt{CB} \end{tabular}
          } (q_1)
    %       edge [bend right=10] node[swap, pos=0.3] {\texttt{0}/\texttt{+}} (q_2)
    %       edge [bend right=90] node[swap] {\texttt{a0}/\texttt{+}} (q_1)
    %       edge [loop below] node {
    %         \begin{tabular}{c} \texttt{+}/\texttt{+}, \{\texttt{0+},\texttt{00+}\}/\texttt{+};\texttt{NUM} \end{tabular}
    %       } ()
;
\end{tikzpicture}
}
\caption{A determinized token-level lexing transducer $\transducer_{\automaton \circ \vocab}$, which is formed by composing $\transducer_\vocab$ from \autoref{fig:detokenizing} and $\transducer_\automaton$ from \autoref{fig:char-transducer}.}
\label{fig:token-transducer}
\end{figure}



Processing LLM tokens character-by-character with the lexing transducer at runtime would incur significant overhead.
To address this problem, we instead construct a token-to-terminal transducer by composing the character-to-terminal lexing transducer with the detokenizing transducer introduced by \citet{koo2024automatabased}.
%\loris{is this diff example than one in sec2?}
A detokenizing transducer simply maps LLM tokens to their corresponding sequence of text characters (i.e., the input alphabet is $\vocab$ and the output alphabet is $\Sigma$).
% 
A detokenizing transducer is nondeterministic and can contain $\varepsilon$-transitions that produce outputs without consuming inputs.

\autoref{fig:detokenizing} illustrates the detokenizing transducer $\transducer_\vocab$ derived from the vocabulary $\vocab = \{\texttt{a}, \texttt{b}, \texttt{c}, \texttt{ab}, \texttt{ac}, \texttt{aba}\}$ in \autoref{fig:overview}.
Note that as an optimization common prefixes of tokens form a trie-like structure (e.g., state $q_\texttt{a}$ denotes the state reached when producing the first \texttt{a} for all tokens that start with that character), reducing computational overhead by reusing shared prefix computation.
\autoref{fig:token-transducer} depicts the combined token-level lexing transducer $\transducer_{\automaton \circ \vocab}=\transducer_\automaton \circ \transducer_{\vocab}$, which is the determinized FST capturing the sequential functional composition of the two transducers---i.e., the output of $\transducer_\vocab$ is fed as input to $\transducer_\automaton$.
This token-to-terminal transducer enables efficient lookup of valid next tokens and produced terminal symbols in each state.


\subsection{Realizable Terminal Sequences}
\label{sec:realizable}

Now that we have defined the formal machinery behind lexing, we are ready to explain how LLM tokens can be mapped to possible sequences of terminals.

When the transducer $\transducer_{\automaton \circ \vocab}$ in \autoref{fig:token-transducer} consumes the LLM token \texttt{aba} from the initial state $q_0$, it produces the terminal \texttt{B} and moves to state $q_1$. 
If we inspect the grammar $\grammar_{\texttt{BC}}$ in \autoref{fig:overview_cfg}, we can deduce that the parser, which receives as input sequences of tokens, expects/requires the next language token to be \texttt{C}.
% \loris{I don't follow next point, since I don't know what automaton I'm looking at}
Since $\transducer_{\automaton \circ \vocab}$ in state $q_1$ does not immediately produce any output when consuming \texttt{b}, 
the generated terminal sequence so far (i.e., \texttt{B}) is still a valid prefix according to the grammar $\grammar_{\texttt{BC}}$.
However, after transitioning to $q_2$, no possible path can generate \texttt{C} next.

As illustrated by the above example, for each transition $q \xrightarrow{t:T_1 \ldots T_k} q'$ in the lexing transducer $\transducer_{\automaton \circ \vocab}$, we should check whether there is a terminal $T$ such that \rone  $T$ can be generated along some path from $q'$, and \rtwo $T_1 \ldots T_k T$ is accepted by the grammar.
%
This observation leads to the following definition, which describes which terminal sequences need to be considered by the parser.

\begin{definition}[Realizable Terminal Sequences]
\label{def:realizable-term-seq}
Given a token vocabulary $\vocab$ and FSA $\automaton$, let $\delta$ be the set of transitions in the token-level lexing transducer $\transducer_\automaton \circ \transducer_\vocab$.
% 
The set of \textit{realizable terminal sequences} $\realizable{\vocab}{\automaton}$ is defined as 
\begin{multline*}
\realizable{\vocab}{\automaton} = \{ T_1 \ldots T_k T \mid q \xrightarrow{t:T_1 \ldots T_k} q' \in \delta \textrm{ and } \\
T \textrm{ can be generated along some path from } q' \}.
\end{multline*}
\end{definition}

Note that the LLM vocabulary $\vocab$ contains all characters in $\alphabet$, ensuring that $\alphabet^\ast = \vocab^\ast$. 
Therefore, any next terminal producible in $\transducer_\automaton$ is also producible in the combined transducer $\transducer_{\automaton \circ \vocab}$ and vice versa. 
This equivalence allows us to simplify producibility checks: instead of analyzing the large combined transducer $\transducer_{\automaton \circ \vocab}$, we need only compute reachability to accepting states within $\transducer_\automaton$ to determine producible next terminals.

\paragraph{Inverse Token Spanner Table}
\setcounter{algorithm}{3}
% 
\begin{algorithm}[!t]
\caption{\tname{BuildInverseTokenSpannerTable}}
\label{alg:inverse-token-table}
\begin{algorithmic}
    \STATE {\bfseries Input:} Combined FST $\transducer_\automaton \circ \transducer_\vocab = (\vocab, \terms, Q, q_0, \delta, F)$
    \STATE {\bfseries Output:} Realizable token sequences $\realizable{\vocab}{\automaton}$, \\
    Inverse lookup table $\inversetable: \realizable{\vocab}{\automaton} \times Q \rightarrow 2^\vocab $
    \STATE $\realizable{\vocab}{\automaton} := \emptyset$, $\inversetable(\cdot, \cdot) := \emptyset$
    \FOR{$q \xrightarrow{t:T_1 \ldots T_k} q' \in \delta$}
        \FOR{$T$ recognized at $q''$, that is reachable from $q'$}
            \STATE $\realizable{\vocab}{\automaton} := \realizable{\vocab}{\automaton} \cup \{ T_1 \ldots T_k T \}$
            \STATE $\inversetable(q, T_1 \ldots T_k T) := \inversetable(q, T_1 \ldots T_k T) \cup \{t\}$
        \ENDFOR
    \ENDFOR
    \RETURN $\realizable{\vocab}{\automaton}$, $\inversetable$
\end{algorithmic}
\end{algorithm}

\autoref{alg:inverse-token-table} computes the set of realizable terminal sequences and constructs the key data structure we use to perform constrained decoding.
\begin{definition}[Inverse Token Spanner Table]
Given a lexer state $q \in Q_{\automaton \circ \vocab}$ and $T_1 \ldots T_k T \in \realizable{\vocab}{\automaton}$, the entry $\inversetable(q, \alpha)$ in the \textit{inverse token spanner table} $\inversetable$ is the set of tokens that generates $T_1 \ldots T_kT$ from state $q$. Formally,
\begin{multline*}
\inversetable(q, T_1 \ldots T_k T) = \{ t \mid q \xrightarrow{t:T_1 \ldots T_k} q' \in \delta \textrm{ and } \\
T \textrm{ can be generated along some path from } q' \}
\end{multline*}    
\end{definition}

% \loris{explain what it does and define}
This table allows a decoder to determine which LLM tokens can result in a given realizable terminal sequence $T_1 \ldots T_k T$ being produced from a specific lexer state.

For the example of \autoref{fig:overview}, the terminal sequence \texttt{BC} is generated by the token \texttt{aba} in state $q_0$ (i.e., $\texttt{aba} \in \inversetable(q_0, \texttt{BC})$). The same sequence is generated by the token \texttt{ac} in state $q_2$ (i.e., $\texttt{ac} \in \inversetable(q_2, \texttt{BC})$).
% \loris{should the two above be more like $\texttt{ac}\in \inversetable(q_2, \texttt{BC})$}


% \khchanged{
% Given a transition relation $\delta$ of transducer $\transducer$, we define the extended transition $\delta^\ast$ as the smallest set defined by \rone $q \xrightarrow{\epsilon:\epsilon}^\ast q \in \delta^\ast$ and \rtwo $q \xrightarrow{wc:\alpha \alpha'}^{\ast} q' \in \delta^\ast$ whenever $q \xrightarrow{w:\alpha}^{\ast} q'' \in \delta^\ast$ and $q'' \xrightarrow{c:\alpha'} q' \in \delta^\ast$.
% }

% \khchanged{
% \begin{theorem}[Soundness]
% Assume $\alphabet \subseteq \vocab$.
% Given a state $q \in Q_{\automaton \circ \vocab}$ and a terminal sequences $T_1 \ldot T_k T \in  \realizable{\vocab}{\automaton}$, 
% for each token $t \in \inversetable(q, T_1 \ldots T_k T)$, there exists $t_1 \ldots t_m \in \vocab^\ast$
% \end{theorem}
% }

% \loris{point to example table in fig1 as example}

\subsection{Parser Preprocessing}
\label{sec:parsing}

We do not formally define context-free grammars (CFG) for brevity and refer the reader to \autoref{app:cfg-definition} for details.
% 
For the sake of this paper, the reader only needs to know that a CFG parser is typically formalized as a pushdown automaton (PDA), an extension of FSA with an execution stack \cite{schutzenberger1963context}.
The definition of a PDA is similar to that of an FSA, but transitions are also allowed to manipulate a stack over symbols $\Pi$ via push and pop operations.
Therefore, in a PDA, a configuration after reading a sequence of input characters is a pair of automaton state $q\in Q$ and execution stack $\gamma\in \Pi^\ast$.
 
We refer the reader to \autoref{app:pda-definition} for the formal definition of a PDA, but informally, each PDA transition $q \xrightarrow{c[\beta \rightarrow \beta']} q'$ can only be activated if the character being read is a $c$ and the top of the current stack configuration $\gamma$ matches to sequence of stack symbols $\beta \in \Pi^\ast$. 
If the transitions is activated, the current state becomes $q'$, and the top $\beta$ elements of the stack are replaced with new stack elements $\beta' \in \Pi^\ast$.
% Notably, widely used LR($k$) and LALR($k$) parsers can be formalized as deterministic PDAs \cite{knuth1965translation}.}

As with lexer preprocessing, our goal is to also preprocess the parser to avoid iterating over every terminal sequence generated by the lexer at runtime.
To achieve this objective, we
directly compose the detokenizing transducer $\realizable{\vocab}{\automaton}$ (\Cref{def:realizable-term-seq}), with the PDA $\pushdown$ produced by a parser generator (and where transitions operate over single terminals)~\cite{allauzen2012pushdown} to obtain a new pushdown automaton $\pushdown \circ \transducer_{\realizable{\vocab}{\automaton}}$ where transitions operate over terminal sequences. 

This last transducer can efficiently determine valid sequences of terminal symbols from each parser state.
% 
We note that preprocessing both the lexer and parser in tandem is a key feature that distinguishes our work from prior work~\cite{beurer2024domino, ugare2024syncode}.

% \loris{should the cardinality beo over T and not RE? I want to move this para a few lines before, let me know once fixed}
% \kh{RE is the range of T (plus reachability), so it is correct I think? }
One key source of efficiency resulting from our approach is that many transitions in the combined transducer $\transducer_{\automaton \circ 
 \vocab}$ produce the same terminal sequence $T_1 \ldots T_k T$, making $|\realizable{\vocab}{\automaton}|$ smaller than $|\vocab|$ or $|\delta|$.
%\khchanged{
% In particular, this set is larger than the set of terminal sequences realizable from any specific state $q$, but not significantly larger than the largest such set from a single state.
% In particular, the \emph{global} set of realizable terminal sequences $\realizable{\vocab}{\automaton}$ enables efficient precomputation of what sequences of terminals the parser should consider, instead of considering realizable terminal sequences \emph{for each} lexer state $q$.}
%
Thus, the set of realizable terminal sequences $\realizable{\vocab}{\automaton}$ enables efficient precomputation of what sequences of terminals the parser should consider.

% \loris{describe first what is the high level thing you are trying to build and start by saying that a parser uses a PDA (you have never said this)}
%

% Specifically, we construct a detokenizing transducer that captures all possible output sequence of terminal symbols.
% \loris{will the PDA receive single tokens or seq of tokens now?}



\paragraph{Context Dependent/Independent Tokens}

Because pushdown automata need a stack to be able to parse arbitrary context-free grammars, we cannot decide entirely at preprocessing time whether a given terminal sequence can be accepted by a PDA at any given state---one has to inspect the content of the stack at execution time.
% 
However, many terminal sequences can be identified as always accepted or always rejected, independent of the current execution stack.

\textit{Stack invariance} provides a sufficient condition for knowing when a sequence of terminals is accepted.

% \loris{this is technical but we have no def of PDAs earlier. You can just have an informal def above where you say a PDA is like an FSA but can also push/pop on stack, therefore configs are pairs (q, gamma) and a push adds to gamma and a pop inspects top of gamma and pops}
\begin{proposition}[Stack Invariance]
If a PDA $\pushdown$ accepts an input sequence $w$ in state $q$ with stack configuration $\gamma$, 
then $w$ is also accepted in the same state $q$ when the stack configuration is $\gamma' \cdot \gamma$ for some $\gamma'$ 
(i.e., when $\gamma$ appears at the top of the stack with additional symbols beneath it).
\end{proposition}

% The proof follows from the fact that stack operations are restricted to the top of the stack, 
% and hence the PDA never attempts to modify or inspect symbols below $\gamma$.


It follows that a terminal sequence accepted by the PDA starting with an empty stack configuration is accepted under any stack configuration. 

The following proposition shows how to construct a stack-free finite-state automaton that overapproximates the set of sequences accepted by a PDA.
\begin{proposition}[Overapproximation via FSA]
\label{prop:overapprox-fsa}
Consider an FSA $\automaton_\pushdown$ obtained by removing all stack operations from a PDA $\pushdown$.
If an input sequence $w$ is not accepted by $\automaton_\pushdown$ in state $q$, then $w$ cannot be accepted by $\pushdown$ in state $q$ with any stack configuration.
\end{proposition}
% Since the FSA $\automaton$ is obtained by removing all stack operations from the PDA $\pushdown$, 
% any input sequence $w$ that is accepted by $\pushdown$ in some state $q$ with stack configuration $\gamma$ must also be accepted by the $\automaton$ in state $q$.

% Not using as it says theorem instead of prop
Following Proposition~\ref{prop:overapprox-fsa}, a terminal sequence is always rejected if it is rejected by the FSA obtained by removing all stack operations.
% 


The above reasoning can be formalized by computing the set of \textit{always-accepted tokens} $A$ and of \textit{context-dependent terminal sequences} $D$.
%
Given a lexer state $q^\automaton$ and a parser state $q^\pushdown$, we denote by $A(q^\automaton, q^\pushdown)$ the set of tokens that are accepted regardless of the stack configuration $\gamma$, and by $D(q^\automaton, q^\pushdown)$ the set of terminal sequences that may be accepted by some configuration $\gamma$. 
%\loris{I thought you were computing the for sure rejected?}
%\kh{that is only used to compute D}

\autoref{alg:preprocess-parser} in Appendix A describes how to preprocess a parser to build a table of always-accepted tokens $A$ and context-dependent sequences $D$.

\setcounter{algorithm}{5}

% \paragraph{Contextual Lexing}

% One challenge in lexical analysis arises when the same input sequence must be interpreted differently based on parsing context. 
% For example, in Java, the end of the nested generic \texttt{List<List<T>{}>} could be erroneously tokenized by a lexer operating under maximal munch as $\texttt{>>}$, the right-shift operator. However, in this context the parser instead expects two consecutive \texttt{>} terminals denoting the closure of a generic type. 
% To resolve this ambiguity, lexer must consider the parser’s current state. 
% One approach involves allowing the lexer to nondeterministically generate both possible tokenizations (e.g., a single \texttt{>{}>} terminal or two separate \texttt{>} terminals) specifically for such cases, enabling the parser to select the valid interpretation based on grammatical constraints.