%%%%%%%% ICML 2025 EXAMPLE LATEX SUBMISSION FILE %%%%%%%%%%%%%%%%%
\pdfoutput=1
\documentclass{article}

% hyperref makes hyperlinks in the resulting PDF.
% If your build breaks (sometimes temporarily if a hyperlink spans a page)
% please comment out the following usepackage line and replace
% \usepackage{icml2025} with \usepackage[nohyperref]{icml2025} above.
\usepackage{hyperref}

% Attempt to make hyperref and algorithmic work together better:
\newcommand{\theHalgorithm}{\arabic{algorithm}}

% Use the following line for the initial blind version submitted for review:
% \usepackage{icml2025}

% If accepted, instead use the following line for the camera-ready submission:
\usepackage[accepted]{icml2025}

\usepackage[utf8]{inputenc} % allow utf-8 input
\usepackage[T1]{fontenc}    % use 8-bit T1 fonts
\usepackage{url}            % simple URL typesetting
\usepackage{booktabs}       % professional-quality tables
\usepackage{amsfonts}       % blackboard math symbols
\usepackage{amsmath}
\usepackage{amssymb} % For semantic brackets
\usepackage{amsthm}
\usepackage{stmaryrd}
\usepackage[scaled=0.75]{beramono}
\usepackage{wrapfig}

% \usepackage{emoji}
\usepackage{nicefrac}       % compact symbols for 1/2, etc.
\usepackage{microtype}      % microtypography
\usepackage{xcolor}         % colors
\usepackage{algorithm}
% \usepackage[noend]{algpseudocode}
\usepackage{listings}
\usepackage{mathtools}
\usepackage{multirow}
\usepackage{multicol}
\usepackage{tcolorbox}
\newtheorem{exmp}{Example}[section]
\usepackage{color, soul}
\definecolor{grey}{rgb}{0.9,0.9,0.9}
\usepackage{balance}
\usepackage{textcomp}
% \usepackage{cleveref}
\newcommand{\ipa}{\textipa}

\usepackage{listings} 
\usepackage{alltt}
\usepackage{latexsym}
\usepackage{xspace}
\usepackage{url}
\usepackage{paralist}
\usepackage{pdfpages}
\usepackage{listings} 
\usepackage{alltt}
% \usepackage{hyperref}
\usepackage{booktabs}  
\usepackage{subcaption} 
\usepackage{url}
\usepackage{hhline}
\usepackage{multicol}
\usepackage{makecell}
% \usepackage{todonotes}
\usepackage[font={small}]{caption}
\usepackage[inline,shortlabels]{enumitem}
\usepackage{color}
\usepackage[export]{adjustbox}
% \usepackage{natbib}
\usepackage{bbm}
\usepackage[symbol]{footmisc}

% if you use cleveref..
\usepackage[capitalize,noabbrev]{cleveref}

% Automata
\usepackage{tikz}
\usetikzlibrary{automata, positioning}

%%%%%%%%%%%%%%%%%%%%%%%%%%%%%%%%
% THEOREMS
%%%%%%%%%%%%%%%%%%%%%%%%%%%%%%%%
\theoremstyle{plain}
\newtheorem{theorem}{Theorem}[section]
\newtheorem{proposition}[theorem]{Proposition}
\newtheorem{lemma}[theorem]{Lemma}
\newtheorem{corollary}[theorem]{Corollary}
\theoremstyle{definition}
\newtheorem{definition}[theorem]{Definition}
\newtheorem{assumption}[theorem]{Assumption}
\theoremstyle{remark}
\newtheorem{remark}[theorem]{Remark}

% Todonotes is useful during development; simply uncomment the next line
%    and comment out the line below the next line to turn off comments
%\usepackage[disable,textsize=tiny]{todonotes}
\usepackage[textsize=tiny]{todonotes}


% The \icmltitle you define below is probably too long as a header.
% Therefore, a short form for the running title is supplied here:
% \icmltitlerunning{Efficient Structured Generation for Large Language Models with Complex Grammars}

\usepackage{bbm}
\usepackage{graphicx}
\usepackage{amsmath,amssymb,amsthm,amsfonts}

\usepackage{paralist}
\usepackage{bm}
\usepackage{xspace}
\usepackage{url}
\usepackage{prettyref}
\usepackage{boxedminipage}
\usepackage{wrapfig}
\usepackage{ifthen}
\usepackage{color}
\usepackage{xspace}

\newcommand{\ii}{{\sc Indicator-Instance}\xspace}
\newcommand{\midd}{{\sf mid}}


\usepackage{amsmath,amsthm,amsfonts,amssymb}
\usepackage{mathtools}
\usepackage{graphicx}


% \usepackage{fullpage}

\usepackage{nicefrac}

\newtheorem{inftheorem}{Informal Theorem}
\newtheorem{claim}{Claim}
\newtheorem*{definition*}{Definition}
\newtheorem{example}{Example}

\DeclareMathOperator*{\argmax}{arg\,max}
\DeclareMathOperator*{\argmin}{arg\,min}
\usepackage{subcaption}

\newtheorem{problem}{Problem}
\usepackage[utf8]{inputenc}
\newcommand{\rank}{\mathsf{rank}}
\newcommand{\tr}{\mathsf{Tr}}
\newcommand{\tv}{\mathsf{TV}}
\newcommand{\opt}{\mathsf{OPT}}
\newcommand{\rr}{\textsc{R}\space}
\newcommand{\alg}{\textsf{Alg}\space}
\newcommand{\sd}{\textsf{sd}_\lambda}
\newcommand{\lblq}{\mathfrak{lq} (X_1)}
\newcommand{\diag}{\textsf{diag}}
\newcommand{\sign}{\textsf{sgn}}
\newcommand{\BC}{\texttt{BC} }
\newcommand{\MM}{\texttt{MM} }
\newcommand{\Nexp}{N_{\mathrm{exp}}}
\newcommand{\Nrep}{N_{\mathrm{replay}}}
\newcommand{\Drep}{D_{\mathrm{replay}}}
\newcommand{\Nsim}{N_{\mathrm{sim}}}
\newcommand{\piBC}{\pi^{\texttt{BC}}}
\newcommand{\piRE}{\pi^{\texttt{RE}}}
\newcommand{\piEMM}{\pi^{\texttt{MM}}}
\newcommand{\mmd}{\texttt{Mimic-MD} }
\newcommand{\RE}{\texttt{RE} }
\newcommand{\dem}{\pi^E}
\newcommand{\Rlint}{\mathcal{R}_{\mathrm{lin,t}}}
\newcommand{\Rlipt}{\mathcal{R}_{\mathrm{lip,t}}}
\newcommand{\Rlin}{\mathcal{R}_{\mathrm{lin}}}
\newcommand{\Rlip}{\mathcal{R}_{\mathrm{lip}}}
\newcommand{\Rmax}{R_{\mathrm{max}}}
\newcommand{\Rall}{\mathcal{R}_{\mathrm{all}}}
\newcommand{\Rdet}{\mathcal{R}_{\mathrm{det}}}
\newcommand{\Fmax}{F_{\mathrm{max}}}
\newcommand{\Nmax}{\mathcal{N}_{\mathrm{max}}}
\newcommand{\piref}{\pi^{\mathrm{ref}}}
\newcommand{\green}{\text{\color{green!75!black} green}\;}
\newcommand{\thetaBC}{\widehat{\theta}^{\textsf{BC}}}
\newcommand{\ent}{\mathcal{E}_{\Theta,n,\delta}}
\newcommand{\eNt}{\mathcal{E}_{\Theta_t,\Nexp,\delta}}
\newcommand{\eNtH}{\mathcal{E}_{\Theta_t,\Nexp,\delta/H}}

\newcommand{\eref}[1]{(\ref{#1})}
\newcommand{\sref}[1]{Sec. \ref{#1}}
\newcommand{\dr}{\widehat{d}_{\mathrm{replay}}}
\newcommand{\figref}[1]{Fig. \ref{#1}}

\usepackage{xcolor}
\definecolor{expert}{HTML}{008000}
\definecolor{error}{HTML}{f96565}
\newcommand{\GKS}[1]{{\textcolor{violet}{\textbf{GKS: #1}}}}
\newcommand{\Q}[1]{{\textcolor{red}{\textbf{Question #1}}}}
\newcommand{\ZSW}[1]{{\textcolor{orange}{\textbf{ZSW: #1}}}}
\newcommand{\JAB}[1]{{\textcolor{teal}{\textbf{JAB: #1}}}}
\newcommand{\jab}[1]{{\textcolor{teal}{\textbf{JAB: #1}}}}
\newcommand{\SAN}[1]{{\textcolor{blue}{\textbf{SC: #1}}}}
\newcommand{\scnote}[1]{\SAN{#1}}
\newcommand{\norm}[1]{\left\lVert #1 \right\rVert}

\usepackage{color-edits}
\addauthor{sw}{blue}

\usepackage{thmtools}
\usepackage{thm-restate}

\usepackage{tikz}
\usetikzlibrary{arrows,calc} 
\newcommand{\tikzAngleOfLine}{\tikz@AngleOfLine}
\def\tikz@AngleOfLine(#1)(#2)#3{%
\pgfmathanglebetweenpoints{%
\pgfpointanchor{#1}{center}}{%
\pgfpointanchor{#2}{center}}
\pgfmathsetmacro{#3}{\pgfmathresult}%
}

\declaretheoremstyle[
    headfont=\normalfont\bfseries, 
    bodyfont = \normalfont\itshape]{mystyle} 
\declaretheorem[name=Theorem,style=mystyle,numberwithin=section]{thm}

% \usepackage{algorithm}
% \usepackage{algorithmic}
\usepackage[linesnumbered,algoruled,boxed,lined,noend]{algorithm2e}

\usepackage{listings}
\usepackage{amsmath}
\usepackage{amsthm}
\usepackage{tikz}
\usepackage{caption}
\usepackage{mdwmath}
\usepackage{multirow}
\usepackage{mdwtab}
\usepackage{eqparbox}
\usepackage{multicol}
\usepackage{amsfonts}
\usepackage{tikz}
\usepackage{multirow,bigstrut,threeparttable}
\usepackage{amsthm}
\usepackage{bbm}
\usepackage{epstopdf}
\usepackage{mdwmath}
\usepackage{mdwtab}
\usepackage{eqparbox}
\usetikzlibrary{topaths,calc}
\usepackage{latexsym}
\usepackage{cite}
\usepackage{amssymb}
\usepackage{bm}
\usepackage{amssymb}
\usepackage{graphicx}
\usepackage{mathrsfs}
\usepackage{epsfig}
\usepackage{psfrag}
\usepackage{setspace}
\usepackage[%dvips,
            CJKbookmarks=true,
            bookmarksnumbered=true,
            bookmarksopen=true,
%						bookmarks=false,
            colorlinks=true,
            citecolor=red,
            linkcolor=blue,
            anchorcolor=red,
            urlcolor=blue
            ]{hyperref}
%\usepackage{algorithm}
\usepackage[linesnumbered,algoruled,boxed,lined]{algorithm2e}
\usepackage{algpseudocode}
\usepackage{stfloats}
\RequirePackage[numbers]{natbib}

\usepackage{comment}
\usepackage{mathtools}
\usepackage{blkarray}
\usepackage{multirow,bigdelim,dcolumn,booktabs}

\usepackage{xparse}
\usepackage{tikz}
\usetikzlibrary{calc}
\usetikzlibrary{decorations.pathreplacing,matrix,positioning}

\usepackage[T1]{fontenc}
\usepackage[utf8]{inputenc}
\usepackage{mathtools}
\usepackage{blkarray, bigstrut}
\usepackage{gauss}

\newenvironment{mygmatrix}{\def\mathstrut{\vphantom{\big(}}\gmatrix}{\endgmatrix}

\newcommand{\tikzmark}[1]{\tikz[overlay,remember picture] \node (#1) {};}

%% Adapted form https://tex.stackexchange.com/questions/206898/braces-for-cases-in-tabular-environment/207704#207704
\newcommand*{\BraceAmplitude}{0.4em}%
\newcommand*{\VerticalOffset}{0.5ex}%  
\newcommand*{\HorizontalOffset}{0.0em}% 
\newcommand*{\blocktextwid}{3.0cm}%
\NewDocumentCommand{\InsertLeftBrace}{%
	O{} % #1 = draw options
	O{\HorizontalOffset,\VerticalOffset} % #2 = optional brace shift options
	O{\blocktextwid} % #3 = optional text width
	m   % #4 = top tikzmark
	m   % #5 = bottom tikzmark
	m   % #6 = node text
}{%
	\begin{tikzpicture}[overlay,remember picture]
	\coordinate (Brace Top)    at ($(#4.north) + (#2)$);
	\coordinate (Brace Bottom) at ($(#5.south) + (#2)$);
	\draw [decoration={brace, amplitude=\BraceAmplitude}, decorate, thick, draw=black, #1]
	(Brace Bottom) -- (Brace Top) 
	node [pos=0.5, anchor=east, align=left, text width=#3, color=black, xshift=\BraceAmplitude] {#6};
	\end{tikzpicture}%
}%
\NewDocumentCommand{\InsertRightBrace}{%
	O{} % #1 = draw options
	O{\HorizontalOffset,\VerticalOffset} % #2 = optional brace shift options
	O{\blocktextwid} % #3 = optional text width
	m   % #4 = top tikzmark
	m   % #5 = bottom tikzmark
	m   % #6 = node text
}{%
	\begin{tikzpicture}[overlay,remember picture]
	\coordinate (Brace Top)    at ($(#4.north) + (#2)$);
	\coordinate (Brace Bottom) at ($(#5.south) + (#2)$);
	\draw [decoration={brace, amplitude=\BraceAmplitude}, decorate, thick, draw=black, #1]
	(Brace Top) -- (Brace Bottom) 
	node [pos=0.5, anchor=west, align=left, text width=#3, color=black, xshift=\BraceAmplitude] {#6};
	\end{tikzpicture}%
}%
\NewDocumentCommand{\InsertTopBrace}{%
	O{} % #1 = draw options
	O{\HorizontalOffset,\VerticalOffset} % #2 = optional brace shift options
	O{\blocktextwid} % #3 = optional text width
	m   % #4 = top tikzmark
	m   % #5 = bottom tikzmark
	m   % #6 = node text
}{%
	\begin{tikzpicture}[overlay,remember picture]
	\coordinate (Brace Top)    at ($(#4.west) + (#2)$);
	\coordinate (Brace Bottom) at ($(#5.east) + (#2)$);
	\draw [decoration={brace, amplitude=\BraceAmplitude}, decorate, thick, draw=black, #1]
	(Brace Top) -- (Brace Bottom) 
	node [pos=0.5, anchor=south, align=left, text width=#3, color=black, xshift=\BraceAmplitude] {#6};
	\end{tikzpicture}%
}%

\usetikzlibrary{patterns}

\definecolor{cof}{RGB}{219,144,71}
\definecolor{pur}{RGB}{186,146,162}
\definecolor{greeo}{RGB}{91,173,69}
\definecolor{greet}{RGB}{52,111,72}

% provide arXiv number if available:
% \arxiv{cs.IT/1502.00326}

% put your definitions there:

%\newtheorem{remark}{Remark} \def\remref#1{Remark~\ref{#1}}
%\newtheorem{conjecture}{Conjecture} \def\remref#1{Remark~\ref{#1}}
%\newtheorem{example}{Example}

%\theorembodyfont{\itshape}
%\newtheorem{theorem}{Theorem}
%\newtheorem{proposition}{Proposition}
%\newtheorem{lemma}{Lemma} \def\lemref#1{Lemma~\ref{#1}}
%\newtheorem{corollary}{Corollary}


%\theorembodyfont{\rmfamily}
%\newtheorem{definition}{Definition}
%\numberwithin{equation}{section}
% \theoremstyle{plain}
% \newtheorem{theorem}{Theorem}
% \newtheorem{Example}{Example}
% \newtheorem{lemma}{Lemma}
% \newtheorem{remark}{Remark}
% \newtheorem{corollary}{Corollary}
% \newtheorem{definition}{Definition}
% \newtheorem{conjecture}{Conjecture}
% \newtheorem{question}{Question}
% \newtheorem*{induction}{Induction Hypothesis}
% \newtheorem*{folklore}{Folklore}
% \newtheorem{assumption}{Assumption}

\def \by {\bar{y}}
\def \bx {\bar{x}}
\def \bh {\bar{h}}
\def \bz {\bar{z}}
\def \cF {\mathcal{F}}
\def \bP {\mathbb{P}}
\def \bE {\mathbb{E}}
\def \bR {\mathbb{R}}
\def \bF {\mathbb{F}}
\def \cG {\mathcal{G}}
\def \cM {\mathcal{M}}
\def \cB {\mathcal{B}}
\def \cN {\mathcal{N}}
\def \var {\mathsf{Var}}
\def\1{\mathbbm{1}}
\def \FF {\mathbb{F}}


\newenvironment{keywords}
{\bgroup\leftskip 20pt\rightskip 20pt \small\noindent{\bfseries
Keywords:} \ignorespaces}%
{\par\egroup\vskip 0.25ex}
\newlength\aftertitskip     \newlength\beforetitskip
\newlength\interauthorskip  \newlength\aftermaketitskip















%%%%%%%%%%%%%%%%%%%%%%%%%%%% by Wu %%%%%%%%%%%%%%%%%%%%%%%%%%%%
\usepackage{xspace}

\newcommand{\Lip}{\mathrm{Lip}}
\newcommand{\stepa}[1]{\overset{\rm (a)}{#1}}
\newcommand{\stepb}[1]{\overset{\rm (b)}{#1}}
\newcommand{\stepc}[1]{\overset{\rm (c)}{#1}}
\newcommand{\stepd}[1]{\overset{\rm (d)}{#1}}
\newcommand{\stepe}[1]{\overset{\rm (e)}{#1}}
\newcommand{\stepf}[1]{\overset{\rm (f)}{#1}}


\newcommand{\floor}[1]{{\left\lfloor {#1} \right \rfloor}}
\newcommand{\ceil}[1]{{\left\lceil {#1} \right \rceil}}

\newcommand{\blambda}{\bar{\lambda}}
\newcommand{\reals}{\mathbb{R}}
\newcommand{\naturals}{\mathbb{N}}
\newcommand{\integers}{\mathbb{Z}}
\newcommand{\Expect}{\mathbb{E}}
\newcommand{\expect}[1]{\mathbb{E}\left[#1\right]}
\newcommand{\Prob}{\mathbb{P}}
\newcommand{\prob}[1]{\mathbb{P}\left[#1\right]}
\newcommand{\pprob}[1]{\mathbb{P}[#1]}
\newcommand{\intd}{{\rm d}}
\newcommand{\TV}{{\sf TV}}
\newcommand{\LC}{{\sf LC}}
\newcommand{\PW}{{\sf PW}}
\newcommand{\htheta}{\hat{\theta}}
\newcommand{\eexp}{{\rm e}}
\newcommand{\expects}[2]{\mathbb{E}_{#2}\left[ #1 \right]}
\newcommand{\diff}{{\rm d}}
\newcommand{\eg}{e.g.\xspace}
\newcommand{\ie}{i.e.\xspace}
\newcommand{\iid}{i.i.d.\xspace}
\newcommand{\fracp}[2]{\frac{\partial #1}{\partial #2}}
\newcommand{\fracpk}[3]{\frac{\partial^{#3} #1}{\partial #2^{#3}}}
\newcommand{\fracd}[2]{\frac{\diff #1}{\diff #2}}
\newcommand{\fracdk}[3]{\frac{\diff^{#3} #1}{\diff #2^{#3}}}
\newcommand{\renyi}{R\'enyi\xspace}
\newcommand{\lpnorm}[1]{\left\|{#1} \right\|_{p}}
\newcommand{\linf}[1]{\left\|{#1} \right\|_{\infty}}
\newcommand{\lnorm}[2]{\left\|{#1} \right\|_{{#2}}}
\newcommand{\Lploc}[1]{L^{#1}_{\rm loc}}
\newcommand{\hellinger}{d_{\rm H}}
\newcommand{\Fnorm}[1]{\lnorm{#1}{\rm F}}
%% parenthesis
\newcommand{\pth}[1]{\left( #1 \right)}
\newcommand{\qth}[1]{\left[ #1 \right]}
\newcommand{\sth}[1]{\left\{ #1 \right\}}
\newcommand{\bpth}[1]{\Bigg( #1 \Bigg)}
\newcommand{\bqth}[1]{\Bigg[ #1 \Bigg]}
\newcommand{\bsth}[1]{\Bigg\{ #1 \Bigg\}}
\newcommand{\xxx}{\textbf{xxx}\xspace}
\newcommand{\toprob}{{\xrightarrow{\Prob}}}
\newcommand{\tolp}[1]{{\xrightarrow{L^{#1}}}}
\newcommand{\toas}{{\xrightarrow{{\rm a.s.}}}}
\newcommand{\toae}{{\xrightarrow{{\rm a.e.}}}}
\newcommand{\todistr}{{\xrightarrow{{\rm D}}}}
\newcommand{\eqdistr}{{\stackrel{\rm D}{=}}}
\newcommand{\iiddistr}{{\stackrel{\text{\iid}}{\sim}}}
%\newcommand{\var}{\mathsf{var}}
\newcommand\indep{\protect\mathpalette{\protect\independenT}{\perp}}
\def\independenT#1#2{\mathrel{\rlap{$#1#2$}\mkern2mu{#1#2}}}
\newcommand{\Bern}{\text{Bern}}
\newcommand{\Poi}{\mathsf{Poi}}
\newcommand{\iprod}[2]{\left \langle #1, #2 \right\rangle}
\newcommand{\Iprod}[2]{\langle #1, #2 \rangle}
\newcommand{\indc}[1]{{\mathbf{1}_{\left\{{#1}\right\}}}}
\newcommand{\Indc}{\mathbf{1}}
\newcommand{\regoff}[1]{\textsf{Reg}_{\mathcal{F}}^{\text{off}} (#1)}
\newcommand{\regon}[1]{\textsf{Reg}_{\mathcal{F}}^{\text{on}} (#1)}

\definecolor{myblue}{rgb}{.8, .8, 1}
\definecolor{mathblue}{rgb}{0.2472, 0.24, 0.6} % mathematica's Color[1, 1--3]
\definecolor{mathred}{rgb}{0.6, 0.24, 0.442893}
\definecolor{mathyellow}{rgb}{0.6, 0.547014, 0.24}


\newcommand{\red}{\color{red}}
\newcommand{\blue}{\color{blue}}
\newcommand{\nb}[1]{{\sf\blue[#1]}}
\newcommand{\nbr}[1]{{\sf\red[#1]}}

\newcommand{\tmu}{{\tilde{\mu}}}
\newcommand{\tf}{{\tilde{f}}}
\newcommand{\tp}{\tilde{p}}
\newcommand{\tilh}{{\tilde{h}}}
\newcommand{\tu}{{\tilde{u}}}
\newcommand{\tx}{{\tilde{x}}}
\newcommand{\ty}{{\tilde{y}}}
\newcommand{\tz}{{\tilde{z}}}
\newcommand{\tA}{{\tilde{A}}}
\newcommand{\tB}{{\tilde{B}}}
\newcommand{\tC}{{\tilde{C}}}
\newcommand{\tD}{{\tilde{D}}}
\newcommand{\tE}{{\tilde{E}}}
\newcommand{\tF}{{\tilde{F}}}
\newcommand{\tG}{{\tilde{G}}}
\newcommand{\tH}{{\tilde{H}}}
\newcommand{\tI}{{\tilde{I}}}
\newcommand{\tJ}{{\tilde{J}}}
\newcommand{\tK}{{\tilde{K}}}
\newcommand{\tL}{{\tilde{L}}}
\newcommand{\tM}{{\tilde{M}}}
\newcommand{\tN}{{\tilde{N}}}
\newcommand{\tO}{{\tilde{O}}}
\newcommand{\tP}{{\tilde{P}}}
\newcommand{\tQ}{{\tilde{Q}}}
\newcommand{\tR}{{\tilde{R}}}
\newcommand{\tS}{{\tilde{S}}}
\newcommand{\tT}{{\tilde{T}}}
\newcommand{\tU}{{\tilde{U}}}
\newcommand{\tV}{{\tilde{V}}}
\newcommand{\tW}{{\tilde{W}}}
\newcommand{\tX}{{\tilde{X}}}
\newcommand{\tY}{{\tilde{Y}}}
\newcommand{\tZ}{{\tilde{Z}}}

\newcommand{\sfa}{{\mathsf{a}}}
\newcommand{\sfb}{{\mathsf{b}}}
\newcommand{\sfc}{{\mathsf{c}}}
\newcommand{\sfd}{{\mathsf{d}}}
\newcommand{\sfe}{{\mathsf{e}}}
\newcommand{\sff}{{\mathsf{f}}}
\newcommand{\sfg}{{\mathsf{g}}}
\newcommand{\sfh}{{\mathsf{h}}}
\newcommand{\sfi}{{\mathsf{i}}}
\newcommand{\sfj}{{\mathsf{j}}}
\newcommand{\sfk}{{\mathsf{k}}}
\newcommand{\sfl}{{\mathsf{l}}}
\newcommand{\sfm}{{\mathsf{m}}}
\newcommand{\sfn}{{\mathsf{n}}}
\newcommand{\sfo}{{\mathsf{o}}}
\newcommand{\sfp}{{\mathsf{p}}}
\newcommand{\sfq}{{\mathsf{q}}}
\newcommand{\sfr}{{\mathsf{r}}}
\newcommand{\sfs}{{\mathsf{s}}}
\newcommand{\sft}{{\mathsf{t}}}
\newcommand{\sfu}{{\mathsf{u}}}
\newcommand{\sfv}{{\mathsf{v}}}
\newcommand{\sfw}{{\mathsf{w}}}
\newcommand{\sfx}{{\mathsf{x}}}
\newcommand{\sfy}{{\mathsf{y}}}
\newcommand{\sfz}{{\mathsf{z}}}
\newcommand{\sfA}{{\mathsf{A}}}
\newcommand{\sfB}{{\mathsf{B}}}
\newcommand{\sfC}{{\mathsf{C}}}
\newcommand{\sfD}{{\mathsf{D}}}
\newcommand{\sfE}{{\mathsf{E}}}
\newcommand{\sfF}{{\mathsf{F}}}
\newcommand{\sfG}{{\mathsf{G}}}
\newcommand{\sfH}{{\mathsf{H}}}
\newcommand{\sfI}{{\mathsf{I}}}
\newcommand{\sfJ}{{\mathsf{J}}}
\newcommand{\sfK}{{\mathsf{K}}}
\newcommand{\sfL}{{\mathsf{L}}}
\newcommand{\sfM}{{\mathsf{M}}}
\newcommand{\sfN}{{\mathsf{N}}}
\newcommand{\sfO}{{\mathsf{O}}}
\newcommand{\sfP}{{\mathsf{P}}}
\newcommand{\sfQ}{{\mathsf{Q}}}
\newcommand{\sfR}{{\mathsf{R}}}
\newcommand{\sfS}{{\mathsf{S}}}
\newcommand{\sfT}{{\mathsf{T}}}
\newcommand{\sfU}{{\mathsf{U}}}
\newcommand{\sfV}{{\mathsf{V}}}
\newcommand{\sfW}{{\mathsf{W}}}
\newcommand{\sfX}{{\mathsf{X}}}
\newcommand{\sfY}{{\mathsf{Y}}}
\newcommand{\sfZ}{{\mathsf{Z}}}


\newcommand{\calA}{{\mathcal{A}}}
\newcommand{\calB}{{\mathcal{B}}}
\newcommand{\calC}{{\mathcal{C}}}
\newcommand{\calD}{{\mathcal{D}}}
\newcommand{\calE}{{\mathcal{E}}}
\newcommand{\calF}{{\mathcal{F}}}
\newcommand{\calG}{{\mathcal{G}}}
\newcommand{\calH}{{\mathcal{H}}}
\newcommand{\calI}{{\mathcal{I}}}
\newcommand{\calJ}{{\mathcal{J}}}
\newcommand{\calK}{{\mathcal{K}}}
\newcommand{\calL}{{\mathcal{L}}}
\newcommand{\calM}{{\mathcal{M}}}
\newcommand{\calN}{{\mathcal{N}}}
\newcommand{\calO}{{\mathcal{O}}}
\newcommand{\calP}{{\mathcal{P}}}
\newcommand{\calQ}{{\mathcal{Q}}}
\newcommand{\calR}{{\mathcal{R}}}
\newcommand{\calS}{{\mathcal{S}}}
\newcommand{\calT}{{\mathcal{T}}}
\newcommand{\calU}{{\mathcal{U}}}
\newcommand{\calV}{{\mathcal{V}}}
\newcommand{\calW}{{\mathcal{W}}}
\newcommand{\calX}{{\mathcal{X}}}
\newcommand{\calY}{{\mathcal{Y}}}
\newcommand{\calZ}{{\mathcal{Z}}}

\newcommand{\bara}{{\bar{a}}}
\newcommand{\barb}{{\bar{b}}}
\newcommand{\barc}{{\bar{c}}}
\newcommand{\bard}{{\bar{d}}}
\newcommand{\bare}{{\bar{e}}}
\newcommand{\barf}{{\bar{f}}}
\newcommand{\barg}{{\bar{g}}}
\newcommand{\barh}{{\bar{h}}}
\newcommand{\bari}{{\bar{i}}}
\newcommand{\barj}{{\bar{j}}}
\newcommand{\bark}{{\bar{k}}}
\newcommand{\barl}{{\bar{l}}}
\newcommand{\barm}{{\bar{m}}}
\newcommand{\barn}{{\bar{n}}}
\newcommand{\baro}{{\bar{o}}}
\newcommand{\barp}{{\bar{p}}}
\newcommand{\barq}{{\bar{q}}}
\newcommand{\barr}{{\bar{r}}}
\newcommand{\bars}{{\bar{s}}}
\newcommand{\bart}{{\bar{t}}}
\newcommand{\baru}{{\bar{u}}}
\newcommand{\barv}{{\bar{v}}}
\newcommand{\barw}{{\bar{w}}}
\newcommand{\barx}{{\bar{x}}}
\newcommand{\bary}{{\bar{y}}}
\newcommand{\barz}{{\bar{z}}}
\newcommand{\barA}{{\bar{A}}}
\newcommand{\barB}{{\bar{B}}}
\newcommand{\barC}{{\bar{C}}}
\newcommand{\barD}{{\bar{D}}}
\newcommand{\barE}{{\bar{E}}}
\newcommand{\barF}{{\bar{F}}}
\newcommand{\barG}{{\bar{G}}}
\newcommand{\barH}{{\bar{H}}}
\newcommand{\barI}{{\bar{I}}}
\newcommand{\barJ}{{\bar{J}}}
\newcommand{\barK}{{\bar{K}}}
\newcommand{\barL}{{\bar{L}}}
\newcommand{\barM}{{\bar{M}}}
\newcommand{\barN}{{\bar{N}}}
\newcommand{\barO}{{\bar{O}}}
\newcommand{\barP}{{\bar{P}}}
\newcommand{\barQ}{{\bar{Q}}}
\newcommand{\barR}{{\bar{R}}}
\newcommand{\barS}{{\bar{S}}}
\newcommand{\barT}{{\bar{T}}}
\newcommand{\barU}{{\bar{U}}}
\newcommand{\barV}{{\bar{V}}}
\newcommand{\barW}{{\bar{W}}}
\newcommand{\barX}{{\bar{X}}}
\newcommand{\barY}{{\bar{Y}}}
\newcommand{\barZ}{{\bar{Z}}}

\newcommand{\hX}{\hat{X}}
\newcommand{\Ent}{\mathsf{Ent}}
\newcommand{\awarm}{{A_{\text{warm}}}}
\newcommand{\thetaLS}{{\widehat{\theta}^{\text{\rm LS}}}}

\newcommand{\jiao}[1]{\langle{#1}\rangle}
\newcommand{\gaht}{\textsc{GoodActionHypTest}\;}
\newcommand{\iaht}{\textsc{InitialActionHypTest}\;}
\newcommand{\true}{\textsf{True}\;}
\newcommand{\false}{\textsf{False}\;}

% \usepackage[capitalize,noabbrev]{cleveref}
% \crefname{lemma}{Lemma}{Lemmas}
% \Crefname{lemma}{Lemma}{Lemmas}
% \crefname{thm}{Theorem}{Theorems}
% \Crefname{thm}{Theorem}{Theorems}
% \Crefname{assumption}{Assumption}{Assumptions}
% \Crefname{inftheorem}{Informal Theorem}{Informal Theorems}
% \crefformat{equation}{(#2#1#3)}

% % if you use cleveref..
% \usepackage[capitalize,noabbrev]{cleveref}
% \crefname{lemma}{Lemma}{Lemmas}
% \crefname{proposition}{Proposition}{Propositions}
% \crefname{remark}{Remark}{Remarks}
% \crefname{corollary}{Corollary}{Corollaries}
% \crefname{definition}{Definition}{Definitions}
% \crefname{conjecture}{Conjecture}{Conjectures}
% \crefname{figure}{Fig.}{Figures}


\begin{document}

\twocolumn[
\icmltitle{Flexible and Efficient Grammar-Constrained Decoding}

% It is OKAY to include author information, even for blind
% submissions: the style file will automatically remove it for you
% unless you've provided the [accepted] option to the icml2025
% package.

% List of affiliations: The first argument should be a (short)
% identifier you will use later to specify author affiliations
% Academic affiliations should list Department, University, City, Region, Country
% Industry affiliations should list Company, City, Region, Country

% You can specify symbols, otherwise they are numbered in order.
% Ideally, you should not use this facility. Affiliations will be numbered
% in order of appearance and this is the preferred way.
% \icmlsetsymbol{equal}{*}

\begin{icmlauthorlist}
\icmlauthor{Kanghee Park}{ucsd}
\icmlauthor{Timothy Zhou}{ucsd}
\icmlauthor{Loris D'Antoni}{ucsd}
\end{icmlauthorlist}

\icmlaffiliation{ucsd}{Department of Computer Science and Engineering, UCSD, San Diego, USA}

\icmlcorrespondingauthor{Kanghee park}{kap022@ucsd.edu}

% You may provide any keywords that you
% find helpful for describing your paper; these are used to populate
% the "keywords" metadata in the PDF but will not be shown in the document
\icmlkeywords{Language Models, Decoding, Context-free Grammars}

\vskip 0.3in
]

% this must go after the closing bracket ] following \twocolumn[ ...

% This command actually creates the footnote in the first column
% listing the affiliations and the copyright notice.
% The command takes one argument, which is text to display at the start of the footnote.
% The \icmlEqualContribution command is standard text for equal contribution.
% Remove it (just {}) if you do not need this facility.

\printAffiliationsAndNotice{}  % leave blank if no need to mention equal contribution
% \printAffiliationsAndNotice{\icmlEqualContribution} % otherwise use the standard text.

\begin{abstract}
Large Language Models (LLMs) are often asked to generate structured outputs that obey precise syntactic rules, such as code snippets or formatted data. Grammar-constrained decoding (GCD) can guarantee that LLM outputs matches such rules by masking out tokens that will provably lead to outputs that do not belong to a specified context-free grammar (CFG). To guarantee soundness, GCD algorithms have to compute how a given LLM subword tokenizer can ``align'' with the tokens used 
 by a given context-free grammar and compute token masks based on this information. Doing so efficiently is challenging and existing GCD algorithms require tens of minutes to preprocess common grammars. We present a new GCD algorithm together with an implementation that offers \nx faster offline preprocessing than existing approaches while preserving state-of-the-art efficiency in online mask computation.
\end{abstract}

\section{Introduction}


\begin{figure}[t]
\centering
\includegraphics[width=0.6\columnwidth]{figures/evaluation_desiderata_V5.pdf}
\vspace{-0.5cm}
\caption{\systemName is a platform for conducting realistic evaluations of code LLMs, collecting human preferences of coding models with real users, real tasks, and in realistic environments, aimed at addressing the limitations of existing evaluations.
}
\label{fig:motivation}
\end{figure}

\begin{figure*}[t]
\centering
\includegraphics[width=\textwidth]{figures/system_design_v2.png}
\caption{We introduce \systemName, a VSCode extension to collect human preferences of code directly in a developer's IDE. \systemName enables developers to use code completions from various models. The system comprises a) the interface in the user's IDE which presents paired completions to users (left), b) a sampling strategy that picks model pairs to reduce latency (right, top), and c) a prompting scheme that allows diverse LLMs to perform code completions with high fidelity.
Users can select between the top completion (green box) using \texttt{tab} or the bottom completion (blue box) using \texttt{shift+tab}.}
\label{fig:overview}
\end{figure*}

As model capabilities improve, large language models (LLMs) are increasingly integrated into user environments and workflows.
For example, software developers code with AI in integrated developer environments (IDEs)~\citep{peng2023impact}, doctors rely on notes generated through ambient listening~\citep{oberst2024science}, and lawyers consider case evidence identified by electronic discovery systems~\citep{yang2024beyond}.
Increasing deployment of models in productivity tools demands evaluation that more closely reflects real-world circumstances~\citep{hutchinson2022evaluation, saxon2024benchmarks, kapoor2024ai}.
While newer benchmarks and live platforms incorporate human feedback to capture real-world usage, they almost exclusively focus on evaluating LLMs in chat conversations~\citep{zheng2023judging,dubois2023alpacafarm,chiang2024chatbot, kirk2024the}.
Model evaluation must move beyond chat-based interactions and into specialized user environments.



 

In this work, we focus on evaluating LLM-based coding assistants. 
Despite the popularity of these tools---millions of developers use Github Copilot~\citep{Copilot}---existing
evaluations of the coding capabilities of new models exhibit multiple limitations (Figure~\ref{fig:motivation}, bottom).
Traditional ML benchmarks evaluate LLM capabilities by measuring how well a model can complete static, interview-style coding tasks~\citep{chen2021evaluating,austin2021program,jain2024livecodebench, white2024livebench} and lack \emph{real users}. 
User studies recruit real users to evaluate the effectiveness of LLMs as coding assistants, but are often limited to simple programming tasks as opposed to \emph{real tasks}~\citep{vaithilingam2022expectation,ross2023programmer, mozannar2024realhumaneval}.
Recent efforts to collect human feedback such as Chatbot Arena~\citep{chiang2024chatbot} are still removed from a \emph{realistic environment}, resulting in users and data that deviate from typical software development processes.
We introduce \systemName to address these limitations (Figure~\ref{fig:motivation}, top), and we describe our three main contributions below.


\textbf{We deploy \systemName in-the-wild to collect human preferences on code.} 
\systemName is a Visual Studio Code extension, collecting preferences directly in a developer's IDE within their actual workflow (Figure~\ref{fig:overview}).
\systemName provides developers with code completions, akin to the type of support provided by Github Copilot~\citep{Copilot}. 
Over the past 3 months, \systemName has served over~\completions suggestions from 10 state-of-the-art LLMs, 
gathering \sampleCount~votes from \userCount~users.
To collect user preferences,
\systemName presents a novel interface that shows users paired code completions from two different LLMs, which are determined based on a sampling strategy that aims to 
mitigate latency while preserving coverage across model comparisons.
Additionally, we devise a prompting scheme that allows a diverse set of models to perform code completions with high fidelity.
See Section~\ref{sec:system} and Section~\ref{sec:deployment} for details about system design and deployment respectively.



\textbf{We construct a leaderboard of user preferences and find notable differences from existing static benchmarks and human preference leaderboards.}
In general, we observe that smaller models seem to overperform in static benchmarks compared to our leaderboard, while performance among larger models is mixed (Section~\ref{sec:leaderboard_calculation}).
We attribute these differences to the fact that \systemName is exposed to users and tasks that differ drastically from code evaluations in the past. 
Our data spans 103 programming languages and 24 natural languages as well as a variety of real-world applications and code structures, while static benchmarks tend to focus on a specific programming and natural language and task (e.g. coding competition problems).
Additionally, while all of \systemName interactions contain code contexts and the majority involve infilling tasks, a much smaller fraction of Chatbot Arena's coding tasks contain code context, with infilling tasks appearing even more rarely. 
We analyze our data in depth in Section~\ref{subsec:comparison}.



\textbf{We derive new insights into user preferences of code by analyzing \systemName's diverse and distinct data distribution.}
We compare user preferences across different stratifications of input data (e.g., common versus rare languages) and observe which affect observed preferences most (Section~\ref{sec:analysis}).
For example, while user preferences stay relatively consistent across various programming languages, they differ drastically between different task categories (e.g. frontend/backend versus algorithm design).
We also observe variations in user preference due to different features related to code structure 
(e.g., context length and completion patterns).
We open-source \systemName and release a curated subset of code contexts.
Altogether, our results highlight the necessity of model evaluation in realistic and domain-specific settings.





\section{Overview}

\revision{In this section, we first explain the foundational concept of Hausdorff distance-based penetration depth algorithms, which are essential for understanding our method (Sec.~\ref{sec:preliminary}).
We then provide a brief overview of our proposed RT-based penetration depth algorithm (Sec.~\ref{subsec:algo_overview}).}



\section{Preliminaries }
\label{sec:Preliminaries}

% Before we introduce our method, we first overview the important basics of 3D dynamic human modeling with Gaussian splatting. Then, we discuss the diffusion-based 3d generation techniques, and how they can be applied to human modeling.
% \ZY{I stopp here. TBC.}
% \subsection{Dynamic human modeling with Gaussian splatting}
\subsection{3D Gaussian Splatting}
3D Gaussian splatting~\cite{kerbl3Dgaussians} is an explicit scene representation that allows high-quality real-time rendering. The given scene is represented by a set of static 3D Gaussians, which are parameterized as follows: Gaussian center $x\in {\mathbb{R}^3}$, color $c\in {\mathbb{R}^3}$, opacity $\alpha\in {\mathbb{R}}$, spatial rotation in the form of quaternion $q\in {\mathbb{R}^4}$, and scaling factor $s\in {\mathbb{R}^3}$. Given these properties, the rendering process is represented as:
\begin{equation}
  I = Splatting(x, c, s, \alpha, q, r),
  \label{eq:splattingGA}
\end{equation}
where $I$ is the rendered image, $r$ is a set of query rays crossing the scene, and $Splatting(\cdot)$ is a differentiable rendering process. We refer readers to Kerbl et al.'s paper~\cite{kerbl3Dgaussians} for the details of Gaussian splatting. 



% \ZY{I would suggest move this part to the method part.}
% GaissianAvatar is a dynamic human generation model based on Gaussian splitting. Given a sequence of RGB images, this method utilizes fitted SMPLs and sampled points on its surface to obtain a pose-dependent feature map by a pose encoder. The pose-dependent features and a geometry feature are fed in a Gaussian decoder, which is employed to establish a functional mapping from the underlying geometry of the human form to diverse attributes of 3D Gaussians on the canonical surfaces. The parameter prediction process is articulated as follows:
% \begin{equation}
%   (\Delta x,c,s)=G_{\theta}(S+P),
%   \label{eq:gaussiandecoder}
% \end{equation}
%  where $G_{\theta}$ represents the Gaussian decoder, and $(S+P)$ is the multiplication of geometry feature S and pose feature P. Instead of optimizing all attributes of Gaussian, this decoder predicts 3D positional offset $\Delta{x} \in {\mathbb{R}^3}$, color $c\in\mathbb{R}^3$, and 3D scaling factor $ s\in\mathbb{R}^3$. To enhance geometry reconstruction accuracy, the opacity $\alpha$ and 3D rotation $q$ are set to fixed values of $1$ and $(1,0,0,0)$ respectively.
 
%  To render the canonical avatar in observation space, we seamlessly combine the Linear Blend Skinning function with the Gaussian Splatting~\cite{kerbl3Dgaussians} rendering process: 
% \begin{equation}
%   I_{\theta}=Splatting(x_o,Q,d),
%   \label{eq:splatting}
% \end{equation}
% \begin{equation}
%   x_o = T_{lbs}(x_c,p,w),
%   \label{eq:LBS}
% \end{equation}
% where $I_{\theta}$ represents the final rendered image, and the canonical Gaussian position $x_c$ is the sum of the initial position $x$ and the predicted offset $\Delta x$. The LBS function $T_{lbs}$ applies the SMPL skeleton pose $p$ and blending weights $w$ to deform $x_c$ into observation space as $x_o$. $Q$ denotes the remaining attributes of the Gaussians. With the rendering process, they can now reposition these canonical 3D Gaussians into the observation space.



\subsection{Score Distillation Sampling}
Score Distillation Sampling (SDS)~\cite{poole2022dreamfusion} builds a bridge between diffusion models and 3D representations. In SDS, the noised input is denoised in one time-step, and the difference between added noise and predicted noise is considered SDS loss, expressed as:

% \begin{equation}
%   \mathcal{L}_{SDS}(I_{\Phi}) \triangleq E_{t,\epsilon}[w(t)(\epsilon_{\phi}(z_t,y,t)-\epsilon)\frac{\partial I_{\Phi}}{\partial\Phi}],
%   \label{eq:SDSObserv}
% \end{equation}
\begin{equation}
    \mathcal{L}_{\text{SDS}}(I_{\Phi}) \triangleq \mathbb{E}_{t,\epsilon} \left[ w(t) \left( \epsilon_{\phi}(z_t, y, t) - \epsilon \right) \frac{\partial I_{\Phi}}{\partial \Phi} \right],
  \label{eq:SDSObservGA}
\end{equation}
where the input $I_{\Phi}$ represents a rendered image from a 3D representation, such as 3D Gaussians, with optimizable parameters $\Phi$. $\epsilon_{\phi}$ corresponds to the predicted noise of diffusion networks, which is produced by incorporating the noise image $z_t$ as input and conditioning it with a text or image $y$ at timestep $t$. The noise image $z_t$ is derived by introducing noise $\epsilon$ into $I_{\Phi}$ at timestep $t$. The loss is weighted by the diffusion scheduler $w(t)$. 
% \vspace{-3mm}

\subsection{Overview of the RTPD Algorithm}\label{subsec:algo_overview}
Fig.~\ref{fig:Overview} presents an overview of our RTPD algorithm.
It is grounded in the Hausdorff distance-based penetration depth calculation method (Sec.~\ref{sec:preliminary}).
%, similar to that of Tang et al.~\shortcite{SIG09HIST}.
The process consists of two primary phases: penetration surface extraction and Hausdorff distance calculation.
We leverage the RTX platform's capabilities to accelerate both of these steps.

\begin{figure*}[t]
    \centering
    \includegraphics[width=0.8\textwidth]{Image/overview.pdf}
    \caption{The overview of RT-based penetration depth calculation algorithm overview}
    \label{fig:Overview}
\end{figure*}

The penetration surface extraction phase focuses on identifying the overlapped region between two objects.
\revision{The penetration surface is defined as a set of polygons from one object, where at least one of its vertices lies within the other object. 
Note that in our work, we focus on triangles rather than general polygons, as they are processed most efficiently on the RTX platform.}
To facilitate this extraction, we introduce a ray-tracing-based \revision{Point-in-Polyhedron} test (RT-PIP), significantly accelerated through the use of RT cores (Sec.~\ref{sec:RT-PIP}).
This test capitalizes on the ray-surface intersection capabilities of the RTX platform.
%
Initially, a Geometry Acceleration Structure (GAS) is generated for each object, as required by the RTX platform.
The RT-PIP module takes the GAS of one object (e.g., $GAS_{A}$) and the point set of the other object (e.g., $P_{B}$).
It outputs a set of points (e.g., $P_{\partial B}$) representing the penetration region, indicating their location inside the opposing object.
Subsequently, a penetration surface (e.g., $\partial B$) is constructed using this point set (e.g., $P_{\partial B}$) (Sec.~\ref{subsec:surfaceGen}).
%
The generated penetration surfaces (e.g., $\partial A$ and $\partial B$) are then forwarded to the next step. 

The Hausdorff distance calculation phase utilizes the ray-surface intersection test of the RTX platform (Sec.~\ref{sec:RT-Hausdorff}) to compute the Hausdorff distance between two objects.
We introduce a novel Ray-Tracing-based Hausdorff DISTance algorithm, RT-HDIST.
It begins by generating GAS for the two penetration surfaces, $P_{\partial A}$ and $P_{\partial B}$, derived from the preceding step.
RT-HDIST processes the GAS of a penetration surface (e.g., $GAS_{\partial A}$) alongside the point set of the other penetration surface (e.g., $P_{\partial B}$) to compute the penetration depth between them.
The algorithm operates bidirectionally, considering both directions ($\partial A \to \partial B$ and $\partial B \to \partial A$).
The final penetration depth between the two objects, A and B, is determined by selecting the larger value from these two directional computations.

%In the Hausdorff distance calculation step, we compute the Hausdorff distance between given two objects using a ray-surface-intersection test. (Sec.~\ref{sec:RT-Hausdorff}) Initially, we construct the GAS for both $\partial A$ and $\partial B$ to utilize the RT-core effectively. The RT-based Hausdorff distance algorithms then determine the Hausdorff distance by processing the GAS of one object (e.g. $GAS_{\partial A}$) and set of the vertices of the other (e.g. $P_{\partial B}$). Following the Hausdorff distance definition (Eq.~\ref{equation:hausdorff_definition}), we compute the Hausdorff distance to both directions ($\partial A \to \partial B$) and ($\partial B \to \partial A$). As a result, the bigger one is the final Hausdorff distance, and also it is the penetration depth between input object $A$ and $B$.


%the proposed RT-based penetration depth calculation pipeline.
%Our proposed methods adopt Tang's Hausdorff-based penetration depth methods~\cite{SIG09HIST}. The pipeline is divided into the penetration surface extraction step and the Hausdorff distance calculation between the penetration surface steps. However, since Tang's approach is not suitable for the RT platform in detail, we modified and applied it with appropriate methods.

%The penetration surface extraction step is extracting overlapped surfaces on other objects. To utilize the RT core, we use the ray-intersection-based PIP(Point-In-Polygon) algorithms instead of collision detection between two objects which Tang et al.~\cite{SIG09HIST} used. (Sec.~\ref{sec:RT-PIP})
%RT core-based PIP test uses a ray-surface intersection test. For purpose this, we generate the GAS(Geometry Acceleration Structure) for each object. RT core-based PIP test takes the GAS of one object (e.g. $GAS_{A}$) and a set of vertex of another one (e.g. $P_{B}$). Then this computes the penetrated vertex set of another one (e.g. $P_{\partial B}$). To calculate the Hausdorff distance, these vertex sets change to objects constructed by penetrated surface (e.g. $\partial B$). Finally, the two generated overlapped surface objects $\partial A$ and $\partial B$ are used in the Hausdorff distance calculation step.
\section{Offline Token Preprocessing}
\label{sec:offline}
% \timothy{I think that ideally the section should start with the high-level flow of how the algorithm works. I'm not actually very clear on the details---I assume the transducers are being composed with a PDA? Is the PDA deterministic (I think an earlier version said it was)? If not, how do we actually run it during decoding? Shouldn't the composition involve a lot of preprocessing time?}

% In this section, we present how our approach efficiently precomputes the mapping from LLM tokens to sets of possible terminal sequences.
% 
Our approach starts by preprocessing the lexer to efficiently construct a lookup table that relates LLM tokens to terminal sequences (\autoref{sec:lexing}) and vice versa (\autoref{sec:realizable}). 
The preprocessed lexer is then used to analyze the parser to determine what terminal sequences are actually possible sequences in the grammar (\autoref{sec:parsing}).

% \loris{mini outline here with sec num}

\subsection{Lexer Preprocessing }
\label{sec:lexing}

% \loris{not sure if this is defined in sec 2, but I'll def here for now. Don't we need to also define that each terminal type is defined by a regex?}
Let $\alphabet$ denote the set of string characters, $\alphabet^\ast$ denote the set of strings over this alphabet, and $\terms$ denote the set of terminals (i.e., grammar-level tokens).
A \emph{lexer} is a function $\lexer$ that given an input string $w \in \alphabet^\ast$, returns a pair $(T_1\ldots T_k, w_r)$, where $T_1 \ldots T_k \in \terms^\ast$ is a sequence of terminals and $w_r \in \alphabet^\ast$ is the suffix of $w$ that has not been lexed (i.e., mapped to language terminals) yet. 

Typically lexers resolve ambiguity by making some simplifying assumptions that also help improve efficiency and avoid backtracking.
We use the same assumptions and describe them next.


\paragraph{Maximal Munch Principle and Lookahead}

Consider a language that contains two different tokens for the increment operator \texttt{++} and the addition operator \text{+}. 
% 
Although the input string \texttt{++} could be tokenized as two separate \texttt{+} addition tokens, in practice lexers prioritize the longer valid token to resolve ambiguity (and which usually captures the intended syntax of the programming language).
% 
This behavior is called the \emph{maximal munch principle}: the lexer matches the longest possible substring starting at the current position that aligns with a defined token pattern.

Under a strict interpretation of the maximal munch principle, if the lexer reaches the end of the input stream while processing a partial triple-quoted Python string \texttt{"""a"}, the lexer should tokenize the input as two strings \texttt{""} and \texttt{"a"}.
% 
However, supporting such cases would require either waiting until the end of the input string to produce any tokens or allowing backtracking.
% 
As such, in practice many lexers (including Python's) will raise an error and stop lexing if the scanned prefix cannot be tokenized as a single terminal.
%To avoid backtracking, many lexers would raise an error when tokenizing the string \texttt{"""a"} rather than backtracking to tokenize it as two strings. \loris{when is the error raised, and what is the error?}
% 
This greedy behavior disallows some strings, but guarantees that the lexer can resolve all tokenization ambiguities by inspecting only the next character at each step.
% \timothy{Should maybe add a sentence stating that this is not too far from how actual lexers work.}

\begin{definition}[1-lookahead]
A lexer $\lexer$ is \emph{1-lookahead} if for every string $w \in \alphabet^\ast$ and valid continuation $c \in \alphabet$ of $\sent$, whenever $\lexer(\sent) = (\term_1 \ldots \term_k, r)$ then $\lexer(\sent c)$ is either $(\term_1 \ldots \term_k \term_{k+1}, c)$ for some $T_{k+1} \in \terms$ or $(\term_1 \ldots \term_k, r c)$.
\end{definition}

Terminals are specified as a set of regular expressions. It is oftentimes convenient to work with a \textit{lexing automaton}, which is the finite state automaton (FSA) that accepts strings matching any terminal definition \cite{mcnaughton1960regular}.
We refer the reader to \autoref{app:fsa-definition} for formal definitions of the semantics of FSA, but recall that an FSA is a tuple $\automaton = (\Sigma, Q, q_0, \delta, F)$ where $\Sigma$ is the alphabet, $Q$ is the set of states with initial state $q_0\in Q$, $\delta$ contains transitions of the form $q \xrightarrow{c} q'$, and $F\subseteq Q$ is the set of final states.

% \loris{you need to end with fact that one can build FSA when you have 1-lookahead and point to example figure, or start next section with that. Next para just assumes you magically have an FSA}

\paragraph{Lexing Transducer}

\begin{figure}
\centering
{\footnotesize
\begin{tikzpicture}[
    shorten >=1pt,
    node distance=2.5cm and 2.5cm,
    on grid,
    auto,
    scale=1.0,
    every state/.style={minimum size=0.8cm}
] 
   \node[state,initial,accepting,initial text={}] (q_0)   {$q_0$}; 
   \node[state] (q_1) [right=of q_0] {$q_1$}; 
   \node[state] (q_2) [right=of q_1, yshift=30pt] {$q^{\texttt{B}}_2$}; 
   \node[state] (q_3) [right=of q_1, yshift=-30pt] {$q^{\texttt{C}}_3$};

    \path[->] 
    (q_0) edge node {\texttt{a}:$\epsilon$} (q_1)

    (q_1) edge [bend right=10] node[swap, pos=0.5] {\texttt{b}:$\epsilon$} (q_2) 
          edge [bend left=10] node[pos=0.5] {\texttt{c}:$\epsilon$} (q_3) 

    (q_2) edge [loop right] node[pos=0.15, above] {
            \texttt{b}:$\epsilon$
            } ()
          edge[bend right=10] node[swap, above=2pt] {
            \texttt{a}:\texttt{B}
          } (q_1)

    (q_3) edge [loop right] node[pos=0.15, above] {\texttt{c}:$\epsilon$} ()
          edge[bend left=10] node[swap, below=2pt] {\texttt{a}:\texttt{C}} (q_1)

;
\end{tikzpicture}
}
\caption{A lexing transducer $\transducer_\automaton$ derived from FSA $\automaton$ in \autoref{fig:overview}. }
\label{fig:char-transducer}
\end{figure}

A 1-lookahead maximal munch lexer can be defined from a lexing automaton as follows:
% Given an FSA that recognizes a given set of language tokens, the maximal munch principle with 1-lookahead operates as follows:
The input is processed character-by-character by transitioning through the FSA's states.
When no valid transition exists for the next character $c$, the lexer checks whether the current state corresponds to a valid language token. 
% 
If it does and the tokenizer has at this point produced a pair $(T_1\ldots T_k, w_r)$, the not-yet tokenized string $w_r$ is tokenized with token $T_{k+1}$ corresponding to the reached state, the FSA is reset to its initial state $q_0$ (and the tokenizer state $(T_1\ldots T_k T_{k+1}, \varepsilon)$ with the empty string $\varepsilon$), and the character $c$ is consumed as the starting character of a new token $q_0$. 
If the current state does not correspond to a valid token or if $c$ cannot be consumed at $q_0$, then $c$ is invalid. 
% 
Invalid characters inform what tokens should be masked during constrained decoding.

% A finite-state transducer (FST) extends an FSA with outputs, formally defined as a tuple $\transducer = (\alphabet, \terms, Q, q_0, \delta, F)$ where $\Sigma, Q, q_0$ and $F$ are as in FSA; $\Gamma$ is the output alphabet, and $\delta \subseteq Q \times (\Sigma \cup \{\epsilon\}) \times \Gamma^\ast \times Q$ is the set of transitions. 
%
% Each transition $(q, c, \term_1 \ldots \term_k, q') \in \delta$ means that $q$, $c$, $q'$ are as in FSA and the transition outputs the sequence $\term_1 \ldots \term_k$.

This process can be formalized as a finite-state transducer (FST), an extension of a finite-state automaton that can produce output terminals when reading characters. Given the original lexing automaton $\automaton$ representing valid tokens, we write $\transducer_\automaton$ to denote the \emph{lexing transducer}, the FST corresponding to $\automaton$.
% 
The construction of $\transducer_\automaton$ from $\automaton$ is formalized in \autoref{alg:fst-construction} in Appendix A, but at a high level the process simply adds transitions to handle cases where no valid transition exists for the current character $c$, outputting terminals and exiting final states. 
\autoref{fig:char-transducer} shows the lexing transducer derived from the FSA in \autoref{fig:overview}.

% \loris{next sence is broken, also you never point at figures}
% When the FSA cannot process $c$ from its current state, the transducer the token recognized up to that point, resets to the initial state $q_0$, and processes $c$ from $q_0$. 

% \begin{lemma}
% If $L_\automaton$ is 1-lookahead, then $\transducer_\automaton(\sent) = T_1 \ldots T_k$ if and only if $L_\automaton(\sent) = (T_1 \ldots T_k, r)$ for some suffix $r$.
% \end{lemma}

% \kh{TODO: Proof, by induction on $w$. }

% \loris{this para name should be more about LLM.}
\paragraph{LLM Token to Terminals}

\begin{figure}
\centering
{\footnotesize
\begin{tikzpicture}[
    shorten >=1pt,
    node distance=2.5cm and 3.0cm,
    on grid,
    auto,
    scale=1.0,
    every state/.style={minimum size=0.8cm}
] 
   \node[state,initial,accepting,initial text={}] (q_0)   {$q_\epsilon$}; 
   \node[state] (q_1) [right=of q_0] {$q_{\texttt{a}}$}; 
   \node[state] (q_2) [right=of q_1] {$q_{\texttt{ab}}$}; 
   % \node[state] (q_4) [below=of q_3] {$q^{(}_4$};
   % \node[state] (q_5) [below=of q_4] {$q^{)}_5$};
    \path[->] 
    (q_0) edge [bend left=10] node {$\epsilon$:\texttt{a}} (q_1)
          edge [loop above] node {\texttt{a}:\texttt{a}, \texttt{b}:\texttt{b}, \texttt{c}:\texttt{c}} ()
          % edge [bend right=30]  node[pos=0.78] {( / $\epsilon$} (q_4)
          % edge [bend right=30]  node[pos=0.81] {) / $\epsilon$} (q_5)
    (q_1) edge [bend left=10] node {\texttt{ab}:\texttt{b}, \texttt{ac}:\texttt{c}} (q_0)
          edge node {$\epsilon$:\texttt{b}} (q_2) 
    (q_2) edge [bend right=40] node[pos=0.4, swap] {\texttt{aba}:\texttt{a}} (q_0)
;
\end{tikzpicture}
}
\caption{Detokenizing transducer for vocabulary $\vocab=\{\texttt{a}, \texttt{b}, \texttt{c}, \texttt{ab}, \texttt{ac}, \texttt{aba}\}$.}
\label{fig:detokenizing}
\end{figure}

\begin{figure}
\centering
{\footnotesize
\begin{tikzpicture}[
    shorten >=1pt,
    node distance=2.5cm and 2.7cm,
    on grid,
    auto,
    scale=1.0,
    every state/.style={minimum size=0.8cm}
] 
   \node[state,initial,accepting,initial text={}] (q_0)   {$q_0$}; 
   \node[state] (q_1) [right=of q_0] {$q_1$}; 
   \node[state] (q_2) [right=of q_1, yshift=40pt] {$q^{\texttt{B}}_2$}; 
   \node[state] (q_3) [right=of q_1, yshift=-40pt] {$q^{\texttt{C}}_3$};
   % \node[state] (q_4) [below=of q_3] {$q^{(}_4$};
   % \node[state] (q_5) [below=of q_4] {$q^{)}_5$};
    \path[->] 
    (q_0) edge node {\texttt{a}:$\epsilon$, \texttt{aba}:\texttt{B}} (q_1)
          edge [bend left=40]  node[pos=0.4, swap] {\texttt{ab}:$\epsilon$} (q_2)
          edge [bend right=40]  node[pos=0.4] {\texttt{ac}:$\epsilon$} (q_3)
          % edge [bend right=60]  node[pos=0.3, right=-5pt] {
          %   \begin{tabular}{c} \texttt{+}/$\epsilon$, \\ \texttt{\{0+,00+\}}\\/\texttt{NUM} \end{tabular}
          % } (q_3)
          % edge [bend right=30]  node[pos=0.78] {( / $\epsilon$} (q_4)
          % edge [bend right=30]  node[pos=0.81] {) / $\epsilon$} (q_5)
    (q_1) edge [bend right=10] node[swap, pos=0.3] {\texttt{b}:$\epsilon$} (q_2) 
          edge [bend left=10] node[pos=0.3] {\texttt{c}:$\epsilon$} (q_3) 
    %       edge [loop above] node[pos=0.8, right] {\texttt{\{a,0,a0\}}/$\epsilon$} ()
    %       edge [bend left=80] node[swap, pos=0.65, left=-8pt] {
    %         \begin{tabular}{c} \texttt{\{0+,00+\}}\\/\texttt{NAME} \end{tabular}
    %       } (q_3)
    (q_2) edge [loop right] node[pos=0.15, above] {
            \begin{tabular}{c} \texttt{b}:$\epsilon$, \\ \texttt{ab}:\texttt{B} \end{tabular}
            } ()
          edge[bend right=10] node[swap, above=2pt] {
            \begin{tabular}{c} \texttt{a}:\texttt{B}, \\ \texttt{aba}:\texttt{BB} \end{tabular}
          } (q_1)
          edge [bend right=10] node[swap, pos=0.4] {\texttt{ac}:\texttt{B}} (q_3)
    %       edge [bend left=40] node[pos=0.2, below=2pt] {\texttt{EOS}/\texttt{NUM};$\$$} (q_0)
    (q_3) edge [loop right] node[pos=0.15, above] {
            \begin{tabular}{c} \texttt{c}:$\epsilon$, \\ \texttt{ac}:\texttt{C} \end{tabular}
            } ()
          edge [bend right=10] node[swap, pos=0.6] {\texttt{ab}:\texttt{C}} (q_2)
          edge[bend left=10] node[swap, below=2pt] {
            \begin{tabular}{c} \texttt{a}:\texttt{C}, \\ \texttt{aba}:\texttt{CB} \end{tabular}
          } (q_1)
    %       edge [bend right=10] node[swap, pos=0.3] {\texttt{0}/\texttt{+}} (q_2)
    %       edge [bend right=90] node[swap] {\texttt{a0}/\texttt{+}} (q_1)
    %       edge [loop below] node {
    %         \begin{tabular}{c} \texttt{+}/\texttt{+}, \{\texttt{0+},\texttt{00+}\}/\texttt{+};\texttt{NUM} \end{tabular}
    %       } ()
;
\end{tikzpicture}
}
\caption{A determinized token-level lexing transducer $\transducer_{\automaton \circ \vocab}$, which is formed by composing $\transducer_\vocab$ from \autoref{fig:detokenizing} and $\transducer_\automaton$ from \autoref{fig:char-transducer}.}
\label{fig:token-transducer}
\end{figure}



Processing LLM tokens character-by-character with the lexing transducer at runtime would incur significant overhead.
To address this problem, we instead construct a token-to-terminal transducer by composing the character-to-terminal lexing transducer with the detokenizing transducer introduced by \citet{koo2024automatabased}.
%\loris{is this diff example than one in sec2?}
A detokenizing transducer simply maps LLM tokens to their corresponding sequence of text characters (i.e., the input alphabet is $\vocab$ and the output alphabet is $\Sigma$).
% 
A detokenizing transducer is nondeterministic and can contain $\varepsilon$-transitions that produce outputs without consuming inputs.

\autoref{fig:detokenizing} illustrates the detokenizing transducer $\transducer_\vocab$ derived from the vocabulary $\vocab = \{\texttt{a}, \texttt{b}, \texttt{c}, \texttt{ab}, \texttt{ac}, \texttt{aba}\}$ in \autoref{fig:overview}.
Note that as an optimization common prefixes of tokens form a trie-like structure (e.g., state $q_\texttt{a}$ denotes the state reached when producing the first \texttt{a} for all tokens that start with that character), reducing computational overhead by reusing shared prefix computation.
\autoref{fig:token-transducer} depicts the combined token-level lexing transducer $\transducer_{\automaton \circ \vocab}=\transducer_\automaton \circ \transducer_{\vocab}$, which is the determinized FST capturing the sequential functional composition of the two transducers---i.e., the output of $\transducer_\vocab$ is fed as input to $\transducer_\automaton$.
This token-to-terminal transducer enables efficient lookup of valid next tokens and produced terminal symbols in each state.


\subsection{Realizable Terminal Sequences}
\label{sec:realizable}

Now that we have defined the formal machinery behind lexing, we are ready to explain how LLM tokens can be mapped to possible sequences of terminals.

When the transducer $\transducer_{\automaton \circ \vocab}$ in \autoref{fig:token-transducer} consumes the LLM token \texttt{aba} from the initial state $q_0$, it produces the terminal \texttt{B} and moves to state $q_1$. 
If we inspect the grammar $\grammar_{\texttt{BC}}$ in \autoref{fig:overview_cfg}, we can deduce that the parser, which receives as input sequences of tokens, expects/requires the next language token to be \texttt{C}.
% \loris{I don't follow next point, since I don't know what automaton I'm looking at}
Since $\transducer_{\automaton \circ \vocab}$ in state $q_1$ does not immediately produce any output when consuming \texttt{b}, 
the generated terminal sequence so far (i.e., \texttt{B}) is still a valid prefix according to the grammar $\grammar_{\texttt{BC}}$.
However, after transitioning to $q_2$, no possible path can generate \texttt{C} next.

As illustrated by the above example, for each transition $q \xrightarrow{t:T_1 \ldots T_k} q'$ in the lexing transducer $\transducer_{\automaton \circ \vocab}$, we should check whether there is a terminal $T$ such that \rone  $T$ can be generated along some path from $q'$, and \rtwo $T_1 \ldots T_k T$ is accepted by the grammar.
%
This observation leads to the following definition, which describes which terminal sequences need to be considered by the parser.

\begin{definition}[Realizable Terminal Sequences]
\label{def:realizable-term-seq}
Given a token vocabulary $\vocab$ and FSA $\automaton$, let $\delta$ be the set of transitions in the token-level lexing transducer $\transducer_\automaton \circ \transducer_\vocab$.
% 
The set of \textit{realizable terminal sequences} $\realizable{\vocab}{\automaton}$ is defined as 
\begin{multline*}
\realizable{\vocab}{\automaton} = \{ T_1 \ldots T_k T \mid q \xrightarrow{t:T_1 \ldots T_k} q' \in \delta \textrm{ and } \\
T \textrm{ can be generated along some path from } q' \}.
\end{multline*}
\end{definition}

Note that the LLM vocabulary $\vocab$ contains all characters in $\alphabet$, ensuring that $\alphabet^\ast = \vocab^\ast$. 
Therefore, any next terminal producible in $\transducer_\automaton$ is also producible in the combined transducer $\transducer_{\automaton \circ \vocab}$ and vice versa. 
This equivalence allows us to simplify producibility checks: instead of analyzing the large combined transducer $\transducer_{\automaton \circ \vocab}$, we need only compute reachability to accepting states within $\transducer_\automaton$ to determine producible next terminals.

\paragraph{Inverse Token Spanner Table}
\setcounter{algorithm}{3}
% 
\begin{algorithm}[!t]
\caption{\tname{BuildInverseTokenSpannerTable}}
\label{alg:inverse-token-table}
\begin{algorithmic}
    \STATE {\bfseries Input:} Combined FST $\transducer_\automaton \circ \transducer_\vocab = (\vocab, \terms, Q, q_0, \delta, F)$
    \STATE {\bfseries Output:} Realizable token sequences $\realizable{\vocab}{\automaton}$, \\
    Inverse lookup table $\inversetable: \realizable{\vocab}{\automaton} \times Q \rightarrow 2^\vocab $
    \STATE $\realizable{\vocab}{\automaton} := \emptyset$, $\inversetable(\cdot, \cdot) := \emptyset$
    \FOR{$q \xrightarrow{t:T_1 \ldots T_k} q' \in \delta$}
        \FOR{$T$ recognized at $q''$, that is reachable from $q'$}
            \STATE $\realizable{\vocab}{\automaton} := \realizable{\vocab}{\automaton} \cup \{ T_1 \ldots T_k T \}$
            \STATE $\inversetable(q, T_1 \ldots T_k T) := \inversetable(q, T_1 \ldots T_k T) \cup \{t\}$
        \ENDFOR
    \ENDFOR
    \RETURN $\realizable{\vocab}{\automaton}$, $\inversetable$
\end{algorithmic}
\end{algorithm}

\autoref{alg:inverse-token-table} computes the set of realizable terminal sequences and constructs the key data structure we use to perform constrained decoding.
\begin{definition}[Inverse Token Spanner Table]
Given a lexer state $q \in Q_{\automaton \circ \vocab}$ and $T_1 \ldots T_k T \in \realizable{\vocab}{\automaton}$, the entry $\inversetable(q, \alpha)$ in the \textit{inverse token spanner table} $\inversetable$ is the set of tokens that generates $T_1 \ldots T_kT$ from state $q$. Formally,
\begin{multline*}
\inversetable(q, T_1 \ldots T_k T) = \{ t \mid q \xrightarrow{t:T_1 \ldots T_k} q' \in \delta \textrm{ and } \\
T \textrm{ can be generated along some path from } q' \}
\end{multline*}    
\end{definition}

% \loris{explain what it does and define}
This table allows a decoder to determine which LLM tokens can result in a given realizable terminal sequence $T_1 \ldots T_k T$ being produced from a specific lexer state.

For the example of \autoref{fig:overview}, the terminal sequence \texttt{BC} is generated by the token \texttt{aba} in state $q_0$ (i.e., $\texttt{aba} \in \inversetable(q_0, \texttt{BC})$). The same sequence is generated by the token \texttt{ac} in state $q_2$ (i.e., $\texttt{ac} \in \inversetable(q_2, \texttt{BC})$).
% \loris{should the two above be more like $\texttt{ac}\in \inversetable(q_2, \texttt{BC})$}


% \khchanged{
% Given a transition relation $\delta$ of transducer $\transducer$, we define the extended transition $\delta^\ast$ as the smallest set defined by \rone $q \xrightarrow{\epsilon:\epsilon}^\ast q \in \delta^\ast$ and \rtwo $q \xrightarrow{wc:\alpha \alpha'}^{\ast} q' \in \delta^\ast$ whenever $q \xrightarrow{w:\alpha}^{\ast} q'' \in \delta^\ast$ and $q'' \xrightarrow{c:\alpha'} q' \in \delta^\ast$.
% }

% \khchanged{
% \begin{theorem}[Soundness]
% Assume $\alphabet \subseteq \vocab$.
% Given a state $q \in Q_{\automaton \circ \vocab}$ and a terminal sequences $T_1 \ldot T_k T \in  \realizable{\vocab}{\automaton}$, 
% for each token $t \in \inversetable(q, T_1 \ldots T_k T)$, there exists $t_1 \ldots t_m \in \vocab^\ast$
% \end{theorem}
% }

% \loris{point to example table in fig1 as example}

\subsection{Parser Preprocessing}
\label{sec:parsing}

We do not formally define context-free grammars (CFG) for brevity and refer the reader to \autoref{app:cfg-definition} for details.
% 
For the sake of this paper, the reader only needs to know that a CFG parser is typically formalized as a pushdown automaton (PDA), an extension of FSA with an execution stack \cite{schutzenberger1963context}.
The definition of a PDA is similar to that of an FSA, but transitions are also allowed to manipulate a stack over symbols $\Pi$ via push and pop operations.
Therefore, in a PDA, a configuration after reading a sequence of input characters is a pair of automaton state $q\in Q$ and execution stack $\gamma\in \Pi^\ast$.
 
We refer the reader to \autoref{app:pda-definition} for the formal definition of a PDA, but informally, each PDA transition $q \xrightarrow{c[\beta \rightarrow \beta']} q'$ can only be activated if the character being read is a $c$ and the top of the current stack configuration $\gamma$ matches to sequence of stack symbols $\beta \in \Pi^\ast$. 
If the transitions is activated, the current state becomes $q'$, and the top $\beta$ elements of the stack are replaced with new stack elements $\beta' \in \Pi^\ast$.
% Notably, widely used LR($k$) and LALR($k$) parsers can be formalized as deterministic PDAs \cite{knuth1965translation}.}

As with lexer preprocessing, our goal is to also preprocess the parser to avoid iterating over every terminal sequence generated by the lexer at runtime.
To achieve this objective, we
directly compose the detokenizing transducer $\realizable{\vocab}{\automaton}$ (\Cref{def:realizable-term-seq}), with the PDA $\pushdown$ produced by a parser generator (and where transitions operate over single terminals)~\cite{allauzen2012pushdown} to obtain a new pushdown automaton $\pushdown \circ \transducer_{\realizable{\vocab}{\automaton}}$ where transitions operate over terminal sequences. 

This last transducer can efficiently determine valid sequences of terminal symbols from each parser state.
% 
We note that preprocessing both the lexer and parser in tandem is a key feature that distinguishes our work from prior work~\cite{beurer2024domino, ugare2024syncode}.

% \loris{should the cardinality beo over T and not RE? I want to move this para a few lines before, let me know once fixed}
% \kh{RE is the range of T (plus reachability), so it is correct I think? }
One key source of efficiency resulting from our approach is that many transitions in the combined transducer $\transducer_{\automaton \circ 
 \vocab}$ produce the same terminal sequence $T_1 \ldots T_k T$, making $|\realizable{\vocab}{\automaton}|$ smaller than $|\vocab|$ or $|\delta|$.
%\khchanged{
% In particular, this set is larger than the set of terminal sequences realizable from any specific state $q$, but not significantly larger than the largest such set from a single state.
% In particular, the \emph{global} set of realizable terminal sequences $\realizable{\vocab}{\automaton}$ enables efficient precomputation of what sequences of terminals the parser should consider, instead of considering realizable terminal sequences \emph{for each} lexer state $q$.}
%
Thus, the set of realizable terminal sequences $\realizable{\vocab}{\automaton}$ enables efficient precomputation of what sequences of terminals the parser should consider.

% \loris{describe first what is the high level thing you are trying to build and start by saying that a parser uses a PDA (you have never said this)}
%

% Specifically, we construct a detokenizing transducer that captures all possible output sequence of terminal symbols.
% \loris{will the PDA receive single tokens or seq of tokens now?}



\paragraph{Context Dependent/Independent Tokens}

Because pushdown automata need a stack to be able to parse arbitrary context-free grammars, we cannot decide entirely at preprocessing time whether a given terminal sequence can be accepted by a PDA at any given state---one has to inspect the content of the stack at execution time.
% 
However, many terminal sequences can be identified as always accepted or always rejected, independent of the current execution stack.

\textit{Stack invariance} provides a sufficient condition for knowing when a sequence of terminals is accepted.

% \loris{this is technical but we have no def of PDAs earlier. You can just have an informal def above where you say a PDA is like an FSA but can also push/pop on stack, therefore configs are pairs (q, gamma) and a push adds to gamma and a pop inspects top of gamma and pops}
\begin{proposition}[Stack Invariance]
If a PDA $\pushdown$ accepts an input sequence $w$ in state $q$ with stack configuration $\gamma$, 
then $w$ is also accepted in the same state $q$ when the stack configuration is $\gamma' \cdot \gamma$ for some $\gamma'$ 
(i.e., when $\gamma$ appears at the top of the stack with additional symbols beneath it).
\end{proposition}

% The proof follows from the fact that stack operations are restricted to the top of the stack, 
% and hence the PDA never attempts to modify or inspect symbols below $\gamma$.


It follows that a terminal sequence accepted by the PDA starting with an empty stack configuration is accepted under any stack configuration. 

The following proposition shows how to construct a stack-free finite-state automaton that overapproximates the set of sequences accepted by a PDA.
\begin{proposition}[Overapproximation via FSA]
\label{prop:overapprox-fsa}
Consider an FSA $\automaton_\pushdown$ obtained by removing all stack operations from a PDA $\pushdown$.
If an input sequence $w$ is not accepted by $\automaton_\pushdown$ in state $q$, then $w$ cannot be accepted by $\pushdown$ in state $q$ with any stack configuration.
\end{proposition}
% Since the FSA $\automaton$ is obtained by removing all stack operations from the PDA $\pushdown$, 
% any input sequence $w$ that is accepted by $\pushdown$ in some state $q$ with stack configuration $\gamma$ must also be accepted by the $\automaton$ in state $q$.

% Not using as it says theorem instead of prop
Following Proposition~\ref{prop:overapprox-fsa}, a terminal sequence is always rejected if it is rejected by the FSA obtained by removing all stack operations.
% 


The above reasoning can be formalized by computing the set of \textit{always-accepted tokens} $A$ and of \textit{context-dependent terminal sequences} $D$.
%
Given a lexer state $q^\automaton$ and a parser state $q^\pushdown$, we denote by $A(q^\automaton, q^\pushdown)$ the set of tokens that are accepted regardless of the stack configuration $\gamma$, and by $D(q^\automaton, q^\pushdown)$ the set of terminal sequences that may be accepted by some configuration $\gamma$. 
%\loris{I thought you were computing the for sure rejected?}
%\kh{that is only used to compute D}

\autoref{alg:preprocess-parser} in Appendix A describes how to preprocess a parser to build a table of always-accepted tokens $A$ and context-dependent sequences $D$.

\setcounter{algorithm}{5}

% \paragraph{Contextual Lexing}

% One challenge in lexical analysis arises when the same input sequence must be interpreted differently based on parsing context. 
% For example, in Java, the end of the nested generic \texttt{List<List<T>{}>} could be erroneously tokenized by a lexer operating under maximal munch as $\texttt{>>}$, the right-shift operator. However, in this context the parser instead expects two consecutive \texttt{>} terminals denoting the closure of a generic type. 
% To resolve this ambiguity, lexer must consider the parser’s current state. 
% One approach involves allowing the lexer to nondeterministically generate both possible tokenizations (e.g., a single \texttt{>{}>} terminal or two separate \texttt{>} terminals) specifically for such cases, enabling the parser to select the valid interpretation based on grammatical constraints.
\subsection{Online retrieval}

% 1. 介绍motivation和整体步骤
% 2. 检索步骤
% 3. 生成步骤

%Our ArchRAG is a cost-efficient generation method designed to support both specific QA and abstract QA tasks.
%
In the online retrieval phase, after obtaining the query vector for a given question, ArchRAG generates the final answer by first conducting hierarchical search on the C-HNSW index, and then analyzing and filtering of retrieved information.

\subsubsection{Hierarchical search.} 
\label{sec:search}
% 
We first introduce the query process of C-HNSW and then describe the hierarchical search.
%
The query process is the core of C-HNSW, which can be used for constructing C-HNSW as illustrated in Algorithm \ref{alg:query}.

\begin{algorithm}[ht]
  \caption{{C-HNSW query}}
  \label{alg:query}
  \small
   \SetKwInOut{Input}{input}\SetKwInOut{Output}{output}
    \Input{$\mathcal{H} = ({\mathcal G},{L_{inter}})$, $q$, $k$, $l$.}
    % \Output{$k$ Nearest neighbor found by C-HNSW.}
    $s\gets $ a random node in the highest layer $L$\;
    \For{$i \gets L,\cdots,l+1$}{
        $c\gets$ SearchLayer(${ G_i}=(V_i,E_i), q, s, 1, i$)\;
        % $s\gets$ get the nearest node from $K$\;
$s\gets$ find the node in layer $i-1$ via the inter-layer links of $c$\;
    }
    $K\gets$ SearchLayer(${ G_l}=(V_l,E_l), q, s, k, l$)\;
    \Return{$K$;}
    
    \textbf{Procedure} SearchLayer(${G_i}=(V_i,E_i), q, s,k,i$)\;
    $V\gets \{s\}$,  $R\gets \{s\}$, $Q\gets $ initialize a queue containing $s$\; 
    \While{$|R|>0$}{
        $c \gets$ nearest node in $Q$\;
        $f \gets$ furthest node in $R$\;
        \lIf{$d(c,q)>d(f,q)$}{
            {\bf break}
        } 
        \For{each neighbor $x\in N(c)$ in $G_i$}{
            \lIf{$x \in V$}{\textbf{continue}}
            $V\gets V \cup \{x\}$\;
            $f \gets$ furthest node in $R$\;
            \If{$d(x,q)<d(f,q)$\text{ or }$|R|<k$ }{
                $Q\gets Q\cup \{x\}$, $R\gets R\cup \{x\}$\;
                \lIf{$|R|>k$}{remove $f$ from $R$}
            }
        }
    }
        
    \Return{${R}$;}
\end{algorithm}

Specifically, given a specified query layer $l$, query point $q$, and the number $k$ of nearest neighbors, the query algorithm can be implemented through the following iterative process:
\begin{enumerate}
    \item Start from a random node at the highest layer $L$, which serves as the starting node for layer $L$ (line 1).
    
    \item For each layer from layers $L$ to $l+1$, begin at the starting node and use a greedy traversal approach (i.e., the procedure SearchLayer) to find the nearest neighbor $c$ of $q$, and then traverse to the next layer using $c$'s inter-layer link as the starting node for the next layer (lines 2-4).
    
    \item In the query layer $l$, use the greedy traversal approach to find the $k$ nearest neighbors of $q$ (line 5).
\end{enumerate}

Specifically, the greedy traversal strategy compares the distance between the query point and the visited nodes during the search process.
% 
It achieves this by maintaining a candidate expansion queue $Q$ and a dynamic nearest neighbor set $R$, which contains $k$ elements:

\begin{itemize}
    \item Expansion Queue $Q$: For each neighbor $x$ of a visited node, if $d(x,q)<d(f,q)$, where $f$ is the furthest node from $R$ to $q$, then $x$ is added to the expansion queue.
    \item Dynamic Nearest Neighbor Set $R$: Nodes added to $C$ are used to update $R$, ensuring that it maintains no more than $k$ elements, where $k$ is the specified number of query results.
\end{itemize}

In the greedy traversal, if a node $x$ expanded from $Q$ satisfies $
d(n,q)>d(n,f)$, where $f$ is the furthest node from $R$ to $q$, then the traversal stops.


Based on the discussions above, we propose a hierarchical search approach, which retrieves the top $k$ relevant elements in each layer using the method outlined in Algorithm \ref{alg:query}.
% 
% We leverage the C-HNSW index and the C-HNSW query algorithm to retrieve the top $k$ nearest neighbors at each level, starting from the top layer.
% 
Once the query at a given layer is complete, we use the inter-layer link of the nearest neighbor in that layer to traverse to the next layer, repeating the process until reaching the bottom layer.
% 
At the bottom layer, we extract the relationships between connected retrieved entities, forming the text information $R_0$ for this subgraph.
% 
We represent the retrieved information from each layer as $R_i$, where $i \in \{0,1,\cdots, L\}$, to be used in the adaptive filtering-based generation process.

% C-HNSW leverages the structure of hierarchical communities and entities to efficiently query the most relevant information at any layer, including both communities and entities.
% % 
% At each layer, we select the top-k communities or entities most relevant to the query embedding to address queries with varying levels of detail.
% % 
% Specifically, starting from any point in the top layer, we progressively traverse and search, maintaining the nearest neighbor queue.
% 



\subsubsection{Adaptive filtering-based generation. }
While some optimized LLMs support longer text inputs, they may still encounter issues such as the ``lost in the middle'' dilemma \cite{liu2024lost}. 
% 
Thus, direct utilization of retrieved information comprising multiple text segments for LLM-based answer generation risks compromising output accuracy.

To mitigate this limitation, we propose an adaptive filtering-based method that harnesses the LLM’s inherent reasoning capabilities.
% 
We first prompt the LLM to extract and generate an analysis report from the retrieved information, identifying the parts that are most relevant to answering the query and assigning relevance scores to these reports.
% 
Then, all analysis reports are integrated and sorted, ensuring that the most relevant content is used to summarize the final response to the query, with any content exceeding the text limit being truncated.
% 
This process can be represented as:
\begin{align}
A_i &= LLM(P_{filter}||R_i) \\
Output&=LLM(P_{merge}||Sort(\{A_0,A_1,\cdots,A_n\}))
\end{align}
where $P_{filter}$ and $P_{merge}$ represent the prompts for extracting relevant information and summarizing, respectively, 
$A_i$, $i \in 0 \cdots n$ denotes the filtered analysis report.
% 
The sort function orders the content based on the relevance scores from the analysis report. 

We also analyze the complexity of ArchRAG in Appendix \ref{sec:complexity}.

% The Map-Reduce Analysis Generation approach enables the LLM to efficiently utilize a greater amount of retrieved information, leading to more accurate responses and summaries.
% % 
% By minimizing the number of LLM calls and leveraging the LLM's analytical reasoning, a more powerful RAG framework is achieved.


\section{Implementation and Evaluation}
\label{sec:evaluation}

We prototype our proposal into a tool \toolName, using approximately 5K lines of OCaml (for the program analysis) and 5K lines of Python code (for the repair). 
In particular, we employ Z3~\cite{DBLP:conf/tacas/MouraB08} as the SMT solver, clingo~\cite{DBLP:books/sp/Lifschitz19} as the ASP solver, and Souffle~\cite{scholz2016fast} as the Datalog engine. %, respectively.
To show the effectiveness, 
we design the experimental evaluation to answer the 
following research questions (RQ):
(Experiments ran on a server with an Intel® Xeon® Platinum 8468V, 504GB RAM, and 192 cores. All the dataset are publicly available from \cite{zenodo_benchmark})

\begin{itemize}[align=left, leftmargin=*,labelindent=0pt]
\item \textbf{RQ1:} How effective is \toolName in verifying CTL properties for relatively small but complex programs, compared to the state-of-the-art tool  \function~\cite{DBLP:conf/sas/UrbanU018}?


\item \textbf{RQ2:} What is the effectiveness of \toolName in detecting real-world bugs, which can be encoded using both CTL and linear temporal logic (LTL), such as non-termination gathered from GitHub \cite{DBLP:conf/sigsoft/ShiXLZCL22} and unresponsive behaviours in protocols  \cite{DBLP:conf/icse/MengDLBR22}, compared with \ultimate~\cite{DBLP:conf/cav/DietschHLP15}?

\item \textbf{RQ3:} How effective is \toolName in repairing CTL violations identified in RQ1 and RQ2? which has not been achieved by any existing tools. 


 

\end{itemize}



% \begin{itemize}[align=left, leftmargin=*,labelindent=0pt]
% \item \textbf{RQ1:} What is the effectiveness of \toolName in verifying CTL properties in a set of relatively small yet challenging programs, compared to the state-of-the-art tools, T2~\cite{DBLP:conf/fmcad/CookKP14},  \function~\cite{DBLP:conf/sas/UrbanU018}, and \ultimate~\cite{DBLP:conf/cav/DietschHLP15}?


% \item \textbf{RQ2:} What is the effectiveness of \toolName in finding  real-world bugs, which can be encoded using CTL properties, such as non-termination 
% gathered from GitHub \cite{DBLP:conf/sigsoft/ShiXLZCL22} and unresponsive behaviours in protocol implementations \cite{DBLP:conf/icse/MengDLBR22}?

% \item \textbf{RQ3:} What is the effectiveness of \toolName in repairing CTL bugs from RQ1--2?

% \end{itemize}

%The benchmark programs are from various sources. More specifically, termination bugs from real-world projects \cite{DBLP:conf/sigsoft/ShiXLZCL22} and CTL analysis \cite{DBLP:conf/fmcad/CookKP14} \cite{DBLP:conf/sas/UrbanU018}, and temporal bugs in real-world protocol implementations \cite{DBLP:conf/icse/MengDLBR22}. 



% \ly{are termination bugs ok? Do we need to add new CTL bugs?}
\subsection{RQ1: Verifying CTL Properties}

% Please add the following required packages to your document preamble:
%  \Xhline{1.5\arrayrulewidth}

\hide{\begin{figure}[!h]
\vspace{-8mm}
\begin{lstlisting}[xleftmargin=0.2em,numbersep=6pt,basicstyle=\footnotesize\ttfamily]
(*@\textcolor{mGray}{//$EF(\m{resp}{\geq}5)$}@*)
int c = *; int resp = 0;
int curr_serv = 5; 
while (curr_serv > 0){ 
 if (*) {  
   c--; 
   curr_serv--;
   resp++;} 
 else if (c<curr_serv){
   curr_serv--; }}
\end{lstlisting} 
\vspace{-2mm}
\caption{A possibly terminating loop} 
\label{fig:terminating_loop}
\vspace{-2mm}
\end{figure}}


%loses precision due to a \emph{dual widening} \cite{DBLP:conf/tacas/CourantU17}, and 

The programs listed in \tabref{tab:comparewithFuntionT2} were obtained from the evaluation benchmark of \function, which includes a total of 83 test cases across over 2,000 lines of code. We categorize these test cases into six groups, labeled according to the types of CTL properties. 
These programs are short but challenging, as they often involve complex loops or require a more precise analysis of the target properties. The \function tends to be conservative, often leading it to return ``unknown" results, resulting in an accuracy rate of 27.7\%. In contrast, \toolName demonstrates advantages with improved accuracy, particularly in \ourToolSmallBenchmark. 
%achieved by the novel loop summaries. 
The failure cases faced by \toolName highlight our limitations when loop guards are not explicitly defined or when LRFs are inadequate to prove termination. 
Although both \function and \toolName struggle to obtain meaningful invariances for infinite loops, the benefits of our loop summaries become more apparent when proving properties related to termination, such as reachability and responsiveness.  




\begin{table}[!t]
\vspace{1.5mm}
\caption{Detecting real-world CTL bugs.}
\normalsize
\label{tab:comparewithCook}
\renewcommand{\arraystretch}{0.95}
\setlength{\tabcolsep}{4pt}  
\begin{tabular}{c|l|c|cc|cc}
\Xhline{1.5\arrayrulewidth}
\multicolumn{1}{l|}{\multirow{2}{*}{\textbf{}}} & \multirow{2}{*}{\textbf{Program}}        & \multirow{2}{*}{\textbf{LoC}} & \multicolumn{2}{c|}{\textbf{\ultimateshort}}   & \multicolumn{2}{c}{\textbf{\toolName}}             \\ \cline{4-7} 
  \multicolumn{1}{l|}{}                           &                                          &                               & \multicolumn{1}{c|}{\textbf{Res.}} & \textbf{Time} & \multicolumn{1}{c|}{\textbf{Res.}} & \textbf{Time} \\ \hline
  1 \xmark                                      & \multirow{2}{*}{\makecell[l]{libvncserver\\(c311535)}}   & 25                            & \multicolumn{1}{c|}{\xmark}      & 2.845         & \multicolumn{1}{c|}{\xmark}      & 0.855         \\  
  1 \cmark                                      &                                          & 27                            & \multicolumn{1}{c|}{\cmark}      & 3.743         & \multicolumn{1}{c|}{\cmark}      & 0.476         \\ \hline
  2 \xmark                                      & \multirow{2}{*}{\makecell[l]{Ffmpeg\\(a6cba06)}}         & 40                            & \multicolumn{1}{c|}{\xmark}      & 15.254        & \multicolumn{1}{c|}{\xmark}      & 0.606         \\  
  2 \cmark                                      &                                          & 44                            & \multicolumn{1}{c|}{\cmark}      & 40.176        & \multicolumn{1}{c|}{\cmark}      & 0.397         \\ \hline
  3 \xmark                                      & \multirow{2}{*}{\makecell[l]{cmus\\(d5396e4)}}           & 87                            & \multicolumn{1}{c|}{\xmark}      & 6.904         & \multicolumn{1}{c|}{\xmark}      & 0.579         \\  
  3 \cmark                                      &                                          & 86                            & \multicolumn{1}{c|}{\cmark}      & 33.572        & \multicolumn{1}{c|}{\cmark}      & 0.986         \\ \hline
  4 \xmark                                      & \multirow{2}{*}{\makecell[l]{e2fsprogs\\(caa6003)}}      & 58                            & \multicolumn{1}{c|}{\xmark}      & 5.952         & \multicolumn{1}{c|}{\xmark}      & 0.923         \\  
  4 \cmark                                      &                                          & 63                            & \multicolumn{1}{c|}{\cmark}      & 4.533         & \multicolumn{1}{c|}{\cmark}      & 0.842         \\ \hline
  5 \xmark                                      & \multirow{2}{*}{\makecell[l]{csound-an...\\(7a611ab)}} & 43                            & \multicolumn{1}{c|}{\xmark}      & 3.654         & \multicolumn{1}{c|}{\xmark}      & 0.782         \\  
  5 \cmark                                      &                                          & 45                            & \multicolumn{1}{c|}{TO}          & -             & \multicolumn{1}{c|}{\cmark}      & 0.648         \\ \hline
  6 \xmark                                      & \multirow{2}{*}{\makecell[l]{fontconfig\\(fa741cd)}}     & 25                            & \multicolumn{1}{c|}{\xmark}      & 3.856         & \multicolumn{1}{c|}{\xmark}      & 0.769         \\  
  6 \cmark                                      &                                          & 25                            & \multicolumn{1}{c|}{Error}       & -             & \multicolumn{1}{c|}{\cmark}      & 0.651         \\ \hline
  7 \xmark                                      & \multirow{2}{*}{\makecell[l]{asterisk\\(3322180)}}       & 22                            & \multicolumn{1}{c|}{\unk}        & 12.687        & \multicolumn{1}{c|}{\unk}        & 0.196         \\  
  7 \cmark                                      &                                          & 25                            & \multicolumn{1}{c|}{\unk}        & 11.325        & \multicolumn{1}{c|}{\unk}        & 0.34          \\ \hline
  8 \xmark                                      & \multirow{2}{*}{\makecell[l]{dpdk\\(cd64eeac)}}          & 45                            & \multicolumn{1}{c|}{\xmark}      & 3.712         & \multicolumn{1}{c|}{\xmark}      & 0.447         \\  
  8 \cmark                                      &                                          & 45                            & \multicolumn{1}{c|}{\cmark}      & 2.97          & \multicolumn{1}{c|}{\unk}        & 0.481         \\ \hline
  9 \xmark                                      & \multirow{2}{*}{\makecell[l]{xorg-server\\(930b9a06)}}   & 19                            & \multicolumn{1}{c|}{\xmark}      & 3.111         & \multicolumn{1}{c|}{\xmark}      & 0.581         \\  
  9 \cmark                                      &                                          & 20                            & \multicolumn{1}{c|}{\cmark}      & 3.101         & \multicolumn{1}{c|}{\cmark}      & 0.409         \\ \hline
  10 \xmark                                      & \multirow{2}{*}{\makecell[l]{pure-ftpd\\(37ad222)}}      & 42                            & \multicolumn{1}{c|}{\cmark}      & 2.555         & \multicolumn{1}{c|}{\xmark}      & 0.933         \\  
  10 \cmark                                      &                                          & 49                            & \multicolumn{1}{c|}{\cmark}        & 2.286         & \multicolumn{1}{c|}{\cmark}      & 0.383         \\ \hline
  11 \xmark  & \multirow{2}{*}{\makecell[l]{live555$_a$\\(181126)}} & 34  & \multicolumn{1}{c|}{\cmark} &  2.715         & \multicolumn{1}{c|}{\xmark}    & 0.513   \\  
  11 \cmark  &     &   37    & \multicolumn{1}{c|}{\cmark} &  2.837         & \multicolumn{1}{c|}{\cmark}      & 0.341 \\ \hline
  12 \xmark  & \multirow{2}{*}{\makecell[l]{openssl\\(b8d2439)}} & 88  & \multicolumn{1}{c|}{\xmark} &  4.15          & \multicolumn{1}{c|}{\xmark}    & 0.78   \\
  12 \cmark  &     &  88     & \multicolumn{1}{c|}{\cmark} &  3.809         & \multicolumn{1}{c|}{\cmark}      & 0.99 \\ \hline
  13 \xmark  & \multirow{2}{*}{\makecell[l]{live555$_b$\\(131205)}} & 83  & \multicolumn{1}{c|}{\xmark} & 2.838         & \multicolumn{1}{c|}{\xmark}    & 0.602     \\  
  13 \cmark  &    &   84     & \multicolumn{1}{c|}{\cmark} &  2.393         & \multicolumn{1}{c|}{\cmark}      & 0.565 \\ \Xhline{1.5\arrayrulewidth}
                                                   & {\bf{Total}}                                  & 1249  & \multicolumn{1}{c|}{\bestBaseLineReal}          & $>$180       & \multicolumn{1}{c|}{\ourToolRealBenchmark}              & 16.01        \\ \Xhline{1.5\arrayrulewidth}
  \end{tabular}
  \end{table}

\subsection{RQ2: CTL Analysis on  Real-world Projects}




Programs in \tabref{tab:comparewithCook} are from real-world repositories, each associated with a Git commit number where developers identify and fix the bug manually. 
In particular, the property used for programs 1-9 (drawn from \cite{DBLP:conf/sigsoft/ShiXLZCL22}) is  \code{AF(Exit())}, capturing non-termination bugs. The properties used for programs 10-13 (drawn from \cite{DBLP:conf/icse/MengDLBR22}) are of the form \code{AG(\phi_1{\rightarrow}AF(\phi_2))}, capturing unresponsive behaviours from the protocol implementation. 
We extracted the main segments of these real-world bugs into smaller programs (under 100 LoC each), preserving features like data structures and pointer arithmetic. Our evaluation includes both buggy (\eg 1\,\xmark) and developer-fixed (\eg 1\,\cmark) versions.
After converting the CTL properties to LTL formulas, we compared our tool with the latest release of UltimateLTL (v0.2.4), a regular participant in SV-COMP \cite{svcomp} with competitive performance. 
Both tools demonstrate high accuracy in bug detection, while \ultimateshort often requires longer processing time. 
This experiment indicates that LRFs can effectively handle commonly seen real-world loops, and \toolName performs a more lightweight summary computation without compromising accuracy. 



%Following the convention in \cite{DBLP:conf/sigsoft/ShiXLZCL22}, t
%Prior works \cite{DBLP:conf/sigsoft/ShiXLZCL22} gathered such examples by extracting 
%\toolName successfully identifies the majority of buggy and correct programs, with the exception of programs 7 and 8. 







{
\begin{table*}[!h]
  \centering
\caption{\label{tab:repair_benchmark}
{Experimental results for repairing CTL bugs. Time spent by the ASP solver is separately recorded. 
}
}
\small
\renewcommand{\arraystretch}{0.95}
  \setlength{\tabcolsep}{9pt}
\begin{tabular}{l|c|c|c|c|c|c|c|c}
  \Xhline{1.5\arrayrulewidth}
  \multicolumn{1}{c|}{\multirow{2}{*}{\textbf{Program}}} & \multicolumn{1}{c|}{\multirow{2}{*}{\shortstack{\textbf{LoC}\\\textbf{(Datalog)}}}} & \multicolumn{3}{c|}{\textbf{Configuration}}                                 & \multicolumn{1}{c|}{\multirow{2}{*}{\textbf{Fixed}}} & \multicolumn{1}{c|}{\multirow{2}{*}{\textbf{\#Patch}}} & \multicolumn{1}{c|}{\multirow{2}{*}{\textbf{ASP(s)}}} & \multirow{2}{*}{\textbf{Total(s)}} \\ \cline{3-5}

  \multicolumn{1}{c|}{}                                  & \multicolumn{1}{c|}{}                              & \multicolumn{1}{c|}{\textbf{Symbols}} & \multicolumn{1}{c|}{\textbf{Facts}} & \multicolumn{1}{c|}{\textbf{Template}} & \multicolumn{1}{c|}{} & \multicolumn{1}{c|}{} & \multicolumn{1}{c|}{}  &                                      \\ \hline

AF\_yEQ5 (\figref{fig:first_Example})                                           & 115                           & 3+0                   & 0+1                & Add                & \cmark     & 1                   & 0.979                              & 1.593                                \\
test\_until.c                                         & 101                            & 0+3                   & 1+0                & Delete                & \cmark     & 1                   & 0.023                              & 0.498                                \\
next.c                                                & 87                            & 0+4                   & 1+0                & Delete                & \cmark     & 1                   & 0.023                              & 0.472                                \\
libvncserver                                          & 118                            & 0+6                   & 1+0                & Delete                & \cmark     & 3                   & 0.049                              & 1.081                                \\
Ffmpeg                                                & 227                           & 0+12                  & 1+0                & Delete                & \cmark     & 4                   & 13.113                              & 13.335                                \\
cmus                                                  & 145                           & 0+12                  & 1+0                & Delete                & \cmark     & 4                   & 0.098                              & 2.052                                \\
e2fsprogs                                             & 109                           & 0+8                   & 1+0                & Delete                & \cmark     & 2                   & 0.075                              & 1.515                                \\
csound-android                                        & 183                           & 0+8                   & 1+0                & Delete                & \cmark     & 4                   & 0.076                              & 1.613                                \\
fontconfig                                            & 190                           & 0+11                  & 1+0                & Delete                & \cmark     & 6                   & 0.098                              & 2.507                                \\
dpdk                                                  & 196                           & 0+12                  & 1+0                & Delete                & \cmark     & 1                   & 0.091                              & 2.006                                \\
xorg-server                                           & 118                            & 0+2                   & 1+0                & Delete                & \cmark     & 2                   & 0.026                              & 0.605                                \\
pure-ftpd                                             & 258                           & 0+21                  & 1+0                & Delete                & \cmark     & 2                   & 0.069                              & 3.590                               \\
live$_a$                                              & 112                            & 3+4                   & 1+1                & Update                & \cmark     & 1                   & 0.552                              & 0.816                                \\
openssl                                               & 315                           & 1+0                   & 0+1                & Add.                & \cmark     & 1                   & 1.188                              & 2.277                                \\
live$_b$                                              & 217                           & 1+0                   & 0+1                & Add                & \cmark     & 1                   & 0.977                              & 1.494                                 \\
  \Xhline{1.5\arrayrulewidth}
\textbf{Total}                                                 & 2491                          &                       &                    &                   &           &                     & 17.437                              & 35.454                               \\ 
  \Xhline{1.5\arrayrulewidth}           
\end{tabular}

\vspace{-2mm}
\end{table*}
}


\subsection{RQ3: Repairing CTL Property Violations} 


\tabref{tab:repair_benchmark} gathers all the program instances (from \tabref{tab:comparewithFuntionT2} and \tabref{tab:comparewithCook}) that violate their specified CTL properties and are sent to \toolName for repair.   
The \textbf{Symbols} column records the number of symbolic constants + symbolic signs, while the \textbf{Facts} column records the number of facts allowed to be removed + added. 
We gradually increase the number of symbols and the maximum number of facts that can be added or deleted. 
The \textbf{Configuration} column shows the first successful configuration that led to finding patches, and we record the total searching time till reaching such configurations. 
We configure \toolName to apply three atomic templates in a breadth-first manner with a depth limit of 1, \ie, \tabref{tab:repair_benchmark} records the patch result after one iteration of the repair. 
The templates are applied sequentially in the order: delete, update, and add. The repair process stops when a correct patch is found or when all three templates have been attempted. 
%without success. 
% Because of this configuration, \toolName only finds one patch for Program 1 (AF\_yEQ5). 
% The patch inserting \plaincode{if (i>10||x==y) \{y=5; return;\}} mentioned in \figref{fig:Patched-program} cannot be found in current configuration, as it requires deleting facts then adding new facts on the updated program.
% The `Configuration' column in \tabref{tab:repair_benchmark} shows the number of symbolic constants and signs, the number of facts allowed to be removed and added, and the template used when a patch is found.

Due to the current configuration, \toolName only finds patch (b) for Program 1 (AF\_yEQ5), while the patch (a) shown in \figref{fig:Patched-program} can be obtained by allowing two iterations of the repair: the first iteration adds the conditional than a second iteration to add a new assignment on the updated program. 
Non-termination bugs are resolved within a single iteration by adding a conditional statement that provides an earlier exit. 
For instance, \figref{fig:term-Patched-program} illustrates the main logic of 1\,\xmark, which enters an infinite loop when \code{\m{linesToRead}{\leq}0}. 
\toolName successfully 
provides a fix that prevents \code{\m{linesToRead}{\leq}0} from occurring before entering the loop. Note that such patches are more desirable which fix the non-termination bug without dropping the loops completely. 
%much like the example shown in  \figref{fig:term-Patched-program}. At the same time, 
Unresponsive bugs involve adding more function calls or assignment modifications. 
%Most repairs were completed within seconds. 

On average, the time taken to solve ASP accounts for 49.2\% (17.437/35.454) of the total repair time. We also keep track of the number of patches that successfully eliminate the CTL violations. More than one patch is available for non-termination bugs, as some patches exit the entire program without entering the loop. 
While all the patches listed are valid, those that intend to cut off the main program logic can be excluded based on the minimum change criteria. 
After a manual inspection of each buggy program shown in \tabref{tab:repair_benchmark}, we confirmed that at least one generated patch is semantically equivalent to the fix provided by the developer. 
As the first tool to achieve automated repair of CTL violations, \toolName successfully resolves all reported bugs. 



\begin{figure}[!t]
\begin{lstlisting}[xleftmargin=6em,numbersep=6pt,basicstyle=\footnotesize\ttfamily]
void main(){ //AF(Exit())
  int lines ToRead = *;
  int h = *;
  (*@\repaircode{if ( linesToRead <= 0 )  return;}@*)
  while(h>0){
    if(linesToRead>h)  
        linesToRead=h; 
    h-=linesToRead;} 
  return;}
\end{lstlisting}
\caption{Fixing a Possible Hang Found in libvncserver \cite{LibVNCClient}}
\label{fig:term-Patched-program}
\end{figure}


\putsec{related}{Related Work}

\noindent \textbf{Efficient Radiance Field Rendering.}
%
The introduction of Neural Radiance Fields (NeRF)~\cite{mil:sri20} has
generated significant interest in efficient 3D scene representation and
rendering for radiance fields.
%
Over the past years, there has been a large amount of research aimed at
accelerating NeRFs through algorithmic or software
optimizations~\cite{mul:eva22,fri:yu22,che:fun23,sun:sun22}, and the
development of hardware
accelerators~\cite{lee:cho23,li:li23,son:wen23,mub:kan23,fen:liu24}.
%
The state-of-the-art method, 3D Gaussian splatting~\cite{ker:kop23}, has
further fueled interest in accelerating radiance field
rendering~\cite{rad:ste24,lee:lee24,nie:stu24,lee:rho24,ham:mel24} as it
employs rasterization primitives that can be rendered much faster than NeRFs.
%
However, previous research focused on software graphics rendering on
programmable cores or building dedicated hardware accelerators. In contrast,
\name{} investigates the potential of efficient radiance field rendering while
utilizing fixed-function units in graphics hardware.
%
To our knowledge, this is the first work that assesses the performance
implications of rendering Gaussian-based radiance fields on the hardware
graphics pipeline with software and hardware optimizations.

%%%%%%%%%%%%%%%%%%%%%%%%%%%%%%%%%%%%%%%%%%%%%%%%%%%%%%%%%%%%%%%%%%%%%%%%%%
\myparagraph{Enhancing Graphics Rendering Hardware.}
%
The performance advantage of executing graphics rendering on either
programmable shader cores or fixed-function units varies depending on the
rendering methods and hardware designs.
%
Previous studies have explored the performance implication of graphics hardware
design by developing simulation infrastructures for graphics
workloads~\cite{bar:gon06,gub:aam19,tin:sax23,arn:par13}.
%
Additionally, several studies have aimed to improve the performance of
special-purpose hardware such as ray tracing units in graphics
hardware~\cite{cho:now23,liu:cha21} and proposed hardware accelerators for
graphics applications~\cite{lu:hua17,ram:gri09}.
%
In contrast to these works, which primarily evaluate traditional graphics
workloads, our work focuses on improving the performance of volume rendering
workloads, such as Gaussian splatting, which require blending a huge number of
fragments per pixel.

%%%%%%%%%%%%%%%%%%%%%%%%%%%%%%%%%%%%%%%%%%%%%%%%%%%%%%%%%%%%%%%%%%%%%%%%%%
%
In the context of multi-sample anti-aliasing, prior work proposed reducing the
amount of redundant shading by merging fragments from adjacent triangles in a
mesh at the quad granularity~\cite{fat:bou10}.
%
While both our work and quad-fragment merging (QFM)~\cite{fat:bou10} aim to
reduce operations by merging quads, our proposed technique differs from QFM in
many aspects.
%
Our method aims to blend \emph{overlapping primitives} along the depth
direction and applies to quads from any primitive. In contrast, QFM merges quad
fragments from small (e.g., pixel-sized) triangles that \emph{share} an edge
(i.e., \emph{connected}, \emph{non-overlapping} triangles).
%
As such, QFM is not applicable to the scenes consisting of a number of
unconnected transparent triangles, such as those in 3D Gaussian splatting.
%
In addition, our method computes the \emph{exact} color for each pixel by
offloading blending operations from ROPs to shader units, whereas QFM
\emph{approximates} pixel colors by using the color from one triangle when
multiple triangles are merged into a single quad.


\section{Conclusion}
In this work, we propose a simple yet effective approach, called SMILE, for graph few-shot learning with fewer tasks. Specifically, we introduce a novel dual-level mixup strategy, including within-task and across-task mixup, for enriching the diversity of nodes within each task and the diversity of tasks. Also, we incorporate the degree-based prior information to learn expressive node embeddings. Theoretically, we prove that SMILE effectively enhances the model's generalization performance. Empirically, we conduct extensive experiments on multiple benchmarks and the results suggest that SMILE significantly outperforms other baselines, including both in-domain and cross-domain few-shot settings.

% \section{Acknowledgements}

\bibliography{main}
\bibliographystyle{icml2025}

%%%%%%%%%%%%%%%%%%%%%%%%%%%%%%%%%%%%%%%%%%%%%%%%%%%%%%%%%%%%%%%%%%%%%%%%%%%%%%%
%%%%%%%%%%%%%%%%%%%%%%%%%%%%%%%%%%%%%%%%%%%%%%%%%%%%%%%%%%%%%%%%%%%%%%%%%%%%%%%
% APPENDIX
%%%%%%%%%%%%%%%%%%%%%%%%%%%%%%%%%%%%%%%%%%%%%%%%%%%%%%%%%%%%%%%%%%%%%%%%%%%%%%%
%%%%%%%%%%%%%%%%%%%%%%%%%%%%%%%%%%%%%%%%%%%%%%%%%%%%%%%%%%%%%%%%%%%%%%%%%%%%%%%
\newpage
\appendix
\onecolumn
\subsection{Lloyd-Max Algorithm}
\label{subsec:Lloyd-Max}
For a given quantization bitwidth $B$ and an operand $\bm{X}$, the Lloyd-Max algorithm finds $2^B$ quantization levels $\{\hat{x}_i\}_{i=1}^{2^B}$ such that quantizing $\bm{X}$ by rounding each scalar in $\bm{X}$ to the nearest quantization level minimizes the quantization MSE. 

The algorithm starts with an initial guess of quantization levels and then iteratively computes quantization thresholds $\{\tau_i\}_{i=1}^{2^B-1}$ and updates quantization levels $\{\hat{x}_i\}_{i=1}^{2^B}$. Specifically, at iteration $n$, thresholds are set to the midpoints of the previous iteration's levels:
\begin{align*}
    \tau_i^{(n)}=\frac{\hat{x}_i^{(n-1)}+\hat{x}_{i+1}^{(n-1)}}2 \text{ for } i=1\ldots 2^B-1
\end{align*}
Subsequently, the quantization levels are re-computed as conditional means of the data regions defined by the new thresholds:
\begin{align*}
    \hat{x}_i^{(n)}=\mathbb{E}\left[ \bm{X} \big| \bm{X}\in [\tau_{i-1}^{(n)},\tau_i^{(n)}] \right] \text{ for } i=1\ldots 2^B
\end{align*}
where to satisfy boundary conditions we have $\tau_0=-\infty$ and $\tau_{2^B}=\infty$. The algorithm iterates the above steps until convergence.

Figure \ref{fig:lm_quant} compares the quantization levels of a $7$-bit floating point (E3M3) quantizer (left) to a $7$-bit Lloyd-Max quantizer (right) when quantizing a layer of weights from the GPT3-126M model at a per-tensor granularity. As shown, the Lloyd-Max quantizer achieves substantially lower quantization MSE. Further, Table \ref{tab:FP7_vs_LM7} shows the superior perplexity achieved by Lloyd-Max quantizers for bitwidths of $7$, $6$ and $5$. The difference between the quantizers is clear at 5 bits, where per-tensor FP quantization incurs a drastic and unacceptable increase in perplexity, while Lloyd-Max quantization incurs a much smaller increase. Nevertheless, we note that even the optimal Lloyd-Max quantizer incurs a notable ($\sim 1.5$) increase in perplexity due to the coarse granularity of quantization. 

\begin{figure}[h]
  \centering
  \includegraphics[width=0.7\linewidth]{sections/figures/LM7_FP7.pdf}
  \caption{\small Quantization levels and the corresponding quantization MSE of Floating Point (left) vs Lloyd-Max (right) Quantizers for a layer of weights in the GPT3-126M model.}
  \label{fig:lm_quant}
\end{figure}

\begin{table}[h]\scriptsize
\begin{center}
\caption{\label{tab:FP7_vs_LM7} \small Comparing perplexity (lower is better) achieved by floating point quantizers and Lloyd-Max quantizers on a GPT3-126M model for the Wikitext-103 dataset.}
\begin{tabular}{c|cc|c}
\hline
 \multirow{2}{*}{\textbf{Bitwidth}} & \multicolumn{2}{|c|}{\textbf{Floating-Point Quantizer}} & \textbf{Lloyd-Max Quantizer} \\
 & Best Format & Wikitext-103 Perplexity & Wikitext-103 Perplexity \\
\hline
7 & E3M3 & 18.32 & 18.27 \\
6 & E3M2 & 19.07 & 18.51 \\
5 & E4M0 & 43.89 & 19.71 \\
\hline
\end{tabular}
\end{center}
\end{table}

\subsection{Proof of Local Optimality of LO-BCQ}
\label{subsec:lobcq_opt_proof}
For a given block $\bm{b}_j$, the quantization MSE during LO-BCQ can be empirically evaluated as $\frac{1}{L_b}\lVert \bm{b}_j- \bm{\hat{b}}_j\rVert^2_2$ where $\bm{\hat{b}}_j$ is computed from equation (\ref{eq:clustered_quantization_definition}) as $C_{f(\bm{b}_j)}(\bm{b}_j)$. Further, for a given block cluster $\mathcal{B}_i$, we compute the quantization MSE as $\frac{1}{|\mathcal{B}_{i}|}\sum_{\bm{b} \in \mathcal{B}_{i}} \frac{1}{L_b}\lVert \bm{b}- C_i^{(n)}(\bm{b})\rVert^2_2$. Therefore, at the end of iteration $n$, we evaluate the overall quantization MSE $J^{(n)}$ for a given operand $\bm{X}$ composed of $N_c$ block clusters as:
\begin{align*}
    \label{eq:mse_iter_n}
    J^{(n)} = \frac{1}{N_c} \sum_{i=1}^{N_c} \frac{1}{|\mathcal{B}_{i}^{(n)}|}\sum_{\bm{v} \in \mathcal{B}_{i}^{(n)}} \frac{1}{L_b}\lVert \bm{b}- B_i^{(n)}(\bm{b})\rVert^2_2
\end{align*}

At the end of iteration $n$, the codebooks are updated from $\mathcal{C}^{(n-1)}$ to $\mathcal{C}^{(n)}$. However, the mapping of a given vector $\bm{b}_j$ to quantizers $\mathcal{C}^{(n)}$ remains as  $f^{(n)}(\bm{b}_j)$. At the next iteration, during the vector clustering step, $f^{(n+1)}(\bm{b}_j)$ finds new mapping of $\bm{b}_j$ to updated codebooks $\mathcal{C}^{(n)}$ such that the quantization MSE over the candidate codebooks is minimized. Therefore, we obtain the following result for $\bm{b}_j$:
\begin{align*}
\frac{1}{L_b}\lVert \bm{b}_j - C_{f^{(n+1)}(\bm{b}_j)}^{(n)}(\bm{b}_j)\rVert^2_2 \le \frac{1}{L_b}\lVert \bm{b}_j - C_{f^{(n)}(\bm{b}_j)}^{(n)}(\bm{b}_j)\rVert^2_2
\end{align*}

That is, quantizing $\bm{b}_j$ at the end of the block clustering step of iteration $n+1$ results in lower quantization MSE compared to quantizing at the end of iteration $n$. Since this is true for all $\bm{b} \in \bm{X}$, we assert the following:
\begin{equation}
\begin{split}
\label{eq:mse_ineq_1}
    \tilde{J}^{(n+1)} &= \frac{1}{N_c} \sum_{i=1}^{N_c} \frac{1}{|\mathcal{B}_{i}^{(n+1)}|}\sum_{\bm{b} \in \mathcal{B}_{i}^{(n+1)}} \frac{1}{L_b}\lVert \bm{b} - C_i^{(n)}(b)\rVert^2_2 \le J^{(n)}
\end{split}
\end{equation}
where $\tilde{J}^{(n+1)}$ is the the quantization MSE after the vector clustering step at iteration $n+1$.

Next, during the codebook update step (\ref{eq:quantizers_update}) at iteration $n+1$, the per-cluster codebooks $\mathcal{C}^{(n)}$ are updated to $\mathcal{C}^{(n+1)}$ by invoking the Lloyd-Max algorithm \citep{Lloyd}. We know that for any given value distribution, the Lloyd-Max algorithm minimizes the quantization MSE. Therefore, for a given vector cluster $\mathcal{B}_i$ we obtain the following result:

\begin{equation}
    \frac{1}{|\mathcal{B}_{i}^{(n+1)}|}\sum_{\bm{b} \in \mathcal{B}_{i}^{(n+1)}} \frac{1}{L_b}\lVert \bm{b}- C_i^{(n+1)}(\bm{b})\rVert^2_2 \le \frac{1}{|\mathcal{B}_{i}^{(n+1)}|}\sum_{\bm{b} \in \mathcal{B}_{i}^{(n+1)}} \frac{1}{L_b}\lVert \bm{b}- C_i^{(n)}(\bm{b})\rVert^2_2
\end{equation}

The above equation states that quantizing the given block cluster $\mathcal{B}_i$ after updating the associated codebook from $C_i^{(n)}$ to $C_i^{(n+1)}$ results in lower quantization MSE. Since this is true for all the block clusters, we derive the following result: 
\begin{equation}
\begin{split}
\label{eq:mse_ineq_2}
     J^{(n+1)} &= \frac{1}{N_c} \sum_{i=1}^{N_c} \frac{1}{|\mathcal{B}_{i}^{(n+1)}|}\sum_{\bm{b} \in \mathcal{B}_{i}^{(n+1)}} \frac{1}{L_b}\lVert \bm{b}- C_i^{(n+1)}(\bm{b})\rVert^2_2  \le \tilde{J}^{(n+1)}   
\end{split}
\end{equation}

Following (\ref{eq:mse_ineq_1}) and (\ref{eq:mse_ineq_2}), we find that the quantization MSE is non-increasing for each iteration, that is, $J^{(1)} \ge J^{(2)} \ge J^{(3)} \ge \ldots \ge J^{(M)}$ where $M$ is the maximum number of iterations. 
%Therefore, we can say that if the algorithm converges, then it must be that it has converged to a local minimum. 
\hfill $\blacksquare$


\begin{figure}
    \begin{center}
    \includegraphics[width=0.5\textwidth]{sections//figures/mse_vs_iter.pdf}
    \end{center}
    \caption{\small NMSE vs iterations during LO-BCQ compared to other block quantization proposals}
    \label{fig:nmse_vs_iter}
\end{figure}

Figure \ref{fig:nmse_vs_iter} shows the empirical convergence of LO-BCQ across several block lengths and number of codebooks. Also, the MSE achieved by LO-BCQ is compared to baselines such as MXFP and VSQ. As shown, LO-BCQ converges to a lower MSE than the baselines. Further, we achieve better convergence for larger number of codebooks ($N_c$) and for a smaller block length ($L_b$), both of which increase the bitwidth of BCQ (see Eq \ref{eq:bitwidth_bcq}).


\subsection{Additional Accuracy Results}
%Table \ref{tab:lobcq_config} lists the various LOBCQ configurations and their corresponding bitwidths.
\begin{table}
\setlength{\tabcolsep}{4.75pt}
\begin{center}
\caption{\label{tab:lobcq_config} Various LO-BCQ configurations and their bitwidths.}
\begin{tabular}{|c||c|c|c|c||c|c||c|} 
\hline
 & \multicolumn{4}{|c||}{$L_b=8$} & \multicolumn{2}{|c||}{$L_b=4$} & $L_b=2$ \\
 \hline
 \backslashbox{$L_A$\kern-1em}{\kern-1em$N_c$} & 2 & 4 & 8 & 16 & 2 & 4 & 2 \\
 \hline
 64 & 4.25 & 4.375 & 4.5 & 4.625 & 4.375 & 4.625 & 4.625\\
 \hline
 32 & 4.375 & 4.5 & 4.625& 4.75 & 4.5 & 4.75 & 4.75 \\
 \hline
 16 & 4.625 & 4.75& 4.875 & 5 & 4.75 & 5 & 5 \\
 \hline
\end{tabular}
\end{center}
\end{table}

%\subsection{Perplexity achieved by various LO-BCQ configurations on Wikitext-103 dataset}

\begin{table} \centering
\begin{tabular}{|c||c|c|c|c||c|c||c|} 
\hline
 $L_b \rightarrow$& \multicolumn{4}{c||}{8} & \multicolumn{2}{c||}{4} & 2\\
 \hline
 \backslashbox{$L_A$\kern-1em}{\kern-1em$N_c$} & 2 & 4 & 8 & 16 & 2 & 4 & 2  \\
 %$N_c \rightarrow$ & 2 & 4 & 8 & 16 & 2 & 4 & 2 \\
 \hline
 \hline
 \multicolumn{8}{c}{GPT3-1.3B (FP32 PPL = 9.98)} \\ 
 \hline
 \hline
 64 & 10.40 & 10.23 & 10.17 & 10.15 &  10.28 & 10.18 & 10.19 \\
 \hline
 32 & 10.25 & 10.20 & 10.15 & 10.12 &  10.23 & 10.17 & 10.17 \\
 \hline
 16 & 10.22 & 10.16 & 10.10 & 10.09 &  10.21 & 10.14 & 10.16 \\
 \hline
  \hline
 \multicolumn{8}{c}{GPT3-8B (FP32 PPL = 7.38)} \\ 
 \hline
 \hline
 64 & 7.61 & 7.52 & 7.48 &  7.47 &  7.55 &  7.49 & 7.50 \\
 \hline
 32 & 7.52 & 7.50 & 7.46 &  7.45 &  7.52 &  7.48 & 7.48  \\
 \hline
 16 & 7.51 & 7.48 & 7.44 &  7.44 &  7.51 &  7.49 & 7.47  \\
 \hline
\end{tabular}
\caption{\label{tab:ppl_gpt3_abalation} Wikitext-103 perplexity across GPT3-1.3B and 8B models.}
\end{table}

\begin{table} \centering
\begin{tabular}{|c||c|c|c|c||} 
\hline
 $L_b \rightarrow$& \multicolumn{4}{c||}{8}\\
 \hline
 \backslashbox{$L_A$\kern-1em}{\kern-1em$N_c$} & 2 & 4 & 8 & 16 \\
 %$N_c \rightarrow$ & 2 & 4 & 8 & 16 & 2 & 4 & 2 \\
 \hline
 \hline
 \multicolumn{5}{|c|}{Llama2-7B (FP32 PPL = 5.06)} \\ 
 \hline
 \hline
 64 & 5.31 & 5.26 & 5.19 & 5.18  \\
 \hline
 32 & 5.23 & 5.25 & 5.18 & 5.15  \\
 \hline
 16 & 5.23 & 5.19 & 5.16 & 5.14  \\
 \hline
 \multicolumn{5}{|c|}{Nemotron4-15B (FP32 PPL = 5.87)} \\ 
 \hline
 \hline
 64  & 6.3 & 6.20 & 6.13 & 6.08  \\
 \hline
 32  & 6.24 & 6.12 & 6.07 & 6.03  \\
 \hline
 16  & 6.12 & 6.14 & 6.04 & 6.02  \\
 \hline
 \multicolumn{5}{|c|}{Nemotron4-340B (FP32 PPL = 3.48)} \\ 
 \hline
 \hline
 64 & 3.67 & 3.62 & 3.60 & 3.59 \\
 \hline
 32 & 3.63 & 3.61 & 3.59 & 3.56 \\
 \hline
 16 & 3.61 & 3.58 & 3.57 & 3.55 \\
 \hline
\end{tabular}
\caption{\label{tab:ppl_llama7B_nemo15B} Wikitext-103 perplexity compared to FP32 baseline in Llama2-7B and Nemotron4-15B, 340B models}
\end{table}

%\subsection{Perplexity achieved by various LO-BCQ configurations on MMLU dataset}


\begin{table} \centering
\begin{tabular}{|c||c|c|c|c||c|c|c|c|} 
\hline
 $L_b \rightarrow$& \multicolumn{4}{c||}{8} & \multicolumn{4}{c||}{8}\\
 \hline
 \backslashbox{$L_A$\kern-1em}{\kern-1em$N_c$} & 2 & 4 & 8 & 16 & 2 & 4 & 8 & 16  \\
 %$N_c \rightarrow$ & 2 & 4 & 8 & 16 & 2 & 4 & 2 \\
 \hline
 \hline
 \multicolumn{5}{|c|}{Llama2-7B (FP32 Accuracy = 45.8\%)} & \multicolumn{4}{|c|}{Llama2-70B (FP32 Accuracy = 69.12\%)} \\ 
 \hline
 \hline
 64 & 43.9 & 43.4 & 43.9 & 44.9 & 68.07 & 68.27 & 68.17 & 68.75 \\
 \hline
 32 & 44.5 & 43.8 & 44.9 & 44.5 & 68.37 & 68.51 & 68.35 & 68.27  \\
 \hline
 16 & 43.9 & 42.7 & 44.9 & 45 & 68.12 & 68.77 & 68.31 & 68.59  \\
 \hline
 \hline
 \multicolumn{5}{|c|}{GPT3-22B (FP32 Accuracy = 38.75\%)} & \multicolumn{4}{|c|}{Nemotron4-15B (FP32 Accuracy = 64.3\%)} \\ 
 \hline
 \hline
 64 & 36.71 & 38.85 & 38.13 & 38.92 & 63.17 & 62.36 & 63.72 & 64.09 \\
 \hline
 32 & 37.95 & 38.69 & 39.45 & 38.34 & 64.05 & 62.30 & 63.8 & 64.33  \\
 \hline
 16 & 38.88 & 38.80 & 38.31 & 38.92 & 63.22 & 63.51 & 63.93 & 64.43  \\
 \hline
\end{tabular}
\caption{\label{tab:mmlu_abalation} Accuracy on MMLU dataset across GPT3-22B, Llama2-7B, 70B and Nemotron4-15B models.}
\end{table}


%\subsection{Perplexity achieved by various LO-BCQ configurations on LM evaluation harness}

\begin{table} \centering
\begin{tabular}{|c||c|c|c|c||c|c|c|c|} 
\hline
 $L_b \rightarrow$& \multicolumn{4}{c||}{8} & \multicolumn{4}{c||}{8}\\
 \hline
 \backslashbox{$L_A$\kern-1em}{\kern-1em$N_c$} & 2 & 4 & 8 & 16 & 2 & 4 & 8 & 16  \\
 %$N_c \rightarrow$ & 2 & 4 & 8 & 16 & 2 & 4 & 2 \\
 \hline
 \hline
 \multicolumn{5}{|c|}{Race (FP32 Accuracy = 37.51\%)} & \multicolumn{4}{|c|}{Boolq (FP32 Accuracy = 64.62\%)} \\ 
 \hline
 \hline
 64 & 36.94 & 37.13 & 36.27 & 37.13 & 63.73 & 62.26 & 63.49 & 63.36 \\
 \hline
 32 & 37.03 & 36.36 & 36.08 & 37.03 & 62.54 & 63.51 & 63.49 & 63.55  \\
 \hline
 16 & 37.03 & 37.03 & 36.46 & 37.03 & 61.1 & 63.79 & 63.58 & 63.33  \\
 \hline
 \hline
 \multicolumn{5}{|c|}{Winogrande (FP32 Accuracy = 58.01\%)} & \multicolumn{4}{|c|}{Piqa (FP32 Accuracy = 74.21\%)} \\ 
 \hline
 \hline
 64 & 58.17 & 57.22 & 57.85 & 58.33 & 73.01 & 73.07 & 73.07 & 72.80 \\
 \hline
 32 & 59.12 & 58.09 & 57.85 & 58.41 & 73.01 & 73.94 & 72.74 & 73.18  \\
 \hline
 16 & 57.93 & 58.88 & 57.93 & 58.56 & 73.94 & 72.80 & 73.01 & 73.94  \\
 \hline
\end{tabular}
\caption{\label{tab:mmlu_abalation} Accuracy on LM evaluation harness tasks on GPT3-1.3B model.}
\end{table}

\begin{table} \centering
\begin{tabular}{|c||c|c|c|c||c|c|c|c|} 
\hline
 $L_b \rightarrow$& \multicolumn{4}{c||}{8} & \multicolumn{4}{c||}{8}\\
 \hline
 \backslashbox{$L_A$\kern-1em}{\kern-1em$N_c$} & 2 & 4 & 8 & 16 & 2 & 4 & 8 & 16  \\
 %$N_c \rightarrow$ & 2 & 4 & 8 & 16 & 2 & 4 & 2 \\
 \hline
 \hline
 \multicolumn{5}{|c|}{Race (FP32 Accuracy = 41.34\%)} & \multicolumn{4}{|c|}{Boolq (FP32 Accuracy = 68.32\%)} \\ 
 \hline
 \hline
 64 & 40.48 & 40.10 & 39.43 & 39.90 & 69.20 & 68.41 & 69.45 & 68.56 \\
 \hline
 32 & 39.52 & 39.52 & 40.77 & 39.62 & 68.32 & 67.43 & 68.17 & 69.30  \\
 \hline
 16 & 39.81 & 39.71 & 39.90 & 40.38 & 68.10 & 66.33 & 69.51 & 69.42  \\
 \hline
 \hline
 \multicolumn{5}{|c|}{Winogrande (FP32 Accuracy = 67.88\%)} & \multicolumn{4}{|c|}{Piqa (FP32 Accuracy = 78.78\%)} \\ 
 \hline
 \hline
 64 & 66.85 & 66.61 & 67.72 & 67.88 & 77.31 & 77.42 & 77.75 & 77.64 \\
 \hline
 32 & 67.25 & 67.72 & 67.72 & 67.00 & 77.31 & 77.04 & 77.80 & 77.37  \\
 \hline
 16 & 68.11 & 68.90 & 67.88 & 67.48 & 77.37 & 78.13 & 78.13 & 77.69  \\
 \hline
\end{tabular}
\caption{\label{tab:mmlu_abalation} Accuracy on LM evaluation harness tasks on GPT3-8B model.}
\end{table}

\begin{table} \centering
\begin{tabular}{|c||c|c|c|c||c|c|c|c|} 
\hline
 $L_b \rightarrow$& \multicolumn{4}{c||}{8} & \multicolumn{4}{c||}{8}\\
 \hline
 \backslashbox{$L_A$\kern-1em}{\kern-1em$N_c$} & 2 & 4 & 8 & 16 & 2 & 4 & 8 & 16  \\
 %$N_c \rightarrow$ & 2 & 4 & 8 & 16 & 2 & 4 & 2 \\
 \hline
 \hline
 \multicolumn{5}{|c|}{Race (FP32 Accuracy = 40.67\%)} & \multicolumn{4}{|c|}{Boolq (FP32 Accuracy = 76.54\%)} \\ 
 \hline
 \hline
 64 & 40.48 & 40.10 & 39.43 & 39.90 & 75.41 & 75.11 & 77.09 & 75.66 \\
 \hline
 32 & 39.52 & 39.52 & 40.77 & 39.62 & 76.02 & 76.02 & 75.96 & 75.35  \\
 \hline
 16 & 39.81 & 39.71 & 39.90 & 40.38 & 75.05 & 73.82 & 75.72 & 76.09  \\
 \hline
 \hline
 \multicolumn{5}{|c|}{Winogrande (FP32 Accuracy = 70.64\%)} & \multicolumn{4}{|c|}{Piqa (FP32 Accuracy = 79.16\%)} \\ 
 \hline
 \hline
 64 & 69.14 & 70.17 & 70.17 & 70.56 & 78.24 & 79.00 & 78.62 & 78.73 \\
 \hline
 32 & 70.96 & 69.69 & 71.27 & 69.30 & 78.56 & 79.49 & 79.16 & 78.89  \\
 \hline
 16 & 71.03 & 69.53 & 69.69 & 70.40 & 78.13 & 79.16 & 79.00 & 79.00  \\
 \hline
\end{tabular}
\caption{\label{tab:mmlu_abalation} Accuracy on LM evaluation harness tasks on GPT3-22B model.}
\end{table}

\begin{table} \centering
\begin{tabular}{|c||c|c|c|c||c|c|c|c|} 
\hline
 $L_b \rightarrow$& \multicolumn{4}{c||}{8} & \multicolumn{4}{c||}{8}\\
 \hline
 \backslashbox{$L_A$\kern-1em}{\kern-1em$N_c$} & 2 & 4 & 8 & 16 & 2 & 4 & 8 & 16  \\
 %$N_c \rightarrow$ & 2 & 4 & 8 & 16 & 2 & 4 & 2 \\
 \hline
 \hline
 \multicolumn{5}{|c|}{Race (FP32 Accuracy = 44.4\%)} & \multicolumn{4}{|c|}{Boolq (FP32 Accuracy = 79.29\%)} \\ 
 \hline
 \hline
 64 & 42.49 & 42.51 & 42.58 & 43.45 & 77.58 & 77.37 & 77.43 & 78.1 \\
 \hline
 32 & 43.35 & 42.49 & 43.64 & 43.73 & 77.86 & 75.32 & 77.28 & 77.86  \\
 \hline
 16 & 44.21 & 44.21 & 43.64 & 42.97 & 78.65 & 77 & 76.94 & 77.98  \\
 \hline
 \hline
 \multicolumn{5}{|c|}{Winogrande (FP32 Accuracy = 69.38\%)} & \multicolumn{4}{|c|}{Piqa (FP32 Accuracy = 78.07\%)} \\ 
 \hline
 \hline
 64 & 68.9 & 68.43 & 69.77 & 68.19 & 77.09 & 76.82 & 77.09 & 77.86 \\
 \hline
 32 & 69.38 & 68.51 & 68.82 & 68.90 & 78.07 & 76.71 & 78.07 & 77.86  \\
 \hline
 16 & 69.53 & 67.09 & 69.38 & 68.90 & 77.37 & 77.8 & 77.91 & 77.69  \\
 \hline
\end{tabular}
\caption{\label{tab:mmlu_abalation} Accuracy on LM evaluation harness tasks on Llama2-7B model.}
\end{table}

\begin{table} \centering
\begin{tabular}{|c||c|c|c|c||c|c|c|c|} 
\hline
 $L_b \rightarrow$& \multicolumn{4}{c||}{8} & \multicolumn{4}{c||}{8}\\
 \hline
 \backslashbox{$L_A$\kern-1em}{\kern-1em$N_c$} & 2 & 4 & 8 & 16 & 2 & 4 & 8 & 16  \\
 %$N_c \rightarrow$ & 2 & 4 & 8 & 16 & 2 & 4 & 2 \\
 \hline
 \hline
 \multicolumn{5}{|c|}{Race (FP32 Accuracy = 48.8\%)} & \multicolumn{4}{|c|}{Boolq (FP32 Accuracy = 85.23\%)} \\ 
 \hline
 \hline
 64 & 49.00 & 49.00 & 49.28 & 48.71 & 82.82 & 84.28 & 84.03 & 84.25 \\
 \hline
 32 & 49.57 & 48.52 & 48.33 & 49.28 & 83.85 & 84.46 & 84.31 & 84.93  \\
 \hline
 16 & 49.85 & 49.09 & 49.28 & 48.99 & 85.11 & 84.46 & 84.61 & 83.94  \\
 \hline
 \hline
 \multicolumn{5}{|c|}{Winogrande (FP32 Accuracy = 79.95\%)} & \multicolumn{4}{|c|}{Piqa (FP32 Accuracy = 81.56\%)} \\ 
 \hline
 \hline
 64 & 78.77 & 78.45 & 78.37 & 79.16 & 81.45 & 80.69 & 81.45 & 81.5 \\
 \hline
 32 & 78.45 & 79.01 & 78.69 & 80.66 & 81.56 & 80.58 & 81.18 & 81.34  \\
 \hline
 16 & 79.95 & 79.56 & 79.79 & 79.72 & 81.28 & 81.66 & 81.28 & 80.96  \\
 \hline
\end{tabular}
\caption{\label{tab:mmlu_abalation} Accuracy on LM evaluation harness tasks on Llama2-70B model.}
\end{table}

%\section{MSE Studies}
%\textcolor{red}{TODO}


\subsection{Number Formats and Quantization Method}
\label{subsec:numFormats_quantMethod}
\subsubsection{Integer Format}
An $n$-bit signed integer (INT) is typically represented with a 2s-complement format \citep{yao2022zeroquant,xiao2023smoothquant,dai2021vsq}, where the most significant bit denotes the sign.

\subsubsection{Floating Point Format}
An $n$-bit signed floating point (FP) number $x$ comprises of a 1-bit sign ($x_{\mathrm{sign}}$), $B_m$-bit mantissa ($x_{\mathrm{mant}}$) and $B_e$-bit exponent ($x_{\mathrm{exp}}$) such that $B_m+B_e=n-1$. The associated constant exponent bias ($E_{\mathrm{bias}}$) is computed as $(2^{{B_e}-1}-1)$. We denote this format as $E_{B_e}M_{B_m}$.  

\subsubsection{Quantization Scheme}
\label{subsec:quant_method}
A quantization scheme dictates how a given unquantized tensor is converted to its quantized representation. We consider FP formats for the purpose of illustration. Given an unquantized tensor $\bm{X}$ and an FP format $E_{B_e}M_{B_m}$, we first, we compute the quantization scale factor $s_X$ that maps the maximum absolute value of $\bm{X}$ to the maximum quantization level of the $E_{B_e}M_{B_m}$ format as follows:
\begin{align}
\label{eq:sf}
    s_X = \frac{\mathrm{max}(|\bm{X}|)}{\mathrm{max}(E_{B_e}M_{B_m})}
\end{align}
In the above equation, $|\cdot|$ denotes the absolute value function.

Next, we scale $\bm{X}$ by $s_X$ and quantize it to $\hat{\bm{X}}$ by rounding it to the nearest quantization level of $E_{B_e}M_{B_m}$ as:

\begin{align}
\label{eq:tensor_quant}
    \hat{\bm{X}} = \text{round-to-nearest}\left(\frac{\bm{X}}{s_X}, E_{B_e}M_{B_m}\right)
\end{align}

We perform dynamic max-scaled quantization \citep{wu2020integer}, where the scale factor $s$ for activations is dynamically computed during runtime.

\subsection{Vector Scaled Quantization}
\begin{wrapfigure}{r}{0.35\linewidth}
  \centering
  \includegraphics[width=\linewidth]{sections/figures/vsquant.jpg}
  \caption{\small Vectorwise decomposition for per-vector scaled quantization (VSQ \citep{dai2021vsq}).}
  \label{fig:vsquant}
\end{wrapfigure}
During VSQ \citep{dai2021vsq}, the operand tensors are decomposed into 1D vectors in a hardware friendly manner as shown in Figure \ref{fig:vsquant}. Since the decomposed tensors are used as operands in matrix multiplications during inference, it is beneficial to perform this decomposition along the reduction dimension of the multiplication. The vectorwise quantization is performed similar to tensorwise quantization described in Equations \ref{eq:sf} and \ref{eq:tensor_quant}, where a scale factor $s_v$ is required for each vector $\bm{v}$ that maps the maximum absolute value of that vector to the maximum quantization level. While smaller vector lengths can lead to larger accuracy gains, the associated memory and computational overheads due to the per-vector scale factors increases. To alleviate these overheads, VSQ \citep{dai2021vsq} proposed a second level quantization of the per-vector scale factors to unsigned integers, while MX \citep{rouhani2023shared} quantizes them to integer powers of 2 (denoted as $2^{INT}$).

\subsubsection{MX Format}
The MX format proposed in \citep{rouhani2023microscaling} introduces the concept of sub-block shifting. For every two scalar elements of $b$-bits each, there is a shared exponent bit. The value of this exponent bit is determined through an empirical analysis that targets minimizing quantization MSE. We note that the FP format $E_{1}M_{b}$ is strictly better than MX from an accuracy perspective since it allocates a dedicated exponent bit to each scalar as opposed to sharing it across two scalars. Therefore, we conservatively bound the accuracy of a $b+2$-bit signed MX format with that of a $E_{1}M_{b}$ format in our comparisons. For instance, we use E1M2 format as a proxy for MX4.

\begin{figure}
    \centering
    \includegraphics[width=1\linewidth]{sections//figures/BlockFormats.pdf}
    \caption{\small Comparing LO-BCQ to MX format.}
    \label{fig:block_formats}
\end{figure}

Figure \ref{fig:block_formats} compares our $4$-bit LO-BCQ block format to MX \citep{rouhani2023microscaling}. As shown, both LO-BCQ and MX decompose a given operand tensor into block arrays and each block array into blocks. Similar to MX, we find that per-block quantization ($L_b < L_A$) leads to better accuracy due to increased flexibility. While MX achieves this through per-block $1$-bit micro-scales, we associate a dedicated codebook to each block through a per-block codebook selector. Further, MX quantizes the per-block array scale-factor to E8M0 format without per-tensor scaling. In contrast during LO-BCQ, we find that per-tensor scaling combined with quantization of per-block array scale-factor to E4M3 format results in superior inference accuracy across models. 

%%%%%%%%%%%%%%%%%%%%%%%%%%%%%%%%%%%%%%%%%%%%%%%%%%%%%%%%%%%%%%%%%%%%%%%%%%%%%%%
%%%%%%%%%%%%%%%%%%%%%%%%%%%%%%%%%%%%%%%%%%%%%%%%%%%%%%%%%%%%%%%%%%%%%%%%%%%%%%%


\end{document}


% This document was modified from the file originally made available by
% Pat Langley and Andrea Danyluk for ICML-2K. This version was created
% by Iain Murray in 2018, and modified by Alexandre Bouchard in
% 2019 and 2021 and by Csaba Szepesvari, Gang Niu and Sivan Sabato in 2022.
% Modified again in 2023 and 2024 by Sivan Sabato and Jonathan Scarlett.
% Previous contributors include Dan Roy, Lise Getoor and Tobias
% Scheffer, which was slightly modified from the 2010 version by
% Thorsten Joachims & Johannes Fuernkranz, slightly modified from the
% 2009 version by Kiri Wagstaff and Sam Roweis's 2008 version, which is
% slightly modified from Prasad Tadepalli's 2007 version which is a
% lightly changed version of the previous year's version by Andrew
% Moore, which was in turn edited from those of Kristian Kersting and
% Codrina Lauth. Alex Smola contributed to the algorithmic style files.
