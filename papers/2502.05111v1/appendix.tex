% \section{Supplementary Material}
\section{Algorithms} \label{app:cd}
In this section we formalize several algorithms presented in the paper. Alg. \ref{alg:gcd} describes the abstract algorithm for general constrained decoding. \ref{alg:fst-construction} describes how to build the lexing transducer from a lexing automaton given as a FSA. Alg. \ref{alg:detokenize} describes the construction of the detokenizing transducer, which converts sequences of tokens to sequences of the characters they contain. Alg. \ref{alg:preprocess-parser} describes the parser preprocessing and how the always-accepted token table and context-dependent sequence table are built.
\setcounter{algorithm}{0}
\begin{algorithm}
\caption{\tname{ConstrainedDecoding}}
\label{alg:gcd}
\begin{algorithmic}
    \STATE {\bfseries Input:} Model $M$, Checker $C$, Tokenized prompt $x$
    \STATE $\vocab := M.\texttt{vocabulary}$
    \REPEAT
    \STATE $m := C(x; \vocab)$
    \STATE $\mathit{logits} := M(x)$   
    \STATE $\token_{\mathit{next}} := \texttt{sample}(\texttt{apply\_mask}(m, \mathit{logits}))$
    \STATE $x := x.\texttt{append}(\token_{\mathit{next}})$
    \UNTIL{$\token_{\mathit{next}} \neq \texttt{EOS}$}
    \RETURN $x$
\end{algorithmic}
\end{algorithm}

\begin{algorithm}
\caption{\tname{BuildLexingFST}}
\label{alg:fst-construction}
\begin{algorithmic}
    \STATE {\bfseries Input:} FSA $\automaton = (\Sigma, Q, q_0, \delta, F)$, Output alphabet $\Gamma$ % , Terminal Map $T: F \rightarrow \Gamma$
    \STATE {\bfseries Output:} FST $\transducer_\automaton = (\Sigma, \terms, Q, q_0, \delta_{\textrm{FST}}, F_{\textrm{FST}})$
    \STATE $\delta_{\textrm{FST}} := \{q \xrightarrow{c:\epsilon}q' \mid q \xrightarrow{c} q' \in \delta\}$, $F_{\textrm{FST}} := \{q_0\}$
    \FOR{state $q$ that recognizes language token $\term \in \terms$ }
        \FOR{$(c, q')$ s.t. $\exists q''.\, q \xrightarrow{c} q'' \notin \delta$ and $q_0 \xrightarrow{c} q' \in \delta $}
            \STATE $\delta_{\textrm{FST}} := \delta_{\textrm{FST}} \,\cup\,  \{q \xrightarrow{c:\term} q'\}$
        \ENDFOR
        \STATE $\delta_{\textrm{FST}} := \delta_{\textrm{FST}} \,\cup\, \{ q \xrightarrow{\texttt{EOS}: \term\$} q_0 \}$
    \ENDFOR
    \RETURN $\transducer_\automaton = (\Sigma, \Gamma, Q, q_0, \delta_{\textrm{FST}}, F_{\textrm{FST}})$
\end{algorithmic}
\end{algorithm}

\begin{algorithm}
\caption{\tname{BuildDetokenizingFST}}
\label{alg:detokenize}
\begin{algorithmic}
    \STATE {\bfseries Input:} Vocabulary $\vocab \subseteq \alphabet^+$
    \STATE {\bfseries Output:} FST $\transducer_\vocab = (\vocab, \alphabet, Q, q_0, \delta, F)$
    \STATE $Q := \{q_\epsilon\}$, $F := \{q_\epsilon\}$, $q_0 := q_\epsilon$, $\delta := \emptyset$
    \FOR{$c_1 \ldots c_k \in \vocab$}
        \STATE $q_{prev} := q_\epsilon$
        \FOR{$i=1$ to $k-1$}
            \STATE $Q := Q \,\cup\, \{ q_{c_1 \ldots c_i} \}$
            \STATE $\delta := \delta \,\cup\, \{q_{prev} \xrightarrow{\epsilon:c_i} q_{c_1 \ldots c_i}\} $
            \STATE $q_{prev} := q_{c_1 \ldots c_i}$
        \ENDFOR
        \STATE $\delta := \delta \,\cup\, \{q_{prev} \xrightarrow{c_1 \ldots c_k:c_k} q_{\epsilon}\}$
    \ENDFOR
    \RETURN $\transducer_\vocab = (\vocab, \alphabet, Q, q_0, \delta, F)$
\end{algorithmic}
\end{algorithm}
\setcounter{algorithm}{4}

\begin{algorithm}
\caption{\tname{PreprocessParser}}
\label{alg:preprocess-parser}
\begin{algorithmic}
    \STATE {\bfseries Input:} PDA $\pushdown$, Realizable sequences $\realizable{\vocab}{\automaton}$, \\
    FSA $\automaton$, Inverse token spannar table $\inversetable$
    \STATE {\bfseries Output:} Always-accepted token table $A$
    \\ Context-dependent sequence table $D$
   
    \STATE $\automaton_{\pushdown} := \textsc{RemoveStackOperations}(\pushdown)$
    \FOR{$q_\pushdown \in Q_\pushdown$}
        \STATE $\bar{A}(q_\pushdown) := \{\alpha \in \realizable{\vocab}{\automaton} \mid q \textrm{ with stack } \epsilon \textrm{ accepts } \alpha \}$
        \STATE $\bar{R}(q_\pushdown) := \{\alpha \in \realizable{\vocab}{\automaton} \setminus A(q) \mid \automaton_{\pushdown} \textrm{ rejects } \alpha \textrm{ in } q  \}$
        \STATE $\bar{D}(q_\pushdown) := \realizable{\vocab}{\automaton} \setminus A(q) \setminus R(q)$
        \FOR{$q_\automaton \in Q_\automaton$}
            \STATE $A(q_\automaton, q_\pushdown) := \bigcup_{\alpha \in \bar{A}(q)} \inversetable(q_\automaton, \alpha) $
            \STATE $D(q_\automaton, q_\pushdown) := \{\alpha \in \bar{D}(q) \mid \inversetable(q_\automaton, \alpha) \neq \emptyset \} $
        \ENDFOR
    \ENDFOR
    \RETURN $A$, $D$
\end{algorithmic}
\end{algorithm}

\section{Formal Definitions}

\subsection{Finite-State Automata}
\label{app:fsa-definition}

A \emph{finite-state automaton} (FSA) is defined as a tuple $\automaton = (\alphabet, Q, q_0, \delta, F)$ where $\Sigma$ is the input alphabet, $Q$ is the set of states, $q_0 \in Q$ is the initial state, $\delta \subseteq Q \times (\alphabet \,\cup\, \{\epsilon\}) \times Q$ is the set of transitions, and $F \subseteq Q$ is the set of accepting states.
Each transition $(q, c, q')$, also denoted by $q \xrightarrow{c} q'$, indicates that, from state $q$, upon reading the input symbol $c$, the automaton transitions to state $q'$.

Given a transition relation $\delta$ of automaton $\automaton$, the extended transition $\delta^\ast \subseteq Q \times \Sigma \times Q$ is the smallest relation defined by \rone $(q, \epsilon, q) \in \delta^\ast$ and \rtwo $(q, wc, q') \in \delta^\ast$ whenever $(q, w, q'') \in \delta^\ast$ and $(q'', c, q) \in \delta$ for some $q'' \in Q$. We also denote $(q, w, q') \in \delta^\ast$ by $q \xrightarrow{w} q'$.
A string $w \in \alphabet^\ast$ is accepted by automaton $\automaton$ when there exists $q' \in F$ such that $q \xrightarrow{w}^\ast q' \in \delta^\ast$.

However, in this paper, we use the term \emph{accepted} in a broader sense than its standard definition. Specifically, we say that a string $w$ is accepted in state $q$ if there exists $q'$ such that $q \xrightarrow{w} q' \in \delta^\ast$, meaningthat a valid transition for $w$ exists from $q$. 

% \subsection{Transducer Composition}
% \label{app:transducer-composition}

\subsection{Context-Free Grammar}
\label{app:cfg-definition}

A \emph{context-free grammar} (CFG) $\grammar$ is a tuple $(\terms, \nonterms, \start, \ruleset)$ where
$\terms$ is a finite set of terminal symbols (e.g. constants, variable names, and keywords), $\nonterms$ is a finite set of non-terminal symbols, $\start \in \nonterms$ is a start nonterminal, $\ruleset$ is a set of production rules $A \to \alpha_1, \dots, \alpha_n$ where $A \in \nonterms$ and $\alpha_i \in \nonterms \cup \terms$.

% A string $\sent$ is in $L(\grammar)$ if one can derive $\sent$ from the start symbol by repeatedly applying the production rules.

%
Formally, a grammar $\grammar$ defines a \emph{single-step derivation} relation on sequences of symbols $\alpha, \beta, \gamma \in (\nonterms \cup \terms)^*$:
$\alpha \nterm \gamma \step \alpha \beta \gamma$ if $\nterm \to \beta \in \ruleset$.
%
The reflexive transitive closure of this relation is called \emph{derivation} and written $\manystep$.
%
A sequence of tokens $\sent$ is a \emph{sentence} if it is derivable from $\start$;
the set of all sentences is called the \emph{language} of the grammar $\grammar$, that is,
$\lang(\grammar) = \{\sent \in \terms^* \mid \start \manystep \sent\}$.

\subsection{Finite State Transducer}
An FST is a tuple $\transducer=(\Sigma, \terms, Q, q_0, \delta, F)$ where all components are analogous to those of a FSA but each FST transition $q \xrightarrow{c:T_1 \ldots T_n} q'$ in $\delta$ denotes that when reading a character $c$ from state $q$, the FST moves to a new state $q'$ and also outputs the sequence $\term_1 \ldots \term_k$ of elements over the output alphabet $\terms$.

\subsection{Pushdown Automata}
\label{app:pda-definition}

A \emph{pushdown automaton} is a tuple $\pushdown = (\alphabet, \Pi, Q, q_0, Z_0, \delta, F)$ where $\alphabet$, $Q$, $q_0$ and $F$ are as in their FSA definitions, $\Pi$ is the stack alphabet, $Z_0 \in \Pi$ is the initial stack symbol, and $\delta \subseteq Q \times (\Sigma \cup \{\epsilon\}) \times \Pi^\ast \times Q \times \Pi^\ast$ is the set of transitions. 
%
Each transition $(q, c, \alpha, q', \beta)$ specifies that, in state $q$, upon reading the input symbol $c$ and matching the top stack symbols to $\alpha$, the PDA transitions to state $q'$ and replaces $\alpha$ with the sequence $\beta$ on the stack.

% \section{Grammars}

% \subsection{Go}

% \subsection{Java}

% \subsection{Python}