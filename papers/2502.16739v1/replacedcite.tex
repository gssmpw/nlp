\section{Related work}
Technologies now manage nearly every aspect of academic activities____, with tech ecosystems constantly evolving with the availability of numerous mobile apps and extensions for other platforms (such as through marketplaces for Zoom ____ and Canvas ____). %without any transparency about or oversight on dataflows____. 
This continuous digitization has led to security and privacy vulnerabilities: in 2023, data breaches at HEIs have cost an average of $3M$ USD____. 
%Simultaneously, reports on the abuse of collected data by service providers and their affiliates for profiling, tracking, advertising, as well as direct selling of the data to data brokers, are on the rise____. 
% Complexity brings vulnerabilities, as was seen when MOVEit, a popular file transfer tool, was compromised, causing data breaches in almost 900 schools. It exposed students' data including name, IDs, date of birth, contact information, and Social Security number____. 

%We lack an understanding of how HEIs conduct security assessments during procurement, how responsibilities for ensuring secure operations are distributed and maintained throughout the life cycle of EdTech, and how liability is negotiated in case of a privacy or security breach.

%Unfortunately, the US does not have any comprehensive data privacy law; the FERPA is the main federal statute that governs the privacy of student data____. It requires HEIs to obtain consent before sharing ``educational records'' unless there is a ``legitimate educational interest.'' However, legal scholar Elana Zeide noted that institutions have almost complete authority to define what is ``legitimate educational interest'' and decide what data are protected by FERPA and what are not____. Russel~\textit{et al.} noted that the revision to FERPA in 2008 gives more leverage to private companies by including them as ``school officials'' and allows disclosure of data to parties such as contractors, consultants, and volunteers; and once data enters the marketplace, it can be freely exchanged since FERPA does not apply to data brokers____. Such loopholes allow vendors to use vague statements about what data can be collected, used, and for how long, as Paris \textit{et al.} reported after reviewing publicly available documents at Rutgers University____. For example, the contract with Canvas\footnote{\url{https://www.structure.com/canvas}} says that upon contract termination, the university will lose access to the data collected by Canvas, without mentioning if the data will be deleted or the vendor may continue to store and use it____. \diff{This situation also poses legal challenges for HEIs who enroll students from other countries with a stricter privacy policy. As reported in the Findings section, many HEIs we interviewed serve students from the European Union (EU) through online degree programs. Data about these students must be handled following the General Data Protection Regulation (GDPR\footnote{https://gdpr-info.eu/}), which requires a lawful basis and stronger consenting requirements before sharing data with third parties and imposes restrictions on secondary data usage.} Thus, there is a critical need for in-depth scrutiny of HEIs' current EdTech acquisition practices---how they leverage federal and state legislature to guide vendor contracts and negotiations to protect data from breaches and misuse, and what challenges they face. %This paper addresses this gap by conducting a semi-structured interview of the policymakers and EdTech experts with 3 to 23 years of experience in EdTech exploration, acquisition, maintenance, and policy governance. \sazz{Needs a better transition to the next section. Maybe we can bring the overview here: ``Research Questions and Study overview''?}

Past research has studied the institutional use of technologies. Radway~\etal investigated if and how universities conform to the Family Education Rights and Privacy Act (FERPA) while sharing directory information____. Chanenson~\etal interviewed K-12 school officials and IT personnel to understand districts' use of technologies and how they manage student privacy and security____. Balash~\etal surveyed university instructors to understand the prevalence of using online exam proctoring apps and why they are (not) adopted____, while Shioji~\etal investigated the same with a target audience of senior administrators____. Kelso~\etal investigated how universities procure technology, their auditing process, and how they maintain institutional security posture. Paris~\etal reported how loopholes in regulations and institutional contracts can be exploited to invade privacy. However, the literature lacks studies on unsanctioned technology use and its impact on \SP, which is crucial for a more comprehensive understanding of educational institutes' \SP posture.