\section{Discussion}
This study sheds light on the nuanced challenges associated with educators' use of unsanctioned applications in educational settings. 
From \sp perceptions to institutional policy awareness and administrative roles, our findings highlight a complex interplay of factors that shape app usage behavior. 
Below, we discuss key themes and their implications.

\subsection{Educators' Security and Privacy Awareness vs. Action}
The findings reveal a striking paradox: while many educators acknowledge the importance of \sp (\S~\ref{sp_perceptions}), these concerns rarely influence app selection or usage behavior. 
For example, 218 participants considered \sp for at least one app, yet more than a third ($n=176$) did not factor these considerations into their decisions at all. Furthermore, even after experiencing breaches ($n=23$), suspecting apps of poor data collection practices ($n=38$), or believing apps sold user data ($n=43$), many educators continued to use these apps. 
This behavior underscores a tendency to prioritize ease of use, engagement, and functionality (\S~\ref{reasons_unsanctioned_app_use}) over compliance and safety, highlighting a gap between perceived risks and actionable responses.  

Compounding this issue is the widespread uncertainty surrounding regulatory compliance. Despite most participants assuming FERPA and HIPAA applied to the apps they used, many were unsure of their compliance status ($n=71$ for FERPA, $n=51$ for HIPAA). 
This lack of clarity not only compromises institutional data security but also demonstrates the inadequacy of current measures to educate and empower educators in \emph{evaluating app security}. 
Institutions must focus on bridging this gap by offering secure alternatives that align with educators' needs while providing clear guidance and training on security and compliance.

\subsection{Limited Effectiveness of Institutional Warnings}
Only $24.8\%$ of participants ($n=107$) were aware of institutional warnings about the risks of unsanctioned app use, and among those, only 33 were influenced to change their behavior (\S~\ref{inst_policy_on_unsanction_app_use}). 
These numbers suggest that current warning mechanisms fail to translate awareness into meaningful action. 
Most educators, even when aware of institutional guidelines, prioritize practicality and familiarity over compliance.  

This highlights the need for institutions to rethink their policy enforcement strategies. 
Instead of relying solely on warnings, institutions should adopt a proactive approach by educating educators on the risks associated with unsanctioned apps and offering clear, actionable alternatives. 
For example, providing training sessions on how to evaluate apps for compliance, encouraging the adoption of institution-approved tools, and implementing educator-friendly policies that balance security with flexibility could enhance adherence. Ensuring that warnings are both relatable and actionable will likely improve their effectiveness in fostering behavior change.

\subsection{The Illusion of Control}
While the ability to download unsanctioned apps may give educators a sense of flexibility and autonomy, it also creates an illusion of control, particularly regarding data security and deletion. 
Our study results highlight a significant gap in educators' understanding of app exit strategies (\S~\ref{discontinue_data_deletion}). 
Of the 85 participants who stopped using at least one app, only nine were aware of a data deletion option, and just eight requested data deletion. Meanwhile, 60 participants were uncertain about whether such options existed.  

This lack of awareness demonstrates that educators are not currently equipped to safeguard institutional data effectively, especially when using unsanctioned apps. 
Without clear exit pathways, sensitive data may remain exposed to risks long after app use ends. 
This emphasizes the critical need for institutions to implement robust policies and educational initiatives that address not only app usage but also secure disengagement practices. 
Training educators on topics such as data retention, deletion policies, and privacy risks can help mitigate these vulnerabilities while reinforcing the importance of compliance.

\subsection{Administration Challenges and Contradictions}
Administrators are often on the frontlines of managing the consequences of unsanctioned app use, including \sp incidents. 
As evidenced by participant feedback (\S~\ref{admin_sp_incidents}), some administrators have firsthand experience with data breaches, such as third-party integrations scraping user data (A4) or faculty auto-forwarding emails outside managed environments, thereby bypassing institutional phishing controls (A12). 
These incidents underscore the broader institutional vulnerabilities posed by unsanctioned apps, particularly regarding the leakage of personally identifiable information (PII) and violations of FERPA and other legal requirements.  

Interestingly, while many administrators expressed frustration with educators' use of unsanctioned apps, others acknowledged their own role in facilitating such practices. 
Some administrators even supported the integration of unsanctioned apps, validating educators' claims that they are occasionally encouraged to use these tools. 
This duality can create gray areas in policy enforcement and addressing this issue will require a collaborative approach between educators and administrators, while maintaining a common goal toward security.
By involving both educators and administrators in developing guidelines and reviewing tools, institutions can create realistic and effective policies that balance security requirements with the practical needs of teaching, ensuring compliance while supporting classroom needs.

\subsection{Conclusion}
Our findings emphasize the urgent need for institutions to rethink their approach to unsanctioned app use. 
Educators' preference for these tools, despite associated risks, reflects a demand for functionality and engagement that institutional offerings often fail to meet. 
At the same time, administrators' experiences with security incidents reveal critical gaps in policy implementation and enforcement. 
Educational institutions must invest in creating educator-friendly, while secure alternatives and provide clear pathways for vetting and approving education tools. 
Moreover, fostering a culture of shared responsibility---where both educators and administrators collaborate to balance innovation with compliance---will be essential for addressing these challenges effectively.