\section{Study 2: Admin survey}
To understand if and how the use of unsanctioned apps by educator leads to any \SP incidents, we conducted another online study targeting administrators, IT personnel, data governance bodies, and technology policymakers from US-based educational institutes.

\subsection{Methods}

\paragraph{Survey design.}
Survey questions were created based on results from the first study. We first asked participants if they recommended unsanctioned apps to instructors, or received requests from instructors to integrate such tools with other institutional tools. Next, we asked them to explain if there was any \SP incident at the institution due to instructors using unsanctioned apps (see \S~\ref{app:study2ques}) for the full questionnaire.

% We asked participants what types of personal tools they have recommended for faculty to use and reasons behind it to gain the administrative point of view. They were asked if they knew of any incidents that took place at their institution that was caused by the use of unsanctioned technology and what were the repercussions. Next they listed what their top concern was about the types of technologies that educators are bringing into the classroom. This allowed the authors to learn about the risks involved from the administrators perspective about the use of unsanctioned technologies. Lastly, they were asked multiple questions about what measures were in place at their institution to prevent leakage and the types of proactive measure in place.

\paragraph{Participant recruitment.} Large-scale recruitment of participants from the target population is infeasible~\cite{edtech-ccs2024} and there is no direct way to reach them through online survey platforms like Prolific. Thus, we attempted to recruit participants through direct emails and posting the study link to Educause platform \cite{Educause} and two Reddit 
\cite{Reddit} groups: \textit{k12sysadmin} and \textit{k12cybersecurity}. %( We also directly emailed 1035 educators who ranged from principals, IT personnel, finance officers, and other administrative roles ranging from higher education to K-12 using personal connections.

% \paragraph{Data analysis}
% \hlfixme{Add data analysis methods}

\subsection{Results}
\subsubsection{Participants}
We collected data from 29 participants, denonated with an 'A'. Among those, four were from K-12 institutions while the rest were from higher education institutions. Of those participants, 12 were actively in roles related to information security while the rest were in administrative roles at their institutions such as business manager, principal, or vice chancellor.

\subsubsection{Use and integration of unsanctioned apps}
Seven participants said that they had recommended unsanctioned apps to instructors, supporting results from Study 1. Their recommendations come from needing ``to meet niche demands that our institutional platforms cannot manage, or where instructors are looking for free alternatives for their use or their student's use.'' (A4). The applications administrators recommend are ``common cloud platforms'' and platforms that the wider population is using like `` Canva, Slack, and ChatGPT'' (A7).

On the other hand, all but three participants had received instructors' requests for unsanctioned tools to be integrated into institutionally licensed tools. %Five participants mentioned how educators want AI tools to be implemented. 
For example, A4 mentioned how they ``have instructors wishing to integrate many tools such as polling, scheduling, citation software for use in classes.''.  One administrator discussed how ``Almost at an individual level, everyone has their "solution" to teaching/learning "better" and brings it into the classroom with disregard.'' (A21) regarding how educators bring unsanctioned technology in classrooms. 

Regarding fulfilling the integration requests, two participants said such requests are never accepted, others' responses varied from `sometimes' to `most of the time'. Twelve participants said such integrations go through IT audit, others were unsure about this.

\subsubsection{Security and privacy incidents}
\label{admin_sp_incidents}
Nine participants indicated they had experienced \SP incidents due to unsanctioned app use. A4 stated that they ``have had third-party integrations scrape user data without being vetted and without contractual controls in place.'' Another participant (A12) stated challenges they faced in ensuring institutional security because ``a significant subset of faculty were in the habit of auto-forwarding internal university email to personal email accounts. This was a problem for many reasons, the biggest being none of our phishing controls or detection would detect anything after such emails had transited outside our managed environment, including phishing attack successes and account compromise [\dots]'' Without providing details, one participant (A19) mentioned an issue with \textit{otter.ai},\cite{OtterAI} that was integrated with \textit{Zoom} video conferencing tool.

Several participants expressed worries about the impact unsanctioned apps can have on institutional security posture and legal compliance. Overall their top concern for the use of those technologies stems from the possibility of leakage of PII, ``Personal information collected and stored incorrectly.'' (A10). One participant explained how using unsanctioned technology ``strips the controls [the institution] very intentionally design, build in and manage'' (A11).  A21 stated that ``In the case of a classroom, [disclosure of information] can commonly meet the criteria for a FERPA breach exposing the institution to serious liabilities''.
Another participant pointed out how ``Instructors [may] not understand that the institution has laws that it must follow to protect its data… And sometimes integrations impact a security plan/posture and we just can't use them'' (A16). Concerns were also growing regarding the use of AI tools like ChatGPT. A20 mentioned how they ``believe faculty are regularly using "free" AI tools with student data'' which could lead to student data being leaked and put into those AI systems.

% \rakib{discussion point: admins are aware of instructors' use of unsanctioned apps, and are the ones who first hand experienced resulting \SP incidents. while many expressed frustration with this situation, few also recommended such tools and helped with integrations, lending support to instructors claim that they are sometimes encouraged...}


% [add qualitative data about tools that were integrated.]

% [add qualitative answers about most severe security issues due to integration]

% 18 said they have proactive mechanisms to prevent security issues, [add examples of most frequently mentioned proactive measures.]


% \subsubsection{Institutional policy}
% Most institutions have policies regarding the use of unsanctioned apps, and these policies have been updated in the past 2 years with a notable range of policies not being updated from ``the last five years'' (P13) and ``over a decade ago'' (P21). %All participants mentioned that their institutions have these policies available online, with four participants stating that they are also sent over email for faculty to look over and understand. 
% Fifteen participants said their schools do not have a clear recommendation about unsanctioned app integration, they do it on a case-by-case basis. %Most participants (80\%) stated that these policies prohibit downloading institutional data on personal devices.  