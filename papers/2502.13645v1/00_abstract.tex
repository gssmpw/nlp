\begin{abstract}

With the increasing prevalence of recorded human speech, spoken language understanding (SLU) is essential for its efficient processing. In order to process the speech, it is commonly transcribed using automatic speech recognition technology. This speech-to-text transition introduces errors into the transcripts, which subsequently propagate to downstream NLP tasks, such as dialogue summarization. While it is known that transcript noise affects downstream tasks, a systematic approach to analyzing its effects across different noise severities and types has not been addressed. We propose a configurable framework for assessing task models in diverse noisy settings, and for examining the impact of transcript-cleaning techniques. The framework facilitates the investigation of task model behavior, which can in turn support the development of effective SLU solutions. We exemplify the utility of our framework on three SLU tasks and four task models, offering insights regarding the effect of transcript noise on tasks in general and models in particular. For instance, we find that task models can tolerate a certain level of noise, and are affected differently by the types of errors in the transcript.\footnote{Code: \url{https://github.com/OriShapira/ENDow}}

\end{abstract}