\begin{figure*}[ht]
    \centering
    \subfloat[Pairwise ranking, Mistral]{
        \includegraphics[width=0.32\linewidth, trim=8 8 7 21, clip]{figures/cleaning/qmsum_cleaning_Mistral7BInstruct_recursive_pairwise_ranking.pdf}
    }
    \subfloat[ROUGE-1, Mistral]{
        \includegraphics[width=0.32\linewidth, trim=8 8 7 21, clip]{figures/cleaning/qmsum_cleaning_Mistral7BInstruct_recursive_rouge1.pdf}
    }
    \subfloat[ROUGE-2, Mistral]{
        \includegraphics[width=0.32\linewidth, trim=8 8 7 21, clip]{figures/cleaning/qmsum_cleaning_Mistral7BInstruct_recursive_rouge2.pdf}
    }
    \hspace{0.2cm}
    \subfloat[Pairwise ranking, Llama3]{
        \includegraphics[width=0.32\linewidth, trim=8 8 7 21, clip]{figures/cleaning/qmsum_cleaning_Llama3Instruct_recursive_pairwise_ranking.pdf}
    }
    \subfloat[ROUGE-1, Llama3]{
        \includegraphics[width=0.32\linewidth, trim=8 8 7 21, clip]{figures/cleaning/qmsum_cleaning_Llama3Instruct_recursive_rouge1.pdf}
    }
    \subfloat[ROUGE-2, Llama3]{
        \includegraphics[width=0.32\linewidth, trim=8 8 7 21, clip]{figures/cleaning/qmsum_cleaning_Llama3Instruct_recursive_rouge2.pdf}
    }
    \hspace{0.2cm}
    \subfloat[Pairwise ranking, Llama3.1]{
        \includegraphics[width=0.32\linewidth, trim=8 8 7 21, clip]{figures/cleaning/qmsum_cleaning_Llama3_1Instruct_truncate_pairwise_ranking.pdf}
    }
    \subfloat[ROUGE-1, Llama3.1]{
        \includegraphics[width=0.32\linewidth, trim=8 8 7 21, clip]{figures/cleaning/qmsum_cleaning_Llama3_1Instruct_truncate_rouge1.pdf}
    }
    \subfloat[ROUGE-2, Llama3.1]{
        \includegraphics[width=0.32\linewidth, trim=8 8 7 21, clip]{figures/cleaning/qmsum_cleaning_Llama3_1Instruct_truncate_rouge2.pdf}
    }
    \hspace{0.2cm}
    \subfloat[Pairwise ranking, GPT]{
        \includegraphics[width=0.32\linewidth, trim=8 8 7 21, clip]{figures/cleaning/qmsum_cleaning_Gpt4oMini_truncate_pairwise_ranking.pdf}
    }
    \subfloat[ROUGE-1, GPT]{
        \includegraphics[width=0.32\linewidth, trim=8 8 7 21, clip]{figures/cleaning/qmsum_cleaning_Gpt4oMini_truncate_rouge1.pdf}
    }
    \subfloat[ROUGE-2, GPT]{
        \includegraphics[width=0.32\linewidth, trim=8 8 7 21, clip]{figures/cleaning/qmsum_cleaning_Gpt4oMini_truncate_rouge2.pdf}
    }
    \caption{The performance of models when applying various cleaning techniques, on the summarization dataset of \textbf{QMSum}. Each point on the ``no\_cleaning'' curve can be compared to the respective point on a cleaning technique's curve. A good cleaning technique should increase the task score (y value) as much as possible, with as little effort as possible (represented by decrease in WER, as the x value). Each cleaning technique is marked with its overall cleaning-effectiveness score which is computed as a function of the change in the task score and in the WER score. The CES scores can be seen also in \autoref{tab_scores_cleaning_all}.}
    \label{fig_cleaning_graphs_all_qmsum}
\end{figure*}