\begin{figure}[h!]
    \centering
    \subfloat[QMSum with pairwise ranking evaluation, for GPT]{
        \includegraphics[width=0.9\columnwidth, trim=8 8 14 21, clip]{figures/cleaning/partial/qmsum_cleaning_Gpt4oMini_truncate_pairwise_ranking.pdf}
    }
    \hspace{0.2cm}
    \subfloat[QAConv with fuzzy matching evaluation, for GPT]{
        \includegraphics[width=0.9\columnwidth, trim=8 8 14 21, clip]{figures/cleaning/partial/qaconv_cleaning_Gpt4oMini_fzr.pdf}
    }
    \hspace{0.2cm}
    \subfloat[MRDA with macro-$F_1$ evaluation, for GPT]{
        \includegraphics[width=0.9\columnwidth, trim=8 8 14 21, clip]{figures/cleaning/partial/mrda_cleaning_Gpt4oMini_f1_macro.pdf}
    }
    \caption{The performance of GPT-4o-mini when applying various cleaning techniques. Compare a point on the ``no\_cleaning'' curve to the respective point on a cleaning technique's curve. Effective cleaning means maximizing gain in task score (y-axis) with minimum effort (x-axis), measured using the cleaning-effectiveness score (CES).
    %which is computed as a function of the change in the task score and in the WER score.
    Additional CES scores are in \autoref{tab_scores_cleaning}, and more graphs are in Figures \ref{fig_cleaning_graphs_all_qmsum}, \ref{fig_cleaning_graphs_all_qaconv} and \ref{fig_cleaning_graphs_all_mrda} in the Appendix.}
    \label{fig_cleaning_graphs}
\end{figure}