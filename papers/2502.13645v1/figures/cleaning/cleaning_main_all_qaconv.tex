\begin{figure*}[ht]
    \centering
    \subfloat[Fuzzy match, Mistral]{
        \includegraphics[width=0.32\linewidth, trim=8 8 7 21, clip]{figures/cleaning/qaconv_cleaning_Mistral7BInstruct_fzr.pdf}
    }
    \subfloat[Token $F_1$, Mistral]{
        \includegraphics[width=0.32\linewidth, trim=8 8 7 21, clip]{figures/cleaning/qaconv_cleaning_Mistral7BInstruct_f1.pdf}
    }
    \subfloat[Exact match, Mistral]{
        \includegraphics[width=0.32\linewidth, trim=8 8 7 21, clip]{figures/cleaning/qaconv_cleaning_Mistral7BInstruct_exact.pdf}
    }
    \hspace{0.2cm}
    \subfloat[Fuzzy match, Llama3]{
        \includegraphics[width=0.32\linewidth, trim=8 8 7 21, clip]{figures/cleaning/qaconv_cleaning_Llama3Instruct_fzr.pdf}
    }
    \subfloat[Token $F_1$, Llama3]{
        \includegraphics[width=0.32\linewidth, trim=8 8 7 21, clip]{figures/cleaning/qaconv_cleaning_Llama3Instruct_f1.pdf}
    }
    \subfloat[Exact match, Llama3]{
        \includegraphics[width=0.32\linewidth, trim=8 8 7 21, clip]{figures/cleaning/qaconv_cleaning_Llama3Instruct_exact.pdf}
    }
    \hspace{0.2cm}
    \subfloat[Fuzzy match, Llama3.1]{
        \includegraphics[width=0.32\linewidth, trim=8 8 7 21, clip]{figures/cleaning/qaconv_cleaning_Llama3_1Instruct_fzr.pdf}
    }
    \subfloat[Token $F_1$,  Llama3.1]{
        \includegraphics[width=0.32\linewidth, trim=8 8 7 21, clip]{figures/cleaning/qaconv_cleaning_Llama3_1Instruct_f1.pdf}
    }
    \subfloat[Exact match,  Llama3.1]{
        \includegraphics[width=0.32\linewidth, trim=8 8 7 21, clip]{figures/cleaning/qaconv_cleaning_Llama3_1Instruct_exact.pdf}
    }
    \hspace{0.2cm}
    \subfloat[Fuzzy match, GPT]{
        \includegraphics[width=0.32\linewidth, trim=8 8 7 21, clip]{figures/cleaning/qaconv_cleaning_Gpt4oMini_fzr.pdf}
    }
    \subfloat[Token $F_1$, GPT]{
        \includegraphics[width=0.32\linewidth, trim=8 8 7 21, clip]{figures/cleaning/qaconv_cleaning_Gpt4oMini_f1.pdf}
    }
    \subfloat[Exact match, GPT]{
        \includegraphics[width=0.32\linewidth, trim=8 8 7 21, clip]{figures/cleaning/qaconv_cleaning_Gpt4oMini_exact.pdf}
    }
    \caption{The performance of models when applying various cleaning techniques, on the question-answering dataset of \textbf{QAConv}. Each point on the ``no\_cleaning'' curve can be compared to the respective point on a cleaning technique's curve. A good cleaning technique should increase the task score (y value) as much as possible, with as little effort as possible (represented by decrease in WER, as the x value). Each cleaning technique is marked with its overall cleaning-effectiveness score which is computed as a function of the change in the task score and in the WER score. The CES scores can be seen also in \autoref{tab_scores_cleaning_all}.}
    \label{fig_cleaning_graphs_all_qaconv}
\end{figure*}