\section{Related work}
Our work builds directly on the concept of exclusion zones introduced by \citet{tomlinson2024moderating}, where we studied them for IRV and plurality with one-dimensional Euclidean preferences. The concepts of Condorcet winning sets and $\theta$-winning sets~\cite{elkind2011choosing,elkind2015condorcet,bloks2018condorcet,charikar2024six} are related in spirit to exclusion zones, in that they describe sets of candidates preferred by voters to candidates outside the set. However, there are a few important distinctions. First, exclusion zones are sets of possible candidate positions across a family of profiles (the voting theoretic term for a collection of voter preferences over a set of candidates) rather than candidates in a specific profile. As a second matter, exclusion zones are defined by the outcome of the election under a given voting system rather than pairwise preferences of voters.

Another notable approach to studying voting systems in metric spaces comes from the literature on utilitarian metric distortion~\cite{procaccia2006distortion,anshelevich2015approximating,anshelevich2018approximating}. In this framework, voters and candidates have unknown positions in a metric space and the \emph{distortion} of a voting system is the worst-case ratio (over positions consistent with expressed voter rankings) between the total distance from voters to the elected candidate and the minimal total distance to any candidate. No voting system can have metric distortion better than 3~\cite{anshelevich2015approximating} and voting systems achieving this bound are known~\cite{gkatzelis2020resolving,kizilkay2022apluralityveto}. The distortions of other voting rules are also known, including for Borda count, plurality, IRV, and Copeland~\cite{anshelevich2018approximating,anagnostides2022dimensionality}. Utilitarian distortion is a valuable tool for comparing voting systems, but answers a different question than our work. We ask what regions of a space are favored by a voting system over all possible candidate sets rather than how bad an outcome can be in a worst case over unknown voter and candidate positions.  

Relating to our study of voting on graphs, graph-based preferences have a long history in the facility location literature, where facilities can be viewed as candidates and customers can be viewed as voters. A Condorcet node is then a facility preferred by more than half of customers to any other facility~\cite{wendell1981new,bandelt1985networks,hansen1986equivalence}. Graph-based preferences also occasionally appear in the social choice literature as a special case of metric preferences~\cite{skowron2017social}. One working paper uses graph-based voting to explain the success of the Medici family in medieval Florence~\cite{telek2016power}. From a different angle, graph-distance voting has recently been proposed as a node centrality measure~\cite{brandes2022voting,skibski2023closeness}.