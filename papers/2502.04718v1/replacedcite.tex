\section{Related Work}
\label{sec:related_work}

Text Style Transfer (\textsc{TST}) tasks are traditionally evaluated using three key dimensions: \emph{style transfer accuracy}, \emph{content preservation}, and \emph{fluency} ____. Prior work underscores the challenge of jointly capturing subtle stylistic nuances and preserving semantic content ____.

\paragraph{Style Transfer Accuracy}
A common approach is to train a dedicated classifier to check whether the transformed text reflects the intended style ____. Alternatively, unsupervised methods rely on distributional shifts in style-related features ____.

\paragraph{Content Preservation}
Metrics such as \emph{BLEU} ____ and embedding-based similarity ____ often serve as proxies for semantic fidelity. However, they may overlook nuances introduced by stylistic transformations in both single-language and multilingual contexts ____, and recent studies highlight the shortcomings of traditional similarity measures when evaluating paraphrase-like modifications ____.

\paragraph{Fluency}
\emph{Fluency} is typically estimated using perplexity from a pre-trained language model such as \emph{GPT-2} ____. Nonetheless, perplexity may fail to capture context-specific grammatical coherence, especially if the style domain diverges from the model’s training data ____, and can yield inconsistent performance across languages ____.