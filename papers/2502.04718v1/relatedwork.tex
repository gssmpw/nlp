\section{Related Work}
\label{sec:related_work}

Text Style Transfer (\textsc{TST}) tasks are traditionally evaluated using three key dimensions: \emph{style transfer accuracy}, \emph{content preservation}, and \emph{fluency} \cite{mukherjee2024textstyletransferintroductory, hu2022text, jin2022deep}. Prior work underscores the challenge of jointly capturing subtle stylistic nuances and preserving semantic content \cite{briakou2021review, tikhonov2019style}.

\paragraph{Style Transfer Accuracy}
A common approach is to train a dedicated classifier to check whether the transformed text reflects the intended style \cite{prabhumoye2018style, shen2017style}. Alternatively, unsupervised methods rely on distributional shifts in style-related features \cite{yang2018unsupervised, tikhonov2019style}.

\paragraph{Content Preservation}
Metrics such as \emph{BLEU} \cite{papineni2002bleu} and embedding-based similarity \cite{rahutomo2012semantic, reimers2019sentence} often serve as proxies for semantic fidelity. However, they may overlook nuances introduced by stylistic transformations in both single-language and multilingual contexts \cite{yamshchikov2021style, briakou2021evaluating}, and recent studies highlight the shortcomings of traditional similarity measures when evaluating paraphrase-like modifications \cite{yamshchikov2021style, briakou2021review}.

\paragraph{Fluency}
\emph{Fluency} is typically estimated using perplexity from a pre-trained language model such as \emph{GPT-2} \cite{radford2019language}. Nonetheless, perplexity may fail to capture context-specific grammatical coherence, especially if the style domain diverges from the model’s training data \cite{tikhonov2019style, briakou2021review}, and can yield inconsistent performance across languages \cite{briakou2021evaluating}.