

\section{Discussion}
\label{sec:discussions}

This paper presents a method for optimizing 3D Gaussian representations while self-calibrating camera parameters and lens distortion. 
Our approach enables the use of large field-of-view captures to achieve efficient and high-quality reconstruction without cumbersome pre-calibration. Even with fewer input captures, our method maintains comprehensive scene coverage and reconstruction quality.
% We compare our method with state-of-the-art techniques and demonstrate superior performance on datasets containing large FoV captures.


\vspace{-1em}
\paragraph{Limitations and Future directions.} 
This work does not account for entrance pupil shift in fisheye lenses, which affects near-field scenes by causing splat misalignment but is negligible for distant scenes. Vignetting and chromatic aberration were also not modeled, leaving room for future extensions. Additionally, intensity discontinuities arise at cubemap face boundaries due to variations in 2D covariances. A simple fix involves switching from depth sorting to distance sorting, while a more refined solution could involve projecting covariances orthogonally to the viewing ray. More discussions can be found in the supplementary.


% \vspace{-1em}
\paragraph{Societal Impact.} 
This research can benefit industries that depend on 3D reconstruction, such as film production and virtual reality. 
However, a potential downside is the environmental impact associated with the increased computational resources required for model optimization.

\section{Acknowledgement}
This work was supported in part by the National Science Foundation under grant 2212084 and the Vannevar Bush Faculty Fellowship.
We want to express gratitude to Xichen Pan, Xiangzhi Tong, Julien Philip, Li Ma, Hansheng Chen, Jan Ackermann, and Eric Chen for their discussion and suggestions on this paper. 

\clearpage