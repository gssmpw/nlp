{
% \begin{table*}[ht]
%   \centering
%   \scalebox{0.71}{
%     \begin{tabular}{ccccccccccccccccc}
%     \toprule
%     \multicolumn{1}{c}{\multirow{2}[0]{*}{Method}} & \multicolumn{1}{c}{\multirow{2}[0]{*}{Validation Set}} & \multicolumn{3}{c}{Blender Kitchen} & \multicolumn{3}{c}{Blender Living} & \multicolumn{3}{c}{Blender Room} & \multicolumn{3}{c}{Garden} & \multicolumn{3}{c}{Studio}\\
%     \cmidrule(lr){3-5}
%     \cmidrule(lr){6-8}
%     \cmidrule(lr){9-11}
%     \cmidrule(lr){12-14}
%     \cmidrule(lr){15-17}
%     % \cmidrule(lr){17-19}
%     \multicolumn{1}{c}{} & \multicolumn{1}{c}{} & SSIM  & PSNR  & LPIPS & SSIM  & PSNR & LPIPS & SSIM  & PSNR & LPIPS & SSIM  & PSNR & LPIPS & SSIM  & PSNR & LPIPS\\
%     \midrule
%     3DGS~\cite{kerbl20233d} & \multirow{2}[0]{*}{Perspective} & 0.470 & 11.09 & 0.435 & 0.595 & 15.27 & 0.406 & \textbf{0.897} & 30.88 & \textbf{0.155} & -- & -- & --  & -- & -- & -- \\
%     Ours & & \textbf{0.794} & \textbf{27.56} & \textbf{0.272} & \textbf{0.686} & \textbf{26.27} & \textbf{0.314} & 0.895 & \textbf{33.21} & 0.172 & -- & -- & -- & -- & -- & -- \\
%     \midrule
%     Fisheye-GS~\cite{liao2024fisheye} & \multirow{2}[0]{*}{Fisheye} & 0.601 & 14.30 & 0.485 & 0.549 & 15.54 & 0.548 & 0.739 & 17.97 & 0.355 & 0.530 & 14.94 & 0.542 & 0.536 & 12.24 & 0.549\\
%     Ours & & \textbf{0.886} & \textbf{30.72} & \textbf{0.146} & \textbf{0.842} & \textbf{28.46} & \textbf{0.180} & \textbf{0.929} & \textbf{31.16} & \textbf{0.095} & \textbf{0.882} & \textbf{27.85} & \textbf{0.144} & \textbf{0.965} & \textbf{33.86} & \textbf{0.044}\\

%     \bottomrule
%     \end{tabular}%
%   } 
%     \vspace{-1mm}
%     \caption{\textbf{Quantitative Evaluation on Wide-Angle Scenes}. We compare our method with vanilla 3DGS~\cite{kerbl20233d} and Fisheye-GS~\cite{liao2024fisheye} on a set of held-out captures, respectively. Two types (\textit{i.e.,} perspective and fisheye) of cameras in ValSet are used and further explained in~\cref{sec:evaluation}. ``--" indicates that no ground truth can be obtained.\TODO{split the table put the number}}
%     \vspace{-2mm}
%     \label{tab:quantitative_large_small}%
% \end{table*}%


\begin{table}[t]
  \centering
  \scalebox{1}{
    \begin{tabular}{ccccc}
        \toprule
        Method & Num & SSIM & PSNR & LPIPS \\
        \midrule
        3DGS~\cite{kerbl20233d} & 200 & 0.654 & 19.08 & 0.332 \\
        \midrule
        \multirow{4}{*}{Ours} 
        & 100 & \textbf{0.800} & \textbf{29.01} & \textbf{0.231} \\
        & 50 & 0.735 & 26.26 & 0.267 \\
        & 25 & 0.709 & 24.59 & 0.292 \\
        & 10 & 0.615 & 22.77 & 0.356 \\
        \bottomrule
    \end{tabular}
  }
  % \vspace{-2mm}
    \caption{\textbf{Evaluation on Mitsuba Scenes}. We compare our method with Vanilla 3DGS~\cite{kerbl20233d} to demonstrate the advantages of using wide-angle cameras over regular-FOV cameras. The testing views are rendered in perspective. Our method achieves better performance and coverage, even with fewer captures.}
    \label{tab:quantitative_large_small}%
  % \vspace{-5mm}
\end{table}
}