{
\begin{figure*}[t]
    \centering
    \setlength{\tabcolsep}{1pt} % Adjust space between columns if needed
    \begin{tabular}{cccc} % 5 columns
         \includegraphics[height=0.22\textwidth]{images/undistort/lego_fish.jpg} &
        \includegraphics[height=0.22\textwidth]{images/undistort/lego_pers.jpg} &
        \includegraphics[height=0.22\textwidth]{images/undistort/car_fish.jpg} &
        \includegraphics[height=0.22\textwidth]{images/undistort/car_pers.jpg}\\
        \multicolumn{2}{c}{(a) Lego} & \multicolumn{2}{c}{(b) Car}
        \vspace{-0mm}
        \\
        \includegraphics[height=0.22\textwidth]{images/undistort/rover_fish.jpg} &
        \includegraphics[height=0.22\textwidth]{images/undistort/rover_pers.jpg} &
        \includegraphics[height=0.22\textwidth]{images/undistort/space_fish.jpg} &
        \includegraphics[height=0.22\textwidth]{images/undistort/space_pers.jpg}\\
        \multicolumn{2}{c}{(c) Rover} & \multicolumn{2}{c}{(d) Spaceship}
    \end{tabular}

    \caption{\textbf{Radial and Perspective Rendering}. We evaluate our method on a synthetic radial distortion dataset. Our approach successfully recovers slight radial distortion during reconstruction and enables perspective rendering upon completion of training.}

    \label{fig:ours_obj_radial}

\end{figure*}
}