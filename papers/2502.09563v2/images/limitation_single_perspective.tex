{
\begin{figure}[t]
    \centering
    \setlength{\tabcolsep}{1pt} % Adjust space between columns if needed
    \begin{tabular}{cc} % 4 columns
        \subcaptionbox{Conventional Paradigm\label{fig:hilbert_180}}{
            \includegraphics[width=0.28\textwidth]{images/illustration_of_single_planar_perspective/undistortion_small_large_fov.jpg}
            \label{fig:conventional}
        } &

        % Fourth image with label (c)
        \subcaptionbox{Ours\label{fig:hilbert_cubemap}}{
            \includegraphics[width=0.16\textwidth]{images/illustration_of_single_planar_perspective/ours_cubemap_fisheye.jpg}
        }
    \end{tabular}
    \vspace{-0.6em}
    \caption{\textbf{Conventional Paradigm vs.\ Our Method}. (a) Conventional approaches require reprojecting the image into perspective views compatible with 3DGS rasterization. As the field of view increases, pixel stretching becomes progressively severe, significantly compromising the quality of the reconstruction. (b) In contrast, our cubemap resampling strategy maintains a consistent pixel density across the entire field of view. This approach, combined with our hybrid distortion field, utilizes the peripheral regions (the annular area outside the blue box) without severe distortion or pixel stretching. Moreover, our method can handle fields of view up to 180°, as demonstrated by the green box, allowing for comprehensive and accurate reconstructions.}
    \label{fig:limitation_single_perspective}
    \vspace{-0.5em}
\end{figure}
}