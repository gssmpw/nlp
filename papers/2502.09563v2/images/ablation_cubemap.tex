{\begin{figure}[t]
    \centering
    \setlength{\tabcolsep}{1pt} % Adjust space between columns if needed
    \scalebox{0.95}{
    \begin{tabular}{ccc} % 4 columns (Vertical Caption | Image Set 1 | Image Set 2 | Image Set 3)

        \includegraphics[width=0.158\textwidth]{images/large_vs_small_fov_jpg/living_without_cubemap_fisheye.jpg} &
        \includegraphics[width=0.158\textwidth]{images/large_vs_small_fov_jpg/kitchen_without_cubemap_fisheye.jpg} &
        \includegraphics[width=0.158\textwidth]{images/large_vs_small_fov_jpg/hilbert_without_cubemap_fisheye.jpg}
        \\

        \includegraphics[width=0.158\textwidth]{images/large_vs_small_fov_jpg/living_ours_fisheye.jpg} &
        \includegraphics[width=0.158\textwidth]{images/large_vs_small_fov_jpg/kitchen_ours_fisheye.jpg} &
        \includegraphics[width=0.158\textwidth]{images/large_vs_small_fov_jpg/hilbert_ours_fisheye.jpg}
        \\
        \multicolumn{1}{c}{(a) Living} & \multicolumn{1}{c}{(b) Kitchen} & \multicolumn{1}{c}{(c) Room}
        \\
    \end{tabular}
    }
    \caption{\textbf{Single Planar Projection with Hybrid Field}. Our hybrid field can be directly applied to a single plane during rasterization. However, the limitation of single planar projection is that it cannot cover the full FOV of the raw images, leading to partial loss of information in the peripheral regions.}
    \label{fig:ablation_cubemap}
\end{figure}
}