%\setlength{\textfloatsep}{0.2cm}
%\setlength{\floatsep}{0.2cm}
\IEEEoverridecommandlockouts
%\normalsize

%\documentclass[twocolumn]{IEEEtran}
%\documentclass[12pt, draftclsnofoot, onecolumn]{IEEEtran}
%\setlength{\oddsidemargin}{-.2 in}
%\setlength{\evensidemargin}{-.2 in}
%\setlength{\textheight}{9.2 in}
%\setlength{\textwidth}{6.9 in}

\usepackage{amsmath, amssymb, cite}
\usepackage{amsthm,bm,bbm}
\usepackage{mathtools}
\usepackage[size=small]{caption}
\usepackage{amsthm,multirow,color,amsfonts}
\usepackage{tabulary}
\usepackage{subfigure}
\usepackage{graphicx}
\usepackage{setspace}
\usepackage{enumerate}
\usepackage{algorithm}
\usepackage{algpseudocode}
\usepackage{comment}
\usepackage{pdfpages}
\usepackage{hyperref}
\usepackage{stfloats}

\addtolength{\topmargin}{1mm}

\usepackage{etoolbox}
\makeatletter
\patchcmd{\@makecaption}
  {\scshape}
  {}
  {}
  {}
\makeatother

\makeatletter
\def\BState{\State\hskip-\ALG@thistlm}
\makeatother


\usepackage[utf8]{inputenc} 
\usepackage[T1]{fontenc}
\usepackage{url}
\usepackage{ifthen}
%\usepackage[cmex10]{amsmath} % Use the [cmex10] option to ensure complicance
                             % with IEEE Xplore (see bare_conf.tex)

%% Please note that the amsthm package must not be loaded with
%% IEEEtran.cls because IEEEtran provides its own versions of
%% theorems. Also note that IEEEXplore does not accepts submissions
%% with hyperlinks, i.e., hyperref cannot be used.




\interdisplaylinepenalty=2500 % As explained in bare_conf.tex




		
% Reduces space around equations and figures 
%\setlength\abovedisplayskip{3pt plus 2pt minus 2pt} 	% Reduce space before equation
%\setlength\belowdisplayskip{3pt plus 2pt minus 2pt}	% Reduce space after equation
%\setlength\textfloatsep{12pt plus 2pt minus 2pt}		% Reduce space between figure caption and text





\DeclareMathOperator*{\Hg}{{\emph{\bf \em H}}} %graph entropy notation




\newcommand\independent{\protect\mathpalette{\protect\independenT}{\perp}}
\def\independenT#1#2{\mathrel{\rlap{$#1#2$}\mkern2mu{#1#2}}}


\usepackage[dvipsnames]{xcolor}
%black, blue, brown, cyan, darkgray, gray, green, lightgray, lime, magenta, olive, orange, pink, purple, red, teal, violet, white, yellow.



\def\rt{\theta}
\def\ro{\omega}
\def\rO{\Omega}



              
\newtheorem{theo}{Theorem}
\newtheorem{defi}{Definition}
\newtheorem{assu}{Assumption}
\newtheorem{prop}{Proposition}
\newtheorem{remark}{Remark}
\newtheorem{note}{Note}
\newtheorem{cor}{Corollary}
\newtheorem{con}{Conjecture}
\newtheorem{ques}{Questions}
\newtheorem{lem}{Lemma}
\newtheorem{ex}{Example}



\newcommand{\E}[1]{\mathbb{E}\left[#1\right]}
\newcommand{\Pro}[1]{\mathbb{P}\left[#1\right]}

\newcommand{\err}{\mathcal{E}r}
\newcommand{\Rcomp}{R'}

\newcommand{\GU}{G_{\cup_{i=1}^{K}f_i}}
\newcommand{\derya}[1]{\textcolor{blue}{ #1}}
\newcommand{\deryar}[1]{\textcolor{red}{ #1}}
\newcommand{\vijith}[1]{\textcolor{red}{#1}}
\newcommand{\chr}[1]{{\em \color{red} #1}}
\newcommand\norm[1]{\left\lVert#1\right\rVert}

\newcommand{\reza}[1]{\textcolor{red}{#1}}


\newcommand{\s}{\mbox{sec}}

\DeclareMathOperator*{\qq}{{\small q}} 
\DeclareMathOperator*{\qtwo}{{\small 2}} 



% \usepackage{chngcntr}
% \newcounter{para}
% \counterwithin{para}{subsection} % makes paragraph depend on subsection
% \newcommand\mypara[1]{\par\refstepcounter{para}\textbf{\thepara)\space#1\space}}
% % \counterwithin{paragraph}{chapter} % makes paragraph depend on chapter


\hyphenation{op-tical net-works semi-conduc-tor}

\interdisplaylinepenalty=2500 % As explained in bare_conf.tex


\newcommand{\coloring}{\mathcal{C}(C_{2K}^n)}
\newcommand{\coloringodd}{\mathcal{C}}



\def\compactify{\itemsep=0pt \topsep=0pt \partopsep=0pt \parsep=0pt}
\let\latexusecounter=\usecounter
\newenvironment{CompactItemize}
  {\def\usecounter{\compactify\latexusecounter}
   \begin{itemize}}
  {\end{itemize}\let\usecounter=\latexusecounter}
\newenvironment{CompactEnumerate}
  {\def\usecounter{\compactify\latexusecounter}
   \begin{enumerate}}
  {\end{enumerate}\let\usecounter=\latexusecounter}

  
\def\independenT#1#2{\mathrel{\rlap{$#1#2$}\mkern2mu{#1#2}}}

\singlespacing



%%%%%%%%%%%%%%%%%%%%%%%%
\ifx\du\undefined
  \newlength{\du}
\fi
\setlength{\du}{4.\unitlength}
\usepackage{siunitx}
\usepackage{tikz}
\usetikzlibrary{shapes.geometric}
% \usepackage{pgfplots}
\usepackage{verbatim}
\usepackage{color,soul}
\usetikzlibrary{spy,calc}
% \usepackage{amsthm}
% \usepackage{caption}

%\usepackage[hyphens]{url}
%\usepackage[hidelinks]{hyperref}
\hypersetup{breaklinks=true}
\urlstyle{same}



%\usepackage[cmex10]{amsmath} % Use the [cmex10] option to ensure complicance
                             % with IEEE Xplore (see bare_conf.tex)


\usepackage{diagbox}


\DeclareMathOperator*{\argminA}{arg\,min}
\tikzstyle{vertex}=[circle,minimum size=1pt,inner sep=0pt]
\tikzstyle{selected vertex} = [vertex, fill=red!24]
\tikzstyle{edge} = [draw,thick,-]
\tikzstyle{weight} = [font=\small]
\tikzstyle{selected edge} = [draw,line width=5pt,-,red!50]
\tikzstyle{ignored edge} = [draw,line width=5pt,-,black!20]