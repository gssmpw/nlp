\begin{table*}[t]
    \centering
    \caption{\textbf{Comparison of multiple protein-generation methods on the ACYP protein dataset.} All models are evaluated over 100 generated sequences. We report average pLDDT and TM-score (higher is better), RMSD and scPPL (lower is better), and H-prob (percentage of samples matching known protein families). 
    % Notably, \method{} attains near-native structural quality (pLDDT $\approx$92.9) while requiring far fewer diffusion steps (64 vs. 1000). Standard deviations are omitted for brevity. \rex{say sth about std in experiment section?}
    }
    \vspace{3pt}
    \begin{adjustbox}{max width=0.85\linewidth}
    \begin{tabular}{lcccc|c}
    \toprule
        \textsc{Model} 
        & \textsc{pLDDT ($\uparrow$)} 
        & \textsc{TM-score ($\uparrow$)} 
        & \textsc{RMSD ($\downarrow$)} 
        & \textsc{H-prob ($\uparrow$)} 
        & \textsc{scPPL ($\downarrow$)}  \\
    \midrule
        {ACYP dataset} 
        & 94.8 & 0.98 & 0.86 & 100\% & 3.02  \\
    \midrule
        D3PM      
        & 82.6 & 0.91 & 1.41 & 90\% & 7.47  \\
        % SEDD (64 steps)       
        % & 76.8 & 0.95 & 0.95 & 98\% & --- & --- \\
        % Transformers          
        % & --- & --- & --- & --- & --- & --- \\
        Discrete Flow Matching
        & 83.6 & 0.89 & 1.48 & 87\% & 7.20  \\
        MDLM      
        & 83.4 & 0.91 & 1.27 & 92\% & 7.02  \\
        UDLM      
        & 84.7 & 0.92 & 1.28 & 92\% & 7.19  \\
        SEDD      
        & 79.7 & 0.95 & 0.91 & 99\% & 5.10  \\
    \midrule
        \textbf{\method{}} (64 steps) 
        & \textbf{92.9} & \textbf{0.97} & \textbf{0.90} & \textbf{100\%} & \textbf{3.19}  \\
    \bottomrule
    \end{tabular}
    \end{adjustbox}
    \label{tab:protein}
    \vspace{-5pt}
\end{table*}