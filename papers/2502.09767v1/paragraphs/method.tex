\begin{figure*}[t]
    \centering
    \includegraphics[width=0.9\linewidth]{paragraphs/figures/model_9.pdf}
    \caption{(a). \textbf{Inference paradigm for a standard causal language model versus \method{}.} In \method{}, each timestep first autoregressively denoises the tokens into $\widetilde{\mathbf{x}}_0$, then re-applies noise via the diffusion kernel to obtain A traditional autoregressive model emerges as the special case of $T=1$, which can be adapted to discrete diffusion by fine-tuning. (b). \textbf{Extending 1D to 2D Rotary Positional Encoding.} Standard rotary encodings for token positions are seamlessly generalized to also encode diffusion timesteps, remaining fully backward-compatible with existing language model architectures.}
    \label{fig:model}
\end{figure*}

\section{Non-Markovian Discrete Diffusion}
\label{sec:framework}

Previous methods have modeled the discrete diffusion process as a Markovian process, where the model learns an instantaneous reverse process to denoise \(\mathbf{x}_t\) and reconstruct \(\mathbf{x}_{t-1}\) by \(p_\theta (\mathbf{x}_{t-1} \mid \mathbf{x}_{t})\)~\citep{d3pm}. Despite efficient generation, this Markovian constraint can limit the model’s ability to capture long-range dependencies within the latent chain. All relevant information is compressed into single state \(\mathbf{x}_t\), potentially leading to a non-robust inference procedure.

In this paper, we extend the non-Markovian diffusion process to discrete data modeling following~\citep{DART}. Specifically, recent studies have demonstrated that the Markovian assumption is not strictly necessary in the inference process. Breaking this assumption allows for the incorporation of the entire temporal trajectory \(\mathbf{x}_{t:T}\) to denoise \(\mathbf{x}_{t-1}\) by \(p_\theta (\mathbf{x}_{t-1} \mid \mathbf{x}_{t:T})\) in autoregressive manner, leading to a more expressive and robust inference process. 

In the following, we will describe how the non-Markovian discrete diffusion process is constructed. Crucially, we will see that the resulting non-Markovian autoregressive inference mechanism essentially aligns with a causal language model plus an additional temporal dimension—laying the groundwork for our unified spatial-temporal framework (Section~\ref{sec:method}).



% \rex{i think it would be clearer if we first lay out the spatial temporal setting and the big picture}



\subsection{Hybrid Non-Markovian Forward Trajectory}
A key challenge in realizing non-Markovian inference is how to design the forward trajectory so that future states \(\mathbf{x}_{t+1: T}\) carry more complementary information about\(\mathbf{x}_t\). The straightforward Markovian absorbing process \(q(\mathbf{x}_{0:T}) = \prod_{t=1}^{T} q(\mathbf{x}_t | \mathbf{x}_{t-1})\) is ill-suited for this purpose, as information is highly limited and redundant. To ensure each timestep retains complementary information about the original data, we (1). \textit{construct the forward trajectory by independently corrupting \(\mathbf{x}_0\).}
(2). \textit{Mix an abosrbing kernel with uniform kernal to produce more diverse noisy states.}

\paragraph{Independent Corruption.}
As shown in Fig.~\ref{fig:non-markov}, the diffusion trajectory \(\mathbf{x}_{0:T}\) is created  where we add \textit{independent} noise to \(\mathbf{x}_0\) at different timesteps, rather than relying on the previous state as the noise source. The forward trajectory is constructed as:
    \begin{align}
    q(\mathbf{x}_{0:T}) :=&  q(\mathbf{x}_0, \mathbf{x}_1, \mathbf{x}_2, ..., \mathbf{x}_T) \\
    =& q(\mathbf{x}_0) \prod_{t=1}^{T} q(\mathbf{x}_t | \mathbf{x}_{0:t-1}) \\
    =& q(\mathbf{x}_0) \prod_{t=1}^{T} q(\mathbf{x}_t | \mathbf{x}_{0}),
    \end{align} 
where \(q(\mathbf{x}_t | \mathbf{x}_{0})\) is the marginal conditional distribution, which is obtained from a standard Markovian diffusion kernel but applied directly to \(\mathbf{x}_0\). For example, in absorbing or uniform diffusion processes, we can write:
    \begin{align}
        q\left(\mathbf{x}_t \mid \mathbf{x}_0\right)&=\operatorname{Cat}\left(\mathbf{x}_t ; \mathbf{x}_0 \overline{\mathbf{Q}}_t\right) \\
        &= \operatorname{Cat}\left(\mathbf{x}_t ; \mathbf{x}_0 \prod_{k=1}^{t} \mathbf{Q}_k\right)
    \end{align}
    where \(\mathbf{Q}_k\) is the transition matrix at step \(k\), and \(\overline{\mathbf{Q}}_t\) is the product of all such transitions up to \(t\). This construction generalizes the Markovian forward trajectory (see Appendix for further details) and creates a sequence of noisy states that better preserve intermediate information across timesteps.

    \begin{proposition}[Discrete Non-Markovian Information Gain]
        \label{thm:non_markov_gain}
        Let \(\{\mathbf{x}_0,\mathbf{x}_1,\ldots,\mathbf{x}_T\}\) be a discrete forward diffusion process that is \emph{not} strictly Markovian.  Suppose there exists at least one timestep \(t\) such that\footnote{it is trivial to find such \(t\) in our case with indendent noise corruption.}
        \[
        \mathbf{x}_{t-1}
        \;\not\!\!\perp\!\!\!\!\perp\;
        \mathbf{x}_{t+1:T}
        \;\big|\;
        \mathbf{x}_t.
        \]
        Then the conditional mutual information 
        \[
        I\bigl(\mathbf{x}_{t-1}; \mathbf{x}_{t+1:T}\,\big|\;\mathbf{x}_t\bigr)
        \;>\; 0,
        \]
        which implies that conditioning on \(\mathbf{x}_{t+1:T}\) in the reverse process \(\,p_\theta(\mathbf{x}_{t-1}\mid \mathbf{x}_{t:T})\) strictly reduces the uncertainty about \(\mathbf{x}_{t-1}\) compared to conditioning on \(\mathbf{x}_t\) alone.
    \end{proposition}
% \rex{discuss its significance}
In other words, when the non-Markovian forward trajectory is designed with independent corruption, future noisy states can complement each other, bolstering the reverse inference process.

\paragraph{Hybrid Diffusion Kernel.} Most prior discrete diffusion methods rely on a single kernel, such as purely absorbing (where tokens are replaced by \texttt{[MASK]}) or purely uniform corruption. However, sticking to one kernel can lead to monotonic or insufficiently diverse noise patterns in the generative trajectory. We therefore mix an absorbing kernel and a uniform kernel:

\begin{equation}
    \overline{\mathbf{Q}}_t = (1 -  \alpha_{t} -\beta_{t})\mathbf{I} 
    + \alpha_{t}\overline{\mathbf{Q}}_T^{\text{absorb}} 
    + \beta_{t}\overline{\mathbf{Q}}_T^{\text{uniform}},
\end{equation}
where $\alpha_t, \beta_t$ are time-dependent schedules with boundary conditions $\alpha_0=0, \alpha_T=1, \beta_0=$ $0, \beta_T=0$. The terms $\overline{\mathbf{Q}}_T^{\text {absorb }}$ and $\overline{\mathbf{Q}}_T^{\text {uniform }}$ are the marginal diffusion kernels (absorbing and uniform, respectively) at the final step. By mixing these kernels, we introduce a spectrum of corruption modes-some tokens may be masked out, while others might be replaced by random symbols, thus yielding a more informative trajectory. 

\begin{algorithm}[t]
    \caption{Inference for Non-Markovian Discrete Diffusion}
    \label{alg:non_markov_inference}
    \begin{algorithmic}[1]
    \State \textbf{Input:} Prior distribution $q(\mathbf{x}_T)$, model parameters $\theta$
    \State \textbf{Output:} Sampled data $\mathbf{x}_0$
    \State Initialize $\mathbf{x}_T \sim q(\mathbf{x}_T)$ as the noisy input data at the final timestep
    \For{$t = T \textbf{ down to } 1$}
    
        % \Comment{Using $x_0$ parameterization}
        \State Predict the clean data $\mathbf{x}_{0}$ from the historical trajectory $\mathbf{x}_{t:T}$ using the model:
        $p_\theta\left( \mathbf{x}_{0} \mid \mathbf{x}_{t:T} \right)$
        \State Sample from the predicted distribution to obtain the clean data at timestep $t-1$: 
        $\widetilde{\mathbf{x}}_{0} \sim p_\theta\left( \mathbf{x}_{0} \mid \mathbf{x}_{t:T} \right)$
        \State Add noise to the predicted clean data to get the next timestep $t-1$:
        $\mathbf{x}_{t-1} \sim q\left( \mathbf{x}_{t-1} \mid \widetilde{\mathbf{x}}_{0} \right)$
        \EndFor
    \State \Return $\mathbf{x}_0$
    \end{algorithmic}
    \end{algorithm}

% \paragraph{Combining Uniform and Absorbing forward diffusion process}

% \TODO{Mixing uniform and absorbing diffusion forward process}
    
\subsection{Non-Markovian Inference Process}
We train the non-Markovian reverse model 
\(
p_\theta\bigl(\mathbf{x}_{t-1} \mid \mathbf{x}_{t:T}\bigr)
\)
by minimizing a weighted ELBO objective derived from the variational perspective of diffusion:

\begin{equation}
    \mathcal{L}_{\text{non-markov}} = 
    \mathbb{E}_{\mathbf{x}_{1: T} \sim q\bigl( \mathbf{x}_{1:T} \mid \mathbf{x}_0\bigr)}
    \Biggl[
        \sum_{t=1}^T \tilde{\omega}_t \,\log p_\theta\bigl(\mathbf{x}_{t-1} \mid \mathbf{x}_{t:T}\bigr)
    \Biggr].
\end{equation}
Here, the key difference from standard discrete diffusion is that each term conditions on \(\mathbf{x}_{t+1: T}\). By Theorem~\ref{thm:non_markov_gain}, this inclusion \textit{strictly reduces} the conditional entropy when the forward process is indeed non-Markovian. Consequently, \(p_\theta\left(\mathbf{x}_{t-1} \mid \mathbf{x}_{t: T}\right)\) has a more robust denoising pathway.

Once trained, sampling proceeds in an autoregressive manner, iterating backward over time by conditioning on the entire future trajectory \(\mathbf{x}_{t:T}\).

% TODO: maybe move to next section
\paragraph{\(\textbf{x}_0\)-Parameterization.}
Similar to standard discrete diffusion approaches~\citep{udlm, gat2024discreteflowmatching}, we adopt an \(x_0\)-parameterization to simplify training. In this view, we directly predict the clean sequence \(\mathbf{x}_0\) at each step, which leads to a simpler denoising objective:
\begin{equation}
    \mathcal{L}_{\text{non-markov}}
    \;=\;
    \mathbb{E}_{\mathbf{x}_{1: T} \sim q\bigl( \mathbf{x}_{1:T} \mid \mathbf{x}_0\bigr)}\!\Biggl[
        \sum_{t=1}^T  
        \log p_\theta\bigl(\mathbf{x}_0 \mid \mathbf{x}_{t:T}\bigr)
    \Biggr].
\end{equation}
(See Appendix for the derivation.) At inference time, we first sample \(\widetilde{\mathbf{x}}_0\) from the learned denoiser \(p_\theta(\mathbf{x}_0 \mid \mathbf{x}_{t:T})\) and then use the forward kernel \(q(\cdot \mid \widetilde{\mathbf{x}}_0)\) to obtain \(\mathbf{x}_{t-1}\). This procedure iterates backward through time until we reach \(\mathbf{x}_0\). Pseudocode is presented in Algorithm~\ref{alg:non_markov_inference}.


\section{\method{}: \underline{Ca}usal \underline{D}iscrete \underline{Di}ffusion Model}
\label{sec:method}

The autoregressive property of non-Markovian inference (Section~\ref{sec:framework}) naturally lends itself to a causal model that predicts tokens in a left-to-right fashion. Motivated by this insight, we propose \textbf{\method{}}, a causal language model that unifies the \textit{sequential} dimension (i.e., token order) with the \textit{temporal} dimension (i.e., discrete diffusion timesteps). Concretely, \method{} leverages a standard left-to-right structure while conditioning on multiple timesteps of the diffusion chain, enabling it to handle non-Markovian discrete diffusion in a single unified framework.

\subsection{Unified Sequential and Temporal Modeling}

In the non-Markovian setting (Section~\ref{sec:framework}), the reverse process is inherently autoregressive: we decode the entire sequence of latent states \(\mathbf{x}_{t:T}\) to retrieve \(\mathbf{x}_{t-1}\). This naturally aligns with the standard left-to-right generation of language models. However, unlike a conventional \textit{single}-dimensional token sequence, we must now account for a {temporal} dimension.

\paragraph{Constructing the Training Trajectory.} 
To accommodate both sequential and temporal dependencies in a single model, \method{} constructs a \textit{non-Markovian} forward trajectory for each data instance:
\begin{equation*}
    \bigl(\mathbf{x}_T^{(0)}, \ldots, \mathbf{x}_T^{(L)},\; \mathbf{x}_{T-1}^{(0)}, \ldots, \mathbf{x}_{0}^{(L)}\bigr),
\end{equation*}
where the upper index \((i)\) denotes the token position in the original sequence of length \(L\), and the subscript denotes the diffusion timestep. We then train a causal language model to predict {the next token} at each position via a standard next-token prediction loss. Crucially, the target for each position \(i\) is always the original clean token \(\mathbf{x}_0^{(i)}\) under \(\textbf{x}_0\)-parameterization.

\paragraph{Context Window.}
In practice, feeding all timesteps into the model can be prohibitively large. Therefore, we restrict the model’s temporal context to the most recent \(n\) timesteps. More formally, for each timestep \(t\) and token position \(i\), we define the context \(\mathcal{C}_{t,i}\) as follows:
\begin{align}
    \mathcal{C}_{t,i} &= 
    \left\{ \mathbf{x}_\tau^{(j)} \;\middle|\; t-1 \leq \tau \leq \min(t+n, T),\; 0 \leq j \leq L \right\} \notag \\
    &\cup 
    \left\{ \widetilde{\mathbf{x}}_0^{(j)} \;\middle|\; 0 \leq j < i \right\}
\end{align}
where \(\widetilde{\mathbf{x}}_0^{(j)}\) is the predicted clean token\footnote{In teacher-forced training, this is the ground truth clean token} at position \(j\) at current timestep \(t\). The token-level training objective is then:
\begin{equation}
    \mathcal{L}^{t, i}_{\text{token}} = \mathbb{E}\left[ -\log p_\theta\left( \mathbf{x}_0^{(i)} \;\middle|\; \mathcal{C}_{t,i} \right) \right].
\end{equation}
This design ensures that \method{} captures both local temporal structure and token-level dependencies in a scalable manner.
In practice, there is a inherent trade-off between scalability and the ability to model long-range temporal dependencies. We found \(n=4\) a reasonable choice through empirical evaluation.
% TODO: add some ending sentence, possibly with parallel training
% \rex{how do we pick $n$ in practice? is model performance sensitive to this?}


\subsection{2D Rotary Positional Encoding for Sequence Position \& Diffusion Timestep}
\label{subsec:2d_rotary}

A modern causal language model typically encodes \textit{only} the sequential dimension, via rotary positional encodings~\citep{roformer}. However, for non-Markovian discrete diffusion, we must capture not just the standard token-level sequence but also a temporal dimension corresponding to diffusion timesteps. To address this, we extend the original 1D rotary scheme to a {2D} variant 
% \rex{how does it compare with the fourier feature https://arxiv.org/abs/2106.02795 ?}
, allowing the model to incorporate positional information across both the sequence index \(i\) and the diffusion timestep \(t\). This enhanced encoding enables the model to more effectively learn joint dependencies between tokens and their progression through multiple diffusion steps.



Specifically, standard RoPE in modern language models like Pythia~\citep{biderman2023pythia} rotates a subset
% \footnote{For models that rotate the full set of dimensions, we could also extend 1D RoPE to 2D RoPE by composing two rotation matrixs.}
of the query/key dimensions according to the token position $i$. If $\mathbf{R}^{(i)}$ denotes the rotation matrix parameterized by $i$, the attention weight between positions $i$ and $j$ becomes $\left(\mathbf{R}^{(i)} \mathbf{q}^{(i)}\right)^{\top}\left(\mathbf{R}^{(j)} \mathbf{k}^{(j)}\right)$, where

% Specifically, in modern language models like Pythia~\citep{biderman2023pythia}, position is integrated by rotating a subset of the query/key dimensions in calculating attention weight\footnote{In some architectures, all dimensions of query/key are roateted. In that case, we can also extend 1D RoPE to 2D RoPE by composing rotation matrix introduced by two dimensions respectively.}:
% \begin{equation*}
% \alpha_{i,j}=\left(\mathbf{R}^{(i)} \mathbf{q}^{(i)}\right)^{\top}\left(\mathbf{R}^{(j)} \mathbf{k}^{(j)}\right)={\mathbf{q}^{(i)}}^{\top} \mathbf{R}^{(j-i)} \mathbf{k}^{(j)},
% \end{equation*}
% where
\begin{equation*}
\label{eq:1d_rotary}
\resizebox{.70\columnwidth}{!}{$
\mathbf{R}^{(i)}=\left(\begin{array}{ccccc}
\cos \left(i \theta_{12}\right) & -\sin \left(i \theta_{12}\right) & 0 & \cdots \\
\sin \left(i \theta_{12}\right) & \cos \left(i \theta_{12}\right) & 0 & \cdots \\
0 & 0 & 0 & \cdots\\
\vdots & \vdots &  \vdots & \ddots
\end{array}\right).
$}
\end{equation*}

As shown in Figure~\ref{fig:model}, we generalize this approach by introducing additional rotation for the timestep dimension:
\begin{equation*}
\label{eq:2d_rotary}
\resizebox{.95\columnwidth}{!}{$
\mathbf{R}_t^{(i)} = \begin{pmatrix}
    \cos \left(i \theta_{12}\right) & -\sin \left(i \theta_{12}\right) & 0 & 0 & \cdots\\
    \sin \left(i \theta_{12}\right) & \cos \left(i \theta_{12}\right)  & 0 & 0 & \cdots\\
    0 & 0 & \cos \left(t \theta_{34}\right) & -\sin \left(t \theta_{34}\right) & \cdots\\
    0 & 0 & \sin \left(t \theta_{34}\right) & \cos \left(t \theta_{34}\right)  & \cdots\\
    \vdots & \vdots & \vdots & \vdots & \ddots
\end{pmatrix}
$}
\end{equation*}
This 2D encoding allows the model to disentangle positional-based rotation (the token dimension \(i\)) from temporal-based rotation (the temporal dimension\(t\)), letting \method{} jointly reason about sequential and temporal positions.

\paragraph{Consistency with Standard Language Modeling.}
By interleaving temporal based rotation in these additional dimensions, it's easy to observe that when two tokens share the same timepoint $t$:
\begin{equation*}
\begin{aligned}
\left(\mathbf{R}_t^{(i)} \mathbf{q}_t^{(i)}\right)^{\top}\left(\mathbf{R}_t^{(j)} \mathbf{k}_t^{(j)}\right) & ={\mathbf{q}_t^{(i)}}^{\top} \mathbf{R}_0^{(j-i)} \mathbf{k}_t^{(j)} \\
& =\left(\mathbf{R}^{(i)} \mathbf{q}_t^{(i)}\right)^{\top}\left(\mathbf{R}^{(j)} \mathbf{k}_t^{(j)}\right)
\end{aligned}
\end{equation*}
which means that in the same timepoint the sequential attention pattern is identical to that of a conventional causal language model, and \(\mathbf{R}_t^{(i)}\) reduces to the usual (1D) rotation in \(i\).  

Hence, \method{} remains \textit{backward-compatible}: if no temporal dimension is present or within the same timepoint, it behaves like a standard causal language model. This design ensures that \method{} smoothly unifies the standard sequential modeling paradigm with the demands of non-Markovian discrete diffusion.

% \begin{figure*}[ht]
% \centering
% \begin{tcolorbox}[
%     width=0.9\textwidth,
%     colback=white,     % Background color
%     colframe=black,    % Frame color
%     boxsep=5pt,        % Box inner sep
%     arc=1pt,           % Rounding corners
%     title=Sample Generations from Our Model
% ]
% % \textbf{Prompt:} The capital of France is \underline{\phantom{~~~~}}. 

% % \textbf{Model Output 1:} 
% % The capital of France is Paris. It is well known for its art, fashion, gastronomy and culture.  

% % \medskip

% % \textbf{Prompt:} Write a short poem about computers. 

% % \textbf{Model Output 2:}
% % Silicon dreams dance  
% % On digital fields so vast  
% % Computers humming  
% \end{tcolorbox}
% \caption{Examples of generated sentences produced by our model. Note that the figure environment spans both columns.}
% \label{fig:sample_generations}
% \end{figure*}

\begin{figure*}[ht]
\centering
\begin{minipage}{0.9\textwidth}  % Adjust width as desired
\hrule height 0.8pt
\vspace{0.5em}
\small

..., I have been advocating for the 'right to speak up if I don’t trust them.' As I have stated before, this is \textcolor{cyan}{something I believe} in deeply. I was once caught in a fantasy, lost in uncertainty, but speaking out is better han remaining silent. Over time ...

\vspace{0.5em}
\hrule
\vspace{0.5em}

... A recent investigation by The Guardian revealed the impact of economic policies on \textcolor{cyan}{housing prices} and other financial factors. However, during a period of heightened anti-terror measures, the push for justice has remained a priority, ...

\vspace{0.5em}
\hrule
\vspace{0.5em}

\textcolor{cyan}{The first season} highlighted the importance of maintaining a strong presence in the league. Injuries \textcolor{cyan}{have made it difficult} for some players, but their role in shaping the team’s success remains crucial. Despite setbacks, their resilience and ...


\vspace{0.5em}
\hrule height 0.8pt
\end{minipage}
\caption{\textbf{Representative text completions illustrating controllable generation with CaDDi, adapted from a pretrained Pythia model.} CaDDi generates meaningful text interleaved with user-provided prompts (highlighted in \textcolor{cyan}{cyan}).}
\label{fig:sample_text_infilling}
\vspace{-5pt}
\end{figure*}

\subsection{Adapt LLMs for Discrete Diffusion}
% Unlocking capability
A key observation is that standard causal language modeling can be seen as a \textbf{special case} of our proposed framework under particular settings: namely, a single-step diffusion (\(T=1\)) and a minimal context window restricted to the current timestep. When \(T=1\), the forward trajectory is simply \(\mathbf{x}_0\), and the reverse process is a single-step denoising process which autoregressively predicts next clean token, closly mirroring the standard language modeling paradigm.

Moreover, as shown in ~\ref{subsec:2d_rotary}, the 2D rotary positional encoding can be seamlessly integrated into existing language models, allowing for a unified treatment of both sequential and temporal dimensions. Given these equivalence, one can take a pretained LLM (trained in a standard causal fashion) and further fine-tune it under our non-Markovian diffusion objective. By allowing the model to condition on the previous timesteps in the diffusion chain \(\mathbf{x}_{t+1:T}\), we equip it with iterative denoising capabilities beyond standard next-token prediction. This straightforward adaptation:
\begin{itemize}
    \item \textbf{Expands Generation Modes:} The LLM can perform text infilling or partial prompting from arbitrary positions, rather than strictly appending text at the end, as shown in Figure~\ref{fig:sample_text_infilling}.
    \item \textbf{Leverages Pretraining Knowledge:} Since large LLMs are already trained on vast corpora, fine-tuning under our discrete diffusion objective benefits from a strong initialization and broad linguistic knowledge.
    \item \textbf{No Architectural Changes:} We only replace the original (causal) loss with a non-Markovian diffusion loss and provide noise-corrupted sequences as training data, preserving the underlying transformer structure.
\end{itemize}

%TODO{Talk about some related experiments}

\subsection{Inference Bottleneck of Naive \method{}}
Naive \method{} inference can be slower than standard discrete diffusion, typically requiring $\mathcal{O}(L \times T)$ function evaluations for a sequence of length $L$ over $T$ timesteps. However, by leveraging the unique properties of causal language modeling, we propose a \textit{semi-speculative decoding} strategy that substantially reduces inference time while maintaining generation quality.

\paragraph{Semi-Speculative Decoding.}

\begin{algorithm}[h]
    \caption{Semi-Speculative Decoding for Non-Markovian Discrete Diffusion}
    \label{alg:semi_spec_decoding}
    \begin{algorithmic}[1]
    \State \textbf{Input:}
        model parameters $\theta$, prior distribution $q(\mathbf{x}_T)$
    \State \textbf{Output:}
        Sampled data $\mathbf{x}_0$
    \State Initialize $\mathbf{x}_T \sim q(\mathbf{x}_T)$ \Comment{Noisy data at final timestep}
    \For{$t = T \textbf{ down to } 1$}
        \State $i \gets 0$
        % \State \textbf{// Verification in parallel}
        \If{$\widetilde{\mathbf{x}}_0^{\text{prev}} \text{ is available}$}
        \Comment{If previous step is available, use it as drafted tokens}
        \State $i \gets \textsc{Verify} (p_\theta, \mathbf{x}_{t:T}, \widetilde{\mathbf{x}}_0^{\text{prev}})$ 
        \Comment{Predict drafted tokens's probability in parallel, verify to find the first rejection on index $i$}
        \State $\widetilde{\mathbf{x}}_0^{i, \text{prev}} \gets \textsc{Correct} (p_\theta, \mathbf{x}_{t:T}, \widetilde{\mathbf{x}}_0^{\text{prev}})$
        \Comment{Correct the first rejection on index $i$ based on criterion}
        \State $\widetilde{\mathbf{x}}_0^{0:i} \gets \widetilde{\mathbf{x}}_0^{0:i, \text{prev}}$
        \EndIf
    

        \While{$i < L$} 
            \State $\widetilde{\mathbf{x}}_0^{i+1} \gets p_\theta\bigl(\mathbf{x}_0 \mid \mathbf{x}_{t:T}, \widetilde{\mathbf{x}}_0^{0:i}\bigr)$
            \State $ i \gets i + 1$
        \EndWhile
        
        \State $\mathbf{x}_{t-1} \sim q\bigl(\mathbf{x}_{t-1} \mid \widetilde{\mathbf{x}}_0\bigr)$

        \State $\widetilde{\mathbf{x}}_0^{\text{prev}} \gets \widetilde{\mathbf{x}}_0$
    \EndFor
    \State \textbf{return} $\mathbf{x}_0$ \Comment{Final predicted clean data}
    \end{algorithmic}
    \end{algorithm}

Although causal language models generate tokens sequentially, they can verify the probabilities of any \textit{pre-drafted} sequence in parallel. Notably, \method{} shares the same denoising target \(\mathbf{x}_0\) across all timesteps. This observation suggests a natural procedure: reuse the previous timestep's predictions \(\widetilde{\mathbf{x}}_0^{\text {prev }}\) as a \textit{draft} for the current timestep (see Algorithm~\ref{alg:semi_spec_decoding}). The model then \textit{verifies} these drafted tokens in parallel, accepting those that meet a specified confidence threshold (e.g. high probability).

This approach closely resembles speculative decoding~\citep{leviathan2023fast}, with one key difference: we do not rely on a separate, smaller model to propose the draft sequence. Instead, \method{}’s own predictions from the preceding timestep serve as the draft. Like speculative decoding, various verification and correction strategies (e.g. greedy, nucleus sampling) can be employed, ensuring either a comparable or identical sampling distribution while significantly reducing the total number of sampling steps.

\section{Protein tasks}
\label{sec:protein}

Our model's capabilities with respect to proteins are assessed through several distinct types of tasks:

\begin{enumerate}
\item Unconditioned protein generation: The model generates protein sequences from scratch without any specific conditions or prompts.
\item Text-guided protein generation: This task involves guiding the model to generate protein sequences based on given natural language descriptions.
% \item Protein Understanding - Classification and Regression: The model utilizes its generative abilities to perform understanding tasks, specifically classification and regression.
\item Antibody design: The model designs the Complementary-Determining Region H3 (CDR-H3) of antibodies to effectively bind to target antigens.
\item Protein description generation: This task focuses on generating explanations or uncovering properties and functions of protein sequences, articulating them in natural language.
\end{enumerate}

\subsection{Unconditioned generation}\label{sec_prot_generation}

The first capability of the model is generating protein sequences from scratch freely, prompted by the start token for proteins only, i.e., \text{$\langle$protein$\rangle$}. 
However, since there is no golden standard for evaluating proteins when no conditions are specified, it is difficult to measure the generation results. We focus on foldability, measured by pLDDT score \cite{Mariani2013-av}, as well as lengths and diversity of the sequences, for the valid sequences.

\begin{table}[!h]
\centering
\begin{tabular}{lccc}
\toprule
Model & Avg Length & Diversity & AVG pLDDT \\
\midrule
Mixtral 8x7b & 53.3 & 0.906 & 69.9 \\
GPT-4 & 45.7 & 0.816 & 65.1 \\
\midrule
\ourM{} (1B) & 288.3 & 0.985 & 69.8 \\
\ourM{} (8B) & 284.5 & 0.973 & 71.8 \\
\ourM{} (8x7B) & 318.4 & 0.989 & 75.9 \\
\bottomrule
\end{tabular}
\caption{Protein Sequence Generation Comparison. The average length of natural proteins (calculated from a subset of proteins randomly sampled from UR50) is about 311. The diversity was calculated by the number of clusters with 50\% sequence identity divided by the total generated sequence count. The pLDDT scores were calculated by OmegaFold \cite{omegafold} on the generated sequences with length less than 100 for a fair comparison. The length distribution is left in Figure \ref{fig:protein:unconditioned_generation_sequence_length}. }
\label{tab:protein:unconditioned_generation}
\end{table}

As shown in Table \ref{tab:protein:unconditioned_generation}, \ourM{} consistently outperform Mixtral 8x7b and GPT-4 in terms of average sequence length, diversity, and average pLDDT score. The \ourM{} (8x7B) model achieves the best performance across all metrics, with an average length of 318.4, diversity of 0.989, and average pLDDT score of 75.9. ProLLAMA~\cite{lv2024prollamaproteinlanguagemodel} a fine-tuned LLM for protein. It generates proteins without explicitly defined constraints on length, achieving a pLDDT score of 66.5. In contrast, our approach, which does not impose length constraints, results in pLDDT scores of 69.8 and 78.1 for the 8B and 8x7B models, respectively, demonstrating our significant advancement in this area.



\subsection{Text-guided protein generation}\label{sec:text_guided_protein_design}

% For text-guided protein generation, we evaluated our models' ability to generate proteins with specific properties based on natural language prompts. We have selected solubility and stability for this assessment and leave more properties as future work. In terms of stability, the models are tasked with generating stable protein sequences. Regarding solubility, given the prevalence of both soluble and insoluble proteins in natural sequences, we have instructed \ourM{} to generate both types of sequences. Exemplary prompts are depicted in Figure \ref{fig:protein:conditioned_prompts} while a comprehensive list of prompts can be found in Figure \ref{fig:protein:conditioned_prompts_full}. 

For text-guided protein generation, we evaluated our models' ability to generate proteins with specific properties based on natural language prompts. In this study, we focused on two key properties: solubility and stability, leaving the exploration of additional properties for future work.
%
For stability, the models were tasked with generating protein sequences that exhibit stable properties. Regarding solubility, since both soluble and insoluble proteins are common in natural sequences, we instructed \ourM{} to generate sequences of both types. 
% To build the dataset used by our models, we adapted prompt templates with data from the PEER benchmark~\cite{xu2022peer}, utilizing only the training data during instruction tuning. 
Sample prompts are shown below, and a full list of prompts can be found in Figure~\ref{fig:protein:conditioned_prompts_full}.

\begin{example} 
$\rhd$ An example prompt for ``stable protein generation''\\
\texttt{I require a stable protein sequence, kindly generate one.}\\
$\rhd$ An example prompt for ``soluble protein generation''\\
\texttt{Generate a soluble protein sequence.}\\
$\rhd$ An example prompt for ``insoluble protein generation''\\
\texttt{Produce a protein sequence that is not soluble.}
\end{example} 

To evaluate the stability and solubility of a generated protein sequence, we utilized two specialist models fine-tuned from the protein foundation model, SFM-Protein~\cite{he2024sfm}, as oracle models. One model was used for stability classification, while the other was used for solubility classification. %  (see Section \TODO{xxxx} for details)
The oracle models provide probabilities that suggest the likelihood of the sequence possessing the desired property. To verify the efficiency of our model against random sampling, we have also chosen a subset of 1000 natural protein sequences from the UR50 dataset and assessed them using the same oracle models.

% \begin{figure}[h]  
%     \centering  
%     \begin{mdframed}[backgroundcolor=white, linecolor=black, linewidth=1pt]  
%     \textbf{Instruction:}
%     \textit{Produce a protein sequence that is soluble.} \\
%     \textbf{Response:} \\
%     \\
%     \underline{\text{$\langle$protein$\rangle$}\text{MSLSELSLQL \ldots KGVLVNK}\text{$\langle$/protein$\rangle$}}
%     \end{mdframed}  
%     \caption{Templates for conditioned generation} \label{fig:protein:conditioned_example}
% \end{figure} 

\begin{figure}[!htbp]
\centering
\subfigure[\ourM{} (1B)]{
% \includegraphics[width=0.33\linewidth]{figures/SFM-Seq_1B.stability.pdf}
\includegraphics[width=0.33\linewidth]{figures/NatureLM_1B.stability.pdf}
}%
\subfigure[\ourM{} (8B)]{
% \includegraphics[width=0.33\linewidth]{figures/SFM-Seq_8B.stability.pdf}
\includegraphics[width=0.33\linewidth]{figures/NatureLM_8B.stability.pdf}
}%
\subfigure[\ourM{} (8x7B)]{
% \includegraphics[width=0.33\linewidth]{figures/SFM-Seq_8x7B.stability.pdf}
\includegraphics[width=0.33\linewidth]{figures/NatureLM_8x7B.stability.pdf}
}
\caption{Stability score distribution of the generated sequences.}
\label{fig:protein:conditioned_generation_stability}
\end{figure}

\begin{table}[!h]
\centering
\begin{tabular}{ccc}
\toprule
Source & AVG Prediction & Data Ratio (Score $>0.5$) \\
\midrule
Natural & 0.552 & 0.704 \\
\ourM{} (1B) & 0.559 & 0.644 \\
\ourM{} (8B) & 0.619 & 0.757 \\
\ourM{} (8x7B) & 0.655 & 0.812 \\
\bottomrule
\end{tabular}
\caption{Stability score ratio of the generated sequences.}
\label{tab:protein:conditioned_generation_stability}
\end{table}

\begin{figure}[!htbp]
\centering
\subfigure[\ourM{} (1B)]{
% \includegraphics[width=0.33\linewidth]{figures/SFM-Seq_1B.solubility.pdf}
\includegraphics[width=0.33\linewidth]{figures/NatureLM_1B.solubility.pdf}
}%
\subfigure[\ourM{} (8B)]{
% \includegraphics[width=0.33\linewidth]{figures/SFM-Seq_8B.solubility.pdf}
\includegraphics[width=0.33\linewidth]{figures/NatureLM_8B.solubility.pdf}
}%
\subfigure[\ourM{} (8x7B)]{
% \includegraphics[width=0.33\linewidth]{figures/SFM-Seq_8x7B.solubility.pdf}
\includegraphics[width=0.33\linewidth]{figures/NatureLM_8x7B.solubility.pdf}
}
\caption{Solubility score distribution of the generated sequences.}
\label{fig:protein:conditioned_generation_solubility}
\end{figure}

% \begin{table}[!h]
% \centering
% \begin{tabular}{c|c|c|c}
% \hline
% Source & AVG Prediction & Score $>0.2$ & Score $>0.5$ \\
% \hline
% Natural & 0.221 & 0.452 & 0.054 \\
% SFM-Seq (1B) [Insoluble] & 0.203 & 0.409 & 0.131 \\
% SFM-Seq (8B) [Insoluble] & 0.229 & 0.479 & 0.151 \\
% SFM-Seq (8x7B) [Insoluble] & 0.207 & 0.436 & 0.112 \\
% SFM-Seq (1B) [Soluble] & 0.467 & 0.845 & 0.372 \\
% SFM-Seq (8B) [Soluble] & 0.539 & 0.900 & 0.493 \\
% SFM-Seq (8x7B) [Soluble] & 0.515 & 0.868 & 0.461 \\
% \hline
% \end{tabular}
% \caption{Solubility score ratio of the generated sequences}
% \label{tab:protein:conditioned_generation_stability}
% \end{table}

Figures \ref{fig:protein:conditioned_generation_stability} and \ref{fig:protein:conditioned_generation_solubility} show the distributions of stability and solubility scores for the generated sequences, respectively. The \ourM{} models demonstrate controlled distribution shift in generating proteins with desired properties compared to the natural sequences. 
%
In the task of generating more stable proteins, as shown in Figure~\ref{fig:protein:conditioned_generation_stability}, a clear trend emerges: as the model size increases, the proportion of sequences classified as stable grows, with a pronounced peak in the \ourM{} (8x7B) results. The quantified data, summarized in Table~\ref{tab:protein:conditioned_generation_stability}, further supports this observation. All three models produce proteins that are more stable than natural sequences based on average stability scores. Additionally, two of the models outperform natural proteins in terms of the number of sequences that exceed a stability threshold of 0.5.
%
For the solubility condition, Figure~\ref{fig:protein:conditioned_generation_solubility} reveals a similar trend. As the model size increases, the separation between the distributions of soluble and insoluble scores becomes more distinct, with less overlap. 
% Given that the oracle model tends to classify proteins as insoluble, it is notable that the SFM-Seq (8x7B), when conditioned on solubility, generates sequences with solubility scores that are consistently higher than the median solubility score of natural sequences.

\subsection{Antigen-binding CDR-H3 design}

The task of antigen-binding CDR-H3 design focuses on constructing the Complementary-Determining Region H3 (CDR-H3) of an antibody to bind effectively to a target antigen. We employed the RAbD benchmark dataset~\cite{adolf2018rabd}, comprising 60 antibody-antigen complexes. The example is shown below:

\begin{mdframed}
\noindent\textbf{Instruction: }\\
\texttt{Using antigen} \pro{}TQVCTGTDMKLR$\cdots$GESSEDCQS\epro{} \texttt{and antibody frameworks} \ant{}IVLTQTPS$\cdots$LAVYYC\eant{} \texttt{and} \ant{}FGGGTRLEIEVQ\eant{}, \texttt{create the CDR3 regions.}\\
\textbf{Response: }\\ 
\ant{}QQYSNYPWT\eant{}
\end{mdframed}  



The generation quality is evaluated by the Amino Acid Recovery (AAR) scores for the CDR-H3 design task. We use $r$ and $\hat{r}$ to represent the reference and generated sequences respectively, while $L(r)$ and $L(\hat{r})$ denote the number of amino acids in $r$ and $\hat{r}$. The $i$-th residue in the two sequences is denoted by $r_i$ and $\hat{r}_i$. The AAR is defined as follows:
\begin{equation}
{\rm AAR}(r,\hat{r}) = \frac{1}{L(r)}\sum_{i=1}^{L(r)}\mathbb{I}(r_i = \hat{r}_i). 
\end{equation}
In case $L(\hat{r})>L(r)$, only the first $L(r)$ elements are verified. If $L(\hat{r})<L(r)$, we assign $\mathbb{I}(r_i = \hat{r}_i)=0$ for $i>L(\hat{r})$.


\begin{table}[!h]
\centering
\begin{tabular}{lc}
\toprule
Method & AAR ($\uparrow$) \\
\midrule
GPT-4 & 0.312 \\
RefineGNN~\cite{jin2021refinegnn} & 0.298 \\
HSRN~\cite{jin2022hsrn} & 0.327 \\
MEAN~\cite{kong2022mean} & 0.368 \\
ABGNN~\cite{gao2023abgnn} & 0.396 \\
% SFM-Protein \cite{he2024sfm} & 0.549\\
\midrule
Llama 3 (8B) & 0.275 \\
\ourM{} (1B) & 0.273 \\
\ourM{} (8B) & 0.368 \\
\ourM{} (8x7B) & 0.376 \\
\bottomrule
\end{tabular}
\caption{AAR of the CDR-H3 design. Please note that the \ourM{} models utilize sequence-only input for this task, whereas the baseline models may incorporate additional information, such as structural data.}
\label{tab:protein:antibody:cdr3GenGiven_antigen}
\end{table}

Table \ref{tab:protein:antibody:cdr3GenGiven_antigen} presents the Amino Acid Recovery (AAR) scores for the CDR-H3 design task. As the model size of \ourM{} increase, the AAR gradually increases. The \ourM{} (8x7B) model achieves competitive performance with an AAR of 0.376, outperforming several specialized GNN-based models. While SFM-protein, a BERT-like model trained on protein sequences, holds the top performance, our results demonstrate the potential of \ourM{} in CDR-H3 design, particularly as the model scales and undergoes further refinement.

\subsection{Protein description generation}\label{sec:protein_to_desc}
Despite the rapid discovery of natural protein sequences facilitated by advanced sequencing techniques, the functions of many of these proteins remain largely unknown. This knowledge gap restricts our ability to exploit these proteins for engineering and therapeutic purposes. In this study, we explored the annotation generation capabilities of the \ourM{} series.

To achieve this, we compiled pairs of protein sequences and their human-readable annotations from various species, sourced from the NCBI database. We divided the dataset temporally: historical data were utilized for training the \ourM{} models, while annotation data from the most recent four months were reserved for testing. Model performance was evaluated using Rouge-L scores. As shown in Table \ref{tab:protein:ncbi_description}, \ourM{} models consistently outperformed Llama 3 8B in Rouge-L scores, with performance differences widening as model size increased. Notably, the \ourM{} (8x7B) model achieved the highest score of 0.585. A detailed analysis presented in Figure \ref{fig:protein:protein_understanding} revealed that the \ourM{} (8x7B) model not only generates protein annotations with greater accuracy but also successfully identifies orthologues and functions of proteins, while \ourM{} (8B) is also able to generate reasonable results in many cases.



\begin{table}[!htbp]
\centering
\begin{tabular}{lc}
\toprule
Model Setting & Rouge-L\\
\midrule
Fine-tuned Llama 3 (8B) & 0.324 \\
\ourM{} (1B) & 0.548 \\
\ourM{} (8B) & 0.572 \\
\ourM{} (8x7B) & 0.585 \\
\bottomrule
\end{tabular}
\caption{Performance of protein description generation, measured by Rouge-L. Llama 3 (8B) serves as a baseline, which is directly fine-tuned on the data collection described in Section \ref{sec:supervised_ft_data}. More details about this baseline in Section \ref{sec:ablation_study}.}
\label{tab:protein:ncbi_description}
\end{table}

\begin{figure}[!htbp]
\centering
\includegraphics[trim=3cm 2cm 7cm 1cm, clip, width=0.85\linewidth]{figures/proteinAnnotation_v3.pdf}
\caption{Generated descriptions for different proteins.}
\label{fig:protein:protein_understanding}
\end{figure}


In conclusion, \ourM{} demonstrates strong performance across a wide range of protein-related tasks, from unconditioned generation to specific design tasks like CDR-H3 design. The scalability of our approach is evident, with larger models consistently outperforming smaller versions and often achieving state-of-the-art results.


\subsection{Heme-binding protein design driven by text and SMILES}\label{sec:heme_case_study}
\ourM{} is capable of designing proteins to meet various requirements. In this case study, we demonstrate how to use \ourM{} to design a protein that binds to heme cofactor.

Heme is an essential molecule, playing critical roles in oxygen transport and various enzymatic processes. Designing a protein that can bind to heme can contribute to advancements in therapeutic development, biochemical research, and synthetic biology.

The protein design practice is outlined as the following 4 steps:

\noindent{\textbf{Step 1: protein generation}}. We prompt \ourM{} to generate proteins using two approaches:
\begin{enumerate}
\item Text-based protein design, where we provide a text instruction (here in English);
\item Small-molecule-based protein design, where we use the SMILES representation of heme as part of the input instruction.
\end{enumerate}

Examples of these prompts and their corresponding responses are shown below:

\begin{example}
{{
\small
\noindent$\rhd$ {Generate proteins based on text} (Section \ref{sec:text_guided_protein_design})

\noindent\textbf{Instruction:} \texttt{Design a heme-binding protein sequence.} 

\noindent\textbf{Response:} \pro{}MSAAEGAVVFSEEKEALVLK$\cdots$\epro{}


\noindent$\rhd$ { Generate proteins based on the SMILES of the heme}

\noindent\textbf{Instruction:} \texttt{Generate a protein that can bind to} \\
\mol{}[Fe+2].C=CC1=C(C)c2cc3[n-]c(cc4[n-]c(cc5nc(cc1n2)C(C)$\cdots$\emol{}

\noindent\textbf{Response:} \pro{}ETIGKRVFVHYCHGCHSQNALGI$\cdots$\epro{}
}}
\end{example}

\noindent{\textbf{Step 2: description generation}}. For each generated protein, we utilize the protein-to-text functionality in \ourM{} (Section \ref{sec:protein_to_desc}), to obtain a description of the properties and potential functions of the generated protein.

\noindent{\textbf{Step 3: screen proteins through keyword matching}}. In this step, we use GPT-4o to generate a keyword list, called \texttt{HemeList}, containing characteristics associated with heme-binding proteins. For every protein description generated in Step 2, we check whether it contains keywords from \texttt{HemeList}. If a description matches these criteria, the corresponding protein is added to a list called \texttt{HemeProtein}.

\noindent{\textbf{Step 4: structure generation and validation}}. For each protein in \texttt{HemeProtein}, we use Protenix~\cite{Protenix2025} to predict the complex structure of the generated proteins bound to heme. The predicted structures are then inspected to ensure that the proteins can form the critical interaction with heme for stable binding.

\begin{figure}[!htpb]
    \centering
    \includegraphics[width=0.9\linewidth]{figures/heme_showCase.pdf}
    \caption{Two examples of proteins with plausibility of binding to heme. The yellow models represent the generated protein structures, while the blue models correspond to the reference structures retrieved using the built-in ``blast protein'' function in ChimeraX \cite{chimerax2023}. In each model, the heme binding region is highlighted by showing the nearby residues in stick representations. We use the protein-to-text functionality of \ourM{} to generate  functional annotations for these proteins, and the original outputs are provided here: (left protein) ``Heme-binding protein''; (right protein) ``Transfers electrons from cytochrome c551 to cytochrome oxidase; C-type cytochrome; Part of the cbb3-type cytochrome c oxidase complex.''}
    \label{fig:heme_bind_prot}
\end{figure}

In the text-based design case study (Fig.\ref{fig:heme_bind_prot} left), two histidine residues are positioned in close proximity to the iron located in the center of heme, enabling the formation of coordinated bonding interactions with the heme group. Similarly, in the SMILES-based design, the algorithm can output proteins with binding motifs similar to those generated in the text-based example (Fig. \ref{fig:SI:moreHemeCases}). However, as shown in Fig. \ref{fig:heme_bind_prot} (right), we show a representative case where a methionine and histidine residue are observed to interact closely with the iron ion (see Fig. \ref{fig:SI:prot_hem_hec} for more discussion on this case). These residues effectively coordinate the metal ion through their respective side chains, demonstrating alternative structural strategies for heme binding. Furthermore, the designed protein sequences differ significantly from those present in the database, indicating that our approach can generate novel sequences with distinct structural properties. Collectively, these results demonstrate the effectiveness of \ourM{} in designing functional heme-binding proteins with diverse and novel structural features. We also compare the apo and holo structures of the generated proteins in Fig. \ref{fig:compare_apo_holo}, which shows that the key residues involved in heme binding, such as histidine and methionine, occupy similar positions in both structures.












\paragraph{Further Acceleration.}
Beyond semi-speculative decoding, \method{} also benefits from \textit{key-value caching (KV-Cache)}, a hallmark of causal generation that is unavailable in bidirectional discrete diffusion models. Additionally, the \(\textbf{x}_0\)-parameterization enables efficient timestep skipping, further accelerating inference. We believe these techniques only begin to illustrate the potential for more advanced optimization and scaling in \method{}, which we leave for future exploration.




% Analyze complexity of inference of naive CaDDI vs standard discrete diffusion vs standard LLM?


% \rex{complexity analysis}