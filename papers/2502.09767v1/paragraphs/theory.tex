\section{Non-Markovian Inverses of Markovian Processes}
\label{sec:non-markovian-inverse}

A core insight underlying our approach is that even if the \emph{forward} process is Markovian, its \emph{inverse} often requires information from the entire trajectory, thereby breaking the Markov property.
\paragraph{Physical Systems.}
For instance, in physics, many systems exhibit this property. \textit{Langevin Dynamics:} Although the forward motion of a Brownian particle (with velocity and position) can be Markovian in the full state space, attempts to reverse the position-only dynamics often require the history of the system to account for friction or random kicks \citep{gardiner2009stochastic, vankampen1992stochastic}. \textit{Quantum Processes:} Tracing out environmental degrees of freedom can yield a Markovian forward evolution, but reconstructing the entire global state upon reversal introduces non-Markovian memory effects \citep{Rivas_2014}.

\paragraph{Diffusion Models in Machine Learning.}
Denoising Diffusion Probabilistic Models (DDPMs) \citep{ho2020denoisingdiffusionprobabilisticmodels} define a simple Markovian chain that adds Gaussian noise, yet the learned \emph{reverse} process leverages global information through a neural network, effectively introducing non-Markovian dependencies. Similarly, Volterra Flow Matching \citep{he2024calmflowvolterraflowmatching} and Score-Based Generative Models \citep{song2021scorebasedgenerativemodelingstochastic} show that reversing a Markovian SDE can demand memory-like effects or knowledge of the full data distribution.

\paragraph{Discrete Diffusion and Our Approach.}
In discrete diffusion, the same principle applies: one can corrupt tokens step-by-step with a Markovian noising schedule, but reversing this corruption accurately often necessitates conditioning on multiple future (noisy) states rather than solely on $x_t$. As we show in Section~\ref{sec:method}, our \method{} framework leverages this insight by explicitly building a non-Markovian reverse pass over the entire corrupted trajectory, leading to more robust and flexible sequence generation.

See Appendix~\ref{apdx:non_mark_details} for further references and a broader theoretical discussion of non-Markovian inverses.


