\section{Methodology}
\label{sec:method}
\section{Methodology}
In this section, we outline the key research questions driving this study, followed by a detailed description of the methodology used to design and conduct the survey.
\subsection{Research Questions}
\begin{enumerate}
    \item[\textbf{RQ1:}] How do developers allocate their time during a typical workweek, and how does this compare to their perception of an \textbf{ideal workweek?}
    \item[\textbf{RQ2:}] How are developer's satisfaction and productivity affected by \textbf{deviations} from their ideal workweek?
     \item[\textbf{RQ3:}] For which tasks do developers prefer using \textbf{AI tools}, and how does the frequency of AI tool usage \textbf{influence} their satisfaction and productivity?
\end{enumerate}

\subsection{Survey Design}
% Describe how the survey was conducted, survey structure, sample size, which activities were selected and how, incentives, etc. 

To gain insights into the types of activities developers engage in during a typical work week, we conducted a series of exploratory interviews with 12 randomly selected participants. These semi-structured interviews provided a qualitative foundation, allowing us to iteratively develop a comprehensive list of higher-level activities that reflect both ideal and actual workweek allocations. The findings from these interviews were instrumental in refining our survey questions and design.

% - When was it distributed
% - How many people were invited
% - how was the survey advertised
% - incentive provided to participants
% - how many responses received (with response rates)
% - Board of ethics description \& instruments
% - Describe the main questions asked in the survey

The survey was distributed in \textcolor{blue}{May 2024} to software engineers working in Microsoft teams across India and the United States. A total of 6000 developers were invited to participate via email. Framed as a study aimed at boosting developer productivity by understanding how they allocate their time in a workday, the survey received 510 complete responses (responses rate of 8.5\%). After finishing the survey, the participants could enter a sweepstake to win one out of ten \$50 Amazon.com Gift Cards.
\textcolor{blue}{description of ethics}.

The main questions in the survey were as follows:
\begin{enumerate}
    \item Their roles and years of experience in the industry/team
    \item The hours spent on various activities in their typical workweek
    \item Ideally, the percentage of time they would want to allocate to each activity in a workweek
    \item How productive and satisfied were they by their past workweek
    \item Activities they find most cognitively challenging
    \item How often do they use AI tools to assist in their daily activities
    \item Two open-ended questions about the activities they would want to automate using AI tools, and advice for new hires to boost their productivity and satisfaction levels 
\end{enumerate}



\subsection{Data Analysis \& Exploration}
% Here, we could start with discussing the survey group:
% - demographic observations
% - distribution of participants (based on the years experience in the industry/team), 

From the exploratory interviews, we identified sixteen key activities, which were subsequently used to quantify the developers' time allocation across their work week. 

\subsection{Limitations}














\subsection{Algorithm Overview}
\label{sec:methodology:overview}
\Cref{fig:Methodology} provides an overview of \sys's workflow. During prefill,  \sys performs K-means clustering on key vectors to group similar tokens. During decode, \sys ranks tokens based on the dot product between cluster centroids and the current query vector. \sys then models the distribution of \as{} with a fitted curve and determines the tokens to meet the desired cumulative \as{} threshold. After token selection, \sys handles the Group Query Attention (GQA) and then performs the attention using FlashInfer~\cite{ye2025flashinfer}.


\subsection{Clustering}
To organize tokens for efficient sorting, \sys performs K-means clustering on the key vectors for each head in every layer during the prefill phase. We empirically choose the average cluster size to be $32$ to balance accuracy and efficiency. Clustering begins by randomly sampling $\text{\textit{SeqLen}} / \text{\textit{Average cluster size}}$ data points as the initial cluster centroids.\footnote{Note that neither multiple initializations nor K-Means ++ initialization drastically improves the clustering quality, and in fact leads to high-performance overhead.}In each iteration, the distance between K-vectors and centroids is computed and the token will be assigned to the nearest cluster. After the assignment step, the centroids are updated as the mean of the key vectors assigned to each cluster. This process repeats until convergence or until a maximum of 10 iterations is reached\footnote{More iterations do not improve the quality of clustering.}. 

\subsection{Querying}

Once the tokens are organized into clusters, \sys identifies critical clusters for a given query vector $Q$ in the decode phase. The criticality of each cluster is determined by the dot product between $Q$ and each cluster centroid\footnote{Compared to distance, dot product directly relates to the attention score, which is more accurate.}. This process produces a sequence of clusters sorted by the criticality, from which we can derive a partially sorted token list. 

\subsection{Fitting Attention Score Distribution}

The next step of \sys is to determine the token budget required to meet the cumulative \as{}. \sys models the distribution of the exponential values of the dot products ($\exp(\frac{QK^\top}{\sqrt{d}})$) for each token using a lightweight function \( y = \frac{a}{x} + b \), where \( a \) and \( b \) are parameters to be determined and \( x \) is the position in the sorted list of tokens.
To estimate these parameters, we select two segments of the tokens in the middle of the curve (e.g., 10\% and 60\% of all the tokens), and calculate the average of tokens within each segment (as labeled in \fig{fig:distribution_curve}). Using these two data points, we can solve for \( a \) and \( b \), which provides an estimation of \as{} for all tokens.

However, initial tokens are often outliers and cannot be accurately described by the curve. Moreover, these tokens feature high \as{}, and thus a bad estimation would cause high deviations of estimated cumulative \as{} which affects the accuracy of \sys.
Luckily, we observed that this only happens within 1-2\% of total tokens. Therefore, \sys directly calculates the exponential values of the dot products for these tokens. A detailed description of the Distribution Fitting stage is provided in \Cref{alg:token_selection}.

\subsection{Taking Union for Group Query Attention models}


Modern models use Grouped Query Attention (GQA) to reduce the KV cache size \cite{llama_3}, where multiple query heads share a single KV head. However, loading KV heads separately for each query head is inefficient. To optimize this, query heads within the same group are batched. A challenge arises when using sparse attention, as different query heads may select to attend to different KV tokens. Finding the minimal set of KV tokens that satisfies the cumulative attention scores (\as{}) across all query heads is NP-hard. To address this, \sys simplifies the problem by taking the union of selected tokens across all query heads and loading them at once, ensuring that each head retains the KV tokens it requires to perform attention while reducing repetitive loading.

\subsection{Attention on Selected Tokens}
Finally, \sys performs actual attention for selected tokens using FlashInfer~\cite{ye2025flashinfer}.
Notably, variations in sparsity across different heads cause an imbalanced attention workload. Traditional implementations primarily address imbalances across varying request lengths but struggle to handle head-level imbalance efficiently. To address this, \sys divides each request into subrequests. Each subrequest processes a KV head and its corresponding Query head, with sequence length determined by the tokens selected for each KV head. This transforms head-level imbalance back into sequence-level imbalance, which Flashinfer handles efficiently.
