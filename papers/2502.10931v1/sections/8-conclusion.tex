\section{Conclusion} \label{sec:conclusion}
% We proposed D-CIPHER, a multi-agent framework that leverages dynamic collaboration to address the complex challenges inherent in Capture the Flag competitions. By integrating specialized agents with distinct roles, D-CIPHER fosters streamlined communication and collaboration, mimicking real-world team dynamics in CTF scenarios. Our key innovation including Planner-Executor architecture for task planning and heuristic inspired Prompter mechanism which delivers robust feedback loops and efficient scenario exploration regarding these CTF challenges, driving significant improvements in smoothness of problem-solving process. Our evaluation across diverse benchmark and large language models from different platform for both commercial and open source models, which highlights the efficacy of multi-agent collaboration in addressing CTF challenges. We conducted a comprehensive case study which demonstrated how D-CIPHER's adaptive strategies and mistake mitigation mechanisms outperformed traditional single-agent baselines with solely prompt and feedback enhancement. The framework's ability collaboration facilitate real-time strategy adjustments and error mitigation through inter-agent collaboration mirrors the successful dynamics observed in human CTF teams, which suggests muli-agent system may continue to evolve toward more team-like problem-solving approaches. D-CIPHER highlights the potential of multi-agent systems in cybersecurity, bridging single-agent limitations and real-world collaboration. Future research could enhance agent specializations, feedback mechanisms, and collaboration protocols, extending these advancements to other domains. This work lays a foundation for robust, adaptable automated systems, addressing complex challenges and advancing autonomous, collaborative intelligence.
% \minghao{TODO: Refine based on the results section}

We present D-CIPHER, an LLM multi-agent framework that autonomously solves CTF challenges. We propose two key innovations: first is the Planner-Executor system with the Planner agent to generate a plan and manage overall problem-solving, along with multiple Executor agents that focus on their assigned tasks; and, second is the the Auto-prompter agent that dynamically generates a prompt based on initial exploration to solve the challenge.
We introduce novel mechanisms to facilitate interaction between agents via function calling. 
By incorporating dynamic interactions and feedback among multiple agents, D-CIPHER mirrors the team dynamics observed in real-world CTF competitions.
With these innovations, D-CIPHER gets higher performance over state-of-the-art on three benchmarks: 22\% on NYU CTF Bench, 22.5\% on Cybench, and 44\% on HackTheBox.


D-CIPHER has limitations which merit consideration and show potential avenues for improvement.
While we notice improved focus and efficiency of the Executors, there is no direct interaction between each Executor and information exchange is bottlenecked via the Planner, which may causes failures as seen in Section~\ref{sec:casestudy_enigma}.
Despite the present limitation, the framework allow versatility to configure different types of multi-agent systems with different interactions. One such extension of D-CIPHER can incorporate interactions between Executors operating simultaneously to alleviate the information bottleneck.
Another limitation is that errors in the initial exploration phase of the Auto-prompter have a severe impact on the generated prompt, which inevitably biases the Planner in the wrong direction, seen clearly in Section~\ref{sec:autoprompter_casestudy}.
The high dependence on the Auto-prompter can be reduced by combining the generated prompt with hard-coded human-written directions and tips.
D-CIPHER also shows improved cost efficiency over existing single-agent systems, despite running multiple agents. This demonstrates the potential of multi-agent systems in cost-constrained deployments. While the current results with combination of stronger and weaker models are not good, they show promise and open new avenues for building cost efficient multi-agent systems for autonomous problem solving.
% These are the proposed future work of D-CIPHER.

% \meet{re-iterate in conclusion and future work} The framework and the special interaction functions allow versatility to configure different types of multi-agent systems.
% For example, a simpler system without the Planner can have an Auto-prompter generate a prompt and a single Executor solve the challenge end-to-end.
% New interactions can easily be defined which may allow, for instance, two Executors to run simultaneously and coordinate to solve the challenge.
% These configurations demonstrate the framework's flexibility to build diverse systems for complex problems.

% We present D-CIPHER, a multi-agent framework that leverages the principles of dynamic collaboration to address the intricate challenges of Capture the Flag (CTF) competitions. By integrating specialized agents with distinct roles, D-CIPHER promotes seamless communication and teamwork that can closely emulate real-world dynamics in CTF scenarios. Central to our approach are two key approaches: the Planner-Executor architecture, which enables efficient task planning and execution, and the Dynamic AutoPrompting mechanism, which provides robust feedback loops and facilitates thorough scenario exploration. These innovations significantly improve the fluidity and effectiveness of the problem-solving process. Our evaluation spanned diverse CTF benchmarks and large language models from different platforms to assess the efficacy of multi-agent collaboration on Offensive Security.
% A comprehensive case study highlighted how D-CIPHER’s adaptive strategies and mistake mitigation mechanisms consistently outperformed traditional single-agent baselines that rely solely on prompt engineering and feedback enhancements. 
% By enabling real-time strategy adjustments and error resolution through inter-agent collaboration, D-CIPHER mirrors the success dynamics observed in human CTF teams.
% Our test results show that it is capable of outputting state-of-the-art results on NYU CTF Bench and Cybench, with \textbf{22\%} and \textbf{22.5\%} success rate respectively. This reflects the potential of multi-agent systems to evolve into more team-like and adaptable problem-solving approaches, paving the way for collaborative, real-world applications. 
% D-CIPHER underscores the transformative potential of multi-agent systems in cybersecurity, bridging the limitations of single-agent approaches and paving the way for more collaborative, real-world applications.
% Future research can further improve agent specializations, refine feedback mechanisms, and optimize collaboration protocols. In doing so, D-CIPHER could potentially extend these improvements to domains beyond CTF competitions. This work establishes a strong foundation for the development of robust, adaptable autonomous systems that can advance the frontiers of collaborative intelligence.


% The future work of this research aims to: 

% \textbf{Reduce costs:} The current framework uses multiple LLM agents as planners, executors, and auto-prompters. While this approach improves collective reasoning, it increases computational costs. Our goal is to extend this framework to reduce costs while maintaining the benefits of multi-agent reasoning and collaboration. This will involve optimizing agent interactions, reducing redundant processing, and exploring lightweight models for specific tasks. By balancing performance and resource use, we aim to create a scalable and cost-effective solution for complex problem solving.

% \textbf{Resource Optimization:} One of the key areas for future enhancement is optimizing computational resources across multi-agent frameworks. As collaborative LLM agents often run parallel processes, the cumulative computational load can become expensive and inefficient. We can explore asynchronous agent processing that would allow inactive agents to reduce their resource consumption until re-engaged by task demands. Implementing selective agent activation based on task complexity can ensure that only the necessary agents are active, minimizing redundant operations. 