\clearpage
\newpage

\section{Hard-coded prompt templates}
\label{sec:hard_coded_prompts}
Figure~\ref{fig:planner_conv1} shows the hard-coded category-wise prompt templates that are used to initiate the Planner agent when Auto-prompter is not present or it fails to generate a prompt.
% \begin{figure}[H]
%     \centering
%     \includegraphics[width=0.7\linewidth]{figures/appendix/prompt_1.pdf}
%     \label{fig:planner_conv2}
% \end{figure}

\begin{figure}[H]
    \centering
    \includegraphics[width=0.49\linewidth]{figures/appendix/prompt_1.pdf}
    \includegraphics[width=0.49\linewidth]{figures/appendix/prompt_2.pdf}
    \caption{Initial Executor Agent prompts for use in \textit{Planner-Executor} Framework; variations by category}
    \label{fig:planner_conv1}
\end{figure}

% \clearpage
% \newpage
\newpage
\section{Impact of Planner}
\label{sec:impact_planner}
In that section, we select the challenge \textit{ezmae} with Claude 3.5 Sonnet to analyze the impact of using a planner, showcasing a detailed example of the benefits it brings to the solution process.

% Figure~\ref{fig:woplanner5} and Figure~\ref{fig:wplanner5} compare the trajectories of the \textit{ezMaze} challenge with and without the Planner. 
\subsection{Without planner}
Figure~\ref{fig:woplanner5} shows a snapshot of the \textit{ezmaze} challenge trajectory, shows the executor directly follows the Auto-prompter generated prompt without using the planner.
\begin{figure}[H]
    \centering
    \includegraphics[width=0.49\linewidth]{figures/appendix/planner/woplanner1.pdf}
    \includegraphics[width=0.49\linewidth]{figures/appendix/planner/woplanner2.pdf}
    \label{fig:woplanner1}
\end{figure}

\begin{figure}[H]
    \centering
    \includegraphics[width=0.49\linewidth]{figures/appendix/planner/woplanner3.pdf}
    \includegraphics[width=0.49\linewidth]{figures/appendix/planner/woplanner4.pdf}
    \caption{Demonstration of failed solution without planner}
    \label{fig:woplanner5}
\end{figure}

\subsection{With planner}
Figure~\ref{fig:wplanner5} shows the trajectory for the \textit{ezmaze} challenge shows that the planner used a hard-coded prompt, while the Autoprompter did not generate the prompt.
\begin{figure}[H]
    \centering
    \includegraphics[width=0.49\linewidth]{figures/appendix/planner/wplanner1.pdf}
    \includegraphics[width=0.49\linewidth]{figures/appendix/planner/wplanner2.pdf}
    \label{fig:wplanner1}
\end{figure}

\begin{figure}[H]
    \centering
    \includegraphics[width=0.49\linewidth]{figures/appendix/planner/wplanner3.pdf}
    \includegraphics[width=0.49\linewidth]{figures/appendix/planner/wplanner4.pdf}
    \label{fig:wplanner3}
\end{figure}

\begin{figure}[H]
    \centering
    \includegraphics[width=0.49\linewidth]{figures/appendix/planner/wplanner5.pdf}
    \includegraphics[width=0.49\linewidth]{figures/appendix/planner/wplanner6.pdf}
    \caption{Demonstration of successful solution with planner}
    \label{fig:wplanner5}
\end{figure}

% \begin{figure*}
%     \centering
%     \includegraphics[width=17cm]{figures/appendix/appendix_w_o_planner.pdf}
%     \caption{A snapshot of the EzMaze challenge trajectory shows the executor directly following the Autoprompter's generated prompt without using the planner}
%     \label{fig:ezmaze_w_o_planner}
% \end{figure*}

% \begin{figure*}
%     \centering
%     \includegraphics[width=15cm]{figures/appendix/appendix_w_planner.pdf}
%     \caption{A snapshot of the trajectory for the EzMaze challenge shows that the planner used a hardcoded prompt, while the Autoprompter did not generate the prompt for this challenge}
%     \label{fig:ezmaze_w_planner}
% \end{figure*}

% \clearpage
% \newpage
\section{Comparison of different LLMs}
\label{sec:compare_llms}
We selected the challenge \textit{target\_practice} to illustrate the behaviors of different models tested in this section, highlighting their successes and failures by comparing their solution processes.
\subsection{Claude 3.5 Sonnet successful case}
Figure~\ref{fig:claude} demonstrates Claude 3.5 Sonnet with default setting to successfully solve challenge \texttt{target\_practice}.
\begin{figure}[H]
    \centering
    \includegraphics[width=0.49\linewidth]{figures/appendix/comparison/claude1.pdf}
    \includegraphics[width=0.49\linewidth]{figures/appendix/comparison/claude2.pdf}
    \caption{Demonstration of successful solution with Claude 3.5 Sonnet}
    \label{fig:claude}
\end{figure}

\subsection{GPT-4o successful case}
Figure~\ref{fig:4o} demonstrates GPT-4o with default setting to successfully solve challenge \texttt{target\_practice}.
\begin{figure}[H]
    \centering
    \includegraphics[width=0.49\linewidth]{figures/appendix/comparison/4o1.pdf}
    \includegraphics[width=0.49\linewidth]{figures/appendix/comparison/4o2.pdf}
    \caption{Demonstration of successful solution with GPT-4o}
    \label{fig:4o}
\end{figure}

% \begin{figure}[H]
%     \centering
%     \includegraphics[width=0.49\linewidth]{figures/appendix/comparison/4o3.pdf}
%     \caption{Demonstration of using GPT-4o with default setting to successfully solve challenge \texttt{target\_practice}.}
%     \label{fig:4o3}
% \end{figure}

% \begin{figure}[H]
%     \centering
%     \includegraphics[width=0.49\linewidth]{figures/appendix/comparison/turbo1.pdf}
%     \label{fig:turbo1}
% \end{figure}

\subsection{GPT-4 Turbo failed case}
Figure~\ref{fig:turbo} demonstrates GPT-4 Turbo with default setting to try solving challenge \textit{target\_practice}, however the executors do not find the flag and decide to giveup.
\begin{figure}[H]
    \centering
    \includegraphics[width=0.49\linewidth]{figures/appendix/comparison/turbo1.pdf}
    \includegraphics[width=0.49\linewidth]{figures/appendix/comparison/turbo2.pdf}
    \caption{Demonstration of failed solution with GPT-4 Turbo}
    \label{fig:turbo}
\end{figure}

% \begin{figure}[H]
%     \centering
    
%     \label{fig:llama1}
% \end{figure}

\subsection{LLaMa 3.1 405B failed case}
Figure~\ref{fig:llama} demonstrates LLaMa 3.1 405B with default setting to try solving challenge \textit{target\_practice}, however the executors do not find the flag and decide to giveup.
\begin{figure}[H]
    \centering
    \includegraphics[width=0.49\linewidth]{figures/appendix/comparison/llama1.pdf}
    \includegraphics[width=0.49\linewidth]{figures/appendix/comparison/llama2.pdf}
    \caption{Demonstration of failed solution with LLaMa 3.1 405B}
    \label{fig:llama}
\end{figure}

\subsection{Gemini 1.5 Flash failed case}
Figure~\ref{fig:gemini} demonstrates Gemini 1.5 Flash with default setting to try solving challenge \textit{target\_practice}, however the executors do not finish the task and do not provide a summary.
\begin{figure}[H]
    \centering
    \includegraphics[width=0.49\linewidth]{figures/appendix/comparison/gemini1.pdf}
    \includegraphics[width=0.49\linewidth]{figures/appendix/comparison/gemini2.pdf}
    \caption{Demonstration of failed solution with Gemini 1.5 Flash}
    \label{fig:gemini}
\end{figure}

% \clearpage
% \newpage
\section{Failure demonstration}
\label{sec:failure_demo}
That section contains examples of the failures discussed in Section~\ref{sec:other_failure}. For each type of failure identified, we provided a representative example from the experiment to illustrate the nature of the issue in detail. By analyzing these examples, we aim to provide a comprehensive understanding of the failure types and their impact on the overall system performance.

\subsection{Calling non-existent function}
Figure~\ref{fig:run_unexistent_tool} shows the model tries to call a function \textit{decode} that is not provided in D-CIPHER's tool set on challenge \textit{babycrypto}.
\begin{figure}[H]
    \centering
    \includegraphics[width=0.5\linewidth]{figures/appendix/call_unexistent_tool.pdf}
    \caption{Calling non-existent functions}
    \label{fig:run_unexistent_tool}
\end{figure}

\subsection{Call command line tools before installation}
Figure~\ref{fig:call_uninstalled_cmd} shows the model tries to call ndisasm tool before install nasm package on challenge \textit{realism}.

\begin{figure}[H]
    \centering
    \includegraphics[width=0.5\linewidth]{figures/appendix/call_uninstalled_cmd.pdf}
    \caption{Call command line tools before installation}
    \label{fig:call_uninstalled_cmd}
\end{figure}

\subsection{Running non-existent commands}
Figure~\ref{fig:run_unexistent_cmd} shows the model tries to use radar2 to analyze a binary program with command pd and px, but run these commands by mistake before entering radar2 interface on challenge \textit{a\_walk\_through\_x86\_part\_1}
\begin{figure}[H]
    \centering
    \includegraphics[width=0.5\linewidth]{figures/appendix/call_unexistent_cmd.pdf}
    \caption{Running non-existent commands}
    \label{fig:run_unexistent_cmd}
\end{figure}


% \begin{figure}[H]
%     \centering
%     \includegraphics[width=\linewidth]{figures/appendix/failed_function_call.pdf}
%     \caption{Failed to generate function call on challenge \textit{thoroughlyStripped}, the model generated a wrong function calling structure.}
%     \label{fig:fail_function_call}
% \end{figure}

% \begin{figure}[H]
%     \centering
%     \includegraphics[width=\linewidth]{figures/appendix/fail_connection.pdf}
%     \caption{Hallucination on server information on challenge \textit{pwnvoltex}, the challenge does not require any challenge server access but the planner provides nonexisted server information for executors to take action due to its hallucination from misleading local challenge files.}
%     \label{fig:fail_connection}
% \end{figure}

% \begin{figure}[H]
%     \centering
%     \includegraphics[width=\linewidth]{figures/appendix/failed_autoprompting.pdf}
%     \caption{Auto prompting Agent fail to generate on challenge \textit{adversarial}, the autoprompter runs prompts to obtain extra information but due to the mistakes in command generation the autoprompter does not get enough information to generate a prompt}
%     \label{fig:fail_autoprompting}
% \end{figure}

\subsection{Reverse error}
Figure~\ref{fig:fail_reverse} shows model tries to extract a function called \textit{\_Z1AIPhEvT\_} from decompiler, which is a weak symbol. In this challenge, the weak symbol may be overridden by a strong symbol, causing the decompiler to fail to locate the function on challenge \textit{arevenge}.
\begin{figure}[H]
    \centering
    \includegraphics[width=0.5\linewidth]{figures/appendix/reverse_failure.pdf}
    \caption{Reverse error}
    \label{fig:fail_reverse}
\end{figure}

\subsection{Lack of tool support}
Figure~\ref{fig:unsupported_tool} shows agent fails to complete the \textit{baby\_boi} challenge with GPT-4o due to the use of an unsupported tool.
\begin{figure}
    \centering
    \includegraphics[width=0.5\linewidth]{figures/appendix/unsupported_tool.pdf}
    \caption{Lack of tool support}
    \label{fig:unsupported_tool}
\end{figure}

% \clearpage
% \newpage

\section{Cost statistics}
\label{sec:appendix_cost}
This section presents a category-wise comparison of the average costs incurred by EnIGMA and D-CIPHER, including both the costs for successful solutions and the overall costs for all solutions, as a supplement to Section~\ref{sec:cost_analysis}.
% \begin{table*}[t]
% \vspace{0.10em}
% \centering
% \begin{tabular}{lccc|ccccc}
%     \toprule
%     & \multicolumn{3}{c}{\textbf{EnIGMA} (\$ Overall)} 
%     & \multicolumn{5}{c}{\textbf{D-CIPHER} (\$ Overall)} \\
%     \cmidrule(lr){2-4} \cmidrule(lr){5-9}
%     \textbf{Category} 
%     & \textbf{Claude 3.5 Sonnet} 
%     & \textbf{GPT-4o} 
%     & \textbf{GPT-4 Turbo}
%     & \textbf{Claude 3.5 Sonnet} 
%     & \textbf{GPT-4 Turbo} 
%     & \textbf{GPT-4o} 
%     & \textbf{LLaMA 3.2 405B} 
%     & \textbf{Gemini 1.5 Flash} \\
%     \midrule
%     \texttt{crypto}    & 0 & 0 & 0 & 0 & 0 & 0 & 0 \\
%     \texttt{forensics} & 0 & 0 & 0 & 0 & 0 & 0 & 0 \\
%     \texttt{pwn}       & 0 & 0 & 0 & 0 & 0 & 0 & 0 \\
%     \texttt{rev}       & 0 & 0 & 0 & 0 & 0 & 0 & 0 \\
%     \texttt{misc}      & 0 & 0 & 0 & 0 & 0 & 0 & 0 \\
%     \texttt{web}       & 0 & 0 & 0 & 0 & 0 & 0 & 0 \\
%     \midrule
%     \textbf{Overall}   & 0 & 0 & 0 & 0 & 0 & 0 & 0 \\
%     \bottomrule
% \end{tabular}
% \caption{
% \minghao{TODO: Write the caption}
% \haoran{Add the solved and cost numbers.}
% }
% \label{tab:category_wised}
% \end{table*}

% \begin{table}[H]
% \vspace{0.25em}
% \centering
% \begin{tabular}{lccc|cccccp{2cm}}
%     \toprule
%     & \multicolumn{2}{c}{\textbf{EnIGMA} (\$)} 
%     & \multicolumn{5}{c}{\textbf{D-CIPHER} (\$)} \\
%     \cmidrule(lr){2-4} \cmidrule(lr){5-9}
%     \textbf{Category} 
%     & \rotatebox{90}{\textbf{Claude 3.5 Sonnet}} 
%     & \rotatebox{90}{\textbf{GPT-4 Turbo}} 
%     & \rotatebox{90}{\textbf{GPT-4o}} 
%     & \rotatebox{90}{\textbf{Claude 3.5 Sonnet}} 
%     & \rotatebox{90}{\textbf{GPT-4 Turbo}} 
%     & \rotatebox{90}{\textbf{GPT-4o}} 
%     & \rotatebox{90}{\textbf{LLaMA 3.2 405B}} 
%     & \rotatebox{90}{\textbf{Gemini 1.5 Flash}} \\
%     \midrule
%     \texttt{crypto}    & 2.74 & 3.05 & 2.72 & 0 & 0 & 0 & 0 & 0 \\
%     \texttt{forensics} & 2.51 & 2.81 & 2.65 & 0 & 0 & 0 & 0 & 0 \\
%     \texttt{pwn}       & 2.45 & 2.77 & 2.67 & 0 & 0 & 0 & 0 & 0 \\
%     \texttt{rev}       & 2.54 & 2.76 & 2.58 & 0 & 0 & 0 & 0 & 0 \\
%     \texttt{misc}      & 2.62 & 2.63 & 2.66 & 0 & 0 & 0 & 0 & 0 \\
%     \texttt{web}       & 3.05 & 2.80 & 2.93 & 0 & 0 & 0 & 0 & 0 \\
%     \midrule
%     \textbf{Overall}   & 2.62 & 2.82 & 2.68 & 0 & 0 & 0 & 0 & 0 \\
%     \bottomrule
% \end{tabular}
% \caption{
% \minghao{TODO: Write the caption}
% \haoran{Add the solved and cost numbers.}
% }
% \label{tab:category_wised}
% \end{table}

\begin{table}[H]
\centering
\small
\caption{
Average cost comparison of all 6 categories for Claude 3.5 Sonnet, GPT-4 Turbo and GPT-4o on all solutions attempted with EniGMA and D-CIPHER on NYU CTF Bench.
}
% \vspace{-1mm}
\begin{tabular}{lccc|cccp{2cm}}
    \toprule
    & \multicolumn{3}{c}{\textbf{EnIGMA} (\$)} 
    & \multicolumn{3}{c}{\textbf{D-CIPHER} (\$)} \\
    \cmidrule(lr){2-4} \cmidrule(lr){5-7}
    \textbf{Category} 
    & \rotatebox{90}{\textbf{Claude 3.5 S.}} 
    & \rotatebox{90}{\textbf{GPT-4 T.}} 
    & \rotatebox{90}{\textbf{GPT-4o}} 
    & \rotatebox{90}{\textbf{Claude 3.5 S.}} 
    & \rotatebox{90}{\textbf{GPT-4 T.}} 
    & \rotatebox{90}{\textbf{GPT-4o}} \\
    \midrule
    \texttt{crypto}    & 2.74 & 3.05 & 2.72 & 2.30 & 1.42 & 0.89 \\
    \texttt{forensics} & 2.51 & 2.81 & 2.65 & 2.13 & 1.47 & 1.04 \\
    \texttt{pwn}       & 2.45 & 2.77 & 2.67 & 2.00 & 1.92 & 1.10 \\
    \texttt{rev}       & 2.54 & 2.76 & 2.58 & 2.12 & 1.52 & 1.10 \\
    \texttt{misc}      & 2.62 & 2.63 & 2.66 & 1.55 & 1.04 & 0.62 \\
    \texttt{web}       & 3.05 & 2.80 & 2.93 & 1.98 & 1.34 & 0.82 \\
    \midrule
    \textbf{Overall}   & 2.62 & 2.82 & 2.68 & 2.06 & 1.49 & 0.96 \\
    \bottomrule
\end{tabular}
% \vspace{-5mm}
\label{tab:overall_cost_category}
\end{table}

\begin{table}[H]
\centering
\small
\caption{
Average cost comparison of all 6 categories for Claude 3.5 Sonnet, GPT-4 Turbo and GPT-4o on succeeded solutions with EniGMA and D-CIPHER on NYU CTF Bench.
}
% \vspace{-1mm}
\begin{tabular}{lccc|cccp{2cm}}
    \toprule
    & \multicolumn{3}{c}{\textbf{EnIGMA} (\$)} 
    & \multicolumn{3}{c}{\textbf{D-CIPHER} (\$)} \\
    \cmidrule(lr){2-4} \cmidrule(lr){5-7}
    \textbf{Category} 
    & \rotatebox{90}{\textbf{Claude 3.5 S.}} 
    & \rotatebox{90}{\textbf{GPT-4 T.}} 
    & \rotatebox{90}{\textbf{GPT-4o}} 
    & \rotatebox{90}{\textbf{Claude 3.5 S.}} 
    & \rotatebox{90}{\textbf{GPT-4 T.}} 
    & \rotatebox{90}{\textbf{GPT-4o}} \\
    \midrule
    \texttt{crypto}    & 0.18 & 1.67 & 1.48 & 0.68 & 0.50 & 0.68  \\
    \texttt{forensics} & 0.33 & 0.73 & 0.75 & 0.46 & 1.37 & 0.08  \\
    \texttt{pwn}       & 0.11 & 0.53 & 0.29 & 0.32 & 0.68 & 0.11  \\
    \texttt{rev}       & 0.52 & 0.99 & 0.35 & 0.66 & 0.12 & 0.14  \\
    \texttt{misc}      & 0.56 & 0.49 & 0.68 & 0.26 & 0.08 & 0.25  \\
    \texttt{web}       & N/A & N/A & 1.29 & 0.06 & 0.78 & 0.08  \\
    \midrule
    \textbf{Overall}   & 0.35 & 0.79 & 0.62 & 0.52 & 0.47 & 0.22 \\
    \bottomrule
\end{tabular}
\label{tab:category_wised}
\end{table}

\begin{figure}[H]
    \centering
    \vspace{-4mm}
    \includegraphics[width=0.94\linewidth]{figures/cost_comparison.pdf}
    \label{fig:cost_compare_bar}
    \caption{Comparison of average cost of solved challenges and overall average cost of D-CIPHER and EnIGMA on NYU CTF Bench.}
    % \vspace{-6mm}
\end{figure}

\section{Failure statistics}
\label{sec:appendix_failure}
This section provides a category-wise comparison of failure statistics across all five models, focusing on three types of failures: Max Rounds, Max Cost, and Give Up. The experiment using D-CIPHER with the default setup on the NYU CTF Bench serves as a supplement to Section~\ref{sec:analysis}.

% \begin{table}[htbp]
%   \centering
%   \begin{tabular}{|l|c|c|c|c|c|c|}
%     \hline
%     \multicolumn{7}{|c|}{\textbf{Maximum Planner Rounds (\%)}} \\
%     \hline
%     \textbf{LLM} & \textbf{cry} & \textbf{for} & \textbf{pwn} & \textbf{rev} & \textbf{web} & \textbf{misc} \\
%     \hline
%     Claude 3.5 Sonnet &1.92 &6.67 &0 &1.96 &0 &0 \\
%     GPT-4 Turbo &0 &0 &0 &0 &0 &0 \\
%     GPT-4o &13.46 &6.67 &12.82 &9.80 &0 &20.83 \\
%     LLaMA 3.2 405B &23.08 &20.00 &28.21 &21.57 &36.84 &8.33 \\
%     Gemini 1.5 Flash &0 &0 &0 &0 &0 &4.17 \\
%     \hline
%   \end{tabular}
  
%   \vspace{10pt}
  
%   \begin{tabular}{|l|c|c|c|c|c|c|}
%     \hline
%     \multicolumn{7}{|c|}{\textbf{Maximum Budget Cost (\%)}} \\
%     \hline
%     \textbf{LLM} & \textbf{cry} & \textbf{for} & \textbf{pwn} & \textbf{rev} & \textbf{web} & \textbf{misc} \\
%     \hline
%     Claude 3.5 Sonnet &63.46 &53.33 &48.72 &52.94 &42.11 &37.50 \\
%     GPT-4 Turbo &11.54 &20.00 &33.33 &13.73 &10.53 &0 \\
%     GPT-4o &3.85 &0 &7.69 &1.96 &5.26 &4.17 \\
%     LLaMA 3.2 405B &0 &0 &0 &0 &0 &0s \\
%     Gemini 1.5 Flash &40.38 &26.67 &58.97 &31.37 &0 &8.33 \\
%     \hline
%   \end{tabular}
%   \caption{Failure ratio for two types of system designed common failure on 5 models examined with all six categories on NYU CTF Bench.}
%   \label{tab:combined}
% \end{table}

\begin{table}[H]
  \centering
\caption{Maximum Planner Rounds ratio on 5 models examined with all six categories on NYU CTF Bench.}
  \begin{tabular}{|l|c|c|c|c|c|c|}
    \hline
    \multicolumn{7}{|c|}{\textbf{Maximum Planner Rounds (\%)}} \\
    \hline
    \textbf{LLM} & \textbf{cry} & \textbf{for} & \textbf{pwn} & \textbf{rev} & \textbf{web} & \textbf{misc} \\
    \hline
    Claude 3.5 Sonnet &1.92 &6.67 &0 &1.96 &0 &0 \\
    GPT-4 Turbo &0 &0 &0 &0 &0 &0 \\
    GPT-4o &3.85 &6.67 &5.13 &7.84 &5.26 &8.33 \\
    LLaMA 3.1 405B &23.08 &20.00 &28.21 &21.57 &36.84 &8.33 \\
    Gemini 1.5 Flash &0 &0 &0 &0 &0 &4.17 \\
    \hline
  \end{tabular}
  
  \vspace{10pt}
  \label{tab:max_planner_rounds}
\end{table}

\begin{table}[H]
  \caption{Failure ratio for two types of system designed common failure on 5 models examined with all six categories on NYU CTF Bench.}
  \centering
  \begin{tabular}{|l|c|c|c|c|c|c|}
    \hline
    \multicolumn{7}{|c|}{\textbf{Maximum Budget Cost (\%)}} \\
    \hline
    \textbf{LLM} & \textbf{cry} & \textbf{for} & \textbf{pwn} & \textbf{rev} & \textbf{web} & \textbf{misc} \\
    \hline
    Claude 3.5 Sonnet &63.46 &60.00 &48.72 &52.94 &42.11 &37.50 \\
    GPT-4 Turbo &11.54 &20.00 &33.33 &13.73 &10.53 &0 \\
    GPT-4o &5.77 &6.67 &10.26 &9.80 &0 &4.17 \\
    LLaMA 3.1 405B &0 &0 &0 &0 &0 &0 \\
    Gemini 1.5 Flash &40.38 &26.67 &58.97 &31.37 &0 &8.33 \\
    \hline
  \end{tabular}
  
  \vspace{10pt}
  \label{tab:max_budget}
\end{table}


% \begin{table*}[htbp]
% \centering
% \normalsize
% \begin{tabular}{lcccccc} % Adjust the number of columns according to your data
% \toprule
% \textbf{Category} & \textbf{GPT 4 Turbo(\%)} & \textbf{GPT 4o(\%)} & \textbf{Claude 3.5 Sonnet(\%)} & \textbf{LLaMA 3.2 405B(\%)} & \textbf{Gemini 1.5 Flash (\%)} \\
% \midrule
% crypto & 82.35 & 61.54 & 23.53 & 74.00 & 57.69 \\
% forensics & 57.14 & 53.33 & 20.00 & 73.33 & 66.67 \\
% misc & 87.50 & 58.33 & 50.00 & 76.19 & 79.17\\
% pwn & 61.54 & 26.32 & 30.77 & 67.65 & 36.11\\
% rev & 78.43 & 47.06 & 17.39 & 74.51 & 64.71\\
% web & 83.33 & 89.47 & 60.00 & 63.16 & 100.0\\
% \textbf{Overall} & 75.05 & 56.01 & 33.61 & 71.47 & 67.39 \\
% \bottomrule
% \end{tabular}
% \caption{Give up ratio on D-CIPHER with different models}
% \label{tab:give_up}
% \end{table*}

% \begin{table}[htbp]
%   \centering
%   \vspace{10pt}
  
%   \begin{tabular}{|l|c|c|c|c|c|c|}
%     \hline
%     \multicolumn{7}{|c|}{\textbf{Give up (\%)}} \\
%     \hline
%     \textbf{LLM} & \textbf{cry} & \textbf{for} & \textbf{pwn} & \textbf{rev} & \textbf{web} & \textbf{misc} \\
%     \hline
%     Claude 3.5 Sonnet &15.38 &13.33 &23.08 &13.73 &52.63 &37.50 \\
%     GPT-4 Turbo &84.62 &53.33 &61.54 &78.43 &78.95 &87.50 \\
%     GPT-4o &71.15 &73.33 &66.67 &74.51 &84.21 &54.17 \\
%     LLaMA 3.2 405B &71.15 &73.33 &58.97 &74.51 &63.16 &66.67 \\
%     Gemini 1.5 Flash &57.69 &66.67 &33.33 &64.71 &100.00 &79.17 \\
%     \hline
%   \end{tabular}
%   \caption{Give up ratio on 5 models examined with all six categories on NYU CTF Bench.}
%   \label{tab:combined}
% \end{table}



% \begin{table}[htbp]
% \centering
% \normalsize
% \begin{tabular}{lcccccc} % Adjust the number of columns according to your data
% \toprule
% \textbf{Category} & 
% \rotatebox{90}{\textbf{GPT 4 Turbo (\%)}} & 
% \rotatebox{90}{\textbf{GPT 4o (\%)}} & 
% \rotatebox{90}{\textbf{Claude 3.5 Sonnet (\%)}} & 
% \rotatebox{90}{\textbf{LLaMA 3.2 405B (\%)}} & 
% \rotatebox{90}{\textbf{Gemini 1.5 Flash (\%)}} \\
% \midrule
% crypto & 82.35 & 61.54 & 23.53 & 74.00 & 57.69 \\
% forensics & 57.14 & 53.33 & 20.00 & 73.33 & 66.67 \\
% misc & 87.50 & 58.33 & 50.00 & 76.19 & 79.17\\
% pwn & 61.54 & 26.32 & 30.77 & 67.65 & 36.11\\
% rev & 78.43 & 47.06 & 17.39 & 74.51 & 64.71\\
% web & 83.33 & 89.47 & 60.00 & 63.16 & 100.0\\
% \textbf{Overall} & 75.05 & 56.01 & 33.61 & 71.47 & 67.39 \\
% \bottomrule
% \end{tabular}
% \caption{Give up ratio on D-CIPHER with different models}
% \label{tab:give_up}
% \end{table}


\begin{table}[H]
  \centering
  \vspace{10pt}
\caption{Give up ratio on 5 models examined with all six categories on NYU CTF Bench.}
  \begin{tabular}{|l|c|c|c|c|c|c|}
    \hline
    \multicolumn{7}{|c|}{\textbf{Give up (\%)}} \\
    \hline
    \textbf{LLM} & \textbf{cry} & \textbf{for} & \textbf{pwn} & \textbf{rev} & \textbf{web} & \textbf{misc} \\
    \hline
    Claude 3.5 Sonnet &15.38 &13.33 &23.08 &13.73 &52.63 &37.50 \\
    GPT-4 Turbo &84.62 &60.00 &61.54 &78.43 &78.95 &87.50 \\
    GPT-4o &80.77 &73.33 &69.23 &64.71 &84.21 &66.67 \\
    LLaMA 3.1 405B &71.15 &73.33 &58.97 &74.51 &63.16 &66.67 \\
    Gemini 1.5 Flash &57.69 &66.67 &33.33 &64.71 &100.00 &79.17 \\
    \hline
  \end{tabular}
  \label{tab:giveup}
\end{table}