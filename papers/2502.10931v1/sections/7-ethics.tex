\subsection{Ethics}


Capture the Flag competitions underscore the ethical challenges associated with using LLMs in cybersecurity. Designed to simulate real-world cyberattacks, these events emphasize the need for robust AI ethics education to prepare students and professionals to navigate the security threats posed by emerging LLMs \cite{jackson2023artificial}. While advancements in LLMs offer significant advantages for cybersecurity, they also introduce risks, including the potential misuse of these models in adversarial scenarios where safeguards are bypassed. CTFs serve as controlled environments to test the ethical and secure deployment of these technologies, providing insights into their strengths and vulnerabilities. As LMs evolve, users and decision-makers must address concerns around data security, user privacy, and malicious exploitation by implementing strategies that balance technical capabilities with ethical responsibility \cite{dabbagh2024ai}. Malicious actors can exploit LLMs for social engineering campaigns or generating harmful code, underscoring the urgent need for robust ethical protocols and governance structures \cite{wu2023privacy}. Moreover, the rapid evolution of AI often outpaces existing legal and regulatory frameworks, raising critical questions about data security, user privacy, and accountability \cite{porsdam2023generative}. Furthermore, insufficient fine-tuning and biased training data can perpetuate discriminatory practices or enable harmful mechanisms of social control \cite{bouschery2023augmenting}. Addressing these challenges requires rigorous policy frameworks, transparent AI methods, and governance structures that align with ethical standards while mitigating misuse \cite{chan2024ai}. CTFs provide a controlled environment for testing LMs, enabling organizations to evaluate their strengths, vulnerabilities, and ethical deployment strategies. Integrating LMs into cybersecurity requires balancing ethical risks with technological potential. Strengthening AI ethics education, especially in CTF competitions, is crucial for preparing future professionals. Organizations must focus on responsible LM development, guided by accountability, ethical policies, and oversight, to ensure defensive use and mitigate misuse.

% trimmed-version-1 Capture the Flag competitions highlight the ethical challenges associated with using large language models (LLMs) in cybersecurity. Designed to simulate real-world cyberattacks, these events emphasize the need for robust AI ethics education to prepare students and professionals to navigate the security threats posed by emerging LMs \cite{jackson2023artificial}. While advancements in LMs 
% % such as OpenAI-o1 \cite{openaio1}, Claude 3.5 \cite{claude35sonnet}, Gemini 2.0 \cite{gemini20}, and LLaMA 3.1 \cite{llama32} 
% offer significant advantages for cybersecurity, they also introduce risks, including the potential misuse of these models in adversarial scenarios where safeguards are bypassed. CTFs serve as controlled environments to test the ethical and secure deployment of these technologies, providing insights into their strengths and vulnerabilities. As LMs evolve, users and decision-makers must address concerns around data security, user privacy, and malicious exploitation by implementing strategies that balance technical capabilities with ethical responsibility \cite{dabbagh2024ai}. Organizations and academic institutions must fully consider the ethical implications of LMs, ensuring their deployment is guided by accountability, moral responsibility, and adherence to responsible AI principles \cite{gennari2024considerations}.

% Malicious actors can exploit LLMs for social engineering campaigns or generating harmful code, underscoring the urgent need for robust ethical protocols and governance structures \cite{wu2023privacy}. The rapid pace of AI innovation often outpaces existing legal and regulatory frameworks, complicating efforts to manage outputs from non-human entities and raising profound legal and moral questions \cite{porsdam2023generative}. Additionally, insufficient fine-tuning and homogeneous training data can embed biases within LMs, potentially enabling discriminatory practices or mechanisms of social control \cite{bouschery2023augmenting}. Addressing these challenges requires the development of rigorous policy frameworks aligned with ethical standards, as well as explainable AI methods to enhance transparency and accountability by revealing the decision-making processes of these models \cite{chan2024ai}. For LMs to be effectively integrated into cybersecurity strategies, a balanced approach that weighs their ethical risks against their technological potential is essential. Strengthening AI ethics education, particularly within the context of Capture the Flag (CTF) competitions, is crucial to preparing the next generation of cybersecurity professionals. Organizations must prioritize the responsible development of LMs for defensive purposes, supported by ongoing oversight, ethical policies, and a commitment to mitigating misuse.



%Minghao Version
% Within Capture the Flag (CTF) competitions, the ethical implications of using large language models (LLMs) come to the forefront. As these events are designed to simulate real-world cyber attacks, it highlights the importance ofintegrating robust AI ethics education into cybersecurity training programs. As AI-driven systems become more prominent, both students and professionals must develop advanced analytical skills and a strong moral compass to navigate the security threats posed by emerging LMs \cite{jackson2023artificial}. The potential for these models to be misused, particularly in scenarios where adversaries jailbreak systems to override content safeguards, underscores the urgency of implementing responsible AI strategies, and ensuring ethical considerations are not overlooked in cybersecurity contexts\cite{dabbagh2024ai}. In this light, organizations and academic institutions that employ LMs must fully appreciate the ethical ramifications of these tools, and ensure they are deployed with due diligence and moral responsibility \cite{gennari2024considerations}.

% Recent developments in LMs, including OpenAI-o1 \cite{openaio1}, Claude 3.5 \cite{claude35sonnet}, Gemini 2.0 \cite{gemini20} and LLaMA 3.1 \cite{llama32}, offer considerable advantages for the cybersecurity field, but they also pose formidable challenges. By intentionally mirroring real attack vectors in a structured environment, CTFs serve as a proving ground for the safe and ethical deployment of advanced AI. As LMs continue to evolve, both those who use these technologies--and those charged with deciding \textit{how} they are used--will need  to address escalating concerns about data security, user privacy, and the likelihood of malicious exploitation. Implementing these technologies responsibly requires strategies that address technical and ethical dimensions.

% Malicious actors can co-opt these models for social engineering campaigns, or to generate harmful code, which illustrates the necessity of sound ethical protocols and governance structures \cite{wu2023privacy}. Existing legal and regulatory frameworks often lag behind the accelerated pace of AI innovation, complicating efforts to manage outputs produced by non-human entities and raising significant legal and moral questions \cite{porsdam2023generative}. Moreover, insufficient fine-tuning and homogeneous training data can inadvertently perpetuate biases within LMs, potentially leading to discriminatory practices or even facilitating broader mechanisms of social control \cite{bouschery2023augmenting}. To address these issues, researchers must develop rigorous policy frameworks that align with ethical standards and protect against misuse. Equally important, they must develop explainable AI methods that contribute to transparency and accountability by revealing the decision-making processes that lie at the heart of  AI models \cite{chan2024ai}.

% Before LMs can be fully integrated into cybersecurity strategies, a balanced strategy that weighs the ethical risks against technological potential of LMs is required. To ensure that the next generation of cybersecurity leaders and professionals is prepared to address these challenges, it is essential to strengthen AI ethics education, especially in the context of CTF competitions. Organizations must be dedicated to the responsible development of LMs for defensive purposes, without permitting harmful exploits. This dedication should include developing and advocating for policies that encourage ethical behavior, and that are supported by ongoing oversight, and the development of ethical awareness.