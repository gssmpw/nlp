\pdfoutput=1 
\documentclass[sn-mathphys-num]{sn-jnl}
\DeclareUnicodeCharacter{2212}{\textminus}

%\usepackage[
%backend=biber,
%style=alphabetic,
%sorting=ynt
%]{biblatex}
%\usepackage{svg-extract}
\usepackage{pgf}
\usepackage{layouts}
\usepackage{geometry}
\usepackage[acronym]{glossaries}
\usepackage{optidef}
\usepackage{soul}
\usepackage{color}
\usepackage{tabularx}
\usepackage{graphicx}%
\usepackage{multirow}%
\usepackage{amsmath,amssymb,amsfonts}%
\usepackage{amsthm}%
\usepackage{mathrsfs}%
\usepackage{pdflscape}
 \usepackage{svg}

%% as per the requirement new theorem styles can be included as shown below
\theoremstyle{thmstyleone}%
\newtheorem{theorem}{Theorem}%  meant for continuous numbers
%%\newtheorem{theorem}{Theorem}[section]% meant for sectionwise numbers
%% optional argument [theorem] produces theorem numbering sequence instead of independent numbers for Proposition
\newtheorem{proposition}[theorem]{Proposition}% 
%%\newtheorem{proposition}{Proposition}% to get separate numbers for theorem and proposition etc.

\theoremstyle{thmstyletwo}%
\newtheorem{example}{Example}%
\newtheorem{remark}{Remark}%

\theoremstyle{thmstylethree}%
\newtheorem{definition}{Definition}%

\raggedbottom




%\date{November 2021}

\begin{document}

\title{Towards Privacy-Preserving Anomaly-Based Intrusion Detection in Energy Communities}
%Towards Secure Energy Communities
%Detecting Cyber Threats in Energy Communities Using Unsupervised Anomaly Detection

\author*[1]{\fnm{Zeeshan} \sur{Afzal}}\email{zeeshan.afzal@liu.se}
\author[2]{\fnm{Giovanni} \sur{Gaggero,}}%\email{iiauthor@gmail.com}
\author[1]{\fnm{Mikael} \sur{Asplund}}%\email{iiauthor@gmail.com}

%\equalcont{These authors contributed equally to this work.}


\affil[1]{Department of Computer and Information Science (IDA) \\ Linköping University, 581 83, Linköping, Sweden}
\affil[2]{Department of Electrical, Electronic and Telecommunications Engineering, and Naval Architecture (DITEN) \\ University of Genoa, Via all'Opera Pia 11A, 16145, Genoa, Italy}

%\maketitle


\abstract{

%Modern power systems are increasingly relying on decentralized energy communities that integrate production, storage, distribution, and consumption. However, this shift also brings heightened risks of cyber threats.

Energy communities consist of decentralized energy production, storage, consumption, and distribution and are gaining traction in modern power systems. However, these communities may increase the vulnerability of the grid to cyber threats. We propose an anomaly-based intrusion detection system to enhance the security of energy communities. The system leverages deep autoencoders to detect deviations from normal operational patterns in order to identify anomalies induced by malicious activities and attacks. Operational data for training and evaluation are derived from a Simulink model of an energy community. The results show that the autoencoder-based intrusion detection system achieves good detection performance across multiple attack scenarios. We also demonstrate potential for real-world application of the system by training a federated model that enables distributed intrusion detection while preserving data privacy.

} 

%We demonstrate potential for real-world application by training a federated model that enables distributed intrusion detection while preserving data privacy, allowing multiple energy communities to collaboratively improve their cybersecurity without sharing raw operational data.
%The work highlights the critical need of securing energy communities to enhance the reliability and stability of the electrical grid.

\keywords{smart grid, distributed energy resources, battery energy storage systems, cloud control systems, energy communities, cyber security} 



\maketitle


\section{Introduction}

Video generation has garnered significant attention owing to its transformative potential across a wide range of applications, such media content creation~\citep{polyak2024movie}, advertising~\citep{zhang2024virbo,bacher2021advert}, video games~\citep{yang2024playable,valevski2024diffusion, oasis2024}, and world model simulators~\citep{ha2018world, videoworldsimulators2024, agarwal2025cosmos}. Benefiting from advanced generative algorithms~\citep{goodfellow2014generative, ho2020denoising, liu2023flow, lipman2023flow}, scalable model architectures~\citep{vaswani2017attention, peebles2023scalable}, vast amounts of internet-sourced data~\citep{chen2024panda, nan2024openvid, ju2024miradata}, and ongoing expansion of computing capabilities~\citep{nvidia2022h100, nvidia2023dgxgh200, nvidia2024h200nvl}, remarkable advancements have been achieved in the field of video generation~\citep{ho2022video, ho2022imagen, singer2023makeavideo, blattmann2023align, videoworldsimulators2024, kuaishou2024klingai, yang2024cogvideox, jin2024pyramidal, polyak2024movie, kong2024hunyuanvideo, ji2024prompt}.


In this work, we present \textbf{\ours}, a family of rectified flow~\citep{lipman2023flow, liu2023flow} transformer models designed for joint image and video generation, establishing a pathway toward industry-grade performance. This report centers on four key components: data curation, model architecture design, flow formulation, and training infrastructure optimization—each rigorously refined to meet the demands of high-quality, large-scale video generation.


\begin{figure}[ht]
    \centering
    \begin{subfigure}[b]{0.82\linewidth}
        \centering
        \includegraphics[width=\linewidth]{figures/t2i_1024.pdf}
        \caption{Text-to-Image Samples}\label{fig:main-demo-t2i}
    \end{subfigure}
    \vfill
    \begin{subfigure}[b]{0.82\linewidth}
        \centering
        \includegraphics[width=\linewidth]{figures/t2v_samples.pdf}
        \caption{Text-to-Video Samples}\label{fig:main-demo-t2v}
    \end{subfigure}
\caption{\textbf{Generated samples from \ours.} Key components are highlighted in \textcolor{red}{\textbf{RED}}.}\label{fig:main-demo}
\end{figure}


First, we present a comprehensive data processing pipeline designed to construct large-scale, high-quality image and video-text datasets. The pipeline integrates multiple advanced techniques, including video and image filtering based on aesthetic scores, OCR-driven content analysis, and subjective evaluations, to ensure exceptional visual and contextual quality. Furthermore, we employ multimodal large language models~(MLLMs)~\citep{yuan2025tarsier2} to generate dense and contextually aligned captions, which are subsequently refined using an additional large language model~(LLM)~\citep{yang2024qwen2} to enhance their accuracy, fluency, and descriptive richness. As a result, we have curated a robust training dataset comprising approximately 36M video-text pairs and 160M image-text pairs, which are proven sufficient for training industry-level generative models.

Secondly, we take a pioneering step by applying rectified flow formulation~\citep{lipman2023flow} for joint image and video generation, implemented through the \ours model family, which comprises Transformer architectures with 2B and 8B parameters. At its core, the \ours framework employs a 3D joint image-video variational autoencoder (VAE) to compress image and video inputs into a shared latent space, facilitating unified representation. This shared latent space is coupled with a full-attention~\citep{vaswani2017attention} mechanism, enabling seamless joint training of image and video. This architecture delivers high-quality, coherent outputs across both images and videos, establishing a unified framework for visual generation tasks.


Furthermore, to support the training of \ours at scale, we have developed a robust infrastructure tailored for large-scale model training. Our approach incorporates advanced parallelism strategies~\citep{jacobs2023deepspeed, pytorch_fsdp} to manage memory efficiently during long-context training. Additionally, we employ ByteCheckpoint~\citep{wan2024bytecheckpoint} for high-performance checkpointing and integrate fault-tolerant mechanisms from MegaScale~\citep{jiang2024megascale} to ensure stability and scalability across large GPU clusters. These optimizations enable \ours to handle the computational and data challenges of generative modeling with exceptional efficiency and reliability.


We evaluate \ours on both text-to-image and text-to-video benchmarks to highlight its competitive advantages. For text-to-image generation, \ours-T2I demonstrates strong performance across multiple benchmarks, including T2I-CompBench~\citep{huang2023t2i-compbench}, GenEval~\citep{ghosh2024geneval}, and DPG-Bench~\citep{hu2024ella_dbgbench}, excelling in both visual quality and text-image alignment. In text-to-video benchmarks, \ours-T2V achieves state-of-the-art performance on the UCF-101~\citep{ucf101} zero-shot generation task. Additionally, \ours-T2V attains an impressive score of \textbf{84.85} on VBench~\citep{huang2024vbench}, securing the top position on the leaderboard (as of 2025-01-25) and surpassing several leading commercial text-to-video models. Qualitative results, illustrated in \Cref{fig:main-demo}, further demonstrate the superior quality of the generated media samples. These findings underscore \ours's effectiveness in multi-modal generation and its potential as a high-performing solution for both research and commercial applications.
\section{Related Work}

\subsection{Large 3D Reconstruction Models}
Recently, generalized feed-forward models for 3D reconstruction from sparse input views have garnered considerable attention due to their applicability in heavily under-constrained scenarios. The Large Reconstruction Model (LRM)~\cite{hong2023lrm} uses a transformer-based encoder-decoder pipeline to infer a NeRF reconstruction from just a single image. Newer iterations have shifted the focus towards generating 3D Gaussian representations from four input images~\cite{tang2025lgm, xu2024grm, zhang2025gslrm, charatan2024pixelsplat, chen2025mvsplat, liu2025mvsgaussian}, showing remarkable novel view synthesis results. The paradigm of transformer-based sparse 3D reconstruction has also successfully been applied to lifting monocular videos to 4D~\cite{ren2024l4gm}. \\
Yet, none of the existing works in the domain have studied the use-case of inferring \textit{animatable} 3D representations from sparse input images, which is the focus of our work. To this end, we build on top of the Large Gaussian Reconstruction Model (GRM)~\cite{xu2024grm}.

\subsection{3D-aware Portrait Animation}
A different line of work focuses on animating portraits in a 3D-aware manner.
MegaPortraits~\cite{drobyshev2022megaportraits} builds a 3D Volume given a source and driving image, and renders the animated source actor via orthographic projection with subsequent 2D neural rendering.
3D morphable models (3DMMs)~\cite{blanz19993dmm} are extensively used to obtain more interpretable control over the portrait animation. For example, StyleRig~\cite{tewari2020stylerig} demonstrates how a 3DMM can be used to control the data generated from a pre-trained StyleGAN~\cite{karras2019stylegan} network. ROME~\cite{khakhulin2022rome} predicts vertex offsets and texture of a FLAME~\cite{li2017flame} mesh from the input image.
A TriPlane representation is inferred and animated via FLAME~\cite{li2017flame} in multiple methods like Portrait4D~\cite{deng2024portrait4d}, Portrait4D-v2~\cite{deng2024portrait4dv2}, and GPAvatar~\cite{chu2024gpavatar}.
Others, such as VOODOO 3D~\cite{tran2024voodoo3d} and VOODOO XP~\cite{tran2024voodooxp}, learn their own expression encoder to drive the source person in a more detailed manner. \\
All of the aforementioned methods require nothing more than a single image of a person to animate it. This allows them to train on large monocular video datasets to infer a very generic motion prior that even translates to paintings or cartoon characters. However, due to their task formulation, these methods mostly focus on image synthesis from a frontal camera, often trading 3D consistency for better image quality by using 2D screen-space neural renderers. In contrast, our work aims to produce a truthful and complete 3D avatar representation from the input images that can be viewed from any angle.  

\subsection{Photo-realistic 3D Face Models}
The increasing availability of large-scale multi-view face datasets~\cite{kirschstein2023nersemble, ava256, pan2024renderme360, yang2020facescape} has enabled building photo-realistic 3D face models that learn a detailed prior over both geometry and appearance of human faces. HeadNeRF~\cite{hong2022headnerf} conditions a Neural Radiance Field (NeRF)~\cite{mildenhall2021nerf} on identity, expression, albedo, and illumination codes. VRMM~\cite{yang2024vrmm} builds a high-quality and relightable 3D face model using volumetric primitives~\cite{lombardi2021mvp}. One2Avatar~\cite{yu2024one2avatar} extends a 3DMM by anchoring a radiance field to its surface. More recently, GPHM~\cite{xu2025gphm} and HeadGAP~\cite{zheng2024headgap} have adopted 3D Gaussians to build a photo-realistic 3D face model. \\
Photo-realistic 3D face models learn a powerful prior over human facial appearance and geometry, which can be fitted to a single or multiple images of a person, effectively inferring a 3D head avatar. However, the fitting procedure itself is non-trivial and often requires expensive test-time optimization, impeding casual use-cases on consumer-grade devices. While this limitation may be circumvented by learning a generalized encoder that maps images into the 3D face model's latent space, another fundamental limitation remains. Even with more multi-view face datasets being published, the number of available training subjects rarely exceeds the thousands, making it hard to truly learn the full distibution of human facial appearance. Instead, our approach avoids generalizing over the identity axis by conditioning on some images of a person, and only generalizes over the expression axis for which plenty of data is available. 

A similar motivation has inspired recent work on codec avatars where a generalized network infers an animatable 3D representation given a registered mesh of a person~\cite{cao2022authentic, li2024uravatar}.
The resulting avatars exhibit excellent quality at the cost of several minutes of video capture per subject and expensive test-time optimization.
For example, URAvatar~\cite{li2024uravatar} finetunes their network on the given video recording for 3 hours on 8 A100 GPUs, making inference on consumer-grade devices impossible. In contrast, our approach directly regresses the final 3D head avatar from just four input images without the need for expensive test-time fine-tuning.


\section{DISCOVER: Data-driven Identification of Sub-activities via Clustering and Visualization for Enhanced Recognition}

\begin{figure}
    \centering
        \includegraphics[width=0.95\linewidth]{figures/main_image.pdf}
    \caption{
        Overview of \ToolName{}, a self-supervised system designed to \textit{discover} fine-grained human activities from unlabeled sensor data without relying on pre-segmentation, consisting of two main stages: clustering and labeling.
        After (0) slicing the raw data into continuous sliding windows of sensor activations \textit{ without assuming any pre-segmentation}, we (1) train a BERT model with mask language modeling task to encode the windows in an embedding space. We then use these embeddings to identify similar activity windows and (2) fine-tune a clustering model using SCAN loss, which results in assigning all data points to \( k \) clusters. We then (3) sample a handful of windows closest to each cluster centroid, replay them on 2D house layouts using our custom built visualization tool, and send these samples to a group of experts for annotation. With minimal labeling effort, we obtain custom granular activity labels for each cluster centroid, and (4) propagate them to the rest of the data points in each cluster. These custom labels are then (5) applied to the original dataset and can later be used for a set of specialized downstream tasks.
    } 
    \vspace*{-1em}
    \label{fig:main_image}
\end{figure}

To address the challenges of human activity recognition in smart homes, we introduce \ToolName{}, a data-driven approach that \textit{discovers} fine-grained human activities from unlabeled sensor data without relying on pre-segmentation. 
%Building on the insights from recent advances in self-supervised learning and clustering, 
\ToolName{} combines unsupervised feature extraction with an interactive annotation tool, enabling more granular and personalized activity labels.

We consider the problem of 
%\emph{self-supervised} 
human activity recognition in smart home environments. Given a time-series dataset of sensor readings collected over a period of time, divided into a collection of \(n\) sliding windows \(\{W_1, W_2, \dots, W_n\}\) of size \( l \), our goal is to assign each of these windows to an activity label. Formally, we seek to learn a mapping
\(
f: W_i \mapsto c_k,
\)
where \(c_k\) represents the activity cluster for window \(W_i\). A natural challenge in this setting is the lack of ground truth labels for the time windows.

\cref{fig:main_image} gives an overview of \ToolName{}, which consists of two main stages: clustering and labeling. The approach starts by pre-processing sensor data into sliding windows in step (0). We then pursue a active learning sub-activity discovery through the following five steps:

\begin{enumerate}
    \item \textbf{Encoder Pre-Training}: Create an embedding representation of sensor readings by training a BERT transformer model with a self-supervised, mask language modeling task. Use these embeddings to create a set of \( m \) nearest neighbors for each window \( W_i \).
    
    \item \textbf{Clustering Model Fine-Tuning}: Fine-tune the pre-trained BERT model using the SCAN loss function \cite{scan2020}, which partitions all windows \( W_i \) into \( k \) distinct clusters, assigning each \( W_i \) to one of them.

    \item \textbf{Centroid Annotation}: Select a handful of windows \( W_i\) closest to each cluster centroid, and send them to expert annotators to have them assign a label to each sample using \ToolName{} custom built visualization tool.
    
    \item \textbf{Label propagation}: Propagate the centroid labels to every data point in the cluster. 
    
    \item \textbf{Re-Annotation of the Original Time-Series Data}: Apply these cluster activity labels to the original time-series data.
\end{enumerate}

\subsection{Encoder Pre-Training}

\begin{figure}
    \centering
        \includegraphics[width=0.95\linewidth]{figures/training_pipeline.pdf}
    \vspace*{-1em}
    \caption{
        \ToolName{} approach -- model training pipeline, consisting of (1) Encoder Pre-Training, and (2) Clustering Model Fine-Tuning. 
        \ToolName{} first trains a BERT model using a Masked Language Modeling head in (1) to obtain initial embeddings for each window \( W_i \). In step 2(a) it uses these embeddings to identify similar activity windows and pairs them together as a a new training set. It then continues training the pre-trained BERT base model with a SCAN loss (2(b)). In the end, the trained SCAN model assigns a cluster \( c_k \) to each input sequence \( W_i \) (2(c)).
    }
    \vspace*{-1em}
    \label{fig:training_pipeline}
\end{figure}

The first step is to learn an informative representation for raw sensor sequences. In this subsection, we describe how we adapt a BERT model through a masked language modeling (MLM) objective to capture context-rich representations of ambient sensor data without any labels.

We build on the approach from \cite{hiremath2022bootstrapping}, which employs a pre-trained BERT Transformer model to derive embeddings for sensor sequences by feeding in new, domain-specific tokens and training for several epochs. However, unlike \cite{hiremath2022bootstrapping}, our pipeline does not rely on pre-segmented data, staying faithful to a more realistic, fully unsupervised scenario. \cref{fig:training_pipeline}(1) shows a diagram of this process. First, we load the original BERT model that was pre-trained on text corpora (BooksCorpus and English Wikipedia) \cite{bert2019} from HuggingFace \cite{huggingface2020} and extend the original vocabulary of size \texttt{30,522} WordPiece tokens by adding sensors and their readings as new tokens to the original vocabulary (which evaluates to roughly \texttt{100} new tokens per dataset). For example, sensor labeled \texttt{M1} and its output signal \texttt{ON} are concatenated into a new token \texttt{M1\_ON}, assigned a new token ID, and initialized with random \texttt{768}-dimensional embedding. Once all training input sequences are processed and encoded, we train BERT model using a self-supervised MLM objective, defined below. 

For every input sequence \( W_i = \{d_1, d_2, \dots, d_l\} \) of sensor tokens \( d \), we choose a proportion \( p \) of these readings to mask. Following \cite{hiremath2020deriving} and the original BERT paper \cite{bert2019}, we set \( p = 0.15 \). Let \( M \subset W_i \) be a subset of positions selected for masking, where \(\lvert M \rvert = \lfloor p \times l \rfloor\). We replace each token in \( M \) with a special \texttt{[MASK]} token, producing a masked sequence \(\hat{W_i}\). The model is then trained to predict the original tokens in \( M \) based on the surrounding context provided by \(\hat{W_i}\).

The training loss \( \mathcal{L}_{MLM} \) is defined as:
\[
\mathcal{L}_{MLM} = - \frac{1}{|M|} \sum_{x \in M} \log P(d_x \mid \hat{W_i})
\]

where \( P(d_x \mid \hat{W_i}) \) is the probability that \(x\)-th masked token \( d \) is correctly predicted by the model.

This loss function incorporates the context from the surrounding, unmasked tokens (i.e., sensor readings) in each sequence. By introducing new tokens for sensor readings (e.g., \texttt{M1\_ON}) into the vocabulary and initializing their embeddings randomly, the model is able to integrate sensor-specific information into the learned representations during pre-training.
Once the training is done, we obtain an embedding for each input sequence \( W_i \) by removing the MLM head and extracting the \texttt{768}-dimensional vector of a special \texttt{[CLS]} token. This \texttt{[CLS]} ("classification") token is specifically designed to represent the entire sequence in a single embedding, a common practice in BERT-based architectures to obtain a fixed-length representation of any input sequence \cite{bert2019}.

\subsection{Fine-Tuning the Clustering Model}
Once we obtain an embedding for each window \(W_i\) in the training data, we use these embeddings to measure the similarity between individual sequence vectors, build a "neighbors" dataset, and train a clustering model using this data.

First, for every input sequence we want to find and store a small set of examples that are most similar to it in the BERT embedding space (\cref{fig:training_pipeline} (2a)). Let \( \text{sim}(W_q, W_r) \) represent the cosine similarity between two embedding vectors \( W_q \) and \( W_r \), defined as:

\[
\text{sim}(W_q, W_r) = \frac{W_q \cdot W_r}{\|W_q\| \|W_r\|}
\]

\noindent For every input sequence \( W_i \), we identify its \( m \)-nearest neighbors using this cosine similarity. Following \cite{scan2020}, we set \( m \) equal to 20. These neighbors will be used as training examples in the following step where the SCAN loss function will be forcing them to be in the same cluster \( c_k\)

Then, we further fine-tune the re-trained BERT model from the previous step by redefining its training objective (\cref{fig:training_pipeline} (2b)). To achieve that, we replace the MLM training head and its cross-entropy loss function with a clustering head paired with the SCAN loss \cite{scan2020}. \textbf{SCAN loss} consists of two components, an \textbf{instance-level contrastive loss} and a \textbf{cluster-level entropy loss}, defined as follows:

\begin{itemize}

    \item 
    \textbf{Instance-Level Contrastive Loss}: Encourages every input sequence and its neighbors to belong to the same cluster. For a sequence \( W_i \), let \( M(W_i) \) denote its set of \( m \)-nearest neighbors. The contrastive loss \( \mathcal{L}_{\text{contrastive}} \) is defined as:

    \[
    \mathcal{L}_{\text{contrastive}} = - \frac{1}{n} \sum_{W_i} \frac{1}{|M(W_i)|} \sum_{W_j \in M(W_i)} \log P(c_i = c_j)
    \]
    
    where \( P(c_i = c_j) \) is the probability that \( W_i \) and \( W_j \) are assigned to the same cluster.

    \item \textbf{Cluster-Level Entropy Loss}: Encourages balanced cluster assignments by ensuring that the distribution of cluster labels is uniform. Let \( P_k \) denote the probability of sequences being assigned to cluster \( k \). The entropy loss \( \mathcal{L}_{\text{entropy}} \) is defined as:
    
    \[
    \mathcal{L}_{\text{entropy}} = - \sum_{k=1}^K P_k \log P_k
    \]

    \item \textbf{Combined SCAN Loss}: The total SCAN loss \( \mathcal{L}_{\text{SCAN}} \) is a sum of the two components:

    \[
    \mathcal{L}_{\text{SCAN}} = \mathcal{L}_{\text{contrastive}} + \lambda \mathcal{L}_{\text{entropy}}
    \]
    
    where \( \lambda \) controls the weight of the entropy loss. We use the default value of \( \lambda = 2\) \cite{scan2020}.

\end{itemize}

SCAN loss enables the BERT model to assign cluster labels 
\( c_i \in \{1, 2, \dots, k\} \) to each sequence \( W_i \). The clustering head is initialized 
with random labels and iteratively learns to group similar sequences together, thus refining the underlying embeddings into more distinct clusters. Along with the cluster labels (\cref{fig:training_pipeline} (2c)), the trained model provides a probability \( P(c_i = k) \), indicating how confidently each sequence 
\( W_i \) is assigned to cluster \( k \). In the end, every input window receives both a cluster assignment and a corresponding probability distribution over all clusters.


\subsection{Cluster Centroid Annotation}

\begin{figure}
    \centering
        \includegraphics[width=0.9\linewidth]{figures/house_map_replays.pdf}
    \caption{
    \ToolName{} custom-built interactive in-browser annotation tool for reviewing sensor activation sequences. The tool displays a 2D house layout, allowing annotators to replay sequences temporally, observe contextual and spatial details, and assign labels via a drop-down menu. The example above shows a sequence of sensor activations following a resident walking to the guest bathroom, that an annotator can label as "Guest Bathroom: Walking In" from the drop-down menu.
    }
    \label{fig:viz_tool}
\end{figure}

Given the set of clusters learned in the previous stage, the next goal is to interpret and contextualize them by assigning meaningful labels. Since the model has so far operated without any supervision, no ready-made labels exist to inform downstream analyses or practical applications. By annotating a small number of representative samples from each cluster, we can translate the resulting clusters into actionable units—such as identifying specific sub-activities or understanding when and where a resident might be performing a certain activity. This process not only validates the clusters internally but also bridges the gap between the automatic discovery of patterns and real-world interpretability.

To minimize the number of sequences we need to review, we will only use \( N \) samples from each cluster that have the highest probability of belonging to that cluster. The probability \( P(c_i = k) \) is taken directly from the SCAN model output. In step (3) in \cref{fig:main_image}, we select \( N \times k \) samples and review them manually using the custom tool that we developed, shown in \cref{fig:viz_tool}. This interactive in-browser annotation tool shows a detailed layout of a given house floor plan and allows to playback the sensor activation sequences temporally, and assign a label from a drop-down menu. This tool lets the annotator see both the contextual information about that sequence (e.g., location of the activity and its surrounding areas), and the temporal dimension (e.g., whether the activity was static, or if it triggered multiple sensor activations throughout the house). For example, \cref{fig:viz_tool} displays a sequence of sensor activations representing a resident walking through the house to the guest bathroom. The annotator can label this sequence as "Guest Bathroom: Walking In." \ToolName{} visualization tool simplifies and speeds up the annotation process and allows for faster iterations. 


\subsection{Label Propagation}
\label{sec:label_propagation}

Once each cluster is assigned a label based on its centroid samples, we propagate labels to all remaining sequences belonging to the cluster. This process creates a fully labeled dataset from initially unlabeled sequences. By grouping each sequence with its dominant cluster label, we enable efficient large-scale annotation of resident activities.

\subsection{Re-Annotation of the Original Time-Series Data}
\label{sec:reannotation}

Having assigned labels to each sequence, we then apply these annotations back to the original time-series data, preserving their chronological order. This comprehensive labeling allows researchers to analyze the temporal progression of activities and identify patterns or shifts in behavior over days, weeks, or longer periods. In doing so, our pipeline not only re-annotates large portions of unlabeled data but also lays the groundwork for an array of downstream tasks, such as behavior monitoring, anomaly detection, and personalized interventions. 

\section{Experimental Results}
In this section, we present the main results in~\secref{sec:main}, followed by ablation studies on key design choices in~\secref{sec:ablation}.

\begin{table*}[t]
\renewcommand\arraystretch{1.05}
\centering
\setlength{\tabcolsep}{2.5mm}{}
\begin{tabular}{l|l|c|cc|cc}
type & model     & \#params      & FID$\downarrow$ & IS$\uparrow$ & Precision$\uparrow$ & Recall$\uparrow$ \\
\shline
GAN& BigGAN~\cite{biggan} & 112M & 6.95  & 224.5       & 0.89 & 0.38     \\
GAN& GigaGAN~\cite{gigagan}  & 569M      & 3.45  & 225.5       & 0.84 & 0.61\\  
GAN& StyleGan-XL~\cite{stylegan-xl} & 166M & 2.30  & 265.1       & 0.78 & 0.53  \\
\hline
Diffusion& ADM~\cite{adm}    & 554M      & 10.94 & 101.0        & 0.69 & 0.63\\
Diffusion& LDM-4-G~\cite{ldm}   & 400M  & 3.60  & 247.7       & -  & -     \\
Diffusion & Simple-Diffusion~\cite{diff1} & 2B & 2.44 & 256.3 & - & - \\
Diffusion& DiT-XL/2~\cite{dit} & 675M     & 2.27  & 278.2       & 0.83 & 0.57     \\
Diffusion&L-DiT-3B~\cite{dit-github}  & 3.0B    & 2.10  & 304.4       & 0.82 & 0.60    \\
Diffusion&DiMR-G/2R~\cite{liu2024alleviating} &1.1B& 1.63& 292.5& 0.79 &0.63 \\
Diffusion & MDTv2-XL/2~\cite{gao2023mdtv2} & 676M & 1.58 & 314.7 & 0.79 & 0.65\\
Diffusion & CausalFusion-H$^\dag$~\cite{deng2024causal} & 1B & 1.57 & - & - & - \\
\hline
Flow-Matching & SiT-XL/2~\cite{sit} & 675M & 2.06 & 277.5 & 0.83 & 0.59 \\
Flow-Matching&REPA~\cite{yu2024representation} &675M& 1.80 & 284.0 &0.81 &0.61\\    
Flow-Matching&REPA$^\dag$~\cite{yu2024representation}& 675M& 1.42&  305.7& 0.80& 0.65 \\
\hline
Mask.& MaskGIT~\cite{maskgit}  & 227M   & 6.18  & 182.1        & 0.80 & 0.51 \\
Mask. & TiTok-S-128~\cite{yu2024image} & 287M & 1.97 & 281.8 & - & - \\
Mask. & MAGVIT-v2~\cite{yu2024language} & 307M & 1.78 & 319.4 & - & - \\ 
Mask. & MaskBit~\cite{weber2024maskbit} & 305M & 1.52 & 328.6 & - & - \\
\hline
AR& VQVAE-2~\cite{vqvae2} & 13.5B    & 31.11           & $\sim$45     & 0.36           & 0.57          \\
AR& VQGAN~\cite{vqgan}& 227M  & 18.65 & 80.4         & 0.78 & 0.26   \\
AR& VQGAN~\cite{vqgan}   & 1.4B     & 15.78 & 74.3   & -  & -     \\
AR&RQTran.~\cite{rq}     & 3.8B    & 7.55  & 134.0  & -  & -    \\
AR& ViTVQ~\cite{vit-vqgan} & 1.7B  & 4.17  & 175.1  & -  & -    \\
AR & DART-AR~\cite{gu2025dart} & 812M & 3.98 & 256.8 & - & - \\
AR & MonoFormer~\cite{zhao2024monoformer} & 1.1B & 2.57 & 272.6 & 0.84 & 0.56\\
AR & Open-MAGVIT2-XL~\cite{luo2024open} & 1.5B & 2.33 & 271.8 & 0.84 & 0.54\\
AR&LlamaGen-3B~\cite{llamagen}  &3.1B& 2.18& 263.3 &0.81& 0.58\\
AR & FlowAR-H~\cite{flowar} & 1.9B & 1.65 & 296.5 & 0.83 & 0.60\\
AR & RAR-XXL~\cite{yu2024randomized} & 1.5B & 1.48 & 326.0 & 0.80 & 0.63 \\
\hline
MAR & MAR-B~\cite{mar} & 208M & 2.31 &281.7 &0.82 &0.57 \\
MAR & MAR-L~\cite{mar} &479M& 1.78 &296.0& 0.81& 0.60 \\
MAR & MAR-H~\cite{mar} & 943M&1.55& 303.7& 0.81 &0.62 \\
\hline
VAR&VAR-$d16$~\cite{var}   & 310M  & 3.30& 274.4& 0.84& 0.51    \\
VAR&VAR-$d20$~\cite{var}   &600M & 2.57& 302.6& 0.83& 0.56     \\
VAR&VAR-$d30$~\cite{var}   & 2.0B      & 1.97  & 323.1 & 0.82 & 0.59      \\
\hline
\modelname& \modelname-B    &172M   &1.72&280.4&0.82&0.59 \\
\modelname& \modelname-L   & 608M   & 1.28& 292.5&0.82&0.62\\
\modelname& \modelname-H    & 1.1B    & 1.24 &301.6&0.83&0.64\\
\end{tabular}
\caption{
\textbf{Generation Results on ImageNet-256.}
Metrics include Fréchet Inception Distance (FID), Inception Score (IS), Precision, and Recall. $^\dag$ denotes the use of guidance interval sampling~\cite{guidance}. The proposed \modelname-H achieves a state-of-the-art 1.24 FID on the ImageNet-256 benchmark without relying on vision foundation models (\eg, DINOv2~\cite{dinov2}) or guidance interval sampling~\cite{guidance}, as used in REPA~\cite{yu2024representation}.
}\label{tab:256}
\end{table*}

\subsection{Main Results}
\label{sec:main}
We conduct experiments on ImageNet~\cite{deng2009imagenet} at 256$\times$256 and 512$\times$512 resolutions. Following prior works~\cite{dit,mar}, we evaluate model performance using FID~\cite{fid}, Inception Score (IS)~\cite{is}, Precision, and Recall. \modelname is trained with the same hyper-parameters as~\cite{mar,dit} (\eg, 800 training epochs), with model sizes ranging from 172M to 1.1B parameters. See Appendix~\secref{sec:sup_hyper} for hyper-parameter details.





\begin{table}[t]
    \centering
    \begin{tabular}{c|c|c|c}
      model    &  \#params & FID$\downarrow$ & IS$\uparrow$ \\
      \shline
      VQGAN~\cite{vqgan}&227M &26.52& 66.8\\
      BigGAN~\cite{biggan}& 158M&8.43 &177.9\\
      MaskGiT~\cite{maskgit}& 227M&7.32& 156.0\\
      DiT-XL/2~\cite{dit} &675M &3.04& 240.8 \\
     DiMR-XL/3R~\cite{liu2024alleviating}& 525M&2.89 &289.8 \\
     VAR-d36~\cite{var}  & 2.3B& 2.63 & 303.2\\
     REPA$^\ddagger$~\cite{yu2024representation}&675M &2.08& 274.6 \\
     \hline
     \modelname-L & 608M&1.70& 281.5 \\
    \end{tabular}
    \caption{
    \textbf{Generation Results on ImageNet-512.} $^\ddagger$ denotes the use of DINOv2~\cite{dinov2}.
    }
    \label{tab:512}
\end{table}

\noindent\textbf{ImageNet-256.}
In~\tabref{tab:256}, we compare \modelname with previous state-of-the-art generative models.
Out best variant, \modelname-H, achieves a new state-of-the-art-performance of 1.24 FID, outperforming the GAN-based StyleGAN-XL~\cite{stylegan-xl} by 1.06 FID, masked-prediction-based MaskBit~\cite{maskgit} by 0.28 FID, AR-based RAR~\cite{yu2024randomized} by 0.24 FID, VAR~\cite{var} by 0.73 FID, MAR~\cite{mar} by 0.31 FID, and flow-matching-based REPA~\cite{yu2024representation} by 0.18 FID.
Notably, \modelname does not rely on vision foundation models~\cite{dinov2} or guidance interval sampling~\cite{guidance}, both of which were used in REPA~\cite{yu2024representation}, the previous best-performing model.
Additionally, our lightweight \modelname-B (172M), surpasses DiT-XL (675M)~\cite{dit} by 0.55 FID while achieving an inference speed of 9.8 images per second—20$\times$ faster than DiT-XL (0.5 images per second). Detailed speed comparison can be found in Appendix \ref{sec:speed}.



\noindent\textbf{ImageNet-512.}
In~\tabref{tab:512}, we report the performance of \modelname on ImageNet-512.
Similarly, \modelname-L sets a new state-of-the-art FID of 1.70, outperforming the diffusion based DiT-XL/2~\cite{dit} and DiMR-XL/3R~\cite{liu2024alleviating} by a large margin of 1.34 and 1.19 FID, respectively.
Additionally, \modelname-L also surpasses the previous best autoregressive model VAR-d36~\cite{var} and flow-matching-based REPA~\cite{yu2024representation} by 0.93 and 0.38 FID, respectively.




\noindent\textbf{Qualitative Results.}
\figref{fig:qualitative} presents samples generated by \modelname (trained on ImageNet) at 512$\times$512 and 256$\times$256 resolutions. These results highlight \modelname's ability to produce high-fidelity images with exceptional visual quality.

\begin{figure*}
    \centering
    \vspace{-6pt}
    \includegraphics[width=1\linewidth]{figures/qualitative.pdf}
    \caption{\textbf{Generated Samples.} \modelname generates high-quality images at resolutions of 512$\times$512 (1st row) and 256$\times$256 (2nd and 3rd row).
    }
    \label{fig:qualitative}
\end{figure*}

\subsection{Ablation Studies}
\label{sec:ablation}
In this section, we conduct ablation studies using \modelname-B, trained for 400 epochs to efficiently iterate on model design.

\noindent\textbf{Prediction Entity X.}
The proposed \modelname extends next-token prediction to next-X prediction. In~\tabref{tab:X}, we evaluate different designs for the prediction entity X, including an individual patch token, a cell (a group of surrounding tokens), a subsample (a non-local grouping), a scale (coarse-to-fine resolution), and an entire image.

Among these variants, cell-based \modelname achieves the best performance, with an FID of 2.48, outperforming the token-based \modelname by 1.03 FID and surpassing the second best design (scale-based \modelname) by 0.42 FID. Furthermore, even when using standard prediction entities such as tokens, subsamples, images, or scales, \modelname consistently outperforms existing methods while requiring significantly fewer parameters. These results highlight the efficiency and effectiveness of \modelname across diverse prediction entities.






\begin{table}[]
    \centering
    \scalebox{0.92}{
    \begin{tabular}{c|c|c|c|c}
        model & \makecell[c]{prediction\\entity} & \#params & FID$\downarrow$ & IS$\uparrow$\\
        \shline
        LlamaGen-L~\cite{llamagen} & \multirow{2}{*}{token} & 343M & 3.80 &248.3\\
        \modelname-B& & 172M&3.51&251.4\\
        \hline
        PAR-L~\cite{par} & \multirow{2}{*}{subsample}& 343M & 3.76 & 218.9\\
        \modelname-B&  &172M& 3.58&231.5\\
        \hline
        DiT-L/2~\cite{dit}& \multirow{2}{*}{image}& 458M&5.02&167.2 \\
         \modelname-B& & 172M&3.13&253.4 \\
        \hline
        VAR-$d16$~\cite{var} & \multirow{2}{*}{scale} & 310M&3.30 &274.4\\
        \modelname-B& &172M&2.90&262.8\\
        \hline
        \baseline{\modelname-B}& \baseline{cell} & \baseline{172M}&\baseline{2.48}&\baseline{269.2} \\
    \end{tabular}
    }
    \caption{\textbf{Ablation on Prediction Entity X.} Using cells as the prediction entity outperforms alternatives such as tokens or entire images. Additionally, under the same prediction entity, \modelname surpasses previous methods, demonstrating its effectiveness across different prediction granularities. }%
    \label{tab:X}
\end{table}

\noindent\textbf{Cell Size.}
A prediction entity cell is formed by grouping spatially adjacent $k\times k$ tokens, where a larger cell size incorporates more tokens and thus captures a broader context within a single prediction step.
For a $256\times256$ input image, the encoded continuous latent representation has a spatial resolution of $16\times16$. Given this, the image can be partitioned into an $m\times m$ grid, where each cell consists of $k\times k$ neighboring tokens. As shown in~\tabref{tab:cell}, we evaluate different cell sizes with $k \in \{1,2,4,8,16\}$, where $k=1$ represents a single token and $k=16$ corresponds to the entire image as a single entity. We observe that performance improves as $k$ increases, peaking at an FID of 2.48 when using cell size $8\times8$ (\ie, $k=8$). Beyond this, performance declines, reaching an FID of 3.13 when the entire image is treated as a single entity.
These results suggest that using cells rather than the entire image as the prediction unit allows the model to condition on previously generated context, improving confidence in predictions while maintaining both rich semantics and local details.





\begin{table}[t]
    \centering
    \scalebox{0.98}{
    \begin{tabular}{c|c|c|c}
    cell size ($k\times k$ tokens) & $m\times m$ grid & FID$\downarrow$ & IS$\uparrow$ \\
       \shline
       $1\times1$ & $16\times16$ &3.51&251.4 \\
       $2\times2$ & $8\times8$ & 3.04& 253.5\\
       $4\times4$ & $4\times4$ & 2.61&258.2 \\
       \baseline{$8\times8$} & \baseline{$2\times2$} & \baseline{2.48} & \baseline{269.2}\\
       $16\times16$ & $1\times1$ & 3.13&253.4  \\
    \end{tabular}
    }
    \caption{\textbf{Ablation on the cell size.}
    In this study, a $16\times16$ continuous latent representation is partitioned into an $m\times m$ grid, where each cell consits of $k\times k$ neighboring tokens.
    A cell size of $8\times8$ achieves the best performance, striking an optimal balance between local structure and global context.
    }
    \label{tab:cell}
\end{table}



\begin{table}[t]
    \centering
    \scalebox{0.95}{
    \begin{tabular}{c|c|c|c}
      previous cell & noise time step &  FID$\downarrow$ & IS$\uparrow$ \\
       \shline
       clean & $t_i=0, \forall i<n$& 3.45& 243.5\\
       increasing noise & $t_1<t_2<\cdots<t_{n-1}$& 2.95&258.8 \\
       decreasing noise & $t_1>t_2>\cdots>t_{n-1}$&2.78 &262.1 \\
      \baseline{random noise}  & \baseline{no constraint} &\baseline{2.48} & \baseline{269.2}\\
    \end{tabular}
    }
    \caption{
    \textbf{Ablation on Noisy Context Learning.}
    This study examines the impact of noise time steps ($t_1, \cdots, t_{n-1} \subset [0, 1]$) in previous entities ($t=0$ represents pure Gaussian noise).
    Conditioning on all clean entities (the ``clean'' variant) results in suboptimal performance.
    Imposing an order on noise time steps, either ``increasing noise'' or ``decreasing noise'', also leads to inferior results. The best performance is achieved with the "random noise" setting, where no constraints are imposed on noise time steps.
    }
    \label{tab:ncl}
\end{table}


\noindent\textbf{Noisy Context Learning.}
During training, \modelname employs Noisy Context Learning (NCL), predicting $X_n$ by conditioning on all previous noisy entities, unlike Teacher Forcing.
The noise intensity of previous entities is contorlled by noise time steps $\{t_1, \dots, t_{n-1}\} \subset [0, 1]$, where $t=0$ corresponds to pure Gaussian noise.
We analyze the impact of NCL in~\tabref{tab:ncl}.
When conditioning on all clean entities (\ie, the ``clean'' variant, where $t_i=0, \forall i<n$), which is equivalent to vanilla AR (\ie, Teacher Forcing), the suboptimal performance is obtained.
We also evaluate two constrained noise schedules: the ``increasing noise'' variant, where noise time steps increase over AR steps ($t_1<t_2< \cdots < t_{n-1}$), and the `` decreasing noise'' variant, where noise time steps decrease ($t_1>t_2> \cdots > t_{n-1}$).
While both settings improve over the ``clean'' variant, they remain inferior to our final ``random noise'' setting, where no constraints are imposed on noise time steps, leading to the best performance.




        


\section{Discussion and Conclusion}

% \begin{quote}
% \textit{"We believe it is unethical for social workers not to learn... about technology-mediated social work."} (\citeauthor{singer_ai_2023}, 2023)
% \end{quote}

In this study, we uncovered multiple ways in which GenAI can be used in social service practice. While some concerns did arise, practitioners by and large seemed optimistic about the possibilities of such tools, and that these issues could be overcome. We note that while most participants found the tool useful, it was far from perfect in its outputs. This is not surprising, since it was powered by a generic LLM rather than one fine-tuned for social service case management. However, despite these inadequacies, our participants still found many uses for most of the tool's outputs. Many flaws pointed out by our participants related to highly contextualised, local knowledge. To tune an AI system for this would require large amounts of case files as training data; given the privacy concerns associated with using client data, this seems unlikely to happen in the near future. What our study shows, however, is that GenAI systems need not aim to be perfect to be useful to social service practitioners, and can instead serve as a complement to the critical "human touch" in social service.

We draw both inspiration and comparisons with prior work on AI in other settings. Studies on creative writing tools showed how the "uncertainty" \cite{wan2024felt} and "randomness" \cite{clark2018creative} of AI outputs aid creativity. Given the promise that our tool shows in aiding brainstorming and discussion, future social service studies could consider AI tools explicitly geared towards creativity - for instance, providing side-by-side displays of how a given case would fit into different theoretical frameworks, prompting users to compare, contrast, and adopt the best of each framework; or allowing users to play around with combining different intervention modalities to generate eclectic (i.e. multi-modal) interventions.

At the same time, the concept of supervision creates a different interaction paradigm to other uses of AI in brainstorming. Past work (e.g. \cite{shaer2024ai}) has explored the use of GenAI for ideation during brainstorming sessions, wherein all users present discuss the ideas generated by the system. With supervision in social service practice, however, there is a marked information and role asymmetry: supervisors may not have had the time to fully read up on their supervisee's case beforehand, yet have to provide guidance and help to the latter. We suggest that GenAI can serve a dual purpose of bringing supervisors up to speed quickly by summarising their supervisee's case data, while simultaneously generating a list of discussion and talking points that can improve the quality of supervision. Generalising, this interaction paradigm has promise in many other areas: senior doctors reviewing medical procedures with newer ones \cite{snowdon2017does} could use GenAI to generate questions about critical parts of a procedure to ask the latter, confirming they have been correctly understood or executed; game studio directors could quickly summarise key developmental pipeline concerns to raise at meetings and ensure the team is on track; even in academia, advisors involved in rather too many projects to keep track of could quickly summarise each graduate student's projects and identify potential concerns to address at their next meeting.

In closing, we are optimistic about the potential for GenAI to significantly enhance social service practice and the quality of care to clients. Future studies could focus on 1) longitudinal investigations into the long-term impact of GenAI on practitioner skills, client outcomes, and organisational workflows, and 2) optimising workflows to best integrate GenAI into casework and supervision, understanding where best to harness the speed and creativity of such systems in harmony with the experience and skills of practitioners at all levels.

% GenAI here thus serves as a tool that supervisors can use before rather then using the session, taking just a few minutes of their time to generate a list of discussion points with their supervisees.

% Traditional brainstorming comes up with new things that users discuss. In supervision, supervisors can use AI to more efficiently generate talking points with their supervisees. These are generally not novel ideas, since an experienced worker would be able to come up with these on their own. However, the interesting and novel use of AI here is in its use as a preparation tool, efficiently generating talking and discussion points, saving supervisors' time in preparing for a session, while still serving as a brainstorming tool during the session itself.

% The idea of embracing imperfect AI echoes the findings of \citeauthor{bossen2023batman} (2023) in a clinical decision setting, which examined the successful implementation of an "error-prone but useful AI tool". This study frames human-AI collaboration as "Batman and Robin", where AI is a useful but ultimately less skilled sidekick that plays second fiddle to Batman. This is similar to \citeauthor{yang2019unremarkable}'s (2019) idea of "unremarkable AI", systems designed to be unobtrusive and only visible to the user when they add some value. As compared to \citeauthor{bossen2023batman}, however, we see fewer instances of our AI system producing errors, and more examples of it providing learning and collaborative opportunities and other new use cases. We build on the idea of "complementary performance" \cite{bansal2021does}, which discusses how the unique expertise of AI enhances human decision-making performance beyond what humans can achieve alone. Beyond decision-making, GenAI can now enable "complementary work patterns", where the nature of its outputs enables humans to carry out their work in entirely new ways. Our study suggests that rather being a sidekick - Robin - AI is growing into the role of a "second Batman" or "AI-Batman": an entity with distinct abilities and expertise from humans, and that contributes in its own unique way. There is certainly still a time and place for unremarkable AI, but exploring uses beyond that paradigm uncovers entirely new areas of system design.

% % \cite{gero2022sparks} found AI to be useful for science writers to translate ideas already in their head into words, and to provide new perspectives to spark further inspiration. \textit{But how is ours different from theirs?}

% \subsection{New Avenues of Human-AI Collaboration}
% \label{subsubsec:discussionhaicollaboration}

% Past HCI literature in other areas \cite{nah2023generative} has suggested that GenAI represents a "leap" \cite{singh2023hide} in human-AI collaboration, 
% % Even when an AI system sometimes produces irrelevant outputs, it can still provide users 
% % Such systems have been proposed as ways to 
% helping users discover new viewpoints \cite{singh2023hide}, scour existing literature to suggest new hypotheses 
% \cite{cascella2023evaluating} and answer questions \cite{biswas2023role}, stimulate their cognitive processes \cite{memmert2023towards}, and overcome "writer's block" \cite{singh2023hide, cooper2023examining} (particularly relevant to SSPs and the vast amount of writing required of them). Our study finds promise for AI to help SSPs in all of these areas. By nature of being more verbose and capable of generating large amounts of content, GenAI seems to create a new way in which AI can complement human work and expertise. Our system, as LLMs tend to do, produced a lot of "bullshit" (S6) \cite{frankfurt2005bullshit} - superficially true statements that were often only "tangentially related" and "devoid of meaning" \cite{halloran2023ai}. Yet, many participants cited the page-long analyses and detailed multi-step intervention plans generated by the AI system to be a good starting point for further discussion, both to better conceptualize a particular case and to facilitate general worker growth and development. Almost like throwing mud at a wall to see what sticks, GenAI can quickly produce a long list of ideas or information, before the worker glances through it and quickly identifies the more interesting points to discuss. Playing the proposed role as a "scaffold" for further work \cite{cooper2023examining}, GenAI, literally, generates new opportunities for novel and more effective processes and perspectives that previous systems (e.g., PRMs) could not. This represents an entirely new mode of human-AI collaboration.
% % This represents a new mode of collaboration not possible with the largely quantitative AI models (like PRMs) of the past.

% Our work therefore supports and extends prior research that have postulated the the potential of AI's shifting roles from decision-maker to human-supporter \cite{wang_human-human_2020}. \citeauthor{siemon2022elaborating} (2022) suggests the role of AI as a "creator" or "coordinator", rather than merely providing "process guidance" \cite{memmert2023towards} that does not contribute to brainstorming. Similarly, \citeauthor{memmert2023towards} (2023) propose GenAI as a step forward from providing meta-level process guidance (i.e. facilitating user tasks) to actively contributing content and aiding brainstorming. We suggest that beyond content-support, AI can even create new work processes that were not possible without GenAI. In this sense, AI has come full circle, becoming a "meta-facilitator".

% % --- WIP BELOW ---

% % Our work echoes and extends previous research on HAI collaboration in tasks requiring a human touch. \cite{gero2023social} found AI to be a safe space for creative writers to bounce ideas off of and document their inner thoughts. \cite{dhillon2024shaping} reference the idea of appropriate scaffolding in argumentative writing, where the user is providing with guidance appropriate for their competency level, and also warns of decreased satisfaction and ownership from AI use. 

% Separately, we draw parallels with the field of creative writing, where HAI collaboration has been extensively researched. Writers note the "irreducibly human" aspect of creativity in writing \cite{gero2023social}, similar to the "human touch" core to social service practice (D1); both groups therefore expressed few concerns about AI taking over core aspects of their jobs. Another interesting parallel was how writers often appreciated the "uncertainty" \cite{wan2024felt} and "randomness" \cite{clark2018creative} of AI systems, which served as a source of inspiration. This echoes the idea of "imperfect AI" "expanding [the] perspective[s]" (S4) of our participants when they simply skimmed through what the AI produced. \cite{wan2024felt} cited how the "duality of uncertainty in the creativity process advances the exploration of the imperfection of GenAI models". While social service work is not typically regarded as "creative", practitioners nonetheless go through processes of ideation and iteration while formulating a case. Our study showed hints of how AI can help with various forms of ideation, but, drawing inspiration from creative writing tools, future studies could consider designs more explicitly geared towards creativity - for instance, by attempting to fit a given case into a number of different theories or modalities, and displaying them together for the user to consider. While many of these assessments may be imperfect or even downnright nonsensical, they may contain valuable ideas and new angles on viewing the case that the practitioner can integrate into their own assessment.

% % \cite{foong2024designing}, describing the design of caregiver-facing values elicitation tools, cites the "twin scenarios" that caregivers face - private use, where they might use a tool to discover their patient's values, and collaborative use, where they discuss the resulting values with other parties close to the patient. This closely mirrors how SSPs in our study reference both individual and collaborative uses of our tool. Unlike in \cite{foong2024designing}, however, we do not see a resulting need to design a "staged approach" with distinct interface features for both stages.

% % --- END OF WIP ---

% Having mentioned algorithm aversion previously, we also make a quick point here on the other end of the spectrum - automation bias, or blind trust in an automated system \cite{brown2019toward}. LLMs risk being perceived as an "ultimate epistemic authority" \cite{cooper2023examining} due to their detailed, life-like outputs. While automation bias has been studied in many contexts, including in the social sector or adjacent areas, we suggest that the very nature of GenAI systems fundamentally inhibits automation bias. The tendency of GenAI to produce verbose, lengthy explanations prompts users to read and think through the machine's judgement before accepting it, bringing up opportunities to disagree with the machine's opinion. This guards against blind acceptance of the system's recommendations, particularly in the culture of a social work agency where constant dialogue - including discussing AI-produced work - is the norm.


% % : Perception of AI in Social Service Work ??

% \subsection{Redefining the Boundary}
% \label{subsubsec:discussiontheoretical}

% As \citeauthor{meilvang_working_2023} (2023) describes, the social service profession has sought to distance itself from comprising mostly "administrative work" \cite{abbott2016boundaries}, and workers have long tried to tried to reduce their considerable time \cite{socialraadgiverforening2010notat} spent on such tasks in favour of actual casework with clients \cite{toren1972social}. Our study, however, suggests a blurring of the line between "manual" administrative tasks and "mental" casework that draws on practitioner expertise. Many tasks our participants cited involve elements of both: for instance, documenting a case recording requires selecting only the relevant information to include, and planning an intervention can be an iterative process of drafting a plan and discussing it with colleagues and superiors. This all stems from the fact that GenAI can produce virtually any document required by the user, but this document almost always requires revision under a watchful human eye.

% \citeauthor{meilvang_working_2023} (2023) also describes a more recent shift in the perceived accepted boundary of AI interventions in social service work. From "defending [the] jurisdiction [of social service work] against artificial intelligence" in the early days of PRM and other statistical assessment tools, the community has started to embrace AI as a "decision-support ... element in the assessment process". Our study concurs and frames GenAI as a source of information that can be used to support and qualify the assessments of SSPs \cite{meilvang_working_2023}, but suggests that we can take a step further: AI can be viewed as a \textit{facilitator} rather than just a supporter. GenAI can facilitate a wide range of discussions that promote efficiency, encourage worker learning and growth, and ultimately enhance client outcomes. This entails a much larger scope of AI use, where practitioners use the information provided by AI in a range of new scenarios. 

% Taken together, these suggest a new focus for boundary work and, more broadly, HCI research. GAI can play a role not just in menial documentation or decision-support, but can be deeply ingrained into every facet of the social service workflow to open new opportunities for worker growth, workflow optimisation, and ultimately improved client outcomes. Future research can therefore investigate the deeper, organisational-level effects of these new uses of AI, and their resulting impact on the role of profession discretion in effective social service work.

% % MH: oh i feel this paragraph is quite new to me! Could we elaborate this more, and truncate the first two paragraphs a bit to adjust the word propotion?


% % Our study extensively documents this for the first time in social service practice, and in the process reveals new insights about how AI can play such a role.



% \subsection{Design Implications}

% % Add link from ACE diagram?

% % EJ: it would be interesting to discuss how LLMs could help "hands-on experience" in the discussion section

% Addressing the struggle of integrating AI amidst the tension between machine assessment and expert judgement, we reframe AI as an \textit{facilitator} rather than an algorithm or decision-support tool, alleviating many concerns about trust and explainablity. We now present a high-level framework (Figure \ref{fig:hai-collaboration}) on human-AI collaboration, presenting a new perspective on designing effective AI systems that can be applied to both the social service sector and beyond.

% \begin{figure}
%     \centering
%     \includegraphics[scale=0.15]{images/designframework.png}
%     \caption{Framework for Human-AI Collaboration}
%     \label{fig:hai-collaboration}
%     \Description{An image showing our framework for Human-AI Collaboration. It shows that as stakeholder level increases from junior to senior, the directness of use shifts from co-creation to provision.}
% \end{figure}
% % MH: so this paradigm is proposed by us? I wonder if this could a part of results as well..?

% % \subsubsection{From Creation to Provision}

% In Section \ref{sec:stage2findings}, we uncovered the different ways in which SSPs of varying seniorities use, evaluate, and suggest uses of AI. These are intrinsically tied to the perspectives and levels of expertise that each stakeholder possesses. We therefore position the role of AI along the scale of \textit{creation} to \textit{provision}. 

% With junior workers, we recommend \textbf{designing tools for co-creation}: systems that aid the least experienced workers in creating the required deliverables for their work. Rather than \textit{telling} workers what to do - a difficult task in any case given the complexity of social work solutions - AI systems should instead \textit{co-create} deliverables required of these workers. These encompass the multitude of use cases that junior workers found useful: creating reports, suggesting perspectives from which to formulate a case, and providing a starting template for possible intervention plans. Notably, since AI outputs are not perfect, we emphasise the "co" in "co-creation": AI should only be a part of the workflow that also includes active engagement on the part of the SSPs and proactive discussion with supervisors. 

% For more experienced SSPs, we recommend \textbf{designing tools for provision}. Again, this is not the mere provision of recommendations or courses of action with clients, but rather that of resources which complement the needs of workers with greater responsibilities. This notably includes supplying materials to aid with supervision, a novel use case that to our knowledge has not surfaced in previous literature. In addition, senior workers also benefit greatly from manual tasks such as routine report writing and data processing. Since these workers are more experienced and can better spot inaccuracies in AI output, we suggest that AI can "provide" a more finished product that requires less vetting and corrections, and which can be used more directly as part of required deliverables.

% % MH: can we seperate here? above is about the guidance to paradigm, below is the practical roadmap for implementation
% In terms of concrete design features, given the constant focus on discussing AI outputs between colleagues in our FGDs, we recommend that AI tools, particularly those for junior workers, \textbf{include collaborative features} that facilitate feedback and idea sharing between users. We also suggest that designers work closely with domain experts (i.e. social work practitioners and agencies) to identify areas where the given AI model tends to make more mistakes, and to build in features that \textbf{highlight potential mistakes or inadequacies} in the AI's output to facilitate further discussion and avoid workers adopting suboptimal suggestions. 

% We also point out a fundamental difference between GenAI systems and previous systems: that GenAI can now play an important role in aiding users \textit{regardless of its flaws}. The nature of GenAI means that it promotes discussion and opens up new workflows by nature of its verbose and potentially incomplete outputs. Rather than working towards more accurate or explainable outcomes, which may in any case have minimal improvement on worker outcomes \cite{li2024advanced}, designers can also focus on \textbf{understanding how GenAI outputs can augment existing user flows and create new ones}.

% % for more senior workers...

% % how to differentiate levels of workers?

% % \subsubsection{Provider}

% % The most basic and obvious role of modern AI that we identify leverages the main strength of LLMs. They have the ability to produce high-quality writing from short, point-form, or otherwise messy and disjoint case notes that user often have \textit{[cite participant here]}. 

% Finally, given the limited expertise of many workers at using AI, it is important that systems \textbf{explicitly guide users to the features they need}, rather than simply relying on the ability of GAI to understand complex user instructions. For example, in the case of flexibility in use cases (Section \ref{subsubsec:control}), systems should include user flows that help combine multiple intervention and assessment modalities in order to directly meet the needs of workers.

% \subsection{Limitations and Future Work}

% While we attempt to mimic a contextual inquiry and work environment in our study design, there is no substitute for real data from actual system deployment. The use of an AI system in day-to-day work could reveal a different set of insights. Future studies could in particular study how the longitudinal context of how user attitudes, behaviours, preferences, and work outputs change with extended use of AI. 

% While we tried to include practitioners from different agencies, roles, and seniorities, social service practice may differ culturally or procedurally in other agencies or countries. Future studies could investigate different kinds of social service agencies and in different cultures to see if AI is similarly useful there.

% As the study was conducted in a country with relatively high technology literacy, participants naturally had a higher baseline understanding and acceptance of AI and other computer systems. However, we emphasise that our findings are not contingent on this - rather, we suggest that our proposed lens of viewing AI in the social sector is a means for engaging in relevant stakeholders and ensuring the effective design and implementation of AI in the social sector, regardless of how participants feel about AI to begin with. 



% % \subsection{Notes}

% % 1) safety and risks and 2) privacy - what does the emphasis on this say about a) design recommendations and b) approach to designing/PD of such systems?


% % W9 was presented with "Strengths" and "SFBT" output options. They commented, "solution focus is always building on the person's strengths". W9 therefore requested being able to output strengths and SFBT at the same time. But this would suggest that the SFBT output does not currently emphasise strengths strongly enough. However, W9 did not specifically evaluate that, and only made this comment because they saw the "strengths" option available, and in their head, strengths are key to SFBT.
% % What does this say about system design and UI in relation to user mental models?
\section{Conclusions}\label{sec-conclusions}
We present Interaction-aware Conformal Prediction (ICP) to explicitly address the mutual influence between robot and humans in crowd navigation problems. We achieve interaction awareness by proposing an iterative process of robot motion planning based on human motion uncertainty and conformal prediction of the human motion dependent on the robot motion plan. Our crowd navigation simulation experiments show ICP strikes a good balance of performance among navigation efficiency, social awareness, and uncertainty quantification compared to previous works. ICP generalizes well to navigation tasks across different crowd densities, and its fast runtime and manageable memory usage indicates potential for real-world applications.

In future work, we will address infeasible robot planning solutions with adaptive failure probability and conduct real-world crowd navigation experiments to evaluate the effectiveness of ICP. As ICP is a task-agnostic algorithm, we would like to explore its applications in manipulation settings, such as collaborative manufacturing.

\begin{credits}
\subsubsection{\ackname} This work was supported by the National Science Foundation under Grant No. 2143435 and by the National Science Foundation under Grant CCF 2236484.
\end{credits}

\backmatter

\section*{Declarations}

\subsection*{Competing interests}
The authors declare that they have no competing interests.

\subsection*{Funding}

The work was conducted as part of the Cybersecurity for Resilient Energy Communities of the Future (CyREC) project, funded by the Swedish innovation agency (reference number 2023-02987), and the RAISE project under the MUR National Recovery and Resilience Plan funded by the European Union - NextGenerationEU. The funderx had no role in the design of the study; in the collection, analyses, or interpretation of data; in the writing of the manuscript; or in the decision to publish the results.

%This work was partially supported by 

\subsection*{Author contribution}

Writing-original draft, Z.A, G.G; conceptualization, Z.A, G.G, M.A; methodology, Z.A, G.G; funding acquisition, M.A. All authors have read and agreed to the published version of the manuscript.

\subsection*{Data availability}
The code and data used in this work are openly available in the repository at~\cite{repo}.

%\subsection*{Acknowledgements}

\subsection*{Ethical approval}
Not applicable.

%\noindent
%If any of the sections are not relevant to your manuscript, please include the heading and write `Not applicable' for that section. 

%%===================================================%%
%% For presentation purpose, we have included        %%
%% \bigskip command. Please ignore this.             %%
%%===================================================%%


%\begin{appendices}

%\section{Section title of first appendix}\label{secA1}



%%=============================================%%
%% For submissions to Nature Portfolio Journals %%
%% please use the heading ``Extended Data''.   %%
%%=============================================%%

%%=============================================================%%
%% Sample for another appendix section			       %%
%%=============================================================%%

%% \section{Example of another appendix section}\label{secA2}%
%% Appendices may be used for helpful, supporting or essential material that would otherwise 
%% clutter, break up or be distracting to the text. Appendices can consist of sections, figures, 
%% tables and equations etc.

%\end{appendices}

%%===========================================================================================%%
%% If you are submitting to one of the Nature Portfolio journals, using the eJP submission   %%
%% system, please include the references within the manuscript file itself. You may do this  %%
%% by copying the reference list from your .bbl file, paste it into the main manuscript .tex %%
%% file, and delete the associated \verb+\bibliography+ commands.                            %%
%%===========================================================================================%%

\bibliography{references.bib}% common bib file
%% if required, the content of .bbl file can be included here once bbl is generated
%%\input sn-article.bbl


\end{document}


