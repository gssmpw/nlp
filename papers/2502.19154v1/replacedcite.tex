\section{Background and Related Work}
\label{section:related_work}
%This section provides an overview of existing work in the field of security for renewable energy communities, anomaly detection methods, autoencoder-based approaches, and IDS applications in smart grids and energy communities.

The evolution of traditional power grids into smart grids represents a significant advancement in the management and distribution of electric power. However, the increased connectivity and reliance on digital technologies also introduce new security vulnerabilities. This makes smart grids a prime target for cyber attacks____. As the smart grid is a critical infrastructure, breaches can lead to severe consequences, including widespread power outages, financial losses, and even threats to national security____. The increasing trend and practice of monitoring and managing distributed energy resources (DERs) and other smart grid assets remotely through cloud-based solutions over the Internet further complicates the security landscape____.

Several studies____ have investigated potential attacks in smart grids and revealed attacks such as time delay switch (TDS) attacks, false data injection (FDI) attacks, denial of service (DoS), and replay attacks. Mahmoud et al.____ showed that DoS attacks can cause instability of power grids and produce lengthy delays between packets being sent and received. The potential risks associated with the control of a high number of DERs across the distribution grid are also studied in the literature____. For instance, load altering attacks____ can be launched by an attacker that physically or remotely controls a large enough part of the load (power) in an area or by somehow manipulating aggregated load communication of demand aggregation systems____. Such large-scale and distributed attacks against the power grid are considered among the most critical threats____. If an attacker gets access to enough flexible assets, such as electric vehicle (EV) chargers, they could coordinate consumption patterns that might impact the stability of the grid. Another study has focused on the poor encryption in the communication between BESSs installed in homes and the manufacturer servers used for remote control____, leading to threats. 

\subsection{Energy Communities and their Security}

Although, the electric grid congestion and imbalance challenges are long known, these problems are lately approached through the perspective of energy flexibility. Flexibility is the ability to adjust power generation or demand to account for grid conditions. Many actors in the grid together contribute to the flexibility of the grid. For example, flexibility is usually offered by electricity consumers or prosumers that own generation and storage facilities and controllable loads. To increase energy flexibility and with the advancement of regulations regarding  reduction in carbon emissions, the European Union published the renewable energy directive (REDII)____ and introduced mechanisms such as energy communities (ECs). There are many definitions for ECs, but the idea with the emergence of ECs is local energy production from distributed and RESs and its storage and sharing among members in the proximity of the production facility to reach common goals such energy independence, environmental or economic. As ECs introduce unique challenges due to their distributed nature and cloud-based management, which could be manipulated at large scale to manipulate energy demand or inject extra energy into grid, it is critical to study and address their security implications. 

To identify existing research literature on security for ECs, we conducted a comprehensive review of the existing literature. The boolean search string (``energy communities” AND *security) is used to search relevant databases for peer-reviewed research including, journal, conference, and workshop papers as well as peer-reviewed book chapters. The search is focused on IEEE Xplore digital library, ACM digital library, Scopus, and Web of Science databases. We excluded matches before the year 2018 and non-peer reviewed abstracts and publications from the search. The string is searched in the title, abstract, and keywords of the article. Table~\ref{tab:papers} shows a summary of identified literature per database. 
%The main reason for limiting the publication years was that the Clean Energy Package in the EU came in 2018 so we wanted to focus on the research that came from that year onwards. 

\begin{table}[htbp!] 
\centering
\caption{Identified papers per source.}
\begin{tabular}{| p{0.2\linewidth} | p{0.2\linewidth} |}
\hline
\textbf{Database} & \textbf{Papers} 
\\ \hline
IEEE & 43 \\ \hline
ACM & 44 \\ \hline
Scopus & 91 \\ \hline
Web of Science & 51 \\ \hline
\end{tabular}
\centering
\label{tab:papers}
\end{table}

%\subsection{Selection Criteria}

Our initial search resulted in a total of 229 papers, as shown in Figure (step 0). In the next step, we removed duplicates and applied the following two exclusion criteria (ECs) on the remaining 162 papers; EC1: the publication is not concerned with energy communities, and EC2: the publication does not relate to any cybersecurity or privacy aspect. In step 2, we read the paper title and abstract to identify 5 relevant papers in step 3.


\begin{figure}
    \centering
    \includegraphics[width=1\linewidth, height=6cm]{figures/process.png}
    \caption{Literature review process.}
    \label{fig:enter-label}
\end{figure}


From the final selection of papers, only a handful of papers actually address the cybersecurity aspects of ECs. Gaggero et al.____ analyze the potential architectures and protocols used to build ECs and evaluated possible vulnerabilities and attack vectors. Three main attack vectors were identified; attacks against communication protocols, attacks against smart gateway, and attacks against the platform. The paper also discussed some solutions which can be employed to mitigate the risk. In another work, Mokarim et al.____ focus on the software platforms responsible for managing and controlling ECs as they handle a lot of sensitive information. The presented attack model in the research work assumes that an attacker can gain full control over such a software platform using common attacks on servers and applications and then disrupt the operation of the distribution grid by altering certain parameters (set points) of the system. The analysis is done by observing how the grid responds to these power manipulations for both low-voltage (LV) as well as medium-voltage (MV) systems. The power manipulations led to either absorption or injection of power in the grid. The paper highlights that renewable energy communities must implement proper cybersecurity monitoring tools and DSOs must maintain the ability to disconnect generators and loads. Another work____ proposes to implement automatic mechanisms in response to cyber attacks in ECs. As the generators and inverters exist in the homes of private individuals, this makes it impossible to have dedicated cybersecurity teams that implement a proper incident response. The proposed approach includes connecting an IDS to every local generator/inverter. In the event of detected attack/s, the IDS signals electrical protection devices such as a circuit breaker to disconnect the particular generator under attack from the grid. The study____ proposes to model a microgrid energy community, in order to simulate different cyberattacks to assess their spread and impact and finally develop methods to protect from those attacks. The considered attacks include FDI where an attacker modifies the measured information neighboring agents by adding false data and hijack attack where the attacker replaces measurements with malicious data. The attacks are simulated on a customized IEEE34 test feeder and results show deteriorated quality of voltage regulation by the affected inverters. 

Sanduleac et al.____ focus on the resilience of an EC. Resilience is seen both as energy resilience and as a cybersecurity aspect against attacks. The work views ECs as citadels that need to build virtual walls to be resilient against any external aggression. This entails that energy communities need to be always operational and sustainable in face of harsh external conditions such as Internet blackout or cyberattacks. The paper presents a communication architecture that an EC can use to secure all its interactions over the Internet with the DSO and other market actors. As such digital interactions are sensitive and need clear rules, the notion of contracts is put forward which can be exchanged between actors. The concept has been implemented in a simulated energy community and the data exchange is further secured by common ICT security solutions such API keys, digital signatures, VPN, and HTTPS. Steinheimer et al.____ present different approaches for a service management framework to control and monitor decentralized energy consumers, storages, and generators. In these approaches the user is integrated in design and configuration of the services for energy management which offer the possibility to follow its personal needs. ECs between  households are discussed as one of the approaches and P2P interaction among the members is shown to offer a solution to solve user rights and privacy restrictions and to achieve common goals. Other literature works focus on case-studies and pilot projects____. Additionally, they focus on energy security rather than cybersecurity____, explore policy implications and legals aspects for ECs____, and examine the application of distributed ledger technologies, such as block chain, in the context of ECs____. 


%Diagnosis of challenges for power system protection – selected aspects of transformation of power systems
%The analysis includes technical as well as legal and formal aspects, social expectations, economic challenges and security requirements. Special attention is given to challenges raised by energy communities as they will undoubtedly contribute to the prevalence of renewable energy sources (RES). Among the mentioned challenges, the lack of effectiveness of power system protection (PSP) e.g., to protect against electric shocks is emphasized. These equipments fill in a very important role in the power system by protecting users against abnormal operating conditions such as detection of short circuits. However, they typically are designed to be only effective at medium and high current values. In a EC, there is often the lack of load or very low load on the connection between EC and the PS. 

%appkication digital twins architecture
%[Digital Twins for Designing Energy Management Systems for Microgrids: Implementation Example Based on TalTech Campulse Project ] 


\subsection{Intrusion Detection in Smart Grids} 
Intrusion detection systems (IDSs) can play a critical role in protecting smart grids and renewable energy systems from cyber-physical threats. Over the years, researchers have studied various rule-based and signature-based techniques to detect threats exploiting vulnerabilities associated with DERs and their communication networks____. However, these methods are limited in their ability to identify zero-day attacks or novel intrusion strategies, which are increasingly common in smart grid environments. To address this limitation, anomaly-based IDS methods have gained significant attention____. These systems analyze operational data to establish a baseline of normal behavior and detect deviations that may indicate malicious activity. Several studies have demonstrated the effectiveness of anomaly-based IDS in identifying a wide range of cyber-physical attacks____. However, the heterogeneity and dynamic nature of ECs introduce unique challenges that require more sophisticated detection mechanisms.

Autoencoders are a type of deep artificial neural network designed for unsupervised machine learning____. They work by learning a compressed representation of normal data and reconstructing it with minimal loss. Anomalies, which deviate from the learned representation, result in higher reconstruction errors, making them detectable. In the context of smart grids, several studies____ have utilized autoencoders to identify anomalies in sensor data, control signals, and energy flows. For example, Takiddin et al.____ propose a detection method for electricity theft in smart grids using deep autoencoders. This approach captures complex consumption patterns and temporal correlations in energy usage data. Simulation results demonstrate that the proposed method enhances detection rate and reduces false alarms as compared to existing solutions. Similarly, Nazir et al.____ employ an autoencoder-based IDS to detect FDI attacks in SCADA systems, achieving high detection accuracy while maintaining low false-positive rates. 

\subsection{Gaps and Opportunities}

While significant progress has been made in the development of intrusion detection mechanisms for smart grids in general, several gaps remain. Existing anomaly detection methods do not consider the complex interdependencies between generation, storage, and consumption. As demonstrated by our literature review, security aspects in ECs are not thoroughly studied. The challenges posed by the integration of cloud-based management platforms to these ECs demand research in this direction. Finally, despite the potential of autoencoders for anomaly detection, their application in ECs to detect a wide range of cyber-physical attacks requires thorough investigation.

This work addresses these gaps by proposing an autoencoder-based IDS tailored specifically to the unique characteristics of ECs. By leveraging unsupervised learning techniques, the system can detect multiple attacks. Furthermore, the proposed IDS is designed to operate within the local EC infrastructure, facilitating real-time detection and response to cyber-physical threats. This paper builds on the insights and methodologies from previous research while extending their application to ECs, thereby contributing to the growing body of knowledge on smart grid security.