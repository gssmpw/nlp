\section{Conclusions}

This paper presents an anomaly-based intrusion detection system developed to enhance the security of energy communities while preserving data privacy. By leveraging deep autoencoders and unsupervised learning, the proposed model effectively detects cyber threats that could lead to economic losses or hazardous conditions. The results demonstrate good detection performance, confirming the potential of the approach for real-world application. The incorporation of federated learning, while introducing a slight reduction in performance, successfully ensures data privacy and decentralization. This approach provides a strong foundation for future development, particularly in scenarios where privacy is a key requirement. Our work represents a promising advancement in cybersecurity for energy communities. With further refinements and real-world validations, the proposed detection system has the potential to become a vital component of secure decentralized energy infrastructures. 

%The proposed anomaly-based intrusion detection system represents a promising advancement in cybersecurity for energy communities. With further refinements and real-world validation, it has the potential to become a component of resilient and secure decentralized energy infrastructures.