\section{Interpretation of Results}

The aim of this paper is to enhance the security of ECs by proposing and demonstrating the effectiveness of an IDS in such environments. A review of the literature revealed that there is a lack of thorough research in the area of security for ECs, especially in detection methods. To address this gap, we propose a detection system based on deep autoencoders, which is a type of unsupervised model. The results of the study demonstrate the potential of the proposed IDS for ECs. The detection algorithm is capable of identifying both economically damaging attacks and those that may lead to unsafe working conditions, highlighting its relevance for real-world deployment. These findings confirm that anomaly-based intrusion detection can be a valuable tool in enhancing the cybersecurity and resilience of energy infrastructures.

The selection of autoencoders as the detection model among other possibilities is mainly motivated by previous research which has found autoencoders to be effective in many cyber-physical settings. For the architecture of autoencoders, there are many possibilities and optimizations that can be applied. Our goal was to achieve a fine balance between model complexity and performance and that is the reason we implemented a model of five layers (two encoding, one repeat, and two decoding). The selection of the optimizer, loss function, number of epochs, detection threshold, and other hyperparameters are also based on the goal of keeping the model simple. While, we have demonstrated that the performance of the model is good for most attack scenarios, these hyperparameters can be fine-tuned to further enhance the performance. 

As data-driven detection systems are as good as the data they train on, we made an attempt to build a model in Simulink that is accurate, well-tuned, and based on real-world patterns (e.g., consumption trends). While this simulated data can not fully replace a real world data, it can be considered very close to it. In a field where data from real systems is impractical to access for many legal and data protections reasons, simulated data enable us to design and test our detection systems. However, real-world testing of the system is necessary for final validation and deployment. 

We have also considered privacy concerns that such a detection system can suffer from in a real-world  setting. The detection system relies on raw measurements about voltages, currents, and power from private households, and that can prove to be a big challenge in reality. To address this concern, we have demonstrated that the detection model can be trained in a collaborate manner using FL. While the decentralized approach slightly reduces detection accuracy, it ensures that sensitive data remains local, mitigating privacy risks. This may introduce a trade-off between performance and privacy but finding the right balance is particularly important in ECs, where data privacy concerns can hinder the adoption of centralized security solutions. Despite the minor decrease in performance, the advantages offered by FL, such as improved data confidentiality and compliance with privacy regulations, make it a viable and valuable approach. 

Nonetheless, there is room for further improvement. One of the main challenges observed in the results is the presence of false negatives, specially for PA attacks. Future research should focus on refining the detection algorithm to enhance performance, particularly by improving model recall without reducing the precision. This could be achieved through more sophisticated feature selection, more complex deep learning models, or integrating more operational data, e.g., network data, into the training phase. The performance of the federated model could also be further improved by finding optimal hyperparameters such as number of rounds and number of epochs per training round. However, this needs to be carefully observed as increasing the number of epochs allows each client to train more on its local data in each round which may improve the performance, but it may lead to overfitting if the data are small. Similarly, increasing the number of rounds allows the global model to improve over time, but it comes at the cost of increased communication overhead. Overall, the results indicate that the proposed approach is a promising step toward secure and privacy-preserving anomaly detection in ECs. Continued research and optimization efforts will be essential to ensure its robustness and practical applicability in dynamic and complex environments such as ECs.
 

%Practical implications for deployment in real-world RECs.
%Limitations of the proposed approach.

%FL model can be further improved by finding optimal hyperparameters (rounds, epochs) 
%Increasing the number of epochs allows each client to train more on its local data in each round, but it may lead to overfitting if the data is small.

%Increasing the number of rounds allows the global model to improve over time, but it increases communication overhead.



