
\section{Geometrical structures} \label{secMathStructures}

In the appendix sections we give rigorous definitions of the data structures and sheaf-related theory mentioned in the main part of the text, and provide a self-contained exposition of the related theory. Here is the quick guide to the appendices.
\begin{enumerate}
  \item The remaining part of Appendix~\ref{secMathStructures}. We start with the review of geometrical structures: partially ordered sets (posets), simplicial complexes, CW-complexes, introduce the notion of cell poset and its variations, among which the most important is homology Morse cell poset. We recall the classical connection of posets with Alexandrov topologies.
  \item Appendix~\ref{secMathSheaves}. We proceed to diagrams over posets and sheaves over topologies, and describe their connection. Global sections and functorial properties of sheaves are defined.
  %\item Appendix~\ref{secMathSpaceRestoredFromShvs}. We formulate some results stating that the category of sheaves ``remembers'' the original space. These results are well-known among specialists in category theory, they provide a mathematical explanation, why consideration of sheaves on a space makes sense, however most applied papers lack this explanation.
  \item Appendix~\ref{secMathCohomology}. We recall the basic machinery of homological algebra: cochain complexes, exact sequences. We construct enough injective objects in the category of diagrams over a poset, and define sheaf cohomology as right derived functors of the global sections. The classical constructions of cochain complexes are presented: Roos complex (also known as standard simplicial resolution or cobar construction) and cellular cochain complex. In subsection~\ref{subsecMathOneShot} and~\ref{subsecMathMinimalComputations} we prove new results. A notion of one-shot sheaf cohomology computation is introduced; we prove that a poset $S$ allows for one-shot computation if and only if $S$ is a homology Morse cell poset. For an arbitrary poset $S$ we prove a theoretical bound for the complexity of sheaf computations on $S$, which is similar in spirit to Morse inequalities. We prove the bound is exact by introducing a minimal cochain complex on $S$. In case of cell posets, this minimal cochain complex coincides with cellular cochain complex.
  \item Appendix~\ref{secMathLaplacians}. Here we recall some basic linear algebra and differential equations, and introduce the notions of euclidean sheaves, Laplacians, higher-order Laplacians, energy functions, and heat/sheaf diffusion. All notions make sense for sheaves on arbitrary posets, not just cellular sheaves, which is important, if one wants to consider hypergraphs. In subsection~\ref{subsecMathDiffuseExamples} we provide examples how the mathematical formalism introduced in the previous subsection describes the internal mechanisms of neural networks used in topological deep learning. Namely, we describe, in a unified manner, diffusion in sheaf neural networks, and linear diffusion in sheaf hypergraph neural networks, and propose the way to things can be generalized.
  \item Appendix~\ref{secMathConnection}. We recall the notion of a connection sheaf (also known as a local system or a vector bundle) and provide basic mathematical claims about them.
\end{enumerate}

\begin{rem}
A small disclaimer: the mathematical appendices do not contain a list of notions despite their relevance to the subject of the review. We make no mention of Grothendieck topologies and sheaves over them; this subject can be found in the book of Rosiak~\cite{Rosiak}. We intentionally avoid mentioning cosheaves' homology (because this story is about reverting the order in a poset) as well as sheaf homology and cosheaf cohomology (which are more interesting); this subject is covered by Curry's thesis~\cite{Curry}. We don't review any of the topics of topological data analysis and applied topology in general, as there are many other sources~\cite{Zom,EdelHarer,Oudot,CarlssonVejdemo,DeyWang}. We also assume that the reader is familiar with the basics of category theory: objects, morphisms, categories, functors. More specific terms will be either explained or provided with a reference as they appear in the text. 
\end{rem}

\subsection{Posets}

Let $S$ be a partially ordered set (poset), or a preordered set (preposet). The non-strict order relation on $S$ is denoted $\leq$, while its strict version is denoted by $<$ (the latter makes sense for posets). 

\begin{con}\label{conCatOfPoset}
For each preposet $S$ define a small category $\cat(S)$, whose objects are elements from $S$, and there is only one morphism from $s_1$ to $s_2$ if $s_1\leq s_2$, and no morphisms otherwise.
\end{con}

A map between (pre)posets $f\colon S\to T$ is called \emph{monotonic}, or a morphism of (pre)posets if the inequality $s_1\leq s_2$ in $S$ implies the inequality $f(s_1)\leq f(s_2)$ in $T$. Each such morphism induces a functor $\cat(f)\colon \cat(S)\to \cat(T)$ between the corresponding small categories.

\begin{con}\label{conLowerUpperSetNotation}
For any (pre)poset $S$ and any element $s\in S$ we define
\[
S_{\geq s}=\{t\in S\mid t\geq s\},\qquad S_{>s}=\{t\in S\mid t>s\}.
\]
Similarly, we define $S_{\leq s}$, $S_{<s}$, etc. Poset $S$ with the dual order is denoted by $S^{\op}$.
\end{con}

\begin{con}\label{conPreposetToPoset}
For a preordered set $(S,\leq)$, let $\sim$ denote the equivalence relation defined by $s_1\sim s_2$ if and only if $s_1\leq s_2$ and $s_2\leq s_1$. The quotient set $S/\sim$ inherits the partial preorder ($[s_1]\leq[s_2]\Leftrightarrow s_1\leq s_2$), which is a partial order. This poset $S/\sim$ with this order is denoted $\bar{S}$. We have a couple of morphisms $S\leftrightarrows \bar{S}$, where one is the natural projection, and another one is inclusion of a representative in each class (for infinite sets the existence of such map follows from the axiom of choice). On the level of categories, $\cat(\bar{S})$ is the category obtained from $\cat{S}$ by taking one representative in each isomorphism class of objects.
\end{con}

\subsection{Alexandrov topologies}\label{subsecMathAlexandrov}

\begin{defin}\label{defAlexandrovTopology}
A topology $\Omega$ on a set $X$ is called an \emph{Alexandrov topology} if the intersection of any (possibly infinite) number of open subsets is open.
\end{defin}

\begin{defin}\label{defT0}
A topology $\Omega$ on a set $X$ \emph{satisfies the Kolmogorov axiom} (separation axiom) $T_0$ if, for each pair of points $x_1\neq x_2\in X$, there exists an open neighborhood $U$ of $x_1$ which does not contain $x_2$, or an open neighborhood $U$ of $x_2$ which does not contain $x_1$.
\end{defin}

The following correspondence is well known in general topology and mathematical logic.

\begin{prop}\label{propPosTop}
The category $\PrePos$ of preordered sets and their morphisms is equivalent to the category $\AlexTop$ of Alexandrov topological spaces with continuous maps. Under this equivalence, posets correspond to Alexandrov topologies satisfying separation axiom $T_0$.
\end{prop}

The proof can be found in the literature, for example,~\cite{Arenas}. We present the main constructions from the proof since they are used in what follows.

\begin{con}\label{conPrePosToSpace}
Let $S$ be a set with a preorder $\leq$. Consider the topology $\Omega_S$ (the family of subsets, which are called open) on the set $S$, which consists of upper order ideals: $U\in \Omega_S$ if and only if $x\in U$ and $y\geq x$ implies $y\in U$. The unions and intersections of any number of upper order ideals is an upper order ideal as well. Hence $X_S=(S,\Omega_S)$ is an Alexandrov topological space.
\end{con}

\begin{con}\label{conSpaceToPrepos}
Let us now review the reverse construction: given an Alexandrov topological space, we build a preorder. Let $X=(M,\Omega_X)$ be an Alexandrov topological space. The key property of Alexandrov spaces is the existence of an open neighbourhood of any point, minimal by inclusion. Indeed, let $x\in M$. Consider the family $\mathcal{U}=\{U\in \Omega_X\mid x\in U\}$ of all open neighbourhoods of a chosen point. Their intersection
\begin{equation}\label{eqDefUx}
U_x=\bigcap_{U\in \mathcal{U}}U
\end{equation}
is an open set by the definition of the Alexandrov topology. It is easy to see that this intersection belongs to any open neighbourhood of $x$, therefore we can call this intersection the minimal open neighbourhood of $x$. Now we can define the partial preorder on the set $M$: we put $x_1\leq x_2$ if $U_{x_1}\subseteq U_{x_2}$.
\end{con}

Checking that the Constructions ~\ref{conPrePosToSpace} and \ref{conSpaceToPrepos} are mutually inverses is a simple exercise. Both constructions are functorial (i.e. monotonic functions are continuous and vice versa), which is another exercise. The last exercise left to the reader is to check that partial orders correspond to $T_0$-spaces under this correspondence.

\begin{rem}\label{remMinNbhdIsACone}
In terms of the (pre)order on $S$, \emph{the minimal open neighbourhood} $U_x$ defined in Construction~\ref{conSpaceToPrepos} is an upper cone over $x$:
\[
U_x=\{s\in S\mid s\geq x\}.
\]
This is the minimal upper order ideal, which contains a point $x$.
\end{rem}

\subsection{Simplicial and cellular complexes}

In this section we mainly deal with finite structures: finite posets, finite simplicial and cell complexes. However, many definitions make sense for infinite sets.

\begin{defin}\label{definSimpComp}
A(n abstract) \emph{simplicial complex} on a (finite) vertex set $M$ is a collection $\ca{K}$ of subsets of $M$ such that
\begin{enumerate}
  \item any singleton\footnote{For our exposition we prefer to forbid ghost vertices.} $\{i\}$ for $i\in M$ lies in $\ca{K}$;
  \item if $I\in \ca{K}$ and $J\subset I$ then $J\in \ca{K}$.
\end{enumerate}
\end{defin}

The element $I\in \ca{K}$ is called \emph{a simplex}, the number $\dim I=\#I-1$ its dimension. The set of nonempty simplices of $\ca{K}$ is partially ordered by inclusion, we denote this poset by $\Cells(\ca{K})$. It is usually assumed that the vertex set $M$ is well ordered, hence may be identified with $[m]=\{1,\ldots,m\}$.

\begin{con}\label{conStandardGeomRealization}
The \emph{standard geometrical realization} of a simplicial complex $\ca{K}$ is the compact space
\begin{equation}\label{eqStdGeomRealiz}
|\ca{K}|=\bigcup_{I\in \Cells(\ca{K})}\triangle_I\subset\Ro^m,
\end{equation}
where $\triangle_I=\ConvHull\{e_i\mid i\in I\}$ and $e_1,\ldots,e_m$ is the standard basis of $\Ro^m$. Each $\Delta_I$ is a (geometrical) simplex, hence $|\ca{K}|$ is a geometrical simplicial complex, that is a space decomposed into cells isomorphic to simplices.
\end{con}

\begin{rem}\label{remAbuseSimpComp}
There exists a common abuse of notation: we call two simplicial complexes $\ca{K}_1,\ca{K}_2$ homeomorphic (homotopically equivalent, etc.) if their geometrical realizations are homeomorphic (resp. homotopically equivalent, etc.).
\end{rem}

\subsection{Order complex}

In combinatorial topology, there is a classical way of turning a (finite) poset into a simplicial complex.

\begin{defin}\label{definGeomRealPoset}
Let $S$ be a poset. \emph{The order complex of} $S$ is a simplicial complex $\ord S$ on the vertex set $S$ whose simplices are chains (i.e. well ordered subsets) of $S$. \emph{The geometric realization of} $S$ denoted by $|S|$ is the geometric realization $|\ord S|$ of the order complex $\ord S$.
\end{defin}

\begin{rem}\label{remSimplexOfOrd}
It follows from the definition, that a $j$-dimensional simplex $\sigma$ of $\ord S$ is encoded by a sequence $s_0<s_1<\ldots<s_j$ in $S$. Equivalently, it is a sequence of $j$ non-identical morphisms in $\cat(S)$.
\end{rem}

\begin{rem}\label{remBarycentric}
If $\ca{K}$ is a simplicial complex, then the simplicial complex
\[
\ca{K}'=\Cells\ord(\ca{K})
\]
is \emph{the barycentric subdivision} of $\ca{K}$. Simplicial complexes $\ca{K}$ and $\ca{K}'$ are homeomorphic. See details in~\cite[\S2.1.5]{KozlovBook}.
\end{rem}

There is a fundamental McCord's theorem, which establishes a connection between Alexandrov spaces and geometric realizations of finite posets.

\begin{rem}\label{remMcCord}
Let $S$ be an arbitrary finite poset. Then there is a canonical map:
\[
h_S\colon |S|\to X_S.
\]
If $x\in |S|$ is a point in the relative interior of a geometrical simplex $\triangle_{\{s_0,\ldots,s_k\}}$ corresponding to a chain $s_0<s_1<\cdots<s_k$, then $h_S(x)=s_k$, the maximal element of the chain in $S$. It is not difficult to check that $h_S$ is continuous (with respect to the Alexandrov topology on $X_S$ and Hausdorff topology on $|S|$). McCord's theorem~\cite{McCord} states that $h_S$ is a weak homotopy equivalence, i.e. induces isomorphism of all standard homotopy invariants. Among other applications, this theorem allows one to think of a finite topology as a nice substitute for a finite cell complexes with a comparable homotopical expressiveness. However, it should be noted that McCord theorem doesn't mention sheaves or their cohomology, so certain care is needed in this topic.
\end{rem}

%###################################

\subsection{Cell complexes} 

Now we switch to cellular complexes. In our review this notion is essentially needed to define a cellular sheaf introduced by Zeeman and Shepard, see~\cite{Curry}. However, there is a certain discrepancy between formal definition of a (regular finite) CW-complex known in topology (see e.g.~\cite{bjorner1984poset}) and how this notion is actually used in applied papers. The intuition that matches both formal definition and the way how it is commonly understood is expressed by the following informal definition.

\begin{difin}\label{difinCWhausdorff}
A \emph{CW-complex} (or a \emph{cell complex} or a \emph{cellular complex}) is a filtered Hausdorff topological space
\[
X=\bigcup_nX_n, \qquad X_0\subseteq X_1\subseteq\cdots
\]
where $X_n$, $n\geq 0$, is called \emph{the $n$-skeleton of} $X$. The space is obtained inductively, that is $X=\lim_{t,\to} X^{(t)}$. The base of induction $X^{(0)}$ is the empty topological space with the trivial filtration; and the space $X^{(t)}$ is the result of attaching a topological disc $D^k$ to the previously constructed space $X^{(t-1)}$ along some attaching map $a_s\colon \dd D^k\to X^{(t-1)}_{k-1}$ in a way that the attached disc contributes to skeleta $X^{(t)}_n$ with $n\geq k$. A cell complex is called \emph{regular} if the attaching maps are homeomorphisms to the image.

In other words, a cell complex is a Hausdorff space constructed inductively from relatively simple pieces $(D^k,\dd D^k)$ called the \emph{cells} of $X$.
\end{difin}

\begin{rem}\label{remBuildingBlocks}
In the sense of the above definition, cell complexes generalize simplicial complexes: instead of simplices we are allowed to use e.g. cubes, pentagons, and almost anything, as long as each building block is homeomorphic to a topological disc $D^k$, and the attachment of this disc is made along its boundary $(k-1)$-sphere $\dd D^k$.
\end{rem}

\begin{rem}\label{remNovikovForman}
Definition~\ref{difinCWhausdorff} is bad from algorithmical perspective since it operates with (usually infinite) Hausdorff spaces. If one chooses to encode all spaces $X_n^{(s)}$ as finite simplicial complexes and all the attachment maps $a_s$ as simplicial maps (or piecewise linear maps), the resulting structure may be too heavy to store and process. Besides, it is algorithmically impossible to say if a given collection of simplicial complexes $\{X_n^{(s)}\}$ is a cell complex or not due to Novikov's theorem~\cite{volodin1974sphere} which states that homeomorphism to a sphere is algorithmically undecidable. %However, sometimes a theoretical guarantee can be given, that a cell is actually a disc, or behaves like a disc. The most well-known approach of this sort is Discrete Morse Theory developed by Forman\cite{forman1998morse}.
\end{rem}

\subsection{Cell posets} 

In the practice of machine learning, attachment maps are usually not considered at all. A cell complex is thought of as an abstract finite poset of cells satisfying certain restrictions. We refer to~\cite{bjorner1984poset} for the extensive review of the related notions and constructions~\footnote{It should be noted that Bj\"{o}rner's formal definition of a cell poset differs from the one given below.}. %First of all, there exists an abstraction to encode ``dimensions'' of cells.

\begin{defin}\label{definGrading}\cite{StanleyComb}
A \emph{grading} on a (finite) poset $S$ is a function $\rk\colon S\to\Zo$ such that
\begin{enumerate}
  \item $\rk$ is strictly monotonic: if $s_1<s_2$ then $\rk s_1<\rk s_2$.
  \item $\rk$ agrees with covering relation: if $s_1<s_2$ is a covering relation i.e. there is no $t$ such that $s_1<t<s_2$, then $\rk s_2=\rk s_1+1$.%is dense
  \item All minimal elements have rank $0$.
\end{enumerate}
A pair $(S,\rk)$, where $\rk$ is a grading on $S$, is called \emph{a graded poset}.
\end{defin}

\begin{rem}\label{remRankUnique}
It can be easily proven that whenever a poset $S$ has a grading, this grading is unique. So far, grading is a property of a poset rather than a complementary structure. A poset is \emph{graded} if, for any $s\in S$, all maximal (saturated) chains descending from $s$:
\[
s_1<s_2<\cdots<s_k<s
\]
have the same length $k$. In this case we can set $\rk s=k$. However, in practice it is more convenient to store the rank function as a complementary data structure.
\end{rem}

\begin{ex}\label{exGradingExistsNot}
The poset $\Cells(\ca{K})$ of nonempty simplices of a simplicial complex $\ca{K}$ is naturally graded by $\rk I=\dim I$. Similarly, the poset of faces of any convex polytope is graded by their dimensions. Geometric lattices~\cite{CrapoComb} constitute another important class of graded posets. On the other hand, there exist posets which do not admit any grading at all, see e.g. the rightmost poset on Fig.~\ref{figPosets}.
\end{ex}

Cells of a cell complex will be encoded by elements of a poset. However, the condition that each cell is a disc attached along the boundary sphere should be somehow described in the language of posets. This is done in the standard way.

\begin{con}\label{conCones}
The \emph{cone} of an element $s\in S$ is the subposet $C(s)=S_{\leq s}=\{t\in S\mid t\leq s\}$. The \emph{boundary} of an element $s\in S$ is the subposet $\dd C(s)=S_{<s}=\{t\in S\mid t<s\}$. Notice that $S_{\leq s}$ is obtained from $S_{<s}$ by adding the greatest element, the point $s$ itself. Therefore, for the geometrical realizations we have $|C(s)|=\Cone|\dd C(s)|$, where $|\dd C(s)|$ is the base and $s$ --- the apex of a cone.
\end{con}

\begin{ex}\label{exDownConeSimplex}
If $S=\Cells(\ca{K})$, and $I=\{i_0,\ldots,i_k\}\in S$ is a simplex of dimension $k$, then $C(I)$ is the poset of faces of the simplex $I$, equiv. the poset of subsets of $I$, equiv. the boolean lattice of rank $k+1$ with the bottom element removed. Its boundary $\dd C(I)$ is the poset of proper faces of $I$, which coincides with $\Cells(\dd\triangle_I)$. This observation explains the name of $\dd C(s)$ in general.
\end{ex}

The above mentioned intuition leads to the following definition of a cell complex, given in the language of posets. We reserve the symbols $\ca{X}$, $\ca{Y}$, etc. for cell posets, and $\sigma$,$\tau$ etc. for their elements, the cells.

\begin{defin}\label{definCellPoset}
A graded poset $(\ca{X},\rk)$ is called a \emph{cell poset} if, for any $\sigma\in \ca{X}$, the boundary $|\dd C(\sigma)|$ is homeomorphic\footnote{We recall the standard formalism known in topology. The sphere $S^{-1}$ is the empty space. Any set homeomorphic to the empty space is the empty space.} to the sphere $S^{\rk\sigma-1}$. If this is the case, an element $\sigma$ of rank $k$ is called a $k$-\emph{dimensional cell}.
\end{defin}

\begin{rem}
As a consequence of Construction~\ref{conCones}, the cone $|C(\sigma)|$ of each cell $\sigma$ of a cell poset is homeomorphic to a disc $|C(\sigma)|=\Cone|\dd C(\sigma)|\cong \Cone S^{\rk \sigma-1}\cong D^{\rk \sigma}$. The pair $(|C(\sigma)|,|\dd C(\sigma)|)$ resembles the building block $(D^k,S^{k-1})$ of a cell complex in terms of Definition~\ref{difinCWhausdorff}.
\end{rem}

\begin{con}\label{conCellPosetSkeleton}
The subposet $\ca{X}_n=\{\sigma\in \ca{X}\mid \rk \sigma\leq n\}$ is called \emph{the $k$-skeleton} of a cell poset $\ca{X}$. It resembles the topological skeleton of a cell complex as made transparent in the proof of the following proposition.
\end{con}

The following claim provides the connection between CW-complexes and cell posets.

\begin{prop}\label{propCellPosetToCpx}
If $X$ is a regular cell complex, then the poset $\Cells(X)$ is a cell poset. If $\ca{X}$ is a cell poset, then the Hausdorff space $|\ca{X}|$ filtered by $|\ca{X}_n|$ is a regular cell complex.
\end{prop}

See~\cite[Prop.3.1]{bjorner1984poset} for the proof.
%
%\begin{prop}
%Let
%\end{prop}
%
%\begin{proof}
%Consider a linear order $\sigma_1,\ldots,\sigma_r$ on $\ca{X}$ extending\footnote{This is called topological sorting in applied literature} the given partial order on $\ca{X}$, so that every element $\sigma_j<\sigma_i$ appears in the list earlier than $\sigma_i$, that is $j<i$. Let $\ca{X}^{(t)}$ denote a subposet $\{\sigma_1,\ldots,\sigma_t\}$ of $\ca{X}$ with the induced partial order and the grading restricted from $\ca{X}$. Then the space $|\ca{X}|$ is the limit (the union) of the expanding sequence $|\ca{X}^{(t)}|$ of filtered spaces, such that $|\ca{X}^{(t)}|=|\ca{X}^{(t-1)}|\cup_{|\dd C(\sigma_t)|}|C(\sigma_t)|$. The attachment map is the natural inclusion. The space $|\dd C(\sigma_t)|$ is homeomorphic to a sphere $S^{\rk \sigma_t-1}$, the space $|C(\sigma_t)|$ is homeomorphic to the disc $D^{\rk \sigma_t}$, and the requirement on filtrations is easily verified, therefore the definition of a cell complex is satisfied.
%\end{proof}


\subsection{Variations of cell posets}\label{subsecMathCellVariations}

When dealing with specific aspects of cell complexes, for example homotopical or homological, Definition~\ref{definCellPoset} can be relaxed in one way or another. For example, we may forget the requirement for all spaces to be filtered. This situation appears in Morse theory~\cite{MilnorMorse} (Morse decomposition of a manifold does not usually give the structure a CW-complex), and in its algebro-geometrical version Bialynicki-Birula theory \cite{BialynizkiDecomp} (the analogues of cell complexes are called affine pavings). The requirement that $|\dd C(\sigma)|$ is homeomorphic to a sphere may be weakened in several obvious ways.

\begin{defin}\label{definHomotopyCellPoset}
A graded poset $(\ca{X},\rk)$ is called a \emph{homotopy cell poset} if, for any $\sigma\in \ca{X}$, the boundary $|\dd C(\sigma)|$ is homotopy equivalent to the sphere $S^{\rk\sigma-1}$.
\end{defin}

\begin{defin}\label{definHomologyCellPoset}
A graded poset $(\ca{X},\rk)$ is called a \emph{homology cell poset} if, for any $\sigma\in \ca{X}$, the boundary $|\dd C(\sigma)|$ has the same $\Zo$-homology as the sphere $S^{\rk\sigma-1}$:
\[
H_i(|\dd C(\sigma)|;\Zo)\begin{cases}
                          \cong\Zo, & \mbox{if } i=\rk\sigma-1 \\
                          =0, & \mbox{otherwise}.
                        \end{cases}
\]
\end{defin}

Definition~\ref{definHomologyCellPoset} has an advantage over Definitions~\ref{definHomotopyCellPoset} and~\ref{definCellPoset}: the property of a graded poset $(\ca{X},\rk)$ to be a homology cell poset is algorithmically decidable, while homotopy cell posets and cell posets are not, see Remark~\ref{remNovikovForman}. Yet another way to weaken the definition of a cell poset is to drop the assumption of being graded. We will see in Subsection~\ref{subsecMathOneShot} that the basic properties of cellular sheaves and their cohomology remain the same if these more general notions are used instead of cell posets.

\begin{defin}\label{definMorseHomologyCellPoset}
A poset $\ca{X}$ is called a \emph{(homology) Morse cell poset} if, for any $\sigma\in \ca{X}$, the boundary $|\dd C(\sigma)|$ has the same singular homology as a sphere $S^{d_s-1}$. In this case, the elements $s\in S$ are called \emph{cells}, and the corresponding number $d_s$ is called \emph{the dimension of a cell} $s$.
\end{defin}

\begin{rem}\label{remHomologyInContext}
Depending on the context, singular homology is taken with specific coefficients. If the target abelian category is $\Abel=\Zo\Mod$, then the coefficients are taken in $\Zo$, so that $|\dd C(\sigma)|$ should have the same $\Zo$-homology as a sphere. If the target abelian category is the category $\ko\Vect$ of vector spaces over a field $\ko$, then $|\dd C(\sigma)|$ should have the same $\ko$-homology as a sphere. The same remark relates to Definition~\ref{definHomologyCellPoset} as well.
\end{rem}

The definition of Morse homotopy cell poset may be given in a straightforward way.

\begin{figure}
  \centering
  \includegraphics[scale=0.25]{pics/morse.pdf}
  \caption{Example of Morse sell poset and the corresponding ``topological intuition''. There are two cellular approximations available for this poset.}\label{figElementaryMorse}
\end{figure}

\begin{ex}\label{exMorseExample}
Fig.~\ref{figElementaryMorse} shows an example of a Morse cell poset which is not graded. There is a geometrical intuition behind such a poset: assume we have 3 vertices $1,2,3$ and two edges $a,b$, and $a$ is attached to $1$ and $2$ in a natural way, while endpoints of $b$ are attached to some middle-point of $a$ and $3$ respectively. This geometrical figure is not a CW-complex, because it contradicts to the assumption that boundaries of cells are attached to lower-dimensional skeleta. However, such situations often (if not always) occur in the classical Morse theory: the (un)stable manifolds of the gradient flow of a Morse function do not form a CW-decomposition.
\end{ex}

\begin{rem}\label{remSubposet}
The posets defined by either of the Definitions~\eqref{definCellPoset},\ref{definHomotopyCellPoset},\ref{definHomologyCellPoset}, or~\ref{definMorseHomologyCellPoset} will be called cellular-like posets. If $\ca{X}$ is cellular-like, then lower order ideals $\ca{Y}\subseteq\ca{X}$ will be called cell subposets. The condition of lower order ideal means that whenever $\sigma\in\ca{Y}$ and $\tau<\sigma$, then $\tau\in\ca{Y}$. It is easily seen that cell subposets inherit the defining property of $\ca{X}$: for example, if $\ca{X}$ is a homology Morse cell poset, then so are its subposets $\ca{Y}$.
\end{rem}

A convention used in the following: notation $S_{\leq s}$, $S_{<s}$ is used for generic posets, while $C(s)$, $\dd C(s)$ is used for cell posets to be defined below. The letter $C$ stands here for ``cell'', and cell is usually assumed to be bounded by a sphere, or something like a sphere.

%###################################




\section{Sheaves and diagrams}\label{secMathSheaves}

\subsection{Diagrams} 

In this section we recall the notion of a diagram on a poset, sheaf on a topological space, and describe their basic relations. We start with a more common and more intuitive notion of a diagram on a poset.

We recall some basic terminology from category theory to fix the notation. If $\Vv$ is a category, $\Ob(\Vv)$ denotes the class of objects of $\Vv$ and $\Hom_\Vv(a,b)$ the set of morphisms from an object $a$ to an object $b$. We usually abuse the notation and write $a\in\Vv$ instead of $a\in\Ob(\Vv)$. The symbol $\Cc$ will denotes generic small categories, while $\Vv$ is reserved for big categories such as
\begin{enumerate}
  \item the category $\Sets$ of sets and their maps;
  \item the category $\Abel$ of abelian groups and group homomorphisms;
  \item the category $\ko\Vect$ of vector spaces over a ground field $\ko$ and linear maps;
  \item the category $R\Mod$ of $R$-modules and module homomorphisms.
\end{enumerate}
Usually it is assumed that $\Vv$ is \emph{complete and cocomplete}, i.e. has all direct and inverse limits.

\begin{defin}\label{definDiagramOnPoset}
A \emph{diagram on a preposet} $S$ valued in a category $\Vv$ is a functor $D\colon \cat(S)\to \Vv$. %More generally, a diagram on a category $\Cc$ valued in $\Vv$ is a functor $D\colon \Cc\to \Vv$.
\end{defin}

\begin{rem}\label{remDiagramInformally}
In simple words, a diagram on a preposet consists of two assignments. (1) To each element $s\in S$ we assign some object $D(s)=V_s$ of a category $\Vv$ (e.g., if the category $\Vv$ is $\ko\Vect$, $V_s$ is a vector space). (2) To each pair $s_1,s_2\in S$ such that $s_1\leq s_2$, we assign a map
\[
f_{s_1s_2}=D(s_1\leq s_2)\colon V_{s_1}\to V_{s_2}.
\]
These maps satisfy compositionality: $f_{ss}=\id_{V_s}$ for any $s\in S$, and $f_{s_1s_2};f_{s_2s_3}=f_{s_1s_3}$ for any $s_1\leq s_2\leq s_3$.
\end{rem}

All diagrams on $S$ form the category $\Diag(S,\Vv)$, in which the morphisms are natural transformations of functors.

\begin{rem}\label{remPreposetsNotNeeded}
Let $S$ be a preordered set, and $\bar{S}$ the corresponding quotient poset, as described in Construction~\ref{conPreposetToPoset}. Assume $D$ is a diagram on $S$. Whenever $s_1\sim s_2$ in $S$, we have two maps
\[
D(s_1\leq s_2)\colon D(s_1)\leftrightarrows D(s_2)\colon D(s_2\leq s_1)
\]
which are inverses of each other, hence provide the isomorphism between $D(s_1)$ and $D(s_2)$. Therefore we can meaningfully define a diagram $\bar{D}$ on $\bar{S}=S/\sim$ by setting $\bar{D}([s])=D(s)$, $\bar{D}([s_1]\leq [s_2])=D(s_1\leq s_2)$. Two maps $S\leftrightarrows \bar{S}$ described in Construction~\ref{conPreposetToPoset} prove equivalence of categories $\Diag(S;\Vv)=\Diag(\bar{S};\Vv)$. For this reason, when we speak of diagrams, sheaves or whatever, we can restrict consideration to posets. Every preordered set can be naturally transformed into a poset so that the category of diagrams do not change.
\end{rem}

%################################################
\subsection{Presheaves and sheaves} 
Now we define the notion of a sheaf and presheaf. For a general topological space $X$ let $\OpSets(X)$ denote the set of all open subsets of $X$. The set $\OpSets(X)$ is partially ordered by inclusion.

\begin{defin}\label{definPresheaf}
A \emph{presheaf} on a topological space $X$ valued in $\Vv$ is a contravariant functor $\ca{F}\colon \cat(\OpSets(X)\setminus\{\varnothing\})^{\op}\to \Vv$.
\end{defin}

\begin{rem}\label{remPresheafInformally}
$\cat(\OpSets(X)\setminus\{\varnothing\})^{\op}$ is a category, opposite to the category of the poset of open subsets of $X$. In other words, presheaf can be considered as a rule which associates, to any non-empty open subset $U\in\Omega_X$, an object $V_U=\ca{F}(U)$ of the category $\Vv$, and, to each pair of open subsets $U_1\subseteq U_2$, a morphism
\[
r_{U_2\supseteq U_1}=\ca{F}(U_2\supseteq U_1)\colon V_{U_2}\to V_{U_1}.
\]
These maps are called \emph{restriction morphisms} of a presheaf. All restriction morphisms should satisfy the compositionality conditions: $r_{U\supseteq U}=\id_{V_U}$ and $r_{U_3\supseteq U_2};r_{U_2\supseteq U_1}=r_{U_3\supseteq U_1}$.
\end{rem}

\begin{rem}\label{remSectionOfPresheaf}
In the common categories $\Vv$ like $\Sets$ or $\ko\Vect$, the notion of an element of an object $V\in\Ob(\Vv)$ make sense. The elements $x\in\ca{F}(U)$ are called \emph{sections} of a presheaf $\ca{F}$ on a set $U$. If $U_1\subseteq U_2$ and $x\in\ca{F}(U_2)$, then $\ca{F}(U_2\supseteq U_1)(x)\in \ca{F}(U_1)$ is called the restriction of the section $x$ to the subset $U_1$ and is denoted by $x|_{U_1}$.
\end{rem}

\begin{defin}\label{definSheaf}
A presheaf $\ca{F}$ is called a \emph{sheaf} if the following two properties are satisfied:
\begin{enumerate}
  \item (Locality) Let $U$ be an open set, which is covered by a family of open subsets, $\{U_{i}\}_{i\in I}$, $U_{i}\subseteq U$ for each $i\in I$, and suppose we are given two elements $s,t\in \ca{F}(U)$. If $s|_{U_{i}}=t|_{U_{i}}$ for each $i\in I$, then $s=t$.
  \item (Gluing) Let $U$ be an open set and $\{U_{i}\}_{i\in I}$ be its open covering by subsets $U_{i}\subseteq U$ and let $\{s_{i}\in \ca{F}(U_{i})\}_{i\in I}$ be a collection of sections. Suppose the sections agree on the overlap of their domains, i.e. $s_{i}|_{U_{i}\cap U_{j}}=s_{j}|_{U_{i}\cap U_{j}}$ for each $i,j\in I$. Then there exists a section $s\in \ca{F}(U)$ such that $s|_{U_{i}}=s_{i}$ for each $i\in I$.
\end{enumerate}
\end{defin}

All $\Vv$-valued presheaves on a topology $X$ form the category $\PreShvs(X,\Vv)$, whose morphisms are natural transformations of functors from $\cat(\OpSets(X)\setminus\{\varnothing\})^{\op}$ to $\Vv$. Sheaves form a complete subcategory $\Shvs(X,\Vv)$ in the category $\PreShvs(X,\Vv)$.

\begin{defin}\label{defStalk}
If $\ca{F}$ is a presheaf on a topological space $X$ and $x\in X$, then
\[
\ca{F}_x=\lim\limits_{\substack{\longrightarrow\\U\in\Omega_X, U\ni x}}\ca{F}(U)\hspace{6pt}\in \Ob(\Vv)
\]
(the direct limit of values of the presheaf over all open subsets which contain $x$), is called a \emph{stalk} of the presheaf $\ca{F}$ at a point $x$.
\end{defin}

The definition makes sense only if the category $\Vv$ has direct limits. From the general categorical definition of a direct limit~\cite{MacLane} it follows that for each $U\ni x$, $U\in \Omega_X$ there exists a canonical morphism $\ca{F}(U\ni x)\colon \ca{F}(U)\to \ca{F}_x$.

%################################################

\subsection{Relation between sheaves and diagrams} 

Apparently, diagrams on posets and sheaves on topologies are connected via the notion Alexandrov topology and Kan's extension. According to Proposition~\ref{propPosTop}, each poset $S$ corresponds to a $T_0$-Alexandrov topology $X_S$. There is a natural one-to-one correspondence between diagrams on $S$ valued in $\Vv$ and sheaves on $X_S$ valued in $\Vv$.

\begin{prop}\label{propDiagSheaf}
The category $\Diag(S,\Vv)$ of diagrams on a poset $S$ is naturally equivalent to the category $\Shvs(X_S,\Vv)$ of sheaves on the corresponding Alexandrov topology.
\end{prop}

For completenes, we review two standard constructions providing transformation of a diagram into a sheaf and vice versa.

\begin{con}\label{conSheafToDiag}
Let $\ca{F}$ be a presheaf on $X_S$. Recall that for each point $x\in S$ there is a minimal open neighbourhood $U_x$ in the topology on $X_S$ (see Construction~\ref{conSpaceToPrepos} and Remark~\ref{remMinNbhdIsACone}). If $x\leq y$, then we have $U_x\supseteq U_y$. Define a diagram $F$ on $S$ by
\[
F(x)=\ca{F}(U_x),\qquad F(x\leq y)=\ca{F}(U_x\supseteq U_y).
\]
It is easy to check that this diagram on $S$ is well-defined.
\end{con}

\begin{rem}
For a presheaf $\ca{F}$ on the Alexandrov topology we have $\ca{F}_x=\lim\limits_{\substack{\longrightarrow\\x\in U}}\ca{F}(U)=\ca{F}(U_x)$ since $U_x$ is the minimal open set containing a point $x$, thus is the terminal object of the diagram, on which the limit is taken. Therefore, Construction~\ref{conSheafToDiag} means that $F$ is actually the diagram of stalks of a presheaf $\ca{F}$.
\end{rem}

\begin{con}\label{conDiagToSheaf}
Let $D$ be a diagram on $S$. We construct the corresponding sheaf $\ca{D}$ on the Alexandrov space $X_S$ with $\Omega_S=\OpSets(X_S)$ the collection of upper order ideals. For each open set $U\in\Omega_S$ define
\begin{equation}\label{eqSheafFromDiagram}
\ca{D}(U)=\lim\limits_{\substack{\leftarrow \\ s\in U}}D(s),
\end{equation}
--- the inverse limit of the restriction of the diagram $D$ to a subcategory $\cat(U)$. For each pair $U_1\subset U_2$ we define the map $\ca{D}(U_2\supset U_1)\colon \ca{D}(U_2)\to \ca{D}(U_1)$ in a natural way, by the universal property of the inverse limit. The uniqueness of the universal object and the corresponding morphisms guarantee that $\ca{D}$ is a presheaf on $X_S$ (i.e. a commutative diagram on $\cat(\Omega_S\setminus\{\varnothing\})^{\op}$). Moreover, simple arguments involving the uniqueness and naturality of the inverse limit establish the properties of locality and gluing, which ensure that $\ca{D}$ is a sheaf on $X_S$. We call it \emph{the sheaf of sections} of the diagram $D$.
\end{con}

\begin{rem}
In more abstract terms, Construction~\ref{conDiagToSheaf} can be defined by the right Kan extension, see~\cite[Theorem 4.2.10]{Curry}.
\end{rem}

Checking that Constructions~\ref{conSheafToDiag} and~\ref{conDiagToSheaf} define the bijective correspondence between sheaves on $X_S$ and diagrams on $S$ is yet another simple exercise for the reader. Functoriality of both constructions is easily checked easily as well proving Proposition~\ref{propDiagSheaf}.

\begin{rem}\label{remSheafification}
We note that Construction~\ref{conSheafToDiag} is formulated for an arbitrary presheaf $\ca{F}$. Let us take a presheaf $\ca{F}$ on an Alexandrov space $X$ and build the diagram $F$ of its stalks by Construction~\ref{conSheafToDiag}, and then build its sheaf of sections $\overline{\ca{F}}$ by Construction~\ref{conDiagToSheaf}. Then the sheaf $\overline{\ca{F}}$ has the same stalks as the presheaf $\ca{F}$. Generally speaking, a sheaf with such properties is called the \emph{sheafification} of a presheaf $\ca{F}$. It is the left adjoint functor of the inclusion functor from $\Shvs(X;\Vv)$ to $\PreShvs(X;\Vv)$.
\end{rem}

\subsection{Global sections}

\begin{con}\label{conGlobalSections}
Let $\ca{F}\in\Shvs(X_S;\Vv)$ be a sheaf. The value
\[
\Gamma(X_S;\ca{F})=\ca{F}(X_S)\in\Vv
\]
of this sheaf on the whole space $X_S$ is called the set\footnote{Formally it is not the set, but rather an object of a category $\Vv$. However, since most common examples of categories $\Vv$ are concrete, it is not a big mistake to call it a set.} of global sections of the sheaf $\ca{F}$. From the definition of a sheaf it follows that global sections are functorial: if we have a morphism $f\colon \ca{D}\to\ca{F}$ of sheaves, we naturally have a map $\Gamma(X_S;f)\colon \ca{D}(X_S)\to\ca{F}(X_S)$ of their values. Henceforth, we have a functor
\begin{equation}\label{eqFunctorOfGlobalSects}
\Gamma(X_S,\cdot)\colon \Shvs(X_S;\Vv)\to \Vv,
\end{equation}
which is called \emph{the functor of global sections}.
\end{con}

\begin{con}\label{conInverseLimit}
Proposition~\ref{propDiagSheaf} states $\Shvs(X_S;\Vv)=\Diag(S;\Vv)$. In the language of diagrams over $S$, we have the following description of the global sections. Let $D\in \Diag(S;\Vv)$ be a diagram. Then $\Gamma(S,D)$ is the value of the corresponding sheaf $\ca{D}$ on the space $X_S$. This value is defined as the inverse limit
\[
\Gamma(S,D)=\lim\limits_{\substack{\leftarrow \\ s\in S}}D(s)
\]
according to the general Construction~\ref{conDiagToSheaf}. This is the reason why the functor of global sections is called \emph{the inverse limit functor} in topological and algebraical sources, see~\cite{careil1956homalg, grothendieck1957tohoku}.
\end{con}

\begin{rem}\label{remConcreteLimSets}
We provide an explicit construction of the inverse limit in the category $\ko\Vect$ in order to make formula~\eqref{eqSheafFromDiagram} more meaningful for readers not familiar with abstract nonsense. Essentially the procedure of taking inverse limits is the most essential ingredient in the applications of sheaves: this part of the theory deals with solving equations. We have
\begin{equation}\label{eqInvLimit}
\ca{D}(U)=\lim\limits_{\substack{\leftarrow \\ s\in U}}D(s)=\left\{(x_s\mid s\in U)\in\prod\nolimits_{s\in U}D(s)\mid D(s_1\leq s_2)x_{s_1}=x_{s_2} \text{ for any }s_1\leq s_2 \right\}.
\end{equation}
In other words, a section $x\in \ca{D}(U)$ on an open set $U$ coincides with a collection of vectors $x_s\in D(s)$, one for each $s\in U$, which are compatible with respect to the maps of the diagram. We also call such collection of vectors coherent. In the form~\eqref{eqInvLimit}, the restriction map $\ca{D}(U_2\supset U_1)$ is written in an straightforward manner, because the family of compatible stalks on the set $U_2$ is also compatible on its subset $U_1$.
\end{rem}

We give an important methodological remark, to which we refer in the analysis of the sheaves' applications.

\begin{rem}\label{remLimsToProductsEqualizers}
A construction similar to~\eqref{eqInvLimit} makes sense in other common categories such us $\Sets$, $R\Mod$, $\Top$, etc. In more general categorical terms, there is a standard statement (see e.g.~\cite[\S V.2]{MacLane}) that procedure of taking inverse limits (over small or finite, respectively, categories) reduces to taking small or finite, respectively, products and equalizers. In practice, this means that dealing with general data-structures $\Vv$ as the values of a (finitely supported) sheaf $\ca{F}$, the generation of a global section $\ca{F}(U)$ reduces to two procedures as follows.
\begin{enumerate}
  \item Generation of finite products. Usually not a problem, since Cartesian products are supported by all common data structures. E.g. concatenation of vectors is ubiquitous in data science, and is rarely explicitly mentioned at all.
  \item Generation of equalizers. In practice this means solving (systems of) equations of the form ``Find $x\in V$, such that $F(x)=G(x)$'' where $V,W\in \Ob(\Vv)$ and $F,G\in \Hom(V, W)$ are given. This step is usually hard, but it makes the theory substantial. The solution of equations (as a process) is the substantial part of sheaves' inspired architectures of neural networks, as defined in subsection~\ref{subsecSheafLearning} and reviewed in Section~\ref{secReviewShvsML}.  
\end{enumerate}
\end{rem}

%################################

\subsection{Functorial properties}\label{subsecMathFunctorial}

Here we discuss how the maps of spaces and posets interact with sheaves and diagrams. We won't retell the whole six-functor formalism of Grothendieck here and restrict to the necessary minimum. In general, a continuous map of topological spaces $f\colon X\to Y$ induces direct and inverse image functors on categories of sheaves
\[
f_*\colon \Shvs(X;\Vv)\to \Shvs(Y;\Vv), \qquad f^*\colon \Shvs(Y;\Vv)\to \Shvs(X;\Vv).
\]
For the details of this construction in general, we refer to~\cite{Iversen}. In case of posets the definition of the functor $f^*$ (inverse image of a sheaf) is significantly simplified at the conceptual level and, as a result, finds numerous applications in applied topology and contemporary approaches to neural networks.

Let $f\colon S\to T$ be a morphism of posets (equiv, a continuous map of the corresponding Alexandrov topological spaces), see Proposition~\ref{propPosTop}.

\begin{defin}\label{defDirectImage}
Let $\ca{F}\in\Shvs(X_S;\Vv)$ be a sheaf on the Alexandrov space $X_S$. Then its \emph{direct image} is a presheaf $f_*\ca{F}$ on the topological space $X_T$ whose values on open subsets $U\in\OpSets(X_S)$ and inclusions $U_1\supseteq U_2$ are defined by the following formulas:
\[
f_*\ca{F}(U)=\ca{F}(f^{-1}(U)),\qquad f_*\ca{F}(U_1\supseteq U_2)=\ca{F}(f^{-1}(U_1)\supseteq f^{-1}(U_2)).
\]
\end{defin}

A quite standard proposition states that the direct image $f_*\ca{F}$ is a sheaf. The map $f_*$ is functorial, i.e. it is a well-defined functor from the category $\Shvs(X_S;\Vv)$ to the category $\Shvs(X_T;\Vv)$. The inverse image functor is more intricate, since an image of an open set under a continuous map may fail to be open. However, for posets and Alexandrov topologies, the construction of the inverse image functor is simplified due to the equivalence between sheaves and diagrams over posets. Assume again that $f\colon S\to T$ is a morphism of posets and $\ca{D}\in\Shvs(X_T;\Vv)$ is a sheaf on the topological space $X_T$. According to Proposition~\ref{propDiagSheaf}, sheaf $\ca{D}$ corresponds to a diagram $D\colon \cat(T)\to\Vv$.

\begin{defin}\label{defInverseImage}
The \emph{inverse image} of a diagram $D$ on $T$ (or the corresponding sheaf $\ca{D}$ on $X_T$) under a morphism $f\colon S\to T$ is
a diagram $f^*D$ on the poset $S$ (or the corresponding sheaf $f^*\ca{D}$ on a space $X_S$), defined by the formulas\footnote{In algebraic geometry there is a difference between $f^{-1}$ and $f^*$, and here we are actually defining $f^{-1}$. Since we do not work with ringed spaces in this review, we prefer to keep notation simple and use symbol $f^*$ for inverse image.}:
\[
f^*D(s)=D(f(s)),\qquad f^*D(s\leq s')=D(f(s)\leq f(s')).
\]
\end{defin}

It can be seen that $f^*$ is functorial, so that we have a functor $f^*\colon\Shvs(X_T;\Vv)\to \Shvs(X_S;\Vv)$. The proof of the next statement can be found in~\cite[Theorem 5.3.1]{Curry} for sheaves on finite spaces, or in~\cite[Theorem 4.8]{Iversen} for general topological spaces.
%in any general book on homological algebra in general

\begin{prop}\label{propAdjointPair}
The functors of direct and inverse image form an adjoint pair. This means there is a natural bijection
\[
\Hom_{\Shvs(X_T,\Vv)}(\ca{D},f_*\ca{F})=\Hom_{\Shvs(X_S,\Vv)}(f^*\ca{D},\ca{F})
\]
for any $\ca{F}\in\Shvs(X_S,\Vv)$ and $\ca{D}\in\Shvs(X_T,\Vv)$.
\end{prop}

Recalling Proposition~\ref{propDiagSheaf}, we have a sequence of natural bijections
\begin{multline}\label{eqAllBijections}
\Hom_{\Diag(T,\Vv)}(D,f_*F)=\Hom_{\Shvs(X_T,\Vv)}(\ca{D},f_*\ca{F})=\\
\Hom_{\Shvs(X_S,\Vv)}(f^*\ca{D},\ca{F})=\Hom_{\Diag(S,\Vv)}(f^*D,F).
\end{multline}



\section{Sheaf cohomology}\label{secMathCohomology}

This section describes the definition of sheaf cohomology, and introduces various computational approaches to cohomology calculations. An important role here is played by additive, and more generally abelian categories, and additive functors between them. The basic examples of abelian categories are the categories $\Abel$ of abelian groups, the category $\ko\Vect$ of vector spaces over a field $\ko$, or the category $R\Mod$ of modules (e.g. left modules) over a ring $R$. For a general introduction to abelian categories we refer to e.g.~\cite{GelMan}. We are mainly interested in Grothendieck categories, the particularly well-behaved subclass of abelian categories, to which the listed examples belong~\cite[Ch.5,\SS8-9]{BucurDeleanu}.



\subsection{Recap on (co)chain complexes}\label{subsecMathCochainRecap}

In this subsection we review the basic toolbox of homological algebra: cochain complexes and exact sequences.

\begin{defin}\label{definCochainCpx}
Assume $\Vv$ is an abelian category. The sequence
\begin{equation}\label{eqCochainCpx}
0\to C^0\stackrel{d_0}{\to} C^1 \stackrel{d_1}{\to} C^2 \stackrel{d_2}{\to}\cdots
\end{equation}
of objects of $\Vv$ and morphisms between them is called a \emph{(connective) cochain complex}\footnote{We consider only connective cochain complexes, i.e. indexed with nonnegative integers, because only these complexes appear in the review.} if $d_j;d_{j+1}=0$ for any $j=0,1,\ldots$ (sometimes the numeration starts with $-1$). For short, a cochain complex~\eqref{eqCochainCpx} is denoted $(C^*,d)$.
\end{defin}

The definition implies the natural inclusion $p_j\colon \im d_j\hookrightarrow \Ker d_{j+1}$.

\begin{defin}\label{definCohomologyCochain}
The \emph{cohomology of the cochain complex} $(C^*,d)$ are defined as the objects
\[
H^j(C^*,d)=\Ker d_{j+1}/\im d_j = \Coker p_j \mbox{ for }j=0,1,\ldots
\]
of the category $\Vv$, indexed by nonnegative numbers.
\end{defin}

A cochain complex $(C^*,d)$ --- or any sequence of the form~\eqref{eqCochainCpx} --- is called \emph{exact} (resp. exact at position $i$), if $H^j(C^*,d)=0$ for any $j$ (resp. for $j=i$). Equivalently, $\im d_j=\Ker d_{j+1}$.

\begin{con}\label{conCategoryCochain}
Connective cochain complexes over $\Vv$ form a category, in which the maps (called \emph{cochain maps}) are defined componentwise. A cochain map $f\colon (C^*,d_C)\to (B^*,d_B)$ is defined if the maps $f_j\colon C^j\to B^j$ are defined for any $j$, commuting with the differentials $f_j;d_b=d_C;f_{j+1}$. Each such map induces the map of cohomology $f_*\colon H^j(C^*,d_C)\to H^j(B^*,d_B)$. Let us denote the category of cochain complexes over $\Vv$ by $\Cochain(\Vv)$. The category $\Cochain(\Vv)$ is an abelian category itself with kernels and cokernels defined componentwise in a straightforward manner.
\end{con}

The following lemma is classical, see e.g.~\cite[Lemma 1.3.2]{weibel1994homalg}.

\begin{lem}[Zig-zag lemma]\label{lemZigZag}
A short exact sequence
\[
0\to (C^*,d_C)\to (B^*,d_B)\to (A^*,d_A)\to 0
\]
in $\Cochain(\Vv)$ induces the long exact sequence of cohomology
\begin{equation}\label{eqLongExactCochain}
\begin{tikzpicture}[descr/.style={fill=white,inner sep=1.5pt}]
        \matrix (m) [
            matrix of math nodes,
            row sep=1em,
            column sep=2.5em,
            text height=1.5ex, text depth=0.25ex
        ]
        { 0 & H^0(C^*,d_C) & H^0(B^*,d_B) & H^0(A^*,d_A) & \\
            & H^1(C^*,d_C) & H^1(B^*,d_B) & H^1(A^*,d_A) & \\
            & H^2(C^*,d_C) & H^2(B^*,d_B) & H^2(A^*,d_A) & \cdots\\
%            & \mbox{}         &                 & \mbox{}         \\
%            & H^n(\mathcal A) & H^n(\mathcal B) & H^n(\mathcal C) \\
        };

        \path[overlay,->, font=\scriptsize,>=latex]
        (m-1-1) edge (m-1-2)
        (m-1-2) edge (m-1-3)
        (m-1-3) edge (m-1-4)
        (m-1-4) edge[out=355,in=175] node[descr,yshift=0.3ex] {} (m-2-2)
        (m-2-2) edge (m-2-3)
        (m-2-3) edge (m-2-4)
        (m-2-4) edge[out=355,in=175] node[descr,yshift=0.3ex] {} (m-3-2)
        (m-3-2) edge (m-3-3)
        (m-3-3) edge (m-3-4)
        (m-3-4) edge (m-3-5);
%        (m-3-4) edge[out=355,in=175,dashed,red] (m-5-2)
%        (m-5-2) edge (m-5-3)
%        (m-5-3) edge (m-5-4);
\end{tikzpicture}
    %0\to H^0(C^*,d_C)\to H^0(B^*,d_B)\to H^0(A^*,d_A)\to\\
%    \to H^1(C^*,d_C)\to H^1(B^*,d_B)\to H^1(A^*,d_A)\to\\
%    \to H^2(C^*,d_C)\to H^2(B^*,d_B)\to H^2(A^*,d_A)\to \cdots
\end{equation}
\end{lem}

\subsection{Sheaf cohomology as derived functors}\label{subsecMathCohomologyDerived}

Sheaf cohomology, as they are understood in algebraic geometry and homological algebra, are the derived functor of the global sections' functor. In this subsection we give the necessary definitions, constructions, and explain the specifics of finite topological spaces in terms of this definition. This subsection is entirely covered by Curry's thesis~\cite{Curry}: it is provided here for convenience and also to fix the notation for some important objects to be defined later. 

Let $\Vv$ be an abelian category. Then, for any topological space $X$, the category $\Shvs(X,\Vv)$ is an abelian category as well. If $X=X_S$ is the Alexandrov space corresponding to a poset $S$, then we have an abelian category of diagrams $\Diag(S,\Vv)=\Shvs(X_S,\Vv)$ (see Propositions~\ref{propPosTop} and~\ref{propDiagSheaf}).

\begin{con}\label{conDiagIsAbelian}
In terms of diagrams over posets, the characteristic properties of the abelian category $\Diag(S,\Vv)$ are the following. As the illustrative example one can imagine diagrams of vector spaces $\Diag(S,\ko\Vect)$.
\begin{enumerate}
  \item The \emph{zero diagram} $0\in \Diag(S,\Vv)$ is the diagram made of zero vector spaces and zero maps.
  \item There exist finite \emph{biproducts}, also called \emph{direct sums}. If $D_1,D_2\in \Diag(S,\Vv)$ are two diagrams, then their direct sum $D_1\oplus D_2$ is defined component-wise:
      \[
      (D_1\oplus D_2)(s)=D_1(s)\oplus D_2(s)
      \]
      and
      \[
      (D_1\oplus D_2)(s\leq t)=D_1(s\leq t)\oplus D_2(s\leq t).
      \]
  \item \emph{Kernel} and \emph{cokernel} of a morphism of diagrams are defined component-wise. If $G\colon D_1\to D_2$ is a morphism of diagrams, then the kernel $K\colon \Ker G\hookrightarrow D_1$ is defined by
      \[
      (\Ker G)(s)=\Ker G(s),\mbox{ the maps } (\Ker G)(s\leq t)\colon \Ker G(s)\to \Ker G(t)
      \]
      are naturally induced by $D_1(s\leq t)$, and $K(s)\colon (\Ker G)(s)\hookrightarrow D_1(s)$ is a natural inclusion. Cokernels are completely similar.
  \item Every monomorphism is a kernel of its cokernel, and every epimorphism is a cokernel of its kernel. Again, easily verified component-wise.
\end{enumerate}
\end{con}

%\begin{con}\label{conAbCats}
%A functor between two abelian categories is called \emph{additive} if it preserves finite direct sums and zeroes. Let us denote by $\AbCat$ the category of all abelian categories and additive functors between them as morphisms. Fixing a particular ground abelian category, say $\ko\Vect$, we obtain a functor
%\[
%\Shvs(\cdot,\ko\Vect)\colon \Top\to\AbCat,\quad X\mapsto \Shvs(X,\ko\Vect).
%\]
%\end{con}

The next proposition follows from the 3-rd item of Construction~\ref{conDiagIsAbelian}. It constitutes an important property of diagrams over posets: exactness of sequences of sheaves can be verified stalk-wise. 

\begin{prop}\label{propExactSequenceDiagrams}
A sequence $0\to D_0\to D_1\to D_2 \to\cdots$ of diagrams over a poset $S$ is exact if and only if the corresponding sequence of stalks
\[
0\to D_0(s)\to D_1(s)\to D_2(s)\to\cdots
\]
is exact for each $s\in S$.
\end{prop}

\textbf{Sheaves have enough injectives.} In order to deal with derived functors, one needs to assure that the abelian category of sheaves has enough injectives. For a general definition of injective object in an abelian category see~\cite[\S2.3]{weibel1994homalg}. An abelian category $\ST{A}$ is said to \emph{have enough injectives} if, for any object $A\in\ST{A}$ there exists a monomorphism from $A$ to an injective object of $\ST{A}$. The categories $\ko\Vect$, $\Abel$, $R\Mod$ are known to have enough injectives~\cite[\S 2.3]{weibel1994homalg}. For example, in the ``most applied'' category $\ko\Vect$ every object is injective. This simplifies many arguments and constructions in this case.

\begin{prop}[{\cite[Thm.1.10.1 and Prop.3.1.1]{grothendieck1957tohoku}}]\label{propEnoughInjectives}
If $\Vv$ has enough injectives, then $\Shvs(X,\Vv)$ also has enough injectives.
\end{prop}

Below, we provide the proof for finite topologies, following Curry~\cite[Def.7.1.3]{Curry}.

\begin{con}\label{conInjSheaves}
Let $s\in S$ and $V\in\Vv$. Consider the diagram $\low{s}{V}$ on $S$, defined by
\begin{equation}\label{eqDefinElementaryInj}
\low{s}{V}(t)=\begin{cases}
               V, & \mbox{if } t\leqslant s \\
               0, & \mbox{otherwise}.
             \end{cases}
\end{equation}
Mappings between the non-zero elements of the diagram are set to be the identity isomorphisms. The diagrams of this form, and their corresponding sheaves, will be called \emph{cone-shaped sheaves} on $S$ (with apex $s$ and value $V$). Notice the analogy with the definition of the cone of an element, Construction~\ref{conCones}. It can be seen that $\low{s}{V}$ coincides with \emph{the skyscraper sheaf} on $X_S$, concentrated in $s$, known in the classical homological algebra~\cite[Ex.2.3.12]{weibel1994homalg}.
\end{con}

\begin{lem}\label{lemElementaryInjective}
If $W$ is an injective object in $\Vv$, then $\low{s}{W}$ is an injective object in $\Diag(S,\Vv)$ (equiv. an injective sheaf on the space $X_S$).
\end{lem}

\begin{proof}
Let $\pt$ be the one-point poset. The category of sheaves on $\pt$ can be naturally identified with the category $\Vv$. Thus, $W$ can be considered as injective sheaf on $\pt$. Consider the inclusion map $i_s\colon \pt\to S$, $i_s(\pt)=s$. It is easily verified that $\low{s}{W}=(i_s)_*W$. Hence
\begin{equation}\label{eqIdentificationHoms}
\Hom_{\Diag(S,\Vv)}(F,\low{s}{W})= \Hom_{\Diag(S,\Vv)}(F,(i_s)_*W)=\Hom_{\Diag(\pt,\Vv)}(i_s^*F,W)=\Hom_{\Vv}(F(s),W).
\end{equation}
It is known~\cite[Lm.2.3.4]{weibel1994homalg} that an object $I$ of an abelian category $\ST{A}$ is injective if and only if the functor $\Hom_{\ST{A}}(\cdot, I)$ is exact. Taking stalks at the point $s$ is an exact functor from $\Diag(S,\Vv)$ to $\Vv$, see Proposition~\ref{propExactSequenceDiagrams}. Therefore, since $W$ is an injective object of $\Vv$, the functor
\[
(\ca{F}\in\Diag(S,\Vv))\mapsto \Hom_{\Vv}(\ca{F}(s),W)=\Hom_{\Diag(S;\Vv)}(\ca{F},\low{s}{W})
\]
is exact, hence the diagram (the sheaf) $\low{s}{W}$ is injective.
\end{proof}

Proposition~\ref{propEnoughInjectives} is proved by the following construction, which is a particular case of the Godement's construction~\cite{Godement1973} applied to a finite topology~\footnote{It should be noticed that general Godement's construction works with direct products which, for infinite indexing sets, do not coincide with direct sums, or coproducts. Therefore the assumption of finiteness of $S$ is important in this and subsequent constructions.}.

\begin{con}\label{conGodementConstruction}
Let $D\in\Diag(S,\Vv)$ be an arbitrary diagram on a poset. For each element $s\in S$ consider a monomorphism $j_s\colon D(s)\to J_s$ to some injective object of $\Vv$ (which exists since we assumed $\Vv$ has enough injectives). Consider the following map
\begin{equation}\label{eqGodementPoset}
j\colon D\to \bigoplus_{s\in S}\low{s}{J_s}, \quad j=\bigoplus_{s\in S}\tilde{j}_s,
\end{equation}
where $\tilde{j}_s\colon D\to\low{s}{J_s}$ is the map which corresponds to $j_s\colon D(s)\to J_s$ via the natural correspondence described in~\eqref{eqIdentificationHoms}. The map $j$ is monomorphic stalk-wise: it is monomorphic on an element $s$ since already its summand $\tilde{j}_s$ (which is basically $j_s$) is monomorphic.

In short, we mapped a diagram $D$ into the direct sum of the cone-shaped diagrams valued by objects in which the values of $D$ inject.
\end{con}

\begin{rem}\label{remInjIsSumOfCones}
Construction~\ref{conGodementConstruction} asserts, that every sheaf embeds monomorphically into a direct sum of cone-shaped sheaves, which is injective. In the following, when we speak about injective sheaves on Alexandrov topologies, we usually mean ``sums of cone-shaped sheaves'' since no other concrete constructions of injective sheaves appear in our work.
\end{rem}

\textbf{Derived functors.} The most important functor in sheaf theory is the functor of global sections described in Construction~\ref{conGlobalSections}. If $\Vv$ is abelian, then the functor of global sections $\Gamma(X_S;\cdot)$ (or $\Gamma(S;\cdot)=\lim\limits_{\substack{\leftarrow, S}}(\cdot)$ in terms of posets) can be seen to be additive.

In homological algebra, given two sufficiently nice abelian categories $\Ww$, $\Vv$ and an additive functor $\Fu$ between them, it is possible to define an infinite sequence of functors, called the derived functors of $\Fu$. %Higher order cohomology of a sheaf are formally defined as a derived functor of the functor of global sections $\Gamma(X_S;\cdot)\colon \Shvs(X_S;\Vv)\to \Vv$, as described below. 
Some steps of this definition appear useful in the following, so we recall the abstract definitions with a moderate level of mathematical details. The main purpose of this rather classical description is to check its algorithmical feasibility for finite diagrams of finite-dimensional vector spaces.

\begin{defin}\label{definExactFunctor}
Let $\Fu\colon \Ww\to \Vv$ be an additive functor between two abelian categories. The functor $\Fu$ is called \emph{left exact} if every short exact sequence in $\Ww$
\[
0\to W_1 \to W_2\to W_3\to 0
\]
induces the sequence
\begin{equation}\label{eqLeftExact}
0\to \Fu(W_1) \to \Fu(W_2) \to \Fu(W_1)
\end{equation}
which is exact at all positions except the rightmost.
\end{defin}

Assume that $\Ww$ has enough injectives. Let us define the right derived functors $R^j\Fu\colon\Ww\to\Vv$. Actually, we will concentrate on constructing $R^j\Fu(W)$ for a given object $W\in\Ww$, not a morphism. The tedious check of functoriality is based on the horseshoe lemma and may be found elsewhere~\cite[Theorem 2.3.7]{weibel1994homalg}.

\begin{con}\label{conDerivedFunctorGeneral}
Let $W\in\Ww$ be an object. Since $\Ww$ has enough injectives, there exists a monomorphic map $\imath_0\colon M\to I_0$ into some injective object. Now let us take an injective hull of $\Coker \imath_0$, which gives a map $\imath_1\colon I_0\to I_1$. Proceeding the same way inductively, we get a potentially infinite exact sequence
\begin{equation}\label{eqInjectiveResolGen}
0\to W\stackrel{i_0}{\to}I_0\stackrel{i_1}{\to}I_1\stackrel{i_2}{\to}I_2\stackrel{i_3}{\to}\cdots
\end{equation}
in which all objects $I_j$ are injective. Such a sequence is called an \emph{injective resolution} of $W$; it may be non-unique. Now apply the functor $\Fu$ to the sequence~\eqref{eqInjectiveResolGen} cut at the leftmost position:
\begin{equation}\label{eqFuncOfInjective}
0\to\Fu(I_0)\stackrel{\Fu(i_1)}{\longrightarrow}\Fu(I_1)\stackrel{\Fu(i_2)}{\longrightarrow}\Fu(I_2)\stackrel{\Fu(i_3)}{\to}\cdots
\end{equation}
This sequence may fail to be exact, but it is a cochain complex in $\Vv$. The cohomology of this cochain complex (see Definition~\ref{definCohomologyCochain}) are called the \emph{values of right derived functors} of $\Fu$ on $W$:
\[
R^j\Fu(W)=H^j(\Fu(I^*,\imath)).
\]
\end{con}

For convenience we provide the list of basic claims about derived functors in general, the proofs can be found e.g. in~\cite[\S 2.5]{weibel1994homalg}.

\begin{prop}\label{propDerivedFunctorProps}
The following properties hold for the derived functors $R^j\Fu$.
\begin{enumerate}
  \item $R^j\Fu(W)$ are well-defined up to natural isomorphism. This basically means that these objects of $\Vv$ do not depend on the choice of injective resolution of $W$.
  \item $R^0\Fu=\Fu$.
  \item Each short exact sequence $0\to W_1\to W_2\to W_3\to 0$ in $\Ww$ induces the long exact sequence
\begin{equation}\label{eqLongExactDerived}
\begin{tikzpicture}[descr/.style={fill=white,inner sep=1.5pt}]
        \matrix (m) [
            matrix of math nodes,
            row sep=1em,
            column sep=2.5em,
            text height=1.5ex, text depth=0.25ex
        ]
        { 0 & R^0\Fu(W_1) & R^0\Fu(W_2) & R^0\Fu(W_3) & \\
            & R^1\Fu(W_1) & R^2\Fu(W_2) & R^1\Fu(W_3) & \\
            & R^2\Fu(W_1) & R^2\Fu(W_2) & R^3\Fu(W_1) & \cdots\\
%            & \mbox{}         &                 & \mbox{}         \\
%            & H^n(\mathcal A) & H^n(\mathcal B) & H^n(\mathcal C) \\
        };

        \path[overlay,->, font=\scriptsize,>=latex]
        (m-1-1) edge (m-1-2)
        (m-1-2) edge (m-1-3)
        (m-1-3) edge (m-1-4)
        (m-1-4) edge[out=355,in=175] node[descr,yshift=0.3ex] {} (m-2-2)
        (m-2-2) edge (m-2-3)
        (m-2-3) edge (m-2-4)
        (m-2-4) edge[out=355,in=175] node[descr,yshift=0.3ex] {} (m-3-2)
        (m-3-2) edge (m-3-3)
        (m-3-3) edge (m-3-4)
        (m-3-4) edge (m-3-5);
%        (m-3-4) edge[out=355,in=175,dashed,red] (m-5-2)
%        (m-5-2) edge (m-5-3)
%        (m-5-3) edge (m-5-4);
\end{tikzpicture}
\end{equation}
%    0\to R^0\Fu(W_1)\to R^0\Fu(W_2)\to R^0\Fu(W_3)\to\\
%    \to R^1\Fu(W_1)\to R^1\Fu(W_2)\to R^1\Fu(W_3)\to\\
%    \to R^2\Fu(W_1)\to R^2\Fu(W_2)\to R^2\Fu(W_3)\to \cdots
%  \end{multline}
  which extends the sequence~\eqref{eqLeftExact}.
\end{enumerate}
\end{prop}

\textbf{Sheaf cohomology.} In order for the definition~\ref{definCohomologyMainPart} of sheaf cohomology to make sense, a bit more preparatory work should be done.

\begin{prop}\label{propGlobSectLeftExact}
The functor $\Gamma(S,\cdot)$ is left exact.
\end{prop}

This is derived from the fact that $\Gamma(S;\cdot)$ is right adjoint to the constant sheaf functor, see~\cite[\S 2.6]{weibel1994homalg}. The abelian category $\Shvs(X_S,\Vv)$ has enough injectives by Proposition~\ref{propEnoughInjectives}, therefore the derived functors of the functor $\Gamma(X_S;\cdot)$ are well defined according to Construction~\ref{conDerivedFunctorGeneral}. The next definition duplicates Definition~\ref{definCohomologyMainPart} given in the main part of the text.

\begin{defin}\label{definSheafCohomologyDerived}
The $j$-th right derived functor $R^j\Gamma(X_S;\cdot)$ is called \emph{the functor of $j$-th cohomology of a sheaf}. We use the notation\footnote{The notation AG stands for algebraic geometry. It may rightfully refer to Alexander Grothendieck.}
\[
H^j_{AG}(X_S;\ca{F})=R^j\Gamma(X_S;\ca{F}).
\]
\end{defin}

Applying Proposition~\ref{propDerivedFunctorProps} to the functor $\Gamma(X_S;\cdot)\colon\Shvs(X_S;\Vv)\to\Vv$ of global sections, we obtain the main properties of the sheaf cohomology.

\begin{prop}\label{propCohomologyProperties}
Consider sheaves on the Alexandrov topology $X_S$ corresponding to a poset $S$. The following properties hold true for the cohomology.
\begin{enumerate}
  \item Cohomology modules $H_{AG}^j(X_S;\ca{F})$ are well-defined and functorial with respect to a sheaf-component.
  \item The 0-degree cohomology $H_{AG}^0(X_S;\ca{F})$ coincide with the global sections $\Gamma(X_S;\ca{F})$.
  \item Each short exact sequence $0\to \ca{F}_1\to \ca{F}_2\to \ca{F}_3\to 0$ of sheaves over $X_S$ induces long exact sequence in cohomology:
%  \begin{multline}
	\begin{equation}\label{eqLongExactDerivedCohomology}
\begin{tikzpicture}[descr/.style={fill=white,inner sep=1.5pt}]
        \matrix (m) [
            matrix of math nodes,
            row sep=1em,
            column sep=2.5em,
            text height=1.5ex, text depth=0.25ex
        ]
        { 0 & \Gamma(X_S;\ca{F}_1) & \Gamma(X_S;\ca{F}_2) & \Gamma(X_S;\ca{F}_3) & \\
            & H_{AG}^1(X_S;\ca{F}_1) & H_{AG}^1(X_S;\ca{F}_2) & H_{AG}^1(X_S;\ca{F}_3) & \\
            & H_{AG}^2(X_S;\ca{F}_1) & H_{AG}^2(X_S;\ca{F}_2) & H_{AG}^2(X_S;\ca{F}_3) & \cdots\\
%            & \mbox{}         &                 & \mbox{}         \\
%            & H^n(\mathcal A) & H^n(\mathcal B) & H^n(\mathcal C) \\
        };

        \path[overlay,->, font=\scriptsize,>=latex]
        (m-1-1) edge (m-1-2)
        (m-1-2) edge (m-1-3)
        (m-1-3) edge (m-1-4)
        (m-1-4) edge[out=355,in=175] node[descr,yshift=0.3ex] {} (m-2-2)
        (m-2-2) edge (m-2-3)
        (m-2-3) edge (m-2-4)
        (m-2-4) edge[out=355,in=175] node[descr,yshift=0.3ex] {} (m-3-2)
        (m-3-2) edge (m-3-3)
        (m-3-3) edge (m-3-4)
        (m-3-4) edge (m-3-5);
\end{tikzpicture}
\end{equation}
\end{enumerate}
\end{prop}

Next, we provide more constructive details on injective resolutions of sheaves over Alexandrov spaces and their cohomology.

\begin{con}\label{conSheafCohomDerivedStyle}
Assume $D$ is a diagram over a poset $S$ (equiv. sheaf $\ca{D}$ over the Alexandrov space $X_S$), and as before, $\Vv$ is an abelian category with enough injectives. In the abelian category $\Diag(S,\Vv)=\Shvs(X_S,\Vv)$ there exists an injective resolution
\begin{equation}\label{eqInjSheaf}
0\to \ca{D}\to\ca{I}_0\to\ca{I}_1\to\ca{I}_3\to\cdots
\end{equation}
that is an exact sequence of sheaves, in which all $\ca{I}_j$ are injective. Moreover, according to Construction~\ref{conGodementConstruction} we may assume that each $\ca{I}_j$ is a direct sum of the sheaves of the form $\low{s}{W}$ with $s\in S$ and $W$ an injective object of $\Vv$. Now we apply the functor of global sections to the non-augmented resolution:
\begin{equation}\label{eqGlobSectionsOfInjRes}
0\to\ca{D}(X_S)\to\ca{I}_0(X_S)\to\ca{I}_1(X_S)\to\ca{I}_2(X_S)\to\cdots
\end{equation}
which, in the language of posets, the same as the following sequence
\begin{equation}\label{eqGlobSectionsOfInjResLimits}
0\to\lim\limits_{\leftarrow S} D\to \lim\limits_{\leftarrow S} I_0\to\lim\limits_{\leftarrow S} I_1\to\lim\limits_{\leftarrow S} I_2\to\cdots
\end{equation}
according to Construction~\ref{conGlobalSections}. The sequence~\eqref{eqGlobSectionsOfInjRes} is obviously a cochain complex in $\Vv$. Its cohomology are called the cohomology of the sheaf $\ca{D}$, or cohomology of category $\cat(S)$ with coefficients in a functor $D$ (or simply cohomology of a diagram $D$):
\begin{equation}\label{eqCohomOfSheafAG}
H^j_{AG}(X_S;\ca{D})=H^j_{AG}(S;D)=\Ker(d\colon \ca{I}_j(X_S)\to \ca{I}_{j+1}(X_S))/\im(d\colon \ca{I}_{j-1}(X_S)\to \ca{I}_j(X_S)).
\end{equation}
\end{con}

In view of this construction, it is useful to compute the global sections of the basic ``injective block'' $\low{s}{W}$ used in the construction of injective resolution.

\begin{lem}\label{lemGlobSecOfInjectives}
For a sheaf $\low{s}{W}$ defined in Construction~\ref{conInjSheaves}, the global sections $\Gamma(X_S;\low{s}{W})=\low{s}{W}(X_S)$ are naturally isomorphic to $W$. Moreover, if $s\leq t$ in $S$ and the map $g\colon \low{s}{W}\to \low{t}{V}$ is induced by $f\colon W\to V$, that is defined on stalks by
\[
g(s')=\begin{cases}
        f\colon W=\low{s}{W}(s')\to V=\low{t}{V}, & \mbox{if } s'\leq s \\
        0, & \mbox{otherwise}.
      \end{cases}
\]
then the induced map $g_*\colon \Gamma(X_S;\low{s}{W})\to \Gamma(X_S;\low{t}{V})$ coincides with $f\colon W\to V$.
\end{lem}

The proof is a simple exercise: on the universality of an inverse limit if it is proved in an arbitrary category, or on the application of Remark~\ref{remConcreteLimSets} if one prefers to stay concrete.

\begin{rem}\label{remOpinionLeader}
It follows from Construction~\ref{conSheafCohomDerivedStyle}, that higher cohomology $H^{>0}(S;\low{s}{W})$ of the basic injective sheaf vanish. Indeed, an injective resolution of $\low{s}{W}$ has the form
\[
0\to \low{s}{W}\to \low{s}{W}\to 0\to 0\to\cdots
\]
so the global sections of its reduced version give the cochain complex $0\to W\to 0\to\cdots$, with no higher order cohomology.

In subsection~\ref{subsecCohomologyMainPart} it was explained that sheaf cohomology provide a language to speak about ways of information distribution and copying in a network. In terms of these allusion, the skyscraper sheaf $\low{s}{W}$ encodes the situation, when we have a single node $s\in S$, the opinion leader, which possesses information $W$, and all lower nodes of a poset copy this information identically (follow the leader). A consensus state of such sheaf is determined entirely by a state $x_s\in \low{s}{W}(s)=W$ of the opinion leader hence we have $H^0(S;\low{s}{W})=\Gamma(S;\low{s}{W})\cong W$ as stated in Lemma~\ref{lemGlobSecOfInjectives}. Each node $t$ receives its state in $\low{s}{W}(t)$ in a single way: by copying it from $\low{s}{W}(s)$ if $t\leq s$, or by setting it to $0$ otherwise. Therefore, higher cohomology of $\low{s}{W}$ vanish. Although the implication ``injective $\Rightarrow$ acyclic'' is mathematically tautological, it is fairly consistent with the common-sense interpretation demonstrated in Example~\ref{exPathVsCircle}.

The procedure of constructing injective resolution can be informally understood as a decomposition of an arbitrary medium (sheaf) $D$ on $S$ into a combination of simpler media $\low{s}{W}$: these media have opinion leaders hence are easier to deal with\footnote{This certainly looks like a good reason for political engineers to start using injective resolutions in their work.}.
\end{rem}

\begin{rem}\label{remComputFeasibilityOfAGcohomology}
Assume that $S$ is finite, and $\Vv=\Abel$ is the category of abelian groups. Is it possible to algorithmically compute cohomology of a sheaf $\ca{F}$ on $X_S$? Potentially, yes, Construction~\ref{conSheafCohomDerivedStyle} describes the algorithm. However, there is one formal detail, which complicates the deal. Namely, we should specify a computationally feasible class of abelian groups. In topological practice, consideration is restricted to finitely generated abelian groups. However, unwinding Constructions~\ref{conSheafCohomDerivedStyle} and~\ref{conGodementConstruction}, we see the necessity to generate injective hulls in $\Vv$, which kicks us out of the class of finitely generated groups. For example, the injective hull of the simplest group $\Zo$ is $\Zo\hookrightarrow \Qo$ so we need to take the groups like $\Qo$, $\Qo/\Zo$, their direct sums, and homomorphisms between them into account. This certainly complicates the computational engine.

The situation is a bit easier in the category $\ko\FinVect$ of finite-dimensional vector spaces over a field, where every object is injective. The computational approach to cohomology, in which injective resolution of a given sheaf is built step by step, looks a bit scary. However, we notice here that our homology-slaying Algorithm~\ref{algMainAlg} to be described in the latter sections, can be reformulated as the construction of a minimal injective resolution which is, in a sense, universal for all sheaves $D$ on a given poset $S$. %Therefore the idea to use injective resolution as the foundation of an algorithm is not that crazy after all.
\end{rem}

\subsection{Roos complex}\label{subsecMathCohomologySimplicial}

The definition of sheaf cohomology given in Subsection~\ref{subsecMathCohomologyDerived} will be used mainly for mathematical reference: it is universal, but seems impractical when it comes to actual calculations as shown in Remark~\ref{remComputFeasibilityOfAGcohomology}. There seems to be a consensus, both in theoretical and applied community, that when it comes to (co)homology computations, it is better to deal directly with (co)chain complexes, see Subsection~\ref{subsecMathCochainRecap}.

%A number of (co)chain complexes and hence (co)homology theories are associated with directed graph (of which posets can be considered a part). The account

Given a diagram $D$ on a poset $S$ (equiv. a sheaf $\ca{D}$ on a finite topological space $X_S$) we want to define a cochain complex $(C^*(S;D),d)$ which is more-or-less easily constructed from $D$, and computes cohomology of $D$, i.e.
\[
H^j(C^*(S;D),d)\cong H^j_{AG}(X_S;\ca{D}).
\]
For any finite poset $S$ there is a classical construction introduced in the paper of Roos~\cite{roos1961derlim}.

\begin{con}\label{conLargestEltOfChain}
Let $\tau$ be a chain $s_0<s_1<\cdots<s_j$ in the poset $S$, which means that $\tau$ is a $j$-dimensional simplex of the order complex $\ord(S)$. Denote by $\tau_{\max}$ the largest element $s_k$ of this chain. If $\tau\subseteq\sigma$ is a set-theoretic inclusion of chains, then clearly $\tau_{\max}\leq \sigma_{\max}$.
\end{con}

\begin{con}\label{conStandardSimplicialIncNumbers}
Assume that $\sigma$ is a $(j+1)$-dimensional simplex of $\ord(S)$, i.e. a chain $s_0<s_1<\cdots<s_{j+1}$, and $\tau$ is a $j$-dimensional simplex. Then we can define the incidence number $\inc{\sigma}{\tau}\in\Zo$ in the standard alternating fashion. If $\sigma$ is a chain $s_0<s_1<\cdots<s_{j+1}$ and $\tau=\sigma\setminus\{s_i\}$, then $\inc{\sigma}{\tau}$ is set equal to $(-1)^i$; in all other cases $\inc{\sigma}{\tau}$ is set equal to $0$. This alternating assumption easily implies that, for any $\sigma_1<\sigma_2$, $\dim\sigma_1=j-1$, $\dim\sigma_2=j+1$, we have
\begin{equation}\label{eqDiamondSimplicial}
\sum_{\tau\colon \dim\tau=j}\inc{\sigma_2}{\tau}\cdot\inc{\tau}{\sigma_1}=0.
\end{equation}
There are only two nonzero summands in this sum.
\end{con}

\begin{defin}\label{definRoosComplex}
Let $D$ be a diagram on a poset $S$. Consider the graded vector space $C^*_{Roos}(S;D)$, and a map $d_{Roos})$ where
\[
C^j_{Roos}(S;D)=\bigoplus_{\tau\in\ord(S),\dim\tau=j}D(\tau_{\max}),\mbox{ and }d_{Roos}\colon C^j_{Roos}(S;D)\to C^{j+1}_{Roos}(S;D)
\]
defined by
\[
d=\bigoplus_{\tau<\sigma, \dim\tau=j,\dim\sigma=j+1}\inc{\sigma}{\tau}F(\tau_{\max}\leq \sigma_{\max}).
\]
It follows from~\eqref{eqDiamondSimplicial} that $d^2=0$. The cochain complex $(C^j_{Roos}(S;D),d_{Roos})$ is called \emph{the Roos complex} of the diagram $D$ on $S$, and its cohomology are called \emph{the Roos cohomology}:
\[
H^j_{Roos}(S;D)=H^j(C^j_{Roos}(S;D),d_{Roos}).
\]
\end{defin}

\begin{rem}\label{remRefinedDiagram}
Given a diagram $D$ on $S$, we can naturally define a refined commutative diagram $\hat{D}\colon\cat(\ord(S))\to \Vv$, given by $\hat{D}(\sigma)=D(\sigma_{\max})$ and $\hat{D}(\sigma\subset\tau)=D(\sigma_{\max}\leq\tau_{\max})$. Roos complex is basically the cellular cochain complex associated with the cellular sheaf (local coefficient system) $\hat{D}$ on $\ord(S)$. However, since cellular sheaves and cellular cochain complex haven't been defined yet, we avoid putting this remark as the definition of Roos complex to avoid any accusation in looping the definitions.
\end{rem}

\begin{con}\label{conConstantSheafOnLowerIdeal}
Let $T\subseteq S$ be a lower order ideal of $S$, i.e. a set with the property that $s\in T$ and $s'<s$ implies $s'\in T$. Equivalently, $T$ is a closed subset of the Alexandrov topological space $X_S$. Let $V\in\Vv$ be an object of the abelian category. Consider the diagram $V_T\in \Diag(S;\Vv)$ defined by
\[
V_T(s)=\begin{cases}
         V, & \mbox{if } s\in T \\
         0, & \mbox{otherwise}
       \end{cases}\mbox{ and }
       V_T(s'\leq s) =\begin{cases}
         \id_V, & \mbox{if } s\in T \\
         0, & \mbox{otherwise}.
       \end{cases}
\]
We denote the corresponding sheaf on $X_S$ with the same letter $V_T$.
\end{con}

The following statement shows the topological nature of the Roos construction. %, known in algebraic topology~\cite{Iversen}.

\begin{prop}\label{propRoosIsSingular}
If $T\subseteq S$ is a lower order ideal, then Roos cohomology $H^*_{Roos}(X_S;V_T)$ is naturally isomorphic to the singular cohomology $H^*(|T|;V)$ of the geometrical realization.
\end{prop}

\begin{proof}
The Roos complex $(C^j_{Roos}(S;V_T),d_{Roos})$ by construction coincides with the simplicial cochain complex of the simplicial complex $\ord(T)$ with coefficients in $V$, hence $H^*_{Roos}(S;V_T)\cong H^*_{simp}(|T|;V)$. The latter modules are isomorphic to simplicial, and hence singular cohomology of the geometrical realization, which is a standard fact in topology~\cite[Thm.2.27]{Hatcher}. %\cite[p.51]{RationalHomotopy}.
\end{proof}

\begin{ex}\label{exInjectiveIsCone}
The basic injective sheaf $\low{s}{W}$ defined in Construction~\ref{conInjSheaves} coincides with $W_{S_{\leq s}}$. We have $H^*_{Roos}(X_S;\low{s}{W})\cong H^*(|S_{\leq s}|;W)$. The latter cohomology is trivial in positive degrees and isomorphic to $W$ in degree $0$, as follows from contractibility of $|S_{\leq s}|$.
\end{ex}

A classical theorem asserts that Roos cohomology of a sheaf are isomorphic to cohomology defined from the injective resolution.

\begin{thm}\label{thmRoosIsAGcohomology}
For any sheaf $D$ on a poset $S$, the Roos cohomology modules are naturally isomorphic to its cohomology:
\[
H^j_{Roos}(S;D)=H^j_{AG}(X_S;\ca{D}).
\]
\end{thm}

Basically, this statement is the instance of the universal delta-functor theorem. We postpone the proof to Subsection~\ref{subsecMathHonestComputations}, where it is formulated in bigger generality. The next claim follows from Theorem~\ref{thmRoosIsAGcohomology} and Proposition~\ref{propRoosIsSingular}.

\begin{rem}\label{remRoos0globalSec}
Theorem~\ref{thmRoosIsAGcohomology} implies $H^0_{Roos}(S;D)\cong \Gamma(S;D)$. This particular statement can be seen directly: it is a simple exercise to prove that, for the Roos differential $d_{Roos}\colon C^0_{Roos}(S;D)\to C^1_{Roos}(S;D)$, the equation $d_{Roos}x=0$ coincides with coherence equations~\ref{eqCoherentStates}.
\end{rem}

A number of important corollaries follow from Theorem~\ref{thmRoosIsAGcohomology}.

\begin{cor}\label{corLowerSheafIsSingularCohomology}
If $T\subseteq S$ is a lower order ideal, then the sheaf cohomology $H^*_{AG}(X_S;V_T)$ is naturally isomorphic to the singular cohomology $H^*(|T|;V)$ of the geometrical realization.
\end{cor}

\begin{ex}\label{exConstantSheaf}
When $T=S$, the sheaf $V_S$ is just the constant sheaf on $S$ (see Construction~\ref{conConstantSheaf}), we denoted it by $\bar{V}$. From the Corollary~\ref{corLowerSheafIsSingularCohomology}, it follows that $H^*_{AG}(X_S;\bar{V})\cong H^*(|X|;V)$; it is nice to have all notions consistent throughout mathematics. In particular, it follows that $\Gamma(X_S;\overline{\ko^1})\cong H^0(|S|;\ko)\cong \ko^{c(|S|)}$, where $c(|S|)$ is the number of connected components of the geometrical realization $|S|$, the latter property is well known for singular (co)homology.
\end{ex}

Another important Corollary of Theorem~\ref{thmRoosIsAGcohomology} states that, for a finite poset, cohomology in sufficiently large degrees vanish.

\begin{cor}\label{corVanishing}
Let $h=\dim\ord S$ be the length of the longest chain in $S$. Then, for any sheaf $\ca{D}$ on $X_S$ we have $H^j_{AG}(X_S;\ca{D})=0$ for $j>h$.
\end{cor}

\begin{proof}
By definition, the graded components $C^j_{Roos}(S;D)$ of the Roos complex vanish for $j>h$: there are no chains of length $>h$ in $S$. Therefore cohomology $H^j_{Roos}(S;D)$ vanish as well, and the rest follows from Theorem~\ref{thmRoosIsAGcohomology}.
\end{proof}

We finish this section with a general remark.

\begin{rem}
The alternative way to reduce cohomology computation to a simplicial structure is provided by the notion of \emph{\v{C}ech cohomology}. In this approach, an arbitrary topological space is replaced with the nerve of its covering, and the computation is performed on this simplicial complex. We do not give a review of the related theory here. However, it should be mentioned that \v{C}ech cohomology of sheaves on finite posets are not necessarily isomorphic to the cohomology $H^*_{AG}$ as observed by Husainov~\cite[Sect.5.4]{Husainov}. The analogue of Theorem~\ref{thmRoosIsAGcohomology} does not hold, therefore one should be careful about which type of cohomology is used.
\end{rem}


\subsection{Algorithmic issues}\label{subsecMathFeasibility}

\begin{defin}\label{definHomologicallyFeasible}
Assume that an abelian category $\Vv$ has finite direct sums. We say that $\Vv$ is \emph{homologically feasible} if, given a triple $V_{-1}\stackrel{d_0}{\rightarrow}V_0\stackrel{d_1}{\rightarrow} V_1$ in $\Vv$ satisfying $d_0;d_1=0$, the problem of describing subquotients $H=\Ker d_1/\im d_{0}$ is algorithmically solvable.
\end{defin}

\begin{ex}\label{exVectFeasible}
The category $\ko\FinVect$ of finite dimensional vector spaces is homologically feasible --- as long as arithmetical properties of the ground field $\ko$ are described somehow. This follows from the Gauss algorithm.
\end{ex}

\begin{ex}\label{exAbelFeasible}
The category $\FinAbel$ of finitely generated abelian groups is homologically feasible as well since the classification of finitely generated abelian groups is constructive.
\end{ex}

\begin{ex}\label{exPolyModulesFeasible}
The category $\ko[t]\FinMod$ of finitely generated modules over a polynomial ring in a single variable is homologically feasible by the same reason. The ring $\ko[t]$ is a principal ideal domain, and the classification of finitely generated modules over a PiD is constructive. The latter example is of particular importance in topological data analysis, since persistence modules, in particular persistent homology, are the objects of the category of (graded) $\ko[t]$-modules.
\end{ex}

The fact that cohomology of Roos complex coincides with cohomology defined in derived manner, has an important corollary.

\begin{cor}\label{corComputabilityPosets}
Assume $X$ is a finite topology, and $\Vv$ is a homologically feasible category. Then, for any sheaf $\ca{D}$ on $X$ valued in $\Vv$, all the cohomology $H^*_{AG}(X;\ca{D})$ are computable.
\end{cor}

\begin{proof}
Proposition~\ref{propPosTop} tells that $X=X_S$ for some preposet $S$, and according to Remark~\ref{remPreposetsNotNeeded}, without loss of generality, $S$ can be assumed a poset. For a poset $S$, Theorem~\ref{thmRoosIsAGcohomology} claims that $H^*_{AG}(X_S;\ca{D})\cong H^*_{Roos}(S;D)$. Roos cohomology are computed as subquotients of the Roos complex $C^*_{Roos}(S;D)$. Since there are only finitely many chains in a finite poset $S$, the whole complex $C^*_{Roos}(S;D)$ is a finite direct sum of stalks. For sufficiently large $j$ (for example for $j>\#S$), the component $C^j_{Roos}(S;D)$ simply vanishes. The assumption of algorithmic feasibility implies that cohomology are computable in all degrees $j$ below $\#S$.
\end{proof}

\subsection{Cohomology of cellular sheaves}\label{subsecMathCohomologyCellular}

Cohomology computation with the use of Roos complex is universal --- it works for all sheaves on Alexandrov spaces. In the case of a finite poset and a diagram valued in finite-dimensional vector spaces, the computation of cohomology can be done algorithmically, as Corollary~\ref{corComputabilityPosets} shows. However, such computation is heavy in practice: the order complex $\ord S$ of a general poset $S$ has many simplices, so the matrices involved in the calculation of Roos cohomology are exponential in the cardinality of $S$, in the worst case.

Another approach to calculate cohomology of a sheaf is more optimal but it narrows the class of posets on which it applies. In this section we review the construction and basic theory of cellular cohomology which are well defined on cell posets, or their analogues, given by Definitions~\ref{definCellPoset}, \ref{definHomotopyCellPoset}, and \ref{definHomologyCellPoset}.%, and~\ref{definMorseHomologyCellPoset}. 
The mathematical results of this subsection were proved in~\cite{EVERITT2015134}, and the definition of a cellular cochain complex appears in all applied papers dealing with cohomology of cellular sheaves. We provide formulations in terms introduced previously in the paper.

\begin{defin}\label{definCellularSheaf}
A \emph{cellular sheaf} $D$ is a diagram on a cell poset $\ca{X}$. Equivalently, a cellular sheaf is a sheaf on the Alexandrov space corresponding to a cell poset.
\end{defin}

The property of ``being cellular'' is actually not the property of a sheaf itself, but rather the property of its domain of definition.

\begin{rem}
The previous sections, especially Proposition~\ref{propDiagSheaf} should have convinced the reader that there is not much operational difference between posets $S$ and Alexandrov topologies $X_S$ when we speak of diagrams/sheaves. To avoid the notational monster $X_{\ca{X}}$, we abuse the notation and denote by $\ca{X}$ both cell posets (or the variations of this notion) and the corresponding Alexandrov topologies.
\end{rem}

In order to define a cochain complex, which computes sheaf cohomology of cellular sheaves in a more optimal way than Roos complex, one needs an important ingredient: incidence numbers~\footnote{At this place the perfect symphony of abstract nonsense is first broken by a down-to-earth notion of a number. The harmony will be restored later on in the text, where we show that incidence numbers are homology as well.} of cells. This notion is classical in the study of CW-complexes in algebraic topology. %There is no such thing as an incidence number for a general (finite) poset, henceforth, straightforward analogues of the cellular cochain complex cannot be defined for general posets.

Let $\sigma$ be a cell of rank $k$ of a cell poset $\ca{X}$. By definition, it means that the space $|\dd C(\sigma)|=|\ca{X}_{<\sigma}|$ has integral homology of a $(k-1)$-dimensional sphere. From the contractibility of the cone $|C(\sigma)|$ and the long exact sequence of the pair $(|C(\sigma)|,|\dd C(\sigma)|)$, it follows that
\[
H^k(|C(\sigma)|,|\dd C(\sigma)|;\Zo)\cong \Zo
\]
while all other relative homology groups vanish: $H^j(|C(\sigma)|,|\dd C(\sigma)|;\Zo)=0$ for $j\neq \rk\sigma$.

\begin{defin}\label{definCellOrientation}
A choice of a generator $\kappa_\sigma$ of the group $H^k(|C(\sigma)|,|\dd C(\sigma)|;\Zo)\cong \Zo$ is called \emph{an orientation} of a cell $\sigma\in\ca{X}$.
\end{defin}

\begin{con}\label{conIncidenceNumbers}
Assume that $\sigma,\tau\in \ca{X}$ be two cells of a cell poset $\ca{X}$ such that $\tau<\sigma$ and $\rk\tau=k$, $\rk\sigma=k+1$ for some $k=0,1,\ldots$. Consider the map $f_{\sigma,\tau}$ given by composing the sequence
\begin{equation}\label{eqMapForIncidenceNumber}
H_{k+1}(|C(\sigma)|,|\dd C(\sigma)|;\Zo)\stackrel{\cong}{\longrightarrow} \Hr_{k}(|\dd C(\sigma)|;\Zo)\to H_{k}(|\dd C(\sigma)|,|\dd C(\sigma)\setminus\{\tau\}|;\Zo)\cong H_{k}(|C(\tau)|,|\dd C(\tau)|;\Zo),
\end{equation}
where
\begin{enumerate}
  \item The first map $f_1\colon H_{k+1}(|C(\sigma)|,|\dd C(\sigma)|;\Zo)\to H_{k}(|\dd C(\sigma)|;\Zo)$ is the connecting homomorphism in the long exact sequence of singular homology of the pair $(|C(\sigma)|,|\dd C(\sigma)|)$. This map is an isomorphism since the cone $|C(\sigma)|$ is contractible.
  \item The second map $H_{k}(|\dd C(\sigma)|;\Zo)\to H_{k}(|\dd C(\sigma)|,|\dd C(\sigma)\setminus\{\tau\}|;\Zo)$ is the natural map to relative homology.
  \item The last isomorphism is due to excision property of singular homology.
\end{enumerate}
Since $\ca{X}$ is a cell complex, both groups $H_{k+1}(|C(\sigma)|,|\dd C(\sigma)|;\Zo)$ and $H_{k}(|C(\tau)|,|\dd C(\tau)|;\Zo)$ are isomorphic to $\Zo$, so the composition of~\eqref{eqMapForIncidenceNumber} belongs to $\Hom(\Zo,\Zo)\cong\Zo$. If the orientations $\kappa_\sigma$ and $\kappa_\tau$ of cells $\sigma$ and $\tau$ are fixed, the map $f_{\sigma,\tau}$ becomes a number as well.
\end{con}

\begin{defin}\label{definIncidenceNumber}
Let $\sigma,\tau\in \ca{X}$ be two cells of a cell poset $\ca{X}$ such that $\rk\sigma=\rk\tau+1$, and let $\kappa_\sigma$ and $\kappa_\tau$ be orientations of these cells. \emph{The incidence number} $\inc{\sigma}{\tau}\in\Zo$ is defined as the unique integer such that
\[
f_{\sigma,\tau}(\kappa_\sigma)=\inc{\sigma}{\tau}\kappa_\tau,
\]
--- in the case $\tau<\sigma$. If $\sigma\ngtr \tau$, we set $\inc{\sigma}{\tau}=0$.
\end{defin}

The following statement is the classical property of incidence numbers, which allows to use them in the definition of a differential in a cellular cochain complex.

\begin{lem}[Diamond property]\label{lemDiamondProperty}
Let $\sigma,\tau$ be two cells of a cell poset $\ca{X}$ such that $\sigma_1<\sigma_2$ and $\rk\sigma_2=\rk\sigma_1+2$. Then
\begin{equation}\label{eqDiamond}
\sum_{\tau\colon \sigma_1<\tau<\sigma_2}\inc{\sigma_2}{\tau}\cdot\inc{\tau}{\sigma_1}=0.
\end{equation}
\end{lem}

\begin{proof}
The classical sketch of proof. Let $\rk\sigma_2=k+1$, so that $\rk\sigma_1=k-1$, and consequently $\rk\tau=k$. Since $\inc{\sigma}{\tau}=0$ when $\tau$ is incomparable with $\sigma$, we can replace the required formula~\eqref{eqDiamond} by the equivalent formula
\begin{equation}\label{eqDiamondRewritten}
\sum_{\tau\colon \rk\tau=k}\inc{\sigma_2}{\tau}\cdot\inc{\tau}{\sigma_1}=0.
\end{equation}
The direct sum $\bigoplus_{\sigma\colon \rk\sigma=k+1}H_{k+1}(|C(\sigma)|,|\dd C(\sigma)|;\Zo)$ is naturally identified with $H_{k+1}(|\ca{X}_{k+1}|,|\ca{X}_k|;\Zo)$ (recall the definition of a skeleton in Construction~\ref{conCellPosetSkeleton}). The direct sum of all maps $f_{\sigma,\tau}$ over all $\sigma$ and $\tau$ of ranks $k+1$ and $k$ respectively becomes identified with the map $\delta_{k+1}\colon H_{k+1}(|\ca{X}_{k+1}|,|\ca{X}_k|;\Zo)\to H_{k}(|\ca{X}_{k}|,|\ca{X}_{k-1}|;\Zo)$, which is the connecting homomorphism in the long exact sequence of the triple $(|\ca{X}_{k+1}|,|\ca{X}_{k}|,|\ca{X}_{k-1}|)$. The choice of orientations of the cells defines the basis of both domain and target abelian group, therefore the map $\delta_{k+1}$ becomes an integral matrix $D_{k+1}$ with entries $\inc{\sigma}{\tau}$.

Equality~\eqref{eqDiamondRewritten} is equivalent to $D_kD_{k+1}=0$. This directly follows from the fact that two connecting homomorphisms $\delta_{k+1}\colon H_{k+1}(|\ca{X}_{k+1}|,|\ca{X}_k|;\Zo)\to H_{k}(|\ca{X}_{k}|,|\ca{X}_{k-1}|;\Zo)$ and $\delta_{k}\colon H_{k}(|\ca{X}_{k}|,|\ca{X}_{k-1}|;\Zo)\to H_{k-1}(|\ca{X}_{k-1}|,|\ca{X}_{k-2}|;\Zo)$ compose to zero. Indeed, these connecting homomorphisms are nothing but the 1-st differential $d_1$ of the spectral sequence in homology, associated with the filtration
\[
\varnothing\subset |\ca{X}_0|\subset |\ca{X}_1|\subset|\ca{X}_2|\subset\cdots,
\]
therefore $d_1^2=0$ is trivially satisfied.
\end{proof}

Less involved proofs of the stated fact may be found in any basic book on algebraic topology, when the notion of cell cohomology is introduced, see e.g.~\cite[p.139]{Hatcher}.

\begin{defin}\label{definCWcochainComplex}
For a cellular sheaf $D$ on a cell poset $\ca{X}$, consider the cochain complex $(C^*_{CW}(\ca{X};D),d_{CW})$
\[
C^j_{CW}(\ca{X};D)=\bigoplus_{\sigma\in \ca{X}, \rk\sigma=j}D(s), \qquad d_{CW}\colon C^j_{CW}(\ca{X};D)\to C^{j+1}_{CW}(\ca{X};D),
\]
where $d_{CW}=\bigoplus_{\tau<\sigma, \rk\tau=j, \rk\sigma=j+1}\inc{\sigma}{\tau}F(\tau<\sigma)$. This cochain complex is called \emph{the cellular cochain complex} of the cellular sheaf $D$. The cohomology of $(C^*_{CW}(\ca{X};D),d_{CW})$ are called \emph{cellular cohomology} of a cellular sheaf $D$ and are denoted
\[
H^j_{CW}(\ca{X};D)=H^j(C^*_{CW}(\ca{X};D),d_{CW}).
\]
\end{defin}

\begin{rem}
The most nontrivial part of this definition is the property $d_{CW}^2=0$. It is easily proven from the diamond property of incidence numbers and commutativity of a diagram $D$ itself. Indeed, if we consider the block of the map $d_{CW}^2$ acting from $D(\sigma_1)\subset C^{k-1}(\ca{X};D)$ to $D(\sigma_2)\subset C^{k+1}(\ca{X};D)$ (where $\sigma_1<\sigma_2$), this block equals $D(\sigma_1<\sigma_2)$ multiplied by $\sum_{\tau\colon \sigma_1<\tau<\sigma_2}\inc{\sigma_2}{\tau}\cdot\inc{\tau}{\sigma_1}$ since we summate the compositions of maps over all saturated paths leading from $\sigma_1$ to $\sigma_2$. The latter sum vanishes due to the diamond property, Lemma~\ref{lemDiamondProperty}.
\end{rem}

\begin{thm}\label{thmCWcohomologyIsAGcohomology}
For any cellular sheaf $D$ on a cell poset $\ca{X}$, the cellular cohomology modules are naturally isomorphic to its cohomology:
\[
H^j_{CW}(\ca{X};D)=H^j_{AG}(\ca{X};D).
\]
\end{thm}

We prove both Theorems~\ref{thmRoosIsAGcohomology} and~\ref{thmCWcohomologyIsAGcohomology} with the same general argument in subsection~\ref{subsecMathHonestComputations}. The next remark explains, why Theorem~\ref{thmCWcohomologyIsAGcohomology} is important even if we don't care about higher order cohomology.

\begin{rem}\label{remCompatibilityReducesToCells}
As stated in Proposition~\ref{propCohomologyProperties}, we have $H^0_{AG}(\ca{X};D)\cong \Gamma(\ca{X};D)$, the space of global sections. In the original formulation, the set $\Gamma(\ca{X};D)$ is defined as the solution to the system of compatibility equations, see Definition~\ref{definGlobalSectionsConcrete} or Remark~\ref{remConcreteLimSets}. However, Theorem~\ref{thmCWcohomologyIsAGcohomology} implies that
\[
H^0_{AG}(\ca{X};D)\cong H^0_{CW}(\ca{X};D)=\Ker d_{CW}\colon C^0_{CW}(\ca{X};D)\to C^1_{CW}(\ca{X};D).
\]
The number of equations in the system $d_{CW}x=0$ is less than the number of all compatibility equations (even restricted to the edges of the Hasse diagram of a poset, as in Remark~\ref{remReduceEquationsToGenerators}), because it only involves relations between vertices and edges, with no higher-dimensional cells involved. Informally this means that, in a cell complex, higher-dimensional cells do not store any compatibility information --- it all lives in the 1-skeleton $\ca{X}_1$. This phenomenon reflects the cellular approximation theorem~\cite[p.349]{Hatcher}, known in algebraic topology: any ``proof of coherence'' is a path through a space, and this path can be deformed to a path lying in the 1-skeleton. The same remark holds true for cellular complexes even if a sheaf takes values in non-abelian category, such as $\Sets$: we don't need all compatibility equations~\eqref{eqCoherentStates} to describe global sections, --- equations between 0- and 1-dimensional cells do the job.
\end{rem}

\subsection{Honest cohomology computations}\label{subsecMathHonestComputations}

Let us fix a general poset $S$ and an abelian category $\Vv$. Consider an additive functor $\Fu\colon \Shvs(X_S;\Vv)\to\Cochain(\Vv)$ endowed with a natural transformation $\Gamma(X_S;\cdot)\to \Cochain(\cdot)^0$ which will be called \emph{an augmentation}. The latter means that, for any sheaf $\ca{D}$ on $X_S$ we have natural maps from the global sections $\ca{D}(S)=\lim\limits_{\leftarrow S}D$ to the 0-degree component $\Fu(\ca{D})^0$ of the cochain complex $\Fu(\ca{D})$.

\begin{defin}\label{definHonestlyComputes}
We say that $\Fu$ \emph{honestly computes sheaf cohomology}, if $H^*(\Fu(\ca{D}))$ is naturally isomorphic to $H_{AG}^*(X_S;\ca{D})$.
\end{defin}

Recall that $\low{s}{W}$ is a basic injective sheaf on $S$ defined in Construction~\ref{conInjSheaves}. The next theorem is a reformulation of Grothendieck's delta-functor theorem~\cite[Thm.2.2.2]{grothendieck1957tohoku} which we find more convenient to use in practice.

\begin{thm}\label{thmHonestCohomology}
Assume that the functor $\Fu\colon \Shvs(X_S;\Vv)\to\Cochain(\Vv)$ is exact and for every injective object $W\in\Vv$ we have $H^j(\Fu(\low{s}{W}))=0$ if $j\neq 0$ and the augmentation functor induces the natural isomorphism $W=\Gamma(S;\low{s}{W})\to H^0(\Fu(\low{s}{W}))$. Then $\Fu$ honestly computes sheaf cohomology.
\end{thm}

\begin{proof}
Let $\ca{D}$ be a sheaf on $X_S$ ($D$ a diagram on $S$). Consider an injective resolution~\eqref{eqInjSheaf} of $\ca{D}$:
\[
0\to \ca{D}\to \ca{I}_0\to \ca{I}_1\to\cdots
\]
where each $\ca{I}_j$ is a direct sum of injectives of the form $\low{s}{W}$. Applying the functor $\Fu$ to this injective resolution we get a bicomplex
\begin{equation}\label{eqBicomplex}
\begin{CD}
@.   @. 0 @. 0 @.\\
@. @. @VVV @VVV @.\\
@. @. \ca{I}_0(S) @>>> \ca{I}_1(S) @>>> \cdots\\
@. @. @VVV @VVV @.\\
0 @>>> \Fu(\ca{D})^0 @>>> \Fu(\ca{I}_0)^0 @>>> \Fu(\ca{I}_1)^0 @>>> \cdots\\
@. @VVV @VVV @VVV @.\\
0 @>>> \Fu(\ca{D})^1 @>>> \Fu(\ca{I}_0)^1 @>>> \Fu(\ca{I}_1)^1 @>>> \cdots\\
@. @VVV @VVV @VVV @.\\
@.\vdots @.\vdots @.\vdots @.\ddots
\end{CD}
\end{equation}
Here we added the augmentation map on top of each column except the leftmost one.

In the first nontrivial row of~\eqref{eqBicomplex}, the global sections of the injective resolution of $\ca{D}$ are written. The cohomology of this complex is, by definition, the sheaf cohomology $H_{AG}^*(X_S;\ca{D})$. All other rows are acyclic, since we assumed the functor $\Fu$ is exact.

In the first nontrivial column of~\eqref{eqBicomplex} the cochain complex $\Fu(\ca{D})$ itself is written. All other columns are acyclic, because the augmented complex $0\to \ca{I}(S)\to \Fu(\ca{I})^*$ is assumed acyclic for any injective of the form $\low{s}{W}$, --- hence for any direct sum of such sheaves.

Now, the standard argument of homological algebra (see e.g.\cite[Theorem 2.15]{SpectralSeq}) shows that cohomology of the first row is naturally isomorphic to the cohomology of the first column: $H^*_{AG}(X_S;\ca{D})\cong H^*(\Fu(\ca{D}))$, which was required.
\end{proof}

\begin{rem}\label{remInjectiveAcyclic}
Notice that the statement somehow converse to Theorem~\ref{thmHonestCohomology} holds true: if $\Fu$ honestly computes cohomology, then for any injective sheaf $\low{s}{W}$, the cochain complex $\Fu(\low{s}{W})$ is acyclic in degrees $j>0$, while $H^0(\Fu(\low{s}{W}))\cong W$. Indeed, each injective sheaf is acyclic, see Remark~\ref{remOpinionLeader}, so this statement is trivial.
\end{rem}

The following are the two main classes of examples of honest cohomology computation.

\begin{ex}\label{exRoos}
Let $D$ be a diagram on a poset $S$. Roos complex $C^*_{Roos}(S;D)$ is defined, see Definition~\ref{definRoosComplex}. This construction defines a functor $\Fu_{Roos}\colon \Diag(S;\Vv)\to \Cochain(\Vv)$ --- the functoriality is easily seen from the definition. It is an exact functor, because each module $C^j_{Roos}(S;D)$ is a direct sum of stalks $D(s)$ of a sheaf, and taking a stalk is an exact functor according to Proposition~\ref{propExactSequenceDiagrams}. The augmentation map $\ca{D}(S)\to C^0_{Roos}(S;D)$ is naturally defined since $C^0_{Roos}(S;D)$ is a direct sum of stalks, and $\ca{D}(S)$ in the inverse limit $\lim\limits_{\leftarrow S}D$ of a diagram, hence maps naturally to all its stalks.

In order to apply Theorem~\ref{thmHonestCohomology}, we need to check that, for any injective $W\in\Vv$, the augmented complex $0\to\low{s}{W}(X_S)\to C^*_{Roos}(S;\low{s}{W})$ is acyclic. This is indeed the case, see Example~\ref{exInjectiveIsCone}. The assumptions of Theorem~\ref{thmHonestCohomology} are fulfilled for the Roos functor $\Fu_{Roos}$, hence $H^*_{AG}(X_S;\ca{D})\cong H_{Roos}^*(S;D)$. This proves Theorem~\ref{thmRoosIsAGcohomology}.
\end{ex}

\begin{ex}\label{exCell}
Let $D$ be a cellular sheaf on a cell poset $\ca{X}$. Cellular cochain complex $C^*_{CW}(\ca{X};D)$ is defined, see~\ref{definCWcochainComplex}. This construction defines a functor $\Fu_{CW}\colon \Diag(\ca{X};\Vv)\to \Cochain(\Vv)$ --- the functoriality is easily seen from the definition. It is an exact functor, because each module $C^j_{CW}(\ca{X};D)$ is a direct sum of stalks $D(s)$ of a sheaf, and taking a stalk is an exact functor. The augmentation map $\ca{D}(\ca{X})\to C^0_{CW}(\ca{X};D)$ is naturally defined since $C^0_{CW}(\ca{X};D)$ is a direct sum of stalks, and $\ca{D}(\ca{X})$ in the inverse limit $\lim\limits_{\leftarrow \ca{X}}D$ of a diagram, hence maps to all its stalks.

Let us check that, for any injective $W\in\Vv$, the augmented complex $0\to\low{\sigma}{W}(\ca{X})\to C^*_{CW}(\ca{X};\low{\sigma}{W})$ is acyclic. The augmented cellular cochain complex takes the form
\begin{equation}\label{eqCellCochainOfCell}
0\to W\to \bigoplus_{\tau\leq\sigma, \rk\tau=0}W\to \bigoplus_{\tau\leq\sigma, \rk\tau=1} W\to \bigoplus_{\tau\leq\sigma, \rk\tau=2} W\to \cdots
\end{equation}
Recall that precise definition of the cellular differential required the auxiliary notion of incidence numbers, which came from topological considerations. It is time to dig these considerations up. Notice that the differential complex~\eqref{eqCellCochainOfCell} is by construction the result of application of a functor $\Hom(\cdot,W)$ to the augmented chain complex
\begin{equation}\label{eqChainComplexOfCell}
0\leftarrow W\leftarrow C_0^{CW}(|C(\sigma)|;\Zo)\leftarrow C_1^{CW}(|C(\sigma)|;\Zo)\leftarrow C_2^{CW}(|C(\sigma)|;\Zo)\leftarrow \cdots
\end{equation}
where each $C_j^{CW}(|C(\sigma)|;\Zo)$ is by definition the relative homology group $C_j(|C(\sigma)_j|,|C(\sigma)_{j-1}|;\Zo)$. The homological spectral sequence of the skeletal filtration $\varnothing\subset |C(\sigma)_0|\subset|C(\sigma)_1|\subset\cdots$ degenerates at $E^2$-page by the vanishing argument. Therefore the homology of the complex~\eqref{eqChainComplexOfCell} coincides with the reduced (singular) homology of the space $|C(\sigma)|$ itself. Since $|C(\sigma)|$ is contractible, its reduced homology vanish. Therefore, the cohomology of the cochain complex~\eqref{eqCellCochainOfCell} vanish as well by the universal coefficients' theorem for cohomology.

The assumptions of Theorem~\ref{thmHonestCohomology} are fulfilled for the CW-cochain functor $\Fu_{CW}$. Therefore $H^*_{AG}(\ca{X};D)\cong H_{CW}^*(\ca{X};D)$ which proves Theorem~\ref{thmCWcohomologyIsAGcohomology}.
\end{ex}

Next example is very particular, but it illustrates that the general technique which proves Theorem~\ref{thmHonestCohomology} works pretty well on more concrete examples, which are not covered by Examples~\ref{exRoos} and~\ref{exCell}.

\begin{ex}\label{exMorse}
Let $\ca{Y}$ be a poset described in Example~\ref{exMorseExample}. It is a Morse cell poset with 3 vertices $1,2,3$ and two edges $a,b$, with the edge $a$ attached to vertices $1$ and $2$, and the edge $b$ attached to the middle of $a$ and the vertex $3$. For every sheaf $D$ on $\ca{Y}$ consider the augmented cochain complex
\[
0\to \lim\limits_{\leftarrow \ca{Y}} D\to C^0(\ca{Y};D)\to C^1(\ca{Y};D)\to 0
\]
where
\[
C^0(\ca{Y};D)=D(1)\oplus D(2)\oplus D(3), \qquad C^1(\ca{Y};D)=D(a)\oplus D(b),
\]
the augmentation is a natural map from the inverse limit, and the differential $d\colon C^0(\ca{Y};D)\to C^1(\ca{Y};D)$ is defined by the block matrix
\begin{equation}\label{eqMatrixMorseExample}
\begin{pmatrix}
  D(1<a) & -D(2<a) & 0 \\
  0 & D(2<b) & -D(3<b)
\end{pmatrix}.
\end{equation}
Then the functor $\Fu = C^*(\ca{Y};\cdot)$ honestly computes sheaf cohomology on $\ca{Y}$. Indeed, all the sheaves $\low{s}{W}$ are $\Fu$-acyclic. The most interesting case is $s=b$: in this case acyclicity of $\low{b}{W}$ can be proven by hand.

Informally, the matrix~\eqref{eqMatrixMorseExample} corresponds to first taking ``CW-approximation'' in the spirit of Whitehead theorem (see Fig.~\ref{figElementaryMorse}), and considering incidence numbers of the resulting cellular replacement. Notice however, that on the level of homological algebra, there is no need to take an actual cellular approximation. Instead of matrix~\eqref{eqMatrixMorseExample} one can define the differential to be
\begin{equation}\label{eqMatrixMorseExampleAlternative}
\begin{pmatrix}
  D(1<a) & -D(2<a) & 0 \\
  k_1D(2<b) & k_2D(2<b) & -D(3<b)
\end{pmatrix}
\end{equation}
with $k_1+k_2=1$, and this cochain complex still honestly computes sheaf cohomology on a given poset $\ca{Y}$. This procedure can be considered ``a fuzzy'' CW-approximation.
\end{ex}

The idea behind this example becomes more transparent in the next subsection, where we deal with general Morse posets.

\subsection{One-shot cohomology computations}\label{subsecMathOneShot}

There is a certain similarity shared by Examples~\ref{exRoos}, \ref{exCell}, and~\ref{exMorse}. In all cases the graded components of the complex $\Fu(\ca{D})$, the one which honestly computes cohomology of a sheaf $\ca{D}$, are direct sums of certain stalks $D(s)$ of $\ca{D}$. This motivates the following generalization. %There may exist functors $\Fu\colon \Shvs(S;\Vv)\to\Cochain(\Vv)$ which do not have this form, however seem to never appear in machine learning practice and applied algebra.

\begin{defin}\label{definConcreteFunctor}
A functor $\Fu\colon \Shvs(S;\Vv)\to\Cochain(\Vv)$ is called \emph{concrete}, if for any $j\geqslant 0$, the graded component $\Fu(\ca{D})_j$ is a direct sum of stalks $D(s)$ of a sheaf $\ca{D}$.
\end{defin}

Each concrete functor is exact, since taking stalks is exact. Also there exists a natural augmentation map $\ca{D}(S)\to \Fu(\ca{D})_0$, because $\ca{D}(S)=\lim\limits_{\leftarrow} D$ maps to each stalk by the universal property of the inverse limit.

However, it is important that in order for the functor to land in $\Cochain(\Vv)$ one needs to define differentials in a uniform manner. We have already seen in Example~\ref{exCell} that this definition may be tricky: in the case of cell posets ``to define differential in a uniform manner'' reads as ``to construct incidence numbers of cells''. This particular task is outsourced to basic algebraic topology books.

The cellular cochain complex $C^*_{CW}(\ca{X};D)$ is much smaller than Roos complex $C^*_{Roos}(S;D)$, hence more usable in practice. One can summarize this optimality by the following two observations. In the cellular case, each stalk $D(\sigma)$ appears in $C^*_{CW}(\ca{X};D)$ exactly once. In the Roos case, each stalk $D(s)$ appears in $C^*_{Roos}(S;D)$ multiple times. The total number of summands in $C^*_{Roos}(S;D)$ is in the worst case exponential in the cardinality $|S|$. See more precise computations in Examples~\ref{exMultiplicityCW} and~\ref{exMultiplicityRoos} below. This difference motivates the following definition.

\begin{defin}\label{definOneShotComplex}
A concrete functor $\Fu\colon \Shvs(S;\Vv)\to\Cochain(\Vv)$ is said to be \emph{one-shot} if each stalk $D(s)$ of a sheaf $\ca{D}\in\Shvs(S;\Vv)$ appears exactly once as a summand in $\Fu(\ca{D})$.
\end{defin}

Of course we are interested in one-shot concrete functors which provide honest cohomology calculations. We have already seen in Example~\ref{exCell} that cellular cochain complex provides one-shot honest cohomology calculation. On the other hand, Example~\ref{exMorse} shows that non-cell poset may support a functor $\Fu$ that is both one-shot and honestly computes cohomology of sheaves.

\begin{rem}\label{remDeletePoint}
Is it possible to honestly compute cohomology with a concrete functor $\Fu$ without using some stalk $D(s)$ at all? Somehow surprisingly, the answer is yes. Whenever $|S_{<s}|$ is acyclic, the element $s\in S$ can be dropped from the poset, and cohomology (of all sheaves) do not change, namely, $H^j(S;D)\cong H^j(S\setminus\{s\};D|_{S\setminus\{s\}})$. This can be seen as a particular case of the theorem of Oberst which holds in more general categorical setting; see~\cite{Oberst} for derived direct limits of diagrams or~\cite[Thm.3.10]{Husainov} for the cohomological version, or the paper~\cite[Prop.3.5]{SepaFR} where this result is reproved in the context of group homology and cohomology.

In the recent paper of Malko~\cite{Malko} this result is connected to beat-removals which were introduced in algebraic topology in the classical work of Stong~\cite{Stong1966FiniteTS}. The idea of beats' removal recently found an application~\cite{BoissonnatEtAl}: it can be used as a preliminary data simplification step before computing persistent homology, and it works impressively well on the data originating from ML applications. We believe, that a similar preprocessing can be applied for more general sheaf cohomology calculations. For example, if $s$ is a downbeat, i.e. there is a unique edge of the Hasse diagram, which enters $s$, then $|S_{<s}|$ is a cone, hence contractible, hence acyclic, hence $s$ can be deleted without any effect on sheaf cohomology.
\end{rem}

On the other hand, if $|S_{<s}|$ is not acyclic, then it is impossible to not to use $D(s)$ in the computations.

\begin{con}\label{conDiracDiagram}
Consider \emph{the Dirac's delta sheaf} $\delta_s(W)$: the sheaf which has a nonzero value $W$ concentrated in a single point $s$. It will be seen in the subsequent arguments, that whenever $|S_{<s}|$ is not acyclic, the cohomology of $\delta_s(W)$ are also nontrivial, hence cannot be honestly computed if $\delta_s(W)(s)=W$ is not present in the cochain complex.
\end{con}

Next we formulate and prove a new result, which indicates the precise class of posets, suitable for one-shot honest cohomology computations. %In a certain sense this theorem gives a mathematical answer to Question~\ref{queWhyCellComplexes} from the introduction.

\begin{thm}[One-shot theorem]\label{thmOneShotTheorem}
Let $\Vv$ be the category $\Abel$ of abelian groups or the category $\ko\Vect$ of vector spaces over arbitrary field. The following two conditions on a finite poset $S$ are equivalent:
\begin{enumerate}
  \item There exists a one-shot concrete functor $\Fu\colon\Shvs(S;\Vv)\to\Cochain(\Vv)$, which honestly computes cohomology.
  \item $S$ is a homology Morse cell poset (see Definition~\ref{definMorseHomologyCellPoset}).
\end{enumerate}
\end{thm}

We prove two implications separately. In both cases the arguments work in slightly bigger generality, and may be useful in their own right.

\begin{prop}\label{propOneShotOnePosition}
Assume that a concrete functor $\Fu\colon\Shvs(S;\Vv)\to\Cochain(\Vv)$ honestly computes cohomology of sheaves, and for an element $s\in S$, the stalk $D(s)$ contributes exactly once in $\Fu(\ca{D})$, namely, for some $k\geqslant 0$, the stalk $D(s)$ is a direct summand of the graded component $\Fu(\ca{D})^k$. Then $|S_{<s}|$ has the same homology as a sphere $S^{k-1}$.
\end{prop}

\begin{proof}
For convenience, we restrict considerations to the abelian category $\Vv=\Abel$, the case $\Vv=\ko\Vect$ being similar, and even a bit easier. Let us denote the cochain complex $\Fu(\ca{D})$ by $C^*(S;D)$. Let $W\in\Vv$ be an arbitrary injective abelian group. Consider two sheaves: the first one is $\low{s}{W}=W_{S_{\leq s}}$ defined in Construction~\ref{conInjSheaves}, and another one is
\begin{equation}\label{eqCutLow}
\low{\bar{s}}{W}=W_{S_{<s}}.
\end{equation}
We have a short exact sequence of diagrams on $S$:
\[
0\to \delta_s(W)\to \low{s}{W}\to \low{\bar{s}}{W}\to 0
\]
where $\delta_s(W)$ is the Dirac sheaf, whose stalk in $s$ is $W$ and zero otherwise. The functor $\Fu$ is concrete, hence exact, hence we have a short exact sequence of cochain complexes
\begin{equation}\label{eqShortExactCochainDelta}
0\to C^*(S;\delta_s(W))\to C^*(S;\low{s}{W})\to C^*(S;\low{\bar{s}}{W})\to 0.
\end{equation}
By zig-zag Lemma (Lemma~\ref{lemZigZag}), we have an induced long exact sequence in cohomology
\[
\cdots\to H^j(C^*(S;\low{s}{W}))\to H^j(C^*(S;\low{\bar{s}}{W}))\to  H^{j+1}(C^*(S;\delta_s(W)))\to H^{j+1}(C^*(S;\low{s}{W}))\to\cdots
\]
Since $\Fu$ honestly computes cohomology, the groups $H^j(C^*(S;\low{s}{W}))\cong H_{AG}^j(X_S;\low{s}{W})$ vanish, since $\low{s}{W}$ is injective (it doesn't vanish in degree 0, where the map $H^j(C^*(S;\low{s}{W}))\to H^j(C^*(S;\delta_s(W)))$ is an identity isomorphism $W\to W$, such cases should be considered separately, as usual). From this vanishing, we have an isomorphism~\footnote{It can be considered a sheaf-theoretical analogue of the suspension isomorphism in singular homology.}
\begin{equation}\label{eqConeIsomorphism}
H^{j+1}(C^*(S;\delta_s(W)))\cong H^j(C^*(S;\low{\bar{s}}{W}))\mbox{ for }j>0.
\end{equation}
Now notice that, by assumption, the stalk at $s$ appears only once in the components of $\Fu$ of degree $k$, and $\delta_s(W)$ is concentrated in the point $s$. Therefore $C^k(S;\delta_s(W))=W$, while other components of this cochain complex vanish. Passing to cohomology we get $H^k(S;\delta_s(W))=W$ and all other cohomology groups vanish. Applying isomorphism~\eqref{eqConeIsomorphism} and remembering that $\Fu$ honestly computes sheaf cohomology we get that
\begin{equation}\label{eqSphereLikeVanishing}
H^{k-1}_{AG}(X_S;\low{\bar{s}}{W})\cong W, \mbox{ and } H^{j}_{AG}(X_S;\low{\bar{s}}{W})=0\mbox{ for }j\neq 0,k-1.
\end{equation}
It remains to notice that $\low{\bar{s}}{W}$ is a sheaf supported on subposet $S_{<s}$ which is a lower order ideal. Corollary~\ref{corLowerSheafIsSingularCohomology} implies that $H^*_{AG}(X_S;\low{\bar{s}}{W})$ is isomorphic to singular cohomology of $|S_{<s}|$ with coefficients in $W$. From the vanishing condition~\eqref{eqSphereLikeVanishing} it follows that
\[
\Hr^j(|S_{<s}|;W)=\begin{cases}
                    W, & \mbox{if } j=k-1 \\
                    0, & \mbox{otherwise}
                  \end{cases}
\]
for reduced cohomology groups of the geometrical realization. This shows that the space $|S_{<s}|$ is a ``$W$-cohomological $(k-1)$-sphere'' for each injective group $W$. We need to show that this latter condition is the same as $|S_{<s}|$ being a $\Zo$-homological sphere. This last step is a bit technical, and can be omitted for vector spaces $\ko\Vect$.

Applying the universal coefficient theorem in cohomology~\cite[p.195]{Hatcher} to the space $X=|S_{<s}|$ we get the short exact sequence
\[
0\to\Ext^1_{\Zo}(H_{j-1}(X;\Zo),W)\to \Hr^j(X;W)\to \Hom(\Hr_j(X;\Zo);W)\to 0.
\]
Since $W$ is injective, the $\Ext$-term vanishes, so there is an isomorphism $\Hr^j(X;W)\cong \Hom(\Hr_j(X;\Zo);W)$. For $j\neq k-1$, the module $\Hr^j(X;W)$ vanishes for any $W$, therefore $\Hr_j(X;\Zo)$ vanishes as well (a nontrivial group can always be nontrivially mapped to its injective hull). For $j=k-1$, we have $\Hom(\Hr_j(X;\Zo);W)\cong \Hr^j(X;W)\cong W$ for any injective $W$. In general there exist various abelian groups $G$ satisfying the property
\begin{equation}\label{eqHomProperty}
\Hom(G,W)\cong W \mbox{ for any injective } W,
\end{equation}
not only $G=\Zo$ (for example $G=\Qo$ works as well). Luckily for us, the space $X=|S_{<s}|$ is a finite simplicial complex, so its homology $\Hr_k(X;\Zo)$ with integral coefficients is a finitely generated abelian group. Among such groups, only $\Zo$ satisfies the property~\eqref{eqHomProperty}. This argument shows that $\Hr_*(|S_{<s}|;\Zo)\cong \Hr_*(S^{k-1};\Zo)$ as required.
\end{proof}

Proposition~\ref{propOneShotOnePosition} proves the implication $1\Rightarrow 2$ in Theorem~\ref{thmOneShotTheorem}. If every stalk $D(s)$ is used once by an honest functor $\Fu$, then every space $|S_{<s}|$ has homology of a sphere, which is the definition of a homology Morse cell poset.

\begin{rem}\label{remOneCohomologyOneBullet}
Informally, the proof of implication $1\Rightarrow 2$ in Theorem~\ref{thmOneShotTheorem} can be characterized by the following mantra. ``Let us treat stalks as bullets. If we want to shoot all cohomology with these bullets, we should be able in particular to shoot the subposets $S_{<s}$. To be able to do this in one shot, the total reduced homology of $S_{<s}$ should be 1-dimensional, i.e. $|S_{<s}|$ has homology of a sphere.''
\end{rem}

Somehow surprisingly the proof of the opposite implication is a bit more subtle. This proof utilizes certain intuitions from the proof of Proposition~\ref{propOneShotOnePosition} and eventually leads to a precise algorithm for cohomology computation for general posets to be described in the next subsection.

\begin{con}\label{conIncidenceRequirements}
Let $\ca{X}$ be a homology Morse poset. If $\sigma\in\ca{X}$ is a cell, then there exists $k\geq 0$ such that $|\dd C(\sigma)|=|\ca{X}_{<\sigma}|$ has homology of $S^{k-1}$. In Subsection \ref{subsecMathCellVariations} we called such $k$ the dimension of a cell $\sigma$. Notice that dimensions defined this way do not form a grading on a poset, they are not even monotonic: a situation may occur that $\sigma_1<\sigma_2$ while $\dim\sigma_1\geq \dim\sigma_2$ as demonstrated in Example~\ref{exMorseExample}. Nevertheless, we can try to construct a Morse chain complex $(C_*^{M}(\ca{X};\Zo),\dd)$ which computes singular homology of $|\ca{X}|$ in a geometrically meaningful way (to be made precise below). We set by definition
\[
C_j^{M}(\ca{X};\Zo)=\bigoplus_{\dim\sigma=j}\Zo,
\]
with $j$-dimensional cells $\sigma$ forming the distinguished basis of this module. The nontrivial part of the construction is to define the chain differential $\dd\colon C_j^{M}(\ca{X};\Zo)\to C_{j-1}^{M}(\ca{X};\Zo)$, i.e. the incidence numbers $\inc{\sigma}{\tau}\in\Zo$ such that $\dd(\sigma)=\sum_{\dim\tau=\dim\sigma-1}\inc{\sigma}{\tau}\tau$. We cannot just initialize incidence numbers at random if want to use them in honest cohomology calculations. They way how incidence numbers are used will tell us which properties they should satisfy.

Let's pretend that we already have some incidence numbers. Then we can construct a functor $\Fu_M\colon\Shvs(\ca{X},\Vv)\to\Cochain(\Vv)$ similar to cellular cochain complex:
\begin{equation}\label{eqMorseCochain}
\Fu_M(D)=C^*_M(\ca{X};D),\quad C^*_M(\ca{X};D)=\bigoplus_{\dim\sigma=j}D(\sigma),\quad d_M=\bigoplus_{\dim\tau=\dim\sigma-1}\inc{\sigma}{\tau} D(\tau<\sigma).
\end{equation}
\end{con}

In order for $\Fu_M$ to be well-defined and honestly compute sheaf cohomology we need three conditions to be satisfied.

\begin{req}\label{reqOnIncidenceNumbers}
\begin{enumerate}
  \item $\inc{\sigma}{\tau}=0$ if $\tau\nless\sigma$. Otherwise there are undefined summands in the differential $d_M$ in formula~\eqref{eqMorseCochain}.
  \item $\Fu_M$ should land in cochain complexes. This means $d_M^2=0$. Equivalently, incidence numbers should satisfy the diamond relation~\eqref{eqDiamond}. Equivalently, we should have $\dd^2=0$, so $C^M_*(\ca{X};\Zo)$ should be a chain complex.
  \item The augmented complex $0\to I(\ca{X})\to \Fu_M(I)$ is acyclic for any injective sheaf $I$ of the form $\low{\sigma}{W}$ with injective group $W$. This latter condition is equivalent to vanishing of the reduced Morse homology of any subposet $C(\sigma)=\ca{X}_{\leq\sigma}$ according to the universal coefficient theorem.
\end{enumerate}
\end{req}

We will use the requirements~\ref{reqOnIncidenceNumbers} to define incidence numbers inductively in Construction~\ref{conMorseIncidenceNumbers} below. To justify the induction step we need a technical construction and a lemma, which resembles a topological analogue of Theorem~\ref{thmHonestCohomology}.

\begin{con}\label{conMorseChainSubposet}
Recall from Remark~\ref{remSubposet} that (Morse) cell subposet of a (Morse) cell poset $\ca{Y}$ is any lower order ideal of $\ca{Y}$. Notice that whenever a Morse chain complex $(C_*^M(\ca{Y};\Zo), \dd_M)$ is defined on $\ca{Y}$ in the way that requirements~\ref{reqOnIncidenceNumbers} are fulfilled, it can be restricted, in a straightforward manner, to any cell subposet $\ca{Z}\subseteq\ca{Y}$. We denote the restricted chain complex by $(C^M_*(\ca{Z},\Zo),\dd_M)$ --- the incidence numbers are inherited from those on $\ca{X}$.
\end{con}

\begin{lem}\label{lemHonestHomology}
Assume that incidence numbers are defined on a homology Morse poset $\ca{Y}$ and satisfy the requirements~\ref{reqOnIncidenceNumbers}. Then, for any cell subposet $\ca{Z}\subseteq\ca{Y}$, the homology of the chain complex $(C^M_*(\ca{Z},\Zo),\dd_M)$ are isomorphic to the integral homology of the geometrical realization of $\ca{Z}$:
\[
H_i((C^M_*(\ca{Z},\Zo),\dd_M))\cong H_i(|\ca{Z}|;\Zo).
\]
\end{lem}

\begin{proof}
We prove the lemma by clumsily reducing it to something already proven~\footnote{Right in the spirit of anecdotes about mathematicians.}. Let $W$ be any injective abelian group. Consider the sheaf $W_{\ca{Z}}$ constantly supported on $\ca{Z}$ as in Construction~\ref{conConstantSheafOnLowerIdeal}. We have a sequence of isomorphisms
\begin{equation}\label{eqSequenceOfHomo}
  \begin{split}
\Hom(H_*(|\ca{Z}|;\Zo),W) & \stackrel{1}{\cong} H^*(|\ca{Z}|;W)\\
 & \stackrel{2}{\cong} H^*_{AG}(\ca{Y};W_Z)\\
 & \stackrel{3}{\cong} H^*(C^*_M(\ca{Y};W_{\ca{Z}}))\\
 & \stackrel{4}{\cong} H^*(\Hom((C_*^M(\ca{Z};\Zo),\dd_M),W))\\
 & \stackrel{5}{\cong} \Hom(H_*(C_*^M(\ca{Z};\Zo),\dd_M),W),
\end{split}
\end{equation}
where (1) is the universal coefficients isomorphism due to $W$ being injective (hence the functor $\Hom(\cdot,W)$ is exact), (2) is by Corollary~\ref{corLowerSheafIsSingularCohomology}, (3) is by the fact that Morse complex functor $\Fu_M$ honestly computes cohomology (since incidence numbers on $\ca{Y}$ are assumed to satisfy requirements~\ref{reqOnIncidenceNumbers}, we can apply Theorem~\ref{thmHonestCohomology} to this functor), (4) is just rewriting the definition of $C^*_M(\ca{Y};\ca{Z}_W)$, and, finally, (5) is once again an exactness of $\Hom(\cdot,W)$ for injective $W$.

If two groups $A,B$ are finitely generated and satisfy $\Hom(A,W)\cong\Hom(B,W)$ for any injective $W$, then $A\cong B$. Both groups $H_*(|\ca{Z}|;\Zo)$ and $H_*(C_*^M(\ca{Y};\Zo),\dd_M)$ are finitely generated, hence the claim follows from~\eqref{eqSequenceOfHomo}.
\end{proof}

Now we are finally ready to construct incidence numbers for any finite Morse cell poset.

\begin{con}\label{conMorseIncidenceNumbers}
At first, choose a topological sorting $\sigma_1,\ldots,\sigma_m$ of all cells of the Morse cell poset $\ca{X}$, i.e. a linear order in which, for any $\sigma_i<\sigma_j$ we have $i<j$ (all cells which are smaller than $\sigma_j$ in the partial order, appear earlier in the list). Let $\ca{X}_{(j)}$ denote the subposet $\{\sigma_1,\ldots,\sigma_j\}$ of $\ca{X}$, with the induced partial order. We also formally set $\ca{X}_{(0)}=\varnothing$. All posets $\ca{X}_{(j)}$ are Morse cell posets. Indeed, $(\ca{X}_{(j)})_{<\sigma}=\ca{X}_{<\sigma}=\dd C(\sigma)$ by the defining property of a topological sorting. The poset $\ca{X}_{(j)}$ is obtained from the poset $\ca{X}_{(j-1)}$ by adding a cell $\sigma_j$. Topologically, the space $|\ca{X}_{(j)}|$ is obtained from $|\ca{X}_{(j-1)}|$ by attaching the cone $|C(\sigma_j)|$ over $\dd|C(\sigma_j)|$ with the apex in $\sigma_j$. We construct incidence numbers $\inc{\sigma_j}{\tau}$ inductively for $j\in\{1,\ldots,m\}$. The base of induction $j=0$ is superfluous: there are no cells in the empty poset, hence no need to construct incidence numbers.

Assume that incidence numbers $\inc{\sigma_i}{\tau}$ are already defined for $i\leq j-1$ in a way that requirements~\ref{reqOnIncidenceNumbers} are satisfied on the poset $\ca{X}_{(j-1)}$. We need to extend this definition to the next filtration term $\ca{X}_{(j)}$. This means that we need to define $\inc{\sigma_j}{\tau}$ for the newly added cell $\sigma_j$. Let $k=\dim\sigma_j$. If $k=0$, there is no need to construct incidence numbers $\inc{\sigma_j}{\tau}$ since there are no cells $\tau$ of dimension $-1$ in the poset. Henceforth, in the following we assume $k\geq 1$.

The first natural requirement tells that $\inc{\sigma_j}{\tau}=0$ if $\tau\nless\sigma_j$. So far, we may restrict to only those cells $\tau$ which satisfy $\tau<\sigma_j$ and $\dim\tau=k-1$. Consider the subposet $\dd C(\sigma_j)=\ca{X}_{<\sigma_j}$ --- it is a cell subposet of $\ca{X}_{j-1}$. Since incidence numbers are already constructed on $\ca{Y}=\ca{X}_{j-1}$, we are in position to apply Lemma~\ref{lemHonestHomology} to the cell subposet $\dd C(\sigma_j)$, which gives an isomorphism
\[
H_*(C^M_*(\dd C(\sigma_j);\Zo),\dd_M)\cong H_*(|\dd C(\sigma_j)|;\Zo)
\]
Now, by assumption, the space $|\ca{X}_{<\sigma_j}|$ has the same integral homology as a sphere $S^{k-1}$, i.e. the homology are concentrated in degrees $0$ and $k-1$. Let us augment the chain complex $C^M_*(\dd C(\sigma_j);\Zo)$ with the term $\Zo$ and the map
\[
\dd\colon C^M_0(\dd C(\sigma_j);\Zo)=\bigoplus_{\dim\sigma=0}\Zo\to C^M_{-1}(\dd C(\sigma_j);\Zo)=\Zo,
\]
which sends all the chosen generators $\sigma$, $\dim\sigma=0$, to $1$. Then reduced homology $\Hr_*$ (i.e. the homology of the augmented complex $0\leftarrow\Zo\leftarrow C^M_*(\dd C(\sigma_j);\Zo)$) are concentrated in a single degree $k-1$. We have
\[
\Zo\cong \Hr_{k-1}(C^M_\ast(\dd C(\sigma_j);\Zo))=\dfrac{Z_{k-1}}{B_{k-1}}=\dfrac{\Ker\dd\colon C^M_{k-1}(\dd C(\sigma_j);\Zo)\to C^M_{k-2}(\dd C(\sigma_j);\Zo)}{\im \dd\colon C^M_k(\dd C(\sigma_j);\Zo)\to C^M_{k-1}(\dd C(\sigma_j);\Zo)}.
\]
Take any chain $c\in Z_{k-1}$ of the Morse chain complex which represents a generator of $\Zo\cong\Hr_{k-1}(C_*^M(\dd C(\sigma_j);\Zo))$. By construction, $c$ is an element of
\[
C^M_{k-1}(\dd C(\sigma_j);\Zo)=\bigoplus_{\tau\colon \dim\tau=k-1, \tau<\sigma_j}\Zo
\]
hence
\[
c=\sum\limits_{\tau\colon \dim\tau=k-1, \tau<\sigma_j}b_\tau \tau.
\]
Finally, we set $\inc{\sigma_j}{\tau}=b_\tau$.
\end{con}

\begin{lem}\label{lemIncNumbersInductionStep}
The incidence numbers on $\ca{X}_{(j)}$ defined by Construction~\ref{conMorseIncidenceNumbers} satisfy the requirements~\ref{reqOnIncidenceNumbers}.
\end{lem}

\begin{proof}
We do not need to check anything for the incidence numbers $\inc{\sigma_i}{\tau}$ with $i<j$, since these numbers were inherited from $\ca{X}_{(j-1)}$ and satisfy the requirements~\ref{reqOnIncidenceNumbers} by inductive hypothesis. So we only need to check the properties for $\inc{\sigma_j}{\tau}$. The first requirement is satisfied by assumption: we have $\inc{\sigma_j}{\tau}=0$ unless $\tau<\sigma_j$. The second requirement is also satisfied: by construction we have
\[
\dd_M(\sigma_j)=\sum\limits_{\tau\colon \dim\tau=k-1, \tau<\sigma_j}b_\tau \tau=c\in Z_{k-1}=\Ker\dd_M,%\colon C^M_{k-1}(\ca{X}_{<\sigma_j};\Zo)\to C^M_{k-2}(\ca{X}_{<\sigma_j};\Zo),
\]
hence $\dd_M^2(\sigma_j)=0$. Finally, the third property is about the subposet $\ca{X}_{\leq\sigma}$ being Morse-acyclic. It holds by construction: the Morse complex of the poset $C(\sigma)$ differs from the Morse complex of $\dd C(\sigma)$ in one term $\Zo$ of degree $k=\dim\sigma$. When passing to homology, this term kills the unique $(k-1)$-homology of $\dd C(\sigma)$ (without any torsion left, since $[c]$ was chosen to be a generator of $H_{k-1}(\dd C(\sigma);\Zo)$), --- and dies itself.
\end{proof}

Lemma~\ref{lemIncNumbersInductionStep} completes the induction step in the construction of incidence numbers on a Morse cell poset $\ca{X}$. Using these incidence numbers, we define the functor $\Fu_M\colon \Shvs(\ca{X};\Vv)\to\Vv$ by~\eqref{eqMorseCochain}. It is a concrete one-shot functor which honestly computes cohomology of sheaves on $\ca{X}$. The implication $2\Rightarrow 1$ of Theorem~\ref{thmOneShotTheorem} is proven.

\begin{rem}
It is possible to consider diagrams over arbitrary finite categories: cohomology can be defined as right derived functors similar to the case of posets. In this generality, one-shot computations of sheaf cohomology can be extended to cellular categories as defined in~\cite{CellCat}.
\end{rem}

\subsection{Structure agnostic cohomology computations}\label{subsecMathMinimalComputations}

One might wonder, what prevents us from extending Remark~\ref{remDeletePoint} (about redundant stalks) and Theorem~\ref{thmOneShotTheorem} (about stalks used exactly once) to more general classes of posets? Literally nothing! We develop more general approach in this subsection and provide an algorithm for sheaf cohomology computations which satisfies two properties.
\begin{enumerate}
  \item The algorithm is applicable to sheaves over all posets. Its complexity (see Definition~\ref{definMultiplicityOfStalk}) is lower than complexity of Roos complex, and is proven to be the minimal possible.
  \item On the class of homology Morse cell posets (in particular on cell posets) this algorithm becomes the classical algorithm of cohomology calculation via cellular (or Morse) cochain complex: in this case the algorithm computes incidence numbers.
\end{enumerate}
The ideas of this subsection are similar to the idea of local homology sheaf applied in stratification learning~\cite{brown2021sheaf}, although the ultimate goals are different. While stratification learning aims to build a convenient filtration on a geometrical structure by utilizing sheaf theory~\footnote{As well as, implicitly Zeeman--McCrory spectral sequence~\cite{McCrory}}, our goal is to construct an optimized algorithm for general sheaf cohomology computation. Our considerations also have much in common with those of~\cite{CIANCI20171}, where similar arguments were applied to constant sheaves on posets.

\begin{defin}\label{definMultiplicityOfStalk}
Assume that a concrete functor $\Fu\colon \Shvs(X_S;\Vv)\to\Cochain(\Vv)$ honestly computes cohomology of sheaves on $S$. The number of times the stalk $D(s)$ appears as a direct summand in the $k$-th graded component $\Fu(\ca{D})^k$ is called \emph{the multiplicity of} $s$, and is denoted $\mult_k(s,\Fu)$. The total number
\[
c(\Fu)=\sum_{s\in S, k\in\Zo}\mult_k(s,\Fu)
\]
is called \emph{the complexity of the functor} $\Fu$.
\end{defin}

\begin{ex}\label{exMultiplicityCW}
For a cell poset $\ca{X}$ and a cellular cochain functor $\Fu_{CW}\colon \Shvs(\ca{X};\Vv)\to\Cochain(\Vv)$ we have
\[
\mult_k(\sigma,\Fu_{CW})=\begin{cases}
                           1, & \mbox{if } k=\dim\sigma \\
                           0, & \mbox{otherwise}.
                         \end{cases}
\]
for any cell $\sigma$. Notice that, whenever $\dim\sigma=0$, we necessarily have $\ca{X}_{<\sigma}=\varnothing$, and formally $\Hr_{-1}(\varnothing;\ko)=\ko$ (this homology comes from the augmentation term). Therefore $0$-dimensional cells are not exceptions in this example. The complexity $c(\Fu_{CW})$ equals the number of cells of $\ca{X}$.
\end{ex}

\begin{ex}\label{exMultiplicityMorse}
For a homology Morse poset $\ca{X}$ and a Morse cochain functor $\Fu_{M}$, similar to the previous example, we have $\mult_k(\sigma,\Fu_M)=1$ if $\dim\sigma=k$, and $0$ otherwise. The complexity $c(\Fu_{M})$ again equals the number of cells of $\ca{X}$. These are precisely all one-shot functors according to Theorem~\ref{thmOneShotTheorem}.
\end{ex}

\begin{ex}\label{exMultiplicityRoos}
For the Roos cochain functor $\Fu_{Roos}$ we have
\[
\mult_{\Fu_{Roos}}(s)=f_{k-1}(\ord(S_{<s})),
\]
where $f_{j}(\ca{K})$ denotes the number of $j$-dimensional simplices of a nonempty simplicial complex $\ca{K}$, while it is formally assumed that $f_{-1}(\ca{K})=1$ for any simplicial complex $\ca{K}$, even for $\ca{K}=\varnothing$. This formula follows easily from the count of chains $s_1<\cdots<s_k<s$ of $S$ which end in a given element $s$. The complexity $c(\Fu_{Roos})$ equals the number of simplices of $\ord(S)$.
\end{ex}

To avoid complications with torsion, for the rest of this subsection we work with the target category $\Vv=\ko\Vect$, the category of vector spaces over a fixed field $\ko$.

\begin{thm}\label{thmMorseBound}
For any functor $\Fu$ which honestly computes cohomology of sheaves on a poset $S$, and for any element $s\in S$, the multiplicity $\mult_k(s,\Fu)$ is at least the total reduced Betti number $\br_{k-1}(|S_{<s}|)=\dim\Hr_{k-1}(|S_{<s}|;\ko)$ of the down-set of $s$.
\end{thm}

\begin{proof}
The proof is completely similar to Proposition~\ref{propOneShotOnePosition}. Consider a short exact sequence of sheaves on $S$:
\begin{equation}\label{eqShortExactDirac}
0\to \delta_s(\ko)\to\low{s}{\ko}\to \low{\bar{s}}{\ko}\to 0,
\end{equation}
where $\low{s}{\ko}$ is the basic injective sheaf defined in Construction~\ref{conInjSheaves}, $\low{\bar{s}}{\ko}$ is defined in~\eqref{eqCutLow}, and $\delta_s(\ko)$ is the Dirac diagram supported on $s$, see Construction~\ref{conDiracDiagram}. Notice that $\dim \Fu(\delta_s(\ko))^k=\mult_k(s,\Fu)$ --- because stalks in all points except $s$ vanish, and $\delta_s(\ko)(s)=\ko$ contributes to the $k$-th graded component of the cochain complex $\mult_k(s,\Fu)$ many times by construction.

Taking long exact sequence of cohomology induced by the short exact sequence~\eqref{eqShortExactDirac}, and noticing that the sheaf $\low{s}{\ko}$ is acyclic, we see that $H_{AG}^*(X_S;\delta_s(\ko))\cong \Hr_{AG}^{*-1}(X_S;\low{\bar{s}}{\ko})$. However, by Corollary~\ref{corLowerSheafIsSingularCohomology} we know that $\Hr_{AG}^{*-1}(X_S;\low{\bar{s}}{\ko})$ is isomorphic to the singular cohomology $\Hr^{*-1}(|S_{<s}|;\ko)$. On the other hand, $H_{AG}^*(X_S;\delta_s(\ko))\cong H^*(\Fu(\delta_s(\ko)))$, since $\Fu$ honestly computes cohomology. Since cohomology is a subquotient of the cochain complex, we have
\[
\br_{k-1}(|S_{<s}|)=\dim H^{k}(\Fu(\delta_s(\ko)))\leq \dim \Fu(\delta_s(\ko))^k=\mult_k(s,\Fu),
\]
as needed.
\end{proof}

\begin{rem}\label{remEulerChar}
It follows from the proof of Theorem~\ref{thmMorseBound} that $\sum_{k=0}^{\infty}(-1)^k\mult_{k,\Fu}=\sum_{k=0}^{\infty}(-1)^k\br_{k-1}(|S_{<s}|)$, whenever the first sum makes sense. Indeed, the first expression is the Euler characteristic of the cochain complex $\Fu(\delta_s(\ko))^k$, while the second one is the Euler characteristic of its cohomology. It is well-known that Euler characteristics of any cochain complex and that of its cohomology coincide (see e.g.~\cite[Thm.2.44]{Hatcher} or~\cite[Lm.5.1]{robinson2014topological}. Together with Theorem~\ref{thmMorseBound}, this remark allows to derive Proposition~\ref{propOneShotOnePosition} in the category $\Vv=\ko\Vect$.
\end{rem}

The next statement asserts that the estimation given by~\ref{thmMorseBound} is exact. There exists a functor $\Fu_{\min}$ which realizes the lower theoretical bound on multiplicity for all elements $s\in S$ at once, hence on the total complexity.

\begin{thm}\label{thmExactBound}
For any finite poset $S$, there exists a concrete functor
\[
\Fu_{\min}\colon\Shvs(X_S;\Vv)\to\Cochain(\Vv)
\]
which honestly computes cohomology and satisfies $\mult_k(s,\Fu_{\min})=\br_{k-1}(|S_{<s}|)$ for any $s\in S$.
\end{thm}

To prove this statement we provide an algorithm which constructs the required functor $\Fu_{\min}$ given a finite poset $S$. First we give some specifications and definitions to simplify the implementation of this algorithm.

\begin{con}\label{conChainCpxDataStructure}
The main step, given an arbitrary poset $S$, is to construct a chain complex $(C_*(S;\ko),\dd)$ which stores information on ``incidence numbers'' (more preciesly, their analogues when no assumption on topology of $|S_{<s}|$ is posed). 

To encode a (finite-dimensional) chain complex $(C_*,d)$ over a field $\ko$ we use the data structure $\ChCpx$ which consists of
\begin{enumerate}
  \item a finite set $A$, denoting a set of homogeneous generators of the total space $\bigoplus_{j\geq 0}C_j$;
  \item a grading $\deg\colon A\to \Zg$;
  \item a map $M\colon A\times A\to \ko$, which denotes the matrix of an operator $\dd$ in the chosen basis $A$ (first component stands for input=indicates the columns, the second stands for the output and indicates rows).
\end{enumerate}
The contract on $\ChCpx$ (assumed but not checked explicitly) is that (1) for each $a,c\in A$ we have $\sum_{b\in A}M(a,b)M(b,c)=0$ (i.e. $d^2=0$), (2) for each $a,b\in A$ with $\deg(a)\neq \deg(b)+1$, $M(a,b)=0$ (the differential lowers degree by $1$).

We also have the augmented version of this structure, $\AChCpx$, where the grading takes values in $\{-1\}\sqcup\Zg$. The cochain complex $\CochCpx$ can be defined by transposing the matrix $M$, i.e. inverting the order of arguments (the only difference from chain complexes is that cohomological differential raises degree by $1$).
\end{con}

Two auxiliary procedures will be used in the algorithm, whose contracts are described below.

\begin{enumerate}
  \item A function HomologyBasis consumes $\ChCpx$ (resp., $\AChCpx$) and outputs representing chains of generators of its homology, i.e. an indexing set $H$ together with the degree function $\deg\colon H\to \Zg$ (resp. $\{-1\}\sqcup\Zg$), and the representative chains $RC\colon H\to (A\to \ko)$, satisfying the condition that $\deg(a)=\deg(i)$ for any nonzero coefficient $RC(i)(a)\neq 0$. The procedure HomologyBasis has standard realizations based on variations of Gauss algorithm, see~\cite{KACZYNSKI199859} and the implementation available in~\cite{Sage}.%, used in topological data analysis, see e.g.~\cite{edelsbrunner2002topological}.
  \item A function Restrict consumes an augmented chain complex $(C_*,d)$, on a set $A$ of generators, and a subset $B\subset A$, and outputs the restricted complex, that is simply $B\times B$-block of the matrix of the differential. We assume the restriction thus defined makes sense, i.e. the differential preserves the vector subspace spanned by $B$.
  \item A function TopoSort consumes a poset $S$ and outputs an ordered list $(s_1,\ldots,s_m)$ of its elements which is a linearization of $S$.
\end{enumerate}

Returning back to the complex associated with a poset, a vector space $C_*(S;\ko)$ will have type $\AChCpx$ and consist of the triple $(G^a,\deg,I)$, where $G^a$ is the set of generators and $I$ is the matrix of ``incidence numbers''. All generators from $G^a$ except the augmentation term will be stitched to specific elements of $S$, so we fix a decomposition $G^a=\{\varnothing\}\sqcup G$, and $G=\bigsqcup_{s\in S}G_s$.
%
%is going to be a graded vector space equipped with a specified homogeneous finite basis $G$ (the set of generators), so that $\dd$ is treated as a square matrix $I$ of size $\#G$. It is believed that $I$ is a sparse matrix, so we would treat it like this: for any $g\in G$, only the nonzero components of the vector $Ig$ are stored i.e. a set $\{(g_1,a_1),\ldots,(g_p,a_p)\}$, where $g_i\in G$, $a_i\in\ko$. Each generator $g\in G$ comes equipped with two data structures: $\deg(g)\in\{-1,0,1,\ldots\}$ is the degree of this element in a chain complex, and $\relel(g)$ is an element of $S$, from which this generator originated. The only exception will be the augmentation generator $\varnothing$, which has degree $-1$ and is not related to any element of $S$.

The set $G^a$ of generators and the matrix $I$ are constructed inductively. We first choose a topological sorting of $S$ and proceed by computing and killing homology of all down-sets $|S_{<s}|$ one by one, according to the following algorithm. The values of $I(a,b)$ which are not explicitly specified are set equal to $0$. Whether this procedure is actually needed depends on the representation format of matrices. For example, if the matrices are represented in sparse format, this procedure is unnecessary. However, in the subsequent algorithm of sheaf cohomology computation, it will be more convenient to work with matrices represented as dense matrices, or double arrays.

\begin{algorithm}
    \caption{Incidence matrix computation}\label{algIncMatrixMain}
    \begin{algorithmic}[1]
    \REQUIRE Poset $S$.
    \STATE \textbf{procedure} $S_{sorted}$=TopoSort($S$)
    \STATE \textbf{create} generator $\varnothing$ with $\deg(\varnothing)=-1$
    \STATE \textbf{define} $C=(\{\varnothing\},\deg,I)$ of type $\AChCpx$
    \FOR{each $s$ in $S_{sorted}$}
        \STATE \textbf{define} $G_s=\{\}$
        \STATE \textbf{define} $S_{<s}=\{t\in S\mid t<s\}$
        \STATE \textbf{define} $G|_{<s}=\{\varnothing\}\sqcup\bigsqcup_{t\in S_{<s}}G_t$
        \STATE \textbf{procedure} $C|_{<s}$=Restrict($C$, $G|_{<s}$)
        \STATE \textbf{procedure} $(H,\deg,RC)$=HomologyBasis($C|_{<s}$)
        \FOR{each homology generator $i$ in $H$}
            \STATE \textbf{create} generator $g$ with $\deg(g)$ equal to $\deg(i)+1$
            \STATE \textbf{append} $g$ to $G_s$
            \FOR{each $h$ in $G$}
                \STATE \textbf{set} $I(h,g)$ equal to $0$
                \IF{$h$ lies in $G|_{<s}$}
                    \STATE \textbf{set} $I(g,h)$ equal to $RC(i)(h)$
                \ELSE
                    \STATE \textbf{set} $I(g,h)$ equal to $0$
                \ENDIF
            \ENDFOR
        \ENDFOR
        \STATE \textbf{append} $G_s$ to $G^a$
    \ENDFOR
    \STATE \textbf{output} set of generators $G^a$, matrix $I$
    \end{algorithmic}
\end{algorithm}

\begin{proof}[Proof of Theorem~\ref{thmExactBound}]
As long as the set of generators $G^a$ and the incidence matrix $I$ are computed by Algorithm~\ref{algIncMatrixMain}, they can be used to compute cohomology of any sheaf $D$ on $S$. Since we have a decomposition $G^a=\{\varnothing\}\sqcup G$ and $G=\bigsqcup_{s\in S}G_s$, it will be convenient to adopt a convention from dependent type theory: if $g\in G$, then $\pr_S(g)\in S$ denotes the index of the corresponding component of the union, and $\pr_G(g)\in G_{\pr_S(g)}$ denotes the corresponding element of this component.

Recall that $I(g,h)$ is the element of the matrix $I$, lying in $g$-th column and $h$-th row. Mathematically, given a diagram $D$ on $S$ (a sheaf $\ca{D}$ on the corresponding Alexandrov topological space $X_S$) we define the cochain complex $(C^*_{\min}(S;D), d_{\min})$ by the formulas
\begin{equation}\label{eqMinComplex}
C^j_{\min}(S;D)=\bigoplus_{g\in G,\deg g=j}D(\pr_S(g)), \qquad d_{\min}^j\colon C^j_{\min}(S;D)\to C^{j+1}_{\min}(S;D)
\end{equation}
\begin{equation}\label{eqMinComplexDifferential}
d_{\min}^j=\bigoplus_{\substack{g_1,g_2\in G}{\deg(g_1)=j,\deg(g_2)=j+1}}I(g_2,g_1)D(\pr_S(g_1)<\pr_S(g_2)).
\end{equation}
for any $j=0,1,\ldots$. We set $\Fu_{\min}(D)=(C^*_{\min}(S;D), d_{\min})$. This construction satisfies the following properties.

\begin{enumerate}
  \item The differential is well-defined since $\inc{g_2}{g_1}$ may be nonzero only if $\pr_S(g_1)<\pr_S(g_2)$, as can be seen directly from Algorithm~\ref{algIncMatrixMain}.
  \item The differential $d_{\min}$ satisfies $d_{\min}^2=0$. The proof is similar to Construction~\ref{conMorseIncidenceNumbers}. Hence $\Fu_{\min}$ lands in $\Cochain(\Vv)$.
  \item Since all summands in~\eqref{eqMinComplex} are functorial, $\Fu_{\min}$ is actually a functor from $\Diag(S;\Vv)$ to $\Cochain(\Vv)$.
  \item Finally, we have $H^*(W\to\Fu_{\min}(\low{s}{W}))=0$ for any $s\in S$ and any vector space $W=\ko^n$ (which is automatically injective). Indeed, we have $\Hr_*(C_*(S_{\leq s};\ko),\dd)=0$ by construction of the incidence matrix in Algorithm~\ref{algIncMatrixMain}. Applying the exact functor $\Hom(\cdot,W)$ we get
      \begin{multline*}
      H^*(W\to\Fu_{\min}(\low{s}{W}))=H^*(\Hom(C_*(S_{\leq s};\ko)\to\ko,W))= \\
      \Hom(H_*(C_*(S_{\leq s};\ko)\to\ko),W)=0.
      \end{multline*}
\end{enumerate}

The assumption of Theorem~\ref{thmHonestCohomology} are satisfied, therefore $\Fu_{\min}$ honestly computes cohomology of sheaves on $S$. The Algorithm~\ref{algIncMatrixMain} can be backtracked and it is seen that the multiplicity
\[
\mult_j(s,\Fu_{\min})=\#G_s
\]
of a stalk $D(s)$ in the vector space $C^j_{\min}(S;D)$ equals the rank $\dim \Hr_j'$ where
\[
\Hr_j'=\Hr_j(C|_{<s}),
\]
is the homology of the complex restricted to generators associated with elements of $S_{<s}$. Notice that $S_{<s}$ is a subset of $S'=\{s_1,\ldots,s\}$, the interval of topological sorting which have already been constructed. By induction, the incidence numbers already defined on $S'$ give honest sheaf cohomology computation for sheaves on $S'$. According to Corollary~\ref{corLowerSheafIsSingularCohomology} (also see details in the proof of Lemma~\ref{lemIncNumbersInductionStep}), the singular homology of lower order ideals of $S'$ can be honestly computed with the same incidence numbers (the restricted chain complex). This is applied to the lower order ideal $S_{<s}\subset S'$. Therefore, it is proven by induction, that
\[
\Hr_j(C|_{<s})\cong \Hr_j(|S_{<s}|;\ko).
\]
Therefore $\mult_j(s,\Fu_{\min})=\dim \Hr_j(|S_{<s}|;\ko)$ which is the predicted lower bound for the multiplicity. Theorem is finally proved.
\end{proof}

\begin{rem}\label{remExternalSolverIntegral}
Notice, that the problem of computing chains representing a basis of homology is outsourced to an external solver HomologyBasis. If the solver can deal with integral matrices, and if it is known that homology of $|S_{<s}|$ are torsion-free, then it is natural to require that the solver outputs not just any basis of homology, but an integral basis, i.e. the set of integral generators. However, things become more tricky if $|S_{<s}|$ has torsion in homology: in this case more generators are required to kill the homology. It is still possible to construct an integral analogue of the functor $\Fu_{\min}$ algorithmically, however, we postpone such description for the future research.
\end{rem}

\begin{rem}\label{remSparsity}
In the case of coefficients in a field, it is natural to define some optimization priors on the work of the homology solver. Notice that the solver should output the generators of a homology group $H_j(C|_{<s})$ which are classes of chains of the corresponding complex. There is a freedom to choose the representatives. A natural prior would be to require the sparsity of the resulting representative chains. The more sparse is the incidence matrix, the easier is the subsequent (co)homology computation. This leads to an interesting practical problem in applied linear algebra.
\end{rem}

\begin{probl}\label{problSparseSolution}
Consider a couple of vector subspaces $B\subset Z\subset C=\ko^n$. Find vectors $c^{(1)},\ldots,c^{(k)}\in C$ such that
\begin{enumerate}
  \item the set $\{[c^{(1)}],\ldots,[c^{(k)}]\}$ is a basis of $Z/B$,
  \item the total number $\sum_{i=1}^{k}\#\{j\in[n]\mid c_j^{(i)}\}$ of vanishing components is maximized.
\end{enumerate}
\end{probl}

An algorithm, which effectively solves Problem~\ref{problSparseSolution} should be preferred in a system which computes cohomology of sheaves.

\begin{rem}\label{remMorseExampleAndApproximation}
Recalling Example~\ref{exMorse}, we see a freedom in the definition of the incidence numbers, namely, we can set $\dd(b)=k_1\mathbf{1}+k_2\mathbf{2}-\mathbf{3}$ with any choice of parameters $(k_1,k_2)\in\Zo^2$ such that $k_1+k_2=1$, and it leads to the honest cohomology computations. According to Remark~\ref{remSparsity}, we should prefer sparse solutions, which are either $(k_1,k_2)=(1,0)$ or $(k_1,k_2)=(0,1)$. Notice that these two ways of defining a differential neatly correspond to the two geometrically meaningful ways of turning a Morse poset $\ca{Y}$ into a cell poset, see Fig.~\ref{figElementaryMorse}. In certain sense, the requirement of sparsity of solutions at each step somehow resembles cellular approximation theorem --- in a homological language. %This observation deserves further study.
\end{rem}

\begin{rem}
It follows from Algorithm~\ref{algIncMatrixMain} that, whenever $|S_{<s}|$ is acyclic, there are no generators $g$ related to the element $s$. This means that the stalk $D(s)$ does not contribute to $C_{\min}^*(S;D)$ at all. In particular, it follows that cohomology of any sheaf $D$ do not depend on the stalk $D(s)$ with acyclic downset $D_{<s}$. This is well-aligned with Remark~\ref{remDeletePoint}.
\end{rem}

For convenience we formulate an algorithm to compute cohomology of arbitrary diagram $D$ on a finite poset $S$.

\begin{con}
First of all, we encode a diagram $D$ by specifying the following data structures
\begin{enumerate}
  \item for each $s\in S$, a finite set $N_s$ is specified. This corresponds to a basis of the stalk $D(s)$;
  \item for each pair $s,t\in S$ such that $s<t$ in $S$, a function $D_{st}\colon N_s\times N_t\to \ko$ is specified. This function encodes the matrix of the operator $D(s<t)$ written in the chosen bases.
\end{enumerate}
Notice that there are many redundant matrices in this structure: since the compositionality holds for a sheaf, it is sufficient to define the matrices $D_{st}$ only for the edges $(s,t)$ of the Hasse diagram of $S$ (see Construction~\ref{conHasseDiag}), while all other matrices can be recovered as their products. However, it will be more convenient to work with this redundant structure: the several-hop matrices
\[
D(s_0<\cdots<s_k)=D(s_{k-1}<s_k)\cdots D(s_0<s_1)
\]
may appear explicitly in the calculation.
\end{con}

It will be assumed that the set $G=\bigsqcup_{s\in S}G_s$ of generating elements, and the incidence matrix $I\colon G\times G\to\ko$ are already precomputed by Algorithm~\ref{algIncMatrixMain}. The following algorithm is mathematically straightforward: we form the structure $C'$ of type $\CochCpx$ representing the cochain complex $(C^*_{\min}(S;D),d_{\min})$, send send it to HomologyBasis which outputs the set of cochains representing cohomology generators together with their degrees. Recall from Construction~\ref{conChainCpxDataStructure} that $C'$ is a triple $(N,\deg',\Dif)$, where $N$ is a finite set of basic vectors, $\deg\colon N\to\Zg$ is the degree function, and $\Dif\colon N\times N\to \ko$ is the matrix of the differential, assumed to shift degrees by $+1$.

As follows from the definition of the cochain complex, we have $N=\bigsqcup_{g\in G}N_{\pr_S(g)}$. Again, given $n\in N$, the notation $\pr_G(n)\in G$ stands for the corresponding index, and $\pr_N(n)\in N_{\pr_S(g)}$ the element of the corresponding component.

\begin{algorithm}
    \caption{Sheaf cohomology computation}\label{algMainAlg}
    \begin{algorithmic}[1]
    \STATE \textbf{set} $N$ equal to $\bigsqcup_{g\in G}N_{\pr_S(g)}$
    \FOR{each $n$ in $N$}
        \STATE \textbf{set} $\deg(n)$ equal to $\deg(\pr_G(n))$
    \ENDFOR
    \FOR{each $(n_1,n_2)$ in $N\times N$}
        \STATE \textbf{set} $\Dif(n_1,n_2)$ equal to $I(\pr_G(n_2),\pr_G(n_1))\cdot D_{\pr_S(\pr_G(n_1))\pr_S(\pr_G(n_2))}(\pr_N(n_1),\pr_N(n_2))$
    \ENDFOR
    \STATE \textbf{define} $C'=(N,\deg,\Dif)$ of type $\CochCpx$
    \STATE \textbf{procedure} $(H',\deg,RC')$=HomologyBasis($C'$)
    \STATE \textbf{output} $(H,\deg,RC)$
    \end{algorithmic}
\end{algorithm}

Notice that the matrix $\Dif$ is well defined. The factor
\[
D_{\pr_S(\pr_G(n_1))\pr_S(\pr_G(n_2))}(\pr_N(n_1),\pr_N(n_2))
\]
is undefined when $\pr_S(\pr_G(n_1))\nless\pr_S(\pr_G(n_2))$. However, in this case the factor
\[
I(\pr_G(n_2),\pr_G(n_1))
\]
vanishes, as guaranteed by Algorithm~\ref{algIncMatrixMain}, so their product is set equal to $0$.

\begin{rem}\label{remAgnosticComputations}
Notice that Theorems~\ref{thmMorseBound} and~\ref{thmExactBound} do not guarantee that, given a poset $S$ and a sheaf $D$, the complex $\Fu_{\min}(D)$ is the most optimal way to compute cohomology of $D$. As stupid example: if $D$ identically vanishes, the cohomology vanish as well, so there is no need to compute matrices $I$ and $\Dif$. Instead, the theorems assert that, given a poset $S$, the complex $\Fu_{\min}(D)$ is the most optimal for all possible diagrams $D$ on $S$. So the application of Algorithms~\ref{algIncMatrixMain} and~\ref{algMainAlg} makes sense in the context when $S$ is given, and there is a need to compute cohomology of several possible sheaves on $S$; the matrix $I$ depends only on $S$ hence can be reused in the calculations.

Certainly, in specific situation, the matrix $I$ can be found without using Algorithm~\ref{algIncMatrixMain}. For example, if $S$ is known to be a simplicial complex, the incidence numbers are defined in the alternating fashion, as in Construction~\ref{conStandardSimplicialIncNumbers}, which doesn't require the calculation of homology of the down-sets. Algorithm~\ref{algIncMatrixMain} is applicable to any finite poset, hence we call it structure agnostic.

The development of algorithms which efficiently compute cohomology for a single diagram $D$ on $S$ is a challenging problem, which is a subject of the future work.
\end{rem}

\section{Laplacians, diffusion, and beyond}\label{secMathLaplacians}

In this section we restrict to the category $\Ro\Vect$ of real vector spaces. The field $\Ro$ has two important features, distinguishing it from general fields $\ko$ and making it suitable for practical applications. (1) It is ordered. (2) It is a complete metric space. By utilizing the first property one can define the general combinatorial Hodge theory. The second property leads to heat diffusion on sheaves which underlies the usage of sheaves in the design of neural networks. Most mathematical claims about Laplacians follow from the basic linear algebra reviewed in the next subsection. 

\subsection{Linear algebra preliminaries}\label{subsecMathLinAlg}

A euclidean space is a real vector space $V\in\Ro\Vect$ with the chosen inner product $\langle\cdot,\cdot\rangle$. Each (finite-dimensional) euclidean space $V$ is naturally identified with its dual $V^*$ via the isomorphism $v\mapsto \langle v,\cdot\rangle$. Any map $f\colon V\to W$ between euclidean spaces induces the conjugate map $f^*\colon W=W^*\to V=V^*$. If a map $f$ is written as a matrix $F$ in some orthogonal basis, then its conjugate $F^*$ is written with the transpose matrix $F^t$. If $V=\Ro^n$ is the arithmetical space, then we assume the scalar product is chosen in a standard way $\langle v,u\rangle=\sum_{i=1}^{n}v_iu_i$.

The next lemma is the classical statement, to be found in any elementary book on linear algebra.

\begin{lem}\label{lemKerFFisKerF}
For a linear map $f\colon V\to W$, we have $\Ker f=\Ker f^*f$.
\end{lem}

\begin{proof}
If $fv=0$, then $f^*fv=0$, so the inclusion $\Ker f\subseteq\Ker f^*f$ is obvious. On the converse, if $f^*fv=0$, then
$0=\langle f^*fv,v\rangle=\langle fv,fv\rangle$, which implies $fv=0$ since the inner product is chosen non-degenerate.
\end{proof}

\begin{con}\label{conSelfAdjoint}
Recall that an operator $A\colon V\to V$ on euclidean space is called self-adjoint (symmetric matrix) if $A^*=A$. Every operator of the form $f^*f$ or $gg^*$ is obviously self-adjoint. The sum of self-adjoint operators is self-adjoint. The spectral theorem asserts that any self-adjoint operator can be diagonalized, i.e. there exists an orthogonal basis of $V$ in which $A$ is written as a diagonal matrix $\Lambda=\diag(\lambda_1,\ldots,\lambda_n)$. In other words, for any real symmetric matrix $A$ there exists a spectral decomposition $A=Q^t\Lambda Q$ where $Q$ is orthogonal, and $\Lambda$ is diagonal. The diagonal entries of $\Lambda$ are eigenvalues of $A$, and the columns of $Q$ are the eigenvectors of $A$. The problem of finding $Q$ and $\Lambda$ for a given symmetric matrix $A$ is called the symmetric diagonalization problem. It can be solved by approximate algorithms of numerical linear algebra with floating point, see e.g.~\cite{NumLinAlg}.
\end{con}

\begin{con}\label{conNonNegative}
An operator $A\colon V\to V$ is called nonnegative~\footnote{We use this term as a shorter version of a more standard ``positive semidefinite''.}, if $\langle Av,v\rangle\geq 0$. The proof of Lemma~\ref{lemKerFFisKerF} shows that any operator of the form $f^*f$ or $gg^*$ is nonnegative. The sum of nonnegative operators is easily seen to be nonnegative. If $A$ is self-adjoint, then $A$ is nonnegative if and only if all its eigenvalues $\lambda_i$ are nonnegative real numbers. It can be seen from the spectral theorem, that if $A$ is nonnegative self-adjoint, then
\begin{equation}\label{eqKernelIsQuadraticMinimum}
\Ker A=\{v\in V \mid \langle Av,v\rangle=0\}.
\end{equation}
An important observation follows from formula~\eqref{eqKernelIsQuadraticMinimum}. With any nonnegative symmetric matrix $A$, we can associate a function $Q_A\colon V\to\Ro$, $Q_A(x)=\langle Ax,x\rangle$, which is (1) quadratic, (2) nonnegative, (3) convex, (4) most importantly,
\begin{equation}\label{eqQuadraticMinimization}
Q_A(x)=0 \Leftrightarrow x\in\Ker A.
\end{equation}
Hence the problem of finding a solution to the equation $Ax=0$ can be faithfully reformulated as a problem of minimization of $Q_A$. 

Notice that, whenever $A=f^*f$, we get 
\begin{equation}\label{eqStupidObservation}
Q_A(x)=\langle x,f^*fx\rangle = \|fx\|^2.
\end{equation}
\end{con}



\begin{rem}\label{remKernelsIntersect}
If $A_1,A_2$ are two nonnegative self-adjoint operators, then so is $A_1+A_2$. Since $Q_{A_1+A_2}=Q_{A_1}+Q_{A_2}$, it easily follows from the considerations of Construction~\ref{conNonNegative}, that
\[
\Ker(A_1+A_2)=\Ker A_1\cap \Ker A_2,
\]
whenever $A_1,A_2$ are nonnegative self-adjoint.
\end{rem}

With all this been written, the next important lemma is not much more complicated than Lemma~\ref{lemKerFFisKerF}. It constitutes the basis of Hodge theory in the finite-dimensional case.

\begin{lem}\label{lemQuotientIsHarmonic}
For a sequence of two linear maps $U\stackrel{g}{\rightarrow} V\stackrel{f}{\rightarrow} W$ between euclidean spaces define an operator $L=f^*f+gg^*\colon V\to V$. Then we have an isomorphism
\[
\Ker L\cong \Ker f/(\Ker f\cap \im g).
\]
\end{lem}

\begin{proof}
Both $L_1=gg^*$ and $L_2=f^*f$ are nonnegative self-adjoint operators on $V$. We have
\begin{equation}\label{eqLinAlg}
\Ker L=\Ker(L_2+L_1)=\Ker L_2\cap \Ker L_1=\Ker f\cap \Ker g^*=\Ker f\cap (\im g)^\bot.
\end{equation}
Whenever $A,B$ are two subspaces of the euclidean space $V$, we have an isomorphism $A\cap B^{\bot}\cong A/(A\cap B)$, given by $a\mapsto [a]$. Therefore, the rightmost space in~\eqref{eqLinAlg} is naturally isomorphic to $\Ker f/(\Ker f\cap \im g)$.
\end{proof}

\subsection{Laplacians and combinatorial Hodge theory}\label{subsecMathHodge}

Recall the definition of cochain complex from subsection~\ref{subsecMathCochainRecap}.

\begin{defin}\label{definLaplaceGeneralCochain}
Consider a connective cochain complex $(C^*,d)$ in the category $\Ro\Vect$:
\[
0\to C^0\stackrel{d_0}{\to} C^1 \stackrel{d_1}{\to} C^2 \stackrel{d_2}{\to}\cdots
\]
Assume an inner product $\langle\cdot,\cdot\rangle$ is somehow chosen on every vector space $C_i$. Then we can define the sequence of nonnegative operators:
\[
\Delta_j=d_j^*d_j+d_{j-1}d_{j-1}^*\colon C^j\to C^j,
\]
called \emph{Laplace operators (or Laplacians)}. The vector subspace $\ca{H}_j=\Ker\Delta_j$ is called \emph{the harmonic subspace} of $C^j$, and its elements --- \emph{harmonic cochains}.
\end{defin}

We have a natural corollary of Lemma~\ref{lemQuotientIsHarmonic}

\begin{cor}\label{corHarmonicIsCohomology}
The $j$-th harmonic subspace of the euclidean cochain complex $(C^*,d)$ is (non-naturally) isomorphic to its $j$-th cohomology module of $(C^*,d)$:
\begin{equation}\label{eqHarmonicIsCohomology}
\ca{H}_j\cong H^j(C^*,d).
\end{equation}
\end{cor}

%\begin{rem}\label{remHarmonicDependOnEverything}
%The definition of harmonic subspaces is not canonical in the sense that it depends essentially on the choice of euclidean structure on vector spaces. It is also important to notice that harmonic cochains are defined inside a particular cochain complex. We have seen in the previous sections that the same cohomology groups can be computed from different cochain complexes. For example, cohomology of a sheaf on a cell poset can be computed from global sections of an injective resolution (and there exist infinitely many injective resolutions!), from Roos complex, or from cellular cochain complex. In each case, harmonic subspace will be defined in its own enveloping vector space of cochains.
%\end{rem}

\begin{con}\label{conInnerProductCellCpx}
Let $D$ be a cellular sheaf defined on a cell poset $\ca{X}$, see Definitions~\ref{definCellPoset} and~\ref{definCellularSheaf}. Assume an inner product is chosen on each stalk $D(\sigma)$. This gives a canonical inner product on the cellular cochain spaces $C^j_{CW}(\ca{X};D)=\bigoplus_{\dim\sigma=j}D(\sigma)$, where direct summands are assumed orthogonal.
\end{con}

\begin{defin}\label{definCellSheafLaplacian}
The Laplacian $\Delta_j$ of the cochain complex $(C^j_{CW}(\ca{X};D),d_{CW})$ is called the \emph{$j$-th cellular sheaf Laplacian}.
\end{defin}

The assumptions needed to define euclidean structure on the cellular cochain complex can be generalized in a natural way.

\begin{defin}\label{definEuclideanSheaf}
A \emph{euclidean sheaf} on a poset $S$ is a sheaf (diagram) $D$ together with a choice of inner product on each stalk $D(s)$, $s\in S$.
\end{defin}

\begin{con}\label{conEuclideanSheaf}
Let $D$ be a euclidean sheaf on $S$. Let $\Fu\colon\Shvs(S;\Ro\Vect)\to\Cochain(\Ro\Vect)$ be a concrete functor in the sense of Definition~\ref{definConcreteFunctor}. Then the cochain complex $(C^*,d)=\Fu(D)$ naturally attains the inner product. Indeed, each graded component $C^j$ is a direct sum of stalks, taken with some multiplicities:
\begin{equation}\label{eqJthComponentOfF}
C^j=\Fu(D)^j=\bigoplus_{s\in S}D(s)^{\oplus \mult_j(s,\Fu)}.
\end{equation}
The inner product on $C^j$ is defined by setting all direct summands pairwise orthogonal, while the inner product in each summand is summand $D(s)$ is given by assumption.

Since we have a euclidean structure on $(C^*,d)$, we can define \emph{$\Fu$-based Laplacians} in the natural way:
\begin{equation}\label{eqFuValuedLaplacians}
\Delta_j^{\Fu}\colon C^j\to C^j,\qquad \Delta_j^{\Fu}=d^*d+dd^*.
\end{equation}
\end{con}

Recall that a concrete functor $\Fu\colon\Shvs(S;\Ro\Vect)\to\Cochain(\Ro\Vect)$ is said to honestly compute sheaf cohomology, if cohomology of $\Fu(D)$ is isomorphic to the sheaf cohomology, see Definition~\ref{definHonestlyComputes}. This definition and Corollary~\ref{corHarmonicIsCohomology} easily imply the next statement.

\begin{prop}\label{propFuBasedLaplacianKernel}
Let $\Fu\colon\Shvs(S;\Ro\Vect)\to\Cochain(\Ro\Vect)$ be a concrete functor that honestly computes sheaf cohomology, and let $D$ be a euclidean sheaf on a poset $S$. Then
\[
\Ker\Delta_j^{\Fu} \cong H^j(S;D).
\]
\end{prop}

\begin{rem}\label{remFuLaplacianToOptimization}
It follows from Construction~\ref{conNonNegative}, that the linear subspace $\Ker\Delta_j^{\Fu}$ coincides with the zero-set of the nonnegative quadratic function $Q_{\Delta_j^{\Fu}}$ defined on the space $\Fu(D)^j$, described in~\eqref{eqJthComponentOfF}. Harmonic $j$-cochains can be found by minimizing this function. In particular, the space of global sections $\Gamma(S;D)\cong H^0(\Fu(D))$ can be found by minimizing the function
\[
Q_{\Delta_0^{\Fu}}(x)=\|d_0x\|^2,
\]
according to~\eqref{eqStupidObservation}. Here $d_0\colon C^0(S;D)\to C^1(S;D)$ is the leftmost differential in the cochain complex $\Fu(D)$.
\end{rem}

\subsection{Normalization}\label{subsecMathNormalize}

Computing cohomology reducing to finding $\Ker L_j$ which can be done by diagonalizing the matrix $L$. However, when it comes to the study of Laplacians, the nonnegative symmetric matrices $L_j$ are valuable and informative in their own right. Sometimes, the matrix $L_j$ is replaced with another matrix $L_j'$, which is better behaved from computational perspective, but preserves the main properties of $L_j$, namely, (1) $\Ker L_j'=\Ker L_j$, (2) $L_j'$ is nonnegative symmetric. Such matrix $L_j'$ are usually called \emph{normalizations} of $L_j$. This notion originated in spectral graph theory~\cite{Chung}, where many properties of general graphs are more naturally formulated in terms of eigenvalues of $L_j'$ rather than eigenvalues of $L_j$. It should be noticed however, that in the case of general sheaves, a normalization can be performed in a variety of ways, which we review below in the case of cellular sheaves.

\begin{con}\label{conNormalizationOfLaplacian}
Consider a cellular sheaf $D$ on a cell poset $\ca{X}$, and let $\sigma\in\ca{X}$ be a cell. Let $k_\sigma$ denote the dimension of a stalk $D(\sigma)$. Assume that an orthogonal basis $v_{\sigma,1},\ldots,v_{\sigma,k_\sigma}$ is chosen in each stalk $D(\sigma)$. Then the collection of vectors $\bigsqcup_{\sigma\colon \dim\sigma=j}\{v_{\sigma,1},\ldots,v_{\sigma,k_\sigma}\}$ becomes an orthogonal basis of $C^j_{CW}(\ca{X};D)$, for each $j\geq 0$. In this basis, the Laplacian $\Delta_j$ is written as a symmetric nonnegatively defined matrix, which we denote by $L_j$ in order to distinguish from the corresponding operator. Notice, that $L_j$ has a natural block structure; given two cells $\sigma,\tau$ of dimension $j$, we denote by $(L_j)_{\sigma,\tau}$ the block of $L_j$ formed by the intersection of columns indexed by $(\sigma,1),\ldots,(\sigma,k_\sigma)$ and rows indexed by $(\tau,1),\ldots,(\tau,k_\tau)$ (i.e. the matrix corresponding to $\Delta_j$ restricted to $D(\sigma)$ in the domain, and $D(\tau)$ in the target space).
\begin{enumerate}
  \item Naive coordinate-wise normalization. Consider the diagonal matrix $K$ filled with the values
  \[
  K_{(\sigma,i),(\sigma,i)}=\begin{cases}
                              (L_j)_{(\sigma,i),(\sigma,i)}^{-1}, & \mbox{if } (L_j)_{(\sigma,i),(\sigma,i)}\neq 0 \\
                              1, & \mbox{otherwise},
                            \end{cases}
  \]
  --- i.e. the pseudoinverse of the diagonal part of $L_j$. Then the matrix
  \begin{equation}\label{eqWeaklyNormalized}
    L_j'=K^{1/2}L_jK^{1/2}
  \end{equation}
  is symmetric nonnegative, has the same kernel as $L_j$, but the diagonal of $L_j'$ is only filled with $0$'s and $1$'s. We call $L_j'$ the weakly normalized sheaf Laplacian.
  \item Stalk-wise normalization. Let $\sigma$ be a cell of $\ca{X}$, and $A_\sigma=(L_j)_{\sigma,\sigma}$ be the corresponding diagonal block of $L_j$. Since $\Delta_j$ is a nonnegative operator, so is its restriction to the subspace $D(\sigma)$, therefore the symmetric matrix $A_\sigma$ is a matrix of a nonnegative operator. By the orthogonal diagonalization, there exists an orthogonal matrix $Q_\sigma$ such that $Q_\sigma^tA_\sigma Q_\sigma$ is a diagonal matrix $\diag(\lambda_{(\sigma,1)},\ldots,\lambda_{(\sigma,k_\sigma)})$ with nonnegative entries. Again, consider the pseudoinverse $N_\sigma=\diag(n_{(\sigma,1)},\ldots, n_{(\sigma,k_\sigma)})$, where $n_{(\sigma,i)}=1/\lambda_{(\sigma,i)}^{-1}$ if the denominator is nonzero, and $1$ otherwise. Let us form a block diagonal orthogonal matrix $Q$ with blocks $Q_\sigma$ and diagonal matrix $N$ with blocks $N_\sigma$. Then the matrix
  \begin{equation}\label{eqStronglyNormalized}
  L''_j=N^{1/2}Q^tLQ
  \end{equation}
  is symmetric nonnegative, has the same kernel as $L_j$, but all the diagonal blocks $(L''_j)_{\sigma,\sigma}$ are diagonal matrices of the form $\diag(1,\ldots,1,0,\ldots,0)$. We call $L''_j$ the strongly normalized Laplacian.
  \item We may take a matrix $M$ which is pseudoinverse of $L_j$ itself. Then $L_j'''=K^{1/2}L_jK^{1/2}$ is a matrix of the form $\diag(1,\ldots,1,0,\ldots,0)$. The computation of $M$ is equivalent to diagonalization of $L_j$, so this case is usually not considered a normalization, despite the fact it is completely analogous to the previous two examples. However, we find it amusing that the study of sheaf Laplacians has explicit hierarchical structure.
\end{enumerate}
\end{con}

\subsection{Heat diffusion}\label{subsecMathDiffusion}

Assume a nonnegative self-adjoint operator $A\colon V\to V$ is defined on a euclidean vector space $V$. According to Construction~\ref{conNonNegative} the question of finding $\Ker A$ is equivalent to minimizing the corresponding quadratic function $Q_A\colon V\to\Rg$. The latter can be done e.g. by means of gradient descent.

\begin{con}\label{conGradientHeat}
Consider a function $Q_A(x)=\langle Ax,x\rangle$ defined on a finite-dimensional euclidean vector space. Its gradient equals
\begin{equation}\label{eqGradientOfQuadratic}
\nabla Q_A(x_0)=(A+A^*)x_0.
\end{equation}
In the following, it will be assumed that $A$ is self-adjoint, so that $\nabla Q_A=2A$. Henceforth, the continuous gradient descent flow of $Q_A$ with a parameter $\eta>0$ takes the form
\begin{equation}\label{eqGradFlowContinuous}
\dot{x}=-2\eta Ax,\qquad x(0)=x_0,
\end{equation}
this is a linear system of differential equations. Its solution is $x(t)=\exp(-2\eta At)x_0$. If $A$ is nonnegative, then there exists $x_{+\infty}=\lim_{t\to+\infty} x(t)$ and it lies in $\Ker A$.

Similarly, the discrete gradient descent for $Q_A(x)$ has the form
\begin{equation}\label{eqGradFlowDiscrete}
x_{k+1}=x_k-2\eta Ax_k\qquad k=0,1,2,\ldots.
\end{equation}
The solution of this cascade is $x_k=(1-2\eta A)^kx_0$. Again, for nonnegative $A$, there exists $x_{+\infty}=\lim_{k\to+\infty} x_k$ and this vector lies in $\Ker A$.
\end{con}

Estimation of the convergence rate of either continuous or discrete gradient flow is a straightforward exercise, but it emphasizes the role of eigenvalues of $A$.

\begin{con}\label{conSolutionDiagonalized}
If $A=Q^t\Lambda Q$ is a spectral decomposition with the diagonal matrix $\Lambda=\diag(\lambda_1,\ldots,\lambda_n)$, as in Construction~\ref{conSelfAdjoint}, then the solution to equation~\eqref{eqGradFlowContinuous} has the form
\[
x(t)=\exp(-2\eta At)x_0=Q^t\diag(e^{-2\eta\lambda_1t},\ldots, e^{-2\eta\lambda_nt})Qx_0.
\]
Assume that $A$ is nonnegative, i.e. $\lambda_i\geq 0$ for all $i$. As $t\to+\infty$, the exponent $e^{-2\eta \lambda_it}$ is constant whenever $\lambda_i=0$, and decreases to $0$ when $\lambda_i>0$. The exponent with the least nonzero $\lambda_i$ has the slowest convergence rate.

Similarly, the solution to a cascase~\eqref{eqGradFlowDiscrete} has the form
\[
x_k=Q^t\diag((1-2\eta\lambda_1)^k,\ldots, (1-2\eta\lambda_n)^k)Qx_0,
\]
and the main term of the asymptotics is given by the one with the smallest positive $\lambda_i$. These considerations prove the following statement.
\end{con}

\begin{prop}\label{propConvergenceRate}
Let $A\neq 0$ be nonnegative self-adjoint operator, and $\lambda_{\min}>0$ be its least positive eigenvalue. Then the convergence rate of the equation~\eqref{eqGradFlowContinuous} is asymptotically equivalent to $e^{-2\eta\lambda_{\min}t}$, for a generic initial value $x_0$. The rate of convergence of the cascade~\eqref{eqGradFlowDiscrete} is asymptotically equivalent to $(1-2\eta\lambda_{\min})^k$.
\end{prop}

It remains to apply these standard arguments to sheaf Laplacians or their normalized versions. Let $\Fu$ be a concrete functor which honestly computes cohomology, let $\Delta_j^{\Fu}$ be the $\Fu$-based Laplacian of a euclidean sheaf $D$ defined in Construction~\ref{conEuclideanSheaf}, let $L_j$ be a matrix of this Laplacian (in some orthogonal basis), and $L_j'$ --- its normalized version.

\begin{defin}\label{definHeatDiffusion}
The function $Q_j(x)=\langle L_j'x,x\rangle$ is called \emph{the ($\Fu$-based, $j$-th order, normalized, Dirichlet) energy} of the euclidean sheaf $D$. Its gradient descent flow with parameter $\eta$, --- either continuous
\[
\dot{x}=-2\eta L_j'(x),\qquad x(0)=x_0,
\]
or discrete
\[
x_{k+1}=(1-2\eta L_j')x_k,\qquad k=0,1,2,\ldots
\]
--- is called a \emph{heat diffusion} on a sheaf $D$.
\end{defin}

By definition, the heat diffusion is the process that minimizes the sheaf's energy, which is physically meaningful. Proposition~\ref{propConvergenceRate} implies the following.

\begin{cor}\label{corSpeedOfHeatDiffusion}
The convergence rate of the heat diffusion on a sheaf $D$ is determined asymptotically by the least nonzero eigenvalue $\lambda_{\min}$ of the corresponding (normalized) sheaf Laplacian $L_j'$.
\end{cor}

\begin{rem}\label{remOtherEigenvalues}
The zero eigenspace of $L_j'$ coincides with $\Ker\Delta_j^{\Fu}$ (see Construction~\ref{conNormalizationOfLaplacian}) which is isomorphic to $H^j(S;D)$ (see Proposition~\ref{propFuBasedLaplacianKernel}). Therefore, the multiplicity of $0$-th eigenvalue of $L_j'$ equals the $j$-th Betti number of a sheaf $D$, i.e. $\dim H^j(S;D)$. The smallest positive eigenvalue $\lambda_{\min}$ measures ``the worst'' asymptotical term of the heat diffusion process. Other eigenvalues of $L_j'$ contain information on lower terms of the asymptotics for the heat diffusion process. It is believed that these eigenvalues reflect some geometrical properties of the underlying poset $S$, which is supported by a variety of results in spectral graph theory~\cite{Chung}. It should be noticed however, that there are no mathematical results relating eigenvalues of higher order Laplacians with higher dimensional discrete structures, such as simplicial complexes and cell posets.
\end{rem}

%\begin{rem}\label{remNotOnlyDiffusion}
%A common applied approach to study the geometry of $S$ (e.g. a graph), is to define a sheaf $D$ on $S$, run heat diffusion on $D$ several times with random initial heat distributions, and take the intermediate results of such computation as a feature-vector, representing the geometry. It should be noticed that this approach seems very non-conceptual and data inefficient. The only features which are actually independent of the random choice of the initial distributions in this computation scheme, are the eigenvalues and eigenvectors of the normalized Laplacian. However, these mathematical invariants can be found by algorithms of applied linear algebra~\cite{NumLinAlg}, which are more efficient than gradient descent.
%\end{rem}

\subsection{Examples of diffusion}\label{subsecMathDiffuseExamples}

We end up with two main examples which demonstrate how the terminology and results introduced in the previous subsections motivate particular constructions used in topological deep learning. The definitions of sheaf diffusion and message passing were given in the main part of the text, see subsection~\ref{subsecSheafLearning}. Here we use cellular cochain complex and Roos cochain complex described in subsections~\ref{subsecMathCohomologyCellular} and~\ref{subsecMathCohomologySimplicial} respectively.

\textbf{Diffusion in cellular sheaves.} Let $S$ be a cell poset, e.g. a poset given by a simple graph, as in Example~\ref{exGraphToPoset}. In this case, cellular cochain complex $\Fu_{CW}(D)=(C^*_{CW}(S;D),d_{CW})$ is defined and honestly computes cohomology.

\begin{con}\label{conCellComplexDiffusion}
By Theorem~\ref{thmCWcohomologyIsAGcohomology}, we have
\[
\Gamma(S;D)\cong H^0(S;D)\cong H^0(C^*_{CW}(S;D),d_{CW})=\Ker d_{CW}\colon C^0_{CW}(S;D)\to C^1_{CW}(S;D).
\]
This formula shows that global sections of $D$ can be defined as a subspace of $C^0_{CW}(S;D)=\bigoplus_{\rk s=0}D(s)$. For any element $s\in S$ of rank 1 (that is an edge), the poset $S_{<s}$ is by definition homeomorphic to a $0$-dimensional sphere, i.e. a pair of points. In other words, in a cell poset, each edge $s$ has two endpoints\footnote{This should not surprise anyone who ever seen an interval.}, vertices $s'$ and $s''$. Coherence relations reduce to the linear system $d_{CW}x=0$ given by
\[
f_{s's}(x_{s'})-f_{s''s}(x_{s''})=0 \mbox{ for each edge }s\mbox{ with endpoints }s',s'',
\] 
where $f_{ts}=D(t<s)$ denote the structure maps of the diagram $D$. The two signs in the above formula alternate, because only this choice of signs (incidence numbers) guarantees that $\Fu_{CW}$ honestly computes cohomology, in particular in degree 0. A global section can be found by minimizing the sheaf energy function $Q_{\Delta_0^{\Fu_{CW}}}$ corresponding to the functor $\Fu_{CW}$; we denote it by $Q$ for simplicity:
\[
Q\colon C^0_{CW}(S;D)\to \Ro,\qquad Q(x)=\sum_{s\colon \rk s=1}\|f_{s's}(x_{s'})-f_{s''s}(x_{s''})\|^2.
\]
See Remark~\ref{remFuLaplacianToOptimization} for the derivation of this expression. 

Notice that, even in the case $\dim |S|>1$, i.e. if a cell poset $S$ has elements other than vertices and edges, the global sections of a sheaf are defined through vertices and edges alone, see Remark~\ref{remCompatibilityReducesToCells}. The variables used in the system live over the vertices of $S$, while the edges are needed to define relations, and to perform linear message passing. 
\end{con}

\begin{rem}\label{remDiffusionOnMorse}
The situation with Morse cell posets described in subsection~\ref{subsecMathOneShot} is completely similar, but the functor $\Fu_{M}$ is used instead of $\Fu_{CW}$. This gives a definition of sheaf diffusion on non-graphical structures similar to the one shown on Fig.~\ref{figElementaryMorse}.
\end{rem}

\textbf{Diffusion in binary relations.} Let $R\subseteq A\times B$ be a binary relation (or a bipartite graph), which can be considered as a poset $S_R$ on $A\sqcup B$, see Example~\ref{exBinRelToPoset}. In particular, $R$ may encode a hypergraph, where $A$ is the set of vertices, $B$ is the set of hyperedges, and $R$ is the relation of inclusion. 

\begin{con}\label{conHypergraphDiffusion}
A diagram $D$ on $S_R$ is, by the general definition, an assignment, to each vertex $a\in A$, a real vector space $D(a)$, to each vertex $b$, a space $D(b)$, and, whenever $aRb$, a linear map $f_{ab}\colon D(a)\to D(b)$. No compositionality restrictions are posed on these maps since $\dim |S_R|=1$.

The graded components of the Roos complex $C^*_{Roos}(S;D)$ have the form:
\[
C^0_{Roos}(S_R;D)=\bigoplus_{a\in A}D(a)\oplus\bigoplus_{b\in B}D(b),\quad C^1_{Roos}(S;D)=\bigoplus_{ab\in R}D(b),
\]
while the differential $d_{Roos}\colon C^0_{Roos}(S_R;D)\to C^1_{Roos}(S_R;D)$ is defined by
\[
d_{Roos}(x_a)=-f_{ab}(x_a) \mbox{ for }x_a\in D(a);\quad d_{Roos}(x_b)=x_b \mbox{ for }x_b\in D(b).
\]
All components $C^j_{Roos}(S;D)$ with $j>1$ vanish. The global sections $\Gamma(S_R;D)\cong \Ker d_{Roos}$ are the vectors $\bigoplus_{a\in A}x_a\oplus \bigoplus_{b\in B}x_b\in C^0_{Roos}(S_R;D)$ such that $f_{ab}(x_a)=x_b$ for any $aRb$. These are simply the coherence relations for the diagram $D$: this is not surprising due to Remark~\ref{remRoos0globalSec}. 

The theory developed in subsection~\ref{subsecMathHodge} claims that $\Ker d_{Roos}=\Ker \Delta$ where $\Delta=d_{Roos}^*d_{Roos}$. The heat diffusion is given by minimization of Dirichlet's energy $Q(x)=\langle x,\Delta x\rangle$ given by
\begin{equation}\label{eqRelationDiriclet}
Q(x)=\sum_{ab\in R}\|x_b-f_{ab}(x_a)\|^2,
\end{equation}
according to Remark~\ref{remFuLaplacianToOptimization}. 
\end{con}

\begin{ex}\label{exSingleToManyHypergraph}
Assume that $A=\{a_1,\ldots,a_m\}$, $B=\{b\}$ is a singleton in Construction~\ref{conHypergraphDiffusion}, and the values $x_a\in D(a)$ are forced to stay constant during the heat diffusion. Then minimization of the energy $Q$ given by~\eqref{eqRelationDiriclet} makes $x_b$ tend to the barycenter of $f(x_{a_1}),\ldots,f(x_{a_m})$, since the barycenter minimizes the sum of squared distances to a fixed finite set.
\end{ex}

\begin{rem}\label{remRelationGrouping}
In general, when $B=\{b_1,\ldots,b_k\}$, each letter $b_j$ can be treated as a hyperedge $A_j=\{a\mid aRb_j\}\subset A$. The minimization of $Q$ makes the vectors $\{f_{ab_j}(x_a)\mid a\in A_j\}$ group around their barycenter $x_{b_j}$, for any hyperedge $b_j\in B$. The barycenters $x_b$ can be removed from consideration: instead of minimizing $Q$ given by~\eqref{eqRelationDiriclet}, one can instead minimize the function
\begin{equation}\label{eqHypergraphDirichlet}
Q(x)=\sum_{b_j\in B} \frac{1}{|A_j|}\sum_{a',a''\in A_j}\|f_{a'b_j}(x_{a'})-f_{a''b_j}(x_{a''})\|^2.
\end{equation}
The latter process makes the vectors $f_{ab_j}(x_a)$ group together for any hyperedge $A_j$. The above formula appears in the paper~\cite{duta2024sheaf} introducing hypergraph sheaf neural networks. Essentially, this is the same function as~\eqref{eqRelationDiriclet} (at least if $x_b$ is set equal to the barycenter of the corresponding vectors). Indeed, given $l$ vectors in $\Ro^d$, the sum of their pairwise squared distances equals $l$ times the sum of squared distances to their barycenter. However, the formula~\eqref{eqRelationDiriclet} is a bit more optimal since it has fewer summands. 
\end{rem}

\begin{rem}\label{remInternalStructureOfHyperedges}
The above construction utilizes Roos complex: it is permutation invariant in the sense that the formulas for the energy function $Q(x)$ (either~\eqref{eqRelationDiriclet} or~\eqref{eqHypergraphDirichlet}) do not depend on the order of variables $a\in A_j$ in each hyperedge $A_j$. It is possible to model the heat diffusion by using smaller number of summands. E.g. in Example~\ref{exSingleToManyHypergraph} we could minimize the function $Q(x)=\sum_{i=1}^{m-1}\|f_{a_{i+1}b}(x_{a_{i+1}})-f_{a_ib}(x_{a_i})\|^2$, the energy defined over a minimal functor $\Fu_{\min}$. However, this expression depends on the linear order of $a_i$'s, hence not permutation invariant anymore. Application of such formulas in constructions of neural architectures may be reasonable if the problem's formulation subsumes some internal structure on hyperedges of a hypergraph.
\end{rem}

\textbf{Derived perspective on binary relations.} Treating hypergraphs as binary relations (or bipartite graphs) has a conceptual advantage: the symmetric role of vertices $A$ and hyperedges $B$ becomes more transparent. Given a relation $R\subset A\times B$, we can swap $A$ and $B$ and get a relation $R^{\op}\subset B\times A$, such that $bR^{\op}a\Leftrightarrow aRb$. The corresponding poset $S_{R^{\op}}$ coincides with $S_R^{\op}$ obtained from $S_R$ by reversing the order. As obvious from the definition of geometrical realization, we have $|S_R^{\op}|=|S_R|$. A sheaf theoretical counterpart of this claim should be mentioned.

\begin{rem}\label{remDerivedOpposite}
It was proved in~\cite[Cor.4.18]{LADKANI2008435} that the derived categories $\Der^b(S_R)$ and $\Der^b(S_{R^{\op}})$ are equivalent, see Remark~\ref{remDerivedCatOnPoset}. This can be seen as an instance of a more general fact: Coxeter functors (see~\cite{BernGelfPonomarev}) provide the derived equivalence of the categories of representations~\cite[Thm.3.19]{Longbottom}. In the absence of compositionality restrictions, the category of quiver representations coincides with the category of sheaves on a poset which implies the claim.

The stated claim has a straightforward practical implication. Whenever a hypergraph is studied by homological methods, one can swap the roles of its elements: consider hyperedges as vertices, and vertices as hyperedges. The equivalence of derived categories guarantees that no essential information is lost. Such swap makes sense if calculations with $R^{\op}$ are less costly than those with $R$.    
\end{rem}

\begin{con}\label{conDowker}
The construction of sheaves on a hypergraph is utilizing the poset $S_R$ which is 1-dimensional; therefore all cohomology in degrees $>1$ of all sheaves on $S_R$ vanish, see Corollary~\ref{corVanishing}. However, in algebraic topology there is another way to transform a binary relation into a higher-order structure, the Dowker complex. A binary relation (a hypergraph) can be turned into a simplicial complex by treating hyperedges as simplices and adding their proper subsets. Dowker complex better resembles the intuitions behind applied papers, since hyperedges with $>2$ vertices actually become higher-dimensional structures and there may be nontrivial higher-order cohomology. 

Swapping the arguments of a binary relation leads to a Dowker duality. The simplicial complexes constructed from $R$ and $R^{\op}$ are homotopically equivalent; an elegant functorial proof of this claim can be found in~\cite{BrunSalbu}. Therefore, the homotopy type of a binary relation is well defined and symmetric in the arguments of the relation. It seems an interesting research problem: to define sheaf theory on binary relations in the way that remembers higher order structure of Dowker complex. 
\end{con}

\textbf{Diffusion in general.} For a general poset $S$, the somehow canonical way to define Laplacian and diffusion on sheaves on $S$ is by using Roos complex $\Fu_{Roos}$, since this construction is universal. This idea of combining Laplacians with general posets was proposed in~\cite{HypergraphsSimpSets} motivated by the particular case of binary relations and hypergraphs. 

Another way to introduce Laplacians and diffusion over arbitrary posets is by using a minimal functor $\Fu_{\min}$. Let $S_{low}$ denote the subset of minimal elements of $S$. It can be seen from the inductive construction of $\Fu_{\min}$ presented in subsection~\ref{subsecMathMinimalComputations}, that $C^0_{\min}(S;D)=\bigoplus_{s\in S_{low}}D(s)$, and the number of summands of $C^1_{\min}(S;D)$ equals, by Definition~\ref{definMultiplicityOfStalk}, to the sum of multiplicities $\sum_{s\in S} \mult_1(s,\Fu_{\min})$, i.e. the number of generators of degree $1$ output by Algorithm~\ref{algIncMatrixMain}. The formula for the energy function $Q_{\Delta_0^{\Fu_{\min}}}$ is determined by this generating set.

We finish with the general remark addressed to the specialists in deep learning.

\begin{rem}
In general, the mathematical pipeline described in Appendix sections~\ref{secMathCohomology} and~\ref{secMathLaplacians}, consists of the following steps 
\begin{enumerate}
  \item choice of cochain complex which honestly computes cohomology (or just global sections),
  \item choice of euclidean structure on a sheaf,
  \item choice of normalization for a Laplacian.
\end{enumerate}
Each of this steps subsumes picking real parameters~\footnote{Sometimes some discrete optimization is involved, like choosing the optimal dimensions of stalks, but in all cases there is a bunch of real numbers to optimize.} hence can be learned end-to-end by gradient descent, or constructed by hand based on the specifics of the problem at hand. 
\end{rem}

\begin{ex}
The mechanism of sheaf attention networks mentioned in Section~\ref{secReviewShvsML} is an example of learning the euclidean structure. The standard euclidean structure on stalks of edges is scaled by the learnable attention weights. 
\end{ex}

\begin{ex}
If hyperedges of a hypergraph have an unknown internal structure, then it can be captured by the learnable cochain complex as outlined in Remark~\ref{remInternalStructureOfHyperedges}. The only mathematical restriction on the complex is that it should honestly compute cohomology, otherwise the whole theory doesn't make much sense.
\end{ex}

\section{Connection sheaves}\label{secMathConnection}

In this section we briefly recall the basic notions related to connection sheaves. Here, similar to Appendices~\ref{secMathCohomology} and~\ref{secMathLaplacians}, $\Vv$ is an abelian category, with $\Vv=\Ro\Vect$ being the main example. For a more detailed overview of the mathematical results related to posets we refer to~\cite{Barmak2012GcoloringsOP}.

\begin{defin}\label{definConnectionSheaf}
A $\Vv$-valued diagram $D$ on a poset $S$ (a sheaf $\ca{D}$ on an Alexandrov space $X_S$) is called a \emph{connection sheaf} if all structure maps $D(s_1\leq s_2)$ are isomorphisms in $\Vv$. More generally, let $G$ be a subgroup of the monoid $\Hom_{\Vv}(V,V)$ for some $V\in\Vv$. Then a \emph{$G$-connection sheaf} on $S$ is a diagram $D$ in which all structure maps $D(s_1\leq s_2)$ belong to $G$.
\end{defin}

In classical algebraic topology connection sheaves are known as local systems; they were introduced by Steenrod in~\cite{Steenrod1943HomologyWL}.

\begin{rem}\label{remDifGeomConnections}
Connection sheaf on a finite poset is a suitable discretization for the notion of a bundle over a manifold, such as tangent bundle of a smooth manifold. In a bundle, we have a real vector space of the same dimension attached to each point of a manifold. In order to take derivatives of tensor fields, the spaces attached to infinitely closed points are related by a certain isomorphism; this is formalized in the notion of connection from differential geometry. This basic and extremely powerful object of differential geometry leads to the notions of parallel transport and Riemann curvature tensor.

A straightforward way to discretize this setting is to take a poset $S$, assign, to each element $s\in S$, a real vector space $D(s)$ of the same dimension, and, whenever $s_1<s_2$ (which is treated as closeness of points) assign an isomorphism between $D(s_1)$ and $D(s_2)$. This seemingly leads to Definition~\ref{definConnectionSheaf} of a sheaf. However, the analogy between connection sheaves on posets and differential connections on manifolds may be a bit misleading if a poset is more complicated than just a graph. In sheaf theory compositionality $D(s_1<s_2);D(s_2<s_3)=D(s_1<s_3)$ is required. In differential geometry, the discrepancy between ``various ways to move around a point'' measures the curvature and may be nontrivial. In this sense, connection sheaves may be seen as discrete models of flat manifolds. General quiver representations with isomorphic maps can model arbitrary manifolds.
\end{rem}

Assume that $S$ is connected (meaning that the geometrical realization $|S|$ is a connected topological space). This means that for any two points $s',s''\in S$ there is a path
\[
p=(s'\lessgtr s_1\lessgtr s_2\lessgtr \cdots \lessgtr s_n\lessgtr s'')
\]
where $a\lessgtr b$ denotes either $a\leq b$ or $a\geq b$. The reversed path $p^{-1}$ from $s''$ to $s'$ is defined in a straightforward manner.

\begin{con}\label{conParallelTransport}
Given a connection sheaf $D$ on $S$ and a path $p$ between $s'$ and $s''$ as above, we can define \emph{the parallel transport} operator
\[
T_p\colon D(s')\to D(s''),\qquad T_p=D(s'\lessgtr s_1);D(s'\lessgtr s_1);\cdots;D(s_n\lessgtr s''),
\]
where we set by definition
\[
D(a\lessgtr b)=\begin{cases}
                 D(a\leq b), & \mbox{if } a\lessgtr b \mbox{ reads as }a\leq b\\
                 D(b\leq a)^{-1}, & \mbox{if } a\lessgtr b \mbox{ reads as }a\geq b.
               \end{cases}
\]
We obviously have $T_{p^{-1}}=T_p^{-1}$. The definition of a sheaf~\ref{definSheafDiagramMainText} directly implies $T_{p'}=T_{p''}$ if a path $p''$ is obtained from $p'$ by deletion of fragment $(s_i=s_{i+1})$ or substituting a fragment $(s_{i-1}\leq s_{i}\leq s_{i+1})$ with $(s_{i-1}\leq s_{i+1})$ or similar operation for the reversed relation $\geq$, --- we call such operations triangular reductions. Therefore, it can be assumed without loss of generality that a parallel transport is determined along paths composed of alternating strict inequalities. This argument proves the following.
\end{con}

\begin{lem}\label{lemHomotopicPaths}
If $p',p''$ are two homotopic paths from $s'$ to $s''$ in $S$, then $T_{p'}=T_{p''}$.
\end{lem}

\begin{proof}
Construction~\ref{conParallelTransport} shows that two paths differing by a triangle $(s_0<s_1<s_2)$ in $|\ord S|=|S|$ give the same parallel transport operators. By the cellular approximation theorem, every homotopy between $s'$ to $s''$ lies in a 2-skeleton of $|S|$. Since all 2-cells of $|S|$ have the form $(s_0<s_1<s_2)$ a homotopy is a sequence of triangular reductions or their inverses, hence preserves the parallel transport operator.
\end{proof}

\begin{rem}\label{remMonodromy}
If the initial and end-points coincide, a path from $s$ to itself is called a loop, and its homotopy class is, by definition, the element of the fundamental group $\pi_1(|S|,s)$. Lemma~\ref{lemHomotopicPaths} implies that the parallel transport map $T_p$ provides a well-defined homomorphism from $\pi_1(|S|,s)$ to $\Iso(D(s))$, which is called a \emph{monodromy} of a connection sheaf. In case $D$ is $G$-connection sheaf, the monodromy is valued in the subgroup $G$. A connection sheaf is called \emph{flat} if its monodromy is a trivial representation. This means that traversing any loop is identical on $D(s)$.
\end{rem}

More generally, let $\ConDiag(S,\Vv)$ denote the full subcategory of connection sheaves in $\Diag(S,\Vv)$. The next proposition and its proof are subsumed by the previous remark, and seem to be folklore. 

\begin{prop}\label{propFundGrpRepresentations}
For a connected poset $S$, the category $\ConDiag(S,\Vv)$ is equivalent to the category $\Repr(\pi_1(|S|),\Vv)$ of $\Vv$-valued representations of the fundamental group $\pi_1(|S|)$.
\end{prop}

\begin{rem}
By McCord's theorem we have $\pi_1(|S|)\cong \pi_1(X_S)$, see Remark~\ref{remMcCord}. The claim of Proposition~\ref{propFundGrpRepresentations} can thus be formulated as an equivalence of the categories $\ConDiag(S,\Vv)$ and $\Repr(\pi_1(X_S),\Vv)$.
\end{rem}

In view of Remark~\ref{remDifGeomConnections} on relation between connection sheaves and connections in differential geometry, this proposition may be considered as a discrete version of Riemann--Hilbert correspondence (rather naive though). To prove it, we choose a base point $s_0\in S$ and a spanning tree $G$ of the 1-skeleton $|(\ord S)_1|$. For any point $s\in S$, there is a unique path $p_s$ from $s_0$ to $s$ lying in $G$. These data will be assumed fixed. The following two constructions emulate the procedure of collapsing/decolapsing a spanning skeleton into one point, and prove Proposition~\ref{propFundGrpRepresentations}.

\begin{con}\label{conConSheafToRepr}
Given a connection sheaf $D$ on $S$, we construct a representation of $\pi_1(|S|,s_0)$ on the space $D(s_0)$. For an element $g=[p]\in\pi_1(|S|,s_0)$ represented by a loop $p$ from $s_0$ to itself, set the action of $g$ on $D(s_0)$ defined by $T_p$. This is well-defined on homotopy classes by Lemma~\ref{lemHomotopicPaths}. This is a group homomorphism from $\pi_1(|S|,s_0)$ to $\End(D(s_0))$ by the definition of the parallel transport.
\end{con}

\begin{con}\label{conReprToConSheaf}
Given a fundamental group representation $R\colon \pi_1(|S|,s_0)\to \End(V)$, we construct a connection sheaf $\bar{V}$ on $S$. Set $\bar{V}(s)=V$ for any $s$. Then for $s_1<s_2$ set
\[
\bar{V}(s_1<s_2)=R([p_{s_1};(s_1<s_2);p_{s_2}^{-1}])\colon V=\bar{V}(s_1)\to V=\bar{V}(s_2),
\]
where $[p_{s_1};(s_1<s_2);p_{s_2}]\in \pi_1(|S|,s_0)$ is the homotopy class of the loop obtained by concatenation of the chosen path $p_{s_1}$ from $s_0$ to $s_1$, the shortcut from $s_1$ to $s_2$, and the reversed chosen path $p_{s_2}^{-1}$ from $s_2$ to $s_0$. It is a simple exercise to prove that $\bar{V}$ is a well-defined connection sheaf.
\end{con}

The proof that Constructions~\ref{conConSheafToRepr} and~\ref{conReprToConSheaf} are inverses to each other up to natural isomorphism (given the tree $G$) is rather straightforward and left as an exercise. Of course, a more conceptual and mathematically correct way to state Proposition~\ref{propFundGrpRepresentations} is via the fundamental groupoid of $|S|$.

\begin{rem}\label{remConnectionCohomology}
It should be noted that the category $\Repr(\pi_1(|S|),\Vv)$ (and hence $\ConDiag(S,\Vv)$ according to Proposition~\ref{propFundGrpRepresentations}) is an abelian category itself, and it has enough injectives if so does $\Vv$. However, the subcategory $\ConDiag(S,\Vv)$ does not inherit all its structure from $\Diag(S;\Vv)$. Injective sheaves on $S$ may fail to be connection sheaves, and injective connection sheaves may fail to be injective sheaves of $\Diag(S,\Vv)$. In particular, the notion of cohomology in $\Diag(S,\Vv)$ and $\ConDiag(S,\Vv)$ vary. When speaking about cohomology of connection sheaves, one should be extremely careful in terminology.
\end{rem}

\begin{ex}\label{exConstantCohomologyConnection}
As an example, consider the constant sheaf $\overline{\Ro^1}$ on $S$. Its cohomology, as defined in the category $\Diag(S,\Ro\Vect)$, coincides with the singular cohomology of the geometrical realization $H^*(|S|;\Ro)$, as discussed in Example~\ref{exConstantSheaf}. However, $\overline{\Ro^1}$ is actually a connection sheaf, $\overline{\Ro^1}\in\ConDiag(S,\Vv)$. In terms of Proposition~\ref{propFundGrpRepresentations} it corresponds to the trivial $\pi_1(|S|)$-module $\Ro^1$, so its cohomology coincides with the group cohomology $H^*(\pi_1(|S|);\Ro^1)$. The modules $H^*(|S|;\Ro)$ and $H^*(\pi_1(|S|);\Ro^1)$ may be different, e.g. if $|S|$ is homeomorphic to 2-sphere.

Similarly, if $D$ is an arbitrary connection sheaf, then $H^*(S;D)$ computed in $\Diag(S;\Vv)$ is the cohomology of $S$ with local coefficients $D$, while $H^*(S;D)$ computed in $\ConDiag(S,\Vv)$ is the cohomology of the $\pi_1(S,s_0)$-module $D(s_0)$. In a sense, passage from the category $\Diag(S;\Vv)$ to the subcategory $\ConDiag(S,\Vv)$ forgets higher homotopical information encoded in $S$.
\end{ex}

We have the following simple corollary of Proposition~\ref{propFundGrpRepresentations} which may seem counterintuitive at first glance.

\begin{cor}\label{corConnSheafOnSimplyConnected}
If $|S|$ is simply connected, then every connection sheaf on $S$ is isomorphic to the constant sheaf with the same stalks' dimension.
\end{cor}

\begin{rem}\label{remWhyLearnConnectionSheaf}
The natural question arises: if the connection sheaves are just the constant sheaf, then what's the point in ``learning'' a connection sheaf as described in Subsection~\ref{subsecSheafLearning} and the Section~\ref{secReviewShvsML} of the review? There are several reasonable answers.
\begin{enumerate}
  \item Data structures, on which connection sheaf learning is performed are usually graphs, and graphs are rarely simply connected.
  \item If data structures are highly dimensional, we actually learn connection quiver representations rather than ordinary connection sheaf and simply forget compositionality relations.
  %\item The features of connection sheaves, used in practice, are not invariants of sheaf isomorphism, so they still make sense. For example, euclidean connection sheaves are learned, not just connection sheaves. Euclidean isomorphism of euclidean sheaves is a stronger relation than an ordinary isomorphism, so it has more invariants.
\end{enumerate}
Anyhow, the question once again leads to Problem~\ref{problRelations} from the list. There won't be any actual higher-dimensional topology in deep learning, until the relations are taken into account.
\end{rem}


\section{Information capacity of sheaves}\label{secMathSpaceRestoredFromShvs}

In this section we discuss to which extent the category of sheaves on a space remembers the space. All the claims made in this subsection are standard from mathematical perspective, but they rigorify the usage of sheaves in geometrical problems and provide a general fundament for sheaf learning. We couldn't find the mentions of these claims in the literature, and add them in our review for completeness.

\subsection{Topos of sheaves}

Given a topological space $X$ and a chosen target category $\Vv$, we can form a category $\Shvs(X,\Vv)$ of $\Vv$-valued sheaves on $X$. The space $X$ is a bare geometrical structure, no computations are available on $X$ itself. We need to borrow certain ``types of computations'' which exist in the target category $\Vv$, and put them into $\Shvs(S,\Vv)$ in order to be able to say something about $X$. The natural question arises: how much information do we loose, when passing from $X$ to the category $\Shvs(X;\Vv)$? In case $\Vv=\Sets$ this is a classical question and the answer is well-known.

\begin{con}\label{conToposOfSheaves}
We refer to~\cite{MacLaneMoerdijk} for the definition of a topos. The category $\Sets$ is a topos, and, for any topological space $X$, the category $\Shvs(X,\Sets)$ is also a topos. Given two topoi $\ST{T}_1,\ST{T}_2$, a geometrical morphism $F$ from $\ST{T}_1$ to $\ST{T}_2$ consists of a pair of adjoint functors $f_*\colon \ST{T}_1\rightleftarrows \ST{T}_2\colon f^*$ such that $f^*$ preserves finite limits. The class of all toposes with geometrical morphisms between them forms a category, which will be denoted $\Topoi$. A continuous map of topological spaces $f\colon X\to Y$, induces direct and inverse image functors $f_*\colon \Shvs(X;\Sets)\rightleftarrows\Shvs(Y;\Sets)\colon f^*$ as described in subsection~\ref{subsecMathFunctorial}, they form a geometrical morphism in the category $\Topoi$. Hence, passing to a topos of $\Sets$-valued sheaves defines a functor
\begin{equation}\label{eqSpaceToTopos}
\Shvs(\cdot,\Sets)\colon \Top\to \Topoi,\quad X\mapsto \Shvs(X,\Sets).
\end{equation}
\end{con}

Topological space $X$ is called \emph{sober} if, roughly speaking, it can be reconstructed from its lattice, or locale, $\OpSets(X)$ of open subsets. See formal definition e.g. in~\cite[p.477]{MacLaneMoerdijk}.

\begin{prop}\label{propTopoiFullyFaithful}
The functor $\Shvs(\cdot,\Sets)$ is fully faithful on the full subcategory $\Top_{sober}$ of sober topological spaces.
\end{prop}

The proof follows from~\cite[Cor.4,p.481]{MacLaneMoerdijk} and~\cite[Prop.2, p.491]{MacLaneMoerdijk}. It is not surprising that the lattice $\OpSets(X)$ is what actually matters: presheaves were defined as diagrams on $\OpSets(X)^{\op}$.

\begin{cor}\label{corToposIsoThenSoberHomeo}
If $X,Y$ are sober topological spaces, and $\Shvs(X,\Sets)$ is isomorphic as a topos to $\Shvs(Y,\Sets)$, then $X\cong Y$.
\end{cor}

\begin{rem}\label{remRecoverXfromTopos}
Unwinding the formal proof of Proposition~\ref{propTopoiFullyFaithful} we can notice that the whole big category $\Shvs(X,\Sets)$ is not actually needed to recover a sober space $X$. Basically, the lattice $\OpSets(X)$ coincides with the collection of subsheaves of the constant $\ast$-valued sheaf, up to isomorphism. Here $\ast$ is the singleton, the final object of $\Sets$.
\end{rem}

\begin{rem}\label{remFiniteT0SpaceSetSheaves}
Any finite $T_0$-space can be shown to be sober. Corollary~\ref{corToposIsoThenSoberHomeo} gives a strict mathematical ground for the belief that $\Sets$-valued diagrams over finite posets (resp., sheaves over corresponding $T_0$-spaces, according to Proposition~\ref{propPosTop}) store the complete information about the underlying topology. It should be noticed however, that not every Alexandrov $T_0$-space is sober: such space is sober if and only if the corresponding poset is Artinian~\cite[Thm.C.6]{Lindenhovius}.
\end{rem}

\begin{rem}\label{remPreposetsNotNeededLocalic}
If $S$ is just a preordered set, and $\bar{S}$ is its corresponding poset, as in Construction~\ref{conPreposetToPoset}, then we have $\OpSets(X_S)\cong \OpSets(X_{\bar{S}})$. Therefore Alexandrov topology $X_S$ and the corresponding $T_0$-topology $X_{\bar{S}}$ are indistinguishable by means of topoi of sheaves. This is the main conceptual reason why we restrict consideration to posets, and not general preordered sets, see Remark~\ref{remPreposetsNotNeeded}.
\end{rem}

The natural question arises in abelian setting (see definition of abelian category in Appendix~\ref{secMathCohomology}). To which extent the abelian category $\Shvs(X,\ko\Vect)$ remembers a space $X$? This situation differs from the case of a topos: it appears that the abelian categories are not that ``rigid'' objects as topoi as shown in the next remark.

\begin{rem}\label{remToAbCatNotFaithful}
The functor $\Shvs(\cdot,\ko\Vect)\colon \Top\to\AbCat$ is not faithful (even on sober spaces). Indeed, there is exactly one continuous map from a singleton $\ast$ to itself, but there are infinitely many additive endofunctors on $\ko\Vect$ (which is identified with $\Shvs(\ast,\ko\Vect)$). E.g. we have an infinite series $V\mapsto V^{\oplus n}$.

It could be noticed that, for sheaves valued in an abelian category, both direct $f_*$ and inverse image $f^*$ maps are additive functors. We could wonder, if it is possible to make category $\AbCat$ more rigid, by considering adjoint pairs of additive functors $(f_*,f^*)$ as morphisms instead of a single additive morphism, i.e. by ``simulating'' a topos. However, this doesn't resolve the previous counterexample: the functor $f_*\colon V\mapsto V^{\oplus n}$ has an adjoint $f^*\colon V\mapsto V^{\oplus n}$.
\end{rem}

Nevertheless, the category of $\ko\Vect$-diagrams on a poset still remembers a poset. 

\begin{prop}\label{propRecoverXfromAbCat}
The locale $\OpSets(X)$ can be recovered from $\Shvs(X,\ko\Vect)$. The functor $\Shvs(\cdot,\ko\Vect)\colon \Top\to\AbCat$ is full on sober spaces.
\end{prop}

\begin{proof}
We only need to prove the first claim, the second follows from the equivalence between the category of spatial locales and sober topological spaces, see~\cite[Cor.4,p.481]{MacLaneMoerdijk}. There is a distinguished element $\ko\in\ko\Vect$, which is a unique up to isomorphism simple object of the abelian category $\ko\Vect$. Hence we can construct a constant $\ko$-valued sheaf $\bar{\ko}\in\Shvs(X,\ko\Vect)$. Subsheaves of $\bar{\ko}$, defined up to isomorphism, form a poset which is isomorphic to the locale $\OpSets(X)$.
\end{proof}

This proof is similar to Remark~\ref{remRecoverXfromTopos}, but now we use $\ko$ instead of $\ast$ and $0$ instead of $\varnothing$. This observation mimics the boolean logic (inherent to the topos $\Sets$) in terms of linear algebra (inherent to the abelian category $\ko\Vect$).

\begin{rem}\label{remTheoryIs1dimStrange}
Proof of Proposition~\ref{propRecoverXfromAbCat} shows that, from mathematical perspective, the complete information about finite $T_0$-topology can be extracted from sheaves, whose stalks are no more than 1-dimensional. This is somehow orthogonal to the deep learning practice, where it is assumed that a single sheaf with higher-dimensional stalks (or a few such sheaves used in the layers of a network) is already expressive enough to describe the properties of the space, see Section~\ref{secReviewShvsML}. This discrepancy between theory and practice seems interesting and deserves further investigation.
\end{rem}

\begin{rem}\label{remDerivedCatOnPoset}
The topos $\Shvs(X;\Sets)$ is interesting in its own right. However, when working in abelian setting, it is common in algebraic geometry to consider not the category $\Shvs(X;\Abel)$ of sheaves itself, but its derived versions which are better suited for homological applications. For a poset $S$, or the corresponding Alexandrov space $X_S$, the bounded derived category $\Der^b(S)$ of finite-dimensional sheaves is defined. A comprehensive study of derived categories of $\Abel$-valued sheaves on posets was done by Ladkani~\cite{LADKANI2008435}. There exist examples of non-isomorphic finite posets which are derived-equivalent.
\end{rem}
