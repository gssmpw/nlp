\section{Literature Review}
Combining or fusing networks/graphs that represent different views into a unified structure is a well-studied problem relating to our proximity network fusion setting. \citet{kang2020multi} propose a multi-graph fusion model that simultaneously performs graph combination and spectral clustering. The idea that multi-view data admit the same clustering structure aligns with the structural consistency assumptions in our proposed method. However, this method, together with other multi-view fusion approaches such as \citet{yang2019adaptive} and \citet{yang2024bidirectional}, cannot naturally consider structural consistency over time, but rather focuses on fusing multiple views into a unified static graph. \citet{zhang2021adaptive} and \citet{hu2022spatio} leverage spatio-temporal fusion to address challenges in predicting remaining useful life and in trajectory data analytics, respectively. However, their fusion modules, being within a neural network architecture, cannot be readily applied to combine multiple levels of proximity, as in our case.

Node similarity in network time series is another key topic in our paper. \citet{gunecs2016link} consider neighborhood-based node similarity scores for link prediction. \citet{yang2019time} develop a diffusion model to drive the dynamic evolution of node states and proposes a novel notion of dissimilarity index. The approach in \citet{meng2018coupled} learns node similarities by incorporating both structural and attribute-based information. However, none of them considers node similarity based on long-term close relationships.

Detecting long-term relationships in dynamic systems has been explored from various perspectives. In long-term studies of primate relationships, scientists report how long certain pairs have a high frequency of interaction, but they do not provide a statistical approach to determine what constitutes a persisting relationship. For example,  \citet{mitani2009male} emphasizes the significance of long-term affiliative relationships in social mammals by considering pairwise affinity indexes between male dyads~\citep{pepper1999general} to quantify long-term relationships, but the indexes neglect consecutive proximities in the time series. \citet{derby2024female} present a Bayesian multimembership approach to test what factors predict the persistence of proximity relationships, but persistence is defined deterministically by being in proximity for more than a fixed length of time period without considering hypothesis-testing based on an expected duration. \citet{qin2019mining} investigate the interplay between temporal interactions and social structures, but their approach is constrained to analyzing periodic behaviors. \citet{escribano2023stability} study the stability of personal relationship networks in a longitudinal study of middle school students by exploring persistence of circle structures, which is different from our novel perspectives of node similarity.

%%%%%%%%%%%%%%%%%%%%%%%%%%%%%%%%