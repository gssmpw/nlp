\section{Conclusion \& Limitations}

In this work, we address the limitations of existing model editing methods in handling long-form and diverse knowledge formats. Current approaches often face challenges due to their reliance on editing a single token’s hidden state, which restricts their effectiveness on complex and unstructured knowledge. To overcome this, we introduced \textbf{AnyEdit}, a novel autoregressive editing paradigm that enables efficient and precise updates by sequentially processing and editing token hidden states over long-form text. Our approach is grounded in theoretical validation using the Chain Rule of mutual information, demonstrating its ability to produce consistent and accurate edits. Additionally, AnyEdit serves as a versatile framework that integrates seamlessly with traditional methods, broadening their applicability to a wider range of knowledge editing tasks. 

Despite its ability to enhance traditional editing approaches and broaden their applicability to complex knowledge editing tasks, AnyEdit still faces two key limitations: Firstly, the current framework is not explicitly optimized for \textit{lifelong editing} scenarios, which demand continuous and iterative knowledge updates over time. Adapting AnyEdit to such dynamic settings—where model parameters or hidden states require periodic refinement—remains a critical challenge for future research. Secondly, AnyEdit is currently confined to \textit{textual knowledge editing} and lacks support for multimodal knowledge integration. Extending its capabilities to handle cross-modal updates (e.g., synchronizing edits across text, images, and audio) would significantly expand its practical utility, offering a promising direction for multimodal language model editing.

\section{Impact Statement}

AnyEdit enhances model editing by enabling precise and efficient updates across diverse knowledge formats, addressing limitations in editing long-form and unstructured knowledge. Its ability to modify knowledge at scale improves the adaptability of LLMs, making them more responsive to real-world updates. However, increased editing flexibility raises concerns about potential misuse, such as unauthorized knowledge injection or model tampering. Mitigating these risks requires careful deployment and oversight to ensure responsible use. We encourage the research community to leverage AnyEdit for advancing trustworthy and beneficial AI applications.

