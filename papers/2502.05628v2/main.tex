%%%%%%%% ICML 2025 EXAMPLE LATEX SUBMISSION FILE %%%%%%%%%%%%%%%%%

\documentclass{article}
\usepackage{tabularx}
\usepackage{siunitx}
\usepackage{multirow}
\usepackage{booktabs}
\usepackage[accepted]{icml2025}
% Recommended, but optional, packages for figures and better typesetting:
\usepackage{microtype}
\usepackage{graphicx}
\usepackage{subfigure}% for professional tables
%%%%% NEW MATH DEFINITIONS %%%%%

% \usepackage{amsmath,amsfonts,bm}
\usepackage{amsmath,amsfonts}

\usepackage{pifont}


\newcommand{\R}{\mathbb{R}}


\def\va{{\mathbf{a}}}
\def\vg{{\mathbf{g}}}

% Sets
\def\sR{\mathbb{R}}
\def\sC{\mathbb{C}}
\def\sZ{\mathbb{Z}}
\def\sN{\mathbb{N}}
\def\sQ{\mathbb{Q}}

\def\sS{\mathcal{S}}



% Vectors
\def\vzero{{\mathbf{0}}}
\def\vone{{\mathbf{1}}}
\def\vmu{{\mathbf{\mu}}}
\def\vtheta{{\mathbf{\theta}}}
\def\va{{\mathbf{a}}}
\def\vb{{\mathbf{b}}}
\def\vc{{\mathbf{c}}}
\def\vd{{\mathbf{d}}}
\def\ve{{\mathbf{e}}}
\def\vf{{\mathbf{f}}}
\def\vg{{\mathbf{g}}}
\def\vh{{\mathbf{h}}}
\def\vi{{\mathbf{i}}}
\def\vj{{\mathbf{j}}}
\def\vk{{\mathbf{k}}}
\def\vl{{\mathbf{l}}}
\def\vm{{\mathbf{m}}}
\def\vn{{\mathbf{n}}}
\def\vo{{\mathbf{o}}}
\def\vp{{\mathbf{p}}}
\def\vq{{\mathbf{q}}}
\def\vr{{\mathbf{r}}}
\def\vs{{\mathbf{s}}}
\def\vt{{\mathbf{t}}}
\def\vu{{\mathbf{u}}}
\def\vv{{\mathbf{v}}}
\def\vw{{\mathbf{w}}}
\def\vx{{\mathbf{x}}}
\def\vy{{\mathbf{y}}}
\def\vz{{\mathbf{z}}}
\def\vzeta{{\mathbf{\zeta}}}

% Matrix
\def\mA{{\mathbf{A}}}
\def\mB{{\mathbf{B}}}
\def\mC{{\mathbf{C}}}
\def\mD{{\mathbf{D}}}
\def\mE{{\mathbf{E}}}
\def\mF{{\mathbf{F}}}
\def\mG{{\mathbf{G}}}
\def\mH{{\mathbf{H}}}
\def\mI{{\mathbf{I}}}
\def\mJ{{\mathbf{J}}}
\def\mK{{\mathbf{K}}}
\def\mL{{\mathbf{L}}}
\def\mM{{\mathbf{M}}}
\def\mN{{\mathbf{N}}}
\def\mO{{\mathbf{O}}}
\def\mP{{\mathbf{P}}}
\def\mQ{{\mathbf{Q}}}
\def\mR{{\mathbf{R}}}
\def\mS{{\mathbf{S}}}
\def\mT{{\mathbf{T}}}
\def\mU{{\mathbf{U}}}
\def\mV{{\mathbf{V}}}
\def\mW{{\mathbf{W}}}
\def\mX{{\mathbf{X}}}
\def\mY{{\mathbf{Y}}}
\def\mZ{{\mathbf{Z}}}
\def\mBeta{{\mathbf{\beta}}}
\def\mPhi{{\mathbf{\Phi}}}
\def\mLambda{{\mathbf{\Lambda}}}
\def\mSigma{{\mathbf{\Sigma}}}


% Expectation
% \def\eE{\mathop{\mathbb{E}}\limits}
\def\eE{\mathbb{E}}

% Probability
\def\pP{\mathbb{P}}

% Tilde
\def\tf{\tilde{f}}
\def\tS{\tilde{S}}
\def\wtF{\widetilde{\mathcal{F}}}
\def\whR{\widehat{R}}
\def\tvx{\tilde{\mathbf{x}}}
\def\ty{\tilde{y}}


\def\defeq{\overset{\textup{def}}{=}}
% \def\defeq{\overset{.}{=}}
\def\defone{\overset{\text{\ding{172}}}{=}}
\def\deftwo{\overset{\text{\ding{173}}}{=}}
\def\leqone{\overset{\text{\ding{172}}}{\leq}}
\def\leqtwo{\overset{\text{\ding{173}}}{\leq}}
\def\leqthree{\overset{\text{\ding{174}}}{\leq}}
\def\leqfour{\overset{\text{\ding{175}}}{\leq}}
\def\eqone{\overset{\text{\ding{172}}}{=}}
\def\eqtwo{\overset{\text{\ding{173}}}{=}}
\def\eqthree{\overset{\text{\ding{174}}}{=}}
\def\eqfour{\overset{\text{\ding{175}}}{=}}
\def\geqfive{\overset{\text{\ding{176}}}{\geq}}
\usepackage{xcolor}         % colors
\usepackage{tcolorbox}
\usepackage{makecell}
\usepackage{enumitem}
\newcommand{\tar}[1]{\texttt{{\color{red}{#1}}}}
\newcommand{\std}[1]{\small{$\pm$#1}}
% hyperref makes hyperlinks in the resulting PDF.
% If your build breaks (sometimes temporarily if a hyperlink spans a page)
% please comment out the following usepackage line and replace
% \usepackage{icml2025} with \usepackage[nohyperref]{icml2025} above.
\usepackage{hyperref}
\usepackage[hyphenbreaks]{breakurl}
\usepackage{flushend}

% Attempt to make hyperref and algorithmic work together better:
\newcommand{\theHalgorithm}{\arabic{algoithm}}

% Use the following line for the initial blind version submitted for review:
\usepackage{icml2025}

% If accepted, instead use the following line for the camera-ready submission:
% \usepackage[accepted]{icml2025}

% For theorems and such
\usepackage{amsmath}
\usepackage{amssymb}
\usepackage{mathtools}
\usepackage{amsthm}

% if you use cleveref..
\usepackage[capitalize,noabbrev]{cleveref}

%%%%%%%%%%%%%%%%%%%%%%%%%%%%%%%%
% THEOREMS
%%%%%%%%%%%%%%%%%%%%%%%%%%%%%%%%
\theoremstyle{plain}
\newtheorem{theorem}{Theorem}[section]
\newtheorem{proposition}[theorem]{Proposition}
\newtheorem{lemma}[theorem]{Lemma}
\newtheorem{corollary}[theorem]{Corollary}
\theoremstyle{definition}
\newtheorem{definition}[theorem]{Definition}
\newtheorem{assumption}[theorem]{Assumption}
\theoremstyle{remark}
\newtheorem{remark}[theorem]{Remark}

% Todonotes is useful during development; simply uncomment the next line
%    and comment out the line below the next line to turn off comments
%\usepackage[disable,textsize=tiny]{todonotes}
\usepackage[textsize=tiny]{todonotes}

\newcommand{\wx}[1]{{\color{blue}#1}}

% The \icmltitle you define below is probably too long as a header.
% Therefore, a short form for the running title is supplied here:
\icmltitlerunning{AnyEdit: Edit Any Knowledge Encoded in Language Models}

\begin{document}

\twocolumn[
% \icmltitle{AutoEdit: Auto-Regressive Knowledge Editing for Language Models}
\icmltitle{AnyEdit: Edit Any Knowledge Encoded in Language Models}  % {BetaEdit: Editing Knowledge of Arbitrary \\ Length  and Structure for Language Models}

% It is OKAY to include author information, even for blind
% submissions: the style file will automatically remove it for you
% unless you've provided the [accepted] option to the icml2025
% package.

% List of affiliations: The first argument should be a (short)
% identifier you will use later to specify author affiliations
% Academic affiliations should list Department, University, City, Region, Country
% Industry affiliations should list Company, City, Region, Country

% You can specify symbols, otherwise they are numbered in order.
% Ideally, you should not use this facility. Affiliations will be numbered
% in order of appearance and this is the preferred way.
\icmlsetsymbol{equal}{*}

\icmlsetsymbol{corr}{*}
\begin{icmlauthorlist}
\icmlauthor{Houcheng Jiang}{ustc}
\icmlauthor{Junfeng Fang}{nus,corr}
\icmlauthor{Ningyu Zhang}{zju}
\icmlauthor{Guojun Ma}{byte}
\\
\icmlauthor{Mingyang Wan}{byte}
\icmlauthor{Xiang Wang}{ustc,corr}
\icmlauthor{Xiangnan He}{ustc}
\icmlauthor{Tat-Seng Chua}{nus}
%\icmlauthor{Firstname7 Lastname7}{comp}
%\icmlauthor{}{sch}
%\icmlauthor{Firstname8 Lastname8}{sch}
%\icmlauthor{Firstname8 Lastname8}{yyy,comp}
%\icmlauthor{}{sch}
%\icmlauthor{}{sch}
\end{icmlauthorlist}

\icmlaffiliation{ustc}{University of Science and Technology of China}
\icmlaffiliation{nus}{National University of Singapore}
\icmlaffiliation{zju}{Zhejiang University}
\icmlaffiliation{byte}{Douyin Co., Ltd}
\icmlcorrespondingauthor{Houcheng Jiang}{janghc@mail.ustc.edu.cn}
%\icmlcorrespondingauthor{Firstname2 Lastname2}{first2.last2@www.uk}

% You may provide any keywords that you
% find helpful for describing your paper; these are used to populate
% the "keywords" metadata in the PDF but will not be shown in the document
\icmlkeywords{Machine Learning, ICML}

\vskip 0.3in
]

% this must go after the closing bracket ] following \twocolumn[ ...

% This command actually creates the footnote in the first column
% listing the affiliations and the copyright notice.
% The command takes one argument, which is text to display at the start of the footnote.
% The \icmlEqualContribution command is standard text for equal contribution.
% Remove it (just {}) if you do not need this facility.

\printAffiliationsAndNotice{*Corresponding Author.}   % leave blank if no need to mention equal contribution
%\printAffiliationsAndNotice{\icmlEqualContribution} % otherwise use the standard text.


\begin{abstract}
Large language models (LLMs) often produce incorrect or outdated information, necessitating efficient and precise knowledge updates. Current model editing methods, however, struggle with long-form knowledge in diverse formats, such as poetry, code snippets, and mathematical derivations. These limitations arise from their reliance on editing a single token’s hidden state, a limitation we term ``efficacy barrier''. To solve this, we propose \textbf{AnyEdit}, a new autoregressive editing paradigm. It decomposes long-form knowledge into sequential chunks and iteratively edits the key token in each chunk, ensuring consistent and accurate outputs. Theoretically, we ground AnyEdit in the Chain Rule of Mutual Information, showing its ability to update any knowledge within LLMs. Empirically, it outperforms strong baselines by 21.5\% on benchmarks including UnKEBench, AKEW, and our new \textbf{EditEverything} dataset for long-form diverse-formatted knowledge. Additionally, AnyEdit serves as a plug-and-play framework, enabling current editing methods to update knowledge with arbitrary length and format, significantly advancing the scope and practicality of LLM knowledge editing. %Our code is available at: \url{https://github.com/jianghoucheng/AnyEdit}.
\end{abstract}

\section{Introduction}
\label{sec:intro}

\begin{figure*}[tb]
    \centering
    \includegraphics[width=0.848\linewidth]{figs/circuitnn.pdf} 
    \caption{Illustration of differentiable CircuitNN. CircuitNN is designed based on differentiable NAND gates. After DAS is guided by PI and PO pairs of the truth table, CircuitNN can get the precise circuit architecture logic equivalent to the truth table.}
    \label{fig:circuitnn}
\end{figure*}

% 1. Describe the importance of logic synthesis
% 2. Existing Problems
% (a) Neural Architecture Search: Unstable, Predefined Setting, etc.
% (b) Circuit Generation: Probabilistic Model, Logic Equivalence

With the rapid advancement of technology, the scale of integrated circuits (ICs) has expanded exponentially. 
This expansion has introduced significant challenges in chip manufacturing, particularly concerning power and area metrics.
A primary objective in IC design is achieving the same circuit function with fewer transistors, thereby reducing power usage and area occupancy.

Logic synthesis~\cite{hachtel2005logicsynth}, a critical step in electronic design automation (EDA), transforms behavioral-level circuit designs into optimized gate-level circuits, ultimately yielding the final IC layout. 
The primary goal of logic synthesis is to identify the physical implementation with the fewest gates for a given circuit function. 
This task constitutes a challenging NP-hard combinatorial optimization problem. 
Current logic synthesis tools~\cite{brayton2010abc, wolf2013yosys} rely on human-designed heuristics, often leading to sub-optimal outcomes.

Differentiable architecture search (DAS) techniques~\cite{liu2018darts, chu2020darts} offer novel perspectives on addressing challenges in this problem.
Circuit functions can be represented through truth tables, which map binary inputs to their corresponding outputs. 
Truth tables provide a precise representation of input-output relationships, ensuring the design of functionally equivalent circuits.
Inspired by this, researchers~\cite{deepmind2024ai4sys, wang2024tnet} have begun exploring the application of DAS to synthesize circuits directly from truth tables.
Specifically, \citet{deepmind2024ai4sys} proposed CircuitNN, a framework that learns differentiable connection structures with logic gates, enabling the automatic generation of logic circuits from truth tables.
This approach significantly reduces the complexity of traditional circuit generation. 
Building on this, \citet{wang2024tnet} introduced T-Net, a triangle-shaped variant of CircuitNN, incorporating regularization techniques to enhance the efficiency of DAS.

Despite these advancements, several challenges remain. 
The computational complexity of DAS grows quadratically with the number of gates, posing scalability issues.
Although triangle-shaped architecture~\cite{wang2024tnet} partially mitigates this problem, redundancy persists. 
%Additionally, DAS is susceptible to converging to local optima, limiting the ability to search architectures that satisfy the given truth tables~\cite{liu2018darts}. 
%Furthermore, hyperparameters (network depth and layer width) require extensive searches, introducing complexity and prolonging the synthesis process. 
Additionally, DAS is susceptible to converging to local optima~\cite{liu2018darts} and hyperparameters (network depth and layer width) require extensive searches. 
The challenges arise from the vast search space in DAS. 
% Even with predefined settings for CircuitNN, finding a configuration that meets the truth table requires extensive trial and error during the DAS process. 
Intuitively, limiting the search space through predefined parameters (network depth, gates per layer, and connection probabilities) can significantly reduce the complexity.

Recent advances~\cite{openai2023gpt4, abramson2024alphafold3, esser2024sd3, li2024mar} in conditional generative models have demonstrated remarkable performance across language, vision, and graph generation tasks. 
Motivated by these developments, we propose a novel approach to circuit generation that generates preliminary circuit structures to guide DAS in generating refined circuits matching specified truth tables. 
Firstly, we introduce CircuitVQ, a tokenizer with a discrete codebook for circuit tokenization. 
Built upon our Circuit AutoEncoder framework~\cite{hou2022graphmae,li2023maskgae,wu2025mgvga}, CircuitVQ is trained through a circuit reconstruction task. 
Specifically, the CircuitVQ encoder encodes input circuits into discrete tokens using a learnable codebook, while the decoder reconstructs the circuit adjacency matrix based on these tokens.
Subsequently, the CircuitVQ encoder serves as a circuit tokenizer for CircuitAR pretraining, which employs a masked autoregressive modeling paradigm~\cite{chang2022maskgit, li2023mage}. 
In this process, the discrete codes function as supervision signals. 
After training, CircuitAR can generate discrete tokens progressively, which can be decoded into initial circuit structures by the decoder of the CircuitVQ. 
These prior insights can guide DAS in producing refined circuits that match the target truth tables precisely.

Our key contributions can be summarized as follows:
\begin{itemize}
\item We introduce CircuitVQ, a circuit tokenizer that facilitates graph autoregressive modeling for circuit generation, based on our Circuit AutoEncoder framework;
\item Develop CircuitAR, a model trained using masked autoregressive modeling, which generates initial circuit structures conditioned on given truth tables;
\item Propose a refinement framework that integrates differentiable architecture search to produce functionally equivalent circuits guided by target truth tables;
\item Comprehensive experiments demonstrating the scalability and capability emergence of our CircuitAR and the superior performance of the proposed circuit generation approach.
\end{itemize}

% Motivation
% (a) Diffusion (Vision, Graph), Autoregressive (Language, Vision)
% (b) Circuit Generation for Predefined Setting
% (c) Neural Architecture Search for Strict Logic Equivalence

% Contribution
% (a) Circuit Tokenizer (new transformer arch, training strategy)
% (b) CircuitAR (train and gen strategies, post-ar strategy)
% (c) Extensive Evaluation including BitD (Bit Distance) for Scalability

% \begin{figure}
%     \centering
%     \includegraphics[width=0.5\linewidth]{Move_teaser.pdf}
%     \caption{Comparison of different dynamic compute approaches. length of arrow indicates residual transformation per token while width indicates velocity of transformation.}
%     \label{fig:enter-label}
% \end{figure}

\section{Method}
\label{sec:method}
Residual connections play a crucial role in shaping token representations, yet their dynamics remain underexplored in the context of efficient decoding. In this work, we delve deeper into transformer residual dynamics and investigate how modulating residual transformation velocity can improve inference efficiency in token-level processing, optimizing both dense and sparse MoE transformers.


\subsection{Residual Dynamics and Motivation for Multi-rate Residuals} \label{sec:motivation}

To analyze how hidden representations evolve across different layers of a transformer architecture, it's crucial to consider the effect of residual connections. Each transformer decoder layer typically has residual connections across attention and MLP submodules. As the residual stream $h_i$ traverses from interval $E_j$ to $E_{j+1}$, it undergoes a residual transformation given by:  
% \begin{equation}
% \label{eq:slow_residual_transformation}
% H_{E_{j+1}} = H_{E_j} \prod_{i=E_j}^{E_{j+1}} \left( I + \mathcal{A}_i \right) \left( I + \mathcal{M}_i \right) \quad \text{where} \quad \mathcal{A}_i = f(c_i, h_{i}), \mathcal{M}_i = g(h_i)
% \end{equation}

\begin{equation} \label{eq:slow_residual_transformation}
h_{E_{j+1}} = h_{E_j} + \sum_{i=E_j}^{E_{j+1}-1} \left( \mathcal{A}_i(h_i) + \mathcal{M}_i(h_i + \mathcal{A}_i(h_i)) \right) \quad \text{where} \quad \mathcal{A}_i = f(c_i, h_{i}), \mathcal{M}_i = g(h_i). 
\end{equation}

Here, \( \mathcal{A}_i \) denotes the non-linear transformation introduced by the multi-head attention mechanism at layer \( i \), while \( \mathcal{M}_i \) corresponds to the non-linear transformation of the MLP block at the same layer. These transformations depend on the input residual stream \( h_i \) and, in the case of \( \mathcal{A}_i \), the previous contextual representation \( c_i \).\footnote{Normalization layers are typically applied in practice but are omitted here for simplicity of the argument.}


% For easy tokens, the magnitude and direction of this delta transformation become progressively smaller with each successive layer as shown in \cref{fig:delta_transformation}. Consequently, it is feasible to predict these tokens after only a few residual connections, whereas harder tokens necessitate more extensive processing through additional layers.

\begin{figure}[ht]
    \centering
    \begin{subfigure}{0.48\textwidth}
        \centering
        \includegraphics[width=\textwidth]{sections/figures/residual_change.pdf}
        \caption{}
        \label{fig:residual_change}
    \end{subfigure}%
    \hfill
    \begin{subfigure}{0.48\textwidth}
        \centering
        \includegraphics[width=\textwidth]{sections/figures/alignment_wrt_dedicated_model.pdf}
        \caption{}
    \label{fig:alignment_wrt_dedicated_model}
    \end{subfigure}
    \caption{(a) As residual streams propagate through the model, the directional shifts in the residuals become progressively smaller. (b) A dedicated model with $k$ layers achieves a faster rate of change in residual streams and higher alignment than base model leveraging early exit mechanisms at layer $k$.}
    \label{fig}
\end{figure}


To examine whether residual transformations can be accelerated across layers, we conducted experiments using a diverse set of prompts on a pre-trained Phi3 model~\cite{phi3_report}. As illustrated in \cref{fig:residual_change}, we measured the directional shift in residual states as \( 1 - \mathcal{C}(h_{i-1}, h_i) \), where \(\mathcal{C}\) denotes normalized cosine similarity. This shift is notably higher in the initial layers, gradually decreasing in subsequent layers. This behavior allows traditional early exit approaches to effectively accelerate decoding by enabling earlier exits for simpler tokens. However, these approaches typically rely on a distance-based approximation, where the full residual transformation of the model is approximated by the residual transformations of the initial layers. To gain deeper insights into the distance versus velocity aspects of residual transformation, we conducted a comparative study. Specifically, we trained an early exit head at layer $k$ of the Phi3 model, which consists of 32 layers, restricting the distance traveled by each token. To accelerate the residual transformation relative to number of layers, we trained a smaller model consisting of only $k$ layers, while keeping all other hyperparameters consistent. We then compared the next-token prediction accuracy of the early exit head of the base model with that of the smaller model. To ensure an equal number of trainable parameters, we inserted low-rank adapters into the smaller model and trained only these adapters, whereas, in the distance-based approach, we trained solely the early exit head. In addition, to accelerate the residual transformation in smaller model, we distilled the residual streams from the larger model by incorporating a distillation loss ~\cite{sanh2019distilbert} between the residual state at layer \(i\) of the smaller model and the residual state at layer \(4 \times i\) of the larger model. As shown in ~\cref{fig:alignment_wrt_dedicated_model} the smaller model demonstrates a significantly faster rate of change in residual streams, leading to higher next token prediction accuracy after $k$ layers compared to the base model that employs traditional early exit mechanisms after $k$ layers \cite{schuster2022confident, chen2023eellm, varshney-etal-2024-investigating}. This experimental setup, which modifies only the rate of change in residual streams while keeping other factors constant, suggests that dense transformers, trained with a fixed number of layers, may inherently possess a slow residual transformation bias.

This observation raises an intriguing question: if the rate of change in residual streams could be accelerated relative to the number of layers, is it possible to facilitate earlier alignment for a greater proportion of tokens? Earlier alignment would be beneficial to not only facilitate dynamic computation but also for generating speculative tokens efficiently with high acceptance rates in speculative decoding setups ~\cite{leviathan2023fast, chen2023accelerating}. 

%thereby enhancing the efficiency of early exiting? 
 % This bias likely constrains the effectiveness of early exiting, particularly for easier tokens. By addressing this limitation through accelerated residual transformations, we hypothesize that it is possible to substantially improve the efficiency and accuracy of early exit strategies in transformer models.

\subsection{Multi-Rate Residual Transformation} \label{m2r2_method}

To address the slow residual transformation bias described in ~\cref{sec:motivation}, we introduce \textit{accelerated residual streams} that operate at rate $R$ relative to original slow residual stream. We pair slow residual stream, $h$ with an accelerated residual stream, $p$, which has an intrinsic bias towards earlier alignment. Relative to ~\cref{eq:slow_residual_transformation}, accelerated residual transformation from interval $E_j$ to $E_{j+1}$ can be represented as: 

% \begin{equation}
% \label{eq:fast_residual_transformation}
% P_{E_{j+1}} = P_{E_j} \prod_{i=E_j}^{E_{j+1}} \left( I + \hat{\mathcal{A}_i} \right) \left( I + \hat{\mathcal{M}_i} \right) \quad \text{where} \quad \hat{\mathcal{A}_i} = \hat{f}(c_i, P_{i}), \hat{\mathcal{M}_i} = \hat{g}(P_{i})
% \end{equation}


\begin{equation} \label{eq:fast_residual_transformation}
p_{E_{j+1}} = p_{E_j} + \sum_{i=E_j}^{E_{j+1}-1} \left( \hat{\mathcal{A}_i}(p_i) + \hat{\mathcal{M}_i}(p_i + \hat{\mathcal{A}_i}(p_i)) \right) \quad \text{where} \quad \hat{\mathcal{A}_i} = \hat{f}(c_i, p_{i}), \hat{\mathcal{M}_i} = \hat{g}(h_i), 
\end{equation}



where $\hat{\mathcal{A}_i}$ and $\hat{\mathcal{M}_i}$ denote non-linear transformation added by layer $i$ to previous accelerated residual $p_{i}$. Similar to $\mathcal{A}_i$, non-linear transformation $\hat{\mathcal{A}_i}$ attends to same context $c_i$ but uses a different transformation $\hat{f}$ for accelerating $p_{E_j}$ relative to $h_{E_j}$. 

We integrate accelerated residual transformation directly into the base network using parallel accelerator adapters such that rank of accelerator adapters $R_p << d$ where $d$ denotes base model hidden dimension. This setup allows the slow residual stream $h_{E_j}$ to pass through the base model layers while the accelerated residual stream $p_{E_j}$ utilizes these parallel adapters as shown in ~\cref{fig:m2r2_main}. Both slow and accelerated residuals are processed in same forward pass via attention masking and incur negligible additional inference latency in memory bound decoding setups, while in compute bound decoding setups where FLOPs optimization is essential, accelerated residual stream utilizes a fraction of attention heads that of slow residual (see ~\cref{sec:flops_optimization}). Additionally, to maximize the utility of accelerated residual transformations without introducing dedicated KV caches, we propose a shared caching mechanism between the slow and accelerated streams which minimally impact alignment benefits of our approach while offering substantial memory savings (see ~\cref{fig:koala_alignment}). Specifically, the attention operation on the slow residuals \( \text{MHA}(h_t, h_{\leq t}, h_{\leq t}) \) is redefined for accelerated residuals as 
\[
\hat{\mathcal{A}} = MHA(p_t, h_{<t} \oplus p_t, h_{<t} \oplus p_t),
\]
where the accelerated residual at time-step $t$, \( p_t \) attends to the slow residual’s KV cache, facilitating the reuse of contextual information across both residual streams without incurring additional caching costs. Here, \(MHA(q, k, v) \) represents multi-head attention between query \( q \), key \( k \), and value \( v \).

\begin{figure}
    \centering
    \includegraphics[width=0.8\linewidth]{sections//figures/m2r2_main2.pdf}
    \caption{Multi-rate Residuals Framework: Slow residual stream of base model is accompanied by a faster stream that operates at a $2-(J+1)\times$ rate relative to the slow stream, undergoing transformations via accelerator adapters as detailed in \cref{m2r2_method}, where J denotes number of early exit intervals. Colors within the slow and fast residual streams indicate similarity, with matching colors representing the most closely aligned residual states. At the beginning of the forward pass and at each exit point, the accelerated residual state is initialized from the corresponding slow residual state to avoid gradient conflict during training (see ~\cref{sec:grad_conflict}). Early exiting decisions are informed by the Accelerated Residual Latent Attention (ARLA) mechanism, described in \cref{method_arla}, which evaluates residual dynamics across consecutive exit gates.}
    \label{fig:m2r2_main}
\end{figure}

% Furthermore. to maximize the benefits of fast residual transformations without using dedicated KV caches, we propose sharing the fast network’s cache with the slow network. Formally speaking, We modify attention operation on slow residuals $MHA(H_t, H_{<=t}, H_{<=t})$ as $MHA(P_{t}, H_{<t} \oplus P_t, H_{<t}  \oplus P_t)$ such that accelerated residuals attend to previous slow context KV cache, where $MHA(q,k,v)$ denotes multi head attention between query, $q$, key $k$ and value $v$.


\subsection{Enhanced Early Residual Alignment}
Early residual alignment is instrumental in optimizing early exiting, speculative decoding, and Mixture-of-Experts (MoE) inference mechanisms. In this section, we provide a detailed analysis of how accelerated residuals enhance these inference setups.

% By aligning the residual states of intermediate layers with the final output representations, the model can maintain high prediction accuracy even when computations are truncated at earlier layers. This enables more reliable early exiting, reducing the overall computational cost while preserving performance. Additionally, in speculative decoding, early residual alignment allows the model to make confident predictions using faster, partial computations, thereby accelerating inference without sacrificing output quality.


\subsubsection{Early Exiting} \label{method_early_exiting}

A prevalent strategy for enabling early exiting at an intermediate layer $E_{j}$ involves approximating the residual transformation between $E_{j}$ and the final layer $N-1$ using a linear, context independent mapping, $\mathcal{T}$, such that $H_{N-1} \approx \mathcal{T}(H_{E_{j}})$. This approximation has been extensively employed in conventional approaches ~\cite{schuster2022confident, chen2023eellm, varshney-etal-2024-investigating}, providing a computationally efficient means to project the output of deeper layers from intermediate states. Specifically, residual state of layer $N-1$ with this approximation can be expressed as:


% \begin{equation}
% \label{eq: vanila_ea_assumption}
% \Phi(H_{E_{j}}) \sim H_{E_{j}} \prod_{i=E_{j}}^{N}\left( I + \mathcal{A}_i \right) \left( I + \mathcal{M}_i \right) \quad \text{where} \quad \Phi \perp C
% \end{equation}

\begin{equation} \label{eq:early_exiting}
h_{E_j} + \sum_{i=E_j}^{N-1} \left( \mathcal{A}_i(h_i) + \mathcal{M}_i(h_i + \mathcal{A}_i(h_i)) \right) \sim \mathcal{T}(h_{E_{j}})  \quad \text{where} \quad \mathcal{T} \perp c. 
\end{equation}


Here, $\mathcal{A}_i$ and $\mathcal{M}_i$ represent the residual contributions of the multi-head attention and MLP layers, respectively, while $\mathcal{T}$ remains independent of $c$, the preceding context.

This approach is inherently limited by two major factors: first, the assumption of linearity between $h_{E_{j}}$ and $h_{N-1}$ may not hold uniformly for all tokens, particularly when $E_j \ll N$. Second, the linear transformation $\mathcal{T}$ disregards the influence of the context $c$ and fails to account for the latent representations of previous contextual states. In contrast, M2R2 accelerated residual states mitigate both of these challenges by approximating the slow residual transformation of all layers via a faster residual transformation of fewer layers as:
% \begin{equation}
% H_{E_j} \prod_{i=E_j}^{N}\left( I + \mathcal{A}_i \right) \left( I + \mathcal{M}_i \right) \sim P_{E_j} \prod_{i=E_j}^{E_j+1}\left( I + \hat{\mathcal{A}_i} \right) \left( I + \hat{\mathcal{M}_i} \right)
% \end{equation}


\begin{equation} \label{eq:m2r2_approximating_ea}
h_{E_j} + \sum_{i=E_j}^{N-1} \left( \mathcal{A}_i(h_i) + \mathcal{M}_i(h_i + \mathcal{A}_i(h_i)) \right) \sim p_{E_j} + \sum_{i=E_j}^{E_{j+1}-1} \left( \hat{\mathcal{A}_i}(p_i) + \hat{\mathcal{M}_i}(p_i + \hat{\mathcal{A}_i}(p_i)) \right), 
\end{equation}

% \begin{equation} \label{eq:fast_residual_transformation}
% p_{E_{j+1}} = p_{E_j} + \sum_{i=E_j}^{E_{j+1}-1} \left( \hat{\mathcal{A}_i}(p_i) + \hat{\mathcal{M}_i}(p_i + \hat{\mathcal{A}_i}(p_i)) \right) \quad \text{where} \quad \hat{\mathcal{A}_i} = \hat{f}(c_i, p_{i}), \hat{\mathcal{M}_i} = \hat{g}(h_i) 
% \end{equation}






where $p_{E_j}$ is initialized from the slow residual state $h_{E_j}$ at each early exit interval $E_j$ using an identity transformation (see ~\cref{fig:m2r2_main}). As shown in ~\cref{fig:m2r2_residual_sim}, accelerated residuals offer a smoother, more consistent shift in residual direction across layers, in contrast to the abrupt changes typically seen at early exit points in standard early exit methods. Moreover, the normalized cosine similarity between accelerated states at early exit intervals and final residual states is substantially higher compared to traditional early exit techniques, highlighting improved alignment with final layer representations. Traditional adaptive compute methods are constrained by two principal factors: the number of tokens eligible for early exit at intermediate layers and the precision of early exit decision. If residual streams fail to saturate early, the majority of tokens remain ineligible for exit, thereby diminishing potential speedups. Additionally, imprecise delineations between tokens suitable for early exit can lead to underthinking (premature exits that adversely affect accuracy) or overthinking (unnecessary processing that compromises efficiency) ~\cite{zhou2020self, dai2020dynamic}. Enhanced early alignment using ~\cref{eq:m2r2_approximating_ea} helps to address  first issue. To address the second issue we introduce Accelerated Residual Latent Attention, which dynamically assesses the saturation of the residual stream, allowing for a more precise differentiation between tokens that can exit early and those requiring further processing.

% This results in uniform change in residual direction    
% % We keep $\mathcal{A} = \hat{\mathcal{A}}$, while $\hat{\mathcal{M}}$ is accelerated by a factor of $2 - (N_{E}+1)X$ relative to the slower residual transformation $\mathcal{M}$, where $N_E$ represents number of early exiting intervals.
% Figure~\cref{fig:rate_change_comparison} illustrates the comparative rate of change between these transformation streams.



% fig:rate_change_comparison
% - grid plot x axis -> layer id (0, 8) , y axis -> layer id -> dark color cell for max similarity , lighter for lower 
% 
-------------------------------------------------------
Let's consider residual stream $h_i$ traverses through interval $E_j$ to $E_{j+1}$ and undergoes residual transformation given by 
\begin{equation}
h_{E_{j+1}} = h_{E_j} \prod_{i=E_j}^{E_{j+1}} \left( 1 + \delta_i \right)    
\end{equation}

where $\delta_i$ denotes non-linear transformation added by layer $i$. Each non-linear transformation of layer $i$ is a function of previous contextual representation, $c_i$ and input residual stream $h_i-1$ as
$\delta_i = f(c_i, h_{i-1})$ 

One way to exit early at exit $E_j+1$ is to assume that residual transformation from $E_j+1$ to final layer $N-1$ can be approximated by a linear function $\phi$ as $h_{N-1} \sim \Phi(h_{E_j+1})$ and most conventional approaches such as \todo{cite EA papers} use this approach. In other words, 

\begin{equation}
\Phi(h_{E_j+1} \sim h_{E_j+1} \prod_{i=E_j+1}^{N} \left( 1 + \delta_i \right)   
\end{equation}

This approach suffers from two primary issues, linearity assumption from $h_E_j+1$ to $H_N-1$ if often incorrect, particularly when $E_j << N$. More importantly, linear transformation $\Phi$ doesn't consider effect of context $C_i$. M2R2  effectively addresses these issues as accelerated residual stream at interval $E_j+1$ can be represented as 

\begin{equation}
r_{E_{j+1}} = r_{E_j} \prod_{i=E_j}^{E_{j+1}} \left( 1 + \gamma_i \right)    
\end{equation}

where $\gamma_i$ denotes non-linear transformation added by layer $i$ to previous accelerated residual $r_i-1$. Similar to $\delta_i$, non-linear transformation $\gamma_i$ considers context $C_i$ as 
$\gamma_i = g(c_i, r_{i-1})$. So in summary, slow residual transformation is approximated by accelerated residual as: 

\begin{equation}
h_{E_j} \prod_{i=E_j}^{N} \left( 1 + \delta_i \right) \sim h_{E_j} \prod_{i=E_j}^{E_j+1} \left( 1 + \gamma_i \right)
\end{equation}

It's worth noting that accelerated residual $r_i$ and slow residual $h_i$ are processed concurrently at layer $i$ by constructing proper attention mask such as attention of slow residual is represented as 

$MHA(H_it, H_{i<=t}, H_{i<=t}$ while attention of fast residual is computed as 

$MHA(r_it, H_{i<=t}, H_{i<=t}$ where $MHA(q,k,v$ denotes multi head attention between query, $q$, key $k$ and value $v$.


------------------------------------------------------------------

Vertical latent attention on accelerated residual is computed as 
$MHA(S_mt, S(Ej<=i<=m)t, S(Ej<=i<=m)t)$ where $Smt$ denotes query/key/value projection in latent domain at layer $m$ at time $t$. 
------------------------------------------------------------------

Gradient conflict Avoidance: 

Let's consider $w_j$ is a trainable parameter that belongs to a layer between $E_j$ and $E_j+1$. Consider early exit loss at gate $E_j+1$, $L_j+1$, gradient propagation of $w_j$ at another trainable parameter $w_j-n$ can be gives as 

$\sum_{k=E_j-n}^{E_j} \beta_k \frac{\partial L_{E_k}}{\partial w_k}$

where $\beta_j$ denotes backward transformation coefficient for weight $w_j$ to reach gate $E_j$. 
 
On the other hand, gradient propagation in proposed approach can be represented as 

\[
\frac{\partial L_{E_j}}{\partial w_j} = 
\begin{cases} 
\beta_j \frac{\partial L_{E_j}}{\partial w_j} & \text{if } E_j \leq w_j \leq E_{j+1} \\
0 & \text{otherwise}
\end{cases}
\]







% \begin{figure}[ht]
%     \centering
%     \includegraphics[width=0.8\textwidth, height=5cm]{rate_change_comparison.png}
%     \caption{Rate of change comparison between fast and slow residual streams.}
%     \label{fig:rate_change_comparison}
% \end{figure}

%vary k and and plot EA accuracy for larger and smaller models. 

% \begin{figure}[ht]
%     \centering
%     \includegraphics[width=0.5\textwidth,height=5cm]{sections/figures/alignment_comparison_dialogsum.pdf}
%     \caption{Alignment of exited tokens for different early exit layers using traditional early exiting heads, dedicated faster networks, and faster residuals.}
%     \label{fig:small_model_early_exiting}
% \end{figure}


\textbf{Accelerated Residual Latent Attention} \label{method_arla}

In the context of residual streams, we observe that the decision to exit at a given layer can be more effectively informed by analyzing the dynamics of residual stream transformations, instead of solely relying on a classification head applied at the early exit interval $E_j$. To capture the subtle dynamics of residual acceleration, we propose a \textit{Accelerated Residual Latent Attention} (ARLA) mechanism. This approach involves making the exit decision at gate $E_j$ by attending to the residuals spanning from gate $E_{j-1}$ to $E_j$, rather than considering only the residual at gate $E_j$. To minimize the computational overhead associated with exit decision-making, the attention mechanism operates within the latent domain as depicted in ~\cref{fig:arla_arch}. Formally, for each interval $[E_j, E_{j+1}]$, the accelerated residuals are projected into Query ($Q^s_{E_j}, \ldots, Q^s_{E_{j+1}}$), Key ($K^s_{E_j}, \ldots, K^s_{E_{j+1}}$), and Value ($V^s_{E_j}, \ldots, V^s_{E_{j+1}}$) vectors, with latent dimension $d^s$ for $Q^s$, $K^s$, and $V^s$ being significantly smaller than hidden dimension of $p$.\footnote{We use $d^s = 64$ for experiments described in ~\cref{sec:experiments}.} Notably, when the router is allowed to make exit decisions at gate $E_j$ based on residual change dynamics, we observe that the attention is not confined to the residual state at $E_j$ but is distributed across residual states from $E_{j-1}$ to $E_j$, %as illustrated in Figure~\ref{fig:vertical_latent_attention_dynamics}. 
This broader focus on residual dynamics significantly reduces decision ambiguity in early exits, as demonstrated in Figure~\ref{fig:roc_arla}, which contrasts routers based on the last hidden state, and the proposed ARLA router.

%show R -> S transformation. 
%show parameter and flop overhead as compared to adapter on last hidden state.

% \begin{figure}[ht]
%     \centering
%     \includegraphics[width=0.5\textwidth,height=5cm]{sections/figures/roc_arla.pdf}
%     \caption{ROC curves of early exit decision strategies: confidence-based methods (CALM/LITE), routers based on the accelerated hidden state, and latent attention routers.}
%     \label{fig:decision_making_comparison}
% \end{figure}

% \begin{figure}[ht]
%     \centering
%     \includegraphics[width=0.5\textwidth,height=5cm]{vertical_latent_attention.png}
%     \caption{Vertical latent attention mechanism for optimizing early exit decisions by considering residuals from gate \(M\) through \(M-1\).}
%     \label{fig:vertical_latent_attention}
% \end{figure}

\begin{figure}[ht]
    \centering
    \begin{subfigure}{0.52\textwidth}
        \centering
        \includegraphics[width=\textwidth, height = 4cm]{sections/figures/arla_arch.pdf}
        \caption{Accelerated Residual Latent Attention (ARLA): Accelerated residuals between early exit gates are projected into latent domain and attention over residual states within the interval is computed to capture residual dynamics and exit decision is made based on residual saturation.}
        \label{fig:arla_arch}
    \end{subfigure}%
    \hfill
    \begin{subfigure}{0.45\textwidth}
        \centering
        \includegraphics[width=\textwidth, height = 4.5cm]{sections/figures/vla_roc.pdf}
        \caption{ROC classification curves of early exit decision strategies using a linear router used on last residual state ~\cite{schuster2022confident, varshney-etal-2024-investigating, chen2023eellm}  and using ARLA approach that considers residual dynamics. }
        \label{fig:roc_arla}
    \end{subfigure}
    \caption{Effectiveness of ARLA in capturing residual dynamics for early exiting decisions.}


\end{figure}



% \begin{figure}[ht]
%     \centering
%     \includegraphics[width=1\textwidth,height=5cm]{sections/figures/arla.pdf}
%     \caption{fig that plots 32 rows 2 cols heatmap showing attention at each gate}
%     \label{fig:vertical_latent_attention_dynamics}
% \end{figure}

\subsubsection{Self Speculative Decoding} \label{method_self_speculative_decoding}

An alternative means to exploit the early alignment properties of our approach is through the use of accelerated residual states for speculative token sampling to accelerate autoregressive decoding. Speculative decoding aims to speed up memory-bound transformer inference by employing a lightweight draft model to predict candidate tokens, while verifying speculated tokens in parallel and advancing token generation by more than one token per full model invocation \cite{leviathan2023fast, chen2023accelerating, xia2023speculative, miao2023specinfer}. Despite its effectiveness in accelerating large language models (LLMs), speculative decoding introduces substantial complexity in both deployment and training. A separate draft model must be specifically trained and aligned with the target model for each application, which increases the training load and operational complexity ~\cite{chen2023accelerating}. Additionally, this approach is resource-inefficient, as it requires both the draft and target models to be simultaneously maintained in memory during inference \cite{leviathan2023fast, chen2023accelerating}. 

One strategy to address this inefficiency is to leverage the initial layers of the target model itself to generate speculative candidates, as depicted in ~\cite{Tang2024}. While this method reduces the autoregressive overhead associated with speculation, it suffers from suboptimal acceptance rates. This occurs because the linear transformation employed for translating hidden states from layer $k$ to the final layer $N$ is typically a poor approximation, as discussed in ~\cref{sec:motivation} and ~\cref{method_early_exiting}. Our approach resolves this limitation by utilizing accelerated residuals, which demonstrate higher fidelity to their slower counterparts. By utilizing accelerated residuals operating at a rate of $N/k$, where $k$ denotes the number of layers used for candidate speculation, we are able to efficiently generate speculative tokens for decoding.\footnote{We typically set $k = 4$ to balance the trade-off between autoregressive drafting overhead and acceptance rate, as discussed in~\cref{sec:experiments}.}
 This technique not only obviates the need for multiple models during inference but also improves the overall efficiency and effectiveness of speculative decoding.

\begin{figure}
    \centering    \includegraphics[width=1\linewidth]{sections/figures/m2r2_aot_loading.pdf}
    \caption{Ahead-of-Time Expert Loading: M2R2 accelerated residual stream predicts experts required for future layers, reducing reliance on on-demand lazy loading. Speculative pre-loading is efficiently overlapped with computation of multi-head attention (MHA) and MLP transformations. Only incorrectly speculated experts are loaded lazily, resulting in faster inference steps and improved computational efficiency. Here, H indicates LBM Host while D indicates HBM Device.}
    \label{fig:moe_expert_aot_loading}
\end{figure}


\subsubsection{Ahead of Time Expert Loading:} \label{method_aot_expert_loading}

Recent advancements in sparse Mixture-of-Experts (MoE) architectures ~\cite{shazeer2017outrageously, fedus2022switch, artetxe2019massively, lepikhin2020gshard, zoph2022designing} have introduced a paradigm shift in token generation by dynamically activating only a subset of experts per input, achieving superior efficiency in comparison to dense models, particularly under memory-bound constraints of autoregressive decoding \cite{fedus2022switch, zoph2022designing}. This sparse activation approach enables MoE-based language models to generate tokens more swiftly, leveraging the efficiency of selective expert usage and avoiding the overhead of full dense layer invocation. In dense transformer models, pre-loading layers is a common strategy to enhance throughput, as computations of current layer can be overlapped with pre-loading of next layer parameters ~\cite{narayanan2021efficient, shoeybi2020megatron}. However, MoE models face a unique challenge: expert selection occurs dynamically based on previous layer’s output, making it infeasible to preload next layer’s experts in parallel. This limitation results in inherent latency, as expert loading becomes a sequential, on-demand process ~\cite{lepikhin2020gshard, fedus2022switch}.

To address this inefficiency, our method introduces a mechanism with \textit{accelerated residuals}, which not only captures key characteristics of base slower residual states but also exhibit high cosine similarity with their final counterparts (as illustrated in \cref{fig:m2r2_residual_sim}). By employing accelerated residual streams, we can effectively predict the necessary experts for future layers well in advance of their actual invocation. Specifically, using a $2\times$ accelerated residual, the experts needed for layers $2i+2$ and $2i+3$ can be identified while still computing in layer $i$, thus overcoming the bottleneck of sequential, on-demand expert selection and mitigating latency in the decoding pipeline, as shown in \cref{fig:moe_expert_aot_loading}. Note that, we use fixed set of accelerator adapters for transforming accelerated residuals (as discussed in ~\cref{m2r2_method}) while slow residual is transformed via expert routing mechanism. 

Furthermore, our approach integrates a Least Recently Used (LRU) caching strategy, which enhances memory efficiency by replacing the least recently used experts with speculated experts that are anticipated to be needed in upcoming layers. This hybrid approach of preemptive expert loading with LRU caching yields substantial improvements over traditional on-demand loading or standalone caching strategies. By minimizing cache misses and efficiently managing memory, this approach addresses both compute and memory bottlenecks, leading to faster, more resource-efficient token generation in MoE architectures. A comprehensive evaluation of this strategy, in relation to state-of-the-art methods, is provided in \cref{experiments_aot}, and the compute and memory traces on an A100 GPU are detailed in \cref{fig:moe_aot_cuda_trace}.



% Recent advancements in sparse Mixture-of-Experts (MoE) architectures have introduced the concept of utilizing distinct computational paths for different tokens \cite{shazeer2017outrageously}. This approach, wherein only a subset of experts are activated per input, enables MoE-based language models to generate tokens more swiftly compared to their dense counterparts due to memory-bound nature of auto-regressive decoding. In dense models, pre-loading layers in advance is a common strategy to enhance computational efficiency. However, this technique is not applicable to MoE models, where expert selection occurs dynamically based on the outputs of previous layers, preventing parallel pre-fetching of experts.

% Our proposed method addresses this inefficiency. Accelerated residuals, which are highly similar to their slower counterparts (see \cref{fig:similarity}), can reliably predict the necessary experts ahead of time. For instance, by utilizing $2X$ accelerated residual stream, we can predict the experts needed for the layer $2i+1$ and $2i+3$ while carrying out computation in layer $i$. This enables us to commence expert loading significantly earlier, as illustrated in \cref{expert_loading}, effectively mitigating the delays observed with the naive on-demand expert loading. Additionally, our method benefits from incorporating a Least Recently Used (LRU) strategy, where speculated experts replace those that are least recently utilized, resulting in improved performance compared to using either strategy alone. For a comprehensive evaluation, refer to \cref{moe_trace}, which provides a CUDA compute and memory trace of our approach executed on <>.



% A naive solution involves using the residual state of the previous layer along with the gating function of the next layer to predict which experts need to be loaded, and initiating the expert loading process in parallel with the attention computation of the next layer. Yet, as shown in \cref{fig:MOE_attn_vs_loading_time}, the attention computation for medium to long contexts is considerably faster than the expert loading time, making this approach inefficient.




\subsection{Training} \label{method_training}
% This approach is feasible due to the absence of gradient conflicts, as discussed in \cref{sec:grad_conflict}.

To accelerate residual streams, we employ parallel accelerator adapters as described in \cref{m2r2_method}.  For the early exiting use-case outlined in \cref{method_early_exiting}, we define the training objective for these adapters using the following loss function, which combines cross-entropy loss at each exit $E_j$ with distillation loss at each layer $i$. Loss weights coefficients $\alpha_0$ and $\alpha_1$ are employed to balance contribution of corresponding losses.

\begin{align} \label{eq:mr_loss}
L_{\text{m2r2}} = \underbrace{-\alpha_0 \sum_{j=1}^{J} \sum_{t=1}^{T} \log p_{\theta} \left( \hat{y}_t^{E_j} \mid y_{<t}, x \right)}_{\text{cross-entropy loss}} 
+ \underbrace{\alpha_1\sum_{i=1}^{E_{J-1}} \sum_{t=1}^{T} \| \mathbf{p}_{t}^{i} - \mathbf{h}_{t}^{((i - E_{j(i)}) \cdot R_i) + E_{j(i)})} \|^2}_{\text{distillation loss}}.
\end{align}

where $\hat{y}_t^{E_j}$ denotes the predictions from the accelerated residual stream at layer $E_j$ and time step $t$, $y_t$ represents the corresponding ground truth tokens, and $x$ indicates previous context tokens. The distillation loss at each layer $i$ is computed by comparing accelerated residuals at layer $i$ with slow residuals at layer $(i - E_{j(i)}) \cdot R_i + E_{j(i)}$, where $R_i$ denotes the rate of accelerated residuals at layer $i$ while $E_{j(i)}$ represents the most recent gate layer index such that $E_{j(i)} <= i$. \( J \) represents the total number of early exit gates, N denotes number of hidden layers and $E_j$ denotes layer index corresponding to gate index $j$ and \( T \) denotes the sequence length. 

In dynamic compute settings, after training of accelerator adapters, we optimize the query, key, and value parameters governing the ARLA routers (see ~\cref{method_arla}) across all exits in parallel on binary cross entropy loss between predicted decision and ground truth exiting decision. The ground truth labels for the router are determined based on whether the application of the final logit head on $\hat{y}_t^{E_j}$ yields the correct next-token prediction. 


% The objective for this optimization is defined by the following loss function:


%TODO are equations required ? 
% \begin{equation} \label{eq:arla_loss_combined}\small
%     L_{\text{arla}} = -\frac{1}{N} \sum_{t=1}^{T} \left( \sum_{j=1}^{E_n} \left[ O_t^{E_j} \log(\hat{O}_t^{E_j}) + (1 - O_t^{E_j}) \log(1 - \hat{O}_t^{E_j}) \right] \right), \quad \text{where} \quad 
%     O_t^{E_j} = \begin{cases} 
%     1, & \text{if } L(\hat{y}_t^{E_j}) = y_t^{E_j} \\
%     0, & \text{otherwise}
%     \end{cases}
% \end{equation}

% where $\hat{O}_t^{E_j}$ represents the binary predicted logits produced by the vertical latent attention router, as described in \cref{sec:arla}, at gate $E_j$ and time step $t$, and $O_t^{E_j}$ denotes the corresponding ground truth labels. The ground truth labels for the router are determined based on whether the application of the logit head on $\hat{y}_t^{E_j}$ yields the correct next-token prediction. The parameters controlling vertical latent attention are trained concurrently to ensure consistency and efficient use of computational resources.

For self-speculative decoding, as described in \cref{method_self_speculative_decoding}, the training objective remains the same as \cref{eq:mr_loss}, but with the number of intervals set to $J = 1$ and the rate of residual transformation set to $R_n = N/k$, where the first $k$ layers generate speculative candidate tokens. In the context of Ahead-of-Time Expert Loading for Mixture-of-Experts (MoE) models (see \cref{method_aot_expert_loading}), setting the rate of residual transformation to $R_n = 2$ typically offers a good trade-off between the accuracy of expert speculation and AoT pre-loading of experts. 

% Thus, we set $J = 1$ and $E_1 = 16$.


~\subsection{FLOPs Optimization} \label{sec:flops_optimization}

Naively implemented, M2R2 incurs higher FLOP overhead compared to traditional speculative decoding and early exiting approaches such as ~\cite{medusa, schuster2022confident, Tang2024}. However, modern accelerators demonstrate compute bandwidth that exceeds memory access bandwidth by an order of magnitude or more~\cite{databricksLLMInference2023, jouppi2021ten}, meaning increased FLOPs do not necessarily translate to increased decoding latency. Nevertheless, to ensure fair comparison and efficiency in compute bound scenarios, we introduce targeted optimizations.

~\textbf{Attention FLOPs Optimization} For medium-to-long context lengths, attention computation dominates FLOPs in the self-attention layer, surpassing the contribution from MLP layers. Specifically, matrix multiplications involving queries, cached keys, and cached values scale with $l_{kv} * l_{q}$ where $l_{kv}$ denotes previous context length and $l_q$ denotes current query length. Since M2R2 pairs accelerated residuals with slow residuals, a naive implementation results in twice the FLOPs consumption compared to a standard attention layer. To address this, we limit the attention of accelerated residual stream to selectively attend to the top-k most relevant tokens, identified by the slow residual stream based on top attention coefficients\footnote{We set to k = 64 and attend to top 64 tokens as identified by the slow residual stream.}. This is possible since slow and accelerated residual streams are processed in same forward pass and accelerated streams have access to attention coefficients of slow stream. Note that, the faster residual stream still retains the flexibility to assign distinct attention coefficients to these tokens. Furthermore, we design the faster residual stream to employ only 8 attention heads, compared to the 32 heads used in the slow residual stream of the Phi-3 model, reducing query, key, value, and output projection FLOPs by a factor of 1/4. ~\cref{fig:m2r2_num_heads_ablation} indicates effect of using a slicker stream on alignment. As depicted, using $\hat{n}_h = 8$ offers a good trade-off between alignment and FLOPs overhead. 

~\textbf{MLP FLOPs Optimization} The accelerator adapters operating on the accelerated residual stream are intentionally designed with lower rank than their counterparts in the base model. This reduces FLOP overhead by a factor proportional to $hiddenSize / rank$. Additionally, since the faster residual stream uses only 8 attention heads (compared to 32 in the slow residual stream of Phi-3), the subsequent MLP layers process a smaller set of activations, further reducing FLOPs by another factor of 1/4.

These optimizations significantly reduce the FLOP overhead per speculative draft generation, as illustrated in ~\cref{fig:flops_optmization}. Notably, while traditional early-exiting speculative approaches such as DEED require propagating the full slow residual state through the initial layers, incurring substantial computational costs, M2R2 achieves efficient token generation via slimmer, low-rank faster residual streams. In contrast, Medusa introduces considerable FLOP overhead due to per-head computations scaling with $d^2+dv$\footnote{Here $d$ denotes hidden state dimension while $v$ denotes vocab size.}, whereas M2R2 employs low-rank layers for both MLP and language modeling heads, maintaining computational efficiency. All experiments involving the M2R2 approach, as detailed in ~\cref{sec:experiments}, are conducted using these FLOPs optimizations.









% \[
% O_t^{E_j} = 
% \begin{cases} 
% 1, & \text{if } L(\hat{y}_t^{E_j}) = y_t^{E_j} \\
% 0, & \text{otherwise}
% \end{cases}
% \]




%add distillation
% We train accelerator adapters described in \cref{m2r2_method} to accelerate residual streams on next token prediction all in parallel since there are no gradient conflict issues as described in \cref{sec:grad_conflict}.

% \begin{align} \label{eq:mr_loss}
% L_{mr} =  & -\sum_{j = 1}^{E_n} (\sum_{t=1}^{T}\log p_{\theta} (\hat{y}_t^{E_j} | \hat{y}_{<t}, x)) \nonumber
% \end{align}

% where $\hat{y_t^{E_j}}$ denotes predicted logits obtained from accelerated residual stream at gate $E_j$ and time-step $t$ while $y_t^{E_j}$ denotes corresponding truth tokens. 

% Upon training of adapters responsible for accelerating residual streams, we train query, key, value parameters responsible for vertical latent attention of all gates in parallel as

% \begin{equation} \label{eq:arla_loss}
%     L_{arla} = -\frac{1}{N} (\sum_{t=1}^{T}(1\sum_{j=1}^{E_n} \left[ O_t^{E_j} \log(\hat{O}_t^{E_j}) + (1 - o_t^{E_j}) \log(1 - \hat{o_t}_{E_j}) \right]))
% \end{equation}

% where $\hat{O_t^{E_j}}$ denotes binary predicted logits obtained from vertical latent attention router described in \cref{sec:arla} at gate $E_j$ and timestep $t$ while $O_t^{E_j}$ denotes corresponding truth label. Truth labels for router are obtained by computing whether logit head application on $\hat{y}_t^j$ results in true next token prediction. Formally speaking, 

% $O_t^{E_j} = 1 if L(\hat{y_t^{E_j}}) == y_t^{E_j} , 0 otherwise$. 

% Parameters responsible for vertical latent attention are also trained in parallel as well. 

%todo: training slow and fast residuals together and distillation can be two training mdoes. 
%Distillation can be an ablation. 




% Although transformer decoding is memory bound on most mainstream accelerators, there could be scenarios where flop savings are crucial. For instance, on on-device settings power consumption is directly correlated with flops per decoding step and reducing flops does help with overall energy consumption. Vanilla early exiting methods help with flop reduction but suffer from mismatch between training and inference due to early exited tokens. If token at decoding step $t$, $T_t$ exited at layer $E_i$, while token $T_{t+k}$ exits at layer $E_j$ such that $E_i < E_j$, hidden state $H_{t+k}l$ does not have corresponding hidden state $H_tl$ to attend to where $E_i < l <= E_j$. One solution that's often used in literature is to rely on last hidden state available, $H_t{E_j}$, however it tends to be sub-optimal and does affect generation quality \cite{ref}.  To alleviate this mismatch while reducing flops, we train router such that attention mask between token $T_{t+k}$ and token $T_{<t+k}$ is given by: 

% \begin{equation}
%     a_{T_{{t+k}{T_{<t+k}}} = 1 if  E_{T_{<t+k}} >= E{T_{t+k}}
%     else 0
% \end{equation}

% This attention mask enables router to account for exited tokens and get trained accordingly. Since attention mechanism during decoding remains exactly same as that during training, impact on generation quality tends to be minimal as noted in \cref{fig:gen_auality_with_and_without_recompute_attention_show_flops}.  Although MoD does not suffer from training and inference mismatch, we observe that it suffers from discountinuity between pre-training and super-vised fine-tuning resulting in sub-optimal perplexity. On the other hand, our method doesn't not require pre-training , doesn't suffer from discountinuity, and achieves much better perplexity in super-vised fine-tuning and instruction tuning setups as shown in \cref{fig:Mod_vs_m2r2_loss_curves}.






% Our techniques are directly applicable in such scenarios.    




%expert loading with cuda streams in experiments
\begin{figure*}[!h]
    \centering
    \begin{subfigure}[b]{0.8\linewidth}
        \centering
        \includegraphics[width=0.45\linewidth]{images/residual/text/CIReVL_Recall5.png}
        \hfil
        \includegraphics[width=0.45\linewidth]{images/residual/text/pic2word_recall5.png}
        \caption{\textbf{PDV-T}: Impact of $\alpha$ scaling on composed text embeddings}
        \label{fig:residual_text_sub}
    \end{subfigure}
    
    \begin{subfigure}[b]{0.8\linewidth}
        \centering
        \includegraphics[width=0.45\linewidth]{images/residual/image/CIReVL_Recall5.png}
        \hfil
        \includegraphics[width=0.45\linewidth]{images/residual/image/pic2word_recall5.png}
        \caption{\textbf{PDV-I}: Impact of $\alpha$ scaling on composed image embeddings}
        \label{fig:residual_image_sub}
    \end{subfigure}
    
    \begin{subfigure}[b]{0.8\linewidth}
        \centering
        \includegraphics[width=0.45\linewidth]{images/residual/fusion/CIReVL_Recall5.png}
        \hfil
        \includegraphics[width=0.45\linewidth]{images/residual/fusion/pic2word_recall5.png}
        \caption{\textbf{PDV-F}: Impact of varying $\beta$ with on composed fused embeddings}
        \label{fig:residual_fusion_sub}
    \end{subfigure}
    \caption{Impact of changing $\alpha$/$\beta$ on Recall@5 performance across different PDV applications. For each row, results are shown for the CIReVL (left) and Pic2Word (right) baseline methods.}
    \label{fig:residual_all}
\end{figure*}

\section{Experiments} 
\label{sec:exp}
\noindent\textbf{Implementation Details.} We utilize the official implementations of four ZS-CIR baseline methods: CIReVL\footnote{https://github.com/ExplainableML/Vision\_by\_Language} and LDRE \footnote{https://github.com/yzy-bupt/LDRE} as representative caption-based feature extraction approaches and Pic2Word\footnote{https://github.com/google-research/composed\_image\_retrieval} and SEARLE\footnote{https://github.com/miccunifi/SEARLE} as representative pseudo tokenization-based methods. All feature extraction processes follow the original implementations provided by these baseline methods. However, to calculate $\Delta_{PDV}$, we need text embeddings without prompts, which are not provided in the original implementations. For CIReVL and LDRE, we obtain these embeddings by passing the generated image captions directly to CLIP. For Pic2Word and SEARL, we construct the base text embedding by passing the phrase ``a photo of $\langle$token$\rangle$" to CLIP, where $\langle$token$\rangle$ represents the extracted image token obtained via text inversion.

\noindent\textbf{Datasets and Base Vision-Language Models.} Following previous work, we evaluated our method on a suite of datasets including Fashion-IQ \cite{wu2021fashion}, CIRR \cite{liu2021image} and CIRCO \cite{baldrati2023zero}. Our proposed method is a plug-and-play approach requiring no additional training, leveraging only pre-trained models. For feature extraction, we use three CLIP variants: ViT-B/32, ViT-L/14, and ViT-G/14, and used the same pre-trained weights as used by the baseline methods. For image tokenization, we employ the pre-trained Pic2Word model. 

\subsection{Effect of Using the PDV}
We now explore the impact of the three proposed uses of the PDV: Using the PDV to augment text queries (PDV-T, see Sec. \ref{sec:exp1}), using the PDV to augment image queries (PDV-I, see Sec. \ref{sec:exp2}), and using the PDV in queries that fuse image and text data (PDV-F, see Sec. \ref{sec:exp3}).

\begin{table*}
	\footnotesize
	\centering
	\begin{tabular}{l|l|c|c|c|cccccccc}
		\hline
		\textbf{Fashion-IQ} & & & & & \multicolumn{2}{c}{\textbf{Shirt}} & \multicolumn{2}{c}{\textbf{Dress}} & \multicolumn{2}{c}{\textbf{Toptee}} & \multicolumn{2}{c}{\textbf{Average}} \\ \hline
		Backbone & Method& $\beta$ & $\alpha_{I}$& $\alpha_{T}$ & R@10 & R@50 & R@10 & R@50 & R@10 & R@50 & R@10 & R@50 \\
		\hline
		\multirow{6}{*}{ViT-B/32} %
		& SEARLE & - & - & - & 24.14 & 41.81 & 18.39 & 38.08 & 25.91 & 47.02 & 22.81 & 42.30 \\
		& SEARLE + \textbf{PDV-F} & 0.9 & 1.1 & 0.9 & \hli{24.83} & 41.71 & \hli{20.13} & \hli{41.40} & \hli{25.96} & \hli{47.17}  & \hli{23.64} & \hli{43.43} \\
		& CIReVL \textdagger &- & -& -& 28.36 & 47.84 & 25.29 & 46.36 & 31.21 & 53.85 & 28.29 & 49.35 \\
		& CIReVL + \textbf{PDV-F} & 0.75 & 1.4 & 1.4 & \hlb{32.88} & \hlb{52.80} & \hlb{32.67} & \hlb{54.49} & \hlb{38.91} & \hlb{61.81} & \hlb{34.82} & \hlb{56.37} \\
		& LDRE \textdagger & - & - & - & 27.38 & 46.27 & 19.97 & 41.84 & 27.07 & 48.78 & 24.81 & 45.63 \\
		& SEIZE \textdagger & - & - & - & \underline{29.38} & \underline{47.97} & \underline{25.37} & \underline{46.84} & \underline{32.07} & \underline{54.78} & \underline{28.94} & \underline{49.86} \\
		\hline
		\multirow{8}{*}{ViT-L/14} & Pic2Word & & & & 25.96 & 43.52 & 19.63 & 40.90 & 27.28 & 47.83 & 24.29 & 44.08 \\
		& Pic2Word + \textbf{PV-F} & 0.8 & 1.0 & 1.0 & \hli{28.21} & \hli{44.55} & \hli{20.92} & \hli{42.24} & \hli{29.02} & \hli{48.90}& \hli{26.05} & \hli{45.23}\\
		& SEARLE & - & - & - & 26.84 & 45.19 & 20.08 & 42.19 & 28.40 & 49.62 & 25.11 & 45.67 \\
		& SEARLE +\textbf{PDV-F} & 0.8 & 1.2 & 1.0 & \hli{28.66} & \hli{46.76} & \hli{23.60} & \hli{46.41} & \hli{31.00} & \hli{52.32} & \hli{27.75} & \hli{48.50} \\
		& CIReVL \textdagger & & & & 29.49 & 47.40 & 24.79 & 44.76 & 31.36 & 53.65 & 28.55 & 48.57 \\
		
		& CIReVL + \textbf{PDV-F} & 0.55 & 1 & 1.3 & \hlb{37.78} & \hlb{54.22} & \hlb{33.61} & \hlb{56.07} & \hlb{41.61} & \hlb{62.16} & \hlb{37.67} & \hlb{57.48} \\
		& LinCIR & - & - & - & 29.10 & 46.81 & 20.92 & 42.44 & 28.81 & 50.18 & 26.82 & 46.49 \\
        & SEIZE & -& -& -& \underline{33.04} & \underline{53.22} & \underline{30.93} & \underline{50.76} & \underline{35.57} & \underline{58.64} & \underline{33.18} & \underline{54.21} \\
		\hline
        \multirow{6}{*}{ViT-G/14} & Pic2Word  & - & - & - & 33.17 & 50.39 & 25.43 & 47.65 & 35.24 & 57.62 & 31.28 & 51.89\\
         & SEARLE  & - & - & - & 36.46 & 55.35 & 28.16 & 50.32 & 39.83 & 61.45 & 34.81 & 55.71\\
		  & CIReVL \textdagger & -& -& -& 33.71 & 51.42 & 27.07 & 49.53 & 35.80 & 56.14 & 32.19 & 52.36 \\
		& CIReVL + \textbf{PV-F} & 0.6 & 1.4 & 1.4 & \hli{41.90} & \hli{58.19} & \hlb{40.70} & \hlb{62.82} & \underline{\hli{48.09}}& \hli{67.77}& \underline{\hli{43.56}}& \hli{62.93}\\
        & LinCIR & - & - & - & \textbf{46.76} & \underline{65.11} & 38.08& 60.88& \textbf{50.48}& \underline{71.09}& \textbf{45.11} & \underline{65.69}\\
        & SEIZE & - & - & - & \underline{43.60} & \textbf{65.42}& \underline{39.61} & \underline{61.02} & 45.94& \textbf{71.12}& 43.05& \textbf{65.85}\\
		\hline
	\end{tabular}
	\caption{Average recall for different methods on Fashion-IQ validation dataset. \textdagger~denotes that numbers are taken from the original paper.}
	\label{tab:fashion_iq_results}
\end{table*}


\begin{table*}
	\centering
	\footnotesize
	\setlength{\tabcolsep}{4pt}
	\begin{tabular}{ll|c|c|c|cccc|cccc|ccc}
		\hline
		\multicolumn{2}{c|}{\textbf{Dataset}} & & & &  \multicolumn{4}{c|}{\textbf{CIRCO}} & \multicolumn{7}{c}{\textbf{CIRR}} \\
		\hline
		\multicolumn{2}{c|}{Metric} & & & & \multicolumn{4}{c|}{mAP@k} & \multicolumn{4}{c|}{Recall@k} &\multicolumn{3}{c}{$R_s$@k} \\
		\cline{3-16}
		Arch & Method & $\beta$ & $\alpha_I$ & $\alpha_T$ & k=5 & k=10 & k=25 & k=50 & k=1 & k=5 & k=10 & k=50 & k=1 & k=2 & k=3 \\
		\hline
		\multirow{8}{*}{ViT-B/32} 
		& PALAVRA\cite{cohen2022my} \textdagger & -& -& -& 4.61 & 5.32 & 6.33 & 6.80 & 16.62 & 43.49 & 58.51 & 83.95 & 41.61 & 65.30 & 80.94 \\
		& SEARLE \textdagger & -& -&- & 9.35 & 9.94 & 11.13 & 11.84 & 24.00 & 53.42 & 66.82 
		& 89.78 & 54.89 & 76.60 & 88.19 \\
		& SEARLE + \textbf{PDV-F} & 0.9 & 1.4 & 1.2 & \hli{9.99} & \hli{10.50}  & \hli{11.70} & \hli{12.40} & \hli{24.53} & \hli{53.71} & \hli{67.33} & \hli{89.81} & \hli{56.94} & \hli{78.05} & \hli{88.99} \\
		&CIReVL \textdagger & - & - & -& 14.94 & 15.42 & 17.00 & 17.82 & 23.94 & 52.51 & 66.00 & 86.95 & 60.17 & 80.05 & 90.19 \\
		& CIReVL + \textbf{PDV-F} & 0.75 & 1.4 & 1.2 & \hlb{19.90} & \hlb{20.61} & \hlb{22.64} & \hlb{23.52} & \hlb{33.25} & \hlb{64.15} & \hlb{75.23} & \hlb{92.43} & \hlb{65.81} &\underline{\hli{83.76}} &\underline{\hli{92.10}} \\
		& LDRE & -& -& -& 17.81 & 18.04 & 19.73 & 20.67 & 25.69 & 55.52 & 68.77 & 89.86 & 60.10 & 80.58 & 91.04 \\
		& LDRE + \textbf{PDV-F} & 0.75 & 1.4 & 1.4 & \hli{17.80} & \hli{18.78} & \hli{20.61} & \hli{21.56} & \underline{\hli{29.30}} & \underline{\hli{60.39}} & \underline{\hli{72.51}} & \underline{\hli{91.42}} & \hli{63.06} & \hli{82.36} & \hli{91.54} \\
        & SEIZE & -&- &- & \underline{19.04} & \underline{19.64} & \underline{21.55}& \underline{22.49}& 27.47 & 57.42& 70.17 & - & \underline{65.59} & \textbf{84.48}& \textbf{92.77} \\
 		\hline
		\multirow{10}{*}{ViT-L/14}
		& Pic2Word & -& -& -& 6.81 & 7.49 & 8.51 & 9.07 & 23.69 & 51.32 & 63.66 & 86.21 & 53.61 & 74.34 & 87.28 \\
		& Pic2Word + \textbf{PDV-F} & 0.85 & 1.2 & 1.0 & \hli{7.74} &  \hli{8.67} & \hli{9.77} & \hli{10.37} & \hli{23.90} & \hli{51.95} & \hli{64.63} & \hli{87.04} & \hli{53.16}  & \hli{74.07} & \hli{87.08}\\
		& SEARLE \textdagger & - & - & - & 11.68 & 12.73 & 14.33 & 15.12 & 24.24 & 52.48 & 66.29 & 88.84 & 53.76 & 75.01 & 88.19 \\
		& SEARLE + \textbf{PDV-F} & 0.85 & 1.4 & 1.2 & \hli{12.58} & \hli{13.57} & \hli{15.30} & \hli{16.07} & \hli{25.64} & \hli{53.61} & \hli{66.58} & \hli{88.55} & \hli{55.83} & \hli{76.48} & \hli{88.53} \\
		& CIReVL \textdagger & -& -& -& 18.57 & 19.01 & 20.89 & 21.80 & 24.55 & 52.31 & 64.92 & 86.34 & 59.54 & 79.88 & 89.69 \\
		& CIReVL + \textbf{PDV-F} & 0.75 & 1.4 & 1.2 & \hlb{25.67} & \hlb{26.61} & \underline{\hli{28.81}} & \hlb{29.95} & \hlb{36.24} & \hlb{66.17} & \hlb{76.96} & \hlb{92.29} & \hlb{68.07} & \hlb{85.35} & \hlb{93.47} \\
		& LDRE & -& -& -& 22.32 & 23.75 & 25.97 & 27.03 & 26.68 &55.45  & 67.49 & 88.65 & 60.39 & 80.53 & 90.15 \\
		& LDRE + \textbf{PDV-F} & 0.75 & 1.4 & 1.4 & \hli{25.23} & \hli{26.52} & \hlb{28.94} & \hlb{29.95} & \underline{\hli{30.16}} & \underline{\hli{59.98}} & \underline{\hli{71.90}} & \underline{\hli{90.87}} & \hli{63.66} & \hli{82.87} & \hli{91.57} \\

        & LinCIR & - & - & - &12.59 &13.58 &15.00 &15.85 &25.04 &53.25 &66.68 & - &57.11 &77.37 &88.89\\
        & SEIZE & -& -& -& 24.98 & 25.82 &28.24 &\underline{29.35}& 28.65 &57.16& 69.23& - &\underline{66.22} &\underline{84.05} &\underline{92.34} \\
        

        
		\hline
		\multirow{7}{*}{ViT-G/14} & CIReVL \textdagger & -& -& -& 26.77 & 27.59 & 29.96 & 31.03 & 34.65 & 64.29 & 75.06 & 91.66 & 67.95 & 84.87 & 93.21 \\

		& CIReVL + \textbf{PDV-F} & 0.75 & 1.4 & 1.2 & \hli{30.02} & \hli{31.46} & \hli{34.01} & \hli{35.08} & \hli{38.15} &\hli{67.93} & \hli{77.90} & \hli{92.77} & \hli{69.37} & \hli{85.37} & \hli{93.45}  \\
		
		& LDRE & -& -& -& \underline{33.30} & \underline{34.32} & \underline{37.17} & \underline{38.27} & 37.40 & 66.96 & 78.17 & 93.66 & 68.84 & 85.64 & 93.90 \\
		& LDRE + \textbf{PDV-F} & 0.75 & 1.4 & 1.4 & \hlb{34.88} & \hlb{36.41} & \hlb{39.12} & \hlb{40.23} & \hlb{42.51} & \hlb{72.22} & \hlb{81.71} & \hlb{94.94} & \underline{\hli{72.39}} & \underline{\hli{88.34}} & \underline{\hli{94.80}} \\
        & SEARLE & - & - & - & 13.20 &13.85 &15.32 &16.04 & 34.80 & 64.07 & 75.11 &-&68.72 &84.70 &93.23 \\
        & LinCIR & - & - & - & 19.71 &21.01 &23.13 &24.18 &35.25 &64.72 &76.05 & - &63.35 &82.22 &91.98 \\
        & SEIZE & -& -& -& 32.46 & 33.77 &36.46 &37.55 &\underline{38.87} & \underline{69.42} & \underline{79.42} & -&\textbf{74.15} & \textbf{89.23} & \textbf{95.71} \\
		\hline
	\end{tabular}
	\caption{Performance comparison on CIRCO and CIRR test datasets. As in previous works, for CIRCO, mAP@k is reported, while for CIRR both Recall@k and $R_s$@k metrics are used. \textdagger~denotes that numbers are taken from the original paper.}
	\label{tab:circo_cirr_results}
\end{table*}

\noindent{\textbf{Analysing the PDV for Text (PDV-T)}}
\label{sec:exp1}
To investigate how scaling the prompt vector, $\Delta_{PDV}$, affects retrieval performance with composed text embeddings, we conducted experiments using two zero-shot approaches (CIReVL and Pic2Word) with different backbone networks across three datasets. We evaluated the performance by varying the scaling parameter, $\alpha$ (Eq. \ref{eqn:text_embedding}), from -0.5 to 3 by an interval of 0.1.

The results are presented in Figure \ref{fig:residual_text_sub}. To account for scale variations across different experiments, we report relative recall values, where a baseline of zero is established at $\alpha=1$. As shown in Figure \ref{fig:residual_text_sub}, varying $\alpha$ leads to significant changes in relative recall performance\footnote{See supplementary material for Recall@10 and Recall@50 figures}. Our analysis reveals method-specific patterns across datasets. With CIReVL, increasing $\alpha$ improves relative recall on both FashionIQ and CIRCO datasets. In contrast, Pic2Word shows no significant improvement on FashionIQ and CIRR when varying $\alpha$, while CIRCO's performance improves when $\alpha$ is reduced to 0.8-1.0. This divergent behavior is fundamentally linked to each method's ability to generate an accurate $\Delta_{PDV}$. As demonstrated in Tables \ref{tab:fashion_iq_results} and \ref{tab:circo_cirr_results}, CIReVL consistently outperforms Pic2Word across various benchmarks, indicating its superior ability to generate a more accuraute composed query, and thus a more accurate $\Delta_{PDV}$. Consequently, increasing $\alpha$ yields greater benefits for CIReVL compared to Pic2Word.

We visualize the top-5 retrieval results using CIReVL with a ViT-B-32 backbone across three datasets (one reference image from each) under varying $\alpha$ values, as shown in Figure \ref{fig:residual_qual}\red{a}. As $\alpha$ increases, the retrieved results show stronger alignment with the prompt. Conversely, when $\alpha$ exceeds 1, the results include semantically related but unseen variations, while $\alpha$ values below 0.5 yields results opposite to the prompt's intent. For instance, ``brighter blue and sleeveless" retrieves ``dark blue with sleeves," ``plain background" yields ``natural/dark background," and ``young boy" returns ``adult" images.





\noindent{\textbf{Analysing the PDV for Image (PDV-I)}}
\label{sec:exp2}
To evaluate whether $\Delta_{PDV}$ enhances the retrieval performance of image embeddings, we conducted experiments following the protocol described in Section~\ref{sec:exp1}. We modified image embeddings by adding $\Delta_{PDV}$ scaled with $\alpha$ values ranging from -0.5 to 2.0, where $\alpha=0$ represents the original image-only embeddings. As shown in Figure \ref{fig:residual_image_sub}, Recall@K exhibits a positive correlation with $\alpha$ for values below 1. This upward trend continues until $\alpha=2.0$ for CIReVL, while Pic2Word's performance peaks when $\alpha$ reaches 1.4.  The performance of PDV-I was evaluated on the CIRR and CIRCO datasets by comparing it with other visual embedding-based methods, as detailed in Table \ref{tab:circo_cirr_results_pdv-I}. The results reveal that PDV-I achieved marginal improvements over existing approaches.

Following the methodology in Section~\ref{sec:exp1}, we conduct similar visualizations, with results shown in Figure \ref{fig:residual_qual}\red{b}. As with PDV-T, increasing $\alpha$ leads to stronger alignment between retrieved results and the prompt. When $\alpha$ exceeds 0.5, the results exhibit semantic relationships to the query, while $\alpha$ values below 0.5 yield results opposing the prompt's intent.
Notably, PDV-I's top retrievals demonstrate higher visual similarity to reference images compared to PDV-F, as evidenced by the preserved design elements in the clothing item (left) and laptop (middle). This characteristic is particularly valuable for applications include fashion search \cite{wu2021fashion} and logo retrieval \cite{tursun2019component}, where visual similarity plays a crucial role.



\begin{figure*}[!tbh]
	\centering
	\includegraphics[width=0.825\linewidth]{images/qualitative/PV_qual_all_mini.pdf}
	\caption{Visualisation of the impact of $\alpha$/$\beta$ scaling on top-5 retrieval results. CIReVL with ViT-B-32 Clip model is the baseline method used. Representative examples with prompts from three datasets: FashionIQ (left), CIRR (middle), and CIRCO (right) are shown at the top. \textbf{\textcolor{boxgreen}{Green}} and \textbf{\textcolor{boxblue}{blue}} bounding boxes indicate true positives and near-true positives, respectively.}
	\label{fig:residual_qual}
	
\end{figure*}

\noindent{\textbf{Analysing PDV Fusion (PDV-F)}}
\label{sec:exp3}
Finally, we evaluate the effectiveness of fusing image and text-composed embeddings by varying the fusion parameter, $\beta$, from 0 to 1 while maintaining $\alpha=1$
for both PDV-I and PDV-F. At $\beta=0$, the model relies solely on composed image embeddings, while at $\beta=1$, it uses only composed text embeddings. As shown in Figure \ref{fig:residual_fusion_sub}, the fusion of both embeddings consistently outperforms using either embedding type alone. Optimal retrieval performance is typically achieved when $\beta$ is between 0.4 and 0.8.

We similarly visualize the top-5 retrieved results across different $\beta$ values. As shown in Figure \ref{fig:residual_qual}\red{c}, when $\beta$ is small, the retrieved results maintain high visual similarity to the reference image. Conversely, as $\beta$ exceeds 0.5, the results demonstrate stronger semantic alignment with the prompt.



\subsection{ZS-CIR Benchmark Comparison}






\begin{table*}
	\centering
	\footnotesize
	\setlength{\tabcolsep}{4pt}
	\begin{tabular}{l|l|c|cccc|cccc|ccc}
		\hline
		\multicolumn{2}{c|}{\textbf{Dataset}} & & \multicolumn{4}{c|}{\textbf{CIRCO}} & \multicolumn{7}{c}{\textbf{CIRR}} \\
		\hline
		& Metric & & \multicolumn{4}{c|}{mAP@k} & \multicolumn{4}{c|}{Recall@k} & \multicolumn{3}{c}{$R_s$@k} \\
		\cline{2-14}
		Arch & Method & $\alpha_I$ & k=5 & k=10 & k=25 & k=50 & k=1 & k=5 & k=10 & k=50 & k=1 & k=2 & k=3 \\
		\hline
		\multirow{6}{*}{ViT-B/32} 
		& Image-only \textdagger & - & 1.34 & 1.60 & 2.12 & 2.41 & 6.89 & 22.99 & 33.68 & 59.23 & 21.04 & 41.04 & 60.31 \\
		& Text-only \textdagger & - & 2.56 & 2.67 & 2.98 & 3.18 & 21.81 & 45.22 & 57.42 & 81.01 & 62.24 & 81.13 & 90.70 \\
		& Image + Text \textdagger & - & 2.65 & 3.25 & 4.14 & 4.54 & 11.71 & 35.06 & 48.94 & 77.49 & 32.77 & 56.89 & 74.96 \\
		& SEARLE + \textbf{PDV-I} & 1.5 & 4.77 & 5.23  & 6.31 & 6.82 & 16.65 & 42.53 & 55.16 & 81.42 & 44.68 & 67.78 & 82.94\\
		& CIReVL + \textbf{PDV-I} & 2.0 & \textbf{10.29 }& \textbf{10.80} & \textbf{12.23} & \textbf{12.93} & \textbf{27.18} & \textbf{56.53} & \textbf{67.76} & \textbf{87.64} & \textbf{59.81} & \textbf{79.59} & \textbf{90.15}\\
		& LDRE + \textbf{PDV-I} & 2.0 & 8.00 & 8.88 & 10.06 & 10.72 & 23.37 & 51.21 & 63.69 & 85.57 & 55.57 & 76.63 & 88.15\\
		\hline
	\end{tabular}
	\caption{PDV-I performance on CIRCO and CIRR test datasets. Note that the image-only approach utilizes the visual embedding of the reference image, whereas the text-only approach employs the text embedding of the prompt.}
	\label{tab:circo_cirr_results_pdv-I}
\end{table*}

We evaluated PDV-F alongside four baseline approaches (CIReVL, LDRE, Pic2Word, and SEARLE) across three benchmarks. Notably, CIReVL was tested with three different backbones on three datasets, as its models and intermediate results are publicly available. However, for the remaining methods, we conducted partial evaluations due to limited open-source availability or restricted support.

The numerical results are presented in Tables \ref{tab:fashion_iq_results} and \ref{tab:circo_cirr_results}.
On the FashionIQ benchmark, PDV-F yields substantial improvements for all baseline approaches, with CIReVL showing particularly strong gains that scale with backbone size. Similarly, all methods demonstrate significant performance improvements on CIRCO and CIRR datasets. Notably, CIReVL achieves larger improvements compared to other methods, with the most substantial gains observed when using small and medium backbone architectures. Our PDV-F implementation within the CIReVL framework consistently outperformed other state-of-the-art methods, including LinCIR and SEIZE, across most evaluation metrics. Similar to SEIZE, PDV-F offers the advantage of being entirely training-free; however, unlike SEIZE, it does not significantly increase feature extraction computational costs. While LinCIR demonstrates exceptional inference speed, it lacks the training-free nature of our approach, requiring dedicated model training before deployment.  






\section{Related Work} \label{sec:related}

% \textbf{Adversarial Attack}
\textbf{Attacks on SLAM.} 
%With the rise of machine learning, 
The robustness of computer vision systems is being actively investigated. With the emergence of adversarial images in the digital domain by adding optimized noise directly to images~\cite{szegedy2013intriguing,carlini2017towards}, researchers find that such attacks also exist physically in the real world \cite{eykholt2018robust,song2018physical,zhao2019seeing}. To fill the gap between attacks in the digital and physical worlds, recent studies have demonstrated that attacks on real-world computer vision systems are practical \cite{eykholt2018robust,li2019adversarial,man2020ghostimage,sharif2016accessorize,zhao2019seeing,zhou2018invisible}. However, attacks on traditional computer vision methods such as SLAM are relatively less explored. \cite{yoshida2022adversarial} proposes an attack against the scan matching algorithm in LiDAR-based SLAM, while most SLAMs in AR/VR devices rely on different sensors like RGB/depth cameras and IMUs. \cite{ikram2022perceptual} and \cite{chen2024adversary} mislead visual SLAM by poisoning the images with special patterns, and \cite{wang2021can} causes the camera to fail using infrared light. In our work, we demonstrate attacks on Visual-Inertial SLAM (VI-SLAM) by perturbing the IMU readings, rather than cameras, and showing its impact on XR user experience. 

\textbf{Acoustic Injection Attacks.} Among various physical attacks, acoustic injection attacks are attractive due to their low cost. Son~\etal~\cite{son2015rocking} were the first to introduce acoustic attacks on MEMS gyroscopes, demonstrating how these attacks could lead to sensor denial-of-service and result in drone crashes. WALNUT~\cite{trippel2017walnut} expanded on this by developing output biasing and control attacks that enable precise manipulation of MEMS accelerometer outputs using modulated sound waves. Wang et al.~\cite{wang2017sonic} demonstrated a sonic gun, showcasing the vulnerability of various smart devices (\eg drones and self-balancing vehicles) to acoustic attacks. Tu et al. \cite{tu2018injected} designed side-swing and switching attacks to alter the outputs of MEMS gyroscopes and accelerometers. Furthermore, Ji et al. \cite{ji2021poltergeist} fool the object detectors by applying acoustic attack to the image stabilizers commonly used in modern cameras. However, none of the existing works study the relationship between the acoustic injections and SLAM outputs on recent XR devices. 

% \zijian{Do we need one session about security in AR/VR?}
% \yicheng{TODO}
%\jiasi{cite the AIVR paper (UMass Amherst?) paper is we have not already. They add IMU perturbation but w/o SLAM, iirc} \yicheng{Cited}

\textbf{XR Security and Privacy.} 
%Security and privacy concerns in XR systems have gained significant attention. 
For single-user XR systems, researchers have demonstrated various side-channel attacks to extract sensitive information (\eg keystrokes) through video feeds~\cite{ling2019know}, head movements~\cite{nair2023unique, slocum2023going}, architectural hints~\cite{zhang2023its,shang2020arspy}, power usage~\cite{li2024dangers}, and EM side-channel leakages~\cite{al2021vr}. In multi-user XR systems, Su et al.~\cite{su2024remote} use avatar motion data to infer keystrokes in shared VR environments. Slocum et al.~\cite{slocum2024doesn} reveal vulnerabilities in the shared state frameworks of multi-user AR. Similarly, Lebeck et al.~\cite{lebeck2017securing} highlight risks like deceptive virtual objects and emphasize access control for managing shared physical and virtual spaces. Ruth et al.~\cite{ruth2019secure} further propose a secure multi-user AR framework focusing on content sharing and permissions.
Chandio et al.~\cite{chandio2024stealthy} %introduced a multi-modal spatiotemporal attack that 
simultaneously manipulated visual and inertial sensors to disrupt XR pose estimation. However, their study evaluated the attack using offline datasets and assumed the attacker's capability to manipulate IMU data streams through acoustic means, without real experiments. Ours is the first to demonstrate acoustic injection attacks on recent XR devices, like the Hololens 2, in the real world.
 


\section{Conclusion}
\label{sec:con}
We introduce the Prompt Directional Vector (PDV), a simple yet effective approach for enhancing Zero-Shot Composed Image Retrieval. PDV captures semantic modifications induced by user prompts without requiring additional training or expensive data collection. Through extensive experiments across multiple benchmarks, we demonstrated three successful applications of PDV: dynamic text embedding synthesis, composed image embedding through semantic transfer, and effective multi-modal fusion. 

Our approach not only improves retrieval performance consistently, but also provides enhanced controllability through the use of scaling factors. PDV serves as a plug-and-play enhancement that can be readily integrated with existing ZS-CIR methods while incurring minimal computational overhead.

We note that PDV's effectiveness correlates strongly with the underlying method's ability to generate accurate compositional embeddings. This insight suggests promising future research directions, including developing more robust compositional embedding techniques and exploring adaptive scaling strategies for PDV. The simplicity and effectiveness of PDV also open possibilities for its application in multi-prompt composed image retrieval (\ie dialogue-based search) and other multi-modal tasks where semantic modifications play a crucial role.



% \section*{Accessibility}
% Authors are kindly asked to make their submissions as accessible as possible for everyone including people with disabilities and sensory or neurological differences.
% Tips of how to achieve this and what to pay attention to will be provided on the conference website \url{http://icml.cc/}.

% \section*{Software and Data}

% If a paper is accepted, we strongly encourage the publication of software and data with the
% camera-ready version of the paper whenever appropriate. This can be
% done by including a URL in the camera-ready copy. However, \textbf{do not}
% include URLs that reveal your institution or identity in your
% submission for review. Instead, provide an anonymous URL or upload
% the material as ``Supplementary Material'' into the OpenReview reviewing
% system. Note that reviewers are not required to look at this material
% when writing their review.

% Acknowledgements should only appear in the accepted version.
% \section*{Acknowledgements}

% \textbf{Do not} include acknowledgements in the initial version of
% the paper submitted for blind review.

% If a paper is accepted, the final camera-ready version can (and
% usually should) include acknowledgements.  Such acknowledgements
% should be placed at the end of the section, in an unnumbered section
% that does not count towards the paper page limit. Typically, this will 
% include thanks to reviewers who gave useful comments, to colleagues 
% who contributed to the ideas, and to funding agencies and corporate 
% sponsors that provided financial support.

% \section*{Impact Statement}

% Authors are \textbf{required} to include a statement of the potential 
% broader impact of their work, including its ethical aspects and future 
% societal consequences. This statement should be in an unnumbered 
% section at the end of the paper (co-located with Acknowledgements -- 
% the two may appear in either order, but both must be before References), 
% and does not count toward the paper page limit. In many cases, where 
% the ethical impacts and expected societal implications are those that 
% are well established when advancing the field of Machine Learning, 
% substantial discussion is not required, and a simple statement such 
% as the following will suffice:

% ``This paper presents work whose goal is to advance the field of 
% Machine Learning. There are many potential societal consequences 
% of our work, none which we feel must be specifically highlighted here.''

% The above statement can be used verbatim in such cases, but we 
% encourage authors to think about whether there is content which does 
% warrant further discussion, as this statement will be apparent if the 
% paper is later flagged for ethics review.


% In the unusual situation where you want a paper to appear in the
% references without citing it in the main text, use \nocite
\nocite{}
\bibliography{references}
\bibliographystyle{icml2025}


%%%%%%%%%%%%%%%%%%%%%%%%%%%%%%%%%%%%%%%%%%%%%%%%%%%%%%%%%%%%%%%%%%%%%%%%%%%%%%%
%%%%%%%%%%%%%%%%%%%%%%%%%%%%%%%%%%%%%%%%%%%%%%%%%%%%%%%%%%%%%%%%%%%%%%%%%%%%%%%
% APPENDIX
%%%%%%%%%%%%%%%%%%%%%%%%%%%%%%%%%%%%%%%%%%%%%%%%%%%%%%%%%%%%%%%%%%%%%%%%%%%%%%%
%%%%%%%%%%%%%%%%%%%%%%%%%%%%%%%%%%%%%%%%%%%%%%%%%%%%%%%%%%%%%%%%%%%%%%%%%%%%%%%
\newpage
\appendix
\onecolumn

\section{Experimental Setup}\label{app:exp}
\subsection{Datasets}
UnKEBench \cite{UnKE} constructs a dataset containing 1,000 counterfactual unstructured texts, where knowledge is presented in an unstructured and relatively lengthy form, going beyond simple knowledge triplets or linear fact chains. These texts originate from ConflictQA \cite{conflictqa}, a benchmark specifically designed to distinguish LLMs' parameter memory from anti-memory. This approach is crucial for preventing the model from merging knowledge obtained during pretraining with knowledge acquired during the editing process. Moreover, it addresses the key challenge of determining whether the model learns target knowledge during training or editing, ensuring a clear boundary between pretraining knowledge and edited knowledge.

AKEW benchmark \cite{AKEW} considers three aspects: (1) \textit{Structured Facts}: Each structured fact consists of an isolated triplet for editing, sourced from existing datasets or knowledge bases. (2) \textit{Unstructured Facts}: Knowledge is presented in unstructured text form. To enable fair comparisons, each unstructured fact contains the same knowledge update as its corresponding structured fact. Compared to structured facts, unstructured facts exhibit greater complexity in natural language format, as they often encapsulate more implicit knowledge. (3) \textit{Extracted Triplets}: Triplets are extracted from unstructured facts using automated methods to investigate whether they can facilitate knowledge editing methods in handling unstructured knowledge. In this work, we primarily focus on unstructured factual knowledge.

EditEverything dataset integrates question-answering data from multiple domains, forming long and diverse knowledge formats that are more challenging to edit. Specifically, for mathematics, we select longer samples from the Orca-Math dataset \cite{math}, which includes grade school math word problems. For coding, we use the MBPP dataset \cite{code}, which consists of approximately 1,000 crowd-sourced Python programming problems solvable by entry-level programmers, covering programming fundamentals and standard library functionalities. For chemistry, we sample from the Camel-Chemistry dataset \cite{chemistry}, which contains problem-solution pairs generated from 25 chemistry topics, each with 25 subtopics and 32 problems per topic-subtopic pair. Lastly, for the news and poetry categories, since they often contain real-world knowledge that LLMs may already possess, we generate synthetic data using GPT-4o to ensure that the information is not already known by the model.

We present sample instances from the dataset in Figure \ref{fig:sample1}, Figure \ref{fig:sample2}, and Figure \ref{fig:sample3}.

\subsection{Evaluation Metrics} \label{app:exp_metric}
Following previous research on model editing for structured knowledge \cite{ROME, MEND}, existing evaluation metrics primarily focus on triplet-structured knowledge, where the goal is to assess the modification of factual triples (\textit{subject, relation, object}). Specifically, given an LLM $f$, an editing knowledge pair $(x, y)$, an equivalent knowledge query $x_e$, and unrelated knowledge pairs $(x_{loc}, y_{loc})$, three standard evaluation metrics are commonly used:

\textbf{Efficacy.} This metric quantifies the success of modifying the target knowledge in $f_{\mathcal{W}}$. It evaluates whether the edited LLM generates the desired target output $y$ when given the input $x$. Formally, it is defined as:
\begin{equation}
\mathbb{E}\left\{y=\mathop{\arg\max}\limits_{y'}\mathbb{P}_{f}(y'\left|x\right.)\right\}.
\end{equation}

\textbf{Generalization.} This metric assesses whether the model has generalized the newly edited knowledge beyond its specific form. It measures if the LLM correctly produces $y$ when given a semantically equivalent input $x_e$, indicating the degree to which the update propagates correctly across paraphrased or restructured queries:
\begin{equation}
\mathbb{E}\left\{y=\mathop{\arg\max}\limits_{y'}\mathbb{P}_{f}(y'\left|x_e\right.)\right\}.
\end{equation}

\textbf{Specificity.} This metric evaluates whether the editing operation is localized, ensuring that unrelated knowledge remains intact. It measures whether the model's response to an unrelated query $x_{loc}$ remains consistent with its original output $y_{loc}$:
\begin{equation}
\mathbb{E}\left\{y_{loc}=\mathop{\arg\max}\limits_{y'}\mathbb{P}_{f}(y'\left|x_{loc}\right.)\right\}.
\end{equation}

While these metrics are well-suited for structured knowledge editing, they are insufficient for evaluating long-form and diverse-formatted knowledge. Such knowledge is often verbose and complex, making it challenging to assess correctness solely based on Efficacy. In these cases, the model may generate an answer that captures the essential information yet fails an exact-match evaluation. To address this, we primarily follow the existing benchmarks for unstructured knowledge editing, incorporating more flexible evaluation methods suited for long-form responses.

Lexical similarity metrics include BLEU \cite{bleu} and various ROUGE scores (ROUGE-1, ROUGE-2, and ROUGE-L) \cite{rouge}. These are computed based on the \textit{original questions}, \textit{paraphrase question}, and \textit{sub-questions}, providing insights into the lexical and n-gram alignment between the model-generated text and the target answer. These metrics serve as the foundation for assessing the surface-level accuracy of edited content.

Semantic similarity is also considered (Bert Score) \cite{bertscore}, as word-level overlap alone is insufficient to capture the nuanced understanding required by the model. To address this, we utilize embedding-based encoders, specifically the all-MiniLM-L6-v2 model \footnote{https://huggingface.co/sentence-transformers/all-MiniLM-L6-v2}, to measure semantic similarity. This ensures a more balanced evaluation that extends beyond lexical matching, quantifying the depth of the model's comprehension.

\subsection{Baseline Methods}
\begin{itemize}
    \item \textbf{FT-L} \cite{FTw} is a knowledge editing approach that fine-tunes specific layers of the LLM using an autoregressive loss function. We reimplemented this method following the hyperparameter from the original paper.
    
    \item \textbf{MEND} \cite{MEND} is a hypernetwork-based efficient knowledge editing method. It trains a hypernetwork to capture patterns in knowledge updates by mapping low-rank decomposed fine-tuning gradients to LLM parameter modifications, enabling efficient and localized edits. Our implementation follows the original hyperparameter settings and completes training over the full dataset. 
    
    \item \textbf{ROME} \cite{ROME} is a method for modifying factual associations in LLM parameters. It identifies critical neuron activations in MLP layers through perturbation-based knowledge localization and modifies MLP layer weights using Lagrange remainders. Since ROME is not designed for large-scale edits, we follow the original paper’s settings and conduct multiple rounds of single-instance editing for evaluation.
    
    \item \textbf{MEMIT} \cite{MEMIT} extends ROME by enabling batch updates of factual knowledge. It utilizes least squares approximation to modify specific layer parameters across multiple layers, allowing simultaneous updates of large numbers of knowledge facts. We evaluate MEMIT in lifelong editing scenarios using the original paper’s configuration.
    
    \item \textbf{AlphaEdit} \cite{AlphaEdit} is a method designed to mitigate interference in LLM lifelong knowledge editing. It introduces a null-space projection mechanism that ensures parameter updates preserve previously edited knowledge while incorporating new updates. AlphaEdit has demonstrated state-of-the-art (SOTA) performance across multiple evaluation metrics while maintaining strong transferability. We follow the original paper’s hyperparameter configuration in our implementation.
    
    \item \textbf{UnKE} \cite{UnKE} improves knowledge editing by refining both the layer and token dimensions. In the layer dimension, it replaces local key-value storage with a non-local block-based mechanism, enhancing the representation capability of key-value pairs while integrating attention-layer knowledge. In the token dimension, it replaces "term-driven optimization" with "cause-driven optimization," which directly edits the final token while preserving contextual coherence. This eliminates the need for explicit term localization and prevents context loss.
\end{itemize}

\subsection{Implementation Details}
Our AnyEdit and AnyEdit* primarily follow the baseline configurations of MEMIT and UnKE, while other baselines adhere to their original implementation settings. All experiments were conducted on a single A100 GPU (80GB).
\begin{itemize}
    \item \textbf{AnyEdit on Llama3-8B-Instruct:} We select layers 4 to 8 for editing and apply a clamp norm factor of 4. The fact token is defined as the last token. The optimization process involves 25 gradient steps for updating the key-value representations, with a learning rate of 0.5. The loss is applied at layer 31, and we use a weight decay of 0.001. To maintain distributional consistency, we introduce a Kullback-Leibler (KL) regularization term with a factor of 0.0625. Furthermore, we enable hyperparameter $\lambda$ with an update weight of 15,000, using 100,000 samples from the Wikipedia dataset with a data type of float32. The module configurations follow MEMIT, where edits are applied to the MLP down projection layers of the selected transformer blocks. Additionally, for chunked editing, we set a chunk size of 40 tokens with no overlap.
    \item \textbf{AnyEdit on Qwen2.5-7B-Instruct:} Same as the above, except that the loss is applied at layer 27 and the chunk size is set to 50 tokens.
    \item \textbf{AnyEdit* on Llama3-8B-Instruct:} We select layer 7 for editing and apply a clamp norm factor of 4. The fact token is defined as the last token. The optimization process involves updating all parameters in both the attention and MLP layers. The gradient descent process utilizes a learning rate of 0.0002 with 50 optimization steps. For updating key-value representations, we use 25 gradient steps with a learning rate of 0.5. The loss is applied at layer 31, and we use a weight decay of 0.001. To preserve original knowledge, we sample 20 data points to constrain parameter updates. Additionally, for chunked editing, we set a chunk size of 40 tokens with no overlap.
    \item \textbf{AnyEdit* on Qwen2.5-7B-Instruct:} Same as the above, except that the loss is applied at layer 27 and the chunk size is set to 50 tokens.
\end{itemize}

\section{Locate-Then-Edit Paradigm \& Related Proof}
\subsection{Locate-Then-Edit Paradigm}\label{app:model_edit}
Following prior works on model editing, the detailed descriptions of specific methods in this section are based on MEMIT \cite{MEMIT}, AlphaEdit \cite{AlphaEdit} and ECE \cite{ECE}. We adhere to their formulations and methodological explanations to ensure consistency and clarity in presenting these approaches.

The locate-then-edit method primarily focuses on triplet-structured knowledge in the form of $(s, r, o)$, such as modifying $(\text{Olympics}, \text{were held in}, \text{Tokyo})$ to $(\text{Olympics}, \text{were held in}, \text{Paris})$. Given new knowledge $(x_e, y_e)$, a triplet can be represented as $x_e = (s, r)$ and $y_e = o$.

We first refine the auto-regressive language model formulation in Section \ref{sec:method:pre}. Let $f$ be a decoder-only model with $L$ layers, processing input sequence $x = (x_0, x_1, \dots, x_T)$ to predict the next token:
\begin{equation}
    \begin{aligned}
        \vh_t^l(x) &= \vh_t^{l - 1}(x) + \va_t^l(x) + \vm_t^l(x), \\
        \va_t^l &= \text{attn}^l(\vh_0^{l - 1}, \vh_1^{l - 1}, \dots, \vh_t^{l - 1}), \\
        \vm_t^l &= \mW_{\text{out}}^l \sigma(\mW_{\text{in}}^l \gamma(\vh_t^{l - 1}+\va_t^l)),
    \end{aligned}
\end{equation}
where $\vh_t^l$ denotes the hidden state of token $t$ at layer $l$, $\va_t^l$ is the attention output, and $\vm_t^l$ is the feedforward (FFN) output. Here, $\mW_{\text{in}}^l$ and $\mW_{\text{out}}^l$ are weight matrices, $\sigma$ is a nonlinear activation function, and $\gamma$ denotes layer normalization.

\textbf{Key-Value Memory Structure}. Locate-then-edit assumes that factual knowledge is stored in the FFN layers and treats them as linear associative memory \cite{key_value}. Specifically, $\mW_{\text{out}}^l$ is conceptualized as a key-value memory structure:
\begin{equation}
    \begin{aligned}
        \underbrace{\vm_t^l}_{\vv} = \mW_{\text{out}}^l \underbrace{\sigma(\mW_{\text{in}}^l \gamma(\vh_t^{l-1}+\va^l))}_{\vk}. \label{eqapp:define_kv}
    \end{aligned}
\end{equation}
Here, the MLP input-output pair at token $t$ and layer $l$ serves as the key-value pair. Casual Tracing is typically used to locate the target token and layer by injecting Gaussian noise into hidden states and incrementally restoring them to analyze output recovery. For more details, please refer to ROME \cite{ROME}.

\textbf{Computing Key-Value.} For editing knowledge $(x_e, y_e)$, we compute its corresponding key-value pair $(\vk^*, \vv^*)$. The key $\vk^*$ is derived via forward propagation of $x_e$, while the value $\vv^*$ is optimized using gradient descent:
\begin{equation}
    \vv^* = \vv + \arg \min_{\bm{\delta}^l} \left( -\log \mathbb{P}_{f(\vh_t^l + \bm{\delta}^l)} [y_e \mid x_e] \right).
\end{equation}
Here, $f(\vh_t^l + \bm{\delta}^l)$ represents the model output when the FFN output $\vh_t^l$ is replaced with $\vh_t^l + \bm{\delta}^l$. 

Methods such as ROME \cite{ROME}, MEMIT \cite{MEMIT}, and AlphaEdit \cite{AlphaEdit} focus on triplets $(s, r, o)$, selecting the last token of the subject $s$ as the target token. In contrast, UnKE \cite{UnKE} extends to unstructured text, using the last token of $x_e$ as the target.

To insert new knowledge $(\vk^*, \vv^*)$ into the key-value memory, we solve the constrained least squares problem:
\begin{align*}
    \min_{\hat{\mW}} &\quad \left\lVert \hat{\mW}\mK - \mV \right\rVert \\
    \text{s.t.} &\quad \hat{\mW}\vk^* = \vv^*.
\end{align*}
The final parameter update can be computed via ROME/MEMIT/AlphaEdit's closed-form solution or UnKE's gradient-based optimization.

For clarity, let $\tilde{\mW}$ denote the edited weight of $\mW_{\text{out}}^l$ in the MLP, and let $\mW$ represent its original weight. The final parameter update can be computed using the closed-form solutions of ROME/MEMIT/AlphaEdit or the gradient-based optimization method in UnKE.

\textbf{Weights Update in ROME.} The ROME method derives a closed-form solution to the constrained least-squares problem for updating MLP parameters:
\begin{equation}
    \tilde{\mW} = \mW + \frac{(\vv^\ast - \mW\vk^\ast) (\mC^{-1} \vk^\ast) ^ {T}}{(\mC^{-1} \vk^\ast) ^ {T} \vk^\ast},
\end{equation}
where $\mC = \mK \mK^T$. The matrix $\mC$ is estimated using 100,000 samples of hidden states $\vk$ obtained from tokens sampled in-context from the entire Wikipedia dataset.

\textbf{Weights Update in MEMIT.} Since the above solution updates only a single knowledge sample at a time, MEMIT improves upon it by avoiding Lagrange multipliers and instead using a relaxed constraint formulation. The problem is reformulated by maintaining a factual set $\{\mK_1, \mV_1\}$ containing $u$ new associations while preserving the original set $\{\mK_0, \mV_0\}$ with $n$ existing associations:
\begin{equation}
\begin{gathered}
    \mK_0 = \left[\vk_1 \mid \vk_2 \mid \dots \mid \vk_n\right], \quad \mV_0 = \left[\vv_1 \mid \vv_2 \mid \dots \mid \vv_n\right], \\
    \mK_1 = \left[\vk^\ast_{n+1} \mid \vk^\ast_{n+2} \mid \dots \mid \vk^\ast_{n+u}\right], \quad \mV_1 = \left[\vv^\ast_{n+1} \mid \vv^\ast_{n+2} \mid \dots \mid \vv^\ast_{n+u}\right].
\end{gathered}
\end{equation}
Here, $\vk$ and $\vv$ are defined as in Eq.~\ref{eqapp:define_kv}, and their subscripts denote knowledge indices. The objective function is given by:
\begin{equation}
    \tilde{\mW} \triangleq \argmin_{\hat{\mW}} \left( \sum_{i=1}^{n} \left\| \hat{\mW} \vk_i - \vv_i \right\|^2 + \sum_{i=n+1}^{n+u} \left\| \hat{\mW} \vk_i - \vv^\ast_i \right\|^2 \right).
\end{equation}
Applying the normal equation \citep{normal_equation}, the closed-form solution is:
\begin{equation}
    \tilde{\mW} = \left( \mV_1 - \mW \mK_1 \right) \mK_1^T \left( \mK_0 \mK_0^T + \mK_1 \mK_1^T \right)^{-1} + \mW.
\end{equation}

\textbf{Weights Update in AlphaEdit.} AlphaEdit addresses the imbalance between old and new knowledge in lifelong learning. It protects existing knowledge using a null-space projection constraint, ensuring that the update $\bm{\Delta}$ to $\mW_{\text{out}}^l$ is always projected onto the null space of $\mK_0 \mK_0^T$. The final parameter update, refining MEMIT, is:
\begin{equation}
    \tilde{\mW} = \left( \mV_1 - \mW \mK_1 \right) \mK_1^T \mP \left( \mK_p \mK_p^T \mP + \mK_1 \mK_1^T \mP + \mI \right)^{-1}+ \mW.
\end{equation}

\textbf{Weights Update in UnKE.} Unlike previous methods, UnKE considers the entire input to layer $l$, denoted as $f^l$, rather than just the MLP input. The output remains $f^l$'s activation values. The parameter update is applied to the entire layer rather than a single weight matrix. Given the knowledge sets $\{\mK_0, \mV_0\}$ and $\{\mK_1, \mV_1\}$, the optimization objective is formulated as:
\begin{equation}
    \tilde{\Theta}^l \triangleq \argmin_{\hat{\Theta}^l} \left( \sum_{i=1}^{n} \left\|  f_{\hat{\Theta}^l}^l(\vk_i) - \vv_i \right\|^2 + \sum_{i=n+1}^{n+u} \left\|  f_{\hat{\Theta}^l}^l(\vk_i) - \vv^\ast_i \right\|^2 \right),
\end{equation}
where $\Theta^l$ denotes the entire set of parameters in layer $l$. Since a closed-form solution is not feasible, UnKE employs gradient descent to iteratively update $\Theta^l$.

\subsection{Proof of Optimization-Conditional Mutual Information Equivalence} \label{app:proof_cmi}
\begin{theorem}
The optimization objective  
\begin{equation}
    \bm{\delta}^* = \argmin_{\bm{\delta}} \left( -\log \mathbb{P}_{f(\vh_t+\bm{\delta})}(Y \mid X) \right), \label{eq:opt}
\end{equation}  
is equivalent to maximizing the conditional mutual information (CMI) between $X$ and $Y$ given the perturbed hidden state $\vh'$:  
\begin{equation}
    \vh' = \argmax_{\vh'} I(X; Y \mid \vh'). \label{eq:cmi}
\end{equation}
\end{theorem}

\begin{proof}
Starting from the definition of CMI, we expand it via the integral form:  
\begin{equation}
I(X; Y \mid \vh') = \int p(x, y, \vh') \log \frac{p(y \mid x, \vh')}{p(y \mid \vh')} \, dx dy d\vh'.
\end{equation}  
% Applying Bayes’ rule $p(y \mid x, \vh') = \frac{p(x, y \mid \vh')}{p(x \mid \vh')}$, we rewrite the integrand:  
% \begin{equation}
% I(X; Y \mid \vh') = \int p(x, y, \vh') \log \frac{p(x, y \mid \vh')}{p(x \mid \vh') p(y \mid \vh')} \, dx dy d\vh'.
% \end{equation}  
This splits into two entropy terms:  
\begin{align}
I(X; Y \mid \vh') = \underbrace{\int p(x, y, \vh') \log p(y \mid x, \vh') \, dx dy d\vh'}_{\text{Term } \mathcal{A}} - \underbrace{\int p(x, y, \vh') \log p(y \mid \vh') \, dx dy d\vh'}_{\text{Term } \mathcal{B}}. \label{eq:split}
\end{align}  

Term $\mathcal{A}$ simplifies to the expectation:  
\begin{equation}
\mathcal{A} = \mathbb{E}_{p(\vh')} \mathbb{E}_{p(x, y \mid \vh')} \left[ \log p(y \mid x, \vh') \right],
\end{equation}  
while Term $\mathcal{B}$ is independent of $X$ given $\vh'$. Since $\mathcal{B}$ does not affect the optimization over $\vh'$, we focus on maximizing $\mathcal{A}$.  

By definition, $\mathbb{P}_{f(\vh')}(Y \mid X) = p(y \mid x, \vh')$. Thus, minimizing the negative log-likelihood in \eqref{eq:opt} directly maximizes $\mathcal{A}$, which is equivalent to maximizing $I(X; Y \mid \vh')$. Substituting $\vh' = \vh_t + \bm{\delta}^*$, we conclude:  
\begin{equation}
\vh' = \argmax_{\vh'} I(X; Y \mid \vh'),
\end{equation}  
thereby establishing the equivalence.  
\end{proof}

\subsection{Proof of the Decomposition of Mutual Information}\label{app:proof_decom}
To rigorously derive Equation \eqref{eq:final_MI}, we start from the mutual information (MI) decomposition given in Equation \eqref{eq:decom}:
\begin{equation}
    I(X; Y \mid \vh'_1, \dots, \vh'_K) = \sum_{k=1}^{K} I(X; Y_k \mid Y_1, \dots, Y_{k-1}, \vh'_1, \dots, \vh'_K).
\end{equation}

\textbf{Step 1: Application of the First Property.}
The first key property states that later hidden states do not influence earlier token generation:
\begin{equation}
    H(Y_p \mid \vh'_q) = H(Y_p), \quad \text{for } p < q.
\end{equation}
Since mutual information is defined as:
\begin{equation}
    I(X; Y_k \mid Y_1, \dots, Y_{k-1}, \vh'_1, \dots, \vh'_K) = H(Y_k \mid Y_1, \dots, Y_{k-1}, \vh'_1, \dots, \vh'_K) - H(Y_k \mid X, Y_1, \dots, Y_{k-1}, \vh'_1, \dots, \vh'_K).
\end{equation}
Since $\vh'_q$ for $q > k$ does not affect $Y_k$, we can simplify:
\begin{equation}
    H(Y_k \mid Y_1, \dots, Y_{k-1}, \vh'_1, \dots, \vh'_K) = H(Y_k \mid Y_1, \dots, Y_{k-1}, \vh'_1, \dots, \vh'_k).
\end{equation}

\textbf{Step 2: Application of the Second Property.}
The second key property states that once $Y_k$ is determined, conditioning on $Y_k$ subsumes conditioning on $\vh'_k$:
\begin{equation}
    H(\cdot \mid Y_k) = H(\cdot \mid Y_k, \vh'_k).
\end{equation}
Using this, we rewrite the MI term:
\begin{equation}
    I(X; Y_k \mid Y_1, \dots, Y_{k-1}, \vh'_1, \dots, \vh'_K) = I(X; Y_k \mid Y_1, \dots, Y_{k-1}, \vh'_k).
\end{equation}

\textbf{Step 3: Applying the Conditional Mutual Information Decomposition.}
Using the decomposition formula for conditional mutual information, each term can be written as:
\begin{equation}
    I(X; Y_k \mid Y_1, \dots, Y_{k-1}, \vh'_k) = I(X, Y_1, \dots, Y_{k-1}; Y_k \mid \vh'_k) - I(Y_1, \dots, Y_{k-1}; Y_k \mid \vh'_k).
\end{equation}
In the optimization process, since we are given $X$, the second term is ignored:
\begin{equation}
    I(X; Y_k \mid Y_1, \dots, Y_{k-1}, \vh'_k) = I(X, Y_1, \dots, Y_{k-1}; Y_k \mid \vh'_k).
\end{equation}
Substituting this result back into our summation, we arrive at the desired decomposition:
\begin{equation}
    I(X; Y \mid \vh'_1, \dots, \vh'_K) = \sum_{k=1}^{K} I(X, Y_1, \dots, Y_{k-1}; Y_k \mid \vh'_k).
\end{equation}
This completes the proof.

\section{More Experimental Results}\label{app:exp_result}
\subsection{Case Study}

We selected several editing samples from the UnKEBench and EditEverything datasets as case studies to analyze the long-form knowledge edit performance. 

From the experimental results, AnyEdit effectively restores the target text in terms of both textual consistency and generalization to paraphrase questions. In contrast, other baseline methods often fail to achieve effective editing. Although the UnKE method also demonstrates strong editing performance, it exhibits a noticeable performance drop on paraphrase questions.


\subsubsection{Case 1}

\begin{tcolorbox}[boxrule=0.5pt, left=0pt, right=0pt, top=2.5pt, bottom=2.5pt, title={Model Editing Case Study on Llama3-8B-Instruct}]
    \centering
    \renewcommand{\arraystretch}{1.8}
    \begin{tabular}{p{0.18\linewidth} p{0.76\linewidth}}
        \makecell[c]{\textbf{Input}} & \makecell[c]{What is John Mayne's background and experience in journalism?} \\
        \hline
        \makecell[c]{\textbf{Paraphrase}} & \makecell[c]{Can you provide information about John Mayne's journalism background and experience?} \\
        \hline
        \makecell[c]{\textbf{Target Output}} & \makecell*[{{p{13cm}}}]{John Mayne is a journalist who has been working in the field for over a decade. He has written for several reputable publications, including \tar{The New York Times} and \tar{The Guardian}. In an interview with \tar{The Huffington Post}, Mayne discussed his passion for journalism and his commitment to reporting on important issues. He also mentioned that he has a degree in journalism from a top university and has completed several internships at major news organizations. Mayne's extensive experience and education make him a highly qualified journalist and a valuable asset to any newsroom.} \\
        \hline
        \multicolumn{2}{c}{\textbf{\large LLM-edited Generation}} \\ % Bold and centered
        \hline
        \makecell[c]{\textbf{MEMIT}} & \makecell*[{{p{13cm}}}]{John Mayne is a journalist and writer, but I couldn't find much about his personal background. However, he has been in the industry for several years, contributing to various publications. According to LinkedIn, he has worked for major outlets including BBC and \tar{The Guardian}(...) John Mayne's experience spans politics, business, and culture, and he has been recognized for his in-depth reporting.} \\
        \hline
        \makecell[c]{\textbf{AlphaEdit}} & \makecell*[{{p{13cm}}}]{John Mayne is a British journalist and author with a long career in media. He has written for major newspapers such as The Times and The Sunday Times. Mayne has covered diverse topics, including politics and culture, and worked as a foreign correspondent reporting from the Middle East, Africa, and Asia... In addition, he has taught journalism at universities and received multiple awards for his writing.} \\
        \hline
        \makecell[c]{\textbf{UnKE}} & \makecell*[{{p{13cm}}}]{John Mayne is a journalist who has been working in the field for several years. He has written for a number of publications and has a strong track record of producing high-quality, well-researched articles. In addition to his writing experience, Mayne has also worked as an editor and has a strong understanding of the journalism industry. He is well-respected by his peers and is known for his ability to produce engaging and informative articles.} \\
        \hline
        \makecell[c]{\textbf{AnyEdit}} & \makecell*[{{p{13cm}}}]{ John Mayne is a journalist who has been working in the field for over a decade. He has written for several reputable publications, including \tar{The New York Times} and \tar{The Guardian}. In an interview with \tar{The Huffington Post}, Mayne discussed his passion for journalism and his commitment to reporting on important issues. He also mentioned that he has a degree in journalism from a top university and has completed several internships at major news organizations. Mayne's extensive experience and education make him a highly qualified journalist and a valuable asset to any newsroom.} \\
    \end{tabular}
\end{tcolorbox}

\subsubsection{Case 2}

\begin{tcolorbox}[boxrule=0.5pt, left=0pt, right=0pt, top=2.5pt, bottom=2.5pt, title={Model Editing Case Study on Qwen2.5-7B-Instruct}]
    \centering
    \renewcommand{\arraystretch}{1.8}
    \begin{tabular}{p{0.18\linewidth} p{0.76\linewidth}}
        \makecell[c]{\textbf{Input}} & \makecell[c]{find the value of $log_y(x^4)$ if $log_x(y^3) = 2$} \\
        \hline
        \makecell[c]{\textbf{Target Output}} & \makecell*[{{p{13cm}}}]{To find the value of $log_y(x^4)$, we can use the property of logarithms that states: $log_b(a) = 1 / log_a(b)$. Given that $log_x(y^3) = 2$, we can rewrite this as: $y^3 = x^2$. Now, we want to find $log_y(x^4)$. We can use the property mentioned above: $log_y(x^4) = 1 / log_x(y^4)$. But we don't have $log_x(y^4)$; we have $log_x(y^3)$. So, we need to manipulate the expression to use the given information. We know that $y^3 = x^2$, so we can raise both sides to the power of $4/3$ to get $y^4$: $(y^3)^{(4/3)} = (x^2)^{(4/3)}$. $y^4 = x^{(8/3)}$. Now we can write $log_x(y^4)$ in terms of $log_x(y^3)$: $log_x(y^4) = log_x((y^3)^{(4/3)})$. Using the power rule of logarithms ($log_b(a^c) = c * log_b(a)$), we get: $log_x(y^4) = (4/3) * log_x(y^3)$. Since we know $log_x(y^3) = 2$, we can substitute this value in: $log_x(y^4) = (4/3) * 2$. $log_x(y^4) = 8/3$. Now we can find $log_y(x^4)$ using the reciprocal property: $log_y(x^4) = 1 / log_x(y^4)$. $log_y(x^4) = 1 / (8/3)$. $log_y(x^4) = 3/8$. Therefore, the value of $log_y(x^4)$ is \tar{$3/8$}.} \\
        \hline
        \multicolumn{2}{c}{\textbf{\large LLM-edited Generation}} \\ % Bold and centered
        \hline
        \makecell[c]{\textbf{MEMIT}} & \makecell*[{{p{13cm}}}]{To find the value of $log_y(x^4)$, we can use the change of base formula and the given information. Given: $log_x(y^3) = 2$. Using the change of base formula, we can rewrite this as: $log_x(y^3) = (log_y(y^3)) / (log_y(x))$. Since $log_y(y^3) = 3$ (because $y^3$ is the argument raised to the power of 3), we have: $2 = 3 / (log_y(x))$. Now, we can solve for $log_y(x)$: $log_y(x) = 3/2$. Next, we want to find $log_y(x^4)$. Using the power rule of logarithms, we can write: $log_y(x^4) = 4 * log_y(x)$. Substituting the value we found for $log_y(x)$: $log_y(x^4) = 4 * (3/2)$. Simplifying: $log_y(x^4) = 6$. Therefore, the value of $log_y(x^4)$ is $6$.} \\
        \hline
        \makecell[c]{\textbf{AlphaEdit}} & \makecell*[{{p{13cm}}}]{To find the value of $log_y(x^4)$, we can use the change of base formula and the given information: $log_x(y^3) = 2$. Using the change of base formula, we can rewrite this as: $y^3 = x^2$. Now, we want to find $log_y(x^4)$. Using the properties of logarithms, we can rewrite this as: $log_y(x^4) = 4 * log_y(x)$. We can express $log_y(x)$ in terms of $log_x(y)$ using the change of base formula: $log_y(x) = 1 / log_x(y)$. Since we know that $log_x(y^3) = 2$, we can write: $log_x(y^3) = 3 * log_x(y) = 2$. Therefore: $log_x(y) = 2/3$. Substituting this into our expression for $log_y(x)$: $log_y(x) = 1 / (2/3) = 3/2$. Finally, substituting this back into our expression for $log_y(x^4)$: $log_y(x^4) = 4 * (3/2) = 6$. Therefore, the value of $log_y(x^4)$ is $6$.} \\
        \hline
        \makecell[c]{\textbf{UnKE}} & \makecell*[{{p{13cm}}}]{To find the value of $log_y(x^4)$, we can use the property of logarithms that states: $log_b(a) = 1 / log_a(b)$. Given that $log_x(y^3) = 2$, we can rewrite this as $y^3 = x^2$. Now, we can use the property to find $log_y(x^4)$: $log_y(x^4) = 1 / log_x(y^3)$ $= 1 / (1 / x^2)$ $= x^2$. So, the value of $log_y(x^4)$ is $x^2$. Therefore, the correct answer is $x^2$.} \\
        \hline
        \makecell[c]{\textbf{AnyEdit}} & \makecell*[{{p{13cm}}}]{To find the value of $log_y(x^4)$, we can use the property of logarithms that states: $log_b(a) = 1 / log_a(b)$. Given that $log_x(y^3) = 2$, we can rewrite this as: $y^3 = x^2$. Now, we want to find $log_y(x^4)$. We can use the property mentioned above: $log_y(x^4) = 1 / log_x(y^4)$. But we don't have $log_x(y^4)$; we have $log_x(y^3)$. So, we need to manipulate the expression to use the given information. We know that $y^3 = x^2$, so we can raise both sides to the power of $4/3$ to get $y^4$: $(y^3)^{(4/3)} = (x^2)^{(4/3)}$. $y^4 = x^{(8/3)}$. Now we can write $log_x(y^4)$ in terms of $log_x(y^3)$: $log_x(y^4) = log_x((y^3)^{(4/3)})$. Using the power rule of logarithms ($log_b(a^c) = c * log_b(a)$), we get: $log_x(y^4) = (4/3) * log_x(y^3)$. Since we know $log_x(y^3) = 2$, we can substitute this value: $log_x(y^4) = (4/3) * 2$. $log_x(y^4) = 8/3$. Now we can use the property of logarithms: $log_y(x^4) = 1 / log_x(y^4)$. $log_y(x^4) = 1 / (8/3)$. $log_y(x^4) = 3/8$. Therefore, the value of $log_y(x^4)$ is \tar{$3/8$}.} \\
    \end{tabular}
\end{tcolorbox}
\clearpage
\newpage

\subsection{Supplementary Experimental Results on RQ1 \& RQ2} \label{app:exp_result_1}
We present a comprehensive evaluation of all metrics on the UnKEBench and AKEW datasets in Table \ref{tab:app_1} and Table \ref{tab:app_2}. The results demonstrate that UnKE consistently outperforms other baselines across both original and paraphrase question evaluations. Notably, UnKE+, which integrates AnyEdit’s autoregressive editing paradigm, achieves even higher scores in lexical similarity (BLEU, ROUGE-1/2/L) and semantic similarity (BERT Score), indicating its superior ability to preserve and generalize edited knowledge. In contrast, MEMIT and AlphaEdit struggle with paraphrase generalization, showing significantly lower performance on the right side of `/', suggesting that these methods fail to robustly transfer edited knowledge across rephrased contexts. While MEMIT+ and AlphaEdit+ improve over their base versions, their performance still lags behind UnKE and UnKE+.

Overall, UnKE+ achieves the best balance between precise knowledge modification and robust generalization, confirming that combining UnKE with autoregressive fine-tuning leads to stronger and more reliable knowledge editing in LLMs.
\begin{table*}[h]
\caption{Performance comparison in UnKEBench. The `+' symbol indicates results incorporating AnyEdit's autoregressive editing paradigm. The left side of `/' represents the LLM's edited output for original questions, while the right side represents the edited output for paraphrase questions.}
    \label{tab:app_1}
    \centering
    \renewcommand{\arraystretch}{1.2}
    \setlength{\tabcolsep}{4pt}
    \resizebox{\textwidth}{!}{
    \begin{tabular}{l cccc ccc}
        \toprule
        \multirow{2}{*}{\textbf{Method}} & \multicolumn{4}{c}{\textbf{Lexical Similarity}} & \multicolumn{1}{c}{\textbf{Semantic Similarity}} & \textbf{Sub Questions} \\
        \cmidrule(lr){2-5} \cmidrule(lr){6-6} \cmidrule(lr){7-7} 
        & BLEU & ROUGE-1 & ROUGE-2 & ROUGE-L & BERT Score & ROUGE-L \\
        \midrule
        \multicolumn{7}{l}{\textbf{Based on Llama3-8B-Instruct}} \\
        \midrule
        UnKE        & 93.56 / 78.09  & 93.61 / 79.26  & 91.42 / 71.73  & 93.33 / 78.42  & 98.34 / 93.38    & 37.87 \\
        UnKE+       & 99.67 / 84.60  & 99.69 / 86.31  & 99.57 / 81.18  & 99.68 / 85.75  & 99.86 / 94.70    & 41.45 \\
        MEMIT       & 25.57 / 22.88  & 32.67 / 30.75  & 14.51 / 12.37  & 30.49 / 28.65  & 76.21 / 74.25    & 22.56 \\
        MEMIT+      & 88.88 / 81.38  & 93.26 / 86.53  & 90.32 / 80.61  & 92.96 / 85.91  & 97.76 / 95.60    & 41.67 \\
        AlphaEdit   & 21.29 / 20.24  & 28.62 / 27.99  & 11.36 / 10.24  & 26.59 / 25.92  & 73.92 / 72.96    & 20.71 \\
        AlphaEdit+  & 75.02 / 66.35  & 81.70 / 73.47  & 74.35 / 62.74  & 80.92 / 72.22  & 94.19 / 91.51    & 40.56 \\
        \midrule
        \multicolumn{7}{l}{\textbf{Based on Qwen2.5-7B-Instruct}} \\
        \midrule
        UnKE        & 91.92 / 70.61  & 91.41 / 68.47  & 87.75 / 56.34  & 91.01 / 67.00  & 96.97 / 89.17    & 38.12 \\
        UnKE+       & 98.52 / 82.48  & 98.85 / 83.36  & 98.43 / 77.03  & 98.82 / 82.60  & 99.35 / 94.81    & 42.24 \\
        MEMIT       & 45.07 / 40.81  & 40.73 / 36.75  & 19.59 / 15.87  & 38.04 / 34.07  & 78.03 / 76.50    & 24.75 \\
        MEMIT+      & 91.31 / 77.23  & 95.10 / 80.88  & 92.93 / 72.50  & 94.89 / 79.98  & 98.05 / 93.56    & 42.38 \\
        AlphaEdit   & 49.71 / 45.21  & 45.42 / 41.06  & 24.63 / 19.85  & 42.77 / 38.26  & 80.48 / 78.38    & 25.37 \\
        AlphaEdit+  & 97.77 / 83.09  & 98.20 / 84.18  & 97.40 / 77.38  & 98.14 / 83.40  & 99.08 / 94.51    & 41.58 \\
        \bottomrule
    \end{tabular}
    }
    
\end{table*}

\begin{table*}[h]
\caption{Performance comparison in AKEW (Counterfact). The `+' symbol indicates results incorporating AnyEdit's autoregressive editing paradigm. The left side of `/` represents the LLM's edited output for original questions, while the right side represents the edited output for paraphrase questions.}
    \label{tab:app_2}
    \centering
    \renewcommand{\arraystretch}{1.2}
    \setlength{\tabcolsep}{4pt}
    \resizebox{\textwidth}{!}{
    \begin{tabular}{l cccc ccc}
        \toprule
        \multirow{2}{*}{\textbf{Method}} & \multicolumn{4}{c}{\textbf{Lexical Similarity}} & \multicolumn{1}{c}{\textbf{Semantic Similarity}} & \textbf{Sub Questions} \\
        \cmidrule(lr){2-5} \cmidrule(lr){6-6} \cmidrule(lr){7-7} 
        & BLEU & ROUGE-1 & ROUGE-2 & ROUGE-L & BERT Score & ROUGE-L \\
        \midrule
        \multicolumn{7}{l}{\textbf{Based on Llama3-8B-Instruct}} \\
        \midrule
        MEMIT       & 33.44 / 18.13  & 34.46 / 17.44  & 16.29 / 4.74   & 32.20 / 16.10  & 76.44 / 47.80  & 39.98\\
        MEMIT+      & 85.41 / 38.78  & 96.07 / 47.61  & 94.21 / 32.37  & 95.87 / 46.00  & 97.76 / 62.63  & 64.07\\
        UnKE        & 98.43 / 36.99  & 98.43 / 34.58  & 97.78 / 19.37  & 98.37 / 32.89  & 99.62 / 59.62  & 63.22\\
        UnKE+       & 99.98 / 45.23  & 99.98 / 46.57  & 99.96 / 35.41  & 99.98 / 45.31  & 99.95 / 64.24  & 59.03\\
        AlphaEdit   & 23.36 / 16.25  & 26.92 / 15.00  & 10.81 / 3.61   & 24.95 / 13.79  & 72.63 / 44.67  & 35.76 \\
        AlphaEdit+  & 79.60 / 40.67  & 84.49 / 41.11  & 78.00 / 26.60  & 83.76 / 39.51  & 96.51 / 65.14  & 57.05 \\
        \midrule
        \multicolumn{7}{l}{\textbf{Based on Qwen2.5-7B-Instruct}} \\
        \midrule
        MEMIT       & 45.29 / 32.83  & 41.68 / 28.01  & 20.38 / 8.79   & 38.95 / 25.73  & 77.19 / 56.04  & 43.51\\
        MEMIT+      & 90.55 / 44.32  & 95.33 / 45.56  & 93.12 / 27.38  & 95.09 / 43.49  & 98.08 / 65.40  & 55.10\\
        UnKE        & 91.53 / 38.59  & 90.91 / 31.53  & 87.06 / 12.11  & 90.44 / 29.27  & 97.34 / 59.29  & 49.97\\
        UnKE+       & 98.95 / 34.68  & 99.01 / 35.23  & 98.59 / 15.59  & 98.99 / 32.95  & 99.63 / 60.78  & 51.58\\
        AlphaEdit   & 49.97 / 34.65  & 48.15 / 30.02  & 27.76 / 10.38  & 45.55 / 27.69  & 80.66 / 56.99  & 45.12\\
        AlphaEdit+  & 97.61 / 46.97  & 97.80 / 47.63  & 96.89 / 30.31  & 97.73 / 45.84  & 99.10 / 66.10  & 54.99\\
        \bottomrule
    \end{tabular}
    }
    
\end{table*}

\subsection{Supplementary Experimental Results on RQ4}\label{app:exp_result_4}
\begin{figure}[t]
\begin{center}
\includegraphics[width=0.6\linewidth, keepaspectratio]{figures/exp_3.png}
\caption{The relationship between AnyEdit's editing performance and chunk size in long-form diverse-formatted knowledge.}
\label{fig:exp_3}
\end{center}
\end{figure}


 The experimental results of relationship between AnyEdit's editing performance and chunk size in long-form diverse-formatted knowledge are presented in Figure \ref{fig:exp_3}. Based on these results, we draw the following observation:.

\begin{itemize}[leftmargin=*]
    \item \textbf{Obs 7: The editing performance of AnyEdit is influenced by chunk size.}  
    As the chunk size increases beyond a certain threshold, the editing performance of AnyEdit declines. Specifically, when the chunk size is smaller, each iteration of editing becomes more manageable, leading to improved overall performance. However, this improvement comes at the cost of increased editing time due to the larger number of iterations required for longer texts. Conversely, when the chunk size is larger, it becomes challenging to achieve effective edits within a single iteration, resulting in degraded performance. Based on this trade-off, we recommend using a balanced chunk size of 40 for most editing scenarios.
\end{itemize}

\begin{figure}[h]
    \centering
    \includegraphics[width=\textwidth]{figures/data1.png}
    \vspace{-5mm}
    \caption{A Sample of the AKEW (Counterfact) dataset.}
    \label{fig:sample1}
\end{figure}

\begin{figure}[h]
    \centering
    \includegraphics[width=\textwidth]{figures/data2.png}
    \vspace{-5mm}
    \caption{A Sample of the UnKEBench dataset.}
    \label{fig:sample2}
\end{figure}

\begin{figure}[h]
    \centering
    \includegraphics[width=\textwidth]{figures/data3.png}
    \vspace{-5mm}
    \caption{Samples of the EditEverything dataset.}
    \label{fig:sample3}
\end{figure}
%%%%%%%%%%%%%%%%%%%%%%%%%%%%%%%%%%%%%%%%%%%%%%%%%%%%%%%%%%%%%%%%%%%%%%%%%%%%%%%
%%%%%%%%%%%%%%%%%%%%%%%%%%%%%%%%%%%%%%%%%%%%%%%%%%%%%%%%%%%%%%%%%%%%%%%%%%%%%%%


\end{document}



% This document was modified from the file originally made available by
% Pat Langley and Andrea Danyluk for ICML-2K. This version was created
% by Iain Murray in 2018, and modified by Alexandre Bouchard in
% 2019 and 2021 and by Csaba Szepesvari, Gang Niu and Sivan Sabato in 2022.
% Modified again in 2023 and 2024 by Sivan Sabato and Jonathan Scarlett.
% Previous contributors include Dan Roy, Lise Getoor and Tobias
% Scheffer, which was slightly modified from the 2010 version by
% Thorsten Joachims & Johannes Fuernkranz, slightly modified from the
% 2009 version by Kiri Wagstaff and Sam Roweis's 2008 version, which is
% slightly modified from Prasad Tadepalli's 2007 version which is a
% lightly changed version of the previous year's version by Andrew
% Moore, which was in turn edited from those of Kristian Kersting and
% Codrina Lauth. Alex Smola contributed to the algorithmic style files.
