\section{Related Works}
\subsection{Patch-Level Glomerular Segmentation}
    
Patch-level glomerular segmentation has been a focal point in renal pathology, aiming to delineate glomeruli within small, manageable regions of kidney histology images \citep{ginley2019computational}. Early methodologies predominantly utilized classical image processing techniques, including thresholding, edge detection, and morphological operations \citep{electronics9030503, ginley2017automatic}, to identify glomerular structures. While these approaches established foundational workflows, they often encountered challenges due to the intricate morphology of glomeruli and variability in staining methods, leading to inconsistent segmentation outcomes \citep{electronics9030503}.

The advent of deep learning, particularly convolutional neural networks (CNNs), has significantly advanced the field of medical image segmentation \citep{GALLEGO2021101865}. Architectures such as U-Net and its derivatives have been extensively applied to patch-level glomerular segmentation, demonstrating enhanced accuracy and efficiency by learning complex patterns inherent in kidney tissues \citep{Samant2023Glomerulus, GALLEGO2021101865, deng2023omni}.
Despite these advancements, several challenges persist in patch-level segmentation. A primary concern is the lack of global contextual information, as models trained on isolated patches may not effectively capture the spatial relationships and broader tissue architecture present in whole slide images (WSIs) \citep{Wu2023DigitalPathology}. This limitation can lead to inaccuracies, particularly when glomeruli are located near the periphery of patches or exhibit atypical presentations. Additionally, variability in image quality, staining techniques, and the inherent complexity of glomerular structures can adversely affect segmentation performance \citep{Wu2023DigitalPathology}. 
To address these issues, recent research has explored hybrid models that integrate CNNs with transformer-based architectures to enhance both local feature extraction and global context understanding \citep{Liu2024, Yin2024, Wu2023DigitalPathology, deng2024prpseg, deng2024hats, deng2023segment, cuienhancing}. For example, a hybrid CNN-TransXNet \citep{Liu2024} approach has been proposed to improve segmentation accuracy by combining the strengths of CNNs in capturing fine-grained local features with the global contextual understanding provided by transformers.


    \subsection{Whole Slides Image-Level Glomerular Detection and Segmentation}

WSI glomerular detection and segmentation have become pivotal in computational kidnet pathology, enabling comprehensive analysis of kidney tissues at high resolution. Unlike patch-based methods, WSI approaches consider the entire tissue context, facilitating more accurate localization and characterization of glomeruli \citep{BUENO2020105273, tang2024holohisto}. 

Recent advancements have introduced holistic frameworks that integrate detection, segmentation, and lesion characterization within unified pipelines. For instance, the Glo-In-One \citep{GloInOne, GloInOne_v2} toolkit performs comprehensive glomerular quantification from WSIs, streamlining the analysis process for non-technical users. Extension plugin GloFinder \citep{GloFinder}, designed for QuPath, enables single-click automated glomeruli detection across entire WSIs, streamlining the analysis process for non-technical users.

Additionally, deep learning models have been employed to enhance glomerular detection in WSIs, such as modified U-Net architecture \citep{jha2021instance}. These models leverage convolutional neural networks to accurately identify glomeruli, demonstrating significant improvements over traditional methods \citep{jha2021instance, JIANG20211431}. 

Despite these advancements, several challenges persist in WSI glomerular analysis. Processing gigapixel WSIs demands substantial computational resources, necessitating efficient algorithms and hardware \citep{BECKER202065}. Differences in staining protocols, imaging modalities, and tissue preparation can introduce variability, affecting model performance \citep{7243333}. Obtaining pixel-level annotations for WSIs is labor-intensive, often resulting in limited training data for supervised learning models \citep{9973240}. 

To address challenges in WSI glomerular detection and segmentation, researchers have developed innovative strategies. For example, Weighted Circle Fusion (WCF) \citep{WCF} has been proposed to enhance detection precision by fusing outputs from multiple models using confidence-weighted circle representations. This method not only improves accuracy but also reduces false positives by effectively merging overlapping detections. Furthermore, the human-in-the-loop (HITL) approach has significantly improved annotation efficiency, combining machine learning detection with human verification. However, WCF is not without limitations. It demonstrates slightly lower recall compared to methods like Non-Maximum Suppression (NMS), and its reliance on careful parameter tuning may restrict generalizability across different datasets \citep{WCF}.