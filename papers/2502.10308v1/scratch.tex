In full generality, the cardinal representation of students' combinatorial valuations over courses requires exponential space. However, real preferences may have some underlying structure (for example, additive, or additive with only pairwise interactions), so that relevant preferences can be expressed via a suitably compact bidding language.
But in reality, one may not know a priori what sorts of bidding languages capture true student preferences; moreover, making reports in these languages may in practice be difficult for the students (as was observed in \citet{budish2021can}).

Natural language is a much more flexible and expressive way to represent these preferences. For instance, consider the following notional description of preferences\footnote{
Similar language is used by the LLMs in our experiments, for example
``Course 19 also excites me, especially given its complementary relationship with Course 12.''
}
\begin{quote}
``I prefer to take courses that are scheduled as closely together as possible so I can have an extra day off. If courses have a laboratory section, I strongly prefer that it be in the morning. Course A and Course B complement each other, and I would prefer to take them together to save time and effort. I do not want to take Course D and E together as they cover similar topics. Nonetheless, I need at least one of them to fulfill my requirements...''
\end{quote}
The textual description encodes combinatorial information about preferences without requiring a commitment to a bidding language, and it allows the students to easily express preferences over whole categories of courses (rather than labeling each one). It also provides a more natural elicitation process for students, who may find it easier to express their preferences in free text rather than answering hundreds of comparison queries. Our goal is to go from this useful yet imprecise natural language expression, to an allocative decision by a mechanism that requires cardinal bids as input.
