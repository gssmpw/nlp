% The type of topic model selected must align with the nuanced needs and requirements of the dataset and research objectives at hand. A model that performs well on one dataset or task may not necessarily be the optimal choice for a different context. Researchers should assess the strengths and limitations of various topic modeling techniques in the context of their own work, rather than making assumptions based on generic performance claims.

While advocates push \mm{}-based solutions for data exploration, their
evaluation is still limited to cluster based evaluation or topic
matching against gold labels.
%
Our more realistic evaluation shows that while all topic models can help humans understand a dataset, \mm{}-based approaches still have their limitations: 
instability, hallucination, and scalability, and inflexibility.
%
Traditional topic models are still the cheapest way for users to run
preliminary and simple data exploration.
%
That said, our results confirm that people like \mm{} outputs.

Our proposed model, \bass{}, helps address some of these concerns
for trickier datasets: it is cheaper (Appendix~\ref{sec:cost}) and less vulnerable (Appendix~\ref{tab:topic_generation_comparison}) to
hallucination.
%
However, the amount of mental effort can be overkill for simple
datasets, and it is best recommended for advanced data exploration or
motivated users.
%
Nonetheless, users can prefer the topics they build
themselves~\cite{NORTON2012453}.
% 
In sum, there is no definite answer which topic model is best in all
circumstances, nor that \mm{}-based methods can replace
traditional or human supervised topic models.
% 
Future work is needed to reduce the cost of \mm{}-based methods by training stronger local models and reduce human involvement efforts by improving \mm{}s' reasoning abilities spanning multiple documents.

% The choice of what tools to use relies on the user goals,
% applications, and resource availabilities to maximize the advantages
% of a tool (Table~\ref{tab:topic_modeling_comparison}).

% \jbgcomment{Points the conclusion needs to hit:

% \url{https://myscp.onlinelibrary.wiley.com/doi/10.1016/j.jcps.2011.08.002}


% \begin{itemize*}
%     \item Most topic overviews help, and users can also do well when they have limited tools
%     \item Users prefer using tools where they have more of a say in how the labels are formed
%     \item This up front investment pays off by shorter time to answer questions (in our contrived example)
%     \item future work should further reduce the cost by using cheaper models, more directly connect the models for labeling and topic distributions, and provide more flexibility for users
% \end{itemize*}

% }


% \jbgcomment{Need bst to cut after 10 authors, share with everyone else}
