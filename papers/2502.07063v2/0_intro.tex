\section{Introduction}

% Zero-Knowledge Proofs (ZKPs) provide a secure cryptographic method for a prover \Prv to prove that they know a secret value $w$ to a verifier \Vrf, without revealing anything about $w$.
Privacy-preserving cryptographic methods have become increasingly vital as privacy and data security evolve into a higher priority in new applications. Zero-Knowledge Proofs (ZKPs) enable a prover \Prv to prove to a verifier \Vrf that a statement is true, without revealing any information beyond the validity of the statement itself. While ZKPs are most prominently known to the general public in blockchain applications \cite{boo2021litezkp, chainOverviewZeroKnowledge, binanceWhatZeroknowledge, vcapko2022state}, they have also been effectively applied in many other real-world domains, such as healthcare \cite{tomaz2020preserving, sharma2020blockchain, gaba2022zero}, traditional finance \cite{rabin2012strictly, thorpe2009zero, lin2021efficient}, and government \cite{bamberger2022verification, landau1988zero}. ZKPs are an excellent solution for verifying data and computation in a secure fashion, however there are still many challenges before they can become a practical privacy-preserving solution.

Although introduced in the 1980s \cite{goldwasser2019knowledge}, recent algorithmic and computing advances have garnered the evolution of ZKPs from a theoretical construct to a relatively practical cryptographic primitive. ZKPs have garnered the interest of researchers and developers as concerns over data privacy grow, which has caused significant improvements in both theory and implementation. The first significant milestone in the practical application of ZKPs was the development of zero-knowledge succinct non-interactive arguments of knowledge (zk-SNARKs) introduced by Ben-Sasson et. al \cite{ben2014succinct} in 2013. 
% zk-SNARKs have been a prominent protocol for quite some time, making them the most well-known and widely used construction of ZKPs. zk-SNARKs were also the initial backbone of one of first at-scale deployments of ZKPs in the real-world: Zcash \cite{hopwood2016zcash}. Zcash utilizes ZKPs to provide privacy for transactions on the blockchain, allowing for transactional data to be validated without revealing any sensitive information.

In the decade since the introduction of zk-SNARKs, the zero-knowledge landscape has evolved to include a diverse set of ZKP constructions, such as zk-STARKs \cite{ben2018scalable}, which build off of zk-SNARKs and are discussed at length in this work. For many of the ZKP constructions that are available, there are several prominent frameworks, stemming from industry and academia, that allow developers to create their own ZKP applications. Despite the availability of these open-source frameworks and the demand of privacy-preservation in real-world systems, the implementation of ZKPs in practical applications has been limited. This can be attributed to three ongoing challenges: 1) performance; 2) usability; 3) accessibility. These are the attributes that we evaluate the chosen open-source frameworks on in this work.
The journey towards making ZKPs a de facto solution for privacy-preserving applications is hamstrung by the performance, due to the complexity of current accessible ZKP protocols. 
% Admittedly, this is not a problem with the schemes, as much as it is a call for action for hardware acceleration of said schemes. While there have been excellent works that aim to accelerate ZKPs \cite{hirner2023proteus, zhang2021pipezk}, they are primarily focused on accelerating specific general operations, such as the number theoretic transform \cite{agarwal1975number} and multiscalar multiplications \cite{doche2009double}. While these are significant contributions, the journey towards practical usage of ZKPs would significantly benefit from an end-to-end accelerator of a \textit{specific} scheme. The main hurdle for this objective is figuring out which scheme is the most suitable for acceleration based on the current performance and the potential of accelerating the computational bottlenecks of said scheme.
In this work, we hope to find the schemes with the best performance for each type of ZKP construction, evaluated over several metrics on CPU and we will discuss the next steps that can be taken towards practical ZKP adoption. 
% \nojan{While we believe that hardware acceleration is necessary for the widespread adoption of ZKPs in current applications, we do note that ZKPs are not too slow for \textit{all} applications and are currently being effectively used in several domains.}

Usability and accessibility are common problems that face privacy-preserving technologies, especially those stemming from academia, and the ZK landscape is no different. Due to the surge of research that has been done in the ZK space, there has been a sudden increase in the number of available frameworks for developers. For a nascent developer of ZKP applications, especially one with little exposure to cryptography and ZK concepts, this can seem like a near-impossible field to navigate. Even for experienced developers, these frameworks are often hard to use, due to their (mostly) poor documentation or reproducible examples. While this is understandable in academic settings, due to time and resource constraints, a significant step towards enabling practical ZKP usage is demystifying the currently existing frameworks and lowering the barrier of entry for experienced and unexperienced developers. Alongside this, it is also currently difficult for developers to decipher whether a framework is usable for their custom application, due to the different ZKP constructions available, each with different underlying arithmetic, security guarantees, and interaction/communication requirements. 

While there is no arguing that the development of the open-source ZKP frameworks has significantly reduced the amount of necessary effort for building new applications, the field is still difficult to navigate. The available open source frameworks have been used to enable secure verification of computation, data, and identity in the domains of machine learning \cite{zhang2020zero}, networking \cite{grubbs2022zero}, IP protection \cite{sheybani2023zkrownn}, and many more. Although there has been this evident uptick in ZKP frameworks and applications, there is no overview of the ZK landscape that is both cryptographer and non-cryptographer friendly. Alongside this, it is hard to find a clear path for where ZKPs can be improved so they can be more broadly integrated into practical real-world applications. 
% As one of the most booming topics in privacy-preserving technology research, there is no lack of interest in development of ZK-based applications.
% There is no doubt that the state-of-the-art works that are accompanied by open-source frameworks are 

This paper aims to provide users with a guide to ZKPs and the available ZKP frameworks, allowing readers to gain a high-level overview of the ZK landscape, while also providing new quantitative benchmarks and details for developers to choose the best ZKP framework for their application. To achieve this goal, we conduct an extensive survey of the ZK landscape, gathering several state-of-the-art frameworks representing the seminal ZKP constructions. We first evaluate these existing tools based on the usability and accessibility of their repositories for a non-experienced cryptographic application developer, highlighting their features and shortcomings from a design standpoint. 
% A detailed overview of each evaluated framework is provided.
We then evaluate a subset of the most accessible and usable frameworks, primarily those that expose a high-level API, based on their performance through an in-depth analysis of their runtime and communication complexity. Performance is measured over two custom benchmarks that represent commonly used functions in privacy-preserving computation: matrix multiplication and SHA-256 compression. 
% Due to the different ZKP constructions that are evaluated, we aim to only compare frameworks representing the same construction (e.g. zk-SNARKs) to each other to maintain fairness.

We provide a discussion of the different constructions at a high-level to guide developers in their choice of framework. Our experimental evaluation, insights, and recommendations should provide a general guide to developers on how to whittle down the available frameworks to ones that fit their application setting, bandwidth, and computational requirements.
We conclude our work with a discussion on some of the cutting-edge applications that ZKPs have been utilized in, the challenges that ZKP applications currently face, and the future of ZKPs.

Unfortunately, many of the prominent works that provide open-source frameworks do not include a proper documentation or reproducible examples, thus hindering developers in integrating these frameworks into their applications. Alongside this, many of the frameworks require complex local environment build dependencies. To combat this hurdle, we provide a new open-source Github repository\footnote{\url{https://github.com/ACESLabUCSD/ZeroKnowledgeFrameworksSurvey}} containing all the tools necessary to build a custom ZKP application with any framework discussed in this survey. Not only do we provide open-source Docker environments for each frameworks with reproducible documented examples, but we also include Docker containers for other helpful tools, such as circuit building and inspection tools. This repository is also well-documented and actively maintained to encourage users to immediately start building custom applications, rather than focusing on setup troubles.

The goal of this survey is to lower the barrier of entry to building ZKP applications by providing an in-depth overview of ZKPs, the existing constructions, the available open-source frameworks and their capabilities, and the usability, accessibility, and performance of each available framework. This paper is written so that a reader with no prior knowledge of ZKPs can garner a high-level understanding of the landscape, while experienced readers can sharpen their knowledge of the details of ZKPs and gain insights on the available tools for ZK-based application development. In short, our scientific contributions are:
\begin{itemize}
    \item We present the first survey of open-source ZKP frameworks, spanning \textit{all} ZKP constructions, with accompanying open-source environments for each framework, including benchmarks and documentation.
    \item We perform extensive analysis of select open-source ZKP frameworks on scalability, runtime, and proof size on two benchmarks representing prominent domains of ZKPs in current practice.
    \item We provide a thorough analysis of the capabilities, usability, and accessibility of each open-source ZKP framework.  Based on the insights of our work, we customize suggestions of frameworks for different use cases based on available compute power, developer experience, and application type. Finally, we provide novel insights on the current state of ZKP and the necessary path to further boost practicality.
\end{itemize}
\nojan{add scientific contribution to this DONE}
% \nojan{mention high-level API exposed more?}

\subsection{Related Work}

To the best of our knowledge, this work is the first to systematically survey and benchmark open-source ZKP frameworks spanning \textit{all} constructions for practical settings and realization. \cite{CelerNetwork2023Pantheon, Delendum2023ZKSystemBenchmarking} have considered a very limited amount of frameworks, but their industry-led work is largely comparing different constructions to each other (zk-SNARK vs. zk-STARK), rather than comparing frameworks of the same construction to each other (zk-SNARK vs. zk-SNARK). While there are very interesting insights made, we believe that our work is much more objective, systematic and extensive, while also adding the element of usability and accessibility analysis.
Another survey on ZKP frameworks has been conducted \cite{9520375}, however this work only focuses on zk-SNARKs and does not look into the usability, accessibility, or performance of the chosen frameworks. Also, the work is largely focused on the application of zk-SNARKs in the blockchain. Similarly, \cite{partala2020non} conducts a survey on non-interactive ZK applications in the blockchain. The main focus of this work is the analysis of privacy-protection schemes for smart contracts. This work approaches ZK scheme analysis from a theoretical standpoint. Rather than focusing on usability and accessibility, this work primarily focuses on analyzing the asymptotics of the available schemes. Unfortunately, this is not fully representative of the performance of these schemes in practice, due to some schemes having high constants in their asymptotic complexities. While we believe that surveying zk-SNARKs is very important, we note that many zk-SNARK schemes are not post-quantum secure. As post-quantum security becomes a rapidly growing concern, our work purposefully inspects every available ZKP construction to provide insights into post-quantum secure frameworks, alongside more established zk-SNARK frameworks. We believe that limiting our work to zk-SNARKs would not be fully representative of the ZK landscape.

We model our paper after the seminal surveys in privacy-preserving technology centered around MPC \cite{hastings2019sok} and FHE \cite{viand2021sok}. 
% As these techniques have been more extensively studied, the accompanying tools are more mature and their insights are more conclusive.
Like these works, we aim to provide as detailed of a description as we can surrounding the usability, accessibility, and performance of our chosen frameworks, while providing a digestible guide for developers choosing a tool for their ZK-based applications.