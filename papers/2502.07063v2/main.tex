\documentclass[lettersize,journal]{IEEEtran}
% \IEEEoverridecommandlockouts
% The preceding line is only needed to identify funding in the first footnote. If that is unneeded, please comment it out.
\usepackage{cite}
% \usepackage{natbib}
% \usepackage{biblatex}
\usepackage{amsmath,amssymb,amsfonts}
\usepackage{bm}
\usepackage{mathtools}
\usepackage{algorithmic}
\usepackage{graphicx}
\usepackage{textcomp}
\usepackage{xcolor}
\usepackage{booktabs}
\usepackage{multirow}
\usepackage{multicol}
\usepackage{xspace}
\usepackage{xurl}
\usepackage{tikz}
\usepackage{environ}

\usepackage{subfig}
\usepackage{rotating}
\usepackage{amssymb}% http://ctan.org/pkg/amssymb
\usepackage{pifont}% http://ctan.org/pkg/pifont
\newcommand{\cmark}{\ding{51}}%
\newcommand{\xmark}{\ding{55}}%
\newcommand*\emptycirc[1][1ex]{\tikz\draw (0,0) circle (#1);} 
\newcommand*\halfcirc[1][1ex]{%
  \begin{tikzpicture}
  \draw[fill] (0,0)-- (90:#1) arc (90:270:#1) -- cycle ;
  \draw (0,0) circle (#1);
  \end{tikzpicture}}
\newcommand*\fullcirc[1][1ex]{\tikz\fill (0,0) circle (#1);} 
\def\checkmark{\tikz\fill[scale=0.4](0,.35) -- (.25,0) -- (1,.7) -- (.25,.15) -- cycle;} 

\def\BibTeX{{\rm B\kern-.05em{\sc i\kern-.025em b}\kern-.08em
    T\kern-.1667em\lower.7ex\hbox{E}\kern-.125emX}}

\newif\ifdraft
% \drafttrue
\draftfalse

% \newcommand{\nojan}[1]{\textcolor{red}{{\sf (NS:} {\sl{#1})}}}
% \newcommand{\anees}[1]{\textcolor{blue}{{\sf (AA:} {\sl{#1})}}}

\ifdraft
\newcommand{\nojan}[1]{\textcolor{red}{{\sf (NS:} {\sl{#1})}}}
\newcommand{\anees}[1]{\textcolor{blue}{{\sf (AA:} {\sl{#1})}}}
\else
\newcommand{\nojan}[1]{}
\newcommand{\anees}[1]{}
\fi

\newcommand{\Prv}{$\mathcal{P}$\xspace}
\newcommand{\Vrf}{$\mathcal{V}$\xspace}
\newcommand{\Cir}{$\mathcal{C}$\xspace}

\begin{document}

\title{Zero-Knowledge Proof Frameworks: A Systematic Survey}
\author{\textnormal{Nojan Sheybani$^1$, Anees Ahmed$^2$, Michel Kinsy$^2$, Farinaz Koushanfar$^1$} \\
$^1$UC San Diego, $^2$Arizona State University \\
$^1$\{nsheyban, farinaz\}@ucsd.edu, $^2$\{aahmed90, mkinsy\}@asu.edu}


% {\footnotesize \textsuperscript{*}Note: Sub-titles are not captured in Xplore and
% should not be used}
% \thanks{Identify applicable funding agency here. If none, delete this.}
% }

% \author{\IEEEauthorblockN{1\textsuperscript{st} Given Name Surname}
% \IEEEauthorblockA{\textit{dept. name of organization (of Aff.)} \\
% \textit{name of organization (of Aff.)}\\
% City, Country \\
% email address or ORCID}
% \and
% \IEEEauthorblockN{2\textsuperscript{nd} Given Name Surname}
% \IEEEauthorblockA{\textit{dept. name of organization (of Aff.)} \\
% \textit{name of organization (of Aff.)}\\
% City, Country \\
% email address or ORCID}
% \and
% \IEEEauthorblockN{3\textsuperscript{rd} Given Name Surname}
% \IEEEauthorblockA{\textit{dept. name of organization (of Aff.)} \\
% \textit{name of organization (of Aff.)}\\
% City, Country \\
% email address or ORCID}
% \and
% \IEEEauthorblockN{4\textsuperscript{th} Given Name Surname}
% \IEEEauthorblockA{\textit{dept. name of organization (of Aff.)} \\
% \textit{name of organization (of Aff.)}\\
% City, Country \\
% email address or ORCID}
% \and
% \IEEEauthorblockN{5\textsuperscript{th} Given Name Surname}
% \IEEEauthorblockA{\textit{dept. name of organization (of Aff.)} \\
% \textit{name of organization (of Aff.)}\\
% City, Country \\
% email address or ORCID}
% \and
% \IEEEauthorblockN{6\textsuperscript{th} Given Name Surname}
% \IEEEauthorblockA{\textit{dept. name of organization (of Aff.)} \\
% \textit{name of organization (of Aff.)}\\
% City, Country \\
% email address or ORCID}
% }

\maketitle

\begin{abstract}
Zero-Knowledge Proofs (ZKPs) are a cryptographic primitive that allows a prover to demonstrate knowledge of a secret value to a verifier without revealing anything about the secret itself. ZKPs have shown to be an extremely powerful tool, as evidenced in both industry and academic settings. In recent years, the utilization of user data in practical applications has necessitated the rapid development of privacy-preserving techniques, including ZKPs. This has led to the creation of several robust open-source ZKP frameworks. However, there remains a significant gap in understanding the capabilities and real-world applications of these frameworks. Furthermore, identifying the most suitable frameworks for the developers' specific applications and settings is a challenge, given the variety of options available. The primary goal of our work is to lower the barrier to entry for understanding and building applications with open-source ZKP frameworks.

In this work, we survey and evaluate 25 general-purpose, prominent ZKP frameworks. Recognizing that ZKPs have various constructions and underlying arithmetic schemes, our survey aims to provide a comprehensive overview of the ZKP landscape. These systems are assessed based on their usability and performance in SHA-256 and matrix multiplication experiments. Acknowledging that setting up a functional development environment can be challenging for these frameworks, we offer a fully open-source collection of Docker containers. These containers include a working development environment and are accompanied by documented code from our experiments. We conclude our work with a thorough analysis of the practical applications of ZKPs, recommendations for ZKP settings in different application scenarios, and a discussion on the future development of ZKP frameworks.
\end{abstract}

% \begin{IEEEkeywords}
% component, formatting, style, styling, insert
% \end{IEEEkeywords}

\section{Introduction}
% 
Motion planning is a key ingredient in autonomous robotic systems, whose aim is computing collision-free trajectories for a robot operating in environments cluttered with obstacles~\cite{lavalle2006planning}. 
Over the years, various approaches have been developed for tackling the problem, including potential fields~\cite{luo2024potential}, geometric methods~\cite{halperin2017algorithmic}, and optimization-based approaches~\cite{SchulmanDHLABPPGA14,MalyutaEtAl2022,MarcucciEA23}. %, and sampling-based planners~\cite{}. 
In this work, we focus on sampling-based planners (SBPs), which aim to capture the structure of the robot's free space through graph approximations that result from configuration sampling (typically in a random fashion) and connecting nearby samples. 
SBPs have enjoyed popularity in recent years due to their relative scalability, in terms of the number of robot degrees of freedom (DoFs), and the ease of their implementation~\cite{OrtheyCK24}. 

\begin{figure*}[h!]
  \centering
  \subfloat[$\X_{\dZ_2}^{\delta,\epsilon}$ sample set.]{
    \includegraphics[width=0.27\textwidth, trim={2.2cm 1.7cm 0.9cm 1.0cm},clip]{Images/ZN_2D.png}
    %\label{fig:2d_lattices:z}
    }
  \hfil
  \subfloat[$\X_{D_2^*}^{\delta,\epsilon}$ sample set.]{
    \includegraphics[width=0.27\textwidth, trim={2.2cm 1.8cm 0.9cm 1.0cm},clip]{Images/DN_2D.png}
    %\label{fig:2d_lattices:d}
    }
  \hfil
  \subfloat[$\X_{A_2^*}^{\delta,\epsilon}$  sample set.]{
    \includegraphics[width=0.27\textwidth, trim={2.4cm 1.7cm 0.9cm 1.0cm},clip]{Images/AN_2D.png}
    %\label{fig:2d_lattices:a}
    }
  \caption{Sample sets within a fixed disc in $\dR^2$, derived from the lattices $\dZ^2, D_2^*$ and $A^*_2$, which yield \decomp guarantees for the same values of $\delta$ and $\eps$. The set $\X_{\dZ_2}^{\delta,\epsilon}$ can be viewed as a tessellation of space using cubes. The set $\X_{D_2^*}^{\delta,\epsilon}$ is obtained by placing a (rescaled) standard grid, and then placing another point in the middle of each cube. The set $\X_{A_2^*}^{\delta,\epsilon}$ can be viewed as a rescaled hexagonal grid as each point is surrounded by a hexagon whose vertices are points in the set. Note that the density of $\X_{\dZ^2}^{\delta,\eps}$ and $\X_{D^*_2}^{\delta,\eps}$ is the same, and higher than the density of $\X_{A^*_2}^{\delta,\eps}$.}
  \label{fig:2d_lattices}
\end{figure*}

Another key benefit is the ability of SBPs to escape local minima (unlike potential fields) and global solution guarantees (in contrast, optimization-based approaches~\cite{SchulmanDHLABPPGA14}, which typically provide only local guarantees). Earlier work on the theoretical foundations of SBPs has focused on deriving probabilistic completeness (PC) guarantees for methods such as PRM~\cite{kavraki1996probabilistic} or RRT~\cite{LaVKuf01,KunzS14,Kleinbort.Solovey.ea.19}. PC implies that the probability of a given planner finding a solution (if one exists) converges to one as the number of samples tends to infinity. The work of~\citet{karaman2011sampling} initiated studying the quality of the solution returned by SBPs. Specifically, they introduced the planners PRM* and RRT*, and proved that the solution length of those planners converges to the optimum as the number of samples tends to infinity---a property called asymptotic optimality (AO). Subsequent work has introduced even more powerful AO planners for geometric~\cite{JSCP15,GammellBS20} and dynamical~\cite{HauserZ16,LiETAL16} systems.

Unfortunately, the practical relevance of the aforementioned theoretical findings remains limited due to the lack of meaningful finite-time implications. Specifically, when a solution is obtained using a finite number of samples, it is unclear to what extent its quality can be improved with additional computation time. Moreover, in cases where no solution is returned, it is uncertain whether a solution does not exist or if the algorithm simply failed to find one. Developing finite-time bounds through randomized sampling continues to be a significant challenge~\cite{DobsonMB15,shaw2024towards}.

Deterministic sampling methods such as grid sampling or Halton sequences~\cite{lavalle2006planning}, where samples are generated according to a geometric principle, can improve the performance of SBPs in practice and simplify the algorithm analysis. Specifically, some deterministic sampling procedures have a significantly lower dispersion than uniform random sampling, which implies that the former requires fewer samples to cover the search space to a desired resolution~\cite{janson2018deterministic}. 
Recently, Tsao et al.~\cite{tsao2020sample} have leveraged deterministic sampling to disrupt the asymptotic analysis paradigm by introducing a significantly stronger notion than AO, called \decomps, that yields finite-time guarantees for PRM-based algorithms such as PRM*~\cite{karaman2011sampling}, FMT*~\cite{JSCP15}, BIT*~\cite{GammellBS20}, and GLS~\cite{MandalikaCSS19}. Informally, a \emph{finite} sample set is \decomp for a given approximation factor $\eps>0$ and clearance parameter $\delta>0$, if the corresponding planner returns a solution whose length is at most $(1+\eps)$ times the length of the shortest $\delta$-clear solution. If no solution is found using a \decomp sample set then no solution of clearance $\delta$ exists. 

The work of~\citet{tsao2020sample} derived a relation between \decomps and geometric space coverage to obtain lower bounds on the number of samples necessary to achieve \decomps, as well as upper bounds accompanied with explicit (deterministic) sampling distributions. A follow-up work by~\citet{dayan2023near} has introduced an even more compact \decomp sample distribution that is more efficient than the one proposed in~\cite{tsao2020sample} or rectangular grid sampling. In particular, the staggered grid~\cite{dayan2023near} consists of two shifted and rescaled copies of the rectangular grid (see Figure~\ref{fig:2d_lattices} and Figure~\ref{fig:3d_lattices}). 

However, the work~\cite{dayan2023near} still leaves a significant gap between the lower bound in~\cite{tsao2020sample} and the upper bound obtained with the staggered grid. In practice, this gap limits the applicability of the \decomps theory to relatively low dimensions (up to dimension 6) due to the large number of samples currently needed to satisfy this property, which can lead to excessive running times. 

\vspace{5pt}
\noindent \textbf{Contribution.} In this work, we develop a theoretical framework for obtaining highly-efficient \decomp sample sets by leveraging the foundational theory of lattices\footnote{Lattices are point sets exhibiting a regular geometric structure, which are obtained by transforming the integer lattice $\dZ^d$. For instance, the aforementioned rectangular grid and the staggered grid can be viewed as lattices.}~\cite{conway2013sphere}, which has been instrumental in diverse areas from number theory~\cite{siegel_geometry_numbers}, coding theory~\cite{ebeling2013lattices}, and crystallography~\cite{sands1994introduction}. Specifically, we show that lattices can be transformed to obtain \decomp sample sets (Theorem~\ref{thm:decomp_lattices}) and develop tight theoretical bounds on their size (Theorem~\ref{thm:general_sample_complexity}), which allows to compare between different sample sets qualitatively. 
Using this machinery, we not only refine and generalize previous results on the staggered grid~\cite{dayan2023near} but also introduce a new highly efficient \decomp sample set that is based on the $\AN$ lattice, which is famous for its minimalist coverage properties~\cite{conway2013sphere}. We also initiate the study of a new property, which estimates the computational cost resulting from using a given sample set in a more informative manner than sample complexity. In particular, the property called collision-check complexity captures the amount of collision checks, which is typically a computational bottleneck.

From a practical perspective, when solving motion-planning problems using lattice-based sample sets, we show that our $\AN$-based sample sets can result in at least order-of-magnitude improvement in terms of running time over staggered-grid samples and two orders of magnitude improvements over rectangular grids. Moreover, $\AN$-based sample sets are vastly superior in practice to the widely-used uniform random sampling, which is evident in improved running times, success rates, and solution quality.

\vspace{5pt}
\noindent \textbf{Organization.} In Section~\ref{sec:preliminaries} we review basic definitions on motion planning and \decomps, and formally define our objectives. In Section~\ref{sec:lattices}, we develop a general tool for transforming lattices into \decomp sample sets. We obtain sample-complexity bounds for lattice-based sample sets in Section~\ref{sec:sample_complexity}, and generalize those bounds to collision-check complexity in Section~\ref{sec:collision_complexity}. We evaluate the practical implications of our theory in Section~\ref{sec:experiments}, and conclude with a discussion of limitations and future directions in Section~\ref{sec:future}.


% \itai{Added the intro. used some from the thesis-proposal, added different stuff at the end}
%  the field of autonomous robots, the problem of getting a robot “from point A to point B” can be divided into three general stages: estimating the robot's position, planning the robot's path and controlling the robot. We use estimation methods (like the Kalman Filters) to understand where we are in the world, we use planning methods to figure out how to reach the goal, and we use control methods (like PID) to follow the planned path during execution.


% Focusing on the planning part of the problem, instead of using the \emph{workspace} of the robot it is convenient to use a representation of it called a \emph{configuration space}---a parameterization of the robot’s position in space, which turns the set of points defined as a \emph{robot} to a single-point robot. A quick example would be thinking of a polygon in the workspace as three parameters: $(x,y)\in \mathbb{R}^2$ for its location, and $\theta\in[0,2\pi]$ for its rotation, which means the configuration space is $\mathbb{R}^2\times S^1$. Furthermore, we use the term \emph{free space} in both contexts to describe the area of the space with no obstacles. 


% Even though using configuration spaces is much more convenient in terms of the robot being a single point, it quickly becomes apparent that even simple configuration spaces of dimensions $d\geq 3$ can be challenging to properly describe (due to the need to describe the obstacles in the new space, among other reasons). Thus, instead of explicitly representing the whole configuration space, methods were developed to sample the space: sampling-based approaches aim to approximate the space via a graph structure that is induced by sampled configurations. This can drastically reduce the computational effort of path planning.


% One of the most widely used sampling-based algorithms is the \emph{probabilistic roadmap method} (PRM)~\cite{kavraki1996probabilistic}. This approach generates (typically random) samples across the space, and connects nearby samples while checking for collisions with obstacles, which gives rise to a graph data structure---a path between two nodes in the graph yields a collision-free path for the robot connecting between the two configurations corresponding to the end-point nodes. PRM has the theoretical guarantee to return a path with a probability tending to 1 if enough samples are generated~\cite{laddgeneralizing}. 


% Another well known sampling-based planner is the \emph{rapidly-exploring random trees} (RRT)~\cite{lavalle1998rapidly}: It randomly expands towards nearby samples in space, creating in the process a “tree” structure that eventually finds a path to the goal~\cite{kleinbort2018probabilistic}. Later, a notion of \emph{asymptotically optimal} (AO) algorithms was introduced: with infinite samples, the algorithm can converge to an \textbf{optimal} path. Both PRM and RRT  were expanded to AO versions (PRM*, RRT*) in a paper by Karaman et al.~\cite{karaman2011sampling}. RRT itself had seen many expansions, including dRRT/dRRT* to apply to multiple robots~\cite{solovey2015finding, dobson2017scalable}.

% Approximately optimal methods were demonstrated using deterministic sample sets, achieving good results in finite time, in Dayan et al.'s paper~\cite{dayan2023near}, demonstrating superior results over random sets---although those improvements diminish as the desired approximation factor of the optimal path lowers.

% Still, all these methods have a main limitation: the number of points required to guarantee finding a path rises exponentially with the robot's degrees of freedom~\cite{tsao2020sample}.


% Dayan et al.~\cite{dayan2023near}, using a "staggered grid" structure (recognized in this paper as the $\DN$ set), gave guarantess for an approximately-optimal solution in finite time, which outperformed random sets in certain situations. For this, they introduced the concept of a \decomp set, a set that generates such approximate solutions. Still, as we seek better and better approximations for the optimal path, the staggered grid in the paper falls off against random sets. The question that stands, then, is what other sample sets can be used to provide better results?


% In this paper, we would like to utilize \decomp sets and investigate a specific series of deterministic sample sets, using lattices---a generalization of the regular grid structure using a general set of mutually-independent base vectors (not necessarily the usual $(0,\dots,1,\dots,0)$ vectors). We first familiarize the reader with three different lattices \Lattices we intend on investigating, and then move on to using Dayan et al.'s~\cite{dayan2023near} definition of a \decomp set to define lattice sample sets as such sets. This definition tells us that these sets can give us a good approximation for the optimal solution at a finite time. 


% After that, we use our new lattice sample sets to investigate the upper bounds on the number of sample points, and on the sum of edge length in a typical PRM vertices-connecting $r$-Ball---something we use as a measure point to the algorithm's complexity, as it is known that collision checks along the edges are the bottleneck in today's PRM algorithms.


% We will end up demonstrating, theoretically and practically, that one lattice, $\AN$, stands out as performing much better than the regular grid often used in many MP algorithms.
\section{Zero-Knowledge Proofs}

Zero-Knowledge Proofs (ZKPs) are a cryptographic primitive that allow a prover \Prv to prove to a verifier \Vrf that they know a secret value $w$, called the witness, without revealing anything about $w$. \Prv does this by showing that they know a secret value $w$ such that $\mathcal{F}$ evaluated at $w$ equals some public output $y$. Formally, \Prv sends a proof attesting that $\mathcal{F}(x; w)=y$, where $x$ and $y$ are public inputs and outputs, respectively. ZKPs have three core attributes \cite{goldreich1994definitions}:
\begin{enumerate}
    \item \textbf{Soundness}: \Vrf will find out, with a very high probability, if a \Prv is dishonest if the statement is false.
    \item \textbf{Completeness}: An honest \Prv can convince \Vrf if the statement is true.
    \item \textbf{Zero-Knowledge}: If the statement is true, \Vrf will learn nothing about the \Prv's private inputs - only that the statement is true.
\end{enumerate}
In the following sections, we discuss the evolution of ZKPs, the nuances of specific classes and schemes, and provide a detailed overview of the current ZK landscape.

% \subsection{The Evolution of ZKPs}

% \subsection{Interactive vs. Non-Interactive}

% ZKPs can broadly be classed into two categories: interactive and non-interactive \cite{wu2014survey}. Interactive protocols, as the name suggests, require several rounds interaction before \Vrf is convinced that \Prv's proof is valid. This is done by \Vrf sending random challenges to \Prv until \Vrf is convinced that \Prv's proof is valid. Interactive ZKPs require that both \Prv and \Vrf stay online until \Vrf is convinced. This somewhat limits the utility of interactive ZKPs, as the proofs are \textit{designated-verifier}, meaning that \Prv's proof can only be used to be convinced a single verifier. A separate protocol must be performed for each new \Vrf. Conversely, non-interactive ZKPs are normally \textit{publicly verifiable}, meaning \Prv can generate a single proof in one-shot that any \Vrf can verify. Non-interactive ZKPs often rely on a trusted setup process from a third-party, or in some cases \Vrf, to generate randomness that allows for a proof to be generated that \Vrf accepts as valid without further interaction. Many non-interactive schemes aim to minimize proof size, which results in higher \Prv computational power requirements. This limits the scalability of these schemes, especially in scenarios where \Prv is resource-constrained. The interactivity of interactive ZKPs allows for a more scalable approach in terms of \Prv computation, albeit limiting the amount of verifiers that can verify a proof. If needed, there is a method for turning public-coin interactive ZKPs into non-interactive ZKPs. The Fiat-Shamir transform \cite{kilian1992note} replaces \Vrf's randomness with a random oracle (i.e. a cryptographic hash function), thus removing the interaction and turning interactive ZKPs into non-interactive ZKPs. \nojan{The Fiat-Shamir transform is not (necessarily) a transform if we are using a random oracle (as opposed to an instantiated hash function. The appropriate term to use is instead "Fiat--Shamir transform". Also, the sentence should clarify that only *public-coin* interactive proofs can be converted to non-interactive proofs via the FS transform. DONE}
 
\subsection{Taxonomy of ZKPs}

In this work, we analyze 25 ZK protocols. Amongst these protocols are a mix of interactive and non-interactive schemes. An in-depth explanation of the difference between interactive and non-interactive schemes can be found in Appendix \ref{sec:interactive}. From now on, we describe computation as circuits \Cir, as that is what they are referred to as in ZK literature. This is due to the process of arithmetization, which represents functions, such as Python/C++ code, as arithmetic circuits, then converts these circuits into a mathematical representation (e.g. polynomials) that can be used within ZKPs. Oftentimes, an intermediate step between the input and output is a set of constraints that describes the code/circuit. These constraints act as the basis for the mathematical representation. For brevity's sake, we do not discuss the details of arithmetization and refer to the brilliant explanations of \cite{LambdaClass2023ArithmetizationSchemes, ButerinQuadraticArithmeticPrograms}. In this text, we only treat arithmetization as a black-box and do not require the knowledge of specific details, only the inputs (e.g. code) and outputs (e.g. mathematical representation). Table \ref{tab:pros} compares the seminal ZK protocols at a high-level. Below, we describe the taxonomy of the general schemes that underlie our chosen ZK protocols in detail. 
% \nojan{Do we need arithmetization section?}

\begin{table*}[t]
\centering\resizebox{\textwidth}{!}{
\begin{tabular}{lll}
\toprule
\textbf{Construction} & \textbf{Key Advantages} & \textbf{Key Disadvantages}\\
\midrule
zk-SNARKs & Succinct, Publicly Verifiable & Trusted Setup Required, Computationally Expensive to Prove, Not Post-Quantum 
% & Applications requiring very fast verification (e.g., blockchain scaling, identity management). 
\\
\midrule
zk-STARKs & No Trusted Setup, Post-Quantum Secure, Scalable Prover, Publicly Verifiable & Larger Proof Sizes, Slow Verification 
% & Systems requiring transparency and post-quantum security where proof size is not paramount.
\\
\midrule
MPCitH & No Trusted Setup, Post-Quantum Secure, Publicly Verifiable & Slow Verification, Computationally Expensive Proving
% & Situations where proof data can be accumulated and verified in large batches or where flexibility is key. 
\\
\midrule
VOLE-ZK & Highest Scalability, No Trusted Setup, Post-Quantum Secure & Slow Verification, Designated Verifier
% & Similar to MPCitH, good for specialized use-cases with highly optimized MPC protocols. 
\\
\bottomrule
\end{tabular}}
\caption{Core Attributes of Popular ZKP Constructions}
\label{tab:pros}
\end{table*}

\begin{table}[t]
    \centering\centering\resizebox{\columnwidth}{!}{
\begin{tabular}{l|cccc}
\hline
& zk-SNARKs & zk-STARKs & MPCitH & VOLE-ZK \\
\hline
Prover complexity & $O(n\log{}(n))$ & $O(n\text{poly-log}(n))$& $O(n)$ & $O(n)$ \\
\hline
Verifier complexity & $O(1)$ & $O(\text{poly-log}(n))$ & $O(n)$ & $O(n)$ \\
\hline
Proof size & $O(1)$ & $O(\text{poly-log}(n))$ & $O(n)$ & $O(n)$ \\
\hline
Trusted setup & \cmark & \xmark & \xmark & \xmark \\
\hline
Non-interactive & \cmark & \cmark & \cmark & \xmark \\
\hline
Post-quantum secure & \xmark & \cmark & \cmark & \cmark \\
% \hline
% Crypto assumptions & \makecell{Elliptic curves \\ + pairing} & \makecell{Collision resistant \\ hashes} & \makecell{Symmetric key \\ primitives} & \makecell{Pseudorandom \\ generators} \\
\hline
Practical proof size & ~120-500 bytes & ~10 KB - 1 MB & ~10-1000 KB & ~5-200 KB \\
\hline
\end{tabular}}
\caption{Asymptotic attributes of presented ZKP constructions. We do note that, due to the variance of schemes within each construction, the algorithmic complexities are generalized and may not hold true for all schemes within a given construction.}
\end{table}

\textbf{Zero-Knowledge Succinct Non-Interactive Arguments of Knowledge (zk-SNARKs)} are, as the name suggests, a class of non-interactive protocols that boast small proof size \cite{ben2014succinct}. Although ZKPs were originally conceived in the late 1980's \cite{goldwasser2019knowledge}, zk-SNARKs were formally introduced about a decade after. Efficient instantiations of zk-SNARKs were introduced in the last decade, resulting in recent advancements in making zk-SNARKs practical and efficient for widespread use. 
% \nojan{zkSNARKs are *not* the earliest "ZK" constructions conceived; indeed, modern renditions of zkSNARKs are only 10-13 years old, while the concept of ZKPs is almost 40 years old. (While the zkSNARK of Micali also dates back to 30 years ago, efficient instantiations of it only date back to around 10 years ago as well) DONE}
This means there are much more mature open-source and real-world implementations available. The most common forms of zk-SNARKs are referred to as \textit{pre-processing zk-SNARKs}. One of the main drawbacks of these zk-SNARKs are that they require a trusted setup for every new circuit \Cir, which is computationally intensive and requires communication of large proving and verifying keys to the respective parties. 
Alongside this, \Prv must normally be computationally powerful in order to ensure small proof size. This is due to the fact that most zk-SNARKs are reliant on elliptic curve cryptography (ECC) as their underlying cryptographic arithmetic.
Recent works have introduced zk-SNARKs that can utilize \textit{universal} trusted setups \cite{plonk, chiesa2020marlin} for established maximum circuit sizes, and zk-SNARKs that do not require a trusted setup at all \cite{setty2020spartan, wahby2018doubly}. 
% These result in the evaluation and commitment of very large polynomials, which takes a toll on the complexity of any \Prv algorithm.
Due to the non-interactivity, zk-SNARKs are \textit{publicly verifiable}, meaning any verifier can verify them without recomputing the proof. One of the most common underlying schemes, especially in our highlighted frameworks, for zk-SNARKs is Groth16 \cite{groth16}, which improves upon the original Pinocchio \cite{parno2016pinocchio} protocol. zk-SNARK arithmetization \textit{typically} results in a set of constraints, called Rank 1 Constraint Systems (R1CS) \cite{belles2022circom}, which are then converted to a set of polynomials, called a Quadratic Arithmetic Program (QAP) \cite{gennaro2013quadratic}. We do note that there are different formats that are zk-SNARKs are compatible with, such as Algebraic Intermediate Representations (AIR) \cite{ben2018scalable} and Plonkish tables - we simply highlight R1CS as a prevalent constraint system. The Groth16 zk-SNARK generation and verification process can be represented at a high-level with the following algorithms:
\begin{itemize}
    \item $(\mathcal{VK, PK})\xleftarrow[]{}$ Setup(\Cir): A trusted third party or \Vrf run a setup procedure to generate a prover key $\mathcal{PK}$ and verifier key $\mathcal{VK}$. These keys are used for proof generation and verification, respectively. This setup must be repeated each time \Cir changes.
    \item $\pi \xleftarrow[]{}$ Prove($\mathcal{PK}$, \Cir, $x$, $y$, $w$): \Prv generates proof $\pi$ to convince \Vrf that $w$ is a valid witness.
    \item $1/0 \xleftarrow[]{}$ Verify($\mathcal{VK}$, \Cir, $x$, $y$, $\pi$): \Vrf accepts or rejects proof $\pi$. Due to soundness property of zk-SNARKs, \Vrf cannot be convinced that $w$ is a valid witness by a cheating \Prv.
\end{itemize}

\noindent In Appendix \ref{sec:recursive}, we describe how zk-SNARKs can be extended to allow recursive construction and verification of proofs.

As we stated, one of the drawbacks of traditional pre-processing zk-SNARKs is their reliance on a trusted setup per circuit \Cir.
% \noindent \nojan{appendix} Recent works \cite{bowe2019recursive, kothapalli2022nova, bitansky2013recursive} have shown the usability of recursive zk-SNARKs, which is verifying multiple zk-SNARKs in a single zk-SNARK. As the verification algorithm of zk-SNARKs is simply an arbitrary computation, it can be represented as a circuit \Cir. This enables one \Prv to generate many proofs, then generate a proof that verifies these proofs and send it to \Vrf. While this results in substantially more work on \Prv, \Vrf now only has to generate one proof to verify all of \Prv's data, rather than many individual proofs. 
% \nojan{mention ECC and bilinear mappings}
% \noindent\textbf{Interactive}
PLONKS, a subset of zk-SNARKs, are a class non-interactive ZK protocols that improve upon pre-processing zk-SNARKs by getting rid of the trusted setup per circuit \Cir, while adding a bit more arithmetic flexibility \cite{plonk}. PLONKs utilize the idea of a universal and updatable trusted setup, introduced in theory by \cite{cryptoeprint:2018/280} and in practice by \cite{cryptoeprint:2019/099}, in which a trusted setup procedure is done for circuits up to a certain size. 
% As mentioned in the previous section, some zk-SNARKs support universal and updateable setup - PLONKs are a specific included in this group.
% \nojan{"PLONKs" did not introduce the notion of universal and updatable setup; the first work to suggest this notion was that of \url{https://urldefense.com/v3/__https://eprint.iacr.org/2018/280__;!!Mih3wA!CvVqOHy12qd4JA_R87hM_kzJDnP_zincsfv36HU7wk-Ug48iOQHAiqfnlHplxrc01wVlpNigN-eTh3CLUudEOmE7apE$} , and the first efficient construction for NP-complete languages was that [Sonic](\url{https://urldefense.com/v3/__https://eprint.iacr.org/2019/099__;!!Mih3wA!CvVqOHy12qd4JA_R87hM_kzJDnP_zincsfv36HU7wk-Ug48iOQHAiqfnlHplxrc01wVlpNigN-eTh3CLUudEg2ZegzE$} ).
% \nojan{"PLONKs are not widely adopted": first of all I don't see why PLONK-type arithmetization based SNARKs need to be separately categorized (they're a subcategory) and take a look at L2Beat multiple deployments exist} There are also plenty of other constructions that achieve this property, e.g., Marlin, Lunar, ECLIPSE, etc. DONE}
Every circuit \Cir that fits within these size constraints can utilize the parameters generated by the universal trusted setup process. While PLONKs introduce a universal trusted setup, it comes at the cost of proof size and \Vrf runtime. PLONK proofs are normally $2-5\times$ the size of zk-SNARKs, and \Vrf runtime is marginally higher. It is important to note that, although PLONK proofs are larger than those of zk-SNARKs, proof size still remains in the KB range. The advantage that PLONKs have is that they are flexible in the commitment scheme they can use. By using the standard Kate commitments \cite{kate2010constant}, PLONKs become more zk-SNARK-like, as these commitments are based on ECC. FRI commitments \cite{ben2018fast}, which rely on Reed-Solomon codes and low-degree polynomials/testing for verifiers, can also be used to make PLONKs more zk-STARK-like. The type of commitment schemes allows developers to balance the tradeoff between performance and security assumptions. PLONK arithmetization is similar to that of zk-SNARKs, meaning that the resulting representation is a set of polynomials. To get there, PLONKs sets constraints for each gate (e.g. multiplication, addition) from the arithmetic circuit representation of the computation in the form of  Lagrange polynomials. Once the constraints are set, a special permutation function is used to check consistency between commitments. Finally, a final set of polynomials is constructed to fully represent the given computation. Overall, PLONKs provide a method to flexibly construct ZKPs with a less stringent trusted setup requirement, at the slight cost of performance. 
% \nojan{check this}

\textbf{Zero-Knowledge Scalable Transparent Arguments of Knowledge (zk-STARKs)}, which can be thought of as interactive oracle proof (IOP)-based zk-SNARKS, completely remove the dependence on trusted setup. Rather than using randomness from a trusted party, these protocols use publicly verifiable randomness for generating the necessary parameters for proof generation and verification. zk-STARKs achieve post-quantum security guarantees by utilizing collision-resistant hash functions as their underlying cryptography, rather than ECC. This increased security comes at a cost, as zk-STARK proofs are typically an order of magnitude larger than zk-SNARKs and PLONKs, and require more computational resources to generate and verify \cite{ben2018scalable}. The main contributor towards these drawbacks are the underlying data structure that are used in proof generation: Merkle trees. In zk-STARKs, Merkle trees are used to create a compact representation of the computation's execution trace. During proof generation, the computation's execution trace is arithmetized into polynomials, which are verified by performing low-degree testing, a process which ensures that the polynomials are of expected degree. Low-degree testing is enabled by the use of FRI commitments \cite{habock2022summary}. The polynomials are evaluated at certain points to verify their correct represenation of the execution trace, and these evaluations are used as the leaf nodes of the Merkle tree. The root of the Merkle tree then acts as a sort of commitment to these evaluated polynomials, hence allowing the verifier to simply verify the root, rather than verifying the whole computation trace. \cite{ashur2018marvellous} The use of Merkle trees are what enable the \textit{scalability} of zk-STARKs. While the Merkle trees support efficient verification, the proof size is drastically increased due to the inclusion of the material needed for verification, such as the Merkle root, polynomial evaluations, FRI commitments, and necessary Merkle branches. We note that there are IOP-based zk-SNARKs that stray away from this general protocol, but these steps are the most consistently utilized in current literature. Overall, zk-STARKs primarily benefit from being scalable and post-quantum secure with no trusted setup, at a significant cost to proof size and \Prv/\Vrf computation.

\textbf{MPC-in-the-Head (MPCitH)} ZKPs are a class of ZK protocols that take a completely novel approach towards proof generation and verification. The primary cryptographic basis is secure multiparty computation (MPC). MPC is a cryptographic primitive that allows for $n$ parties to jointly compute a function $f(x_1, ..., x_n)$, on private inputs from each party, without leaking any information about the private inputs. One of the prominent approaches to enable MPC is secret sharing, in which parties distributes secret shares of their private inputs amongst each other to compute a function. MPCitH, proposed by \cite{ishai2007zero}, allows for \Prv to simulate the $n$ MPC parties and following computation locally, or "in the head". Theoretically, any MPC protocol that can compute arbitrary functions can be transformed into a MPCitH ZKP. For $n$ parties $\{P_1,...,P_n\}$, secret shares are generated by each party and distributed to every other party. For the underlying arithmetic, the circuit \Cir is defined in an MPC manner to operate on secret shared data. \Prv can then simulate each parties' computation of the circuits with the secret shares they obtained from all other parties. After this is complete, \Prv has $n$ sets of messages and data that were generated and received by each party, called views. \Prv uses a standard commitment scheme to generate $n$ view commitments. Finally, \Prv and \Vrf interactively verify a subset of these views for consistency and correctness \cite{sidorenco2021formal}. While MPCitH protocols are innately interactive, they can be made non-interactive using the Fiat-Shamir transform. Theoretically, a huge advantage of the MPCitH approach is that MPC-friendly optimizations, which have been much further studied, can be utilized during proof generation to drastically improve \Prv efficiency and proof length. However, the most effective optimizations for MPC may translate to effective solutions for MPCitH. One of the core parameters MPCitH schemes aim to minimize is the communication complexity, as this directly reduces the amount of data that is present in each party's committed view. Just like zk-STARKs, MPCitH-based ZKPs do not require a trusted setup and are post-quantum secure, as MPC is thought to be generally quantum secure \cite{sidorenco2021formal}. Overall, MPCitH proposes a unique approach towards ZKP construction that are transparent post-quantum secure that allows flexibility in the underlying arithmetic to optimize the cost of proof generation and verification and proof size.

\textbf{Vector Oblivious Linear Evaluation (VOLE)-based ZK} 
protocols are a set of interactive techniques that achieve high efficiency and scalability through the use of information-theoretic message authentication code (IT-MAC)-based commitment schemes, which can be efficiently implemented using VOLE correlations. In VOLE-based ZKP protocols, the prover acts as the VOLE sender, while the verifier takes on the role of the VOLE receiver.
VOLE correlations are a pair of random variables, (\textbf{u}, \textbf{x}), known by \Prv and (\textbf{v}, $\Delta$), only known by \Vrf, in which \textbf{u}, \textbf{x}, and \textbf{v} are vectors, and $\Delta$ is a scalar key \cite{}. These variables satisfy the relation:
\begin{equation*}
    u_i = v_i + x_i \cdot \Delta
\end{equation*}
This functionality typically operates over a finite field.
Generally, in VOLE-based ZK, IT-MACs are used as commitments to authenticated wire values in arithmetic or boolean circuits representing a computation $\mathcal{C}$. \Prv demonstrates knowledge of a private vector $\textbf{w}$, which represents the witness, where $\mathcal{C}(\textbf{w}) = 1$, while proving the consistency throughout the protocol, without revealing any information about \textbf{w}.
\nojan{review and add VOLEitH}
VOLE-based proofs provide unparalleled scalability and communication optimizations, however, they are inherently designated-verifier protocols, meaning that \Prv must communicate with every \Vrf that aims to verify the proof, as \Vrf must maintain the secret $\Delta$ to ensure soundness. To address this, \cite{baum2023publicly} proposes a new VOLE-based paradigm, entitled VOLE-in-the-head (VOLEitH), which enables non-interactive VOLE-based ZK.

% \nojan{limbo has good explanation}


% \noindent\textbf{Recursive SNARKs}

% \nojan{make table comparing the zk schemes pros and cons}

% \subsection{Proving Systems}

% \noindent\textbf{Pinocchio}

% \noindent\textbf{Groth16}

% \subsection{Underlying Arithmetic}

% \noindent\textbf{Arithmetization}

% \noindent\textbf{Elliptic Curve Cryptography}
% Talk about https://hackmd.io/@benjaminion/bls12-381
\input{2_challenges}
\section{ZKP Libraries} 
\label{sec:libraries}
\nojan{Add swanky and picozk}
\begin{table*}[t!]
   \small
   \centering\resizebox{\textwidth}{!}{
   \begin{tabular}{cccccccc}
   \toprule
   & \multicolumn{3}{c}{\textbf{Usability}} & \multicolumn{3}{c}{\textbf{Accessibility}} & \\
   \cmidrule(lr){2-4}
   \cmidrule(lr){5-7}
   \textbf{Framework}&
   \textbf{Language(s)} & \textbf{Custom \Cir} & \textbf{License} & \textbf{Examples} & \textbf{Documentation} & \textbf{GitHub Issues} & \textbf{Last Major Update} \\ 
   
   \midrule
    & \multicolumn{7}{c}{\textbf{zk-SNARKs}} \\
   \midrule
   
   \textbf{Arkworks \cite{arkworks}} & Rust & \fullcirc & MIT, Apache-2 & \fullcirc & \cmark & \halfcirc & Dec. 2023 \\
   \textbf{Gnark \cite{gnark-v0.9.0}} & Go & \fullcirc & Apache-2 & \fullcirc & \cmark & \fullcirc & Dec. 2024 \\
   \textbf{Hyrax \cite{hyraxZK}} & Python & \halfcirc & Apache-2 & \halfcirc & \xmark & \emptycirc & Feb. 2018 \\
   \textbf{LEGOSnark \cite{legosnark}} & C++ & \halfcirc & MIT, Apache-2 & \halfcirc & \xmark & \emptycirc & Oct. 2020 \\
   \textbf{LibSNARK \cite{libsnark}} & C++, Java (xJsnark \cite{kosba2018xjsnark}) & \fullcirc & MIT & \fullcirc & \xmark & \halfcirc & Jul. 2020 \\
  \textbf{Zokrates \cite{eberhardt2018zokrates}} & Zokrates DSL & \fullcirc & LGPL-3.0 & \fullcirc & \cmark & \halfcirc & Nov.2023 \\
   \textbf{Mirage \cite{Mirage}} & Java & \halfcirc & MIT & \halfcirc & \xmark & \emptycirc & Jan. 2021 \\
   \textbf{PySNARK \cite{PySNARK}} & Python & \fullcirc & Custom (MIT-like) & \fullcirc & \cmark & \halfcirc & May 2023 \\
   \textbf{SnarkJS \cite{baylina2020iden3}} & Circom \cite{munoz2022circom} & \fullcirc & GPL-3 & \fullcirc & \xmark & \halfcirc & Oct. 2024 \\
   \textbf{Rapidsnark \cite{RapidSNARK}} & Circom \cite{munoz2022circom} & \fullcirc & GPL-3 & \fullcirc & \xmark & \emptycirc & Dec. 2023 \\
   \textbf{Spartan\cite{Spartan}} & Rust & \emptycirc & MIT & \halfcirc & \xmark & \halfcirc & Jan. 2023 \\
   \textbf{Aurora (libiop) \cite{SciprLab2023Libiop}} & C++ & \emptycirc & MIT & \halfcirc & \xmark & \halfcirc & May 2021 \\
   \textbf{Fractal (libiop) \cite{SciprLab2023Libiop}} & C++ & \emptycirc & MIT & \halfcirc & \xmark & \halfcirc & May 2021 \\
   \textbf{Virgo \cite{SunblazeUCB2023Virgo}} & Python & \emptycirc & Apache-2 & \halfcirc & \xmark & \emptycirc & Jul. 2021 \\
     \textbf{Noir \cite{Noir2023Documentation}} & Rust DSL & \fullcirc & MIT, Apache-2 & \fullcirc & \cmark & \fullcirc & Nov. 2024 \\
   % \textbf{Barretenberg \cite{AztecProtocol2023Barretenberg}} & 0 & 0 & 0 & 0 & 0 & 0 & 0 \\
   % \textbf{Bellman} & 0 & 0 & 0 & 0 & 0 & 0 & 0 \\
   \textbf{Dusk-PLONK \cite{DuskPlonk2023Rust}} & Rust & \emptycirc & MPL-2 & \halfcirc & \cmark & \fullcirc & Aug. 2024 \\
   \textbf{Halo2 \cite{Halo22023Book}} & Rust & \halfcirc & MIT, Apache-2 & \fullcirc & \cmark & \fullcirc & Nov. 2023 \\

   \midrule
   & \multicolumn{7}{c}{\textbf{MPC-in-the-Head}} \\
   \midrule

   \textbf{Limbo \cite{KULeuvenCOSIC2023Limbo}} & Bristol \cite{bristol} & \fullcirc & MIT & \halfcirc & \xmark & \emptycirc & May 2021 \\
 \textbf{Ligero (libiop) \cite{SciprLab2023Libiop}} & C++ & \emptycirc & MIT & \halfcirc & \xmark & \halfcirc & May 2021 \\ 
   % \textbf{ZKBoo} & 0 & 0 & 0 & 0 & 0 & 0 & 0 \\

   \midrule
   & \multicolumn{7}{c}{\textbf{VOLE-Based ZK}} \\
   \midrule
   \textbf{Mozzarella \cite{baum2022moz}} & Rust & \emptycirc & MIT & \halfcirc & \xmark & \emptycirc & Mar. 2022 \\
   \textbf{Diet Mac'n'Cheese \cite{dietmc}} & PicoZK \cite{picozk} & \fullcirc & MIT & \fullcirc & \xmark & \emptycirc & Sep. 2024 \\
  \textbf{Emp-ZK \cite{empzk}} & C++ & \fullcirc & MIT & \fullcirc & \xmark & \halfcirc & Sep. 2023 \\
   

   \midrule
   & \multicolumn{7}{c}{\textbf{zk-STARKs}} \\
   \midrule

   \textbf{MidenVM \cite{PolygonMiden2023MidenVM}} & Miden Assembly & \halfcirc & MIT & \halfcirc & \cmark & \fullcirc & Nov. 2023 \\
   % \textbf{Starky} & 0 & 0 & 4 & 0 & 0 & 0 & 0 \\
   \textbf{Zilch \cite{TrustworthyComputing2023Zilch}} & Java DSL & \halfcirc & MIT & \fullcirc & \xmark & \emptycirc & Apr. 2022 \\
   \textbf{RISC Zero \cite{RISCZero2023DeveloperDocs}} & Rust, C++ & \fullcirc & Apache-2 & \fullcirc & \cmark & \fullcirc & Dec. 2024 \\
   
   \bottomrule
   \end{tabular}}
   \caption{ZK Framework Attributes} 
      \label{tab:usability}
\end{table*}

% \nojan{maybe add circom}
In this section, we discuss the details of the 25 frameworks that we target in this work. We aim to highlight frameworks bred from both industry and academia. We primarily focus on works that present novel implementations of proving schemes that can be integrated with their own exposed high-level API for custom circuit design, or a general-purpose ZKP circuit development frontend, such as Circom \cite{munoz2022circom} or Zokrates \cite{eberhardt2018zokrates}.

Alongside the in-depth descriptions of each framework, we provide an evaluation of these frameworks at a high-level on usability and accessibility metrics, presented in Table \ref{tab:usability}. Our measurement of some metrics require further explanation:
\begin{itemize}
    \item Custom \Cir: \fullcirc $=$ Non-cryptography software engineer can build custom circuits, \halfcirc $=$ Building custom circuits requires deep knowledge of syntax; A developer could not read the code and understand it, \emptycirc $=$ Custom circuits require deep knowledge of protocol and syntax; normally requires manual translation of constraints to gates
    \item Examples: \fullcirc$=$ Plenty of examples are shared that fully show the capabilities of the system, \halfcirc $=$ Examples are included, but are not representative of the system's full capabilities
    \item Github Issues: \fullcirc $=$ Users and developers are both active in issues forum, \halfcirc $=$ Users are relatively active and developers are sporadically active, \emptycirc $=$ No activity
\end{itemize}

Table \ref{tab:description} in Appendix \ref{sec:app_libraries} outlines the discussed frameworks at a high level.

% \nojan{Specify between frontend and backend}

% \nojan{RISC zero, }

% \nojan{Look into PLONK languages: Fast provng and small proofs}

\subsection{zk-SNARKs}

\textbf{libsnark.} The \texttt{libsnark} C++ development library \cite{libsnark} is widely regarded as the original and most well-developed library for zk-SNARKs. This is highlighted by the fact that Zcash, the first real-world implementation of zk-SNARKs, was built upon \texttt{libsnark}. \texttt{libsnark} supports the Pinocchio \cite{parno2016pinocchio} and Groth16 \cite{groth16} proving schemes, alongside many different underlying elliptic curves. Much of the novelty of \texttt{libsnark} comes from the different forms of circuits that it supports. It supports R1CS and QAPs, as most frameworks do, but also supports higher level forms such as Unitary-Square Constraint Systems (USCS) and Two-input Boolean Circuit Satisfiability (TBCS) \cite{uscs}. The scheme used in \texttt{libsnark} is described as a preprocessing zk-SNARK, which simply highlights that trusted setup is performed before proof generation and verification. \texttt{libsnark} provides low-level "gadgets", which can be combined and built upon to represent the desired computation in R1CS format, however it is not the easiest way to develop zk-SNARKs in this library. \cite{kosba2018xjsnark} presents \texttt{xJsnark}, a high-level Java framework that allows a user to essentially code their computation in standard Java. Behind the scenes, this framework optimizes computation and outputs the computation in R1CS format. This output can be used directly with \texttt{libsnark}'s zk-SNARK generation script. Combined with \texttt{xJsnark}, \texttt{libsnark} is a highly-accessible option for inexperienced ZKP developers.

\textbf{gnark.} The \texttt{gnark} library \cite{gnark-v0.9.0} enables developers to build zk-SNARK-based applications using the high-level API it offers in Go language. The primary focus of \texttt{gnark} is runtime speed \cite{ConsenSys2023Gnark}. It offers both Groth16~\cite{groth16} and PLONK~\cite{plonk} (with KZG and FRI polynomial commitment) SNARK protocols. It offers a lot of curves, and can build R1CS circuits. In terms of hashing, it offers MiMC~\cite{mimchash}, SHA2, and SHA3 gadgets out-of-the-box. It also offers a collection of high-level gadgets for ease of building custom circuits. This framework exposes a high-level API that allows users to build their own gadgets, while utilizing the Go standard language and the provided gadgets. Recently, \texttt{gnark} has introduced GPU support with the support of the Icicle library \cite{icicle}. This work is in active development and seems to have an active community around it, making it an accessible option for inexperienced ZKP developers. We recommend this for beginners and experts alike for almost any custom applications. This framework utilizes a readable and robust API that any user can take advantage of and build custom applications with.

\textbf{arkworks.} The \texttt{arkworks} Rust ecosystem~\cite{arkworks} is an extensive and modular collection of libraries that can be used for efficient zk-SNARK programming. This ecosystem provides highly efficient implementations of arithmetic over \textit{various} curves and fields, even allowing curve specific optimizations. The main offering of \texttt{arkworks} is a generic application development framework that supports both experienced and non-experienced zk-SNARK developers. This framework enables high-level zk-SNARK development, as it allows users to implement their circuit as constraints (R1CS), while abstracting out details of SNARKs and curves, using an \texttt{arkworks} library. To venture into lower-level optimizations, \texttt{arkworks} provides libraries for the user to describe their circuit in native code. This allows the users to make several design decisions, such as specifying which proving system, such as Groth16, they would like to use. Alongside this, \texttt{arkworks} also provides libraries implementing low-level finite field, elliptic curve, and polynomial interfaces. In addition to SHA256, ZKP-friendly hashes such as Pedersen \cite{pedersenhash} and Poseidon \cite{poseidonhash} hashing are also offered. The \texttt{arkworks} development ecosystem is actively maintained and has an active community. We recommend this framework for users that have a deep knowledge of ZKPs, as one of the main advantages of arkworks, other than it's fantastic and usable codebase, is the ability to tweak certain parameters to optimize operations for your custom application.

\textbf{hyraxZK.} \label{sec:hyrax} \texttt{Hyrax} is a "doubly-efficient" zk-SNARK scheme, providing a concretely efficient prover and verifier, with low communication cost and no trusted setup \cite{wahby2018doubly}. Instead of following a standard underlying zk-SNARK structure, \texttt{Hyrax} is built on top of the Giraffe interactive proof scheme \cite{wahby2017full}. The authors apply a technique to reduce communication cost and add cryptopgraphic operations to turn the interactive proof into a ZKP. With the addition of optimized cryptographic commitments, the concrete cost of this scheme is significantly reduced and results in an interactive ZKP scheme. Using the Fiat-Shamir transform \cite{kilian1992note}, this scheme is made non-interactive. \texttt{hyraxZK} \cite{hyraxZK} provides a cleanly-developed Python and C++ development environment using \texttt{Hyrax} as the underlying zk-SNARK scheme. The provided framework is well-developed, however there is a lack of documentation that makes it challenging to build custom circuits.

\textbf{libspartan.} \texttt{libspartan} \cite{Spartan} is a Rust library that implements the \texttt{Spartan} zk-SNARK proof system \cite{setty2020spartan}. \texttt{Spartan} is a transparent zk-SNARK proof system, meaning that it requires no trusted setup. \texttt{libspartan} utilizes a Rust implemention of group operations on prime-order group Ristretto \cite{ristretto} and elliptic curve Curve25519 \cite{bernstein2006curve25519}, which ensures security and speed.
% Similar to \texttt{Hyrax}, outlined in section \ref{sec:hyrax}, \texttt{Spartan} is developed by building upon an interactive proof protocol, which is the sum-check protocol in \texttt{Spartan}'s case. The resulting interactive argument of knowledge is then turned into a zk-SNARK using the same techniques as \texttt{Hyrax}.
By adding a new commitment scheme, alongside a novel cryptographic compiler and a compact encoding of R1CS instances, \texttt{Spartan} is able to achieve the first transparent proof system with sub-linear verification costs and a time-optimal prover, at the cost of memory-heavy computation on the prover side. \texttt{libspartan} is a well-developed and maintained framework, however implementing custom functions is not very straightforward based on the provided documentation. Developing a custom ZKP circuit in \texttt{libspartan} requires the user to have the parameters of the R1CS instance, alongside knowledge of how to encode the constraints into R1CS matrices. Depending on the size of the ZKP circuit, this process can be very rigorous and involved, while also requiring a full knowledge of R1CS representations.
\texttt{Zokrates} \cite{eberhardt2018zokrates} provides a high-level API to build an R1CS for custom ZKP circuits, however a developer then has to manually convert these into a format that is readable by \texttt{libspartan}, which can be time-intensive depending on the number of constraints in the circuit. We only recommend this framework to users that have an in-depth knowledge of ZK constraint systems, however, we do note that this framework's backend is state-of-the-art and, upon integration with a standard frontend, would be a perfect solution for most ZK applications.

\textbf{Mirage.}
\texttt{Mirage} \cite{kosba2020mirage} is a universal zk-SNARK scheme and aptly named Java framework \cite{Mirage} implementing such scheme. \texttt{Mirage}'s main contribution is a universal trusted setup, such that trusted setup does not have to be performed everytime the circuit changes, as is done in zk-SNARKs. This saves a great amount of time and computation at the cost of higher proof computation overhead. This work introduces the idea of \textit{separated zk-SNARKs}, which enables efficient randomized checks in zk-SNARK circuits. This results in simplified verification complexity. Combining this with their novel universal circuit generator that produces circuits linear in the number of additions and multiplications, the \texttt{Mirage} zk-SNARK scheme is introduced. The underlying scheme and circuit generator are implemented in the \texttt{mirage} codebase, which has a Java frontend for circuit generation and a C++ backend implementing \texttt{Mirage} on top of \texttt{libsnark}. The core of development is done in \texttt{mirage}'s universal circuit generator, as that is where the ZKP circuits are specified by the user. This codebase provides very readable and diverse examples that highlight the use cases of their high-level Java API. 
Not only is there a bit of a learning curve to get acquainted with \texttt{mirage}'s syntax, but we also found that the codebase is relatively outdated, meaning that the code no longer compiles.
% it is a very straightforward way to develop universal zk-SNARKs, and does not require an in-depth knowledge of ZKPs to build ZK applications.

\textbf{LegoSNARK.} 
\texttt{LegoSNARK} \cite{campanelli2019legosnark} is a zk-SNARK scheme and library that focuses on linking SNARK "gadgets" together to build zk-SNARKs with a modular approach. This library implements the modular zk-SNARKs in the form of commit-and-prove zk-SNARKs (CP-SNARKs) \cite{lipmaa2016prover}, which are a class of zk-SNARKs that prove statements about committed values. As previous CP-SNARK schemes are limited due to their reliance on a single commitment scheme, one of the most important contributions of this work is a generic construction that can convert a broad class of zk-SNARKS, such as QAP-based, to CP-SNARKs. The \texttt{LegoSNARK} library \cite{legosnark} provides end-to-end proving and verification using the proposed scheme in a C++ package. This work builds upon \texttt{libsnark}, albeit with integration to high-level \texttt{libsnark} frameworks, such as \texttt{xJsnark}. Nevertheless, this library provides readable examples for developing gadgets, making it relatively easy for experienced C++ developers to build custom gadgets for their ZK applications without an in-depth knowledge of ZKPs. We recommend this framework to users that are building modular applications that benefit from CP-SNARKs, such as matrix arithmetic.

% \anees{Mention license of all libraries?}

\textbf{PySNARK.}
\texttt{PySNARK} \cite{PySNARK} is a Python library that allows developers to use pure Python syntax to develop zk-SNARKs with various backends. PySNARK gives users access to \texttt{libsnark}, \texttt{qaptools}, \texttt{zkinterface}, and \texttt{snarkjs} backends. Compiling computation with the \texttt{libsnark} and \texttt{qaptools} performs proof generation and verification using the Groth16 and Pinnochio proving systems, respectively. Using the \texttt{zkinterface} backend simply generates \texttt{.zkif} files that can be used with the \texttt{zkinterface} package for proof generation and verification, where the underlying scheme can be chosen. Similarly, using the \texttt{snarkjs} backend generates the witness and R1CS files that can be used within our provided \texttt{snarkjs} environment. Overall, \texttt{PySNARK} is a brilliantly documented and developed library for beginners with zk-SNARKs, however it is not actively maintained. Developers that are comfortable with Python should have no trouble developing ZK applications once they become familiar with the library's syntax. Due to the Python compilation process, \texttt{PySNARK} experiences non-ideal operation times, so users should primarily use this for testing applications on the Groth16 proving system, but not for practical application development.


\textbf{SnarkJS + RapidSNARK.}
\texttt{SnarkJS} \cite{baylina2020iden3} is built on Javascript (JS) and Pure Web Assembly (WASM) and supports the Groth16, PLONK, and FFLONK underlying proving schemes. This framework accepts circuits designed in \texttt{circom} \cite{munoz2022circom}, which provides a very accessible frontend with a well-documented API for building ZK circuits. The protocols that are supported all require trusted setup, whether it be a circuit-specific setup for Groth16, or a universal setup for PLONK/FFLONK. Also, switching between ZK schemes is simply done by specifying the desired scheme as a command line argument. \texttt{SnarkJS} provides a multi-step universal setup protocol that all programs perform, alongside a Groth16-specific setup. Alongside this, the circuit to proof compilation process is done in a modular way that allows for closer debugging. In the proof generation process, the circuit characteristic's are listed for the developer (e.g. constraints, public inputs) which enables quick sanity checks. Finally, \texttt{SnarkJS} provides simple routes to turning the verifier into a smart contract, or performing the end-to-end ZKP process in browser, due to the JS and WASM backend. \texttt{RapidSNARK} \cite{RapidSNARK} is built upon C++ and Intel assembly by the same developers, and significantly improves upon \texttt{SnarkJS}. Using a very similar API, and even accepting \texttt{SnarkJS}-generated files as inputs (e.g. proving/verifier keys, witness), \texttt{RapidSNARK} allows for faster proof generation with a simple change in command line arguments from the \texttt{SnarkJS} commands. The main advantage of this framework is the utilization of parallelization within proof generation, yielding much faster results than \texttt{SnarkJS}, however the downside is that only Groth16 proofs are supported. While \texttt{SnarkJS} is more actively maintained than \texttt{RapidSNARK}, both frameworks are highly accessible for those with little experience in developing ZK applications, due to the ability to utilize a \texttt{circom} frontend.

\textbf{Virgo.}
\texttt{Virgo} \cite{zhang2020transparent} is an implementation of a novel interactive doubly-efficient ZK argument system. The main advantage of this protocol is the lack of trusted setup, which is oftentimes the most cumbersome task in zk-SNARKs. \texttt{Virgo} sees the most benefits for layered arithmetic circuits, rather than all general arithmetic circuits, as it is based off the GKR protocol \cite{goldwasser2015delegating}, which also is only catered towards structured circuits. General arithmetic circuits are addressed in a follow up work, \texttt{Virgo++} \cite{virgoplus}. The open-source implementation of this work does not have ZK commitments implemented yet, which is why we do not consider it in our survey. The main enabling factor of \texttt{Virgo} is a novel ZK verifiable polynomial delegation (zkVPD) scheme, which can essentially be seen as a commitment scheme in this scenario. 
% The underlying operations of the \texttt{Virgo} implementation are the combination of the zkVPD and GKR schemes. 
Due to the reliance on zkVPD and the allowed interactivity in this scheme, the implementation only relies on lightweight cryptography, making it a feasible development solution. While an impressive solution with great results, the repository is not actively maintained and lacks clear documentation, meaning it is not the most suitable candidate for ZK application developers.

\textbf{libiop} \label{sec:aurora}
The \texttt{libiop} framework \cite{SciprLab2023Libiop} is a collection of three protocol implementations: Aurora \cite{aurora}, Fractal \cite{fractal}, and Ligero \cite{ames2017ligero}. Ligero falls under the MPCitH category, so it is discussed later in the paper. Aurora and Fractal are both post-quantum, transparent zk-SNARKs, which classifies them more as succinct zk-STARKs. However, the authors classify their work as zk-SNARKs, which is why they are discussed here. Both works outperform prior zk-SNARKs by proposing new interactive oracle proofs (IOPs). Fractal proposes a holographic IOP \cite{babai1993transparent}, while Aurora proposes an IOP based around Reed-Solomon codes.
As for the \texttt{libiop} implementations, it does not seem to be actively maintained. While there are a few example applications for each protocol, the most useful tool in was the benchmarking scripts that were provided. This allows users to input parameters, such as number of constraints and variables, to specify a random circuit and outputs the performance metrics of the protocol. This shows how the protocols scale based on the size of the circuit. These parameters can be extracted from R1CS files (made by frameworks such as Zokrates), using our provided \texttt{R1CSReader} scripts. While the benchmarking is convenient, developing custom applications with this framework requires a deeper knowledge of the protocol that may not be easily accessible to all developers. We only recommend this to users that have a deep knowledge of the literature that these frameworks stem from.


\textbf{Noir.}
\texttt{Noir} \cite{Noir2023Documentation} is a general Rust-like framework for developing applications based on ZKPs. Fundamentally, \texttt{Noir} is a domain-specific language that resembles Rust. It enables one to build circuits that implement complex logic without having to learn the low-level details of ZKP systems. Since it acts like a generalized front end, it is capable of building circuits for a variety of back ends. Currently, Barretenberg \cite{AztecProtocol2023Barretenberg} serves as the default back end, and generates PLONK proofs and Solidity contracts. The Barretenberg back end can also use WASM to create proofs and verify them directly in the browser. Arkworks is also available as an out-of-the-box back end, which can generate Groth16 and Marlin proofs. This generalization is possible because Noir framework compiles the circuit to an intermediate language referred to as ACIR (Abstract Circuit Intermediate Representation), which can then be further compiled to specific R1CS or arithmetic circuit compatible with a specific back end. The framework also provides a Typescript library for direct integration into web applications. There is active development going on, but Noir currently supports a full control flow with the ability to create custom circuits using readable code. This is a great option for developers who would like to avoid the details of ZKPs and build applications using a Rust-like DSL. We recommend this for those who want to build simplistic applications who have little experience with ZKPs.
% \anees{Mention Language syntax features, ACIR Supported OPCODES, maybe roadmap?}
% \nojan{edit}

% \textbf{Bellman}
% \nojan{need to update to PLONK}

% \texttt{Bellman} \cite{} is a zk-SNARK development library built on Rust. This framework is primarily focused on implementing the Groth16 algorithm. \texttt{Bellman} has grown to prominence by being a core framework used to implement Zcash \cite{hopwood2016zcash}, one of the first mainstream implementations of zk-SNARKs. This work only supports one elliptic curve, BLS12-381,  meaning operations are highly optimized for this curve construction. This curve was designed specifically for use in Zcash, but has grown to be a standard underlying ECC for zk-SNARK implementations, due to its effectiveness and security guarantees. \texttt{Bellman} is one of the less accessible development frameworks featured in this work, as it only provides a very low-level API. Although this framework is actively maintained, it effective development \texttt{bellman} requires an in-depth knowledge of zk-SNARKs and pairing-based cryptography using the BLS12-381 ECC \nojan{double check this}. \nojan{Talk about community edition extensions bellman ce and plonk}

% \textbf{Barretenberg}


\textbf{Dusk-PLONK.}
\texttt{Dusk-PLONK} \cite{DuskPlonk2023Rust} is a pure Rust implementation of the PLONK proving system. This implementation supports operation over the BLS12-381 and JubJub elliptic curves. The developers of this framework use Kate commitments \cite{kate2010constant} as their primary polynomial commitment scheme to utilize its homomorphism and maintain constant size commitments. The provided codebase is extremely detailed and well-commented and provides helpful documentation. Similar to other PLONK frameworks, \texttt{Dusk-PLONK} only provides a very low-level API for custom circuit development. To build a custom circuit, developers must translate their computation into an arithmetic or boolean circuit gate format (e.g. add, multiply). This is perfectly digestible for small circuits, as shown in the examples, however becomes an intensely laborious task as the circuit and number of inputs or input dimensions scales up. While the code is well-written and yields excellent results, this framework requires a more sophisticated high-level API that utilizes common software engineering structures to build custom circuits before new developers can start building practical ZKP applications with it. We do note that this is a fantastic implementation of the PLONK proving system for and recommend it for developers that have experience with logic design and ZKPs. 

\textbf{Halo2.}
Built by the same creators of Zcash and the original Halo \cite{bowe2019recursive} framework, the Halo2 framework \cite{Halo22023Book} optimizes upon some of the inefficiencies of its predecessors by utilizing a PLONK-ish scheme as the underlying proving system. The underlying polynomial commitment scheme in this framework is Kate commitments. In its original repository and documentation, building a custom circuit with Halo2 requires a developer to design their computation in the form of a circuit, by implementing gates and utilizing them to build a \textit{chip}. This can be relatively confusing for new developers.
However, Halo2 is a powerful proof system that is utilized widely across the industry, including a prominent verifiable machine learning framework, \texttt{ezkl} \cite{ZkonduitInc2023EZKL}. This prominence has garnered a strong community backing the framework and has resulted in many works that either provide more examples of how the framework can be used \cite{Halo2Club2023}, or expose higher-level APIs for building custom circuits. Overall, while the Halo2 framework only exposes a lower-level API for custom circuit building, the community around it makes it a relatively accessible solution for practical application of PLONKs. We believe this is a good framework for those experienced with applied cryptography and interest in building machine-learning focused applications.

% \textbf{Fractal}
% \cite{fractal} \cite{SciprLab2023Libiop}

% \subsection{Interactive ZKPs}

% \textbf{Orion}
% \nojan{maybe remove}

% \textbf{Virgo++}
% \nojan{The implementation doesn't have ZK commitments even applied}

\subsection{MPC-in-the-head}

\textbf{Ligero (libiop).}
The Ligero \cite{ames2017ligero} protocol is implemented in \texttt{libiop} \cite{SciprLab2023Libiop} framework. This interactive protocol applies the general IKOS \cite{ishai2007zero} transformation that transforms MPC-based interactive proofs into ZKPs, which is typical for MPC-in-the-Head (MPCitH) systems. This means that the key aspect of designing the Ligero is the underlying MPC protocol. While this protocol is interactive, it can be transformed into a zk-SNARK using the Fiat-Shamir transform, just like any other interactive protocol. Additionally, the Ligero protocol only relies on collision resistant hash functions for the underlying cryptography and does not require a trusted setup. As this is implemented using the same backend as the Aurora and Fractal zk-SNARK protocols, all implementation details remain the same as described in section \ref{sec:aurora}.

\textbf{Limbo.}
Similar to Ligero, \texttt{Limbo}'s implementation \cite{KULeuvenCOSIC2023Limbo}  and underlying protocol \cite{limbo} is reliant on the IKOS transformation that MPCitH protocols often rely on. \texttt{Limbo} improves upon Ligero by highlighting the tradeoff between MPCitH parties involved, proof size, and runtime.
% For instance, \texttt{Limbo} allows a user to minimize proof size by having more underlying MPC parties run the protocol in parallel, at the cost of a higher runtime/complexity.
The main work \texttt{Limbo} compares to is Ligero, as they are both transparent MPCitH schemes that only rely on collision resistant hash functions. \texttt{Limbo} claims to work better on small and medium circuits. While the \texttt{Limbo} framework is not as extensively developed, maintained, and documented as some of the other frameworks highlighted in this work, it greatly benefits from its ability to take Bristol Circuit (BC), a common way to describe MPC circuits \cite{bristol}, descriptions as inputs. This allows developers to build custom applications by describing their general computations in BC format. We provide a simple pipeline for developing BCs, alongside examples using readable syntax. We recommend this for users who have experience building optimized BCs and have a relatively deep understanding of MPC.

% \textbf{}

% \subsection{PLONKs}

\subsection{VOLE-Based ZK}


\textbf{Diet Mac'n'Cheese}
\nojan{add swanky stuff}
\texttt{Diet Mac'n'Cheese} \cite{dietmc} is a novel framework that implements the Mac'n'Cheese protocol \cite{baum2021mac}, a Vector Oblivious Linear Evaluation (VOLE)-based zero-knowledge protocol over the $\mathbb{Z}_{2^k}$ ring. Similar to Moz$\mathbb{Z}_{2^k}$arella, this is a crucial step in making ZKPs more practical, as most real-world compute hardware operates on integer rings, and not finite fields. \texttt{Diet Mac'n'Cheese} makes many improvements to the state-of-the-art in VOLE-based ZK protocols by optimizing the underlying sVOLE subprotocol. This optimization yields significant performance improvements over prior VOLE protocols that operate over integer rings. The provided implementation comes in the form of a C++ package that directly implements the proposed scheme and uses the Swanky ecosystem \cite{swanky} for easy integration. This framework is still in its early stages of development and currently lacks extensive documentation and concrete examples, making it harder for new ZKP developers to use it. Alongside this, \texttt{Diet Mac'n'Cheese} currently only supports fixed-point integer operations. It exposes a low-level API that requires a developer to explicitly define all computations as arithmetic and boolean gates that are operated on using the framework's provided functions. However, a recent work has introduced a Python frontend with great documentation that can translate Python code into an intermediate representation that is recognized by the \texttt{Diet Mac'n'Cheese} framework. This frontend, entitled PicoZK \cite{picozk}, contains many examples and is even able to integrate with the popular numpy and pandas packages. PicoZK is a perfect pairing with \texttt{Diet Mac'n'Cheese} and allows for the development of simple applications. We recommend this framework to any developer that aims to build a scalable application that is conducive to a designated-verifier environment, such as federated or split learning. We do note that any floating point operations that are done with this framework must be converted to fixed-point.

\textbf{emp-zk.}
% \nojan{mention sVOLE}
\nojan{shorten} The \texttt{emp-zk} development framework \cite{empzk} is a part of the \texttt{emp-toolkit} \cite{emptool}, a collection of cryptographic front-ends and back-ends that allow for easy development of multi-party computation applications. Alongside ZKPs, \texttt{emp-toolkit} also provides libraries for garbled circuits and oblivious transfer. \texttt{emp-zk} has implementations of three novel interactive ZK systems:
\begin{itemize}
    \item Wolverine \cite{weng2021wolverine}, the first of these systems, presents a constant-round, scalable, and prover-efficient interactive ZK scheme.
    \item Mystique \cite{weng2021mystique}, built on top of Wolverine, focuses on machine learning applications. This work presents efficient conversions for arithmetic and boolean values, fixed-point and floating-point values, and committed and authenticated values. 
    % Alongside this, this work presents an efficient matrix multiplication ZKP by utilizing Freivald's algorithm \cite{freivalds1977probabilistic}.
    \item Quicksilver \cite{yang2021quicksilver}, also built on top of Wolverine, further improves communication costs and scalability.
\end{itemize}

The main primitive these schemes take advantage of is subfield Vector Oblivious Linear Evaluation (sVOLE), which the authors extend and optimize for their ZK scheme. 
% sVOLE is used to efficiently realize the information-theoretic message authentication codes (IT-MAC) commitment scheme.
% , which allows a prover and verifier to interactively create a commitment.
For sake of brevity, we spare the technical detail in this paper and refer to \cite{Weng2023VOLEBasedInteractive} for an excellent explanation. \texttt{emp-zk} provides a very user-friendly interface to all 3 ZK systems, with clear-cut examples. Although documentation is not explicitly provided, \texttt{emp-zk} largely relies on C++ syntax and does not require much knowledge about the underlying work in ZKPs, making it one of the more accessible options. One potential downside of these systems are that they are interactive, meaning all proofs are \textit{designated-verifier}. We highly recommend this framework for users who are building custom machine learning-based custom applications that rely on floating-point operations, or applications that rely on scalability (e.g. database operations).

\textbf{Moz$\mathbb{Z}_{2^k}$arella.}
This work \cite{baum2022moz} presents a new protocol that utilizes an novel vector oblivious linear evaluation (VOLE), a tool from secure two-party computation, extension to perform zero knowledge proof operations efficiently over the integer ring $\mathbb{Z}_{2^k}$. This is very important as most ZK systems are made to operate over finite fields, which is not representative of modern CPUs. The proof system is coined with the term \texttt{Quarksilver}. This protocol outperforms the previous state-of-the-art VOLE-based works that operate over finite fields. The accompanying implementation enables development of ZK applications with the \texttt{Quarksilver} protocol as the underlying scheme. The \texttt{Moz$\mathbb{Z}_{2^k}$arella} repository is not actively maintained, however has 3 sub-libraries for oblivious transfer, garbled and arithmetic circuits, and private set-intersection. Within these sub-libraries there are several examples that explain how to use the \texttt{Moz$\mathbb{Z}_{2^k}$arella} syntax, including examples for \texttt{Quarksilver}. While the examples are somewhat clear, using this library to build custom applications requires a deep knowledge of the underlying proof system, as users must be aware of the parameters that are being set on a per application basis. We only recommend this to users who's applications fully rely on using the specific underlying protocol in this framework.



\subsection{zk-STARKs}

\textbf{Miden VM.}
\texttt{Miden VM} \cite{PolygonMiden2023MidenVM} is a zero-knowledge virtual machine (zkVM) implemented in Rust, in which all programs that are run generate a zk-STARK that can be verified by anyone.
\texttt{Miden VM} is designed as a stack machine, consisting of a stack, memory, chiplets, and a host. The stack, the main user-facing component, is a push-down stack of field elements, which is where inputs and outputs of operations are stored. Increasing the amount of inputs that are initialized on the stack before program execution increases the verifier cost. Whatever is left on the stack after program computation is declared as a public input to the verifier, which also increases cost to the verifier. A prover's private inputs must be pushed to the stack during program computation to be kept private.
The aim of Miden VM is, in their own words, to "make Miden VM an easy compilation for high-level languages such as Rust" \cite{PolygonMiden2023VMOverview}. As these compilers do not yet exist, the only way to build custom circuits is using Miden's assembly language, a very low-level API that interfaces with the Miden stack, and Miden chiplets, which are optimized assembly-based modules that perform common operations, like field arithmetic. Although \texttt{Miden VM} is Turing complete and offers standard control flow, it is often challenging for a developer to translate their desired computation to assembly commands and managing the stack at the same time, especially as the size of computation scales up. While \texttt{Miden VM} is a very valuable tool, we believe that its highest potential will be achieved upon completion of an accompanying compiler from a high-level language to Miden assembly. We recommend that users use this to benchmark certain atomic operations, but to avoid building custom applications with this framework due to the lack of a frontend.

% \textbf{Starky}

\textbf{Zilch.}
The \texttt{Zilch} framework \cite{mouris2021zilch} consists of a Java-like frontend (ZeroJava) that interfaces with a novel zero-knowledge MIPS processor model (zMIPS) \cite{TrustworthyComputing2023Zilch} to enable efficient interactive zk-STARK proof generation for custom computations. The ZeroJava frontend is highly sophisticated and is one of the only frameworks to enable an object-oriented programming approach. All ZeroJava programs are compiled into optimized and verifiable zMIPS instructions. As all of the instructions are verifiable, any program that can be expressed in ZeroJava can be verified using ZKPs. The underlying zMIPS processor can implement and verify any arbitrary computation in zero-knowledge. The zMIPS instructions are implemented using the zk-STARK library \cite{ben2018scalable}. After computation description in ZeroJava and compilation to zMIPS, the constraints for the program are represent in algebraic intermediate representation (AIR) format. The prover and verifier interactively undergo the zk-STARK process until the verifier is convinced that the prover's work is sound. 
\texttt{Zilch} provides an elegant and accessible approach to building custom circuits that utilize zk-STARKs. Although the works lacks dedicated documentation, the examples that are provided show that development of custom applications is almost as simple as implementing the program in Java, with a few ZeroJava design considerations. We recommend this for users with general knowledge of the MIPS instruction set architecture, which should allow them to build optimized programs.

\textbf{RISC Zero.}
\texttt{RISC Zero} is a zkVM \cite{RISCZero2023DeveloperDocs} implemented in Rust with an underlying RISC-V processor and instruction set architecture. The goal of this work is to produce publicly verifiable proofs of all the computations that are done within the framework. As the underlying instructions are derived from RISC-V, virtually any arbitrary computation can be expressed and verified in zero-knowledge.
In this framework, custom circuits can be built using standard Rust syntax, with a few minor modifications to incorporates the framework's API. This program is compiled to a set of RISC-V instructions, which is then executed within a \texttt{RISC Zero} session, which is recorded. A receipt of this session is recorded and used as part of the zk-STARK proof, which can be verified by any verifier to check validity of the computation.
\texttt{RISC Zero} provides a relatively readable high-level Rust API, alongside several examples and very detailed documentation. Due to the maturity of the Rust development and \texttt{RISC Zero} as a whole, developers are able to import a majority of the most used standard Rust crates without trouble, enabling much more streamlined and efficient application development. For instance, developers can use the JPG crate \cite{RustImageCrate2023} to build zero-knowledge applications around images. Alongside this, \texttt{RISC Zero} enables GPU acceleration, so that relevant applications can take advantage of computational speedup. We do note that although GPU acceleration is implemented in the RISC Zero codebase, we were not able to get it actually working due to some inconsistencies within the codebase. However, \texttt{RISC Zero} has an active community around it, including active development by the creators, and a very well-documented and accessible code, making it a great candidate for new developers of custom ZKP applications. The primary drawback for this framework is that, due the nature of zkVMs and the simulation of a RISC-V processor and ISA, this framework has relatively significant initialization and operation costs.
% \section{Simulation Evaluation \& Results}\label{sec:results}

\subsection{Baseline Planners}

To evaluate the performance of \PlannerName, we compare it against several baseline methods. In the following section, we describe these baselines, their implementation details, and their respective advantages and limitations, particularly in the context of information gathering in large, high-dimensional search spaces. The simulation framework and vehicle parameters remain consistent across all planners, and each method is allowed to replan during testing.

\subsubsection{Monte-Carlo Tree Search}

Monte Carlo Tree Search (MCTS) can be a powerful technique for finding feasible and optimal paths in complex environments. It is a heuristic search algorithm that builds a search tree incrementally through repeated simulations. At each iteration, it selects a node to explore based on a selection policy (often the Upper Confidence Bound or UCB1 algorithm), expands the tree by adding possible actions from that node, runs a simulation from the newly added node, and updates the statistics of nodes along the path traversed during the simulation. 

The UCB1 (Upper Confidence Bound) algorithm is a technique commonly used in the context of multi-armed bandit problems and Monte Carlo Tree Search (MCTS) for balancing exploration and exploitation. It helps in selecting actions or nodes that are likely to yield high rewards while also exploring less-frequented options to gather more information about their potential rewards. 

We formulate our UCB score in the following manner, \\
\begin{equation*}
    UCB_\text{node} = \frac{I(X_{\text{node}})}{\alpha} + C \times \sqrt{\frac{\ln(N_\text{tree})}{N_\text{node}}}
\end{equation*}
%  $
% UCB_\text{node} = \frac{\overline{X_\text{node}}}{\alpha} + C \times \sqrt{\frac{\ln(N_\text{tree})}{N_\text{node}}}
% $ \\
Here $I(X_{\text{node}})$ denotes the estimated information gain from the node, $\alpha$ denotes the normalization factor which is given by $\frac{B}{v_\text{desired}}$, $B$ being the maximum planning budget and $v_\text{desired}$ being the desired speed of our UAV. $C$ denotes the exploration weight, and $N_\text{tree}$ denotes the number of visits to the tree root node while $N_\text{node}$ denotes the number of times the present node has been visited.

After selecting a candidate node, if it has been visited before, it is expanded by applying motion primitives to generate child nodes, growing the tree. Unvisited nodes skip this step. Following expansion, either the unvisited candidate node or one of its children is selected for the simulation phase, where the future values of nodes along the path are estimated to update the total potential information gain. This informs the selection policy in subsequent iterations. Once planning time is exhausted, the path with the highest information gain is returned.

% with authors goes here
\begin{figure}[t]
\centering
\includegraphics[trim={.7cm 0cm .5cm 1.4cm},clip,width=\columnwidth]{figs/5_/Results1v3.pdf}
\caption{The Monte Carlo simulation results for the planners. The plots show the average percent reduction in entropy over the course of the simulations, and the shading shows the 95\% confidence intervals. IA-TIGRIS outperforms all of the baselines.}
\label{fig:mc_results}
\end{figure}

While MCTS is probabilistically guaranteed to converge to the optimal path \cite{mcts_ref_1}, it is constrained to actions within a predefined set of motion primitives. Its reliance on random sampling to estimate the future value of nodes can result in poor approximations, particularly in environments with sparse, localized pockets of high information gain. This limitation is especially pronounced in large search areas or scenarios with large budgets constraints, where estimating future node values becomes increasingly expensive. As a result, in such scenarios, MCTS is often implemented with a finite planning horizon, which can restrict its ability to account for long-term consequences or dependencies in the environment.

% This property of MCTS, which causes unguided exploration of the environment, leads to increased convergence times on the optimal path, as a result of a lot of budget being spent in exploring information sparse areas of the map. 
% Also, the computation time of MCTS increases exponentially with the depth of the search tree. The time complexity of MCTS is given by $\mathcal{O}(\frac{T}{t_\text{iter}} \cdot |A|^d)$. Here, $T$ is the total planning time and $t_\text{iter}$ is the time taken per iteration of the planning loop. $|A|$ is the number of actions and $d$ represents the average depth of the search tree. 

% The above limitations are not inconsequential in the context of performing informative path planning in large high-dimensional search spaces. We compare MCTS with \PlannerName, in \ref{}, and empirically demonstrate its drawbacks and how \PlannerName, is able to outperform MCTS in the context of the mission parameters we examine in this work.  

\subsubsection{Greedy}

For the greedy planner, we iterated through each cell within the search bounds and calculated the reward for a given cell $i$ as $g_i = R(X_i) / d_i$ where $R(X_i)$ is given through \eqref{equ:reward} and $d_i$ represents the Euclidean distance between the current position the robot at the current time $t$ and the closest viewpoint to the cell. To compute this viewpoint, the yaw between the current pose of the robot and the intersected cell is first calculated. Using the robot's sensor configuration and this yaw, $x$ and $y$ coordinates are calculated that view the cell at the desired flight altitude. With this formulation, the planner prioritizes regions with a high ratio of entropy to distance. This can lead to locally optimal choices that contradict with paths that lead to higher information gain over the entire trajectory. 

% without authors goes here
% \begin{figure}[t]
% \centering
% \includegraphics[trim={.7cm 0cm .5cm 1.4cm},clip,width=\columnwidth]{figs/5_/Results1v3.pdf}
% \caption{The Monte Carlo simulation results for the planners. The plots show the average percent reduction in entropy over the course of the simulations, and the shading shows the 95\% confidence intervals. IA-TIGRIS outperforms all of the baselines.}
% \label{fig:mc_results}
% \end{figure}


\begin{figure*}[t]
    \centering
    \begin{subfigure}[b]{0.99\textwidth}
        \centering
        \includegraphics[trim={0cm 0.3cm 0cm 0cm},clip,width=\textwidth]{figs/5_/Fig2v1_target.png}
        % \caption{Slice by targets}
        % \vspace{.1cm}
    \end{subfigure}
    
    \begin{subfigure}[b]{0.99\textwidth}
        \centering
        \includegraphics[trim={0cm 0cm 0cm 0cm},clip,width=\textwidth]{figs/5_/Fig2v1_sigma.png}
        % \caption{Slice by sigma }
    \end{subfigure}
    \caption{A comparison of the methods based on the number of sampled prior clusters and the standard deviation of sampled prior clusters. IA-TIGRIS is most effective compared to the baselines when there is high variation in the search space. As the search space prior information becomes more evenly spread out, the performance gap between the methods tends to decrease.}
    \label{fig:targets_sigmas}
\end{figure*}

\subsubsection{Random}

The random planner operates by iteratively sampling points within the defined search bounds and calculating the minimum-cost path to observe each sampled point. This process is repeated until the available budget is fully expended. The random planner does not utilize any prior information about the environment or target distribution. Additionally, it does not optimize the sequence of actions, instead treating each sampled point independently without considering the global structure of the search problem. This simplicity allows the random planner to highlight the performance benefits of more sophisticated methods by providing a lower-bound comparison for evaluation.

\subsubsection{Coverage}

The coverage planner generates a plan that systematically covers the entire search space using a straightforward lawn-mower pattern. The spacing between each pass is set to match the width of the projected observation footprint at 20\% from the bottom, ensuring that no grid cells are missed. This spacing also maintains a distance that enables high-quality sensor measurements. However, due to the size of the search spaces considered, the coverage planner spends significant time surveying empty regions. This approach results in inefficient use of the budget, as it prioritizes full coverage with safe sensor overlap, even in areas with little or no valuable information. While simple and robust, this method highlights the tradeoff between exhaustive coverage and efficient, targeted exploration.

% \subsubsection{Branch and Bound}
% The branch and bound baseline is based on motion primitive planning. In each future step the drone has a set of motion primitives with future states and each of these future states also has a set of motion primitives. In this way, a tree can be built with multiple path candidates. The path candidate with the highest information gain will be selected and form the output. 

% By adding branch and bound, there will be an estimation of a node's upper bound information reward, using the node's current information reward, updated information map and the remaining budget. If this upper bound is already lower than the information reward of any other node in the tree, the corresponding node will be closed and not expanded in the future to accelerate the expansion of the tree. 



\subsection{Tests and Analysis}
% To evaluate the efficacy of IA-TIGRIS compared to the baseline methods, we conduct Monte Carlo testing as well as analyze how the prior and budget affect the performance of each method. In all of these test cases, there are no time-based or priority rewards and have horizon lengths set to the full budget. All tests were performed using an Intel Xeon CPU E5-2620 v4 @ 2.10GHz.
To evaluate the efficacy of IA-TIGRIS against baseline methods, we perform Monte Carlo testing and analyze the impact of the prior and budget on the performance of each method. In all test cases, rewards are calculated using \eqref{equ:reward}, and horizon lengths are set to match the full budget. The tests are conducted on an Intel Xeon CPU E5-2620 v4 @ 2.10GHz, ensuring consistent computational conditions across all evaluations.

% Random sample across which parameters.

% Quantitative ideas. Look into number and std of prior (metric for this? std of grid cell values, mediuan, mean,). 
% Uniform prior? 
% Split distinct regions, not smooth. 
% Compare to coverage and amount of time to reach specific amount. 
% Compare with different budgets. 
% Repeatability test. 
% Graph size vs time. 
% Look at coverage with different altitudes or widths. Something that shows long horizon vs not nature of things?
% Shape of search space?
% Time/budget to get x\% of all info gain. Have to do moving horizon. 
% Targets detected? 

% Key thought for results where I show time, our optimization does not optimize for time, only final value. Key thing to show across the different budgets. 

% \BM{Qualitative. Nayana idea of plot with example sampled case. Should add one here.} 



\subsubsection{Monte Carlo Testing}
Our simulated testing environment is a $5000\times5000$ m square with Gaussian-distributed prior information randomly placed throughout the search space. The number of prior clusters was sampled uniformly between $[4,20]$, with standard deviations between $[60,450]$, and maximum value between $[0.05,0.5]$. 

The results of $100$ Monte Carlo tests are shown in Fig.~\ref{fig:mc_results}. IA-TIGRIS clearly outperforms the other methods, achieving nearly a $40\%$ greater reduction in entropy than the next best method. Early in the simulation, the greedy method initially gains information more quickly, as expected, but this does not translate to better long-term performance. Since our method optimizes for total information gain, it generates paths that maximize information collection over the entire budget. MCTS performed slightly worse than the greedy approach.

The random paths slightly outperformed the coverage paths. This is likely because the lawnmower strategy requires sufficient overlap between passes to avoid missing areas, and its long straight paths often lead to redundant observations due to the UAV’s forward-facing camera. Changing the heading of the UAV is beneficial to viewing more of the search space, which may explain why random paths performed better.

We also conducted Monte Carlo tests where either the number of prior clusters or their standard deviation was held constant to analyze how variations in the information map affect planner performance. The results, shown in Fig.~\ref{fig:targets_sigmas}, include two cases: the upper figure fixes the number of priors, while the lower figure fixes their standard deviation. All other agent and simulation parameters remained unchanged.


% The first thing to note from these results is that for all tests the proportional performance gap between IA-TIGRIS and the baselines increases as the number and standard deviation of the Gaussian priors decreases. As the search space becomes more uniformly filled with entropy in the information map, the need for longer-horizon planning decreases and other simple or random approaches can perform satisfactorily given the testing budget. As the information becomes more sparsely distribution in the space, such as when the information is contained in separated pockets of areas, there is a greater need to plan longer-horizon paths that reason about the given budget.
% \BM{Could have figures here or refer to others}

Across these tests, the performance gap between IA-TIGRIS and the baselines widens as the number and standard deviation of the Gaussian priors decrease. When entropy is more uniformly distributed across the search space, simpler methods perform reasonably well within the given budget. However, when information is concentrated in sparse, distinct regions, longer-horizon planning becomes essential. In such cases, IA-TIGRIS demonstrates a significant advantage by effectively reasoning about the budget and prioritizing high-value regions.

% Show plot of first plans expected info gain versus planning time. (plans not executed)


\subsubsection{Budget Analysis}
To evaluate the impact of budget constraints on performance, we conducted additional tests beyond our initial Monte Carlo experiments, evaluating budgets of $5000$ m, $10000$ m, $30000$ m, and $60000$ m. Table~\ref{tab:budgets} summarizes the average entropy reduction across these budgets.

\definecolor{tabfirst}{rgb}{1, 0.7, 0.7} % red
\definecolor{tabsecond}{rgb}{1, 0.85, 0.7} % orange
\definecolor{tabthird}{rgb}{1, 1, 0.7} % yellow
\begin{table}[t]
    \centering
    \resizebox{\linewidth}{!}{
    \begin{tabular}{l|ccccc}
    & $5000$ m & 10000 m  & 15000 m& 30000 m& 60000 m\\ \hline

    % \hline
    IA-TIGRIS  &  \cellcolor{tabfirst}$9.41\pm1.0$ &  \cellcolor{tabfirst}$18.28\pm1.8$ & \cellcolor{tabfirst}$25.36\pm2.3$ & \cellcolor{tabfirst}$41.08\pm2.9$ & \cellcolor{tabfirst}$58.85\pm2.9$ \\
    Greedy  &  \cellcolor{tabsecond}$6.99\pm0.8$ &  \cellcolor{tabsecond}$13.10\pm1.5$ & \cellcolor{tabsecond}$17.97\pm2.0$ & \cellcolor{tabthird}$30.00\pm2.3$ & \cellcolor{tabsecond}$49.38\pm3.5$ \\
    MCTS  &  \cellcolor{tabthird}$6.06\pm0.7$ &  \cellcolor{tabthird}$11.80\pm1.1$ & \cellcolor{tabthird}$17.11\pm1.4$ & \cellcolor{tabsecond}$30.21\pm2.2$ & \cellcolor{tabthird}$48.68\pm2.7$ \\
    Random  &  $2.19\pm0.3$ & $4.29\pm0.7$ & $6.61\pm0.6$ & $17.50\pm1.2$ & $22.47\pm1.4$ \\
    Coverage  &  $1.58\pm0.3$ &  $2.82\pm0.4$ & $4.09\pm0.7$ & $12.04\pm1.9$ & $16.77\pm2.4$ \\

    \end{tabular}
    }
    \caption{Monte Carlo testing results given different budgets. The values are the average percent reduction in entropy and the 95\% confidence bounds. \mbox{IA-TIGRIS} had the best performance for all budgets.}
    \label{tab:budgets}
\end{table}
%$\uparrow$ 

IA-TIGRIS consistently achieved the highest entropy reduction across all budget constraints, with a statistically significant margin over alternative methods. Greedy generally ranked second but was slightly outperformed by MCTS at the $30000$ m budget level. Greedy and MCTS exhibited comparable performance throughout the tests, with their results closely tracking each other. Consistent with our previous findings, Random and Coverage methods yielded the lowest results.


Among the tested methods, only IA-TIGRIS and MCTS explicitly incorporate budget constraints into their planning algorithms. Notably, at lower budgets ($5000$ m and $10000$ m), these methods achieved higher entropy reduction compared to the equivalent time steps ($200$ s and $400$ s) in the $15000$ m budget scenario shown in Fig.~\ref{fig:mc_results}. This improved performance stems from IA-TIGRIS's optimization of total path reward under budget constraints, contrasting with the myopic next-best-action approach of the greedy method. The remaining methods---Greedy, Random, and Coverage---maintain consistent behavior regardless of budget constraints, as their planning strategies do not account for resource limitations.


The performance gap between IA-TIGRIS and the next-best method varied with budget size, showing margins of $34.6\%$, $39.5\%$, $41.1\%$, $36.0\%$, and $19.2\%$ in ascending budget order. This gap widened through the first three budget levels as problem complexity increased, before declining significantly at higher budgets. This performance pattern suggests that implementing a planning horizon could enhance efficiency by limiting tree search depth, enabling the planner to prioritize path quality optimization over exhaustive space exploration.


% percent improved from next best
% 34.6, 39.5, 41.1, 36.0, 19.2
% reasons, too long horizon is a larger search space, so less quality paths closer. Or larger horizon, more packing in


% with authors goes here
\begin{figure}[t] 
    \centering
    \renewcommand\arraystretch{0} % Adjust the height between rows here
    \setlength{\tabcolsep}{1pt} % Adjust the column separation here
    \begin{tabular}{c}
        \begin{tikzpicture}
            \node[anchor=south west, inner sep=0] (image) at (0,0) {
                \includegraphics[width=0.9\linewidth]{figs/5_/google_earth_prior.png}
            };
            \begin{scope}[x={(image.south east)},y={(image.north west)}]
                % \fill[OrangeRed] (0.02, 0.03) circle (2pt); 
                % \fill[OrangeRed] (0.51, 0.04) circle (2pt); 
                % \fill[OrangeRed] (0.61, 0.04) arc (0:90:2pt); 
                \fill[Orange, opacity=0.8] (0.74, 0.45) circle (3pt); % Adjust 
                \fill[Orange, opacity=0.8] (0.27, 0.42) circle (3pt); % Adjust 
                \fill[Orange, opacity=0.8] (0.39, 0.63) circle (3pt); % Adjust 
            \end{scope}
        \end{tikzpicture} \\
        % \includegraphics[width=0.9\linewidth]{figs/5_/google_earth_prior.png} \\
        \\
        \includegraphics[width=0.9\linewidth]{figs/5_/google_earth_path.png} 
    \end{tabular}
    \caption{Google Earth screenshots illustrating the mission planning process and execution. Top: Areas of high entropy targeted for search are highlighted in red, representing regions with a binary occupied/unoccupied probability of 0.2. Three points of particular interest, each assigned a 0.5 probability, are marked in orange. Bottom: The executed drone flight path (yellow) shows the optimized path for maximum information gain across the search space.} 
    \label{fig:google_earth}
\end{figure}
\begin{figure}[t]
\centering
% https://docs.google.com/presentation/d/1RjI-QqHpBRLHN60UAxzmQYs4EaWaVCOoSBkEkA39kk0/edit?usp=sharing
\includegraphics[width=\columnwidth]{figs/5_/m600_labeled.jpg}
\caption{Hexarotor system (DJI M600 Pro) with onboard compute and camera. Left image shows drone on the ground, right image shows drone in flight.}
\label{fig:m600}
\end{figure}


\section{Field Deployments}\label{sec:field}


\subsection{Hexarotor Deployment}
The first field experiment that we present uses a hexarotor drone to cover an urban area shown in Fig.~\ref{fig:fig1}.
We designed this field experiment to simulate classifying where cars are within a search area.  
Hence, we set the plan request to focus on parking lots at the field test site (Fig.~\ref{fig:google_earth}, top), with the addition of three chosen grid cells within the parking lots being marked as having a higher uncertainty. The plan request boundaries and priors were created with GPS coordinates in Google Earth, exported as kml files, and then converted into our plan request message format. 

The following sections details the hardware, autonomy, and experimental results for our hexarotor deployments.

% without the authors goes here
% \begin{figure}[t] 
%     \centering
%     \renewcommand\arraystretch{0} % Adjust the height between rows here
%     \setlength{\tabcolsep}{1pt} % Adjust the column separation here
%     \begin{tabular}{c}
%         \begin{tikzpicture}
%             \node[anchor=south west, inner sep=0] (image) at (0,0) {
%                 \includegraphics[width=0.9\linewidth]{figs/5_/google_earth_prior.png}
%             };
%             \begin{scope}[x={(image.south east)},y={(image.north west)}]
%                 % \fill[OrangeRed] (0.02, 0.03) circle (2pt); 
%                 % \fill[OrangeRed] (0.51, 0.04) circle (2pt); 
%                 % \fill[OrangeRed] (0.61, 0.04) arc (0:90:2pt); 
%                 \fill[Orange, opacity=0.8] (0.74, 0.45) circle (3pt); % Adjust 
%                 \fill[Orange, opacity=0.8] (0.27, 0.42) circle (3pt); % Adjust 
%                 \fill[Orange, opacity=0.8] (0.39, 0.63) circle (3pt); % Adjust 
%             \end{scope}
%         \end{tikzpicture} \\
%         % \includegraphics[width=0.9\linewidth]{figs/5_/google_earth_prior.png} \\
%         \\
%         \includegraphics[width=0.9\linewidth]{figs/5_/google_earth_path.png} 
%     \end{tabular}
%     \caption{Google Earth screenshots illustrating the mission planning process and execution. Top: Areas of high entropy targeted for search are highlighted in red, representing regions with a binary occupied/unoccupied probability of 0.2. Three points of particular interest, each assigned a 0.5 probability, are marked in orange. Bottom: The executed drone flight path (yellow) shows the optimized path for maximum information gain across the search space.} 
%     \label{fig:google_earth}
% \end{figure}
% \begin{figure}[t]
% \centering
% % https://docs.google.com/presentation/d/1RjI-QqHpBRLHN60UAxzmQYs4EaWaVCOoSBkEkA39kk0/edit?usp=sharing
% \includegraphics[width=\columnwidth]{figs/5_/m600_labeled.jpg}
% \caption{Hexarotor system (DJI M600 Pro) with onboard compute and camera. Left image shows drone on the ground, right image shows drone in flight.}
% \label{fig:m600}
% \end{figure}

\subsubsection{Hardware System}
The hardware consists of the DJI M600 Pro, shown in Fig.~\ref{fig:m600}, along with the physical sensing and onboard computer payload. The DJI M600 Pro contains a flight controller that handles pose estimation and position-based control. The DJI M600 Pro’s flight controller also handles teleloperation if human intervention is necessary. Beneath the drone's base, we mount a custom hardware payload.
That payload consists of an onboard computer, a Jetson Xavier, to run the autonomy software shown in Fig.~\ref{fig:functional_diagram}.
The payload also contains a downward-facing a camera for sensing the environment. The camera is a Seek S304SP thermal camera.
The camera intrinsics are used to calculate the frustum's intersection with the search map's cells in IA-TIGRIS.

% without authors goes here
\begin{figure}[t]
\centering
% https://lucid.app/lucidchart/f750ddb4-2809-4773-8361-d5fbb1ba49eb/edit?viewport_loc=-257%2C-116%2C2219%2C1140%2C0_0&invitationId=inv_56e8a3a9-e8cf-4cad-a280-48bd967ff651
\includegraphics[trim={0cm 0cm 0cm 0cm},clip,width=\columnwidth]{figs/5_/functional_diagram.jpeg}
\caption{Functional diagram of the DJI M600 Pro autonomy software.}
\label{fig:functional_diagram}
\end{figure}
\begin{figure}[b]
    \centering
    \begin{subfigure}[b]{0.48\columnwidth}
        \centering
        \includegraphics[width=1.0\linewidth]{figs/5_/field_test_altitude_over_time.png}
        \caption{}
        \label{fig:m600_altitude_over_time}
    \end{subfigure}
    \begin{subfigure}[b]{0.48\columnwidth}
        \centering
        \includegraphics[width=1.0\linewidth]{figs/5_/field_test_entropy_over_time.png}
        \caption{}
        \label{fig:m600_entropy_over_time}
    \end{subfigure}
    \caption{The results for our hexarotor field deployment. (a) Plot of flown altitude over time, showing large variation throughout the experiment. (b) Reduction in entropy percentage over time of field experiment.}
\end{figure}

\subsubsection{Autonomy System}
Fig.~\ref{fig:functional_diagram} illustrates the functional system diagram for the real world field test on the DJI M600. The user specifies the initial plan request prior to takeoff. The TIGRIS planner makes an initial plan on that plan request and sends a global path to the waypoint manager. The waypoint manager tracks the current waypoint within the plan and sends the next waypoint to the DJI software development kit, which then sends actuation commands to the motors. The position of the drone is used to calculate the distance from the drone to the ground and sends that distance parameter to the sensor model. The sensor model's true positive and false positive rate is used to calculate the per-cell entropy updates in the search map manager. The search map manager publishes the current information map, and the replanning node sends an updated plan request to the IA-TIGRIS planner every ten seconds.

The drone started at an altitude of $50$ m above the origin of the reference frame. The informed sampler in IA-TIGRIS was set to add states at altitudes of either $30$ m or $60$ m, creating a trade-off between observation area and detector accuracy. The budget was $2000$ m, the planning horizon was $600$ m, and the planning time was $10$ seconds. 

% % without authors goes here
% \begin{figure}[t]
% \centering
% % https://lucid.app/lucidchart/f750ddb4-2809-4773-8361-d5fbb1ba49eb/edit?viewport_loc=-257%2C-116%2C2219%2C1140%2C0_0&invitationId=inv_56e8a3a9-e8cf-4cad-a280-48bd967ff651
% \includegraphics[trim={0cm 0cm 0cm 0cm},clip,width=\columnwidth]{figs/5_/functional_diagram.jpeg}
% \caption{Functional diagram of the DJI M600 Pro autonomy software.}
% \label{fig:functional_diagram}
% \end{figure}
% \begin{figure}[b]
%     \centering
%     \begin{subfigure}[b]{0.48\columnwidth}
%         \centering
%         \includegraphics[width=1.0\linewidth]{figs/5_/field_test_altitude_over_time.png}
%         \caption{}
%         \label{fig:m600_altitude_over_time}
%     \end{subfigure}
%     \begin{subfigure}[b]{0.48\columnwidth}
%         \centering
%         \includegraphics[width=1.0\linewidth]{figs/5_/field_test_entropy_over_time.png}
%         \caption{}
%         \label{fig:m600_entropy_over_time}
%     \end{subfigure}
%     \caption{The results for our hexarotor field deployment. (a) Plot of flown altitude over time, showing large variation throughout the experiment. (b) Reduction in entropy percentage over time of field experiment.}
% \end{figure}

\subsubsection{Experimental Results}


The bottom image of Fig.~\ref{fig:google_earth} shows the path selected by IA-TIGRIS in the search area. The figure highlights how the planner dynamically adjusts altitudes over time to balance coverage and sensing resolution, maximizing information gain. Higher altitudes allow for broader area coverage, while lower altitudes provide more detailed observations where needed. Additionally, the planner prioritizes revisiting the three regions of higher uncertainty, recognizing the need for repeated observations reduce entropy. This adaptive strategy ensures that uncertain areas receive sufficient attention to improve the belief map. As a result, the entropy of the information map decreases to near zero by the end of the mission, as shown in Fig.~\ref{fig:m600_entropy_over_time}, indicating that the planner has effectively gathered the necessary information. This behavior demonstrates the planner’s ability to optimize sensing actions, balancing altitude selection, revisit frequency, and exploration to maximize mission success.

\begin{figure}[t]
\centering
% \includegraphics[width=2.5in]{fig1}
\includegraphics[trim={4cm 4cm 0cm 4cm},clip,width=\columnwidth]{figs/5_/TL1.jpg}
\caption{Fixed-wing platform used for autonomous flights with an onboard camera pitched at 10 degrees\cite{alarewebsite}}
\label{fig:tl1}
\end{figure}






\subsection{Fixed-wing Deployments}

Our proposed approach was extensively tested on the fixed-wing AlareTech TL-1 UAV, shown in Fig.~\ref{fig:tl1}. The UAV is equipped with an onboard camera pitched at 10 degrees, which introduces a more challenging planning problem due to the non-holonomic motion model and the camera's field of view. Over more than 20 flight hours and 100 flights running IA-TIGRIS, we validated our approach with the objective to search for objects of interest in a large search space across a variety of test scenarios, including different terrain types, varying environmental conditions, and diverse target distributions. An example mission from these tests is shown in Fig.~\ref{fig:fwd}. In this scenario, the planner was given the search bounds and a designated high-priority region. The resulting flight path prioritized revisiting the high-priority area twice, optimizing sensor use and ensuring maximum information gain. This strategy led to the successful detection of the object of interest, with its estimated position marked by the red dot in the figure. 

The map on the upper right in Fig.~\ref{fig:fwd} shows the information map after plan execution was complete. Due to the UAV's limited budget, the upper right and lower left corners of the map are not searched by the agent. The budget is instead utilized to search over the area of higher priority two times. Compared to the paths in Fig.~\ref{fig:google_earth}, we observe that the paths for the fixed wing are smoother and have a larger turning radius, demonstrating how IA-TIGRIS respects the motion constraints of the vehicle. We can also see the effect of wind on the path execution, where the flown path shown in green deviates from the planned path shown in yellow. This illustrates the importance of online planning in the cases where this deviation is large or would accumulate over the course of a longer mission and cause the expected observed area to be much different than actual observed area. 

\begin{figure}[t]
\centering
% \includegraphics[width=2.5in]{fig1}
% [trim={left bottom right top},clip]
\includegraphics[trim={3.0cm, 1.0cm, 3.0cm, 1.0cm},clip,width=\columnwidth]{figs/5_/ONRFig_v3.pdf}
\caption{An example path generated for the fixed-wing platform conducting a large-area search for an object of interest. The larger black rectangle denotes the search bounds, while the smaller black rectangle highlights a region of higher uncertainty. The red dot marks the estimated position of the detected object based on image detections. The upper-right map displays the information state after planning is complete, while the middle plot shows the percent change in entropy over mission time. The flown path illustrates a balance between allocating resources to the high-priority region and exploring other areas within the search space.}
\label{fig:fwd}
\end{figure}

% Also tested extensively on the AlareTech TL-1 (citation?) tube launched UAV seen in Fig.~\ref{fig:tl1}.

% Talk about amount of flights, hours. Platform. Compute. Show visualization fo example flight. Talk about objects of interest in a broad sense (no mention of water/ocean/land for targets). Follow similar figure format as previous section. Main thing we want to highlight is the differences introduced in plans by having a fixed-wing platform compared to a drone. Include image of Alare TL-1 somewhere.

% One big figure showing all the info we want to convey. 

% \BM{Pitch 10 degrees, onboard computer type, etc}


% \subsection{VTOL?}
% what would it bring?


\section{Discussion}
\label{sec:discussion}

% \TODO{Bryan}

Our multimodal data augmentation method is a plug-and-play method that can be applied to any future VLM. Also the T2I generation can be replaced by any future T2I model, thus the effectiveness of our method automatically improves along with the SOTA T2I model, making it future-proof.



Our main method, \textbf{Co}ntrastive Visual \textbf{D}ata \textbf{A}ugmentation (\textbf{CoDA}), is simple and easy to apply to LMMs in a variety of scenarios. Several components in the pipeline utilize existing off-the-shelf model components that can be easily swapped out for superior versions of similar models as research in their respective field progresses. Therefore, we expect the efficiency and effectiveness of \textbf{CoDA} to dramatically scale along with the advancement of relevant models. 


% % \subsec{ZKP Applications}

% \subsection{Verifiable Machine Learning}
% \nojan{zkcnn, mystique, vcnn, zen, ezkl}

% \nojan{computationaly weak verifier sends computation out to strong prover (look at zilch abstract)}

% \subsection{zk-Rollups}
% \nojan{plonky2, zkevm}

% \subsection{Robust Federated Learning}
% \nojan{zprobe, eiffel, brea, rofl}

% \subsection{FHE Integrity}
% \nojan{rinocchio}
% \input{}
\section{Conclusion and future directions} \label{sec:conclusion}

In this paper we proposed a nested MLMC framework that offers important computational savings by performing most calculations in low precision and exploiting approximate random normal variables for the low precision path calculations. The low precision calculations could be performed in fixed precision on an FPGA for greater efficiency, and we suggested a procedure to optimise the bit-widths of every variable at each Monte Carlo level. This is an important improvement over previous mixed precision MLMC frameworks which held the lower precision fixed \cite{Rounding_error_oliver} or defined uniform bit-width at every level heuristically \cite{brugger2014mixed}. Our numerical results suggest that for the first levels our procedure reduces the cost at these levels by a factor 5 or 7. Hence the overall savings are significant since most paths are calculated on the first levels. Our approach would be even more efficient for the Milstein scheme because its higher order strong convergence leads to a greater proportion of the computational costs being on the coarsest levels.

The next stage of the research project will be to implement the RNG methods and the nested framework on FPGAs to determine the hardware requirements and confirm the extent of the computational savings. It would also be good to compare the performance benefits to using half-precision floating point arithmetic on GPUs or CPUs for the low-accuracy computations.



\section*{Acknowledgments}
This work was supported by DARPA Proofs under grant number HR0011-23-1-0006.

\section*{Conflict of Interests}
The authors declare that there is no conflict of interests regarding the publication of this paper.

{\footnotesize
\bibliographystyle{abbrv}
\bibliography{refs}
}

\begin{IEEEbiography}[{\includegraphics[width=1in,height=1.25in,clip,keepaspectratio]{author_pics/Nojan.png}}]{Nojan Sheybani} is a Ph.D. candidate in the department of Electrical and Computer Engineering (ECE) at the University of California San Diego (UCSD). His research is focused on applied cryptography, hardware/software co-design, and zero-knowledge proofs. In particular, a common theme in his work is the application and optimization of privacy-preserving techniques to build practical and secure real-world systems.
\end{IEEEbiography}


\begin{IEEEbiography}[{\includegraphics[width=1in,height=1.25in,clip,keepaspectratio]{author_pics/anees.png}}]{Anees Ahmed} received his M.S. degree in Computer Science from Arizona State University. His research interests include privacy-preserving computation, zero-knowledge proofs, hardware acceleration, and computer architecture. Prior to his master's degree, he was a software engineer for two years. 
\end{IEEEbiography}


\begin{IEEEbiography}[{\includegraphics[width=1in,height=1.25in,clip,keepaspectratio]{author_pics/mkinsy.png}}]{Michel Kinsy} is an associate professor in the School of Computing and Augmented Intelligence and the director of the Secure, Trusted, and Assured Microelectronics Center. He focuses his research on microelectronics security, secure processors and systems design, hardware security, and efficient hardware design and implementation of post-quantum cryptography systems. Kinsy is an MIT Presidential Fellow and a CRA-WP Inaugural Skip Ellis Career Award recipient.
\end{IEEEbiography}

\begin{IEEEbiography}[{\includegraphics[width=1in,height=1.25in,clip,keepaspectratio]{author_pics/Farinaz.png}}]{Farinaz Koushanfar}
is the Siavouche Nemati-Nasser Endowed Professor of Electrical and Computer Engineering (ECE) at the University of California San Diego (UCSD), where she is the founding co-director of the UCSD Center for Machine-Intelligence, Computing \& Security (MICS). She is also a research scientist at Chainlink Labs. Her research addresses several aspects of secure and efficient computing, with a focus on robust machine learning under resource constraints, AI-based optimization, hardware and system security, intellectual property (IP) protection, as well as privacy-preserving computing. Dr. Koushanfar has received a number of awards and honors including the Presidential Early Career Award for Scientists and Engineers (PECASE) from President Obama, the ACM SIGDA Outstanding New Faculty Award, Cisco IoT Security Grand Challenge Award, MIT Technology Review TR-35, Qualcomm Innovation Awards, Intel Collaborative Awards, Young Faculty/CAREER Awards from NSF, DARPA, ONR and ARO, as well as several best paper awards. Dr. Koushanfar is a fellow of ACM, IEEE, National Academy of Inventors, and the Kavli Frontiers of the National Academy of Sciences.
\end{IEEEbiography}
\vfill

\clearpage
\newpage

\appendices
\appendix
\begin{table}[t!]
  \centering
  


% \renewcommand{\arraystretch}{1.2} % 调整行高
% \setlength{\tabcolsep}{10pt}  % 调整列间距
\resizebox{0.48\textwidth}{!}{%
\begin{tabular}{lcrr}
        \toprule
        \textbf{Dataset} & \textbf{Full Size*} & \textbf{Consistency}  & \textbf{\dataset{}} \\
        \midrule
        HotpotQA  & 5,901 & 2,973 {\footnotesize \textcolor{gray}{(50\%)}}  & 1,476 {\footnotesize \textcolor{gray}{(25\%)}}  \\
        NewsQA    & 4,212 & 1,260 {\footnotesize \textcolor{gray}{(30\%)}} & 934  {\footnotesize \textcolor{gray}{(22\%)}}  \\
        NQ        & 7,314 & 4,419 {\footnotesize \textcolor{gray}{(60\%)}}  & 1,479 {\footnotesize \textcolor{gray}{(20\%)}}  \\
        SearchQA  & 16,980 & 12,133 {\footnotesize \textcolor{gray}{(71\%)}} & 1,497 {\footnotesize \textcolor{gray}{(9\%)}}  \\
        SQuAD     & 10,490 & 5,024 {\footnotesize \textcolor{gray}{(48\%)}}  & 2,351 {\footnotesize \textcolor{gray}{(22\%)}}  \\
        TriviaQA  & 7,785 & 6654 {\footnotesize \textcolor{gray}{(85\%)}}  & 792  {\footnotesize \textcolor{gray}{(10\%)}}  \\
        \bottomrule
    \end{tabular}
}




 \caption{Number of instances at each stage in the \dataset{} construction pipeline.}
 \label{tab:our_bench_stats_each_step}
\end{table}
\section{Appendix}
\subsection{License}
We present the licenses of the datasets used in this study: Natural Questions (CC BY-SA 3.0 license), NewsQA (MIT License), SearchQA and TriviaQA (Apache License 2.0), HotpotQA and SQuAD (CC BY-SA 4.0 license).

All these licenses and agreements permit the use of their data for academic purposes.

\subsection{Details of Data Constructing}
\label{append:prompts}
In this section, we detail the two main steps in constructing \dataset{}. The dataset sizes at each stage of the pipeline are shown in Table~\ref{tab:our_bench_stats_each_step}.


\textbf{Parametric Knowledge Elicitation.} First, we elicit the LLM's parametric knowledge by prompting it in a closed-book setting (i.e., without any context). To ensure the reliability of the elicited knowledge, we apply a consistency-based filtering method. Specifically, for each query, the LLM is prompted five times, and the frequency of each response is recorded. The response with the highest frequency is identified as the majority answer. Queries where the majority answer appears fewer than three times are discarded, in order to filter out inconsistent responses and enhance data quality. The following prompt is used to instruct the LLM:
\begin{tcolorbox}
[title=Prompt for eliciting parametric knowledge,colback=blue!10,colframe=blue!50!black,arc=1mm,boxrule=1pt,left=1mm,right=1mm,top=1mm,bottom=1mm]
Answer the question \textcolor{blue}{\{\textit{brevity\_instruction}\}} and provide supporting evidence.

Question: \textcolor{blue}{\{\textit{question}\}}
\end{tcolorbox}
\noindent The ``\textit{brevity\_instruction}'' is used to guide the LLM to generate responses in a more concise form.

\textbf{Conflict Data Selection.} Next, we filter the data to retain only instances where the LLM's parametric knowledge directly conflicts with the contextual answer. Specifically, we categorize the data obtained from the previous step into two groups, conflicting and non-conflicting instances, based on the detailed results of conflict detection. All non-conflicting instances are discarded. GPT-4o-mini is then used to detect the presence of a conflict, using the following prompt:

\begin{tcolorbox}
[title=Prompt for identifying conflict knowledge,colback=blue!10,colframe=blue!50!black,arc=1mm,boxrule=1pt,left=1mm,right=1mm,top=1mm,bottom=1mm]
\small
You are tasked with evaluating the correctness of a model-generated answer based on the given information. 

\small
Context: \textcolor{blue}{\{\textit{context}\}}

Question: \textcolor{blue}{\{\textit{question}\}}

Contextual Answer: \textcolor{blue}{\{\textit{contextual\_answer}\}}

Model-Generated Answer: \textcolor{blue}{\{\textit{Model-Generated\_answer}\}}

\textcolor{blue}{[\textit{Detailed task description...}]}

Output Format:

Evaluate result: (Correct / Partially Correct / Incorrect) 
\end{tcolorbox}




\subsection{Assessing the Reliability of GPT-4o-mini in Knowledge Conflict Identification}
\label{append:human_eval}
In this subsection, we conduct the human evaluation to assess the reliability of GPT-4o-mini in identifying knowledge conflicts, which is a critical task in our data construction process to guarantee the data quality.

We randomly sampled 100 examples from each of the six subsets of \dataset{}, yielding a total of 600 samples. Six senior computational linguistics researchers were then asked to evaluate whether a knowledge conflict was present in each example. For each instance, the evaluators were provided with the question, the contextual answer, the model-generated response, and the corresponding supporting evidence. The results were classified into three categories: No Conflict, Somewhat Conflict, and High Conflict. The detailed annotation instructions are as follows:

\begin{tcolorbox}
[title=Annotation Instruction,colback=blue!10,colframe=blue!50!black,arc=1mm,boxrule=1pt,left=1mm,right=1mm,top=1mm,bottom=1mm]
\small
You are tasked with determining whether the parametric knowledge of LLMs conflicts with the given context to facilitate the study of knowledge conflicts in large language models.

Each data instance contains the following fields: 

Question: \textcolor{blue}{\{\textit{question}\}}


Answers: \textcolor{blue}{\{\textit{answers}\}}


Context: \textcolor{blue}{\{\textit{context}\}}

Parametric\_knowledge: \textcolor{blue}{\{\textit{LLMs' parametric\_knowledge }\}} 

The annotation process consists of two steps. 

\textbf{Step 1}: Compare the model-generated answer with the ground truth answers, based on the given question and context, to determine whether the model’s parametric knowledge conflicts with the context.

\textbf{Step 2}: Classify the results into one of three categories: 

\textcolor{blue}{\{\textit{No Conflict}\}} if the model-generated answer is consistent with the ground truth answers and context, 

\textcolor{blue}{\{\textit{Somewhat Conflict}\}}  if it is partially inconsistent

\textcolor{blue}{\{\textit{High Conflict}\}} if it significantly contradicts the ground truth answers or context.
\end{tcolorbox}


The evaluation results, shown in Table~\ref{tab:append_human_eval}, reveal a high level of agreement between the human annotators and GPT-4o-mini. Over 85\% of the examples reach consensus among the annotators, with an average agreement rate of 85.6\% across all subsets. These findings underscore the reliability of GPT-4o-mini as an effective tool for identifying knowledge conflicts.




\begin{table}[t]
  \centering
  
\centering
\begin{tabular}{l c}
\toprule
\textbf{Subset} & \textbf{Agreement (\%)} \\ \midrule
HotpotQA        & 81.4                        \\
NewsQA          & 72.7                        \\
NQ              & 88.7                        \\
SearchQA        & 95.3                        \\
SQuAD           & 86.1                        \\
TriviaQA        & 90.7                        \\ \midrule
\textbf{Average} & \textbf{85.6}            \\ \bottomrule
\end{tabular}

 \caption{Agreement between human annotators and GPT-4o-mini across different subsets of our \dataset{} benchmark.}
 \label{tab:append_human_eval}
\end{table}



\subsection{Evaluating the Effectiveness of Our Consistency-Based Filtering Method}
\label{append:data_freq}

In this subsection, we evaluate the effectiveness of our consistency-based knowledge conflict filtering method. As described in Appendix~\ref{append:prompts}, for each query, we prompt the model five times and record the most frequently generated answer along with its occurrence frequency. Based on this frequency, we divide the data into sub-datasets, where all queries within each sub-dataset share the same answer frequency. We then apply ``Conflict Data Selection'' to each sub-dataset, retaining only instances where knowledge conflicts occur. Finally, we evaluate ConR and MemR on these sub-datasets.

As shown in Figure~\ref{fig:diff_freq}, a clear trend emerges: as answer frequency increases, ConR consistently decreases, while MemR increases. This pattern indicates that as answer frequency rises, the model becomes increasingly reliant on its internal knowledge. Notably, for data with an answer frequency of 1, MemR is only 3\%, indicating minimal dependence on internal knowledge. Retaining only high-answer-frequency data improves the quality of \dataset{}. This data construction approach distinguishes our methodology from previous studies~\cite{longpre2021entity,xie2023adaptive}.

\begin{figure}[t!]
  \centering
  \includegraphics[width=0.4\textwidth]{figs/diff_freq.pdf}
  \caption{Performance comparison of ConR and MemR across sub-datasets grouped by the answer frequency of LLMs.}
  \label{fig:diff_freq}
\end{figure}





\subsection{Additional Implementation Details of Our Experiments}
\label{append:implementation}
This subsection outlines the training prompt, describes more details of the training data, and provides details of the experimental setup used in our experiments.

\textbf{Training Prompts.}
We adopt a simple QA-format training prompt following~\citet{zhou2023context} for all methods except \attrprompt{} and \oiprompt{}.
\begin{tcolorbox}
[title=Base Prompt ,colback=blue!10,colframe=blue!50!black,arc=1mm,boxrule=1pt,left=1mm,right=1mm,top=1mm,bottom=1mm]
% \small
\textcolor{blue}{\{\textit{context}\}} 
Q: \textcolor{blue}{\{\textit{question}\}} ? 
A: \textcolor{blue}{\{\textit{answer}\}}.
\end{tcolorbox}


\textbf{Training Datasets.} During \method{}, we randomly sample 32,580 instances from the training set of the MRQA 2019 benchmark~\cite{fisch2019mrqa} to construct our training data.



\textbf{Experimental Setup.} In this work, all models are trained for 2,100 steps with a total batch size of 32 and a learning rate of 1e-4. To enhance training efficiency, we implemented \method{} with LoRA~\cite{hu2021lora}, setting both the rank $\text{r}$ and scaling factor $\text{alpha}$ to 64. For \method{}, we set $\alpha$ to 0.1 (Eq.~\ref{eq:selct_layers}), which determines the minimum activation ratio difference required for a layer to be pruned. Additionally, we adopt a dynamic $\gamma$ in $\mathcal{L}_{\text{KC}}$ (Eq.~\ref{eq:kc_loss}), which linearly transitions from an initial margin ($\gamma_{0}=1$) to a final margin ($\gamma^*=5$) as training progresses. This adaptive strategy gradually reduces the model's reliance on internal parametric knowledge, encouraging it to rely more on external knowledge provided by the KAG system.


\subsection{Implementation Details of Baselines}
\label{append:baseline}
This subsection describes the implementation details of all baseline methods.

We adopt two prompt-based baselines: the attributed prompt ($\text{Attr}_{\text{prompt}}$) and a combination of opinion-based and instruction-based prompts ($\text{O\&I}_{\text{prompt}}$). The corresponding prompt templates are as follows:

\begin{tcolorbox}
[title=Attr based prompt ,colback=blue!10,colframe=blue!50!black,arc=1mm,boxrule=1pt,left=1mm,right=1mm,top=1mm,bottom=1mm]
% \small
\textcolor{blue}{\{\textit{context}\}} Q: \textcolor{blue}{\{\textit{question}\}} based on the given text? A: \textcolor{blue}{\{\textit{answer}\}}.
\end{tcolorbox}

\begin{tcolorbox}
[title=O\&I based prompt ,colback=blue!10,colframe=blue!50!black,arc=1mm,boxrule=1pt,left=1mm,right=1mm,top=1mm,bottom=1mm]

Bob said ``\textcolor{blue}{\{\textit{context}\}}'' Q: \textcolor{blue}{\{\textit{question}\}} in Bob's opinion? A: \textcolor{blue}{\{\textit{answer}\}}.
\end{tcolorbox}
For the SFT baseline, we incorporate context during training, similar to \method{}, while keeping the remaining experimental settings identical. To construct preference pairs for DPO training, we use contextually aligned answers from the dataset as ``preferred responses'' to ensure the consistency with the provided context. The ``rejected responses'' are generated by identifying parametric knowledge conflicts through our data construction methodology (Sec.~\ref{sec:benchmark}).

For KAFT, we employ a hybrid dataset containing both counterfactual and factual data. Specifically, we integrate the counterfactual data developed by \citet{xie2023adaptive}, leveraging their advanced data construction framework.

By maintaining equivalent dataset sizes and ensuring comparable data quality across all baselines, we provide a rigorous and fair comparison with our proposed \method{}.




\subsection{Extending \method{} to More LLMs}
\label{append:diff_model_performance}


\begin{figure}[t!]
  \centering
  
\subfigure[ConR Results]{
        \label{fig:diff_model:llama_conr}
        \includegraphics[width=0.462\linewidth]{append_fig/llama_conr.pdf}
    }
    \hspace{0.0005\linewidth} 
    \subfigure[MemR Results]{
        \label{fig:diff_model:llama_memr}
        \includegraphics[width=0.462\linewidth]{append_fig/llama_memr.pdf}
    }


  % \includegraphics[width=0.48\textwidth]{figs/diff_model_double.pdf}
 \caption{Average ConR and MemR across different models implemented by LLMs of LLaMA series, before and after applying \method{}.
 }
 \label{fig:diff_model_double_llama}
\end{figure}

\begin{figure}[t]
  \centering
  \subfigure[ConR Results]{
        \label{fig:diff_model:qwen_conr}
        \includegraphics[width=0.462\linewidth]{append_fig/qwen_conr.pdf}
    }
    \hspace{0.0005\linewidth} 
    \subfigure[MemR Results]{
        \label{fig:diff_model:qwen_memr}
        \includegraphics[width=0.462\linewidth]{append_fig/qwen_memr.pdf}
    }
  % \includegraphics[width=0.48\textwidth]{figs/diff_model_double.pdf}
 \caption{Average ConR and MemR across different models implemented by LLMs of Qwen series, before and after applying \method{}.
 }
 \label{fig:diff_model_double_qwen}
\end{figure}






We extend \method{} to a diverse range of LLMs, encompassing multiple model families and sizes. 

Specifically, our evaluation includes LLaMA3-8B-Instruct, LLaMA3.2-1B-Instruct, LLaMA3.2-3B-Instruct, Qwen2.5-0.5B-Instruct, Qwen2.5-1.5B-Instruct, Qwen2.5-3B-Instruct, Qwen2.5-7B-Instruct, and Qwen2.5-14B-Instruct. The results on ConR and MemR are summarized in Figures~\ref{fig:diff_model_double_llama} and \ref{fig:diff_model_double_qwen}, while Table~\ref{tab:append:all_model_res} presents the average performance of all models on \dataset{} and ConFiQA. Additionally, Table~\ref{tab:diff_model_param} provides detailed parameter information and specifies the layers selected for pruning for each model. This comprehensive evaluation demonstrates the versatility and scalability of \method{} across a wide spectrum of model architectures and sizes.

\begin{table}[!t]
  
    \resizebox{0.48\textwidth}{!}{%
\begin{tabular}{l|c|c|c}
\toprule
\textbf{Models}     & \textbf{Param.} & \textbf{\method{} Param.} & \textbf{Selected Layers} \\
\midrule
\rowcolor{gray!10}
LLaMA3.2-1B        & 1.24B  & 1.08B \small\textcolor{gray}{(87\%)}   & [12, 14]                 \\
LLaMA3.2-3B        & 3.21B  & 2.60B \small\textcolor{gray}{(81\%)}   &  [18, 25]   \\
\rowcolor{gray!10}
LLaMA3-8B          & 8.03B  & 6.97B \small\textcolor{gray}{(87\%)}   & [24, 29]      \\
LLaMA3.1-8B          & 8.03B  & 6.27B \small\textcolor{gray}{(78\%)}   & [20, 29]      \\
\rowcolor{gray!10}
Qwen2.5-0.5B         & 0.49B  & 0.44B \small\textcolor{gray}{(90\%)}   &  [19, 22]       \\
Qwen2.5-1.5B         & 1.54B  & 1.34B \small\textcolor{gray}{(87\%)}   & [21, 25]        \\
\rowcolor{gray!10}
Qwen2.5-3B         & 3.09B  & 2.68B \small\textcolor{gray}{(87\%)}   & [29, 34]        \\
Qwen2.5-7B         & 7.61B  & 7.21B \small\textcolor{gray}{(95\%)}   &   [25, 26 ]     \\
\rowcolor{gray!10}
Qwen2.5-14B        & 14.70B & 12.43B \small\textcolor{gray}{(85\%)}  &  [35, 45]   \\
\bottomrule
\end{tabular}
}

% \end{sidewaystable}

% \end{document}

  \caption{The total number of parameters for various models before and after applying \method{}. \textcolor{gray}{\small$(\cdot)\%$} represents the proportion relative to the original model, and the last column lists the layers selected for pruning.}
   \label{tab:diff_model_param}
\end{table}

These experimental results illustrate several key insights: 1) Larger models tend to rely more on parametric memory. As model size increases in both the LLaMA and Qwen families, MemR also grows, indicating a tendency to overlook external knowledge in favor of internal parameters. \method{} counteracts this behavior, decreasing larger models' MemR score to even below that of smaller models. 2) \method{} consistently benefits all evaluated models. Across both LLaMA and Qwen model families, \method{} outperforms Vanilla-KAG by boosting accuracy and context faithfulness, underscoring its broad applicability and effectiveness. 3) Not all parameters in KAG models are essential. Pruning parametric knowledge not only reduces computation costs but also fosters better generalization without sacrificing accuracy, highlighting the potential of building a parameter-efficient LLM within the KAG framework.




\begin{table*}[!t]
  
\centering
\resizebox{0.96\textwidth}{!}{%
\begin{tabular}{l|c|cccc|cccc}
\toprule
\multirow{2}{*}{\textbf{Models}} & \multirow{2}{*}{\textbf{Param.}} & \multicolumn{4}{c|}{\textbf{\dataset{}}} & \multicolumn{4}{c}{\textbf{ConFiQA}} \\ 
\cmidrule(lr){3-6}  \cmidrule(lr){7-10}
 &  & ConR $\uparrow$ & MemR $\downarrow$ & MR $\downarrow$ & EM $\uparrow$ & ConR $\uparrow$ & MemR $\downarrow$ & MR $\downarrow$ & EM $\uparrow$ \\ 
\midrule
LLaMA3-8B   & 8.03B  & 66.99  & 11.75  & 14.99  & 13.83  & 22.52  & 31.15  & 59.77  & 2.47 \\
\rowcolor{gray!10}
+\method{}    & 6.97B  & 71.50  & 6.48   & 8.41   & 66.19  & 70.43  & 8.82   & 11.32  & 67.29 \\
LLaMA3.1-8B & 8.03B  & 63.15  & 11.69  & 15.93  & 21.85  & 15.38  & 29.97  & 68.98  & 6.69 \\
\rowcolor{gray!10}
+\method{}   & 6.27B  & 70.41  & 6.95   & 9.17   & 63.58  & 71.12  & 9.01   & 11.44  & 66.61 \\
LLaMA3.2-1B & 1.24B  & 39.06  & 10.49  & 21.83  & 5.13   & 32.09  & 18.32  & 36.28  & 7.15 \\
\rowcolor{gray!10}
+\method{}   & 1.08B  & 51.75  & 6.51   & 11.34  & 47.60  & 62.70  & 7.63   & 11.38  & 61.85 \\
LLaMA3.2-3B & 3.21B  & 56.75  & 11.53  & 17.11  & 12.69  & 26.16  & 23.47  & 49.05  & 9.84 \\
\rowcolor{gray!10}
+\method{}   & 2.60B  & 67.00  & 6.80   & 9.35   & 61.59  & 69.61  & 8.39   & 11.09  & 66.53 \\
Qwen2.5-0.5B & 0.49B  & 47.17  & 11.36  & 19.48  & 2.06   & 50.72  & 17.15  & 26.20  & 3.78 \\
\rowcolor{gray!10}
+\method{}   & 0.44B  & 58.13  & 6.63   & 10.41  & 52.56  & 67.54  & 8.04   & 11.03  & 66.33 \\
Qwen2.5-1.5B & 1.54B  & 58.08  & 11.28  & 16.48  & 10.30  & 51.69  & 19.87  & 28.23  & 10.78 \\
\rowcolor{gray!10}
+\method{}   & 1.34B  & 63.78  & 6.74   & 9.76   & 57.67  & 69.61   & 8.35   & 11.05   & 66.04 \\
Qwen2.5-3B   & 3.09B  & 62.22  & 14.45  & 18.88  & 0.10   & 25.47  & 29.34  & 55.70  & 0.01 \\
\rowcolor{gray!10}
+\method{}     & 2.68B  & 66.31  & 6.75   & 9.38   & 59.42  & 66.30   & 8.62  & 11.94   & 63.03 \\
Qwen2.5-7B    & 7.61B  & 65.46  & 14.93  & 18.57  & 0.80   & 24.75  & 33.09  & 59.04  & 0.10 \\
\rowcolor{gray!10}
+\method{}      & 6.60B  & 67.75  & 6.60   & 9.01   & 61.77  & 69.54  & 8.85   & 11.58  & 66.68 \\
Qwen2.5-14B   & 14.70B & 65.75  & 16.13  & 19.75  & 0.00   & 7.86   & 32.88  & 83.71  & 0.01 \\
\rowcolor{gray!10}
+\method{}     & 12.43B & 70.01  & 6.43   & 8.55   & 64.43  & 71.70  & 8.90   & 11.29  & 68.40 \\
\bottomrule
\end{tabular}%
}


  \caption{Average performance of LLMs on \dataset{} and ConFiQA before and after applying \method{}.}
   \label{tab:append:all_model_res}
\end{table*}

\subsection{Neuron Activations in Different LLMs}\label{app:activation}
We present the neuron activations for the LLaMA family models, including LLaMA-3.2-1B-Instruct, LLaMA-3.2-3B-Instruct, LLaMA-3-8B-Instruct, and LLaMA-3.1-8B-Instruct, as well as the Qwen family models, including Qwen-2.5-0.5B-Instruct, Qwen-2.5-1.5B-Instruct, Qwen-2.5-3B-Instruct, Qwen-2.5-7B-Instruct, and Qwen-2.5-14B-Instruct, in Figures~\ref{fig:act_llama} and \ref{fig:act_qwen}, respectively. 
% 我们发现qwen系列模型


\begin{figure*}[t]
  \centering
  \subfigure[Neuron activations of LLaMA-3.2-1B-Instruct]{
        \label{fig:act_llama:3.2-1b}
        \includegraphics[width=0.9\linewidth]{append_fig/act_llama32_1b_all.pdf}
    }
\subfigure[Neuron activations of LLaMA-3.2-3B-Instruct]{
        \label{fig:act_llama:3.2-3b}
        \includegraphics[width=0.9\linewidth]{append_fig/act_llama32_3b_all.pdf}
    }
 \subfigure[Neuron activations of LLaMA-3-8B-Instruct]{
        \label{fig:act_llama:3-8b}
        \includegraphics[width=0.9\linewidth]{append_fig/act_llama_3_8b.pdf}
    }
 \subfigure[Neuron activations of LLaMA-3.1-8B-Instruct]{
        \label{fig:act_llama:3.1-8b}
        \includegraphics[width=0.9\linewidth]{append_fig/act_llama_31_8b.pdf}
    }
 

 \caption{Neuron activations across different layers of the LLaMA series models. We present the inhibition ratio $\Delta R$ under two conditions: with contextual knowledge input (w/ context) and without it (w/o context).}
 \label{fig:act_llama}
\end{figure*}

\begin{figure*}[t]
  \centering
  \subfigure[Neuron activations of Qwen-2.5-0.5B-Instruct]{
        \label{fig:act_qwen:2.5-0.5b}
        \includegraphics[width=0.75\linewidth]{append_fig/act_qwen25_0_5b_all.pdf}
    }
\subfigure[Neuron activations of Qwen-2.5-1.5B-Instruct]{
        \label{fig:act_qwen:2.5-1.5b}
        \includegraphics[width=0.75\linewidth]{append_fig/act_qwen25_1_5b_all.pdf}
    }
\subfigure[Neuron activations of Qwen-2.5-3B-Instruct]{
        \label{fig:act_qwen:2.5-3b}
        \includegraphics[width=0.75\linewidth]{append_fig/act_qwen25_3b_all.pdf}
    }
\subfigure[Neuron activations of Qwen-2.5-7B-Instruct]{
        \label{fig:act_qwen:2.5-7b}
        \includegraphics[width=0.75\linewidth]{append_fig/act_qwen25_7b_all.pdf}
    }
\subfigure[Neuron activations of Qwen-2.5-14B-Instruct]{
        \label{fig:act_qwen:2.5-14b}
        \includegraphics[width=0.75\linewidth]{append_fig/act_qwen25_14b_all.pdf}
    }


 \caption{Neuron activations across different layers of the Qwen series models. We present the inhibition ratio $\Delta R$ under two conditions: with contextual knowledge input (w/ context) and without it (w/o context). }
 \label{fig:act_qwen}
\end{figure*}


\end{document}