\documentclass[lettersize,journal]{IEEEtran}
% \IEEEoverridecommandlockouts
% The preceding line is only needed to identify funding in the first footnote. If that is unneeded, please comment it out.
\usepackage{cite}
% \usepackage{natbib}
% \usepackage{biblatex}
\usepackage{amsmath,amssymb,amsfonts}
\usepackage{bm}
\usepackage{mathtools}
\usepackage{algorithmic}
\usepackage{graphicx}
\usepackage{textcomp}
\usepackage{xcolor}
\usepackage{booktabs}
\usepackage{multirow}
\usepackage{multicol}
\usepackage{xspace}
\usepackage{xurl}
\usepackage{tikz}
\usepackage{environ}

\usepackage{subfig}
\usepackage{rotating}
\usepackage{amssymb}% http://ctan.org/pkg/amssymb
\usepackage{pifont}% http://ctan.org/pkg/pifont
\newcommand{\cmark}{\ding{51}}%
\newcommand{\xmark}{\ding{55}}%
\newcommand*\emptycirc[1][1ex]{\tikz\draw (0,0) circle (#1);} 
\newcommand*\halfcirc[1][1ex]{%
  \begin{tikzpicture}
  \draw[fill] (0,0)-- (90:#1) arc (90:270:#1) -- cycle ;
  \draw (0,0) circle (#1);
  \end{tikzpicture}}
\newcommand*\fullcirc[1][1ex]{\tikz\fill (0,0) circle (#1);} 
\def\checkmark{\tikz\fill[scale=0.4](0,.35) -- (.25,0) -- (1,.7) -- (.25,.15) -- cycle;} 

\def\BibTeX{{\rm B\kern-.05em{\sc i\kern-.025em b}\kern-.08em
    T\kern-.1667em\lower.7ex\hbox{E}\kern-.125emX}}

\newif\ifdraft
% \drafttrue
\draftfalse

% \newcommand{\nojan}[1]{\textcolor{red}{{\sf (NS:} {\sl{#1})}}}
% \newcommand{\anees}[1]{\textcolor{blue}{{\sf (AA:} {\sl{#1})}}}

\ifdraft
\newcommand{\nojan}[1]{\textcolor{red}{{\sf (NS:} {\sl{#1})}}}
\newcommand{\anees}[1]{\textcolor{blue}{{\sf (AA:} {\sl{#1})}}}
\else
\newcommand{\nojan}[1]{}
\newcommand{\anees}[1]{}
\fi

\newcommand{\Prv}{$\mathcal{P}$\xspace}
\newcommand{\Vrf}{$\mathcal{V}$\xspace}
\newcommand{\Cir}{$\mathcal{C}$\xspace}

\begin{document}

\title{Zero-Knowledge Proof Frameworks: A Systematic Survey}
\author{\textnormal{Nojan Sheybani$^1$, Anees Ahmed$^2$, Michel Kinsy$^2$, Farinaz Koushanfar$^1$} \\
$^1$UC San Diego, $^2$Arizona State University \\
$^1$\{nsheyban, farinaz\}@ucsd.edu, $^2$\{aahmed90, mkinsy\}@asu.edu}


% {\footnotesize \textsuperscript{*}Note: Sub-titles are not captured in Xplore and
% should not be used}
% \thanks{Identify applicable funding agency here. If none, delete this.}
% }

% \author{\IEEEauthorblockN{1\textsuperscript{st} Given Name Surname}
% \IEEEauthorblockA{\textit{dept. name of organization (of Aff.)} \\
% \textit{name of organization (of Aff.)}\\
% City, Country \\
% email address or ORCID}
% \and
% \IEEEauthorblockN{2\textsuperscript{nd} Given Name Surname}
% \IEEEauthorblockA{\textit{dept. name of organization (of Aff.)} \\
% \textit{name of organization (of Aff.)}\\
% City, Country \\
% email address or ORCID}
% \and
% \IEEEauthorblockN{3\textsuperscript{rd} Given Name Surname}
% \IEEEauthorblockA{\textit{dept. name of organization (of Aff.)} \\
% \textit{name of organization (of Aff.)}\\
% City, Country \\
% email address or ORCID}
% \and
% \IEEEauthorblockN{4\textsuperscript{th} Given Name Surname}
% \IEEEauthorblockA{\textit{dept. name of organization (of Aff.)} \\
% \textit{name of organization (of Aff.)}\\
% City, Country \\
% email address or ORCID}
% \and
% \IEEEauthorblockN{5\textsuperscript{th} Given Name Surname}
% \IEEEauthorblockA{\textit{dept. name of organization (of Aff.)} \\
% \textit{name of organization (of Aff.)}\\
% City, Country \\
% email address or ORCID}
% \and
% \IEEEauthorblockN{6\textsuperscript{th} Given Name Surname}
% \IEEEauthorblockA{\textit{dept. name of organization (of Aff.)} \\
% \textit{name of organization (of Aff.)}\\
% City, Country \\
% email address or ORCID}
% }

\maketitle

\begin{abstract}
Zero-Knowledge Proofs (ZKPs) are a cryptographic primitive that allows a prover to demonstrate knowledge of a secret value to a verifier without revealing anything about the secret itself. ZKPs have shown to be an extremely powerful tool, as evidenced in both industry and academic settings. In recent years, the utilization of user data in practical applications has necessitated the rapid development of privacy-preserving techniques, including ZKPs. This has led to the creation of several robust open-source ZKP frameworks. However, there remains a significant gap in understanding the capabilities and real-world applications of these frameworks. Furthermore, identifying the most suitable frameworks for the developers' specific applications and settings is a challenge, given the variety of options available. The primary goal of our work is to lower the barrier to entry for understanding and building applications with open-source ZKP frameworks.

In this work, we survey and evaluate 25 general-purpose, prominent ZKP frameworks. Recognizing that ZKPs have various constructions and underlying arithmetic schemes, our survey aims to provide a comprehensive overview of the ZKP landscape. These systems are assessed based on their usability and performance in SHA-256 and matrix multiplication experiments. Acknowledging that setting up a functional development environment can be challenging for these frameworks, we offer a fully open-source collection of Docker containers. These containers include a working development environment and are accompanied by documented code from our experiments. We conclude our work with a thorough analysis of the practical applications of ZKPs, recommendations for ZKP settings in different application scenarios, and a discussion on the future development of ZKP frameworks.
\end{abstract}

% \begin{IEEEkeywords}
% component, formatting, style, styling, insert
% \end{IEEEkeywords}

\begin{figure}
    \centering
    \begin{tikzpicture}[font=\footnotesize]
        \node (img) {\includegraphics[width=0.7\columnwidth]{figpaper/nfe_vs_fvd_vs_ep_ffs_teaser.pdf}};
            \node[anchor=north west, xshift=25pt, yshift=-5pt] at (img.north west) {
                \begin{tabular}{ll}
                \scriptsize
                    \textcolor[HTML]{A0A0A0}{\rule{6pt}{6pt}} &Rolling Diffusion \cite{ruhe2024rollingdiffusionmodels} \\
                    \textcolor[HTML]{e9cbc4}{\rule{6pt}{6pt}} &Diffusion Forcing \cite{chen2024diffusionforcing} \\
                    \textcolor[HTML]{F4A700}{\rule{6pt}{6pt}} &MaskFlow (\textit{Ours})
                \end{tabular}
            };
    \end{tikzpicture}
    \vspace{-7pt}
    \caption{\textbf{Our method (MaskFlow) improves video quality compared to baselines while simultaneously requiring fewer function evaluations (NFE)} when generating videos $2\times$, $5\times$, and $10\times$ longer than the training window.
}
    \label{fig:teaser}
    \vspace{-10pt}
\end{figure}

\section{Introduction}

Due to the high computational demands of both training and sampling processes, long video generation remains a challenging task in computer vision. Many recent state-of-the-art video generation approaches train on fixed sequence lengths \cite{blattmann2023stable,blattmann2023align_videoldm,ho2022video} and thus struggle to scale to longer sampling horizons. Many use cases not only require long video generation, but also require the ability to generate videos with varying length. A common way to address this is by adopting an autoregressive diffusion approach similar to LLMs \cite{gao2024vid}, where videos are generated frame by frame. This has other downsides, since it requires traversing the entire denoising chain for every frame individually, which is computationally expensive. Since autoregressive models condition the generative process recursively on previously generated frames, error accumulation, specifically when rolling out to videos longer than the training videos, is another challenge.
\par
Several recent works \cite{ruhe2024rollingdiffusionmodels, chen2024diffusionforcing} have attempted to unify the flexibility of autoregressive generation approaches with the advantages of full sequence generation. These approaches are built on the intuition that the data corruption process in diffusion models can serve as an intermediary for injecting temporal inductive bias. Progressively increasing noise schedules \cite{xie2024progressive,ruhe2024rollingdiffusionmodels} are an example of a sampling schedule enabled by this paradigm. These works impose monotonically increasing noise schedules w.r.t. frame position in the window during training, limiting their flexibility in interpolating between fully autoregressive, frame-by-frame generation and full-sequence generation. This is alleviated in \cite{chen2024diffusionforcing}, where independent, uniformly sampled noise levels are applied to frames during training, and the diffusion model is trained to denoise arbitrary sequences of noisy frames. All of these works use continuous representations.
\par
We transfer this idea to a discrete token space for two main reasons: First, it allows us to use a masking-based data corruption process, which enables confidence-based heuristic sampling that drastically speeds up the generative process. This becomes especially relevant when considering frame-by-frame autoregressive generation. Second, it allows us to use discrete flow matching dynamics, which provide a more flexible design space and the ability to further increase our sampling speed. Specifically, we adopt a \emph{frame-level masking} scheme in training (versus a \emph{constant-level masking} baseline, see Figure~\ref{fig:training}), which allows us to condition on an arbitrary number of previously generated frames while still being consistent with the training task. This makes our method inherently versatile, allowing us to generate videos using both full-sequence and autoregressive frame-by-frame generation, and use different sampling modes. We show that confidence-based masked generative model (MGM) style sampling is uniquely suited to this setting, generating high-quality results with a low number of function evaluations (NFE), and does not degrade quality compared to diffusion-like flow matching (FM)-style sampling that uses larger NFE. 
Combining frame-level masking during training with MGM-style sampling enables highly efficient long-horizon rollouts of our video generation models beyond $10 \times$ training frame lengths without degradation. We also demonstrate that this sampling method can be applied in a timestep-\emph{independent} setting that omits explicit timestep conditioning, even when models were trained in a timestep-dependent manner, which further underlines the flexibility of our approach. In summary, our contributions are the following:

\begin{itemize}
    \item To the best of our knowledge, we are the first to unify the paradigms of discrete representations in video flow matching with rolling out generative models to generate arbitrary-length videos. 
    \item We introduce MaskFlow, a frame-level masking approach that supports highly flexible sampling methods in a single unified model architecture.
    \item We demonstrate that MaskFlow with MGM-style sampling generates long videos faster while simultaneously preserving high visual quality (as shown in Figure~\ref{fig:teaser}).
    \item Additionally, we demonstrate an additional increase in quality when using full autoregressive generation or partial context guidance combined with MaskFlow for very long sampling horizons.
    \item We show that we can apply MaskFlow to both timestep-dependent and timestep-independent model backbones without re-training.
\end{itemize}

\begin{figure}
    \centering
    \includegraphics[width=0.75\linewidth]{figpaper/training.pdf}
    \caption{\textbf{MaskFlow Training:} For each video, Baseline training applies a single masking ratios to all frames, whereas our method samples masking ratios independently for each frame.}
    \vspace{-10pt}
    \label{fig:training}
\end{figure}

















\section{Zero-Knowledge Proofs}

Zero-Knowledge Proofs (ZKPs) are a cryptographic primitive that allow a prover \Prv to prove to a verifier \Vrf that they know a secret value $w$, called the witness, without revealing anything about $w$. \Prv does this by showing that they know a secret value $w$ such that $\mathcal{F}$ evaluated at $w$ equals some public output $y$. Formally, \Prv sends a proof attesting that $\mathcal{F}(x; w)=y$, where $x$ and $y$ are public inputs and outputs, respectively. ZKPs have three core attributes \cite{goldreich1994definitions}:
\begin{enumerate}
    \item \textbf{Soundness}: \Vrf will find out, with a very high probability, if a \Prv is dishonest if the statement is false.
    \item \textbf{Completeness}: An honest \Prv can convince \Vrf if the statement is true.
    \item \textbf{Zero-Knowledge}: If the statement is true, \Vrf will learn nothing about the \Prv's private inputs - only that the statement is true.
\end{enumerate}
In the following sections, we discuss the evolution of ZKPs, the nuances of specific classes and schemes, and provide a detailed overview of the current ZK landscape.

% \subsection{The Evolution of ZKPs}

% \subsection{Interactive vs. Non-Interactive}

% ZKPs can broadly be classed into two categories: interactive and non-interactive \cite{wu2014survey}. Interactive protocols, as the name suggests, require several rounds interaction before \Vrf is convinced that \Prv's proof is valid. This is done by \Vrf sending random challenges to \Prv until \Vrf is convinced that \Prv's proof is valid. Interactive ZKPs require that both \Prv and \Vrf stay online until \Vrf is convinced. This somewhat limits the utility of interactive ZKPs, as the proofs are \textit{designated-verifier}, meaning that \Prv's proof can only be used to be convinced a single verifier. A separate protocol must be performed for each new \Vrf. Conversely, non-interactive ZKPs are normally \textit{publicly verifiable}, meaning \Prv can generate a single proof in one-shot that any \Vrf can verify. Non-interactive ZKPs often rely on a trusted setup process from a third-party, or in some cases \Vrf, to generate randomness that allows for a proof to be generated that \Vrf accepts as valid without further interaction. Many non-interactive schemes aim to minimize proof size, which results in higher \Prv computational power requirements. This limits the scalability of these schemes, especially in scenarios where \Prv is resource-constrained. The interactivity of interactive ZKPs allows for a more scalable approach in terms of \Prv computation, albeit limiting the amount of verifiers that can verify a proof. If needed, there is a method for turning public-coin interactive ZKPs into non-interactive ZKPs. The Fiat-Shamir transform \cite{kilian1992note} replaces \Vrf's randomness with a random oracle (i.e. a cryptographic hash function), thus removing the interaction and turning interactive ZKPs into non-interactive ZKPs. \nojan{The Fiat-Shamir transform is not (necessarily) a transform if we are using a random oracle (as opposed to an instantiated hash function. The appropriate term to use is instead "Fiat--Shamir transform". Also, the sentence should clarify that only *public-coin* interactive proofs can be converted to non-interactive proofs via the FS transform. DONE}
 
\subsection{Taxonomy of ZKPs}

In this work, we analyze 25 ZK protocols. Amongst these protocols are a mix of interactive and non-interactive schemes. An in-depth explanation of the difference between interactive and non-interactive schemes can be found in Appendix \ref{sec:interactive}. From now on, we describe computation as circuits \Cir, as that is what they are referred to as in ZK literature. This is due to the process of arithmetization, which represents functions, such as Python/C++ code, as arithmetic circuits, then converts these circuits into a mathematical representation (e.g. polynomials) that can be used within ZKPs. Oftentimes, an intermediate step between the input and output is a set of constraints that describes the code/circuit. These constraints act as the basis for the mathematical representation. For brevity's sake, we do not discuss the details of arithmetization and refer to the brilliant explanations of \cite{LambdaClass2023ArithmetizationSchemes, ButerinQuadraticArithmeticPrograms}. In this text, we only treat arithmetization as a black-box and do not require the knowledge of specific details, only the inputs (e.g. code) and outputs (e.g. mathematical representation). Table \ref{tab:pros} compares the seminal ZK protocols at a high-level. Below, we describe the taxonomy of the general schemes that underlie our chosen ZK protocols in detail. 
% \nojan{Do we need arithmetization section?}

\begin{table*}[t]
\centering\resizebox{\textwidth}{!}{
\begin{tabular}{lll}
\toprule
\textbf{Construction} & \textbf{Key Advantages} & \textbf{Key Disadvantages}\\
\midrule
zk-SNARKs & Succinct, Publicly Verifiable & Trusted Setup Required, Computationally Expensive to Prove, Not Post-Quantum 
% & Applications requiring very fast verification (e.g., blockchain scaling, identity management). 
\\
\midrule
zk-STARKs & No Trusted Setup, Post-Quantum Secure, Scalable Prover, Publicly Verifiable & Larger Proof Sizes, Slow Verification 
% & Systems requiring transparency and post-quantum security where proof size is not paramount.
\\
\midrule
MPCitH & No Trusted Setup, Post-Quantum Secure, Publicly Verifiable & Slow Verification, Computationally Expensive Proving
% & Situations where proof data can be accumulated and verified in large batches or where flexibility is key. 
\\
\midrule
VOLE-ZK & Highest Scalability, No Trusted Setup, Post-Quantum Secure & Slow Verification, Designated Verifier
% & Similar to MPCitH, good for specialized use-cases with highly optimized MPC protocols. 
\\
\bottomrule
\end{tabular}}
\caption{Core Attributes of Popular ZKP Constructions}
\label{tab:pros}
\end{table*}

\begin{table}[t]
    \centering\centering\resizebox{\columnwidth}{!}{
\begin{tabular}{l|cccc}
\hline
& zk-SNARKs & zk-STARKs & MPCitH & VOLE-ZK \\
\hline
Prover complexity & $O(n\log{}(n))$ & $O(n\text{poly-log}(n))$& $O(n)$ & $O(n)$ \\
\hline
Verifier complexity & $O(1)$ & $O(\text{poly-log}(n))$ & $O(n)$ & $O(n)$ \\
\hline
Proof size & $O(1)$ & $O(\text{poly-log}(n))$ & $O(n)$ & $O(n)$ \\
\hline
Trusted setup & \cmark & \xmark & \xmark & \xmark \\
\hline
Non-interactive & \cmark & \cmark & \cmark & \xmark \\
\hline
Post-quantum secure & \xmark & \cmark & \cmark & \cmark \\
% \hline
% Crypto assumptions & \makecell{Elliptic curves \\ + pairing} & \makecell{Collision resistant \\ hashes} & \makecell{Symmetric key \\ primitives} & \makecell{Pseudorandom \\ generators} \\
\hline
Practical proof size & ~120-500 bytes & ~10 KB - 1 MB & ~10-1000 KB & ~5-200 KB \\
\hline
\end{tabular}}
\caption{Asymptotic attributes of presented ZKP constructions. We do note that, due to the variance of schemes within each construction, the algorithmic complexities are generalized and may not hold true for all schemes within a given construction.}
\end{table}

\textbf{Zero-Knowledge Succinct Non-Interactive Arguments of Knowledge (zk-SNARKs)} are, as the name suggests, a class of non-interactive protocols that boast small proof size \cite{ben2014succinct}. Although ZKPs were originally conceived in the late 1980's \cite{goldwasser2019knowledge}, zk-SNARKs were formally introduced about a decade after. Efficient instantiations of zk-SNARKs were introduced in the last decade, resulting in recent advancements in making zk-SNARKs practical and efficient for widespread use. 
% \nojan{zkSNARKs are *not* the earliest "ZK" constructions conceived; indeed, modern renditions of zkSNARKs are only 10-13 years old, while the concept of ZKPs is almost 40 years old. (While the zkSNARK of Micali also dates back to 30 years ago, efficient instantiations of it only date back to around 10 years ago as well) DONE}
This means there are much more mature open-source and real-world implementations available. The most common forms of zk-SNARKs are referred to as \textit{pre-processing zk-SNARKs}. One of the main drawbacks of these zk-SNARKs are that they require a trusted setup for every new circuit \Cir, which is computationally intensive and requires communication of large proving and verifying keys to the respective parties. 
Alongside this, \Prv must normally be computationally powerful in order to ensure small proof size. This is due to the fact that most zk-SNARKs are reliant on elliptic curve cryptography (ECC) as their underlying cryptographic arithmetic.
Recent works have introduced zk-SNARKs that can utilize \textit{universal} trusted setups \cite{plonk, chiesa2020marlin} for established maximum circuit sizes, and zk-SNARKs that do not require a trusted setup at all \cite{setty2020spartan, wahby2018doubly}. 
% These result in the evaluation and commitment of very large polynomials, which takes a toll on the complexity of any \Prv algorithm.
Due to the non-interactivity, zk-SNARKs are \textit{publicly verifiable}, meaning any verifier can verify them without recomputing the proof. One of the most common underlying schemes, especially in our highlighted frameworks, for zk-SNARKs is Groth16 \cite{groth16}, which improves upon the original Pinocchio \cite{parno2016pinocchio} protocol. zk-SNARK arithmetization \textit{typically} results in a set of constraints, called Rank 1 Constraint Systems (R1CS) \cite{belles2022circom}, which are then converted to a set of polynomials, called a Quadratic Arithmetic Program (QAP) \cite{gennaro2013quadratic}. We do note that there are different formats that are zk-SNARKs are compatible with, such as Algebraic Intermediate Representations (AIR) \cite{ben2018scalable} and Plonkish tables - we simply highlight R1CS as a prevalent constraint system. The Groth16 zk-SNARK generation and verification process can be represented at a high-level with the following algorithms:
\begin{itemize}
    \item $(\mathcal{VK, PK})\xleftarrow[]{}$ Setup(\Cir): A trusted third party or \Vrf run a setup procedure to generate a prover key $\mathcal{PK}$ and verifier key $\mathcal{VK}$. These keys are used for proof generation and verification, respectively. This setup must be repeated each time \Cir changes.
    \item $\pi \xleftarrow[]{}$ Prove($\mathcal{PK}$, \Cir, $x$, $y$, $w$): \Prv generates proof $\pi$ to convince \Vrf that $w$ is a valid witness.
    \item $1/0 \xleftarrow[]{}$ Verify($\mathcal{VK}$, \Cir, $x$, $y$, $\pi$): \Vrf accepts or rejects proof $\pi$. Due to soundness property of zk-SNARKs, \Vrf cannot be convinced that $w$ is a valid witness by a cheating \Prv.
\end{itemize}

\noindent In Appendix \ref{sec:recursive}, we describe how zk-SNARKs can be extended to allow recursive construction and verification of proofs.

As we stated, one of the drawbacks of traditional pre-processing zk-SNARKs is their reliance on a trusted setup per circuit \Cir.
% \noindent \nojan{appendix} Recent works \cite{bowe2019recursive, kothapalli2022nova, bitansky2013recursive} have shown the usability of recursive zk-SNARKs, which is verifying multiple zk-SNARKs in a single zk-SNARK. As the verification algorithm of zk-SNARKs is simply an arbitrary computation, it can be represented as a circuit \Cir. This enables one \Prv to generate many proofs, then generate a proof that verifies these proofs and send it to \Vrf. While this results in substantially more work on \Prv, \Vrf now only has to generate one proof to verify all of \Prv's data, rather than many individual proofs. 
% \nojan{mention ECC and bilinear mappings}
% \noindent\textbf{Interactive}
PLONKS, a subset of zk-SNARKs, are a class non-interactive ZK protocols that improve upon pre-processing zk-SNARKs by getting rid of the trusted setup per circuit \Cir, while adding a bit more arithmetic flexibility \cite{plonk}. PLONKs utilize the idea of a universal and updatable trusted setup, introduced in theory by \cite{cryptoeprint:2018/280} and in practice by \cite{cryptoeprint:2019/099}, in which a trusted setup procedure is done for circuits up to a certain size. 
% As mentioned in the previous section, some zk-SNARKs support universal and updateable setup - PLONKs are a specific included in this group.
% \nojan{"PLONKs" did not introduce the notion of universal and updatable setup; the first work to suggest this notion was that of \url{https://urldefense.com/v3/__https://eprint.iacr.org/2018/280__;!!Mih3wA!CvVqOHy12qd4JA_R87hM_kzJDnP_zincsfv36HU7wk-Ug48iOQHAiqfnlHplxrc01wVlpNigN-eTh3CLUudEOmE7apE$} , and the first efficient construction for NP-complete languages was that [Sonic](\url{https://urldefense.com/v3/__https://eprint.iacr.org/2019/099__;!!Mih3wA!CvVqOHy12qd4JA_R87hM_kzJDnP_zincsfv36HU7wk-Ug48iOQHAiqfnlHplxrc01wVlpNigN-eTh3CLUudEg2ZegzE$} ).
% \nojan{"PLONKs are not widely adopted": first of all I don't see why PLONK-type arithmetization based SNARKs need to be separately categorized (they're a subcategory) and take a look at L2Beat multiple deployments exist} There are also plenty of other constructions that achieve this property, e.g., Marlin, Lunar, ECLIPSE, etc. DONE}
Every circuit \Cir that fits within these size constraints can utilize the parameters generated by the universal trusted setup process. While PLONKs introduce a universal trusted setup, it comes at the cost of proof size and \Vrf runtime. PLONK proofs are normally $2-5\times$ the size of zk-SNARKs, and \Vrf runtime is marginally higher. It is important to note that, although PLONK proofs are larger than those of zk-SNARKs, proof size still remains in the KB range. The advantage that PLONKs have is that they are flexible in the commitment scheme they can use. By using the standard Kate commitments \cite{kate2010constant}, PLONKs become more zk-SNARK-like, as these commitments are based on ECC. FRI commitments \cite{ben2018fast}, which rely on Reed-Solomon codes and low-degree polynomials/testing for verifiers, can also be used to make PLONKs more zk-STARK-like. The type of commitment schemes allows developers to balance the tradeoff between performance and security assumptions. PLONK arithmetization is similar to that of zk-SNARKs, meaning that the resulting representation is a set of polynomials. To get there, PLONKs sets constraints for each gate (e.g. multiplication, addition) from the arithmetic circuit representation of the computation in the form of  Lagrange polynomials. Once the constraints are set, a special permutation function is used to check consistency between commitments. Finally, a final set of polynomials is constructed to fully represent the given computation. Overall, PLONKs provide a method to flexibly construct ZKPs with a less stringent trusted setup requirement, at the slight cost of performance. 
% \nojan{check this}

\textbf{Zero-Knowledge Scalable Transparent Arguments of Knowledge (zk-STARKs)}, which can be thought of as interactive oracle proof (IOP)-based zk-SNARKS, completely remove the dependence on trusted setup. Rather than using randomness from a trusted party, these protocols use publicly verifiable randomness for generating the necessary parameters for proof generation and verification. zk-STARKs achieve post-quantum security guarantees by utilizing collision-resistant hash functions as their underlying cryptography, rather than ECC. This increased security comes at a cost, as zk-STARK proofs are typically an order of magnitude larger than zk-SNARKs and PLONKs, and require more computational resources to generate and verify \cite{ben2018scalable}. The main contributor towards these drawbacks are the underlying data structure that are used in proof generation: Merkle trees. In zk-STARKs, Merkle trees are used to create a compact representation of the computation's execution trace. During proof generation, the computation's execution trace is arithmetized into polynomials, which are verified by performing low-degree testing, a process which ensures that the polynomials are of expected degree. Low-degree testing is enabled by the use of FRI commitments \cite{habock2022summary}. The polynomials are evaluated at certain points to verify their correct represenation of the execution trace, and these evaluations are used as the leaf nodes of the Merkle tree. The root of the Merkle tree then acts as a sort of commitment to these evaluated polynomials, hence allowing the verifier to simply verify the root, rather than verifying the whole computation trace. \cite{ashur2018marvellous} The use of Merkle trees are what enable the \textit{scalability} of zk-STARKs. While the Merkle trees support efficient verification, the proof size is drastically increased due to the inclusion of the material needed for verification, such as the Merkle root, polynomial evaluations, FRI commitments, and necessary Merkle branches. We note that there are IOP-based zk-SNARKs that stray away from this general protocol, but these steps are the most consistently utilized in current literature. Overall, zk-STARKs primarily benefit from being scalable and post-quantum secure with no trusted setup, at a significant cost to proof size and \Prv/\Vrf computation.

\textbf{MPC-in-the-Head (MPCitH)} ZKPs are a class of ZK protocols that take a completely novel approach towards proof generation and verification. The primary cryptographic basis is secure multiparty computation (MPC). MPC is a cryptographic primitive that allows for $n$ parties to jointly compute a function $f(x_1, ..., x_n)$, on private inputs from each party, without leaking any information about the private inputs. One of the prominent approaches to enable MPC is secret sharing, in which parties distributes secret shares of their private inputs amongst each other to compute a function. MPCitH, proposed by \cite{ishai2007zero}, allows for \Prv to simulate the $n$ MPC parties and following computation locally, or "in the head". Theoretically, any MPC protocol that can compute arbitrary functions can be transformed into a MPCitH ZKP. For $n$ parties $\{P_1,...,P_n\}$, secret shares are generated by each party and distributed to every other party. For the underlying arithmetic, the circuit \Cir is defined in an MPC manner to operate on secret shared data. \Prv can then simulate each parties' computation of the circuits with the secret shares they obtained from all other parties. After this is complete, \Prv has $n$ sets of messages and data that were generated and received by each party, called views. \Prv uses a standard commitment scheme to generate $n$ view commitments. Finally, \Prv and \Vrf interactively verify a subset of these views for consistency and correctness \cite{sidorenco2021formal}. While MPCitH protocols are innately interactive, they can be made non-interactive using the Fiat-Shamir transform. Theoretically, a huge advantage of the MPCitH approach is that MPC-friendly optimizations, which have been much further studied, can be utilized during proof generation to drastically improve \Prv efficiency and proof length. However, the most effective optimizations for MPC may translate to effective solutions for MPCitH. One of the core parameters MPCitH schemes aim to minimize is the communication complexity, as this directly reduces the amount of data that is present in each party's committed view. Just like zk-STARKs, MPCitH-based ZKPs do not require a trusted setup and are post-quantum secure, as MPC is thought to be generally quantum secure \cite{sidorenco2021formal}. Overall, MPCitH proposes a unique approach towards ZKP construction that are transparent post-quantum secure that allows flexibility in the underlying arithmetic to optimize the cost of proof generation and verification and proof size.

\textbf{Vector Oblivious Linear Evaluation (VOLE)-based ZK} 
protocols are a set of interactive techniques that achieve high efficiency and scalability through the use of information-theoretic message authentication code (IT-MAC)-based commitment schemes, which can be efficiently implemented using VOLE correlations. In VOLE-based ZKP protocols, the prover acts as the VOLE sender, while the verifier takes on the role of the VOLE receiver.
VOLE correlations are a pair of random variables, (\textbf{u}, \textbf{x}), known by \Prv and (\textbf{v}, $\Delta$), only known by \Vrf, in which \textbf{u}, \textbf{x}, and \textbf{v} are vectors, and $\Delta$ is a scalar key \cite{}. These variables satisfy the relation:
\begin{equation*}
    u_i = v_i + x_i \cdot \Delta
\end{equation*}
This functionality typically operates over a finite field.
Generally, in VOLE-based ZK, IT-MACs are used as commitments to authenticated wire values in arithmetic or boolean circuits representing a computation $\mathcal{C}$. \Prv demonstrates knowledge of a private vector $\textbf{w}$, which represents the witness, where $\mathcal{C}(\textbf{w}) = 1$, while proving the consistency throughout the protocol, without revealing any information about \textbf{w}.
\nojan{review and add VOLEitH}
VOLE-based proofs provide unparalleled scalability and communication optimizations, however, they are inherently designated-verifier protocols, meaning that \Prv must communicate with every \Vrf that aims to verify the proof, as \Vrf must maintain the secret $\Delta$ to ensure soundness. To address this, \cite{baum2023publicly} proposes a new VOLE-based paradigm, entitled VOLE-in-the-head (VOLEitH), which enables non-interactive VOLE-based ZK.

% \nojan{limbo has good explanation}


% \noindent\textbf{Recursive SNARKs}

% \nojan{make table comparing the zk schemes pros and cons}

% \subsection{Proving Systems}

% \noindent\textbf{Pinocchio}

% \noindent\textbf{Groth16}

% \subsection{Underlying Arithmetic}

% \noindent\textbf{Arithmetization}

% \noindent\textbf{Elliptic Curve Cryptography}
% Talk about https://hackmd.io/@benjaminion/bls12-381
% \section{Challenges}
\section{ZKP Libraries} 
\label{sec:libraries}
\nojan{Add swanky and picozk}
\begin{table*}[t!]
   \small
   \centering\resizebox{\textwidth}{!}{
   \begin{tabular}{cccccccc}
   \toprule
   & \multicolumn{3}{c}{\textbf{Usability}} & \multicolumn{3}{c}{\textbf{Accessibility}} & \\
   \cmidrule(lr){2-4}
   \cmidrule(lr){5-7}
   \textbf{Framework}&
   \textbf{Language(s)} & \textbf{Custom \Cir} & \textbf{License} & \textbf{Examples} & \textbf{Documentation} & \textbf{GitHub Issues} & \textbf{Last Major Update} \\ 
   
   \midrule
    & \multicolumn{7}{c}{\textbf{zk-SNARKs}} \\
   \midrule
   
   \textbf{Arkworks \cite{arkworks}} & Rust & \fullcirc & MIT, Apache-2 & \fullcirc & \cmark & \halfcirc & Dec. 2023 \\
   \textbf{Gnark \cite{gnark-v0.9.0}} & Go & \fullcirc & Apache-2 & \fullcirc & \cmark & \fullcirc & Dec. 2024 \\
   \textbf{Hyrax \cite{hyraxZK}} & Python & \halfcirc & Apache-2 & \halfcirc & \xmark & \emptycirc & Feb. 2018 \\
   \textbf{LEGOSnark \cite{legosnark}} & C++ & \halfcirc & MIT, Apache-2 & \halfcirc & \xmark & \emptycirc & Oct. 2020 \\
   \textbf{LibSNARK \cite{libsnark}} & C++, Java (xJsnark \cite{kosba2018xjsnark}) & \fullcirc & MIT & \fullcirc & \xmark & \halfcirc & Jul. 2020 \\
  \textbf{Zokrates \cite{eberhardt2018zokrates}} & Zokrates DSL & \fullcirc & LGPL-3.0 & \fullcirc & \cmark & \halfcirc & Nov.2023 \\
   \textbf{Mirage \cite{Mirage}} & Java & \halfcirc & MIT & \halfcirc & \xmark & \emptycirc & Jan. 2021 \\
   \textbf{PySNARK \cite{PySNARK}} & Python & \fullcirc & Custom (MIT-like) & \fullcirc & \cmark & \halfcirc & May 2023 \\
   \textbf{SnarkJS \cite{baylina2020iden3}} & Circom \cite{munoz2022circom} & \fullcirc & GPL-3 & \fullcirc & \xmark & \halfcirc & Oct. 2024 \\
   \textbf{Rapidsnark \cite{RapidSNARK}} & Circom \cite{munoz2022circom} & \fullcirc & GPL-3 & \fullcirc & \xmark & \emptycirc & Dec. 2023 \\
   \textbf{Spartan\cite{Spartan}} & Rust & \emptycirc & MIT & \halfcirc & \xmark & \halfcirc & Jan. 2023 \\
   \textbf{Aurora (libiop) \cite{SciprLab2023Libiop}} & C++ & \emptycirc & MIT & \halfcirc & \xmark & \halfcirc & May 2021 \\
   \textbf{Fractal (libiop) \cite{SciprLab2023Libiop}} & C++ & \emptycirc & MIT & \halfcirc & \xmark & \halfcirc & May 2021 \\
   \textbf{Virgo \cite{SunblazeUCB2023Virgo}} & Python & \emptycirc & Apache-2 & \halfcirc & \xmark & \emptycirc & Jul. 2021 \\
     \textbf{Noir \cite{Noir2023Documentation}} & Rust DSL & \fullcirc & MIT, Apache-2 & \fullcirc & \cmark & \fullcirc & Nov. 2024 \\
   % \textbf{Barretenberg \cite{AztecProtocol2023Barretenberg}} & 0 & 0 & 0 & 0 & 0 & 0 & 0 \\
   % \textbf{Bellman} & 0 & 0 & 0 & 0 & 0 & 0 & 0 \\
   \textbf{Dusk-PLONK \cite{DuskPlonk2023Rust}} & Rust & \emptycirc & MPL-2 & \halfcirc & \cmark & \fullcirc & Aug. 2024 \\
   \textbf{Halo2 \cite{Halo22023Book}} & Rust & \halfcirc & MIT, Apache-2 & \fullcirc & \cmark & \fullcirc & Nov. 2023 \\

   \midrule
   & \multicolumn{7}{c}{\textbf{MPC-in-the-Head}} \\
   \midrule

   \textbf{Limbo \cite{KULeuvenCOSIC2023Limbo}} & Bristol \cite{bristol} & \fullcirc & MIT & \halfcirc & \xmark & \emptycirc & May 2021 \\
 \textbf{Ligero (libiop) \cite{SciprLab2023Libiop}} & C++ & \emptycirc & MIT & \halfcirc & \xmark & \halfcirc & May 2021 \\ 
   % \textbf{ZKBoo} & 0 & 0 & 0 & 0 & 0 & 0 & 0 \\

   \midrule
   & \multicolumn{7}{c}{\textbf{VOLE-Based ZK}} \\
   \midrule
   \textbf{Mozzarella \cite{baum2022moz}} & Rust & \emptycirc & MIT & \halfcirc & \xmark & \emptycirc & Mar. 2022 \\
   \textbf{Diet Mac'n'Cheese \cite{dietmc}} & PicoZK \cite{picozk} & \fullcirc & MIT & \fullcirc & \xmark & \emptycirc & Sep. 2024 \\
  \textbf{Emp-ZK \cite{empzk}} & C++ & \fullcirc & MIT & \fullcirc & \xmark & \halfcirc & Sep. 2023 \\
   

   \midrule
   & \multicolumn{7}{c}{\textbf{zk-STARKs}} \\
   \midrule

   \textbf{MidenVM \cite{PolygonMiden2023MidenVM}} & Miden Assembly & \halfcirc & MIT & \halfcirc & \cmark & \fullcirc & Nov. 2023 \\
   % \textbf{Starky} & 0 & 0 & 4 & 0 & 0 & 0 & 0 \\
   \textbf{Zilch \cite{TrustworthyComputing2023Zilch}} & Java DSL & \halfcirc & MIT & \fullcirc & \xmark & \emptycirc & Apr. 2022 \\
   \textbf{RISC Zero \cite{RISCZero2023DeveloperDocs}} & Rust, C++ & \fullcirc & Apache-2 & \fullcirc & \cmark & \fullcirc & Dec. 2024 \\
   
   \bottomrule
   \end{tabular}}
   \caption{ZK Framework Attributes} 
      \label{tab:usability}
\end{table*}

% \nojan{maybe add circom}
In this section, we discuss the details of the 25 frameworks that we target in this work. We aim to highlight frameworks bred from both industry and academia. We primarily focus on works that present novel implementations of proving schemes that can be integrated with their own exposed high-level API for custom circuit design, or a general-purpose ZKP circuit development frontend, such as Circom \cite{munoz2022circom} or Zokrates \cite{eberhardt2018zokrates}.

Alongside the in-depth descriptions of each framework, we provide an evaluation of these frameworks at a high-level on usability and accessibility metrics, presented in Table \ref{tab:usability}. Our measurement of some metrics require further explanation:
\begin{itemize}
    \item Custom \Cir: \fullcirc $=$ Non-cryptography software engineer can build custom circuits, \halfcirc $=$ Building custom circuits requires deep knowledge of syntax; A developer could not read the code and understand it, \emptycirc $=$ Custom circuits require deep knowledge of protocol and syntax; normally requires manual translation of constraints to gates
    \item Examples: \fullcirc$=$ Plenty of examples are shared that fully show the capabilities of the system, \halfcirc $=$ Examples are included, but are not representative of the system's full capabilities
    \item Github Issues: \fullcirc $=$ Users and developers are both active in issues forum, \halfcirc $=$ Users are relatively active and developers are sporadically active, \emptycirc $=$ No activity
\end{itemize}

Table \ref{tab:description} in Appendix \ref{sec:app_libraries} outlines the discussed frameworks at a high level.

% \nojan{Specify between frontend and backend}

% \nojan{RISC zero, }

% \nojan{Look into PLONK languages: Fast provng and small proofs}

\subsection{zk-SNARKs}

\textbf{libsnark.} The \texttt{libsnark} C++ development library \cite{libsnark} is widely regarded as the original and most well-developed library for zk-SNARKs. This is highlighted by the fact that Zcash, the first real-world implementation of zk-SNARKs, was built upon \texttt{libsnark}. \texttt{libsnark} supports the Pinocchio \cite{parno2016pinocchio} and Groth16 \cite{groth16} proving schemes, alongside many different underlying elliptic curves. Much of the novelty of \texttt{libsnark} comes from the different forms of circuits that it supports. It supports R1CS and QAPs, as most frameworks do, but also supports higher level forms such as Unitary-Square Constraint Systems (USCS) and Two-input Boolean Circuit Satisfiability (TBCS) \cite{uscs}. The scheme used in \texttt{libsnark} is described as a preprocessing zk-SNARK, which simply highlights that trusted setup is performed before proof generation and verification. \texttt{libsnark} provides low-level "gadgets", which can be combined and built upon to represent the desired computation in R1CS format, however it is not the easiest way to develop zk-SNARKs in this library. \cite{kosba2018xjsnark} presents \texttt{xJsnark}, a high-level Java framework that allows a user to essentially code their computation in standard Java. Behind the scenes, this framework optimizes computation and outputs the computation in R1CS format. This output can be used directly with \texttt{libsnark}'s zk-SNARK generation script. Combined with \texttt{xJsnark}, \texttt{libsnark} is a highly-accessible option for inexperienced ZKP developers.

\textbf{gnark.} The \texttt{gnark} library \cite{gnark-v0.9.0} enables developers to build zk-SNARK-based applications using the high-level API it offers in Go language. The primary focus of \texttt{gnark} is runtime speed \cite{ConsenSys2023Gnark}. It offers both Groth16~\cite{groth16} and PLONK~\cite{plonk} (with KZG and FRI polynomial commitment) SNARK protocols. It offers a lot of curves, and can build R1CS circuits. In terms of hashing, it offers MiMC~\cite{mimchash}, SHA2, and SHA3 gadgets out-of-the-box. It also offers a collection of high-level gadgets for ease of building custom circuits. This framework exposes a high-level API that allows users to build their own gadgets, while utilizing the Go standard language and the provided gadgets. Recently, \texttt{gnark} has introduced GPU support with the support of the Icicle library \cite{icicle}. This work is in active development and seems to have an active community around it, making it an accessible option for inexperienced ZKP developers. We recommend this for beginners and experts alike for almost any custom applications. This framework utilizes a readable and robust API that any user can take advantage of and build custom applications with.

\textbf{arkworks.} The \texttt{arkworks} Rust ecosystem~\cite{arkworks} is an extensive and modular collection of libraries that can be used for efficient zk-SNARK programming. This ecosystem provides highly efficient implementations of arithmetic over \textit{various} curves and fields, even allowing curve specific optimizations. The main offering of \texttt{arkworks} is a generic application development framework that supports both experienced and non-experienced zk-SNARK developers. This framework enables high-level zk-SNARK development, as it allows users to implement their circuit as constraints (R1CS), while abstracting out details of SNARKs and curves, using an \texttt{arkworks} library. To venture into lower-level optimizations, \texttt{arkworks} provides libraries for the user to describe their circuit in native code. This allows the users to make several design decisions, such as specifying which proving system, such as Groth16, they would like to use. Alongside this, \texttt{arkworks} also provides libraries implementing low-level finite field, elliptic curve, and polynomial interfaces. In addition to SHA256, ZKP-friendly hashes such as Pedersen \cite{pedersenhash} and Poseidon \cite{poseidonhash} hashing are also offered. The \texttt{arkworks} development ecosystem is actively maintained and has an active community. We recommend this framework for users that have a deep knowledge of ZKPs, as one of the main advantages of arkworks, other than it's fantastic and usable codebase, is the ability to tweak certain parameters to optimize operations for your custom application.

\textbf{hyraxZK.} \label{sec:hyrax} \texttt{Hyrax} is a "doubly-efficient" zk-SNARK scheme, providing a concretely efficient prover and verifier, with low communication cost and no trusted setup \cite{wahby2018doubly}. Instead of following a standard underlying zk-SNARK structure, \texttt{Hyrax} is built on top of the Giraffe interactive proof scheme \cite{wahby2017full}. The authors apply a technique to reduce communication cost and add cryptopgraphic operations to turn the interactive proof into a ZKP. With the addition of optimized cryptographic commitments, the concrete cost of this scheme is significantly reduced and results in an interactive ZKP scheme. Using the Fiat-Shamir transform \cite{kilian1992note}, this scheme is made non-interactive. \texttt{hyraxZK} \cite{hyraxZK} provides a cleanly-developed Python and C++ development environment using \texttt{Hyrax} as the underlying zk-SNARK scheme. The provided framework is well-developed, however there is a lack of documentation that makes it challenging to build custom circuits.

\textbf{libspartan.} \texttt{libspartan} \cite{Spartan} is a Rust library that implements the \texttt{Spartan} zk-SNARK proof system \cite{setty2020spartan}. \texttt{Spartan} is a transparent zk-SNARK proof system, meaning that it requires no trusted setup. \texttt{libspartan} utilizes a Rust implemention of group operations on prime-order group Ristretto \cite{ristretto} and elliptic curve Curve25519 \cite{bernstein2006curve25519}, which ensures security and speed.
% Similar to \texttt{Hyrax}, outlined in section \ref{sec:hyrax}, \texttt{Spartan} is developed by building upon an interactive proof protocol, which is the sum-check protocol in \texttt{Spartan}'s case. The resulting interactive argument of knowledge is then turned into a zk-SNARK using the same techniques as \texttt{Hyrax}.
By adding a new commitment scheme, alongside a novel cryptographic compiler and a compact encoding of R1CS instances, \texttt{Spartan} is able to achieve the first transparent proof system with sub-linear verification costs and a time-optimal prover, at the cost of memory-heavy computation on the prover side. \texttt{libspartan} is a well-developed and maintained framework, however implementing custom functions is not very straightforward based on the provided documentation. Developing a custom ZKP circuit in \texttt{libspartan} requires the user to have the parameters of the R1CS instance, alongside knowledge of how to encode the constraints into R1CS matrices. Depending on the size of the ZKP circuit, this process can be very rigorous and involved, while also requiring a full knowledge of R1CS representations.
\texttt{Zokrates} \cite{eberhardt2018zokrates} provides a high-level API to build an R1CS for custom ZKP circuits, however a developer then has to manually convert these into a format that is readable by \texttt{libspartan}, which can be time-intensive depending on the number of constraints in the circuit. We only recommend this framework to users that have an in-depth knowledge of ZK constraint systems, however, we do note that this framework's backend is state-of-the-art and, upon integration with a standard frontend, would be a perfect solution for most ZK applications.

\textbf{Mirage.}
\texttt{Mirage} \cite{kosba2020mirage} is a universal zk-SNARK scheme and aptly named Java framework \cite{Mirage} implementing such scheme. \texttt{Mirage}'s main contribution is a universal trusted setup, such that trusted setup does not have to be performed everytime the circuit changes, as is done in zk-SNARKs. This saves a great amount of time and computation at the cost of higher proof computation overhead. This work introduces the idea of \textit{separated zk-SNARKs}, which enables efficient randomized checks in zk-SNARK circuits. This results in simplified verification complexity. Combining this with their novel universal circuit generator that produces circuits linear in the number of additions and multiplications, the \texttt{Mirage} zk-SNARK scheme is introduced. The underlying scheme and circuit generator are implemented in the \texttt{mirage} codebase, which has a Java frontend for circuit generation and a C++ backend implementing \texttt{Mirage} on top of \texttt{libsnark}. The core of development is done in \texttt{mirage}'s universal circuit generator, as that is where the ZKP circuits are specified by the user. This codebase provides very readable and diverse examples that highlight the use cases of their high-level Java API. 
Not only is there a bit of a learning curve to get acquainted with \texttt{mirage}'s syntax, but we also found that the codebase is relatively outdated, meaning that the code no longer compiles.
% it is a very straightforward way to develop universal zk-SNARKs, and does not require an in-depth knowledge of ZKPs to build ZK applications.

\textbf{LegoSNARK.} 
\texttt{LegoSNARK} \cite{campanelli2019legosnark} is a zk-SNARK scheme and library that focuses on linking SNARK "gadgets" together to build zk-SNARKs with a modular approach. This library implements the modular zk-SNARKs in the form of commit-and-prove zk-SNARKs (CP-SNARKs) \cite{lipmaa2016prover}, which are a class of zk-SNARKs that prove statements about committed values. As previous CP-SNARK schemes are limited due to their reliance on a single commitment scheme, one of the most important contributions of this work is a generic construction that can convert a broad class of zk-SNARKS, such as QAP-based, to CP-SNARKs. The \texttt{LegoSNARK} library \cite{legosnark} provides end-to-end proving and verification using the proposed scheme in a C++ package. This work builds upon \texttt{libsnark}, albeit with integration to high-level \texttt{libsnark} frameworks, such as \texttt{xJsnark}. Nevertheless, this library provides readable examples for developing gadgets, making it relatively easy for experienced C++ developers to build custom gadgets for their ZK applications without an in-depth knowledge of ZKPs. We recommend this framework to users that are building modular applications that benefit from CP-SNARKs, such as matrix arithmetic.

% \anees{Mention license of all libraries?}

\textbf{PySNARK.}
\texttt{PySNARK} \cite{PySNARK} is a Python library that allows developers to use pure Python syntax to develop zk-SNARKs with various backends. PySNARK gives users access to \texttt{libsnark}, \texttt{qaptools}, \texttt{zkinterface}, and \texttt{snarkjs} backends. Compiling computation with the \texttt{libsnark} and \texttt{qaptools} performs proof generation and verification using the Groth16 and Pinnochio proving systems, respectively. Using the \texttt{zkinterface} backend simply generates \texttt{.zkif} files that can be used with the \texttt{zkinterface} package for proof generation and verification, where the underlying scheme can be chosen. Similarly, using the \texttt{snarkjs} backend generates the witness and R1CS files that can be used within our provided \texttt{snarkjs} environment. Overall, \texttt{PySNARK} is a brilliantly documented and developed library for beginners with zk-SNARKs, however it is not actively maintained. Developers that are comfortable with Python should have no trouble developing ZK applications once they become familiar with the library's syntax. Due to the Python compilation process, \texttt{PySNARK} experiences non-ideal operation times, so users should primarily use this for testing applications on the Groth16 proving system, but not for practical application development.


\textbf{SnarkJS + RapidSNARK.}
\texttt{SnarkJS} \cite{baylina2020iden3} is built on Javascript (JS) and Pure Web Assembly (WASM) and supports the Groth16, PLONK, and FFLONK underlying proving schemes. This framework accepts circuits designed in \texttt{circom} \cite{munoz2022circom}, which provides a very accessible frontend with a well-documented API for building ZK circuits. The protocols that are supported all require trusted setup, whether it be a circuit-specific setup for Groth16, or a universal setup for PLONK/FFLONK. Also, switching between ZK schemes is simply done by specifying the desired scheme as a command line argument. \texttt{SnarkJS} provides a multi-step universal setup protocol that all programs perform, alongside a Groth16-specific setup. Alongside this, the circuit to proof compilation process is done in a modular way that allows for closer debugging. In the proof generation process, the circuit characteristic's are listed for the developer (e.g. constraints, public inputs) which enables quick sanity checks. Finally, \texttt{SnarkJS} provides simple routes to turning the verifier into a smart contract, or performing the end-to-end ZKP process in browser, due to the JS and WASM backend. \texttt{RapidSNARK} \cite{RapidSNARK} is built upon C++ and Intel assembly by the same developers, and significantly improves upon \texttt{SnarkJS}. Using a very similar API, and even accepting \texttt{SnarkJS}-generated files as inputs (e.g. proving/verifier keys, witness), \texttt{RapidSNARK} allows for faster proof generation with a simple change in command line arguments from the \texttt{SnarkJS} commands. The main advantage of this framework is the utilization of parallelization within proof generation, yielding much faster results than \texttt{SnarkJS}, however the downside is that only Groth16 proofs are supported. While \texttt{SnarkJS} is more actively maintained than \texttt{RapidSNARK}, both frameworks are highly accessible for those with little experience in developing ZK applications, due to the ability to utilize a \texttt{circom} frontend.

\textbf{Virgo.}
\texttt{Virgo} \cite{zhang2020transparent} is an implementation of a novel interactive doubly-efficient ZK argument system. The main advantage of this protocol is the lack of trusted setup, which is oftentimes the most cumbersome task in zk-SNARKs. \texttt{Virgo} sees the most benefits for layered arithmetic circuits, rather than all general arithmetic circuits, as it is based off the GKR protocol \cite{goldwasser2015delegating}, which also is only catered towards structured circuits. General arithmetic circuits are addressed in a follow up work, \texttt{Virgo++} \cite{virgoplus}. The open-source implementation of this work does not have ZK commitments implemented yet, which is why we do not consider it in our survey. The main enabling factor of \texttt{Virgo} is a novel ZK verifiable polynomial delegation (zkVPD) scheme, which can essentially be seen as a commitment scheme in this scenario. 
% The underlying operations of the \texttt{Virgo} implementation are the combination of the zkVPD and GKR schemes. 
Due to the reliance on zkVPD and the allowed interactivity in this scheme, the implementation only relies on lightweight cryptography, making it a feasible development solution. While an impressive solution with great results, the repository is not actively maintained and lacks clear documentation, meaning it is not the most suitable candidate for ZK application developers.

\textbf{libiop} \label{sec:aurora}
The \texttt{libiop} framework \cite{SciprLab2023Libiop} is a collection of three protocol implementations: Aurora \cite{aurora}, Fractal \cite{fractal}, and Ligero \cite{ames2017ligero}. Ligero falls under the MPCitH category, so it is discussed later in the paper. Aurora and Fractal are both post-quantum, transparent zk-SNARKs, which classifies them more as succinct zk-STARKs. However, the authors classify their work as zk-SNARKs, which is why they are discussed here. Both works outperform prior zk-SNARKs by proposing new interactive oracle proofs (IOPs). Fractal proposes a holographic IOP \cite{babai1993transparent}, while Aurora proposes an IOP based around Reed-Solomon codes.
As for the \texttt{libiop} implementations, it does not seem to be actively maintained. While there are a few example applications for each protocol, the most useful tool in was the benchmarking scripts that were provided. This allows users to input parameters, such as number of constraints and variables, to specify a random circuit and outputs the performance metrics of the protocol. This shows how the protocols scale based on the size of the circuit. These parameters can be extracted from R1CS files (made by frameworks such as Zokrates), using our provided \texttt{R1CSReader} scripts. While the benchmarking is convenient, developing custom applications with this framework requires a deeper knowledge of the protocol that may not be easily accessible to all developers. We only recommend this to users that have a deep knowledge of the literature that these frameworks stem from.


\textbf{Noir.}
\texttt{Noir} \cite{Noir2023Documentation} is a general Rust-like framework for developing applications based on ZKPs. Fundamentally, \texttt{Noir} is a domain-specific language that resembles Rust. It enables one to build circuits that implement complex logic without having to learn the low-level details of ZKP systems. Since it acts like a generalized front end, it is capable of building circuits for a variety of back ends. Currently, Barretenberg \cite{AztecProtocol2023Barretenberg} serves as the default back end, and generates PLONK proofs and Solidity contracts. The Barretenberg back end can also use WASM to create proofs and verify them directly in the browser. Arkworks is also available as an out-of-the-box back end, which can generate Groth16 and Marlin proofs. This generalization is possible because Noir framework compiles the circuit to an intermediate language referred to as ACIR (Abstract Circuit Intermediate Representation), which can then be further compiled to specific R1CS or arithmetic circuit compatible with a specific back end. The framework also provides a Typescript library for direct integration into web applications. There is active development going on, but Noir currently supports a full control flow with the ability to create custom circuits using readable code. This is a great option for developers who would like to avoid the details of ZKPs and build applications using a Rust-like DSL. We recommend this for those who want to build simplistic applications who have little experience with ZKPs.
% \anees{Mention Language syntax features, ACIR Supported OPCODES, maybe roadmap?}
% \nojan{edit}

% \textbf{Bellman}
% \nojan{need to update to PLONK}

% \texttt{Bellman} \cite{} is a zk-SNARK development library built on Rust. This framework is primarily focused on implementing the Groth16 algorithm. \texttt{Bellman} has grown to prominence by being a core framework used to implement Zcash \cite{hopwood2016zcash}, one of the first mainstream implementations of zk-SNARKs. This work only supports one elliptic curve, BLS12-381,  meaning operations are highly optimized for this curve construction. This curve was designed specifically for use in Zcash, but has grown to be a standard underlying ECC for zk-SNARK implementations, due to its effectiveness and security guarantees. \texttt{Bellman} is one of the less accessible development frameworks featured in this work, as it only provides a very low-level API. Although this framework is actively maintained, it effective development \texttt{bellman} requires an in-depth knowledge of zk-SNARKs and pairing-based cryptography using the BLS12-381 ECC \nojan{double check this}. \nojan{Talk about community edition extensions bellman ce and plonk}

% \textbf{Barretenberg}


\textbf{Dusk-PLONK.}
\texttt{Dusk-PLONK} \cite{DuskPlonk2023Rust} is a pure Rust implementation of the PLONK proving system. This implementation supports operation over the BLS12-381 and JubJub elliptic curves. The developers of this framework use Kate commitments \cite{kate2010constant} as their primary polynomial commitment scheme to utilize its homomorphism and maintain constant size commitments. The provided codebase is extremely detailed and well-commented and provides helpful documentation. Similar to other PLONK frameworks, \texttt{Dusk-PLONK} only provides a very low-level API for custom circuit development. To build a custom circuit, developers must translate their computation into an arithmetic or boolean circuit gate format (e.g. add, multiply). This is perfectly digestible for small circuits, as shown in the examples, however becomes an intensely laborious task as the circuit and number of inputs or input dimensions scales up. While the code is well-written and yields excellent results, this framework requires a more sophisticated high-level API that utilizes common software engineering structures to build custom circuits before new developers can start building practical ZKP applications with it. We do note that this is a fantastic implementation of the PLONK proving system for and recommend it for developers that have experience with logic design and ZKPs. 

\textbf{Halo2.}
Built by the same creators of Zcash and the original Halo \cite{bowe2019recursive} framework, the Halo2 framework \cite{Halo22023Book} optimizes upon some of the inefficiencies of its predecessors by utilizing a PLONK-ish scheme as the underlying proving system. The underlying polynomial commitment scheme in this framework is Kate commitments. In its original repository and documentation, building a custom circuit with Halo2 requires a developer to design their computation in the form of a circuit, by implementing gates and utilizing them to build a \textit{chip}. This can be relatively confusing for new developers.
However, Halo2 is a powerful proof system that is utilized widely across the industry, including a prominent verifiable machine learning framework, \texttt{ezkl} \cite{ZkonduitInc2023EZKL}. This prominence has garnered a strong community backing the framework and has resulted in many works that either provide more examples of how the framework can be used \cite{Halo2Club2023}, or expose higher-level APIs for building custom circuits. Overall, while the Halo2 framework only exposes a lower-level API for custom circuit building, the community around it makes it a relatively accessible solution for practical application of PLONKs. We believe this is a good framework for those experienced with applied cryptography and interest in building machine-learning focused applications.

% \textbf{Fractal}
% \cite{fractal} \cite{SciprLab2023Libiop}

% \subsection{Interactive ZKPs}

% \textbf{Orion}
% \nojan{maybe remove}

% \textbf{Virgo++}
% \nojan{The implementation doesn't have ZK commitments even applied}

\subsection{MPC-in-the-head}

\textbf{Ligero (libiop).}
The Ligero \cite{ames2017ligero} protocol is implemented in \texttt{libiop} \cite{SciprLab2023Libiop} framework. This interactive protocol applies the general IKOS \cite{ishai2007zero} transformation that transforms MPC-based interactive proofs into ZKPs, which is typical for MPC-in-the-Head (MPCitH) systems. This means that the key aspect of designing the Ligero is the underlying MPC protocol. While this protocol is interactive, it can be transformed into a zk-SNARK using the Fiat-Shamir transform, just like any other interactive protocol. Additionally, the Ligero protocol only relies on collision resistant hash functions for the underlying cryptography and does not require a trusted setup. As this is implemented using the same backend as the Aurora and Fractal zk-SNARK protocols, all implementation details remain the same as described in section \ref{sec:aurora}.

\textbf{Limbo.}
Similar to Ligero, \texttt{Limbo}'s implementation \cite{KULeuvenCOSIC2023Limbo}  and underlying protocol \cite{limbo} is reliant on the IKOS transformation that MPCitH protocols often rely on. \texttt{Limbo} improves upon Ligero by highlighting the tradeoff between MPCitH parties involved, proof size, and runtime.
% For instance, \texttt{Limbo} allows a user to minimize proof size by having more underlying MPC parties run the protocol in parallel, at the cost of a higher runtime/complexity.
The main work \texttt{Limbo} compares to is Ligero, as they are both transparent MPCitH schemes that only rely on collision resistant hash functions. \texttt{Limbo} claims to work better on small and medium circuits. While the \texttt{Limbo} framework is not as extensively developed, maintained, and documented as some of the other frameworks highlighted in this work, it greatly benefits from its ability to take Bristol Circuit (BC), a common way to describe MPC circuits \cite{bristol}, descriptions as inputs. This allows developers to build custom applications by describing their general computations in BC format. We provide a simple pipeline for developing BCs, alongside examples using readable syntax. We recommend this for users who have experience building optimized BCs and have a relatively deep understanding of MPC.

% \textbf{}

% \subsection{PLONKs}

\subsection{VOLE-Based ZK}


\textbf{Diet Mac'n'Cheese}
\nojan{add swanky stuff}
\texttt{Diet Mac'n'Cheese} \cite{dietmc} is a novel framework that implements the Mac'n'Cheese protocol \cite{baum2021mac}, a Vector Oblivious Linear Evaluation (VOLE)-based zero-knowledge protocol over the $\mathbb{Z}_{2^k}$ ring. Similar to Moz$\mathbb{Z}_{2^k}$arella, this is a crucial step in making ZKPs more practical, as most real-world compute hardware operates on integer rings, and not finite fields. \texttt{Diet Mac'n'Cheese} makes many improvements to the state-of-the-art in VOLE-based ZK protocols by optimizing the underlying sVOLE subprotocol. This optimization yields significant performance improvements over prior VOLE protocols that operate over integer rings. The provided implementation comes in the form of a C++ package that directly implements the proposed scheme and uses the Swanky ecosystem \cite{swanky} for easy integration. This framework is still in its early stages of development and currently lacks extensive documentation and concrete examples, making it harder for new ZKP developers to use it. Alongside this, \texttt{Diet Mac'n'Cheese} currently only supports fixed-point integer operations. It exposes a low-level API that requires a developer to explicitly define all computations as arithmetic and boolean gates that are operated on using the framework's provided functions. However, a recent work has introduced a Python frontend with great documentation that can translate Python code into an intermediate representation that is recognized by the \texttt{Diet Mac'n'Cheese} framework. This frontend, entitled PicoZK \cite{picozk}, contains many examples and is even able to integrate with the popular numpy and pandas packages. PicoZK is a perfect pairing with \texttt{Diet Mac'n'Cheese} and allows for the development of simple applications. We recommend this framework to any developer that aims to build a scalable application that is conducive to a designated-verifier environment, such as federated or split learning. We do note that any floating point operations that are done with this framework must be converted to fixed-point.

\textbf{emp-zk.}
% \nojan{mention sVOLE}
\nojan{shorten} The \texttt{emp-zk} development framework \cite{empzk} is a part of the \texttt{emp-toolkit} \cite{emptool}, a collection of cryptographic front-ends and back-ends that allow for easy development of multi-party computation applications. Alongside ZKPs, \texttt{emp-toolkit} also provides libraries for garbled circuits and oblivious transfer. \texttt{emp-zk} has implementations of three novel interactive ZK systems:
\begin{itemize}
    \item Wolverine \cite{weng2021wolverine}, the first of these systems, presents a constant-round, scalable, and prover-efficient interactive ZK scheme.
    \item Mystique \cite{weng2021mystique}, built on top of Wolverine, focuses on machine learning applications. This work presents efficient conversions for arithmetic and boolean values, fixed-point and floating-point values, and committed and authenticated values. 
    % Alongside this, this work presents an efficient matrix multiplication ZKP by utilizing Freivald's algorithm \cite{freivalds1977probabilistic}.
    \item Quicksilver \cite{yang2021quicksilver}, also built on top of Wolverine, further improves communication costs and scalability.
\end{itemize}

The main primitive these schemes take advantage of is subfield Vector Oblivious Linear Evaluation (sVOLE), which the authors extend and optimize for their ZK scheme. 
% sVOLE is used to efficiently realize the information-theoretic message authentication codes (IT-MAC) commitment scheme.
% , which allows a prover and verifier to interactively create a commitment.
For sake of brevity, we spare the technical detail in this paper and refer to \cite{Weng2023VOLEBasedInteractive} for an excellent explanation. \texttt{emp-zk} provides a very user-friendly interface to all 3 ZK systems, with clear-cut examples. Although documentation is not explicitly provided, \texttt{emp-zk} largely relies on C++ syntax and does not require much knowledge about the underlying work in ZKPs, making it one of the more accessible options. One potential downside of these systems are that they are interactive, meaning all proofs are \textit{designated-verifier}. We highly recommend this framework for users who are building custom machine learning-based custom applications that rely on floating-point operations, or applications that rely on scalability (e.g. database operations).

\textbf{Moz$\mathbb{Z}_{2^k}$arella.}
This work \cite{baum2022moz} presents a new protocol that utilizes an novel vector oblivious linear evaluation (VOLE), a tool from secure two-party computation, extension to perform zero knowledge proof operations efficiently over the integer ring $\mathbb{Z}_{2^k}$. This is very important as most ZK systems are made to operate over finite fields, which is not representative of modern CPUs. The proof system is coined with the term \texttt{Quarksilver}. This protocol outperforms the previous state-of-the-art VOLE-based works that operate over finite fields. The accompanying implementation enables development of ZK applications with the \texttt{Quarksilver} protocol as the underlying scheme. The \texttt{Moz$\mathbb{Z}_{2^k}$arella} repository is not actively maintained, however has 3 sub-libraries for oblivious transfer, garbled and arithmetic circuits, and private set-intersection. Within these sub-libraries there are several examples that explain how to use the \texttt{Moz$\mathbb{Z}_{2^k}$arella} syntax, including examples for \texttt{Quarksilver}. While the examples are somewhat clear, using this library to build custom applications requires a deep knowledge of the underlying proof system, as users must be aware of the parameters that are being set on a per application basis. We only recommend this to users who's applications fully rely on using the specific underlying protocol in this framework.



\subsection{zk-STARKs}

\textbf{Miden VM.}
\texttt{Miden VM} \cite{PolygonMiden2023MidenVM} is a zero-knowledge virtual machine (zkVM) implemented in Rust, in which all programs that are run generate a zk-STARK that can be verified by anyone.
\texttt{Miden VM} is designed as a stack machine, consisting of a stack, memory, chiplets, and a host. The stack, the main user-facing component, is a push-down stack of field elements, which is where inputs and outputs of operations are stored. Increasing the amount of inputs that are initialized on the stack before program execution increases the verifier cost. Whatever is left on the stack after program computation is declared as a public input to the verifier, which also increases cost to the verifier. A prover's private inputs must be pushed to the stack during program computation to be kept private.
The aim of Miden VM is, in their own words, to "make Miden VM an easy compilation for high-level languages such as Rust" \cite{PolygonMiden2023VMOverview}. As these compilers do not yet exist, the only way to build custom circuits is using Miden's assembly language, a very low-level API that interfaces with the Miden stack, and Miden chiplets, which are optimized assembly-based modules that perform common operations, like field arithmetic. Although \texttt{Miden VM} is Turing complete and offers standard control flow, it is often challenging for a developer to translate their desired computation to assembly commands and managing the stack at the same time, especially as the size of computation scales up. While \texttt{Miden VM} is a very valuable tool, we believe that its highest potential will be achieved upon completion of an accompanying compiler from a high-level language to Miden assembly. We recommend that users use this to benchmark certain atomic operations, but to avoid building custom applications with this framework due to the lack of a frontend.

% \textbf{Starky}

\textbf{Zilch.}
The \texttt{Zilch} framework \cite{mouris2021zilch} consists of a Java-like frontend (ZeroJava) that interfaces with a novel zero-knowledge MIPS processor model (zMIPS) \cite{TrustworthyComputing2023Zilch} to enable efficient interactive zk-STARK proof generation for custom computations. The ZeroJava frontend is highly sophisticated and is one of the only frameworks to enable an object-oriented programming approach. All ZeroJava programs are compiled into optimized and verifiable zMIPS instructions. As all of the instructions are verifiable, any program that can be expressed in ZeroJava can be verified using ZKPs. The underlying zMIPS processor can implement and verify any arbitrary computation in zero-knowledge. The zMIPS instructions are implemented using the zk-STARK library \cite{ben2018scalable}. After computation description in ZeroJava and compilation to zMIPS, the constraints for the program are represent in algebraic intermediate representation (AIR) format. The prover and verifier interactively undergo the zk-STARK process until the verifier is convinced that the prover's work is sound. 
\texttt{Zilch} provides an elegant and accessible approach to building custom circuits that utilize zk-STARKs. Although the works lacks dedicated documentation, the examples that are provided show that development of custom applications is almost as simple as implementing the program in Java, with a few ZeroJava design considerations. We recommend this for users with general knowledge of the MIPS instruction set architecture, which should allow them to build optimized programs.

\textbf{RISC Zero.}
\texttt{RISC Zero} is a zkVM \cite{RISCZero2023DeveloperDocs} implemented in Rust with an underlying RISC-V processor and instruction set architecture. The goal of this work is to produce publicly verifiable proofs of all the computations that are done within the framework. As the underlying instructions are derived from RISC-V, virtually any arbitrary computation can be expressed and verified in zero-knowledge.
In this framework, custom circuits can be built using standard Rust syntax, with a few minor modifications to incorporates the framework's API. This program is compiled to a set of RISC-V instructions, which is then executed within a \texttt{RISC Zero} session, which is recorded. A receipt of this session is recorded and used as part of the zk-STARK proof, which can be verified by any verifier to check validity of the computation.
\texttt{RISC Zero} provides a relatively readable high-level Rust API, alongside several examples and very detailed documentation. Due to the maturity of the Rust development and \texttt{RISC Zero} as a whole, developers are able to import a majority of the most used standard Rust crates without trouble, enabling much more streamlined and efficient application development. For instance, developers can use the JPG crate \cite{RustImageCrate2023} to build zero-knowledge applications around images. Alongside this, \texttt{RISC Zero} enables GPU acceleration, so that relevant applications can take advantage of computational speedup. We do note that although GPU acceleration is implemented in the RISC Zero codebase, we were not able to get it actually working due to some inconsistencies within the codebase. However, \texttt{RISC Zero} has an active community around it, including active development by the creators, and a very well-documented and accessible code, making it a great candidate for new developers of custom ZKP applications. The primary drawback for this framework is that, due the nature of zkVMs and the simulation of a RISC-V processor and ISA, this framework has relatively significant initialization and operation costs.
% \begin{table}[!t]
% \centering
% \scalebox{0.68}{
%     \begin{tabular}{ll cccc}
%       \toprule
%       & \multicolumn{4}{c}{\textbf{Intellipro Dataset}}\\
%       & \multicolumn{2}{c}{Rank Resume} & \multicolumn{2}{c}{Rank Job} \\
%       \cmidrule(lr){2-3} \cmidrule(lr){4-5} 
%       \textbf{Method}
%       &  Recall@100 & nDCG@100 & Recall@10 & nDCG@10 \\
%       \midrule
%       \confitold{}
%       & 71.28 &34.79 &76.50 &52.57 
%       \\
%       \cmidrule{2-5}
%       \confitsimple{}
%     & 82.53 &48.17
%        & 85.58 &64.91
     
%        \\
%        +\RunnerUpMiningShort{}
%     &85.43 &50.99 &91.38 &71.34 
%       \\
%       +\HyReShort
%         &- & -
%        &-&-\\
       
%       \bottomrule

%     \end{tabular}
%   }
% \caption{Ablation studies using Jina-v2-base as the encoder. ``\confitsimple{}'' refers using a simplified encoder architecture. \framework{} trains \confitsimple{} with \RunnerUpMiningShort{} and \HyReShort{}.}
% \label{tbl:ablation}
% \end{table}
\begin{table*}[!t]
\centering
\scalebox{0.75}{
    \begin{tabular}{l cccc cccc}
      \toprule
      & \multicolumn{4}{c}{\textbf{Recruiting Dataset}}
      & \multicolumn{4}{c}{\textbf{AliYun Dataset}}\\
      & \multicolumn{2}{c}{Rank Resume} & \multicolumn{2}{c}{Rank Job} 
      & \multicolumn{2}{c}{Rank Resume} & \multicolumn{2}{c}{Rank Job}\\
      \cmidrule(lr){2-3} \cmidrule(lr){4-5} 
      \cmidrule(lr){6-7} \cmidrule(lr){8-9} 
      \textbf{Method}
      & Recall@100 & nDCG@100 & Recall@10 & nDCG@10
      & Recall@100 & nDCG@100 & Recall@10 & nDCG@10\\
      \midrule
      \confitold{}
      & 71.28 & 34.79 & 76.50 & 52.57 
      & 87.81 & 65.06 & 72.39 & 56.12
      \\
      \cmidrule{2-9}
      \confitsimple{}
      & 82.53 & 48.17 & 85.58 & 64.91
      & 94.90&78.40 & 78.70& 65.45
       \\
      +\HyReShort{}
       &85.28 & 49.50
       &90.25 & 70.22
       & 96.62&81.99 & \textbf{81.16}& 67.63
       \\
      +\RunnerUpMiningShort{}
       % & 85.14& 49.82
       % &90.75&72.51
       & \textbf{86.13}&\textbf{51.90} & \textbf{94.25}&\textbf{73.32}
       & \textbf{97.07}&\textbf{83.11} & 80.49& \textbf{68.02}
       \\
   %     +\RunnerUpMiningShort{}
   %    & 85.43 & 50.99 & 91.38 & 71.34 
   %    & 96.24 & 82.95 & 80.12 & 66.96
   %    \\
   %    +\HyReShort{} old
   %     &85.28 & 49.50
   %     &90.25 & 70.22
   %     & 96.62&81.99 & 81.16& 67.63
   %     \\
   % +\HyReShort{} 
   %     % & 85.14& 49.82
   %     % &90.75&72.51
   %     & 86.83&51.77 &92.00 &72.04
   %     & 97.07&83.11 & 80.49& 68.02
   %     \\
      \bottomrule

    \end{tabular}
  }
\caption{\framework{} ablation studies. ``\confitsimple{}'' refers using a simplified encoder architecture. \framework{} trains \confitsimple{} with \RunnerUpMiningShort{} and \HyReShort{}. We use Jina-v2-base as the encoder due to its better performance.
}
\label{tbl:ablation}
\end{table*}

\section{Results}
\label{sec:results}

In this section, we present detailed results demonstrating \emph{CellFlow}'s state-of-the-art performance in cellular morphology prediction under perturbations, outperforming existing methods across multiple datasets and evaluation metrics.

\subsection{Datasets}

Our experiments were conducted using three cell imaging perturbation datasets: BBBC021 (chemical perturbation)~\cite{caie2010high}, RxRx1 (genetic perturbation)~\cite{sypetkowski2023rxrx1}, and the JUMP dataset (combined perturbation)~\cite{chandrasekaran2023jump}. We followed the preprocessing protocol from IMPA~\cite{palma2023predicting}, which involves correcting illumination, cropping images centered on nuclei to a resolution of 96×96, and filtering out low-quality images. The resulting datasets include 98K, 171K, and 424K images with 3, 5, and 6 channels, respectively, from 26, 1,042, and 747 perturbation types. Examples of these images are provided in Figure~\ref{fig:comparison}. Details of datasets are provided in \S\ref{sec:data}.

\subsection{Experimental Setup}

\textbf{Evaluation metrics.} We evaluate methods using two types of metrics: (1) FID and KID, which measure image distribution similarity via Fréchet and kernel-based distances, computed on 5K generated images for BBBC021 and 100 randomly selected perturbation classes for RxRx1 and JUMP; we report both overall scores across all samples and conditional scores per perturbation class. (2) Mode of Action (MoA) classification accuracy, which assesses biological fidelity by using a trained classifier to predict a drug’s effect from perturbed images and comparing it to its known MoA from the literature.

\textbf{Baselines.} We compare our approach against two baselines, PhenDiff~\cite{bourou2024phendiff} and IMPA~\cite{palma2023predicting}, the only two baselines that incorporate control images into their model design --- a crucial setup for distinguishing true perturbation effects from artifacts such as batch effects. PhenDiff uses diffusion models to first map control images to noise and then transform the noise into target images. In contrast, IMPA employs GANs with an AdaIN layer to transfer the style of control images to target images, specifically designed for paired image-to-image mappings. Our method uses flow matching, which is tailored for distribution-to-distribution mapping, providing a more suitable solution for our problem. We reproduce these baselines with official codes.

\textbf{Training details.} \emph{CellFlow} employs a UNet-based velocity field with a four-stage design. Perturbations are encoded following IMPA~\cite{palma2023predicting}. Training is conducted for 100 epochs on 4 A100 GPUs. Details are in \S\ref{sec:experimental}.


\begin{table*}[ht!]
\centering
\footnotesize
\renewcommand{\arraystretch}{1.2}
\begin{tabular}{p{2.15cm} p{14.05cm}}
\toprule
 & \includegraphics[height=2.6cm]{figure/qualitative_example.pdf}\\
\midrule
Original \phantom{00000} Description & Bromazepam binds to the GABA receptor GABA-A, causing a conformational change and \textbf{increasing inhibitory effects of GABA}. Other neurotransmitters are not influenced. One of the \textbf{benzodiazepines} that is \textbf{used in the treatment of anxiety disorders.} \\
\midrule
User & Explain the features of the given molecule that are most distinct from those of other molecules. \\
\midrule
Mol-LLaMA (Ours) & The given molecule is a member of the \textcolor{tabblue}{benzodiazepine class}, which is characterized by a fusion of a benzene ring and a seven-membered diazepine ring. This structure is crucial for its biological activity, particularly its role as \textcolor{tabblue}{a sedative and hypnotic agent}. ...$<$omitted$>$...\\

& The given molecule is also characterized by a ketone group within the diazepine ring, which is a common feature in benzodiazepines. \textcolor{tabblue}{This functional group is important for the interaction with the GABA-A receptor, enhancing the inhibitory effects of GABA}. ...$<$omitted$>$...\\

\midrule
GPT-4o & 1. **Quinazoline Core**: The molecule contains a \textcolor{tabred}{quinazoline core}, ...$<$omitted$>$... \\
 & Overall, the combination of a \textcolor{tabred}{quinazoline} structure with unique substitution patterns, including bromine halogenation and the presence of a pyridinyl group, ...$<$omitted$>$... . \\
\midrule
LLaMo & The molecule has a \textcolor{tabblue}{benzodiazepine} structure with a bromo substituent at the 7-position and \textcolor{tabred}{a methyl group at the 1-position}. It is distinct from other molecules in that it contains a bromine atom, a nitrogen atom, and a methyl group, which are not present in the other molecules. \\
\midrule
3D-MoLM & ...$<$omitted$>$... It belongs to the class of compounds called \textcolor{tabred}{quinazolines}, which are characterized by a bicyclic structure consisting of a benzene ring fused to a pyrimidine ring. ...$<$omitted$>$...\\
& The molecule's structure suggests potential applications in medicinal chemistry, as quinazolines have been found to possess various biological activities, including \textcolor{tabred}{antitumor, antimicrobial, and anti-inflammatory properties.} \\
\midrule
Mol-Instructions & The molecule is a \textcolor{tabred}{quinoxaline derivative}. \\
\bottomrule
\end{tabular}
\vspace{-0.1in}
\caption{\small Case study to compare molecular understanding and reasoning ability. Mol-LLaMA accurately understands the molecular features, answering a correct molecular taxonomy and providing its distinct properties that are relevant to the given molecule.}
\label{tab:qualitative}
\vspace{-0.1in}
\end{table*}

\subsection{Main Results}

\textbf{\emph{CellFlow} generates highly realistic cell images.}  
\emph{CellFlow} outperforms existing methods in capturing cellular morphology across all datasets (Table~\ref{tab:results}a), achieving overall FID scores of 18.7, 33.0, and 9.0 on BBBC021, RxRx1, and JUMP, respectively --- improving FID by 21\%–45\% compared to previous methods. These gains in both FID and KID metrics confirm that \emph{CellFlow} produces significantly more realistic cell images than prior approaches.

\textbf{\emph{CellFlow} accurately captures perturbation-specific morphological changes.}  
As shown in Table~\ref{tab:results}a, \emph{CellFlow} achieves conditional FID scores of 56.8 (a 26\% improvement), 163.5, and 84.4 (a 16\% improvement) on BBBC021, RxRx1, and JUMP, respectively. These scores are computed by measuring the distribution distance for each specific perturbation and averaging across all perturbations.   
Table~\ref{tab:results}b further highlights \emph{CellFlow}’s performance on six representative chemical and three genetic perturbations. For chemical perturbations, \emph{CellFlow} reduces FID scores by 14–55\% compared to prior methods.
The smaller improvement (5–12\% improvements) on RxRx1 is likely due to the limited number of images per perturbation type.

\textbf{\emph{CellFlow} preserves biological fidelity across perturbation conditions.} 
Table~\ref{tab:ablation}a presents mode of action (MoA) classification accuracy on the BBBC021 dataset using generated cell images. MoA describes how a drug affects cellular function and can be inferred from morphology. To assess this, we train an image classifier on real perturbed images and test it on generated ones. \emph{CellFlow} achieves 71.1\% MoA accuracy, closely matching real images (72.4\%) and significantly surpassing other methods (best: 63.7\%), demonstrating its ability to maintain biological fidelity across perturbations. Qualitative comparisons in Figure~\ref{fig:comparison} further highlight \emph{CellFlow}’s accuracy in capturing key biological effects. For example, demecolcine produces smaller, fragmented nuclei, which other methods fail to reproduce accurately.

\textbf{\emph{CellFlow} generalizes to out-of-distribution (OOD) perturbations.}  
On BBBC021, \emph{CellFlow} demonstrates strong generalization to novel chemical perturbations never seen during training (Table~\ref{tab:ablation}b). It achieves 6\% and 28\% improvements in overall and conditional FID over the best baseline. This OOD generalization is critical for biological research, enabling the exploration of previously untested interventions and the design of new drugs.

\textbf{Ablations highlight the importance of each component in \emph{CellFlow}.}  
Table~\ref{tab:ablation}c shows that removing conditional information, classifier-free guidance, or noise augmentation significantly degrades performance, leading to higher FID scores. These underscore the critical role of each component in enabling \emph{CellFlow}’s state-of-the-art performance.  

\begin{figure*}[!tb]
    \centering
     \includegraphics[width=\linewidth]{imgs/interpolation.pdf}
     \vspace{-2em}
    \caption{
    \textbf{\emph{CellFlow} enables new capabilities.} 
\textit{(a.1) Batch effect calibration.}  
\emph{CellFlow} initializes with control images, enabling batch-specific predictions. Comparing predictions from different batches highlights actual perturbation effects (smaller cell size) while filtering out spurious batch effects (cell density variations).  
\textit{(a.2) Interpolation trajectory.}  
\emph{CellFlow}'s learned velocity field supports interpolation between cell states, which might provide insights into the dynamic cell trajectory. 
\textit{(b) Diffusion model comparison.}  
Unlike flow matching, diffusion models that start from noise cannot calibrate batch effects or support interpolation.  
\textit{(c) Reverse trajectory.}  
\emph{CellFlow}'s reversible velocity field can predict prior cell states from perturbed images, offering potential applications such as restoring damaged cells.
    }
    \label{fig:interpolation}
    \vspace{-1em}
\end{figure*}

\subsection{New Capabilities}

\textbf{\emph{CellFlow} addresses batch effects and reveals true perturbation effects.}  
\emph{CellFlow}’s distribution-to-distribution approach effectively addresses batch effects, a significant challenge in biological experimental data collection. As shown in Figure~\ref{fig:interpolation}a, when conditioned on two distinct control images with varying cell densities from different batches, \emph{CellFlow} consistently generates the expected perturbation effect (cell shrinkage due to mevinolin) while recapitulating batch-specific artifacts, revealing the true perturbation effect. Table~\ref{tab:ablation}d further quantifies the importance of conditioning on the same batch. By comparing generated images conditioned on control images from the same or different batches against the target perturbation images, we find that same-batch conditioning reduces overall and conditional FID by 21\%. This highlights the importance of modeling control images to more accurately capture true perturbation effects—an aspect often overlooked by prior approaches, such as diffusion models that initialize from noise (Figure~\ref{fig:interpolation}b).

\textbf{\emph{CellFlow} has the potential to model cellular morphological change trajectories.}
Cell trajectories could offer valuable information about perturbation mechanisms, but capturing them with current imaging technologies remains challenging due to their destructive nature. Since \emph{CellFlow} continuously transforms the source distribution into the target distribution, it can generate smooth interpolation paths between initial and final predicted cell states, producing video-like sequences of cellular transformation based on given source images (Figure~\ref{fig:interpolation}a). This suggests a possible approach for simulating morphological trajectories during perturbation response, which diffusion methods cannot achieve (Figure~\ref{fig:interpolation}b). Additionally, the reversible distribution transformation learned through flow matching enables \emph{CellFlow} to model backward cell state reversion (Figure~\ref{fig:interpolation}c), which could be useful for studying recovery dynamics and predicting potential treatment outcomes.

%% New Disucssion 
Our study reveals how heavy users integrate LLMs into their daily tasks through distinct patterns. Rather than simple tool usage, participants demonstrated sophisticated cognitive offloading strategies that transformed their decision-making processes. In our study, we observed participants delegating social and interpersonal reasoning to LLMs, suggesting ways users might leverage AI collaboration to support their social cognition processes.

Participants' mental models of LLMs directly influenced their cognitive strategies---those viewing LLMs as rational entities engaged in cognitive complementarity by leveraging LLM capabilities where they perceived personal limitations, while those viewing LLMs as average decision-makers used cognitive benchmarking, establishing baseline standards while reserving higher-order tasks for themselves.
% While delegating a broad range of decisions raised potential concerns about over-reliance and diminished critical thinking, our findings also highlight a nuanced form of human-AI collaboration where users and LLMs develop complementary relationships. Participants showed diverse usage strategies, treating LLMs as an emerging problem-solving tool and developing sophisticated prompting techniques. Most notably, participants frequently sought LLM guidance on social appropriateness and interpersonal situations. Although some users expressed concerns about potential skill degradation and a sense of unease, LLM consultations often led to a more thorough consideration of social factors and an enhanced understanding of different perspectives.

This raises questions for future research on redefining how we conceptualize and measure over-reliance on LLMs. Current metrics typically assess over-reliance through simplified quantitative measures in controlled settings, primarily focusing on users' acceptance rates of LLM outputs ~\cite{bo2024rely, kim2024rely}. However, our findings reveal more complex patterns of engagement. Participants did not blindly adopt LLM outputs, even in cases where they eventually accepted them. Instead, participants demonstrated thoughtful delegation strategies, using LLMs to validate existing decisions, automate routine tasks, or navigate unfamiliar situations. The critical concern was not users' acceptance of LLM outputs, but rather instances where users adopted LLM reasoning without exploring alternative perspectives. Future research should expand the definition of over-reliance beyond simple acceptance rates to examine how users critically engage with alternative lines of reasoning.

Another key direction for future research involves capturing diverse user contexts. Our participants valued the ability of LLMs to extract necessary contextual information when not initially provided. They appreciated that they could receive meaningful responses without extensively explaining background information, even for context-heavy topics like relationship advice. Future research should explore ways to incorporate multi-modal inputs beyond text-based interactions, allowing users to convey context through various channels. Additionally, LLMs' ability to elicit implicit user intentions without explicit prompting is crucial, as demonstrated by recent advances in reasoning-focused LLM architectures that can proactively identify and address underlying user needs.

The development of active usage patterns with LLMs appeared more prominent among younger users who had less experience managing tasks without these systems. Participants with extensive pre-LLM experience maintained clearer boundaries and showed greater awareness of system limitations. In contrast, users with less experience with LLMs demonstrated fewer reservations, viewing LLM interaction itself as a skill and actively developing their prompting strategies. Conducting design studies focused on younger generations, to better understand and support these emerging interaction patterns represents a crucial direction for future research.
% % \subsec{ZKP Applications}

% \subsection{Verifiable Machine Learning}
% \nojan{zkcnn, mystique, vcnn, zen, ezkl}

% \nojan{computationaly weak verifier sends computation out to strong prover (look at zilch abstract)}

% \subsection{zk-Rollups}
% \nojan{plonky2, zkevm}

% \subsection{Robust Federated Learning}
% \nojan{zprobe, eiffel, brea, rofl}

% \subsection{FHE Integrity}
% \nojan{rinocchio}
% \input{}
\section{Discussion}\label{sec:discussion}



\subsection{From Interactive Prompting to Interactive Multi-modal Prompting}
The rapid advancements of large pre-trained generative models including large language models and text-to-image generation models, have inspired many HCI researchers to develop interactive tools to support users in crafting appropriate prompts.
% Studies on this topic in last two years' HCI conferences are predominantly focused on helping users refine single-modality textual prompts.
Many previous studies are focused on helping users refine single-modality textual prompts.
However, for many real-world applications concerning data beyond text modality, such as multi-modal AI and embodied intelligence, information from other modalities is essential in constructing sophisticated multi-modal prompts that fully convey users' instruction.
This demand inspires some researchers to develop multimodal prompting interactions to facilitate generation tasks ranging from visual modality image generation~\cite{wang2024promptcharm, promptpaint} to textual modality story generation~\cite{chung2022tale}.
% Some previous studies contributed relevant findings on this topic. 
Specifically, for the image generation task, recent studies have contributed some relevant findings on multi-modal prompting.
For example, PromptCharm~\cite{wang2024promptcharm} discovers the importance of multimodal feedback in refining initial text-based prompting in diffusion models.
However, the multi-modal interactions in PromptCharm are mainly focused on the feedback empowered the inpainting function, instead of supporting initial multimodal sketch-prompt control. 

\begin{figure*}[t]
    \centering
    \includegraphics[width=0.9\textwidth]{src/img/novice_expert.pdf}
    \vspace{-2mm}
    \caption{The comparison between novice and expert participants in painting reveals that experts produce more accurate and fine-grained sketches, resulting in closer alignment with reference images in close-ended tasks. Conversely, in open-ended tasks, expert fine-grained strokes fail to generate precise results due to \tool's lack of control at the thin stroke level.}
    \Description{The comparison between novice and expert participants in painting reveals that experts produce more accurate and fine-grained sketches, resulting in closer alignment with reference images in close-ended tasks. Novice users create rougher sketches with less accuracy in shape. Conversely, in open-ended tasks, expert fine-grained strokes fail to generate precise results due to \tool's lack of control at the thin stroke level, while novice users' broader strokes yield results more aligned with their sketches.}
    \label{fig:novice_expert}
    % \vspace{-3mm}
\end{figure*}


% In particular, in the initial control input, users are unable to explicitly specify multi-modal generation intents.
In another example, PromptPaint~\cite{promptpaint} stresses the importance of paint-medium-like interactions and introduces Prompt stencil functions that allow users to perform fine-grained controls with localized image generation. 
However, insufficient spatial control (\eg, PromptPaint only allows for single-object prompt stencil at a time) and unstable models can still leave some users feeling the uncertainty of AI and a varying degree of ownership of the generated artwork~\cite{promptpaint}.
% As a result, the gap between intuitive multi-modal or paint-medium-like control and the current prompting interface still exists, which requires further research on multi-modal prompting interactions.
From this perspective, our work seeks to further enhance multi-object spatial-semantic prompting control by users' natural sketching.
However, there are still some challenges to be resolved, such as consistent multi-object generation in multiple rounds to increase stability and improved understanding of user sketches.   


% \new{
% From this perspective, our work is a step forward in this direction by allowing multi-object spatial-semantic prompting control by users' natural sketching, which considers the interplay between multiple sketch regions.
% % To further advance the multi-modal prompting experience, there are some aspects we identify to be important.
% % One of the important aspects is enhancing the consistency and stability of multiple rounds of generation to reduce the uncertainty and loss of control on users' part.
% % For this purpose, we need to develop techniques to incorporate consistent generation~\cite{tewel2024training} into multi-modal prompting framework.}
% % Another important aspect is improving generative models' understanding of the implicit user intents \new{implied by the paint-medium-like or sketch-based input (\eg, sketch of two people with their hands slightly overlapping indicates holding hand without needing explicit prompt).
% % This can facilitate more natural control and alleviate users' effort in tuning the textual prompt.
% % In addition, it can increase users' sense of ownership as the generated results can be more aligned with their sketching intents.
% }
% For example, when users draw sketches of two people with their hands slightly overlapping, current region-based models cannot automatically infer users' implicit intention that the two people are holding hands.
% Instead, they still require users to explicitly specify in the prompt such relationship.
% \tool addresses this through sketch-aware prompt recommendation to fill in the necessary semantic information, alleviating users' workload.
% However, some users want the generative AI in the future to be able to directly infer this natural implicit intentions from the sketches without additional prompting since prompt recommendation can still be unstable sometimes.


% \new{
% Besides visual generation, 
% }
% For example, one of the important aspect is referring~\cite{he2024multi}, linking specific text semantics with specific spatial object, which is partly what we do in our sketch-aware prompt recommendation.
% Analogously, in natural communication between humans, text or audio alone often cannot suffice in expressing the speakers' intentions, and speakers often need to refer to an existing spatial object or draw out an illustration of her ideas for better explanation.
% Philosophically, we HCI researchers are mostly concerned about the human-end experience in human-AI communications.
% However, studies on prompting is unique in that we should not just care about the human-end interaction, but also make sure that AI can really get what the human means and produce intention-aligned output.
% Such consideration can drastically impact the design of prompting interactions in human-AI collaboration applications.
% On this note, although studies on multi-modal interactions is a well-established topic in HCI community, it remains a challenging problem what kind of multi-modal information is really effective in helping humans convey their ideas to current and next generation large AI models.




\subsection{Novice Performance vs. Expert Performance}\label{sec:nVe}
In this section we discuss the performance difference between novice and expert regarding experience in painting and prompting.
First, regarding painting skills, some participants with experience (4/12) preferred to draw accurate and fine-grained shapes at the beginning. 
All novice users (5/12) draw rough and less accurate shapes, while some participants with basic painting skills (3/12) also favored sketching rough areas of objects, as exemplified in Figure~\ref{fig:novice_expert}.
The experienced participants using fine-grained strokes (4/12, none of whom were experienced in prompting) achieved higher IoU scores (0.557) in the close-ended task (0.535) when using \tool. 
This is because their sketches were closer in shape and location to the reference, making the single object decomposition result more accurate.
Also, experienced participants are better at arranging spatial location and size of objects than novice participants.
However, some experienced participants (3/12) have mentioned that the fine-grained stroke sometimes makes them frustrated.
As P1's comment for his result in open-ended task: "\emph{It seems it cannot understand thin strokes; even if the shape is accurate, it can only generate content roughly around the area, especially when there is overlapping.}" 
This suggests that while \tool\ provides rough control to produce reasonably fine results from less accurate sketches for novice users, it may disappoint experienced users seeking more precise control through finer strokes. 
As shown in the last column in Figure~\ref{fig:novice_expert}, the dragon hovering in the sky was wrongly turned into a standing large dragon by \tool.

Second, regarding prompting skills, 3 out of 12 participants had one or more years of experience in T2I prompting. These participants used more modifiers than others during both T2I and R2I tasks.
Their performance in the T2I (0.335) and R2I (0.469) tasks showed higher scores than the average T2I (0.314) and R2I (0.418), but there was no performance improvement with \tool\ between their results (0.508) and the overall average score (0.528). 
This indicates that \tool\ can assist novice users in prompting, enabling them to produce satisfactory images similar to those created by users with prompting expertise.



\subsection{Applicability of \tool}
The feedback from user study highlighted several potential applications for our system. 
Three participants (P2, P6, P8) mentioned its possible use in commercial advertising design, emphasizing the importance of controllability for such work. 
They noted that the system's flexibility allows designers to quickly experiment with different settings.
Some participants (N = 3) also mentioned its potential for digital asset creation, particularly for game asset design. 
P7, a game mod developer, found the system highly useful for mod development. 
He explained: "\emph{Mods often require a series of images with a consistent theme and specific spatial requirements. 
For example, in a sacrifice scene, how the objects are arranged is closely tied to the mod's background. It would be difficult for a developer without professional skills, but with this system, it is possible to quickly construct such images}."
A few participants expressed similar thoughts regarding its use in scene construction, such as in film production. 
An interesting suggestion came from participant P4, who proposed its application in crime scene description. 
She pointed out that witnesses are often not skilled artists, and typically describe crime scenes verbally while someone else illustrates their account. 
With this system, witnesses could more easily express what they saw themselves, potentially producing depictions closer to the real events. "\emph{Details like object locations and distances from buildings can be easily conveyed using the system}," she added.

% \subsection{Model Understanding of Users' Implicit Intents}
% In region-sketch-based control of generative models, a significant gap between interaction design and actual implementation is the model's failure in understanding users' naturally expressed intentions.
% For example, when users draw sketches of two people with their hands slightly overlapping, current region-based models cannot automatically infer users' implicit intention that the two people are holding hands.
% Instead, they still require users to explicitly specify in the prompt such relationship.
% \tool addresses this through sketch-aware prompt recommendation to fill in the necessary semantic information, alleviating users' workload.
% However, some users want the generative AI in the future to be able to directly infer this natural implicit intentions from the sketches without additional prompting since prompt recommendation can still be unstable sometimes.
% This problem reflects a more general dilemma, which ubiquitously exists in all forms of conditioned control for generative models such as canny or scribble control.
% This is because all the control models are trained on pairs of explicit control signal and target image, which is lacking further interpretation or customization of the user intentions behind the seemingly straightforward input.
% For another example, the generative models cannot understand what abstraction level the user has in mind for her personal scribbles.
% Such problems leave more challenges to be addressed by future human-AI co-creation research.
% One possible direction is fine-tuning the conditioned models on individual user's conditioned control data to provide more customized interpretation. 

% \subsection{Balance between recommendation and autonomy}
% AIGC tools are a typical example of 
\subsection{Progressive Sketching}
Currently \tool is mainly aimed at novice users who are only capable of creating very rough sketches by themselves.
However, more accomplished painters or even professional artists typically have a coarse-to-fine creative process. 
Such a process is most evident in painting styles like traditional oil painting or digital impasto painting, where artists first quickly lay down large color patches to outline the most primitive proportion and structure of visual elements.
After that, the artists will progressively add layers of finer color strokes to the canvas to gradually refine the painting to an exquisite piece of artwork.
One participant in our user study (P1) , as a professional painter, has mentioned a similar point "\emph{
I think it is useful for laying out the big picture, give some inspirations for the initial drawing stage}."
Therefore, rough sketch also plays a part in the professional artists' creation process, yet it is more challenging to integrate AI into this more complex coarse-to-fine procedure.
Particularly, artists would like to preserve some of their finer strokes in later progression, not just the shape of the initial sketch.
In addition, instead of requiring the tool to generate a finished piece of artwork, some artists may prefer a model that can generate another more accurate sketch based on the initial one, and leave the final coloring and refining to the artists themselves.
To accommodate these diverse progressive sketching requirements, a more advanced sketch-based AI-assisted creation tool should be developed that can seamlessly enable artist intervention at any stage of the sketch and maximally preserve their creative intents to the finest level. 

\subsection{Ethical Issues}
Intellectual property and unethical misuse are two potential ethical concerns of AI-assisted creative tools, particularly those targeting novice users.
In terms of intellectual property, \tool hands over to novice users more control, giving them a higher sense of ownership of the creation.
However, the question still remains: how much contribution from the user's part constitutes full authorship of the artwork?
As \tool still relies on backbone generative models which may be trained on uncopyrighted data largely responsible for turning the sketch into finished artwork, we should design some mechanisms to circumvent this risk.
For example, we can allow artists to upload backbone models trained on their own artworks to integrate with our sketch control.
Regarding unethical misuse, \tool makes fine-grained spatial control more accessible to novice users, who may maliciously generate inappropriate content such as more realistic deepfake with specific postures they want or other explicit content.
To address this issue, we plan to incorporate a more sophisticated filtering mechanism that can detect and screen unethical content with more complex spatial-semantic conditions. 
% In the future, we plan to enable artists to upload their own style model

% \subsection{From interactive prompting to interactive spatial prompting}


\subsection{Limitations and Future work}

    \textbf{User Study Design}. Our open-ended task assesses the usability of \tool's system features in general use cases. To further examine aspects such as creativity and controllability across different methods, the open-ended task could be improved by incorporating baselines to provide more insightful comparative analysis. 
    Besides, in close-ended tasks, while the fixing order of tool usage prevents prior knowledge leakage, it might introduce learning effects. In our study, we include practice sessions for the three systems before the formal task to mitigate these effects. In the future, utilizing parallel tests (\textit{e.g.} different content with the same difficulty) or adding a control group could further reduce the learning effects.

    \textbf{Failure Cases}. There are certain failure cases with \tool that can limit its usability. 
    Firstly, when there are three or more objects with similar semantics, objects may still be missing despite prompt recommendations. 
    Secondly, if an object's stroke is thin, \tool may incorrectly interpret it as a full area, as demonstrated in the expert results of the open-ended task in Figure~\ref{fig:novice_expert}. 
    Finally, sometimes inclusion relationships (\textit{e.g.} inside) between objects cannot be generated correctly, partially due to biases in the base model that lack training samples with such relationship. 

    \textbf{More support for single object adjustment}.
    Participants (N=4) suggested that additional control features should be introduced, beyond just adjusting size and location. They noted that when objects overlap, they cannot freely control which object appears on top or which should be covered, and overlapping areas are currently not allowed.
    They proposed adding features such as layer control and depth control within the single-object mask manipulation. Currently, the system assigns layers based on color order, but future versions should allow users to adjust the layer of each object freely, while considering weighted prompts for overlapping areas.

    \textbf{More customized generation ability}.
    Our current system is built around a single model $ColorfulXL-Lightning$, which limits its ability to fully support the diverse creative needs of users. Feedback from participants has indicated a strong desire for more flexibility in style and personalization, such as integrating fine-tuned models that cater to specific artistic styles or individual preferences. 
    This limitation restricts the ability to adapt to varied creative intents across different users and contexts.
    In future iterations, we plan to address this by embedding a model selection feature, allowing users to choose from a variety of pre-trained or custom fine-tuned models that better align with their stylistic preferences. 
    
    \textbf{Integrate other model functions}.
    Our current system is compatible with many existing tools, such as Promptist~\cite{hao2024optimizing} and Magic Prompt, allowing users to iteratively generate prompts for single objects. However, the integration of these functions is somewhat limited in scope, and users may benefit from a broader range of interactive options, especially for more complex generation tasks. Additionally, for multimodal large models, users can currently explore using affordable or open-source models like Qwen2-VL~\cite{qwen} and InternVL2-Llama3~\cite{llama}, which have demonstrated solid inference performance in our tests. While GPT-4o remains a leading choice, alternative models also offer competitive results.
    Moving forward, we aim to integrate more multimodal large models into the system, giving users the flexibility to choose the models that best fit their needs. 
    


\section{Conclusion}\label{sec:conclusion}
In this paper, we present \tool, an interactive system designed to help novice users create high-quality, fine-grained images that align with their intentions based on rough sketches. 
The system first refines the user's initial prompt into a complete and coherent one that matches the rough sketch, ensuring the generated results are both stable, coherent and high quality.
To further support users in achieving fine-grained alignment between the generated image and their creative intent without requiring professional skills, we introduce a decompose-and-recompose strategy. 
This allows users to select desired, refined object shapes for individual decomposed objects and then recombine them, providing flexible mask manipulation for precise spatial control.
The framework operates through a coarse-to-fine process, enabling iterative and fine-grained control that is not possible with traditional end-to-end generation methods. 
Our user study demonstrates that \tool offers novice users enhanced flexibility in control and fine-grained alignment between their intentions and the generated images.

\section*{Acknowledgments}
This work was supported by DARPA Proofs under grant number HR0011-23-1-0006.

\section*{Conflict of Interests}
The authors declare that there is no conflict of interests regarding the publication of this paper.

{\footnotesize
\bibliographystyle{abbrv}
\bibliography{refs}
}

\begin{IEEEbiography}[{\includegraphics[width=1in,height=1.25in,clip,keepaspectratio]{author_pics/Nojan.png}}]{Nojan Sheybani} is a Ph.D. candidate in the department of Electrical and Computer Engineering (ECE) at the University of California San Diego (UCSD). His research is focused on applied cryptography, hardware/software co-design, and zero-knowledge proofs. In particular, a common theme in his work is the application and optimization of privacy-preserving techniques to build practical and secure real-world systems.
\end{IEEEbiography}


\begin{IEEEbiography}[{\includegraphics[width=1in,height=1.25in,clip,keepaspectratio]{author_pics/anees.png}}]{Anees Ahmed} received his M.S. degree in Computer Science from Arizona State University. His research interests include privacy-preserving computation, zero-knowledge proofs, hardware acceleration, and computer architecture. Prior to his master's degree, he was a software engineer for two years. 
\end{IEEEbiography}


\begin{IEEEbiography}[{\includegraphics[width=1in,height=1.25in,clip,keepaspectratio]{author_pics/mkinsy.png}}]{Michel Kinsy} is an associate professor in the School of Computing and Augmented Intelligence and the director of the Secure, Trusted, and Assured Microelectronics Center. He focuses his research on microelectronics security, secure processors and systems design, hardware security, and efficient hardware design and implementation of post-quantum cryptography systems. Kinsy is an MIT Presidential Fellow and a CRA-WP Inaugural Skip Ellis Career Award recipient.
\end{IEEEbiography}

\begin{IEEEbiography}[{\includegraphics[width=1in,height=1.25in,clip,keepaspectratio]{author_pics/Farinaz.png}}]{Farinaz Koushanfar}
is the Siavouche Nemati-Nasser Endowed Professor of Electrical and Computer Engineering (ECE) at the University of California San Diego (UCSD), where she is the founding co-director of the UCSD Center for Machine-Intelligence, Computing \& Security (MICS). She is also a research scientist at Chainlink Labs. Her research addresses several aspects of secure and efficient computing, with a focus on robust machine learning under resource constraints, AI-based optimization, hardware and system security, intellectual property (IP) protection, as well as privacy-preserving computing. Dr. Koushanfar has received a number of awards and honors including the Presidential Early Career Award for Scientists and Engineers (PECASE) from President Obama, the ACM SIGDA Outstanding New Faculty Award, Cisco IoT Security Grand Challenge Award, MIT Technology Review TR-35, Qualcomm Innovation Awards, Intel Collaborative Awards, Young Faculty/CAREER Awards from NSF, DARPA, ONR and ARO, as well as several best paper awards. Dr. Koushanfar is a fellow of ACM, IEEE, National Academy of Inventors, and the Kavli Frontiers of the National Academy of Sciences.
\end{IEEEbiography}
\vfill

\clearpage
\newpage

\appendices


\newpage
\appendix
\section{Applicability of SparseTransX for dense graphs} 
\label{A:density}
Even for fully dense graphs, our KGE computations remain highly sparse. This is because our SpMM leverages the incidence matrix for triplets, rather than the graph's adjacency matrix. In the paper, the sparse matrix $A \in \{-1,0,1\}^{M \times (N+R)}$ represents the triplets, where $N$ is the number of entities, $R$ is the number of relations, and $M$ is the number of triplets. This representation remains extremely sparse, as each row contains exactly three non-zero values (or two in the case of the "ht" representation). Hence, the sparsity of this formulation is independent of the graph's structure, ensuring computational efficiency even for dense graphs.

\section{Computational Complexity}
\label{A:complexity}
 For a sparse matrix $A$ with $m \times k$ having $nnz(A)=$ number of non zeros and dense matrix $X$ with $k \times n$ dimension, the computational complexity of the SpMM is $O(nnz(A) \cdot n)$ since there are a total of $nnz(A)$ number of dot products each involving $n$ components. Since our sparse matrix contains exactly three non-zeros in each row, $nnz(A) = 3m$. Therefore, the complexity of SpMM is $O(3m \cdot n)$ or $O(m \cdot n)$, meaning the complexity increases when triplet counts or embedding dimension is increased. Memory access pattern will change when the number of entities is increased and it will affect the runtime, but the algorithmic complexity will not be affected by the number of entities/relations.

\section{Applicability to Non-translational Models}
\label{A:non_trans}
Our paper focused on translational models using sparse operations, but the concept extends broadly to various other knowledge graph embedding (KGE) methods. Neural network-based models, which are inherently matrix-multiplication-based, can be seamlessly integrated into this framework. Additionally, models such as DistMult, ComplEx, and RotatE can be implemented with simple modifications to the SpMM operations. Implementing these KGE models requires modifying the addition and multiplication operators in SpMM, effectively changing the semiring that governs the multiplication.   

In the paper, the sparse matrix $A \in \{-1,0,1\}^{M \times (N+R)}$ represents the triplets, and the dense matrix $E \in \mathbb{R}^{(N+R) \times d}$ represents the embedding matrix, where $N$ is the number of entities, $R$ is the number of relations, and $M$ is the number of triplets. TransE’s score function, defined as $h + r - t$, is computed by multiplying $A$ and $E$ using an SpMM followed by the L2 norm. This operation can be generalized using a semiring-based SpMM model: $Z_{ij} = \bigoplus_{k=1}^{n} (A_{ik} \otimes E_{kj})$

Here, $\oplus$ represents the semiring addition operator, and $\otimes$ represents the semiring multiplication operator. For TransE, these operators correspond to standard arithmetic addition and multiplication, respectively.

\subsection*{DistMult} 
DistMult’s score function has the expression $h \odot r \odot t$. To adapt SpMM for this model, two key adjustments are required: The sparse matrix $A$ stores $+1$ at the positions corresponding to $h_{\text{idx}}$, $t_{\text{idx}}$, and $r_{\text{idx}}$. Both the semiring addition and multiplication operators are set to arithmetic multiplication. These changes enable the use of SpMM for the DistMult score function.

\subsection*{ComplEx} 
ComplEx’s score function has $h \odot r \odot \bar{t}$, where embeddings are stored as complex numbers (e.g., using PyTorch). In this case, the semiring operations are similar to DistMult, but with complex number multiplication replacing real number multiplication.

\subsection*{RotatE} 
RotatE’s score function has $h \odot r - t$. For this model, the semiring requires both arithmetic multiplication and subtraction for $\oplus$. With minor modifications to our SpMM implementation, the semiring addition operator can be adapted to compute $h \odot r - t$.

\subsection*{Support from other libraries}
Many existing libraries, such as GraphBLAS (Kimmerer, Raye, et al., 2024), Ginkgo (Anzt, Hartwig, et al., 2022), and Gunrock (Wang, Yangzihao, et al., 2017), already support custom semirings in SpMM. We can leverage C++ templates to extend support for KGE models with minimal effort.


\begin{figure*}[t]
\centering     %%% not \center
\includegraphics[width=\textwidth]{figures/all-eval.pdf}
\caption{Loss curve for sparse and non-sparse approach. Sparse approach eventually reaches the same loss value with similar Hits@10 test accuracy.}
\label{fig:loss_curve}
\end{figure*}

\section{Model Performance Evaluation and Convergence}
\label{A:eval}
SpTransX follows a slightly different loss curve (see Figure \ref{fig:loss_curve}) and eventually converges with the same loss as other non-sparse implementations such as TorchKGE. We test SpTransX with the WN18 dataset having embedding size 512 (128 for TransR and TransH due to memory limitation) and run 200-1000 epochs. We compute average Hits@10 of 9 runs with different initial seeds and a learning rate scheduler. The results are shown below. We find that Hits@10 is generally comparable to or better than the Hits@10 achieved by TorchKGE.

\begin{table}[h]
\centering
\caption{Average of 9 Hits@10 Accuracy for WN18 dataset}
\begin{tabular}{|c|c|c|}
\hline
\textbf{Model} & \textbf{TorchKGE} & \textbf{SpTransX} \\ \hline
TransE         & 0.79 ± 0.001700   & 0.79 ± 0.002667   \\ \hline
TransR         & 0.29 ± 0.005735   & 0.33 ± 0.006154   \\ \hline
TransH         & 0.76 ± 0.012285   & 0.79 ± 0.001832   \\ \hline
TorusE         & 0.73 ± 0.003258   & 0.73 ± 0.002780   \\ \hline
\end{tabular}
\label{table:perf_eval}
\end{table}

% We also plot the loss curve for different models in Figure \ref{fig:loss_curve}. We observe that the sparse approach follows a similar loss curve and eventually converges to the same final loss.

\section{Distributed SpTransX and Its Applicability to Large KGs}
\label{A:dist}
SpTransX framework includes several features to support distributed KGE training across multi-CPU, multi-GPU, and multi-node setups. Additionally, it incorporates modules for model and dataset streaming to handle massive datasets efficiently. 

Distributed SpTransX relies on PyTorch Distributed Data Parallel (DDP) and Fully Sharded Data Parallel (FSDP) support to distribute sparse computations across multiple GPUs. 

\begin{table}[h]
\centering
\caption{Average Time of 15 Epochs (seconds). Training time of TransE model with Freebase dataset (250M triplets, 77M entities. 74K relations, batch size 393K)  on 32 NVIDIA A100 GPUs. FSDP enables model training with larger embedding when DDP fails.}
\begin{tabular}{|p{2cm}|p{2.5cm}|p{2.5cm}|}
\hline
\textbf{Embedding Size} & \textbf{DDP (Distributed Data Parallel)} & \textbf{FSDP (Fully Sharded Data Parallel)} \\ \hline
16                      & 65.07 ± 1.641                            & 63.35 ± 1.258                               \\ \hline
20                      & Out of Memory                            & 96.44 ± 1.490                               \\ \hline
\end{tabular}
\end{table}

We run an experiment with a large-scale KG to showcase the performance of distributed SpTransX. Freebase (250M triplets, 77M entities. 74K relations, batch size 393K) dataset is trained using the TransE model on 32 NVIDIA A100 GPUs of NERSC using various distributed settings. SpTransX’s Streaming dataset module allows fetching only the necessary batch from the dataset and enables memory-efficient training. FSDP enables model training with larger embedding when DDP fails.

\section{Scaling and Communication Bottlenecks for Large KG Training}
\label{A:scaling}
Communication can be a significant bottleneck in distributed KGE training when using SpMM. However, by leveraging Distributed Data-Parallel (DDP) in PyTorch, we successfully scale distributed SpTransX to 64 NVIDIA A100 GPUs with reasonable efficiency. The training time for the COVID-19 dataset with 60,820 entities, 62 relations, and 1,032,939 triplets is in Table \ref{table:scaling}. 
% \vspace{-.3cm}
\begin{table}[h]
\centering
\caption{Scaling TransE model on COVID-19 dataset}
\begin{tabular}{|c|c|}
\hline
\textbf{Number of GPUs} & \textbf{500 epoch time (seconds)} \\ \hline
4                       & 706.38                            \\ \hline
8                       & 586.03                            \\ \hline
16                      & 340.00                               \\ \hline
32                      & 246.02                            \\ \hline
64                      & 179.95                            \\ \hline
\end{tabular}
\label{table:scaling}
\end{table}
% \vspace{-.2cm}
It indicates that communication is not a bottleneck up to 64 GPUs. If communication becomes a performance bottleneck at larger scales, we plan to explore alternative communication-reducing algorithms, including 2D and 3D matrix distribution techniques, which are known to minimize communication overhead at extreme scales. Additionally, we will incorporate model parallelism alongside data parallelism for large-scale knowledge graphs.

\section{Backpropagation of SpMM}
\label{A:backprop}
 Our main computational kernel is the sparse-dense matrix multiplication (SpMM). The computation of backpropagation of an SpMM w.r.t. the dense matrix is also another SpMM. To see how, let's consider the sparse-dense matrix multiplication $AX = C$ which is part of the training process. As long as the computational graph reduces to a single scaler loss $\mathfrak{L}$, it can be shown that $\frac{\partial C}{\partial X} = A^T$. Here, $X$ is the learnable parameter (embeddings), and $A$ is the sparse matrix. Since $A^T$ is also a sparse matrix and $\frac{\partial \mathfrak{L}}{\partial C}$ is a dense matrix, the computation $\frac{\partial \mathfrak{L}}{\partial X} = \frac{\partial C}{\partial X} \times \frac{\partial \mathfrak{L}}{\partial C} = A^T \times \frac{\partial \mathfrak{L}}{\partial C} $ is an SpMM. This means that both forward and backward propagation of our approach benefit from the efficiency of a high-performance SpMM.

\subsection*{Proof that $\frac{\partial C}{\partial X} = A^T$}
 To see why $\frac{\partial C}{\partial X} = A^T$ is used in the gradient calculation, we can consider the following small matrix multiplication without loss of generality.
\begin{align*}
A &= \begin{bmatrix}
a_1 & a_2 \\
a_3 & a_4
\end{bmatrix} \\ 
 X &= \begin{bmatrix}
x_1 & x_2 \\
x_3 & x_4
\end{bmatrix} \\
 C &=  \begin{bmatrix}
c_1 & c_2 \\
c_3 & c_4
\end{bmatrix}
\end{align*}
Where $C=AX$, thus-
\begin{align*}
c_1&=f(x_1, x_3) \\
c_2&=f(x_2, x_4) \\
c_3&=f(x_1, x_3) \\
c_4&=f(x_2, x_4) \\
\end{align*}
Therefore-
\begin{align*}
\frac{\partial \mathfrak{L}}{\partial x_1} &= \frac{\partial \mathfrak{L}}{\partial c_1} \times \frac{\partial c_1}{\partial x_1} + \frac{\partial \mathfrak{L}}{\partial c_2} \times \frac{\partial c_2}{\partial x_1} + \frac{\partial \mathfrak{L}}{\partial c_3} \times \frac{\partial c_3}{\partial x_1} + \frac{\partial \mathfrak{L}}{\partial c_4} \times \frac{\partial c_4}{\partial x_1}\\
&= \frac{\partial \mathfrak{L}}{\partial c_1} \times \frac{\partial \mathfrak{c_1}}{\partial x_1} + 0 + \frac{\partial \mathfrak{L}}{\partial c_3} \times \frac{\partial \mathfrak{c_3}}{\partial x_1} + 0\\
&= a_1 \times \frac{\partial \mathfrak{L}}{\partial c_1} + a_3 \times \frac{\partial \mathfrak{L}}{\partial c_3}\\
\end{align*}

Similarly-
\begin{align*}
\frac{\partial \mathfrak{L}}{\partial x_2}
&= a_1 \times \frac{\partial \mathfrak{L}}{\partial c_2} + a_3 \times \frac{\partial \mathfrak{L}}{\partial c_4}\\
\frac{\partial \mathfrak{L}}{\partial x_3}
&= a_2 \times \frac{\partial \mathfrak{L}}{\partial c_1} + a_4 \times \frac{\partial \mathfrak{L}}{\partial c_3}\\
\frac{\partial \mathfrak{L}}{\partial x_4}
&= a_2 \times \frac{\partial \mathfrak{L}}{\partial c_2} + a_4 \times \frac{\partial \mathfrak{L}}{\partial c_4}\\
\end{align*}
This can be expressed as a matrix equation in the following manner-
\begin{align*}
\frac{\partial \mathfrak{L}}{\partial X} &= \frac{\partial C}{\partial X} \times \frac{\partial \mathfrak{L}}{\partial C}\\
\implies \begin{bmatrix}
\frac{\partial \mathfrak{L}}{\partial x_1} & \frac{\partial \mathfrak{L}}{\partial x_2} \\
\frac{\partial \mathfrak{L}}{\partial x_3} & \frac{\partial \mathfrak{L}}{\partial x_4}
\end{bmatrix} &= \frac{\partial C}{\partial X} \times \begin{bmatrix}
\frac{\partial \mathfrak{L}}{\partial c_1} & \frac{\partial \mathfrak{L}}{\partial c_2} \\
\frac{\partial \mathfrak{L}}{\partial c_3} & \frac{\partial \mathfrak{L}}{\partial c_4}
\end{bmatrix}
\end{align*}
By comparing the individual partial derivatives computed earlier, we can say-

\begin{align*}
\begin{bmatrix}
\frac{\partial \mathfrak{L}}{\partial x_1} & \frac{\partial \mathfrak{L}}{\partial x_2} \\
\frac{\partial \mathfrak{L}}{\partial x_3} & \frac{\partial \mathfrak{L}}{\partial x_4}
\end{bmatrix} &= \begin{bmatrix}
a_1 & a_3 \\
a_2 & a_4
\end{bmatrix} \times \begin{bmatrix}
\frac{\partial \mathfrak{L}}{\partial c_1} & \frac{\partial \mathfrak{L}}{\partial c_2} \\
\frac{\partial \mathfrak{L}}{\partial c_3} & \frac{\partial \mathfrak{L}}{\partial c_4}
\end{bmatrix}\\
\implies \begin{bmatrix}
\frac{\partial \mathfrak{L}}{\partial x_1} & \frac{\partial \mathfrak{L}}{\partial x_2} \\
\frac{\partial \mathfrak{L}}{\partial x_3} & \frac{\partial \mathfrak{L}}{\partial x_4}
\end{bmatrix} &= A^T \times \begin{bmatrix}
\frac{\partial \mathfrak{L}}{\partial c_1} & \frac{\partial \mathfrak{L}}{\partial c_2} \\
\frac{\partial \mathfrak{L}}{\partial c_3} & \frac{\partial \mathfrak{L}}{\partial c_4}
\end{bmatrix}\\
\implies \frac{\partial \mathfrak{L}}{\partial X} &= A^T \times \frac{\partial \mathfrak{L}}{\partial C}\\
\therefore \frac{\partial C}{\partial X} &= A^T \qed
\end{align*}


\end{document}