\section{Related Work}
% Paragraph: Talk about DT
\paragraph{Advancements in Decision Transformers}
Classic DT encounters significant challenges, including limited trajectory stitching capabilities and difficulties in adapting to online environments. To address these issues, several enhancements have been proposed. The Q-DT~\cite{qdt} improves the ability to derive optimal policies from sub-optimal trajectories by relabeling return-to-go values in the training data. Elastic DT~\cite{10.5555/3666122.3666936} enhances classic DT by enabling trajectory stitching during action inference at test time, while Multi-Game DT~\cite{10.5555/3600270.3602295} advances its task generalization capabilities. The Online DT~\cite{online_dt,3056976d4116471a86ab3fa345b1695d} extends DTs to online settings by combining offline pretraining with online fine-tuning, facilitating continuous policy updates in dynamic environments. Additionally, adaptations for offline safe RL incorporate cost tokens alongside rewards~\cite{liu2023constrained, pmlr-v238-hong24a}. DT has also been effectively applied to real-world domains, such as healthcare~\cite{Zhang2023} and chip design~\cite{pmlr-v202-lai23c}, showcasing its versatility and practical utility.

\paragraph{RL for EV Smart Charging}
RL algorithms offer notable advantages for EV dispatch, including the ability to handle nonlinear models, robustly quantify uncertainty, and deliver faster computation than traditional mathematical programming~\cite{QIU2023113052}. Popular methods, such as DDPG~\cite{JIN2022120140}, SAC~\cite{9211734}, and batch RL~\cite{8727484}, show promise but often lack formal constraint satisfaction guarantees and struggle to scale with high-dimensional state-action spaces~\cite{isgt2024, 9465776}. Safe RL frameworks address these drawbacks by imposing constraints via constrained MDPs, but typically sacrifice performance and scalability~\cite{ZHANG2023121490, chen2022deep}. Multiagent RL techniques distribute complexity across multiple agents, e.g. charging points, stations, or aggregators~\cite{KAMRANI2025100620}, yet still encounter convergence challenges and may underperform in large-scale applications. 
To the best of our knowledge, no study has used DTs for solving the complex EV charging problem, despite DT's potential to handle sparse rewards effectively.