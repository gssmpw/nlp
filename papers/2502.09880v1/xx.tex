\begin{filecontents}{si.aux}
\relax 
\providecommand\zref@newlabel[2]{}
\providecommand\babel@aux[2]{}
\@nameuse{bbl@beforestart}
\providecommand\hyper@newdestlabel[2]{}
\providecommand\HyField@AuxAddToFields[1]{}
\providecommand\HyField@AuxAddToCoFields[2]{}
\bibstyle{pnas-new}
\babel@aux{english}{}
\@writefile{toc}{\contentsline {section}{\numberline {1}Data of r/place and compositions}{2}{section.1}\protected@file@percent }
\@writefile{toc}{\contentsline {subsection}{\numberline {A}r/place data and generalities}{2}{subsection.1.1}\protected@file@percent }
\newlabel{secSI:rplacedata}{{1.A}{2}{r/place data and generalities}{subsection.1.1}{}}
\@writefile{toc}{\contentsline {paragraph}{Canvas basic rules and statistics.}{2}{figure.caption.1}\protected@file@percent }
\@writefile{toc}{\contentsline {paragraph}{Raw data.}{2}{figure.caption.1}\protected@file@percent }
\@writefile{toc}{\contentsline {paragraph}{Canvas extension history.}{2}{figure.caption.1}\protected@file@percent }
\@writefile{lot}{\contentsline {table}{\numberline {S1}{\ignorespaces Time points at which the canvas was extended or offered more color choices.}}{2}{table.caption.2}\protected@file@percent }
\newlabel{tabSI:extensions}{{S1}{2}{Time points at which the canvas was extended or offered more color choices}{table.caption.2}{}}
\@writefile{toc}{\contentsline {paragraph}{Cooldown.}{2}{table.caption.2}\protected@file@percent }
\@writefile{toc}{\contentsline {paragraph}{Redundant changes.}{2}{table.caption.2}\protected@file@percent }
\@writefile{toc}{\contentsline {paragraph}{Moderators.}{2}{table.caption.2}\protected@file@percent }
\@writefile{lof}{\contentsline {figure}{\numberline {S1}{\ignorespaces General statistics of the r/place pixel change data from 2022 and 2023. \textbf  {(a), (b)} For 2022 (a) and 2023 (b), the heat map of the number of pixel changes over the course of the entire event (left), and the distribution of pixel changes across available colors (right); note that some colors and canvas regions were only available later in the game. \textbf  {(c)} For 2022 (red) and 2023 (blue), distributions of the number of changes per pixel (left), and distributions of the number of pixel changes per user (right).}}{3}{figure.caption.1}\protected@file@percent }
\providecommand*\caption@xref[2]{\@setref\relax\@undefined{#1}}
\newlabel{figSI:rplace}{{S1}{3}{General statistics of the r/place pixel change data from 2022 and 2023. \textbf {(a), (b)} For 2022 (a) and 2023 (b), the heat map of the number of pixel changes over the course of the entire event (left), and the distribution of pixel changes across available colors (right); note that some colors and canvas regions were only available later in the game. \textbf {(c)} For 2022 (red) and 2023 (blue), distributions of the number of changes per pixel (left), and distributions of the number of pixel changes per user (right)}{figure.caption.1}{}}
\citation{haagmansPlaceAtlasInitiative2024}
\@writefile{toc}{\contentsline {paragraph}{Multiple accounts and bots.}{4}{table.caption.2}\protected@file@percent }
\@writefile{toc}{\contentsline {subsection}{\numberline {B}Data of compositions}{4}{subsection.1.2}\protected@file@percent }
\newlabel{secSI:compodata}{{1.B}{4}{Data of compositions}{subsection.1.2}{}}
\@writefile{toc}{\contentsline {paragraph}{Crowdsourced data.}{4}{subsection.1.2}\protected@file@percent }
\@writefile{toc}{\contentsline {paragraph}{Data cleaning.}{4}{subsection.1.2}\protected@file@percent }
\@writefile{toc}{\contentsline {section}{\numberline {2}Transitions}{5}{section.2}\protected@file@percent }
\newlabel{secSI:transitions}{{2}{5}{Transitions}{section.2}{}}
\@writefile{toc}{\contentsline {paragraph}{Stability of pre-transition systems.}{5}{section.2}\protected@file@percent }
\@writefile{toc}{\contentsline {paragraph}{Transitions from a patchwork.}{5}{section.2}\protected@file@percent }
\@writefile{toc}{\contentsline {paragraph}{Time of the transition.}{5}{section.2}\protected@file@percent }
\@writefile{toc}{\contentsline {section}{\numberline {3}Composition-level time series and variables}{5}{section.3}\protected@file@percent }
\newlabel{secSI:variables}{{3}{5}{Composition-level time series and variables}{section.3}{}}
\@writefile{toc}{\contentsline {subsection}{\numberline {A}Normalizations}{5}{subsection.3.1}\protected@file@percent }
\@writefile{lof}{\contentsline {figure}{\numberline {S2}{\ignorespaces Examples of compositions where no transitions are tagged despite a large image change. \textbf  {(a)} The fraction of differing pixels (\texttt  {diff pixels reference}) passes the transition threshold requirements, but the pre-transition image is a patchwork of compositions. \textbf  {(b)} Fraction of differing pixels shows too slow of a change to a new composition. In addition, the pre-transition image is not an Atlas composition, hence not a monitored subsystem.}}{6}{figure.caption.3}\protected@file@percent }
\newlabel{figSI:notrans}{{S2}{6}{Examples of compositions where no transitions are tagged despite a large image change. \textbf {(a)} The fraction of differing pixels (\texttt {diff pixels reference}) passes the transition threshold requirements, but the pre-transition image is a patchwork of compositions. \textbf {(b)} Fraction of differing pixels shows too slow of a change to a new composition. In addition, the pre-transition image is not an Atlas composition, hence not a monitored subsystem}{figure.caption.3}{}}
\@writefile{toc}{\contentsline {subsection}{\numberline {B}State variable}{6}{subsection.3.2}\protected@file@percent }
\@writefile{toc}{\contentsline {subsection}{\numberline {C}Variance}{6}{subsection.3.3}\protected@file@percent }
\@writefile{toc}{\contentsline {paragraph}{Differing pixels and other linear variations.}{7}{subsection.3.3}\protected@file@percent }
\@writefile{toc}{\contentsline {paragraph}{Quadratic pixel variations.}{7}{subsection.3.3}\protected@file@percent }
\@writefile{toc}{\contentsline {paragraph}{Variations of an existing time series.}{8}{subsection.3.3}\protected@file@percent }
\@writefile{toc}{\contentsline {subsection}{\numberline {D}Autocorrelation}{8}{subsection.3.4}\protected@file@percent }
\citation{martinianiQuantifyingHiddenOrder2019}
\citation{martinianiSweetsourcod2019}
\citation{mullerCompressionCulturalEvolution2018}
\citation{martinianiQuantifyingHiddenOrder2019}
\citation{martinianiQuantifyingHiddenOrder2019}
\citation{bisoiCalculationFractalDimension2001}
\@writefile{toc}{\contentsline {subsection}{\numberline {E}Attack duration and return rate}{9}{subsection.3.5}\protected@file@percent }
\@writefile{toc}{\contentsline {subsection}{\numberline {F}Image complexity}{9}{subsection.3.6}\protected@file@percent }
\newlabel{secSI:imagecomplex}{{3.F}{9}{Image complexity}{subsection.3.6}{}}
\@writefile{toc}{\contentsline {paragraph}{Entropy.}{9}{subsection.3.6}\protected@file@percent }
\@writefile{toc}{\contentsline {paragraph}{Fractal dimension.}{9}{subsection.3.6}\protected@file@percent }
\@writefile{toc}{\contentsline {subsection}{\numberline {G}User activity}{10}{subsection.3.7}\protected@file@percent }
\@writefile{toc}{\contentsline {subsection}{\numberline {H}Time-independent per-composition}{10}{subsection.3.8}\protected@file@percent }
\@writefile{toc}{\contentsline {subsection}{\numberline {I}Variables selected in the training}{10}{subsection.3.9}\protected@file@percent }
\newlabel{secSI:varselect}{{3.I}{10}{Variables selected in the training}{subsection.3.9}{}}
\@writefile{toc}{\contentsline {section}{\numberline {4}Data preparation and training of the machine learning algorithm}{10}{section.4}\protected@file@percent }
\@writefile{toc}{\contentsline {subsection}{\numberline {A}Memory of time series embedded in the features}{10}{subsection.4.1}\protected@file@percent }
\newlabel{secSI:memory}{{4.A}{10}{Memory of time series embedded in the features}{subsection.4.1}{}}
\@writefile{lof}{\contentsline {figure}{\numberline {S3}{\ignorespaces Correlation coefficients between all training variables and the time-to-transition, over all time instances used in the training.}}{11}{figure.caption.5}\protected@file@percent }
\newlabel{figSI:correlations}{{S3}{11}{Correlation coefficients between all training variables and the time-to-transition, over all time instances used in the training}{figure.caption.5}{}}
\@writefile{lot}{\contentsline {table}{\numberline {S2}{\ignorespaces Variables used in the training. The short name is that used in figures and in the text when referring to this exact variable. The last column is checked for variables whose memory is recorded in 9 rather than 12 time range features.}}{12}{table.caption.4}\protected@file@percent }
\newlabel{tabSI:variables}{{S2}{12}{Variables used in the training. The short name is that used in figures and in the text when referring to this exact variable. The last column is checked for variables whose memory is recorded in 9 rather than 12 time range features}{table.caption.4}{}}
\@writefile{toc}{\contentsline {subsection}{\numberline {B}Filtering of training dataset}{12}{subsection.4.2}\protected@file@percent }
\newlabel{secSI:filter}{{4.B}{12}{Filtering of training dataset}{subsection.4.2}{}}
\@writefile{lot}{\contentsline {table}{\numberline {S3}{\ignorespaces Time ranges of the memory stored for each feature and time instance. Larger times correspond to earlier times in the memory. The central column is the midpoint of the time range, which is used as shortcuts in figures. The last two columns indicate whether a given time range feature is used for coarse or non-coarse variables, which are listed in Table~\ref {tabSI:variables}.}}{13}{table.caption.6}\protected@file@percent }
\newlabel{tabSI:timeranges}{{S3}{13}{Time ranges of the memory stored for each feature and time instance. Larger times correspond to earlier times in the memory. The central column is the midpoint of the time range, which is used as shortcuts in figures. The last two columns indicate whether a given time range feature is used for coarse or non-coarse variables, which are listed in Table~\ref {tabSI:variables}}{table.caption.6}{}}
\@writefile{toc}{\contentsline {subsection}{\numberline {C}Feature selection}{13}{subsection.4.3}\protected@file@percent }
\newlabel{secSI:featureselec}{{4.C}{13}{Feature selection}{subsection.4.3}{}}
\citation{chenXGBoostScalableTree2016}
\@writefile{toc}{\contentsline {subsection}{\numberline {D}Target value, weights and loss term}{14}{subsection.4.4}\protected@file@percent }
\newlabel{secSI:target}{{4.D}{14}{Target value, weights and loss term}{subsection.4.4}{}}
\@writefile{toc}{\contentsline {subsection}{\numberline {E}Algorithm setup, training, and hyperparameters}{14}{subsection.4.5}\protected@file@percent }
\newlabel{secSI:algo}{{4.E}{14}{Algorithm setup, training, and hyperparameters}{subsection.4.5}{}}
\citation{hylandEarlyPredictionCirculatory2020}
\@writefile{toc}{\contentsline {section}{\numberline {5}Testing and performance evaluation}{15}{section.5}\protected@file@percent }
\newlabel{secSI:testing}{{5}{15}{Testing and performance evaluation}{section.5}{}}
\@writefile{toc}{\contentsline {subsection}{\numberline {A}Calibration of model output}{15}{subsection.5.1}\protected@file@percent }
\newlabel{secSI:calib}{{5.A}{15}{Calibration of model output}{subsection.5.1}{}}
\@writefile{toc}{\contentsline {subsection}{\numberline {B}ROC and PR curves}{15}{subsection.5.2}\protected@file@percent }
\newlabel{secSI:ROCPR}{{5.B}{15}{ROC and PR curves}{subsection.5.2}{}}
\@writefile{toc}{\contentsline {subsection}{\numberline {C}Per-composition results}{15}{subsection.5.3}\protected@file@percent }
\newlabel{secSI:percompo}{{5.C}{15}{Per-composition results}{subsection.5.3}{}}
\@writefile{toc}{\contentsline {subsection}{\numberline {D}cooldown warning system}{15}{subsection.5.4}\protected@file@percent }
\newlabel{secSI:cooldownWarn}{{5.D}{15}{cooldown warning system}{subsection.5.4}{}}
\@writefile{toc}{\contentsline {subsection}{\numberline {E}Sensitivity analysis of parameter values}{15}{subsection.5.5}\protected@file@percent }
\newlabel{secSI:sensitivity}{{5.E}{15}{Sensitivity analysis of parameter values}{subsection.5.5}{}}
\@writefile{lof}{\contentsline {figure}{\numberline {S4}{\ignorespaces \textbf  {(a)} ROC curves for our machine learning algorithm trained and tested on 2022 r/place data (dark blue) and for a standard single-variable warning signal of variance (light blue), for four different warning ranges. \textbf  {(b)} Histogram of areas under the ROC curves for time instances grouped by composition, for warning ranges of $20$~min (blue) and $1$~h (purple)}}{16}{figure.caption.7}\protected@file@percent }
\newlabel{figSI:ROC}{{S4}{16}{\textbf {(a)} ROC curves for our machine learning algorithm trained and tested on 2022 r/place data (dark blue) and for a standard single-variable warning signal of variance (light blue), for four different warning ranges. \textbf {(b)} Histogram of areas under the ROC curves for time instances grouped by composition, for warning ranges of $20$~min (blue) and $1$~h (purple)}{figure.caption.7}{}}
\@writefile{toc}{\contentsline {section}{\numberline {6}Interpretation of predictions}{16}{section.6}\protected@file@percent }
\@writefile{toc}{\contentsline {subsection}{\numberline {A}SHAP trends}{16}{subsection.6.1}\protected@file@percent }
\newlabel{secSI:shap}{{6.A}{16}{SHAP trends}{subsection.6.1}{}}
\@writefile{lof}{\contentsline {figure}{\numberline {S5}{\ignorespaces \textbf  {(a)} Precision-Recall (PR) curves for our machine learning algorithm trained and tested on 2022 r/place data (dark blue) and for a standard single-variable warning signal of variance (light blue), at four different time thresholds for warnings. \textbf  {(b)} Area under the PR curves as a function of warning threshold time for the machine learning warning signal (dark blue), a standard single-variable warning signal of variance (light blue), and for the machine learning warning signal tested on 2023 r/place data (red).}}{17}{figure.caption.8}\protected@file@percent }
\newlabel{figSI:PR}{{S5}{17}{\textbf {(a)} Precision-Recall (PR) curves for our machine learning algorithm trained and tested on 2022 r/place data (dark blue) and for a standard single-variable warning signal of variance (light blue), at four different time thresholds for warnings. \textbf {(b)} Area under the PR curves as a function of warning threshold time for the machine learning warning signal (dark blue), a standard single-variable warning signal of variance (light blue), and for the machine learning warning signal tested on 2023 r/place data (red)}{figure.caption.8}{}}
\@writefile{lof}{\contentsline {figure}{\numberline {S6}{\ignorespaces Performance varies only marginally with transition detection parameters and sliding window width. For both panels: combinations of values of sliding window reference width, absolute threshold, and relative threshold for transition detection are shown in different colors according to the legend. The nominal value chosen in the main text is shown in dark purple. \textbf  {(a)} Receiver operating characteristic (ROC) curves for our machine learning algorithm trained and tested on 2022 r/place data, for the different parameter combinations at four different time thresholds for warnings. \textbf  {(b)} Area under the ROC curves for our machine learning warning signal for the different parameter combinations as a function of the warning threshold time.}}{18}{figure.caption.9}\protected@file@percent }
\newlabel{figSI:sensitivity}{{S6}{18}{Performance varies only marginally with transition detection parameters and sliding window width. For both panels: combinations of values of sliding window reference width, absolute threshold, and relative threshold for transition detection are shown in different colors according to the legend. The nominal value chosen in the main text is shown in dark purple. \textbf {(a)} Receiver operating characteristic (ROC) curves for our machine learning algorithm trained and tested on 2022 r/place data, for the different parameter combinations at four different time thresholds for warnings. \textbf {(b)} Area under the ROC curves for our machine learning warning signal for the different parameter combinations as a function of the warning threshold time}{figure.caption.9}{}}
\@writefile{toc}{\contentsline {paragraph}{Contradictory variable interpretations.}{19}{subsection.6.1}\protected@file@percent }
\@writefile{toc}{\contentsline {paragraph}{Contradictory trend interpretations.}{19}{subsection.6.1}\protected@file@percent }
\@writefile{toc}{\contentsline {paragraph}{Robustness versus stability requirement and prediction quality.}{19}{subsection.6.1}\protected@file@percent }
\@writefile{toc}{\contentsline {paragraph}{Drawbacks of SHAP.}{19}{subsection.6.1}\protected@file@percent }
\@writefile{toc}{\contentsline {subsection}{\numberline {B}Toy model of canvas}{19}{subsection.6.2}\protected@file@percent }
\newlabel{secSI:toymodel}{{6.B}{19}{Toy model of canvas}{subsection.6.2}{}}
\citation{ritchieEarlywarningIndicatorsRateinduced2016}
\@writefile{lof}{\contentsline {figure}{\numberline {S7}{\ignorespaces Probability density distributions of SHAP values over time instances for each variable, calculated using kernel density estimation. The SHAP for a variable is the sum of the SHAP values of all time features associated with this variable. Variables are ordered according to the mean of the absolute values of the SHAP values.}}{21}{figure.caption.10}\protected@file@percent }
\newlabel{figSI:shapdistr}{{S7}{21}{Probability density distributions of SHAP values over time instances for each variable, calculated using kernel density estimation. The SHAP for a variable is the sum of the SHAP values of all time features associated with this variable. Variables are ordered according to the mean of the absolute values of the SHAP values}{figure.caption.10}{}}
\@writefile{lof}{\contentsline {figure}{\numberline {S8}{\ignorespaces All SHAP curves for each time-dependent variable and corresponding time features. Plots show the variable percentile against the mean SHAP value across time instances. Plot limits were chosen to show detail at smaller values. The legend identifies time features using color.}}{22}{figure.caption.11}\protected@file@percent }
\newlabel{figSI:SHAPall}{{S8}{22}{All SHAP curves for each time-dependent variable and corresponding time features. Plots show the variable percentile against the mean SHAP value across time instances. Plot limits were chosen to show detail at smaller values. The legend identifies time features using color}{figure.caption.11}{}}
\@writefile{lof}{\contentsline {figure}{\numberline {S9}{\ignorespaces SHAP curves for the variables that are recorded without memory. The variable percentile is plotted on the y-axis for \texttt  {area}, \texttt  {age} and \texttt  {entropy}. The raw variable is plotted for \texttt  {border corner center} and \texttt  {canvas quadrant} since these two variables span only three distinct values.}}{23}{figure.caption.12}\protected@file@percent }
\newlabel{figSI:SHAPnotime}{{S9}{23}{SHAP curves for the variables that are recorded without memory. The variable percentile is plotted on the y-axis for \texttt {area}, \texttt {age} and \texttt {entropy}. The raw variable is plotted for \texttt {border corner center} and \texttt {canvas quadrant} since these two variables span only three distinct values}{figure.caption.12}{}}
\@writefile{lof}{\contentsline {figure}{\numberline {S10}{\ignorespaces Performance significantly decreases when using a cooldown warning system. Plots show receiver operating characteristic (ROC) curves for our cooldown machine learning warning system as compared to the machine learning warning system presented in the main text. Both warning systems are trained and tested on 2022 r/place data, at three different time thresholds for warnings.}}{23}{figure.caption.12}\protected@file@percent }
\newlabel{figSI:continuous-warn}{{S10}{23}{Performance significantly decreases when using a cooldown warning system. Plots show receiver operating characteristic (ROC) curves for our cooldown machine learning warning system as compared to the machine learning warning system presented in the main text. Both warning systems are trained and tested on 2022 r/place data, at three different time thresholds for warnings}{figure.caption.12}{}}
\@writefile{lof}{\contentsline {figure}{\numberline {S11}{\ignorespaces Additional interpretation of model predictions with SHAP values. \textbf  {(a-f)} Feature percentiles versus mean SHAP values, classified based on curve trends into six pre-transition behaviors, which are described in the panel title. Top titles describe the signal of a coming transition, and italicized subtitles provide an interpretation of the associated dynamics of users and images of the canvas. Grey arrows indicate the qualitative trend of the curves. The text in each legend label describes the curve of the same color as the label's text. Curves are drawn thinner when they show a trend that is not the focus of their respective panel.}}{24}{figure.caption.13}\protected@file@percent }
\newlabel{figSI:SHAPinterp}{{S11}{24}{Additional interpretation of model predictions with SHAP values. \textbf {(a-f)} Feature percentiles versus mean SHAP values, classified based on curve trends into six pre-transition behaviors, which are described in the panel title. Top titles describe the signal of a coming transition, and italicized subtitles provide an interpretation of the associated dynamics of users and images of the canvas. Grey arrows indicate the qualitative trend of the curves. The text in each legend label describes the curve of the same color as the label's text. Curves are drawn thinner when they show a trend that is not the focus of their respective panel}{figure.caption.13}{}}
\bibdata{references}
\bibcite{haagmansPlaceAtlasInitiative2024}{{1}{}{{}}{{}}}
\bibcite{martinianiQuantifyingHiddenOrder2019}{{2}{}{{}}{{}}}
\bibcite{martinianiSweetsourcod2019}{{3}{}{{}}{{}}}
\bibcite{mullerCompressionCulturalEvolution2018}{{4}{}{{}}{{}}}
\bibcite{bisoiCalculationFractalDimension2001}{{5}{}{{}}{{}}}
\bibcite{chenXGBoostScalableTree2016}{{6}{}{{}}{{}}}
\bibcite{hylandEarlyPredictionCirculatory2020}{{7}{}{{}}{{}}}
\bibcite{ritchieEarlywarningIndicatorsRateinduced2016}{{8}{}{{}}{{}}}
\providecommand\NAT@force@numbers{}\NAT@force@numbers
\expandafter\ifx\csname c@section@totc\endcsname\relax\newcounter{section@totc}\fi\setcounter{section@totc}{6}
\expandafter\ifx\csname c@figure@totc\endcsname\relax\newcounter{figure@totc}\fi\setcounter{figure@totc}{11}
\expandafter\ifx\csname c@table@totc\endcsname\relax\newcounter{table@totc}\fi\setcounter{table@totc}{3}
\expandafter\ifx\csname c@NAT@ctr@totc\endcsname\relax\newcounter{NAT@ctr@totc}\fi\setcounter{NAT@ctr@totc}{8}
\expandafter\ifx\csname c@movie@totc\endcsname\relax\newcounter{movie@totc}\fi\setcounter{movie@totc}{0}
\expandafter\ifx\csname c@dataset@totc\endcsname\relax\newcounter{dataset@totc}\fi\setcounter{dataset@totc}{0}
\expandafter\ifx\csname c@SItext@totc\endcsname\relax\newcounter{SItext@totc}\fi\setcounter{SItext@totc}{1}
\newlabel{LastPage}{{6.B}{25}{Toy model of canvas}{page.25}{}}
\gdef\lastpage@lastpage{25}
\gdef\lastpage@lastpageHy{25}
\gdef \@abspage@last{25}


\end{filecontents}