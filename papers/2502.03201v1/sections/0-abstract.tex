\begin{abstract}
\label{sec:abstract}

Node Anomaly Detection (NAD) has gained significant attention in the deep learning community due to its diverse applications in real-world scenarios. 
Existing NAD methods primarily embed graphs within a single Euclidean space, while overlooking the potential of non-Euclidean spaces. 
Besides, to address the prevalent issue of limited supervision in real NAD tasks, previous methods tend to leverage synthetic data to collect auxiliary information, which is not an effective solution as shown in our experiments.
To overcome these challenges, we introduce a novel SpaceGNN model designed for NAD tasks with extremely limited labels. 
Specifically, we provide deeper insights into a task-relevant framework by empirically analyzing the benefits of different spaces for node representations, based on which, we design a Learnable Space Projection function that effectively encodes nodes into suitable spaces.
Besides, we introduce the concept of weighted homogeneity, which we empirically and theoretically validate as an effective coefficient during information propagation. This concept inspires the design of the Distance Aware Propagation module. 
Furthermore, we propose the Multiple Space Ensemble module, which extracts comprehensive information for NAD under conditions of extremely limited supervision. Our findings indicate that this module is more beneficial than data augmentation techniques for NAD. Extensive experiments conducted on 9 real datasets confirm the superiority of SpaceGNN, which outperforms the best rival by an average of 8.55\% in AUC and 4.31\% in F1 scores. Our code is available at https://github.com/xydong127/SpaceGNN. 

\end{abstract}