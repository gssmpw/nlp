\appendix
\newpage

\section*{Appendix}
\label{sec:appendix}

\section{Proofs}
\label{subsec:proof}
{\bf Proof of Theorem 1.}
Let $p$ denote $WH^\kappa$, then the information a normal node can gain within its neighborhood during a propagation process follows $\mathcal{N}(p\vect{\mu}_n+(1-p)\vect{\mu}_a, p^2\vect{\Sigma}_n+(1-p)^2\vect{\Sigma}_a)$ according to the linear properties of independent Gaussian variables. 

Let $\vect{X}$ and $\vect{Y}$ denote the distribution of the normal node and the information over $\mathbb{R}^d$, respectively. We then use Fréchet inception distance \citep{fid17heusel} to describe the distance between two distributions as follows: 
\begin{equation*}
\begin{aligned}
    F(\vect{X}, \vect{Y})^2&=(\inf_{\gamma\in \Gamma(\vect{X}, \vect{Y})}\int_{\mathbb{R}^d \times \mathbb{R}^d}||\vect{x}-\vect{y}||^2d\gamma(\vect{x}, \vect{y})), \\
    &=(\inf_{\gamma\in \Gamma(\vect{X}, \vect{Y})}\mathbb{E}_{(\vect{x}, \vect{y})\sim\gamma}[||\vect{x}-\vect{y}||^2]),
\end{aligned}
\end{equation*}
where $\Gamma(\vect{X}, \vect{Y})$ is the set of all measures on $\mathbb{R}^d\times\mathbb{R}^d$ with marginals $\vect{X}$ and $\vect{Y}$ on the first and second factors, separately. Hence, we have the following equation: 
\begin{equation*}
\begin{aligned}
    &\mathbb{E}_{(\vect{x}, \vect{y})\sim\gamma}[||\vect{x}-\vect{y}||^2]\\
    =&\mathbb{E}_{(\tilde{\vect{x}}, \tilde{\vect{y}})\sim\tilde{\gamma}}[||(\tilde{\vect{x}}+\vect{\mu_x})-(\tilde{\vect{y}}+\vect{\mu_y})||^2]\\
    =&\mathbb{E}_{(\tilde{\vect{x}}, \tilde{\vect{y}})\sim\tilde{\gamma}}[||\tilde{\vect{x}}-\tilde{\vect{y}}||^2+||\vect{\mu_x}-\vect{\mu_y}||^2+2\langle \tilde{\vect{x}}-\tilde{\vect{y}}, \vect{\mu_x}-\vect{\mu_y}\rangle]\\
    =&||\vect{\mu_x}-\vect{\mu_y}||^2+\mathbb{E}_{(\tilde{\vect{x}}, \tilde{\vect{y}})\sim\tilde{\gamma}}[||\tilde{\vect{x}}-\tilde{\vect{y}}||^2]
\end{aligned}
\end{equation*}
where $\vect{\mu_x}$ and $\vect{\mu_y}$ represent the mean value of distributions $\vect{X}$ and $\vect{Y}$, and $\tilde{\vect{x}}$ and $\tilde{\vect{y}}$ represent vectors following distribution $\tilde{\vect{X}}$ and $\tilde{\vect{Y}}$, which have 0 mean value and the same variance value as $\vect{X}$ and $\vect{Y}$, respectively. Hence, the Fréchet inception distance can be decomposed as:
\begin{equation*}
F(\vect{X}, \vect{Y})^2=||\vect{\mu_x}-\vect{\mu_y}||^2+F(\tilde{\vect{X}}, \tilde{\vect{Y}})^2
\end{equation*}
This result shows the distance between the distribution of the normal node and the information is determined by two parts, the mean value and the variance value. Specifically, we can assume $\vect{\Sigma}_n\approx\vect{\Sigma}_a\approx c\vect{I}$ in real NAD tasks, where $c$ is a small constant, due to the independent similar behaviors of nodes in the same category. Thus, we have $F(\tilde{\vect{X}}, \tilde{\vect{Y}})^2\approx0$ and $F(\vect{X}, \vect{Y})^2=||\vect{\mu_x}-\vect{\mu_y}||^2$. 

Then, we check the distance between mean values of $\vect{X}$ and $\vect{Y}$. Specifically, it can be written as: 
\begin{equation*}
||\vect{\mu_x}-\vect{\mu_y}||^2=(1-p)^2||\vect{\mu}_n-\vect{\mu}_a||^2
\end{equation*}
which concludes that if $||\vect{\mu}_n-\vect{\mu}_a||^2$ remains the same, as $p$ increases, the distance between the distribution of the normal node and the information will decrease, and thus the probability of a normal node following its original distribution after a propagation process increases as $WH_\kappa$ increases. 

The situation of an anomalous node can be analyzed accordingly. This solution concludes that weighted homogeneity can benefit the propagation procedure for NAD tasks. {\hfill \qedsymbol}

{\bf Proof of Theorem 2.} First, we apply Taylor expansion on $\tan_\kappa^{-1}(t)$ for a fixed $t$ when $\kappa\rightarrow 0^+$: 
\begin{equation*}
\begin{aligned}
    \tan_\kappa^{-1}(t)=&\kappa^{-\frac{1}{2}}\tan(\kappa^\frac{1}{2}t)\\
    =&\kappa^{-\frac{1}{2}}(\kappa^\frac{1}{2}t+\kappa^\frac{3}{2}\frac{t^3}{3}+\mathcal{O}(\kappa^\frac{5}{2}))\\
    =&t+\kappa\frac{t^3}{3} + \mathcal{O}(\kappa^2)
\end{aligned}
\end{equation*}
When $\kappa\rightarrow 0^-$:
\begin{equation*}
\begin{aligned}
    \tan_\kappa^{-1}(t)=&(-\kappa)^{-\frac{1}{2}}\tanh((-\kappa)^\frac{1}{2}t)\\
    =&(-\kappa)^{-\frac{1}{2}}((-\kappa)^\frac{1}{2}t-(-\kappa)^\frac{3}{2}\frac{t^3}{3}+\mathcal{O}(\kappa^\frac{5}{2}))\\
    =&t+\kappa\frac{t^3}{3}+\mathcal{O}(\kappa^2)
\end{aligned}
\end{equation*}
When $\kappa\rightarrow 0$, we also have $\tan_\kappa^{-1}(t)=t-\kappa\frac{t^3}{3}+\mathcal{O}(\kappa^2)$. Hence, we conclude that near 0, $\tan_\kappa^{-1}(t)=t-\kappa\frac{t^3}{3}+\mathcal{O}(\kappa^2)$. 

Then, we need to use the Tayler expansion for $||\cdot||$. Specifically, $||\vect{x}+\vect{o}||=||\vect{x}||+\langle \vect{x}, \vect{o}\rangle+\mathcal{O}(||\vect{o}||^2)$ when $\vect{o}\rightarrow \vect{0}$. 

After that, we derive the Tayler expansion for $\vect{x}\oplus_\kappa\vect{y}$ when $\kappa$ near 0: 
\begin{equation*}
\begin{aligned}
\vect{x}\oplus\vect{y}=&\frac{(1-2\kappa\vect{x^T}\vect{y}-\kappa||\vect{y}||^2)\vect{x}+(1+\kappa||\vect{x}||^2)\vect{y}}{1-2\kappa\vect{x}^T\vect{y}+\kappa^2||\vect{x}||^2||\vect{y}||^2}\\
=&((1-2\kappa\vect{x^T}\vect{y}-\kappa||\vect{y}||^2)\vect{x}+(1+\kappa||\vect{x}||^2)\vect{y})(1+2\kappa\vect{x^T}\vect{y}+\mathcal{O}(\kappa^2))\\
=&(1-2\kappa\vect{x^T}\vect{y}-\kappa||\vect{y}||^2)\vect{x}+(1+\kappa||\vect{x}||^2)\vect{y}+2\kappa\vect{x^T}\vect{y}(\vect{x}+\vect{y})+\mathcal{O}(\kappa^2)\\
=&(1-\kappa||\vect{y}||^2)\vect{x}+(1+\kappa||\vect{x}||^2)\vect{y}+2\kappa(\vect{x^T}\vect{y})\vect{y}+\mathcal{O}(\kappa^2)\\
=&\vect{x}+\vect{y}+\kappa(||\vect{x}||^2\vect{y}-||\vect{y}||^2\vect{x}+2(\vect{x^T}\vect{y})\vect{y})+\mathcal{O}(\kappa^2)
\end{aligned}
\end{equation*}

By combining the above three Tayler expansions and ignore $\mathcal{O}(\kappa^2)$, we have the following equation: 
\begin{equation*}
\begin{aligned}
d_\kappa(\vect{x}, \vect{y})=&2\tan_\kappa^{-1}(||(-\vect{x})\oplus_\kappa\vect{y}||)\\
=&2(||\vect{x}-\vect{y}||+\kappa((-\vect{x})^T\vect{y})||\vect{x}-\vect{y}||^2)(1-\frac{\kappa}{3}(||\vect{x}-\vect{y}||^2))\\
=&2||\vect{x}-\vect{y}||-2\kappa((\vect{x^T}\vect{y})||\vect{x}-\vect{y}||^2+\frac{||\vect{x}-\vect{y}||^3}{3})
\end{aligned}
\end{equation*}
which concludes our theorem. {\hfill \qedsymbol}

{\bf Proof of {\update Proposition} \ref{thm:ensemble}.} We use the weighted cross-entropy loss of $m$ single models to subtract the ensemble cross-entropy loss: 
\begin{equation*}
\begin{aligned}
\sum_{i=1}^m\alpha_i\mathcal{L}(\vect{p}, \vect{q}_i)-\mathcal{L}(\vect{p}, \bar{\vect{q}})=&\sum_{c=1}^C\vect{p}^c\log\bar{\vect{q}}^c-\sum_{i=1}^m\sum_{c=1}^C\alpha_i\vect{p}^c\log\vect{q}^c_i\\
=&\sum_{c=1}^C\vect{p}^c\log\bar{\vect{q}}^c-\sum_{c=1}^C\vect{p}^c\log(\prod_i(\vect{q}_i^c)^{\alpha_i})\\
=&\sum_{c=1}^C\vect{p}^c\log(\frac{\bar{\vect{q}}^c}{\prod_i(\vect{q}_i^c)^{\alpha_i}})\\
=&\Omega(\vect{p})
\end{aligned}
\end{equation*}

By applying the weighted AM–GM inequality, we have $\bar{\vect{q}}\geq\prod_i(\vect{q}_i^c)^{\alpha_i}$, which means the term inside the log function is greater or equal to 1, and so the $\Omega(\vect{p})$ is non-negative. Thus, it finishes the proof of the {\update Proposition}. {\hfill \qedsymbol}

{\bf Proof of {\update Proposition} \ref{thm:expectation}.} We take the expectation of the equation in {\update Proposition} \ref{thm:ensemble} over entire graph $G$ and apply KL bias-variance decomposition in previous study \citep{enstheo23wood}, then we have: 
\begin{equation*}
\begin{aligned}
\mathbb{E}_G[\mathcal{L}(\vect{p}, \bar{\vect{q}})]=&\mathbb{E}_G[\sum_{i=1}^m\alpha_i\mathcal{L}(\vect{p}, \vect{q}_i)]-\mathbb{E}_G[\Omega(\vect{p})]\\
=&\sum_{i=1}^m\alpha_i\mathcal{L}(\vect{p}, \vect{\hat{q}}_i)+\sum_{i=1}^m\alpha_i\mathbb{E}_G[\KL(\vect{\hat{q}}_i||\vect{q}_i)]-\mathbb{E}_G[\Omega(\vect{p})]\\
=&\sum_{i=1}^m\alpha_i\mathcal{L}(\vect{p}, \vect{\hat{q}}_i)+\Theta(\vect{p}), 
\end{aligned}
\end{equation*}
where $\Theta(\vect{p})$ is demonstrated as non-negative in previous study \citep{enstheo23wood}, and thus this {\update Proposition} is proven. {\hfill \qedsymbol}

\section{Datasets and Baselines}
\label{subsec:dataset-baseline}
\textbf{Datasets.} The datasets used in our experiments are from the most recent benchmark paper \citep{gadbench23tang}, according to which, Weibo, Reddit, Questions, and T-Social aim to detect anomalous accounts on social media, Tolokers, Amazon, and YelpChi are proposed for malicious comments detection in review platforms, and T-Finance and DGraph-Fin focus on fraud detection in financial networks. The statistics of these 9 real-world datasets are shown in Table \ref{tab:datasets}. 

\begin{table*}[ht]
    \footnotesize
    \centering
    \renewcommand{\arraystretch}{1.1} % Adjusts the row spacing
    \resizebox{16cm}{!} 
    { 
    \begin{tblr}{hline{1,2,Z} = 0.8pt, hline{3-Y} = 0.2pt,
                 colspec = {Q[l,m, 13em] Q[l,m, 6em] Q[c,m, 8em] Q[c,m, 5em] Q[l,m, 14em]},
                 colsep  = 4pt,
                 row{1}  = {0.4cm, font=\bfseries, bg=gray!30},
                 row{2-Z} = {0.2cm},
                 }
\textbf{Dataset}       & \textbf{Table Source} & \textbf{\# Tables / Statements} & \textbf{\# Words / Statement} & \textbf{Explicit Control}\\ 
\SetCell[c=5]{c} \textit{Single-sentence Table-to-Text}\\
ToTTo \cite{parikh2020tottocontrolledtabletotextgeneration}   & Wikipedia        & 83,141 / 83,141                  & 17.4                          & Table region      \\
LOGICNLG \cite{chen2020logicalnaturallanguagegeneration} & Wikipedia        & 7,392 / 36,960                  & 14.2                          & Table regions      \\ 
HiTab \cite{cheng-etal-2022-hitab}   & Statistics web   & 3,597 / 10,672                  & 16.4                          & Table regions \& reasoning operator \\ 
\SetCell[c=5]{c} \textit{Generic Table Summarization}\\
ROTOWIRE \cite{wiseman2017challengesdatatodocumentgeneration} & NBA games      & 4,953 / 4,953                   & 337.1                         & \textbf{\textit{X}}                   \\
SciGen \cite{moosavi2021scigen} & Sci-Paper      & 1,338 / 1,338                   & 116.0                         & \textbf{\textit{X}}                   \\
NumericNLG \cite{suadaa-etal-2021-towards} & Sci-Paper   & 1,355 / 1,355                   & 94.2                          & \textbf{\textit{X}}                    \\
\SetCell[c=5]{c} \textit{Table Question Answering}\\
FeTaQA \cite{nan2021fetaqafreeformtablequestion}     & Wikipedia      & 10,330 / 10,330                 & 18.9                          & Queries rewritten from ToTTo \\
\SetCell[c=5]{c} \textit{Query-Focused Table Summarization}\\
QTSumm \cite{zhao2023qtsummqueryfocusedsummarizationtabular}                        & Wikipedia      & 2,934 / 7,111                   & 68.0                          & Queries from real-world scenarios\\ 
\textbf{eC-Tab2Text} (\textit{ours})                           & e-Commerce products      & 1,452 / 3,354                   & 56.61                          & Queries from e-commerce products\\
    \end{tblr}
    }
\caption{Comparison between \textbf{eC-Tab2Text} (\textit{ours}) and existing table-to-text generation datasets. Statements and queries are used interchangeably. Our dataset specifically comprises tables from the e-commerce domain.}
\label{tab:datasets}
\end{table*}

\textbf{Baselines.}  The first group is generalized models:
\begin{itemize}[topsep=0.5mm, partopsep=0pt, itemsep=0pt, leftmargin=10pt]
    \item MLP \citep{mlp58f}: A type of neural network with multiple layers of fully connected artificial neurons;
    \item GCN \citep{gcn17kipf}: A type of GNN that leverages convolution function on a graph to propagate information within the neighborhood of each node;
    \item GraphSAGE \citep{graphsage17hamilton}: A type of GNN that uses sampling technique to aggregate features from the neighborhood;
    \item GAT \citep{gat18velickovic}: A type of GNN that adopts an attention mechanism to assign different importance to different nodes within the neighborhood of each node;
    \item GIN \citep{gin19xu}: A type of GNN that captures the properties of a graph while following graph isomorphism;
    \item HNN \citep{hnn18ganea}: A type of neural network that projects data features into non-Euclidean space;
    \item HGCN \citep{hgcn19chami}: A type of GNN that embeds node representations into non-Euclidean space and propagates accordingly;
    \item HYLA \citep{hyla23yu}: A type of GNN combines both laplacian characteristics within a graph and the information from non-Euclidean space.
\end{itemize}
The second group is specialized models:
\begin{itemize}[topsep=0.5mm, partopsep=0pt, itemsep=0pt, leftmargin=10pt]
    \item AMMNet \citep{amnet22chai}: A method proposed to capture both low- and high-frequency spectral information to detect anomalies;
    \item BWGNN \citep{bwgnn22tang}: A method designed to handle the 'right-shift' phenomenon of graph anomalies in spectral space;
    \item GDN \citep{gdn23gao}: A method that aims to learn information from a graph of the dependence relationships between sensors;
    \item SparseGAD \citep{sparsegad23gong}: A method that leverages sparsification to mitigate the heterophily issues within the neighborhood of each node;
    \item GHRN \citep{ghrn23gao}: A method that tackles the heterophily problem in the spectral space of graph anomaly detection;
    \item GAGA \citep{gaga23wang}: A method that uses group aggregation to reduce the influence of low homophily;
    \item XGBGraph \citep{gadbench23tang}: A method that combines XGB and GIN to boost the expressiveness;
    \item CONSISGAD \citep{consisgad24chen}: A method that applies a pseudo-label generation technique to solve the limited supervision problem;
\end{itemize}

\section{Algorithm}
\label{subsec:algorithm}
\IncMargin{1em}
\vspace{-2mm}
\begin{algorithm}

\caption{$exp_{\vect{o}}^{\kappa}$/$log_{\vect{o}}^{\kappa}$}\label{alg:exp-log}
\KwIn{$G$}
\KwOut{$\vect{H}^\kappa$}
%$\hat{\kappa} \leftarrow $\;
$\hat{\vect{H}} \leftarrow \text{NORMALIZE}(\vect{H})$\;
\If {$\kappa<0$} {
    $\vect{H}^\kappa \leftarrow \frac{tanh(\sqrt{|\kappa|}\hat{\vect{H}})\vect{H}}{\sqrt{|\kappa|}\hat{\vect{H}}}$ if $exp_{\vect{o}}^{\kappa}$, else  $\frac{arctanh(\sqrt{|\kappa|}\hat{\vect{H}})\vect{H}}{\sqrt{|\kappa|}\hat{\vect{H}}}$\;
}
\ElseIf {$\kappa>0$} {
    $\vect{H}^\kappa \leftarrow \frac{tan(\sqrt{|\kappa|}\hat{\vect{H}})\vect{H}}{\sqrt{|\kappa|}\hat{\vect{H}}}$  if $exp_{\vect{o}}^{\kappa}$, else  $\frac{arctan(\sqrt{|\kappa|}\hat{\vect{H}})\vect{H}}{\sqrt{|\kappa|}\hat{\vect{H}}}$\;
}
\Else{
    $\vect{H}^\kappa \leftarrow \vect{H}$\;    
}

Return $\vect{H}^\kappa$\;

\end{algorithm}
\vspace{-2mm}
\DecMargin{1em}
\IncMargin{1em}
\vspace{-2mm}
\begin{algorithm}

\caption{$CLAMP_{\kappa}$}\label{alg:clamp}
\KwIn{$G$}
\KwOut{$\vect{H}^\kappa$}
$\hat{\vect{H}} \leftarrow \text{NORMALIZE}(\vect{H})$\;
$\epsilon\leftarrow 1^{-8}$\;
$\tau \leftarrow \frac{1-\epsilon}{\sqrt{|\kappa|}}$\;
\For{$i=1$ to $n$}{
    \For{$j=1$ to $d$} {
        \If {$\hat{\vect{H}}_{ij}>\tau$} {
            $\vect{H}^\kappa_{ij}\leftarrow \frac{\vect{H}_{ij}}{\tau\hat{\vect{H}}_{ij}}$
        }
        \Else {
            $\vect{H}^\kappa_{ij}\leftarrow \vect{H}_{ij}$
        }
    }
}

Return $\vect{H}^\kappa$\;

\end{algorithm}
\vspace{-2mm}
\DecMargin{1em}
\usepackage[inkscapearea=page]{svg}

\usepackage{mathtools}
\usepackage{hyperref}
\usepackage{amssymb}
\usepackage{bm}
\usepackage{graphicx}
\usepackage{float}
\usepackage{amsmath}
% \usepackage{algpseudocode}
% \usepackage{algpseudocode}
\usepackage{braket}
\usepackage{tikz}
\usepackage{bbm}
\usepackage{multirow}
\usetikzlibrary{positioning, backgrounds, fit, shapes.arrows}
\usetikzlibrary{decorations.markings}
\usepackage{amsthm}
\usepackage{cuted}
\usepackage{enumitem}
\usepackage{caption}
\usepackage{subcaption}
\newtheorem{definition}{Definition}[section]
\newtheorem{theorem}{Theorem}[section]
\newtheorem{corollary}{Corollary}[theorem]
\newtheorem{lemma}[theorem]{Lemma}
\newtheorem{prop}{Proposition}
\newtheorem*{problem}{Problem Statement}
\newtheorem{assumption}{Assumption}[section]
% \usepackage[table]{xcolor}
% \usepackage{booktabs}
% \usepackage{multirow}
% \usepackage{tcolorbox}
% \usepackage{bbding}
% \usepackage{amssymb}% http://ctan.org/pkg/amssymb
\usepackage{pifont}% http://ctan.org/pkg/pifont
\IncMargin{1em}
\vspace{-2mm}
\begin{algorithm}[H]
\caption{SpaceGNN}\label{alg:spacegnn}
\KwIn{$G$, $L$}
\KwOut{$\vect{Z}$}
$\vect{Z}^{\vect{\kappa}^+}\leftarrow f_{\vect{\kappa}^+}^L(G)$, $\vect{Z}^{\vect{\kappa}^-}\leftarrow f_{\vect{\kappa}^-}^L(G)$, $\vect{Z}^{\vect{0}}\leftarrow f_{\vect{0}}^L(G)$\;
$\vect{Z}\leftarrow (1-\beta)((1-\alpha)\vect{Z}^{\vect{\kappa}^-}+\alpha\vect{Z}^{\vect{\kappa}^+})+\beta\vect{Z}^{\vect{0}}$\;
Return $\vect{Z}$\;

\end{algorithm}
\vspace{-2mm}
\DecMargin{1em}
We provide the detailed algorithm in this Section. In Algorithm \ref{alg:exp-log}, we calculate $exp_{\vect{o}}^\kappa(\cdot)$ and $log_{\vect{o}}^\kappa(\cdot)$ based on the original point $\vect{o}$ of the space with curvature $\kappa$. Besides, to satisfy the range of $log_{\vect{o}}^\kappa(\cdot)$, we utilized Algorithm \ref{alg:clamp} to prune the node representations. Moreover, we construct Algorithm \ref{alg:base} by utilizing Algorithms \ref{alg:exp-log} and \ref{alg:clamp}. Specifically, we use the approximated distance to calculate the similarities between nodes and their neighbors, and then leverage them as the corresponding coefficients during the propagation process. Notice, for each layer $l$ during the propagation, we assign a different learnable $\kappa^l$ to capture comprehensive information from different spaces. To simplify our architecture, we only use three base models, $f_{\vect{\kappa}^+}^L$, $f_{\vect{\kappa}^-}^L$, and $f_{\vect{0}}^L$, for constructing SpaceGNN. This simplification can reduce the running time cost, and allow us to investigate the effectiveness of different spaces on different datasets easily through the corresponding hyperparameters. After obtaining probability matrix $\vect{Z}$ from Algorithm \ref{alg:spacegnn}, we use the cross-entropy loss to update the framework. 

\section{Experimental Settings}
\label{subsec:setting}
We study (stochastic) gradient descent on the empirical risk
\begin{equation*}
\cL(w) = \frac{1}{n}\sum_{i=1}^n l(p_i(w))\, ,
\end{equation*}
where the loss function $l$ and the functions  $(p_i)_{i=1}^n$  are specified in the following assumptions. Note that the empirical risk for binary classification from Equation~\eqref{def:emp_risk_intro} is a special case of the above objective.

\begin{assumption}\label{hyp:loss_exp_log}\phantom{=}
  \begin{enumerate}[label=\roman*)]
    \item The loss is either the exponential loss, $l(q) = e^{-q}$, or the logistic loss, $l(q) = \log(1{+}e^{-q})$.
    \item There exists an integer $L \in \mathbb{N}^*$  such that, for all $1 \leq i \leq n$, the function $p_i$ is $L$-homogeneous\footnote{We recall that a mapping $f : \mathbb{R}^d \rightarrow \mathbb{R}$ is positively $L$-homogeneous if $f(\lambda w) = \lambda^L f(w)$ for all $w \in \mathbb{R}^d$ and $\lambda >0$.}, locally Lipschitz continuous and semialgebraic.
  \end{enumerate}
\end{assumption}
If the $p_i$'s were differentiable with respect to $w$, the chain rule would guarantee that
\begin{align*}
\nabla \mathcal{L}(w) = \frac{1}{n}\sum_{i=1}^n  l'(p_i(w)) \nabla p_i(w)\enspace.
\end{align*}
However, we only assume that the $p_i$'s are semialgebraic. While we could consider Clarke subgradients, the Clarke subgradient of operations on functions (e.g., addition, composition, and minimum) is only contained within the composition of the respective Clarke subgradients. This, as noted in Section~\ref{sec:cons_field}, implies that the output of backpropagation is usually not an element of a Clarke subgradient but a selection of some conservative set-valued field.
Consequently, for $1\leq i \leq n$, we consider $D_i : \bbR^d \rightrightarrows\bbR^d$, a conservative set-valued field of $p_i$, and a function $\sa_i : \bbR^d \rightarrow \bbR^d$ such that for all $w \in \bbR^d$, $\sa_i(w) \in D_i(w)$. Given a step-size $\gamma >0$, gradient descent (GD)\footnote{More precisely, this refers to conservative gradient descent. We use the term GD for simplicity, as conservative gradients behave similarly to standard gradients.} is then expressed as
\begin{equation*}\label{eq:gd_new}\tag{GD}
  w_{k+1} = w_k - \frac{\gamma}{n} \sum_{i=1}^n l'(p_i(w_k))\sa_i(w_k)\,.
\end{equation*}
For its stochastic counterpart, stochastic gradient descent (SGD), we fix a batch-size $1\leq n_b \leq n$. At each iteration $k \in \bbN$, we randomly and uniformly draw a batch $B_k \subset \{1, \ldots, n \}$ of size $n_b$. The update rule is then given by 
\begin{equation*}\label{eq:sgd_new}\tag{SGD}
  w_{k+1} = w_k -  \frac{\gamma}{n_b}\sum_{i\in B_k} l'(p_i(w_k)) \sa_i(w_k)\, .
\end{equation*}
The considered conservative set-valued fields will satisfy an Euler lemma-type assumption.
%\nic{Smoother transition}
\begin{assumption}\phantom{=}\label{hyp:conserv}
  For every $i \leq n$, $\sa_i$ is measurable and $D_i$ is semialgebraic. Moreover, for every $w \in \bbR^d$ and $\lambda \geq 0$, $\sa_i(w)  \in D_i(w)$,
  \begin{equation*}
    D_i(\lambda w) = \lambda^{L-1} D_i(w)\, , \textrm{ and } \quad   L p_i(w) = \scalarp{\sa_i(w)}{w}\, .
  \end{equation*}
\end{assumption}
%\nic{Smoother transition}
Having in mind the binary classification setting, in which $p_i(w) = y_i \Phi(x_i, w)$, we define the margin
\begin{equation}\label{def:marg}
  \sm: \bbR^d \rightarrow \bbR, \quad \sm(w) = \min_{1\leq i \leq n} p_i(w)\, .
\end{equation}
It quantifies the quality of a prediction rule $\Phi(\cdot, w)$. In particular,  the training data is perfectly separated when $\sm(w) >0$. A binary prediction for $x$ is given by the sign of $\Phi(x, w)$, and under the homogeneity assumption, it depends only on the normalized direction $w / \norm{w}$. Consequently, we will focus on the sequence of directions $u_k := w_k / \norm{w_k}$. Our final assumption ensures that the normalized directions $(u_k)$ have stabilized in a region where the training data is correctly classified.

\begin{assumption}\label{hyp:marg_lowb}
  Almost surely, $\liminf \sm(u_k) >0$.
\end{assumption}
Before presenting our main result, we comment on our assumptions.

\paragraph{On Assumption~\ref{hyp:loss_exp_log}.} As discussed in the introduction, the primary example we consider is when $p_i(w) = y_i \Phi(x_i;w)$ is the signed prediction of a feedforward neural network without biases and with piecewise linear activation functions on a labeled dataset $((x_i,y_i))_{i \leq n}$. In this case,
\begin{equation}\label{eq:NN}
 p_i(w) = y_i \Phi(w;x_i) = y_i V_L(W_L) \sigma(V_{L-1}(W_{L-1}) \sigma(V_{L-1}(W_{L-2}) \ldots \sigma(V_{1}(W_1 x_i))))\, ,
\end{equation}
where $w = [W_1, \ldots, W_L]$, $W_i$ represents the weights of the $i$-th layer, $V_i$ is a linear function in the space of matrices (with $V_i$ being the identity for fully-connected layers) and $\sigma$ is a coordinate-wise activation function such as $z \mapsto \max(0,z)$ ($\ReLU$), $z \mapsto \max(az, z)$ for a small parameter $a>0$ (LeakyReLu) or $z \mapsto z$. Note that the mapping $w \mapsto p_i(w)$ is semialgebraic and $L$-homogeneous for any of these activation functions. Regarding the loss functions, the logistic and exponential losses are among the most commonly studied and widely used. In Appendix~\ref{app:gen_sett}, we extend our results to a broader class of losses, including $l(q) = e^{-q^a}$ and $l(q) = \ln (1 + e^{-q^a})$ for any $a \geq 1$.

\paragraph{On Assumption~\ref{hyp:conserv}.} Assumption~\ref{hyp:conserv} holds automatically  if $D_i$ is the Clarke subgradient of $p_i$. Indeed, at any vector $w \in \bbR^d$, where $p_i$ is differentiable it holds that $p_i(\lambda w) = \lambda^{L} p_i(w)$. Differentiating relatively to $w$ and $\lambda$ (noting that $p_i$ remains differentiable at $\lambda w$ due to homogeneity), we obtain $\lambda \nabla p_i(\lambda w) = \lambda^{L} \nabla p_i(w)$ and $\scalarp{\nabla p_i(\lambda w)}{w} = L \lambda^{L-1} p_i(w)$. The expression for any element of the Clarke subgradient then follows from~\eqref{eq:def_clarke}. 

However, for an arbitrary conservative set-valued field, Assumption~\ref{hyp:conserv} does not necessarily hold. For instance, $D(x) = \mathds{1}(x \in \mathbb{N})$ is a conservative set-valued field for $p \equiv 0$, which does not satisfy Assumption~\ref{hyp:conserv}. Nevertheless, in practice, conservative set-valued fields naturally arise from a formal application of the chain rule. For a non-smooth but homogeneous activation function $\sigma$, one selects an element $e \in \partial \sigma (0)$, and computes $\sa_i(w)$ via backpropagation. Whenever a gradient candidate of $\sigma$ at zero is required (i.e., in~\eqref{eq:NN}, for some $j$, $V_j(W_j)$ contains a zero entry), it is replaced by $e$. 
Since $V_j(W_j)$ and $V_j(\lambda W_j)$ have the same zero elements, it follows that for every such $w$, $
\sa_i(\lambda w) = \lambda^L \sa_i(w)$. The conservative set-valued field $D_i$ is then obtained by associating to each $w$ the set of all possible outcomes of the chain rule, with $e$ ranging over all elements of $\partial \sigma(0)$. Thus, for such fields, Assumption~\ref{hyp:conserv} holds.


\paragraph{On Assumption~\ref{hyp:marg_lowb}.} Training typically continues even after the training error reaches zero.
Assumption~\ref{hyp:marg_lowb} characterizes this late-training phase, where our result applies. 
As noted earlier, since $\sm$ is $L$-homogeneous, the classification rule is determined by the direction of the  iterates $u_k=w_k/\norm{w_k}$. Assumption~\ref{hyp:marg_lowb} then states that, beyond some iteration, the normalized margin remains positive. 
This assumption is natural in the context of studying the implicit bias of SGD: we \emph{assume} that we reached the phase in which the dataset is correctly classified and \emph{then} characterize the limit points. A similar perspective was taken in  \cite{nacson2019lexicographic}, where the implicit bias of GF was analyzed under the assumption that the sequence of directions and the loss converge. However, unlike their approach, ours does not require assuming such convergence a priori.

Earlier works such as \cite{ji2020directional,Lyu_Li_maxmargin}, which analyze subgradient flow or smooth GD, establish convergence by assuming the existence of a single iterate $w_{k_0}$ satisfying $\sm(w_{k_0}) > \varepsilon$ and then proving that $\lim \sm(u_{k}) > 0$. Their approach relies on constructing a smooth approximation of the margin, which increases during training, ensuring that $\sm(u_k) > 0$ for all iterates with $k \geq k_0$. This is feasible in their setting, as they study either subgradient flow or GD with smooth $p_i$’s, allowing them to leverage the descent lemma.

In contrast, our analysis considers a nonsmooth and stochastic setting, in which, even if an iterate $w_{k_0}$ satisfying $\sm(w_{k_0}) > \varepsilon$ exists, there is no a priori assurance that subsequent iterates remain in the region where Assumption~\ref{hyp:marg_lowb} holds. From this perspective, Assumption~\ref{hyp:marg_lowb} can be viewed as a stability assumption, ensuring that iterates continue to classify the dataset correctly. Establishing stability for stochastic and nonsmooth algorithms is notoriously hard, and only partial results in restrictive settings exist \cite{borkar2000ode,ramaswamy2017generalization,josz2024global}.

%Finally, note that Assumption~\ref{hyp:marg_lowb} only needs to hold almost surely. Specifically, with probability 1, there exist $k_0$ and $\varepsilon$ such that for all $k \geq k_0$, $\sm(u_k) \geq \varepsilon > 0$. In the case of~\eqref{eq:sgd_new}, $k_0$ and $\delta$ are random variables and may take different values across different realizations. 

%\paragraph{On constant stepsizes.}
%We allow the step size to be a constant of arbitrary magnitude, subject to the stability Assumption~\ref{hyp:marg_lowb}. This may seem surprising in a nonsmooth and stochastic setting, where a vanishing step size is typically required to ensure convergence (see, e.g., \cite{majewski2018analysis, dav-dru-kak-lee-19, bolte2023subgradient, le2024nonsmooth}).
\begin{figure}[t]
\centering
  \begin{small}
    \begin{tabular}{cc}
        \multicolumn{2}{c}{\includegraphics[height=4mm]{figures/hyperparameter1/legend1.eps}}  \\ [-3mm]
        \hspace{-4mm}
        \includegraphics[height=33mm]{figures/hyperparameter1/hdim.eps} &
        \hspace{-4mm}
        \includegraphics[height=33mm]{figures/hyperparameter1/layer.eps} \\ [-2mm]
        \hspace{-2mm}
        (a) Varying Hidden Dimension & 
        \hspace{-2mm}
        (b) Varying Layer \\ 
    \end{tabular}
    \vspace{-2mm}
    \caption{Varying the Hidden Dimension and Layer.}
    \label{fig:hyperparameter1}
    \vspace{-4mm}
  \end{small}
\end{figure}
\begin{figure}[t!]
\centering
  \begin{small}
  
  \vspace{-4mm}
    \begin{tabular}{ccc}
        %\multicolumn{3}{c}{\includegraphics[height=10mm]{figure/observation/mcf/mcf.eps}}  \\[-6mm]
        \hspace{-10mm}
        \includegraphics[height=42mm]{figures/hyperparameter2/Amazon.eps} &
        \hspace{-10mm}
        \includegraphics[height=42mm]{figures/hyperparameter2/T-Finance.eps} &
        \hspace{-10mm}
        \includegraphics[height=42mm]{figures/hyperparameter2/T-Social.eps} \\ [-0mm]
        \hspace{-9mm}
        (a) Amazon & 
        \hspace{-9mm}
        (b) T-Finannce &
        \hspace{-9mm}
        (c) T-Social \\ 
    \end{tabular}
    \vspace{-2mm}
    \caption{Varying $\alpha$ and $\beta$ on different datasets.}
    \label{fig:hyperparameter2}
  \vspace{-8mm}
  \end{small}
\end{figure}

Table \ref{tab:setting} provides a comprehensive list of our hyperparameters. We use grid search to train the model that yields the best F1 score on the validation set and report the corresponding test performance. Specifically, Learning Rate is searched from the set $\{0.001, 0.0001\}$, Hidden Dimension is chosen from the set $\{32, 64, 128, 256\}$, Layer ranges from $1$ to $6$, Dropout is obtained from the set $\{0, 0.05, 0.1\}$, Batch Size is fixed based on the size of the training set, and $\alpha$ and $\beta$ are from the set $\{0, 0.5, 1\}$, respectively. In the following Section \ref{subsec:parameter}, we will further analyze the influence of Hidden Dimension, Layer, and $\alpha$ and $\beta$ on the F1 scores of different datasets. 

\section{Parameter Analysis}
\label{subsec:parameter}


In this Section, we investigate the impact of Hidden Dimension, Layer, and $\alpha$ and $\beta$ on three different datasets, and present their F1 scores. 

Figure \ref{fig:hyperparameter1} reports the F1 score of SpaceGNN as we vary the Hidden Dimension from $32$ to $256$, and the Layer from $1$ to $6$. As we can observe, when we set the Hidden Dimension to $128$, SpaceGNN achieves relatively satisfactory performances on these three datasets. When we vary the Layer, we find for different datasets, the optimal value can be different. Specifically, we set it to $3$ for Amazon and T-Finance, and $6$ for T-Social to get the best performance. 

Figure \ref{fig:hyperparameter2} reports the F1 score of SpaceGNN as we vary $\alpha$ and $\beta$ from $0$ to $1$. These two hyperparameters represent the influence of diverse spaces on the datasets, so for different datasets, the optimal value will be distinct. Specifically, we set $\alpha$ to $0$ for Amazon, $1$ for T-Finance, and $0.5$ for T-Social, and we set $\beta$ to $0$ for Amazon, $1$ for T-Finance and T-Social to get the satisfactory performance. 

\section{Ablation Study}
\label{subsec:ablation}



% Table generated by Excel2LaTeX from sheet 'abaltion'
\begin{table}[t]
  \centering
  \resizebox{\linewidth}{!}{
    \begin{tabular}{rccccc}
    \toprule
    Method & ID & CAG   & PopQA & WebQuestion & \multirow{2}[2]{*}{Avg}\\
        & F1    & EM    & EM    & EM    &  \\
    \midrule
    \multicolumn{1}{l}{DeepRAG} & \textbf{52.40} & \textbf{61.92 } & \textbf{47.80 } & \textbf{45.24 } & \textbf{47.67 } \\
    all-node & 50.92   & 50.47  & 41.50  & 32.70  & 45.30  \\
    sentence-wise & 30.16   & 12.46  & 20.00  & 12.90  & 21.14 \\
    \bottomrule
    \end{tabular}%
    }
  \caption{Experiment results of the ablation study on the Chain of Calibration Stage.}
  \label{tab:dpo-abla}%
\end{table}%


To investigate the usefulness of the LSP and DAP components, we provide the ablation study of them on 9 datasets in Table \ref{tab:ablation}. Specifically, we set the $\kappa$ as fixed values for different spaces during the w/o LSP experiment and set the $\hat{\vect{s}}_i$ as $\vect{1}$ for each node $i$ during the w/o DAP experiment. As shown in Table \ref{tab:ablation}, SpaceGNN consistently outperforms w/o LSP and w/o DAP by a large margin, which demonstrates the benefits of these two components. 

\section{Additional Experimental Results}
\label{subsec:Additional}
\begin{table}[t]
\caption{AUC and F1 scores (\%) on 9 datasets with random split, compared with generalized models, where OOM represents out-of-memory.}
\vspace{-2mm}
\small
\centering
\scalebox{1}{
\setlength\tabcolsep{3.5pt}
\label{tab:general10}
\begin{tabular}{cc|ccccccccc}
\hline \hline
Datasets                    & Metrics & MLP    & GCN    & SAGE   & GAT    & GIN    & HNN             & HGCN   & HYLA            & SpaceGNN        \\ \hline
\multirow{2}{*}{Weibo}      & AUC     & 0.2856 & 0.6223 & 0.6176 & 0.8077 & 0.4452 & 0.4844          & 0.8020 & \textbf{0.9351} & 0.8364          \\
                            & F1      & 0.5347 & 0.6215 & 0.4732 & 0.5127 & 0.5230 & 0.4727          & 0.4721 & 0.7008          & \textbf{0.7158} \\ \hline
\multirow{2}{*}{Reddit}     & AUC     & 0.5320 & 0.5865 & 0.5820 & 0.5421 & 0.5217 & 0.5266          & 0.5196 & 0.4734          & \textbf{0.5868} \\
                            & F1      & 0.4916 & 0.4916 & 0.4916 & 0.4916 & 0.4916 & 0.4916          & 0.4916 & 0.4916          & \textbf{0.4916} \\ \hline
\multirow{2}{*}{Tolokers}   & AUC     & 0.4632 & 0.5355 & 0.4119 & 0.6453 & 0.6030 & 0.5718          & 0.6495 & 0.4927          & \textbf{0.6952} \\
                            & F1      & 0.4375 & 0.5093 & 0.4423 & 0.5075 & 0.5621 & 0.5127          & 0.5522 & 0.5004          & \textbf{0.6026} \\ \hline
\multirow{2}{*}{Amazon}     & AUC     & 0.8550 & 0.7954 & 0.6162 & 0.5210 & 0.8254 & 0.7098          & 0.7468 & 0.7185          & \textbf{0.8722} \\
                            & F1      & 0.8261 & 0.6441 & 0.3725 & 0.3668 & 0.2256 & 0.4822          & 0.4822 & 0.5879          & \textbf{0.8641} \\ \hline
\multirow{2}{*}{T-Finance}  & AUC     & 0.9058 & 0.6796 & 0.6512 & 0.6327 & 0.7567 & 0.8761          & 0.0719 & 0.3917          & \textbf{0.9349} \\
                            & F1      & 0.7856 & 0.3068 & 0.4627 & 0.5293 & 0.7681 & \textbf{0.8204} & 0.4883 & 0.4919          & 0.8031          \\ \hline
\multirow{2}{*}{YelpChi}    & AUC     & 0.5064 & 0.4845 & 0.4975 & 0.5546 & 0.6082 & 0.3702          & 0.4745 & 0.5426          & \textbf{0.6191} \\
                            & F1      & 0.5064 & 0.4608 & 0.4980 & 0.5220 & 0.5332 & 0.4608          & 0.4608 & 0.4644          & \textbf{0.5475} \\ \hline
\multirow{2}{*}{Questions}  & AUC     & 0.5299 & 0.4684 & 0.5654 & 0.5554 & 0.5597 & 0.5177          & 0.5057 & 0.4049          & \textbf{0.5851} \\
                            & F1      & 0.4645 & 0.4924 & 0.4371 & 0.4923 & 0.4492 & \textbf{0.5089} & 0.5083 & 0.4924          & 0.4924          \\ \hline
\multirow{2}{*}{DGraph-Fin} & AUC     & 0.4356 & 0.3900 & 0.5794 & 0.4103 & 0.4088 & 0.3282          & 0.3322 & OOM             & \textbf{0.6515} \\
                            & F1      & 0.4531 & 0.4815 & 0.4994 & 0.4282 & 0.3787 & 0.4968          & 0.3312 & OOM             & \textbf{0.5030} \\ \hline
\multirow{2}{*}{T-Social}   & AUC     & 0.5377 & 0.6183 & 0.6948 & 0.6958 & 0.5554 & 0.4694          & 0.4297 & OOM             & \textbf{0.9019} \\
                            & F1      & 0.1373 & 0.2554 & 0.5421 & 0.5501 & 0.3733 & 0.4924          & 0.4923 & OOM             & \textbf{0.7320} \\ \hline \hline
\end{tabular}
}\vspace{-4mm}
\end{table}


\begin{table}[h]
\caption{AUC and F1 scores (\%) on 9 datasets with random split, compared with specialized models, where TLE represents the experiment can not be conducted successfully within 72 hours. }
\vspace{-2mm}
\small
\centering
\scalebox{0.91}{
\setlength\tabcolsep{2pt}
\label{tab:gad10}
\begin{tabular}{cc|ccccccccc}
\hline \hline
Datasets                    & Metrics & AMNet           & BWGNN  & GDN    & SparseGAD & GHRN            & GAGA   & XGBGraph & CONSISGAD       & SpaceGNN        \\ \hline
\multirow{2}{*}{Weibo}      & AUC     & 0.4206          & 0.7557 & 0.7751 & 0.4558    & 0.6349          & 0.7597 & 0.5660   & 0.3972          & \textbf{0.8364} \\
                            & F1      & 0.5328          & 0.7106 & 0.1664 & 0.4724    & 0.6379          & 0.6574 & 0.5061   & 0.4447          & \textbf{0.7158} \\ \hline
\multirow{2}{*}{Reddit}     & AUC     & \textbf{0.6002} & 0.5815 & 0.4436 & 0.5263    & 0.5513          & 0.5068 & 0.5030   & 0.5536          & 0.5868          \\
                            & F1      & 0.4365          & 0.4617 & 0.4916 & 0.4721    & 0.4093          & 0.4916 & 0.4916   & 0.4514          & \textbf{0.4916} \\ \hline
\multirow{2}{*}{Tolokers}   & AUC     & 0.5627          & 0.5725 & 0.6159 & 0.4792    & 0.5688          & 0.6327 & 0.6083   & 0.5843          & \textbf{0.6952} \\
                            & F1      & 0.4451          & 0.5312 & 0.4665 & 0.4388    & 0.5449          & 0.4388 & 0.4989   & 0.5258          & \textbf{0.6026} \\ \hline
\multirow{2}{*}{Amazon}     & AUC     & 0.8356          & 0.7702 & 0.8335 & 0.7249    & 0.8028          & 0.7795 & 0.7666   & 0.8435          & \textbf{0.8722} \\
                            & F1      & 0.7144          & 0.5834 & 0.1685 & 0.4822    & 0.6777          & 0.6674 & 0.4822   & 0.8475          & \textbf{0.8641} \\ \hline
\multirow{2}{*}{T-Finance}  & AUC     & 0.8302          & 0.7318 & 0.5899 & 0.3650    & 0.7895          & 0.8157 & 0.8570   & 0.8503          & \textbf{0.9349} \\
                            & F1      & 0.5692          & 0.5025 & 0.5568 & 0.4883    & 0.5652          & 0.4894 & 0.7406   & \textbf{0.8316} & 0.8031          \\ \hline
\multirow{2}{*}{YelpChi}    & AUC     & 0.4738          & 0.5058 & 0.4893 & 0.5190    & 0.4231          & 0.4671 & 0.4927   & 0.5927          & \textbf{0.6191} \\
                            & F1      & 0.4875          & 0.4614 & 0.4977 & 0.4608    & 0.4608          & 0.4919 & 0.4608   & 0.5403          & \textbf{0.5475} \\ \hline
\multirow{2}{*}{Questions}  & AUC     & 0.4971          & 0.4125 & 0.5094 & 0.5185    & 0.5062          & 0.5361 & 0.5122   & 0.5492          & \textbf{0.5851} \\
                            & F1      & 0.4843          & 0.4924 & 0.4855 & 0.4988    & \textbf{0.5125} & 0.4944 & 0.4924   & 0.4935          & 0.4924          \\ \hline
\multirow{2}{*}{DGraph-Fin} & AUC     & 0.3812          & 0.6343 & 0.3200 & 0.3346    & 0.3734          & TLE    & 0.5009   & 0.6469          & \textbf{0.6515} \\
                            & F1      & 0.4128          & 0.4909 & 0.2641 & 0.4970    & 0.4871          & TLE    & 0.4968   & 0.4224          & \textbf{0.5030} \\ \hline
\multirow{2}{*}{T-Social}   & AUC     & 0.4745          & 0.6408 & 0.5480 & 0.3317    & 0.6319          & TLE    & 0.5066   & 0.8614          & \textbf{0.9019} \\
                            & F1      & 0.4810          & 0.4487 & 0.5124 & 0.4923    & 0.3435          & TLE    & 0.4923   & 0.5890          & \textbf{0.7320} \\ \hline \hline
\end{tabular}
}\vspace{-4mm}
\end{table}
In addition to the experiments in Section \ref{sec:experiments}, we further compare our SpaceGNN with baseline models on datasets with different sizes of training/validation/testing sets. Specifically, in experiments of Tables \ref{tab:general10} and \ref{tab:gad10}, we randomly divide each dataset into 10/10 for training/validation, and the rest of the nodes for testing, and in experiments of Tables \ref{tab:general100} and \ref{tab:gad100}, we randomly divide each dataset into 100/100 for training/validation, and the rest of the nodes for testing. 

\begin{table}[t]
\caption{AUC and F1 scores (\%) on 9 datasets with random split, compared with generalized models, where OOM represents out-of-memory.}
\vspace{-2mm}
\small
\centering
\scalebox{1}{
\setlength\tabcolsep{3.5pt}
\label{tab:general100}
\begin{tabular}{cc|ccccccccc}
\hline \hline
Datasets                    & Metrics & MLP    & GCN    & SAGE   & GAT    & GIN    & HNN    & HGCN   & HYLA   & SpaceGNN        \\ \hline
\multirow{2}{*}{Weibo}      & AUC     & 0.4438 & 0.9066 & 0.8157 & 0.8401 & 0.8541 & 0.7243 & 0.8545 & 0.9159 & \textbf{0.9521} \\
                            & F1      & 0.6538 & 0.8444 & 0.4746 & 0.7941 & 0.6561 & 0.6545 & 0.7620 & 0.4965 & \textbf{0.8481} \\ \hline
\multirow{2}{*}{Reddit}     & AUC     & 0.5907 & 0.5725 & 0.5767 & 0.6001 & 0.4793 & 0.5330 & 0.5315 & 0.4714 & \textbf{0.6232} \\
                            & F1      & 0.4916 & 0.4916 & 0.4916 & 0.4916 & 0.4916 & 0.4916 & 0.4916 & 0.4916 & \textbf{0.5228} \\ \hline
\multirow{2}{*}{Tolokers}   & AUC     & 0.7050 & 0.6952 & 0.7101 & 0.7139 & 0.7067 & 0.7063 & 0.7115 & 0.6402 & \textbf{0.7140} \\
                            & F1      & 0.5427 & 0.5978 & 0.5776 & 0.5844 & 0.5835 & 0.4711 & 0.5510 & 0.4864 & \textbf{0.6040} \\ \hline
\multirow{2}{*}{Amazon}     & AUC     & 0.8647 & 0.7936 & 0.7748 & 0.8808 & 0.9186 & 0.8635 & 0.7719 & 0.7188 & \textbf{0.9428} \\
                            & F1      & 0.7273 & 0.6167 & 0.6310 & 0.4354 & 0.7272 & 0.7711 & 0.5620 & 0.4822 & \textbf{0.9069} \\ \hline
\multirow{2}{*}{T-Finance}  & AUC     & 0.8960 & 0.8916 & 0.6722 & 0.8647 & 0.8087 & 0.8768 & 0.9329 & 0.3982 & \textbf{0.9486} \\
                            & F1      & 0.5737 & 0.7507 & 0.6042 & 0.8025 & 0.7680 & 0.8384 & 0.8753 & 0.4883 & \textbf{0.8789} \\ \hline
\multirow{2}{*}{YelpChi}    & AUC     & 0.7113 & 0.5160 & 0.5217 & 0.7249 & 0.7052 & 0.7119 & 0.5642 & 0.5508 & \textbf{0.7321} \\
                            & F1      & 0.6081 & 0.4608 & 0.4838 & 0.6256 & 0.6164 & 0.5919 & 0.4833 & 0.4608 & \textbf{0.6256} \\ \hline
\multirow{2}{*}{Questions}  & AUC     & 0.4707 & 0.6130 & 0.6000 & 0.5847 & 0.5083 & 0.5098 & 0.5081 & 0.4055 & \textbf{0.6476} \\
                            & F1      & 0.4961 & 0.4924 & 0.4999 & 0.5020 & 0.4819 & 0.4923 & 0.4924 & 0.4924 & \textbf{0.5386} \\ \hline
\multirow{2}{*}{DGraph-Fin} & AUC     & 0.5752 & 0.6117 & 0.5487 & 0.6505 & 0.6408 & 0.3260 & 0.3298 & OOM    & \textbf{0.6545} \\
                            & F1      & 0.4820 & 0.4769 & 0.4225 & 0.5000 & 0.5037 & 0.4968 & 0.3321 & OOM    & \textbf{0.5097} \\ \hline
\multirow{2}{*}{T-Social}   & AUC     & 0.5896 & 0.7611 & 0.7268 & 0.6968 & 0.7248 & 0.5959 & 0.4245 & OOM    & \textbf{0.9428} \\
                            & F1      & 0.4126 & 0.5666 & 0.5549 & 0.5770 & 0.5433 & 0.4936 & 0.4898 & OOM    & \textbf{0.7828} \\ \hline \hline
\end{tabular}
}\vspace{-4mm}
\end{table}


\begin{table}[t]
\caption{AUC and F1 scores (\%) on 9 datasets with random split, compared with specialized models, where TLE represents the experiment can not be conducted successfully within 72 hours. }
\vspace{-2mm}
\small
\centering
\scalebox{0.91}{
\setlength\tabcolsep{2pt}
\label{tab:gad100}
\begin{tabular}{cc|ccccccccc}
\hline \hline
Datasets                    & Metrics & AMNet  & BWGNN  & GDN             & SparseGAD & GHRN   & GAGA   & XGBGraph        & CONSISGAD & SpaceGNN        \\ \hline
\multirow{2}{*}{Weibo}      & AUC     & 0.6902 & 0.8430 & 0.4353          & 0.8570    & 0.8286 & 0.8283 & 0.9496          & 0.7838    & \textbf{0.9521} \\
                            & F1      & 0.7156 & 0.7925 & 0.6680          & 0.6369    & 0.7918 & 0.7433 & 0.7431          & 0.7217    & \textbf{0.8481} \\ \hline
\multirow{2}{*}{Reddit}     & AUC     & 0.6011 & 0.5833 & 0.5840          & 0.4864    & 0.5823 & 0.4430 & 0.5518          & 0.5704    & \textbf{0.6232} \\
                            & F1      & 0.4916 & 0.4916 & 0.4916          & 0.4916    & 0.4916 & 0.4916 & 0.4909          & 0.4916    & \textbf{0.5228} \\ \hline
\multirow{2}{*}{Tolokers}   & AUC     & 0.6939 & 0.7100 & \textbf{0.7325} & 0.6879    & 0.7197 & 0.4817 & 0.6804          & 0.7088    & 0.7140          \\
                            & F1      & 0.5910 & 0.5943 & 0.5834          & 0.4711    & 0.5871 & 0.2794 & 0.5954          & 0.5932    & \textbf{0.6040} \\ \hline
\multirow{2}{*}{Amazon}     & AUC     & 0.8812 & 0.8742 & 0.8939          & 0.7263    & 0.8843 & 0.7476 & 0.9124          & 0.9325    & \textbf{0.9428} \\
                            & F1      & 0.8768 & 0.8893 & 0.8732          & 0.5939    & 0.8286 & 0.7224 & 0.8607          & 0.8990    & \textbf{0.9069} \\ \hline
\multirow{2}{*}{T-Finance}  & AUC     & 0.7774 & 0.8907 & 0.7863          & 0.9247    & 0.8982 & 0.8387 & 0.9407          & 0.9359    & \textbf{0.9486} \\
                            & F1      & 0.7528 & 0.7726 & 0.7779          & 0.4883    & 0.7805 & 0.5482 & \textbf{0.8831} & 0.8722    & 0.8789          \\ \hline
\multirow{2}{*}{YelpChi}    & AUC     & 0.7201 & 0.7022 & 0.7165          & 0.5504    & 0.6974 & 0.5107 & 0.7239          & 0.7152    & \textbf{0.7321} \\
                            & F1      & 0.6202 & 0.6119 & 0.6161          & 0.4608    & 0.5575 & 0.4608 & 0.6232          & 0.6187    & \textbf{0.6256} \\ \hline
\multirow{2}{*}{Questions}  & AUC     & 0.5959 & 0.5939 & 0.4870          & 0.5080    & 0.5931 & 0.5088 & 0.5217          & 0.5652    & \textbf{0.6476} \\
                            & F1      & 0.5145 & 0.4992 & 0.4936          & 0.4955    & 0.4993 & 0.4924 & 0.4926          & 0.5174    & \textbf{0.5386} \\ \hline
\multirow{2}{*}{DGraph-Fin} & AUC     & 0.5392 & 0.5887 & 0.5407          & 0.3418    & 0.6012 & TLE    & 0.5080          & 0.5108    & \textbf{0.6545} \\
                            & F1      & 0.5043 & 0.5073 & 0.4968          & 0.4487    & 0.4973 & TLE    & 0.4974          & 0.5058    & \textbf{0.5097} \\ \hline
\multirow{2}{*}{T-Social}   & AUC     & 0.5114 & 0.7544 & 0.4846          & 0.7199    & 0.6353 & TLE    & 0.7381          & 0.9192    & \textbf{0.9428} \\
                            & F1      & 0.4653 & 0.5731 & 0.4697          & 0.4924    & 0.4923 & TLE    & 0.5547          & 0.7170    & \textbf{0.7828} \\ \hline \hline
\end{tabular}
}
%\vspace{-4mm}
\end{table}

In Tables \ref{tab:general10} and \ref{tab:gad10}, we can observe that our SpaceGNN can consistently surpass both generalized and specialized models on 9 datasets. In short, SpaceGNN outperforms the best rival 10.84\% and 5.46\% on average in terms of AUC and F1 scores, respectively. 

Similarly, in Tables \ref{tab:general100} and \ref{tab:gad100}, it is easy to find out that SpaceGNN is able to beat both generalized and specialized models on 9 datasets. In summary, compared with the best rival, SpaceGNN takes a lead by 4.98\% and 3.02\% on average in terms of AUC and F1 scores, separately. 

\section{Alternative Model}
\label{subsec:alternative}







 
\documentclass[pdflatex,sn-mathphys-num]{sn-jnl}%


\usepackage{graphicx}%
\usepackage{multirow}%
\usepackage{amsmath,amssymb,amsfonts}%
\usepackage{amsthm}%
\usepackage{mathrsfs}%
\usepackage[title]{appendix}%
\usepackage{xcolor}%
\usepackage{textcomp}%
\usepackage{manyfoot}%
\usepackage{booktabs}%
\usepackage{algorithm}%
\usepackage{algpseudocode}%
\usepackage{listings}%
\usepackage{lmodern}
\usepackage{bm}
\usepackage{placeins}



\DeclareMathOperator*{\argmax}{arg\,max}
\DeclareMathOperator*{\argmin}{arg\,min}
\renewcommand{\thefootnote}{\fnsymbol{footnote}}
\theoremstyle{definition}
\newtheorem{prob}{Problem}

\begin{document}

\title[Summarising local explanations via proxies]{{\sc ExplainReduce}: Summarising local explanations via proxies}



\author*[1]{\fnm{Lauri} \sur{Sepp\"al\"ainen}}\email{lauri.seppalainen@helsinki.fi}

\author[1]{\fnm{Mudong} \sur{Guo}}\email{mudong.guo@helsinki.fi}

\author[1]{\fnm{Kai} \sur{Puolam\"aki}}\email{kai.puolamaki@helsinki.fi}

\affil[1]{%
\orgname{University of Helsinki}, \orgaddress{\street{P.O. Box 64}, \city{Helsinki}, \postcode{00014}, \country{Finland}}}



\abstract{Most commonly used non-linear machine learning methods are closed-box models, uninterpretable to humans. The field of explainable artificial intelligence (XAI) aims to develop tools to examine the inner workings of these closed boxes. An often-used model-agnostic approach to XAI involves using simple models as local approximations to produce so-called local explanations; examples of this approach include {\sc lime},  {\sc shap}, and {\sc slisemap}. This paper shows how a large set of local explanations can be reduced to a small ``proxy set'' of simple models, which can act as a generative global explanation. This reduction procedure, {\sc ExplainReduce}, can be formulated as an optimisation problem and approximated efficiently using greedy heuristics.}



\keywords{Explainable artificial intelligence, XAI, local explanations, interpretability, machine learning}



\maketitle

\section{Introduction}\label{sec:intro}
Explainable artificial intelligence (XAI) aims to elucidate the inner workings of ``closed-box'' machine learning (ML) models: models that are not readily interpretable to humans.
As machine learning has found applications in almost all fields, the need for interpretability has likewise led to the use of XAI in medicine \cite{band2023medical}, manufacturing \cite{peng2022industrial} and atmospheric chemistry \cite{seppalainen2023using}, among many other domains.
In the past two decades, many different XAI methods have been developed to meet the diverse requirements \cite{guidotti2018survey}.
These methods produce explanations, i.e., distillations of a closed-box model's decision patterns.
An ideal XAI method should be model-agnostic -- applicable to a wide range of model types -- and its explanations would be succinct, easily interpretable by the intended user, and stable.
Additionally, such explanations would be global, allowing the user to comprehend the entire mechanism of the model.
However, producing global explanations is often a challenge.
For example, if a model approximates a complex function, describing its behaviour may require describing the complex function itself, thereby defeating the purpose of interpretability.

A common approach to producing model-agnostic explanations is to relax the requirement for explaining the model globally and instead focus on local behaviour \cite{guidotti2018survey}.
Assuming a degree of smoothness, approximating the closed-box function in a small neighbourhood is often feasible using simple, interpretable functions, such as sparse linear models, decision trees, or decision rules.

We argue that these local explanations are inherently unstable.
We initially observed this phenomenon with {\sc slisemap} \cite{bjorklund2023slisemap}, where we noted that most items could be accurately approximated with several different local models.
In the same vein, take two commonly used local explanation methods, {\sc lime} \cite{ribeiro2016} and {\sc shap} \cite{lundberg2017unified}.
Both methods estimate feature importance locally by averaging predictions from the closed-box model for items sampled near the one being explained.
Such explanations can be interpreted as linear approximations for the gradient of the closed-box model.
It is well known that the loss landscapes of, e.g., deep neural network models, are not smooth.
Therefore, it can be conjectured that the predictions from closed-box can be likewise unstable.
Indeed, research has shown that local explanations of neural networks can be manipulated due to the geometry of the closed-box function \cite{dombrowski2019explanations}.
Furthermore, numerous variants of {\sc lime} which aim to increase stability (such as \cite{shankar2019alime, zafar2019dlime}) indicate that instability of {\sc lime} is a key issue to address.
We hypothesise that this instability is a property of all local explanation methods that use simple models to approximate a complex function.
Even in the theoretically ideal case, where a complex model is defined by an exact mathematical formula and the local explanations take the form of gradients, there can be points (e.g., sharp peaks) where the gradient is ill-defined, as in Fig. \ref{fig:pyramid_example}.
In practice, especially when working with noisy data and averaging over random samples, the ambiguity is exacerbated; one item may have several nearly equally viable explanations.
At the same time, some local explanations may accurately approximate many items.
This suggests that if we generate a large set of local explanations, we may be able to find a smaller subset thereof, which could effectively replace the full set without sacrificing accuracy.
As an added bonus, while the local explanations may be unstable, this subset of explanations may be more stable, as observed in \cite{seppalainen2023using}.

This paper introduces a procedure, coined {\sc ExplainReduce}, that can reduce large sets of local explanations to a small subset of so-called proxy models.
This small proxy set can be used as a global explanation for the closed-box model, among other possible applications.
Fig. \ref{fig:pyramid_example} provides insight into the process.
In the left panel, we show noisy samples (blue dots) from a closed-box function (marked with a solid blue line).
The middle panel shows a set of local explanations (as blue and orange dashed lines) produced by the XAI method {\sc slisemap} \cite{bjorklund2023slisemap}, with one explanation per data item.
The right panel shows a reduction from this large set of local models to two (solid lines), maximising the coverage of items within an error tolerance (shaded area).
The procedure offers a trade-off between interpretability, coverage, and fidelity of the local explanation model by finding a minimal subset of local models that can still approximate the closed-box model for most points with reasonable accuracy.

We start with an overview of potential applications of local explanation subsets and introduce related research.
We then define the problem of reducing a large set of local models to a proxy set as an optimisation problem.
We also outline the {\sc ExplainReduce} algorithm and our performance metrics.
In the results section, we first show how a small proxy set provides a global explanation for the closed-box model with simulated data and a practical example.
Second, we find that a proxy set of modest size attains an adherence to the closed-box model on unseen data comparable to, or even better than, the complete set of explanations.
We continue to show that we can find a good proxy set even when starting from a limited set of initial explanations.
We then demonstrate how greedy approximation algorithms can efficiently solve the problem of finding the proxy set.
The code for the procedure and to recreate each of the experiments presented in this paper can be found at \url{https://github.com/edahelsinki/explainreduce}.

\begin{figure}
    \centering
    \includegraphics[width=\textwidth]{figures/pyramid_example.pdf}
    \caption{A simple example of the idea behind {\sc ExplainReduce}. A closed-box model (left) can have many local explanations (middle). We can reduce the size of the local explanation set to get a global explanation consisting of two simple models (right).}
    \label{fig:pyramid_example}
\end{figure}

\section{Applications for reduced local explanation sets}

\emph{Global explanations}:
Producing succinct global explanations for complex closed-box functions remains a challenge.
Given a local explanation method, a naive approach would be simply producing a local explanation for each item in a given training set.
However, this leaves the user with $n$ local explanations without guaranteeing global interpretability.
In practical settings, many of these local explanations are also likely to be similar and thus redundant.
Summarising the large set of local explanations with a reduced set of just a few local models would be much more palatable to the user as a global explanation.

\noindent\emph{Interpretable replacement for the closed-box model}:
Assuming that the method used to produce local explanations is faithful, i.e., able to accurately predict the behaviour of the closed-box function locally, the set of all local explanations can replace the closed-box model with little loss to accuracy.
However, a large set of (e.g., $500$) local models can hardly be called interpretable.
If there is sufficient redundancy in the explanations, selecting a small number of representative models yields a global explanation and provides an interpretable surrogate for the closed-box model.

\noindent\emph{Exploratory data analysis and XAI method comparison}:
Studying how the set of local explanations can be reduced can offer interesting insights about the data and XAI method used.
If, for example, different subsets of the data are consistently associated with very similar proxy sets, it implies that the closed-box model can be replaced with a reasonable loss of accuracy by a small set of simple functions.
Similarly, comparing the reduced proxy sets with individual explanations from different XAI methods could provide new insights into the explanation generation mechanisms in various settings.

\noindent\emph{Outlier detection}:
Given a novel observation, we can study how well the local models in the reduced set can predict it compared to similar items in the training dataset.
If the novel item is better explained by a different model than the one used for other similar items, or if no explanation in the reduced set accurately captures the relationship, the novel item could be considered an outlier.

\section{Related work}


Modern machine learning tools have become indispensable in almost all industry and research fields.
As these tools find more and more applications, awareness of their limitations has simultaneously spread.
Chief among these is the lack of interpretability inherent to many of the most potent ML methods.
Understanding why ML models produce a specific prediction instead of something else poses obstacles in their adoption in, e.g., high-trust applications \cite{dosilovic2018survey}.
Such opaque methods are colloquially called black-box or closed-box methods, in contrast with white- or open-box methods, characterised by their interpretability to humans.
Examples of closed-box methods include random forests and deep neural networks, whereas, e.g., statistical models like linear regression models are often considered open-box methods.
As a result, the field of XAI has grown substantially in the last decade.
This section describes several post-hoc local explanation and model aggregation methods to give the reader context for the {\sc ExplainReduce} procedure.

XAI methods generally take a dataset and a closed-box function as inputs and produce an explanation that describes the relationship between the inputs and outputs of the closed-box model.
One commonly used approach is to produce post-hoc explanations using local (open-box) surrogate models.
In this approach, given a closed-box method $f: \mathcal{X} \rightarrow \mathcal{Y}, f(\bm{X}) = \hat{\bm{y}}$, a local surrogate model $g$ is another function that replicates the behaviour of the closed-box method in some region of the input space.
Mathematically, given a loss measure $\ell$ and a point in the input space $\bm{x}_i$, we might consider models $g$ such that $\ell(f(\bm{x}'_i), g(\bm{x}'_i)) \leq \varepsilon \; \forall \bm{x}'_i \in \{\bm{x} \in \mathcal{X} | D(\bm{x}'_i, \bm{x}_i) \leq \delta \}$ as local surrogate models.

Many methods have been proposed to produce local surrogate models as explanations.
Theoretically, the simplest way to produce a local surrogate model would be to calculate the gradient of the closed-box function.
This XAI method is often called {\sc vanillagrad} in the literature.
However, in practice, the gradients of machine learning models can be very noisy, as demonstrated by the effectiveness of adversarial attacks that exploit small perturbations in neural networks \cite{szegedy2014intriguing}.
On the other hand, many commonly used machine learning models, such as random forests, do not have well-defined gradients.
Hence, more involved approaches are warranted. 

{\sc smoothgrad} \cite{smilkov2017smoothgrad} attempts to solve the gradient noise problem in {\sc vanillagrad} by averaging over gradients sampled from the vicinity of the point under explanation.
Although the original paper only applies {\sc smoothgrad} to classification, the method can easily be extended to regression. 

Moving away from directly analysing the gradient, {\sc lime} \cite{ribeiro2016} and {\sc shap} \cite{lundberg2017unified} are perhaps the most widely known examples of practical local explanation generation methods.
In this paper, we focus on the {\sc kernel-shap} variant, which combines linear {\sc lime} with {\sc shap}.
Both {\sc lime} and {\sc kernel-shap} produce explanations in the form of additive linear models ($\hat{y} = \phi^T \bm{x}$).
Given an item $\bm{x}_i$, the methods sample novel items $\bm{x}'_i \in \mathcal{X}'$ in the neighbourhood of the first item, use the closed-box model to predict the labels for the novel items and find the linear model that best fits them \cite{lundberg2017unified},
\begin{equation}
    \hat{\phi} = \arg \min\nolimits_{\phi} \sum\nolimits_{\bm{x}'_i \in \mathcal{X}'} [f(\bm{x}'_i) - \phi^T \bm{x}'_i]^2 \pi_x(\bm{x}'_i) + \Omega(\phi),
\end{equation}
where $\pi_x$ represents a distance measure and $\Omega(\phi)$ is a regularisation term.
The difference between the methods lies in the choice of distance measure, which defines the notion of neighbourhood for $\bm{x}_i$; {\sc lime} most often uses either $L^2$ or cosine distance, whereas {\sc kernel-shap} utilizes results in game theory \cite{slack2021reliable}.

In addition to a procedure for generating local explanations, the authors of {\sc lime} also propose a method which constructs a global explanation by selecting a set of items whose explanations best capture the global behaviour.
They term this procedure \emph{submodular pick algorithm}.
The procedure first generates a local explanation for each item within the dataset. 
Then, a global feature importance score is calculated as the sum of the square roots of the feature attributions aggregated across all local explanations.
If we only pick items according to the presence of the most important features, there is a danger of ending up with many similar explanations.
Hence, \emph{submodular pick algorithm} encourages diversity by framing the problem as a weighted covering problem, balancing feature importance and representativeness such that the user sees a diverse set of explanations.

{\sc smoothgrad}, {\sc lime} and {\sc shap} are based on sampling novel items, which, while simple to implement, introduces a unique set of challenges.
First, formulating a reliable data generation process or sampling scheme for all possible datasets is difficult, if not impossible \cite{guidotti2018survey, laugel2018defining}.
For example, images generated by adding random noise to the pixel values rarely resemble natural images.
Second, randomly generating new items might produce items that cannot occur naturally due to, for example, violating the laws of physics.
{\sc slisemap} \cite{bjorklund2023slisemap} and its variant, {\sc slipmap} \cite{bjorklund2024SLIPMAP}, produce both a low-dimensional embedding for visualization and a local model for all training items without sampling any new points.
{\sc slisemap} finds both the embedding and local models by optimising a loss function consisting of two parts: an embedding term, where items with similar local explanations attract each other while repelling dissimilar ones and a local loss term for the explanation of each item:
\begin{equation}
    \min_{g_i} \mathcal{L}_i = \sum\nolimits_{i=1}^n \frac{\exp(-D(\bm{z}_i, \bm{z}_j))}{\sum_{k=1}^n \exp(-D(\bm{z}_k, \bm{z}_j))} \ell(g_i(\bm{x}_j), \bm{y}_j) + \Omega(g_i)
\end{equation}
where $D(\cdot, \cdot)$ is the euclidean distance in the embedding, $g_i$ represents the local model, $\ell$ is the local loss function, and $\Omega$ again denotes regularisation term(s).
In {\sc slisemap}, each item is fitted to its local models; in the {\sc slipmap} variant, the number of local models is fixed to some $p$, usually much less than the number of data items $n$, and the training items are mapped to one of the $p$ local models.

Another large class of explanations which deserves mention is the so-called case-based or example-based XAI methods, which use representative samples in the training set to explain novel ones \cite{agnar1994casebased, kim2016examples}.
One such method is using prototypes, which present the user with the most similar ``prototype items'' as an explanation, such as showing images of birds with similar plumage as a basis for classification. 

A shared property of post-hoc local surrogate models is the lack of uniqueness; for a given item, many local surrogates may exist with similar performance.
The phenomenon is documented for {\sc slisemap} in \cite{bjorklund2023slisemap} and implied for other methods based on the results in \cite{dombrowski2019explanations}, as well as for many publications aimed at fixing the inherent instability of {\sc lime} \cite{shankar2019alime, zafar2019dlime, zhao2021baylime}.
We argue that the existence of such alternative explanations is an inherent feature of using local surrogate models.
Intuitively, we can imagine the $n$-surface of a complex closed-box function and consider local surrogates as planes with dimensionality $n-1$.
There are many ways to orient a local surrogate on the curved surface of the closed-box function while retaining reasonable local fidelity.
Interestingly, the existence of alternative explanations implies that there may be surrogates which perform well for many items in the data distribution.
Therefore, we might be able to reduce a large set of local models to a small set of widely applicable surrogates, providing a global explanation of the model's behaviour.

The method proposed in the previous paragraph falls under \emph{model aggregation}.
The submodular pick algorithm mentioned when discussing {\sc lime} is an example of a model aggregation method.
Other methods include Global Aggregations of Local Explanations ({\sc gale}) \cite{vanderlinden2019global}, {\sc GLocalX} \cite{setzu2021glocalx}, and an integer programming-based approach introduced by Li et al. \cite{li2022optimal}.
{\sc gale} offers alternative ways to calculate feature importance for the submodular pick algorithm, as the authors argue that the way these importance values are calculated in the original \cite{ribeiro2016} is only applicable to a limited number of scenarios.
Furthermore, they show how the choice of the best-performing importance value definition is task-dependent.
In {\sc GLocalX} \cite{setzu2021glocalx}, the authors propose a method to merge rule-based explanations to find a global explanation.
In the programming-based approach (\cite{li2022optimal}), the authors take a similar approach to the one proposed in this paper.
They also attempt to find a representative subset of local models, and formulate model aggregation as an optimisation problem with fidelity and coverage constraints.
However, their work has some limitations.
First, their method relies on the definition of applicability radii for the local models, i.e., radii within which the explanation holds.
Second, the framework only functions in classification tasks.
Third, to satisfy the optimisation constraints, the framework requires the inclusion of tens of local models into the aggregation, limiting the interpretability of the aggregation as a global model.
Finally, they only tested their framework with random forest models and two datasets.
Because the model aggregation methods described above cannot be directly applied to an arbitrary set of local explanations, in this paper we opt to measure the performance of {\sc ExplainReduce} against the full set of local explanations instead.



\section{Methods}

In this section, we describe the idea and implementation of {\sc ExplainReduce}.
Assuming that many items in a dataset can have many alternative explanations of similar performance, a proper subset of explanations can accurately model most of the items in the dataset. 
We refer to this smaller set as a set of ``proxy models''.
Thus, the method combines aspects of local surrogate explanations with prototype items; instead of representative data items, we use representative local surrogate models to summarise global model performance.

The {\sc ExplainReduce} procedure works as follows: after training a closed-box model, we generate a large set of local explanations for the closed-box and then find a covering subset of the local models, which acts as a global explanation.
In this section, we first define the problem of finding the subset of local models and then move on to cover the reduction methods and algorithms that generate these proxy sets.
We also introduce the quality metrics used to evaluate the performance of reduced sets.

\subsection{Problem definition}
\label{sec:problem}
A dataset $\mathcal{D} = \{(\bm{x}_1, \bm{y}_1) , \ldots, (\bm{x}_n, \bm{y}_n)\}$ consists $n$ of data items (covariates) $\bm{x}_i \in \mathcal{X}$ and labels (responses) $\bm{y}_i \in \mathcal{Y}$. We use ${\bm X}$ denote a matrix of $n$ rows such that ${\bm X}_{i\cdot}=\bm{x}_i$.
If we have access to a trained supervised learning algorithm $f(\bm{x}_i) = \hat{\bm{y}}_i$, we can instead replace the true labels $\bm{y}_i$ with the predictions from the learning algorithm $\hat{\bm{y}}_i$.
A local explanation for a data item $(\bm{x}_i, \bm{y}_i)$ is a simple model $g(\bm{x}_i) = \Tilde{\bm{y}}_i$  which locally approximates either the connection between the data items and the labels or the behaviour of the closed-box functions. 
In the previous section, we gave multiple examples of generating such explanations.
Assume we have generated a large set of such local explanations $\bm{G} = \{g_j\mid j\in[m]=\{1,\ldots,m\}\}$ and a mapping from data items to models $\Phi:{\mathcal{X}}\mapsto[m]$ using one of these methods.
Assume also that these local models can be identified by a set of $p$ real parameters (such as coefficients of linear models), which we will denote with $\bm{B}\in{\mathbb{R}}^{m\times p}$.
We define a loss function $\ell:\mathcal{Y} \times \mathcal{Y} \rightarrow \mathbf{R}_{\geq 0}$ and a loss matrix $\bm{L}\in{\mathbb{R}}_{\ge 0}^{m\times n}$, where individual items are defined as $\bm{L}_{ij} = \ell(g_i(\bm{x}_j), \bm{y}_j)$.
A straightforward example of a mapping $\Phi$ would then be chosen from the local models, the one with the lowest loss for each item: $\Phi(\bm{x}) = \arg\min\nolimits_{i\in[m]}{\ell(g_i(\bm{x}_j), \bm{y}_j)}$. 

Finally, we assume that the large set $\bm{G}$ contains a reasonable approximation for each item in the dataset ${\cal D}$: for all $i \in [n]$ and for a given $\varepsilon\in{\mathbb{R}}_{>0}$ there exists $j\in[m]$ such that $\ell (g_j(\bm{x}_i), \bm{y}_i) \leq \varepsilon$.
This can be achieved by, e.g., learning a local explanation for each item in ${\cal D}$.

We are interested in how many items can be explained by a given set of local surrogates to a satisfactory degree.
To measure this, we use coverage $C$, defined as the proportion of data items which can be explained sufficiently by at least one local model in a subset ${\bm{S}} \subseteq[m]$.
Mathematically, given a loss threshold $\varepsilon\in{\mathbb{R}}_{>0}$, the coverage can be calculated as
\begin{equation}
    C({\bm{S}}, \varepsilon) = (1/n) \left| \{j\in [n]\mid\min\nolimits_{i\in{\bm{S}}}{} \ell(g_i(\bm{x}_j), \bm{y}_j) \leq \varepsilon \; \} \right|.
\end{equation}
A $c$-covering subset of local models ${\bm{S}} \subseteq [m]$ is a set for which $C({\bm{S}}, \varepsilon) \geq c$. 


Next, we define three computational problems to address the task described earlier. Later, we will introduce algorithms to solve each of the problems.

The first formulation attempts to minimise the number of items for which a satisfactory local explanation is not included in the subset ${\bm{S}}_c$.
\begin{prob} ({\sc maximum coverage})\label{prob:1}
Given $k$ and $\varepsilon\in{\mathbb{R}}_{>0}$, find a subset ${\bm{S}}_c$ of cardinality $k$ that maximises coverage, or
\begin{equation}
{\bm{S}}_c=\arg\max\nolimits_{{\bm{S}}\subseteq [m]_k} C({\bm{S}}, \varepsilon), 
\end{equation}
where we have used $[m]_k=\{{\bm S}\subseteq[m]\mid\left|{\bm S}\right|=k\}$ to denote the subsets of cardinality $k$.
\end{prob}



The second definition attempts to capture a subset of explanations that can be used as a proxy model with a small average loss.
\begin{prob} ({\sc minimum loss})\label{prob:2}
Given $k$, find a subset ${\bm{S}}_c$ with cardinality $k$ such the average loss when picking the lowest loss model from ${\bm{S}}_c$ is minimised, or
\begin{equation}
{\bm{S}}_c=\arg\min\nolimits_{{\bm{S}}\in[m]_k}{\left(
\frac 1n\sum\nolimits_{j=1}^n{
\min\nolimits_{i\in{\bm{S}}}{\ell(g_i({\bf x}_j), {\bf y}_j)}
}
\right)}.
\end{equation}
\end{prob}


The final formulation is a combination of problems \ref{prob:1} and \ref{prob:2}.
\begin{prob} ({\sc coverage-constrained minimum loss})\label{prob:3}
Given $k$, $\varepsilon\in{\mathbb{R}}_{>0}$, and minimum coverage $c\in (0, 1]$, find a $c$-covering  ${\bm{S}}_c$ with cardinality $k$ such the average loss when picking the lowest loss model from ${\bm{S}}_c$ is minimised, or
\begin{equation}
 {\bm{S}}_c=\arg\min\nolimits_{{\bm{S}}\in[m]_{k,c}}{\left(
\frac 1n\sum\nolimits_{j=1}^n{
\min\nolimits_{i\in{\bm{S}}}{\ell(g_i({\bf x}_j), {\bf y}_j)}
}
\right)},
\end{equation}
where $[m]_{k,c}=\{{\bm S}\in[m]_k\mid C({\bm{S}},\varepsilon)\ge c\}$ are the $c$-coverings of cardinality $k$.
\end{prob}

\subsection{General procedure}\label{sec:procedure}

The {\sc ExplainReduce} algorithm is outlined in Algorithm \ref{alg:main}. 
Given a dataset $\mathcal{D}$, an explanation method $\text{Exp}$ that generates a set of $m$ local explanations (a special case being one local explanation for each training data point, in which case $m=n$) and a reduction algorithm explained later in Sect. \ref{sec:reduction_algs}, we first use the explanation method to generate $m\le n$ explanations for $m$ items sampled without replacement from $\mathcal{D}$.
If a closed-box function is provided, we replace the original labels in $\mathcal{D}$ with predictions from $f$.
We then apply the reduction algorithm to the generated set of local explanations $[m]$ and receive the proxy set ${\bm{S}}_c \subseteq [m]$.
Finally, for each sampled data item, we pick the local explanation in the proxy set ${\bm{S}}_c$ with minimal loss, mapping the items and the proxies.


\begin{algorithm}
\caption{{\sc ExplainReduce} Procedure to find a subset of explanations.}
\label{alg:main}
\hspace*{\algorithmicindent} \textbf{Input:} $\mathcal{D} \gets \{(\bm{x}_i, \hat{\bm{y}}_i) | i \in [n]\}$: dataset; ${\bm G}$: the set of $m$ local explanations; $\textrm{reduce}$: method to find ${\bm S}_c$, parametrised optionally by $\varepsilon$, $c$, or $k$, see Sect. \ref{sec:reduction_algs}. \\
\hspace*{\algorithmicindent} \textbf{Output:} ${\bm S}_c$: reduced set of explanations; $\text{map}$: $\text{map}[i]: [n]\mapsto{\bm S}_c$, mapping from the local dataset $[n]$ to explanations in ${\bm S}_c$.
\begin{algorithmic}
    \Procedure{ExplainReduce}{}    
    \State $\bm{S}_c \gets \textrm{reduce}(\bm{G}, \varepsilon, c, k)$ \Comment{$\textrm{reduce}$ is defined in Sect. \ref{sec:reduction_algs}}
    \State $\text{map} \gets \{\}$ \Comment{a mapping between items in $\mathcal{D}$ and the local models}
    \For{$i \in [n]$}
       \State $\text{map}[i] \gets \arg\min_{j\in{\bm S}_c}{\ell(g_j(\bm{x}_i), \bm{y}_i)}$
    \EndFor \\
    \Return ${\bm S}_c,\; \text{map}$\\
    \EndProcedure
\end{algorithmic}
\end{algorithm}
 
\subsection{Reduction algorithms}
\label{sec:reduction_algs}

In this section, we briefly overview the practical implementations of the reduction algorithms (function ${\textrm{reduce}}$ in Alg. \ref{alg:main}) used to solve the problems outlined in the previous section.

{\sc max coverage}: 
Problem 1 is a variant of the NP-complete partial set covering problem, sometimes called {\sc max $k$-cover}.
The equivalence is obvious if we consider an item $(\bm{x}_j, \bm{y}_j) \in \mathcal{D}$ covered by model $g_i$ if $\ell(g_i(\bm{x}_j), \bm{y}_j) \leq \varepsilon$.
We solve Prob. \ref{prob:1} exactly using integer programming (implemented by the {\sc pulp} Python library \cite{mitchell2024pulp}) and approximately using a greedy algorithm.
In the greedy approach, given fixed $k$, we iteratively pick local models $g \in \bm{G}$ such that the marginal increase in coverage is maximised with each iteration until $k$ models have been chosen.
Our problem is submodular, as adding each new model to the subset cannot decrease the coverage.
It has been shown that in this case, the greedy algorithm has a guaranteed lower bound to achieve coverage at least $1 - \left((k - 1)/k\right)^k$ times the optimal solution \cite{nemhauser1978analysis}.

Notably, this approximation ratio only applies to the original set of data items and their associated local explanations.
If we apply the proxy sets to novel data, we should use standard machine learning tools --- such as a separate validation set --- to ensure that the model performs appropriately.

{\sc min loss}:
Problem \ref{prob:2} is an example of a supermodular minimisation problem.
Let $f(\bm{S}) = \left(\frac 1n\sum_{j=1}^n{\min_{i\in{\bm{S}}}{\ell(g_i({\bf x}_j),{\bf y}_j)}} \right)$ and $\bm{A}, \bm{B}: \; \bm{A} \subset \bm{B} \subset \bm{G}$ be subsets of local models.
Additionally, let $v$ be a local model not contained in $\bm{B}$.
Clearly, the decrease in loss by adding $v$ to the larger set $\bm{B}$ must be, at most, as great as adding the same model $v$ to $\bm{A}$.
In other words,
\begin{equation}
    f(\bm{A} \cup \{v\}) - f(\bm{A}) \leq f(\bm{B} \cup \{v\}) - f(\bm{B}) \quad \forall \bm{A} \subset \bm{B} \subset \bm{G}, \; v \notin \bm{B},
\end{equation}
which is the definition of supermodularity \cite{mccormick2005submodular}.

Supermodular minimisation problems are known to be NP-hard.
Hence, we only use a greedy ascent algorithm to solve Problem \ref{prob:2}, as finding an exact solution is computationally expensive due to the continuous nature of loss.
In the worst case, the search for the optimal subset would require $\binom{m}{k}$ comparisons.
Like above, we iteratively pick local models with the best possible decrease in marginal loss until we reach $k$ chosen local models.
In the general case, a multiplicative approximation ratio for a supermodular minimisation problem may not exist due to the optimal solution having a value $f(\bm{S}^*) = 0$, while the greedy algorithm may converge to a solution with non-zero loss.
However, in our case, finding a subset of models with exact zero loss for all items is unlikely.
In \cite{ilev2001approximation}, the author derives a curvature-based approximation ratio for a greedy descent (worst out) algorithm.
Based on this analysis, authors of \cite{bounia2023approximating} derive an approximation ratio for greedy ascent for probabilistic decision trees.
Neither of these approaches is directly applicable to our setting, and hence, we cannot give a closed-form approximation ratio.
However, as we show in Section \ref{sec:greedy}, the empirical approximation ratios for the greedy ascent algorithm are reasonable.
Moreover, we show how the greedy approximation performs nearly equally, if not better, on unseen data compared to the exact solution for our datasets.

{\sc const min loss}:
We solve Problem 3 again with a greedy ascent algorithm, i.e., iteratively picking the best model to include in the subset ${\bm S}_c$ until the maximum coverage has been met.
The only difference to {\sc min loss} algorithm is how the models are scored.
If the coverage constraint has been met, we simply follow the procedure in {\sc min loss} and select the model with the best possible marginal loss decrease.
Otherwise, we divide the marginal loss decrease $\Delta \ell_i$ we would get by including model $i$ in the proxy set with the marginal coverage increase of including the same model $\Delta c_i$.
This reduction method can be implemented with a hard constraint by having the algorithm throw an exception if the coverage is not met or with a soft constraint where the resulting proxy set is returned regardless of satisfying the constraint.
In this paper, we opt to use the softly constrained variant exclusively.
We can see that in the case where the coverage constraint is not met, Problem 3 with this scoring scheme is also a supermodular minimisation problem and equivalent to Problem 2 with a different optimisation objective, namely the coverage-normalised loss.
Similar to Problem 2, we cannot give an exact approximation ratio for this algorithm, but the empirical results in Section \ref{sec:greedy} suggest good performance.

{\sc clustering}:
Above, we have treated the problem as a set covering problem.
We can also approach the problem from the unsupervised clustering perspective: can we cluster the training data to find a proxy set?
Given the number of clusters $k$, we can cluster the training data items $\bm{X}$, the local model parameters $\bm{B}$, or the training loss matrix $\bm{L}$ using some clustering algorithm.
In this paper, we use {\sc k-means} with Euclidean distance for $\bm{X}$ and $\bm{L}$, and cosine distance for the local model parameters $\bm{B}$.
After performing the clustering, we pick the local model closest to the cluster centroid for each cluster to form the proxy set.

\subsection{Performance measures}

\bmhead{Fidelity} Fidelity \cite{guidotti2018survey} measures the adherence of a surrogate model to the complex closed-box function.
Given a closed-box function $f$, we fidelity is the loss between the closed-box model prediction $\hat{\bm{y}} = f(\bm{x})$ and the surrogate model prediction:
\begin{equation}
    \textrm{fidelity} = \frac{1}{n} \sum\nolimits_{i=1}^{n} \ell(g_i(\bm{x}_i), \hat{y}_i),
    \label{eq:fidelity}
\end{equation}
where $g_i$ is the relevant local model either from the full set of local explanations or the proxy model set.

\bmhead{Instability} Instability \cite{guidotti2018survey} (sometimes also referred to as \emph{stability}, despite lower values denoting better performance) measures how much a slight change in the input changes the explanation.
In practice, we model the slight change by measuring the loss of a given local model $g_i$ associated with item $\bm{x}_i$ with its $\kappa$ nearest neighbours:
\begin{equation}
    \textrm{instability} = \frac{1}{n} \sum\nolimits_{i=1}^{n} \frac{1}{\kappa} \sum\nolimits_{j \in \textrm{NN}_\kappa(i)} \ell(g_i(\bm{x}_j), y_j). 
    \label{eq:instability}
\end{equation}
This paper uses a fixed number of $\kappa=5$ nearest neighbours when we report instability values ($\left|\textrm{NN}_5(i)\right|=5$).

\section{Experiments}

In this section, we demonstrate the performance of the proxy sets produced via different XAI methods and reduction strategies.
We experiment on a variety of different datasets, consisting of both classification and regression tasks.
For each of the datasets used, we train a closed-box model and produce the initial set of local explanations using that model.
A brief description of the datasets can be found in Table \ref{tab:datasets}, and further details in Appendix \ref{a:datasets}.

\begin{table}[b]
    \centering
\begin{tabular}{l|cccc}
Dataset Name    & Size        & Task           & Closed-box model  & Citation                                    \\ \hline
Synthetic       & 5000 × 11   & Regression     & Random Forest     & \cite{bjorklund2023slisemap}                \\
Air Quality     & 7355 × 12   & Regression     & Random Forest     & \cite{oikarinenDetectingVirtualConcept2021} \\
Life Expectancy & 2938 × 22   & Regression     & Neural Network    & \cite{rajarshi2017life}                     \\
Vehicle         & 2059 × 12   & Regression     & SVR               & \cite{birla2022vehicles}                    \\
Gas Turbine     & 36733 × 9   & Regression     & Adaboost          & \cite{2019gas}                              \\
QM9             & 133766 × 27 & Regression     & Neural Network    & \cite{ramakrishnan2014Quantum}              \\
Jets            & 266421 × 7  & Classification & Random Forest     & \cite{CMS:opendata}                         \\
HIGGS           & 100000 × 28 & Classification & Gradient Boosting & \cite{whiteson2014higgs}                    \\ \hline
\end{tabular}
    \caption{Summary of datasets used in experiments.}
    \label{tab:datasets}
\end{table}

In the experiments, we measure fidelity mostly with test data, i.e., data not used in generating the full set of local explanations.
This metric better captures the adherence of the explainer to the closed-box model than calculating the same value on the data used to generate the local explanations.

The local explanation methods discussed in this paper are generally not generative, that is, there is no obvious mapping from a new covariate ${\bm{x}}$ to an explanation model in ${\bm{S}}_c$.
Therefore, to use either the full explanation sets or the proxies to produce predictions, we need to generate another mapping between novel items and local models in the explanation set.
The simplest of such mappings is to use the distance in the data space: $\Phi_{\textrm{new}}( \bm{x}) = \arg\min\nolimits_{i\in{\bm{S}}_c} ||\bm{x}_i - \bm{x}||^p, \; i \in (1, ..., n)$, where $||a||^p$ is the $p$-norm.
This new mapping is used for unlabelled items only and is in addition to the loss-minimising mapping used in the reduction process to find an initial mapping with the ``training'' data (data used to generate the local explanations) described in Section \ref{sec:procedure}.

\subsection{Case studies}
We begin our examination by demonstrating the {\sc ExplainReduce} procedure with two case studies to give the reader a better intuition of the procedure and its possible usage.

\subsubsection{Synthetic data}

To give a simple example of the procedure, we apply it to a synthetic dataset.
The dataset, which is described in detail in Appendix \ref{a:datasets}, is generated by producing $k=4$ clusters in the input space and generating a random, different linear model for each cluster.
We then generate labels by applying a local model to the data items based on their cluster ID and adding Gaussian noise.
The dataset is then randomly split into a training set and a test set, and we train a {\sc smoothgrad} explainer on the training data.
For reduction, we use a greedy {\sc max coverage} algorithm, where $\varepsilon$ is defined as the 10th percentile of the loss matrix $\bm{L}$.

In Fig. \ref{fig:case_PCA}, we show a PCA of the items in the test set on the left, coloured with the ground truth cluster labels, and how the test items are mapped to proxy models on the right.
Overall, we can see that most items get mapped to the correct proxy model for that particular cluster.
The small impurity in the clusters stems both from the Gaussian noise and the approximative nature of the greedy coverage-maximising algorithm.
Furthermore, as Fig. \ref{fig:case_radar} shows, the reduced proxy models (red) correspond well with the ground truth models (blue).
The proxy set thus serves well as a generative global explanation for the dataset.

\begin{figure}
    \centering
    \includegraphics[width=\textwidth]{figures/case_PCA.pdf}
    \caption{Ground PCA of a synthetic test dataset (left) and the same dataset where colours correspond to reduced model indices, based on {\sc smoothgrad} local models and reduced using a greedy coverage-maximising algorithm (right). We can see that the reduction is able to faithfully approximate the ground truth clustering.}
    \label{fig:case_PCA}
\end{figure}

\begin{figure}
    \centering
    \includegraphics[width=\textwidth]{figures/case_radar.pdf}
    \caption{Radar plots showing ground truth local model parameters (blue) and corresponding proxy model parameters (red). The reduction method is able to find very similar surrogate models to the ground truth.}
    \label{fig:case_radar}
\end{figure}

\subsubsection{Particle jet classification}

In a previous work \cite{seppalainen2023using}, we analysed a dataset containing simulated LHC proton-proton collisions using {\sc slisemap}.
These collisions can create either quarks or gluons, which decay into cascades of stable particles called jets.
These jets can then be detected, and we can train the classifier to distinguish between jets created by quarks and gluons based on the jet's properties.

We first trained a random forest model on the data and then applied {\sc slisemap} to find local explanations.
We clustered the {\sc slisemap} local explanations to 5 clusters and analysed the average models for each cluster, and found them to adhere well to physical theory.
When we generate 500 {\sc slisemap} local explanations on the same dataset and apply the {\sc const min loss} reduction algorithm with $k=4$ proxies, we find the proxy set depicted in Fig. \ref{fig:jets_example}.
The left panel shows a swarm plot where each item is depicted on a horizontal line based on the random forest-predicted probability of the jet corresponding to a gluon jet and coloured based on which proxy model is associated with the item.
The right panel shows the coefficients of the proxy models, which are regularised logistic regression models.
We find that the proxies are similar to the cluster mean models shown in the previous publication and show similar adherence to the underlying quantum chromodynamic theory \cite{cms2013performance}.
For example, wider jets (high {\sc jetGirth} and {\sc QG\_axis2}) are generally more gluon-like, and therefore these parameters are essential in classifying the jets.
Incidentally, proxy 3 (red) is associated with the most quark- and gluon-like jets and has high positive coefficients for both parameters.
Similarly, proxy 0 (blue) has negative coefficients for momentum ({\sc jetPt}) and {\sc QG\_ptD}, which measures the degree to which the total momentum parallels the jet.
Both of these features indicate a more quark-like jet.

\begin{figure}
    \centering
    \includegraphics[width=\textwidth]{figures/jets_example.pdf}
    \caption{Example of the {\sc ExplainReduce} procedure on a particle jet classification task. The dataset consists of LHC proton-proton collision particle jets that are created by decaying gluons or quarks. We train a random forest classifier on the dataset and use {\sc slisemap} to generate 500 explanations, which are then reduced to $k=4$ proxies using the {\sc const min loss} reduction algorithm. The left panel shows a swarm plot of the 500 items sorted horizontally based on the predicted probability of corresponding to a gluon; the y-axis has no significance. The right panel shows the coefficients of the logistic regression proxy models, which match the underlying physical theory.}
    \label{fig:jets_example}
\end{figure}

\subsection{Fidelity of proxy sets}

The interpretability of a set of local models as a global explanation is directly related to the size of that set.
As explained in a previous section, local explanation models often need to balance interpretability and granularity: a large set of local models may more easily cover a larger portion of the closed-box function, but the increased number of models makes the result less understandable.
In Fig. \ref{fig:fidelity_k}, we show the test fidelity (as defined in Eq. \eqref{eq:fidelity}) as a function of the size of the reduced explanation set for a selection of datasets and local explanation algorithms.
The fidelity is calculated with respect to the closed-box predicted labels, with a fidelity of 0 representing perfect adherence to the closed-box model.
As the figure shows, a set as small as $k=5$ proxies can reach or even surpass the fidelity of a set of $500$ local explanations for all three selected datasets\footnote{For full coverage on all of the seven datasets, we refer to Appendix \ref{a:fidelity_k_full}.}.
We find that optimisation-based approaches perform particularly well, especially the two greedy loss-minimising algorithms.
The clustering-based reduction methods seem to work almost equally well, as does a random selection of local models, although optimisation-based approaches consistently outperform the latter.
This implies that for proxy set generation methods that require choosing $k$ \emph{a priori}, performance is not very sensitive to the choice of $k$.
Therefore, it is enough for a user of {\sc ExplainReduce} to specify a reasonable value of $k$ for a faithful and interpretable global surrogate.
\begin{figure}
    \centering
    \includegraphics[width=\linewidth]{figures/fidelity_k.pdf}
    \caption{Test set fidelity of explanations as a function of proxy set size $k$. The rows show different datasets, and the columns show different XAI generation methods. The different line colours and styles denote the reduction strategy, and the black horizontal line shows the performance of the full explanation set. The loss-minimising reduction methods consistently reach a fidelity comparable or even better than the full explanation set.}
    \label{fig:fidelity_k}
\end{figure}

If a few proxies are enough to find a faithful representation, how many local explanations do we need to generate to find a good proxy set?
The procedure may be infeasible if a very large set of initial local explanations is required.
For example, {\sc slisemap} scales as $\mathcal{O}(n^2m)$, where $n = |\mathcal{D}|$ and $m$ is the number of features \cite{bjorklund2023slisemap}, which makes generating thousands of explanations expensive.

To study the generalisation performance of the proxy sets, we artificially limit the set of all local models $\bm{G}$ via random sampling before reduction.
That is to say, we generate local approximations for each item in a randomly sampled subset of points in the training set ${\cal D}$.
We then calculate the test fidelity of the proxy set trained on the limited universe as above and compare that to the performance of the full explanation set.
As Fig. \ref{fig:fidelity_n} shows, the proxy sets generalize well even starting from a limited set of local models\footnote{Full version shown in Appendix \ref{a:fidelity_n_full}.}.
The coloured lines represent the various reduction algorithms, while the thick dashed black line denotes the performance of the entire set of local explanations.
The loss-minimising greedy algorithms again achieve the best performance across the studied datasets and explanation methods studied, often matching and sometimes outperforming the full explanation set.
On the other hand, the clustering-based approaches fall behind here, as in the previous section.
These results imply that the procedure of {\sc ExplainReduce} does not require training local models for each item; instead, a randomly sampled subset of models suffices to find a good proxy set.
Based on these results, we set the default number of subsamples to $n=500$ in this paper unless otherwise stated.
\begin{figure}
    \centering
    \includegraphics[width=\linewidth]{figures/fidelity_n.pdf}
    \caption{Test set fidelity of explanations as a function of local model subsample size $n=|\mathcal{D}|$. Different datasets are depicted as rows with the XAI methods used to produce the initial local explanations shown as rows. The coloured lines depict the test fidelity of the proxy sets produced with different reduction algorithms, with the thick black line denoting the performance of the entire local explanation set.}
    \label{fig:fidelity_n}
\end{figure}

We also experimented with the impact of the hyperparameters $\varepsilon$, the loss threshold, and $c$, the minimum coverage constraint, and found the reduction algorithms to have similar fidelity performance across a wide range of these hyperparameter values.
We refer to Appendix \ref{a:cov_eps_sensitivity}
for detailed results.

\subsection{Coverage and stability of the proxy sets}

It should be no surprise that methods directly optimising for loss also show the best fidelity results.
However, an ideal global explanation method should be able to accurately explain most, if not all, of the items of interest.
Additionally, explanations are expected to be locally consistent, meaning that similar data items should generally have similar explanations. 
In Fig. \ref{fig:coverage_stability}, we show the evolution of training coverage and instability of reduction methods as a function of the proxy set size $k$ with respect to the closed-box predicted labels\footnote{Full versions shown in Appendices \ref{a:coverage_full} and \ref{a:stability_full} for coverage and stability, respectively.}.
The results follow a similar pattern to the previous sections: a small number of proxies is enough to achieve high coverage and low instability, with optimisation-based approaches outperforming clustering-based alternatives.
Unsurprisingly, the coverage-optimising reduction methods achieve the best coverage with the smallest number of proxies.
However, the loss-minimising algorithms perform surprisingly well even compared to the coverage-optimising variants.
Notably, the balanced approach -- the {\sc const min loss} reduction algorithm -- almost matches the reduction algorithms directly optimising coverage.

Regarding instability, the loss-minimising reduction methods have an edge over the clustering- and coverage-based approaches, but the distinction between methods is not pronounced.
\begin{figure}
    \centering
    \includegraphics[width=\linewidth]{figures/coverage_stability_k_small.pdf}
    \caption{Coverage and instability of various reduction methods as a function of $k$ on the Gas Turbine dataset. The optimisation-based methods outperform the clustering-based approaches within both metrics.}
    \label{fig:coverage_stability}
\end{figure}

Combining the results from the fidelity, coverage, and instability experiments, the greedy loss-minimising approach with a soft coverage constraint appears to offer the best balance in performance across all three metrics.
The proxy set produced by this reduction method consistently achieves similar, if not better, performance than the full explanation set, even with a few proxies.
Based on these experiments, the various clustering-based approaches do not offer performance gains based on any of the metrics used.
While these approaches may produce more interpretable proxy sets by inherently accommodating the structures found in the data, they struggle to generate sufficiently covering proxy sets and lag behind optimisation-based approaches in both fidelity and instability.

\begin{table}[t]
    \centering
    \begin{tabular}{l@{\hspace{3mm}} r@{\hspace{3mm}}r@{\hspace{3mm}}r@{\hspace{3mm}}r@{\hspace{3mm}}r@{\hspace{3mm}}r@{\hspace{3mm}}r}
\hline \\
\multicolumn{6}{|c|}{Coverage $\uparrow$}\\
\hline\\
\bfseries Dataset & \bfseries G. Max $c$ & \bfseries A. ratio & \bfseries G. Min Loss (min $c$) & \bfseries A. ratio & \bfseries Max $c$ \\
\midrule
Air Quality & $0.82 \pm 0.18$ & $0.97 \pm 0.04$ & $0.9 \pm 0.05$ & $0.93 \pm 0.03$ & $0.84 \pm 0.18$ \\
Gas Turbine & $0.86 \pm 0.17$ & $0.99 \pm 0.01$ & $0.93 \pm 0.06$ & $0.97 \pm 0.03$ & $0.87 \pm 0.18$ \\
Jets & $0.86 \pm 0.17$ & $0.99 \pm 0.02$ & $0.95 \pm 0.06$ & $0.98 \pm 0.03$ & $0.87 \pm 0.18$ \\
QM9 & $0.78 \pm 0.17$ & $0.98 \pm 0.03$ & $0.89 \pm 0.04$ & $0.96 \pm 0.02$ & $0.8 \pm 0.18$ \\
\hline \\
\multicolumn{6}{|c|}{Train fidelity $\downarrow$}\\
\hline\\
\bfseries Dataset & \bfseries G. Min Loss & \bfseries A. ratio & \bfseries G. Min Loss (min $c$) & \bfseries A. ratio & \bfseries Min Loss \\
\midrule
Air Quality & $0.36 \pm 0.35$ & $1.25 \pm 0.21$ & $0.16 \pm 0.12$ & $1.53 \pm 0.53$ & $0.32 \pm 0.35$ \\
Gas Turbine & $0.21 \pm 0.23$ & $2.08 \pm 0.9$ & $0.07 \pm 0.07$ & $1.53 \pm 0.43$ & $0.17 \pm 0.23$ \\
Jets & $0.02 \pm 0.02$ & $1.4 \pm 0.4$ & $0.01 \pm 0.0$ & $1.41 \pm 0.27$ & $0.01 \pm 0.02$ \\
QM9 & $0.53 \pm 0.46$ & $1.25 \pm 0.21$ & $0.24 \pm 0.15$ & $1.37 \pm 0.24$ & $0.48 \pm 0.47$ \\
\hline \\
\multicolumn{6}{|c|}{Test fidelity $\downarrow$}\\
\hline\\
\bfseries Dataset & \bfseries G. Min Loss & \bfseries A. ratio & \bfseries G. Min Loss (min $c$) & \bfseries A. ratio & \bfseries Min Loss \\
\midrule
Air Quality & $0.4 \pm 0.29$ & $1.11 \pm 0.22$ & $0.22 \pm 0.07$ & $1.15 \pm 0.17$ & $0.37 \pm 0.28$ \\
Gas Turbine & $0.31 \pm 0.17$ & $1.2 \pm 0.22$ & $0.21 \pm 0.06$ & $1.1 \pm 0.11$ & $0.28 \pm 0.18$ \\
Jets & $0.02 \pm 0.01$ & $1.09 \pm 0.18$ & $0.01 \pm 0.0$ & $1.04 \pm 0.1$ & $0.02 \pm 0.01$ \\
QM9 & $1.19 \pm 0.13$ & $1.01 \pm 0.08$ & $1.09 \pm 0.06$ & $0.96 \pm 0.09$ & $1.18 \pm 0.11$ \\
\bottomrule
\end{tabular}

    \caption{Comparison of the training fidelity and coverage, alongside the observed approximation ratios (A. ratio), between the analytical and approximate optimisation algorithms when applied to $n=100$ explanations produced with {\sc lime} and with $k=5$ proxies. G. Max $c$ corresponds to the greedy {\sc max coverage} algorithm, while G. Min Loss and G. Min Loss (min $c$) correspond to the greedy {\sc min loss} and {\sc const min loss} algorithms, respectively.}
    \label{tab:greedy}
\end{table}
\subsection{Performance of greedy algorithms}
\label{sec:greedy}

In section \ref{sec:reduction_algs}, we discussed the performance of greedy approximation algorithms used to solve Problems \ref{prob:1}--\ref{prob:3}.
As we have seen in previous sections, the performance of the greedy and analytical coverage-optimising algorithms is almost equal as both a function of the proxy set size and the initial local explanation set size.
As mentioned in the section above, solving for the exact loss-minimising proxy set is a computationally challenging problem which requires $\binom{m}{k}$ comparisons in the worst case.
Hence, the analysis in this section was limited to $n=m=100, k=5$ for computational tractability.
Table \ref{tab:greedy} shows how the greedy approximation algorithms perform comparably to the analytically optimal variants.
For the {\sc max coverage} algorithm, experimental approximation ratios are substantially better than the worst-case bound described in Sect. \ref{sec:reduction_algs}.
Furthermore, as we saw in the previous sections, while on the train set the greedy heuristics do not reach the optimal solution fidelity-wise, on the test the differences between the exact solution and the greedy heuristics vanish.
The {\sc const min loss} algorithm especially shows poorer performance compared to the greedy {\sc min loss} algorithm but generalises better on the test set while achieving good coverage performance as well.
Overall, we can conclude that the greedy approximations offer a computationally feasible approach to generating proxy sets.






\section{Discussion}

Using local surrogate models as explanations is a common approach for model-agnostic XAI methods.
In this paper, we have shown that the {\sc ExplainReduce} procedure can find a post-hoc global explanation consisting of a small subset of simple models when provided a modest set of pre-trained local explanations.
The procedure is agnostic to both the underlying closed-box model and the XAI method, assuming that the XAI method can function as a generative (predictive) model.
These proxy sets achieve comparable fidelity, stability, and coverage to the full explanation set while remaining succinct enough to be easily interpretable by humans.
The loss-minimising reduction method with a soft coverage constraint especially offers a balance of fidelity, coverage, and stability across different datasets.
Furthermore, we have shown how we can use greedy algorithms to find these proxy sets with minimal computational overhead with nearly equal performance to analytically optimal counterparts.
The methods are not sensitive to hyperparameters, allowing for easy adoption.

As with any work, there are shortcomings.
The presented implementation for the procedure simply finds a set of proxies that is (approximately) optimal from a global loss or coverage perspective.
In other words, there is no consideration of the spatial distribution of the data with respect to which proxy was associated with each item.
To deepen the insight provided by the method, an interesting idea would be to combine the reduction process with manifold learning techniques, enabling visualisation alongside the proxy set, similar to {\sc slisemap} and {\sc slipmap}.
For example, we might try to find an embedding where data items associated with a particular proxy would form continuous regions, significantly increasing interpretability and augmenting the method's capabilities as an exploratory data analysis tool.

Another interesting way to leverage the set of applicable local models is uncertainty quantification.
Instead of trying to find local models applicable to many data items, we could instead turn our attention to all applicable models for a given item.
This set of applicable models could be considered as samples from the global distribution of explanations for that particular item.
Thus, we could use such sets (alongside the produced explanation) as a way to find, e.g., confidence intervals for explanations produced by a wide variety of XAI methods.

\bmhead{Acknowledgements}

We thank the Research Council of Finland for funding (decisions 364226, VILMA Centre for Excellence) and the Finnish Computing Competence Infrastructure (FCCI) for supporting this project with computational resources.

\bibliography{alternative2024}%

\newpage

\appendix

\section{Dataset introduction}
\label{a:datasets}
Below, we describe each of the datasets used in the experiments in the paper.

\noindent\textit{Synthetic} \cite{bjorklund2023slisemap} is a synthetic regression dataset with $N$ data points, $M$ features, $k$ clusters, and cluster spread $s$.
Each cluster is associated with a linear regression model and a centroid.
Coefficient vectors $\beta_j \in \mathbb{R}^{M+1}$ and centroids $c_j \in \mathbb{R}^M$ are sampled from normal distributions, with repeated samplings if either the coefficients or the centroids are too similar between the clusters.
Each data point is assigned to a cluster $j_i$, with features $x_i$ sampled around $c_{j_i}$ and standardized.
The target variable is $y_i = x_i^\top \beta_{j_i} + \epsilon_i$, where $\epsilon_i$ is Gaussian noise with standard deviation $\sigma_e$.
In the experiment section, unless otherwise stated, we set the number of data points to $N = 5000$, the number of features to $M = 11$, the number of clusters to $k = 5$, and the standard deviation of Gaussian noise to $\sigma_e = 2.0$.
As the closed-box model, we use a random forest regressor.

\vspace{0.2cm}

\noindent\textit{Air Quality} \cite{oikarinenDetectingVirtualConcept2021} contains 7355 hourly observations of 12 different air quality measurements.
One of the measured qualities is chosen as the label, and the other values are used a covariates.
With this dataset, we use a random forest regressor as the closed-box model.

\vspace{0.2cm}

\noindent\textit{Life Expectancy} \cite{rajarshi2017life} comprising 2,938 instances of 22 distinct health-related measurements, this dataset spans the years 2000 to 2015 across 193 countries.
The expected lifespan from birth in years is used as the dependent variable, while other features act as covariates.
The closed-box model we use with this dataset is a neural network.

\vspace{0.2cm}

\noindent\textit{Vehicle} \cite{birla2022vehicles} contains 2059 instances of 12 different car-related features, with the target being the resale value of the instance.
We use a support vector regressor as the closed-box model with the Vehicles dataset.

\vspace{0.2cm}

\noindent\textit{Gas Turbine} \cite{2019gas} is a regression dataset with 36,733 instances of 9 sensor measurements on a gas turbine to study gas emissions.
With this dataset, we used a Adaboost regression as the closed-box model.

\vspace{0.2cm}

\noindent\textit{QM9} \cite{ramakrishnan2014Quantum} is a regression dataset comprising 133,766 small organic molecules. Features are created with the Mordred molecular description calculator \cite{moriwaki2018mordred}.
The QM9 closed-box model is a neural network.

\vspace{0.2cm}

\noindent\textit{HIGGS} \cite{whiteson2014higgs} is a two-class classification dataset consisting of signal processes that produce Higgs bosons or are background.
The dataset contains nearly $100000$ instances with 28 features.
We used a gradient boosting classifier with this dataset as the closed-box model.

\vspace{0.2cm}

\noindent\textit{Jets} \cite{CMS:opendata} contains simulated LHC proton-proton collisions.
The collisions produce quarks and gluons that decay into cascades of stable particles called jets.
The classification task is to distinguish between jets generated by quarks and gluons. 
The dataset has $266421$ instances with 7 features.
The closed-box model we used with this dataset is a random forest classifier.

\clearpage  %

\section{Fidelity of the proxy sets as a function of proxy number}\label{a:fidelity_k_full}
Figure \ref{fig:fidelity_k_full} shows the complete fidelity plots as a function of $k$. Each row corresponds to a different dataset, and each column represents a local model generation method. A general trend emerges, indicating improved fidelity with increasing $k$.

\begin{figure}[h]
    \centering
    \includegraphics[width=\linewidth]{figures/fidelity_k_full.pdf}
    \caption{Fidelity of the reduction algorithms as a function of the proxy set size $k$. The black dashed line indicates the performance of the full set of explanations.}
    \label{fig:fidelity_k_full}
\end{figure}
\FloatBarrier  %
\clearpage  %

\section{Fidelity of the proxy sets as a function of subsample size}\label{a:fidelity_n_full}
Figure \ref{fig:fidelity_n_full} shows the complete fidelity plots as a function of subsample size. Each row corresponds to a different dataset, and each column represents a local model generation method. A general trend emerges, indicating improved fidelity with increasing subsample size.
\begin{figure}[h]
    \centering
    \includegraphics[width=\linewidth]{figures/fidelity_n_full.pdf}
    \caption{Fidelity of the reduction algorithms as a function of subsample size. The black dashed line indicates the performance of the full set of local explanations.}
    \label{fig:fidelity_n_full}
\end{figure}

\FloatBarrier  %
\clearpage  %

\section{Coverage of the proxy sets as a function of proxy number}\label{a:coverage_full}
Figure \ref{fig:coverage_full} shows the complete results of the coverages of the proxy sets. Each row corresponds to a different dataset, and each column represents a local model generation method. A general trend emerges, indicating broader coverage with increasing $k$.
\begin{figure}[h]
    \centering
    \includegraphics[width=\linewidth]{figures/coverage_full.pdf}
    \caption{Coverage of the reduction algorithms as a function of proxy set size $k$.}
    \label{fig:coverage_full}
\end{figure}

\FloatBarrier  %
\clearpage  %

\section{Stability of the proxy sets as a function of proxy number}\label{a:stability_full}
Figure \ref{fig:stability_full} shows the complete results of the stabilities of the proxy sets. Each row corresponds to a different dataset, and each column represents a local model generation method. A general trend emerges, indicating more stable performance with increasing $k$.
\begin{figure}[h]
    \centering
    \includegraphics[width=\linewidth]{figures/stability_full.pdf}
    \caption{Instability of the reduction algorithms as a function of the proxy set size $k$.}
    \label{fig:stability_full}
\end{figure}

\FloatBarrier  %
\clearpage  %

\section{Full sensitivity analysis}\label{a:cov_eps_sensitivity}
 
Many of the reduction methods outlined in this paper require the user to select an error threshold $\varepsilon$, with some additionally requiring a minimum coverage constraint $c$.
Since $\varepsilon$ is a parameter in computing coverage, its value can significantly impact the performance of the {\sc ExplainReduce} procedure.
In this section, we study the sensitivity of the procedure to these hyperparameters.

As the error tolerance $\varepsilon$ is defined in comparison to the absolute loss, the exact value of the parameter is task- and data-dependent.
In this section, we set the error tolerance as a quantile of the training loss of the underlying closed-box function $f$, i.e., $\varepsilon = q(\bm{L}_{BB}, p)$, where $q$ denotes the percentile function, $\bm{L}_{BB, i} = \ell(f(\bm{x}_i), \bm{y}_i)$ and $p$ is the percentile value.

As Fig. \ref{fig:coverage_p_full} shows, the reduction methods that require these parameters produce results with nearly identical fidelity across a wide range of both $c$ and $\varepsilon$.
The test fidelity of the proxy set for all tested reduction methods exhibits low variance when the error tolerance is set anywhere between the 10th and 50th percentiles of the closed-box training loss values.
Additionally, the loss-minimising reduction method with soft coverage produces similar fidelity results across the entire tested range of minimum coverage values, though it often fails to meet the coverage constraint at higher values.
It can be concluded that the reduction methods are not sensitive with respect to these hyperparameters.
For simplicity, we suggest using $c = 0.8$ and $\varepsilon = q(\bm{L}_{BB}, 0.2)$ as default arguments.
When not stated otherwise, these are also the $c$ and $\varepsilon$ values used in other experiments in this paper.

\begin{figure}[ht]
    \centering
    \includegraphics[width=\linewidth]{figures/coverage_p_sensitivity_full.pdf}
    \caption{Sensitivity of the reduction methods as a function of error tolerance and minimum coverage in case of the {\sc const min loss} algorithm.}
    \label{fig:coverage_p_full}
\end{figure}

\end{document}

The most common non-Euclidean GNN is based on either the Poincaré Ball model \citep{hgnn19liu} or the Lorentz model \citep{lorentz18nickle}. We discover that the Poincaré Ball model can be a special form of $\kappa$-stereographic model when setting the $\kappa$ to $-1$, which inspires us to investigate the general form of the Lorentz model. Following the definition of the $\kappa$-stereographic model, we generalize the Lorentz model as the $\kappa$-Lorentz model. Notice that we only provide a similar form to the $\kappa$-stereographic model, serving as the projection functions without considering the physical meaning. The $exp_{\vect{o}}^\kappa(\cdot)$ and $log_{\vect{o}}^\kappa(\cdot)$ for $\vect{x}\in \mathbb{R}^d$ are defined as follows: 
\begin{equation*}
exp_{\vect{x}'}^\kappa(\vect{x})=cos_\kappa(||\vect{x}||_L)\vect{x}'+sin_\kappa(||\vect{x}||_L)\frac{\vect{x}}{||\vect{x}||_L}
\end{equation*}
\begin{equation*}
log_{\vect{x}'}^\kappa(\vect{x})=d_\kappa(\vect{x}, \vect{x}')\frac{\vect{x}+\frac{1}{\kappa}\langle\vect{x}, \vect{x}'\rangle_L\vect{x}'}{||\vect{x}+\frac{1}{\kappa}\langle\vect{x}, \vect{x}'\rangle_L\vect{x}'||_L}
\end{equation*}
where $\langle\vect{x}, \vect{x}'\rangle_L=-x_0x_0'+x_1x_1'+...+x_dx_d'$, $||\vect{x}||_L=\sqrt{\langle\vect{x}, \vect{x}'\rangle_L}$, $d_\kappa(\vect{x}, \vect{x}')=cos_\kappa^{-1}(-\langle\vect{x}, \vect{x}'\rangle_L)$, and $cos_\kappa$ and $sin_\kappa$ are defined as: 
\begin{equation*}
\begin{aligned}
    \cos_\kappa(\vect{x})=
    \begin{cases}
    \frac{1}{\sqrt{-\kappa}}\cosh(\sqrt{-\kappa}\vect{x}), &\kappa < 0,\\
    \vect{x}, &\kappa=0,\\
    \frac{1}{\sqrt{\kappa}}\cos(\sqrt{\kappa}\vect{x}), &\kappa>0. 
    \end{cases}
\end{aligned}
\end{equation*}
\begin{equation*}
\begin{aligned}
    \sin_\kappa(\vect{x})=
    \begin{cases}
    \frac{1}{\sqrt{-\kappa}}\sinh(\sqrt{-\kappa}\vect{x}), &\kappa < 0,\\
    \vect{x}, &\kappa=0,\\
    \frac{1}{\sqrt{\kappa}}\sin(\sqrt{\kappa}\vect{x}), &\kappa>0. 
    \end{cases}
\end{aligned}
\end{equation*}
We replace the corresponding functions in our SpaceGNN framework to get SpaceGNN-L. As shown in Table \ref{tab:alternative}, SpaceGNN and SpaceGNN-L can have similar performance in terms of all the 9 datasets, which shows SpaceGNN-L can also outperform other baselines. These results demonstrate that our framework can be generalized to other base models. 

\section{Learned $\kappa$}
\label{subsec:learnedk}
\begin{table}[t]

\vspace{-2mm}
\small
\centering
\caption{Learned $\vect{\kappa}$}
\scalebox{0.9}{
\setlength\tabcolsep{3pt}
\label{tab:kappa}
\begin{tabular}{c|cccccc|cccccc}
\hline \hline
Datasets   & $\kappa_1^-$ & $\kappa_2^-$ & $\kappa_3^-$ & $\kappa_4^-$ & $\kappa_5^-$ & $\kappa_6^-$ & $\kappa_1^+$ & $\kappa_2^+$ & $\kappa_3^+$ & $\kappa_4^+$ & $\kappa_5^+$ & $\kappa_6^+$ \\ \hline
Weibo      & -0.1272 & -0.0766 & -0.0795 & -0.1176 & -0.1239 & -0.1103 & 0.0989  & 0.1233  & 0.0936  & 0.0961  & 0.0747  & 0.1116  \\
Reddit     & -0.0898 & -0.1012 & -0.1028 & -0.1048 & -0.1106 & -       & 0.0839  & 0.0990  & 0.1223  & 0.0939  & 0.0963  & -       \\
Tolokers   & -0.1046 & -       & -       & -       & -       & -       & 0.1259  & -       & -       & -       & -       & -       \\
Amazon     & -0.0873 & -0.1095 & -0.1054 & -       & -       & -       & 0.0768  & 0.0685  & 0.0864  & -       & -       & -       \\
T-Finance  & -0.1347 & -0.0701 & -0.0738 & -       & -       & -       & 0.0744  & 0.0652  & 0.0850  & -       & -       & -       \\
YelpChi    & -0.1014 & -       & -       & -       & -       & -       & 0.1369  & -       & -       & -       & -       & -       \\
Questions  & -0.0999 & -0.1013 & -0.0992 & -0.1004 & -0.0983 & -0.1008 & 0.1006  & 0.0998  & 0.0998  & 0.1006  & 0.1001  & 0.0998  \\
DGraph-Fin & -0.1262 & -0.0774 & -0.0803 & -0.1169 & -       & -       & 0.0784  & 0.0906  & 0.0994  & 0.1130  & -       & -       \\
T-Social   & -0.1440 & -0.0621 & -0.0669 & -0.1284 & -0.1386 & -0.1167 & 0.0992  & 0.1181  & 0.0950  & 0.0970  & 0.0804  & 0.1090 \\ \hline \hline
\end{tabular}
}\vspace{-4mm}
\end{table}



{\update In this section, we report the learned $\vect{\kappa}$ of the experiments in Tables \ref{tab:general50} and \ref{tab:gad50}. Notice that, for simplicity, we only include 1 Euclidean GNN, 1 Hyperbolic GNN, and 1 Spherical GNN in our framework. Recap from Section \ref{sec:method}, in our final architecture, we utilize a hyperparameter $L$ to control the number of layers of all three GNNs, and the number of entries in $\vect{\kappa}$ for each GNN is the same as the number of layers of it. Specifically, if $L$ is set to be 6, then there will be 6 entries in $\vect{\kappa}^0$ for Euclidean GNN, 6 entries in $\vect{\kappa}^-$ for Hyperbolic GNN, and 6 entries in $\vect{\kappa}^+$ for Spherical GNN. For $\vect{\kappa}^0$, we want the GNN to stay in the Euclidean space, so we set each entry in it to 0. For $\vect{\kappa}^-$ and $\vect{\kappa}^+$, we want the Hyperbolic GNN and Spherical GNN to search for the optimal curvatures for different datasets, so we set these two as learnable curvatures. Notice that, according to Table \ref{tab:setting}, the optimal $L$ varies by datasets, so the number of entries in learned $\vect{\kappa}^-$ and $\vect{\kappa}^+$ will also be different, as shown in Table \ref{tab:kappa}, where $"-"$ represents no such entry in the vector. 

As we can see from Table \ref{tab:kappa}, the learned values stay close to 0 after the learning process, which is aligned with the analysis of Section \ref{subsec:LSP}. As shown in Figure \ref{fig:curvature}, the largest $ER_\kappa$ will be obtained around 0, which further demonstrates our findings are effective for graph anomaly detection tasks. 
}


\section{Time Complexity Analysis}
\label{subsec:time}
{\update As shown in Section \ref{subsec:MSE}, our framework is composed of 1 Euclidean GNN, $H$ Hyperbolic GNN, and $S$ Spherical GNN. The differences between these GNNs are the projection functions and the Distance Aware
Propagation (DAP) component, but the time complexity of them is the same for different GNNs. Hence, We only need to analyze one of the GNNs. 

Our analysis of the GNN time complexity is primarily based on the Algorithm \ref{alg:base} in Appendix \ref{subsec:algorithm}, which illustrates the base architecture of each GNN. For simplicity, we focus on a single layer in the base architecture (Lines 3-7). 

First, in Line 3, we apply a two-layer MLP with time complexity of $O(|V|dd_1+|V|d_1d_2)$ followed by a projection function with time complexity of $O(|V|d_2)$, where $|V|$ is the total number of nodes in the graph, $d$ is the dimension of the node feature, and $d_1, d_2$ are the output dimension of the two MLPs, respectively. Thus, the total time complexity of Line 3 is $O(|V|dd_1+|V|d_1d_2)$. 

Then, in Line 5, we have to calculate the $\vect{\hat{s}}_{i}^{\kappa^l}$ for each node $i$. Specifically, for each edge connected to node $i$, we have a time complexity of $O(d_2)$ to get the corresponding coefficient. Thus, the total time complexity for all nodes in Line 5 would be $O(|E|d_2)$, where $|E|$ is the total number of edges in this graph. 

Afterward, in Line 6, for each edge between nodes $i$ and $j$, we need to calculate the $\omega_{ij}^{\kappa^l}$ with time complexity of $O(d_2^2)$, where the input dimension of the MLP is $2d_2$ and the output dimension of it is $d_2$, so the total complexity for all edges in Line 6 would be $O(|E|d_2^2)$.

Next, in Line 7, we also have to propagate the node embeddings for each edge in the graph, so the total time complexity of Line 7 is $O(|E|d_2)$. 

Finally, we combine the results before to get the time complexity of a single layer in the base architecture, i.e., $O(|V|dd_1+|V|d_1d_2+|E|d_2^2)$. 

According to the time analysis of GAT \citep{gat18velickovic}, one of the most popular architectures in the area of graph learning, the time complexity of a single GAT attention head computing $F_0$ features can be expressed as O($|V|FF_0 + |E|F_0$), where $F$ is the number of input features, and $|V|$ and $|E|$ are the numbers of nodes and edges in the graph, respectively. 

Hence, each layer of our proposed GNN has a similar time complexity to GAT by choosing the proper hyperparameters $d_1$ and $d_2$ in our architecture. In the experiments, we find that combining 1 Euclidean GNN, 1 Hyperbolic GNN, and 1 Spherical GNN in our framework is enough to achieve superior performance over all the other baselines, so the increase of time complexity by the Multiple Space Ensemble component will not be the limitation of our models in real applications. 
}

\section{Performance with More Training Data}
\label{subsec:moredata}

{ 
\begin{figure}[t]
\centering

  \begin{small}
    \begin{tabular}{cc}
        \multicolumn{2}{c}{\includegraphics[height=20mm]{figures/trainsz/legend.eps}}  \\ %[-5mm]
        \hspace{0.5mm}
        \includegraphics[height=40mm]{figures/trainsz/redditauc.eps} &
        \hspace{-16mm}
        \includegraphics[height=40mm]{figures/trainsz/redditf1.eps} \\ [-5mm]
        \hspace{-2mm}
        (a) Reddit AUC & 
        \hspace{-2mm}
        (b) Reddit F1 \\ 
    \end{tabular}
    \vspace{-3mm}
    \caption{Varying the training set size of Reddit.}
    \label{fig:redditsz}
    \vspace{-4mm}
  \end{small}
\end{figure}
\begin{figure}[t]
\centering
 \vspace{-3mm}
  \begin{small}
    \begin{tabular}{cc}
        %\multicolumn{2}{c}{\includegraphics[height=20mm]{figures/trainsz/legend.eps}}  \\ [-5mm]
        \hspace{-4mm}
        \includegraphics[height=40mm]{figures/trainsz/questionsauc.eps} &
        \hspace{-12mm}
        \includegraphics[height=40mm]{figures/trainsz/questionsf1.eps} \\ [-5mm]
        \hspace{-2mm}
        (a) Questions AUC & 
        \hspace{-2mm}
        (b) Questions F1 \\ 
    \end{tabular}
    \vspace{-3mm}
    \caption{Varying the training set size of Questions.}
    \label{fig:questionssz}
    \vspace{-4mm}
  \end{small}
\end{figure}
\begin{figure}[t!]
\centering
 \vspace{-3mm}
  \begin{small}
    \begin{tabular}{cc}
        %\multicolumn{2}{c}{\includegraphics[height=20mm]{figures/trainsz/legend.eps}}  \\ [-5mm]
        \hspace{-4mm}
        \includegraphics[height=40mm]{figures/trainsz/dgraphfinauc.eps} &
        \hspace{-12mm}
        \includegraphics[height=40mm]{figures/trainsz/dgraphfinf1.eps} \\ [-5mm]
        \hspace{-2mm}
        (a) DGraph-Fin AUC & 
        \hspace{-2mm}
        (b) DGraph-Fin F1 \\ 
    \end{tabular}
    \vspace{-3mm}
    \caption{Varying the training set size of DGraph-Fin.}
    \label{fig:dgraphfinsz}
    \vspace{-4mm}
  \end{small}
\end{figure}

\update To further demonstrate the superior ability of our proposed framework, we provide the performance on Reddit, Questions, and DGraph-Fin varying by the size of the training set, as shown in Figures \ref{fig:redditsz}, \ref{fig:questionssz}, and \ref{fig:dgraphfinsz}. Note that, since HYLA and GAGA can not successfully run on DGraph-Fin, we only report the performance of the other 14 baselines and our proposed SpaceGNN in Figure \ref{fig:dgraphfinsz}. 

As we can see, the red lines, which represent the performance of our SpaceGNN, are always on the top of the figures, which demonstrates that with more training data, our SpaceGNN can still outperform all the other baselines consistently in terms of both AUC and F1. In summary, such experiments further make our SpaceGNN a more general and practical algorithm.
}

\section{Performance on GADBench \citep{gadbench23tang} semi-supervised setting}
\label{subsec:gadbench}
{\update 

\begin{table}[t!]
\caption{AUC, AUPRC, and Rec@K scores (\%) on 9 datasets with data split of the semi-supervised setting in GADBench \citep{gadbench23tang}, compared with generalized models, where OOM represents out-of-memory.}
\vspace{-2mm}
\small
\centering
\scalebox{0.97}{
\setlength\tabcolsep{4pt}
\label{tab:gadbench1}
\begin{tabular}{cc|ccccccccc}
\hline \hline
Datasets                    & Metrics & MLP   & GCN   & SAGE  & GAT   & GIN   & HNN   & HGCN  & HYLA  & SpaceGNN       \\ \hline
\multirow{3}{*}{Weibo}      & AUC     & 0.666 & 0.935 & 0.818 & 0.864 & 0.838 & 0.747 & 0.942 & 0.960 & \textbf{0.964} \\
                            & AUPRC   & 0.562 & 0.860 & 0.585 & 0.733 & 0.676 & 0.312 & 0.808 & 0.727 & \textbf{0.864} \\
                            & Rec@K   & 0.532 & 0.792 & 0.634 & 0.702 & 0.665 & 0.371 & 0.757 & 0.736 & \textbf{0.795} \\ \hline
\multirow{3}{*}{Reddit}     & AUC     & 0.591 & 0.569 & 0.603 & 0.605 & 0.600 & 0.619 & 0.625 & 0.523 & \textbf{0.637} \\
                            & AUPRC   & 0.044 & 0.042 & 0.045 & 0.047 & 0.043 & 0.045 & 0.045 & 0.038 & \textbf{0.050} \\
                            & Rec@K   & 0.065 & 0.062 & 0.058 & 0.065 & 0.048 & 0.055 & 0.052 & 0.064 & \textbf{0.077} \\ \hline
\multirow{3}{*}{Tolokers}   & AUC     & 0.681 & 0.642 & 0.676 & 0.681 & 0.668 & 0.690 & 0.699 & 0.618 & \textbf{0.715} \\
                            & AUPRC   & 0.333 & 0.330 & 0.340 & 0.330 & 0.318 & 0.327 & 0.335 & 0.290 & \textbf{0.362} \\
                            & Rec@K   & 0.355 & 0.334 & 0.352 & 0.351 & 0.336 & 0.346 & 0.351 & 0.311 & \textbf{0.366} \\ \hline
\multirow{3}{*}{Amazon}     & AUC     & 0.922 & 0.820 & 0.814 & 0.924 & 0.916 & 0.861 & 0.792 & 0.717 & \textbf{0.947} \\
                            & AUPRC   & \textbf{0.830} & 0.328 & 0.425 & 0.816 & 0.754 & 0.785 & 0.306 & 0.168 & 0.812          \\
                            & Rec@K   & \textbf{0.793} & 0.369 & 0.480 & 0.771 & 0.704 & 0.776 & 0.356 & 0.236 & 0.782          \\ \hline
\multirow{3}{*}{T-Finance}  & AUC     & 0.899 & 0.883 & 0.689 & 0.850 & 0.845 & 0.880 & 0.933 & 0.615 & \textbf{0.949} \\
                            & AUPRC   & 0.534 & 0.605 & 0.117 & 0.289 & 0.448 & 0.677 & 0.799 & 0.063 & \textbf{0.849} \\
                            & Rec@K   & 0.599 & 0.606 & 0.185 & 0.362 & 0.544 & 0.638 & 0.760 & 0.080 & \textbf{0.796} \\ \hline
\multirow{3}{*}{YelpChi}    & AUC     & 0.647 & 0.512 & 0.589 & 0.656 & 0.629 & 0.662 & 0.480 & 0.551 & \textbf{0.726} \\
                            & AUPRC   & 0.236 & 0.164 & 0.209 & 0.250 & 0.237 & 0.263 & 0.141 & 0.174 & \textbf{0.331} \\
                            & Rec@K   & 0.265 & 0.169 & 0.229 & 0.281 & 0.265 & 0.296 & 0.149 & 0.200 & \textbf{0.366} \\ \hline
\multirow{3}{*}{Questions}  & AUC     & 0.612 & 0.600 & 0.612 & 0.623 & 0.622 & 0.601 & 0.575 & 0.619 & \textbf{0.650} \\
                            & AUPRC   & 0.077 & 0.061 & 0.055 & 0.073 & 0.067 & 0.047 & 0.039 & 0.057 & \textbf{0.097} \\
                            & Rec@K   & 0.120 & 0.098 & 0.088 & 0.109 & 0.103 & 0.051 & 0.030 & 0.106 & \textbf{0.145} \\ \hline
\multirow{3}{*}{DGraph-Fin} & AUC     & \textbf{0.691} & 0.662 & 0.648 & 0.672 & 0.657 & 0.644 & 0.638 & OOM   & 0.678          \\
                            & AUPRC   & 0.023 & 0.023 & 0.020 & 0.022 & 0.020 & 0.022 & 0.023 & OOM   & \textbf{0.025} \\
                            & Rec@K   & 0.034 & 0.036 & 0.025 & 0.031 & 0.021 & 0.013 & 0.027 & OOM   & \textbf{0.040} \\ \hline
\multirow{3}{*}{T-Social}   & AUC     & 0.591 & 0.716 & 0.720 & 0.754 & 0.704 & 0.473 & 0.435 & OOM   & \textbf{0.947} \\
                            & AUPRC   & 0.039 & 0.084 & 0.078 & 0.092 & 0.062 & 0.027 & 0.024 & OOM   & \textbf{0.642} \\
                            & Rec@K   & 0.032 & 0.102 & 0.095 & 0.116 & 0.053 & 0.009 & 0.001 & OOM   & \textbf{0.667} \\  \hline  \hline
\end{tabular}
}
%\vspace{-4mm}
\end{table}
\begin{table}[htbp!]
\caption{AUC, AUPRC, and Rec@K scores (\%) on 9 datasets with data split of the semi-supervised setting in GADBench \citep{gadbench23tang}, compared with specialized models, where TLE represents the experiment can not be conducted successfully within 72 hours. }
\vspace{-2mm}
\small
\centering
\scalebox{0.91}{
\setlength\tabcolsep{2pt}
\label{tab:gadbench2}
\begin{tabular}{cc|ccccccccc}
\hline \hline
Datasets                    & Metrics & AMNet & BWGNN & GDN   & SparseGAD & GHRN  & GAGA  & XGBGraph & CONSISGAD & SpaceGNN       \\ \hline
\multirow{3}{*}{Weibo}      & AUC     & 0.824 & 0.936 & 0.682 & 0.897     & 0.916 & 0.732 & \textbf{0.964}    & 0.873     & \textbf{0.964} \\
                            & AUPRC   & 0.671 & 0.806 & 0.582 & 0.696     & 0.770 & 0.376 & 0.759    & 0.654     & \textbf{0.864} \\
                            & Rec@K   & 0.621 & 0.751 & 0.560 & 0.678     & 0.724 & 0.324 & 0.689    & 0.583     & \textbf{0.795} \\ \hline
\multirow{3}{*}{Reddit}     & AUC     & 0.629 & 0.577 & 0.596 & 0.634     & 0.575 & 0.501 & 0.592    & 0.629     & \textbf{0.637} \\
                            & AUPRC   & 0.049 & 0.042 & 0.043 & 0.047     & 0.042 & 0.032 & 0.041    & 0.046     & \textbf{0.050} \\
                            & Rec@K   & 0.068 & 0.060 & 0.052 & 0.074     & 0.063 & 0.019 & 0.049    & 0.061     & \textbf{0.077} \\ \hline
\multirow{3}{*}{Tolokers}   & AUC     & 0.617 & 0.685 & 0.713 & 0.673     & 0.690 & 0.636 & 0.675    & 0.709     & \textbf{0.715} \\
                            & AUPRC   & 0.286 & 0.353 & 0.353 & 0.318     & 0.359 & 0.293 & 0.341    & 0.337     & \textbf{0.362} \\
                            & Rec@K   & 0.305 & 0.355 & 0.363 & 0.346     & 0.361 & 0.318 & \textbf{0.366}    & 0.364     & \textbf{0.366} \\ \hline
\multirow{3}{*}{Amazon}     & AUC     & 0.928 & 0.918 & 0.868 & 0.935     & 0.909 & 0.504 & \textbf{0.947}    & 0.933     & \textbf{0.947} \\
                            & AUPRC   & 0.824 & 0.817 & 0.691 & 0.800     & 0.807 & 0.148 & \textbf{0.844}    & 0.792     & 0.812          \\
                            & Rec@K   & 0.778 & 0.777 & 0.652 & \textbf{0.788}     & 0.777 & 0.143 & 0.782    & 0.775     & 0.782          \\ \hline
\multirow{3}{*}{T-Finance}  & AUC     & 0.926 & 0.921 & 0.900 & 0.944     & 0.926 & 0.725 & 0.948    & 0.932     & \textbf{0.949} \\
                            & AUPRC   & 0.602 & 0.609 & 0.671 & 0.835     & 0.634 & 0.252 & 0.783    & 0.815     & \textbf{0.849} \\
                            & Rec@K   & 0.657 & 0.649 & 0.656 & 0.794     & 0.677 & 0.400 & 0.724    & 0.758     & \textbf{0.796} \\ \hline
\multirow{3}{*}{YelpChi}    & AUC     & 0.648 & 0.643 & 0.670 & 0.639     & 0.645 & 0.549 & 0.640    & 0.715     & \textbf{0.726} \\
                            & AUPRC   & 0.239 & 0.237 & 0.244 & 0.213     & 0.238 & 0.173 & 0.248    & 0.330     & \textbf{0.331} \\
                            & Rec@K   & 0.266 & 0.264 & 0.278 & 0.222     & 0.269 & 0.187 & 0.268    & 0.358     & \textbf{0.366} \\ \hline
\multirow{3}{*}{Questions}  & AUC     & 0.636 & 0.602 & 0.609 & 0.574     & 0.605 & 0.513 & 0.614    & 0.649     & \textbf{0.650} \\
                            & AUPRC   & 0.074 & 0.065 & 0.070 & 0.036     & 0.065 & 0.039 & 0.077    & 0.085     & \textbf{0.097} \\
                            & Rec@K   & 0.127 & 0.109 & 0.097 & 0.032     & 0.111 & 0.072 & 0.106    & 0.092     & \textbf{0.145} \\ \hline
\multirow{3}{*}{DGraph-Fin} & AUC     & 0.671 & 0.655 & 0.660 & 0.674     & 0.671 & TLE   & 0.624    & 0.635     & \textbf{0.678} \\
                            & AUPRC   & 0.022 & 0.021 & 0.022 & 0.023     & 0.023 & TLE   & 0.019    & 0.017     & \textbf{0.025} \\
                            & Rec@K   & 0.026 & 0.031 & 0.032 & 0.022     & 0.034 & TLE   & 0.025    & 0.011     & \textbf{0.040} \\ \hline
\multirow{3}{*}{T-Social}   & AUC     & 0.537 & 0.775 & 0.716 & 0.766     & 0.787 & TLE   & 0.852    & 0.940     & \textbf{0.947} \\
                            & AUPRC   & 0.031 & 0.159 & 0.104 & 0.256     & 0.162 & TLE   & 0.406    & 0.484     & \textbf{0.642} \\
                            & Rec@K   & 0.016 & 0.243 & 0.199 & 0.362     & 0.246 & TLE   & 0.430    & 0.535     & \textbf{0.667} \\  \hline \hline
\end{tabular}
}
%\vspace{-4mm}
\end{table}

For a fair comparison, we also provide AUC, AUPRC, and Rec@K scores on 9 datasets with data split of the semi-supervised setting in GADBench \citep{gadbench23tang}. Specifically, in this setting, we use 20 positive labels (anomalous nodes) and 80 negative labels (normal nodes) for both the training set and the validation set in each dataset, separately. Note that, for the baselines in GADBench, we use the reported performance in it, and for baselines not in GADBench, we obtain the source code of all competitors from GitHub and execute these models using the default parameter settings suggested by their authors. The hyperparameters of SpaceGNN are set based on the same setting in GADBench, i.e., random search.

As we can see from Tables \ref{tab:gadbench1} and \ref{tab:gadbench2}, our proposed model can still outperform all the baselines on almost all the datasets consistently using AUC, AUPRC, and Rec@K scores as metrics, which demonstrates the effectiveness of our SpaceGNN. 
}

