\appendix
\newpage

\section*{Appendix}
\label{sec:appendix}

\section{Proofs}
\label{subsec:proof}
{\bf Proof of Theorem 1.}
Let $p$ denote $WH^\kappa$, then the information a normal node can gain within its neighborhood during a propagation process follows $\mathcal{N}(p\vect{\mu}_n+(1-p)\vect{\mu}_a, p^2\vect{\Sigma}_n+(1-p)^2\vect{\Sigma}_a)$ according to the linear properties of independent Gaussian variables. 

Let $\vect{X}$ and $\vect{Y}$ denote the distribution of the normal node and the information over $\mathbb{R}^d$, respectively. We then use Fréchet inception distance \citep{fid17heusel} to describe the distance between two distributions as follows: 
\begin{equation*}
\begin{aligned}
    F(\vect{X}, \vect{Y})^2&=(\inf_{\gamma\in \Gamma(\vect{X}, \vect{Y})}\int_{\mathbb{R}^d \times \mathbb{R}^d}||\vect{x}-\vect{y}||^2d\gamma(\vect{x}, \vect{y})), \\
    &=(\inf_{\gamma\in \Gamma(\vect{X}, \vect{Y})}\mathbb{E}_{(\vect{x}, \vect{y})\sim\gamma}[||\vect{x}-\vect{y}||^2]),
\end{aligned}
\end{equation*}
where $\Gamma(\vect{X}, \vect{Y})$ is the set of all measures on $\mathbb{R}^d\times\mathbb{R}^d$ with marginals $\vect{X}$ and $\vect{Y}$ on the first and second factors, separately. Hence, we have the following equation: 
\begin{equation*}
\begin{aligned}
    &\mathbb{E}_{(\vect{x}, \vect{y})\sim\gamma}[||\vect{x}-\vect{y}||^2]\\
    =&\mathbb{E}_{(\tilde{\vect{x}}, \tilde{\vect{y}})\sim\tilde{\gamma}}[||(\tilde{\vect{x}}+\vect{\mu_x})-(\tilde{\vect{y}}+\vect{\mu_y})||^2]\\
    =&\mathbb{E}_{(\tilde{\vect{x}}, \tilde{\vect{y}})\sim\tilde{\gamma}}[||\tilde{\vect{x}}-\tilde{\vect{y}}||^2+||\vect{\mu_x}-\vect{\mu_y}||^2+2\langle \tilde{\vect{x}}-\tilde{\vect{y}}, \vect{\mu_x}-\vect{\mu_y}\rangle]\\
    =&||\vect{\mu_x}-\vect{\mu_y}||^2+\mathbb{E}_{(\tilde{\vect{x}}, \tilde{\vect{y}})\sim\tilde{\gamma}}[||\tilde{\vect{x}}-\tilde{\vect{y}}||^2]
\end{aligned}
\end{equation*}
where $\vect{\mu_x}$ and $\vect{\mu_y}$ represent the mean value of distributions $\vect{X}$ and $\vect{Y}$, and $\tilde{\vect{x}}$ and $\tilde{\vect{y}}$ represent vectors following distribution $\tilde{\vect{X}}$ and $\tilde{\vect{Y}}$, which have 0 mean value and the same variance value as $\vect{X}$ and $\vect{Y}$, respectively. Hence, the Fréchet inception distance can be decomposed as:
\begin{equation*}
F(\vect{X}, \vect{Y})^2=||\vect{\mu_x}-\vect{\mu_y}||^2+F(\tilde{\vect{X}}, \tilde{\vect{Y}})^2
\end{equation*}
This result shows the distance between the distribution of the normal node and the information is determined by two parts, the mean value and the variance value. Specifically, we can assume $\vect{\Sigma}_n\approx\vect{\Sigma}_a\approx c\vect{I}$ in real NAD tasks, where $c$ is a small constant, due to the independent similar behaviors of nodes in the same category. Thus, we have $F(\tilde{\vect{X}}, \tilde{\vect{Y}})^2\approx0$ and $F(\vect{X}, \vect{Y})^2=||\vect{\mu_x}-\vect{\mu_y}||^2$. 

Then, we check the distance between mean values of $\vect{X}$ and $\vect{Y}$. Specifically, it can be written as: 
\begin{equation*}
||\vect{\mu_x}-\vect{\mu_y}||^2=(1-p)^2||\vect{\mu}_n-\vect{\mu}_a||^2
\end{equation*}
which concludes that if $||\vect{\mu}_n-\vect{\mu}_a||^2$ remains the same, as $p$ increases, the distance between the distribution of the normal node and the information will decrease, and thus the probability of a normal node following its original distribution after a propagation process increases as $WH_\kappa$ increases. 

The situation of an anomalous node can be analyzed accordingly. This solution concludes that weighted homogeneity can benefit the propagation procedure for NAD tasks. {\hfill \qedsymbol}

{\bf Proof of Theorem 2.} First, we apply Taylor expansion on $\tan_\kappa^{-1}(t)$ for a fixed $t$ when $\kappa\rightarrow 0^+$: 
\begin{equation*}
\begin{aligned}
    \tan_\kappa^{-1}(t)=&\kappa^{-\frac{1}{2}}\tan(\kappa^\frac{1}{2}t)\\
    =&\kappa^{-\frac{1}{2}}(\kappa^\frac{1}{2}t+\kappa^\frac{3}{2}\frac{t^3}{3}+\mathcal{O}(\kappa^\frac{5}{2}))\\
    =&t+\kappa\frac{t^3}{3} + \mathcal{O}(\kappa^2)
\end{aligned}
\end{equation*}
When $\kappa\rightarrow 0^-$:
\begin{equation*}
\begin{aligned}
    \tan_\kappa^{-1}(t)=&(-\kappa)^{-\frac{1}{2}}\tanh((-\kappa)^\frac{1}{2}t)\\
    =&(-\kappa)^{-\frac{1}{2}}((-\kappa)^\frac{1}{2}t-(-\kappa)^\frac{3}{2}\frac{t^3}{3}+\mathcal{O}(\kappa^\frac{5}{2}))\\
    =&t+\kappa\frac{t^3}{3}+\mathcal{O}(\kappa^2)
\end{aligned}
\end{equation*}
When $\kappa\rightarrow 0$, we also have $\tan_\kappa^{-1}(t)=t-\kappa\frac{t^3}{3}+\mathcal{O}(\kappa^2)$. Hence, we conclude that near 0, $\tan_\kappa^{-1}(t)=t-\kappa\frac{t^3}{3}+\mathcal{O}(\kappa^2)$. 

Then, we need to use the Tayler expansion for $||\cdot||$. Specifically, $||\vect{x}+\vect{o}||=||\vect{x}||+\langle \vect{x}, \vect{o}\rangle+\mathcal{O}(||\vect{o}||^2)$ when $\vect{o}\rightarrow \vect{0}$. 

After that, we derive the Tayler expansion for $\vect{x}\oplus_\kappa\vect{y}$ when $\kappa$ near 0: 
\begin{equation*}
\begin{aligned}
\vect{x}\oplus\vect{y}=&\frac{(1-2\kappa\vect{x^T}\vect{y}-\kappa||\vect{y}||^2)\vect{x}+(1+\kappa||\vect{x}||^2)\vect{y}}{1-2\kappa\vect{x}^T\vect{y}+\kappa^2||\vect{x}||^2||\vect{y}||^2}\\
=&((1-2\kappa\vect{x^T}\vect{y}-\kappa||\vect{y}||^2)\vect{x}+(1+\kappa||\vect{x}||^2)\vect{y})(1+2\kappa\vect{x^T}\vect{y}+\mathcal{O}(\kappa^2))\\
=&(1-2\kappa\vect{x^T}\vect{y}-\kappa||\vect{y}||^2)\vect{x}+(1+\kappa||\vect{x}||^2)\vect{y}+2\kappa\vect{x^T}\vect{y}(\vect{x}+\vect{y})+\mathcal{O}(\kappa^2)\\
=&(1-\kappa||\vect{y}||^2)\vect{x}+(1+\kappa||\vect{x}||^2)\vect{y}+2\kappa(\vect{x^T}\vect{y})\vect{y}+\mathcal{O}(\kappa^2)\\
=&\vect{x}+\vect{y}+\kappa(||\vect{x}||^2\vect{y}-||\vect{y}||^2\vect{x}+2(\vect{x^T}\vect{y})\vect{y})+\mathcal{O}(\kappa^2)
\end{aligned}
\end{equation*}

By combining the above three Tayler expansions and ignore $\mathcal{O}(\kappa^2)$, we have the following equation: 
\begin{equation*}
\begin{aligned}
d_\kappa(\vect{x}, \vect{y})=&2\tan_\kappa^{-1}(||(-\vect{x})\oplus_\kappa\vect{y}||)\\
=&2(||\vect{x}-\vect{y}||+\kappa((-\vect{x})^T\vect{y})||\vect{x}-\vect{y}||^2)(1-\frac{\kappa}{3}(||\vect{x}-\vect{y}||^2))\\
=&2||\vect{x}-\vect{y}||-2\kappa((\vect{x^T}\vect{y})||\vect{x}-\vect{y}||^2+\frac{||\vect{x}-\vect{y}||^3}{3})
\end{aligned}
\end{equation*}
which concludes our theorem. {\hfill \qedsymbol}

{\bf Proof of {\update Proposition} \ref{thm:ensemble}.} We use the weighted cross-entropy loss of $m$ single models to subtract the ensemble cross-entropy loss: 
\begin{equation*}
\begin{aligned}
\sum_{i=1}^m\alpha_i\mathcal{L}(\vect{p}, \vect{q}_i)-\mathcal{L}(\vect{p}, \bar{\vect{q}})=&\sum_{c=1}^C\vect{p}^c\log\bar{\vect{q}}^c-\sum_{i=1}^m\sum_{c=1}^C\alpha_i\vect{p}^c\log\vect{q}^c_i\\
=&\sum_{c=1}^C\vect{p}^c\log\bar{\vect{q}}^c-\sum_{c=1}^C\vect{p}^c\log(\prod_i(\vect{q}_i^c)^{\alpha_i})\\
=&\sum_{c=1}^C\vect{p}^c\log(\frac{\bar{\vect{q}}^c}{\prod_i(\vect{q}_i^c)^{\alpha_i}})\\
=&\Omega(\vect{p})
\end{aligned}
\end{equation*}

By applying the weighted AM–GM inequality, we have $\bar{\vect{q}}\geq\prod_i(\vect{q}_i^c)^{\alpha_i}$, which means the term inside the log function is greater or equal to 1, and so the $\Omega(\vect{p})$ is non-negative. Thus, it finishes the proof of the {\update Proposition}. {\hfill \qedsymbol}

{\bf Proof of {\update Proposition} \ref{thm:expectation}.} We take the expectation of the equation in {\update Proposition} \ref{thm:ensemble} over entire graph $G$ and apply KL bias-variance decomposition in previous study \citep{enstheo23wood}, then we have: 
\begin{equation*}
\begin{aligned}
\mathbb{E}_G[\mathcal{L}(\vect{p}, \bar{\vect{q}})]=&\mathbb{E}_G[\sum_{i=1}^m\alpha_i\mathcal{L}(\vect{p}, \vect{q}_i)]-\mathbb{E}_G[\Omega(\vect{p})]\\
=&\sum_{i=1}^m\alpha_i\mathcal{L}(\vect{p}, \vect{\hat{q}}_i)+\sum_{i=1}^m\alpha_i\mathbb{E}_G[\KL(\vect{\hat{q}}_i||\vect{q}_i)]-\mathbb{E}_G[\Omega(\vect{p})]\\
=&\sum_{i=1}^m\alpha_i\mathcal{L}(\vect{p}, \vect{\hat{q}}_i)+\Theta(\vect{p}), 
\end{aligned}
\end{equation*}
where $\Theta(\vect{p})$ is demonstrated as non-negative in previous study \citep{enstheo23wood}, and thus this {\update Proposition} is proven. {\hfill \qedsymbol}

\section{Datasets and Baselines}
\label{subsec:dataset-baseline}
\textbf{Datasets.} The datasets used in our experiments are from the most recent benchmark paper \citep{gadbench23tang}, according to which, Weibo, Reddit, Questions, and T-Social aim to detect anomalous accounts on social media, Tolokers, Amazon, and YelpChi are proposed for malicious comments detection in review platforms, and T-Finance and DGraph-Fin focus on fraud detection in financial networks. The statistics of these 9 real-world datasets are shown in Table \ref{tab:datasets}. 

\begin{table}[h] \footnotesize  \centering\resizebox{0.48\textwidth}{!}{\begin{tabular}{c|l|c|c|c}
\toprule
\textbf{Task} & \textbf{Dataset} & \textbf{N-shot} & \multirowcell{\textbf{Train texts} \\ \textbf{for STMD}} & \multirowcell{\textbf{Evaluation} \\ \textbf{texts}} \\
\midrule
\multirow{3}{*}{\multirowcell{Text \\ Summarization}} & CNN/DailyMail & 0 & 2,000 & 2,000 \\
& XSum & 0 & 2,000 & 2,000 \\
& SamSum & 0 & 2,000 & 819 \\
\midrule
\multirow{4}{*}{\multirowcell{QA \\ Long answer}} & PubMedQA & 0 & 2,000 & 2,000 \\
& MedQUAD & 5 & 2,000 & 2,000 \\
& TruthfulQA & 5 & 408 & 409 \\
& GSM8k & 5 & 2,000 & 1,319 \\
\midrule
\multirow{4}{*}{\multirowcell{QA \\ Short answer}} & SciQ & 0 & 5,000 & 1,000 \\
& CoQA & \multirowcell{all preceding \\ questions} & 5,000 & 2,000 \\
& TriviaQA & 5 & 5,000 & 2,000 \\
\midrule
\multirow{1}{*}{\multirowcell{MCQA}} & MMLU & 5 & 5,000 & 2,000 \\
\bottomrule
\end{tabular}
}\caption{\label{tab:dataset_stat} The statistics of the datasets used for evaluation.}
\end{table}

\textbf{Baselines.}  The first group is generalized models:
\begin{itemize}[topsep=0.5mm, partopsep=0pt, itemsep=0pt, leftmargin=10pt]
    \item MLP \citep{mlp58f}: A type of neural network with multiple layers of fully connected artificial neurons;
    \item GCN \citep{gcn17kipf}: A type of GNN that leverages convolution function on a graph to propagate information within the neighborhood of each node;
    \item GraphSAGE \citep{graphsage17hamilton}: A type of GNN that uses sampling technique to aggregate features from the neighborhood;
    \item GAT \citep{gat18velickovic}: A type of GNN that adopts an attention mechanism to assign different importance to different nodes within the neighborhood of each node;
    \item GIN \citep{gin19xu}: A type of GNN that captures the properties of a graph while following graph isomorphism;
    \item HNN \citep{hnn18ganea}: A type of neural network that projects data features into non-Euclidean space;
    \item HGCN \citep{hgcn19chami}: A type of GNN that embeds node representations into non-Euclidean space and propagates accordingly;
    \item HYLA \citep{hyla23yu}: A type of GNN combines both laplacian characteristics within a graph and the information from non-Euclidean space.
\end{itemize}
The second group is specialized models:
\begin{itemize}[topsep=0.5mm, partopsep=0pt, itemsep=0pt, leftmargin=10pt]
    \item AMMNet \citep{amnet22chai}: A method proposed to capture both low- and high-frequency spectral information to detect anomalies;
    \item BWGNN \citep{bwgnn22tang}: A method designed to handle the 'right-shift' phenomenon of graph anomalies in spectral space;
    \item GDN \citep{gdn23gao}: A method that aims to learn information from a graph of the dependence relationships between sensors;
    \item SparseGAD \citep{sparsegad23gong}: A method that leverages sparsification to mitigate the heterophily issues within the neighborhood of each node;
    \item GHRN \citep{ghrn23gao}: A method that tackles the heterophily problem in the spectral space of graph anomaly detection;
    \item GAGA \citep{gaga23wang}: A method that uses group aggregation to reduce the influence of low homophily;
    \item XGBGraph \citep{gadbench23tang}: A method that combines XGB and GIN to boost the expressiveness;
    \item CONSISGAD \citep{consisgad24chen}: A method that applies a pseudo-label generation technique to solve the limited supervision problem;
\end{itemize}

\section{Algorithm}
\label{subsec:algorithm}
\IncMargin{1em}
\vspace{-2mm}
\begin{algorithm}

\caption{$exp_{\vect{o}}^{\kappa}$/$log_{\vect{o}}^{\kappa}$}\label{alg:exp-log}
\KwIn{$G$}
\KwOut{$\vect{H}^\kappa$}
%$\hat{\kappa} \leftarrow $\;
$\hat{\vect{H}} \leftarrow \text{NORMALIZE}(\vect{H})$\;
\If {$\kappa<0$} {
    $\vect{H}^\kappa \leftarrow \frac{tanh(\sqrt{|\kappa|}\hat{\vect{H}})\vect{H}}{\sqrt{|\kappa|}\hat{\vect{H}}}$ if $exp_{\vect{o}}^{\kappa}$, else  $\frac{arctanh(\sqrt{|\kappa|}\hat{\vect{H}})\vect{H}}{\sqrt{|\kappa|}\hat{\vect{H}}}$\;
}
\ElseIf {$\kappa>0$} {
    $\vect{H}^\kappa \leftarrow \frac{tan(\sqrt{|\kappa|}\hat{\vect{H}})\vect{H}}{\sqrt{|\kappa|}\hat{\vect{H}}}$  if $exp_{\vect{o}}^{\kappa}$, else  $\frac{arctan(\sqrt{|\kappa|}\hat{\vect{H}})\vect{H}}{\sqrt{|\kappa|}\hat{\vect{H}}}$\;
}
\Else{
    $\vect{H}^\kappa \leftarrow \vect{H}$\;    
}

Return $\vect{H}^\kappa$\;

\end{algorithm}
\vspace{-2mm}
\DecMargin{1em}
\IncMargin{1em}
\vspace{-2mm}
\begin{algorithm}

\caption{$CLAMP_{\kappa}$}\label{alg:clamp}
\KwIn{$G$}
\KwOut{$\vect{H}^\kappa$}
$\hat{\vect{H}} \leftarrow \text{NORMALIZE}(\vect{H})$\;
$\epsilon\leftarrow 1^{-8}$\;
$\tau \leftarrow \frac{1-\epsilon}{\sqrt{|\kappa|}}$\;
\For{$i=1$ to $n$}{
    \For{$j=1$ to $d$} {
        \If {$\hat{\vect{H}}_{ij}>\tau$} {
            $\vect{H}^\kappa_{ij}\leftarrow \frac{\vect{H}_{ij}}{\tau\hat{\vect{H}}_{ij}}$
        }
        \Else {
            $\vect{H}^\kappa_{ij}\leftarrow \vect{H}_{ij}$
        }
    }
}

Return $\vect{H}^\kappa$\;

\end{algorithm}
\vspace{-2mm}
\DecMargin{1em}
\IncMargin{1em}
\vspace{-2mm}
\begin{algorithm}
\caption{$f_{\vect{\kappa}}^L$}\label{alg:base}
\KwIn{$G$}
\KwOut{$\vect{Z}^{\vect{\kappa}}$}
$\vect{H}^0\leftarrow \vect{X}$\;
\For{$l=0$ to $L - 1$}{
    $\vect{{\update E}}^{l}\leftarrow CLAMP_{\kappa^{l}}(\sigma(exp_{\vect{o}}^{\kappa^l}(\text{MLP}(\sigma(\text{MLP}(\vect{H}^l))))))$\;
    \For{$i=1$ to $n$}{
        $\hat{\vect{s}}_{i}^{\kappa^l}\leftarrow \vect{1}-\sigma([2||\vect{{\update E}}^{l}_i-\vect{{\update E}}^{l}_j||-2\kappa^l(\vect{{\update E}}^{l}_i)^T\vect{{\update E}}^{l}_j||\vect{{\update E}}^{l}_i-\vect{{\update E}}^{l}_j||^2+\frac{||\vect{{\update E}}^{l}_i-\vect{{\update E}}^{l}_j||^3}{3}: j\in N(i)])$\;
        $\omega_{ij}^{\kappa^l}\leftarrow \text{MLP}(\text{CONCAT}(\vect{{\update E}}^{l}_i, \hat{\vect{s}}_{ij}^{\kappa^l}\vect{{\update E}}^{l}_j))$\;
        $\vect{H}^{l+1}_i\leftarrow \text{SELU}(log_{\vect{o}}^{\kappa^l}(\vect{{\update E}}^{l}_i)+\sum_{j\in N(i)}\omega_{ij}^{\kappa^l}log_{\vect{o}}^{\kappa^l}(\vect{{\update E}}^{l}_j))$\;
        
    }
    %$\vect{H}\leftarrow\vect{H}^{l+1}$\;
}
$\vect{Z}^{\vect{\kappa}}\leftarrow \text{SOFTMAX}(\text{MLP}(\text{CONCAT}(\vect{H}^{0}, \vect{H}^{1}, ..., \vect{H}^{L})))$\;
Return $\vect{Z}^{\vect{\kappa}}$\;

\end{algorithm}
\vspace{-2mm}
\DecMargin{1em}
\IncMargin{1em}
\vspace{-2mm}
\begin{algorithm}[H]
\caption{SpaceGNN}\label{alg:spacegnn}
\KwIn{$G$, $L$}
\KwOut{$\vect{Z}$}
$\vect{Z}^{\vect{\kappa}^+}\leftarrow f_{\vect{\kappa}^+}^L(G)$, $\vect{Z}^{\vect{\kappa}^-}\leftarrow f_{\vect{\kappa}^-}^L(G)$, $\vect{Z}^{\vect{0}}\leftarrow f_{\vect{0}}^L(G)$\;
$\vect{Z}\leftarrow (1-\beta)((1-\alpha)\vect{Z}^{\vect{\kappa}^-}+\alpha\vect{Z}^{\vect{\kappa}^+})+\beta\vect{Z}^{\vect{0}}$\;
Return $\vect{Z}$\;

\end{algorithm}
\vspace{-2mm}
\DecMargin{1em}
We provide the detailed algorithm in this Section. In Algorithm \ref{alg:exp-log}, we calculate $exp_{\vect{o}}^\kappa(\cdot)$ and $log_{\vect{o}}^\kappa(\cdot)$ based on the original point $\vect{o}$ of the space with curvature $\kappa$. Besides, to satisfy the range of $log_{\vect{o}}^\kappa(\cdot)$, we utilized Algorithm \ref{alg:clamp} to prune the node representations. Moreover, we construct Algorithm \ref{alg:base} by utilizing Algorithms \ref{alg:exp-log} and \ref{alg:clamp}. Specifically, we use the approximated distance to calculate the similarities between nodes and their neighbors, and then leverage them as the corresponding coefficients during the propagation process. Notice, for each layer $l$ during the propagation, we assign a different learnable $\kappa^l$ to capture comprehensive information from different spaces. To simplify our architecture, we only use three base models, $f_{\vect{\kappa}^+}^L$, $f_{\vect{\kappa}^-}^L$, and $f_{\vect{0}}^L$, for constructing SpaceGNN. This simplification can reduce the running time cost, and allow us to investigate the effectiveness of different spaces on different datasets easily through the corresponding hyperparameters. After obtaining probability matrix $\vect{Z}$ from Algorithm \ref{alg:spacegnn}, we use the cross-entropy loss to update the framework. 

\section{Experimental Settings}
\label{subsec:setting}
We study (stochastic) gradient descent on the empirical risk
\begin{equation*}
\cL(w) = \frac{1}{n}\sum_{i=1}^n l(p_i(w))\, ,
\end{equation*}
where the loss function $l$ and the functions  $(p_i)_{i=1}^n$  are specified in the following assumptions. Note that the empirical risk for binary classification from Equation~\eqref{def:emp_risk_intro} is a special case of the above objective.

\begin{assumption}\label{hyp:loss_exp_log}\phantom{=}
  \begin{enumerate}[label=\roman*)]
    \item The loss is either the exponential loss, $l(q) = e^{-q}$, or the logistic loss, $l(q) = \log(1{+}e^{-q})$.
    \item There exists an integer $L \in \mathbb{N}^*$  such that, for all $1 \leq i \leq n$, the function $p_i$ is $L$-homogeneous\footnote{We recall that a mapping $f : \mathbb{R}^d \rightarrow \mathbb{R}$ is positively $L$-homogeneous if $f(\lambda w) = \lambda^L f(w)$ for all $w \in \mathbb{R}^d$ and $\lambda >0$.}, locally Lipschitz continuous and semialgebraic.
  \end{enumerate}
\end{assumption}
If the $p_i$'s were differentiable with respect to $w$, the chain rule would guarantee that
\begin{align*}
\nabla \mathcal{L}(w) = \frac{1}{n}\sum_{i=1}^n  l'(p_i(w)) \nabla p_i(w)\enspace.
\end{align*}
However, we only assume that the $p_i$'s are semialgebraic. While we could consider Clarke subgradients, the Clarke subgradient of operations on functions (e.g., addition, composition, and minimum) is only contained within the composition of the respective Clarke subgradients. This, as noted in Section~\ref{sec:cons_field}, implies that the output of backpropagation is usually not an element of a Clarke subgradient but a selection of some conservative set-valued field.
Consequently, for $1\leq i \leq n$, we consider $D_i : \bbR^d \rightrightarrows\bbR^d$, a conservative set-valued field of $p_i$, and a function $\sa_i : \bbR^d \rightarrow \bbR^d$ such that for all $w \in \bbR^d$, $\sa_i(w) \in D_i(w)$. Given a step-size $\gamma >0$, gradient descent (GD)\footnote{More precisely, this refers to conservative gradient descent. We use the term GD for simplicity, as conservative gradients behave similarly to standard gradients.} is then expressed as
\begin{equation*}\label{eq:gd_new}\tag{GD}
  w_{k+1} = w_k - \frac{\gamma}{n} \sum_{i=1}^n l'(p_i(w_k))\sa_i(w_k)\,.
\end{equation*}
For its stochastic counterpart, stochastic gradient descent (SGD), we fix a batch-size $1\leq n_b \leq n$. At each iteration $k \in \bbN$, we randomly and uniformly draw a batch $B_k \subset \{1, \ldots, n \}$ of size $n_b$. The update rule is then given by 
\begin{equation*}\label{eq:sgd_new}\tag{SGD}
  w_{k+1} = w_k -  \frac{\gamma}{n_b}\sum_{i\in B_k} l'(p_i(w_k)) \sa_i(w_k)\, .
\end{equation*}
The considered conservative set-valued fields will satisfy an Euler lemma-type assumption.
%\nic{Smoother transition}
\begin{assumption}\phantom{=}\label{hyp:conserv}
  For every $i \leq n$, $\sa_i$ is measurable and $D_i$ is semialgebraic. Moreover, for every $w \in \bbR^d$ and $\lambda \geq 0$, $\sa_i(w)  \in D_i(w)$,
  \begin{equation*}
    D_i(\lambda w) = \lambda^{L-1} D_i(w)\, , \textrm{ and } \quad   L p_i(w) = \scalarp{\sa_i(w)}{w}\, .
  \end{equation*}
\end{assumption}
%\nic{Smoother transition}
Having in mind the binary classification setting, in which $p_i(w) = y_i \Phi(x_i, w)$, we define the margin
\begin{equation}\label{def:marg}
  \sm: \bbR^d \rightarrow \bbR, \quad \sm(w) = \min_{1\leq i \leq n} p_i(w)\, .
\end{equation}
It quantifies the quality of a prediction rule $\Phi(\cdot, w)$. In particular,  the training data is perfectly separated when $\sm(w) >0$. A binary prediction for $x$ is given by the sign of $\Phi(x, w)$, and under the homogeneity assumption, it depends only on the normalized direction $w / \norm{w}$. Consequently, we will focus on the sequence of directions $u_k := w_k / \norm{w_k}$. Our final assumption ensures that the normalized directions $(u_k)$ have stabilized in a region where the training data is correctly classified.

\begin{assumption}\label{hyp:marg_lowb}
  Almost surely, $\liminf \sm(u_k) >0$.
\end{assumption}
Before presenting our main result, we comment on our assumptions.

\paragraph{On Assumption~\ref{hyp:loss_exp_log}.} As discussed in the introduction, the primary example we consider is when $p_i(w) = y_i \Phi(x_i;w)$ is the signed prediction of a feedforward neural network without biases and with piecewise linear activation functions on a labeled dataset $((x_i,y_i))_{i \leq n}$. In this case,
\begin{equation}\label{eq:NN}
 p_i(w) = y_i \Phi(w;x_i) = y_i V_L(W_L) \sigma(V_{L-1}(W_{L-1}) \sigma(V_{L-1}(W_{L-2}) \ldots \sigma(V_{1}(W_1 x_i))))\, ,
\end{equation}
where $w = [W_1, \ldots, W_L]$, $W_i$ represents the weights of the $i$-th layer, $V_i$ is a linear function in the space of matrices (with $V_i$ being the identity for fully-connected layers) and $\sigma$ is a coordinate-wise activation function such as $z \mapsto \max(0,z)$ ($\ReLU$), $z \mapsto \max(az, z)$ for a small parameter $a>0$ (LeakyReLu) or $z \mapsto z$. Note that the mapping $w \mapsto p_i(w)$ is semialgebraic and $L$-homogeneous for any of these activation functions. Regarding the loss functions, the logistic and exponential losses are among the most commonly studied and widely used. In Appendix~\ref{app:gen_sett}, we extend our results to a broader class of losses, including $l(q) = e^{-q^a}$ and $l(q) = \ln (1 + e^{-q^a})$ for any $a \geq 1$.

\paragraph{On Assumption~\ref{hyp:conserv}.} Assumption~\ref{hyp:conserv} holds automatically  if $D_i$ is the Clarke subgradient of $p_i$. Indeed, at any vector $w \in \bbR^d$, where $p_i$ is differentiable it holds that $p_i(\lambda w) = \lambda^{L} p_i(w)$. Differentiating relatively to $w$ and $\lambda$ (noting that $p_i$ remains differentiable at $\lambda w$ due to homogeneity), we obtain $\lambda \nabla p_i(\lambda w) = \lambda^{L} \nabla p_i(w)$ and $\scalarp{\nabla p_i(\lambda w)}{w} = L \lambda^{L-1} p_i(w)$. The expression for any element of the Clarke subgradient then follows from~\eqref{eq:def_clarke}. 

However, for an arbitrary conservative set-valued field, Assumption~\ref{hyp:conserv} does not necessarily hold. For instance, $D(x) = \mathds{1}(x \in \mathbb{N})$ is a conservative set-valued field for $p \equiv 0$, which does not satisfy Assumption~\ref{hyp:conserv}. Nevertheless, in practice, conservative set-valued fields naturally arise from a formal application of the chain rule. For a non-smooth but homogeneous activation function $\sigma$, one selects an element $e \in \partial \sigma (0)$, and computes $\sa_i(w)$ via backpropagation. Whenever a gradient candidate of $\sigma$ at zero is required (i.e., in~\eqref{eq:NN}, for some $j$, $V_j(W_j)$ contains a zero entry), it is replaced by $e$. 
Since $V_j(W_j)$ and $V_j(\lambda W_j)$ have the same zero elements, it follows that for every such $w$, $
\sa_i(\lambda w) = \lambda^L \sa_i(w)$. The conservative set-valued field $D_i$ is then obtained by associating to each $w$ the set of all possible outcomes of the chain rule, with $e$ ranging over all elements of $\partial \sigma(0)$. Thus, for such fields, Assumption~\ref{hyp:conserv} holds.


\paragraph{On Assumption~\ref{hyp:marg_lowb}.} Training typically continues even after the training error reaches zero.
Assumption~\ref{hyp:marg_lowb} characterizes this late-training phase, where our result applies. 
As noted earlier, since $\sm$ is $L$-homogeneous, the classification rule is determined by the direction of the  iterates $u_k=w_k/\norm{w_k}$. Assumption~\ref{hyp:marg_lowb} then states that, beyond some iteration, the normalized margin remains positive. 
This assumption is natural in the context of studying the implicit bias of SGD: we \emph{assume} that we reached the phase in which the dataset is correctly classified and \emph{then} characterize the limit points. A similar perspective was taken in  \cite{nacson2019lexicographic}, where the implicit bias of GF was analyzed under the assumption that the sequence of directions and the loss converge. However, unlike their approach, ours does not require assuming such convergence a priori.

Earlier works such as \cite{ji2020directional,Lyu_Li_maxmargin}, which analyze subgradient flow or smooth GD, establish convergence by assuming the existence of a single iterate $w_{k_0}$ satisfying $\sm(w_{k_0}) > \varepsilon$ and then proving that $\lim \sm(u_{k}) > 0$. Their approach relies on constructing a smooth approximation of the margin, which increases during training, ensuring that $\sm(u_k) > 0$ for all iterates with $k \geq k_0$. This is feasible in their setting, as they study either subgradient flow or GD with smooth $p_i$’s, allowing them to leverage the descent lemma.

In contrast, our analysis considers a nonsmooth and stochastic setting, in which, even if an iterate $w_{k_0}$ satisfying $\sm(w_{k_0}) > \varepsilon$ exists, there is no a priori assurance that subsequent iterates remain in the region where Assumption~\ref{hyp:marg_lowb} holds. From this perspective, Assumption~\ref{hyp:marg_lowb} can be viewed as a stability assumption, ensuring that iterates continue to classify the dataset correctly. Establishing stability for stochastic and nonsmooth algorithms is notoriously hard, and only partial results in restrictive settings exist \cite{borkar2000ode,ramaswamy2017generalization,josz2024global}.

%Finally, note that Assumption~\ref{hyp:marg_lowb} only needs to hold almost surely. Specifically, with probability 1, there exist $k_0$ and $\varepsilon$ such that for all $k \geq k_0$, $\sm(u_k) \geq \varepsilon > 0$. In the case of~\eqref{eq:sgd_new}, $k_0$ and $\delta$ are random variables and may take different values across different realizations. 

%\paragraph{On constant stepsizes.}
%We allow the step size to be a constant of arbitrary magnitude, subject to the stability Assumption~\ref{hyp:marg_lowb}. This may seem surprising in a nonsmooth and stochastic setting, where a vanishing step size is typically required to ensure convergence (see, e.g., \cite{majewski2018analysis, dav-dru-kak-lee-19, bolte2023subgradient, le2024nonsmooth}).
\begin{figure}[t]
\centering
  \begin{small}
    \begin{tabular}{cc}
        \multicolumn{2}{c}{\includegraphics[height=4mm]{figures/hyperparameter1/legend1.eps}}  \\ [-3mm]
        \hspace{-4mm}
        \includegraphics[height=33mm]{figures/hyperparameter1/hdim.eps} &
        \hspace{-4mm}
        \includegraphics[height=33mm]{figures/hyperparameter1/layer.eps} \\ [-2mm]
        \hspace{-2mm}
        (a) Varying Hidden Dimension & 
        \hspace{-2mm}
        (b) Varying Layer \\ 
    \end{tabular}
    \vspace{-2mm}
    \caption{Varying the Hidden Dimension and Layer.}
    \label{fig:hyperparameter1}
    \vspace{-4mm}
  \end{small}
\end{figure}
\begin{figure}[t!]
\centering
  \begin{small}
  
  \vspace{-4mm}
    \begin{tabular}{ccc}
        %\multicolumn{3}{c}{\includegraphics[height=10mm]{figure/observation/mcf/mcf.eps}}  \\[-6mm]
        \hspace{-10mm}
        \includegraphics[height=42mm]{figures/hyperparameter2/Amazon.eps} &
        \hspace{-10mm}
        \includegraphics[height=42mm]{figures/hyperparameter2/T-Finance.eps} &
        \hspace{-10mm}
        \includegraphics[height=42mm]{figures/hyperparameter2/T-Social.eps} \\ [-0mm]
        \hspace{-9mm}
        (a) Amazon & 
        \hspace{-9mm}
        (b) T-Finannce &
        \hspace{-9mm}
        (c) T-Social \\ 
    \end{tabular}
    \vspace{-2mm}
    \caption{Varying $\alpha$ and $\beta$ on different datasets.}
    \label{fig:hyperparameter2}
  \vspace{-8mm}
  \end{small}
\end{figure}

Table \ref{tab:setting} provides a comprehensive list of our hyperparameters. We use grid search to train the model that yields the best F1 score on the validation set and report the corresponding test performance. Specifically, Learning Rate is searched from the set $\{0.001, 0.0001\}$, Hidden Dimension is chosen from the set $\{32, 64, 128, 256\}$, Layer ranges from $1$ to $6$, Dropout is obtained from the set $\{0, 0.05, 0.1\}$, Batch Size is fixed based on the size of the training set, and $\alpha$ and $\beta$ are from the set $\{0, 0.5, 1\}$, respectively. In the following Section \ref{subsec:parameter}, we will further analyze the influence of Hidden Dimension, Layer, and $\alpha$ and $\beta$ on the F1 scores of different datasets. 

\section{Parameter Analysis}
\label{subsec:parameter}


In this Section, we investigate the impact of Hidden Dimension, Layer, and $\alpha$ and $\beta$ on three different datasets, and present their F1 scores. 

Figure \ref{fig:hyperparameter1} reports the F1 score of SpaceGNN as we vary the Hidden Dimension from $32$ to $256$, and the Layer from $1$ to $6$. As we can observe, when we set the Hidden Dimension to $128$, SpaceGNN achieves relatively satisfactory performances on these three datasets. When we vary the Layer, we find for different datasets, the optimal value can be different. Specifically, we set it to $3$ for Amazon and T-Finance, and $6$ for T-Social to get the best performance. 

Figure \ref{fig:hyperparameter2} reports the F1 score of SpaceGNN as we vary $\alpha$ and $\beta$ from $0$ to $1$. These two hyperparameters represent the influence of diverse spaces on the datasets, so for different datasets, the optimal value will be distinct. Specifically, we set $\alpha$ to $0$ for Amazon, $1$ for T-Finance, and $0.5$ for T-Social, and we set $\beta$ to $0$ for Amazon, $1$ for T-Finance and T-Social to get the satisfactory performance. 

\section{Ablation Study}
\label{subsec:ablation}



% Table generated by Excel2LaTeX from sheet 'abaltion'
\begin{table}[t]
  \centering
  \resizebox{\linewidth}{!}{
    \begin{tabular}{rccccc}
    \toprule
    Method & ID & CAG   & PopQA & WebQuestion & \multirow{2}[2]{*}{Avg}\\
        & F1    & EM    & EM    & EM    &  \\
    \midrule
    \multicolumn{1}{l}{DeepRAG} & \textbf{52.40} & \textbf{61.92 } & \textbf{47.80 } & \textbf{45.24 } & \textbf{47.67 } \\
    all-node & 50.92   & 50.47  & 41.50  & 32.70  & 45.30  \\
    sentence-wise & 30.16   & 12.46  & 20.00  & 12.90  & 21.14 \\
    \bottomrule
    \end{tabular}%
    }
  \caption{Experiment results of the ablation study on the Chain of Calibration Stage.}
  \label{tab:dpo-abla}%
\end{table}%


To investigate the usefulness of the LSP and DAP components, we provide the ablation study of them on 9 datasets in Table \ref{tab:ablation}. Specifically, we set the $\kappa$ as fixed values for different spaces during the w/o LSP experiment and set the $\hat{\vect{s}}_i$ as $\vect{1}$ for each node $i$ during the w/o DAP experiment. As shown in Table \ref{tab:ablation}, SpaceGNN consistently outperforms w/o LSP and w/o DAP by a large margin, which demonstrates the benefits of these two components. 

\section{Additional Experimental Results}
\label{subsec:Additional}
\begin{table}[t]
\caption{AUC and F1 scores (\%) on 9 datasets with random split, compared with generalized models, where OOM represents out-of-memory.}
\vspace{-2mm}
\small
\centering
\scalebox{1}{
\setlength\tabcolsep{3.5pt}
\label{tab:general10}
\begin{tabular}{cc|ccccccccc}
\hline \hline
Datasets                    & Metrics & MLP    & GCN    & SAGE   & GAT    & GIN    & HNN             & HGCN   & HYLA            & SpaceGNN        \\ \hline
\multirow{2}{*}{Weibo}      & AUC     & 0.2856 & 0.6223 & 0.6176 & 0.8077 & 0.4452 & 0.4844          & 0.8020 & \textbf{0.9351} & 0.8364          \\
                            & F1      & 0.5347 & 0.6215 & 0.4732 & 0.5127 & 0.5230 & 0.4727          & 0.4721 & 0.7008          & \textbf{0.7158} \\ \hline
\multirow{2}{*}{Reddit}     & AUC     & 0.5320 & 0.5865 & 0.5820 & 0.5421 & 0.5217 & 0.5266          & 0.5196 & 0.4734          & \textbf{0.5868} \\
                            & F1      & 0.4916 & 0.4916 & 0.4916 & 0.4916 & 0.4916 & 0.4916          & 0.4916 & 0.4916          & \textbf{0.4916} \\ \hline
\multirow{2}{*}{Tolokers}   & AUC     & 0.4632 & 0.5355 & 0.4119 & 0.6453 & 0.6030 & 0.5718          & 0.6495 & 0.4927          & \textbf{0.6952} \\
                            & F1      & 0.4375 & 0.5093 & 0.4423 & 0.5075 & 0.5621 & 0.5127          & 0.5522 & 0.5004          & \textbf{0.6026} \\ \hline
\multirow{2}{*}{Amazon}     & AUC     & 0.8550 & 0.7954 & 0.6162 & 0.5210 & 0.8254 & 0.7098          & 0.7468 & 0.7185          & \textbf{0.8722} \\
                            & F1      & 0.8261 & 0.6441 & 0.3725 & 0.3668 & 0.2256 & 0.4822          & 0.4822 & 0.5879          & \textbf{0.8641} \\ \hline
\multirow{2}{*}{T-Finance}  & AUC     & 0.9058 & 0.6796 & 0.6512 & 0.6327 & 0.7567 & 0.8761          & 0.0719 & 0.3917          & \textbf{0.9349} \\
                            & F1      & 0.7856 & 0.3068 & 0.4627 & 0.5293 & 0.7681 & \textbf{0.8204} & 0.4883 & 0.4919          & 0.8031          \\ \hline
\multirow{2}{*}{YelpChi}    & AUC     & 0.5064 & 0.4845 & 0.4975 & 0.5546 & 0.6082 & 0.3702          & 0.4745 & 0.5426          & \textbf{0.6191} \\
                            & F1      & 0.5064 & 0.4608 & 0.4980 & 0.5220 & 0.5332 & 0.4608          & 0.4608 & 0.4644          & \textbf{0.5475} \\ \hline
\multirow{2}{*}{Questions}  & AUC     & 0.5299 & 0.4684 & 0.5654 & 0.5554 & 0.5597 & 0.5177          & 0.5057 & 0.4049          & \textbf{0.5851} \\
                            & F1      & 0.4645 & 0.4924 & 0.4371 & 0.4923 & 0.4492 & \textbf{0.5089} & 0.5083 & 0.4924          & 0.4924          \\ \hline
\multirow{2}{*}{DGraph-Fin} & AUC     & 0.4356 & 0.3900 & 0.5794 & 0.4103 & 0.4088 & 0.3282          & 0.3322 & OOM             & \textbf{0.6515} \\
                            & F1      & 0.4531 & 0.4815 & 0.4994 & 0.4282 & 0.3787 & 0.4968          & 0.3312 & OOM             & \textbf{0.5030} \\ \hline
\multirow{2}{*}{T-Social}   & AUC     & 0.5377 & 0.6183 & 0.6948 & 0.6958 & 0.5554 & 0.4694          & 0.4297 & OOM             & \textbf{0.9019} \\
                            & F1      & 0.1373 & 0.2554 & 0.5421 & 0.5501 & 0.3733 & 0.4924          & 0.4923 & OOM             & \textbf{0.7320} \\ \hline \hline
\end{tabular}
}\vspace{-4mm}
\end{table}


\begin{table}[h]
\caption{AUC and F1 scores (\%) on 9 datasets with random split, compared with specialized models, where TLE represents the experiment can not be conducted successfully within 72 hours. }
\vspace{-2mm}
\small
\centering
\scalebox{0.91}{
\setlength\tabcolsep{2pt}
\label{tab:gad10}
\begin{tabular}{cc|ccccccccc}
\hline \hline
Datasets                    & Metrics & AMNet           & BWGNN  & GDN    & SparseGAD & GHRN            & GAGA   & XGBGraph & CONSISGAD       & SpaceGNN        \\ \hline
\multirow{2}{*}{Weibo}      & AUC     & 0.4206          & 0.7557 & 0.7751 & 0.4558    & 0.6349          & 0.7597 & 0.5660   & 0.3972          & \textbf{0.8364} \\
                            & F1      & 0.5328          & 0.7106 & 0.1664 & 0.4724    & 0.6379          & 0.6574 & 0.5061   & 0.4447          & \textbf{0.7158} \\ \hline
\multirow{2}{*}{Reddit}     & AUC     & \textbf{0.6002} & 0.5815 & 0.4436 & 0.5263    & 0.5513          & 0.5068 & 0.5030   & 0.5536          & 0.5868          \\
                            & F1      & 0.4365          & 0.4617 & 0.4916 & 0.4721    & 0.4093          & 0.4916 & 0.4916   & 0.4514          & \textbf{0.4916} \\ \hline
\multirow{2}{*}{Tolokers}   & AUC     & 0.5627          & 0.5725 & 0.6159 & 0.4792    & 0.5688          & 0.6327 & 0.6083   & 0.5843          & \textbf{0.6952} \\
                            & F1      & 0.4451          & 0.5312 & 0.4665 & 0.4388    & 0.5449          & 0.4388 & 0.4989   & 0.5258          & \textbf{0.6026} \\ \hline
\multirow{2}{*}{Amazon}     & AUC     & 0.8356          & 0.7702 & 0.8335 & 0.7249    & 0.8028          & 0.7795 & 0.7666   & 0.8435          & \textbf{0.8722} \\
                            & F1      & 0.7144          & 0.5834 & 0.1685 & 0.4822    & 0.6777          & 0.6674 & 0.4822   & 0.8475          & \textbf{0.8641} \\ \hline
\multirow{2}{*}{T-Finance}  & AUC     & 0.8302          & 0.7318 & 0.5899 & 0.3650    & 0.7895          & 0.8157 & 0.8570   & 0.8503          & \textbf{0.9349} \\
                            & F1      & 0.5692          & 0.5025 & 0.5568 & 0.4883    & 0.5652          & 0.4894 & 0.7406   & \textbf{0.8316} & 0.8031          \\ \hline
\multirow{2}{*}{YelpChi}    & AUC     & 0.4738          & 0.5058 & 0.4893 & 0.5190    & 0.4231          & 0.4671 & 0.4927   & 0.5927          & \textbf{0.6191} \\
                            & F1      & 0.4875          & 0.4614 & 0.4977 & 0.4608    & 0.4608          & 0.4919 & 0.4608   & 0.5403          & \textbf{0.5475} \\ \hline
\multirow{2}{*}{Questions}  & AUC     & 0.4971          & 0.4125 & 0.5094 & 0.5185    & 0.5062          & 0.5361 & 0.5122   & 0.5492          & \textbf{0.5851} \\
                            & F1      & 0.4843          & 0.4924 & 0.4855 & 0.4988    & \textbf{0.5125} & 0.4944 & 0.4924   & 0.4935          & 0.4924          \\ \hline
\multirow{2}{*}{DGraph-Fin} & AUC     & 0.3812          & 0.6343 & 0.3200 & 0.3346    & 0.3734          & TLE    & 0.5009   & 0.6469          & \textbf{0.6515} \\
                            & F1      & 0.4128          & 0.4909 & 0.2641 & 0.4970    & 0.4871          & TLE    & 0.4968   & 0.4224          & \textbf{0.5030} \\ \hline
\multirow{2}{*}{T-Social}   & AUC     & 0.4745          & 0.6408 & 0.5480 & 0.3317    & 0.6319          & TLE    & 0.5066   & 0.8614          & \textbf{0.9019} \\
                            & F1      & 0.4810          & 0.4487 & 0.5124 & 0.4923    & 0.3435          & TLE    & 0.4923   & 0.5890          & \textbf{0.7320} \\ \hline \hline
\end{tabular}
}\vspace{-4mm}
\end{table}
In addition to the experiments in Section \ref{sec:experiments}, we further compare our SpaceGNN with baseline models on datasets with different sizes of training/validation/testing sets. Specifically, in experiments of Tables \ref{tab:general10} and \ref{tab:gad10}, we randomly divide each dataset into 10/10 for training/validation, and the rest of the nodes for testing, and in experiments of Tables \ref{tab:general100} and \ref{tab:gad100}, we randomly divide each dataset into 100/100 for training/validation, and the rest of the nodes for testing. 

\begin{table}[t]
\caption{AUC and F1 scores (\%) on 9 datasets with random split, compared with generalized models, where OOM represents out-of-memory.}
\vspace{-2mm}
\small
\centering
\scalebox{1}{
\setlength\tabcolsep{3.5pt}
\label{tab:general100}
\begin{tabular}{cc|ccccccccc}
\hline \hline
Datasets                    & Metrics & MLP    & GCN    & SAGE   & GAT    & GIN    & HNN    & HGCN   & HYLA   & SpaceGNN        \\ \hline
\multirow{2}{*}{Weibo}      & AUC     & 0.4438 & 0.9066 & 0.8157 & 0.8401 & 0.8541 & 0.7243 & 0.8545 & 0.9159 & \textbf{0.9521} \\
                            & F1      & 0.6538 & 0.8444 & 0.4746 & 0.7941 & 0.6561 & 0.6545 & 0.7620 & 0.4965 & \textbf{0.8481} \\ \hline
\multirow{2}{*}{Reddit}     & AUC     & 0.5907 & 0.5725 & 0.5767 & 0.6001 & 0.4793 & 0.5330 & 0.5315 & 0.4714 & \textbf{0.6232} \\
                            & F1      & 0.4916 & 0.4916 & 0.4916 & 0.4916 & 0.4916 & 0.4916 & 0.4916 & 0.4916 & \textbf{0.5228} \\ \hline
\multirow{2}{*}{Tolokers}   & AUC     & 0.7050 & 0.6952 & 0.7101 & 0.7139 & 0.7067 & 0.7063 & 0.7115 & 0.6402 & \textbf{0.7140} \\
                            & F1      & 0.5427 & 0.5978 & 0.5776 & 0.5844 & 0.5835 & 0.4711 & 0.5510 & 0.4864 & \textbf{0.6040} \\ \hline
\multirow{2}{*}{Amazon}     & AUC     & 0.8647 & 0.7936 & 0.7748 & 0.8808 & 0.9186 & 0.8635 & 0.7719 & 0.7188 & \textbf{0.9428} \\
                            & F1      & 0.7273 & 0.6167 & 0.6310 & 0.4354 & 0.7272 & 0.7711 & 0.5620 & 0.4822 & \textbf{0.9069} \\ \hline
\multirow{2}{*}{T-Finance}  & AUC     & 0.8960 & 0.8916 & 0.6722 & 0.8647 & 0.8087 & 0.8768 & 0.9329 & 0.3982 & \textbf{0.9486} \\
                            & F1      & 0.5737 & 0.7507 & 0.6042 & 0.8025 & 0.7680 & 0.8384 & 0.8753 & 0.4883 & \textbf{0.8789} \\ \hline
\multirow{2}{*}{YelpChi}    & AUC     & 0.7113 & 0.5160 & 0.5217 & 0.7249 & 0.7052 & 0.7119 & 0.5642 & 0.5508 & \textbf{0.7321} \\
                            & F1      & 0.6081 & 0.4608 & 0.4838 & 0.6256 & 0.6164 & 0.5919 & 0.4833 & 0.4608 & \textbf{0.6256} \\ \hline
\multirow{2}{*}{Questions}  & AUC     & 0.4707 & 0.6130 & 0.6000 & 0.5847 & 0.5083 & 0.5098 & 0.5081 & 0.4055 & \textbf{0.6476} \\
                            & F1      & 0.4961 & 0.4924 & 0.4999 & 0.5020 & 0.4819 & 0.4923 & 0.4924 & 0.4924 & \textbf{0.5386} \\ \hline
\multirow{2}{*}{DGraph-Fin} & AUC     & 0.5752 & 0.6117 & 0.5487 & 0.6505 & 0.6408 & 0.3260 & 0.3298 & OOM    & \textbf{0.6545} \\
                            & F1      & 0.4820 & 0.4769 & 0.4225 & 0.5000 & 0.5037 & 0.4968 & 0.3321 & OOM    & \textbf{0.5097} \\ \hline
\multirow{2}{*}{T-Social}   & AUC     & 0.5896 & 0.7611 & 0.7268 & 0.6968 & 0.7248 & 0.5959 & 0.4245 & OOM    & \textbf{0.9428} \\
                            & F1      & 0.4126 & 0.5666 & 0.5549 & 0.5770 & 0.5433 & 0.4936 & 0.4898 & OOM    & \textbf{0.7828} \\ \hline \hline
\end{tabular}
}\vspace{-4mm}
\end{table}


\begin{table}[t]
\caption{AUC and F1 scores (\%) on 9 datasets with random split, compared with specialized models, where TLE represents the experiment can not be conducted successfully within 72 hours. }
\vspace{-2mm}
\small
\centering
\scalebox{0.91}{
\setlength\tabcolsep{2pt}
\label{tab:gad100}
\begin{tabular}{cc|ccccccccc}
\hline \hline
Datasets                    & Metrics & AMNet  & BWGNN  & GDN             & SparseGAD & GHRN   & GAGA   & XGBGraph        & CONSISGAD & SpaceGNN        \\ \hline
\multirow{2}{*}{Weibo}      & AUC     & 0.6902 & 0.8430 & 0.4353          & 0.8570    & 0.8286 & 0.8283 & 0.9496          & 0.7838    & \textbf{0.9521} \\
                            & F1      & 0.7156 & 0.7925 & 0.6680          & 0.6369    & 0.7918 & 0.7433 & 0.7431          & 0.7217    & \textbf{0.8481} \\ \hline
\multirow{2}{*}{Reddit}     & AUC     & 0.6011 & 0.5833 & 0.5840          & 0.4864    & 0.5823 & 0.4430 & 0.5518          & 0.5704    & \textbf{0.6232} \\
                            & F1      & 0.4916 & 0.4916 & 0.4916          & 0.4916    & 0.4916 & 0.4916 & 0.4909          & 0.4916    & \textbf{0.5228} \\ \hline
\multirow{2}{*}{Tolokers}   & AUC     & 0.6939 & 0.7100 & \textbf{0.7325} & 0.6879    & 0.7197 & 0.4817 & 0.6804          & 0.7088    & 0.7140          \\
                            & F1      & 0.5910 & 0.5943 & 0.5834          & 0.4711    & 0.5871 & 0.2794 & 0.5954          & 0.5932    & \textbf{0.6040} \\ \hline
\multirow{2}{*}{Amazon}     & AUC     & 0.8812 & 0.8742 & 0.8939          & 0.7263    & 0.8843 & 0.7476 & 0.9124          & 0.9325    & \textbf{0.9428} \\
                            & F1      & 0.8768 & 0.8893 & 0.8732          & 0.5939    & 0.8286 & 0.7224 & 0.8607          & 0.8990    & \textbf{0.9069} \\ \hline
\multirow{2}{*}{T-Finance}  & AUC     & 0.7774 & 0.8907 & 0.7863          & 0.9247    & 0.8982 & 0.8387 & 0.9407          & 0.9359    & \textbf{0.9486} \\
                            & F1      & 0.7528 & 0.7726 & 0.7779          & 0.4883    & 0.7805 & 0.5482 & \textbf{0.8831} & 0.8722    & 0.8789          \\ \hline
\multirow{2}{*}{YelpChi}    & AUC     & 0.7201 & 0.7022 & 0.7165          & 0.5504    & 0.6974 & 0.5107 & 0.7239          & 0.7152    & \textbf{0.7321} \\
                            & F1      & 0.6202 & 0.6119 & 0.6161          & 0.4608    & 0.5575 & 0.4608 & 0.6232          & 0.6187    & \textbf{0.6256} \\ \hline
\multirow{2}{*}{Questions}  & AUC     & 0.5959 & 0.5939 & 0.4870          & 0.5080    & 0.5931 & 0.5088 & 0.5217          & 0.5652    & \textbf{0.6476} \\
                            & F1      & 0.5145 & 0.4992 & 0.4936          & 0.4955    & 0.4993 & 0.4924 & 0.4926          & 0.5174    & \textbf{0.5386} \\ \hline
\multirow{2}{*}{DGraph-Fin} & AUC     & 0.5392 & 0.5887 & 0.5407          & 0.3418    & 0.6012 & TLE    & 0.5080          & 0.5108    & \textbf{0.6545} \\
                            & F1      & 0.5043 & 0.5073 & 0.4968          & 0.4487    & 0.4973 & TLE    & 0.4974          & 0.5058    & \textbf{0.5097} \\ \hline
\multirow{2}{*}{T-Social}   & AUC     & 0.5114 & 0.7544 & 0.4846          & 0.7199    & 0.6353 & TLE    & 0.7381          & 0.9192    & \textbf{0.9428} \\
                            & F1      & 0.4653 & 0.5731 & 0.4697          & 0.4924    & 0.4923 & TLE    & 0.5547          & 0.7170    & \textbf{0.7828} \\ \hline \hline
\end{tabular}
}
%\vspace{-4mm}
\end{table}

In Tables \ref{tab:general10} and \ref{tab:gad10}, we can observe that our SpaceGNN can consistently surpass both generalized and specialized models on 9 datasets. In short, SpaceGNN outperforms the best rival 10.84\% and 5.46\% on average in terms of AUC and F1 scores, respectively. 

Similarly, in Tables \ref{tab:general100} and \ref{tab:gad100}, it is easy to find out that SpaceGNN is able to beat both generalized and specialized models on 9 datasets. In summary, compared with the best rival, SpaceGNN takes a lead by 4.98\% and 3.02\% on average in terms of AUC and F1 scores, separately. 

\section{Alternative Model}
\label{subsec:alternative}
\section{Alternative views}
\label{sec:alternative}

One broadly applicable metric for regulating AI agents is the amount of computation they require. Here, we explore how existing regulations and recommendations from AI researchers often center on computational resources. While these efforts primarily address the computational resources for training, we expand the discussion to include computational demands for inference. We then argue that limiting the focus to these aspects alone is inadequate for the existential risks associated with advanced AI agents.

\subsection{Computational resources for development}

In this section, we briefly review existing regulations, focusing on the EU AI Act and President Biden's executive order, along with key recommendations from AI researchers.

\subsubsection{EU AI Act}
%https://yourlearning.ibm.com/activity/UDEMY-5863828
% https://www.noandt.com/wp-content/uploads/2021/04/technology_no6.pdf

\citet{EUAIAct2024} has established a set of rules (the EU AI Act) for the development, deployment, and use of AI within European Union.  Its Chapter 5 defines ``general-purpose AI models with systemic risk'' and lists obligations for providers of such models.  Here, general-purpose AI models essentially refer to FMs, which are pre-trained with self-supervised learning and can (be adapted to) perform a wide range of downstream tasks (see Article 3(63)).  Also, systemic risk refers to ``a risk that is specific to the high-impact capabilities of general-purpose AI models, having a significant impact ... on public health, safety, public security, fundamental rights, or the society as a whole, that can be propagated at scale across the value chain'' (see Article 3(65)).

In particular, a ``general-purpose AI model shall be presumed to have high impact capabilities ... when the cumulative amount of computation used for its training measured in [FLOPs] is greater than $10^{25}$'' (see Article 51(2)).  When this or other specified conditions are met, the provider of a general-purpose AI model is required to fulfill certain obligations, such as providing technical documentation about the model.

The advanced AI agent that we consider can certainly be classified as general-purpose AI.  Also, the existential risks that we consider can be considered as systemic risks, although human extinction and irreversible global catastrophes are not discussed in the EU AI Act.

\subsubsection{Biden's executive order and related bills}

In October 2023, Joe Biden, then president of the US, signed the Executive Order on the Safe, Secure, and Trustworthy Development and Use of Artificial Intelligence \cite{biden2023executive}\footnote{This executive order was repealed by President Trump.}.  Its Section 4.2 is dedicated to ensuring safe and reliable AI.  In particular, it requires companies to report on ``any model that was trained using a quantity of computing power greater than $10^{26}$ integer or [FLOPs]'' until a set of technical conditions for models are defined by specified authorities (see Section 4.2(b)).

In this executive order, particular attention is paid to a dual-use FM, which refers to a FM that exhibits ``high levels of performance at tasks that pose a serious risk to security, national economic security, national public health or safety, or any combination of those matters, such as by ... permitting the evasion of human control or oversight through means of deception or obfuscation'' (see Section 3(k)).

% https://web.archive.org/web/20250118020619/https://www.whitehouse.gov/briefing-room/presidential-actions/2023/10/30/executive-order-on-the-safe-secure-and-trustworthy-development-and-use-of-artificial-intelligence/

Following this executive order, almost 700 AI-related bills are introduced in 45 states across the United States in 2024 \cite{bsa2024state}.  A particularly interesting one is California Senate Bill 1047 (Safe and Secure Innovation for Frontier Artificial Intelligence Models Act) \cite{wiener2024senate}\footnote{The bill had passed the state legislature but was later vetoed by the Governor.  The technical feasibility of the requirements has also been questioned by the community \cite{ai2024statement}.}.  Its Chapter 22.6 is devoted to safe and secure innovation for frontier artificial intelligence models, which cover ``[a]n artificial intelligence model trained using a quantity of computing power greater than $10^{26}$ integer or floating-point operations.''  In particular, the senate bill requires that ``[b]efore beginning to initially train a covered model, the developer shall ... [i]mplement the capability to promptly enact a full shutdown,'' which completely halts the operations of the model.

% https://leginfo.legislature.ca.gov/faces/billStatusClient.xhtml?bill_id=202320240SB1047

The necessity of such an off-switch \cite{hadfieldmenell2017offswitch} is motivated to prevent ``critical harms,'' which include ``[m]ass casualties or at least five hundred million dollars (\$500,000,000) of damage resulting from an artificial intelligence model engaging in conduct that ... [a]cts with limited human oversight, intervention, or
supervision.''

% Key 2024 Statistics: State lawmakers across the United States introduced almost 700 AI-related bills in 2024 across 45 states % https://www.bsa.org/news-events/news/2025-state-ai-wave-building-after-700-bills-in-2024

\subsubsection{Recommendations by scientists}

We have seen that the computational resources used for training is one of the major criteria to judge whether an AI model can have unacceptably high risk.  Although specific values of the threshold such as $10^{25}$ or $10^{26}$ FLOPs are subject to change, the amount of computation for training is one of the most reliable metrics that AI researchers can currently provide for approximating the performance and potential risks of a wide range of FMs.

For example, \citet{anderljung2023frontier} recommend to identify sufficiently dangerous frontier AI models on the basis of whether they are trained with more than $10^{26}$ FLOPs of computation.  More recently, \citet{cohen2024regulating} argue that ``[s]ystems should be considered `dangerously capable' if they are trained with enough resources to potentially exhibit those dangerous capabilities, and regulators should not permit the development of dangerously capable LTPAs.''  Although they do not specify exactly what is considered as enough resources, the amount of computation for training is the only specific criterion that they suggest to determine whether an LTPA can have existential risk. 

\citet{future2023policymaking} have also provided recommendations for governments on managing AI risks.  The recommendations include mechanisms such as auditing, certification, and regulation, grounded in the assumption that ``[t]he amount of compute used to train a general-purpose system largely correlates with ... the magnitude of its risks'' \cite{future2023policymaking}.  These recommendations have been made by following an open letter that called for ``all AI labs to immediately pause for at least 6 months the training of AI systems more powerful than GPT-4,'' which was issued in response to the severe societal risks posed by advanced AI systems and signed by AI researchers \cite{future2023pause}.

In fact, the amount of computation used for training may serve as a sufficiently reliable predictor of the performance and risks of today’s LLMs. This is because most existing LLMs share the same Transformer architecture, differing primarily in their size and the volume of training data. Several studies have examined scaling laws that describe the relationship between a model's optimal size, the amount of training data, and the computational budget required \cite{kaplan2020scaling,hoffmann2022training}.  Research has also shown that various abilities, such as multi-step reasoning, tend to emerge as the computational resources used for training increase \cite{wei2022emergent}.

These scaling laws can, in turn, be used to estimate the FLOPs needed to train existing LLMs.  For example, \citet{anil2023palm2} propose a heuristic suggesting that an LLM should be trained with $6\,N\,D$ FLOPs, where $D$ is the amount of training data, and $N$ is the model size.  Using this heuristics, Llama 3.1 405B trained on 15 trillion tokens is estimated to require approximately $4 \times 10^{25}$ FLOPs---an amount closely aligning with the thresholds specified in the EU AI Act and President Biden's executive order.


\subsection{Computational resources for operation}
\label{sec:alternative:operation}

We have seen that the amount of computation used for training is one of the key criteria that is used today to regulate highly capable FMs.  Although such regulations may be effective for AI agents whose capabilities primarily and directly stem from traditional FMs, they obviously fail to regulate advanced AI agents that gain substantial reasoning capabilities from computation at inference time.  

A natural approach is to regulate advanced AI agents based on the amount of computation used not only for training but also for inference. While this approach may be effective for some AI agents, we argue that it is insufficient for advanced AI agents, at least for the following four reasons.

First, scaling laws for inference-time computation are much less established than those for training-time computation. Recently, scaling laws have been studied for a limited number of inference strategies in LLMs \cite{chen2024simple,brown2024large,snell2024scaling,wu2024inference}. However, there are many potential inference strategies, and their performance can scale very differently. The scaling law for inference also depends on how difficult the task is and what LLM is used \cite{snell2024scaling,openai2024openai}. 
%As a result, we cannot simply assume that inference is safe simply because it does not require heavy computation.

%\cite{chen2024simple}: inference scaling law for a particular inference strategy (sample many, and select one with tournament)
%\cite{brown2024large}: inference scaling law for a particular inference strategy (sample many, and select one)
%\cite{snell2024scaling,wu2024inference}: inference scaling laws for several selected inference strategies

Second, inference can be performed in parallel by multiple entities. For example, several entities may operate the same AI agent that perform reasoning with MCTS \cite{luo2024improve,zhang2024restmcts}, either collaboratively or independently, possibly without knowing each other. Since MCTS is a randomized algorithm, the likelihood that one of the agents optimally solves the task increases with the number of agents. However, this also means that this agent may exploit a loophole, solving the task super-optimally in a way that violates critical constraints, potentially leading to catastrophic outcomes. 
%Therefore, even if each entity’s computation is negligible, a group of entities could collectively exceed a threshold that ensures the safety of inference.

Third, it is not always clear what constitutes a single run of inference. For instance, the results of one inference run may be stored and used in another. Intermediate results could also be stored and later retrieved by a different AI agent, who may or may not be aware that the information is from a previous inference run. More broadly, reasoning can be enhanced through retrieval augmentation \cite{pouplin2024retrieval}, where retrieved information may have been generated with substantial computational resources.
%Therefore, regulating the computational resources spent on a single, arbitrarily defined inference run may not be sufficient.

Finally, AI agents may continually learn over time, which makes it difficult to separate inference from training. For example, an AI agent may perform reasoning with MCTS, with an LLM performing a step in the process. Once the agent identifies a good sequence of steps, it may fine-tune the LLM in a way that the LLM can perform the entire sequence in a single step, bypassing the reasoning process.  As this learning progresses, the agent will gain the ability to perform high-level reasoning with limited computation.
%However, since training and inference can be performed by different entities, tracking the total computational resources used by an AI agent becomes challenging.

% MCTS \cite{kocsis2006bandit,browne2012survey} at run time.  counterfactual regret minimization \cite{zinkevich2007regret}.

% 民間企業においては、AI の開発に際し、政府機関に安全性テストの報告等が必要になる。ただし、EU の AI 規制(案)のように一部の AI の開発・利用を禁止するものではなく、AI 大統領令は政府機関が策定する基準やベストプラクティスに基づいた対応を求めるにとどまる。
% 。いずれも連邦当局に対して基準やベストプラクティスの策定、一定の対象者に報告義務を課す等の緩やかな規制であり、EU の AI 規則(案)7のように非常にリスクの高い AI を禁止する等の AI の利用そのものを制限するものではない。
% https://www.dir.co.jp/report/research/law-research/law-others/20231130_024115.pdf

% いずれも連邦当局に対して基準やベストプラクティスの策定、一定の対象者に報告義務を課す等の緩やかな規制であり、EU の AI 規則(案)7のように非常にリスクの高い AI を禁止する等の AI の利用そのものを制限するものではない。いずれも連邦当局に対して基準やベストプラクティスの策定、一定の対象者に報告義務を課す等の緩やかな規制であり、EU の AI 規則(案)7のように非常にリスクの高い AI を禁止する等の AI の利用そのものを制限するものではない。
% dual use FMs:悪用されると安全保障、国家経済安全保障、国家公衆衛生・安全に対する深刻なリスクをもたらしうるAIモデル dual use FMsの開発者に対し、AI red-teamingテストの結果やトレーニングに関する情報の報告義務を課す 4.2項(a)(i)
% https://www.noandt.com/wp-content/uploads/2023/11/technology_no43_1.pdf

The most common non-Euclidean GNN is based on either the Poincaré Ball model \citep{hgnn19liu} or the Lorentz model \citep{lorentz18nickle}. We discover that the Poincaré Ball model can be a special form of $\kappa$-stereographic model when setting the $\kappa$ to $-1$, which inspires us to investigate the general form of the Lorentz model. Following the definition of the $\kappa$-stereographic model, we generalize the Lorentz model as the $\kappa$-Lorentz model. Notice that we only provide a similar form to the $\kappa$-stereographic model, serving as the projection functions without considering the physical meaning. The $exp_{\vect{o}}^\kappa(\cdot)$ and $log_{\vect{o}}^\kappa(\cdot)$ for $\vect{x}\in \mathbb{R}^d$ are defined as follows: 
\begin{equation*}
exp_{\vect{x}'}^\kappa(\vect{x})=cos_\kappa(||\vect{x}||_L)\vect{x}'+sin_\kappa(||\vect{x}||_L)\frac{\vect{x}}{||\vect{x}||_L}
\end{equation*}
\begin{equation*}
log_{\vect{x}'}^\kappa(\vect{x})=d_\kappa(\vect{x}, \vect{x}')\frac{\vect{x}+\frac{1}{\kappa}\langle\vect{x}, \vect{x}'\rangle_L\vect{x}'}{||\vect{x}+\frac{1}{\kappa}\langle\vect{x}, \vect{x}'\rangle_L\vect{x}'||_L}
\end{equation*}
where $\langle\vect{x}, \vect{x}'\rangle_L=-x_0x_0'+x_1x_1'+...+x_dx_d'$, $||\vect{x}||_L=\sqrt{\langle\vect{x}, \vect{x}'\rangle_L}$, $d_\kappa(\vect{x}, \vect{x}')=cos_\kappa^{-1}(-\langle\vect{x}, \vect{x}'\rangle_L)$, and $cos_\kappa$ and $sin_\kappa$ are defined as: 
\begin{equation*}
\begin{aligned}
    \cos_\kappa(\vect{x})=
    \begin{cases}
    \frac{1}{\sqrt{-\kappa}}\cosh(\sqrt{-\kappa}\vect{x}), &\kappa < 0,\\
    \vect{x}, &\kappa=0,\\
    \frac{1}{\sqrt{\kappa}}\cos(\sqrt{\kappa}\vect{x}), &\kappa>0. 
    \end{cases}
\end{aligned}
\end{equation*}
\begin{equation*}
\begin{aligned}
    \sin_\kappa(\vect{x})=
    \begin{cases}
    \frac{1}{\sqrt{-\kappa}}\sinh(\sqrt{-\kappa}\vect{x}), &\kappa < 0,\\
    \vect{x}, &\kappa=0,\\
    \frac{1}{\sqrt{\kappa}}\sin(\sqrt{\kappa}\vect{x}), &\kappa>0. 
    \end{cases}
\end{aligned}
\end{equation*}
We replace the corresponding functions in our SpaceGNN framework to get SpaceGNN-L. As shown in Table \ref{tab:alternative}, SpaceGNN and SpaceGNN-L can have similar performance in terms of all the 9 datasets, which shows SpaceGNN-L can also outperform other baselines. These results demonstrate that our framework can be generalized to other base models. 

\section{Learned $\kappa$}
\label{subsec:learnedk}
\begin{table}[t]

\vspace{-2mm}
\small
\centering
\caption{Learned $\vect{\kappa}$}
\scalebox{0.9}{
\setlength\tabcolsep{3pt}
\label{tab:kappa}
\begin{tabular}{c|cccccc|cccccc}
\hline \hline
Datasets   & $\kappa_1^-$ & $\kappa_2^-$ & $\kappa_3^-$ & $\kappa_4^-$ & $\kappa_5^-$ & $\kappa_6^-$ & $\kappa_1^+$ & $\kappa_2^+$ & $\kappa_3^+$ & $\kappa_4^+$ & $\kappa_5^+$ & $\kappa_6^+$ \\ \hline
Weibo      & -0.1272 & -0.0766 & -0.0795 & -0.1176 & -0.1239 & -0.1103 & 0.0989  & 0.1233  & 0.0936  & 0.0961  & 0.0747  & 0.1116  \\
Reddit     & -0.0898 & -0.1012 & -0.1028 & -0.1048 & -0.1106 & -       & 0.0839  & 0.0990  & 0.1223  & 0.0939  & 0.0963  & -       \\
Tolokers   & -0.1046 & -       & -       & -       & -       & -       & 0.1259  & -       & -       & -       & -       & -       \\
Amazon     & -0.0873 & -0.1095 & -0.1054 & -       & -       & -       & 0.0768  & 0.0685  & 0.0864  & -       & -       & -       \\
T-Finance  & -0.1347 & -0.0701 & -0.0738 & -       & -       & -       & 0.0744  & 0.0652  & 0.0850  & -       & -       & -       \\
YelpChi    & -0.1014 & -       & -       & -       & -       & -       & 0.1369  & -       & -       & -       & -       & -       \\
Questions  & -0.0999 & -0.1013 & -0.0992 & -0.1004 & -0.0983 & -0.1008 & 0.1006  & 0.0998  & 0.0998  & 0.1006  & 0.1001  & 0.0998  \\
DGraph-Fin & -0.1262 & -0.0774 & -0.0803 & -0.1169 & -       & -       & 0.0784  & 0.0906  & 0.0994  & 0.1130  & -       & -       \\
T-Social   & -0.1440 & -0.0621 & -0.0669 & -0.1284 & -0.1386 & -0.1167 & 0.0992  & 0.1181  & 0.0950  & 0.0970  & 0.0804  & 0.1090 \\ \hline \hline
\end{tabular}
}\vspace{-4mm}
\end{table}



{\update In this section, we report the learned $\vect{\kappa}$ of the experiments in Tables \ref{tab:general50} and \ref{tab:gad50}. Notice that, for simplicity, we only include 1 Euclidean GNN, 1 Hyperbolic GNN, and 1 Spherical GNN in our framework. Recap from Section \ref{sec:method}, in our final architecture, we utilize a hyperparameter $L$ to control the number of layers of all three GNNs, and the number of entries in $\vect{\kappa}$ for each GNN is the same as the number of layers of it. Specifically, if $L$ is set to be 6, then there will be 6 entries in $\vect{\kappa}^0$ for Euclidean GNN, 6 entries in $\vect{\kappa}^-$ for Hyperbolic GNN, and 6 entries in $\vect{\kappa}^+$ for Spherical GNN. For $\vect{\kappa}^0$, we want the GNN to stay in the Euclidean space, so we set each entry in it to 0. For $\vect{\kappa}^-$ and $\vect{\kappa}^+$, we want the Hyperbolic GNN and Spherical GNN to search for the optimal curvatures for different datasets, so we set these two as learnable curvatures. Notice that, according to Table \ref{tab:setting}, the optimal $L$ varies by datasets, so the number of entries in learned $\vect{\kappa}^-$ and $\vect{\kappa}^+$ will also be different, as shown in Table \ref{tab:kappa}, where $"-"$ represents no such entry in the vector. 

As we can see from Table \ref{tab:kappa}, the learned values stay close to 0 after the learning process, which is aligned with the analysis of Section \ref{subsec:LSP}. As shown in Figure \ref{fig:curvature}, the largest $ER_\kappa$ will be obtained around 0, which further demonstrates our findings are effective for graph anomaly detection tasks. 
}


\section{Time Complexity Analysis}
\label{subsec:time}
{\update As shown in Section \ref{subsec:MSE}, our framework is composed of 1 Euclidean GNN, $H$ Hyperbolic GNN, and $S$ Spherical GNN. The differences between these GNNs are the projection functions and the Distance Aware
Propagation (DAP) component, but the time complexity of them is the same for different GNNs. Hence, We only need to analyze one of the GNNs. 

Our analysis of the GNN time complexity is primarily based on the Algorithm \ref{alg:base} in Appendix \ref{subsec:algorithm}, which illustrates the base architecture of each GNN. For simplicity, we focus on a single layer in the base architecture (Lines 3-7). 

First, in Line 3, we apply a two-layer MLP with time complexity of $O(|V|dd_1+|V|d_1d_2)$ followed by a projection function with time complexity of $O(|V|d_2)$, where $|V|$ is the total number of nodes in the graph, $d$ is the dimension of the node feature, and $d_1, d_2$ are the output dimension of the two MLPs, respectively. Thus, the total time complexity of Line 3 is $O(|V|dd_1+|V|d_1d_2)$. 

Then, in Line 5, we have to calculate the $\vect{\hat{s}}_{i}^{\kappa^l}$ for each node $i$. Specifically, for each edge connected to node $i$, we have a time complexity of $O(d_2)$ to get the corresponding coefficient. Thus, the total time complexity for all nodes in Line 5 would be $O(|E|d_2)$, where $|E|$ is the total number of edges in this graph. 

Afterward, in Line 6, for each edge between nodes $i$ and $j$, we need to calculate the $\omega_{ij}^{\kappa^l}$ with time complexity of $O(d_2^2)$, where the input dimension of the MLP is $2d_2$ and the output dimension of it is $d_2$, so the total complexity for all edges in Line 6 would be $O(|E|d_2^2)$.

Next, in Line 7, we also have to propagate the node embeddings for each edge in the graph, so the total time complexity of Line 7 is $O(|E|d_2)$. 

Finally, we combine the results before to get the time complexity of a single layer in the base architecture, i.e., $O(|V|dd_1+|V|d_1d_2+|E|d_2^2)$. 

According to the time analysis of GAT \citep{gat18velickovic}, one of the most popular architectures in the area of graph learning, the time complexity of a single GAT attention head computing $F_0$ features can be expressed as O($|V|FF_0 + |E|F_0$), where $F$ is the number of input features, and $|V|$ and $|E|$ are the numbers of nodes and edges in the graph, respectively. 

Hence, each layer of our proposed GNN has a similar time complexity to GAT by choosing the proper hyperparameters $d_1$ and $d_2$ in our architecture. In the experiments, we find that combining 1 Euclidean GNN, 1 Hyperbolic GNN, and 1 Spherical GNN in our framework is enough to achieve superior performance over all the other baselines, so the increase of time complexity by the Multiple Space Ensemble component will not be the limitation of our models in real applications. 
}

\section{Performance with More Training Data}
\label{subsec:moredata}

{ 
\begin{figure}[t]
\centering

  \begin{small}
    \begin{tabular}{cc}
        \multicolumn{2}{c}{\includegraphics[height=20mm]{figures/trainsz/legend.eps}}  \\ %[-5mm]
        \hspace{0.5mm}
        \includegraphics[height=40mm]{figures/trainsz/redditauc.eps} &
        \hspace{-16mm}
        \includegraphics[height=40mm]{figures/trainsz/redditf1.eps} \\ [-5mm]
        \hspace{-2mm}
        (a) Reddit AUC & 
        \hspace{-2mm}
        (b) Reddit F1 \\ 
    \end{tabular}
    \vspace{-3mm}
    \caption{Varying the training set size of Reddit.}
    \label{fig:redditsz}
    \vspace{-4mm}
  \end{small}
\end{figure}
\vspace{-5pt}
\section{Key Questions} \label{sec:questions}
This section addresses key questions about \lov.
%In this section, we address several key questions regarding the implementation and deployment aspects of \lov.

\textit{Does ML-Classifier need to retrain?}
In this work, we do not retrain the ML classifier with new data, primarily because the features used can help reduce issues with model generalization to newly emerging data, as described in Section \ref{subsubsec:model_general}.
Nevertheless, expanding the size and diversity of the training data could potentially enhance the classifier's effectiveness and adaptability to various scenarios.

\textit{Why is post-analyzer necessary?}
The post-analyzer not only identifies potential benign conflicts mistakenly flagged as hijacks by the ML classifier but also detects possible hijacking events.
The classifier identifies each hijack as an individual instance within an event, without automatically linking them to hijacking events. To associate hijacks with a specific event, the first step is to identify the perpetrator AS that initiated the event, along with the occurrence time. The post-analyzer helps in identifying these initiators and the event time, assisting security analysts in subsequent event analysis.

\textit{Can \lov\ be deployed in individual networks?}
\lov\ can be deployed within a network that has ROV in practice, to create a local whitelist, ensuring that legitimate traffic with benign ROA conflicts is not mistakenly filtered by ROV. This approach removes the reliance on, and potential distrust of, third-party whitelists. However, to prevent any impact on BGP convergence, \lov\ must operate separately from ROV. For instance, \lov\ can function in the background without interfering with the ongoing ROV processes.

Additionally, \lov\ employs various mechanisms, including ML-based, signature-based, heuristic-based, and analyst-driven approaches, which may require security personnel with expertise in areas such as ML technologies. In addition, deploying \lov\ independently may also incur significant costs associated with hardware, software, and maintenance.

We do not recommend directly incorporating the ML classifier into the existing ROV mechanism (as SROV is expected to do). Although this approach might seem straightforward to implement and deploy, the ML-based system is vulnerable to adversarial attacks, which could impair the ROV's ability to detect hijacks.

\textit{What are the differences between SROV and \lov?}
Both SROV and \lov\ incorporate classifiers to distinguish between benign conflicts and real hijacks. The primary difference between the two classifiers is not the use of different technologies—non-ML versus ML—but their design goals. \lov\ aims to preserve as many benign conflicts as possible while maintaining the ROV system's existing ability to prevent hijacks. In contrast, SROV seems to overlook the importance of this balance, leading to an impractical approach. Furthermore, we do not integrate SROV's route duration-based method into \lov\ because it relies on longitudinal route observation, which could affect the whitelist's timeliness. However, during quarantine, \lov\ monitors route behavior (e.g., its duration), complementing the ML classifier.

\textit{What are the differences between \lov\ and BGPmon?}
BGPmon is often considered a reliable monitoring service for BGP anomalies.
\lov\ and BGPmon have a different focus: \lov\ identifies and analyzes benign conflicts, while BGPmon targets the detection of BGP route anomalies such as hijacks. Looking towards the future, there are opportunities for collaboration between \lov\ and BGPmon. For instance, hijacks or events detected by BGPmon could enhance \lov\, such as updating \lov's classifier with new hijacks. Conversely, BGPmon could leverage benign conflicts detected by \lov\ to minimize potential false positives.

Unlike BGPmon, the hijacking events identified by \lov\ are not publicly disclosed. \lov\ focuses exclusively on confirming hijacking events through email surveys to provide evidence for possible human intervention. Consequently, the accuracy in identifying these events has a limited impact on the reliability of the whitelist.

\textit{What are the advantages, limitations, and security considerations of \lov?}
The whitelist provided by \lov\ is easy to update, maintain, and manage, with minimal resource overhead. More importantly, the additional step after ROV enforcement—checking whether the RPKI-invalid route is on the list—incurs negligible time cost, thus minimizing the impact on BGP convergence.
However, behavior monitoring and human intervention in quarantine often require days of route observation, potentially delaying the addition of some benign conflicts to the whitelist and affecting its timeliness. Future work will explore more efficient inspection mechanisms. In rare cases where the victim network is the provider of the attacker's network, the ML classifier may fail to detect the hijack. However, proximity between the victim and malicious networks can often facilitate quicker attack detection and mitigation, provided appropriate monitoring systems are in place. Future work will incorporate additional features to address such attacks.
Additionally, all codes, data, and models are kept confidential and are not publicly available. The operational process of \lov\ is also not transparent.
While these measures can help prevent potential vulnerability exposure to adversaries, ensuring data security and service integrity, they may lead to distrust in the whitelist and hinder its broader adoption. One potential solution might be to involve authoritative organizations (e.g., cybersecurity agencies) to regularly monitor and assess \lov, certifying its effectiveness and reliability. The potential misuse of the whitelist represents another limitation in the deployment of \lov. As previously mentioned, users can access the whitelist through the APIs we provide. Similarly, adversaries might acquire the whitelist, alter it, and distribute a falsified version under the guise of \lov.
One way to mitigate this risk could be to restrict access to the whitelist of benign conflicts to users who verify their identities. For example, users might be required to submit an email request from an organizational address and provide certification of their identity before gaining access.













\begin{figure}[t!]
\centering
 \vspace{-3mm}
  \begin{small}
    \begin{tabular}{cc}
        %\multicolumn{2}{c}{\includegraphics[height=20mm]{figures/trainsz/legend.eps}}  \\ [-5mm]
        \hspace{-4mm}
        \includegraphics[height=40mm]{figures/trainsz/dgraphfinauc.eps} &
        \hspace{-12mm}
        \includegraphics[height=40mm]{figures/trainsz/dgraphfinf1.eps} \\ [-5mm]
        \hspace{-2mm}
        (a) DGraph-Fin AUC & 
        \hspace{-2mm}
        (b) DGraph-Fin F1 \\ 
    \end{tabular}
    \vspace{-3mm}
    \caption{Varying the training set size of DGraph-Fin.}
    \label{fig:dgraphfinsz}
    \vspace{-4mm}
  \end{small}
\end{figure}

\update To further demonstrate the superior ability of our proposed framework, we provide the performance on Reddit, Questions, and DGraph-Fin varying by the size of the training set, as shown in Figures \ref{fig:redditsz}, \ref{fig:questionssz}, and \ref{fig:dgraphfinsz}. Note that, since HYLA and GAGA can not successfully run on DGraph-Fin, we only report the performance of the other 14 baselines and our proposed SpaceGNN in Figure \ref{fig:dgraphfinsz}. 

As we can see, the red lines, which represent the performance of our SpaceGNN, are always on the top of the figures, which demonstrates that with more training data, our SpaceGNN can still outperform all the other baselines consistently in terms of both AUC and F1. In summary, such experiments further make our SpaceGNN a more general and practical algorithm.
}

\section{Performance on GADBench \citep{gadbench23tang} semi-supervised setting}
\label{subsec:gadbench}
{\update 

\begin{table}[t!]
\caption{AUC, AUPRC, and Rec@K scores (\%) on 9 datasets with data split of the semi-supervised setting in GADBench \citep{gadbench23tang}, compared with generalized models, where OOM represents out-of-memory.}
\vspace{-2mm}
\small
\centering
\scalebox{0.97}{
\setlength\tabcolsep{4pt}
\label{tab:gadbench1}
\begin{tabular}{cc|ccccccccc}
\hline \hline
Datasets                    & Metrics & MLP   & GCN   & SAGE  & GAT   & GIN   & HNN   & HGCN  & HYLA  & SpaceGNN       \\ \hline
\multirow{3}{*}{Weibo}      & AUC     & 0.666 & 0.935 & 0.818 & 0.864 & 0.838 & 0.747 & 0.942 & 0.960 & \textbf{0.964} \\
                            & AUPRC   & 0.562 & 0.860 & 0.585 & 0.733 & 0.676 & 0.312 & 0.808 & 0.727 & \textbf{0.864} \\
                            & Rec@K   & 0.532 & 0.792 & 0.634 & 0.702 & 0.665 & 0.371 & 0.757 & 0.736 & \textbf{0.795} \\ \hline
\multirow{3}{*}{Reddit}     & AUC     & 0.591 & 0.569 & 0.603 & 0.605 & 0.600 & 0.619 & 0.625 & 0.523 & \textbf{0.637} \\
                            & AUPRC   & 0.044 & 0.042 & 0.045 & 0.047 & 0.043 & 0.045 & 0.045 & 0.038 & \textbf{0.050} \\
                            & Rec@K   & 0.065 & 0.062 & 0.058 & 0.065 & 0.048 & 0.055 & 0.052 & 0.064 & \textbf{0.077} \\ \hline
\multirow{3}{*}{Tolokers}   & AUC     & 0.681 & 0.642 & 0.676 & 0.681 & 0.668 & 0.690 & 0.699 & 0.618 & \textbf{0.715} \\
                            & AUPRC   & 0.333 & 0.330 & 0.340 & 0.330 & 0.318 & 0.327 & 0.335 & 0.290 & \textbf{0.362} \\
                            & Rec@K   & 0.355 & 0.334 & 0.352 & 0.351 & 0.336 & 0.346 & 0.351 & 0.311 & \textbf{0.366} \\ \hline
\multirow{3}{*}{Amazon}     & AUC     & 0.922 & 0.820 & 0.814 & 0.924 & 0.916 & 0.861 & 0.792 & 0.717 & \textbf{0.947} \\
                            & AUPRC   & \textbf{0.830} & 0.328 & 0.425 & 0.816 & 0.754 & 0.785 & 0.306 & 0.168 & 0.812          \\
                            & Rec@K   & \textbf{0.793} & 0.369 & 0.480 & 0.771 & 0.704 & 0.776 & 0.356 & 0.236 & 0.782          \\ \hline
\multirow{3}{*}{T-Finance}  & AUC     & 0.899 & 0.883 & 0.689 & 0.850 & 0.845 & 0.880 & 0.933 & 0.615 & \textbf{0.949} \\
                            & AUPRC   & 0.534 & 0.605 & 0.117 & 0.289 & 0.448 & 0.677 & 0.799 & 0.063 & \textbf{0.849} \\
                            & Rec@K   & 0.599 & 0.606 & 0.185 & 0.362 & 0.544 & 0.638 & 0.760 & 0.080 & \textbf{0.796} \\ \hline
\multirow{3}{*}{YelpChi}    & AUC     & 0.647 & 0.512 & 0.589 & 0.656 & 0.629 & 0.662 & 0.480 & 0.551 & \textbf{0.726} \\
                            & AUPRC   & 0.236 & 0.164 & 0.209 & 0.250 & 0.237 & 0.263 & 0.141 & 0.174 & \textbf{0.331} \\
                            & Rec@K   & 0.265 & 0.169 & 0.229 & 0.281 & 0.265 & 0.296 & 0.149 & 0.200 & \textbf{0.366} \\ \hline
\multirow{3}{*}{Questions}  & AUC     & 0.612 & 0.600 & 0.612 & 0.623 & 0.622 & 0.601 & 0.575 & 0.619 & \textbf{0.650} \\
                            & AUPRC   & 0.077 & 0.061 & 0.055 & 0.073 & 0.067 & 0.047 & 0.039 & 0.057 & \textbf{0.097} \\
                            & Rec@K   & 0.120 & 0.098 & 0.088 & 0.109 & 0.103 & 0.051 & 0.030 & 0.106 & \textbf{0.145} \\ \hline
\multirow{3}{*}{DGraph-Fin} & AUC     & \textbf{0.691} & 0.662 & 0.648 & 0.672 & 0.657 & 0.644 & 0.638 & OOM   & 0.678          \\
                            & AUPRC   & 0.023 & 0.023 & 0.020 & 0.022 & 0.020 & 0.022 & 0.023 & OOM   & \textbf{0.025} \\
                            & Rec@K   & 0.034 & 0.036 & 0.025 & 0.031 & 0.021 & 0.013 & 0.027 & OOM   & \textbf{0.040} \\ \hline
\multirow{3}{*}{T-Social}   & AUC     & 0.591 & 0.716 & 0.720 & 0.754 & 0.704 & 0.473 & 0.435 & OOM   & \textbf{0.947} \\
                            & AUPRC   & 0.039 & 0.084 & 0.078 & 0.092 & 0.062 & 0.027 & 0.024 & OOM   & \textbf{0.642} \\
                            & Rec@K   & 0.032 & 0.102 & 0.095 & 0.116 & 0.053 & 0.009 & 0.001 & OOM   & \textbf{0.667} \\  \hline  \hline
\end{tabular}
}
%\vspace{-4mm}
\end{table}
\begin{table}[htbp!]
\caption{AUC, AUPRC, and Rec@K scores (\%) on 9 datasets with data split of the semi-supervised setting in GADBench \citep{gadbench23tang}, compared with specialized models, where TLE represents the experiment can not be conducted successfully within 72 hours. }
\vspace{-2mm}
\small
\centering
\scalebox{0.91}{
\setlength\tabcolsep{2pt}
\label{tab:gadbench2}
\begin{tabular}{cc|ccccccccc}
\hline \hline
Datasets                    & Metrics & AMNet & BWGNN & GDN   & SparseGAD & GHRN  & GAGA  & XGBGraph & CONSISGAD & SpaceGNN       \\ \hline
\multirow{3}{*}{Weibo}      & AUC     & 0.824 & 0.936 & 0.682 & 0.897     & 0.916 & 0.732 & \textbf{0.964}    & 0.873     & \textbf{0.964} \\
                            & AUPRC   & 0.671 & 0.806 & 0.582 & 0.696     & 0.770 & 0.376 & 0.759    & 0.654     & \textbf{0.864} \\
                            & Rec@K   & 0.621 & 0.751 & 0.560 & 0.678     & 0.724 & 0.324 & 0.689    & 0.583     & \textbf{0.795} \\ \hline
\multirow{3}{*}{Reddit}     & AUC     & 0.629 & 0.577 & 0.596 & 0.634     & 0.575 & 0.501 & 0.592    & 0.629     & \textbf{0.637} \\
                            & AUPRC   & 0.049 & 0.042 & 0.043 & 0.047     & 0.042 & 0.032 & 0.041    & 0.046     & \textbf{0.050} \\
                            & Rec@K   & 0.068 & 0.060 & 0.052 & 0.074     & 0.063 & 0.019 & 0.049    & 0.061     & \textbf{0.077} \\ \hline
\multirow{3}{*}{Tolokers}   & AUC     & 0.617 & 0.685 & 0.713 & 0.673     & 0.690 & 0.636 & 0.675    & 0.709     & \textbf{0.715} \\
                            & AUPRC   & 0.286 & 0.353 & 0.353 & 0.318     & 0.359 & 0.293 & 0.341    & 0.337     & \textbf{0.362} \\
                            & Rec@K   & 0.305 & 0.355 & 0.363 & 0.346     & 0.361 & 0.318 & \textbf{0.366}    & 0.364     & \textbf{0.366} \\ \hline
\multirow{3}{*}{Amazon}     & AUC     & 0.928 & 0.918 & 0.868 & 0.935     & 0.909 & 0.504 & \textbf{0.947}    & 0.933     & \textbf{0.947} \\
                            & AUPRC   & 0.824 & 0.817 & 0.691 & 0.800     & 0.807 & 0.148 & \textbf{0.844}    & 0.792     & 0.812          \\
                            & Rec@K   & 0.778 & 0.777 & 0.652 & \textbf{0.788}     & 0.777 & 0.143 & 0.782    & 0.775     & 0.782          \\ \hline
\multirow{3}{*}{T-Finance}  & AUC     & 0.926 & 0.921 & 0.900 & 0.944     & 0.926 & 0.725 & 0.948    & 0.932     & \textbf{0.949} \\
                            & AUPRC   & 0.602 & 0.609 & 0.671 & 0.835     & 0.634 & 0.252 & 0.783    & 0.815     & \textbf{0.849} \\
                            & Rec@K   & 0.657 & 0.649 & 0.656 & 0.794     & 0.677 & 0.400 & 0.724    & 0.758     & \textbf{0.796} \\ \hline
\multirow{3}{*}{YelpChi}    & AUC     & 0.648 & 0.643 & 0.670 & 0.639     & 0.645 & 0.549 & 0.640    & 0.715     & \textbf{0.726} \\
                            & AUPRC   & 0.239 & 0.237 & 0.244 & 0.213     & 0.238 & 0.173 & 0.248    & 0.330     & \textbf{0.331} \\
                            & Rec@K   & 0.266 & 0.264 & 0.278 & 0.222     & 0.269 & 0.187 & 0.268    & 0.358     & \textbf{0.366} \\ \hline
\multirow{3}{*}{Questions}  & AUC     & 0.636 & 0.602 & 0.609 & 0.574     & 0.605 & 0.513 & 0.614    & 0.649     & \textbf{0.650} \\
                            & AUPRC   & 0.074 & 0.065 & 0.070 & 0.036     & 0.065 & 0.039 & 0.077    & 0.085     & \textbf{0.097} \\
                            & Rec@K   & 0.127 & 0.109 & 0.097 & 0.032     & 0.111 & 0.072 & 0.106    & 0.092     & \textbf{0.145} \\ \hline
\multirow{3}{*}{DGraph-Fin} & AUC     & 0.671 & 0.655 & 0.660 & 0.674     & 0.671 & TLE   & 0.624    & 0.635     & \textbf{0.678} \\
                            & AUPRC   & 0.022 & 0.021 & 0.022 & 0.023     & 0.023 & TLE   & 0.019    & 0.017     & \textbf{0.025} \\
                            & Rec@K   & 0.026 & 0.031 & 0.032 & 0.022     & 0.034 & TLE   & 0.025    & 0.011     & \textbf{0.040} \\ \hline
\multirow{3}{*}{T-Social}   & AUC     & 0.537 & 0.775 & 0.716 & 0.766     & 0.787 & TLE   & 0.852    & 0.940     & \textbf{0.947} \\
                            & AUPRC   & 0.031 & 0.159 & 0.104 & 0.256     & 0.162 & TLE   & 0.406    & 0.484     & \textbf{0.642} \\
                            & Rec@K   & 0.016 & 0.243 & 0.199 & 0.362     & 0.246 & TLE   & 0.430    & 0.535     & \textbf{0.667} \\  \hline \hline
\end{tabular}
}
%\vspace{-4mm}
\end{table}

For a fair comparison, we also provide AUC, AUPRC, and Rec@K scores on 9 datasets with data split of the semi-supervised setting in GADBench \citep{gadbench23tang}. Specifically, in this setting, we use 20 positive labels (anomalous nodes) and 80 negative labels (normal nodes) for both the training set and the validation set in each dataset, separately. Note that, for the baselines in GADBench, we use the reported performance in it, and for baselines not in GADBench, we obtain the source code of all competitors from GitHub and execute these models using the default parameter settings suggested by their authors. The hyperparameters of SpaceGNN are set based on the same setting in GADBench, i.e., random search.

As we can see from Tables \ref{tab:gadbench1} and \ref{tab:gadbench2}, our proposed model can still outperform all the baselines on almost all the datasets consistently using AUC, AUPRC, and Rec@K scores as metrics, which demonstrates the effectiveness of our SpaceGNN. 
}

