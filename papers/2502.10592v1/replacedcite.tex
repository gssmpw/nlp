\section{Related work}
% \paragraph{Deploying fair allocation mechanisms}
% \paragraph{Open Source Datasets}
% \paragraph{Fair Course Allocation Mechanisms}
The course allocation problem is well-studied in combinatorial optimization, with various mechanisms proposed to balance efficiency, fairness, and strategic considerations. 
____ shows that serial dictatorships are the only Pareto efficient, strategyproof and non-bossy mechanisms; ____ focuses on additive preferences, whereas we study more general submodular utilities. 
____ examine the Draft (Round-Robin) mechanism for course allocation. 
____ show that the Draft mechanism retains high efficiency guarantees in practice, despite students misreporting their preferences. ____ also collect data on a similar scale to ours (900 students at Harvard Business School), with students ranking classes on a Likert scale. 
The bidding point mechanism is widely used for course allocation ____: students distribute a fixed number of points among their preferred courses. 
Bids are then processed in decreasing order and honored if the course and the student's schedule are not at capacity. This approach focuses solely on efficiency, neglecting equity, and is highly susceptible to strategic manipulation, as students may misreport their preferences to get better outcomes ____. 
Furthermore, since bids serve as proxies for preferences, the inferred preferences may not accurately reflect true student demand, leading to market distortions ____. 

____ propose a Gale-Shapley Pareto-dominant market mechanism that separates the dual role of bids by assigning courses through a matching mechanism considering students preferences, while using bids solely to break ties. In any case, inefficiencies arise from students overbidding on courses they could have gotten with fewer points. 
____ introduce multi-round algorithms, based on matching and second-price concepts, which addresses these inefficiencies and show promising empirical results. 

____ study course allocation as a two sided matching problem, where courses have preferences over students, induced by student priority. 
They analyze the Gale-Shapley Student-Optimal Stable mechanism --- a modified version of the Gale-Shapley deferred acceptance algorithm ____ --- and the Efficiency Adjusted Deferred Acceptance mechanism ____, which reduces the welfare losses produced by the deferred acceptance algorithm, but is not strategyproof. 
Similar to other approaches ____, these fail to incorporate conflicts between classes. %Handling scheduling conflicts or course sections is a crucial aspect of course allocation. Some works address this issue by modeling the problem using interval graphs ____ and conflict graphs ____.

____ introduces the Approximate Competitive Equilibrium from Equal Incomes (A-CEEI) mechanism, which is strategy-proof at large, and approximates envy-freeness and maximin share fairness, both in theory and empirically (see also ____). 
In the context of course allocation, this mechanism faces some challenges. 
A-CEEI assumes that all students have approximately equal budgets and, therefore, equal priority. Due to academic performance, seniority, or graduation requirements, some students may require higher priority when enrolling in courses. ____ propose the Pseudo-Market with Priorities mechanism, which incorporates competitive equilibrium with fake money while accounting for student priorities. 
More importantly, A-CEEI approximation errors can result in capacity constraint violations. In addition, early implementations faced scalability issues ____.

Course Match ____, an A-CEEI-based approach, addresses these issues by introducing two additional stages to ensure feasibility, together with an interface for students to report utilities, and a software system capable of handling reasonably sized instances. 
Course Match relies on students reporting their complete ordinal preferences over all possible course bundles --- an impractical requirement. Instead, students only rate individual courses and course pairs. 
At the same time, Course Match does not guarantee optimal outcomes if students misreport or fail to articulate their true preferences. To mitigate this issue, ____ propose a machine-learning-powered version of Course Match to reduce reporting errors by iteratively asking student to rate personalized pairwise comparison queries. 

Despite these efforts, Course Match is not guaranteed to find a price vector and corresponding allocation that respects course capacities, and may not run in reasonable time for large instances ____. 
Furthermore, Course Match is a commercial product. 
To our knowledge, there is currently no publicly available implementation of Course Match, nor is there large-scale empirical data to evaluate its effectiveness in real-world settings. 
Lastly, Course Match places a significant elicitation burden on students. 
Students are encouraged to study a 12-page manual in order to understand the mechanism and enroll in courses ____. 
While this may be reasonable for MBA students or those in advanced economics-related programs, it may not be suitable for undergraduates from diverse majors and academic backgrounds at a large public university. 

____ analyze the complexity of finding fair and efficient course allocations under course conflicts (see also ____). They establish that the problem is computationally intractable, unless the number of agents is small. At \UMass, course schedules are very standardized, which makes course schedule conflicts well-structured and tractable. 

Fair allocation algorithms and solution concepts have been extensively studied in recent years, with a particular focus on leveraging the structure of agent preferences (see survey by ____). 
Recent works focus on settings where agents have \emph{binary submodular} utilities, i.e., items have a marginal utility of either $0$ or $1$, and agents exhibit diminishing marginal returns on items as their bundles grow ____.
Following these works, we implement the Yankee Swap algorithm ____ as one of our benchmarks. 

The Yankee Swap algorithm outputs a \emph{Lorenz dominating} allocation, which is guaranteed to satisfy several theoretical guarantees ____. In addition to maximizing the minimal utility of any agent (which is implied by the leximin property), it is approximately envy-free, maximizes utilitarian welfare, and offers each agent at least half of their maximin share guarantee. Furthermore, Yankee Swap is a strategyproof mechanism, and can encode student priorities ____. Our empirical evaluation confirms the effectiveness of the Yankee Swap framework, providing evidence for its potential efficacy as a practical course allocation mechanism.
% Since understanding the algorithm is essential for students to accurately report their true preferences, this complexity could significantly undermine the guarantees that Course Match offers.