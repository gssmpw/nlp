\documentclass{article}

\usepackage{microtype}
\usepackage{graphicx}
\usepackage{booktabs}
\usepackage{hyperref}
\usepackage{tcolorbox}

\newcommand{\theHalgorithm}{\arabic{algorithm}}

\usepackage[accepted]{icml2025}

\usepackage{amsmath}
\usepackage{amssymb}
\usepackage{mathtools}
\usepackage{amsthm}
\usepackage{subcaption}
\usepackage{tabularx}
\usepackage{dsfont}
\usepackage{enumitem}
\DeclareMathOperator*{\argmax}{arg\,max}
\usepackage[capitalize,noabbrev]{cleveref}

\newtcolorbox{highlight}[1][]{
    colback=yellow!10,
    colframe=gray!30,
    boxrule=0.5pt,
    arc=2pt,
    leftrule=3pt,
    rightrule=3pt,
    toprule=1pt,
    bottomrule=1pt,
    #1
}

\newcommand{\ourprm}{VersaPRM}
\newcommand{\ourdatatrain}{MMLU-Pro-CoT-Train (Labeled)}
\newcommand{\ourdataeval}{MMLU-Pro-CoT-Eval (Unlabeled)}


\icmltitlerunning{VersaPRM: Multi-Domain Process Reward Model via Synthetic Reasoning Data}

\begin{document}

\twocolumn[
\icmltitle{VersaPRM: Multi-Domain Process Reward Model via Synthetic Reasoning Data}

\icmlsetsymbol{equal}{*}

\begin{icmlauthorlist}
\icmlauthor{Thomas Zeng}{yyy}
\icmlauthor{Shuibai Zhang}{yyy}
\icmlauthor{Shutong Wu}{yyy}
\icmlauthor{Christian Classen}{yyy}
\icmlauthor{Daewon Chae}{zzz}
\icmlauthor{Ethan Ewer}{yyy}
\icmlauthor{Minjae Lee}{fff}
\icmlauthor{Heeju Kim}{fff}
\icmlauthor{Wonjun Kang}{fff}
\icmlauthor{Jackson Kunde}{yyy}
\icmlauthor{Ying Fan}{yyy}
\icmlauthor{Jungtaek Kim}{yyy}
\icmlauthor{Hyung Il Koo}{fff}
\icmlauthor{Kannan Ramchandran}{bbb}
\icmlauthor{Dimitris Papailiopoulos}{yyy}
\icmlauthor{Kangwook Lee}{yyy}

\end{icmlauthorlist}

\icmlaffiliation{yyy}{University of Wisconsin--Madison}
\icmlaffiliation{zzz}{Korea University}
\icmlaffiliation{fff}{FuriosaAI}
\icmlaffiliation{bbb}{University of California, Berkeley}

\icmlcorrespondingauthor{Kangwook Lee}{kangwook.lee@wisc.edu}


\icmlkeywords{Process Reward Models, Multi-Domain Process Reward Models, Synthetic Reasoning Data}

\vskip 0.3in
]


\printAffiliationsAndNotice{}


\begin{abstract}
Process Reward Models (PRMs) have proven effective at enhancing mathematical reasoning for Large Language Models (LLMs) by leveraging increased inference-time computation. However, they are predominantly trained on mathematical data and their generalizability to non-mathematical domains has not been rigorously studied. In response, this work first shows that current PRMs have poor performance in other domains. To address this limitation, we introduce \textbf{\emph{VersaPRM}}, a multi-domain PRM trained on synthetic reasoning data generated using our novel data generation and annotation method. VersaPRM achieves consistent performance gains across diverse domains. For instance, in the MMLU-Pro category of Law, VersaPRM via weighted majority voting, achieves a 7.9\% performance gain over the majority voting baseline---surpassing Qwen2.5-Math-PRM's gain of 1.3\%. We further contribute to the community by open-sourcing all data, code and models for VersaPRM.
\end{abstract}



\setlength{\belowdisplayskip}{1pt}
\setlength{\belowdisplayshortskip}{1pt}
\setlength{\abovedisplayskip}{1pt}
\setlength{\abovedisplayshortskip}{1pt}


\section{Introduction}
\section{Introduction}
\label{sec:introduction}
The business processes of organizations are experiencing ever-increasing complexity due to the large amount of data, high number of users, and high-tech devices involved \cite{martin2021pmopportunitieschallenges, beerepoot2023biggestbpmproblems}. This complexity may cause business processes to deviate from normal control flow due to unforeseen and disruptive anomalies \cite{adams2023proceddsriftdetection}. These control-flow anomalies manifest as unknown, skipped, and wrongly-ordered activities in the traces of event logs monitored from the execution of business processes \cite{ko2023adsystematicreview}. For the sake of clarity, let us consider an illustrative example of such anomalies. Figure \ref{FP_ANOMALIES} shows a so-called event log footprint, which captures the control flow relations of four activities of a hypothetical event log. In particular, this footprint captures the control-flow relations between activities \texttt{a}, \texttt{b}, \texttt{c} and \texttt{d}. These are the causal ($\rightarrow$) relation, concurrent ($\parallel$) relation, and other ($\#$) relations such as exclusivity or non-local dependency \cite{aalst2022pmhandbook}. In addition, on the right are six traces, of which five exhibit skipped, wrongly-ordered and unknown control-flow anomalies. For example, $\langle$\texttt{a b d}$\rangle$ has a skipped activity, which is \texttt{c}. Because of this skipped activity, the control-flow relation \texttt{b}$\,\#\,$\texttt{d} is violated, since \texttt{d} directly follows \texttt{b} in the anomalous trace.
\begin{figure}[!t]
\centering
\includegraphics[width=0.9\columnwidth]{images/FP_ANOMALIES.png}
\caption{An example event log footprint with six traces, of which five exhibit control-flow anomalies.}
\label{FP_ANOMALIES}
\end{figure}

\subsection{Control-flow anomaly detection}
Control-flow anomaly detection techniques aim to characterize the normal control flow from event logs and verify whether these deviations occur in new event logs \cite{ko2023adsystematicreview}. To develop control-flow anomaly detection techniques, \revision{process mining} has seen widespread adoption owing to process discovery and \revision{conformance checking}. On the one hand, process discovery is a set of algorithms that encode control-flow relations as a set of model elements and constraints according to a given modeling formalism \cite{aalst2022pmhandbook}; hereafter, we refer to the Petri net, a widespread modeling formalism. On the other hand, \revision{conformance checking} is an explainable set of algorithms that allows linking any deviations with the reference Petri net and providing the fitness measure, namely a measure of how much the Petri net fits the new event log \cite{aalst2022pmhandbook}. Many control-flow anomaly detection techniques based on \revision{conformance checking} (hereafter, \revision{conformance checking}-based techniques) use the fitness measure to determine whether an event log is anomalous \cite{bezerra2009pmad, bezerra2013adlogspais, myers2018icsadpm, pecchia2020applicationfailuresanalysispm}. 

The scientific literature also includes many \revision{conformance checking}-independent techniques for control-flow anomaly detection that combine specific types of trace encodings with machine/deep learning \cite{ko2023adsystematicreview, tavares2023pmtraceencoding}. Whereas these techniques are very effective, their explainability is challenging due to both the type of trace encoding employed and the machine/deep learning model used \cite{rawal2022trustworthyaiadvances,li2023explainablead}. Hence, in the following, we focus on the shortcomings of \revision{conformance checking}-based techniques to investigate whether it is possible to support the development of competitive control-flow anomaly detection techniques while maintaining the explainable nature of \revision{conformance checking}.
\begin{figure}[!t]
\centering
\includegraphics[width=\columnwidth]{images/HIGH_LEVEL_VIEW.png}
\caption{A high-level view of the proposed framework for combining \revision{process mining}-based feature extraction with dimensionality reduction for control-flow anomaly detection.}
\label{HIGH_LEVEL_VIEW}
\end{figure}

\subsection{Shortcomings of \revision{conformance checking}-based techniques}
Unfortunately, the detection effectiveness of \revision{conformance checking}-based techniques is affected by noisy data and low-quality Petri nets, which may be due to human errors in the modeling process or representational bias of process discovery algorithms \cite{bezerra2013adlogspais, pecchia2020applicationfailuresanalysispm, aalst2016pm}. Specifically, on the one hand, noisy data may introduce infrequent and deceptive control-flow relations that may result in inconsistent fitness measures, whereas, on the other hand, checking event logs against a low-quality Petri net could lead to an unreliable distribution of fitness measures. Nonetheless, such Petri nets can still be used as references to obtain insightful information for \revision{process mining}-based feature extraction, supporting the development of competitive and explainable \revision{conformance checking}-based techniques for control-flow anomaly detection despite the problems above. For example, a few works outline that token-based \revision{conformance checking} can be used for \revision{process mining}-based feature extraction to build tabular data and develop effective \revision{conformance checking}-based techniques for control-flow anomaly detection \cite{singh2022lapmsh, debenedictis2023dtadiiot}. However, to the best of our knowledge, the scientific literature lacks a structured proposal for \revision{process mining}-based feature extraction using the state-of-the-art \revision{conformance checking} variant, namely alignment-based \revision{conformance checking}.

\subsection{Contributions}
We propose a novel \revision{process mining}-based feature extraction approach with alignment-based \revision{conformance checking}. This variant aligns the deviating control flow with a reference Petri net; the resulting alignment can be inspected to extract additional statistics such as the number of times a given activity caused mismatches \cite{aalst2022pmhandbook}. We integrate this approach into a flexible and explainable framework for developing techniques for control-flow anomaly detection. The framework combines \revision{process mining}-based feature extraction and dimensionality reduction to handle high-dimensional feature sets, achieve detection effectiveness, and support explainability. Notably, in addition to our proposed \revision{process mining}-based feature extraction approach, the framework allows employing other approaches, enabling a fair comparison of multiple \revision{conformance checking}-based and \revision{conformance checking}-independent techniques for control-flow anomaly detection. Figure \ref{HIGH_LEVEL_VIEW} shows a high-level view of the framework. Business processes are monitored, and event logs obtained from the database of information systems. Subsequently, \revision{process mining}-based feature extraction is applied to these event logs and tabular data input to dimensionality reduction to identify control-flow anomalies. We apply several \revision{conformance checking}-based and \revision{conformance checking}-independent framework techniques to publicly available datasets, simulated data of a case study from railways, and real-world data of a case study from healthcare. We show that the framework techniques implementing our approach outperform the baseline \revision{conformance checking}-based techniques while maintaining the explainable nature of \revision{conformance checking}.

In summary, the contributions of this paper are as follows.
\begin{itemize}
    \item{
        A novel \revision{process mining}-based feature extraction approach to support the development of competitive and explainable \revision{conformance checking}-based techniques for control-flow anomaly detection.
    }
    \item{
        A flexible and explainable framework for developing techniques for control-flow anomaly detection using \revision{process mining}-based feature extraction and dimensionality reduction.
    }
    \item{
        Application to synthetic and real-world datasets of several \revision{conformance checking}-based and \revision{conformance checking}-independent framework techniques, evaluating their detection effectiveness and explainability.
    }
\end{itemize}

The rest of the paper is organized as follows.
\begin{itemize}
    \item Section \ref{sec:related_work} reviews the existing techniques for control-flow anomaly detection, categorizing them into \revision{conformance checking}-based and \revision{conformance checking}-independent techniques.
    \item Section \ref{sec:abccfe} provides the preliminaries of \revision{process mining} to establish the notation used throughout the paper, and delves into the details of the proposed \revision{process mining}-based feature extraction approach with alignment-based \revision{conformance checking}.
    \item Section \ref{sec:framework} describes the framework for developing \revision{conformance checking}-based and \revision{conformance checking}-independent techniques for control-flow anomaly detection that combine \revision{process mining}-based feature extraction and dimensionality reduction.
    \item Section \ref{sec:evaluation} presents the experiments conducted with multiple framework and baseline techniques using data from publicly available datasets and case studies.
    \item Section \ref{sec:conclusions} draws the conclusions and presents future work.
\end{itemize}

\section{Related Work}
\section{Related Works}
\label{sec:related_works}


\noindent\textbf{Diffusion-based Video Generation. }
The advancement of diffusion models \cite{rombach2022high, ramesh2022hierarchical, zheng2022entropy} has led to significant progress in video generation. Due to the scarcity of high-quality video-text datasets \cite{Blattmann2023, Blattmann2023a}, researchers have adapted existing text-to-image (T2I) models to facilitate text-to-video (T2V) generation. Notable examples include AnimateDiff \cite{Guo2023}, Align your Latents \cite{Blattmann2023a}, PYoCo \cite{ge2023preserve}, and Emu Video \cite{girdhar2023emu}. Further advancements, such as LVDM \cite{he2022latent}, VideoCrafter \cite{chen2023videocrafter1, chen2024videocrafter2}, ModelScope \cite{wang2023modelscope}, LAVIE \cite{wang2023lavie}, and VideoFactory \cite{wang2024videofactory}, have refined these approaches by fine-tuning both spatial and temporal blocks, leveraging T2I models for initialization to improve video quality.
Recently, Sora \cite{brooks2024video} and CogVideoX \cite{yang2024cogvideox} enhance video generation by introducing Transformer-based diffusion backbones \cite{Peebles2023, Ma2024, Yu2024} and utilizing 3D-VAE, unlocking the potential for realistic world simulators. Additionally, SVD \cite{Blattmann2023}, SEINE \cite{chen2023seine}, PixelDance \cite{zeng2024make} and PIA \cite{zhang2024pia} have made significant strides in image-to-video generation, achieving notable improvements in quality and flexibility.
Further, I2VGen-XL \cite{zhang2023i2vgen}, DynamicCrafter \cite{Xing2023}, and Moonshot \cite{zhang2024moonshot} incorporate additional cross-attention layers to strengthen conditional signals during generation.



\noindent\textbf{Controllable Generation.}
Controllable generation has become a central focus in both image \citep{Zhang2023,jiang2024survey, Mou2024, Zheng2023, peng2024controlnext, ye2023ip, wu2024spherediffusion, song2024moma, wu2024ifadapter} and video \citep{gong2024atomovideo, zhang2024moonshot, guo2025sparsectrl, jiang2024videobooth} generation, enabling users to direct the output through various types of control. A wide range of controllable inputs has been explored, including text descriptions, pose \citep{ma2024follow,wang2023disco,hu2024animate,xu2024magicanimate}, audio \citep{tang2023anytoany,tian2024emo,he2024co}, identity representations \citep{chefer2024still,wang2024customvideo,wu2024customcrafter}, trajectory \citep{yin2023dragnuwa,chen2024motion,li2024generative,wu2024motionbooth, namekata2024sg}.


\noindent\textbf{Text-based Camera Control.}
Text-based camera control methods use natural language descriptions to guide camera motion in video generation. AnimateDiff \cite{Guo2023} and SVD \cite{Blattmann2023} fine-tune LoRAs \cite{hu2021lora} for specific camera movements based on text input. 
Image conductor\cite{li2024image} proposed to separate different camera and object motions through camera LoRA weight and object LoRA weight to achieve more precise motion control.
In contrast, MotionMaster \cite{hu2024motionmaster} and Peekaboo \cite{jain2024peekaboo} offer training-free approaches for generating coarse-grained camera motions, though with limited precision. VideoComposer \cite{wang2024videocomposer} adjusts pixel-level motion vectors to provide finer control, but challenges remain in achieving precise camera control.

\noindent\textbf{Trajectory-based Camera Control.}
MotionCtrl \cite{Wang2024Motionctrl}, CameraCtrl \cite{He2024Cameractrl}, and Direct-a-Video \cite{yang2024direct} use camera pose as input to enhance control, while CVD \cite{kuang2024collaborative} extends CameraCtrl for multi-view generation, though still limited by motion complexity. To improve geometric consistency, Pose-guided diffusion \cite{tseng2023consistent}, CamCo \cite{Xu2024}, and CamI2V \cite{zheng2024cami2v} apply epipolar constraints for consistent viewpoints. VD3D \cite{bahmani2024vd3d} introduces a ControlNet\cite{Zhang2023}-like conditioning mechanism with spatiotemporal camera embeddings, enabling more precise control.
CamTrol \cite{hou2024training} offers a training-free approach that renders static point clouds into multi-view frames for video generation. Cavia \cite{xu2024cavia} introduces view-integrated attention mechanisms to improve viewpoint and temporal consistency, while I2VControl-Camera \cite{feng2024i2vcontrol} refines camera movement by employing point trajectories in the camera coordinate system. Despite these advancements, challenges in maintaining camera control and scene-scale consistency remain, which our method seeks to address. It is noted that 4Dim~\cite{watson2024controlling} introduces absolute scale but in  4D novel view synthesis (NVS) of scenes.





\section{Process Reward Models}
% !TEX root =  ../main.tex
\section{Background on causality and abstraction}\label{sec:preliminaries}

This section provides the notation and key concepts related to causal modeling and abstraction theory.

\spara{Notation.} The set of integers from $1$ to $n$ is $[n]$.
The vectors of zeros and ones of size $n$ are $\zeros_n$ and $\ones_n$.
The identity matrix of size $n \times n$ is $\identity_n$. The Frobenius norm is $\frob{\mathbf{A}}$.
The set of positive definite matrices over $\reall^{n\times n}$ is $\pd^n$. The Hadamard product is $\odot$.
Function composition is $\circ$.
The domain of a function is $\dom{\cdot}$ and its kernel $\ker$.
Let $\mathcal{M}(\mathcal{X}^n)$ be the set of Borel measures over $\mathcal{X}^n \subseteq \reall^n$. Given a measure $\mu^n \in \mathcal{M}(\mathcal{X}^n)$ and a measurable map $\varphi^{\V}$, $\mathcal{X}^n \ni \mathbf{x} \overset{\varphi^{\V}}{\longmapsto} \V^\top \mathbf{x} \in \mathcal{X}^m$, we denote by $\varphi^{\V}_{\#}(\mu^n) \coloneqq \mu^n(\varphi^{\V^{-1}}(\mathbf{x}))$ the pushforward measure $\mu^m \in \mathcal{M}(\mathcal{X}^m)$. 


We now present the standard definition of SCM.

\begin{definition}[SCM, \citealp{pearl2009causality}]\label{def:SCM}
A (Markovian) structural causal model (SCM) $\scm^n$ is a tuple $\langle \myendogenous, \myexogenous, \myfunctional, \zeta^\myexogenous \rangle$, where \emph{(i)} $\myendogenous = \{X_1, \ldots, X_n\}$ is a set of $n$ endogenous random variables; \emph{(ii)} $\myexogenous =\{Z_1,\ldots,Z_n\}$ is a set of $n$ exogenous variables; \emph{(iii)} $\myfunctional$ is a set of $n$ functional assignments such that $X_i=f_i(\parents_i, Z_i)$, $\forall \; i \in [n]$, with $ \parents_i \subseteq \myendogenous \setminus \{ X_i\}$; \emph{(iv)} $\zeta^\myexogenous$ is a product probability measure over independent exogenous variables $\zeta^\myexogenous=\prod_{i \in [n]} \zeta^i$, where $\zeta^i=P(Z_i)$. 
\end{definition}
A Markovian SCM induces a directed acyclic graph (DAG) $\mathcal{G}_{\scm^n}$ where the nodes represent the variables $\myendogenous$ and the edges are determined by the structural functions $\myfunctional$; $ \parents_i$ constitutes then the parent set for $X_i$. Furthermore, we can recursively rewrite the set of structural function $\myfunctional$ as a set of mixing functions $\mymixing$ dependent only on the exogenous variables (cf. \cref{app:CA}). A key feature for studying causality is the possibility of defining interventions on the model:
\begin{definition}[Hard intervention, \citealp{pearl2009causality}]\label{def:intervention}
Given SCM $\scm^n = \langle \myendogenous, \myexogenous, \myfunctional, \zeta^\myexogenous \rangle$, a (hard) intervention $\iota = \operatorname{do}(\myendogenous^{\iota} = \mathbf{x}^{\iota})$, $\myendogenous^{\iota}\subseteq \myendogenous$,
is an operator that generates a new post-intervention SCM $\scm^n_\iota = \langle \myendogenous, \myexogenous, \myfunctional_\iota, \zeta^\myexogenous \rangle$ by replacing each function $f_i$ for $X_i\in\myendogenous^{\iota}$ with the constant $x_i^\iota\in \mathbf{x}^\iota$. 
Graphically, an intervention mutilates $\mathcal{G}_{\mathsf{M}^n}$ by removing all the incoming edges of the variables in $\myendogenous^{\iota}$.
\end{definition}

Given multiple SCMs describing the same system at different levels of granularity, CA provides the definition of an $\alpha$-abstraction map to relate these SCMs:
\begin{definition}[$\abst$-abstraction, \citealp{rischel2020category}]\label{def:abstraction}
Given low-level $\mathsf{M}^\ell$ and high-level $\mathsf{M}^h$ SCMs, an $\abst$-abstraction is a triple $\abst = \langle \Rset, \amap, \alphamap{} \rangle$, where \emph{(i)} $\Rset \subseteq \datalow$ is a subset of relevant variables in $\mathsf{M}^\ell$; \emph{(ii)} $\amap: \Rset \rightarrow \datahigh$ is a surjective function between the relevant variables of $\mathsf{M}^\ell$ and the endogenous variables of $\mathsf{M}^h$; \emph{(iii)} $\alphamap{}: \dom{\Rset} \rightarrow \dom{\datahigh}$ is a modular function $\alphamap{} = \bigotimes_{i\in[n]} \alphamap{X^h_i}$ made up by surjective functions $\alphamap{X^h_i}: \dom{\amap^{-1}(X^h_i)} \rightarrow \dom{X^h_i}$ from the outcome of low-level variables $\amap^{-1}(X^h_i) \in \datalow$ onto outcomes of the high-level variables $X^h_i \in \datahigh$.
\end{definition}
Notice that an $\abst$-abstraction simultaneously maps variables via the function $\amap$ and values through the function $\alphamap{}$. The definition itself does not place any constraint on these functions, although a common requirement in the literature is for the abstraction to satisfy \emph{interventional consistency} \cite{rubenstein2017causal,rischel2020category,beckers2019abstracting}. An important class of such well-behaved abstractions is \emph{constructive linear abstraction}, for which the following properties hold. By constructivity, \emph{(i)} $\abst$ is interventionally consistent; \emph{(ii)} all low-level variables are relevant $\Rset=\datalow$; \emph{(iii)} in addition to the map $\alphamap{}$ between endogenous variables, there exists a map ${\alphamap{}}_U$ between exogenous variables satisfying interventional consistency \cite{beckers2019abstracting,schooltink2024aligning}. By linearity, $\alphamap{} = \V^\top \in \reall^{h \times \ell}$ \cite{massidda2024learningcausalabstractionslinear}. \cref{app:CA} provides formal definitions for interventional consistency, linear and constructive abstraction.

\section{Limitations of Process Reward Models Trained on Math Domain Data}
\label{sec:math-lim}



We introduce various math PRMs used for comparison in~\Cref{sec:existing-mathprm}, present our multi-domain evaluation dataset in~\Cref{sec:mmlu-eval}, and provide a detailed analysis of the evaluation results in~\Cref{sec:mathprm-eval}.


\subsection{Open-Source Math PRMs}
\label{sec:existing-mathprm}



For evaluation,
we conduct experiments on a diverse set of models.
Our analysis includes four open-source math PRMs:
Math-PSA~\citep{wang2024openr},
Math-Shepherd~\citep{wang2024math},
RLHFLow-Deepseek~\citep{xiong2024rlhflowmath},
and Qwen-2.5-Math-PRM~\citep{zheng2024processbench}.


In addition to the open-source models,
two math PRMs based on open-source models are specifically trained in this work.
They are denoted as \emph{LlamaPRM800K} and \emph{QwenPRM800K}. More details are given in~\Cref{sec:open-mathprm}



\subsection{Multi-Domain Evaluation Dataset}
\label{sec:mmlu-eval}


For our multi-domain evaluation dataset, we curate questions sampled from the MMLU-Pro dataset~\citep{wang2024mmlupro}.
MMLU-Pro is designed to benchmark the reasoning abilities of LLMs and consists of college-level multiple choice questions in the following 14 domains:
\emph{Math}, \emph{Physics}, \emph{Chemistry}, \emph{Law}, \emph{Engineering}, \emph{Other}, \emph{Economics}, \emph{Health}, \emph{Psychology}, \emph{Business}, \emph{Biology}, \emph{Philosophy}, \emph{Computer Science}, and \emph{History}.


To craft our evaluation dataset, we randomly sample 150 questions from each domain. Due to duplicate questions, we discard 41 questions---23 from Biology, 10 from Health, 5 from Law, and 1 each from Business, Economics, and Philosophy. For each remaining question, we generate 128 candidate solutions using Llama-3.1-8B-Instruct~\citep{dubey2024llama} for MV, WMV, and BoN test-time inference algorithms.
Prompt details and generation parameters are provided in~\Cref{sec:synth-gen-prompts}.
We refer to this multi-domain evaluation dataset as \emph{\ourdataeval}.


\begin{table}[t]
\caption{Results of two open-source math PRMs on different domains in~\ourdataeval~when using WMV with min-aggregation on 16 CoTs generated per question using Llama-3.1-8B-Instruct. In parenthesis we report absolute difference between WMV and MV (WMV$-$MV). While WMV using math PRMs exhibits greater improvement in Math and Math-adjacent domains, there is no significant improvement on MV in other domains.}
\label{tab:math_prm_on_multi_domainsec4}
\small
\centering
\setlength{\tabcolsep}{3pt}
\begin{tabularx}{\linewidth}{l|c|
>{\centering\arraybackslash}X
>{\centering\arraybackslash}X}
\toprule
\textbf{Category} & \textbf{MV}  & \textbf{Math-Shepherd} & \textbf{Qwen-2.5-Math-PRM}   \\
\midrule
All & 57.15  & 57.66 (+0.51) & 58.17 (+1.02)\\
All except math & 56.61  & 57.01 (+0.40) & 57.32 (+0.71) \\
Math & 62.40  & 64.13 (+1.73) & 67.20 (+4.80)   \\
\midrule
Chemistry & 58.67  & 60.13 (+1.46) & 60.67 (+2.00) \\
Physics & 58.53 & 61.87 (+3.34) & 61.47 (+2.94) \\
\midrule
Biology & 75.38 & 75.38 (+0.00) & 75.69 (+0.31) \\
Psychology & 61.60  & 61.47 (-0.13) & 62.27 (+0.67)   \\
Law & 35.93 & 37.24 (+1.31) & 36.28 (+0.35)   \\
History & 49.20 & 49.87 (+0.67) & 49.40 (+0.20)  \\
Philosophy & 44.83  & 44.70 (-0.13) & 45.17 (+0.34) \\
\bottomrule
\end{tabularx}

    \vskip -0.2in
\end{table}



\subsection{Multi-Domain Performance of Math PRMs}
\label{sec:mathprm-eval}





We conduct comprehensive analyses on a diverse set of models. For clarity, we report results for two representative models here, with additional evaluations available in~\Cref{sec:mathprm-fullevals}. The first model, Math-Shepherd~\citep{wang2024math},
is trained on synthetically generated math data labeled via a rollout-based method.
The second model,
Qwen-2.5-Math-PRM~\citep{zheng2024processbench},
is a best-performing open-source PRM,
trained on the high-quality expert labeled PRM800K math dataset~\citep{lightman2023let}.


The PRMs are applied using WMV with min-aggregation. While math PRMs show significant improvements in mathematical reasoning domains,
their effectiveness in broader, non-mathematical areas remains limited.
Notably, in the Math category,
Qwen-2.5-Math-PRM and Math-Shepherd achieve relative gains of $+4.80$ and $+1.73$,
respectively,
outperforming the MV baseline.
Similar improvements are observed in Math-adjacent disciplines:
Chemistry ($+2.00$ for Qwen-2.5-Math-PRM) and Physics ($+3.34$ for Math-Shepherd),
underscoring their utility in tasks requiring mathematical reasoning.

\begin{highlight}
    \paragraph{Finding 1:} 
    \emph{Math PRMs struggle to generalize to broader domains.}
\end{highlight}

However, the benefits diminish sharply in non-mathematical areas.
For example, in Philosophy and History, we see gains of only $+0.34$ and +0.20\% respectively for the most performant PRM Qwen-2.5-Math-PRM.

The ``All except math'' aggregate further underscores this disparity, with PRMs achieving a maximum gain of $+0.71$ (Qwen-2.5-Math-PRM) compared with the majority voting baseline.

These results highlight a critical limitation: math PRMs trained exclusively on mathematical data lack the versatility to generalize beyond mathematical reasoning tasks. While they excel in contexts aligned with their training---quantitative reasoning---their capacity to evaluate reasoning quality in broader domains remains insufficient.


\section{Automatic Generation of Multi-Domain Reasoning Data with Labels}
\label{sec:synth-data-gen}

\begin{figure*}[ht]
    \begin{center}
        \includegraphics[width=0.91\textwidth]{figures/Synthetic_Data_Pipeline.pdf}
        \caption{A diagram of the synthetic data generation pipeline. In the CoT Generation Stage, each question is used to generate 16 CoT solutions. Then, in the Auto-Labeling Stage, each CoT is evaluated to create step-wise labels. If a CoT step is labeled as \textbf{BAD}, all subsequent steps will be discarded.}
        \label{fig:synth-data-pipeline-diagram}
    \end{center}
    \vskip -0.2in
\end{figure*}




In order to obtain step-wise reasoning data for non-Math domains,
we devise a pipeline,
as outlined in~\Cref{fig:synth-data-pipeline-diagram},
to generate synthetic reasoning CoTs from existing question-answering data.
These CoTs are then given step-wise labels based on reasoning correctness.
We detail the synthetic data generation process in~\Cref{sec:cot-gen,sec:auto},
including methods to create and annotate reasoning steps. We also provide additional analysis on the quality of the generation pipeline in~\Cref{sec:auto-analysis-ablate}.


\subsection{Chain-of-Thought Generation}
\label{sec:cot-gen}




For the generation of CoTs,
we prompt Llama-3.1-8B-Instruct to produce step-by-step reasoning for each input question. For training, we source questions from the MMLU-Pro dataset~\citep{wang2024mmlupro}, selected for its high-quality, challenging problems spanning diverse topics.
From this dataset, we randomly sample up to 500 questions per domain, ensuring that it is disjoint to the subset used for evaluation. We then generate 16 CoTs for each sampled question. Post-generation, we filter out CoTs exceeding the 2,048-token limit or containing unparsable answers.


\subsection{Auto-Labeling}
\label{sec:auto}


To annotate our synthetic CoT data, we adopt an approach inspired by the critic models in the work of~\citet{zheng2024processbench}.
Specifically, we utilize Llama-3.1-70B-Instruct as a strong LLM to evaluate each CoT using step-by-step reasoning, locating the earliest erroneous step, if any. To enhance accuracy and consistency, we identified two key additional components.


First, we incorporate explicit step evaluation definitions,
inspired by~\citet{lightman2023let},
into the system prompt.
Steps are categorized as \textbf{GOOD},
\textbf{OK},
or \textbf{BAD}:
\textbf{BAD} for incorrect, unverifiable, or irrelevant steps;
\textbf{GOOD} for correct, verifiable, and well-aligned steps;
\textbf{OK} for intermediate cases.
Second, we also provide the reference ground-truth answer in the prompt.
The full prompt is detailed in~\Cref{sec:synth-gen-prompts}.


To convert the auto-labeling output to stepwise labels, we apply the following rule:
if no steps are detected as incorrect, all steps in the CoT are labeled as $1$.
If a step is detected as incorrect, all preceding steps are labeled as $1$, the incorrect step is labeled as $-1$,
and all subsequent steps are discarded.



In total, we sample 5,750 questions from MMLU-Pro. Among the 84,098 generated CoTs that passed filtering, 36,935 were labeled as having no incorrect steps and 47,163 were labeled as having at least one (see Table \ref{tab:dataset_composition}). This dataset, denoted as \emph{\ourdatatrain}, is the first open-source multi-domain reasoning dataset with step-wise labels.






To assess the quality of our auto-labeled data,
we conduct a manual evaluation on a random sample of 30 questions from the dataset.
For each question, we randomly select one CoT classified as entirely correct and two CoTs flagged as containing an incorrect step.
We then manually validate whether the auto-labeled judgments align with our own assessments.


For the CoTs labeled as correct by the auto-labeler, we observed an agreement rate of 83\% with our manual evaluations.
For CoTs labeled as incorrect, the agreement rate was 70\%.

Based on these results, we estimate that approximately 75\% of the CoTs in the entire dataset are correctly labeled.
This level of accuracy is comparable to that of manually-labeled CoT datasets,
such as PRM800K~\citep{lightman2023let},
which is estimated to achieve around 80\% accuracy.\footnote{Refer to \href{https://github.com/openai/prm800k/issues/12}{this GitHub issue} for a discussion on PRM800K's accuracy.}


\subsection{Auto-Labeling Prompt Analysis}
\label{sec:auto-analysis-ablate}


To further understand the factors influencing auto-labeling performance,
we conduct an evaluation of the auto-labeling using a simplified prompt.
Specifically, we remove the system prompt defining the types of reasoning steps and exclude the reference ground-truth answer from the prompt.
When re-evaluating the auto-labeling quality,
we observed a drastic drop in performance, with the agreement rate for CoTs labeled as correct by the original auto-labeler decreasing by over 70\%,
from 83\% to 7\%, while
the agreement rate for CoTs labeled as incorrect decreased
from 70\% to 62\%.


These results highlight the importance of providing both a well-defined prompt with step label definitions and access to the ground-truth answer in achieving high auto-labeling accuracy.
The ground-truth answer provides essential context on CoT final correctness and enhances the model's ability to evaluate reasoning steps effectively.


\subsection{Counterfactual Augmentation}


To generate additional examples of incorrect reasoning,
we explore methods for instructing an LLM to modify steps in our correct CoTs,
introducing specific types of errors.
We refer to this process broadly as counterfactual augmentation.
However, incorporating counterfactual error steps during PRM training was not observed to significantly improve performance.
Therefore, we defer specific details and experiments using counterfactual augmentation to~\Cref{sec:counter-aug}.


\section{Multi-Domain Process Reward Model}
\label{sec:multi-eval}

We present the implementation and evaluation of \ourprm, structured as follows.
First, \Cref{sec:mdprm-train} covers the various training configurations used.  
We then evaluate \ourprm via BoN and WMV in \Cref{sec:math-v-mdprm}, showing improved domain generalization compared to math PRMs.  
In \Cref{sec:m-v-mdprm-search}, we additionally discuss results using Beam Search and MCTS. Lastly, we examine \ourprm's ability to scale test-time compute for larger models such as Deepseek-R1~\cite{guo2025deepseek} in~\Cref{sec:deepseek}.



\subsection{Training of Our Multi-Domain PRM}
\label{sec:mdprm-train}

To train \ourprm, we employ a classification head atop an LLM,
optimizing with a cross-entropy loss applied to a special classification token appended at the end of each CoT step in \ourdatatrain.
Detailed specifics and hyperparameters are provided in~\Cref{sec:prm-train}.


We explore several training configurations,
including:
1) LoRA~\citep{hu2022lora} vs.~full fine-tuning for efficient training,
2) a base LLM vs.~a math PRM for initializing the PRM,
and 3) a Qwen-based PRM vs.~a Llama-based PRM for training.
Comprehensive experimental results for these studies are presented in the next section.
Based on those findings, our final,
our final multi-domain PRM, named~\ourprm, is initialized from our LlamaPRM800K---see~\Cref{sec:add-prm-train} for its details---fine-tuned using LoRA on our multi-domain training dataset.
% \vspace{-6mm}
\subsection{Math PRM vs.~\ourprm~on Reranking Based Inference-Time Methods}
\label{sec:math-v-mdprm}

We first report results of the reranking methods WMV and BoN on \ourdataeval.
For both methods, we adopt Min-aggregation, as it outperforms Average and Last in aggregating PRM step scores;
see~\Cref{sec:agg-comp} for comparison.
We also include MV as a baseline.



\textbf{Comparison with Math Open-Source PRMs.}
We evaluate our multi-domain PRM, \ourprm, against open-source math PRMs by partitioning~\ourdataeval~into three groups:
1) \emph{Math},
2) \emph{Math-adjacent}, i.e., Chemistry, Computer Science, Engineering, Physics,
and 3) \emph{non-Math-adjacent} domains.
As shown in~\Cref{fig:math-wmv-min},
our model consistently outperforms baselines in both WMV and BoN across all domain groups.


\begin{highlight}
    \paragraph{Finding 2:} 
    \emph{Fine-tuning with synthetic multi-domain data enhances the generalizability of PRM.}
\end{highlight}


For WMV, we can see the relative performance difference increase with domain distance from core mathematics. While performance of open-source math PRMs converges to the majority voting baseline in non-mathematical domains, our multi-domain PRM maintains robust generalization.


In BoN the superiority of our multi-domain PRM is even more pronounced. Unlike open-source math PRMs, which fail to surpass the baseline of MV in Math-adjacent and non-Math-adjacent domains,
our model consistently surpasses it across all domain groups.


See~\Cref{sec:bon-mv-bycat} for more fine-grained details where we plot WMV and BoN for every domain of~\ourdataeval.
The results are consistent with~\Cref{fig:math-wmv-min}, and~\ourprm~outperforms math PRMs in all domains.



\begin{figure*}[t]
    \begin{center}
        \includegraphics[width=.94\linewidth]{figures/new_figures/math_vs_nonmath_prm_min_agg.pdf}        
        \caption{Comparison of WMV (top) and BoN (bottom) using \ourprm~against open-source math PRMs on~\ourdataeval. We use min-aggregation and the CoTs are generated using Llama-3.1-8B-Instruct. \ourprm~has consistently better performance than math PRMs, and the differences become larger in domains not adjacent to Math.}
        \label{fig:math-wmv-min}
    \end{center}
    \vskip -0.2in
\end{figure*}


\textbf{Ablation Experiments Using Multi-Domain PRM Trained on Math Only Subset vs.~Random Subset.}

We further conduct an ablation study to evaluate the impact of training data diversity on the performance of our LlamaPRM800K Math PRM.
Specifically, we train one PRM using only the math subset of our multi-domain training data and another using a random subset of the \emph{same} size.
We refer to these two models as \ourprm~(Math subset) and \ourprm~(random subset), respectively.
This experiment tests that the improved performance of our multi-domain PRM is due to the domain-diversity of the CoT data and not merely from learning the in-distribution question and CoT formats of MMLU-Pro questions. If the latter is the case, both PRMs should perform similarly, given that they are exposed to the same amount of questions and CoT examples with the in-distribution format.

\begin{highlight}
    \paragraph{Finding 3:} 
    \emph{Domain diversity of CoTs in a training dataset plays an integral role in generalization of PRMs to multiple domains.}
\end{highlight}

As shown in \Cref{fig:prm-ablation}, \ourprm~(random subset) achieves superior performance in WMV compared to \ourprm~(Math subset).
This trend holds across both Math and non-Math domains. These findings suggest two key insights.
First, our PRM is not simply learning the question format but is acquiring knowledge on how to label reasoning across diverse domains. This is why training on diverse data enables better overall performance than training on same sized data in only one domain. Second, \ourprm~(random subset) also demonstrates slightly better performance in the math domain, indicating that training on a diverse dataset may facilitate positive transfer, where insights from other domains enhance reasoning in the Math domain.



\begin{figure}[t]
    \centering
    \includegraphics[width=\columnwidth]{figures/new_figures/prm_diff_train_subset_min_agg.pdf}
    \caption{Comparison of WMV using LlamaPRM800K, \ourprm~(Math subset) and \ourprm~(random subset). \ourprm~(random subset) achieves better performance than \ourprm~(Math subset) in Math and non-Math.}
    \label{fig:prm-ablation}
    \vskip -0.2in
\end{figure}



\textbf{Experiments Using Other Training Configurations.}
While our final version of~\ourprm~is trained from LlamaPRM800K on our synthetic data using LoRA, we also test the following training configurations on our multi-domain dataset:
\begin{itemize}[leftmargin=10px]

    \item \textbf{\ourprm~(Llama Base)}: We initialize training from Llama-3.1-8B-Instruct, and use LoRA fine-tuning with our multi-domain dataset.

    \item \textbf{\ourprm~(Qwen)}: We initialize training from QwenPRM800K PRM, and utilize LoRA fine-tuning with our multi-domain dataset.

    \item \textbf{\ourprm~(full-tuned)}: We initialize training from LlamaPRM800K PRM, and do \emph{full} fine-tuning with our multi-domain dataset.

\end{itemize}

The results are presented in~\Cref{fig:multiprm-trainexps}.
Comparing \ourprm~(Qwen) and \ourprm~(Llama), we observe that the QwenPRM800K~\ourprm~performs worse.
This highlights the importance of base model choices. Although Qwen-2.5-Math-7B, the base model for QwenPRM800K, is specialized in mathematical reasoning, its limitations in general-domain knowledge hinder its ability to fully leverage multi-domain training data.


\begin{highlight}
    \paragraph{Finding 4:} 
    \emph{Exposure to mathematical data beforehand can enhance a PRMs' ability to effectively leverage multi-domain CoT fine-tuning.}
\end{highlight}


Next, comparing \ourprm~(Llama Base) with \ourprm, we find that the latter achieves superior performance in Math while maintaining comparable performance in non-Math domains. This suggests that prior exposure to mathematical data enhances the model’s ability to benefit from further domain-specific training.

We note that \ourprm~(full-tuned) has worse performance than \ourprm.
This may be due to suboptimal hyperparameters leading to overfitting during full fine-tuning.


\begin{figure}[t]
    \centering
    \includegraphics[width=\columnwidth]{figures/new_figures/prm_diff_min_agg.pdf}
    \caption{Comparison of MVW using \ourprm~against other multi-domain PRMs trained using different configurations. \ourprm~has better WMV performance than all other models in both Math and non-Math domains.}
    \label{fig:multiprm-trainexps}
\end{figure}



    
\subsection{Math PRM vs. Multi-Domain PRM on Search Based Inference-Time Methods}
\label{sec:m-v-mdprm-search}

We evaluate the performance of math PRMs (using LlamaPRM800K) and~\ourprm~with beam search and MCTS on~\ourdataeval.
The results over questions in all domains, presented in~\Cref{fig:prm-mcts},
show that MCTS outperforms beam search and that they both do better than the MV baseline.
Regardless of the search algorithm used, consistent with our WMN and BoN results,~\ourprm~gives boosted performance over the math PRM.
Details by category results are presented in~\Cref{sec:mcts-detailed}.


\begin{figure}[t]
    \centering
    \includegraphics[width=.95\columnwidth]{figures/new_figures/mcts_vs_beam.pdf}
    \caption{Comparison of \ourprm~and LlamaPRM800K with beam search and MCTS. Overall in the diverse domains from~\ourdataeval, \ourprm~achieves better performance.}
    \label{fig:prm-mcts}
    \vskip -0.2in
\end{figure}


\subsection{Does PRM with Test-Time Compute help Reasoning Models?}
\label{sec:deepseek}



\begin{figure}[t]
    \centering
    \includegraphics[width=\columnwidth]{figures/weighted_majority_voting_comparison_min.pdf}
    \caption{Comparison of WMV using~\ourprm~against Qwen-2.5-Math-PRM on DeepSeek-R1 generated CoTs for the Law subset. \ourprm~has better performance than all other math PRMs.}
    \label{fig:multiprm-deepseek}
\end{figure}


We have shown that~\ourprm~can effectively leverage inference-time compute to increase LLM performance,
a natural question is whether this effectiveness extends to renowned strong reasoning models,
e.g., DeepSeek-R1~\citep{guo2025deepseek}.
Given that a well-trained reasoning model may already generate coherent and correct reasoning steps due to being trained for reasoning,
one might hypothesize that reranking methods like WMV and BoN brings marginal improvement over WM.


To test this, we evaluate the performance of~\ourprm~via WMV on DeepSeek-R1.
Due to budget constraints, we focus on the Law subset and sample 16 CoT responses per question. As shown in~\Cref{fig:multiprm-deepseek}, \ourprm~provides a slight but noticeable performance boost to DeepSeek-R1 during test-time inference despite the limited CoT samples. Significantly it outperforms both the math PRM \emph{and} the MV baseline.
This finding, though preliminary, nullifies the aforementioned hypothesis and suggests that---in fact---large reasoning models \emph{can} still benefit from PRMs during inference to further boost their performance beyond MV.






\section{Discussion and Future Directions}
\section{Discussion of Assumptions}\label{sec:discussion}
In this paper, we have made several assumptions for the sake of clarity and simplicity. In this section, we discuss the rationale behind these assumptions, the extent to which these assumptions hold in practice, and the consequences for our protocol when these assumptions hold.

\subsection{Assumptions on the Demand}

There are two simplifying assumptions we make about the demand. First, we assume the demand at any time is relatively small compared to the channel capacities. Second, we take the demand to be constant over time. We elaborate upon both these points below.

\paragraph{Small demands} The assumption that demands are small relative to channel capacities is made precise in \eqref{eq:large_capacity_assumption}. This assumption simplifies two major aspects of our protocol. First, it largely removes congestion from consideration. In \eqref{eq:primal_problem}, there is no constraint ensuring that total flow in both directions stays below capacity--this is always met. Consequently, there is no Lagrange multiplier for congestion and no congestion pricing; only imbalance penalties apply. In contrast, protocols in \cite{sivaraman2020high, varma2021throughput, wang2024fence} include congestion fees due to explicit congestion constraints. Second, the bound \eqref{eq:large_capacity_assumption} ensures that as long as channels remain balanced, the network can always meet demand, no matter how the demand is routed. Since channels can rebalance when necessary, they never drop transactions. This allows prices and flows to adjust as per the equations in \eqref{eq:algorithm}, which makes it easier to prove the protocol's convergence guarantees. This also preserves the key property that a channel's price remains proportional to net money flow through it.

In practice, payment channel networks are used most often for micro-payments, for which on-chain transactions are prohibitively expensive; large transactions typically take place directly on the blockchain. For example, according to \cite{river2023lightning}, the average channel capacity is roughly $0.1$ BTC ($5,000$ BTC distributed over $50,000$ channels), while the average transaction amount is less than $0.0004$ BTC ($44.7k$ satoshis). Thus, the small demand assumption is not too unrealistic. Additionally, the occasional large transaction can be treated as a sequence of smaller transactions by breaking it into packets and executing each packet serially (as done by \cite{sivaraman2020high}).
Lastly, a good path discovery process that favors large capacity channels over small capacity ones can help ensure that the bound in \eqref{eq:large_capacity_assumption} holds.

\paragraph{Constant demands} 
In this work, we assume that any transacting pair of nodes have a steady transaction demand between them (see Section \ref{sec:transaction_requests}). Making this assumption is necessary to obtain the kind of guarantees that we have presented in this paper. Unless the demand is steady, it is unreasonable to expect that the flows converge to a steady value. Weaker assumptions on the demand lead to weaker guarantees. For example, with the more general setting of stochastic, but i.i.d. demand between any two nodes, \cite{varma2021throughput} shows that the channel queue lengths are bounded in expectation. If the demand can be arbitrary, then it is very hard to get any meaningful performance guarantees; \cite{wang2024fence} shows that even for a single bidirectional channel, the competitive ratio is infinite. Indeed, because a PCN is a decentralized system and decisions must be made based on local information alone, it is difficult for the network to find the optimal detailed balance flow at every time step with a time-varying demand.  With a steady demand, the network can discover the optimal flows in a reasonably short time, as our work shows.

We view the constant demand assumption as an approximation for a more general demand process that could be piece-wise constant, stochastic, or both (see simulations in Figure \ref{fig:five_nodes_variable_demand}).
We believe it should be possible to merge ideas from our work and \cite{varma2021throughput} to provide guarantees in a setting with random demands with arbitrary means. We leave this for future work. In addition, our work suggests that a reasonable method of handling stochastic demands is to queue the transaction requests \textit{at the source node} itself. This queuing action should be viewed in conjunction with flow-control. Indeed, a temporarily high unidirectional demand would raise prices for the sender, incentivizing the sender to stop sending the transactions. If the sender queues the transactions, they can send them later when prices drop. This form of queuing does not require any overhaul of the basic PCN infrastructure and is therefore simpler to implement than per-channel queues as suggested by \cite{sivaraman2020high} and \cite{varma2021throughput}.

\subsection{The Incentive of Channels}
The actions of the channels as prescribed by the DEBT control protocol can be summarized as follows. Channels adjust their prices in proportion to the net flow through them. They rebalance themselves whenever necessary and execute any transaction request that has been made of them. We discuss both these aspects below.

\paragraph{On Prices}
In this work, the exclusive role of channel prices is to ensure that the flows through each channel remains balanced. In practice, it would be important to include other components in a channel's price/fee as well: a congestion price  and an incentive price. The congestion price, as suggested by \cite{varma2021throughput}, would depend on the total flow of transactions through the channel, and would incentivize nodes to balance the load over different paths. The incentive price, which is commonly used in practice \cite{river2023lightning}, is necessary to provide channels with an incentive to serve as an intermediary for different channels. In practice, we expect both these components to be smaller than the imbalance price. Consequently, we expect the behavior of our protocol to be similar to our theoretical results even with these additional prices.

A key aspect of our protocol is that channel fees are allowed to be negative. Although the original Lightning network whitepaper \cite{poon2016bitcoin} suggests that negative channel prices may be a good solution to promote rebalancing, the idea of negative prices in not very popular in the literature. To our knowledge, the only prior work with this feature is \cite{varma2021throughput}. Indeed, in papers such as \cite{van2021merchant} and \cite{wang2024fence}, the price function is explicitly modified such that the channel price is never negative. The results of our paper show the benefits of negative prices. For one, in steady state, equal flows in both directions ensure that a channel doesn't loose any money (the other price components mentioned above ensure that the channel will only gain money). More importantly, negative prices are important to ensure that the protocol selectively stifles acyclic flows while allowing circulations to flow. Indeed, in the example of Section \ref{sec:flow_control_example}, the flows between nodes $A$ and $C$ are left on only because the large positive price over one channel is canceled by the corresponding negative price over the other channel, leading to a net zero price.

Lastly, observe that in the DEBT control protocol, the price charged by a channel does not depend on its capacity. This is a natural consequence of the price being the Lagrange multiplier for the net-zero flow constraint, which also does not depend on the channel capacity. In contrast, in many other works, the imbalance price is normalized by the channel capacity \cite{ren2018optimal, lin2020funds, wang2024fence}; this is shown to work well in practice. The rationale for such a price structure is explained well in \cite{wang2024fence}, where this fee is derived with the aim of always maintaining some balance (liquidity) at each end of every channel. This is a reasonable aim if a channel is to never rebalance itself; the experiments of the aforementioned papers are conducted in such a regime. In this work, however, we allow the channels to rebalance themselves a few times in order to settle on a detailed balance flow. This is because our focus is on the long-term steady state performance of the protocol. This difference in perspective also shows up in how the price depends on the channel imbalance. \cite{lin2020funds} and \cite{wang2024fence} advocate for strictly convex prices whereas this work and \cite{varma2021throughput} propose linear prices.

\paragraph{On Rebalancing} 
Recall that the DEBT control protocol ensures that the flows in the network converge to a detailed balance flow, which can be sustained perpetually without any rebalancing. However, during the transient phase (before convergence), channels may have to perform on-chain rebalancing a few times. Since rebalancing is an expensive operation, it is worthwhile discussing methods by which channels can reduce the extent of rebalancing. One option for the channels to reduce the extent of rebalancing is to increase their capacity; however, this comes at the cost of locking in more capital. Each channel can decide for itself the optimum amount of capital to lock in. Another option, which we discuss in Section \ref{sec:five_node}, is for channels to increase the rate $\gamma$ at which they adjust prices. 

Ultimately, whether or not it is beneficial for a channel to rebalance depends on the time-horizon under consideration. Our protocol is based on the assumption that the demand remains steady for a long period of time. If this is indeed the case, it would be worthwhile for a channel to rebalance itself as it can make up this cost through the incentive fees gained from the flow of transactions through it in steady state. If a channel chooses not to rebalance itself, however, there is a risk of being trapped in a deadlock, which is suboptimal for not only the nodes but also the channel.

\section{Conclusion}
This work presents DEBT control: a protocol for payment channel networks that uses source routing and flow control based on channel prices. The protocol is derived by posing a network utility maximization problem and analyzing its dual minimization. It is shown that under steady demands, the protocol guides the network to an optimal, sustainable point. Simulations show its robustness to demand variations. The work demonstrates that simple protocols with strong theoretical guarantees are possible for PCNs and we hope it inspires further theoretical research in this direction.


\section*{Acknowledgements}

Kangwook Lee is supported by NSF CAREER Award CCF-2339978, an Amazon Research Award, and a grant from FuriosaAI. In addition, Thomas Zeng acknowledges support from NSF under NSF Award DMS-202323 and Daewon Chae was supported by the Hyundai Motor Chung Mong-Koo Foundation.


\section*{Impact Statement}

Given the potential for LLMs to be used in unethical ways, such as spreading misinformation or manipulating public opinion, our Multi-Domain PRM could inadvertently contribute to such misuse. To mitigate these risks, it is essential to implement robust safeguards in training and inference.


\bibliography{refs}
\bibliographystyle{icml2025}


\newpage
\appendix
\onecolumn

\section{More Details on Synthetic Data Generation Pipeline}
\subsection{Dataset Composition}
The total composition of \ourdatatrain~is as follows.

\begin{table}[ht]
    \centering
    \caption{Composition of \textit{\ourdatatrain}}
    \small
    \begin{tabular}{cccc} \toprule
          & \textbf{Total} & \textbf{Fully Correct} & \textbf{Incorrect}   \\ \midrule
         Number of CoTs & 84098 & 36935 & 47163 \\
         Number of Steps & 487380 & 440217 & 47163\\
         \bottomrule
    \end{tabular} 
    \label{tab:dataset_composition}
\end{table}


\subsection{Data Generation Pipeline Prompts}
\label{sec:synth-gen-prompts}


To generate chain-of-thought (CoT) reasoning for MMLU-Pro questions, we utilize the prompt shown in~\Cref{fig:cot-gen-prompt-mmlu}.
To ensure the generated CoT adhere to the proper format---where steps are separated by two newline characters and the final step follows the structure ``the answer is (X)''---we include five few-shot examples. These examples are derived from the CoTs provided in the validation split of MMLU-Pro, with additional processing to ensure each step is delimited. The code for generating the complete prompt will be open-sourced alongside the rest of our code and data.


During generation, we use a temperature of $0.8$ and set the maximum generation length to 2,048 tokens. During auto-labeling, we use a temperature of 0, and the maximum generation length remains at 2,048 tokens.

\begin{figure}[ht]
    \centering
    \begin{minipage}{6in}
    \begin{tcolorbox}[width=6in, sharp corners=all, colback=white!95!black]
The following is a multiple choice question and its ground truth answer. You are also given a students solution (split into step, enclosed with tags and indexed from 0):

\-

[Multiple Choice Question]

\{question\}

\-

[Ground Truth Answer]

\{answer\}

\-

[Student Solution]

\{$<$step\_0$>$\\
Student solution step 0\\
$<$/step\_0$>$

\-\\
$<$step\_1$>$\\
Student solution step 0\\
$<$/step\_1$>$

\-\\...\}

\end{tcolorbox}
    \end{minipage}
    \caption{User prompt template for auto-labeling.}
    \label{fig:v6-auto-label-prompt}
\end{figure}

\begin{figure}[ht]
    \centering
    \begin{minipage}{6in}
    \begin{tcolorbox}[width=6in, sharp corners=all, colback=white!95!black]

You are an experienced evaluator specializing in assessing the quality of reasoning steps in problem-solving. Your task is to find the first BAD step in a student's solution to a multiple choice question.

\-\\
You will judge steps as GOOD, OK or BAD based on the following criteria:\\
1. GOOD Step\\
A step is classified as GOOD if it meets all of these criteria:\\
- Correct: Everything stated is accurate and aligns with known principles or the given problem.\\
- Verifiable: The step can be verified using common knowledge, simple calculations, or a quick reference (e.g., recalling a basic theorem). If verifying requires extensive effort (e.g., detailed calculations or obscure references), mark it BAD instead.\\
- Appropriate: The step fits logically within the context of the preceding steps. If a prior mistake exists, a GOOD step can correct it.\\
- Insightful: The step demonstrates reasonable problem-solving direction. Even if ultimately progress in the wrong direction, it is acceptable as long as it represents a logical approach.

\-\\
2. OK Step\\
A step is classified as OK if it is:\\
- Correct and Verifiable: Contains no errors and can be verified.\\
- Unnecessary or Redundant: Adds little value, such as restating prior information or providing basic encouragement (e.g., “Good job!”).\\
- Partially Progressing: Makes some progress toward the solution but lacks decisive or significant advancement.

\-\\
3. BAD Step\\
A step is classified as BAD if it:\\
- Is Incorrect: Contains factual errors, misapplies concepts, derives an incorrect result, or contradicts the ground truth answer.\\
- Is Hard to Verify: Requires significant effort to confirm due to poor explanation.\\
- Is Off-Topic: Includes irrelevant or nonsensical information.\\
- Derails: Leads to dead ends, circular reasoning, or unreasonable approaches.

\-\\
\#\#\#\# Task Description\\
You will be provided with:\\
1. A Question\\
2. A Ground Truth Answer\\
3. A Reference explanation of the answer\\
4. A Student's Step-by-Step Solution, where each step is enclosed with tags and indexed from 0

\-\\
You may use the ground truth answer and reference explanation in classifying the type of each step.\\
A student's final answer is considered correct if it matches the ground truth answer or only differs due to differences in how the answer is rounded.
Once you identify a BAD step, return the index of the earliest BAD step. Otherwise,
return the index of -1 (which denotes all steps are GOOD or OK).
Please put your final answer (i.e., the index) in $\backslash\backslash$boxed{}.
\end{tcolorbox}
    \end{minipage}
    \caption{System prompt for auto-labeling.}
    \label{fig:v5-auto-label-prompt}
\end{figure}

% 

\clearpage

\subsection{Counterfactual Augmentation}
\label{sec:counter-aug}


\begin{figure*}[ht]
    \begin{center}
        \includegraphics[width=0.9\textwidth]{figures/Counterfactual_Augmentation_Pipeline.pdf}
         \caption{Diagram of the counterfactual augmentation pipeline}
        \label{fig:neg-aug-pipeline}
    \end{center}
\end{figure*}


After generating and labeling our synthetic reasoning CoTs (as described in~\Cref{sec:synth-data-gen}), we attempted to create additional incorrect steps by augmenting the correct reasoning steps. Our pipeline is depicted in~\Cref{fig:neg-aug-pipeline}.
We provide the full CoT to Llama-3.1-70B-Instruct, instructing it to select and rewrite a step where it would be appropriate to introduce an error.
Additionally, we define a list of possible fine-grained error types. To encourage the generation of a variety of different error types, we only include a random selection of two of these error types in each system prompt, forcing the LLM to choose one. The error types are:
\begin{itemize}
    \item Conflicting Steps: The reasoning step includes statements that contradict previous steps.
    \item Non-sequitur: The reasoning step introduces information that is irrelevant to the question.
    \item Factual: The reasoning step contains incorrect statements of factual information.
    \item False Assumption: The reasoning step makes an incorrect assumption about the question.
    \item Contextual: The reasoning step misinterprets information given within the question/context.
\end{itemize}



For the prompt format used in counterfactual augmentation,
see~\Cref{fig:neg-augmentation-system-prompt}.
In total, we generated 73,829 augmented incorrect steps.


\begin{figure}[ht]
    \centering
    \begin{minipage}{6in}
    \begin{tcolorbox}[width=6in, sharp corners=all, colback=white!95!black]
    The following are multiple choice questions (with answers). Think step by step and then finish your answer with "the answer is (X)" where X is the correct letter choice.
    \end{tcolorbox}
    \end{minipage}
    \caption{Prompt to generate CoTs for MMLU Pro.}
    \label{fig:cot-gen-prompt-mmlu}
\end{figure}


\begin{figure}[ht]
    \centering
    \begin{minipage}{6in}
    \begin{tcolorbox}[width=6in, sharp corners=all, colback=white!95!black]
\small You are a highly knowledgeable philosopher with expertise across many domains, tasked with analyzing reasoning processes. 
Your goal is to identify how a reasoning process could naturally deviate toward an incorrect conclusion through the introduction of subtle errors.

\-\\
Here are a list of potential error types, all of which are equally valid:\\
\textrm{[ERROR TYPE 1]: [ERROR TYPE 1 DEFINITION]}\\
\textrm{[ERROR TYPE 2]: [ERROR TYPE 2 DEFINITION]}

\-\\
Instructions:\\
You will be provided with:\\
1. A question.\\
2. A complete chain of reasoning steps, where each step is numbered (e.g., Step X).

\-\\
Your task is to:
1. Identify the major factual information, reasoning, and conclusions within the reasoning steps.\\
3. Explain how to generate an incorrect step to replace one of the existing steps. This should include:\\
   - Identifying a step where the reasoning could naturally deviate.\\
   - Speculating what type of error would be most appropriate to introduce at the chosen step.\\
4. Introduce an incorrect next step that aligns stylistically with the previous steps. This incorrect step should:\\
   - Reflect a deviation in reasoning that significantly harms the correctness.\\
   - Appear natural and believable in the context of the reasoning process.\\
5. Clearly explain how the incorrect step is an error, highlighting the specific logical or conceptual flaw.

\-\\
Output Format:

\-\\
STEP\_SUMMARY:\\
\textrm{[Summarize the reasoning within the steps in 1-2 sentences, identifying major information, logical steps, and conclusions.]}

\-\\
INCORRECT\_STEP\_GEN:\\
\textrm{[Explain how the reasoning at a specific step could deviate naturally into being incorrect. Clearly describe the type of error that could be introduced at this step.]}

\-\\
ERROR\_TYPE:\\
\textrm{[The name of the type of error chosen to be introduced.]}

\-\\
STEP\_NUM:\\
\textrm{[The number of the step that was identified as a place where the reasoning could naturally deviate. Only include the number here.]}

\-\\
INCORRECT\_STEP:\\
\textrm{[Write the incorrect step in the same tone and style as the other steps. Wrap the incorrect step inside curly braces (e.g. \{incorrect step\}).]}

\-\\
ERROR\_EXPLANATION:\\
\textrm{[}Explain how the incorrect step fits the definition of the selected error type, identifying the specific flaw.\textrm{]}
\end{tcolorbox}
    \end{minipage}
    \caption{System prompt for counterfactual augmentation.}
    \label{fig:neg-augmentation-system-prompt}
\end{figure}




\clearpage

\section{Additional Search Algorithm Details}
\label{sec:search-algs}

\begin{algorithm}[ht]
\caption{Beam Search with Process Reward Model}
\label{alg:beam}
\begin{algorithmic}[1]
\REQUIRE Large Language Model $\text{LLM}(\cdot)$, Process Reward Model $\text{PRM}(\cdot)$, Prompt $s_0$, Number of Beams $N$, Beam width $M$, Maximum step length $L$
\STATE $\mathcal{B} \gets [s_0]$
\STATE $\mathcal{Q} \gets [0]$
\FOR{$i = 1$ to $L$}
    \STATE $\mathcal{B} \gets \text{Expand}(\mathcal{B}, \frac{N}{\operatorname{len}(\mathcal{B})})$

    \STATE $\mathcal{B} \gets \text{LLM.step}(\mathcal{B})$


    \STATE $\mathcal{Q} \gets \text{Aggr}(\mathcal{B})$

    \STATE $\texttt{best\_idxs} \gets$ Indexes of the highest $\frac{N}{M}$ scores in $\mathcal{Q}$

    \STATE $\mathcal{B} \gets \mathcal{B}[\texttt{best\_idxs}]$
    \STATE $\mathcal{Q} \gets \mathcal{Q}[\texttt{best\_idxs}]$
    
    \IF{All sequences in $\mathcal{B}$ contain a terminal leaf node}
        \STATE \textbf{break}
    \ENDIF
\ENDFOR
\STATE Return the sequence with the highest score from $\mathcal{B}$

\end{algorithmic}
\end{algorithm}



\Cref{alg:beam} is a greedy search algorithm that uses a PRM select the best CoT during search. More details are given in \Cref{sec:inference-time-methods}.


\clearpage


\begin{algorithm}[ht]
\caption{Monte Carlo Tree Search with Process Reward Model}
\label{alg:mcts}
\begin{algorithmic}[1]
    \REQUIRE Large Language Model $\text{LLM}(\cdot)$, Process Reward Model $\text{PRM}(\cdot)$, Prompt $s_0$, Maximum step length $L$, Number of roll-outs $K$, Number of generated child nodes $d$, Exploration weight $w$
    \STATE Initialize the value function $Q : \mathcal S \mapsto \mathbb R$ and
    visit counter $N : \mathcal S \mapsto \mathbb N$ 
    \FOR {$n \gets 0, \dots, K - 1$}
        \STATE // \textit{Selection}
        \STATE $t \gets 0$
        \WHILE {$s_t$ is not a leaf node}
            \STATE $N(s_t) \gets N(s_t) + 1$ 
            \STATE $s_{t+1} \gets \arg\max_{\text{children}(s_t)} \left[ Q(\text{child}(s_t)) + w \sqrt{\frac{\ln N(s_t)}{N(\text{child}(s_t))}} \right]$
            \STATE $t \gets t + 1$
        \ENDWHILE
        \STATE // \textit{Expansion \& Simulation} (equivalent to the beam search with $N=M=d$)
        \STATE $\mathcal{B} \gets [s_t]$
        \WHILE {$s_t$ is not a terminal leaf node $\wedge$ $t \leq L$}
            \STATE $N(s_t) \gets N(s_t) + 1$
            \STATE $\mathcal{B} \gets \text{Expand}(\mathcal{B}, d)$
            \STATE $\mathcal{B} \gets \text{LLM.step}(\mathcal{B})$
    
        \FOR {$s \in \mathcal{B}$} 
            \STATE $Q(s) \gets \text{Aggr}(s)$
            %\STATE $N(s) \gets N(s) + 1$
            %\STATE $Q(s) \gets \text{PRM}(s)$ // or $\text{Aggr}(s)$
        \ENDFOR
        
            %\STATE $\texttt{best\_idx} \gets$ Index of the highest score in $\mathcal{S}$
            \STATE $s_{t+1} \gets \arg\max_{s \in \mathcal{B}} Q(s)$ 
            \STATE $t \gets t + 1$ 
            \STATE $\mathcal{B} \gets [s_t]$
        \ENDWHILE
        \STATE // \textit{Back Propagation}
        \FOR {$t' \gets t, \dots, 0$}
            \STATE $Q(s_{t'}) \gets \max (Q(s_{t'}), Q(s_{t}))$
        \ENDFOR
    \ENDFOR
    \STATE Return the sequence with the highest score among the terminal nodes
\end{algorithmic}
\end{algorithm}



\Cref{alg:mcts} is a tree-based search algorithm that iteratively expands a search tree to find the CoT with the highest PRM score. MCTS iteratively builds a search tree through the following steps:
\begin{enumerate}
    \item \textbf{Selection}: Starting from the root node, the algorithm traverses the tree by selecting child nodes according to a selection policy.
    \item \textbf{Expansion and Simulation}: Upon reaching a non-terminal leaf node, the tree is expanded iteratively by generating a fixed number of child nodes and then greedily selecting the child node with the highest value (which for us is determined by the PRM). This process continues until a terminal node is reached.
    \item \textbf{Backpropagation}: The results from the simulation are propagated back through the tree, updating value estimates and visit counts for each node along the path.
\end{enumerate}
These steps are repeated for a fixed number of iterations or until a computational or time limit is reached. To determine the final prediction, we choose the terminal node with the highest value.


\clearpage


\section{Additional PRM Training Details}
\label{sec:add-prm-train}
\subsection{Open-Source Math PRM Training Details}
\label{sec:open-mathprm}


The open-source PRMs evaluated in this work utilize CoT training data derived from two mathematical datasets:
MATH~\citep{hendrycks2measuring} and GSM8K~\citep{cobbe2021training}.
The Math-Shepherd and RLHFlow/Deepseek-PRM-Data datasets are synthetically generated following the rollout method proposed by~\citet{wang2024math}.
Similarly, the MATH-APS dataset is produced using the synthetic generation technique introduced by~\citet{luo2024improve}. PRM800K, in contrast, consists of manually annotated labels and was specifically curated for the study by~\citet{lightman2023let}.


All PRMs are trained using the base LLMs of comparable model size and class,
including Mistral-7B~\citep{jiang2023mistral},
Llama-3.1-8B-Instruct~\citep{dubey2024llama},
and Qwen-2.5-Math 8B~\citep{yang2024qwen2}.


\begin{table*}[ht]
    \centering
    \caption{Training details of various Math PRMs}
    \resizebox{\linewidth}{!}
    {
    \begin{tabular}{lccr}
    \toprule
    \textbf{PRM} & \textbf{Base Model} & \textbf{Training Data} & \textbf{Training Method} \\
    \midrule
    Math-PSA & Qwen-2.5-Math-7B-Instruct & PRM800K, Math-Shepherd and MATH-APS & LoRA \\
    Math-Shepherd & Mistral-7B & Math-Shepherd & Full fine-tuning \\
    Qwen-2.5-Math-PRM & Qwen-2.5-Math-7B-Instruct & PRM800K & Full fine-tuning \\
    RLHFLow-Deepseek & Llama3.1-8B-Instruct & RLHFlow/Deepseek-PRM-Data & Full fine-tuning \\
    \midrule
    LlamaPRM800K & Llama3.1-8B-Instruct & PRM800K & Full fine-tuning \\
    QwenPRM800K & Qwen-2.5-Math-7B-Instruct & PRM800K & Full fine-tuning \\
    \bottomrule
    \end{tabular}
    }
    \label{tab:math_prm_details}
\end{table*}


\subsection{Details of PRM Training}
\label{sec:prm-train}




For training, we extract logits from the tokens \texttt{+} and \texttt{-} in the final layer of the LLM. The logit for \texttt{+} corresponds to a correct reasoning step, while the logit for \texttt{-} represents an incorrect step. We use four newline characters \texttt{\textbackslash n\textbackslash n\textbackslash n\textbackslash n} as the classification token, which is appended to the end of each reasoning step. We use standard cross-entropy loss and only compute it over our classification token.

For training our math PRMs on the PRM800K dataset (QwenPRM800K and LlamaPRM800K),
we employ a batch size of 128 and perform full fine-tuning. For experiments on mixed-domain datasets, we reduce the batch size to 32 due to smaller dataset size.

All training is conducted over a single epoch. For full fine-tuning, we use a learning rate of $1.25 \times 10^{-6}$, while for LoRA-based fine-tuning, we use a learning rate of $1.0 \times 10^{-4}$.


\clearpage

\section{Additional PRM Training and Evaluation Experiments}
\subsection{Evaluation Results for Math PRMs and~\ourprm~Across all Categories}
\label{sec:mathprm-fullevals}





\begin{table*}[ht]
\caption{Comparison among various math PRMs and~\ourprm~on different domains in~\ourdataeval~when using WMV with min-aggregation on $N=16$ CoTs generated per question using Llama3.1-8B-Instruct. In parenthesis we report the relative difference between WMV and the MV baseline (WMV$-$MV). While WMV using math PRMs exhibit greater improvement in math and math-adjacent domains, there is no significant improvement on MV in other domains.}
\small
\centering
\setlength{\tabcolsep}{4pt}
\resizebox{\textwidth}{!}{
\begin{tabular}{l|c|cccccc}
\toprule
\textbf{Category} & \textbf{MV (Baseline)} & \textbf{Math-PSA} & \textbf{Math-Shepherd} & \textbf{Qwen-2.5-Math-PRM} & \textbf{RLHFLow-Deepseek} & \textbf{LlamaPRM800K} & \textbf{\ourprm} \\
\midrule
All & 57.15 & 57.87(+0.72) & 57.66(+0.51) & 58.17(+1.02) & 57.59(+0.44) & 58.16(+1.01) & \textbf{61.22(+4.07)} \\
All except math & 56.61 & 56.82(+0.21) & 57.01(+0.40) & 57.32(+0.71) & 56.96(+0.35) & 57.71(+1.10) & \textbf{60.29(+3.68)} \\
Math & 62.40 & 64.20(+1.80) & 64.13(+1.73) & 67.20(+4.80) & 64.07(+1.67) & 65.40(+3.00) & \textbf{68.87(+6.47)} \\
Math-Adjacent & 56.75 & 57.98(+1.23) & 57.48(+0.73) & 58.30(+1.55) & 57.33(+0.58) & 58.27(+1.52) & \textbf{61.22(+4.47)} \\
Non-Math-Adjacent & 56.69 & 56.79(+0.10) & 57.14(+0.45) & 57.09(+0.40) & 57.02(+0.33) & 57.55(+0.86) & \textbf{60.00(+3.31)} \\
\midrule
Chemistry & 58.67 & 60.47(+1.80) & 60.13(+1.46) & 60.67(+2.00) & 59.13(+0.46) & 60.47(+1.80) & \textbf{66.13(+7.46)} \\
Computer Science & 55.80 & 56.93(+1.13) & 56.07(+0.27) & 56.13(+0.33) & 56.07(+0.27) & 56.40(+0.60) & \textbf{58.60(+2.80)} \\
Engineering & 51.67 & 50.67(-1.00) & 51.07(-0.60) & 53.13(+1.46) & 51.87(+0.20) & 52.27(+0.60) & \textbf{55.27(+3.60)} \\
Physics & 58.53 & 61.87(+3.34) & 61.87(+3.34) & 61.47(+2.94) & 60.80(+2.27) & 61.47(+2.94) & \textbf{64.87(+6.34)} \\
\midrule
Biology & 75.38 & 75.23(-0.15) & 75.38(+0.00) & 75.69(+0.31) & 75.77(+0.39) & 76.38(+1.00) & \textbf{80.00(+4.62)} \\
Health & 63.36 & 63.00(-0.36) & 63.93(+0.57) & 63.50(+0.14) & 63.57(+0.21) & 64.50(+1.14) & \textbf{65.50(+2.14)} \\
Psychology & 61.60 & 61.47(-0.13) & 61.47(-0.13) & 62.27(+0.67) & 61.47(-0.13) & 61.87(+0.27) & \textbf{64.53(+2.93)} \\
Business & 61.34 & 61.95(+0.61) & 62.21(+0.87) & 63.02(+1.68) & 62.21(+0.87) & 62.62(+1.28) & \textbf{64.50(+3.16)} \\
Economics & 62.00 & 62.67(+0.67) & 62.33(+0.33) & 62.53(+0.53) & 62.67(+0.67) & 62.40(+0.40) & \textbf{64.27(+2.27)} \\
Law & 35.93 & 35.72(-0.21) & 37.24(+1.31) & 36.28(+0.35) & 36.07(+0.14) & 36.90(+0.97) & \textbf{43.86(+7.93)} \\
History & 49.20 & 49.00(-0.20) & 49.87(+0.67) & 49.40(+0.20) & 49.40(+0.20) & 49.87(+0.67) & \textbf{50.67(+1.47)} \\
Philosophy & 44.83 & 44.90(+0.07) & 44.70(-0.13) & 45.17(+0.34) & 44.56(-0.27) & 45.30(+0.47) & \textbf{49.13(+4.30)} \\
Other & 55.53 & 55.80(+0.27) & 55.47(-0.06) & 56.07(+0.54) & 55.87(+0.34) & 57.07(+1.54) & \textbf{59.00(+3.47)} \\
\bottomrule
\end{tabular}
}
\vskip -0.1in
\label{tab:math_prm_on_multi_domain}
\end{table*}


\subsection{PRM Training with Counterfactual Augmented Data}

\begin{figure*}[ht]
    \begin{center}
        \includegraphics[width=\linewidth]{figures/new_figures/prm_noaugs_vs_augs_min_agg.pdf}        
        \caption{Comparison of WMV (top) and BoN (bottom) using our two multi-domain PRMs (w/ and w/o counterfactually augmented training data) on the categories of~\ourdataeval. We use min-aggregation and the CoTs are generated using Llama-3.1-8B-Instruct. When using WMV, counterfactual augmented data can further improve the performance of PRM on non-math-adjacent domains.}
        \label{fig:counteraug}
    \end{center}
\end{figure*}


\clearpage


\subsection{WMV and BoN using different aggregation methods}
\label{sec:agg-comp3}


\begin{figure*}[ht]
    \begin{center}
        \includegraphics[width=\linewidth]{figures/new_figures/prm_diff_agg.pdf}        
        \caption{Comparison of WMV (left) and BoN (right) using \ourprm~with different reward aggregations on~\ourdataeval. The CoTs are generated using Llama 3.1 8B Instruct. Overall, min-aggregation brings the largest inference performance boost.}
        \label{fig:prm-diff-agg2}
    \end{center}
\end{figure*}


\clearpage


\subsection{Larger Generator Inference with PRM Rewarding}
\label{sec:agg-comp}


\begin{figure*}[ht]
    \begin{center}
        \includegraphics[width=\linewidth]{figures/new_figures/prm_on_70b_cot.pdf}        
        \caption{Comparison of WMV (left) and BoN (right) using \ourprm~against math PRMs on~\ourdataeval. We use min-aggregation and the CoTs are generated using Llama-3.1-70B-Instruct. Similar trends to using 8B model as the generator are observed, indicating that our Multi-Domain PRM can generalize across generators with different capacities.}
        \label{fig:prm-diff-agg}
    \end{center}
\end{figure*}


\clearpage


\subsection{Inference with Compact PRM}
\label{sec:agg-comp2}


\begin{figure*}[ht]
    \begin{center}
        \includegraphics[width=\linewidth]{figures/new_figures/prm_with_3b_min_agg.pdf}        
        \caption{Comparison of WMV (top) and BoN (bottom) using \ourprm~(Llama3B Base), a compact PRM based on Llama-3.2-3B-Instruct and trained on our multi-domain dataset. We use min-aggregation and the CoTs are generated using Llama-3.1-8B-Instruct. Compared with using \ourprm~(Llama Base), which applies the same training data and configurations but is based on Llama-3.1-8B-Instruct, \ourprm~(Llama3B Base) brings a less significant performance boost. However, the overall trends are similar, indicating the efficacy of the inference pipeline using PRM.}
        \label{fig:prm-diff-agg1}
    \end{center}
\end{figure*}

\clearpage


\subsection{Comparison of~\ourprm~against Other Open-Source Math PRMs on WMV and BoN by Category}
\label{sec:bon-mv-bycat}


\begin{figure*}[ht]
    \begin{center}
        \includegraphics[width=\linewidth]{figures/new_figures/prm_all_domains_wmv_min_agg.pdf}        
        \caption{Comparison of WMV using \ourprm~against open-source PRMs on more other categories of~\ourdataeval. We use min-aggregation and the CoTs are generated using Llama-3.1-8B-Instruct.}
        \label{fig:prm-wmv-more-domains}
    \end{center}
\end{figure*}


\begin{figure*}[ht]
    \begin{center}
        \includegraphics[width=\linewidth]{figures/new_figures/prm_all_domains_bon_min_agg.pdf}        
        \caption{Comparison of BoN using \ourprm~against open-source PRMs on more other categories of~\ourdataeval. 
        We use min-aggregation and the CoTs are generated using Llama-3.1-8B-Instruct.}
        \label{fig:prm-bon-more-domains1}
    \end{center}
\end{figure*}


\clearpage

\subsection{Comparison of~\ourprm~against Other Open-Source Math PRMs on MCTS and Beam Search by Category}
\label{sec:mcts-detailed}


\begin{figure*}[ht]
    \begin{center}
        \includegraphics[width=\linewidth]{figures/new_figures/all_domains_mcts_vs_beam.pdf}        
        \caption{Comparison of \ourprm~and LlamaPRM800K
        with beam search and MCTS. In more other categories from~\ourdataeval, \ourprm~achieves better performance.}
        \label{fig:prm-wmv-more-domains2}
    \end{center}
\end{figure*}


\end{document}


% This document was modified from the file originally made available by
% Pat Langley and Andrea Danyluk for ICML-2K. This version was created
% by Iain Murray in 2018, and modified by Alexandre Bouchard in
% 2019 and 2021 and by Csaba Szepesvari, Gang Niu and Sivan Sabato in 2022.
% Modified again in 2023 and 2024 by Sivan Sabato and Jonathan Scarlett.
% Previous contributors include Dan Roy, Lise Getoor and Tobias
% Scheffer, which was slightly modified from the 2010 version by
% Thorsten Joachims & Johannes Fuernkranz, slightly modified from the
% 2009 version by Kiri Wagstaff and Sam Roweis's 2008 version, which is
% slightly modified from Prasad Tadepalli's 2007 version which is a
% lightly changed version of the previous year's version by Andrew
% Moore, which was in turn edited from those of Kristian Kersting and
% Codrina Lauth. Alex Smola contributed to the algorithmic style files.
