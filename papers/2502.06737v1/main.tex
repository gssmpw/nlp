\documentclass{article}

\usepackage{microtype}
\usepackage{graphicx}
\usepackage{booktabs}
\usepackage{hyperref}
\usepackage{tcolorbox}

\newcommand{\theHalgorithm}{\arabic{algorithm}}

\usepackage[accepted]{icml2025}

\usepackage{amsmath}
\usepackage{amssymb}
\usepackage{mathtools}
\usepackage{amsthm}
\usepackage{subcaption}
\usepackage{tabularx}
\usepackage{dsfont}
\usepackage{enumitem}
\DeclareMathOperator*{\argmax}{arg\,max}
\usepackage[capitalize,noabbrev]{cleveref}

\newtcolorbox{highlight}[1][]{
    colback=yellow!10,
    colframe=gray!30,
    boxrule=0.5pt,
    arc=2pt,
    leftrule=3pt,
    rightrule=3pt,
    toprule=1pt,
    bottomrule=1pt,
    #1
}

\newcommand{\ourprm}{VersaPRM}
\newcommand{\ourdatatrain}{MMLU-Pro-CoT-Train (Labeled)}
\newcommand{\ourdataeval}{MMLU-Pro-CoT-Eval (Unlabeled)}


\icmltitlerunning{VersaPRM: Multi-Domain Process Reward Model via Synthetic Reasoning Data}

\begin{document}

\twocolumn[
\icmltitle{VersaPRM: Multi-Domain Process Reward Model via Synthetic Reasoning Data}

\icmlsetsymbol{equal}{*}

\begin{icmlauthorlist}
\icmlauthor{Thomas Zeng}{yyy}
\icmlauthor{Shuibai Zhang}{yyy}
\icmlauthor{Shutong Wu}{yyy}
\icmlauthor{Christian Classen}{yyy}
\icmlauthor{Daewon Chae}{zzz}
\icmlauthor{Ethan Ewer}{yyy}
\icmlauthor{Minjae Lee}{fff}
\icmlauthor{Heeju Kim}{fff}
\icmlauthor{Wonjun Kang}{fff}
\icmlauthor{Jackson Kunde}{yyy}
\icmlauthor{Ying Fan}{yyy}
\icmlauthor{Jungtaek Kim}{yyy}
\icmlauthor{Hyung Il Koo}{fff}
\icmlauthor{Kannan Ramchandran}{bbb}
\icmlauthor{Dimitris Papailiopoulos}{yyy}
\icmlauthor{Kangwook Lee}{yyy}

\end{icmlauthorlist}

\icmlaffiliation{yyy}{University of Wisconsin--Madison}
\icmlaffiliation{zzz}{Korea University}
\icmlaffiliation{fff}{FuriosaAI}
\icmlaffiliation{bbb}{University of California, Berkeley}

\icmlcorrespondingauthor{Kangwook Lee}{kangwook.lee@wisc.edu}


\icmlkeywords{Process Reward Models, Multi-Domain Process Reward Models, Synthetic Reasoning Data}

\vskip 0.3in
]


\printAffiliationsAndNotice{}


\begin{abstract}
Process Reward Models (PRMs) have proven effective at enhancing mathematical reasoning for Large Language Models (LLMs) by leveraging increased inference-time computation. However, they are predominantly trained on mathematical data and their generalizability to non-mathematical domains has not been rigorously studied. In response, this work first shows that current PRMs have poor performance in other domains. To address this limitation, we introduce \textbf{\emph{VersaPRM}}, a multi-domain PRM trained on synthetic reasoning data generated using our novel data generation and annotation method. VersaPRM achieves consistent performance gains across diverse domains. For instance, in the MMLU-Pro category of Law, VersaPRM via weighted majority voting, achieves a 7.9\% performance gain over the majority voting baseline---surpassing Qwen2.5-Math-PRM's gain of 1.3\%. We further contribute to the community by open-sourcing all data, code and models for VersaPRM.
\end{abstract}



\setlength{\belowdisplayskip}{1pt}
\setlength{\belowdisplayshortskip}{1pt}
\setlength{\abovedisplayskip}{1pt}
\setlength{\abovedisplayshortskip}{1pt}


\section{Introduction}
\documentclass[../main.tex]{subfiles}
\graphicspath{{../images/}}
\makeatletter
\def\input@path{{../images/}}
\makeatother
\begin{document}
\section{Introduction}
\begin{figure}
\centering
\begin{tikzpicture}
\node[inner sep=0pt] (ws) at (0, 0) {
\includegraphics[height=.4\textwidth, trim={10cm 0 10cm 0},clip]{world_space.png}};
\node[inner sep=0pt] (cs) at (6,0) {\includegraphics[height=.4\textwidth, trim={10cm 1cm 10cm 4cm},clip]{conf_space.png}};
\end{tikzpicture}
\vspace{-5pt}
\label{fig:pbrm_intro}
\caption{\textbf{Left}: Shows world space obstacles as grey spheres. Robots start and goal configuration is colored red and green, respectively. Configurations along the computed path are colored transparent blue. \textbf{Right:} Mapped world space scenario to configuration space. Obstacle region is the grey mesh. Red spheres are collision-free regions computed by the neural SCDF. The optimized shortest path in the convex corridor is the blue curve.}
\vspace{-25pt}
\end{figure}
Motion planning is the problem of finding a collision-free trajectory that connects a given start and goal configuration. The planning takes place in the configuration space of the robot. For single body robots, like mobile robots or drones, the configuration space and the world space are usually the same. This simplifies the planning, since explicit obstacle representations are available which enables geometrical tools like separating hyperplanes, smallest distance to obstacles etc., to be used when designing motion planning algorithms. For multi-body robots like manipulators, the situation is completely different. The world space obstacles are usually mapped to non-convex regions, and to make the problem even harder, the mapping is usually not known. Forming explicit representations of the obstacle region in the configuration space is usually too expensive or intractable. Despite all of this, sampling based planners are used with great success, which mainly is due to their use of implicit representations of the obstacle region. The basic idea is to construct a graph in the configuration space that covers and connects the collision-free region. From this graph, a path can be extracted that connects a given start and goal configuration. The approach is computationally expensive, since the graph is constructed with the smallest geometrical building block available, points, which represents a collision-check. Furthermore, the extracted paths from the graph are non-smooth and jagged due to the stochastic nature of the approach. This adds an additional post-processing step to the process, where the paths are shortcutted and smoothened, before the path can be used for tracking. Clearly a lot of time is invested to form this graph and produce smooth paths. Thus, if the obstacles start to move, then all of this work is done in no use, since all points that make up this graph need to be re-verified, which is simply too time consuming to be done in real time.
\\\\
In this work, we want to address the existing drawbacks of the sampling based planners. Our main contribution is an improved motion planner where each vertex in the graph covers a collision-free region in the form of a sphere instead of a point and where the edges are formed with neighboring intersecting spheres. This representation has the advantage of instead of returning piecewise linear paths, returning a sequence of overlapping spheres, i.e. a convex corridor, that connects a given start and goal configuration, illustrated in Figure \ref{fig:pbrm_intro}. This convex corridor allows us to use convex optimization to produce smooth trajectories, instead of computationally expensive post-processing methods. The representation further allows us to estimate the coverage of the collision-free space, which gives us awareness and feedback in the offline roadmap construction phase. Finally, our representation is simple to adapt to moving obstacles, simply requery for the new radii and recheck for intersections. 
\\\\
The spherical collision-free regions are formed using a signed distance function (SDF), which is a function that returns the smallest distance from an arbitrary point to the boundary of an obstacle. As the name implies, the distance is signed, thus if the point is inside the obstacle it is negative otherwise positive. If the distance is positive, a sphere with radius equal to the distance is guaranteed to cover a collision-free region. Using an SDF in motion planning is not new, but what is novel about our approach is that we express the distance in the configuration space instead of the world space and by doing so allows us to form these convex collision-free regions. We refer to the resulting SDF as a signed configuration distance function (SCDF). Computing an SCDF analytically is non-trivial, our approach is therefore to parameterize the SCDF with a deep neural network and learn the mapping by supervised learning. Our resulting neural SCDF can compute distances for different parameter values of obstacle shapes and we also show how multiple distances can be combined, thus making our approach flexible.
\section{Related work}
Motion planning algorithms can roughly be divided into three families, grid-based, sampling based and optimization based methods. Grid-based methods (GBM) discretize the planning space from which a graph is then compiled. A standard search method is A$^\star$ \citep{a_star}, which is classified as an \textit{informed} search method, since it employs a heuristic function to speed up the search. A$^\star$ guarantees to return an optimal path at the level of discretization used. GBMs usually discretize the planning space by a regular lattice and this limits the GBMs to problems with low dimensionality due to the curse of dimensionality. Thus, GBMs are usually limited to single-body robots where the degrees of freedom (DOF) are low. To overcome the inherent scaling problem with the GBMs, stochastic methods are usually used for multi-body robots. These methods are termed as sampling-based methods (SBM) and core members within this family are the rapidly-exploring random trees (RRT) \citep{rrt} and the probabilistic roadmap (PRM) \citep{prm}. RRT grows a tree from the start configuration and explores the collision-free region in a rapid way until it is able to connect to the goal region. RRT is usually improved by bi-directional planning \citep{rrt_connect}, i.e. an additional tree is grown from the goal configuration and the trees are tested for connection after any tree has been expanded. RRT is a single-query method, thus it searches for a path from scratch each time it is queried. Contrary to this, PRM is a multi-query method, which solves for multiple queries without starting from scratch. PRM does this by creating a roadmap (graph) that covers the collision-free space as an offline step. The graph is then used to solve for multiple queries. PRMs are used in cases where the environment does not change since the extra offline step is too computationally costly and needs to be re-done if the environment is changed. In our work, we address this inherent issue by using a different roadmap representation. Our vertices in the graph cover a collision-free region in the form of spheres and we form the edges by checking for intersecting spheres. If something in the environment changes, we recompute the spheres radii and recheck the intersections, without relying on collision detection. We use a trained neural network to compute the sphere radius, therefore querying for the radius can be done fast, hence our representation enables the PRM for dynamic environments.
\\\\
In the recent decades, optimization based methods (OBM) \citep{chomp, schulman, itomp, stomp} have been introduced as an alternative to SBM for multi-body robots. Like the SBM, the OBMs scale well to higher dimensional problems and produce smoother motion. It is common to use a SDF in the optimization since it is a smooth function, thus enabling gradient-based methods. However, the standard way of expressing the SDF is in world space. The distance therefore needs to be mapped to the configuration space by the forward kinematics. This mapping makes the optimization problem a non-linear program (NLP), which is computationally expensive to solve. Recently, a different approach has been proposed. In \cite{mp_gcs} motion planning is formulated as a convex optimization problem by using the graph of convex sets framework \citep{gcs}. The underlying idea is to decompose the collision-free space into intersecting convex sets from which a convex optimization problem is formulated. In cases where an explicit representation of the obstacles in the configuration space exists, like for single-body robots, creating collision-free convex regions can be done fast \citep{iris}. For multi-body robots, this is non-trivial. Existing work does this successfully \citep{iris_nlp, iris_c} by an optimization based approach, but the methods are still too time consuming to be used in the presence of moving obstacles. Our approach is instead to use deep learning to learn an SDF expressed in the configuration space. With this, we can query for shortest distances to the collision boundary, which allows us to expand spherical regions which are collision-free. Our approach is fast and therefore enables our suggested roadmap planner to be used in dynamic environments.
\\\\
Recent research has focused on learning collision detection \citep{fk_kernel_distance, diffco, graphdistnet} by predicting the signed distance between the robot links and the surrounding obstacles in the world space. The learned SDF is used in trajectory optimization but since the distance is expressed in the world space, the problem becomes an NLP and therefore takes a long time to solve. We take a novel approach and suggest to instead express the signed distance in the configuration space. This allows us to improve the PRM at the same time as it enables convex optimization for trajectory optimization, which runs faster and is more reliable than NLP solvers. In \cite{cspf} a learned signed distance function in the configuration space is proposed similar to our approach. However, their approach is restricted to point cloud representations, while we propose to represent the obstacles as parameterized geometric shapes, e.g. spheres. Furthermore, we also show how to use our learned SCDF to improve an existing roadmap planner.
\section{Problem formulation}
A robot is located in the world space, $\W \subset \R^3 $. The unique location of the robot is given by its configuration $\q \in \C$, where $\C$ is the configuration space. The set of points covered by the robots bodies at a certain configuration is expressed as $\B(\q) \subset \W$. The robot is surrounded by $\NrObst$ obstacles $\O = \bigcup_{i=1}^{\NrObst} \O_i$, where  $\O_i \subset \W$. The representation of the obstacle in the configuration space is the set $\C\O_i = \{\q \in \C \: |\: \B(\q) \cap \O_i \neq \emptyset \}$. The obstacle space is formed as $\Co = \bigcup_{i=1}^{\NrObst} \C \O_i$. The complement is referred to as the free space, $\Cf = \C \setminus \Co$. The path planning problem is a tuple, ($\Cf$, $\qStart$, $\qGoal$), where we want to connect a query pair, consisting of a start, $\qStart$, and goal configuration, $\qGoal$, with a geometric path, $\q(s): [0, 1] \mapsto \Cf$, such that $\q(0)=\qStart$ and $\q(1)=\qGoal$, or report correctly when such a path does not exist.
\end{document}


\section{Related Work}


\subsection{Plasticity in Neural Networks}
In recent years, various methods have been proposed to address plasticity loss.
Several works have focused on maintaining active units \cite{abbas2023loss, elsayed2024addressing} or re-initializing dead units \cite{sokar2023dormant, dohare2024loss}.
Other studies have explored limiting deviations from the initial statistics of model parameters \cite{kumar2023maintaining, lewandowski2023curvature, elsayed2024weight}.
Additionally, some methods rely on architectural modifications \cite{nikishin2024deep, lee2024slow, lewandowski2024plastic}.  
Plasticity loss also occurs in the reinforcement learning due to its inherent non-stationary. \citet{nikishin2022primacy} proposed resetting the model, while \citet{asadi2024resetting} suggested resetting the optimizer state. 

As noted by \citet{berariu2021study}, loss of plasticity can be divided into two distinct aspects: a decreased ability of networks to minimize training loss on new data (trainability) and a decreased ability to generalize to unseen data (generalizability).
While most previous works focused on trainability, \citet{lee2024slow} addressed generalizability loss.
They demonstrated that plasticity loss also occurs under a stationary distribution, as in a warm-start learning scenario where the model is pretrained on a subset of the training data and then fine-tuned on the full dataset.

Most existing studies have focused on only one of the following challenges: trainability, generalizability, or reinforcement learning.
However, in this study, we validate our AID method across all three aspects, demonstrating its effectiveness in each scenario.



\subsection{Activation Function}
Our AID method is a stochastic approach similar to Dropout while also functioning as an activation function.
Therefore, we aim to discuss previously proposed probabilistic activation functions.
Although the field of probabilistic activation functions has not seen extensive research, two noteworthy studies exist.
The first is the Randomized ReLU (RReLU) function, introduced in the Kaggle NDSB Competition \cite{xu2015empirical}.
The original ReLU function maps all negative values to zero, whereas RReLU maps negative values linearly based on a random slope.
During testing, negative values are mapped using the mean of the slope distribution.
Their experimental results suggest that RReLU effectively prevents overfitting.
Another example of a probabilistic activation function is DropReLU \cite{liang2021drop}.
DropReLU randomly determines whether a node's activation is processed through a ReLU function or a linear function.
The authors claim that DropReLU improves the generalization performance of neural networks.
The fundamental distinction between these probabilistic activation functions and our method lies in the generality of our approach.
Unlike simple probabilistic activation functions, our method encompasses techniques such as Dropout and ReLU, providing a more comprehensive framework.

Another related approach involves activation functions designed to address plasticity loss.
\citep{abbas2023loss} proposed the Concatenated Rectified Linear Units (CReLU), which concatenates the outputs of the standard ReLU applied to the input and its negation.
This structure prevents the occurrence of dead units, thereby improving plasticity.
Additionally, trainable activation functions have also been shown to effectively mitigate plasticity loss in reinforcement learning \citep{delfosseadaptive}.
Specifically, they introduced a trainable rational activation function that prevents value overfitting and overestimation in reinforcement learning.



\begin{figure*}[ht!]
    \centering
    \includegraphics[width=0.3\textwidth]{figures/sources/mainnet_pls_acc.pdf}
    \includegraphics[width=0.3\textwidth]{figures/sources/subnet_pls_acc.pdf}
    \includegraphics[width=0.3\textwidth]{figures/sources/warm_start_dropout.pdf}
    \caption{\textbf{Left.} Random label MNIST experiment using an 8-layer MLP. Higher dropout probabilities result in significant trainability loss. 
    \textbf{Middle.} Accuracy of the subnetworks trained on random target. Each subnetworks are sampled from original network after each epoch. Subnetworks of the Dropout also experience trainability loss. \textbf{Right.} Warm-start scenario of Resnet-18 model with CIFAR100 dataset. Dropout improves generalization performance; however, the reduction in accuracy compared to the cold-start scenario is nearly identical to that of the vanilla model.}
    \label{exp_dropout}
\end{figure*}





\section{Process Reward Models}

\section{Preliminaries}\label{sec:preliminaries}



%We denote by $(\Ac(x_\Ac),\Bc(x_\Bc))(z)$ a random execution of $\pi$ with private inputs $(x_\Ac,y_\Ac)$, and common input $z$.

%\Jnote{Move to DP}
% At the end of such an execution, the protocol outputs a public transcript denoted by the random variable $\trans_\pi(x_\Ac,x_\Ac,z)$ we denotes the common as $\out(\trans_\pi(x_\Ac,x_\Ac,z)$, and each party $\Pc \in \set{\Ac,\Bc}$ obtains his view denoted $\view^\Pc_\pi(x_\Ac,x_\Bc,z)$, which may also contain a ``local output'' \Jnote{Local} $\out^\Pc(x_\Ac,x_\Bc,z)$ (if the protocol specifies such an output). \Jnote{Common output, and parties output}


\subsection{Distributions and Random Variables}\label{sec:prelim:dist}
The support of a distribution $P$ over a finite set $\cS$ is defined by $\Supp(P) \eqdef \set{x\in \cS: P(x)>0}$. For a distribution or a random variable $D$, let $d\from D$ denote that $d$ was sampled according to $D$. Similarly,  for a set $\cS$, let $x \from \cS$ denote that $x$ is drawn uniformly from $\cS$, and denote by $\cU_{\cS}$ the uniform distribution over $\cS$. For a finite set $\cX$ and a distribution $C_X$ over $\cX$, we use the capital letter $X$ to denote the random variable that takes values in $\cX$ and is sampled according to $C_X$. The {\sf statistical distance} (\aka {\sf~variation distance}) of two distributions $P$ and $Q$ over a discrete domain $\cX$ is defined by $\sdist{P}{Q} \eqdef \max_{\cS\subseteq \cX} \size{P(\cS)-Q(\cS)} = \frac{1}{2} \sum_{x \in \cS}\size{P(x)-Q(x)}$. 
For a vector $x = (x_1,\ldots,x_n)$ and index $i\in [n]$, we let $x_{-i} = (x_1,\ldots,x_{i-1},x_{i+1},\ldots,x_n)$ and $x^{(i)} = (x_1,\ldots,x_{i-1}, -x_i, x_{i+1},\ldots,x_n)$, for a set $\cS \subseteq [n]$ we let $x_{\cS} = (x_i)_{i \in \cS}$ and $x_{-\cS} = (x_i)_{i \in [n]\setminus \cS}$, and for a vector $r \in \zo^n$ we let $x_r = (x_i)_{\set{i \colon r_i = 1}}$ and $x_{-r} = (x_i)_{\set{i \colon r_i = 0}}$.

%For $n \in \N$ we let $U_n$ be the uniform distribution over $\oo^n$, and let $S_n$ be the distribution induces by the sum of $n$ i.i.d.\ random variables, each is distributed according to $U_1$. Let $\cN(0,1)$ be the standard normal distribution.
%For a distribution $\cD$ and a function $f$, we define by $f(\cD)$ the distribution that is induced by the output of $f(x)$ for $x \from \cD$. 





% \begin{theorem}[\cite{McGregorMPRTV10}]\label{thm:sv-extracotr}
% 	\Enote{Remove if not needed}
% 	There is a constant $c$ to make the following holds. Let $X$ be an $\alpha$-SV source on $\{0,1\}^n$, let $Y$ be a source on $\{0,1\}^n$ with min-entropy at least $\beta n$ (independent from $X$), and let $Z=\ip{X,Y}\mbox{mod m}$ for some $m\in\mathbb{N}$. Then for every $\delta\in[0,1]$, the random variable $(Y,Z)$ is $\delta$-close to $(Y,U)$ where $U$ is uniform on $\mathbb{Z}_m$ and independent of $Y$, provided that
% 	$$
% 	n\geq c\cdot\frac{m^2}{\alpha\beta}\cdot\log(\frac{m}{\beta})\cdot\log(\frac{m}{\delta}).
% 	$$
% \end{theorem}



\Enote{I removed the definition of DP since it already appears in the intro}
\remove{
\subsection{Differential Privacy}\label{sec:prelim:DP}
We use the following standard definition of (information theoretic) differential privacy, due to \citet{DMNS06}. For notational convenience, we focus on databases over $\oo$.
\begin{definition}[Differentially private mechanisms]\label{def:mech}
	A randomized function $f\colon\oo^n\mapsto \zs$ is an {\sf $n$-size, $(\eps,\delta)$-differentially private mechanism} (denoted $(\eps,\delta)$-\DP) if for every neighboring $w,w'\in \oo^n$ and every function $g\colon \zs\mapsto \zo$, it holds that 
	$$
	\pr{g(f(w))=1}\leq e^{\eps}\cdot \pr{g(f(w'))=1} +\delta.
	$$ 	
	If $\delta=0$, we omit it from the notation.
\end{definition}
}


\subsubsection{Computational Differential Privacy}
There are several ways for defining computational differential privacy (see \cref{sec:related-works}). We use the most relaxed version due to \cite{BNO08}. For notational convenience, we focus on databases over $\oo$.
\begin{definition}[Computational differentially private mechanisms]\label{def:ComMech}
	A randomized function ensemble $f=\set{f_\pk\colon\oo^{n(\pk)}\mapsto \zs}$ is an {\sf $n$-size, $(\eps,\delta)$-computationally differentially private} (denoted $(\eps,\delta)$-$\CDP$) if for every poly-size circuit family $\set{\Ac_\pk}_{\pk\in \N}$, the following holds for every large enough $\pk$ and every neighboring $w,w'\in\oo^{n(\pk)}$:
	$$
	\pr{\Ac_\pk(f_\pk(w))=1}\leq e^{\eps(\pk)}\cdot \pr{\Ac_\pk(f_\pk(w'))=1} +\delta(\pk).
	$$ 
	If $\delta(\pk) = \negl(\pk)$, we omit it from the notation. 
\end{definition}



\subsubsection{Two-Party Differential Privacy}\label{sec:DP}
In this section we formally define distributed differential privacy mechanism (\ie protocols). %For the ease of notation, we consider protocol with no common input.

\begin{definition}\label{def:DP}%\Nnote{fix security parameter}
	A two-party protocol $\Pi=(\Ac,\Bc)$ is {\sf $(\eps,\delta)$-differentially private}, denoted $(\eps,\delta)$-$\DP$, if the following holds for every algorithm $\Dc$: let $\V^\Pc(x,y)(\pk)$ be the view of party $\Pc$ in a random execution of $\Pi(x,y)(1^\pk)$. Then for every $\pk,n \in \N$, $x\in \oo^n$ and neighboring $y,y'\in\oo^n$:
	\begin{align*}
	\pr{\Dc(V^\Ac(x,y)(\pk))=1}\le e^{\eps(\pk)}\cdot \pr{\Dc(V^\Ac (x,y')(\pk))=1}+\delta(\pk),
	\end{align*} 
	and for every $y\in \oo^n$ and neighboring $x,x'\in\oo^{n}$:
	\begin{align*}
	\pr{\Dc(V^\Bc(x,y)(\pk))=1}\le e^{\eps(\pk)}\cdot \pr{\Dc(V^\Bc (x',y)(\pk))=1}+\delta(\pk).
	\end{align*} 	
	Protocol $\Pi$ is {\sf $(\eps,\delta)$-computational differentially private}, denoted $(\eps,\delta)$-$\CDP$, if the above inequalities only hold for a non-uniform \ppt $\Dc$ and large enough $\pk$. We omit $\delta = \negl(\pk)$ from the notation. \footnote{Note that define we give for two-party differentially private protocols is a semi-honest definition, in which we ask for the security to hold when the parties interact in an honest execution of the protocol. Since we are proving a lower bound, starting from this weaker guarantee (as opposed to security against malicious players), yields a stronger result.}
\end{definition}
%We omit $\delta$ from the notation if $\delta$ is a negligible function of $n$.

%\Enote{simulation-based}
\begin{remark}[The definition for computational differential privacy we use]\label{rem:comDPChannel} 
	An alternative, stronger definition of computational differential privacy, known as simulation-based computational differential privacy, requires that the distribution of each party’s view be computationally indistinguishable from a distribution that ensures privacy in an information-theoretic sense. \cref{def:DP} is a weaker notion in comparison. Consequently, establishing a lower bound for a protocol that satisfies this weaker guarantee (as we do in this work) yields a stronger result.%Actually, our lower bound only requires the privacy to hold against \emph{uniform} external observer.
	%\Nnote{Maybe add: When only interesting in \Dp against external observer, the two definitions can be achieve using key-agreement and (single-party) \Dp mechanism. }
\end{remark}




\subsection{Useful Claims}
\remove{
In this section, we state generic lemmas and propositions that we will use later in our proofs.

The following lemma which we prove in \cref{sec:missing-proofs:distance-I}, measures the distance between two uniform stings conditioned one a random index $i$ either being fixed to $0$ or to $1$.

\def\distanceILemma{
    Let $R \la \zo^n$. For any (randomized) function $f:\{0,1\}^n\rightarrow \{0,1\}$ and $\alpha > 0$, it holds that
    \begin{align}\label{eq:f-alpha}
        \ppr{i \la [n]}{\size{\:\ex{f(R) \mid R_i = 0}-\ex{f(R) \mid R_i = 1}\:}\geq \alpha} \leq \frac{2}{n \alpha^2},
    \end{align}
    where the expectations are taken over $R$ and the randomness of $f$.
}

\begin{lemma}\label{lem:distance-I}
    \distanceILemma
\end{lemma}
}

The following two propositions state that given the output of a differentially private function, it is not possible to predict well even a random index (even if all other indexes are leaked). The first proposition handles the information-theoretic case and the second handles the computation case. Both propositions are proven in \cref{sec:missing-proofs:hard-to-guess}. 

\def\propHardToGuessInf{
    Let $f\colon \oo^n \rightarrow \cY$ be an $(\eps,\delta)$-\DP function, let $g \colon [n] \times \oo^{n-1} \times \cY \rightarrow \set{-1,1,\bot}$ be a (randomized) function, and let $X = (X_1,\ldots,X_n) \la \oo^n$. Then the following holds for every $i \in [n]$ where $X_i^* = g(i,X_{-i},f(X_1,\ldots,X_n))$:
    \begin{align*}
        \pr{X_i^* = X_i} \leq e^{\eps}\cdot \pr{X_i^* = -X_i} + \delta.
    \end{align*}
}

\begin{proposition}\label{prop:hard-to-guess-inf}
    \propHardToGuessInf
\end{proposition}


\def\propHardToGuessComp{
    Let $f = \set{f_{\pk} \colon \oo^{n(\pk)} \rightarrow \zo^{m(\pk)}}_{\pk \in \bbN}$ be an $(\eps,\delta)$-\CDP function ensemble, and let $\set{g_{\pk}}_{\pk \in \bbN}$ be a poly-size circuit family. Then, for large enough $\pk$ and $X = (X_1,\ldots,X_{n(\pk)}) \la \oo^{n(\pk)}$, the following holds for every $i \in [n(\pk)]$ where $X_i^* = g_{\pk}(i,X_{-i},f_{\pk}(X_1,\ldots,X_n))$:
    \begin{align*}
        \pr{X_i^* = X_i} \leq e^{\eps(\pk)}\cdot \pr{X_i^* = -X_i} + \delta(\pk).
    \end{align*}
}

\begin{proposition}\label{prop:hard-to-guess-comp}
    \propHardToGuessComp
\end{proposition}





\remove{
\Enote{Chao's old statement:}
\begin{lemma}\label{lem:distance-I-old}
        Let $R \la \zo^n$. 
	For any function $f:\{0,1\}^n\rightarrow \{0,1\}$ and $\alpha<0.01$, it holds that
	$$
	\Pr_{i\la[n]}\left[\: \size{\:\mathbb{E}[f(R) \mid R_i = 0]-\mathbb{E}[f(R) \mid R_i = 1]\:}\geq \alpha\right]\leq \frac{2+2\log(\frac{1}{\alpha})}{n\alpha^2}.
	$$
\end{lemma}
\begin{proof}
	Define $S_1=\{r \in \zo^n \colon f(r)=1\}$. Then for any $i\in[n]$, we have
	$$
	\begin{array}{rl}
		\size{\mathbb{E}[f(R) \mid R_i = 0]-\mathbb{E}[f(R) \mid R_i = 1]}
		&=\size{\Pr[R\in S_1|R_i=0]-\Pr[R\in S_1|R_i=1]}\\
		&=\size{\frac{\Pr[R_i=0|R\in S_1]\cdot\Pr[R\in S_1]}{\Pr[R_i=0]}-\frac{\Pr[R_i=1|R\in S_1]\cdot\Pr[R\in S_1]}{\Pr[R_i=1]}}\\
		&=\frac{2\size{S_1}}{2^n}\size{\Pr[R_i=0|R\in S_1]-\Pr[R_i=1|R\in S_1]}
	\end{array}
	$$
	When $|S_1|\leq \alpha\cdot 2^{n-1}$, we have $\size{\mathbb{E}[f(R) \mid R_i = 0]-\mathbb{E}[f(R) \mid R_i = 1]}\leq\frac{2\size{S_1}}{2^n}\leq \alpha$ for any $i\in[n]$. Hence, in the following, we assume $|S_1|> \alpha\cdot 2^{n-1}$.

	%Define $I_{bad}=\{i|\size{\Pr[R_i=0|R\in S_1]-\Pr[R_i=1|R\in S_1]}>2\alpha\}$ and $k=\size{I_{bad}}$, then for any $i\notin I_{bad}$, we have 
    %$$
    %\begin{array}{rl}
    %    2\alpha&\geq \size{\Pr[R_i=0|R\in S_1]-\Pr[R_i=1|R\in S_1]}\\
    %    &=\size{\frac{\Pr[R\in S_1|R_i=0]\cdot\Pr[R_i=0]}{\Pr[R\in S_1]}-\frac{\Pr[R\in S_1|R_i=1]\cdot\Pr[R_i=1]}{\Pr[R\in S_1]}}\\
    %    &=\size{\Pr[R\in S_1|R_i=0]-\Pr[R\in S_1|R_i=1]}\cdot\frac{1}{2\Pr[R\in S_1]}\\
    %    &\geq \size{\mathbb{E}[f(R) \mid R_i = 0]-\mathbb{E}[f(R) \mid R_i = 1]}\cdot \frac{1}{2},
    %\end{array}
    %$$ 
    %where the last inequality is because $\Pr[R\in S_1]\leq 1$. So that $\size{\mathbb{E}}[f(R) \mid R_i = 0]-\mathbb{E}[f(R) \mid R_i = 1]\leq %4\alpha$.
    Define $I_{bad}=\{i \colon \size{\Pr[R_i=0|R\in S_1]-\Pr[R_i=1|R\in S_1]} \geq 2\alpha\}$ and $k=\size{I_{bad}}$, and denote $I_{bad}=\{i_1,\dots,i_k\}$. Define $(X_{i_1}, \ldots X_{i_k}) = (R_{i_1},\dots,R_{i_k})\mid_{R \in S_1}$. 
    Consider the min-entropy
	$$
	\begin{array}{rl}
		H_{min}(X_{i_1},\dots,X_{i_k})&\leq H(X_{i_1},\dots,X_{i_k})\\
		&\leq \sum_{j=1}^k H(X_{i_j})\\
		&\leq k\cdot \left(-(\frac{1}{2}+2\alpha)\cdot\log(\frac{1}{2}+2\alpha)-(\frac{1}{2}-2\alpha)\cdot\log(\frac{1}{2}-2\alpha)\right)\\
            &=k\cdot \left(-(\frac{1}{2}+2\alpha)\cdot(\log(1+4\alpha)-1)-(\frac{1}{2}-2\alpha)\cdot(\log(1-4\alpha)-1)\right)\\
            &=k\cdot \left(1-(\frac{1}{2}+2\alpha)\cdot\log(1+4\alpha)-(\frac{1}{2}-2\alpha)\cdot\log(1-4\alpha)\right),
		
	\end{array}
	$$
	where $H_{min}(Y)$ is the minimum entropy of $Y$ and $H(Y)$ is the Shannon entropy of $Y$.\Enote{add to preliminaries.}
        The third inequality holds since by the definition of $I_{bad}$, for every $j \in [k]$ it holds that $\size{\pr{X_{i_j} = 1}-\pr{X_{i_j} = 0}} > 2\alpha$, and therefore $H(X_{i_j}) \leq H(1/2 + 2\alpha)$\Enote{define}.
	
	Therefore, there exists $b_1,\dots,b_k\in\{0,1\}$, such that 
	
	\begin{align}\label{eq:min-entropy-result}
		\Pr\left[(R_{i_1},\ldots,R_{i_k}) = (b_1,\ldots,b_k) \mid R\in S_1\right]
		&= \pr{(X_{i_1},\ldots,X_{i_k}) = (b_1,\ldots,b_k)}\\
		&= 2^{-H_{min}(X_{i_1},\dots,X_{i_k})}\nonumber\\
		&\geq 2^{k\cdot \left(-1+(\frac{1}{2}+2\alpha)\cdot\log(1+4\alpha)+(\frac{1}{2}-2\alpha)\cdot\log(1-4\alpha)\right)}.\nonumber
	\end{align}
	
	Let $S_{bad}=\{r \in \zo^n  \colon \set{(r_{i_1},\ldots,r_{i_k}) = (b_1,\ldots,b_k)} \land \set{r\in S_1}\}$.
	It holds that
	\begin{align*}
		|S_{bad}|
		&= \size{S_1} \cdot \Pr\left[(R_{i_1},\ldots,R_{i_k}) = (b_1,\ldots,b_k) \mid R\in S_1\right]\\
		&\geq \alpha\cdot 2^{n-1}\cdot2^{k\cdot \left(-1+(\frac{1}{2}+2\alpha)\cdot\log(1+4\alpha)+(\frac{1}{2}-2\alpha)\cdot\log(1-4\alpha)\right)},
	\end{align*} 
	where the inequality holds by \cref{eq:min-entropy-result} and since $\size{S_1} \geq \alpha\cdot 2^{n-1}$.
	Notice that any string in $S_{bad}$ depends on at most $n-k$ bits. It implies that $|S_{bad}|\leq 2^{n-k}$. Therefore, we have
	$$
	\begin{array}{rl}
		&2^{n-k}\geq \alpha\cdot 2^{n-1}\cdot2^{k\cdot \left(-1+(\frac{1}{2}+2\alpha)\cdot\log(1+4\alpha)+(\frac{1}{2}-2\alpha)\cdot\log(1-4\alpha)\right)} \\
		\Rightarrow& n-k \geq \log \alpha+n-1+k\cdot \left(-1+(\frac{1}{2}+2\alpha)\cdot\log(1+4\alpha)+(\frac{1}{2}-2\alpha)\cdot\log(1-4\alpha)\right)\\
		\Rightarrow& 1-\log \alpha \geq k\cdot((\frac{1}{2}+2\alpha)\cdot\log(1+4\alpha)+(\frac{1}{2}-2\alpha)\cdot\log(1-4\alpha))\\
		\Rightarrow& 1-\log \alpha \geq k\cdot(4\alpha\cdot\log(1+4\alpha)+(\frac{1}{2}-2\alpha)\cdot\log(1-16\alpha^2))\\
        \Rightarrow& 1-\log\alpha \geq k\cdot(15.9\alpha^2-8\alpha^2+32\alpha^3)=k\cdot(7.9\alpha^2+32\alpha^3)>0.5k\alpha^2\\
		\Rightarrow& k\leq \frac{2-2\log \alpha}{\alpha^2} = \frac{2+2\log (1/\alpha)}{\alpha^2},
	\end{array}
	$$
	Where the third transition holds since 
	\begin{align*}
		\lefteqn{(\frac{1}{2}+2\alpha)\cdot\log(1+4\alpha)+(\frac{1}{2}-2\alpha)\cdot\log(1-4\alpha)}\\
		&= 4\alpha\cdot\log(1+4\alpha) + (\frac{1}{2}-2\alpha)\paren{\log(1+4\alpha)+\log(1-4\alpha)}\\
		&= 4\alpha\cdot\log(1+4\alpha)+(\frac{1}{2}-2\alpha)\cdot\log(1-16\alpha^2),
	\end{align*}
	and the forth transition holds since $4\alpha\cdot\log(1+4\alpha)+(\frac{1}{2}-2\alpha)\cdot\log(1-16\alpha^2) > 15.9\alpha^2-8\alpha^2+32\alpha^3$ for $\alpha < 0.01$.
	Thus, we conclude that 
	$$
	\Pr_{i\la[n]}\left[\size{\mathbb{E}[f(R) \mid R_i=0]-\mathbb{E}[f(R) \mid R_i = 1]}\geq \alpha\right]\leq \frac{k}{n}\leq \frac{2+2\log (1/\alpha)}{n\alpha^2}.
	$$
\end{proof}
}


\subsection{Channels and Two-Party Protocols}\label{sec:protocol}

\paragraph{Channels.}A channel is simply a distribution of a pair of tuples defined as follows. 
\begin{definition}[Channels]\label{def:channel} A {\sf channel} $C_{(X,U)(Y,V)}$ of size $\isize$ over alphabet $\Sigma$ is a probability distribution over $(\Sigma^\isize \times\zo^\ast) \times(\Sigma^\isize \times\zo^\ast)$. The ensemble $C_{(X,U)(Y,V)}= \set{C_{(X_\pk,U_\pk)(Y_\pk,V_\pk)}}_{\pk\in \N}$ is an $\isize$-size channel ensemble, if for every $\pk\in \N$, $C_{(X_\pk,U_\pk)(Y_\pk,V_\pk)}$ is an $\isize(\pk)$-size channel. %We denote a channel of size one by a \emph{single-bit} channel. 
We refer to $X$ and $Y$ as the {\sf local outputs}, and to $U$ and $V$ as the {\sf views}.	
\end{definition}

We view a  channel as the experiment in which there are two parties $\Ac$ and $\Bc$.  Party $\Ac$ receives ``output'' $X$ and ``view'' $U$, and party $\Bc$ receives ``output'' $Y$ and ``view'' $V$. Unless stated otherwise, the channels we consider are over the alphabet $\Sigma = \oo$. We naturally identify channels with the distribution that characterizes their output.








\subsubsection{Two-Party Protocols}

A two-party protocol $\Pi=(\Ac,\Bc)$ is \ppt if the running time of both parties is polynomial in their input length. We let $\Pi(x,y)(z)$ or $(\Ac(x),\Bc(y))(z)$ denote a random execution of $\Pi$ on a common input $z$, and private inputs $x,y$.%We assume \wlg that a protocol has a common output (part of its transcript).\Jnote{This is not really the case we consider in this paper..}

\begin{definition}[Oracle-aided protocols]\label{def:ChannelAidedProtocol}
	In a two-party protocol $\Pi$ with oracle access to a {\sf protocol} $\Psi$, denoted $\Pi^\Psi$, the parties make use of the \textit{next-message function} of $\Psi$.\footnote{The function that on a partial view of one of the parties, returns its next message.} In a two-party protocol $\Pi$ with oracle access to a {\sf channel} $C_{Z W}$, denoted $\Pi^C$, the parties can jointly invoke $C$ for several times. In each call, an independent pair $(z,w)$ is sampled according to $C_{Z W}$, one party gets $z$, the other gets $w$.
\end{definition}


\begin{definition}[The channel of a protocol]\label{def:ChannlOfProtocol}
	For a no-input two-party protocol $\Pi= (\Ac,\Bc)$, we associate the channel $C_\Pi$, defined by $\C_\Pi= C_{(X, U),(Y, V)}$, where $X$ and $Y$ are the local outputs of $\Ac$ and $\Bc$ (respectively) and
	$U$ and $V$ are the local views of $\Ac$ and $\Bc$ (respectively).
    
	For a two-party protocol $\Pi$ that gets a security parameter $1^\pk$ as its (only, common) input, we associate the channel ensemble $ \set{C_{\Pi(1^\pk)}}_{\pk\in \N}$. 
\end{definition}

\begin{definition}[$(\alpha,\gamma)$-Accurate channel]\label{def:accurate-func}
	A channel $C = C_{(X, U),(Y, V)}$ is {\sf $(\alpha,\gamma)$-accurate for the function $f$}, if $\ppr{C}{\size{\out(V)-f(X,Y)}\leq \alpha}\ge \gamma$, where $\out(V)$ is the designated output.
    A channel ensemble $C_{(X, U),(Y, V)}= \set{C_{(X_\pk, U_\pk),(Y_\pk, V_\pk)}}_{\pk\in \N}$ is  $(\alpha,\gamma)$-accurate for  $f$ if $C_{(X_\pk, U_\pk),(Y_\pk, V_\pk)}$ is $(\alpha(\pk),\gamma(\pk))$-accurate for $f$, for every $\pk \in \N$.
\end{definition}

\subsubsection{Differentially Private Channels}\label{sec:DPChannel}
Differentially private channels are naturally defined as follows:
\begin{definition}[Differentially private channels]\label{def:DPChannel}
	An $n$-size channel $C = C_{(X, U),(Y, V)}$ with $X, Y$ over $\oo^n$ 
	is {\sf$(\eps,\delta)$-differentially private} (denoted $(\eps,\delta)$-$\DP$) if for every $x \in \Supp(X)$ there exists an $n$-size $(\eps,\delta)$-$\DP$ mechanisms $\Mc_x$ such that $(X,Y,U) \equiv (X,Y,\Mc_X(Y))$, and for every $y \in \Supp(Y)$ there exists an $n$-size $(\eps,\delta)$-$\DP$ mechanisms $\Mc_y'$ such that $(X,Y,V) \equiv (X,Y,\Mc_Y'(X))$. In addition, we say that the channel is \emph{uniform} if $X$ and $Y$ are independent random variables uniformly distributed in $\oo^n$. 
\end{definition}

\begin{definition}[Computational differentially private channels]\label{def:CDPChannel}
	An $n$-size channel ensemble $C = \set{C_{(X_\pk, U_\pk),(Y_\pk, V_\pk)}}_{\pk\in\N}$ with $X_\pk, Y_\pk$ over $\oo^n$ 
	is {\sf$(\eps,\delta)$-computationally differentially private} (denoted $(\eps,\delta)$-$\CDP$) if for every ensemble $\set{x_\pk \in \Supp(X_\pk)}_{\pk\in\N}$ there exists an $n$-size $(\eps,\delta)$-\CDP mechanisms ensemble $\set{\Mc_{x_\pk}}_{\pk\in\N}$ such that $(X_\pk,Y_\pk,U_\pk) \equiv (X_\pk,Y_\pk,\Mc_{X_\pk}(Y_\pk))$, for every $\pk\in\N$, and for every ensemble $\set{y_\pk \in \Supp(Y_\pk)}_{\pk\in\N}$ there exists an $n$-size $(\eps,\delta)$-$\CDP$ mechanisms ensemble $\set{\Mc'_{y_\pk}}_{\pk\in\N}$ such that $(X_\pk,Y_\pk,V_\pk) \equiv (X_\pk,Y_\pk,\Mc_{Y_\pk}'(X_\pk))$ for every $\pk\in \N$. In addition, we say that the channel is \emph{uniform} if $X_\pk$ and $Y_\pk$ are independent random variables uniformly distributed in $\{\pm 1\}^n$ for all $\pk\in\N$.
\end{definition}




% \begin{lemma}~\label{lem:dp-sv-source}
% 	Let $P$ be an $\varepsilon$-DP randomized protocol. Let $X$ and $Y$ be independent random variables uniformly distributed in $\{\pm 1\}^n$ and let random variable $\Pi(X,Y)$ denote the transcript of running $P(X,y)$. Then for every $\pi\in Supp(\Pi)$, the random variables corresponding to the inputs conditioned on transcript $\pi$, $X_\pi$ and $Y_\pi$, are independent $e^{-\varepsilon}$-strong SV source.
% \end{lemma}





\subsubsection{Weak Erasure Channel (\WEC)}

\begin{definition}[\WEC]\label{def:WEC}
	A channel $((O_A,V_A), (O_B,V_B))$ with $O_A \in \set{0,1}$ and $O_B \in \set{0,1,\bot}$ is a {\sf weak erasure channel}, denoted $(\alpha,p,q)$-$\WEC$, if:
	\begin{itemize}
		%\item $O_A\in \set{-1,1}$ and $O_B\in \set{-1,1,\bot}$.
		\item Random erasure: $\pr{O_B = \perp} = 1/2$.
		
		\item Agreement: $\pr{O_A\ne O_B\mid O_B\ne \bot}\le \alpha$.
		
		\item Secrecy:
		
		\begin{enumerate}
			\item For every algorithm $\Dc$ it holds that\label{WEC:item:A}
			\begin{align*}
				%\size{\pr{\Ac(O_A,V_A) = 1 \mid O_B \neq \perp} - \pr{\Ac(O_A,V_A) = 1 \mid O_B = \perp}} \le p
				\size{\pr{\Dc(V_A) = 1 \mid O_B \neq \perp} - \pr{\Dc(V_A) = 1 \mid O_B = \perp}} \le p
			\end{align*}
			(Alice doesn't know if $O_B = \perp$.)
			
			\item For every algorithm $\Dc$ it holds that\label{WEC:item:B}
			\begin{align*}
				\pr{\Dc(V_B) = O_A \mid O_B=\bot} \leq \frac{1+q}{2}.
			\end{align*}
			(i.e., if $O_B=\bot$, Bob don't know what is the value of $O_A$).
			
			%\item $SD((O_A U|O_B=\bot),(O_A U|O_B\ne \bot))\le p$ (The sender don't know if $O_B=\bot$).
			
			%\item $SD(V O_A|O_B=\bot,V(-O_A)|O_B=\bot)\le q$ (If $O_B=\bot$, Bob don't know what the value of $O_A$).
		\end{enumerate}
	\end{itemize}
   We say that a channel ensemble $C=\set{C_\pk}_{\pk\in N}$ is a {\sf computational weak erasure channel}, denoted $(\alpha,p,q)$-\CompWEC, if for every \ppt algorithm $\Dc$ and every sufficiently large $\pk\in\N$, $C_\pk$ satisfies the properties stated in the items above, where the secrecy property holds with respect to a \ppt algorithm $\Dc$. A protocol $\Lambda$ is said to be $(\alpha,p,q)$-$\CompWEC$, if the ensemble induces by the protocol (that is, $C=\set{C_{\Lambda(\pk)}}_{\pk\in\N}$) is $(\alpha,p,q)$-$\CompWEC$.  
\end{definition}



\subsubsection{Approximate Weak Erasure Channel (\AWEC)}\label{sec:AWEC}

\begin{definition}[\AWEC]\label{def:AWEC}
	A channel $C = ((O_A,V_A), (O_B,V_B))$ over $([-n,n] \times \zo^*) \times (([-n,n] \cup \bot)  \times \zo^*)$ is an {\sf approximate weak erasure channel}, denoted $(\ell,\alpha,p,q)$-\AWEC if:
	\begin{itemize}
		
		\item Random erasure: $\pr{O_B = \perp} = 1/2$.
		
		\item Accuracy: $\pr{\size{O_A - O_B} > \ell \mid O_B \ne \bot}\le \alpha$.
		
		\item Secrecy:
		
		\begin{enumerate}
			\item For every algorithm $\Dc$ it holds that\label{AWEC:item:A}
			\begin{align*}
				%\size{\pr{\Ac(O_A,V_A) = 1 \mid O_B \neq \perp} - \pr{\Ac(O_A,V_A) = 1 \mid O_B = \perp}} \le p
				\size{\pr{\Dc(V_A) = 1 \mid O_B \neq \perp} - \pr{\Dc(V_A) = 1 \mid O_B = \perp}} \le p
			\end{align*}
			(Alice doesn't know if $O_B=\bot$).
			
			\item For every algorithm $\Dc$ it holds that\label{AWEC:item:B}
			\begin{align*}
				\pr{\size{\Dc(V_B) - O_A} \leq 1000 \ell \mid O_B=\bot} \leq q.
			\end{align*}
			(i.e., if $O_B=\bot$, Bob can't estimate the value of $O_A$ with error $\leq 1000 \ell$).
		\end{enumerate}
	\end{itemize}
     We say that a channel ensemble $C=\set{C_\pk}_{\pk\in N}$ is a {\sf computational approximate weak erasure channel}, denoted $(\ell,\alpha,p,q)$-\CompAWEC, if for every \ppt algorithm $\Dc$ and every sufficiently large $\pk\in\N$, $C_\pk$ satisfies the properties stated in the items above. A protocol $\Gamma$ is said to be $(\ell,\alpha,p,q)$-$\CompAWEC$, if the ensemble induced by the protocol (that is, $C=\set{C_{\Gamma(\pk)}}_{\pk\in\N}$) is $(\ell,\alpha,p,q)$-$\CompAWEC$.  
\end{definition}

We will make use of the following lemma, which shows that for some choices of the parameters, \AWEC implies \WEC. The lemma is proven in \cref{sec:AWEC-to-WEC}.

\begin{lemma}\label{lemma:AWEC-to-WEC}
	For every $\ell> 0$, there exists a \ppt protocol $\Lambda = (\Pc_1,\Pc_2)$ such that given an oracle access to an $(\ell,\alpha,p,q)$-\AWEC $C$, the channel $\tilde{C}$ induced by $\Lambda^C$ is $(\alpha'=\alpha+0.001,\: p' = p ,\:  q' = 1/2 + 2(q+0.01))$-\WEC.
	Furthermore, the proof is constructive in a black-box manner:
	\begin{enumerate}
		\item There exists an oracle-aided \ppt algorithm $\Ec_1$ such that for every channel $C = ((\OA,\VA), (\OB,\VB))$ and algorithm $\Dc$ violating the \WEC secrecy property~\ref{WEC:item:A} of $\tilde{C}$, algorithm $\Ec_1^{\Dc}$ violates the \AWEC secrecy property~\ref{AWEC:item:A} of $C$.
		
		\item There exists an oracle-aided \ppt algorithm $\Ec_2$ such that for every channel $C = ((\OA,\VA), (\OB,\VB))$ and algorithm $\Dc$ violating the \WEC secrecy property~\ref{WEC:item:B} of $\tilde{C}$, algorithm $\Ec_2^{\Dc}$ violates the \AWEC secrecy property~\ref{AWEC:item:B} of $C$.
	\end{enumerate}
\end{lemma}

Since \cref{lemma:AWEC-to-WEC} is constructive, the following is an immediate corollary.
\begin{corollary}\label{cor:CompAWEC to CompWEC}
There exists an oracle aided \ppt protocol $\Lambda$, such that given a protocol $\Gamma$ that induces $(\ell,\alpha,p,q)$-\CompAWEC, it holds that $\Lambda^\Gamma$ is $(\alpha'=\alpha+0.001,\: p' = p ,\:  q' = 1/2 + 2(q+0.01))$-\CompWEC.  
\end{corollary}
\begin{proof}[Proof of \ref{cor:CompAWEC to CompWEC}]
Let $\Lambda$ be the \ppt algorithm guaranteed  by Lemma \ref{lemma:AWEC-to-WEC}. Given an $(\ell,\alpha,p,q)$-\CompAWEC protocol $\Gamma$, we define $\Lambda(\pk)={\Lambda^{\Gamma(\pk)}(\pk)}$. Assume towards a contradiction that $\Lambda$ is not a $(\alpha',p',q')$-\CompWEC. It follows that there exists a \ppt $\Dc$ that for infinity many $\pk\in\N$ contradicts one of the \WEC secrecy properties of channel ensemble $\set{C_{\Lambda(\pk)}}_{\pk\in\N}$. Fix $\pk\in\N$ for which this holds. By Lemma \ref{lemma:AWEC-to-WEC}, there exists a \ppt $\Ec^\Dc$ that for every such $\pk$  contradicts one of the secrecy properties of the channel $C_{\Gamma(\pk)}$. This implies that for infinity many $\pk\in\N$, $\Ec^\Dc$  contradict the secrecy of the channel ensemble $\set{C_{\Gamma(\pk)}}_{\pk\in\N}$, which is a contradiction since this would means that $\Gamma$ is not a $(\ell,\alpha,p,q)$-\CompAWEC.       
\end{proof}



\subsection{Oblivious Transfer (\OT)}

\paragraph{Secure Computation.}
We use the standard notion of securely computing a functionality, \cf  \cite{Goldreich04}.
\begin{definition}[Secure computation]\label{def:SFE}
	A two-party protocol {\sf securely computes a functionality $f$}, if it does so according to the real/ideal paradigm.   We add the term perfectly/statistically/computationally/non-uniform computationally, if the simulator's output is  perfect/statistical/computationally indistinguishable/  non-uniformly indistinguishable from  the real distribution.  The protocol have the above notions of security {\sf against semi-honest  adversaries}, if its security only  guaranteed to holds against an adversary that follows the prescribed protocol.   Finally, for the case of perfectly secure computation, we naturally apply the above notion also to the non-asymptotic case: the protocol with no security parameter perfectly  compute a functionality $f$.
	
	A two-party protocol {\sf securely computes a functionality ensemble $f$ with oracle to a channel $C$}, if it does so according to the above definition when the parties have access to a trusted party computing $C$. All the above adjectives naturally extend to this setting.
\end{definition}

\paragraph{Oblivious Transfer.}
The (one-out-of-two) oblivious transfer functionality is defined as follows.
\begin{definition}[oblivious transfer functionality $f_{\OT}$]\label{def:OTfunc}
	The oblivious transfer functionality over $\zo \times (\zs)^2$ is defined by  $f_{\OT} (i,(\sigma_0,\sigma_1)) = (\perp,\sigma_i)$.
\end{definition}
A protocol is $\ast$ secure OT,   for \\$\ast\in \set{\text{semi-honest statistically/computationally/computationally non-uniform}}$, if it  compute the $f_{\OT}$  functionality with $\ast$ security.





% \begin{definition}[Computational oblivious transfer, semi-honest model]
% A protocol $\Pi=(\Ac,\Bc)$ is a semi-honest 1-out-of-2 computational oblivious transfer (comp-OT) protocol if the following holds. Given a common input $1^{\pk}$, the parties $\Ac$ and $\Bc$ run the protocol $\Pi(1^\pk)$ (in an honest manner) and    
% $\Ac$ outputs $X=(m_1,m_2)\in \zo\times\zo$ and has a view $U$ and $\Bc$ outputs $Y=(i,\hat{m})\in\zo\times\zo$ and has a view $V$, and the following properties are satisfied:
% \begin{enumerate}
%     \item \textbf{Correctness:} 
%     $\pr{\hat{m}\neq m_i}<\negl(\pk).$ 
    
%     \item \textbf{A's Privacy:} For every \ppt $\Dc$ and every sufficiently large $\pk$:
%     $\pr{\Dc(V)=m_{i-1}}<(1+\negl(\pk))/2$
    
%     \item \textbf{B's Privacy:} For every \ppt $\Dc$ and every sufficiently large $\pk$:
%     $\pr{\Dc(U)=i}<(1+\negl(\pk))/2$  
% \end{enumerate}
% \end{definition}

We make use of the following useful results by Wullschleger on oblivious transfer amplification from weak channels.
\begin{theorem}[\cite{Wullschleger09}, from \WEC to statistically secure \OT]\label{thm:WEC TO OT IT}
    There exists an oracle aided protocol $\Pi$ such that the following holds: Given a $(\alpha,p,q)$-\WEC $C$, if $44(\alpha+p)\le 1-q$ then $\Pi^{C}(1^\pk)$ is a semi-honest statistically secure \OT.
\end{theorem}

The following computational version of \cref{thm:WEC TO OT IT} is implicit in \cite{Wullschleger09} and is based on the computational proof explicitly stated in \cite{Wul07} (see Section 6 in \cite{Wullschleger09} for discussion).   

\begin{theorem}[\cite{Wullschleger09,   Wul07}, from \CompWEC to computinally secure \OT]\label{thm:WEC TO OT Comp}
    There exists an oracle aided protocol $\Pi$ such that the following holds: Given a $(\alpha,p,q)$-\CompWEC protocol $\Lambda$, if $44(\alpha+p)\le 1-q$ then $\Pi^{\Lambda}$ is a semi-honest computational secure \OT.
\end{theorem}



% \begin{definition}[Computational 1-out-of-2 Oblivious Transfer, semi-honest model]
% A protocol $\Pi=(\Ac,\Bc)$ is a semi-honest 1-out-of-2 $(\eps,\alpha,\beta)$-oblivious transfer (OT) protocol if the following holds. 

% The parties $\Ac$ and $\Bc$ run the protocol (in an honest manner) and    
% $\Ac$ outputs $X=(m_1,m_2)\in \zo\times\zo$ and has a view $U$ and $\Bc$ outputs $Y=(i,\hat{m})\in\zo\times\zo$ and has a view $V$, and following properties are satisfied:
% \begin{enumerate}
%     \item \textbf{Correctness:} 
%     $\pr{\hat{m}\neq m_i}<\eps.$ 
    
%     \item \textbf{A's Privacy:} For every adversary $\Dc$:
%     $\pr{\Dc(V)=m_{i-1}}<(1+\alpha)/2$
    
%     \item \textbf{B's Privacy:} For every adversary $\Dc$: $\pr{\Dc(U)=i}<(1+\beta)/2$  
% \end{enumerate}
% \end{definition}

\section{Limitations of Process Reward Models Trained on Math Domain Data}
\label{sec:math-lim}



We introduce various math PRMs used for comparison in~\Cref{sec:existing-mathprm}, present our multi-domain evaluation dataset in~\Cref{sec:mmlu-eval}, and provide a detailed analysis of the evaluation results in~\Cref{sec:mathprm-eval}.


\subsection{Open-Source Math PRMs}
\label{sec:existing-mathprm}



For evaluation,
we conduct experiments on a diverse set of models.
Our analysis includes four open-source math PRMs:
Math-PSA~\citep{wang2024openr},
Math-Shepherd~\citep{wang2024math},
RLHFLow-Deepseek~\citep{xiong2024rlhflowmath},
and Qwen-2.5-Math-PRM~\citep{zheng2024processbench}.


In addition to the open-source models,
two math PRMs based on open-source models are specifically trained in this work.
They are denoted as \emph{LlamaPRM800K} and \emph{QwenPRM800K}. More details are given in~\Cref{sec:open-mathprm}



\subsection{Multi-Domain Evaluation Dataset}
\label{sec:mmlu-eval}


For our multi-domain evaluation dataset, we curate questions sampled from the MMLU-Pro dataset~\citep{wang2024mmlupro}.
MMLU-Pro is designed to benchmark the reasoning abilities of LLMs and consists of college-level multiple choice questions in the following 14 domains:
\emph{Math}, \emph{Physics}, \emph{Chemistry}, \emph{Law}, \emph{Engineering}, \emph{Other}, \emph{Economics}, \emph{Health}, \emph{Psychology}, \emph{Business}, \emph{Biology}, \emph{Philosophy}, \emph{Computer Science}, and \emph{History}.


To craft our evaluation dataset, we randomly sample 150 questions from each domain. Due to duplicate questions, we discard 41 questions---23 from Biology, 10 from Health, 5 from Law, and 1 each from Business, Economics, and Philosophy. For each remaining question, we generate 128 candidate solutions using Llama-3.1-8B-Instruct~\citep{dubey2024llama} for MV, WMV, and BoN test-time inference algorithms.
Prompt details and generation parameters are provided in~\Cref{sec:synth-gen-prompts}.
We refer to this multi-domain evaluation dataset as \emph{\ourdataeval}.


\begin{table}[t]
\caption{Results of two open-source math PRMs on different domains in~\ourdataeval~when using WMV with min-aggregation on 16 CoTs generated per question using Llama-3.1-8B-Instruct. In parenthesis we report absolute difference between WMV and MV (WMV$-$MV). While WMV using math PRMs exhibits greater improvement in Math and Math-adjacent domains, there is no significant improvement on MV in other domains.}
\label{tab:math_prm_on_multi_domainsec4}
\small
\centering
\setlength{\tabcolsep}{3pt}
\begin{tabularx}{\linewidth}{l|c|
>{\centering\arraybackslash}X
>{\centering\arraybackslash}X}
\toprule
\textbf{Category} & \textbf{MV}  & \textbf{Math-Shepherd} & \textbf{Qwen-2.5-Math-PRM}   \\
\midrule
All & 57.15  & 57.66 (+0.51) & 58.17 (+1.02)\\
All except math & 56.61  & 57.01 (+0.40) & 57.32 (+0.71) \\
Math & 62.40  & 64.13 (+1.73) & 67.20 (+4.80)   \\
\midrule
Chemistry & 58.67  & 60.13 (+1.46) & 60.67 (+2.00) \\
Physics & 58.53 & 61.87 (+3.34) & 61.47 (+2.94) \\
\midrule
Biology & 75.38 & 75.38 (+0.00) & 75.69 (+0.31) \\
Psychology & 61.60  & 61.47 (-0.13) & 62.27 (+0.67)   \\
Law & 35.93 & 37.24 (+1.31) & 36.28 (+0.35)   \\
History & 49.20 & 49.87 (+0.67) & 49.40 (+0.20)  \\
Philosophy & 44.83  & 44.70 (-0.13) & 45.17 (+0.34) \\
\bottomrule
\end{tabularx}

    \vskip -0.2in
\end{table}



\subsection{Multi-Domain Performance of Math PRMs}
\label{sec:mathprm-eval}





We conduct comprehensive analyses on a diverse set of models. For clarity, we report results for two representative models here, with additional evaluations available in~\Cref{sec:mathprm-fullevals}. The first model, Math-Shepherd~\citep{wang2024math},
is trained on synthetically generated math data labeled via a rollout-based method.
The second model,
Qwen-2.5-Math-PRM~\citep{zheng2024processbench},
is a best-performing open-source PRM,
trained on the high-quality expert labeled PRM800K math dataset~\citep{lightman2023let}.


The PRMs are applied using WMV with min-aggregation. While math PRMs show significant improvements in mathematical reasoning domains,
their effectiveness in broader, non-mathematical areas remains limited.
Notably, in the Math category,
Qwen-2.5-Math-PRM and Math-Shepherd achieve relative gains of $+4.80$ and $+1.73$,
respectively,
outperforming the MV baseline.
Similar improvements are observed in Math-adjacent disciplines:
Chemistry ($+2.00$ for Qwen-2.5-Math-PRM) and Physics ($+3.34$ for Math-Shepherd),
underscoring their utility in tasks requiring mathematical reasoning.

\begin{highlight}
    \paragraph{Finding 1:} 
    \emph{Math PRMs struggle to generalize to broader domains.}
\end{highlight}

However, the benefits diminish sharply in non-mathematical areas.
For example, in Philosophy and History, we see gains of only $+0.34$ and +0.20\% respectively for the most performant PRM Qwen-2.5-Math-PRM.

The ``All except math'' aggregate further underscores this disparity, with PRMs achieving a maximum gain of $+0.71$ (Qwen-2.5-Math-PRM) compared with the majority voting baseline.

These results highlight a critical limitation: math PRMs trained exclusively on mathematical data lack the versatility to generalize beyond mathematical reasoning tasks. While they excel in contexts aligned with their training---quantitative reasoning---their capacity to evaluate reasoning quality in broader domains remains insufficient.


\section{Automatic Generation of Multi-Domain Reasoning Data with Labels}
\label{sec:synth-data-gen}

\begin{figure*}[ht]
    \begin{center}
        \includegraphics[width=0.91\textwidth]{figures/Synthetic_Data_Pipeline.pdf}
        \caption{A diagram of the synthetic data generation pipeline. In the CoT Generation Stage, each question is used to generate 16 CoT solutions. Then, in the Auto-Labeling Stage, each CoT is evaluated to create step-wise labels. If a CoT step is labeled as \textbf{BAD}, all subsequent steps will be discarded.}
        \label{fig:synth-data-pipeline-diagram}
    \end{center}
    \vskip -0.2in
\end{figure*}




In order to obtain step-wise reasoning data for non-Math domains,
we devise a pipeline,
as outlined in~\Cref{fig:synth-data-pipeline-diagram},
to generate synthetic reasoning CoTs from existing question-answering data.
These CoTs are then given step-wise labels based on reasoning correctness.
We detail the synthetic data generation process in~\Cref{sec:cot-gen,sec:auto},
including methods to create and annotate reasoning steps. We also provide additional analysis on the quality of the generation pipeline in~\Cref{sec:auto-analysis-ablate}.


\subsection{Chain-of-Thought Generation}
\label{sec:cot-gen}




For the generation of CoTs,
we prompt Llama-3.1-8B-Instruct to produce step-by-step reasoning for each input question. For training, we source questions from the MMLU-Pro dataset~\citep{wang2024mmlupro}, selected for its high-quality, challenging problems spanning diverse topics.
From this dataset, we randomly sample up to 500 questions per domain, ensuring that it is disjoint to the subset used for evaluation. We then generate 16 CoTs for each sampled question. Post-generation, we filter out CoTs exceeding the 2,048-token limit or containing unparsable answers.


\subsection{Auto-Labeling}
\label{sec:auto}


To annotate our synthetic CoT data, we adopt an approach inspired by the critic models in the work of~\citet{zheng2024processbench}.
Specifically, we utilize Llama-3.1-70B-Instruct as a strong LLM to evaluate each CoT using step-by-step reasoning, locating the earliest erroneous step, if any. To enhance accuracy and consistency, we identified two key additional components.


First, we incorporate explicit step evaluation definitions,
inspired by~\citet{lightman2023let},
into the system prompt.
Steps are categorized as \textbf{GOOD},
\textbf{OK},
or \textbf{BAD}:
\textbf{BAD} for incorrect, unverifiable, or irrelevant steps;
\textbf{GOOD} for correct, verifiable, and well-aligned steps;
\textbf{OK} for intermediate cases.
Second, we also provide the reference ground-truth answer in the prompt.
The full prompt is detailed in~\Cref{sec:synth-gen-prompts}.


To convert the auto-labeling output to stepwise labels, we apply the following rule:
if no steps are detected as incorrect, all steps in the CoT are labeled as $1$.
If a step is detected as incorrect, all preceding steps are labeled as $1$, the incorrect step is labeled as $-1$,
and all subsequent steps are discarded.



In total, we sample 5,750 questions from MMLU-Pro. Among the 84,098 generated CoTs that passed filtering, 36,935 were labeled as having no incorrect steps and 47,163 were labeled as having at least one (see Table \ref{tab:dataset_composition}). This dataset, denoted as \emph{\ourdatatrain}, is the first open-source multi-domain reasoning dataset with step-wise labels.






To assess the quality of our auto-labeled data,
we conduct a manual evaluation on a random sample of 30 questions from the dataset.
For each question, we randomly select one CoT classified as entirely correct and two CoTs flagged as containing an incorrect step.
We then manually validate whether the auto-labeled judgments align with our own assessments.


For the CoTs labeled as correct by the auto-labeler, we observed an agreement rate of 83\% with our manual evaluations.
For CoTs labeled as incorrect, the agreement rate was 70\%.

Based on these results, we estimate that approximately 75\% of the CoTs in the entire dataset are correctly labeled.
This level of accuracy is comparable to that of manually-labeled CoT datasets,
such as PRM800K~\citep{lightman2023let},
which is estimated to achieve around 80\% accuracy.\footnote{Refer to \href{https://github.com/openai/prm800k/issues/12}{this GitHub issue} for a discussion on PRM800K's accuracy.}


\subsection{Auto-Labeling Prompt Analysis}
\label{sec:auto-analysis-ablate}


To further understand the factors influencing auto-labeling performance,
we conduct an evaluation of the auto-labeling using a simplified prompt.
Specifically, we remove the system prompt defining the types of reasoning steps and exclude the reference ground-truth answer from the prompt.
When re-evaluating the auto-labeling quality,
we observed a drastic drop in performance, with the agreement rate for CoTs labeled as correct by the original auto-labeler decreasing by over 70\%,
from 83\% to 7\%, while
the agreement rate for CoTs labeled as incorrect decreased
from 70\% to 62\%.


These results highlight the importance of providing both a well-defined prompt with step label definitions and access to the ground-truth answer in achieving high auto-labeling accuracy.
The ground-truth answer provides essential context on CoT final correctness and enhances the model's ability to evaluate reasoning steps effectively.


\subsection{Counterfactual Augmentation}


To generate additional examples of incorrect reasoning,
we explore methods for instructing an LLM to modify steps in our correct CoTs,
introducing specific types of errors.
We refer to this process broadly as counterfactual augmentation.
However, incorporating counterfactual error steps during PRM training was not observed to significantly improve performance.
Therefore, we defer specific details and experiments using counterfactual augmentation to~\Cref{sec:counter-aug}.


\section{Multi-Domain Process Reward Model}
\label{sec:multi-eval}

We present the implementation and evaluation of \ourprm, structured as follows.
First, \Cref{sec:mdprm-train} covers the various training configurations used.  
We then evaluate \ourprm via BoN and WMV in \Cref{sec:math-v-mdprm}, showing improved domain generalization compared to math PRMs.  
In \Cref{sec:m-v-mdprm-search}, we additionally discuss results using Beam Search and MCTS. Lastly, we examine \ourprm's ability to scale test-time compute for larger models such as Deepseek-R1~\cite{guo2025deepseek} in~\Cref{sec:deepseek}.



\subsection{Training of Our Multi-Domain PRM}
\label{sec:mdprm-train}

To train \ourprm, we employ a classification head atop an LLM,
optimizing with a cross-entropy loss applied to a special classification token appended at the end of each CoT step in \ourdatatrain.
Detailed specifics and hyperparameters are provided in~\Cref{sec:prm-train}.


We explore several training configurations,
including:
1) LoRA~\citep{hu2022lora} vs.~full fine-tuning for efficient training,
2) a base LLM vs.~a math PRM for initializing the PRM,
and 3) a Qwen-based PRM vs.~a Llama-based PRM for training.
Comprehensive experimental results for these studies are presented in the next section.
Based on those findings, our final,
our final multi-domain PRM, named~\ourprm, is initialized from our LlamaPRM800K---see~\Cref{sec:add-prm-train} for its details---fine-tuned using LoRA on our multi-domain training dataset.
% \vspace{-6mm}
\subsection{Math PRM vs.~\ourprm~on Reranking Based Inference-Time Methods}
\label{sec:math-v-mdprm}

We first report results of the reranking methods WMV and BoN on \ourdataeval.
For both methods, we adopt Min-aggregation, as it outperforms Average and Last in aggregating PRM step scores;
see~\Cref{sec:agg-comp} for comparison.
We also include MV as a baseline.



\textbf{Comparison with Math Open-Source PRMs.}
We evaluate our multi-domain PRM, \ourprm, against open-source math PRMs by partitioning~\ourdataeval~into three groups:
1) \emph{Math},
2) \emph{Math-adjacent}, i.e., Chemistry, Computer Science, Engineering, Physics,
and 3) \emph{non-Math-adjacent} domains.
As shown in~\Cref{fig:math-wmv-min},
our model consistently outperforms baselines in both WMV and BoN across all domain groups.


\begin{highlight}
    \paragraph{Finding 2:} 
    \emph{Fine-tuning with synthetic multi-domain data enhances the generalizability of PRM.}
\end{highlight}


For WMV, we can see the relative performance difference increase with domain distance from core mathematics. While performance of open-source math PRMs converges to the majority voting baseline in non-mathematical domains, our multi-domain PRM maintains robust generalization.


In BoN the superiority of our multi-domain PRM is even more pronounced. Unlike open-source math PRMs, which fail to surpass the baseline of MV in Math-adjacent and non-Math-adjacent domains,
our model consistently surpasses it across all domain groups.


See~\Cref{sec:bon-mv-bycat} for more fine-grained details where we plot WMV and BoN for every domain of~\ourdataeval.
The results are consistent with~\Cref{fig:math-wmv-min}, and~\ourprm~outperforms math PRMs in all domains.



\begin{figure*}[t]
    \begin{center}
        \includegraphics[width=.94\linewidth]{figures/new_figures/math_vs_nonmath_prm_min_agg.pdf}        
        \caption{Comparison of WMV (top) and BoN (bottom) using \ourprm~against open-source math PRMs on~\ourdataeval. We use min-aggregation and the CoTs are generated using Llama-3.1-8B-Instruct. \ourprm~has consistently better performance than math PRMs, and the differences become larger in domains not adjacent to Math.}
        \label{fig:math-wmv-min}
    \end{center}
    \vskip -0.2in
\end{figure*}


\textbf{Ablation Experiments Using Multi-Domain PRM Trained on Math Only Subset vs.~Random Subset.}

We further conduct an ablation study to evaluate the impact of training data diversity on the performance of our LlamaPRM800K Math PRM.
Specifically, we train one PRM using only the math subset of our multi-domain training data and another using a random subset of the \emph{same} size.
We refer to these two models as \ourprm~(Math subset) and \ourprm~(random subset), respectively.
This experiment tests that the improved performance of our multi-domain PRM is due to the domain-diversity of the CoT data and not merely from learning the in-distribution question and CoT formats of MMLU-Pro questions. If the latter is the case, both PRMs should perform similarly, given that they are exposed to the same amount of questions and CoT examples with the in-distribution format.

\begin{highlight}
    \paragraph{Finding 3:} 
    \emph{Domain diversity of CoTs in a training dataset plays an integral role in generalization of PRMs to multiple domains.}
\end{highlight}

As shown in \Cref{fig:prm-ablation}, \ourprm~(random subset) achieves superior performance in WMV compared to \ourprm~(Math subset).
This trend holds across both Math and non-Math domains. These findings suggest two key insights.
First, our PRM is not simply learning the question format but is acquiring knowledge on how to label reasoning across diverse domains. This is why training on diverse data enables better overall performance than training on same sized data in only one domain. Second, \ourprm~(random subset) also demonstrates slightly better performance in the math domain, indicating that training on a diverse dataset may facilitate positive transfer, where insights from other domains enhance reasoning in the Math domain.



\begin{figure}[t]
    \centering
    \includegraphics[width=\columnwidth]{figures/new_figures/prm_diff_train_subset_min_agg.pdf}
    \caption{Comparison of WMV using LlamaPRM800K, \ourprm~(Math subset) and \ourprm~(random subset). \ourprm~(random subset) achieves better performance than \ourprm~(Math subset) in Math and non-Math.}
    \label{fig:prm-ablation}
    \vskip -0.2in
\end{figure}



\textbf{Experiments Using Other Training Configurations.}
While our final version of~\ourprm~is trained from LlamaPRM800K on our synthetic data using LoRA, we also test the following training configurations on our multi-domain dataset:
\begin{itemize}[leftmargin=10px]

    \item \textbf{\ourprm~(Llama Base)}: We initialize training from Llama-3.1-8B-Instruct, and use LoRA fine-tuning with our multi-domain dataset.

    \item \textbf{\ourprm~(Qwen)}: We initialize training from QwenPRM800K PRM, and utilize LoRA fine-tuning with our multi-domain dataset.

    \item \textbf{\ourprm~(full-tuned)}: We initialize training from LlamaPRM800K PRM, and do \emph{full} fine-tuning with our multi-domain dataset.

\end{itemize}

The results are presented in~\Cref{fig:multiprm-trainexps}.
Comparing \ourprm~(Qwen) and \ourprm~(Llama), we observe that the QwenPRM800K~\ourprm~performs worse.
This highlights the importance of base model choices. Although Qwen-2.5-Math-7B, the base model for QwenPRM800K, is specialized in mathematical reasoning, its limitations in general-domain knowledge hinder its ability to fully leverage multi-domain training data.


\begin{highlight}
    \paragraph{Finding 4:} 
    \emph{Exposure to mathematical data beforehand can enhance a PRMs' ability to effectively leverage multi-domain CoT fine-tuning.}
\end{highlight}


Next, comparing \ourprm~(Llama Base) with \ourprm, we find that the latter achieves superior performance in Math while maintaining comparable performance in non-Math domains. This suggests that prior exposure to mathematical data enhances the model’s ability to benefit from further domain-specific training.

We note that \ourprm~(full-tuned) has worse performance than \ourprm.
This may be due to suboptimal hyperparameters leading to overfitting during full fine-tuning.


\begin{figure}[t]
    \centering
    \includegraphics[width=\columnwidth]{figures/new_figures/prm_diff_min_agg.pdf}
    \caption{Comparison of MVW using \ourprm~against other multi-domain PRMs trained using different configurations. \ourprm~has better WMV performance than all other models in both Math and non-Math domains.}
    \label{fig:multiprm-trainexps}
\end{figure}



    
\subsection{Math PRM vs. Multi-Domain PRM on Search Based Inference-Time Methods}
\label{sec:m-v-mdprm-search}

We evaluate the performance of math PRMs (using LlamaPRM800K) and~\ourprm~with beam search and MCTS on~\ourdataeval.
The results over questions in all domains, presented in~\Cref{fig:prm-mcts},
show that MCTS outperforms beam search and that they both do better than the MV baseline.
Regardless of the search algorithm used, consistent with our WMN and BoN results,~\ourprm~gives boosted performance over the math PRM.
Details by category results are presented in~\Cref{sec:mcts-detailed}.


\begin{figure}[t]
    \centering
    \includegraphics[width=.95\columnwidth]{figures/new_figures/mcts_vs_beam.pdf}
    \caption{Comparison of \ourprm~and LlamaPRM800K with beam search and MCTS. Overall in the diverse domains from~\ourdataeval, \ourprm~achieves better performance.}
    \label{fig:prm-mcts}
    \vskip -0.2in
\end{figure}


\subsection{Does PRM with Test-Time Compute help Reasoning Models?}
\label{sec:deepseek}



\begin{figure}[t]
    \centering
    \includegraphics[width=\columnwidth]{figures/weighted_majority_voting_comparison_min.pdf}
    \caption{Comparison of WMV using~\ourprm~against Qwen-2.5-Math-PRM on DeepSeek-R1 generated CoTs for the Law subset. \ourprm~has better performance than all other math PRMs.}
    \label{fig:multiprm-deepseek}
\end{figure}


We have shown that~\ourprm~can effectively leverage inference-time compute to increase LLM performance,
a natural question is whether this effectiveness extends to renowned strong reasoning models,
e.g., DeepSeek-R1~\citep{guo2025deepseek}.
Given that a well-trained reasoning model may already generate coherent and correct reasoning steps due to being trained for reasoning,
one might hypothesize that reranking methods like WMV and BoN brings marginal improvement over WM.


To test this, we evaluate the performance of~\ourprm~via WMV on DeepSeek-R1.
Due to budget constraints, we focus on the Law subset and sample 16 CoT responses per question. As shown in~\Cref{fig:multiprm-deepseek}, \ourprm~provides a slight but noticeable performance boost to DeepSeek-R1 during test-time inference despite the limited CoT samples. Significantly it outperforms both the math PRM \emph{and} the MV baseline.
This finding, though preliminary, nullifies the aforementioned hypothesis and suggests that---in fact---large reasoning models \emph{can} still benefit from PRMs during inference to further boost their performance beyond MV.






\section{Discussion and Future Directions}
This work identifies signal collapse as a critical bottleneck in one-shot neural network pruning. Performance loss in pruned networks is due to \textbf{signal collapse} in addition to the removal of critical parameters. We propose \textbf{REFLOW} (\textbf{Re}storing \textbf{F}low of \textbf{Low}-variance signals), a simple yet effective method that mitigates signal collapse without computationally expensive weight updates. By focusing on signal preservation, REFLOW highlights the importance of mitigating signal collapse in sparse networks and enables magnitude pruning to match or surpass state-of-the-art one-shot pruning methods such as CHITA, CBS, and WF.

REFLOW consistently achieves state-of-the-art accuracy across diverse architectures, restoring ResNeXt-101 from under 4.1\% to 78.9\% top-1 accuracy at 80\% sparsity on ImageNet. Its lightweight design makes it a practical solution for both research and deployment, delivering high-quality sparse models without the overhead of traditional approaches. These findings challenge the traditional emphasis on weight selection strategies and underscore the critical role of signal propagation for achieving high-quality sparse networks in the context of one-shot pruning.





\section*{Acknowledgements}

Kangwook Lee is supported by NSF CAREER Award CCF-2339978, an Amazon Research Award, and a grant from FuriosaAI. In addition, Thomas Zeng acknowledges support from NSF under NSF Award DMS-202323 and Daewon Chae was supported by the Hyundai Motor Chung Mong-Koo Foundation.


\section*{Impact Statement}

Given the potential for LLMs to be used in unethical ways, such as spreading misinformation or manipulating public opinion, our Multi-Domain PRM could inadvertently contribute to such misuse. To mitigate these risks, it is essential to implement robust safeguards in training and inference.


\bibliography{refs}
\bibliographystyle{icml2025}


\newpage
\appendix
\onecolumn

\section{More Details on Synthetic Data Generation Pipeline}
\subsection{Dataset Composition}
The total composition of \ourdatatrain~is as follows.

\begin{table}[ht]
    \centering
    \caption{Composition of \textit{\ourdatatrain}}
    \small
    \begin{tabular}{cccc} \toprule
          & \textbf{Total} & \textbf{Fully Correct} & \textbf{Incorrect}   \\ \midrule
         Number of CoTs & 84098 & 36935 & 47163 \\
         Number of Steps & 487380 & 440217 & 47163\\
         \bottomrule
    \end{tabular} 
    \label{tab:dataset_composition}
\end{table}


\subsection{Data Generation Pipeline Prompts}
\label{sec:synth-gen-prompts}


To generate chain-of-thought (CoT) reasoning for MMLU-Pro questions, we utilize the prompt shown in~\Cref{fig:cot-gen-prompt-mmlu}.
To ensure the generated CoT adhere to the proper format---where steps are separated by two newline characters and the final step follows the structure ``the answer is (X)''---we include five few-shot examples. These examples are derived from the CoTs provided in the validation split of MMLU-Pro, with additional processing to ensure each step is delimited. The code for generating the complete prompt will be open-sourced alongside the rest of our code and data.


During generation, we use a temperature of $0.8$ and set the maximum generation length to 2,048 tokens. During auto-labeling, we use a temperature of 0, and the maximum generation length remains at 2,048 tokens.

\begin{figure}[ht]
    \centering
    \begin{minipage}{6in}
    \begin{tcolorbox}[width=6in, sharp corners=all, colback=white!95!black]
The following is a multiple choice question and its ground truth answer. You are also given a students solution (split into step, enclosed with tags and indexed from 0):

\-

[Multiple Choice Question]

\{question\}

\-

[Ground Truth Answer]

\{answer\}

\-

[Student Solution]

\{$<$step\_0$>$\\
Student solution step 0\\
$<$/step\_0$>$

\-\\
$<$step\_1$>$\\
Student solution step 0\\
$<$/step\_1$>$

\-\\...\}

\end{tcolorbox}
    \end{minipage}
    \caption{User prompt template for auto-labeling.}
    \label{fig:v6-auto-label-prompt}
\end{figure}

\begin{figure}[ht]
    \centering
    \begin{minipage}{6in}
    \begin{tcolorbox}[width=6in, sharp corners=all, colback=white!95!black]

You are an experienced evaluator specializing in assessing the quality of reasoning steps in problem-solving. Your task is to find the first BAD step in a student's solution to a multiple choice question.

\-\\
You will judge steps as GOOD, OK or BAD based on the following criteria:\\
1. GOOD Step\\
A step is classified as GOOD if it meets all of these criteria:\\
- Correct: Everything stated is accurate and aligns with known principles or the given problem.\\
- Verifiable: The step can be verified using common knowledge, simple calculations, or a quick reference (e.g., recalling a basic theorem). If verifying requires extensive effort (e.g., detailed calculations or obscure references), mark it BAD instead.\\
- Appropriate: The step fits logically within the context of the preceding steps. If a prior mistake exists, a GOOD step can correct it.\\
- Insightful: The step demonstrates reasonable problem-solving direction. Even if ultimately progress in the wrong direction, it is acceptable as long as it represents a logical approach.

\-\\
2. OK Step\\
A step is classified as OK if it is:\\
- Correct and Verifiable: Contains no errors and can be verified.\\
- Unnecessary or Redundant: Adds little value, such as restating prior information or providing basic encouragement (e.g., “Good job!”).\\
- Partially Progressing: Makes some progress toward the solution but lacks decisive or significant advancement.

\-\\
3. BAD Step\\
A step is classified as BAD if it:\\
- Is Incorrect: Contains factual errors, misapplies concepts, derives an incorrect result, or contradicts the ground truth answer.\\
- Is Hard to Verify: Requires significant effort to confirm due to poor explanation.\\
- Is Off-Topic: Includes irrelevant or nonsensical information.\\
- Derails: Leads to dead ends, circular reasoning, or unreasonable approaches.

\-\\
\#\#\#\# Task Description\\
You will be provided with:\\
1. A Question\\
2. A Ground Truth Answer\\
3. A Reference explanation of the answer\\
4. A Student's Step-by-Step Solution, where each step is enclosed with tags and indexed from 0

\-\\
You may use the ground truth answer and reference explanation in classifying the type of each step.\\
A student's final answer is considered correct if it matches the ground truth answer or only differs due to differences in how the answer is rounded.
Once you identify a BAD step, return the index of the earliest BAD step. Otherwise,
return the index of -1 (which denotes all steps are GOOD or OK).
Please put your final answer (i.e., the index) in $\backslash\backslash$boxed{}.
\end{tcolorbox}
    \end{minipage}
    \caption{System prompt for auto-labeling.}
    \label{fig:v5-auto-label-prompt}
\end{figure}

% 

\clearpage

\subsection{Counterfactual Augmentation}
\label{sec:counter-aug}


\begin{figure*}[ht]
    \begin{center}
        \includegraphics[width=0.9\textwidth]{figures/Counterfactual_Augmentation_Pipeline.pdf}
         \caption{Diagram of the counterfactual augmentation pipeline}
        \label{fig:neg-aug-pipeline}
    \end{center}
\end{figure*}


After generating and labeling our synthetic reasoning CoTs (as described in~\Cref{sec:synth-data-gen}), we attempted to create additional incorrect steps by augmenting the correct reasoning steps. Our pipeline is depicted in~\Cref{fig:neg-aug-pipeline}.
We provide the full CoT to Llama-3.1-70B-Instruct, instructing it to select and rewrite a step where it would be appropriate to introduce an error.
Additionally, we define a list of possible fine-grained error types. To encourage the generation of a variety of different error types, we only include a random selection of two of these error types in each system prompt, forcing the LLM to choose one. The error types are:
\begin{itemize}
    \item Conflicting Steps: The reasoning step includes statements that contradict previous steps.
    \item Non-sequitur: The reasoning step introduces information that is irrelevant to the question.
    \item Factual: The reasoning step contains incorrect statements of factual information.
    \item False Assumption: The reasoning step makes an incorrect assumption about the question.
    \item Contextual: The reasoning step misinterprets information given within the question/context.
\end{itemize}



For the prompt format used in counterfactual augmentation,
see~\Cref{fig:neg-augmentation-system-prompt}.
In total, we generated 73,829 augmented incorrect steps.


\begin{figure}[ht]
    \centering
    \begin{minipage}{6in}
    \begin{tcolorbox}[width=6in, sharp corners=all, colback=white!95!black]
    The following are multiple choice questions (with answers). Think step by step and then finish your answer with "the answer is (X)" where X is the correct letter choice.
    \end{tcolorbox}
    \end{minipage}
    \caption{Prompt to generate CoTs for MMLU Pro.}
    \label{fig:cot-gen-prompt-mmlu}
\end{figure}


\begin{figure}[ht]
    \centering
    \begin{minipage}{6in}
    \begin{tcolorbox}[width=6in, sharp corners=all, colback=white!95!black]
\small You are a highly knowledgeable philosopher with expertise across many domains, tasked with analyzing reasoning processes. 
Your goal is to identify how a reasoning process could naturally deviate toward an incorrect conclusion through the introduction of subtle errors.

\-\\
Here are a list of potential error types, all of which are equally valid:\\
\textrm{[ERROR TYPE 1]: [ERROR TYPE 1 DEFINITION]}\\
\textrm{[ERROR TYPE 2]: [ERROR TYPE 2 DEFINITION]}

\-\\
Instructions:\\
You will be provided with:\\
1. A question.\\
2. A complete chain of reasoning steps, where each step is numbered (e.g., Step X).

\-\\
Your task is to:
1. Identify the major factual information, reasoning, and conclusions within the reasoning steps.\\
3. Explain how to generate an incorrect step to replace one of the existing steps. This should include:\\
   - Identifying a step where the reasoning could naturally deviate.\\
   - Speculating what type of error would be most appropriate to introduce at the chosen step.\\
4. Introduce an incorrect next step that aligns stylistically with the previous steps. This incorrect step should:\\
   - Reflect a deviation in reasoning that significantly harms the correctness.\\
   - Appear natural and believable in the context of the reasoning process.\\
5. Clearly explain how the incorrect step is an error, highlighting the specific logical or conceptual flaw.

\-\\
Output Format:

\-\\
STEP\_SUMMARY:\\
\textrm{[Summarize the reasoning within the steps in 1-2 sentences, identifying major information, logical steps, and conclusions.]}

\-\\
INCORRECT\_STEP\_GEN:\\
\textrm{[Explain how the reasoning at a specific step could deviate naturally into being incorrect. Clearly describe the type of error that could be introduced at this step.]}

\-\\
ERROR\_TYPE:\\
\textrm{[The name of the type of error chosen to be introduced.]}

\-\\
STEP\_NUM:\\
\textrm{[The number of the step that was identified as a place where the reasoning could naturally deviate. Only include the number here.]}

\-\\
INCORRECT\_STEP:\\
\textrm{[Write the incorrect step in the same tone and style as the other steps. Wrap the incorrect step inside curly braces (e.g. \{incorrect step\}).]}

\-\\
ERROR\_EXPLANATION:\\
\textrm{[}Explain how the incorrect step fits the definition of the selected error type, identifying the specific flaw.\textrm{]}
\end{tcolorbox}
    \end{minipage}
    \caption{System prompt for counterfactual augmentation.}
    \label{fig:neg-augmentation-system-prompt}
\end{figure}




\clearpage

\section{Additional Search Algorithm Details}
\label{sec:search-algs}

\begin{algorithm}[ht]
\caption{Beam Search with Process Reward Model}
\label{alg:beam}
\begin{algorithmic}[1]
\REQUIRE Large Language Model $\text{LLM}(\cdot)$, Process Reward Model $\text{PRM}(\cdot)$, Prompt $s_0$, Number of Beams $N$, Beam width $M$, Maximum step length $L$
\STATE $\mathcal{B} \gets [s_0]$
\STATE $\mathcal{Q} \gets [0]$
\FOR{$i = 1$ to $L$}
    \STATE $\mathcal{B} \gets \text{Expand}(\mathcal{B}, \frac{N}{\operatorname{len}(\mathcal{B})})$

    \STATE $\mathcal{B} \gets \text{LLM.step}(\mathcal{B})$


    \STATE $\mathcal{Q} \gets \text{Aggr}(\mathcal{B})$

    \STATE $\texttt{best\_idxs} \gets$ Indexes of the highest $\frac{N}{M}$ scores in $\mathcal{Q}$

    \STATE $\mathcal{B} \gets \mathcal{B}[\texttt{best\_idxs}]$
    \STATE $\mathcal{Q} \gets \mathcal{Q}[\texttt{best\_idxs}]$
    
    \IF{All sequences in $\mathcal{B}$ contain a terminal leaf node}
        \STATE \textbf{break}
    \ENDIF
\ENDFOR
\STATE Return the sequence with the highest score from $\mathcal{B}$

\end{algorithmic}
\end{algorithm}



\Cref{alg:beam} is a greedy search algorithm that uses a PRM select the best CoT during search. More details are given in \Cref{sec:inference-time-methods}.


\clearpage


\begin{algorithm}[ht]
\caption{Monte Carlo Tree Search with Process Reward Model}
\label{alg:mcts}
\begin{algorithmic}[1]
    \REQUIRE Large Language Model $\text{LLM}(\cdot)$, Process Reward Model $\text{PRM}(\cdot)$, Prompt $s_0$, Maximum step length $L$, Number of roll-outs $K$, Number of generated child nodes $d$, Exploration weight $w$
    \STATE Initialize the value function $Q : \mathcal S \mapsto \mathbb R$ and
    visit counter $N : \mathcal S \mapsto \mathbb N$ 
    \FOR {$n \gets 0, \dots, K - 1$}
        \STATE // \textit{Selection}
        \STATE $t \gets 0$
        \WHILE {$s_t$ is not a leaf node}
            \STATE $N(s_t) \gets N(s_t) + 1$ 
            \STATE $s_{t+1} \gets \arg\max_{\text{children}(s_t)} \left[ Q(\text{child}(s_t)) + w \sqrt{\frac{\ln N(s_t)}{N(\text{child}(s_t))}} \right]$
            \STATE $t \gets t + 1$
        \ENDWHILE
        \STATE // \textit{Expansion \& Simulation} (equivalent to the beam search with $N=M=d$)
        \STATE $\mathcal{B} \gets [s_t]$
        \WHILE {$s_t$ is not a terminal leaf node $\wedge$ $t \leq L$}
            \STATE $N(s_t) \gets N(s_t) + 1$
            \STATE $\mathcal{B} \gets \text{Expand}(\mathcal{B}, d)$
            \STATE $\mathcal{B} \gets \text{LLM.step}(\mathcal{B})$
    
        \FOR {$s \in \mathcal{B}$} 
            \STATE $Q(s) \gets \text{Aggr}(s)$
            %\STATE $N(s) \gets N(s) + 1$
            %\STATE $Q(s) \gets \text{PRM}(s)$ // or $\text{Aggr}(s)$
        \ENDFOR
        
            %\STATE $\texttt{best\_idx} \gets$ Index of the highest score in $\mathcal{S}$
            \STATE $s_{t+1} \gets \arg\max_{s \in \mathcal{B}} Q(s)$ 
            \STATE $t \gets t + 1$ 
            \STATE $\mathcal{B} \gets [s_t]$
        \ENDWHILE
        \STATE // \textit{Back Propagation}
        \FOR {$t' \gets t, \dots, 0$}
            \STATE $Q(s_{t'}) \gets \max (Q(s_{t'}), Q(s_{t}))$
        \ENDFOR
    \ENDFOR
    \STATE Return the sequence with the highest score among the terminal nodes
\end{algorithmic}
\end{algorithm}



\Cref{alg:mcts} is a tree-based search algorithm that iteratively expands a search tree to find the CoT with the highest PRM score. MCTS iteratively builds a search tree through the following steps:
\begin{enumerate}
    \item \textbf{Selection}: Starting from the root node, the algorithm traverses the tree by selecting child nodes according to a selection policy.
    \item \textbf{Expansion and Simulation}: Upon reaching a non-terminal leaf node, the tree is expanded iteratively by generating a fixed number of child nodes and then greedily selecting the child node with the highest value (which for us is determined by the PRM). This process continues until a terminal node is reached.
    \item \textbf{Backpropagation}: The results from the simulation are propagated back through the tree, updating value estimates and visit counts for each node along the path.
\end{enumerate}
These steps are repeated for a fixed number of iterations or until a computational or time limit is reached. To determine the final prediction, we choose the terminal node with the highest value.


\clearpage


\section{Additional PRM Training Details}
\label{sec:add-prm-train}
\subsection{Open-Source Math PRM Training Details}
\label{sec:open-mathprm}


The open-source PRMs evaluated in this work utilize CoT training data derived from two mathematical datasets:
MATH~\citep{hendrycks2measuring} and GSM8K~\citep{cobbe2021training}.
The Math-Shepherd and RLHFlow/Deepseek-PRM-Data datasets are synthetically generated following the rollout method proposed by~\citet{wang2024math}.
Similarly, the MATH-APS dataset is produced using the synthetic generation technique introduced by~\citet{luo2024improve}. PRM800K, in contrast, consists of manually annotated labels and was specifically curated for the study by~\citet{lightman2023let}.


All PRMs are trained using the base LLMs of comparable model size and class,
including Mistral-7B~\citep{jiang2023mistral},
Llama-3.1-8B-Instruct~\citep{dubey2024llama},
and Qwen-2.5-Math 8B~\citep{yang2024qwen2}.


\begin{table*}[ht]
    \centering
    \caption{Training details of various Math PRMs}
    \resizebox{\linewidth}{!}
    {
    \begin{tabular}{lccr}
    \toprule
    \textbf{PRM} & \textbf{Base Model} & \textbf{Training Data} & \textbf{Training Method} \\
    \midrule
    Math-PSA & Qwen-2.5-Math-7B-Instruct & PRM800K, Math-Shepherd and MATH-APS & LoRA \\
    Math-Shepherd & Mistral-7B & Math-Shepherd & Full fine-tuning \\
    Qwen-2.5-Math-PRM & Qwen-2.5-Math-7B-Instruct & PRM800K & Full fine-tuning \\
    RLHFLow-Deepseek & Llama3.1-8B-Instruct & RLHFlow/Deepseek-PRM-Data & Full fine-tuning \\
    \midrule
    LlamaPRM800K & Llama3.1-8B-Instruct & PRM800K & Full fine-tuning \\
    QwenPRM800K & Qwen-2.5-Math-7B-Instruct & PRM800K & Full fine-tuning \\
    \bottomrule
    \end{tabular}
    }
    \label{tab:math_prm_details}
\end{table*}


\subsection{Details of PRM Training}
\label{sec:prm-train}




For training, we extract logits from the tokens \texttt{+} and \texttt{-} in the final layer of the LLM. The logit for \texttt{+} corresponds to a correct reasoning step, while the logit for \texttt{-} represents an incorrect step. We use four newline characters \texttt{\textbackslash n\textbackslash n\textbackslash n\textbackslash n} as the classification token, which is appended to the end of each reasoning step. We use standard cross-entropy loss and only compute it over our classification token.

For training our math PRMs on the PRM800K dataset (QwenPRM800K and LlamaPRM800K),
we employ a batch size of 128 and perform full fine-tuning. For experiments on mixed-domain datasets, we reduce the batch size to 32 due to smaller dataset size.

All training is conducted over a single epoch. For full fine-tuning, we use a learning rate of $1.25 \times 10^{-6}$, while for LoRA-based fine-tuning, we use a learning rate of $1.0 \times 10^{-4}$.


\clearpage

\section{Additional PRM Training and Evaluation Experiments}
\subsection{Evaluation Results for Math PRMs and~\ourprm~Across all Categories}
\label{sec:mathprm-fullevals}





\begin{table*}[ht]
\caption{Comparison among various math PRMs and~\ourprm~on different domains in~\ourdataeval~when using WMV with min-aggregation on $N=16$ CoTs generated per question using Llama3.1-8B-Instruct. In parenthesis we report the relative difference between WMV and the MV baseline (WMV$-$MV). While WMV using math PRMs exhibit greater improvement in math and math-adjacent domains, there is no significant improvement on MV in other domains.}
\small
\centering
\setlength{\tabcolsep}{4pt}
\resizebox{\textwidth}{!}{
\begin{tabular}{l|c|cccccc}
\toprule
\textbf{Category} & \textbf{MV (Baseline)} & \textbf{Math-PSA} & \textbf{Math-Shepherd} & \textbf{Qwen-2.5-Math-PRM} & \textbf{RLHFLow-Deepseek} & \textbf{LlamaPRM800K} & \textbf{\ourprm} \\
\midrule
All & 57.15 & 57.87(+0.72) & 57.66(+0.51) & 58.17(+1.02) & 57.59(+0.44) & 58.16(+1.01) & \textbf{61.22(+4.07)} \\
All except math & 56.61 & 56.82(+0.21) & 57.01(+0.40) & 57.32(+0.71) & 56.96(+0.35) & 57.71(+1.10) & \textbf{60.29(+3.68)} \\
Math & 62.40 & 64.20(+1.80) & 64.13(+1.73) & 67.20(+4.80) & 64.07(+1.67) & 65.40(+3.00) & \textbf{68.87(+6.47)} \\
Math-Adjacent & 56.75 & 57.98(+1.23) & 57.48(+0.73) & 58.30(+1.55) & 57.33(+0.58) & 58.27(+1.52) & \textbf{61.22(+4.47)} \\
Non-Math-Adjacent & 56.69 & 56.79(+0.10) & 57.14(+0.45) & 57.09(+0.40) & 57.02(+0.33) & 57.55(+0.86) & \textbf{60.00(+3.31)} \\
\midrule
Chemistry & 58.67 & 60.47(+1.80) & 60.13(+1.46) & 60.67(+2.00) & 59.13(+0.46) & 60.47(+1.80) & \textbf{66.13(+7.46)} \\
Computer Science & 55.80 & 56.93(+1.13) & 56.07(+0.27) & 56.13(+0.33) & 56.07(+0.27) & 56.40(+0.60) & \textbf{58.60(+2.80)} \\
Engineering & 51.67 & 50.67(-1.00) & 51.07(-0.60) & 53.13(+1.46) & 51.87(+0.20) & 52.27(+0.60) & \textbf{55.27(+3.60)} \\
Physics & 58.53 & 61.87(+3.34) & 61.87(+3.34) & 61.47(+2.94) & 60.80(+2.27) & 61.47(+2.94) & \textbf{64.87(+6.34)} \\
\midrule
Biology & 75.38 & 75.23(-0.15) & 75.38(+0.00) & 75.69(+0.31) & 75.77(+0.39) & 76.38(+1.00) & \textbf{80.00(+4.62)} \\
Health & 63.36 & 63.00(-0.36) & 63.93(+0.57) & 63.50(+0.14) & 63.57(+0.21) & 64.50(+1.14) & \textbf{65.50(+2.14)} \\
Psychology & 61.60 & 61.47(-0.13) & 61.47(-0.13) & 62.27(+0.67) & 61.47(-0.13) & 61.87(+0.27) & \textbf{64.53(+2.93)} \\
Business & 61.34 & 61.95(+0.61) & 62.21(+0.87) & 63.02(+1.68) & 62.21(+0.87) & 62.62(+1.28) & \textbf{64.50(+3.16)} \\
Economics & 62.00 & 62.67(+0.67) & 62.33(+0.33) & 62.53(+0.53) & 62.67(+0.67) & 62.40(+0.40) & \textbf{64.27(+2.27)} \\
Law & 35.93 & 35.72(-0.21) & 37.24(+1.31) & 36.28(+0.35) & 36.07(+0.14) & 36.90(+0.97) & \textbf{43.86(+7.93)} \\
History & 49.20 & 49.00(-0.20) & 49.87(+0.67) & 49.40(+0.20) & 49.40(+0.20) & 49.87(+0.67) & \textbf{50.67(+1.47)} \\
Philosophy & 44.83 & 44.90(+0.07) & 44.70(-0.13) & 45.17(+0.34) & 44.56(-0.27) & 45.30(+0.47) & \textbf{49.13(+4.30)} \\
Other & 55.53 & 55.80(+0.27) & 55.47(-0.06) & 56.07(+0.54) & 55.87(+0.34) & 57.07(+1.54) & \textbf{59.00(+3.47)} \\
\bottomrule
\end{tabular}
}
\vskip -0.1in
\label{tab:math_prm_on_multi_domain}
\end{table*}


\subsection{PRM Training with Counterfactual Augmented Data}

\begin{figure*}[ht]
    \begin{center}
        \includegraphics[width=\linewidth]{figures/new_figures/prm_noaugs_vs_augs_min_agg.pdf}        
        \caption{Comparison of WMV (top) and BoN (bottom) using our two multi-domain PRMs (w/ and w/o counterfactually augmented training data) on the categories of~\ourdataeval. We use min-aggregation and the CoTs are generated using Llama-3.1-8B-Instruct. When using WMV, counterfactual augmented data can further improve the performance of PRM on non-math-adjacent domains.}
        \label{fig:counteraug}
    \end{center}
\end{figure*}


\clearpage


\subsection{WMV and BoN using different aggregation methods}
\label{sec:agg-comp3}


\begin{figure*}[ht]
    \begin{center}
        \includegraphics[width=\linewidth]{figures/new_figures/prm_diff_agg.pdf}        
        \caption{Comparison of WMV (left) and BoN (right) using \ourprm~with different reward aggregations on~\ourdataeval. The CoTs are generated using Llama 3.1 8B Instruct. Overall, min-aggregation brings the largest inference performance boost.}
        \label{fig:prm-diff-agg2}
    \end{center}
\end{figure*}


\clearpage


\subsection{Larger Generator Inference with PRM Rewarding}
\label{sec:agg-comp}


\begin{figure*}[ht]
    \begin{center}
        \includegraphics[width=\linewidth]{figures/new_figures/prm_on_70b_cot.pdf}        
        \caption{Comparison of WMV (left) and BoN (right) using \ourprm~against math PRMs on~\ourdataeval. We use min-aggregation and the CoTs are generated using Llama-3.1-70B-Instruct. Similar trends to using 8B model as the generator are observed, indicating that our Multi-Domain PRM can generalize across generators with different capacities.}
        \label{fig:prm-diff-agg}
    \end{center}
\end{figure*}


\clearpage


\subsection{Inference with Compact PRM}
\label{sec:agg-comp2}


\begin{figure*}[ht]
    \begin{center}
        \includegraphics[width=\linewidth]{figures/new_figures/prm_with_3b_min_agg.pdf}        
        \caption{Comparison of WMV (top) and BoN (bottom) using \ourprm~(Llama3B Base), a compact PRM based on Llama-3.2-3B-Instruct and trained on our multi-domain dataset. We use min-aggregation and the CoTs are generated using Llama-3.1-8B-Instruct. Compared with using \ourprm~(Llama Base), which applies the same training data and configurations but is based on Llama-3.1-8B-Instruct, \ourprm~(Llama3B Base) brings a less significant performance boost. However, the overall trends are similar, indicating the efficacy of the inference pipeline using PRM.}
        \label{fig:prm-diff-agg1}
    \end{center}
\end{figure*}

\clearpage


\subsection{Comparison of~\ourprm~against Other Open-Source Math PRMs on WMV and BoN by Category}
\label{sec:bon-mv-bycat}


\begin{figure*}[ht]
    \begin{center}
        \includegraphics[width=\linewidth]{figures/new_figures/prm_all_domains_wmv_min_agg.pdf}        
        \caption{Comparison of WMV using \ourprm~against open-source PRMs on more other categories of~\ourdataeval. We use min-aggregation and the CoTs are generated using Llama-3.1-8B-Instruct.}
        \label{fig:prm-wmv-more-domains}
    \end{center}
\end{figure*}


\begin{figure*}[ht]
    \begin{center}
        \includegraphics[width=\linewidth]{figures/new_figures/prm_all_domains_bon_min_agg.pdf}        
        \caption{Comparison of BoN using \ourprm~against open-source PRMs on more other categories of~\ourdataeval. 
        We use min-aggregation and the CoTs are generated using Llama-3.1-8B-Instruct.}
        \label{fig:prm-bon-more-domains1}
    \end{center}
\end{figure*}


\clearpage

\subsection{Comparison of~\ourprm~against Other Open-Source Math PRMs on MCTS and Beam Search by Category}
\label{sec:mcts-detailed}


\begin{figure*}[ht]
    \begin{center}
        \includegraphics[width=\linewidth]{figures/new_figures/all_domains_mcts_vs_beam.pdf}        
        \caption{Comparison of \ourprm~and LlamaPRM800K
        with beam search and MCTS. In more other categories from~\ourdataeval, \ourprm~achieves better performance.}
        \label{fig:prm-wmv-more-domains2}
    \end{center}
\end{figure*}


\end{document}


% This document was modified from the file originally made available by
% Pat Langley and Andrea Danyluk for ICML-2K. This version was created
% by Iain Murray in 2018, and modified by Alexandre Bouchard in
% 2019 and 2021 and by Csaba Szepesvari, Gang Niu and Sivan Sabato in 2022.
% Modified again in 2023 and 2024 by Sivan Sabato and Jonathan Scarlett.
% Previous contributors include Dan Roy, Lise Getoor and Tobias
% Scheffer, which was slightly modified from the 2010 version by
% Thorsten Joachims & Johannes Fuernkranz, slightly modified from the
% 2009 version by Kiri Wagstaff and Sam Roweis's 2008 version, which is
% slightly modified from Prasad Tadepalli's 2007 version which is a
% lightly changed version of the previous year's version by Andrew
% Moore, which was in turn edited from those of Kristian Kersting and
% Codrina Lauth. Alex Smola contributed to the algorithmic style files.
