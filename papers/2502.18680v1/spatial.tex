This section analyzes spatial imbalance in multi-GPU jobs.
To determine an appropriate time window length for Equation 2, we evaluated
time windows of 1, 5, 15, and 30 minutes and found that a 1-minute window
enables us to identify patterns and issues most clearly in the data.

We first examine jobs with low (\textless30\%), medium (30\%-70\%), and
high (\textgreater70\%) mean of GPU\_UTIL. Next, we investigate
whether the spatial imbalance arises from multiple underutilized GPUs or a
single GPU driving the imbalance. Finally, we explore the impact of the GPU core used
and job type (ML vs non-ML) on spatial imbalance. This section addresses
the following question:
%
\begin{RQcallout}
    {\bf RQ3 }{\it \RQthree{}}
\end{RQcallout}

Figure~\ref{fig:gputil_spatial} shows the spatial imbalance distribution of
GPU\_UTIL across different mean counter value ranges. Low-utilization jobs
(left plot) exhibit the highest imbalance (peaking at 0.75),
which indicates a significant unevenness in resource distribution. 45.7\% of these
jobs have an imbalance below 0.5. Medium-utilization jobs (middle plot)  are more
balanced but still include both balanced and imbalanced jobs. 83\% of these jobs have
imbalance below 0.5. High-utilization jobs (right plot) have less spatial imbalance
and most of them are concentrated in the lower range. 97.6\% fall below 0.5 which
suggests that higher mean of GPU\_UTIL is associated with more uniform utilization
across GPUs.

In addition to spatial imbalance, we analyze the coefficient of variation (CV)
of GPU\_UTIL across GPUs for jobs with a spatial imbalance of GPU\_UTIL greater
than 0.5. CV quantifies variability relative to the mean. It enables comparison
across jobs with different mean of GPU\_UTIL levels. It is defined as:
$
    \mathit{CV} = \frac{\sigma}{\mu}\times 100
$, where $\sigma$ is the standard deviation and $\mu$ is the mean.

\begin{figure}[h]
    % \vspace{-0.1in}
    \centering
    % \includegraphics[width=0.65\columnwidth]{figs/imbalance/spatial/spatial_burstiness_05.pdf}
    \includegraphics[width=0.65\columnwidth]{figs/imbalance/spatial/gputil_spatial_cv.pdf}
    \includegraphics[width=0.65\columnwidth]{figs/imbalance/spatial/spatial_proportion_underutilized.pdf}
    \caption{
        The plots show CV of total GPU\_UTIL (top) and underutilized
        GPU proportions (bottom) for the jobs with spatial imb. of GPU\_UTIL value
        of greater than 0.5. 52.4\% exceeded a CV of 100\%, and 51\% of jobs have
        75\% underutilized GPU, mostly allocating four GPUs but using only one.}
    \label{fig:spatial_extra}
\end{figure}

The top plot in Figure~\ref{fig:spatial_extra} shows the CV of
total GPU\_UTIL across GPUs for each job.
52.4\% of jobs with high spatial imbalance (\textgreater0.5) have a CV exceeding
100\%. This indicates substantial variation, where
some GPUs are more utilized than others.

As a complementary analysis, we
examine the proportion of underutilized GPUs in jobs with high spatial
imbalance (\textgreater0.5). A GPU is considered underutilized if its total
utilization is at least 50\% lower than the maximum total utilization among GPUs
assigned to the same job. The bottom plot in Figure~\ref{fig:spatial_extra} shows
that 51\% of jobs have 75\% underutilized GPU. We also found that among these jobs,
96\% allocate four GPUs (one node) but actively use only one.

\begin{figure}[h]
    % \vspace{-0.1in}
    \centering
    \includegraphics[width=0.65\columnwidth]{figs/imbalance/spatial/precision_type_spatial_imbalance.pdf}
    % \includegraphics[width=0.65\columnwidth]{figs/imbalance/spatial/precision_type_spatial_imbalance_tensor.pdf}
    \includegraphics[width=0.65\columnwidth]{figs/imbalance/spatial/job_type_spatial_imbalance.pdf}
    \caption{The plots show spatial imbalance of GPU\_UTIL
        across GPU core categories (top) and job type (bottom).
        FP32 and FP64 jobs (without tensor cores) show higher imbalance.
        ML jobs have a flatter distribution, while non-ML jobs cluster
        around 0.5 imbalance.}
    \label{fig:spatial_imbalance_type}
\end{figure}

Figure~\ref{fig:spatial_imbalance_type} presents the spatial
imbalance of GPU\_UTIL across two dimensions: GPU cores (top) and job
type (bottom). Dense areas
indicate a higher concentration of jobs. We normalize the
density of each violin so that the total area remains consistent across all
violins. The overall area is comparable across different categories.

\begin{figure*}[t]
    % \vspace{-0.1in}
    \centering
    Histogram and CDF of temporal imbalance of GPU\_UTIL \par\medskip
    \vspace{-0.05in}
    \includegraphics[width=0.65\columnwidth]{figs/imbalance/temporal/gputil_temporal_imbalance_30.pdf}
    \includegraphics[width=0.65\columnwidth]{figs/imbalance/temporal/gputil_temporal_imbalance_30_70.pdf}
    \includegraphics[width=0.65\columnwidth]{figs/imbalance/temporal/gputil_temporal_imbalance_70.pdf}
    \caption{The plots show distribution of temporal imbalance of GPU\_UTIL for
        jobs grouped by mean of GPU\_UTIL ranges (0–30\%, 31–69\%, 70–100\%, left to
        right). The distribution of imbalance shifts toward lower values as mean
        GPU\_UTIL increases.}
    \label{fig:gputil_temporal}
\end{figure*}

\begin{figure*}[t]
    % \vspace{-0.1in}
    \centering
    Histogram and CDF of burstiness of GPU\_UTIL \par\medskip
    \vspace{-0.05in}
    \includegraphics[width=0.65\columnwidth]{figs/burstiness/burstiness_gputil_30.pdf}
    \includegraphics[width=0.65\columnwidth]{figs/burstiness/burstiness_gputil_30_70.pdf}
    \includegraphics[width=0.65\columnwidth]{figs/burstiness/burstiness_gputil_70.pdf}
    \caption{The plots show distribution of burstiness of GPU\_UTIL for
        jobs grouped by mean GPU\_UTIL ranges (0–30\%,
        31–69\%, 70–100\%, left to right). High-utilization jobs show more irregular bursts
        compared to low- and medium-utilization jobs.}
    \label{fig:burstiness_gputil}
\end{figure*}

The top plot shows violin plots of spatial imbalance
of different GPU core counters, including FP16, FP32, FP64,
Tensor, and their combinations. FP16* includes all jobs that use FP16
core, regardless of whether they also use FP32 or FP64 cores. In contrast,
all other categories represent jobs that exclusively use the specified core(s).
These categories are derived from Figure~\ref{fig:precision_upsetplot}.
% The distribution of spatial imbalance across different categories is less
% distinct compared to temporal imbalance. 

The spatial imbalance is generally low, except for FP32 and FP64 (without
tensor cores). Jobs in the FP32 (without tensor) category are concentrated
around 0.75 imbalance, aligning with the peak observed in
Figure~\ref{fig:gputil_spatial} (left). The jobs in these categories
exhibit lower imbalance when they utilize tensor cores.

The bottom plot compares the spatial imbalance between ML and non-ML jobs.
% Unlike the temporal imbalance comparison, DL/ML jobs do not exhibit better
% spatial imbalance. 
ML jobs exhibit a flatter spatial imbalance distribution, while non-ML jobs
mostly cluster around 0.5 imbalance, but extend up to 1.0. In
contrast, ML jobs reach up to 0.9 imbalance.

% \textbf{Observations: } The first observation is that the jobs with 
% low spatial imbalance tend to have higher GPU utilization. 
% Additionally, the data shows that memcopy utilization imbalance has a 
% stronger effect on GPU utilization compared to SM spatial imbalance. 
% This behavior contrasts with the findings from the temporal imbalance 
% analysis, where SM temporal imbalance had a more significant impact on 
% GPU utilization than memcopy temporal imbalance.

