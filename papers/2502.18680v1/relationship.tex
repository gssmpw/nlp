This section demonstrates the relationships between SM\_ACTV, MEM\_UTIL,
and GPU\_UTIL. It also explains the correlations among counters and imbalance
metrics. Heatmaps provide insights into how different
factors interact. This section investigates the following question:
%
\begin{RQcallout}
    {\bf RQ5 }{\it \RQfive{}}
\end{RQcallout}

\subsection{Relationship between SM\_ACTV, MEM\_UTIL and GPU\_UTIL}

We analyze the relationship between SM\_ACTV, MEM\_UTIL and GPU\_UTIL
to understand how memory transfer (MEM\_UTIL) and
computation (SM\_ACTV) components contribute to GPU\_UTIL.
Figure~\ref{fig:memcopy_sm_spatial_gputil} visualizes
GPU\_UTIL as a function of mean (left), spatial imbalance (middle),
and temporal imbalance (right) metric values of SM\_ACTV and MEM\_UTIL.
Colors represent the corresponding metric for GPU\_UTIL. Cells with no
jobs shown in orange.

In these heatmaps, each cell in the plots represents a binned value range for
SM\_ACTV and MEM\_UTIL. For instance, in the left plot, we divide the mean
counter values into 10 bins. Jobs with mean counter values falling within a
specific bin are grouped into the
corresponding cell. The color of each cell represents the mean GPU\_UTIL of all
jobs within that bin.

% To assess consistency, we compute mean GPU\_UTIL for each cell and
% evaluate its stability using the Standard Error of the Mean (SEM). SEM
% is defined as: $\text{SEM} = \frac{\sigma}{\sqrt{n}}$, where $\sigma$ is the
% standard deviation and $n$ is the sample size.
% Since SEM values are consistently low, we do not report them in this paper.

The left plot in Figure~\ref{fig:memcopy_sm_spatial_gputil} shows
that high mean of GPU\_UTIL (dark blue) occurs when both
mean SM\_ACTV and MEM\_UTIL are high, which indicates efficient use
of both compute and memory transfer resources. The cells with
mean GPU\_UTIL value of greater than 70\% are annotated.

The middle plot in Figure~\ref{fig:memcopy_sm_spatial_gputil} presents spatial
imbalance using the same axes. Annotated cells represent spatial imbalance of
GPU\_UTIL values below 0.3. Unlike the middle plot, the brighter cells are mostly
clustered around the low spatial imbalance of MEM\_UTIL values (\textless0.2) with
a few exceptions on top left. As spatial imbalance of MEM\_UTIL increases
(moving up the y-axis), spatial imbalance in GPU\_UTIL decreases,
even when spatial imbalance in SM\_ACTV remains low. However, when the
spatial imbalance of MEM\_UTIL is low (\textless0.2), the spatial
imbalance of GPU\_UTIL remains relatively more stable as the spatial
imbalance of SM\_ACTV increases. This suggests that spatial imbalance of
MEM\_UTIL has a stronger effect on spatial imbalance of GPU\_UTIL.

The right plot in Figure~\ref{fig:memcopy_sm_spatial_gputil} visualizes
temporal imbalances of SM\_ACTV and MEM\_UTIL, colored by temporal imbalance
of GPU\_UTIL. Cells with mean temporal imbalance of GPU\_UTIL values
below 0.3 are annotated. As SM\_ACTV increases (moving right along the x-axis),
the temporal imbalance of GPU\_UTIL also increases (darker colors), regardless of
MEM\_UTIL. However, increasing temporal imbalance of MEM\_UTIL (moving up) has
a weaker effect on GPU\_UTIL. This suggests that temporal imbalance
of SM\_ACTV has a stronger influence on GPU\_UTIL.

\subsection{Correlation among Hardware Counters}

Figure~\ref{fig:correlations} presents three correlation
heatmaps examining relationship between mean counters values (left),
spatial imbalances (middle), and temporal imbalances (right). We calculate
correlations using jobs with at least one active precision counter.
Values are ranged from -1 (perfect negative correlation) to +1 (perfect
positive correlation), and 0 indicates no monotonic relationship.

The left heatmap in Figure~\ref{fig:correlations} shows correlations
between mean counter values per job. Mean of SM\_ACTV
has the strongest correlations with GPU\_UTIL (0.72), while MEM\_UTIL
exhibits a relatively lower correlation (0.59). This indicates that
computation activity plays a more direct role in mean GPU\_UTIL compared
to data transfer activity (from GPU to CPU or vice versa). HBM usage (0.46)
and GPU cores show weaker correlations, which suggests that
they are less critical in determining mean of GPU\_UTIL.

The middle heatmap in Figure~\ref{fig:correlations} demonstrates spatial imbalance
correlations. Spatial imbalance of GPU\_UTIL is strongly correlated with MEM\_UTIL
(0.86) and slightly less with SM\_ACTV (0.77), consistent with
Figure~\ref{fig:memcopy_sm_spatial_gputil}. Spatial imbalance of SM\_ACTV
has high correlations with spatial imbalance of tensor (93\%), FP16 (91\%),
FP32 (97\%), and FP64 core usage (97\%). Spatial imbalance of MEM\_UTIL is more
strongly correlated with FP32 (0.92) than other precisions. Spatial imbalance of
FP16\_ACTV shows stronger correlations with other counters than its temporal
imbalance.

The right heatmap in Figure~\ref{fig:correlations} examines temporal imbalance
correlations. Temporal imbalance of GPU\_UTIL is strongly correlated
with temporal imbalance of SM\_ACTV (0.89) and MEM\_UTIL (0.83). This
indicates that fluctuations in GPU\_UTIL closely related to variations in compute
and memory transfer activity. Temporal imbalance of SM\_ACTV also shows
high correlations with FP64\_ACTV (0.93) and FP32 (0.85).
Correlation between temporal and spatial imbalance of FP16\_ACTV and FP64\_ACTV
cannot be calculated (black cells) due to insufficient data, as only two jobs use both.
