In this section, we present the temporal behavior of GPU workloads on
Perlmutter focusing on two key metrics: temporal imbalance and burstiness. This
section investigates the following question:
%
\begin{RQcallout}
    {\bf RQ4 }{\it \RQfour{}}
\end{RQcallout}

\subsection{Temporal Imbalance}
\label{sub:temporal-imbalance}

This section examines the distribution of temporal imbalance, impact
of GPU core used, and compare imbalance of ML and non-ML jobs.
Figure~\ref{fig:gputil_temporal} shows the temporal imbalance distribution
of GPU\_UTIL for jobs categorized by mean of GPU\_UTIL: low
(\textless30\%), medium (30\%-70\%), and high (\textgreater70\%).

The left plot in Figure~\ref{fig:gputil_temporal} shows that low-utilization jobs
form two main groups: those with very low imbalance (\textless0.3) and those with
very high imbalance (\textgreater0.7). This suggests that some jobs in this range
consistently use minimal GPU resources, while others exhibit sporadic bursts of
activity or long idle periods. Medium-utilization jobs (middle plot) are more evenly
distributed around 0.5, suggesting mix of steady and fluctuating jobs. High-utilization
exhibit lower imbalance. This indicates that consistent GPU activity is
associated with higher mean counter values, while uneven workloads are more common in
lower utilization ranges.

\begin{figure}[h]
    % \vspace{-0.1in}
    \centering
    \includegraphics[width=0.65\columnwidth]{figs/imbalance/temporal/precision_type_temporal_imbalance.pdf}
    % \includegraphics[width=0.65\columnwidth]{figs/imbalance/temporal/precision_type_temporal_imbalance_tensor.pdf}
    \includegraphics[width=0.65\columnwidth]{figs/imbalance/temporal/job_type_temporal_imbalance.pdf}
    \caption{The plots show temporal imbalance of GPU\_UTIL
        across GPU core categories (top) and job type (bottom).
        FP64 (without tensor) have highest imbalance. FP16 and
        FP32 (with tensor) show lower imbalance than their non-tensor
        counterparts. ML jobs exhibit lower temporal imbalance.}
    \label{fig:imbalance_type}
\end{figure}

Figure~\ref{fig:imbalance_type} shows the temporal imbalance distribution of
GPU\_UTIL for different GPU core combinations (top) and job types (bottom).
Each violin plot shows the spread and density of temporal imbalance values for
jobs utilizing specific counter combinations. We apply the same
normalization as explained in Figure~\ref{fig:spatial_imbalance_type}.

The top plot in Figure~\ref{fig:imbalance_type} presents violin plots of
temporal imbalance of GPU\_UTIL for jobs utilizing specific floating-point
and tensor core combinations. FP32 and FP32+FP64 (without tensor cores)
show higher density around 0.2, while jobs in the same categories
exhibit slight skew towards higher imbalance when using tensor cores.
FP64 (without tensor) have highest imbalance. FP16 and
FP32 (with tensor) show lower imbalance than their non-tensor
counterparts.

The bottom plot in Figure~\ref{fig:imbalance_type} compares temporal imbalance
of GPU\_UTIL between ML and non-ML jobs. Non-ML jobs have a higher density at
high imbalance values (\tweakedsim 0.9). In contrast, ML jobs have a higher
density around \tweakedsim 0.2. This suggests that non-ML jobs exhibit more
variability in GPU\_UTIL over time.

\begin{table*}[t]
    \centering
    \caption{Categorization of jobs by temporal imbalance and burstiness of
        GPU\_UTIL (low and high). The table includes the
        total number of jobs, total GPU hours, and the proportion of jobs
        with low (\textless30\%), medium (30\%-70\%), and high (\textgreater70\%) mean
        of GPU\_UTIL in each category.}
    \label{tab:burst-imbalance}
    \begin{tabular}{l|c|c|c|c}
        \hline
        \textbf{Burstiness}             & \textbf{Temporal Imb. of GPU\_UTIL} & \textbf{Number of Jobs} & \textbf{Total GPU Hours} & \textbf{Proportion of Jobs (Low/Med./High)} \\
        \hline
        \multirow{2}{*}{Low $(\leq 0)$} & Low $(\leq 0.5)$                    & 104,730                 & 1,462,583                & 32/42/26\%                                  \\
                                        & High $(> 0.5)$                      & 145,971                 & 2,657,522                & 54/45/1\%                                   \\
        \hline
        \multirow{2}{*}{High $(> 0)$}   & Low $(\leq 0.5)$                    & 55,639                  & 2,865,362                & 16/41/43\%                                  \\
                                        & High $(> 0.5)$                      & 38,603                  & 3,534,739                & 75/24/1\%                                   \\
        \hline
    \end{tabular}
\end{table*}

\subsection{Burstiness over Time}

In this section, we analyze burstiness for jobs with
low(\textless30\%), medium (30\%-70\%), and high (\textgreater70\%) mean of
GPU\_UTIL. To characterize burstiness, we define bursts as an
increase of a counter value exceeding 15\% from the previous value,
as detailed in Section~\ref{sec:method}.

Figure~\ref{fig:burstiness_gputil} presents burstiness distributions of GPU\_UTIL
across the three mean of GPU\_UTIL ranges. The left plot shows that
85.5\% of jobs with low mean of GPU\_UTIL have a burstiness value below 0. This
indicates that jobs in this category exhibit mostly regular fluctuations.
Jobs with values near -1 show consistent, steady interevent times with regular
GPU usage or idling patterns. The lack of positive burstiness
suggests these jobs do not exhibit irregular bursts of GPU activity, likely due to
their light workload demands.

Medium-utilization jobs (middle plot) have a more spread-out
distribution but still remain centered around 0. 82\% of medium-utilization
jobs still have a burstiness value below 0. It indicates a mix of steady and
moderately irregular bursts. High-utilization jobs (right plot) show a slight
skew towards positive burstiness. 39\% of high-utilization jobs have a burstiness
value above 0. This suggests a greater irregularity among high-utilization jobs.

\begin{figure}[h]
    % \vspace{-0.1in}
    \centering
    \includegraphics[width=0.65\columnwidth]{figs/imbalance/spatial/spatial_burstiness_05.pdf}
    % \includegraphics[width=0.65\columnwidth]{figs/imbalance/spatial/gputil_spatial_cv.pdf}
    % \includegraphics[width=0.65\columnwidth]{figs/imbalance/spatial/spatial_proportion_underutilized.pdf}
    \caption{
        The plot show burstiness in spatial imbalance of GPU\_UTIL for jobs with
        spatial imbalance value of greater than 0.5. Bursts in spatial imbalance
        is mostly regular.}
    \label{fig:spatial_burstiness}
\end{figure}

Our spatial imbalance metric is averaged across time windows. Therefore, it
allows us to apply burstiness to measure irregular fluctuations in spatial
imbalance over a job's lifetime. A burst in spatial imbalance is defined as
a change in imbalance exceeding 0.15 from the previous time window. We analyze
burstiness in spatial imbalance of GPU\_UTIL for jobs with high spatial
imbalance (\textgreater0.5).

Figure~\ref{fig:spatial_burstiness} shows that burstiness in spatial imbalance
peaks around 0, and 8.2\% of jobs are close to -1. This indicates regular
fluctuations. Although some GPUs are heavily loaded compared to
others (Figure~\ref{fig:spatial_extra}), significant changes in spatial imbalance
occur at regular intervals rather than irregularly.

\subsection{Categorizing Jobs by Burstiness and Temporal Imbalance}

We further investigate the interaction between burstiness and temporal imbalance
of GPU\_UTIL by categorizing jobs into a matrix based on burstiness (low
or high) and temporal imbalance (low or high) metrics, as shown in
Table~\ref{tab:burst-imbalance}. We also examine the proportion of jobs
with low (\textless30\%), (30\%-70\%), and high (\textgreater70\%) mean
of GPU\_UTIL in each category This analysis provides deeper insights into
the temporal behavior of GPU workloads.

\begin{figure*}[t]
    \centering
    % GPU\_UTIL Patterns Based on SM\_ACTV and MEM\_UTIL \par\medskip
    % \vspace{-0.05in}
    \includegraphics[width=0.65\columnwidth]{figs/memcopy_smactive_gputil/memcopy_sm_mean_gputil_mean_heatmap_sem.pdf}
    \includegraphics[width=0.65\columnwidth]{figs/memcopy_smactive_gputil/memcopy_sm_spatial_gputil_spatial_heatmap_sem.pdf}
    \includegraphics[width=0.65\columnwidth]{figs/memcopy_smactive_gputil/memcopy_sm_temporal_gputil_temporal_heatmap.pdf}
    \caption{The plots show relationships between SM\_ACTV,
        MEM\_UTIL, and GPU\_UTIL.
        High mean of GPU\_UTIL occurs when both SM\_ACTV and MEM\_UTIL are high (left).
        Spatial imb. of MEM\_UTIL has a stronger impact on spatial imbalance of
        GPU\_UTIL (middle).
        Temporal imb. of SM\_ACTV has a stronger impact on temporal imbalance
        of GPU\_UTIL (right).
        Cells with no jobs are shown in orange.}
    \label{fig:memcopy_sm_spatial_gputil}
\end{figure*}

\begin{figure*}[t]
    % \vspace{-0.1in}
    \centering
    \includegraphics[width=0.69\columnwidth]{figs/corr/correlation_heatmap_overall.pdf}
    \includegraphics[width=0.69\columnwidth]{figs/corr/correlation_heatmap_spatial_all.pdf}
    \includegraphics[width=0.69\columnwidth]{figs/corr/correlation_heatmap_temporal_all.pdf}
    \caption{The heatmaps show Spearman correlation between derivations of counter values.
        Mean of SM\_ACTV has the strongest correlation with mean of GPU\_UTIL (left).
        Spatial imbalance of MEM\_UTIL and FP32\_ACTV has stronger effect on GPU\_UTIL (middle).
        Fluctuations over time in GPU\_UTIL are closely tied to SM\_ACTV and MEM\_UTIL (right).}
    \label{fig:correlations}
\end{figure*}

\paragraph{Low Burstiness$(\leq0)$, Low Imbalance$(\leq 0.5)$} Jobs in this
category exhibit steady usage patterns and regularly stay near to their maximum
counter value. These jobs are generally desirable for HPC systems when mean
of GPU\_UTIL is relatively high, as they consistently push the GPU. The table
shows that there are a total of 104,730 jobs in this category and 32\% of jobs
have low mean of GPU\_UTIL, 42\% have medium, and 26\% have high mean of GPU\_UTIL.
The jobs with low mean represent consistent underutilization. This
may be due to inefficient GPU usage or the minimal demand of the jobs.

\paragraph{Low Burstiness$(\leq0)$, High Imbalance$(>0.5)$} Jobs in this
category are relatively steady but it's generally far below the maximum they
can reach. Most jobs, with a total of 145,971, fall into this category. When
the mean of GPU\_UTIL is low (54\% of the jobs in this category), these jobs
often underutilize GPU resources with some minor irregular bursts. This
underutilization also might occur due to low resource demands of the jobs.

\paragraph{High Burstiness$(>0)$, Low Imbalance$(\leq 0.5)$} Jobs in this
category exhibit irregular bursts (high burstiness) but still remain close to
their maximum (low imbalance). For example, the jobs with high mean
of GPU\_UTIL fluctuate between high and very high counter values, such as
oscillating between 80\% and 100\% with irregular changes. This category is not
a major concern since the total number of jobs in this category is relatively
low (64,433) and the most jobs have medium (41\%) and high (43\%) mean of GPU\_UTIL.

\paragraph{High Burstiness$(>0)$, High Imbalance$(>0.5)$} This category
represents the most problematic jobs and possibly indicates more anomalies. The
bursts are irregular, and the jobs rarely stay near their peak. The jobs that
have low mean of GPU\_UTIL allocate GPU resources for a few intense bursts while
leaving it underutilized most of the remaining runtime. This category has
fewer jobs (38,603) but most of them have low mean of GPU\_UTIL (75\%).
