\section{Case Study}
\label{sec:casestudy}

In this section we present a case study conducted based on the framework, the case study was intentionally designed to be relatively straightforward, ensuring that it is easily understandable while primarily demonstrating the core functionality of the proposed method. 
In this case study we investigate the use case of remote controllable \ac{HEMS} systems and the impact on the grid when the central control infrastructure of the \ac{HEMS} provider gets compromised.

\subsection{Scenario Description}
\label{subsec:scenario}

The basis for the scenario are combined semi urban medium and low voltage grid from \textit{SimBench}\cite{simbench}.
This benchmark case consists of one 110 kv transformer station and 110 secondary substations connected via a 20 kV network in a ring topology.
Furthermore the low voltage grids have 8982 buses in total connected in radial topologies with 7823 households of which some also have \acp{PV}, \acp{BSS}, heat pumps and \acp{EV} connected to them, additionally commercial loads are part of the model.
The benchmark case also contains production and load time series for the different units, for this paper we analysed the time steps with the highest and lowest sum of transformer loading.
For the augmentation, every bus connected to a household load is defined as household \ac{PCC} and a \textit{errol:HouseHold} gets created, which then owns all units connected to this bus.
If the household contains any controllable loads or production a \ac{HEMS} consisting of a host, a functional actor, and the function block for remote contractility are added, as shown in figure\ref{fig:kg}.
Three different \ac{HEMS} templates for different manufacturers exist, they get added which the probability of 50\%, 30\% and 20\%.
For each of the manufacturers a backend system gets created and the function blocks of the \ac{HEMS} of this manufacturer get connected via an information object flow.
For all controllable units \textit{errol:controlValues} are created referencing the active power of the units, which then get connected to the corresponding information object flow.
Because the objective of this case study is a worst case analysis of the impact of a compromised \ac{HEMS} operator for the grid, the augmentation rules identifies all values which are controllable by a specific backend and either set them to the maximal or minimal possible value.
The underlying assumption for the \ac{BSS} is that in normal operation the state of charge is not fully utilised, because of battery health concerns, and therefore the full power can be used for a short time attack.
We analysed two attacks for each manufacturer one trying to maximise the load in the grid and one trying to minimise it.

\subsection{Result and Discussion}
\label{subsec:result}

Figure \ref{fig:pmw_diff_high} shows the difference in the loading of the transformers of the secondary substations for the high load case.
Every load gets compared to the loading in the scenario with out an attack.
"\textit{Max. 1}", for example, is the difference in load for the scenario that the attacker gains access to manufacture 1 and wants to maximise the load in the grid.
\vspace*{-0.5cm} 
\begin{figure}[H]

	\centering
	\includegraphics[width=\linewidth]{pmw_diff34422.pdf}
    \vspace*{-0.3cm} 
	\caption{
		Load difference of transformers in secondary substations for different attack scenarios in MW in the high load case.
	}
	\label{fig:pmw_diff_high}
\end{figure}
\vspace*{-0.85cm} 
\begin{figure}[H]

	\centering
	\includegraphics[width=\linewidth]{pmw_diff_13867.pdf}
    \vspace*{-0.3cm} 
	\caption{
		Load difference of transformers in secondary substations for different attack scenarios in MW in the low load case.
	}
	\label{fig:pmw_diff_low}
\end{figure}


Therefore the results for all maximising attacks are above zero and for all minimising attacks are below zero. As expected the attacks on manufacture 1 have the greatest impact in general because it was distributed with the highest probability in the augmentation process.
Nevertheless, for certain substations, this does not hold true because, by coincidence, a larger portion of the load connected to these substations is controllable by a different manufacturer.
Figure \ref{fig:pmw_diff_low} shows the corresponding results for the low load case. When comparing the results it can be seen that the maximising attacks in this case have a slightly greater effect with a load change up to 0.182 MW compared to 0.117 MW.
The delta is mainly caused by turning off the \ac{PV} production.
On the other hand the impact of the minimising attacks is lower because in this case the total power of controllable loads is smaller and \ac{BSS} with the goal of self consumption optimisation are already charging.
In general it can be stated that the impact of the attacks does not lead to problematic grid states, i.e. violation of voltage or line and transformer limits.
This is additionally supported by a specific analysis of the different voltage levels in the scenario.
Figure \ref{fig:box_high} presents all medium and low voltage levels as box plots for the different attack scenarios.
No clear deterioration can be observed, which is consistent with our experience from working with the \textit{SimBench} models, as they exhibit a high robustness to voltage deviations.
However, a study using real network models is necessary to demonstrate the specific impact on the networks.
The presented results nevertheless showcase the functionality of the developed framework and the feasibility of it for these kind of analysis.
\vspace*{-0.4cm} 
\begin{figure}[H]
	\centering
	\includegraphics[width=\linewidth]{vol_box34422.pdf}
	\caption{
		Voltages of all busses differentiated by different voltage levels as box plots. For the different attacks scenarios in the high load case.
	}
	\label{fig:box_high}
\end{figure}