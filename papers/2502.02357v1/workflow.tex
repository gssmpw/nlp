\section{Workflow of Impact Analysis for Cyber-Attacks}

Figure \ref{fig:errolflow} shows the workflow of the impact analysis developed in this work.
The starting point for the investigations done in this work are \textit{PandaPower}\cite{pandapower_2018} models, which get imported into the presented graph format.
The power system model than gets validated against the \ac{SHACL} shapes for the power system.

Following this the control infrastructure is augmented automatically based on predefined augmentation rules.
This augmentation process contains the hardware of the control infrastructure as well the function blocks needed for the application and also the information flow between these function blocks.
Furthermore, information like the manufacturer and firmware version can be added to identify the scope of a cyber-attack which may only work on a specific firmware version.
\begin{figure}[H]
\begin{tikzpicture}

\node (pp) [data, align=center] {Power System Model};
\node (import) [process, left of=pp, xshift=-\myboxwidth]{Importer};
\node (pprdf) [data, align=center, below of=import, yshift=-0.25* \myboxheigth] {Power System Model};
\node (val1) [process, below of=pprdf, yshift=-0.25* \myboxheigth]{Validation};
\node (sh1) [data, align=center, right of=val1, xshift=\myboxwidth]{Power System\\\textit{SHACL} Shapes};
\node (aug) [process, below of=val1, yshift=-0.25* \myboxheigth]{Augmentation};
\node (cpsrdf) [data, align=center, below of=aug, yshift=-0.25* \myboxheigth] {Cyber-Physical System Model};
\node (rul) [data, align=center, right of=aug, xshift=\myboxwidth]{Augmentation Rules};
\node (val2) [process, below of=cpsrdf, yshift=-0.25 * \myboxheigth]{Validation};
\node (sh2) [data, align=center, right of=val2, xshift=\myboxwidth]{Cyber-Physical System \\\textit{SHACL} Shapes};
\node (impact) [process, below of=val2, yshift=-1 * \myboxheigth]{Identifing impact by cyber-attack};
\node (cyberrul) [data, align=center, right of=impact, xshift=\myboxwidth]{Cyber Attack\\Augmentation Rules};
\node (exp) [process, below of=impact, yshift=-1 * \myboxheigth]{Export};
\node (exppp) [data, align=center, right of=exp, xshift=\myboxwidth]{Power System Model};


\draw [arrow] (pp) -- (import);
\draw [arrow] (import) -- (pprdf);
\draw [arrow] (pprdf) -- (val1);
\draw [arrow] (val1) -- (aug);
\draw [arrow] (rul) -- (aug);
\draw [arrow] (sh1) -- (val1);
\draw [arrow] (aug) -- (cpsrdf);
\draw [arrow] (cpsrdf) -- (val2);
\draw [arrow] (sh2) -- (val2);
\draw [arrow] (val2) -- (impact);
\draw [arrow] (impact) -- (exp);
\draw [arrow] (exp) -- (exppp);
\draw [arrow] (cyberrul) -- (impact);
  
\end{tikzpicture}
    \centering
    \caption{The workflow starting from a \textit{PandaPower} model showing the different steps, including the augmentation step, impact analysis, and export.}
    \label{fig:errolflow}
\end{figure}

The augmentation rules allow probabilistic distributions, for example equip 30 percent of electric vehicles with a specific \ac{HEMS} than connect these to a backend system depending on the manufacturer of the \ac{EV}.
\ac{SPARQL} is the foundation of the augmentation engine, which allows the definition of augmentation rules in an declarative way which which is a significantly more effective approach than solely imperative implementation the other parts are implemented in python.
The main advantage of this approach is the possibility to make probabilistic rules with probabilistic behaviour.
The augmentation rules are also realised as \ac{RDF} graph, bringing the same advantages of validation and re-usability.
Three kinds of rules are implemented \textit{add} rules allowing adding new object based on templates, \textit{change} rules allow to change or append a specific triple, and \textit{delete} rules.
A \ac{SPARQL} query is used to define which elements should be modified or to which element the new objects should be connected.
The probabilistic behaviour is defined by two probabilities the first defines if the rule is applied for specific element and the second defines which template should be used.
After the augmentation the result is a complete model of the cyber-physical energy system, which get validated by the corresponding \ac{SHACL} shapes.
The next step is the simulation of the cyber-attack, for this the type attack vector and the scope must be defined.
The cyber-attacks used for this paper are also implemented as \textit{change} rules in the augmentation engine.
An example may be a compromised backend of an \ac{EV} manufacturer allowing an attacker to arbitrarily control charging of connected \acp{EV}.
Based on the connections modelled in the graph model the change in the \textit{process zone} can be applied before exporting a power system model in the \textit{pandapower} format.
In the last step  power flow simulations are performed to examine the impact of different attack vectors on the power system.


