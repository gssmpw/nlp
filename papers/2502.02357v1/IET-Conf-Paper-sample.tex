%%%IET Conferences full paper LaTeX template
\documentclass{IET-Conf-Paper}

%\usepackage[draft,columns=2]{typogrid}

\newtheorem{theorem}{Theorem}{}
\newtheorem{corollary}{Corollary}{}
\newtheorem{remark}{Remark}{}

\usepackage{amsmath,amssymb,amsfonts}
\usepackage{algorithmic}
\usepackage{graphicx}
\usepackage{textcomp}
\usepackage{xcolor}
\usepackage[printonlyused,nolist]{acronym}
\usepackage{tikz}
\usetikzlibrary{shapes.geometric, arrows}
%\def\BibTeX{{\rm B\kern-.05em{\sc i\kern-.025em b}\kern-.08em
%    T\kern-.1667em\lower.7ex\hbox{E}\kern-.125emX}}
%\usepackage{cite} %for dealing with citations
\usepackage{hyperref} %for linking the cross references\usepackage{graphics} %for dealing with figures.
%\usepackage{graphicx}
\usepackage{algorithmic} %for describing algorithms
%\usepackage{subfig} %for getting the subfigures e.g., "Figure 1a and 1b" etc.
\usepackage{authblk}
%\usepackage[margin=1.2cm]{geometry}
\usepackage{multicol}
\usepackage{blindtext}
\usepackage{epstopdf}
\usepackage{float}
\usepackage{listings}
\epstopdfsetup{update}

\begin{document}

\title{\vspace{-2em}Graph-based Impact Analysis of Cyber-Attacks on Behind-the-Meter Infrastructure
}

\author{Immanuel Hacker\ad{1,2}\corr, Ömer Sen\ad{1,2}, Florian Klein-Helmkamp \ad{2}, Andreas Ulbig\ad{1,2} }

\address{\add{1}{Digital Energy Fraunhofer FIT, Aachen, Germany}
\add{2}{IAEW at RWTH Aachen University, Aachen, Germany}
\email{immanuel.hacker@fit.fraunhofer.de}}


\keywords{cyber-physical systems, cyber-attacks, cyber-resilience, smart grids, SGAM, ontology}

\begin{abstract}
Behind-the-Meter assets are getting more interconnected to realise new applications like flexible tariffs. Cyber-attacks on the resulting control infrastructure may impact a large number of devices, which can result in severe impact on the power system.
To analyse the possible impact of such attacks we developed a graph model of the cyber-physical energy system, representing interdependencies between the control infrastructure and the power system. This model is than used for an impact analysis of cyber-attacks with different attack vectors.
\end{abstract}

\maketitle

\vspace{-2em}
\begin{center}
    \textit{This paper is a preprint of a paper submitted to 14th Mediterranean Conference on Power Generation Transmission, Distribution and Energy Conversion (MEDPOWER 2024) and is subject to Institution of Engineering and Technology Copyright. If accepted, the copy of record will be available at IET Digital Library}
\end{center}
\vspace{-1em}

\begin{acronym}
\acro{gan}[GANs]{Generative Adversarial Networks}
\acro{rl}[RL]{Reinforcement Learning}
\acro{pae}[PAE]{Periodic Autoencoder}
\acro{fld}[FLD]{Fourier Latent Dynamics}
\acro{ppo}[PPO]{Proximal Policy Optimization}
\acro{fft}[FFT]{Fast Fourier Transform}
\acro{pca}[PCA]{Principal Component Analysis}
\acro{dfm}[DFM]{Deep Fourier Mimic}
\acro{dof}[DoF]{Degrees of Freedom}
\acro{mlp}[MLPs]{Multi-Layer Perceptrons}
\end{acronym}


\newlength{\myboxheigth}
\setlength{\myboxheigth}{0.5cm}
\newlength{\myboxwidth}
\setlength{\myboxwidth}{3cm}

\tikzstyle{data} = [rectangle, rounded corners, 
minimum width=\myboxwidth, 
minimum height=\myboxheigth,
text centered, 
draw=black, 
fill=red!30]

\tikzstyle{io} = [trapezium, 
trapezium stretches=true, % A later addition
trapezium left angle=70, 
trapezium right angle=110, 
minimum width=\myboxwidth, 
minimum height=\myboxheigth, text centered, 
draw=black, fill=blue!30]

\tikzstyle{process} = [rectangle, 
minimum width=\myboxwidth, 
minimum height=\myboxheigth, 
text centered, 
text width=\myboxwidth, 
draw=black, 
fill=orange!30]

\tikzstyle{decision} = [diamond, 
minimum width=\myboxwidth, 
minimum height=\myboxheigth, 
text centered, 
draw=black, 
fill=green!30]
\tikzstyle{arrow} = [thick,->,>=stealth]


\lstdefinelanguage{turtle}{
    basicstyle=\small,
    morekeywords={@prefix, a},
    morestring=[b][\color{blue}]",
    morecomment=[s][\color{gray}]{<}{>}
}


\section{Introduction}
The energy transition has led to two developments that increase the significance of \ac{BTM} infrastructure. On the one hand, electrification of the heating and mobility sectors, and on the other, decentralised energy production through \ac{PV} systems. Additionally, storage systems are increasingly integrated to enable economically optimal combined utilisation of production and consumption. These \ac{BTM} installations are interconnected through \ac{ICT} infrastructure often via the internet, due to the requirements from various stakeholders.
The applications that necessitate this connectivity are highly diverse.
The simplest example is the customer's need to remotely control activities such as charging their own \ac{EV} or heating their house.
However, more complex applications such as dynamic electricity tariffs, which control \ac{BTM} devices based on market prices, are already a reality.
From the perspective of network operators, due to the new challenges for the electricity grid, it is crucial that \ac{BTM} installations can be controlled in a  grid-serving manner to ensure the safe operation of the electricity grid at all times.
These applications are either implemented directly via the customer's internet connection using \ac{HEMS} or through a dedicated advanced metering infrastructure.
The interconnection of the installations is imperative to implement these applications.
However, the networking of many small assets via a common \ac{ICT} infrastructure or a common software stack carries the risk that in the event of a cyber-attack, a large number of assets may be simultaneously affected, which could cumulatively have a significant impact on the power system.
Therefore, we propose a method to quantitatively assess this impact for various attack scenarios.

\subsection{Related Work}
\label{subsec:furtherwork}
The goal of this paper is to use approaches from the \ac{SW} to build the graph model, therefore the related work section focuses on  
\ac{SW} technologies are being explored in many areas for data modelling and describing semantic relationships between entities and attributes. Much research has also focused on how this technology can be connected to smart grid research areas.
%\todo{CGMS abgernzung einbauen für Produktiv Systeme verschiedenenn usecases im berreich Forschung aandere anforderung an Daten oft deutlich geruuinger manchmal anders -> perspektifich interoperabilität}
Several approaches exist to build ontologies for power systems using \ac{RDF}.
For instance, \cite{devanand_ontopowsys_2020} focuses on multi-domain energy systems, while \cite{chun_knowledge_2018} concentrates on micro-grid communities and on the service side of the system.
Both use \ac{SW} technologies, but not with the purpose of bridging different domains and enable cyber resilience research.
The proposed models do not consider a holistic framework for describing smart grid use cases with all associated layers in both cases. In particular, they lack the connection to the \ac{SGAM}, which brings the concept in line with a widely used way to describe use cases, making the models much more accessible for researchers.
\cite{sandbergdeliverable} uses the \ac{SGAM} as basis for an ontology designed for threat modelling.
Although this work involves modelling cross-domain inter-dependencies of cyber-physical systems, it does not utilise SHACL, which allows validation through rules.
In contrast to the work presented, our paper proposes an approach that provides a holistic picture that presents a complete framework and shows how \ac{SW} technologies can be useful in all steps of definition, description, validation, and evaluation.
One of these additional aspects is a rule-based augmentation for transforming power systems in \ac{CPES}.
\cite{klaer_sg3_2020} presents an approach of graph-based modelling for \ac{CPES} and emphasises the need for an automated augmentation process. However, the work focuses only on a specific \ac{SCADA} use case and does not utilise \ac{SW} technologies. 


\section{\label{sec:method}Methodology}

Each SIEM system uses its own RDL to define threat detection rules, and each RDL has its own schema.
For example, the Splunk SIEM uses the SPL to define its threat detection rules.
The task of understanding threat detection rules and recommending relevant MITRE ATT\&CK techniques (or sub-techniques) requires complex reasoning skills.
In the case of LLMs, this can be achieved with a technique called prompt chaining in which each task is divided into multiple sub-tasks in order to understand the complex reasoning behind the task.
Therefore, we employ a multi-phase architecture based on prompt chaining that leverages the power of LLMs to take a SIEM rule defined in any RDL and map it to relevant MITRE ATT\&CK techniques using the power of LLMs.
Our approach is based on the following intuitions:
\begin{itemize}[nosep,leftmargin=*]
    \item \textit{LLMs' implicit knowledge}: LLMs possess deep understanding of diverse RDLs. This enables them to interpret any rule, regardless of the RDL it is defined in, and convert it into comprehensible natural language text.
    \item \textit{LLMs' similarity comparison capability}: LLMs are adept at analyzing and comparing textual descriptions. 
    They can intelligently assess the similarity between two textual inputs to establish a meaningful connection.
\end{itemize}

\methodName has two main phases: (1) the rule to text translation phase, and (2) the MITRE ATT\&CK techniques recommendation phase.
These two phases in the pipeline include six key steps to determine relevant TTPs, as illustrated in Figure~\ref{fig:r2t}.

Although LLMs excel at translating SIEM rules into natural language, they often lack critical domain-specific contextual information related to IoCs in the rules.
To overcome this limitation, the \textit{rule to text translation} phase includes three steps: IoC extraction, contextual information retrieval, and natural language translation.

The workflow begins with the extraction of IoCs from the rules (for example, processes, log source, event codes, and file names) that the rule searches for in the logs (step (1)).In the next sstep a web search agent performs the task of obtaining additional contextual information about the IoCs discovered ((step 2)).
By incorporating this additional domain-specific information, the pipeline enhances the language translation, resulting in a more accurate and meaningful interpretation of SIEM rules.
The rule itself and the IoCs' contextual additional information from the previous stage are then used to translate the rule from RDL to natural language (step (3)).

The \textit{MITRE ATT\&CK techniques} recommendation phase of the pipeline includes the following three steps.
The rule in processed in data source identification step in which the probable origin of the data is identified (step (4)).
The description of the rule is then used to determine probable MITRE ATT\&CK techniques based on the implicit knowledge of the LLM (step (5)).
Finally, using chain-of-thought~\cite{wei2022chain} prompting, the most relevant techniques are extracted from the list of probable techniques (step (6)).
Each step of our method is further described in detail below.


% [bb=0 0 1440 900,width=1.43\linewidth,height=0.9\textwidth]
\begin{figure*}[htbp]
   \includegraphics[width=\textwidth]{Images/stages.jpg}
    
   \caption{An illustration of the different steps in \methodName.}
   \label{fig:stages}
\end{figure*} 

\subsection{IoC Extraction}
The context associated with a SIEM detection rule is crucial for its accurate interpretation and effective application. 
Obtaining this contextual understanding requires comprehensive analysis of the embedded IoCs in the SIEM rule.
In the first step, \methodName systematically identifies and extracts all IoCs, identifying the types of IoCs and their corresponding values that form the foundational elements of the detection rules. 
Leveraging the LLM's inherent understanding of rule structures and IoCs, we employ a zero-shot prompting approach for this task. 
Zero-shot prompting enables the direct extraction of IoCs from the rules without requiring extensive pre-training on specific datasets.

\noindent The result of this stage is a dictionary structure, where:
\begin{itemize}[nosep,leftmargin=*]
    \item Keys represent types of IoC, such as processes, files, IP addresses, and log sources.
    \item Values are lists containing specific IoC details, such as process names, file names, IP addresses, and log source identifiers.
\end{itemize}

In the example depicted in Figure~\ref{fig:stages}(a), the pipeline processes a rule for which relevant MITRE ATT\&CK techniques need to be recommended. 
The IoC extractor LLM produces a dictionary structure as output, organizing the IoCs in a structured format to support subsequent stages in the analysis pipeline. 



\subsection{Contextual Information Retrieval}
In this step, an LLM agent is employed to retrieve relevant information pertaining to the IoCs extracted from the rule.
A REACT agent~\cite{react} was used in this case to generate both reasoning traces and task-specific actions in an interleaved manner.
REACT agents interact with external tools to retrieve additional information that leads to more factual and reliable responses.
The LLM agent conducts a systematic search across web resources to gather additional contextual information for each IoC value present in the rule. 
This step addresses LLMS' lack of up-to-date knowledge or specialized domain expertise (which is critical to understanding the role and significance of the IoCs in the rule), without the need for retraining or fine-tuning.
Figure~\ref{fig:stages}(b) presents an example in which the rule includes the process name \texttt{soaphound.exe} as an IoC.
As can be seen, the web search results indicate that \texttt{soaphound.exe} is being used for active directory (AD) enumeration, which is important for the understanding of the attack. 

\subsection{Natural Language Translation}

The translation of detection rules into natural language textual descriptions fulfills three key objectives:
\begin{enumerate}[nosep,leftmargin=*]
    \item \textbf{Ensures that \methodName is format-agnostic}: It converts rules defined in various RDL formats into a generic, unstructured text format, ensuring compatibility with different SIEM systems, regardless of the specific rule format.
    \item \textbf{Provides contextual explanation}: It includes all relevant contextual information to produce a concise and comprehensible explanation of the rule.
    \item \textbf{Enhances the comprehension for LLMs}: It enables LLMs to more effectively compare the translated rule with descriptions in the MITRE ATT\&CK framework by providing a unified textual representation.
\end{enumerate}
To achieve these objectives, a zero-shot prompting technique is employed. 
The input to the LLM comprises two components:
\begin{itemize}
    \item \textbf{Syntactical information}: The rule itself, providing the structural and operational details.
    \item \textbf{Contextual information}: Details of the IoCs extracted from the rule, providing semantic insights into the rule's intent and function.
\end{itemize}
The LLM utilizes these inputs to generate a natural language textual description of the rule. 
This transformation not only ensures a more interpretable representation but also facilitates further steps of analysis and comparison, particularly in aligning the rule with MITRE ATT\&CK techniques and sub-techniques.



\subsection{Data Source or Mitigation Identification}
Identifying the most relevant data component or mitigation associated with the rule description in this step is critical for filtering out irrelevant MITRE ATT\&CK techniques (or sub-techniques) in subsequent steps of the pipeline.
In the MITRE ATT\&CK framework, data sources represent various categories of information that can be gathered from sensors or logs. 
These data sources include \textit{data components}, which are specific attributes or properties within a data source that are directly relevant to detecting a particular technique or sub-technique~. 
For example, in the context of the rule described in Figure~\ref{fig:stages}(a), the term \texttt{Endpoint.Processes} indicates that the activity is happening on an endpoint. 
Presence of the terms such as, \texttt{soaphound.exe}, \texttt{--buildcache}, \texttt{--certdump} and etc. indicate that the rule searches for command line execution of an executable named \texttt{soaphound.exe} with specific parameters. 
Therefore, the appropriate data source in this example is \textit{Command}, with the corresponding data component being \textit{Command Execution}.
Additionally, \textit{mitigations} are defined as categories of technologies or strategies that can prevent or reduce the impact of specific techniques or sub-techniques. 
The MITRE ATT\&CK framework explicitly establishes relationships between data components, mitigations, and techniques (or sub-techniques), enabling a systematic approach for identifying relevant elements.

To identify the most relevant data component or mitigation associated with a given rule description, we utilize agentic retrieval augmented generation (RAG), which incorporates an AI Agent-based implementation of the RAG framework.
Data from the MITRE ATT\&CK framework, specifically related to data components and mitigations, is stored in a vector database (e.g., ChromaDB). 
The process begins with the rule description from the previous stage, which serves as the input to the AI Agent. 
The LLM-powered agent automatically generates a search query tailored to retrieve relevant information from the RAG database.

For each query, the system retrieves the five most similar documents from the database, each containing contextual information about data components or mitigations. 
These documents are then utilized by the LLM agent to contextualize the rule description. 
By comparing the content of these retrieved documents with the rule description, the LLM agent determines and outputs the most relevant data component or mitigation along with a chain-of-thought as to why the data component or mitigation is related to the rule.


\subsection{Probable Technique Recommendation}

In this step, an LM agent is utilized to propose probable MITRE ATT\&CK techniques (and sub-techniques) that may be relevant to the description of the provided rule.
We used a REACT agent in this step as well to utilize both implicit and explicit knowledge during reasoning.
For explicit knowledge, the agent searches the MITRE ATT\&CK framework to obtain the list of probable techniques (and sub-techniques).
The natural language description of the rule from the previous step serves as input to the LLM agent.
The output of this stage consists of a list of JSON objects, each containing the MITRE technique ID, technique name, and technique description as seen in Figure~\ref{fig:stages}(c).

Throughout our experiments, we observed that as the number of recommendations ($k$) increases, both the framework's average recall and precision initially improve, however beyond a certain threshold of $k$, the %average 
precision begins to decline.
Based on these observations(please refer Table~\ref{tab:results3}), we selected a $k$-value of 11 to ensure a high recall.



\subsection{Relevant Technique Extraction}
In this step, \methodName refines the set of probable MITRE ATT\&CK techniques identified in the previous stage by eliminating irrelevant entries.
This step in the pipeline serves two primary purposes: (1) to enhance precision while maintaining recall achieved in previous step, and (2) to provide a clear rationale for the selection of the labels, ensuring transparency and interpretability of the mapping process.
This refinement process is grounded in the assumption that LLMs are effective for text similarity matching tasks.

The process comprises two key steps:
\begin{itemize}
    \item \textit{Rule-technique comparison}: The description of each technique in the set of probable techniques is compared with the rule description. 
    A chain-of-thought technique is then applied to elucidate the reasoning behind the association of each technique with the rule.
    \item \textit{Confidence calculation}: The generated chain-of-thought rationale for each technique (or sub-technique) is compared with the rule description to compute a relevance (or confidence) score, as done in prior work~\cite{freitas2024ai}.
    % \item \textbf{Reasoning}: \new{Add here the reasoning that it provides - explaining in NLP why it was selected...}
\end{itemize}

Techniques with higher confidence scores are deemed more relevant to the rule. 
Conversely, techniques with scores falling below a predefined threshold are excluded.
The techniques retained after this filtering step represent the most relevant techniques corresponding to the given rule's description. 


The chain-of-thought (CoT) rationale generated during the comparison of each rule to its probable technique is also provided as an output in this step.
This rationale offers a detailed natural language explanation, articulating why a particular technique is relevant to the given rule. 
Such explanations are highly valuable for security analysts, as they provide clear and transparent reasoning behind the mapping, enabling analysts to better understand and validate the association between the rule and the technique.
Other classification models proposed in previous works within this domain also suffer from the limitation of being black-box models, which lack the ability to provide clear reasoning or explanations. 
Unlike \methodName, these models fail to generate transparent, CoT rationales that explain why a particular rule is mapped to a specific technique, making them less interpretable and less useful for security analysts.
\begin{tikzpicture}[
    node distance=2em and 3em,
    every node/.style={rectangle, draw, rounded corners, text centered, minimum height=1em},
    arrow/.style={-Stealth, thick},
    decision/.style={diamond, draw, text centered, inner sep=0pt, aspect=2},
]

% Nodes
\node[decision] (decision) {\faUser};
\node[above right=1em and 6em of decision] (generator) {\ggen ($\S\text{\ref{sec:generator}}$)};
\node[below right=1em and 6em of decision] (manual) {Write Manually};
% \node[right=of decision, text width=12em] (fuzzing) {Automated obtaining from programs\\ or Manually Written};
\node[right=22em of decision] (sippy) {\tool ($\S\text{\ref{sec:positive} \&}~\S\text{\ref{sec:SLsynthesis}}$)} ;
\node[above right=1em and 12em of sippy] (verifiers) {Verification ($\S\text{\ref{sec:verification}}$)};
\node[right=12em of sippy] (synthesizers) {Synthesis ($\S\text{\ref{sec:synthesis}}$)};
\node[below right=1em and 12em of sippy] (others) {Other applications};


% Edges with labels
\draw[arrow] (decision) -- node[above, draw=none, align=center, sloped] {Program} (generator);
\draw[arrow] (decision) -- node[below, draw=none, align=center, sloped] {or} (manual);
\draw[arrow] (manual) -- node[above, draw=none, align=left] {Memory Graphs~~} (sippy);
\draw[arrow] (generator) --  (sippy);
\draw[arrow] (sippy) -- node[above, draw=none, align=center] {Heap Predicates} node[below,draw=none,align=center] {for further applications} ++(14em, 0) coordinate (branch);
% \draw[arrow] (mid) -- node[above, draw=none, rectangle, align=center, sloped] {2} ++(1em, 0) coordinate (branch);
\draw[arrow] (branch) |- (verifiers);
\draw[arrow] (branch) |- (synthesizers);
\draw[arrow] (branch) |- (others);

\end{tikzpicture}

\begin{table*}[ht]
  \centering
  \setlength{\fboxsep}{0.7pt}
  \resizebox{\linewidth}{!}{
  \begin{tabular}{p{1\textwidth}}
    \hline
    \textbf{Query:} What is australia's location in the world and region?    \; \textbf{Ground Truth: Oceania } \\
    \textbf{Documents:}\\

    (1) \ldots Australia is both a continent and a country. It is located to the south of Indonesia and Papua New Guinea, in the southeast part of the Indian Ocean. New Zealand lies to its east, and most of the \colorbox{skyblue}{Indian Ocean lies to its west}. \ldots Australia is the world’s \colorbox{lightred}{smallest } continent, though it is also the \colorbox{lightred}{sixth-largest} country in the world. The total land area of Australia is \colorbox{lightred}{7,682,300 sq. km}.\\

    (2) \ldots Australia is the smallest continent and the largest country in \colorbox{skyblue}{Oceania} located between the \colorbox{skyblue}{Indian Ocean and Pacific Ocean} in the \colorbox{skyblue}{Southern hemisphere}. \ldots Australia is separated from Papua New Guinea by the \colorbox{skyblue}{Coral Sea} and Torres Strait to the \colorbox{skyblue}{northeast}; from Indonesia by the \colorbox{skyblue}{Timor and Arafura} seas to the northwest; from Coral Sea Islands Territory by the Great Barrier Reef; from New Zealand by Tasman Sea to the southeast;\\

    (3) \ldots Australia is located \colorbox{skyblue}{south of the equator} and is bordered by the Indian Ocean and the Pacific Ocean.
    \ldots\\
    % (4) \ldots It is located between the \colorbox{lightred}{Indian Ocean and the South Pacific Ocean}. \ldots It has 23,859 km of coastlines including \ldots \\

    % (5) Australia is an island continent in the \colorbox{lightred}{Southern Hemisphere}, \ldots It is surrounded by the \colorbox{skyblue}{Indian Ocean to the  west}; \colorbox{skyblue}{the Timor, Arafura, and Coral Seas to the north}; \colorbox{skyblue}{the Pacific Ocean  to the east}; and \colorbox{skyblue}{the Tasman Sea and Southern Ocean to the south}. \ldots. The area of the continent is 2.97  million square miles (7.69 million square kilometers). \ldots\\
    \hline
    % \multicolumn{1}{c}{\textbf{Vanilla RAG:}} \\
    % \textbf{Answer:} According to the text, Australia is: * Located in the \colorbox{lightbluegreen}{Southern Hemisphere} * \colorbox{lightbluegreen}{Between the Indian Ocean and the Pacific Ocean} * Surrounded by the Indian Ocean. \textcolor{red}{(\ding{55})} \\
    % \hline
    \multicolumn{1}{c}{\textbf{Summary:}} \\
    \textbf{Summary:} Here is a concise summary: Australia is located in the \colorbox{lightred}{Southern Hemisphere}, between the Indian Ocean and the Pacific Ocean. It is the world's \colorbox{lightred}{smallest} continent and the \colorbox{lightred}{sixth-largest} country, covering an area of \colorbox{lightred}{7,682,300 square kilometers}. Australia is surrounded by the Indian Ocean to the west, the Timor, Arafura, and Coral Seas to the north, the Pacific Ocean to the east, and the Tasman Sea and Southern Ocean to the south. The country is made up of six states (Western Australia, South Australia, Victoria, Tasmania, New South Wales, and Queensland) and two territories (the Northern Territory and the Australian Capital Territory) \\
    \textbf{Answer:} Australia is located in the Southern Hemisphere, between the Indian Ocean and the Pacific Ocean. \textcolor{red}{(\ding{55})}\\
    \hline
    \multicolumn{1}{c}{\textbf{RankCoT:}} \\
    \textbf{Chain-of-Thought:} According to the passages, Australia is:
    
    1. An island continent located in the \colorbox{skyblue}{Southern Hemisphere}.
    
    2. Situated between the \colorbox{skyblue}{Indian Ocean and the Pacific Ocean}.
    
    3. Located \colorbox{skyblue}{south of the equator}.
    
    4. Surrounded by the \colorbox{skyblue}{Indian Ocean to the west}, the \colorbox{skyblue}{Timor, Arafura, and Coral Seas to the north}, the Pacific Ocean to the east, and the Tasman Sea and Southern Ocean to the south.
    
    5. An island continent \colorbox{skyblue}{in Oceania, which is a region} that includes Australia, New Zealand, and many Pacific island nations. 
    
    So, Australia's location in the world is in the Southern Hemisphere, \colorbox{lightpurple}{in the region of Oceania}.\\
    \textbf{Answer:} Australia is an island continent located in the Southern Hemisphere, in the \colorbox{lightpurple}{region of Oceania}. \textcolor{green}{(\ding{51})}\\
    \hline
    
  \end{tabular}
  }
  \caption{\label{case study}
    Case Study. We randomly sample one case from the NQ dataset to show the knowledge refinement result. Different colors are used to annotate the key information from the retrieved knowledge retained by different knowledge refinement models: \colorbox{lightred}{Pink} for Summary, \colorbox{skyblue}{Blue} for RankCoT. And we also highlight the key points that can help answer the query in \colorbox{lightpurple}{Purple}.
  }
\end{table*}
\section*{Conclusion}
This paper aims to enhance our understanding of the computational complexity of computing various Shapley value variants. We found that for various ML models --- including decision trees, regression tree ensembles, weighted automata, and linear regression --- both local and global interventional and baseline SHAP can be computed in polynomial time under HMM modeled distributions. This extends popular algorithms, such as TreeSHAP, beyond their empirical distributional scope. We also establish strict complexity gaps between the various SHAP variants (baseline, interventional, and conditional) and prove the intractability of computing SHAP for tree ensembles and neural networks in simplified scenarios. Overall, we present SHAP as a versatile framework whose complexity depends on four key factors: \begin{inparaenum}[(i)] \item model type, \item SHAP variant, \item distribution modeling approach, \item and local vs. global explanations\end{inparaenum}. We believe this perspective provides deeper insight into the computational complexity of SHAP, paving the way for future work.




%We believe that our framework provides a more intricate understanding of SHAP computation complexity across different models, distributions, and variants, paving the way for further research.

Our work opens promising directions for future research. First, expanding our computational analysis to other SHAP-related metrics, such as asymmetric SHAP~\citep{frye20} and SAGE~\citep{covert2020understanding}, would be valuable. Additionally, we aim to explore more expressive distribution classes and relaxed assumptions beyond those in Section \ref{sec:tractable} while maintaining tractable SHAP computation. Finally, when exact computation is intractable (Section \ref{sec:intractable}), investigating the approximability of SHAP metrics through approximation and parameterized complexity theory~\citep{downey2012parameterized} is an important direction.

%Our work opens several promising avenues for future research on the computational properties of explainable AI methods, with a particular focus on SHAP. First, it would be interesting to broaden the computational analysis conducted in this work to include other popular SHAP-related metrics in the literature, such as asymmetric SHAP \cite{frye20} and SAGE \cite{covert2020understanding}. Also, in the future, we aim to explore more expressive distribution classes and relaxed distributional assumptions—extending beyond those examined in Section \ref{sec:tractable} —that still yield tractable SHAP computation. Finally, when exact computation proves intractable (Section \ref{sec:intractable}), it is worthwhile to theoretically investigate the question of the approximability of computing the SHAP metrics across various configurations, through the lens of approximation and parametrized complexity theory \cite{arora2009computational}.

%This paper aims to deepen our understanding of the computational complexity involved in obtaining different Shapley value variants. We found that for a variety of ML models, including decision trees, tree ensembles for regression, weighted automata, and linear regression models — computing both local and global interventional and baseline SHAP can be done in polynomial time when distributions are modeled by HMMs. This extends the distributional scope of popular algorithms like TreeSHAP, which is limited to empirical distributions. Additionally, we demonstrate a strict complexity gap between SHAP variants, showing that interventional and baseline SHAP can be strictly easier to compute than conditional SHAP. Despite these positive results, we uncovered intractability for various SHAP variants in neural networks and tree ensembles. Finally, we provided generalized complexity relations across SHAP variants. We believe that our framework offers a deeper understanding of the complexity involved in computing SHAP across various variants, models, distributions, as well as in both local and global computations, laying the groundwork for future research.


\section{Acknowledgment}
\noindent
\begin{minipage}{0.60\columnwidth}%
This work has received funding from the \ac{BMBF} under project funding reference 03SF0694A.
\end{minipage}
\hspace{0.02\columnwidth}
\begin{minipage}{0.35\columnwidth}%
    \includegraphics[width=\textwidth]{BMBF}
\end{minipage}
\vspace{-1em}
\bibliographystyle{iet}
\bibliography{references.bib}

\end{document}
