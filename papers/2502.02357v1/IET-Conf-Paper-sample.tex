%%%IET Conferences full paper LaTeX template
\documentclass{IET-Conf-Paper}

%\usepackage[draft,columns=2]{typogrid}

\newtheorem{theorem}{Theorem}{}
\newtheorem{corollary}{Corollary}{}
\newtheorem{remark}{Remark}{}

\usepackage{amsmath,amssymb,amsfonts}
\usepackage{algorithmic}
\usepackage{graphicx}
\usepackage{textcomp}
\usepackage{xcolor}
\usepackage[printonlyused,nolist]{acronym}
\usepackage{tikz}
\usetikzlibrary{shapes.geometric, arrows}
%\def\BibTeX{{\rm B\kern-.05em{\sc i\kern-.025em b}\kern-.08em
%    T\kern-.1667em\lower.7ex\hbox{E}\kern-.125emX}}
%\usepackage{cite} %for dealing with citations
\usepackage{hyperref} %for linking the cross references\usepackage{graphics} %for dealing with figures.
%\usepackage{graphicx}
\usepackage{algorithmic} %for describing algorithms
%\usepackage{subfig} %for getting the subfigures e.g., "Figure 1a and 1b" etc.
\usepackage{authblk}
%\usepackage[margin=1.2cm]{geometry}
\usepackage{multicol}
\usepackage{blindtext}
\usepackage{epstopdf}
\usepackage{float}
\usepackage{listings}
\epstopdfsetup{update}

\begin{document}

\title{\vspace{-2em}Graph-based Impact Analysis of Cyber-Attacks on Behind-the-Meter Infrastructure
}

\author{Immanuel Hacker\ad{1,2}\corr, Ömer Sen\ad{1,2}, Florian Klein-Helmkamp \ad{2}, Andreas Ulbig\ad{1,2} }

\address{\add{1}{Digital Energy Fraunhofer FIT, Aachen, Germany}
\add{2}{IAEW at RWTH Aachen University, Aachen, Germany}
\email{immanuel.hacker@fit.fraunhofer.de}}


\keywords{cyber-physical systems, cyber-attacks, cyber-resilience, smart grids, SGAM, ontology}

\begin{abstract}
Behind-the-Meter assets are getting more interconnected to realise new applications like flexible tariffs. Cyber-attacks on the resulting control infrastructure may impact a large number of devices, which can result in severe impact on the power system.
To analyse the possible impact of such attacks we developed a graph model of the cyber-physical energy system, representing interdependencies between the control infrastructure and the power system. This model is than used for an impact analysis of cyber-attacks with different attack vectors.
\end{abstract}

\maketitle

\vspace{-2em}
\begin{center}
    \textit{This paper is a preprint of a paper submitted to 14th Mediterranean Conference on Power Generation Transmission, Distribution and Energy Conversion (MEDPOWER 2024) and is subject to Institution of Engineering and Technology Copyright. If accepted, the copy of record will be available at IET Digital Library}
\end{center}
\vspace{-1em}

\newacronym{rl}{RL}{Reinforcement Learning}
\newacronym{drl}{DRL}{Deep Reinforcement Learning}
\newacronym{mdp}{MDP}{Markov Decision Process}
\newacronym{ppo}{PPO}{Proximal Policy Optimization}
\newacronym{sac}{SAC}{Soft Actor-Critic}
\newacronym{epvf}{EPVF}{Explicit Policy-conditioned Value Function}
\newacronym{unf}{UNF}{Universal Neural Functional}
\newlength{\myboxheigth}
\setlength{\myboxheigth}{0.5cm}
\newlength{\myboxwidth}
\setlength{\myboxwidth}{3cm}

\tikzstyle{data} = [rectangle, rounded corners, 
minimum width=\myboxwidth, 
minimum height=\myboxheigth,
text centered, 
draw=black, 
fill=red!30]

\tikzstyle{io} = [trapezium, 
trapezium stretches=true, % A later addition
trapezium left angle=70, 
trapezium right angle=110, 
minimum width=\myboxwidth, 
minimum height=\myboxheigth, text centered, 
draw=black, fill=blue!30]

\tikzstyle{process} = [rectangle, 
minimum width=\myboxwidth, 
minimum height=\myboxheigth, 
text centered, 
text width=\myboxwidth, 
draw=black, 
fill=orange!30]

\tikzstyle{decision} = [diamond, 
minimum width=\myboxwidth, 
minimum height=\myboxheigth, 
text centered, 
draw=black, 
fill=green!30]
\tikzstyle{arrow} = [thick,->,>=stealth]


\lstdefinelanguage{turtle}{
    basicstyle=\small,
    morekeywords={@prefix, a},
    morestring=[b][\color{blue}]",
    morecomment=[s][\color{gray}]{<}{>}
}


\section{Introduction}
The energy transition has led to two developments that increase the significance of \ac{BTM} infrastructure. On the one hand, electrification of the heating and mobility sectors, and on the other, decentralised energy production through \ac{PV} systems. Additionally, storage systems are increasingly integrated to enable economically optimal combined utilisation of production and consumption. These \ac{BTM} installations are interconnected through \ac{ICT} infrastructure often via the internet, due to the requirements from various stakeholders.
The applications that necessitate this connectivity are highly diverse.
The simplest example is the customer's need to remotely control activities such as charging their own \ac{EV} or heating their house.
However, more complex applications such as dynamic electricity tariffs, which control \ac{BTM} devices based on market prices, are already a reality.
From the perspective of network operators, due to the new challenges for the electricity grid, it is crucial that \ac{BTM} installations can be controlled in a  grid-serving manner to ensure the safe operation of the electricity grid at all times.
These applications are either implemented directly via the customer's internet connection using \ac{HEMS} or through a dedicated advanced metering infrastructure.
The interconnection of the installations is imperative to implement these applications.
However, the networking of many small assets via a common \ac{ICT} infrastructure or a common software stack carries the risk that in the event of a cyber-attack, a large number of assets may be simultaneously affected, which could cumulatively have a significant impact on the power system.
Therefore, we propose a method to quantitatively assess this impact for various attack scenarios.

\subsection{Related Work}
\label{subsec:furtherwork}
The goal of this paper is to use approaches from the \ac{SW} to build the graph model, therefore the related work section focuses on  
\ac{SW} technologies are being explored in many areas for data modelling and describing semantic relationships between entities and attributes. Much research has also focused on how this technology can be connected to smart grid research areas.
%\todo{CGMS abgernzung einbauen für Produktiv Systeme verschiedenenn usecases im berreich Forschung aandere anforderung an Daten oft deutlich geruuinger manchmal anders -> perspektifich interoperabilität}
Several approaches exist to build ontologies for power systems using \ac{RDF}.
For instance, \cite{devanand_ontopowsys_2020} focuses on multi-domain energy systems, while \cite{chun_knowledge_2018} concentrates on micro-grid communities and on the service side of the system.
Both use \ac{SW} technologies, but not with the purpose of bridging different domains and enable cyber resilience research.
The proposed models do not consider a holistic framework for describing smart grid use cases with all associated layers in both cases. In particular, they lack the connection to the \ac{SGAM}, which brings the concept in line with a widely used way to describe use cases, making the models much more accessible for researchers.
\cite{sandbergdeliverable} uses the \ac{SGAM} as basis for an ontology designed for threat modelling.
Although this work involves modelling cross-domain inter-dependencies of cyber-physical systems, it does not utilise SHACL, which allows validation through rules.
In contrast to the work presented, our paper proposes an approach that provides a holistic picture that presents a complete framework and shows how \ac{SW} technologies can be useful in all steps of definition, description, validation, and evaluation.
One of these additional aspects is a rule-based augmentation for transforming power systems in \ac{CPES}.
\cite{klaer_sg3_2020} presents an approach of graph-based modelling for \ac{CPES} and emphasises the need for an automated augmentation process. However, the work focuses only on a specific \ac{SCADA} use case and does not utilise \ac{SW} technologies. 

\section{Research Methodology}~\label{sec:Methodology}

In this section, we discuss the process of conducting our systematic review, e.g., our search strategy for data extraction of relevant studies, based on the guidelines of Kitchenham et al.~\cite{kitchenham2022segress} to conduct SLRs and Petersen et al.~\cite{PETERSEN20151} to conduct systematic mapping studies (SMSs) in Software Engineering. In this systematic review, we divide our work into a four-stage procedure, including planning, conducting, building a taxonomy, and reporting the review, illustrated in Fig.~\ref{fig:search}. The four stages are as follows: (1) the \emph{planning} stage involved identifying research questions (RQs) and specifying the detailed research plan for the study; (2) the \emph{conducting} stage involved analyzing and synthesizing the existing primary studies to answer the research questions; (3) the \emph{taxonomy} stage was introduced to optimize the data extraction results and consolidate a taxonomy schema for REDAST methodology; (4) the \emph{reporting} stage involved the reviewing, concluding and reporting the final result of our study.

\begin{figure}[!t]
    \centering
    \includegraphics[width=1\linewidth]{fig/methodology/searching-process.drawio.pdf}
    \caption{Systematic Literature Review Process}
    \label{fig:search}
\end{figure}

\subsection{Research Questions}
In this study, we developed five research questions (RQs) to identify the input and output, analyze technologies, evaluate metrics, identify challenges, and identify potential opportunities. 

\textbf{RQ1. What are the input configurations, formats, and notations used in the requirements in requirements-driven
automated software testing?} In requirements-driven testing, the input is some form of requirements specification -- which can vary significantly. RQ1 maps the input for REDAST and reports on the comparison among different formats for requirements specification.

\textbf{RQ2. What are the frameworks, tools, processing methods, and transformation techniques used in requirements-driven automated software testing studies?} RQ2 explores the technical solutions from requirements to generated artifacts, e.g., rule-based transformation applying natural language processing (NLP) pipelines and deep learning (DL) techniques, where we additionally discuss the potential intermediate representation and additional input for the transformation process.

\textbf{RQ3. What are the test formats and coverage criteria used in the requirements-driven automated software
testing process?} RQ3 focuses on identifying the formulation of generated artifacts (i.e., the final output). We map the adopted test formats and analyze their characteristics in the REDAST process.

\textbf{RQ4. How do existing studies evaluate the generated test artifacts in the requirements-driven automated software testing process?} RQ4 identifies the evaluation datasets, metrics, and case study methodologies in the selected papers. This aims to understand how researchers assess the effectiveness, accuracy, and practical applicability of the generated test artifacts.

\textbf{RQ5. What are the limitations and challenges of existing requirements-driven automated software testing methods in the current era?} RQ5 addresses the limitations and challenges of existing studies while exploring future directions in the current era of technology development. %It particularly highlights the potential benefits of advanced LLMs and examines their capacity to meet the high expectations placed on these cutting-edge language modeling technologies. %\textcolor{blue}{CA: Do we really need to focus on LLMs? TBD.} \textcolor{orange}{FW: About LLMs, I removed the direct emphase in RQ5 but kept the discussion in RQ5 and the solution section. I think that would be more appropriate.}

\subsection{Searching Strategy}

The overview of the search process is exhibited in Fig. \ref{fig:papers}, which includes all the details of our search steps.
\begin{table}[!ht]
\caption{List of Search Terms}
\label{table:search_term}
\begin{tabularx}{\textwidth}{lX}
\hline
\textbf{Terms Group} & \textbf{Terms} \\ \hline
Test Group & test* \\
Requirement Group & requirement* OR use case* OR user stor* OR specification* \\
Software Group & software* OR system* \\
Method Group & generat* OR deriv* OR map* OR creat* OR extract* OR design* OR priorit* OR construct* OR transform* \\ \hline
\end{tabularx}
\end{table}

\begin{figure}
    \centering
    \includegraphics[width=1\linewidth]{fig/methodology/search-papers.drawio.pdf}
    \caption{Study Search Process}
    \label{fig:papers}
\end{figure}

\subsubsection{Search String Formulation}
Our research questions (RQs) guided the identification of the main search terms. We designed our search string with generic keywords to avoid missing out on any related papers, where four groups of search terms are included, namely ``test group'', ``requirement group'', ``software group'', and ``method group''. In order to capture all the expressions of the search terms, we use wildcards to match the appendix of the word, e.g., ``test*'' can capture ``testing'', ``tests'' and so on. The search terms are listed in Table~\ref{table:search_term}, decided after iterative discussion and refinement among all the authors. As a result, we finally formed the search string as follows:


\hangindent=1.5em
 \textbf{ON ABSTRACT} ((``test*'') \textbf{AND} (``requirement*'' \textbf{OR} ``use case*'' \textbf{OR} ``user stor*'' \textbf{OR} ``specifications'') \textbf{AND} (``software*'' \textbf{OR} ``system*'') \textbf{AND} (``generat*'' \textbf{OR} ``deriv*'' \textbf{OR} ``map*'' \textbf{OR} ``creat*'' \textbf{OR} ``extract*'' \textbf{OR} ``design*'' \textbf{OR} ``priorit*'' \textbf{OR} ``construct*'' \textbf{OR} ``transform*''))

The search process was conducted in September 2024, and therefore, the search results reflect studies available up to that date. We conducted the search process on six online databases: IEEE Xplore, ACM Digital Library, Wiley, Scopus, Web of Science, and Science Direct. However, some databases were incompatible with our default search string in the following situations: (1) unsupported for searching within abstract, such as Scopus, and (2) limited search terms, such as ScienceDirect. Here, for (1) situation, we searched within the title, keyword, and abstract, and for (2) situation, we separately executed the search and removed the duplicate papers in the merging process. 

\subsubsection{Automated Searching and Duplicate Removal}
We used advanced search to execute our search string within our selected databases, following our designed selection criteria in Table \ref{table:selection}. The first search returned 27,333 papers. Specifically for the duplicate removal, we used a Python script to remove (1) overlapped search results among multiple databases and (2) conference or workshop papers, also found with the same title and authors in the other journals. After duplicate removal, we obtained 21,652 papers for further filtering.

\begin{table*}[]
\caption{Selection Criteria}
\label{table:selection}
\begin{tabularx}{\textwidth}{lX}
\hline
\textbf{Criterion ID} & \textbf{Criterion Description} \\ \hline
S01          & Papers written in English. \\
S02-1        & Papers in the subjects of "Computer Science" or "Software Engineering". \\
S02-2        & Papers published on software testing-related issues. \\
S03          & Papers published from 1991 to the present. \\ 
S04          & Papers with accessible full text. \\ \hline
\end{tabularx}
\end{table*}

\begin{table*}[]
\small
\caption{Inclusion and Exclusion Criteria}
\label{table:criteria}
\begin{tabularx}{\textwidth}{lX}
\hline
\textbf{ID}  & \textbf{Description} \\ \hline
\multicolumn{2}{l}{\textbf{Inclusion Criteria}} \\ \hline
I01 & Papers about requirements-driven automated system testing or acceptance testing generation, or studies that generate system-testing-related artifacts. \\
I02 & Peer-reviewed studies that have been used in academia with references from literature. \\ \hline
\multicolumn{2}{l}{\textbf{Exclusion Criteria}} \\ \hline
E01 & Studies that only support automated code generation, but not test-artifact generation. \\
E02 & Studies that do not use requirements-related information as an input. \\
E03 & Papers with fewer than 5 pages (1-4 pages). \\
E04 & Non-primary studies (secondary or tertiary studies). \\
E05 & Vision papers and grey literature (unpublished work), books (chapters), posters, discussions, opinions, keynotes, magazine articles, experience, and comparison papers. \\ \hline
\end{tabularx}
\end{table*}

\subsubsection{Filtering Process}

In this step, we filtered a total of 21,652 papers using the inclusion and exclusion criteria outlined in Table \ref{table:criteria}. This process was primarily carried out by the first and second authors. Our criteria are structured at different levels, facilitating a multi-step filtering process. This approach involves applying various criteria in three distinct phases. We employed a cross-verification method involving (1) the first and second authors and (2) the other authors. Initially, the filtering was conducted separately by the first and second authors. After cross-verifying their results, the results were then reviewed and discussed further by the other authors for final decision-making. We widely adopted this verification strategy within the filtering stages. During the filtering process, we managed our paper list using a BibTeX file and categorized the papers with color-coding through BibTeX management software\footnote{\url{https://bibdesk.sourceforge.io/}}, i.e., “red” for irrelevant papers, “yellow” for potentially relevant papers, and “blue” for relevant papers. This color-coding system facilitated the organization and review of papers according to their relevance.

The screening process is shown below,
\begin{itemize}
    \item \textbf{1st-round Filtering} was based on the title and abstract, using the criteria I01 and E01. At this stage, the number of papers was reduced from 21,652 to 9,071.
    \item \textbf{2nd-round Filtering}. We attempted to include requirements-related papers based on E02 on the title and abstract level, which resulted from 9,071 to 4,071 papers. We excluded all the papers that did not focus on requirements-related information as an input or only mentioned the term ``requirements'' but did not refer to the requirements specification.
    \item \textbf{3rd-round Filtering}. We selectively reviewed the content of papers identified as potentially relevant to requirements-driven automated test generation. This process resulted in 162 papers for further analysis.
\end{itemize}
Note that, especially for third-round filtering, we aimed to include as many relevant papers as possible, even borderline cases, according to our criteria. The results were then discussed iteratively among all the authors to reach a consensus.

\subsubsection{Snowballing}

Snowballing is necessary for identifying papers that may have been missed during the automated search. Following the guidelines by Wohlin~\cite{wohlin2014guidelines}, we conducted both forward and backward snowballing. As a result, we identified 24 additional papers through this process.

\subsubsection{Data Extraction}

Based on the formulated research questions (RQs), we designed 38 data extraction questions\footnote{\url{https://drive.google.com/file/d/1yjy-59Juu9L3WHaOPu-XQo-j-HHGTbx_/view?usp=sharing}} and created a Google Form to collect the required information from the relevant papers. The questions included 30 short-answer questions, six checkbox questions, and two selection questions. The data extraction was organized into five sections: (1) basic information: fundamental details such as title, author, venue, etc.; (2) open information: insights on motivation, limitations, challenges, etc.; (3) requirements: requirements format, notation, and related aspects; (4) methodology: details, including immediate representation and technique support; (5) test-related information: test format(s), coverage, and related elements. Similar to the filtering process, the first and second authors conducted the data extraction and then forwarded the results to the other authors to initiate the review meeting.

\subsubsection{Quality Assessment}

During the data extraction process, we encountered papers with insufficient information. To address this, we conducted a quality assessment in parallel to ensure the relevance of the papers to our objectives. This approach, also adopted in previous secondary studies~\cite{shamsujjoha2021developing, naveed2024model}, involved designing a set of assessment questions based on guidelines by Kitchenham et al.~\cite{kitchenham2022segress}. The quality assessment questions in our study are shown below:
\begin{itemize}
    \item \textbf{QA1}. Does this study clearly state \emph{how} requirements drive automated test generation?
    \item \textbf{QA2}. Does this study clearly state the \emph{aim} of REDAST?
    \item \textbf{QA3}. Does this study enable \emph{automation} in test generation?
    \item \textbf{QA4}. Does this study demonstrate the usability of the method from the perspective of methodology explanation, discussion, case examples, and experiments?
\end{itemize}
QA4 originates from an open perspective in the review process, where we focused on evaluation, discussion, and explanation. Our review also examined the study’s overall structure, including the methodology description, case studies, experiments, and analyses. The detailed results of the quality assessment are provided in the Appendix. Following this assessment, the final data extraction was based on 156 papers.

% \begin{table}[]
% \begin{tabular}{ll}
% \hline
% QA ID & QA Questions                                             \\ \hline
% Q01   & Does this study clearly state its aims?                  \\
% Q02   & Does this study clearly describe its methodology?        \\
% Q03   & Does this study involve automated test generation?       \\
% Q04   & Does this study include a promising evaluation?          \\
% Q05   & Does this study demonstrate the usability of the method? \\ \hline
% \end{tabular}%
% \caption{Questions for Quality Assessment}
% \label{table:qa}
% \end{table}

% automated quality assessment

% \textcolor{blue}{CA: Our search strategy focused on identifying requirements types first. We covered several sources, e.g., ~\cite{Pohl:11,wagner2019status} to identify different formats and notations of specifying requirements. However, this came out to be a long list, e.g., free-form NL requirements, semi-formal UML models, free-from textual use case models, UML class diagrams, UML activity diagrams, and so on. In this paper, we attempted to primarily focus on requirements-related aspects and not design-level information. Hence, we generalised our search string to include generic keywords, e.g., requirement*, use case*, and user stor*. We did so to avoid missing out on any papers, bringing too restrictive in our search strategy, and not creating a too-generic search string with all the aforementioned formats to avoid getting results beyond our review's scope.}


%% Use \subsection commands to start a subsection.



%\subsection{Study Selection}

% In this step, we further looked into the content of searched papers using our search strategy and applied our inclusion and exclusion criteria. Our filtering strategy aimed to pinpoint studies focused on requirements-driven system-level testing. Recognizing the presence of irrelevant papers in our search results, we established detailed selection criteria for preliminary inclusion and exclusion, as shown in Table \ref{table: criteria}. Specifically, we further developed the taxonomy schema to exclude two types of studies that did not meet the requirements for system-level testing: (1) studies supporting specification-driven test generation, such as UML-driven test generation, rather than requirements-driven testing, and (2) studies focusing on code-based test generation, such as requirement-driven code generation for unit testing.




\section{Problem definition}
\label{sec:problem-def}

The first step in building and using a \CSE{} or SciML model is defining the problem scope: the model's intended purpose, application domain and operating environment, required quantities of interest (QoI) and their scales, and how prior knowledge informs model conceptualization.

\subsection{Model purpose}

\begin{essrec}[Specify prior knowledge and model purpose]
Define the model's intended use and document the essential model properties that must be satisfied. Document any known limitations and constraints of the chosen approach. This ensures appropriate data selection and physics-informed objectives while preventing model misuse outside its intended scope.
\end{essrec}

A SciML model's purpose, as discussed in Section~\ref{sec:scope}, dictates all subsequent modeling choices.
This purpose determines required outer-loop processes and essential properties.

To highlight the importance of specifying the target outer-loop process, consider a model used for explanatory modeling. An explanatory model must simulate all system processes, like ice-sheet thickness and velocity evolution. In contrast, a risk assessment model needs only decision-relevant quantities, like ice-sheet mass loss under varying emissions scenarios. Design and control models, meanwhile, have different requirements than those for risk assessment.
The model purpose dictates the data types and formulations needed to train a SciML model. The impact of this purpose on data requirements and physics-informed objectives varies by application domain. Thus, the exact model formulation should be chosen in light of these problem-specific considerations.


\subsection{Verification, calibration, validation and application domain}

\begin{essrec}[Specify verification, calibration, validation, and application domains]
Define the specific conditions under which the model will operate across the verification, calibration, validation, and prediction phases. These domains are specified by relevant boundary conditions, forcing functions, geometry, and timescales. Account for potential differences between these domains and address any data distribution shifts that could affect model performance. This ensures the selected model architecture and training data align with the intended use while preventing unreliable predictions when operating outside validated conditions.
\end{essrec}

The trustworthy development and deployment of a model (see Figure~\ref{fig:model-development}) requires using the model in regards to verification, calibration, validation, and application domains. These domains are defined by the conditions under which the model operates during these respective phases and must be defined before model construction because they determine viable model classes. Key features include boundary conditions, forcing functions, geometry, and timescales. For ice sheets, examples include surface mass balance, land mass topography, and ocean temperatures.

Each domain will often require the prediction of different quantities of interest under different conditions. Moreover the complexity of the processes being modeled typically increase when transitioning from verification to calibration, to validation to prediction. Additionally the amount of data to complement or inform the model decreases as we move through these domains. For example, the verification domain for our ice-sheet examplar predicts the entire state of the ice-sheet for simple manufactured or analytical solutions. The calibration domain predicts Humboldt Glacier surface velocity under steady-state preindustrial conditions. The validation domain predicts grounding-line change rates from the first decade of this century. The application domain predicts glacier mass change in 2100. Figure~\ref{fig:computational-domains} illustrates these distinct domains. When transitioning between domains, data shifts across domains must be considered. For example, a model trained only on calibration data from recent ice-sheet forcings may fail to predict ice-sheet properties under different future conditions.

\begin{figure}[htb]
    \centering
    \includegraphics[width=0.65\linewidth]{application-domain.pdf}
    \caption{Verification, validation, calibration and application domains.}
    \label{fig:computational-domains}
\end{figure}


\subsection{Quantities of interest}

\begin{essrec}[Carefully select and specify the quantities of interest]
Select and specify the model outputs (quantities of interest, QoI) required for the intended use, considering their form and scale. For risk assessment and design applications, identify the minimal set of QoIs needed for decision-making or optimization. For explanatory modeling, specify the broader range of QoIs needed to capture system behavior. This choice fundamentally determines the required model complexity, training data requirements, and computational approach needed to achieve reliable predictions.
\end{essrec}

Quantities of interest (QoI) are the model outputs required by users. Their form and scale depend on modeling purpose and application domain. We now discuss key considerations for QoI selection.

Risk assessment requires reproducing only decision-critical QoI. For ice-sheets, these include sea-level rise from mass loss and infrastructure damage costs. Design applications similarly need few QoI to evaluate objectives and constraints, like thermal and structural stresses in aerospace vehicles. Design models need accurate QoI predictions only along optimizer trajectories\footnote{For each design iteration the model may still need to be accurate across all uncertain model inputs}, while risk assessment models must predict across many conditions. Explanatory modeling demands more extensive QoI sets, such as complete ice-sheet depth and velocity fields for studying calving. Therefore, simple surrogates often suffice for risk assessment and design, but explanatory modeling may require operators or reduced order models.


\subsection{Model conceptualization}


\begin{essrec}[Select and document model structure]
Select a model structure that fits the model's purpose, domain, and quantities of interest based on relevant prior knowledge such as conservation laws or system properties. Document the alternative model structures considered and the reasoning behind the final selection, including how available resources and computational constraints influenced the choice. This systematic approach ensures the model balances usability, reliability, and feasibility while maintaining transparency about structural assumptions and limitations.
\end{essrec}

Model conceptualization, which follows problem definition, involves selecting model structure based on prior knowledge. While essential to \CSE{} model development~\cite{Jakeman_LN_EMS_2006}, this step requires clear identification of the application domain and relevant QoI.

Model structure selection draws on key prior knowledge: conservation laws, system invariances like rotational and translational symmetries. These guide method selection---for example, symplectic time integrators~\cite{ruth1983canonical} preserve system dynamics properties. Moreover, this knowledge informs and justifies the selection of candidate model structures.
A \CSE{} modeler chooses between model types like lumped versus distributed PDE models, and linear versus nonlinear PDEs. The optimal choice depends on application domain, QoI, and available resources. For example, linear PDEs may introduce more error but their lower computational cost enables better error and uncertainty characterization for tasks like optimal design.
Similar considerations guide SciML model selection. For example, Gaussian processes excel at predicting scalar QoI with few inputs and limited data, but become intractable for larger datasets without variational inference approximations~\cite{Liu_CO_KBS_2018}. In contrast, deep neural networks handle high-dimensional data but require large datasets. The intended use also shapes model structure and training, e.g., optimization applications require controlling derivative errors~\cite{bouhlel2020scalable,o2024derivative} to ensure convergence~\cite{cao2024lazy,luo2023efficient}. These approximation errors must be understood and quantified where possible.

\CSE{} has a strong history of using prior knowledge to formulate governing equations for complex phenomena and deriving numerical methods that respect important physical properties. However, all models are approximate and the best model must balance usability, reliability, and feasibility~\cite{Hamilton_PSFJEMS_2022}. While SciML methods can be usable and feasible, more attention is needed to establish their trustworthiness. In the following two sections, we discuss how \CSE{} V\&V can improve the trustworthiness of SciML research.

\section{Verification}
\label{sec:verification}

Verification increases the trustworthiness of numerical models by demonstrating that the numerical method can adequately solve the equations of the desired mathematical model and the code correctly implements the algorithm. Verification consists of code verification and solution verification, which enhance credibility and trust in the model's predictions. Code and solution verification are well-established in \CSE{} to reduce algorithmic errors. However, verification for SciML models has received less attention due to the field's young age and unique challenges. Moreoever, because SciML models heavily rely on data, unlike \CSE{} models, existing \CSE{} verification notions need to be adapted for SciML.

\subsection{Code verification}
\label{sec:code-verification}

\begin{essrec}[Verify code implementation with test problems]
Evaluate the SciML model's accuracy on simple manufactured test problems using verification data that is independent from training data. Assess how the model error responds to variations in training data samples and optimization parameters while increasing both model complexity and training data size. This systematic testing approach reveals implementation issues, quantifies the impact of sampling and optimization choices, and establishes confidence in the model's numerical implementation.
\end{essrec}

Code verification ensures that a computer code correctly implements the intended mathematical model. For \CSE{} models, this involves confirming that numerical methods and algorithms are free from programming errors (``bugs"). PDE-based \CSE{} models commonly use the method of manufactured solutions (MMS) to verify code on arbitrarily complex solutions. MMS substitutes a user-provided solution into the governing equations, then differentiates it to obtain the exact forcing function and boundary conditions. These solutions check if the code produces the known theoretical convergence rate as the numerical discretization is refined. If the observed order of convergence is less than theoretical, causes such as software bugs, insufficient mesh refinement, or singularities and discontinuities affecting convergence must be identified.

Code verification for SciML models is important but challenging due to the large role of data and nonconvex numerical optimization. Three main challenges limit code verification for many SciML models.
First, while theoretical analysis of SciML models is increasing~\cite{schwab2019deep,schwab2023deep,opschoor2022exponential,leshno1993multilayer,lanthaler2023curse,kovachki2021universal,kovachki2023neural}, many models like neural networks do not generally admit known convergence rates outside specific map classes~\cite{schwab2019deep,schwab2023deep,opschoor2022exponential,herrmann2024neural}, despite their universal approximation properties~\cite{hornik1989multilayer,cybenko1989approximation,leshno1993multilayer}.
Second, generalizable procedures to refine models, such as consistently refining neural-network width and depth as data increases, do not exist.
Finally, regardless of data amount and model unknowns, modeling error often plateaus at a much higher level than machine precision due to nonconvex optimization issues like local minima and saddle points~\cite{Dauphin_PGCGB_NIPS_2014,Bottouleon_CN_SIAMR_2018}.

Developing theoretical and algorithmic advances to address the three main challenges limiting code verification can substantially improve the trustworthiness of SciML models. Convergence-based code verification is currently possible only for certain SciML models with theory that bounds approximation errors in terms of model complexity and training data amount, such as operator methods~\cite{Turnage_et_al_arxiv_2024}, polynomial chaos expansions~\cite{Cohen_M_SMAIJCM_2017,xiu2002wiener}, and Gaussian processes~\cite{Burt_RV_PMLR_2019}.

For SciML models without supporting theory, convergence tests should still be conducted and reported. Studies providing evidence of model convergence engender greater trustworthiness than those that do not, even when the empirically estimated convergence rate cannot be compared to theoretical rates. For example, observing Monte Carlo-type sampling rates in a regime of interest for a fixed overparametrized model can provide intuition into whether the model should be enhanced.

To account for the heavy reliance of SciML models on training data optimization, code verification should be adapted in two ways.
First, report errors in the ML model for a given complexity and data amount for different realizations of the training data to quantify the impact of sampling error, which is not present in \CSE{} models.
Second, because most SciML algorithms introduce optimization error, conduct verification studies that artificially generate data from a random realization of an ML model, then compare the recovered parameter values with the true parameter values or compare the predictions of the learned and true approximations, or at the very least compare the predictions of the two models. Additionally, quantify the sensitivity of the SciML model error to randomness in the optimization by varying the random seed and initial guess passed to the optimizer (see Section~\ref{sec:loss-and-opt}).
All verification tests must employ test data or \emph{verification data}, independent of the training data, to measure the accuracy of the SciML model.


\subsection{Solution verification}

\begin{essrec}[Verify solution accuracy with realistic benchmarks]
Test the model's performance on well-designed, realistic benchmark problems that reflect the intended application domain. Quantify how the model error varies with different training data samples and optimization parameters. When feasible, examine error patterns across different model complexities and data amounts; otherwise, focus on verifying the specific configuration intended for deployment. This ensures the model meets accuracy requirements under realistic conditions while accounting for uncertainties in training and optimization.
\end{essrec}

Code verification establishes a code's ability to reproduce known idealized solutions, while solution verification, performed after code verification, assesses the code's accuracy on more complex yet tractable problems defined by more realistic boundary conditions, forcing, and data. For example, code verification of ice sheets may use manufactured solutions, whereas solution verification may use more realistic MISMIP benchmarks~\cite{Cornford_et_al_TC_2020}. In solution verification, the numerical solution cannot be compared to a known exact solution, and the convergence rate to a known solution cannot be established. Instead, solution verification must use other procedures to estimate the error introduced by the numerical discretization.

Solution verification establishes whether the exact conditions of a model result in the expected theoretical convergence rate or if unexpected features like shocks or singularities prevent it. The most common approach for \CSE{} models compares the difference between consecutive solutions as the numerical discretization is refined and uses Richardson extrapolation to estimate errors. A posteriori error estimation techniques that require solving an adjoint equation can also be used.

While thorough solution verification of CSE models is challenging, these difficulties are further amplified for SciML models. Currently, solution verification of SciML models simply consists of evaluating a trained model's performance using test data separate from the training data. However, this is insufficient as solution verification requires quantifying the impact of increasing data and model complexity on model error. Yet, unfortunately, performing a posteriori error estimation for many SciML models using techniques like Richardson extrapolation is difficult due to the confounding of model expressivity, statistical sampling errors, and variability introduced by converging to local solutions or saddle points of nonconvex optimization, making it challenging to monotonically decrease the error of SciML models such as neural networks. 

Until convergence theory for SciML models improves and automated procedures are developed to change SciML model hyperparameters as data increases, solution verification of SciML models should repeat the sensitivity tests proposed for code verification (Section~\ref{sec:code-verification}) with two key differences:
First, verification experiments used to generate verification data must be specifically designed for solution verification, as not all verification data equally informs solution verification efforts, similar to observations made when creating validation datasets for \CSE{} models~\cite{Oberkampf_T_PAS_2002}. See Section~\ref{sec:data-sources} for more information on important properties of verification benchmarks.
Second, while ideally the convergence of SciML errors on realistic benchmarks should be investigated, it may be computationally impractical. Thus, solution verification should prioritize quantifying errors using the model complexity and data amount that will be used when deploying the SciML model to its application domain.

\section{Validation}
\label{sec:validation}

Verification establishes if a model can accurately produce the behavior of a system described by governing equations. In contrast, validation assesses whether a \CSE{} model's governing equations---or data for SciML models---and the model's implementation can reproduce the physical system's important properties, as determined by the model's purpose.

Validation requires three main steps: (1) solve an inverse problem to calibrate the model to observational data; (2) compare the model's output with observational data collected explicitly for validation; and (3) quantify the uncertainty in model predictions when interpolating or extrapolating from the validation domain to the application domain. We will expand on these steps below.
But first note that the issues affecting the verification of SciML models also affect calibration and validation. Consequently, we will not revisit them here but rather will highlight the unique challenges in validating SciML models.

\subsection{Calibration}

\begin{essrec}[Perform probabilistic calibration]
Calibrate the trained SciML model using observational data to optimize its predictive accuracy for the application domain. Implement Bayesian inference when possible to generate probabilistic parameter estimates and quantify model uncertainty. Choose calibration metrics that account for both model and experimental uncertainties, and select calibration data strategically to maximize information content within experimental constraints. This approach enables reliable uncertainty estimation and optimal use of available observational data.
\end{essrec}

Once a \CSE{} model has been verified, it must be calibrated to match experimental data that contains observational noise. This calibration requires solving an inverse problem~\cite{Stuart_AN_2010}, which can be either deterministic or statistical (e.g., Bayesian). The deterministic approach formulates the inverse problem as a (nonlinear) optimization problem that minimizes the mismatch between model and experimental data. This formulation requires regularization to ensure well-posedness, typically chosen using the L-curve~\cite{hansen1999curve} or the Morozov discrepancy principle~\cite{anzengruber2009morozov}. The Bayesian approach replaces the misfit with a likelihood function based on the noise model, while using a prior distribution for regularization. This prior distribution ensures well-posedness while encoding typical parameter ranges and correlation lengths. We recommend Bayesian methods for calibration because they provide insight into the uncertainty of the reconstructed model parameters. 

The calibration of SciML operator, reduced-order, and hybrid CSE-SciML models is distinct from SciML training and follows similar principles to \CSE{} model calibration. These models are first trained using simulation data for solution verification. Next, observational data (called \emph{calibration data}) determines the optimal model input values that match experimental outputs. For instance, calibrating a SciML ice-sheet model such as that of Ref.~\cite{He_PKS_JCP_2023} requires finding optimal friction field parameters of a trained SciML model, which best predict observed glacier surface velocities, given the noise in the observational data.

Calibration typically improves a model's predictive accuracy on its application domain, but the informative value of calibration data varies significantly. Therefore, researchers should select calibration data strategically to maximize information content within their experimental budget. See Section~\ref{sec:data-sources} for further discussion on collecting informative data.

\subsection{Model validation}

\begin{essrec}[Validate model against purpose-specific requirements]
Define validation metrics that align with the model's intended purpose. Then validate the model using independent data that was not used for training or calibration, ensuring it captures essential physics and boundary conditions of interest. If validation reveals inadequate performance, iterate by collecting additional training data, refining the model structure, or gathering more calibration data until the model achieves satisfactory accuracy for its intended application. This systematic approach will help ensure the model meets stakeholder requirements while maintaining scientific rigor.
\end{essrec}

Model validation is the ``substantiation that a model within its domain of applicability possesses a satisfactory range of accuracy consistent with the intended application of the model''~\cite{Refsgaard_H_AWR_2004}. Validation involves comparing computational results with observational data, then determining if the agreement meets the model's intended purpose~\cite{Lee_et_all_AIAA_2016}. For \CSE{} models with unacceptable validation agreement, modelers must either collect additional calibration data or refine the model structure until reaching acceptable accuracy. SciML models follow a similar iterative process but offer an additional option: to collect more training data.

Model validation must occur after calibration and requires independent data not used for calibration or training. For our conceptual ice-sheet model, calibration matches surface velocities assumed to represent pre-industrial conditions, while validation assesses the calibrated model's ability to predict grounding line change rates at the start of this decade. Performance metrics must target the specific modeling purpose. For optimization tasks, metrics should measure the distance from true optima obtained via the SciML model or bound the associated error~\cite{cao2024lazy}. For uncertainty estimation, metrics should quantify errors in uncertainty statistics through moment discrepancies or density-based measures like (shifted) reverse and forward Kullback--Leibler divergences.
For explanatory SciML modeling, validation metrics must also assess physical fidelity: adherence to physical laws, conservation properties (such as mass and energy), and other constraints. As with verification, the validation concept should encompass \emph{data validation}, particularly whether training data adequately represents the application space.

Validation determines whether a model is acceptable for its specific purpose rather than universally correct. The definition of acceptable is subjective, depending on validation metrics and accuracy requirements established by model stakeholders in alignment with the problem definition and model purpose (see Section~\ref{sec:problem-def}). Moreoever, validation itself does not constitute final model acceptance, which must be based on model accuracy in the application domain, as discussed in Section~\ref{sec:prediction}.

Two additional considerations complete our discussion of model validation. First, this validation differs from the concept of \emph{cross validation}, which estimates ML model accuracy on data representative of the training domain during development. The validation described here assesses accuracy in a separate validation domain. Second, validation data varies in informative value. Validation experiments should ``capture the essential physics of interest, including all relevant physical modeling data and initial and boundary conditions required by the code''~\cite{Oberkampf_T_NED_2008}. Most critically, validation data must remain independent from training and calibration data. 

\subsection{Prediction}
\label{sec:prediction}

\begin{essrec}[Quantify prediction uncertainties]
Assess and quantify all sources of uncertainty affecting model predictions in the application domain, including numerical approximation errors, input and parameter uncertainties, sampling errors from finite training data, and optimization errors. Propagate these uncertainties through the model using appropriate techniques to estimate relevant statistics that match validation criteria. Define acceptance thresholds for prediction uncertainty to ensure the model's reliability for its intended use while acknowledging inherent limitations in uncertainty quantification.
\end{essrec}

Although extensive data may be available for model calibration, validation data is typically scarcer and may not represent the model's intended application domain. According to Schwer~\cite{Schwer_EWC_2007}, ``The original reason for developing a model was to make predictions for applications of the model where no experimental data could, or would, be obtained.'' Therefore, minimizing validation metrics at nominal conditions cannot sufficiently validate a model. Modelers must also quantify accuracy and uncertainty when predictions are extrapolated to the application domain.

SciML models, like \CSE{} models, are subject to numerous sources of uncertainty. The impact of these uncertainties on model predictions must be quantified. Several sources of uncertainty affect \CSE{} models. These include: numerical errors, from approximating the solution to governing equations; input uncertainty arises, which is caused by inexact knowledge of model inputs; parameter uncertainty, which stems from inexact knowledge of model coefficients; and model structure error representing the difference between the model and reality. SciML models contain all these uncertainties. They also incorporate additional uncertainties from sampling and optimization errors, as discussed previously.

Sampling error arises from training a model with a finite amount of possibly noisy data. For a fixed ML model structure and zero optimization error, this error decreases as the amount of data increases. Optimization error represents the difference between the optimized solution, which is often a local optimum, and the global solution for fixed training data. Optimization error can enter \CSE{} models during calibration. Optimization error affects SciML models more significantly because it occurs both during calibration and training. Linear approximations, for example, based on polynomials, achieve zero optimization error during training to machine precision. However, nonlinear approximations such as neural networks often produce non-trivial optimization errors. Stochastic gradient descent demonstrates this by producing different parameter estimates due to stochastic optimization randomness and initial guesses.

The identified sources of modeling uncertainty require parameterization for sampling. Expert knowledge typically guides the construction of prior distributions that represent parametric uncertainty. This parameterization should occur during problem definition. Bayesian calibration updates these priors into posterior distributions using calibration data. The model must then propagate all uncertainties onto predictions in the application domain. Monte Carlo quadrature accomplishes this propagation by drawing random samples from the uncertainty distributions. The method collects model predictions at these samples and computes empirical estimates of important statistics defined by validation criteria, such as mean and variance.

We emphasize the impact of all sources of error and uncertainty must be quantified. Simply estimating the impact of error caused by using finite sample sets, for example estimated by generative models such as variational autoencoders of Gaussian processes is insufficient. Moreover, complete elimination of uncertainty is impossible. Consequently, model acceptance, like validation, must rely on subjective accuracy criteria established through stakeholder communication. For instance, acceptance criteria for predicted sea-level change from melting ice-sheets at year 2100 may specify that prediction precision reaches 1\% of the mean value. Yet, engineering applications, such as those focused on aerospace design, may have much higher accuracy requirements.

The aforementioned Monte-Carlo based UQ procedure effectively quantifies the impact of parameterized uncertainties on model predictions. However, model structure error remains difficult to parameterize in both SciML and \CSE{} modeling. Validation can partially assess model structure error. However, experiments rarely cover all conditions of use. Specifically, validation tests only the model's interpolation ability within the convex hull of available data and assumptions. This limitation creates challenges when applying the model outside its validation domain. Some progress exists in quantifying extrapolation error for ``models based upon highly-reliable theory that is augmented with less-reliable embedded models''~\cite{Oliver_TSM_CMAME_2015}. However, such hybrid CSE-SciML models rely on well-established physics-based governing equations to support extrapolation confidence. Pure SciML models still require substantial research to develop reliable methods for estimating model structure uncertainty.


\section{Case Study on SafeSPLE}
We now demonstrate via a case study one way to implement a SafeSPL and parameterized safety cases.  The first part of our process is a hazard analysis. We then build a feature model. The features are then used to parameterize our safety case. Lastly, we can generate safety-case instances as requested for any of the concrete combinations of features.  

\subsection{Hazard Analysis}
To begin the SafeSPLE process for a UAS flight, we analyze the hazards of that flight, which is an important first step before creating a safety case \cite{Knig12}. A hazard is a state or event that can potentially result in an accident \cite{ericson2015hazard}. In this work, we do not describe this part of the process in depth but rather list a few of the key hazards we identified. We utilized several sources to create our list of hazards. First, we referenced several papers describing hazard analysis or safety cases for sUAS flights \cite{denpai2016, clodenpai2017, sora}. Next, we discussed sUAS hazards with colleagues and experts who have studied sUAS and flown them. This investigation gave us an extensive list of hazards, which was too long and broad to include in this paper. We narrowed down this extensive list to focus on the following hazards. 

\begin{itemize}
    \item Too much precipitation
    \item Insufficient visibility
    \item Temperatures outside the operating specifications of the sUAS
    \item Wind gusts outside the operating specifications of the sUAS
    \item Insufficient battery for the mission
\end{itemize}

The hazards above are not intended to be fully described or defined, and we do not include prevention or recovery controls or escalation factors for any of these hazards (see \cite{denpai2016} for a more in-depth discussion of hazard analysis). The ultimate consequences of each of the above hazards are generally either loss of separation from the ground or loss of separation from other air traffic. Either of these consequences could lead to the destruction of property, injury, or death. A complete risk analysis of these consequences is likewise beyond the scope of this paper. We illustrate our family-based approach below using a subset of the identified hazards in order to show how the parameterized safety case addresses the hazards for different sUAS.       

\subsection{Feature Model}

\begin{figure}[ht]
    \centering
    \includegraphics[width=.8\textwidth]{Figures/safety-case-blow-up.pdf}
    \caption{Two parts of the feature model that we focus on for this case study.}
    \label{fig:feature_model_focus}
\end{figure}

The next step in the SafeSPLE process is to create a partial feature model that could apply to a wide variety of sUAS models and missions in controlled airspace as described in Section \ref{sec:SafeSPLE} and Figure \ref{fig:featuremodel}. Since this feature model includes information about the pilot, airspace, mission, vehicle, and weather (among other things), it allows for a wide variety of different types of parameters to be used in our parameterized safety case. In figure \ref{fig:feature_model_focus} we show the two parts of the feature model that are the focus of our safety cases here - the pilot and the weather. These parameterized safety cases are described in the next section. 


\subsection{Parameterized Safety Case}

\begin{figure}[ht]
    \centering
    \includegraphics[width=.7\textwidth]{Figures/pilot_only.pdf}
    \caption{Pilot Safety Case: A safety case based only on whether the pilot is certified and has sufficient experience.}
    \label{fig:pilot_only}
\end{figure}

The next step in our case study (based on SafeSPLE) is to create two illustrative parameterized safety cases for our controlled airspace. The first safety case, seen in Figure \ref{fig:pilot_only}, is based solely on the pilot. It checks whether the pilot is certified and has sufficient flight hours. We assume that in non-commercial airspaces, flight regulations would trust a certified pilot with sufficient reputation (i.e., no significant history of problems) to perform safety checks consistent with the lower-level details of our safety cases. In other words, the pilot is in charge of ensuring a safe flight in whatever airspace they are in. Regulators often do not exclude pilots legally allowed to be in the airspace unless there is some serious prior issue \cite{FAA_TRUST, FAA_part107}. So it is our belief that any UAS Traffic Management system will likely allow certified pilots to enter the airspace unless it has some reason not to.

As shown in Figure \ref{fig:pilot_only}, our safety case checks to see if the pilot is certified to fly their sUAS, here represented using the FAA's Part 107 certification \cite{FAA_part107}. We also check to see if the pilot has sufficient flight hours to be competent to complete this flight, which is something that our managed airspace should know.  In the future, this flight-hours check might be replaced or augmented with different checks, such as the pilot's score on a competency-reputation metric, future certifications, or temporary notices to pilots that the FAA might put out. If evidence of these checks confirms that the pilot is certified and has sufficient experience to enter the controlled airspace, the associated strategy node (S1) in the safety case is satisfied.

\begin{figure}[ht]
    \centering
    \includegraphics[width=\columnwidth]{Figures/wind_only.pdf}
    \caption{Wind Safety Case: A parameterized safety case based only on the weather and the drone's capabilities. This safety case creates the instances seen in Figures \ref{fig:instance_1} and \ref{fig:instance_2}.}
    \label{fig:wind_only}
\end{figure}

Our second safety case is relevant when the pilot lacks the evidence required to satisfy our initial safety case above. There needs to be an opportunity for newer pilots to learn and fly if such flights can be done safely. Thus, our second safety case focuses on giving such pilots the information that they will need in order to complete a safe flight. This second safety case (Figure \ref{fig:wind_only}) focuses on the weather because poor weather is a common reason for a pilot to decide that a flight will not be safe or for in-flight failures \cite{weather_hazards_for_UAV}. The weather portion of the feature model also has several parameters that can map to portions of our safety case. This sort of weather-focused safety case would normally involve far more attributes than we show in Figure \ref{fig:wind_only}, but we focus only on the weather and a small amount of information (evidence) about the battery here.

In the wind safety case from Figure \ref{fig:wind_only}, we constructed a general safety case that involves a number of parameters that are found in our feature model. These parameters are indicated using square brackets, such as [Precipitation] and [UASAllowedPrecipitation]. The data types of each parameter are left intentionally vague, as there are a number of ways for these parameters to be stored. We assume that information about each [Vehicle] is publicly available and that published sUAS specifications can be converted into the same data format and type that the feature model and safety case parameters have. If a [Vehicle] does not contain information in its specifications for certain parameters, then there is an option to assume some default values that could apply to almost all drones. 

For instance, most sUAS specifications will include information about the maximum allowed wind speed within which the manufacturer states the sUAS can operate. Likewise, most sUAS specifications include both maximum and minimum allowed temperatures in which to operate (often from -10 \textcelsius \;or 0 \textcelsius \;up to 40 \textcelsius) \cite{DEERCD20, DJI_MiniPro_4_Specs}. Fewer sUAS specifications contain specific information about visibility requirements since those depend on the type of mission being flown, especially whether it needs to be flown in a visual line of sight (VLOS) or beyond a visual line of sight (BVLOS). If the pilot does not provide visibility requirement information, we thus assume that the flight must take place VLOS and proceed accordingly. Similarly, if no information is provided about an sUAS's ability to fly in various forms of precipitation, we assume that the sUAS can only operate with no precipitation. 

Note that in the wind safety case (Figure \ref{fig:wind_only}), many of the goals share a similar structure. For instance, "The forecast precipitation is within acceptable level..." and "The forecast visibility is within acceptable levels...". The repetition of these elements is intentional and allows for greater ease of human understanding of the safety case, as well as for simpler extension of the safety case when we add additional hazards we need to mitigate. 

Some of the values of the parameters in the safety case may not be available at the time of a flight request. For instance, if a pilot is applying to complete a flight several weeks or months in the future, the forecast weather conditions will be unreliable. In such a case, the safety case might not contain concrete values until closer to the flight. The pilot could still access the parameterized safety case in order to study the safety requirements for the flight. As the time of the flight approaches, a more fully instantiated safety case could be sent to the pilot. 

The information for instantiating these parameterized safety cases will need to be pulled from a variety of sources, such as publicly available weather data and manufacturers' specifications for commercially available sUAS. However, some of the parameters' information will need to come from the pilot, including their certification status, the sUAS model they will fly, their flight plan, and any additional sUAS capabilities they have added (such as detect-and-avoid systems). 
In the event that the sUAS being flown was completely home-built, there may be no public documentation of its abilities, and all of its specifications will need to be provided (or inferred) by the pilot. 
Therefore, some of the individual safety cases will necessarily contain a fair amount of uncertainty while still serving as a guideline for the pilot. 


\subsection{Instances}
As a final step in our SafeSPLE process, we demonstrate how to create instances of our parameterized safety case. This process involves obtaining the information required for all parameters and checking if all the solution nodes of the safety case remain true. In all of the safety case diagrams in Figures \ref{fig:pilot_only}, \ref{fig:wind_only}, \ref{fig:instance_1}, \ref{fig:instance_2}, these solution nodes are the bottom nodes labeled E1-E6, and have propagated from the context. If any solution node becomes false, then we can say that the pilot should either reconsider the flight, or should implement further mitigations to reduce the risk from the relevant hazard. For instance, if the safety case shows that the current wind gusts are too high, the pilot might delay the flight until the wind calms, or the pilot might decide to make the flight with a larger and more capable UAS (if available).

\begin{figure}[ht]
    \centering
    \includegraphics[width=.95\textwidth]{Figures/Instance_1.pdf}
    \caption{Safety Case Instance 1: An instance of the wind safety case (Figure \ref{fig:wind_only}) based on a mission with a DJI Mini 4 Pro drone.}
    \label{fig:instance_1}
\end{figure}

The first instance of our parameterized safety case is shown in Figure \ref{fig:instance_1}. This mission will be performed by a DJI Mini 4 Pro, a widely available drone that currently sells for just over \$1000, depending on accessories. The Mini 4 Pro is fully charged, and the mission, as planned, should take 16 minutes, flown entirely within VLOS of the pilot. This information about battery charge and the mission plan is provided by the pilot. The wind is gusting up to 6 meters/sec, with temperatures in the mid 20s \textcelsius, unlimited visibility, and no precipitation. This weather information is provided to the safety case by a commercial or governmental weather service. 

Once we obtain the information about the make and model of the drone, we can look up the DJI's published specifications. According to DJI \cite{DJI_MiniPro_4_Specs}, the Mini 4 Pro is able to fly in wind speeds up to 10 m/s, and with a fully charged battery can fly up to 34 minutes. The Mini 4 Pro can operate in temperatures between -10 \textcelsius \;and 40 \textcelsius. Using all this information, we can instantiate the safety case seen in Figure \ref{fig:instance_1}. Note that every solution node 
(labeled E1-E6) is satisfied by the above information. There is no precipitation; visibility is unlimited; the temperatures are not too hot or cold; the wind gusts are below the max allowed for the drone; and the battery reserves are more than twice as much as needed. So in this instance of the safety case the the top-level goal is satisfied. 

\begin{figure}[ht]
    \centering
    \includegraphics[width=.95\textwidth]{Figures/Instance_2.pdf}
    \caption{Safety Case Instance 2: An instance of the wind safety case (Figure \ref{fig:wind_only}) based on a mission with a DEERC D20 drone. Note that this instance fails to fulfill our safety requirements at node E4 (marked in darker red).}
    \label{fig:instance_2}
\end{figure}
%Nice examples!

In Figure \ref{fig:instance_2} we can see a second instance of our safety case. This mission will be performed by a DEERC D20 drone, another widely available drone that currently sells for around \$50. The D20 is also fully charged, and the planned mission will only take 5 minutes of flying, entirely within VLOS. The wind is gusting up to 8 m/s, with temperatures in the mid-30s \textcelsius, 3 km visibility, and no precipitation.

According to the DEERC documentation \cite{DEERCD20}, the D20 drone is capable of about 10 minutes of flight time in temperatures between 0 \textcelsius \ and 40 \textcelsius. However, the D20 documentation does not specify the maximum speed of the winds that the drone is capable of flying in. Instead, the documentation reads, "DO NOT use this drone in adverse weather conditions such as rain, snow, fog, and wind." Therefore the safety case takes a conservative approach and assigns a default value of 3 m/s to the variable [MaxAllowedWindSpd] (3 m/s is slightly less than 7 mph). This default value could, of course, be set to 0 m/s, although this seems unrealistic for most outdoor flying. Other default values might be justified.

Plugging in all of these values, we see that while most solution nodes are satisfied, the current wind conditions (gusts up to 8 m/s) do not allow for a safe flight with the D20 (default max wind speed of 3 m/s). In Figure \ref{fig:instance_2}, this is shown at solution node E4, which is colored a darker red than the other solution nodes. The safety case is designed to serve as input to the UTM on-entry decision. At this point there are two main options for how the UAS Traffic Manager could behave. The UTM could refuse entry to this pilot until the wind speed is lower, or the UTM could send the safety case to the pilot with the recommendation that the pilot make modifications to the flight plan while leaving the ultimate flight decision up to the pilot.

Creating instances of safety cases with SafeSPL should be quick and relatively straightforward, if the information it needs is available. If information on the drone's capabilities is lacking, default values can still allow the safety case to create a reasonable instance. If information about the weather is unknown, then those portions of the safety case can be left uninstantiated until more detailed information becomes available. At the very least, we can generate a partially instantiated safety case so the pilot can see the areas where information is lacking or is based on default values. This information could allow the pilot to focus on mitigation measures in those areas if needed. 

\subsection{Connecting to Safe Entry}
\label{sec:safe_entry}

The parameterized safety cases created by SafeSPLE and described above could play an important role in a to-be-developed UTM system. When a pilot requests permission to fly in the airspace controlled by the UTM, the information needed to instantiate the safety case is either submitted by the pilot or looked up by the UTM system. Once a safety case has been created for that flight, there are at least two options for what the UTM system might do with it. 

\begin{enumerate}
    \item Closed Access: The UTM system accepts or denies requests based on whether each generated safety case "passes" or "fails". In other words, if the safety case goals are not satisfied, the UTM system  denies the flight. 
    \item Open Access: The UTM system accepts or denies the flight based solely on whether the pilot is certified or trusted. The safety case then becomes a guideline that can be provided to the pilot as something of a checklist to encourage a safer flight.
\end{enumerate}

Which action the UTM should take is an ongoing discussion with no immediate correct answer.
Currently the regulations in the US appear to generally favor approach (2), the open-access model. Regardless of which approach is taken for a specific controlled airspace, we believe the use of SafeSPLE will generate valuable on-the-fly information.  This information may offer an effective and useful checklist for decision-making. 

\section{Conclusion}
In this work, we propose a simple yet effective approach, called SMILE, for graph few-shot learning with fewer tasks. Specifically, we introduce a novel dual-level mixup strategy, including within-task and across-task mixup, for enriching the diversity of nodes within each task and the diversity of tasks. Also, we incorporate the degree-based prior information to learn expressive node embeddings. Theoretically, we prove that SMILE effectively enhances the model's generalization performance. Empirically, we conduct extensive experiments on multiple benchmarks and the results suggest that SMILE significantly outperforms other baselines, including both in-domain and cross-domain few-shot settings.


\section{Acknowledgment}
\noindent
\begin{minipage}{0.60\columnwidth}%
This work has received funding from the \ac{BMBF} under project funding reference 03SF0694A.
\end{minipage}
\hspace{0.02\columnwidth}
\begin{minipage}{0.35\columnwidth}%
    \includegraphics[width=\textwidth]{BMBF}
\end{minipage}
\vspace{-1em}
\bibliographystyle{iet}
\bibliography{references.bib}

\end{document}
