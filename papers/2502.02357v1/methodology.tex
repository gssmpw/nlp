\section{Modelling Approach}

The goal of the graph model is to not only describe the power system but also the \ac{ICT} infrastructure, and the functional design of the cyber-physical system.
The reason is that the resulting model should be a holistic approach for all aspects of the smart grid and should be expandable to future applications.
Furthermore, a goal of this model is to provide a basis for exchange scenarios in the research community.

\subsection{Smart Grid Architecture Model}
\label{subsec:sgam}

We chose to start off with an existing model, which offers a categorisation we can build up on and therefore ensure that an enhancement of the model happens in a consistent and interoperable way.
The \ac{SGAM} is categorised along three dimensions(Fig.: \ref{fig:sgam}) the \textit{Domains}, \textit{Zones} and, the \textit{Interoperability Layer}, the individual objects in our model are mapped to these dimensions\cite{sgam2012}.
The intent of \ac{SGAM} is a holistic approach to model use cases in smart grids.
Therefore, the original use case differs from the one in this paper as we want to model a system with instantiated models, not just a generic use case.
An advantage of this approach is that use cases, developed with \ac{SGAM}, can easily be transferred to models for analysis.
The main advantage of using \ac{SGAM} is its division into various interoperability layers.
This allows, for example, the examination of information flows independently of the protocols used, enabling a more universally applicable investigation. Similarly, the functions that different actors in an application have are considered independently of the hardware.
This stringent structuring allows for the impact analysis of different attack-vectors on various subareas of the cyber-physical energy system, such as a specific protocol or an attack on a particular functional actor.

\begin{figure}[H]
	\centering
	\includegraphics[width=\linewidth]{sgam_exp.png}
	\caption{
		Visualisation of the Smart Grid Architecture Model showing the three-dimensional structure of the model\protect\cite{sgam2012}.
	}
	\label{fig:sgam}
\end{figure}

\subsection{Closed World Assumption}
\label{subsec:cwa}
The \ac{OWA} and \ac{CWA} are two contrasting principles on how to model data.
Hereby the \ac{OWA} allows unknown or unrepresented data, so the absence of information is not considered evidence of absence and, therefore, not as invalid.
The \ac{OWA} is often used in \ac{SW} applications.
Still, because we need to verify that all the needed information for the later applications exists in the appropriate form, we follow the \ac{CWA} considering that a statement is true only if it is explicitly stated and all unknown information is considered false.
\ac{SHACL} allows us to follow the \ac{CWA}.


\section{Semantic Web Technologies}
\label{sec:technologies}

In the following, we will present \ac{SW} based technology, used in this paper.
\ac{SW} references the idea of a version of the internet where the information is linked and machine-readable so that the knowledge can be accessed and used by software tools.
To realise this vision, different technologies have been developed over the years, where the main challenges they tried to solve have been interoperability between different data sources and domains of knowledge.

\subsection{Resource Description Framework}
\label{subsec:rdf}

The \ac{RDF} is the main standard in the field of \ac{SW} to represent the information and the relationship between different objects.
Every information is represented by a triple consisting of a \textit{Subject}, \textit{Predicate} and, \textit{Object}.
An example would be \textit{"John hasAge 25"}: the resource John has the attribute age which has the value 25.
Instead of a value, the object can be another resource which creates a \ac{KG} representing the relationship between the different references.
Graphs have been proven to be a sophisticated way to model smart grids, in comparison to relational databases\cite{klaer_sg3_2020,jain_semantic_2023}.
The references used in \ac{RDF} are \ac{URI}s, so the linkage of data even outside the \ac{KG} is possible.
\ac{RDFS} are a mechanism in \ac{RDF} to define classes, properties and, basic constraints in \ac{KG}, therefore allowing us to build class hierarchies.
Because of the wide usage of \ac{RDF}, frameworks and tools for handling the data and automated reasoning are widely available.
As well as a high compatibility with different data formats for storage like \textit{XML} and \textit{TURTL}, listing \ref{lis:statgen} shows an extract from such a \textit{TURTL} file.

\subsection{SPARQL Protocol And RDF Query Language}
\label{subsec:sparql}

\ac{SPARQL} is designed to interact with \ac{RDF} graphs and supports simple and complex graph pattern matching which enables sophisticated querying capabilities to analyse relationships between objects.
In addition to information extraction, \ac{SPARQL} also allows graph modification by adding, deleting, and changing data. 

\subsection{Shapes Constraint Language}
\label{subsec:shacl}

An important challenge of the \ac{SW} is to ensure data quality and interoperability, to tackle this, constraints can be specified and validated with \ac{SHACL}.
Although \ac{RDFS} already provides capabilities for simple constraints,  SHACL provides a formal syntax and vocabulary for expressing these constraints and offers a more expressive and flexible validation mechanism compared to \ac{RDFS}.
Shapes define the expected patterns and constraints, which can be rules like required properties, data types and, value ranges.
An example where these are more advanced than \ac{RDFS} is checking for cardinality and value ranges which is essential for the \ac{KG} we want to build.
Because the constraints are defined separately from the data graph, different shapes can be applied for the validation of a data graph depending on the context of the investigation, for example, if a detailed model of the communication network is needed or just abstract communication channels.

\section{Ontology}
\label{sec:data_model}

The ontology section outlines the way we modelled the data for this paper, split into the domains: power system, communication network and operational technology.
To do so, we show components of the ontology based on \ac{SGAM}, entailing the interoperability layers which we use to organise the \ac{KG}.
Using this approach enables the independent design of the interoperability layers for rapid prototyping and error detection.

\subsection{Power System}
\label{subsec:powersystem}
The degree of detail that we envision is to model the component layer describing the power system with power flow calculations in a stationary time horizon.
Therefore we begin by modelling the power system in sufficient detail to enable these simulations, the data is mainly in the \textit{Component Layer}.
However, it is important to note that the model can be enhanced in the future to address additional needs.
We started off by modelling the power system based on the data model of \textit{PandaPower}\cite{pandapower_2018} as one of the most common open source frameworks used, so the compatibility to import models from other tools is quite high.
The original data model is a relational model, in contrast to the graph model we use, which closely resembles the character of the system and allows us to use graph algorithms.
Furthermore \ac{RDFS} enables us to build a class hierarchy with inheritance and \ac{SHACL} lets us define specific constraints for the classes.
This already exhibits several benefits over using the original model because, as with the \ac{SHACL} based validation, errors in the \ac{KG} can be found, which wouldn't be checked otherwise, ensuring data quality.
Listing \ref{lis:statgen} shows the \textit{Turtl} representation of a static generator; it defines the parent class in \ac{RDFS} and the properties, including the allowed value range, for example, the active power has to be a negative value because positive values are defined as power consumption.

\begin{lstlisting}[language=turtle, caption={Representation of the \ac{RDFS} and \ac{SHACL} shapes of a static genarotor in \textit{Turtl}.}, label={lis:statgen}]
errol:StaticGenerator
  rdf:type rdfs:Class ;
  rdf:type sh:NodeShape ;
  rdfs:label "Static generator" ;
  rdfs:subClassOf errol:GenerationConsumption ;
  sh:and (
      [
        sh:path errol:p_mw ;
        sh:maxInclusive "0"^^xsd:decimal ;
      ]
    ) ;
  sh:property [
      rdf:type sh:PropertyShape ;
      sh:path errol:q_mvar ;
      sh:datatype xsd:decimal ;
      sh:description "reactive power" ;
      sh:maxCount 1 ;
      sh:minCount 1 ;
      sh:name "q mvar" ;
    ] ;
  sh:property [
      rdf:type sh:PropertyShape ;
      sh:path errol:type ;
      sh:datatype xsd:string ;
      sh:description "sgen type" ;
      sh:maxCount 1 ;
      sh:name "type" ;
    ] ;
  sh:targetClass errol:StaticGenerator ;
\end{lstlisting}

\subsection{Control Infrastructure}
\label{subsec:control}

The control infrastructure of the layers \textit{function}, \textit{Information}, and \textit{communication} but also of the physical \ac{ICT} infrastructure like hosts the systems are running on and the network components.
Because we manly focus on internet based solutions the internal infrastructure of the internet service provider is out of scope.
Figure \ref{fig:kg} shows the simplified \ac{KG} used in this paper for one household with a remote controllable heat pump.
\textit{errol}, the acronym for "Energy Resilience Research Ontology Library", is the namespace used for the newly developed ontology and indicates classes and predicates we defined.
\textit{errol:HouseHold} is a subclass of \textit{errol:AssetGroups}, which implements the idea of \textit{Zones} from \ac{SGAM} for an instantiated model, in this example an household is equivalent to \textit{station}.
Functional actors are elements which represent one functional entity in the system like a \ac{HEMS}.
Hereby the functions can be abstracted from the technical implementation, an functional actor may be implemented on a specific controller or may be implemented in a distributed manner.
\cite{neureiter_domain-specific_2017} describes a similar implementation to achieve the abstraction of the functional layer.
Functional blocks are the concrete implementations of individual functions of a functional actor.
A functional actor can hold multiple functional blocks simultaneously.
In a \ac{HEMS}, for example, one functional block might be self-consumption optimisation, while another functional block could be the remote controllability of a smart home.
In the example shown in figure \ref{fig:kg}, there is a functional actor for the cloud backend of the HEMS provider and another for the HEMS within the smart home. Each of these has a functional block that implements remote controllability.
The relationship between the functional blocks is realised through information object flows, which represent the flow of information and thereby the functional connections between the different actors.
\vspace*{-0.4cm}
\begin{figure}[H]
	\centering
	\includegraphics[width=\linewidth]{240823_medpower_kg.pdf}
	\caption{
		Simplified visualisation of the \ac{KG} used in this work, consisting of a \ac{HEMS} solution.
	}
	\label{fig:kg}
\end{figure}
These flows are directed, making the direction of the information flow clear.
As a result, control direction and monitoring must be represented in two separate information flows.
The transmission of measurement values has been omitted in this example, as it is not relevant for the intended impact analysis.
These elements are all part of the \textit{function layer}.
Each information object flow references the information objects that are being transmitted.
\textit{errol:ControlValue} is a subclass of \textit{errol:InformationObject}, specifically for control signals.
Information objects can stand alone or reference physical components, such as in this example, a heat pump whose power output can be controlled.
In this way, the information objects represent the relationship between the functional layer and the physical components.
Furthermore the hosts on which the functional actors are realised are added in the component layer, which allows for analysis of the dependency of applications on the physical \ac{ICT} infrastructure.
