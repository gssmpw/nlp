\section
{Conclusion and Future Work}

This work presents a comprehensive approach to modelling and analysing the impact of cyber-attacks on interconnected \ac{BTM} infrastructure within a cyber-physical energy system.
By utilising Semantic Web technologies, \ac{SHACL},  we have developed a graph model based on \ac{SGAM} that effectively represents the interdependencies between control infrastructure and the power system.
This model is used in our framework which automatically augments the control infrastructure to the electrical grid and allows the impact analysis of cyber-attacks.
Our case study demonstrates the feasibility of the developed model and framework, showing that the analysed benchmark case is not jeopardised by the examined attack vectors.
Future work will focus on increasing the model's complexity to identify less obvious risks in future energy systems and integrating scenarios that are closer to real-world grid conditions.
Additionally, the framework will be used to generate models of \ac{CPES} as input data for more sophisticated simulation environments, such as co-simulation platforms, providing a foundation for the development of resilience-enhancing measures.