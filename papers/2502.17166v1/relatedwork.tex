\section{Related Work}
In legal AI, much work has introduced legal theories to enhance reasoning and improve model accuracy and interpretability. For example, legal syllogism prompting (LoT)\cite{jiang2023legal} teaches LLMs for legal judgment prediction by instructing legal syllogism, Chain of Logic\cite{servantez2024chain} guides models in reasoning about compositional rules by decomposing logical statements based on the IRAC (Issue, Rule, Application, Conclusion) paradigm.  Among these, the Four-Elements Theory (FET) of Crime Constitution is a widely adopted framework\cite{yuan2024can,deng2023syllogistic}.

The Four-Element Theory is one of the most widely recognized criminal theories in Chinese judicial practice~\cite{liang2017vicissitudes}. It specifies four essential elements that must be satisfied to establish criminal liability: \textbf{Subject, Object, Subjective aspect, and Objective aspect}. For example, the four elements of the Crime of Affray can be briefly summarized as follows:

(1) Subject: Principal organizers and other active participants who have reached the age of criminal responsibility. (2) Object: Public order. (3) Objective Aspect: The act of assembling brawl, engaging in a brawl, resulting in the following consequences of serious injury. (4) Subjective Aspect: Direct intent, where the person knowingly and willfully engages in organizing or participating in the act of assembling brawl.

Before discussing the Four-Element Theory (FET), it is necessary to briefly compare it with another key theory in Chinese criminal law, the Hierarchical Theory of Crime Constitution\cite{zhou2017hierarchical, zhang2010justification}, and the main distinction between these theories lies in whether a hierarchical structure is considered, with ongoing debates in practice\cite{gao2009rationality, chen2010crime, chen2017comparative, zhou2017debate}. We chose FET as our foundational template for following reasons: 1) its dominance in Chinese judicial practice aligns with real-world criminal judgments; (2) its clear distinction between objective aspects and subjective intent offers direct reasoning checkpoints compared to the Three-Tier Theory; (3) its four-element annotation is flexible and can be adapted to the Three-Tier Theory by prioritizing objective analysis before subjective evaluation\cite{li2006no_reconstruction, zhang2017judicial}.

Recent approaches have leveraged the FET framework to model expert reasoning. For example, breaking down legal rules into FET-aligned components using automated planning techniques \cite{yuan2024can}. Employing model-generated four-element structures as minor premises in legal judgment analysis \cite{deng2023syllogistic}. While these methods have demonstrated improved performance on downstream tasks, they generally assume that the LLMs inherently understand the FET, without systematically validating this assumption. Notably, prior research on criminal charge prediction \cite{an2022charge} suggests that the models may misinterpret key legal concepts and may not be sensitive enough to the subtle differences in fact descriptions of confusing charges, highlighting the need to incorporate expert annotations to support LLM reasoning.