\documentclass{article}

% Recommended, but optional, packages for figures and better typesetting:
\usepackage{microtype}
\usepackage{graphicx}
\usepackage{subfigure}
\usepackage{booktabs} % for professional tables
\usepackage{amsmath}
\usepackage{amsfonts}
\usepackage{subcaption}
\usepackage{grffile}
\usepackage{booktabs}
\usepackage{tabularx}

% hyperref makes hyperlinks in the resulting PDF.
% If your build breaks (sometimes temporarily if a hyperlink spans a page)
% please comment out the following usepackage line and replace
% \usepackage{mlsys2024} with \usepackage[nohyperref]{mlsys2024} above.
\usepackage{hyperref}

% Attempt to make hyperref and algorithmic work together better:
\newcommand{\theHalgorithm}{\arabic{algorithm}}

% Use the following line for the initial blind version submitted for review:
\usepackage{format}
% If accepted, instead use the following line for the camera-ready submission:
% \usepackage[accepted]{mlsys2024}

% The \mlsystitle you define below is probably too long as a header.
% Therefore, a short form for the running title is supplied here:

\mlsystitlerunning{Style Extraction on Text Embeddings via VAE}


\begin{document}

\twocolumn[
\mlsystitle{Style Extraction on Text Embeddings\\ Using VAE and Parallel Dataset}

\vspace{-1.5em}
\begin{center}
    \textbf{September 2, 2024}
\end{center}

\begin{mlsysauthorlist}
\mlsysauthor{InJin Kong}{snu}
\mlsysauthor{Shinyee Kang}{snu}
\mlsysauthor{Yuna Park}{snu}
\mlsysauthor{Sooyong Kim}{snu}
\mlsysauthor{Sanghyun Park}{snu}
\end{mlsysauthorlist}

\mlsysaffiliation{snu}{Seoul National University}
% It is OKAY to include author information, even for blind
% submissions: the style file will automatically remove it for you
% unless you've provided the [accepted] option to the mlsys2024
% package.

% List of affiliations: The first argument should be a (short)
% identifier you will use later to specify author affiliations
% Academic affiliations should list Department, University, City, Region, Country
% Industry affiliations should list Company, City, Region, Country

% You can specify symbols, otherwise they are numbered in order.
% Ideally, you should not use this facility. Affiliations will be numbered
% in order of appearance and this is the preferred way.
\mlsyssetsymbol{equal}

\vskip 0.3in

\begin{abstract}
This study investigates the stylistic differences among various Bible translations using a Variational Autoencoder (VAE) model. By embedding textual data into high-dimensional vectors, the study aims to detect and analyze stylistic variations between translations, with a specific focus on distinguishing the American Standard Version (ASV) from other translations. The results demonstrate that each translation exhibits a unique stylistic distribution, which can be effectively identified using the VAE model. These findings suggest that the VAE model is proficient in capturing and differentiating textual styles, although it is primarily optimized for distinguishing a single style. The study highlights the model's potential for broader applications in AI-based text generation and stylistic analysis, while also acknowledging the need for further model refinement to address the complexity of multi-dimensional stylistic relationships. Future research could extend this methodology to other text domains, offering deeper insights into the stylistic features embedded within various types of textual data.
\end{abstract}]


\section{Introduction}

In recent years, with advancements in generative models and the expansion of training datasets, text-to-speech (TTS) models \cite{valle, voicebox, ns3} have made breakthrough progress in naturalness and quality, gradually approaching the level of real recordings. However, low-latency and efficient dual-stream TTS, which involves processing streaming text inputs while simultaneously generating speech in real time, remains a challenging problem \cite{livespeech2}. These models are ideal for integration with upstream tasks, such as large language models (LLMs) \cite{gpt4} and streaming translation models \cite{seamless}, which can generate text in a streaming manner. Addressing these challenges can improve live human-computer interaction, paving the way for various applications, such as speech-to-speech translation and personal voice assistants.

Recently, inspired by advances in image generation, denoising diffusion \cite{diffusion, score}, flow matching \cite{fm}, and masked generative models \cite{maskgit} have been introduced into non-autoregressive (NAR) TTS \cite{seedtts, F5tts, pflow, maskgct}, demonstrating impressive performance in offline inference.  During this process, these offline TTS models first add noise or apply masking guided by the predicted duration. Subsequently, context from the entire sentence is leveraged to perform temporally-unordered denoising or mask prediction for speech generation. However, this temporally-unordered process hinders their application to streaming speech generation\footnote{
Here, “temporally” refers to the physical time of audio samples, not the iteration step $t \in [0, 1]$ of the above NAR TTS models.}.


When it comes to streaming speech generation, autoregressive (AR) TTS models \cite{valle, ellav} hold a distinct advantage because of their ability to deliver outputs in a temporally-ordered manner. However, compared to recently proposed NAR TTS models,  AR TTS models have a distinct disadvantage in terms of generation efficiency \cite{MEDUSA}. Specifically, the autoregressive steps are tied to the frame rate of speech tokens, resulting in slower inference speeds.  
While advancements like VALL-E 2 \cite{valle2} have boosted generation efficiency through group code modeling, the challenge remains that the manually set group size is typically small, suggesting room for further improvements. In addition,  most current AR TTS models \cite{dualsteam1} cannot handle stream text input and they only begin streaming speech generation after receiving the complete text,  ignoring the latency caused by the streaming text input. The most closely related works to SyncSpeech are CosyVoice2 \cite{cosyvoice2.0} and IST-LM \cite{yang2024interleaved}, both of which employ interleaved speech-text modeling to accommodate dual-stream scenarios. However, their autoregressive process generates only one speech token per step, leading to low efficiency.



To seamlessly integrate with  upstream LLMs and facilitate dual-stream speech synthesis, this paper introduces \textbf{SyncSpeech}, designed to keep the generation of streaming speech in synchronization with the incoming streaming text. SyncSpeech has the following advantages: 1) \textbf{low latency}, which means it begins generating speech in a streaming manner as soon as the second text token is received,
and
2) \textbf{high efficiency}, 
which means for each arriving text token, only one decoding step is required to generate all the corresponding speech tokens.

SyncSpeech is based on the proposed \textbf{T}emporal \textbf{M}asked generative \textbf{T}ransformer (TMT).
During inference, SyncSpeech adopts the Byte Pair Encoding (BPE) token-level duration prediction, which can access the previously generated speech tokens and performs top-k sampling. 
Subsequently, mask padding and greedy sampling are carried out based on  the duration prediction from the previous step. 

Moreover, sequence input is meticulously constructed to incorporate duration prediction and mask prediction into a single decoding step.
During the training process, we adopt a two-stage training strategy to improve training efficiency and model performance. First, high-efficiency masked pretraining is employed to establish a rough alignment between text and speech tokens within the sequence, followed by fine-tuning the pre-trained model to align with the inference process.

Our experimental results demonstrate that, in terms of generation efficiency, SyncSpeech operates at 6.4 times the speed of the current dual-stream TTS model for English and at 8.5 times the speed for Mandarin. When integrated with LLMs, SyncSpeech achieves latency reductions of 3.2 and 3.8 times, respectively, compared to the current dual-stream TTS model for both languages.
Moreover, with the same scale of training data, SyncSpeech performs comparably to traditional AR models in terms of the quality of generated English speech. For Mandarin, SyncSpeech demonstrates superior quality and robustness compared to current dual-stream TTS models. This showcases the potential of  SyncSpeech as a foundational model to integrate with upstream LLMs.



\section{Related Work}


\minisection{Pivot-based approaches}
Pivot translation is an approach that decomposes the translation task into two sequential steps~\cite{wu-wang-2007-pivot, utiyama-isahara-2007-comparison}.
By transferring knowledge from high-resource pivot languages, pivoting is especially effective in translation between low-resource languages \cite{zoph-etal-2016-transfer, aji-etal-2020-neural, he-etal-2022-tencent}.
In this study, pivot translation enables us to obtain high-quality candidates for the ensemble.
\citet{kim-etal-2019-pivot} discusses a pivot-based transfer learning technique where source$\rightarrow$pivot and pivot$\rightarrow$target models are first trained separately, then use pre-trained models to initialize the source$\rightarrow$target model, allowing effective training of a single, direct NMT model.
\citet{zhang-etal-2022-triangular} further investigate the transfer learning approach by utilizing auxiliary monolingual data.


Pivot translation typically employs English as the bridge language.
Nonetheless, previous studies have explored the use of diverse pivot languages, taking into account factors such as data size and the relationships between languages~\cite{paul2009importance, dabre-etal-2015-leveraging}.
By leveraging the ability of pivot translation to produce diverse outputs, several studies have focused on generating paraphrases~\cite{mallinson-etal-2017-paraphrasing, guo2019zeroshot}.
More recently, \citet{mohammadshahi-etal-2024-investigating} uses pivot translation for ensemble, but it requires computing token-level probabilities and fails to improve translation.
Our work shares the motivation with these studies, generating translations depending on the pivot path to obtain a variety of candidates.


\minisection{Ensemble in NLG tasks}
Ensemble learning is a widely adopted strategy to obtain more accurate predictions by employing multiple systems~\cite{sagi2018ensemble}.
In NMT, the traditional approach involves averaging the probability distributions of the next target token, which is predicted at each decoding step by multiple models ~\cite{bojar-etal-2014-findings} or by different snapshots~\cite{huang2017snapshot}.
When multiple sources are available, an ensemble can be conducted with predictions obtained by different sources~\cite {firat-etal-2016-zero}.
Also, a token-level ensemble through vocabulary alignment across LLMs has also been proposed~\cite{eva}.
However, these methods are not applicable to recent black-box models as they cannot compute token-level probabilities at decoding time.


Selection-based ensemble has also been explored, which chooses the final output among the existing candidates.
This can be achieved through majority voting by selecting the most frequent one~\cite{wang2022rationaleaugmented} or selecting the best candidate with QE~\cite{fernandes-etal-2022-quality, howgood}.
Recently, MBR decoding~\cite{GOEL2000115, mbr}, which aims to find the hypothesis with the highest expected utility, has gained attention.
However, this approach limits the final output space to the existing candidate pool.


\begin{figure*}[t]
  \centering
  \includegraphics[width=0.95\textwidth]{Figures/overview.pdf} 
  \caption{Overview of \ours framework.}
  \label{fig:overall}
\end{figure*}


On the other hand, the generation-based ensemble method involves generating a new final prediction.
Fusion-in-Decoder~\cite{fid} proposes an architecture that aggregates additional information with a given input.
More recently, within the context of LLMs, \citet{llm-blender} and \citet{exchangeofthought} investigate a method of using LLMs to generate multiple outputs and aggregate them.
Generating new output through LLMs offers the benefit of explicitly harnessing their pre-trained knowledge within the ensemble process.

\section{Methodology}
\label{sec:method}

Solving the \gls{acr:mdp} as defined in Section \ref{sec:prob_form} presents several key challenges from a dynamic decision-making perspective. A primary challenge is the need for a model that describes the environment's transition dynamics, which is often complex and challenging to characterize. Another significant challenge is the intractable state space arising from the stochastic arrival of requests and gig workers, which renders exhaustive computation of optimal policies infeasible. Therefore, effective generalization across states is essential, as explicitly enumerating all possible states is unrealistic in many real-world scenarios.

Our proposed method offers several solutions to address these challenges. First, we utilize the \gls{acr:mnl} model to describe gig workers' decision-making process, capturing essential transition dynamics without requiring a fully known environment model. To manage the complexity of large state spaces, we leverage a post-decision state formulation, which eases computing optimal policies from a tractability perspective. This formulation also facilitates solving the inner optimization problem and deriving an exact form for optimal prices based on the post-decision state values. Our method is designed to ensure scalability, making it suitable for real-world applications where exhaustive enumeration of all possible states is infeasible. To this end, we incorporate value function approximators to generalize effectively across states, which enables us to estimate value functions without exhaustive state enumeration.  

This section details our approach. Section \ref{sec:bellman} introduces the Bellman equation for the \gls{acr:mdp} as defined in Section~\ref{sec:prob_form}. Section \ref{sec:mnl} discusses the \gls{acr:mnl} model for capturing essential transition dynamics, addresses the state space intractability by deriving the post-decision state formulation and presents an analytical solution to the optimization problem with optimal prices based on post-decision state values. Section \ref{sec:statistical_models} presents value function approximators and an algorithm for learning appropriate parameterizations. Section \ref{sec:training} details the statistical model used for gig worker utility estimation and presents the training procedure for the full algorithm. Finally, Section \ref{sec:meth_as} discusses methodological assumptions. 

\subsection{Bellman equation}
\label{sec:bellman}
In the following, we derive the Bellman equation for the \gls{acr:mdp} as defined in Section \ref{sec:prob_form}. To do so, we first elaborate on optimal value functions in general to elucidate the connection between pre- and post-decision states, which then allows us to derive the pre- and post-decision state value function accordingly. 

\noindent \textit{Optimal value functions:} The optimal value function of a pre-decision state $S\hspace{-0.1em}_t = (\mathcal{R}\hspace{0.05em}_{t},\mathcal{G}_t)$ or a post-decision state $\mathcal{R}\hspace{0.05em}_t^{p}$, denoted as $V_t(\mathcal{R}\hspace{0.05em}_{t},\mathcal{G}_t)$ and $V^{\mathrm{p}}_{t}(\mathcal{R}\hspace{0.05em}_{t}^{p})$, describes the maximum expected reward achievable from that state onward by following the optimal policy. In this context, the transition from a post-decision state to a successor pre-decision state exclusively involves the stochastic realization of the arrival processes of on-demand requests and gig workers. Therefore, a straightforward relationship between the two value functions is the following: 
\begin{align}
V^{\mathrm{p}}_{t}(\mathcal{R}\hspace{0.05em}_t^{p}) = \mathbb{E}_{\mathcal{R}^{\mathrm{new}},\mathcal{G}^{\mathrm{new}}}[V_{t+1}(\mathcal{R}\hspace{0.05em}_t^{p} \cup \mathcal{R}^{\mathrm{new}},\mathcal{G}^{\mathrm{new}})] \label{eq:value_function}
\end{align}
where the expectation over $\mathcal{R}^{\mathrm{new}},\mathcal{G}^{\mathrm{new}}$ indicates the dynamics of new request and new gig worker arrivals in the system correspondingly (see Figure~\ref{fig:state_trans}). Essentially, Equation \ref{eq:value_function} states that the value of a post-decision state $\mathcal{R}\hspace{0.05em}_{t}$ is equal to the expected value over all possible successor pre-decision states. 
%The expected immediate reward at time step $t$ when following the optimal policy. 

\noindent \textit{Bellman equation using the pre-decision state value function:}
The classic definition of the Bellman equation of the pre-decision state $S\hspace{-0.1em}_t = (\mathcal{R}\hspace{0.05em}_{t},\mathcal{G}_t)$ at time step $t = 0,1,...,T$ reads: 
\begin{align}
V_t(S\hspace{-0.1em}_t) & = \max_{\mathbf{c_t}} \{  R_t(\mathbf{c_t}) + \mathbb{E}_{S' \sim P(\cdot|S\hspace{-0.1em}_t,\mathbf{c_t})}[V_{t+1}(S')] \}
\end{align}
where $R_t(\mathbf{c_t})$ accounts for the expected immediate reward of being in state $S\hspace{-0.1em}_t = (\mathcal{R}\hspace{0.05em}_{t},\mathcal{G}_t)$ and following action $\mathbf{c_t}$, while the expectation $\mathbb{E}_{S' \sim P(\cdot|s,\mathbf{c_t})}[V_{t+1}(S')]$ accounts for the expected future reward that can be obtained from successor pre-decision states $S'$. In this standard definition, the value of each pre-decision state is decomposed using the value of the successor pre-decision state. Notably, the transitions from one pre-decision state to the next include two different sources of stochasticity. The first source of stochasticity results from the utility of the gig workers while the second source results from the stochastic nature of the request and gig worker arrivals. 
This formulation of the Bellman equation is highly intractable due to the complexities introduced by both sources of stochasticity. The primary challenge arises from the second source of stochasticity, as the arrival of new requests and gig workers leads to an explosion in the state transition space, making it impossible to model or track transitions to the next state accurately. In contrast, the first source of stochasticity, which depends solely on the gig workers' choice model, is less complex. To address this issue, we can alternatively formulate the Bellman equation using the post-decision state value function.

\noindent \textit{Bellman equation using the post-decision state value function:} Let us denote as $\mathcal{R}\hspace{0.05em}_t^{\mathrm{exp}}$ the set of requests in~$\mathcal{R}\hspace{0.05em}_{t}$ that expire, and let $\mathcal{R}\hspace{0.05em}_t^{'}$ correspond to the set of requests in $\mathcal{R}\hspace{0.05em}_t$ that do not expire after time step $t$, i.e., $\mathcal{R}\hspace{0.05em}_t^{'} = \mathcal{R}\hspace{0.05em}_{t} \setminus \mathcal{R}\hspace{0.05em}_t^{\mathrm{exp}}$. Furthermore, from here on, we assume for ease of notation that $\mathcal{R}\hspace{0.05em}_t^{'} \setminus \{\emptyset\}$ is equivalent to $\mathcal{R}\hspace{0.05em}_t^{'} $. Then, the Bellman equation of the pre-decision state $S\hspace{-0.1em}_t = (\mathcal{R}\hspace{0.05em}_{t},\mathcal{G}_t)$ at time step $t = 0,1,...,T$ is given by: 
\begin{align}
V_t(S\hspace{-0.1em}_t) & = \max_{\mathbf{c_t}} \big\{  R_t(\mathbf{c_t}) + \underbrace{(1 - \mathbbm{1}_{|\mathcal{G}_t|=1})  V^{\mathrm{p}}_{t}(\mathcal{R}\hspace{0.05em}_t^{'})\vphantom{\sum_{i \in \mathcal{R}\hspace{0.05em}_{t} \cup \{\emptyset\}}}}_{\text{(I)}}  + \underbrace{\mathbbm{1}_{|\mathcal{G}_t|=1} \! \sum_{i \in \mathcal{R}\hspace{0.05em}_{t} \cup \{\emptyset\}} \! P^i_{t}(\mathbf{c_t}) V^{\mathrm{p}}_{t}(\mathcal{R}\hspace{0.05em}_t^{'} \setminus \{i\})}_{\text{(II)}}  \big\}
\label{eq:bell_post}
\end{align}

\noindent where the term $R_t(\mathbf{c_t})$ accounts for the expected immediate reward of being in state $S\hspace{-0.1em}_t = (\mathcal{R}\hspace{0.05em}_{t},\mathcal{G}_t)$ and following action $\mathbf{c_t}$, while the remaining terms account for the expected future reward obtained from the successor post-decision state. Specifically, term (I) accounts for the expected future reward in the case where the pre-decision state does not contain any gig workers. Conversely, term (II) represents the expected future reward in the case where the pre-decision state contains a gig worker. In this case, the probabilities of request selection made by the gig worker determine the transition to post-decision states. Consequently, with a probability of \(P^i_{t}\), we transition to the post-decision state \(\mathcal{R}_{t}^{'} \setminus \{i\}\). 

Bellman Equation \ref{eq:bell_post} makes an explicit distinction between the two different sources of stochasticity in the \gls{acr:mdp}. The source of uncertainty which results from the gig workers' stochastic utility is reflected by the transition to all possible post-decision states, while the second source of uncertainty is embedded within the value function of the post-decision states. By separating the transition dynamics in this manner, the complexity of the Bellman equation is reduced, making the problem more tractable by directly depending only on the transition into the post-decision states.

\subsection{Representing gig worker's utility using the \gls{acr:mnl} model and deriving an analytical solution}
\label{sec:mnl}

To effectively utilize the tractability offered by the post-decision state formulation of the Bellman equation, it is essential to model the probability distribution for transitioning to each post-decision state. This necessitates a model that accurately describes gig worker behavior, and the \gls{acr:mnl} model is one of the most commonly used approaches in the literature on discrete choice behavior for both research and practical applications. Its popularity stems from the fact that it offers a closed-form expression for acceptance probabilities and its balance of simplicity and performance, which renders it both versatile and effective. Accordingly, we adopt the \gls{acr:mnl} model as the foundation for our analysis.

\noindent\textit{Multinomial Logit models:} In the \gls{acr:mnl} model, the random variables $\{e_{ij}\}_{i \in \mathcal{R}\hspace{0.05em}_{t}}$ of the utility function (see Equation \ref{eq:util}) are independent and identically distributed (i.i.d), following a Gumbel distribution. We utilize a special case of the MNL model which assumes a linear relationship between the offered compensation and Gumbel-distributed random variables with zero mean. Formally, this variation considers that the utility function of a gig worker $j$ for a request $i$ is:
\begin{equation} U_{ij} = u_{ij} + c_{i} + e_i \end{equation} where $u_{ij}$ is a value indicating the attractiveness of request $i$ to gig worker $j$, and whose value is determined by observable characteristics of the request, $c_i$ is the offered compensation to the gig worker for accepting the request, and $e_i$ are i.i.d. zero mean Gumbel variables with variance equal to $(\mu_j\pi)^2/6$ for some $\mu_j> 0$. Under this \gls{acr:mnl} model, when the request state is $\mathcal{R}\hspace{0.05em}_{t}$, the probability of a gig worker $j$ in time step $t$ accepting request $i$ for offered compensations $\mathbf{c_t} = (c^i_t)_{i \in \mathcal{R}\hspace{0.05em}_{t}}$ equals:
\begin{align} P^i_{t}(\mathbf{c_t}) = \frac{\exp((u_{ij} + c_t^i)/\mu_j)}{\sum_{l \in \mathcal{R}\hspace{0.05em}_{t}}(\exp((u_{lj} + c_t^j)/\mu_j) + \exp(u_0/\mu_j)},\label{eq:mnl1}\end{align} 
while the probability of gig worker $j$ not accepting any requests reads:
\begin{align} P^\emptyset_{t}(\mathbf{c_t}) = \frac{ \exp(u_0/\mu_j)}{\sum_{l \in \mathcal{R}\hspace{0.05em}_{t}}(\exp((u_{lj} + c_t^j)/\mu_j) + \exp(u_0/\mu_j)},\label{eq:mnl2}\end{align} 
where $u_0 \geq 0$ is a constant indicating the attractiveness of the all-reject alternative. 

In the simplest case, all gig workers share the same utility function. Formally, $u_{ij} = u_i$ for all requests $i \in \mathcal{R}\hspace{0.05em}_{t}$ and for all gig workers $j \in \mathcal{G}_t$, and $\mu_j = \mu$ for all gig workers $j \in \mathcal{G}_t$. However, to introduce variability in preferences among the population while maintaining model simplicity, we can assume that the population is divided into $D$ gig worker groups. This formulation corresponds to a Mixed Multinomial Logit (Mixed MNL) model, where within each group, individuals exhibit similar preferences. Formally, for each group $d \in [1,\dots,D]$, the utility function is $u_{ij} = u_i^d$  for all requests $i \in \mathcal{R}\hspace{0.05em}_{t}$ and all gig workers $j \in \mathcal{G}^d_t$, and $\mu_j = \mu_d$ for all gig workers $j \in \mathcal{G}^d_t$.

To solve Bellman equation \ref{eq:bell_post}, we adapt and extend the theory from \cite{dong2009dynamic}, which studies dynamic pricing in the context of inventory control of substitute products, to our problem setting. Specifically, we demonstrate that there exists an alternative formulation of the optimization problem defined by the post-decision state Bellman equation. This alternative formulation uses acceptance probabilities as decision variables and is concave. We then derive the optimal solution by solving the first-order condition.

To this end, we first reformulate the value function by applying Equation \eqref{eq:imm_rew} and representing $P^\emptyset_{t}(c_t^i)$ as $ 1 - \sum_{i \in \mathcal{R}\hspace{0.05em}_{t}}P^i_{t}(\mathbf{c_t})$ and derive the following lemma.
\begin{lemma}\label{lem:dynamic_programming}
The value of the pre-decision state $S\hspace{-0.1em}_t = (\mathcal{R}\hspace{0.05em}_{t},\mathcal{G}_t)$ is equal to:
$$ V_t(\mathcal{R}\hspace{0.05em}_{t},\mathcal{G}_t) = \max_{c_t} \{ \phi_t(\mathcal{R}\hspace{0.05em}_{t},\mathcal{G}_t,\mathbf{c_t}) \} + V^{\mathrm{p}}_{t}(\mathcal{R}\hspace{0.05em}_t^{'})  + \sum_{i \in \mathcal{R}\hspace{0.05em}_t^{\mathrm{exp}}}\! \beta_i $$
where 
$\phi_t(\mathcal{R}\hspace{0.05em}_{t},\mathcal{G}_t,\mathbf{c_t}) = \mathbbm{1}_{|\mathcal{G}_t|=1} \mathbb{E}_{i \sim P_t(\mathbf{c_t})} [r_i - c_t^i - \Delta^i_{V_{t}}(\mathcal{R}\hspace{0.05em}_t^{'}) -\beta_i\mathbbm{1}_{i \in \mathcal{R}\hspace{0.05em}_t^{\mathrm{exp}}}]$ and
$ \Delta^i_{V_{t}}(\mathcal{R}\hspace{0.05em}_t^{'}) = V^{\mathrm{p}}_{t}(\mathcal{R}\hspace{0.05em}_t^{'}) - V^{\mathrm{p}}_{t}(\mathcal{R}\hspace{0.05em}_t^{'} \setminus \{i\}).$

\end{lemma}
The proof of Lemma 1 can be found in Appendix \ref{sec:p_1}. 

As a result of this reformulation, we observe that finding the optimal prices $\mathbf{c_t}$ reduces to optimizing $\phi_t(\mathcal{R}\hspace{0.05em}_{t},\mathcal{G}_t,\mathbf{c_t})$. The function $\phi_t(\mathcal{R}\hspace{0.05em}_{t},\mathcal{G}_t,\mathbf{c_t})$ concerns only scenarios where a gig worker is available, as no decision is required otherwise. It accounts for the probability of each request $i \in \mathcal{R}\hspace{0.05em}_{t}$ being accepted given a compensation $c_t^i$, the corresponding reward and penalty, and an additional term $\Delta^i_{V_{t}}(\mathcal{R}\hspace{0.05em}_t^{'})$. The term $\Delta^i_{V_{t}}(\mathcal{R}\hspace{0.05em}_t^{'})$ reflects the difference in expected reward when request $i \in \mathcal{R}\hspace{0.05em}_{t}$ is accepted immediately versus when it remains unfulfilled, commonly referred to as the opportunity cost of request $i \in \mathcal{R}\hspace{0.05em}_{t}$. As demonstrated by \cite{hanson1996optimizing}, $\phi_t(\mathcal{R}\hspace{0.05em}_{t},\mathcal{G}_t,\mathbf{c_t})$ is not guaranteed to be concave in $\mathbf{c_t}$. 

We can address such optimization problems by expressing the compensation decision $c^t_i$ in terms of the acceptance probabilities. Using the \gls{acr:mnl} model to describe gig worker behavior, we combine Equations \eqref{eq:mnl1} and \eqref{eq:mnl2} to establish a bijection between the compensation \(c^t_i\) and the acceptance probabilities \(P_t^i\). Specifically, the compensation for each request can be expressed as a function of acceptance probabilities as follows:
\begin{align}
\frac{P_t^i(\mathbf{c_t})}{P_t^{\emptyset}(\mathbf{c_t})} = \exp((u_{ij} + c_t^i - u_0)/\mu_j) \Leftrightarrow c_t^i = -u_{ij} + u_0 + \mu_j \ln{P_t^i} - \mu_j \ln{P_t^{\emptyset}}.
\label{eq:mnl3}
\end{align}
Consequently, using the above expression we reformulate $\phi_t$ as a function of $P_t$ and establish the following result.
\begin{lemma}\label{lem:concavity}
$\phi_t(\mathcal{R}\hspace{0.05em}_{t},\mathcal{G}_t,P_t) = \mathbbm{1}_{|\mathcal{G}_t|=1}\mathbb{E}_{i \sim P_t}[r_i + u_{ij} - u_0 - \mu_j \ln{P_t^i} + \mu_j \ln{P_t^{\emptyset}}  - \Delta^i_{V_{t}}(\mathcal{R}\hspace{0.05em}_t^{'}) -\beta_i\mathbbm{1}_{i \in \mathcal{R}\hspace{0.05em}_t^{\mathrm{exp}}}]$ is concave in~$P_t$.
\end{lemma}
The proof of Lemma 2 can be found in Appendix \ref{sec:p_2}. 

By proving the concavity of $\phi_t$ as a function of $P_t$ we can solve the first order condition of $\phi_t$ and derive an optimal solution as a function of the acceptance probabilities $P_t$.
\begin{lemma}\label{lem:exact}
The optimal price $c^{i*}_t$ is given by:
$c^{i*}_t = r_i - \beta_i \mathbbm{1}_{i \in \mathcal{R}\hspace{0.05em}_t^{\mathrm{exp}}} 
 - \Delta^i_{V_{t}}(\mathcal{R}\hspace{0.05em}_t^{'}) - m_t$ where $m_t$ results from solving: 
 $(\frac{m_t}{\mu_j} - 1) \exp\{ \frac{m_t}{\mu_j} - 1\} = \sum_{i \in \mathcal{R}\hspace{0.05em}_{t}} \exp\{ \frac{1}{\mu_j} (r_i + u_{ij} - u_0 - \beta_i \mathbbm{1}_{i \in \mathcal{R}\hspace{0.05em}_t^{\mathrm{exp}}} - \Delta^i_{V_{t}}(\mathcal{R}\hspace{0.05em}_t^{'}) - \mu_j)\}$
\end{lemma}
The proof of Lemma \ref{lem:exact} can be found in Appendix \ref{sec:p_3}. 

Lemma \ref{lem:exact} states that the optimal price for a request~$i$ is equal to the reward of that request, reduced by the following terms: its penalty in the case that it expires within the current time step, the expected value discrepancy $\Delta^i_{V_{t}}(\mathcal{R}\hspace{0.05em}_t^{'})$ and the term $m_t$, which is a state-specific factor that adjusts pricing based on the overall system state rather than on individual requests. In essence, $m_t$ captures system-wide influences on the optimal pricing decision at any given time.

While we define feasible compensations as non-negative in our problem setting, the optimal compensation described in Lemma \ref{lem:exact} may result in negative values under certain conditions, e.g., due to specific worker-request dynamics or model assumptions. In practice, when implementing the optimal compensations, we set any negative values of $\mathbf{c_t} = (c^i_t)_{i \in \mathcal{R}\hspace{0.05em}_{t}}$ to zero to ensure all compensations remain feasible. Since gig workers would not accept requests with either negative or zero compensation, this adjustment aligns with realistic worker behavior and does not compromise the model's practical effectiveness. This adjustment allows for practical feasibility and ensures the model’s applicability in real-world scenarios, where negative compensations would be infeasible, while retaining most of the theoretical solution’s structure and insights.

\subsection{Value function approximation}
\label{sec:statistical_models}
Although our optimization problem has an analytical solution, derived in Section \ref{sec:mnl}, its dependence on the post-decision value function makes precise computation infeasible due to the vast and intractable state space. To overcome this, we employ statistical models to approximate the post-decision value function.

\noindent\underline{Statistical models for the post-decision value function:} A statistical model for the post-decision value function is defined as a function $\hat{V}$ parameterized by $\theta$ which receives as input a post-decision state $\mathcal{R}^{post}\hspace{0.05em}$ and predicts a value $\hat{v}$ for that state: $\hat{V}^\theta: \mathcal{R}^{post}\hspace{0.05em} \mapsto \hat{v} \in \mathbbm{R}$. The primary challenge in designing an effective statistical model to approximate the post-decision state value lies in the fact that the request state is represented as a set, containing a variable number of requests. Consequently, an effective statistical model for the post-decision value function approximation must be able to handle this dynamic structure. To account for such a variability, we utilize a neural network architecture that incorporates an attention mechanism.

\begin{figure}[b]
\centering
\resizebox{\textwidth}{!}{ % Use full width of the page
\begin{tikzpicture}[node distance=2cm]

    % Input node on the left, outside the enclosing box
    \node [block] (input) {$i \in \mathcal{R}^{post}\hspace{0.05em}$};

    % Horizontal structure of nodes inside the enclosing box
    \node [block, right=of input] (mlp_emb) {MLP$^{emb}$};
    \node [block, right=of mlp_emb] (mlp_w) {MLP$^W$};
    \node [block, right=of mlp_w] (cell) {Context \\[-0.3cm] $C = \sum_{i \in \mathcal{R}^{post}\hspace{0.05em}} \beta_{i} \cdot e_{i}$};

    % Enlarged enclosing box around MLP^emb, MLP^W, and Context nodes, including the extra arrow
    \node[fit=(mlp_emb) (mlp_w) (cell), draw, inner sep=10pt, rounded corners,  yshift=-0.18cm] (enclosing_box) {};

    % Arrows
    \draw [arrow] (input) -- (mlp_emb);
    \draw [arrow] (mlp_emb) -- node[above] {$e_i$} (mlp_w);
    \draw [arrow] (mlp_w) -- node[above] {$\beta_i$} (cell);
    
    % Direct arrow from MLP^emb to Context with centered label
    \draw [arrow] (mlp_emb) -- ++(0,-0.8cm) -| node[pos=0.12, above] {$e_i$} (cell);
\end{tikzpicture}
}
\caption{\textnormal{Attention mechanism}}
\label{fig:tikz_diagram}
\end{figure}

\noindent \textit{Attention mechanism:} The attention mechanism (see Figure \ref{fig:tikz_diagram}) calculates an embedding vector $e_i$ for each request $i \in \mathcal{R}^{post}\hspace{0.05em}$ using a multi-layer perceptron (MLP). Subsequently, $e_i$ is processed by a second MLP which performs the following calculation: $\beta_{i} = \sigma(w \cdot \tanh(W \cdot e_{i}))$, where $\sigma$ is the sigmoid function, $w$ is a trainable vector of weights, and $W$ is a trainable matrix of weights. Lastly, it calculates the context vector $C$ as $C = \sum_{i \in \mathcal{R}^{post}\hspace{0.05em}} \beta_{i} \cdot e_{i}$. The context vector is then given as an input to the rest of the neural network architecture. 

\noindent \textit{Neural network architecture:} The neural network for the post-decision value function approximation consists of an attention mechanism that comprises a feedforward layer with 32 units and a Swish activation \citep{ramachandran2017searching}, which provides the embedding vector $e_i$ for each request $i \in \mathcal{R}^{post}\hspace{0.05em}$. It then calculates the context vector using weights $w \in \mathbbm{R}^{64}$ and $W \in \mathbbm{R}^{64 \times 32}$. This context vector, along with other relevant state features (e.g., the number of requests, the most urgent deadline), constitutes the input to the subsequent layers of the network. These layers include two feedforward layers, each with 16 units and a Swish activation. Additional information about neural network training and hyperparameters can be found in Appendix \ref{sec:nn_app}. 

\noindent \underline{Algorithm for training the value function approximator:} To learn the parameters of the statistical model, we follow an approximate value iteration for the post-decision state value function, adapted from \cite{powell2021reinforcement}, to accommodate the continuous action space of our problem setting and the probabilistic transitions to post-decision states. Algorithm~1 shows a pseudocode of our learning procedure: initially, the algorithm receives estimates of the \gls{acr:mnl} parameters $\hat{\mathit{u}}$ and $\hat{\mu}$ (L.1). Then we randomly initialize a parameterization $\theta$ for the statistical model of the post-decision value function (L.2), and repeat the following procedure: first, we gather experiences by interacting with a simulated or real-world environment (L.3). During this time, we select the compensation by considering the estimated optimal price $\hat{c}_{t}$ using the latest estimate of the value function approximation, and we perturb the estimated optimal value using a Gaussian perturbation (L.4) to ensure exploration. After a sufficient amount of experiences is gathered, we update the parameterization $\theta$ (L.5-L.8). For updating the post-decision value function approximation around the observed post-decision state $\mathcal{R}\hspace{0.05em}_{t-1}^{post}$ of time step $t-1$, we use the (estimated) optimal value $v_t^*$ of the observed pre-decision state $S\hspace{-0.1em}_t$ of the following time step $t$.

\begin{algorithm}[b!]
\caption{Approximate value iteration using the post-decision value function}
\begin{algorithmic}[1]  % The [1] here enables line numbering
\REQUIRE MNL model utilities $\hat{\mathit{u}}$, and Gumbel parameter $\hat{\mu}$
\STATE Initialize a value function parameterization $\theta$ for $\hat{V}^\theta$ 
\FOR{episode = 1,\ldots,M}
    \STATE Step 1. Gather experiences using the e-greedy policy:
    \STATE \hspace{\algorithmicindent} $\begin{aligned}[t]
            \mathbf{\hat{c}}_{t} = (\max(0,r_i - \beta_i \mathbbm{1}_{i \in \mathcal{R}\hspace{0.05em}_t^{\mathrm{exp}}} - \Delta_{\hat{V}}^i(\mathcal{R}\hspace{0.05em}_t^{'}) - m_t + \varepsilon_{i}))_{i \in \mathcal{R}\hspace{0.05em}_{t}} 
            \end{aligned}$
            where $\varepsilon_{i} \sim \mathcal{N}(0,\delta)$  $\forall i \in \mathcal{R}\hspace{0.05em}_{t}$
    \STATE Step 2. Update $\theta$ using the target:
    \STATE \hspace{\algorithmicindent} $\begin{aligned}[t]
            \hat{V}^{\theta}(\mathcal{R}\hspace{0.05em}_{t-1}^{post}) \leftarrow v_t^*
            \end{aligned}$
    \STATE where $\mathcal{R}\hspace{0.05em}_{t-1}^{post}$ is the \underline{observed} post-decision state at time step $t-1$ and $v_t^*$ is the estimated optimal value of the successor pre-decision state $S\hspace{-0.1em}_t$:     
    \STATE \hspace{\algorithmicindent} $\begin{aligned}
            v_t^* = \max_{\mathbf{c_t}} \{ R_t(\mathbf{c_t}) +  \gamma \mathbbm{1}_{|\mathcal{G}_t|=1} \sum_{i \in \mathcal{R}\hspace{0.05em}_t\cup\{\emptyset\}}  P_t^i(\mathbf{c_t}) \cdot \hat{V}^{\theta}_t(\mathcal{R}\hspace{0.05em}_t^{'}\backslash\{i\}) \} + (1 - \mathbbm{1}_{|\mathcal{G}_t|=1}) \cdot \hat{V}^{\theta}_t(\mathcal{R}\hspace{0.05em}_t^{'}).
            \end{aligned}$
    \ENDFOR
\end{algorithmic}
\end{algorithm}

\subsection{Training procedure}
\label{sec:training}
\noindent \underline{Statistical model of the gig worker's utility:} In practice, the true \gls{acr:mnl} model parameters for the utilities $u_{ij}$ and the Gumbel parameter $\mu$ are not known. To estimate the parameters of the \gls{acr:mnl} model for each gig worker group, we use observed accept/reject gig worker decisions based on the characteristics of the on-demand request $\mathbf{x}_i$ and offered compensation $c_i$. This data comes either from interactions with the environment under any reasonable policy or from pre-existing historical data. For each gig worker group $d~\in~[1,\dots,D]$ we train a logistic regression model using the log-odds function: $(\mathbf{\hat{w}}_d^T\mathbf{x}_i + c_i)/\beta_d$ where $\mathbf{\hat{w}}_d$ and $\beta_d$ are trainable parameters. Finally, we estimate the utility $u_{ij}$ of a gig worker $j$ for request $i$ using the mapping $\hat{\mathit{u}}_j : \mathbf{x}_i \mapsto \mathbf{\hat{w}}_d^T \mathbf{x}_i$, so that $\hat{u}_{ij} = \hat{\mathit{u}}_j(\mathbf{x}_i)$, and $\beta_d$ by $\hat{\mu}_d = \beta_d$.

\noindent \underline{Training pipeline:} Our training procedure for each scenario involves the following steps: Initially, we gather experiences by interacting with training scenarios. The compensation policy employed sets the compensation for each request $i$  by randomly selecting a value between 40\%-85\% of the request's reward. From these experiences, we train the \gls{acr:mnl} estimator as previously defined, using stochastic gradient descent for optimization. Since the gig worker always chooses the offer that maximizes their utility, the data skews toward higher-compensation offers, especially when the sampling policy is non-optimal and tends to over-offer. Therefore, we use \(L2\) regularization on the weight $\beta_d$ to prevent the weight $\beta_d$ from increasing excessively. We then proceed to train the post-decision value function approximation as outlined in Algorithm 1. In order to mitigate the effect of the network weight initialization, we repeat this process with 5 different random seeds, resulting in 5 distinct models. We select the final model based on the one that demonstrates the best performance on the validation scenarios.

\subsection{Discussion}
\label{sec:meth_as}
Our proposed algorithmic paradigm adopts a model-free approach in many aspects, while it relies on the \gls{acr:mnl} model to capture gig worker behavior. As a result, it requires some knowledge of gig worker groups or at least observable characteristics of gig workers. Below, we discuss key considerations and limitations of our algorithm.

\noindent \textit{Modeling gig worker utility:} Modeling all stochastic aspects of the environment is impractical due to the complexity of real-world scenarios. Instead, selectively modeling key elements provides a practical balance between model-free and model-based approaches. For example, while the variability in gig worker arrivals and requests (e.g., weekday vs. weekend patterns) is too complex to be modeled precisely, focusing on gig worker decision-making is reasonable. In our approach, we use the \gls{acr:mnl} model to represent gig worker decisions. This model is widely used in research and practice, offering flexibility to capture diverse preferences across gig worker groups. However, it assumes a specific mathematical structure (e.g., independence of irrelevant alternatives), which may not fully reflect real-world decision-making. Despite this, its practicality and generalizability justify its use in our algorithm.

\noindent \textit{Knowledge of gig worker groups:} When the gig worker population displays heterogeneous preferences, i.e., when more that one gig worker group exists, the performance of our algorithm relies on having knowledge of the different groups. However, the availability of such information varies across platforms, influencing their ability to identify groups with similar preferences. Platforms seeking to identify potential groups can refer to existing studies, such as \cite{marcucci2017connected}, \cite{bathke2023occasional}, and \cite{miller2017crowdsourced}, which offer valuable insights into gig worker decision-making and subgroup behavior.


\section{Results}\label{sec:results}

\subsection{Training Convergence and Loss Analysis}

For all 36 hyperparameter combinations, both the training loss and test loss decreased and eventually converged, indicating that the models successfully learned from the data and reached a stable state in terms of reconstruction error. Detailed loss curves and analysis are provided in Figure \ref{fig:appendix_loss}.

\begin{figure}[htbp]
    \centering
    \includegraphics[width=0.5\textwidth]{test_set_loss.png}
    \caption{Test set loss during training. The x-axis represents the number of epochs, and the y-axis represents the mean error. The hyperparameters of each model are as follows: starting from left the 1st, 2nd, and 3rd columns represent feature dimensions of 8, 64, and 256, respectively, and the starting from top 1st, 2nd, and 3rd rows represent 1, 3, and 6 hidden layers, respectively.}
    \label{fig:appendix_loss}
\end{figure}

\subsection{L2 Error Distribution and FLD Analysis}

The L2 error distribution for each model is presented in Figure \ref{fig:appendix_l2_error}. The minimum Fisher’s Linear Discriminant (FLD) between the L2 norm distributions of the reconstructed sentence vectors from the trained dataset (ASV) and the anomaly datasets (NET, ASVS, Coverdale, Geneva, KJV Strongs) across the 36 models is shown in Figure \ref{fig:appendix_fld}.

\begin{figure}[htbp]
    \centering
    \includegraphics[width=0.5\textwidth]{distribution.jpg}
    \caption{L2 error distribution on ASV, NET, ASVS, Coverdale, Geneva, and KJV Strongs. The x-axis represents the L2 error between the original and reconstructed sentence vector, and the y-axis represents the distribution density. The hyperparameters of each model are as follows: starting from left the 1st, 2nd, and 3rd columns represent feature dimensions of 8, 64, and 256, respectively, and starting from top the 1st, 2nd, and 3rd rows represent 1, 3, and 6 hidden layers, respectively.}
    \label{fig:appendix_l2_error}
\end{figure}

\begin{figure}[htbp]
    \centering
    \includegraphics[width=0.5\textwidth]{FLD.jpg}
    \caption{(Left) Minimum and (Right) Maximum of FLD between ASV and other 5 anomaly datasets (NET, ASVS, Coverdale, Geneva, and KJV Strongs). A higher minimum FLD indicates better differentiation between ASV and anomaly L2 error distributions.}
    \label{fig:appendix_fld}
\end{figure}

The minimum FLD is more important than the maximum FLD for determining the separation between normal and anomaly data. A high minimum FLD represents the model that has the most differentiation between the ASV original and the anomaly reconstructions, indicating the best-performing model in terms of distinguishing between the original and anomalous styles based on the L2 norm distribution. Figure \ref{fig:appendix_fld} shows that the minimum FLD is maximized in models with 3 hidden layers and a feature dimension size between 32 and 128. Models with too small or too large hidden layers and feature dimensions tend to perform poorly in anomaly differentiation.

Across the 36 models, the anomaly dataset that produced the minimum FLD most frequently was Geneva, appearing 31 times, followed by KJV Strongs, which appeared 5 times. This suggests that the L2 error distribution of the Geneva dataset was generally the closest to that of ASV, making it the hardest to distinguish from ASV. Conversely, the anomaly dataset that consistently produced the maximum FLD in all 36 models was Coverdale, indicating that it was the easiest to distinguish from ASV based on the L2 error distribution. This result highlights the distinctiveness of Coverdale's style compared to ASV, while Geneva's style appears more similar.

\subsection{Impact of Context Subtraction on VAE Performance}

\begin{figure}[htbp]
    \centering
    \includegraphics[width=0.5\textwidth]{no_sub_distribution.jpg}
    \caption{L2 error distribution on ASV, NET, ASVS, Coverdale, Geneva, and KJV Strongs, without parallel sentence (KJV) subtraction. The x-axis represents the L2 error between the original and reconstructed sentence vector, and the y-axis represents the distribution density. The hyperparameters of each model are as follows: starting from left the 1st, 2nd, and 3rd columns represent feature dimensions of 8, 64, and 256, respectively, and the starting from top 1st, 2nd, and 3rd rows represent 1, 3, and 6 hidden layers, respectively.}
    \label{fig:appendix_no_sub_l2_error}
\end{figure}

\begin{figure}[htbp]
    \centering
    \includegraphics[width=0.5\textwidth]{no_sub_FLD.jpg}
    \caption{(Left) Minimum and (Right) Maximum of FLD between ASV and other 5 anomaly datasets (NET, ASVS, Coverdale, Geneva, and KJV Strongs), without parallel sentence (KJV) subtraction. A higher minimum FLD indicates better differentiation between ASV and anomaly L2 error distributions.}
    \label{fig:appendix_no_sub_fld}
\end{figure}

Training the VAE without subtracting context parallel sentence (KJV) vectors demonstrated that both the training loss and test loss decreased and converged, indicating successful learning. However, as shown in Figure \ref{fig:appendix_no_sub_l2_error} in Appendix F, the mean L2 error across all distributions was higher compared to the models trained with parallel sentence subtraction.

When comparing Figures \ref{fig:appendix_no_sub_fld} and \ref{fig:appendix_fld}, the Fisher’s Linear Discriminant (FLD) for the no-subtraction case (from context parallel sentence vectors) is significantly lower than for the subtracted case. Specifically, the mean of the minimum FLD across the 36 models in the subtracted case is 1.111, while the mean for the no-subtraction case is 0.116, making the FLD approximately 9.6 times lower without subtraction.

Furthermore, the highest maximum FLD in the no-subtraction case (1.000) is nearly the same as the lowest minimum FLD in the subtracted case (0.983). This stark difference in FLD highlights that when trained without subtracting the context parallel sentence vectors, the VAE's ability to distinguish anomalies from normal (trained domain) data is significantly diminished. This result reinforces the idea that the subtraction of context helps the VAE better capture stylistic differences, leading to clearer separation between ASV and other translations.

\section{Discussion}\label{sec:discussion}

This study extracted the styles of various Bible translations and utilized a Variational Autoencoder (VAE) model to analyze how these styles differ, particularly in comparison to the American Standard Version (ASV). The results revealed that the styles of each Bible translation followed a normal distribution, and these distributions could be clearly distinguished from that of the ASV. This indicates that there are stylistic differences between the ASV and other translations, and that these differences can be effectively detected using the VAE model.

After optimizing the VAE model’s hyperparameters, the process of distinguishing between the ASV and other translation styles resulted in a Type 1 error of 8.7\% and a Type 2 error of 6.7\%, with a total error rate of 15.3\%. Conversely, the model achieved an accuracy of 84.7\%, demonstrating its ability to effectively differentiate styles. This level of accuracy suggests that the model can clearly recognize the distribution of a specific style and use it as a basis to distinguish between the styles of different translations.

However, the VAE model was optimized for distinguishing a single style. While it was useful for detecting differences between a specific translation style and the ASV, it had limitations when it came to distinguishing multiple styles simultaneously or understanding the relationships between complex, multi-dimensional styles. These limitations stem from the structural characteristics of the VAE, which compresses the data’s features during learning, making it inherently challenging to fully capture the complex characteristics of the data. Therefore, to distinguish multiple styles simultaneously, it may be necessary to use other models or train the VAE model in a more sophisticated manner.

The ability to extract a specific style suggests that the style’s characteristics can be quantified and represented as a probability distribution. This means that AI can utilize this quantified style representation to generate text that adheres to a specific style. For example, in text generation tasks where a particular writing style or tone is required, a 'style metric' could be used as a numerical and comparable indicator to assess and ensure that the generated text conforms to the desired style.

The approach taken in this study opens up the possibility of expanding the research to other parallel text datasets. By applying this methodology to other text domains, researchers can study the stylistic differences and their implications within each domain. For example, the approach could be extended to analyze the styles of different translations of literary works, legal document translations, or works by various authors.

We have demonstrated that the VAE model can distinguish between the original and anomaly data using the reconstruction L2 error. To measure the overall accuracy, False Positive Rate (FPR), and False Negative Rate (FNR) of the model, we created an Accuracy Test Dataset using data not included in the training set. This dataset consisted of 1,000 samples, with 50\% of the samples being from ASV and the remaining 50\% from five anomaly datasets (NET, ASVS, Coverdale, Geneva, and KJV Strongs).

\begin{table}[t]
\caption{Accuracy \& Error Rates of Models 1, 2, and 3 on Anomaly Detection}
\label{tab:accuracy}
\vskip 0.15in
\begin{center}
\begin{small}
\begin{sc}
\setlength{\tabcolsep}{4pt} % Default value: 6pt. Reduce this to make the table narrower.
\begin{tabular}{|c|c|c|c|}
\hline
\textbf{Model} & \textbf{Accuracy} & \textbf{Type I Error} & \textbf{Type II Error} \\
\hline
Model 1 & 83.5\% & 9.8\% & 6.7\% \\
Model 2 & 82.9\% & 10.1\% & 7.0\% \\
Model 3 & 83.4\% & 9.8\% & 6.8\% \\
\hline
\textbf{Average} & \textbf{83.3\%} & \textbf{9.9\%} & \textbf{6.8\%} \\
\hline
\end{tabular}
\end{sc}
\end{small}
\end{center}
\vskip -0.1in
\end{table}


The binary classification results showed that the lowest overall error rate was achieved when the threshold was set at mean + 0.8 std. The average overall error rate across the three models was 16.8\%. The relatively high FNR, particularly with Geneva, suggests that modern English Bible translations inherently do not exhibit distinct stylistic differences.

The results of the anomaly detection using the VAE in this study also show trends similar to what would be expected when humans classify ASV and other Bible versions. In this study, the L2 error distributions of ASV and Geneva had a significant overlap, making it difficult to classify them with a low error rate using a specific threshold. In contrast, the L2 error distribution of Coverdale barely overlapped with ASV, and the FLD was the highest across all models. More typically, 67.8\% (ASV 34.6\%, KJV Strongs 33.2\%) of type 2 error in anomaly detection is from Geneva and KJV Strongs.

\begin{table}[t]
\caption{Original Sentences of 3 Different Versions: ASV, Geneva, Coverdale}
\label{tab:verses}
\vskip 0.15in
\begin{center}
\begin{small}
\begin{sc}
\begin{tabularx}{0.45\textwidth}{lX}
\toprule
\textbf{Verse} & \textbf{Translation} \\
\midrule
\textbf{Gen 1:1} & \textbf{ASV:} In the beginning God created the heavens and the earth. \newline
\textbf{Geneva:} In the beginning God created the heauen and the earth. \newline
\textbf{Coverdale:} In ye begynnynge God created heauen \& earth: \\
\midrule
\textbf{Mat 1:1} & \textbf{ASV:} The book of the generation of Jesus Christ, the son of David, the son of Abraham. \newline
\textbf{Geneva:} The book of the generation of Jesus Christ the son of David, the son of Abraham. \newline
\textbf{Coverdale:} This is the boke of the generacion of Iesus Christ ye sonne of Dauid, the sonne of Abraham. \\
\midrule
\textbf{Joh 3:16} & \textbf{ASV:} For God so loved the world, that he gave his only begotten Son, that whosoever believeth on him should not perish, but have eternal life. \newline
\textbf{Geneva:} For God so loveth the world, that he hath given his only begotten Son, that whosoever believeth in him, should not perish, but have everlasting life. \newline
\textbf{Coverdale:} For God so loued the worlde, that he gaue his onely sonne, that who so euer beleueth in hi, shulde not perishe, but haue euerlastinge life. \\
\bottomrule
\end{tabularx}
\end{sc}
\end{small}
\end{center}
\vskip -0.1in
\end{table}

Table \ref{tab:verses} illustrates the textual differences between three versions of the Bible (ASV, Geneva, Coverdale), which could influence the VAE's ability to distinguish anomalies. The relatively low accuracy of anomaly detection using the VAE in this study may be attributed to the subtle stylistic differences between the texts. This implies that using sentences with clearer stylistic differences and more varied contexts in future experiments could result in better accuracy.

\section{Conclusion}


In this work, we introduced \ours, a pivot-based single model ensemble framework, to enhance translation in scenarios where parallel data are scarce.
By transferring knowledge from diverse pivot languages, we were able to obtain not only diverse but also high-quality candidates.
And the optimal path to generating the best candidate varies per sentence, our study underscores the significance of exploiting a spectrum of pivot languages.
Moreover, the single model generation process offers cost savings compared to multi-model ensemble approaches. 
Empirical results and qualitative analyses show that the proposed method can yield contextually suitable translations for the given source sentences by leveraging pivoted candidates.

\bibliography{citation}
\bibliographystyle{format}

\end{document}
