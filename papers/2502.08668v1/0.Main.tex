\documentclass{article}

% Recommended, but optional, packages for figures and better typesetting:
\usepackage{microtype}
\usepackage{graphicx}
\usepackage{subfigure}
\usepackage{booktabs} % for professional tables
\usepackage{amsmath}
\usepackage{amsfonts}
\usepackage{subcaption}
\usepackage{grffile}
\usepackage{booktabs}
\usepackage{tabularx}

% hyperref makes hyperlinks in the resulting PDF.
% If your build breaks (sometimes temporarily if a hyperlink spans a page)
% please comment out the following usepackage line and replace
% \usepackage{mlsys2024} with \usepackage[nohyperref]{mlsys2024} above.
\usepackage{hyperref}

% Attempt to make hyperref and algorithmic work together better:
\newcommand{\theHalgorithm}{\arabic{algorithm}}

% Use the following line for the initial blind version submitted for review:
\usepackage{format}
% If accepted, instead use the following line for the camera-ready submission:
% \usepackage[accepted]{mlsys2024}

% The \mlsystitle you define below is probably too long as a header.
% Therefore, a short form for the running title is supplied here:

\mlsystitlerunning{Style Extraction on Text Embeddings via VAE}


\begin{document}

\twocolumn[
\mlsystitle{Style Extraction on Text Embeddings\\ Using VAE and Parallel Dataset}

\vspace{-1.5em}
\begin{center}
    \textbf{September 2, 2024}
\end{center}

\begin{mlsysauthorlist}
\mlsysauthor{InJin Kong}{snu}
\mlsysauthor{Shinyee Kang}{snu}
\mlsysauthor{Yuna Park}{snu}
\mlsysauthor{Sooyong Kim}{snu}
\mlsysauthor{Sanghyun Park}{snu}
\end{mlsysauthorlist}

\mlsysaffiliation{snu}{Seoul National University}
% It is OKAY to include author information, even for blind
% submissions: the style file will automatically remove it for you
% unless you've provided the [accepted] option to the mlsys2024
% package.

% List of affiliations: The first argument should be a (short)
% identifier you will use later to specify author affiliations
% Academic affiliations should list Department, University, City, Region, Country
% Industry affiliations should list Company, City, Region, Country

% You can specify symbols, otherwise they are numbered in order.
% Ideally, you should not use this facility. Affiliations will be numbered
% in order of appearance and this is the preferred way.
\mlsyssetsymbol{equal}

\vskip 0.3in

\begin{abstract}
This study investigates the stylistic differences among various Bible translations using a Variational Autoencoder (VAE) model. By embedding textual data into high-dimensional vectors, the study aims to detect and analyze stylistic variations between translations, with a specific focus on distinguishing the American Standard Version (ASV) from other translations. The results demonstrate that each translation exhibits a unique stylistic distribution, which can be effectively identified using the VAE model. These findings suggest that the VAE model is proficient in capturing and differentiating textual styles, although it is primarily optimized for distinguishing a single style. The study highlights the model's potential for broader applications in AI-based text generation and stylistic analysis, while also acknowledging the need for further model refinement to address the complexity of multi-dimensional stylistic relationships. Future research could extend this methodology to other text domains, offering deeper insights into the stylistic features embedded within various types of textual data.
\end{abstract}]


\section{Introduction}

In recent years, with advancements in generative models and the expansion of training datasets, text-to-speech (TTS) models \cite{valle, voicebox, ns3} have made breakthrough progress in naturalness and quality, gradually approaching the level of real recordings. However, low-latency and efficient dual-stream TTS, which involves processing streaming text inputs while simultaneously generating speech in real time, remains a challenging problem \cite{livespeech2}. These models are ideal for integration with upstream tasks, such as large language models (LLMs) \cite{gpt4} and streaming translation models \cite{seamless}, which can generate text in a streaming manner. Addressing these challenges can improve live human-computer interaction, paving the way for various applications, such as speech-to-speech translation and personal voice assistants.

Recently, inspired by advances in image generation, denoising diffusion \cite{diffusion, score}, flow matching \cite{fm}, and masked generative models \cite{maskgit} have been introduced into non-autoregressive (NAR) TTS \cite{seedtts, F5tts, pflow, maskgct}, demonstrating impressive performance in offline inference.  During this process, these offline TTS models first add noise or apply masking guided by the predicted duration. Subsequently, context from the entire sentence is leveraged to perform temporally-unordered denoising or mask prediction for speech generation. However, this temporally-unordered process hinders their application to streaming speech generation\footnote{
Here, “temporally” refers to the physical time of audio samples, not the iteration step $t \in [0, 1]$ of the above NAR TTS models.}.


When it comes to streaming speech generation, autoregressive (AR) TTS models \cite{valle, ellav} hold a distinct advantage because of their ability to deliver outputs in a temporally-ordered manner. However, compared to recently proposed NAR TTS models,  AR TTS models have a distinct disadvantage in terms of generation efficiency \cite{MEDUSA}. Specifically, the autoregressive steps are tied to the frame rate of speech tokens, resulting in slower inference speeds.  
While advancements like VALL-E 2 \cite{valle2} have boosted generation efficiency through group code modeling, the challenge remains that the manually set group size is typically small, suggesting room for further improvements. In addition,  most current AR TTS models \cite{dualsteam1} cannot handle stream text input and they only begin streaming speech generation after receiving the complete text,  ignoring the latency caused by the streaming text input. The most closely related works to SyncSpeech are CosyVoice2 \cite{cosyvoice2.0} and IST-LM \cite{yang2024interleaved}, both of which employ interleaved speech-text modeling to accommodate dual-stream scenarios. However, their autoregressive process generates only one speech token per step, leading to low efficiency.



To seamlessly integrate with  upstream LLMs and facilitate dual-stream speech synthesis, this paper introduces \textbf{SyncSpeech}, designed to keep the generation of streaming speech in synchronization with the incoming streaming text. SyncSpeech has the following advantages: 1) \textbf{low latency}, which means it begins generating speech in a streaming manner as soon as the second text token is received,
and
2) \textbf{high efficiency}, 
which means for each arriving text token, only one decoding step is required to generate all the corresponding speech tokens.

SyncSpeech is based on the proposed \textbf{T}emporal \textbf{M}asked generative \textbf{T}ransformer (TMT).
During inference, SyncSpeech adopts the Byte Pair Encoding (BPE) token-level duration prediction, which can access the previously generated speech tokens and performs top-k sampling. 
Subsequently, mask padding and greedy sampling are carried out based on  the duration prediction from the previous step. 

Moreover, sequence input is meticulously constructed to incorporate duration prediction and mask prediction into a single decoding step.
During the training process, we adopt a two-stage training strategy to improve training efficiency and model performance. First, high-efficiency masked pretraining is employed to establish a rough alignment between text and speech tokens within the sequence, followed by fine-tuning the pre-trained model to align with the inference process.

Our experimental results demonstrate that, in terms of generation efficiency, SyncSpeech operates at 6.4 times the speed of the current dual-stream TTS model for English and at 8.5 times the speed for Mandarin. When integrated with LLMs, SyncSpeech achieves latency reductions of 3.2 and 3.8 times, respectively, compared to the current dual-stream TTS model for both languages.
Moreover, with the same scale of training data, SyncSpeech performs comparably to traditional AR models in terms of the quality of generated English speech. For Mandarin, SyncSpeech demonstrates superior quality and robustness compared to current dual-stream TTS models. This showcases the potential of  SyncSpeech as a foundational model to integrate with upstream LLMs.



\section{Related Work}


\minisection{Pivot-based approaches}
Pivot translation is an approach that decomposes the translation task into two sequential steps~\cite{wu-wang-2007-pivot, utiyama-isahara-2007-comparison}.
By transferring knowledge from high-resource pivot languages, pivoting is especially effective in translation between low-resource languages \cite{zoph-etal-2016-transfer, aji-etal-2020-neural, he-etal-2022-tencent}.
In this study, pivot translation enables us to obtain high-quality candidates for the ensemble.
\citet{kim-etal-2019-pivot} discusses a pivot-based transfer learning technique where source$\rightarrow$pivot and pivot$\rightarrow$target models are first trained separately, then use pre-trained models to initialize the source$\rightarrow$target model, allowing effective training of a single, direct NMT model.
\citet{zhang-etal-2022-triangular} further investigate the transfer learning approach by utilizing auxiliary monolingual data.


Pivot translation typically employs English as the bridge language.
Nonetheless, previous studies have explored the use of diverse pivot languages, taking into account factors such as data size and the relationships between languages~\cite{paul2009importance, dabre-etal-2015-leveraging}.
By leveraging the ability of pivot translation to produce diverse outputs, several studies have focused on generating paraphrases~\cite{mallinson-etal-2017-paraphrasing, guo2019zeroshot}.
More recently, \citet{mohammadshahi-etal-2024-investigating} uses pivot translation for ensemble, but it requires computing token-level probabilities and fails to improve translation.
Our work shares the motivation with these studies, generating translations depending on the pivot path to obtain a variety of candidates.


\minisection{Ensemble in NLG tasks}
Ensemble learning is a widely adopted strategy to obtain more accurate predictions by employing multiple systems~\cite{sagi2018ensemble}.
In NMT, the traditional approach involves averaging the probability distributions of the next target token, which is predicted at each decoding step by multiple models ~\cite{bojar-etal-2014-findings} or by different snapshots~\cite{huang2017snapshot}.
When multiple sources are available, an ensemble can be conducted with predictions obtained by different sources~\cite {firat-etal-2016-zero}.
Also, a token-level ensemble through vocabulary alignment across LLMs has also been proposed~\cite{eva}.
However, these methods are not applicable to recent black-box models as they cannot compute token-level probabilities at decoding time.


Selection-based ensemble has also been explored, which chooses the final output among the existing candidates.
This can be achieved through majority voting by selecting the most frequent one~\cite{wang2022rationaleaugmented} or selecting the best candidate with QE~\cite{fernandes-etal-2022-quality, howgood}.
Recently, MBR decoding~\cite{GOEL2000115, mbr}, which aims to find the hypothesis with the highest expected utility, has gained attention.
However, this approach limits the final output space to the existing candidate pool.


\begin{figure*}[t]
  \centering
  \includegraphics[width=0.95\textwidth]{Figures/overview.pdf} 
  \caption{Overview of \ours framework.}
  \label{fig:overall}
\end{figure*}


On the other hand, the generation-based ensemble method involves generating a new final prediction.
Fusion-in-Decoder~\cite{fid} proposes an architecture that aggregates additional information with a given input.
More recently, within the context of LLMs, \citet{llm-blender} and \citet{exchangeofthought} investigate a method of using LLMs to generate multiple outputs and aggregate them.
Generating new output through LLMs offers the benefit of explicitly harnessing their pre-trained knowledge within the ensemble process.

\section{Methodology}\label{sec:methodology}

\subsection{Data Collection and Preprocessing}
This study utilizes biblical data collected from \textit{Bible SuperSearch} \cite{biblesupersearch}, a platform operating under the GNU GPL open source license. Ten different versions were initially considered: KJV, NET, ASV, ASVS, Coverdale, Geneva, KJV\_Strongs, Bishops, Tyndale, and WEB. However, Bishops, Tyndale, and WEB were excluded due to insufficient parallel data. The remaining versions were selected for their linguistic diversity and historical backgrounds to enhance the depth of our style classification study.

The biblical texts are publicly available under the GNU GPL license, allowing free use for research purposes. Our study adhered to these guidelines without altering the original texts. In the preprocessing phase, we extracted the biblical data in JSON format and encoded all text files using UTF-8 to handle special characters. The initial data quality was high, minimizing the need for extensive text cleaning.

\subsection{Embedding and Model Training}
We employed OpenAI's text-embedding-3-small model to embed each biblical sentence into 1536-dimensional vectors. This model was chosen for its balance between performance and computational efficiency, making it suitable for our research. These high-dimensional vectors capture the nuanced language style of the sentences, providing foundational data for style-based classification.

\subsection{Style Extraction}\label{subsec:style_extraction}
Text embedding is assumed to include both content and style, as represented by the following equation:

\[
\text{text\_embedding} = \text{style\_embedding} + \text{content\_embedding}
\]

Under this assumption, text embedding can be seen as simultaneously containing both the content and stylistic features of the text. In this study, we utilized this assumption to perform an analysis based on Bible data. The Bible data consists of the same verse expressed in multiple translations in a parallel structure, where the content remains the same, but the style varies. This characteristic of Bible data justifies the assumption that each translation’s content embedding is identical. That is, the differences between the translations are primarily due to style, allowing for style analysis to be conducted. The core assumption of this study is that the difference in text embeddings between translations reflects the difference in styles. This can be expressed mathematically as follows:

\begin{multline*}
\text{KJV\_embedding} - \text{Other\_embedding} = \\
\text{KJV\_style\_embedding} - \text{Other\_style\_embedding}
\end{multline*}

Through this relationship, we calculated the difference between the two text embeddings and, based on this, measured the difference in style between the translations. Specifically, the goal of the study was to analyze the text embedding differences between KJV (King James Version) and other translations (e.g., ASV (American Standard Version)) to quantify the stylistic features. To do this, we calculated the difference between embeddings, represented as 1536-dimensional vectors, and used Variational Autoencoder (VAE) as a tool to analyze the distribution of these vectors.

The VAE is an unsupervised learning method that models the distribution of data in a latent space. In this study, we aimed to utilize the VAE to classify the embedding differences between translations and detect stylistic differences through anomaly detection. By compressing the input data and reconstructing it, VAE retains the important features while learning the distribution, allowing for the modeling of stylistic differences between translations.

During the training process of the VAE, we used the distribution differences between \textit{KJV\_embedding} and \textit{ASV\_embedding}. The VAE learned the difference between KJV and ASV embeddings in the latent space and then measured the similarity between the reconstructed distribution and the original distribution. We computed the L2-norm in this reconstruction process to quantitatively evaluate the stylistic similarity or difference between the translations. This allowed us to analyze the stylistic differences between KJV and ASV, as well as conduct comparative analyses with other translations.

\begin{table}[htbp]
\caption{Notation used throughout this article.}
\label{tab:notation}
\vskip 0.15in
\begin{center}
\begin{small}
\begin{sc}
\begin{tabular}{p{0.12\textwidth} p{0.25\textwidth}}
\toprule
\textbf{Symbol} & \textbf{Description} \\
\midrule
$\mathbf{k}^{(i)}$ & Embedding of KJV, \newline $i = 1, \cdots, N$ \\
$\mathbf{a}^{(i)}$ & Embedding of ASV \\
$\mathbf{y}^{(i)}_j$ & Embedding of other Bibles,\newline $j = 1, \cdots, 5$ \\
$\mathbf{x}^{(i)}$ & KJV\_style\_embedding \newline $-$ ASV\_style\_embedding \\
$\mathbb{R}^d$ & $d$-dimensional input space \\
$\mathbb{R}^p$ & $p$-dimensional \newline feature space ($p < d$) \\
$\psi:\mathbb{R}^d \rightarrow \mathbb{R}^p$ & Encoder of VAE \\
$\theta:\mathbb{R}^p \rightarrow \mathbb{R}^d$ & Decoder of VAE \\
\bottomrule
\end{tabular}
\end{sc}
\end{small}
\end{center}
\vskip -0.1in
\end{table}


\begin{algorithm}[htbp]
   \caption{VAE Training Process}
   \label{alg:vae_training}
\begin{algorithmic}
   \STATE {\bfseries Input:} $\mathbf{x}^{(i)}$ \quad \COMMENT{Training data}
   \STATE {\bfseries Output:} $w_{\psi}$ (encoder parameters), $w_{\theta}$ (decoder parameters)
   \STATE Initialize parameters $w_{\psi}$, $w_{\theta}$
   \REPEAT
      \FOR{$i=1$ {\bfseries to} $N$}
         \STATE $z^{(i)} = \psi(\mathbf{x}^{(i)}, w_{\psi})$ \quad \COMMENT{Generate feature vectors}
         \STATE $\hat{\mathbf{x}}^{(i)} = \theta(z^{(i)}, w_{\theta})$ \quad \COMMENT{Reconstruct original embedding}
      \ENDFOR
      \STATE $\mathcal{L}_{\text{mse}} = \frac{1}{N} \sum_{i=1}^{N} (\mathbf{x}^{(i)} - \hat{\mathbf{x}}^{(i)})^2$
      \STATE Update $w_{\psi}$ and $w_{\theta}$ using gradients of $\mathcal{L}_{\text{mse}}$
   \UNTIL{parameters $w_{\psi}$ and $w_{\theta}$ converge}
   \STATE \textbf{return} $w_{\psi}$, $w_{\theta}$
\end{algorithmic}
\end{algorithm}



 In conclusion, this study evaluated the stylistic differences between Bible translations using VAE for anomaly detection. Through this process, we effectively quantified the stylistic similarities and differences between various translations. Based on the VAE model, trained on the difference between \textit{KJV\_embedding} and \textit{ASV\_embedding}, we similarly analyzed the stylistic differences between other translations. This methodology enabled sophisticated text analysis that went beyond merely examining content features to include stylistic features. Thus, we provided new insights into how stylistic differences manifest within the embedding space

\subsection{Model Architecture and Training Details}
The VAE model used in this study has an input dimension of 1536, and both encoder and decoder use fully connected (FC) layers. The size of each hidden layer follows a geometric sequence from the input dimension of 1536 to the final feature dimension (rounded to the nearest integer). Batch normalization is applied to all layers except the final output layers of both the encoder and decoder. The activation function used is Leaky ReLU ($\alpha$=1e-2) except for the final output layer of the encoder and decoder. The final output layer of the decoder uses a Sigmoid-based activation function to ensure that the output distribution lies within the range [-1,1].

The hyperparameters are as follows: 6 values for the number of hidden layers (ranging from 1 to 6) and 6 values for the feature dimension (ranging from $2^3$ to $2^8$), resulting in 36 total combinations.

We split 13,823 sentence vectors into training and test sets with a 9:1 ratio, using KJV-ASV differences as training data. The model employs fully connected layers with batch normalization and Leaky ReLU activation, and is trained using the Adam optimizer and MSE loss function. A schematic of the model structure is provided in Figure \ref{fig:model structure}.

\begin{figure}[htbp]
    \centering
    \includegraphics[width=0.5\textwidth]{VAE_Model_Structure.png}
    \caption{A schematic illustration of the VAE model. The encoder receives a 1,536-dimensional original (sentence embedding) vector as input and outputs a feature vector of the feature dimension. The decoder takes the feature vector of the feature dimension as input and outputs a 1,536-dimensional reconstructed vector.}
    \label{fig:model structure}
\end{figure}

\subsection{Evaluation Metrics}\label{subsec:evaluation_metrics}
According to our hypothesis, the KJV-ASV vector is expected to contain information related to the style of ASV, with KJV as the reference point. If a VAE with a sufficiently small feature dimension can effectively reconstruct this vector, it suggests that the VAE is leveraging specific stylistic features during the encoding-decoding process. On the other hand, if data not included in the model’s training process are reconstructed through the VAE, the reconstruction quality is expected to be poor compared to the original. Based on this characteristic, we aim to perform anomaly detection using the VAE.

We aim to verify whether the VAE, trained using KJV-ASV vectors, has effectively learned the unique style of ASV. To do so, the trained VAE will be applied to six Bible translations (ASV, NET, ASVS, Coverdale, Geneva, and KJV Strongs), and we will examine if the model successfully distinguishes ASV’s unique style compared to other translations. For the test dataset (not used during model training), ASV will serve as the normal data, and the other five translations (NET, ASVS, Coverdale, Geneva, and KJV Strongs) will serve as anomaly data, consisting of sentence embedding vectors corresponding to the same Bible verses as in the test dataset. To remove the context of KJV during ASV training, the VAE was trained on the differences between the sentence vectors of ASV and KJV (KJV-ASV). Similarly, the anomaly data from the other Bible translations will be processed by subtracting the corresponding KJV sentence vectors, following the same procedure.

Among the 36 hyperparameter sets, the model that most clearly differentiates the reconstruction L2 error distribution between the training data and the anomalies will be considered the most effective in detecting the unique style of ASV. We will evaluate how well the original data and anomaly data are distinguished using Fisher’s Linear Discriminant (FLD). FLD increases as the squared difference between the means of the two distributions becomes larger, and the sum of their variances becomes smaller. The formula for FLD \( S \) is as follows:

\[
S = \frac{(\mu_1 - \mu_2)^2}{\sigma_1^2 + \sigma_2^2}
\]

where \( \mu_1 \) and \( \mu_2 \) are the means of the original data and anomaly data distributions, respectively, and \( \sigma_1 \) and \( \sigma_2 \) are the variances of the original data and anomaly data distributions, respectively. This metric will help quantify how well the model separates the 
original data from anomalies based on reconstruction errors.

\begin{algorithm}[htbp]
   \caption{Anomaly Detection by VAE}
   \label{alg:anomaly_detection}
\begin{algorithmic}[1]
   \STATE \textbf{Input:} $\mathbf{a}^{(i)}, \mathbf{y}^{(i)}_j$, Trained parameters by $\mathbf{x}^{(i)}$: $w_{\psi}, w_{\theta}$, $\alpha = 0.1, \ldots, 1.4$
   \STATE \textbf{Output:} Fisher’s Linear Discriminant (FLD): $S_j$
   \FOR{$i = 1$ \textbf{to} $N$}
      \STATE $z^{(i)} = \psi(\mathbf{a}^{(i)}, w_{\psi})$
      \STATE $\hat{\mathbf{a}}^{(i)} = \theta(z^{(i)}, w_{\theta})$
      \FOR{$j = 1$ \textbf{to} $5$}
         \STATE $z^{(i)}_j = \psi(\mathbf{y}^{(i)}_j, w_{\psi})$
         \STATE $\hat{\mathbf{y}}^{(i)}_j = \theta(z^{(i)}_j, w_{\theta})$
      \ENDFOR
   \ENDFOR
   \STATE $\ell_{2,\mathbf{a}} = \|\mathbf{a}^{(i)} - \hat{\mathbf{a}}^{(i)}\|_2$
   \STATE $\ell_{2,\mathbf{y}_j} = \|\mathbf{y}^{(i)}_j - \hat{\mathbf{y}}^{(i)}_j\|_2$
   \STATE $\mu_{\mathbf{a}}, \sigma_{\mathbf{a}}, \mu_{\mathbf{y}_j}, \sigma_{\mathbf{y}_j} \leftarrow$ mean and standard deviation of $\ell_{2,\mathbf{a}}, \ell_{2,\mathbf{y}_j}$
   \STATE Find the threshold minimizing total error: $\gamma = \mu_{\mathbf{a}} + \alpha \sigma_{\mathbf{a}}$
   \STATE $S_j = \frac{(\mu_{\mathbf{a}} - \mu_{\mathbf{y}_j})^2}{\sigma_{\mathbf{a}}^2 + \sigma_{\mathbf{y}_j}^2}$
   \IF{$S_j > \gamma$}
      \STATE $\mathbf{y}_j$ is anomaly
   \ELSE
      \STATE $\mathbf{y}_j$ is not anomaly
   \ENDIF
   \STATE \textbf{return} $S_j$
\end{algorithmic}
\end{algorithm}

\section{Results}\label{sec:results}

\subsection{Training Convergence and Loss Analysis}

For all 36 hyperparameter combinations, both the training loss and test loss decreased and eventually converged, indicating that the models successfully learned from the data and reached a stable state in terms of reconstruction error. Detailed loss curves and analysis are provided in Figure \ref{fig:appendix_loss}.

\begin{figure}[htbp]
    \centering
    \includegraphics[width=0.5\textwidth]{test_set_loss.png}
    \caption{Test set loss during training. The x-axis represents the number of epochs, and the y-axis represents the mean error. The hyperparameters of each model are as follows: starting from left the 1st, 2nd, and 3rd columns represent feature dimensions of 8, 64, and 256, respectively, and the starting from top 1st, 2nd, and 3rd rows represent 1, 3, and 6 hidden layers, respectively.}
    \label{fig:appendix_loss}
\end{figure}

\subsection{L2 Error Distribution and FLD Analysis}

The L2 error distribution for each model is presented in Figure \ref{fig:appendix_l2_error}. The minimum Fisher’s Linear Discriminant (FLD) between the L2 norm distributions of the reconstructed sentence vectors from the trained dataset (ASV) and the anomaly datasets (NET, ASVS, Coverdale, Geneva, KJV Strongs) across the 36 models is shown in Figure \ref{fig:appendix_fld}.

\begin{figure}[htbp]
    \centering
    \includegraphics[width=0.5\textwidth]{distribution.jpg}
    \caption{L2 error distribution on ASV, NET, ASVS, Coverdale, Geneva, and KJV Strongs. The x-axis represents the L2 error between the original and reconstructed sentence vector, and the y-axis represents the distribution density. The hyperparameters of each model are as follows: starting from left the 1st, 2nd, and 3rd columns represent feature dimensions of 8, 64, and 256, respectively, and starting from top the 1st, 2nd, and 3rd rows represent 1, 3, and 6 hidden layers, respectively.}
    \label{fig:appendix_l2_error}
\end{figure}

\begin{figure}[htbp]
    \centering
    \includegraphics[width=0.5\textwidth]{FLD.jpg}
    \caption{(Left) Minimum and (Right) Maximum of FLD between ASV and other 5 anomaly datasets (NET, ASVS, Coverdale, Geneva, and KJV Strongs). A higher minimum FLD indicates better differentiation between ASV and anomaly L2 error distributions.}
    \label{fig:appendix_fld}
\end{figure}

The minimum FLD is more important than the maximum FLD for determining the separation between normal and anomaly data. A high minimum FLD represents the model that has the most differentiation between the ASV original and the anomaly reconstructions, indicating the best-performing model in terms of distinguishing between the original and anomalous styles based on the L2 norm distribution. Figure \ref{fig:appendix_fld} shows that the minimum FLD is maximized in models with 3 hidden layers and a feature dimension size between 32 and 128. Models with too small or too large hidden layers and feature dimensions tend to perform poorly in anomaly differentiation.

Across the 36 models, the anomaly dataset that produced the minimum FLD most frequently was Geneva, appearing 31 times, followed by KJV Strongs, which appeared 5 times. This suggests that the L2 error distribution of the Geneva dataset was generally the closest to that of ASV, making it the hardest to distinguish from ASV. Conversely, the anomaly dataset that consistently produced the maximum FLD in all 36 models was Coverdale, indicating that it was the easiest to distinguish from ASV based on the L2 error distribution. This result highlights the distinctiveness of Coverdale's style compared to ASV, while Geneva's style appears more similar.

\subsection{Impact of Context Subtraction on VAE Performance}

\begin{figure}[htbp]
    \centering
    \includegraphics[width=0.5\textwidth]{no_sub_distribution.jpg}
    \caption{L2 error distribution on ASV, NET, ASVS, Coverdale, Geneva, and KJV Strongs, without parallel sentence (KJV) subtraction. The x-axis represents the L2 error between the original and reconstructed sentence vector, and the y-axis represents the distribution density. The hyperparameters of each model are as follows: starting from left the 1st, 2nd, and 3rd columns represent feature dimensions of 8, 64, and 256, respectively, and the starting from top 1st, 2nd, and 3rd rows represent 1, 3, and 6 hidden layers, respectively.}
    \label{fig:appendix_no_sub_l2_error}
\end{figure}

\begin{figure}[htbp]
    \centering
    \includegraphics[width=0.5\textwidth]{no_sub_FLD.jpg}
    \caption{(Left) Minimum and (Right) Maximum of FLD between ASV and other 5 anomaly datasets (NET, ASVS, Coverdale, Geneva, and KJV Strongs), without parallel sentence (KJV) subtraction. A higher minimum FLD indicates better differentiation between ASV and anomaly L2 error distributions.}
    \label{fig:appendix_no_sub_fld}
\end{figure}

Training the VAE without subtracting context parallel sentence (KJV) vectors demonstrated that both the training loss and test loss decreased and converged, indicating successful learning. However, as shown in Figure \ref{fig:appendix_no_sub_l2_error} in Appendix F, the mean L2 error across all distributions was higher compared to the models trained with parallel sentence subtraction.

When comparing Figures \ref{fig:appendix_no_sub_fld} and \ref{fig:appendix_fld}, the Fisher’s Linear Discriminant (FLD) for the no-subtraction case (from context parallel sentence vectors) is significantly lower than for the subtracted case. Specifically, the mean of the minimum FLD across the 36 models in the subtracted case is 1.111, while the mean for the no-subtraction case is 0.116, making the FLD approximately 9.6 times lower without subtraction.

Furthermore, the highest maximum FLD in the no-subtraction case (1.000) is nearly the same as the lowest minimum FLD in the subtracted case (0.983). This stark difference in FLD highlights that when trained without subtracting the context parallel sentence vectors, the VAE's ability to distinguish anomalies from normal (trained domain) data is significantly diminished. This result reinforces the idea that the subtraction of context helps the VAE better capture stylistic differences, leading to clearer separation between ASV and other translations.

\section{Discussion}\label{sec:discussion}

This study extracted the styles of various Bible translations and utilized a Variational Autoencoder (VAE) model to analyze how these styles differ, particularly in comparison to the American Standard Version (ASV). The results revealed that the styles of each Bible translation followed a normal distribution, and these distributions could be clearly distinguished from that of the ASV. This indicates that there are stylistic differences between the ASV and other translations, and that these differences can be effectively detected using the VAE model.

After optimizing the VAE model’s hyperparameters, the process of distinguishing between the ASV and other translation styles resulted in a Type 1 error of 8.7\% and a Type 2 error of 6.7\%, with a total error rate of 15.3\%. Conversely, the model achieved an accuracy of 84.7\%, demonstrating its ability to effectively differentiate styles. This level of accuracy suggests that the model can clearly recognize the distribution of a specific style and use it as a basis to distinguish between the styles of different translations.

However, the VAE model was optimized for distinguishing a single style. While it was useful for detecting differences between a specific translation style and the ASV, it had limitations when it came to distinguishing multiple styles simultaneously or understanding the relationships between complex, multi-dimensional styles. These limitations stem from the structural characteristics of the VAE, which compresses the data’s features during learning, making it inherently challenging to fully capture the complex characteristics of the data. Therefore, to distinguish multiple styles simultaneously, it may be necessary to use other models or train the VAE model in a more sophisticated manner.

The ability to extract a specific style suggests that the style’s characteristics can be quantified and represented as a probability distribution. This means that AI can utilize this quantified style representation to generate text that adheres to a specific style. For example, in text generation tasks where a particular writing style or tone is required, a 'style metric' could be used as a numerical and comparable indicator to assess and ensure that the generated text conforms to the desired style.

The approach taken in this study opens up the possibility of expanding the research to other parallel text datasets. By applying this methodology to other text domains, researchers can study the stylistic differences and their implications within each domain. For example, the approach could be extended to analyze the styles of different translations of literary works, legal document translations, or works by various authors.

We have demonstrated that the VAE model can distinguish between the original and anomaly data using the reconstruction L2 error. To measure the overall accuracy, False Positive Rate (FPR), and False Negative Rate (FNR) of the model, we created an Accuracy Test Dataset using data not included in the training set. This dataset consisted of 1,000 samples, with 50\% of the samples being from ASV and the remaining 50\% from five anomaly datasets (NET, ASVS, Coverdale, Geneva, and KJV Strongs).

\begin{table}[t]
\caption{Accuracy \& Error Rates of Models 1, 2, and 3 on Anomaly Detection}
\label{tab:accuracy}
\vskip 0.15in
\begin{center}
\begin{small}
\begin{sc}
\setlength{\tabcolsep}{4pt} % Default value: 6pt. Reduce this to make the table narrower.
\begin{tabular}{|c|c|c|c|}
\hline
\textbf{Model} & \textbf{Accuracy} & \textbf{Type I Error} & \textbf{Type II Error} \\
\hline
Model 1 & 83.5\% & 9.8\% & 6.7\% \\
Model 2 & 82.9\% & 10.1\% & 7.0\% \\
Model 3 & 83.4\% & 9.8\% & 6.8\% \\
\hline
\textbf{Average} & \textbf{83.3\%} & \textbf{9.9\%} & \textbf{6.8\%} \\
\hline
\end{tabular}
\end{sc}
\end{small}
\end{center}
\vskip -0.1in
\end{table}


The binary classification results showed that the lowest overall error rate was achieved when the threshold was set at mean + 0.8 std. The average overall error rate across the three models was 16.8\%. The relatively high FNR, particularly with Geneva, suggests that modern English Bible translations inherently do not exhibit distinct stylistic differences.

The results of the anomaly detection using the VAE in this study also show trends similar to what would be expected when humans classify ASV and other Bible versions. In this study, the L2 error distributions of ASV and Geneva had a significant overlap, making it difficult to classify them with a low error rate using a specific threshold. In contrast, the L2 error distribution of Coverdale barely overlapped with ASV, and the FLD was the highest across all models. More typically, 67.8\% (ASV 34.6\%, KJV Strongs 33.2\%) of type 2 error in anomaly detection is from Geneva and KJV Strongs.

\begin{table}[t]
\caption{Original Sentences of 3 Different Versions: ASV, Geneva, Coverdale}
\label{tab:verses}
\vskip 0.15in
\begin{center}
\begin{small}
\begin{sc}
\begin{tabularx}{0.45\textwidth}{lX}
\toprule
\textbf{Verse} & \textbf{Translation} \\
\midrule
\textbf{Gen 1:1} & \textbf{ASV:} In the beginning God created the heavens and the earth. \newline
\textbf{Geneva:} In the beginning God created the heauen and the earth. \newline
\textbf{Coverdale:} In ye begynnynge God created heauen \& earth: \\
\midrule
\textbf{Mat 1:1} & \textbf{ASV:} The book of the generation of Jesus Christ, the son of David, the son of Abraham. \newline
\textbf{Geneva:} The book of the generation of Jesus Christ the son of David, the son of Abraham. \newline
\textbf{Coverdale:} This is the boke of the generacion of Iesus Christ ye sonne of Dauid, the sonne of Abraham. \\
\midrule
\textbf{Joh 3:16} & \textbf{ASV:} For God so loved the world, that he gave his only begotten Son, that whosoever believeth on him should not perish, but have eternal life. \newline
\textbf{Geneva:} For God so loveth the world, that he hath given his only begotten Son, that whosoever believeth in him, should not perish, but have everlasting life. \newline
\textbf{Coverdale:} For God so loued the worlde, that he gaue his onely sonne, that who so euer beleueth in hi, shulde not perishe, but haue euerlastinge life. \\
\bottomrule
\end{tabularx}
\end{sc}
\end{small}
\end{center}
\vskip -0.1in
\end{table}

Table \ref{tab:verses} illustrates the textual differences between three versions of the Bible (ASV, Geneva, Coverdale), which could influence the VAE's ability to distinguish anomalies. The relatively low accuracy of anomaly detection using the VAE in this study may be attributed to the subtle stylistic differences between the texts. This implies that using sentences with clearer stylistic differences and more varied contexts in future experiments could result in better accuracy.


\section{Conclusion}
This paper presents SyncSpeech, a dual-stream speech generation model built on a temporal masked transformer. SyncSpeech can efficiently generate low-latency streaming speech from the real-time text input, maintaining the high quality and robustness of the generated speech. We conducted comprehensive performance evaluations and analysis experiments in both English and Mandarin, demonstrating its capability as a foundational model for integration with upstream LLMs. In the future, SyncSpeech will be trained on larger datasets to further improve its performance.
 

\section{Limitations}
In this section, we will analyze the limitations of
SyncSpeech and discuss potential future work. SyncSpeech requires token-level alignment information, which is challenging to achieve for sentences with mixed languages, and preprocessing becomes time-consuming on large-scale datasets. In the future, we will explore semi-supervised duration prediction, which only requires the duration of a complete sentence without strict token-level alignment information, and integrate SyncSpeech into SLLM as a speech generation module. In addition, since the off-and-shelf streaming speech decoder relies on flow matching, it limits the off-the-shelf RTF and the FPL. Moreover,` current single-codebook acoustic tokens, such as WavTokenizer \cite{wavtokenizer}, do not support streaming decoding. In the future, we will investigate efficient and low-latency streaming speech decoders.

\bibliography{citation}
\bibliographystyle{format}

\end{document}
