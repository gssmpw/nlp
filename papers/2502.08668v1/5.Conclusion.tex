\section{Conclusion}\label{sec:conclusion}

This study has successfully demonstrated the application of a Variational Autoencoder (VAE) model to analyze and distinguish the stylistic differences among various Bible translations, with a particular focus on the American Standard Version (ASV). By embedding textual data into high-dimensional vectors and applying anomaly detection techniques, the study identified unique stylistic distributions for each translation, showcasing the model’s capability to differentiate between these styles with a notable accuracy rate of 84.7\%. The findings confirm that the VAE model can effectively capture and differentiate textual styles, though it is primarily optimized for distinguishing a single style.

Despite these successes, the study also highlighted certain limitations inherent in the VAE model, particularly its challenges in simultaneously distinguishing multiple styles or comprehending complex, multi-dimensional stylistic relationships. This limitation underscores the need for further refinement of the model, potentially through the integration of more advanced machine learning techniques, such as deep learning models capable of classification or other unsupervised learning methods. Addressing these limitations could enhance the model’s ability to manage more complex stylistic differentiation tasks.

The implications of this research extend beyond academic inquiry, offering significant potential applications in the field of AI-driven text generation. The ability to extract and measure specific stylistic features opens up possibilities for generating texts with targeted stylistic attributes, which can be invaluable in automated writing tools, personalized content creation, and stylistic analysis of literary works. Moreover, the methodology employed in this study can be adapted to other text domains, providing a framework for analyzing stylistic differences across various types of textual data, including literary translations, legal documents, and author-specific writing styles.

In conclusion, while this study provides a solid foundation for the analysis of textual styles using VAE, it also sets the stage for future research to explore more sophisticated models and methodologies. By expanding this approach to other text domains and enhancing the model’s capabilities, future work can continue to deepen our understanding of textual styles and their applications in AI and beyond.
