

\begin{abstract}
    Large Language Models (LLMs) have shown remarkable capabilities as AI agents. However, existing methods for enhancing LLM-agent abilities often lack a focus on data quality, leading to inefficiencies and suboptimal results in both fine-tuning and prompt engineering. To address this issue, we introduce EDGE, a novel approach for identifying informative samples without needing golden answers. We propose the Guideline Effectiveness (GE) metric, which selects challenging samples by measuring the impact of human-provided guidelines in multi-turn interaction tasks. A low GE score indicates that the human expertise required for a sample is missing from the guideline, making the sample more informative. By selecting samples with low GE scores, we can improve the efficiency and outcomes of both prompt engineering and fine-tuning processes for LLMs. Extensive experiments validate the performance of our method. Our method achieves competitive results on the HotpotQA and WebShop and datasets, requiring 75\% and 50\% less data, respectively, while outperforming existing methods. We also provide a fresh perspective on the data quality of LLM-agent fine-tuning.
\end{abstract}

% Large Language Models (LLMs) have shown remarkable capabilities as AI agents. However, existing methods for enhancing LLM-agent abilities often lack a focus on data quality, leading to inefficiencies and suboptimal results in both fine-tuning and prompt engineering. To address this issue, we introduce EDGE, a novel approach for identifying informative samples without needing golden answers. We propose the Guideline Effectiveness (GE) metric, which selects challenging samples by measuring the impact of human-provided guidelines in problem-solving scenarios. A low GE score indicates that the human expertise required for a sample is missing from the guideline, making the sample more informative. By selecting samples with low GE scores, we can improve the efficiency and outcomes of both prompt engineering and fine-tuning processes for LLMs. Extensive experiments validate the performance of our method. Our method achieves competitive results on the WebShop and HotpotQA datasets, requiring 75\% and 50\% less data, respectively, while outperforming existing methods.


