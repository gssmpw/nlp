We have demonstrated that robotic manipulation problems can be effectively represented as heterogeneous graphs, comprising two sub-graphs to capture the geometric structure of the environment. Building on this, we introduced HEPi, a graph-based policy featuring multiple equivariant message-passing networks as its backbone. These networks are constrained to be equivariant under $SE(3)$ transformations, which significantly improves sample efficiency. Furthermore, HEPi explicitly models heterogeneity by assigning distinct network parameters for each interaction type, reducing message mixing and improving expressiveness. This approach has proven less prone to converging on sub-optimal solutions. To assess the effectiveness of our approach, we developed a new reinforcement learning benchmark focused on manipulating objects with diverse geometries and deformable materials. Our results show that HEPi outperforms both the state-of-the-art Transformer and its non-heterogeneous, non-equivariant counterparts.

\textbf{Limitation} In our current setup, we abstract away the robot body, focusing solely on end-effector movements. Future work could explore incorporating a more structured representation of actuator nodes, potentially leveraging the robot's full morphology. Moreover, although our approach does not require full object meshes, we assume that the \rebuttal{keypoint} coordinates are readily available as our main observation. This limitation could be addressed by integrating state-of-the-art computer vision techniques to extract keypoints from cameras \citep{dino_tracker_2024, hou2024keygrid}, using these as object nodes, thus increasing its applicability in real-world scenarios.
