\begin{figure*}[t]
    \makebox[\textwidth][c]{
    \input{ICLR_2025/Figure_Wrappers/Legend/equi_five_methods}
    }
    \centering
    % First row of figures
    \begin{subfigure}[b]{0.32\linewidth}
        \includegraphics[width=\textwidth]{ICLR_2025/Figures/eval_cloth_hanging_equi/eval_full_Isaac-Cloth-Hanging-Multi-v0_eval_all.pdf}
        \caption{}
    \end{subfigure}
    \hfill
    \begin{subfigure}[b]{0.32\linewidth}
        \includegraphics[width=\textwidth]{ICLR_2025/Figures/eval_cloth_hanging_equi/eval_half_yaw_Isaac-Cloth-Hanging-Multi-v0_eval_all.pdf}
        \caption{}
    \end{subfigure}
    \hfill
    \begin{subfigure}[b]{0.32\linewidth}
        \includegraphics[width=\textwidth]{ICLR_2025/Figures/eval_cloth_hanging_equi/eval_quater_yaw_Isaac-Cloth-Hanging-Multi-v0_eval_all.pdf}
        \caption{}
    \end{subfigure}
    
    \medskip
    % Second row of figures - centered by wrapping in a minipage
    \begin{minipage}{0.65\textwidth}
    \centering
    \begin{subfigure}[b]{0.49\linewidth}
        \includegraphics[width=\textwidth]{ICLR_2025/Figures/eval_cloth_hanging_equi/appx_eval_quater_half_yaw_Isaac-Cloth-Hanging-Multi-v0_eval_all.pdf}
        \caption{}
    \end{subfigure}
    \hfill
    \begin{subfigure}[b]{0.49\linewidth}
        \includegraphics[width=\textwidth]{ICLR_2025/Figures/eval_cloth_hanging_equi/appx_eval_one_yaw_Isaac-Cloth-Hanging-Multi-v0_eval_all.pdf}
        \caption{}
    \end{subfigure}
    \end{minipage}
    \caption{Performance of different models on the \emph{Cloth-Hanging} task across various sample spaces. Assuming the global scene located at $r=[0,1,0]^T$, then from left to right, we generate sample by rotating $r$ by (a) $\theta_{\text{roll}} \in (-\pi/4, \pi/2)$, $\theta_{\text{yaw}} \in (-\pi, \pi)$, (b) $\theta_{\text{yaw}} \in (-\pi/2, \pi/2)$, and (c) $\theta_{\text{yaw}} \in (-\pi/4, \pi/4)$. Meanwhile, the bottom row shows results for (d) $\theta_{\text{yaw}}\in (-\pi/8, \pi/8)$, and (e) the fixed orientation at $\theta_{\text{roll}}=0, \theta_{\text{yaw}}=0$. As the sample space decreases, performance improves across all models, with HEPi consistently outperforming the baselines. The additional plot with fixed orientation on the bottom are averaged over 5 seeds while the others with 10 seeds.}
    \vspace{-0.2cm}
    \label{fig:appx_eval_equi}
\end{figure*}
