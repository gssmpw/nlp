In this section, we compare our testbed to other fluid-based testbeds, and provide information regarding the sharing of our data and code.
%
\scaleSubsection
\subsection{Comparison with Other Testbeds}\label{ssec:comparison}
\scaleSubsectionBelow
%
\begin{table}[!tbp]
    \caption{Overview of Selected Fluid-Based \ac{MC} Testbeds}
    \centering
    \resizebox{1\columnwidth}{!}{
        \begin{tabular}{cccccc}
            \hline\hline
             Signaling Molecule (Biocompatibility) & Data rate (bit/s) & \ac{BER} (\# transmitted bits) & Detection Method & Reference\\
             \hline\hline
             Green Fluorescent Protein Dreiklang (GFPD) (\cmark) & 0.6 & 0 (5370) & Differential Signal + Adaptive Threshold & This paper\\
             \hline
             Superparamagnetic Iron Oxide & & & & \\[-0.2cm]
             Nanoparticles (SPIONs) (\cmark) & 10 & 0.0843 (600) & Convolutional Neural Network &  \cite{bartunik2023development}\\ % transmission efficiency is 160 bit/Ml M = 8 (2000 symbols were transmitted, but only 200 = 600bits were used for testing the detection
             & &  & $\mu$ Link Decoder with Channel Estimation& & \\[-0.2cm]
             Sodium chloride (NaCl) (\cmark) & 5 & 0.002 (100 x 100) & and Sequence Detection & \cite{wang2020understanding} \\
             Colored Ink (\xmark) & 1.5 & 1/312 (312) & Threshold & \cite{wietfeld2024evaluation}\\ % three colors simultaneously 3xlog(2) = 3 bit/symbol, data rate = 1.5 bps, 1/312 BER (3 color molecules) Threshold detector on individual colors
             Acid/Base (\xmark) & 2.63 & 0 (1080) & Recurrent Neural Network & \cite{farsad2017novel}\\ % SVM/RNN based detection
             Sodium Hydroxide (NaOH) (\xmark) & 1/375 & 0.04 (100) & Threshold &  \cite{walter2023real}\\ % threshold detection, 6.25*60, supplementary file
             Hydrogen Chloride (HCl) (\xmark) & 1/27 & 0.022 (1000) & Adaptive Threshold & \cite{khaloopour2019experimental}\\ % 10 cm^3 solution per bit (?) , adaptive threshold detection
             Transfer DNA (tDNA) (\cmark) & 1/120 & 0 (20) & Differential Signal + Threshold & \cite{kuscu2021fabrication}\\ %differential detection with threshold detector
             Glucose ($\textrm{C}_6 \textrm{H}_{12} \textrm{0}_6$) (\cmark) & 2 & 0.05 (654) & Machine Learning & \cite{koo2020deep}\\ % Module based machine learning, some of the measurements had to be discarded as the transmitter was running out of molecules earlier than expected
             \hline\hline
        \end{tabular}
    }
    \label{tab:experiments}
     \vspace{-1.0cm}
\end{table}
Our comparison is focused on the experimentally determined \acp{BER}, the detection schemes used, and the efficiency of data transmission. \Table{tab:experiments} provides a selected overview of fluid-based \ac{MC} testbeds, listing for each testbed the signaling molecule used, the lowest reported \ac{BER}, the data rate achieving this \ac{BER}, the length of the bit sequence transmitted, and the employed detection method.

Providing a fair and still meaningful comparison between existing \ac{MC} testbeds and the proposed testbed is challenging due to some fundamental differences.
In particular, none of the existing works has considered a closed-loop and self-contained topology. As a result, only our testbed is confronted with the challenge of resolving inter-loop \ac{ISI}, offset \ac{ISI}, and permanent \ac{ISI}. In addition, all other testbeds require and use a much larger number of signaling molecules due to their open topology. The resulting low molecule efficiency, which in \cite{bartunik2021increasing} has been defined as the number of bit transmitted per volume of dissolved signaling molecule solution, is a significant challenge in existing testbeds. The issue of low efficiency has been recognized and discussed \cite{bartunik2023development, koo2020deep, lin2024testbed}: In \cite{bartunik2023development}, an efficiency of $160 \,\si{\bit \per \milli\liter}$ was achieved, which is still low compared to our testbed with $90,000/9 \, \si{\bit \per \milli\liter} = 10,000 \,\si{\bit \per \milli\liter}$; in \cite{koo2020deep}, the authors reported that some measurements had to be discarded because the \ac{TX} ran out of molecules earlier than expected; and in \cite{lin2024testbed}, high costs were associated with the use of the testbed, resulting in a maximum evaluated bit sequence length of $8$ bit\footnote{The authors in \cite{lin2024testbed} also reported that for such a short sequence a meaningful estimation of the \ac{BER} was not possible. We believe that this is an example where the discussed alternative performance metrics, such as \ac{AMED} and eye diagrams, are particularly useful.}.

Despite the different topologies and the challenges our testbed has to overcome, we achieve a data rate that is only an order of magnitude lower than the largest data rates reported in the \ac{MC} literature, which are from \cite{bartunik2023development}\footnote{For the sake of completeness, we note that, to the best of our knowledge, the highest \textit{achievable} data rate was reported in \cite{huang2024non}.}. At the same time, our testbed provides very reliable communication, i.e., very low \acp{BER}, while employing a low-complexity detection scheme that involves only differentiation of the received signal and an adaptive threshold detector. Quantifying the computational complexity of the detection schemes listed here is beyond the scope of this paper. However, methods based on machine learning, such as Convolutional Neural Networks \cite{bartunik2023development} and Recurrent Neural Networks \cite{farsad2017novel}, as well as methods that explicitly take into account the memory effects caused by \ac{ISI} \cite{wang2020understanding}, even if efficiently implemented using the Viterbi algorithm, are believed to have comparatively higher computational complexities.
%
\scaleSubsection
\subsection{Data Handling and Code Sharing}\label{ssec:datasharing}
\scaleSubsectionBelow
%
To promote transparency and enable experimental evaluation of new communication algorithms developed by the \ac{MC} community, we publish our experimental data and the Python code for synchronization and detection in Zenodo and a Git repository under the CC BY and the MIT licenses, respectively. When used, the data and/or code can and should be cited using the corresponding Zenodo \ac{DOI} \cite{scherer2025Zenodo}. The link to the Git repository can be found on Zenodo.
