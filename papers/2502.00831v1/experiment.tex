In this section, an overview of the testbed is provided.
A schematic representation of the building blocks of the testbed and their functionality is displayed in \Figure{fig:overview}, while \Figure{fig:lab_overview} shows the testbed during operation. 
%
\begin{figure}[!tbp]
    \vspace{-0.5cm}

    % Figure a
    \begin{subfigure}[t]{0.99\textwidth} % Volle Breite für Figure a
        \caption{}
        \centering
        \includegraphics[width=\textwidth]{fig/2d_schmetatic_final_a.pdf}
        \label{fig:overview} 
    \end{subfigure}

    \vspace{-1.2cm} % Abstand zwischen den Reihen


    % Figure d und e (besondere Positionierung)
    % Figure d: Breiter als die Mitte
    \begin{subfigure}[t]{0.8\textwidth} 
        \caption{}
        \includegraphics[height=6cm]{fig/2d_schmetatic_final_d.pdf}        
        \label{fig:lab_overview}
    \end{subfigure}
    \hspace{0.05\textwidth} % Kleiner Abstand
    % Figure e: Ganz rechts
    \begin{subfigure}[t]{0.1\textwidth} % Schmale Breite für Figure e
        \caption{}
        \includegraphics[height=6cm]{fig/2d_schmetatic_final_e.pdf}
        \label{fig:led_array}
    \end{subfigure}
    
    \vspace{-0.5cm}
    \caption{(a) Schematic representation of the testbed consisting of the propagation channel, pump, test tube, two LED-arrays serving as \ac{TX} and \ac{EX}, respectively, a flow cell, and a spectrometer acting as \ac{RX}. The switching processes of \ac{GFPD} are depicted next to the components in which they occur. (b) Photo of the experimental setup during operation. (c) \ac{LED} array of the \ac{TX} module (the \ac{LED} module of the \ac{EX} looks similar). Individual \acp{LED} of the array are highlighted by red circles.}
    \label{fig:total_overview}
    \vspace*{-11mm}
\end{figure}
\scaleSubsection
\subsection{Closed-Loop System: Tube System and Pump}\label{subsec:structure}
\scaleSubsectionBelow
For our testbed, we use a \ac{FEP} tube with radius $r_\mathrm{T} = 8 \, \times 10^{-4} \, \si{\meter}$ and total length $L_\mathrm{T} = 2.74 \,\si{\meter}$. To achieve the closed-loop structure, both ends of the tube are put into a test tube (\textit{Falcon}$\,\textsuperscript{\tiny\textregistered}$). The \textit{Falcon}$\,\textsuperscript{\tiny\textregistered}$ is filled with $V_\mathrm{R} = 2.50 \,\si{\milli\liter}$ of fluid\footnote{In fact, the fluid is a solution of dissolved \ac{GFPD}. Details on the composition of the \ac{GFPD} solution and its introduction into the system are provided in \Section{SubSec:Procedure}.}. Hence, as long as both ends of the tube are immersed in the fluid of the \textit{Falcon}$\,\textsuperscript{\tiny\textregistered}$, they are hereby connected. To ensure this, $V_\mathrm{R} = 2.50 \,\si{\milli\liter}$ are suitable to immerse the tube ends in the fluid even if the tubes move slightly during the pumping process.

For pumping, a gear pump (\textit{Ismatec REGLO-Z}), attached to the tube, is used. The pump drives the fluid in the system with an effective volume flux of $Q = 9.45 \, \mathrm{mL}\;\mathrm{min}^{-1}$, leading to an effective flow velocity of $\veff = 0.139 \,\si{\meter \per\second}$ for the given radius of the tube $r_\mathrm{T}$. 
%
\scaleSubsection
\subsection{Signaling Molecule: GFPD and Its Properties}\label{subsec:GFPD}
\scaleSubsectionBelow
\ac{GFPD} is a photoswitchable and biocompatible \ac{GFP} introduced in \cite{brakemann2011reversibly}. The authors in \cite{brakemann2011reversibly} show that \ac{GFPD} has some unique properties, which make it well suited for the use in our testbed. In particular, the switching efficiency, robustness to photobleaching, and fluorescence intensity are high for \ac{GFPD}. In the following, we discuss these properties of \ac{GFPD} in detail. 
%
\scaleSubsubsection
\subsubsection{Reversible Photoswitchability -- Modulation und Resetting via Active Switching}\label{par:switching}
\scaleSubsubsectionBelow
In general, \textit{photoswitchability} means that the properties of a molecule can be changed upon exposure to light of specific wavelengths. In the case of \ac{GFPD}, photoswitchability allows for a reversible switching of its fluorescence between two\footnote{In fact, a third state exists, which we denote as \textit{equilibrium state}. In the absence of light, \ac{GFPD} may switch to this state, which shows a fluorescence similar to that of the ON state. As the equilibrium state's fluorescence is only slightly higher, in this work, for simplicity of presentation, we do not differentiate between the ON state and the equilibrium state.} distinct states, namely the ON and OFF states, which can be controlled by irradiation of light with different wavelengths.
In particular, \ac{GFPD} may be switched to the ON and OFF states upon irradiation with light of wavelengths $\lambda_\mathrm{ON} = 365 \, \si{\nm}$ and $\lambda_\mathrm{OFF} = 405 \, \si{\nm}$, respectively. The success of the switching depends on the intensity of the irradiation.
The switching property allows the modulation and deletion of information into and from the state of the \ac{GFPD}, facilitated by a \ac{TX} and \ac{EX} module in our testbed, respectively, which are discussed in detail in \Section{SubSec:InfComp}.
%
\scaleSubsubsection
\subsubsection{Fluorescence-Based State Readout}\label{par:readout}
\scaleSubsubsectionBelow
Only \acp{GFPD} in the ON state show a high fluorescence, i.e., emit light at a wavelength of $\lambda_\mathrm{E} = 529 \, \si{\nm}$ after being excited by light of wavelength $\lambda_\mathrm{T} = 500 \, \si{\nm}$. Hence, the wavelengths responsible for switching, i.e., $365 \, \si{\nm}$ and $405 \, \si{\nm}$, are decoupled from the fluorescence excitation and detection wavelengths. To read the fluorescence, we use a spectrometer in our testbed, which is detailed in \Section{SubSec:InfComp}.
%
\scaleSubsubsection
\subsubsection{Thermal Relaxation to the ON State}\label{par:relaxation}
\scaleSubsubsectionBelow
%
Thermal relaxation denotes the process by which a perturbed system spontaneously returns to its equilibrium state. For \ac{GFPD}, this means that an OFF state \ac{GFPD} can spontaneously switch back to its ON state. The thermal relaxation of \ac{GFPD} to the ON state has a half-life of $T_{1/2} = 600 \,\si{\second}$ at room temperature \cite{brakemann2011reversibly}. As the thermal relaxation process depends on temperature, a laboratory with constant room temperature, $21 \, \si{\celsius}$ in our case, supports reproducible experimental results.

We note that only the ON state of \ac{GFPD} is thermodynamically stable \cite{brakemann2011reversibly}. Consequently, the ON state is predominant in equilibrium in the proposed testbed.
%
\scaleSubsubsection
\subsubsection{High Robustness to Photobleaching}\label{par:photobleaching}
\scaleSubsubsectionBelow
After multiple photoswitching cycles and readouts, photobleaching of \ac{GFPD} may occur \cite{brakemann2011reversibly}. Photobleaching is a general phenomenon when irradiating \ac{GFP} and its variants such as \ac{GFPD} and refers to the rare event of destruction of the fluorophore molecule upon exposure to light. Photobleaching is energy dependent, i.e., it increases with higher irradiation power or longer irradiation duration \cite[Suppl. Fig. 7]{brakemann2011reversibly}.
A photobleached \ac{GFPD} irreversibly loses its ability to be fluorescent \cite{wang2016gmars, brakemann2011reversibly}, which can be interpreted as \ac{GFPD} degradation, visible as a gradual decrease of fluorescence intensity upon repeated exposure to light.
%
\scaleSubsubsection
\subsubsection{Biocompatibility}\label{sec:GFPD_bio}
\scaleSubsubsectionBelow
Like all \acp{GFP}, \ac{GFPD} is \textit{biocompatible} \cite{richards2003safety}. This property allows the use of \ac{GFP}-based proteins in a wide range of living organisms without causing significant harm or disruption to their normal functions \cite{richards2003safety}. The biocompatibility of \ac{GFPD} also makes our laboratory experiments safe.
%
\scaleSubsection
\subsection{Experimental Procedure} \label{SubSec:Procedure}
\scaleSubsectionBelow
%
Before the testbed can be used for information transmission, the \ac{GFPD} solution must be prepared and filled into the testbed. Hence, each experiment involves the following steps.
%
\scaleSubsubsection
\subsubsection{Step 1 (Preparation of GFPD Solution)} \label{SubSec:GFPD_Solution}
\scaleSubsubsectionBelow
%
Phosphate-buffered saline (PBS) buffer, a water-based salt solution, is used as the liquid buffer medium. Specifically, a high-concentration stock of \ac{GFPD} is diluted with PBS buffer and the detergent Triton X-100, which is also needed to stabilize \ac{GFPD} for long-term experiments, to a total volume of $9 \,\si{\milli\liter}$. In the end, the \ac{GFPD} solution contains a \ac{GFPD} concentration of $\con = 0.3 \,\si{\milli\gram \per\milli\liter}$ and a Triton X-100 concentration of $C_{\mathrm{TRITON}} = 0.125 \,\si{\milli\gram \per\milli\liter}$.
%
\scaleSubsubsection
\subsubsection{Step 2 (Filling the GFPD Solution into the Closed-Loop System)} \label{SubSec:Filling}
\scaleSubsubsectionBelow
%
The \ac{GFPD} solution is pumped into the tube, which has a volume of $\volTube = 5.43 \,\si{\milli\liter}$, and all segments within the pump and flow cell, which have a joint volume of $V_\mathrm{PF} = 1.07 \,\si{\milli\liter}$, until the test tube is filled with $V_\mathrm{R} = 2.50 \,\si{\milli\liter}$, which requires in total $9 \,\si{\milli\liter}$ of \ac{GFPD} solution. At this point, both ends of the tubing are placed in the test tube to close the loop.
%
\scaleSubsubsection
\subsubsection{Step 3 (Information Transmission)} \label{SubSec:Transmission}
\scaleSubsubsectionBelow
%
While the pump constantly transports the \ac{GFPD} solution in the loop, information can be modulated from outside the tube into the state of the \acp{GFPD} by an \ac{LED}-based \ac{TX}. Later, the \ac{RX} reads out the state of the \acp{GFPD} via fluorescence triggered by an \ac{LED}. Additionally, an \ac{LED}-based \ac{EX} can be used to erase the modulated information. Note that the average lap time that one \ac{GFPD} molecule needs to circulate one time through the testbed is $T_\mathrm{L} = \frac{V_\mathrm{PF} + \volTube}{Q} = 41 \,\si{\second}$ assuming, on average, no additional delay is induced by the \textit{Falcon}$\,\textsuperscript{\tiny\textregistered}$. The components that facilitate information transmission, i.e., the \ac{TX}, the \ac{TX}-\ac{RX} link, which we refer to as channel, the \ac{RX}, and the \ac{EX} are described next.
%
\scaleSubsection
\subsection{Information Transmission Components} \label{SubSec:InfComp}
\scaleSubsectionBelow
\scaleSubsubsection
\subsubsection{The Transmitter - Optical-to-Chemical-Signal Conversion}\label{subsubsec:tx}
\scaleSubsubsectionBelow
%
The \ac{TX} is used to modulate information via media modulation into the state of the \acp{GFPD}. Since the ON state is the idle state of the \acp{GFPD}, the \ac{TX} switches \ac{GFPD} to the OFF state to convey information.
The \ac{TX} consists of an \ac{LED} array with $12$ equally spaced \acp{LED} (\textit{PowerBar V3, Lumitronix}), cf. \Figure{fig:led_array}, which can emit light at a wavelength of $\lambda_\mathrm{OFF} = 405 \, \si{\nano\meter}$. The array is attached to an aluminum circuit board with a total length of $L_\mathrm{TX} = 29 \,\si{\cm}$ that is connected to a passive cooling element to increase the service life of the \acp{LED}. In addition, the \acp{LED} are individually wired for maximum flexibility. This design enables control of the irradiation intensity of each \ac{LED} by a microcontroller (\textit{ESP32-S2, Espressif Systems}) during the experiments. In this work, however, all \acp{LED} are switched on and off simultaneously.
The \ac{TX} is placed on top of the tubing, cf. \Figure{fig:total_overview}.

To encode information, the \ac{TX} emits light for a duration of $\Ti$ and hereby controls the ON state concentration of the irradiated \ac{GFPD} molecules in the tubing. The irradiation intensity is adapted according to the symbol to be transmitted (cf. \Section{Sec:Modulation} for mathematical details). The switching of \ac{GFPD} is an energy-dependent probabilistic process \cite{brakemann2011reversibly, Brand2022MediaModulation}. In our testbed, both the irradiation duration $\Ti$ and the irradiation intensity jointly determine the switching probability. This is because both determine the total number of photons released during $\Ti$, which in turn determines the probability that a photon will hit a \ac{GFPD} molecule and thus trigger a switch.
$\Ti$ is followed by a guard interval of duration $\Tg$, during which the \ac{TX} is always inactive. This results in a symbol duration of $\Ts = \Ti + \Tg$.
%
\scaleSubsubsection
\subsubsection{The Channel - Pump-Driven Advection}\label{subsubsec:channel}
\scaleSubsubsectionBelow
In the testbed, information is transmitted over the communication channel of length $d_{\mathrm{TX},\mathrm{RX}}$, which we define to be the tubing between the end of the \ac{TX} and the beginning of the \ac{RX} (framed in red in \Figures{fig:overview}{fig:lab_overview}). In this work, two channel lengths are investigated, $d_{\mathrm{TX},\mathrm{RX}} = 6 \, \si{\centi\meter}$ and $d_{\mathrm{TX},\mathrm{RX}} = 35 \, \si{\centi\meter}$. Note that our testbed design allows for easy modification of $d_{\mathrm{TX},\mathrm{RX}}$ by moving the \ac{TX}, which is loosely placed on top of the tube. In principle, $d_{\mathrm{TX},\mathrm{RX}}$ could even be varied dynamically during experiments. In this work, however, the \ac{TX} is not moved during experiments.

During the propagation from the \ac{TX} to the \ac{RX}, modulated \ac{GFPD} molecules may spontaneously switch back to their ON state due to thermal relaxation.
In our experiments, the effect on the received signal is small as $T_{1/2} = 600 \,\si{\second}$ is large compared to the average time required to propagate the distance $d_{\mathrm{TX},\mathrm{RX}}$ ($0.4 \, \si{\second}$ and $2.5 \,\si{\second}$ for $d_{\mathrm{TX}, \mathrm{RX}} = 6 \, \si{\centi\meter}$ and $d_{\mathrm{TX},\mathrm{RX}} = 35 \, \si{\centi\meter}$, respectively, assuming $\veff = 0.139 \,\si{\meter \per\second}$). However, for future experiments with a longer channel, thermal relaxation may become more relevant.
%
\scaleSubsubsection
\subsubsection{The Receiver - Fluorescence-Based Readout}\label{subsubsec:rx}
\scaleSubsubsectionBelow
The \ac{RX} is used to read out the current states of the \acp{GFPD}. 
As \ac{RX}, a fluorescence flow cell (\textit{FIA-SMA-FL-ULT, Ocean Optics}), equipped with an \ac{LED} that can emit light at a wavelength of $\lambda_\mathrm{T} = 500 \, \si{\nm}$, is placed at a distance of $d_{\mathrm{TX},\mathrm{RX}}$ from the end of the \ac{TX}. The \ac{LED} is used to trigger the fluorescence of the \ac{GFPD}. The light emitted by the \acp{GFPD} upon fluorescence, which has a wavelength of $\lambda_\mathrm{E} = 529 \, \si{\nm}$, is guided by an optical fiber and measured by a compact spectrometer (\textit{CCS 100M, Thorlabs}). The measured signal is recorded by the \textit{ThorSpectra software}. 
%
\scaleSubsubsection
\subsubsection{The Eraser - Deleting the Modulated Information}\label{subsubsec:ex}
\scaleSubsubsectionBelow
The \ac{EX} can be used to reset the state of the \acp{GFPD} to the ON state. 
The \ac{EX} consists of an \ac{LED} array with $12$ equally spaced \acp{LED} (\textit{PowerBar V3, Lumitronix}), which can emit light at a wavelength of $\lambda_\mathrm{ON} = 365 \, \si{\nano\meter}$. Apart from the type of \acp{LED} used, the \ac{EX} is constructed and wired identical to the \ac{TX}, i.e., the hardware design and dimensions of the \ac{EX} are identical to those of the \ac{TX}. The \ac{EX} is placed upstream next to the \ac{TX}.

Compared to the \ac{TX}, the use of the \ac{EX} is optional. Consequently, two scenarios are considered: one with and one without the use of the \ac{EX}. In the case without \ac{EX}, \ac{GFPD} can only switch back from the OFF state to the ON state by thermal relaxation. If we use the \ac{EX}, the \ac{EX} is turned on at the beginning of the experiment, i.e., the \ac{EX} is continuously active. Hence, the \ac{EX} enables the active reset of the \acp{GFPD} from the OFF state back to the ON state.  In consequence, with the \ac{EX} turned on, the number of ON state molecules available at the \ac{TX} is increased. In addition, the \ac{EX} effectively reduces memory effects in the system, i.e., it can be used to reduce the forms of \ac{ISI} specific to closed-loop \ac{MC} systems. We will discuss these forms of \ac{ISI} in detail in \Section{SubSec:ISI}. As a side effect, the \ac{EX} also contributes to photobleaching, cf. \Section{par:photobleaching}. Thus, the use of the \ac{EX} causes a tradeoff between \ac{ISI} mitigation and additional photobleaching.