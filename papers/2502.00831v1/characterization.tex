In this section, we discuss the physical effects observed in the testbed, such as the involved fluid mechanics and different forms of \ac{ISI} present including those that occur exclusively in closed-loop systems.
%
\scaleSubsection
\subsection{Signaling Molecule Propagation}\label{subsec:propagation}
\scaleSubsectionBelow
%
In our testbed, the propagation of the \ac{GFPD} signaling molecules is mainly affected by advection, facilitated by the bulk background flow of the buffer liquid. In the following, we analyze the flow characteristics and its parameters in the testbed.

When a fluid of kinematic viscosity $\nu$ flows with average flow speed $v_{\mathrm{eff}}$ through a pipe of radius $r_\mathrm{T}$, either laminar flow or turbulent flow occurs. 
Which type of flow is dominant can be predicted by the Reynolds number $\text{Re} = \frac{2 r_\mathrm{T} \veff}{\nu}$ \cite[p. 14]{darby2017chemical}. For increasing Reynolds numbers, the transition from the laminar to the turbulent flow regime occurs around $\text{Re} \approx 2300$ \cite[p. 12]{schlichting2016boundary}. The viscosity of the buffer solution in our testbed equals the viscosity of water \cite{hink2000structural}, i.e., $\nu = 1.037\, \times 10^{-6} \,\si{ \meter \squared \per \second}$ (at $21\, \si{\celsius}$). 
Here, for the parameter regime in which the testbed is operated, we obtain $\text{Re} = 167$, i.e., laminar flow is dominant. This value of $\text{Re}$ corresponds to a medium-sized artery, where the Reynolds number is typically between 100 and 1000 \cite{caro2012mechanics}.
Furthermore, we determine the relative influence of diffusion on the fluid transport, which can be characterized by the dimensionless Péclet number $\text{Pe} = \frac{r_\mathrm{T} \veff}{\D}$ \cite[eq. (4.44)]{tabeling2023introduction}. Here, utilizing the diffusion coefficient $ \D = 1\, \times 10^{-10} \,\si{\meter\squared \per \second}$ of \ac{GFPD} \cite{Junghans2016DiffusionGFPD}, $\text{Pe} = 1.11\, \times 10^{6} \gg 1$ follows. Therefore, flow dominates over diffusion in the system, and the latter can be neglected as a result.
%
\scaleSubsection
\subsection{ISI Caused by Closed-Loop Operation}\label{SubSec:ISI}
\scaleSubsectionBelow
%
The closed-loop character of our testbed gives rise to four distinct forms of \ac{ISI}: \textit{channel \ac{ISI}}, \textit{inter-loop \ac{ISI}}, \textit{offset \ac{ISI}}, and \textit{permanent \ac{ISI}}. The latter three forms can only be observed in closed-loop systems. 
All forms of \ac{ISI} are discussed in detail in the following. 
%
\scaleSubsubsection
\subsubsection{Channel ISI}\label{subsubsec:channel_isi}
\scaleSubsubsectionBelow
Within the channel, the propagation of OFF state \ac{GFPD} molecules from \ac{TX} to \ac{RX} may result in the overlap of consecutive transmitted symbols, leading to \ac{ISI}.
The existence of channel \ac{ISI} has been demonstrated with existing \ac{MC} testbeds, e.g., \cite{grebenstein2018biological, wang2020understanding}.
Channel \ac{ISI} can be mitigated by introducing a guard interval $T_\mathrm{G}$, during which the \ac{TX} is inactive, where the effectiveness of \ac{ISI} mitigation is contingent on the duration of the guard interval. However, increasing $T_\mathrm{G}$ also results in a reduction of the achievable data rate.
\scaleSubsubsection
\subsubsection{Inter-Loop ISI}\label{subsubsec:loop_isi}
\scaleSubsubsectionBelow
Due to the closed-loop design of the testbed, inter-loop \ac{ISI} occurs, as the OFF state \ac{GFPD} molecules remain in the system and reenter the \ac{RX} after completing one loop. Hence, these molecules interfere not only with neighboring symbols (as is the case for channel \ac{ISI}) but also with symbols transmitted much later in time. In particular, OFF state \ac{GFPD} molecules, which are switched
to the OFF state by the \ac{TX}, may travel several times through the loop before they spontaneously switch back to the ON state. Thus, these signaling molecules are likely to affect the received signal multiple times, while the intensity, timing, and number of recurrences of inter-loop \ac{ISI} depend on the length of the loop and the half-life $T_{1/2}$ of \ac{GFPD}. The intensity of inter-loop \ac{ISI} can be reduced by using the \ac{EX}.
%
\scaleSubsubsection
\subsubsection{Offset ISI}\label{subsubsec:offset_isi}
\scaleSubsubsectionBelow
As \ac{GFPD} molecules in the OFF state disperse in the system over multiple loops, the inter-loop \ac{ISI}, characterized by distinct drops in fluorescence intensity, undergoes a gradual transformation to a temporal offset in the observed fluorescence intensity, which we refer to as offset \ac{ISI}. The intensity of offset \ac{ISI} depends on the irradiation intensity used at the \ac{TX}, symbol duration $\Ti$, and half-life $T_{1/2}$ of \ac{GFPD}. The intensity of offset \ac{ISI} can be reduced by using the \ac{EX}.
%
\scaleSubsubsection
\subsubsection{Permanent ISI}\label{subsubsec:permanent_isi}
\scaleSubsubsectionBelow
After prolonged exposure to light, \ac{GFPD} molecules may undergo an irreversible degradation as a consequence of photobleaching effects (cf. \Section{par:photobleaching}). While the decrease in fluorescence intensity resulting from offset \ac{ISI} is reversible, photobleaching is not. Thus, photobleaching causes an irreversible reduction of the number of functional \ac{GFPD} molecules available for modulation. Consequently, this leads to an overall decrease in the received fluorescence signal intensity over time. As a result, we observe a deterioration of the communication performance.
Since \ac{GFPD} molecules are irradiated by the \ac{EX}, the \ac{RX}, and the \ac{TX} in the testbed, all three light sources contribute to photobleaching. Therefore, we refer to the fractions caused by the \ac{TX}, \ac{EX}, and \ac{RX} as permanent \ac{ISI}, \ac{EX} bleaching, and measurement bleaching, respectively. Note that only photobleaching caused by the \ac{TX} is interpreted as \ac{ISI}, since it can be varied by varying the transmitted signal, whereas the photobleaching caused by \ac{EX} and \ac{RX} is independent of the transmitted sequence.