\documentclass[journal, 12pt, onecolumn,draftclsnofoot, hidelinks, xcolor=dvipsnames]{IEEEtran}

\pdfoutput=1
\IEEEoverridecommandlockouts

\usepackage{amsmath}
\usepackage[english]{babel}
\usepackage{cite}
\usepackage[shortlabels,inline]{enumitem}
\usepackage{graphicx}
\usepackage{hyperref}
\usepackage{bookmark}
\usepackage{ifthen}
\usepackage[utf8]{inputenc}
\usepackage[locale=US,group-separator={,}]{siunitx}
\usepackage[dvipsnames]{xcolor}
\usepackage[T1]{fontenc}
\usepackage[subnum]{cases}
\usepackage[toc,symbols]{glossaries}
\usepackage{bm}
\usepackage{array}
\usepackage{tabularray}
\usepackage{acro}
\usepackage{microtype}
\usepackage{pifont}


%Maike
\usepackage[utopia, cal=cmcal]{mathdesign} % Teena: Option cal=cmcal
\DeclareSIUnit{\bitsperminute}{\text{bit} \, \text{min}^{-1}}
\newcommand{\xmark}{\ding{55}}
\newcommand{\cmark}{\ding{51}}%

%Louis
\usepackage{subcaption}

\captionsetup[subfigure]{font={small}, skip=1pt, singlelinecheck=false}
\allowdisplaybreaks

\newtheorem{corollary}{Corollary}
\newtheorem{remark}{Remark}
\newtheorem{theo}{Theorem}
\newtheorem{lem}{Lemma}
\newtheorem{defin}{Definition}
\newtheorem{prop}{Proposition}
\newtheorem{assump}{Assumption}
\newtheorem{cond}{Condition}

% macros for referencing figures, tables, equations and so on

\newcommand{\Figure}[1]{Fig.~\ref{#1}}
\newcommand{\Figures}[2]{Figs.~\ref{#1} and~\ref{#2}}
\newcommand{\FigureList}[2]{Figs.~\ref{#1} --~\ref{#2}}

\newcommand{\SubFigure}[2]{Fig.~\ref{#1}~\protect\subref{#2}}
\newcommand{\SubFigures}[3]{Figs.~\ref{#1}~\protect\subref{#2} and \protect\subref{#3}}
\newcommand{\SubFigureRange}[3]{Figure~\ref{#1}(#2)--(#3)}

\newcommand{\FigureStart}[1]{Figure~\ref{#1}}
\newcommand{\SubFigureStart}[2]{Figure~\ref{#1}(#2)}
\newcommand{\SubFiguresStart}[3]{Figures~\ref{#1}(#2) and \ref{#1}(#3)}

\newcommand{\Equation}[1]{\eqref{#1}}
\newcommand{\Equations}[2]{\eqref{#1} and~\eqref{#2}}
\newcommand{\EquationsList}[2]{\eqref{#1} --~\eqref{#2}}
\newcommand{\Table}[1]{Table~\ref{#1}}
\newcommand{\Tables}[2]{Tables~\ref{#1}~and~\ref{#2}}
\newcommand{\Section}[1]{Section~\ref{#1}}
\newcommand{\Subsection}[1]{Subsection~\ref{#1}}
\newcommand{\Subsubsection}[1]{Subsubsection~\ref{#1}}
\newcommand{\Sections}[2]{Sections~\ref{#1}~and~\ref{#2}}
\newcommand{\SectionsThree}[3]{Sections~\ref{#1}~,~\ref{#2},~and~\ref{#3}}
\newcommand{\Sectionss}[2]{Sections~\ref{#1} --\ref{#2}}
\newcommand{\Chapter}[1]{Chapter~\ref{#1}}
\newcommand{\Chapters}[2]{Chapters~\ref{#1}~and~\ref{#2}}
\newcommand{\Appendix}[1]{Appendix~\ref{#1}}
\newcommand{\Algorithm}[1]{Algorithm~\ref{#1}}
\newcommand{\Proposition}[1]{Proposition~\ref{#1}}
\newcommand{\Assumption}[1]{\textit{Assumption~\ref{#1}}}

%%%%%%%%%%%%%%
% Operations %
%%%%%%%%%%%%%%
\renewcommand{\vec}[1]{\mathbf{#1}}

%%%%%%%%%%%%%%%%%%%%%
% System Parameters %
%%%%%%%%%%%%%%%%%%%%%
\newcommand{\radius}[0]{r_{\mathrm{T}}} % tube radius
\newcommand{\con}[0]{C_{\mathrm{GFPD}}} % concentration of GFPD
\newcommand{\volTube}[0]{V_{\mathrm{T}}} % tube volume
\newcommand{\restime}[0]{V_{\mathrm{R}}} % Residence time

\newcommand{\integrationTime}[0]{T_{\mathrm{RX,Int}}}

%%%%%%%%%%%%%
% Variables %
%%%%%%%%%%%%%
\newcommand{\cir}[1]{h(#1)}
\newcommand{\systemResponse}[1]{g(#1)}
\newcommand{\systemSpecificResponse}[1]{f(#1)}
\newcommand{\Nobs}[0]{\bar{N}_{\mathrm{obs}}(d, t)} % avg molecule number after instantaneous release
\newcommand{\pswitch}[0]{p_{\mathrm{S}}^{\mathrm{TX}}} % switching probability at the TX
\newcommand{\Tg}[0]{T_{\mathrm{G}}} % guard interval duration
\newcommand{\volRX}[0]{V_{\mathrm{RX}}} % volume of RX
\newcommand{\veff}[0]{v_{\mathrm{eff}}} % effective velocity
\newcommand{\D}[0]{D} % diffusion coefficient
\newcommand{\rect}[2]{\mathrm{rect}_{#1}(#2)} % rect signal
\newcommand{\Tleff}{L_{\mathrm{T, eff}}}% effective tube length
\newcommand{\Tres}{\bar{T}_{\mathrm{res}}}% average reservoir time
\newcommand{\scalingFactor}{\bar{\rho}_{\mathrm{res}}}% scaling factor of what portion of switched molecules re-enters the system before having switched back to the equilibrium state
\newcommand{\bitest}[1]{\hat{b}[#1]} % estimated bit at time #1
\newcommand{\symest}[1]{\hat{s}_{m[#1]}} % estimated bit at time #1
\newcommand{\sym}[1]{s_{m[#1]}} % symbol at time #1
\newcommand{\bit}[1]{b[#1]} % bit at time #1
\newcommand{\recDiff}[1]{d'[#1]} % received signal at time #1
\newcommand{\coh}[0]{n_{\mathrm{c}}} % coherence time
\newcommand{\nTX}[0]{N_{1}} % num of switched molecules
\newcommand{\nRX}[0]{N_{0}} % typical number of molecules at the RX
\newcommand{\volTX}[0]{V_{\mathrm{TX}}} % volume of TX
\newcommand{\disTX}[0]{d_{\mathrm{TX,RX}}}
\newcommand{\wei}[0]{m_{\mathrm{GFPD}}} % weight of GFPD

\newcommand{\Nobstx}[0]{\bar{N}^{\mathrm{TX}}_{\mathrm{obs}}(d, t)} % avg number of switched off molecules at RX
\newcommand{\switching}[0]{a_{\mathrm{D}}} % switching function
\newcommand{\switchingc}[0]{\kappa} % switching constant

\newcommand{\minED}[0]{d_{\mathrm{min}}}
\newcommand{\minSED}[0]{d^2_{\mathrm{min}}}


% Detection scheme
\newcommand{\threshR}[0]{\xi_{\mathrm{R}}} % regular threshold
\newcommand{\threshD}[0]{\xi_{\mathrm{D}}} % differential based threshold
\newcommand{\threshA}[0]{\xi_{\mathrm{A}}} % adaptive threshold
\newcommand{\nSymb}[0]{N_{\mathrm{Sym}}} % number of symbols
\newcommand{\nBits}[0]{N_{\mathrm{Bit}}} % number of bits
\newcommand{\idxPilots}[0]{\mathcal{K}^{\mathrm{P}}} % set of indices belonging to pilot symbols
\newcommand{\idxSym}[1]{\mathcal{K}^{\mathrm{P}}_{#1}} % set of bit #1s in pilot sequence

%%%%%%%%%
% Units %
%%%%%%%%%
\newcommand{\s}[0]{\text{s}} % seconds

%%%%%%%%%%%%%%%%%%%%%%%%%%%%%%%%%%%%%%%%%%%%%%%%%%%%%%%%%%%%%%%%%%%%%%%%%%%%%%%%%%%%%%%%%%%%%%%%%
%%%%%%%%%%%%%
% Operators %
%%%%%%%%%%%%%
\newcommand{\transpose}[1]{#1^{\mathrm{T}}}
\DeclareMathOperator*{\argmax}{\mathrm{arg\,max}}
\DeclareMathOperator*{\argmin}{\mathrm{arg\,min}}

%%%%%%%%%%%%%
% Indices %
%%%%%%%%%%%%%

%General
\newcommand{\n}[0]{n} % time index
\newcommand{\tn}[0]{t_n} % time index
\newcommand{\altTimeIdx}{\kappa} % alternative time index
\newcommand{\symIdx}[0]{k} % symbol index

%Transmission start detection
\newcommand{\trainIdx}[0]{\tau}

%Detection
\newcommand{\detecThreshIdx}[0]{j}
\newcommand{\detecThreshSetIdx}[0]{l}


%%%%%%%%%%%%%%%%%%%
% Variables %
%%%%%%%%%%%%%%%%%%%

%General
\newcommand{\symVar}[0]{i}
\newcommand{\Ts}[0]{T_{\mathrm{S}}} % symbol duration
\newcommand{\Ti}[0]{T_{\mathrm{I}}} % impulse duration
\newcommand{\modOrder}[0]{M} % order of modulation
\newcommand{\SIR}[0]{s}
\newcommand{\SR}[0]{s_{\mathrm{r}}}
\newcommand{\avgSR}[0]{\bar{s}_{\mathrm{r}}}
\newcommand{\recSig}[0]{r}
\newcommand{\sampleInt}[0]{\Delta t}


%Modulation
\newcommand{\ratioInten}[0]{\rho}
\newcommand{\txInten}[0]{I^{\mathrm{TX}}} % TX intensity
\newcommand{\maxTxInten}[0]{I^{\mathrm{TX}}_{\mathrm{max}}} % max TX intensity

%Transmission start detection
\newcommand{\trainLength}[0]{N_\mathrm{T}}
\newcommand{\trainSet}[0]{\mathcal{K}}
\newcommand{\trainMean}[0] {\hat{\mu}}
\newcommand{\trainVariance}[0]{\hat{\sigma}^2}
\newcommand{\trainThresh}[0]{\xi_\mathrm{T}}
\newcommand{\pFA}[0]{p_\mathrm{FA}} % False probability
\newcommand{\tts}[0]{t_\mathrm{TS}} % Time of detected transmission start

%Synchronization
\newcommand{\ts}[0]{t_\mathrm{s}}
\newcommand{\tsEst}[0]{\hat{t}_\mathrm{s}}
\newcommand{\tsInit}[0]{\Tilde{t}_\mathrm{s}}
\newcommand{\avgSIR}[0]{\bar{\SIR}}
\newcommand{\searchInt}[0]{\mathcal{T}^{\mathrm{search}}}
\newcommand{\searchRadius}[0]{r}
\newcommand{\hypothesisSet}[0]{\mathcal{H}}
\newcommand{\syncMetric}[0]{\Lambda}
\newcommand{\filterLength}[0]{\mathrm{N}}

\newcommand{\corrMarker}[0]{\mathrm{C}}
\newcommand{\diffMarker}[0]{\mathrm{D}}
\newcommand{\blindMarker}[0]{\mathrm{B}}
\newcommand{\corr}[0]{\mathrm{corr}}
\newcommand{\dcorr}[0]{\mathrm{dcorr}}

\newcommand{\modRecSig}[0]{\recSig'}
\newcommand{\filter}[0]{g}
\newcommand{\corrFilter}[0]{\filter_\corrMarker}
\newcommand{\vecModRecSig}[0]{\vec{\recSig}'}
\newcommand{\vecCorrFilter}[0]{\vec{\filter}_\corrMarker}
\newcommand{\corrMetric}[0]{\syncMetric^\corr}
\newcommand{\corrFunc}[0]{\varphi}
\newcommand{\tsCorr}[0]{\tsEst^\corr}

\newcommand{\diffRecSig}[0]{\recSig_{\diffMarker}}
\newcommand{\vecDiffRecSig}[0]{\vec{\recSig}_{\diffMarker}}
\newcommand{\diffCorrFilter}[0]{\filter_\diffMarker}
\newcommand{\vecDiffCorrFilter}[0]{\vec{\filter}_\diffMarker}
\newcommand{\dcorrMetric}[0]{\syncMetric^\dcorr}
\newcommand{\dcorrFunc}[0]{\varphi_\diffMarker}
\newcommand{\tsDcorr}[0]{\tsEst^\dcorr}

%Blind filters
\newcommand{\blindCorrFilter}[0]{g_\corrMarker^{\blindMarker}}
\newcommand{\blindDiffCorrFilter}[0]{\filter_\diffMarker^\blindMarker}
\newcommand{\vecBlindCorrFilter}[0]{\vecCorrFilter^\blindMarker}
\newcommand{\vecBlindDiffCorrFilter}[0]{\vecDiffCorrFilter^\blindMarker}

%Detection
\newcommand{\detecSample}[0]{d}
\newcommand{\estSym}[0]{\hat{{\symVar}}}
\newcommand{\detecThresh}[0]{\xi}
\newcommand{\detecThreshSet}[0]{\Xi}
\newcommand{\nPilots}[0]{P} % number of pilot symbols
\newcommand{\nCoherence}[0]{F} % number of symbols the threshold stays fixed
\newcommand{\nWindow}[0]{W} % number of symbols used for reevaluation of thresholds
\newcommand{\nSkip}[0]{\chi} % number of symbols used for reevaluation of thresholds


\newcommand{\symbolSampleSet}[0]{\mathcal{S}}
\newcommand{\avgSymbolSampleSet}[0]{\bar{S}}
\newcommand{\threshDiff}[0]{\delta}
\newcommand{\avgThreshDiff}[0]{\Delta}

\newcommand{\AMED}[0]{\mathrm{AMED}}

\begin{acronym}
\acro{gan}[GANs]{Generative Adversarial Networks}
\acro{rl}[RL]{Reinforcement Learning}
\acro{pae}[PAE]{Periodic Autoencoder}
\acro{fld}[FLD]{Fourier Latent Dynamics}
\acro{ppo}[PPO]{Proximal Policy Optimization}
\acro{fft}[FFT]{Fast Fourier Transform}
\acro{pca}[PCA]{Principal Component Analysis}
\acro{dfm}[DFM]{Deep Fourier Mimic}
\acro{dof}[DoF]{Degrees of Freedom}
\acro{mlp}[MLPs]{Multi-Layer Perceptrons}
\end{acronym}



\makeatletter
\long\def\@makecaption#1#2{\ifx\@captype\@IEEEtablestring%
    \footnotesize\begin{center}{\normalfont\footnotesize #1}\\
        {\normalfont\footnotesize\scshape #2}\end{center}%
    \@IEEEtablecaptionsepspace
    \else
    \@IEEEfigurecaptionsepspace
    \setbox\@tempboxa\hbox{\normalfont\footnotesize {#1.}~~ #2}%
    \ifdim \wd\@tempboxa >\hsize%
    \setbox\@tempboxa\hbox{\normalfont\footnotesize {#1.}~~ }%
    \parbox[t]{\hsize}{\normalfont\footnotesize \noindent\unhbox\@tempboxa#2}%
    \else
    \hbox to\hsize{\normalfont\footnotesize\hfil\box\@tempboxa\hfil}\fi\fi}
\makeatother

\renewcommand{\baselinestretch}{1.25}
\setlength{\skip\footins}{7pt}
\addtolength{\topmargin}{+0.07cm}
\addtolength{\textwidth}{0.22cm}
\addtolength{\textheight}{0.22cm}

% Increase row spacing in tables
\renewcommand{\arraystretch}{1}
\captionsetup{belowskip=0pt,aboveskip=1pt}

\newcommand{\scaleSection}{\vspace*{-0.35cm}}
\newcommand{\scaleSubsection}{\vspace*{-0.52cm}}
\newcommand{\scaleSubsubsection}{\vspace*{-0.01cm}}
\newcommand{\scaleSectionBelow}{\vspace*{-0.1cm}}
\newcommand{\scaleSubsectionBelow}{\vspace*{-0.02cm}}
\newcommand{\scaleSubsubsectionBelow}{\vspace*{-0.01cm}}
\newcommand{\scaleAlign}{\vspace*{-0.1cm}}


\newcommand\todo[1]{\textcolor{red}{#1}}
%\newcommand\todo[1]{\textcolor{red}{}}

\begin{document}
%\tableofcontents
\bstctlcite{IEEEexample:BSTcontrol}
\vspace{-0.4cm}
\title{Closed-Loop Long-Term Experimental Molecular Communication System}
\author{\vspace{-0.3cm}
\IEEEauthorblockN{Maike Scherer$^\star$, Lukas Brand$^\star$, Louis Wolf, Teena tom Dieck, Maximilian Schäfer, Sebastian Lotter, Andreas Burkovski, Heinrich Sticht, Robert Schober, and Kathrin Castiglione\\
\thanks{$^\star$ Co-first authors.\\This paper was presented in part at the IEEE International Conference on Communications, 2024 \cite{brand2024closed}.}}\IEEEauthorblockA{\small Friedrich-Alexander-Universit\"at Erlangen-N\"urnberg, Erlangen, Germany}}
%
\maketitle
\vspace{-1.9cm}
\begin{abstract}
\vspace*{-0.3cm}
\begin{abstract}
Retrieval-Augmented Generation (RAG) is often used with Large Language Models (LLMs) to infuse domain knowledge or user-specific information. In RAG, given a user query, a retriever extracts chunks of relevant text from a knowledge base. These chunks are sent to an LLM as part of the input prompt. Typically, any given chunk is repeatedly retrieved across user questions. However, currently, for every question, attention-layers in LLMs fully compute the key values (KVs) repeatedly for the input chunks, as state-of-the-art methods cannot reuse KV-caches when chunks appear at arbitrary locations with arbitrary contexts. Naive reuse leads to output quality degradation.  This leads to potentially redundant computations on expensive GPUs and increases latency. In this work, we propose \sys, a system for managing and reusing precomputed KVs corresponding to the text chunks (we call \textit{chunk-caches}) in RAG-based systems. We present how to identify \hl{\textit{chunk-caches} that are reusable}, how to efficiently perform a small fraction of recomputation to \textit{fix} the cache to maintain output quality, and how to efficiently store and evict \textit{chunk-caches} in the hardware for maximizing reuse while masking any overheads. With real production workloads as well as synthetic datasets, we show that \sys reduces redundant computation by \textbf{51\%} over SOTA prefix-caching and \textbf{75\%} over full recomputation.
\hl{Additionally, with continuous batching on a real production workload, we get a \textbf{1.6$\times$} speedup in throughput and a \textbf{2$\times$} reduction in end-to-end response latency over prefix-caching while maintaining quality, for both the \llama-3-8B and \llama-3-70B models. 
}
\end{abstract}





\end{abstract}
\setlength{\belowdisplayskip}{2pt}
\setlength{\belowdisplayshortskip}{2pt}
\vspace*{-0.4cm}
\acresetall
%
\scaleSection
\section{Introduction}\label{sec:intro}
\scaleSectionBelow
\documentclass[../main.tex]{subfiles}
\graphicspath{{../images/}}
\makeatletter
\def\input@path{{../images/}}
\makeatother
\begin{document}
\section{Introduction}
\begin{figure}
\centering
\begin{tikzpicture}
\node[inner sep=0pt] (ws) at (0, 0) {
\includegraphics[height=.4\textwidth, trim={10cm 0 10cm 0},clip]{world_space.png}};
\node[inner sep=0pt] (cs) at (6,0) {\includegraphics[height=.4\textwidth, trim={10cm 1cm 10cm 4cm},clip]{conf_space.png}};
\end{tikzpicture}
\vspace{-5pt}
\label{fig:pbrm_intro}
\caption{\textbf{Left}: Shows world space obstacles as grey spheres. Robots start and goal configuration is colored red and green, respectively. Configurations along the computed path are colored transparent blue. \textbf{Right:} Mapped world space scenario to configuration space. Obstacle region is the grey mesh. Red spheres are collision-free regions computed by the neural SCDF. The optimized shortest path in the convex corridor is the blue curve.}
\vspace{-25pt}
\end{figure}
Motion planning is the problem of finding a collision-free trajectory that connects a given start and goal configuration. The planning takes place in the configuration space of the robot. For single body robots, like mobile robots or drones, the configuration space and the world space are usually the same. This simplifies the planning, since explicit obstacle representations are available which enables geometrical tools like separating hyperplanes, smallest distance to obstacles etc., to be used when designing motion planning algorithms. For multi-body robots like manipulators, the situation is completely different. The world space obstacles are usually mapped to non-convex regions, and to make the problem even harder, the mapping is usually not known. Forming explicit representations of the obstacle region in the configuration space is usually too expensive or intractable. Despite all of this, sampling based planners are used with great success, which mainly is due to their use of implicit representations of the obstacle region. The basic idea is to construct a graph in the configuration space that covers and connects the collision-free region. From this graph, a path can be extracted that connects a given start and goal configuration. The approach is computationally expensive, since the graph is constructed with the smallest geometrical building block available, points, which represents a collision-check. Furthermore, the extracted paths from the graph are non-smooth and jagged due to the stochastic nature of the approach. This adds an additional post-processing step to the process, where the paths are shortcutted and smoothened, before the path can be used for tracking. Clearly a lot of time is invested to form this graph and produce smooth paths. Thus, if the obstacles start to move, then all of this work is done in no use, since all points that make up this graph need to be re-verified, which is simply too time consuming to be done in real time.
\\\\
In this work, we want to address the existing drawbacks of the sampling based planners. Our main contribution is an improved motion planner where each vertex in the graph covers a collision-free region in the form of a sphere instead of a point and where the edges are formed with neighboring intersecting spheres. This representation has the advantage of instead of returning piecewise linear paths, returning a sequence of overlapping spheres, i.e. a convex corridor, that connects a given start and goal configuration, illustrated in Figure \ref{fig:pbrm_intro}. This convex corridor allows us to use convex optimization to produce smooth trajectories, instead of computationally expensive post-processing methods. The representation further allows us to estimate the coverage of the collision-free space, which gives us awareness and feedback in the offline roadmap construction phase. Finally, our representation is simple to adapt to moving obstacles, simply requery for the new radii and recheck for intersections. 
\\\\
The spherical collision-free regions are formed using a signed distance function (SDF), which is a function that returns the smallest distance from an arbitrary point to the boundary of an obstacle. As the name implies, the distance is signed, thus if the point is inside the obstacle it is negative otherwise positive. If the distance is positive, a sphere with radius equal to the distance is guaranteed to cover a collision-free region. Using an SDF in motion planning is not new, but what is novel about our approach is that we express the distance in the configuration space instead of the world space and by doing so allows us to form these convex collision-free regions. We refer to the resulting SDF as a signed configuration distance function (SCDF). Computing an SCDF analytically is non-trivial, our approach is therefore to parameterize the SCDF with a deep neural network and learn the mapping by supervised learning. Our resulting neural SCDF can compute distances for different parameter values of obstacle shapes and we also show how multiple distances can be combined, thus making our approach flexible.
\section{Related work}
Motion planning algorithms can roughly be divided into three families, grid-based, sampling based and optimization based methods. Grid-based methods (GBM) discretize the planning space from which a graph is then compiled. A standard search method is A$^\star$ \citep{a_star}, which is classified as an \textit{informed} search method, since it employs a heuristic function to speed up the search. A$^\star$ guarantees to return an optimal path at the level of discretization used. GBMs usually discretize the planning space by a regular lattice and this limits the GBMs to problems with low dimensionality due to the curse of dimensionality. Thus, GBMs are usually limited to single-body robots where the degrees of freedom (DOF) are low. To overcome the inherent scaling problem with the GBMs, stochastic methods are usually used for multi-body robots. These methods are termed as sampling-based methods (SBM) and core members within this family are the rapidly-exploring random trees (RRT) \citep{rrt} and the probabilistic roadmap (PRM) \citep{prm}. RRT grows a tree from the start configuration and explores the collision-free region in a rapid way until it is able to connect to the goal region. RRT is usually improved by bi-directional planning \citep{rrt_connect}, i.e. an additional tree is grown from the goal configuration and the trees are tested for connection after any tree has been expanded. RRT is a single-query method, thus it searches for a path from scratch each time it is queried. Contrary to this, PRM is a multi-query method, which solves for multiple queries without starting from scratch. PRM does this by creating a roadmap (graph) that covers the collision-free space as an offline step. The graph is then used to solve for multiple queries. PRMs are used in cases where the environment does not change since the extra offline step is too computationally costly and needs to be re-done if the environment is changed. In our work, we address this inherent issue by using a different roadmap representation. Our vertices in the graph cover a collision-free region in the form of spheres and we form the edges by checking for intersecting spheres. If something in the environment changes, we recompute the spheres radii and recheck the intersections, without relying on collision detection. We use a trained neural network to compute the sphere radius, therefore querying for the radius can be done fast, hence our representation enables the PRM for dynamic environments.
\\\\
In the recent decades, optimization based methods (OBM) \citep{chomp, schulman, itomp, stomp} have been introduced as an alternative to SBM for multi-body robots. Like the SBM, the OBMs scale well to higher dimensional problems and produce smoother motion. It is common to use a SDF in the optimization since it is a smooth function, thus enabling gradient-based methods. However, the standard way of expressing the SDF is in world space. The distance therefore needs to be mapped to the configuration space by the forward kinematics. This mapping makes the optimization problem a non-linear program (NLP), which is computationally expensive to solve. Recently, a different approach has been proposed. In \cite{mp_gcs} motion planning is formulated as a convex optimization problem by using the graph of convex sets framework \citep{gcs}. The underlying idea is to decompose the collision-free space into intersecting convex sets from which a convex optimization problem is formulated. In cases where an explicit representation of the obstacles in the configuration space exists, like for single-body robots, creating collision-free convex regions can be done fast \citep{iris}. For multi-body robots, this is non-trivial. Existing work does this successfully \citep{iris_nlp, iris_c} by an optimization based approach, but the methods are still too time consuming to be used in the presence of moving obstacles. Our approach is instead to use deep learning to learn an SDF expressed in the configuration space. With this, we can query for shortest distances to the collision boundary, which allows us to expand spherical regions which are collision-free. Our approach is fast and therefore enables our suggested roadmap planner to be used in dynamic environments.
\\\\
Recent research has focused on learning collision detection \citep{fk_kernel_distance, diffco, graphdistnet} by predicting the signed distance between the robot links and the surrounding obstacles in the world space. The learned SDF is used in trajectory optimization but since the distance is expressed in the world space, the problem becomes an NLP and therefore takes a long time to solve. We take a novel approach and suggest to instead express the signed distance in the configuration space. This allows us to improve the PRM at the same time as it enables convex optimization for trajectory optimization, which runs faster and is more reliable than NLP solvers. In \cite{cspf} a learned signed distance function in the configuration space is proposed similar to our approach. However, their approach is restricted to point cloud representations, while we propose to represent the obstacles as parameterized geometric shapes, e.g. spheres. Furthermore, we also show how to use our learned SCDF to improve an existing roadmap planner.
\section{Problem formulation}
A robot is located in the world space, $\W \subset \R^3 $. The unique location of the robot is given by its configuration $\q \in \C$, where $\C$ is the configuration space. The set of points covered by the robots bodies at a certain configuration is expressed as $\B(\q) \subset \W$. The robot is surrounded by $\NrObst$ obstacles $\O = \bigcup_{i=1}^{\NrObst} \O_i$, where  $\O_i \subset \W$. The representation of the obstacle in the configuration space is the set $\C\O_i = \{\q \in \C \: |\: \B(\q) \cap \O_i \neq \emptyset \}$. The obstacle space is formed as $\Co = \bigcup_{i=1}^{\NrObst} \C \O_i$. The complement is referred to as the free space, $\Cf = \C \setminus \Co$. The path planning problem is a tuple, ($\Cf$, $\qStart$, $\qGoal$), where we want to connect a query pair, consisting of a start, $\qStart$, and goal configuration, $\qGoal$, with a geometric path, $\q(s): [0, 1] \mapsto \Cf$, such that $\q(0)=\qStart$ and $\q(1)=\qGoal$, or report correctly when such a path does not exist.
\end{document}

%
\scaleSection
\section{Testbed Overview}\label{sec:experiment}
\scaleSectionBelow
\section{Experiments}\label{sec_exp}
%\hp{Accelerating IM simulation~\cite{tang2015influence}}

% \begin{itemize}
%     \item 6.1. Problem setting of three COPs, including the general model and three specific CO problems 
%     \item 6.2. Experiment Setting (hyperparameters, details of training, evaluation, and test) 写在appendix里吧
%     \item 6.3. Performance analysis 这个要占半页
% \end{itemize}

%\hp{need to think of a way to compress these tables / visuals.} 

%\hp{\cancel{Baselines}; hyperparamters; \cancel{metrics}; etc.}

With theoretical guarantees on the existence and convergence of NE for ACCES games, we are also interested in how our proposed algorithm CCDO-RL works empirically. To evaluate this, we conduct experiments of CCDO-RL on three distinct ACCES game instances introduced in Section \ref{sub_exp_ins} and analyze the performance of CCDO-RL in Section \ref{sub_train_eval}. Section 6.2.1 aims to empirically demonstrate the convergence (Figures \ref{fig_exploit_20} and \ref{fig_exploit_50}) of the algorithm CCDO-RL over realistic CO problems, and show its consistency with Theorem \ref{CCDOA}. Section 6.2.2 intends to show the average reward (to seen training graphs) as well as the generalizability (to unseen test graphs) of the combinatorial player in real-world ACCES games (shown in Tables \ref{tab_aver}, and \ref{tab_gene}).

\subsection{Three Instances of ACCES Games} \label{sub_exp_ins}
% \hp{This para does not make much sense. Need to follow the framework in the Preliminaries section.}
% For combinatorial optimization problems in real-world applications, situations are more complicated and intractable due to changeable environmental or physical parameters. The form of parameter sets is very crucial because different types have different solvability and computation complexity. Forms of parameter sets mainly contain discrete sets, interval sets \cite{buchheim2018robust} like polyhedral and ellipsoid, probability distributions \cite{carlsson2018wasserstein}, and variable functions \cite{krause2008robust}.

% In reality, these parameters are often impacted by some common factors, such as conditions of weather, transportation, and individual personalities. \cite{kalimeris2019robust} proposed an assumption that real instances (e.g. demands in CVRP, coverages in CSP) 
%Considering affected or attacked COPs, the real instance $\{\theta_{i}\}$ always relied on the estimated value $\{\hat{\theta}_{i}$\} and the variation determined by independent factors $\{g_{i}\}$ and environment/physical parameters/attacker actions $\{\eta\}$. The concrete parameter influence model is stated as follows:

We consider a certain COP which is parameterized with $\{\theta_{i}\}$, where $i$ is the index of nodes (such as a target in security games) -- e.g., such parameters can be interpreted as attack probability of targets.
%coverage radius, customer's demands, or attack probability of targets. 
In real-world applications, we often need to estimate such parameters before solving the COPs. Unfortunately, the estimation $\{\hat{\theta}_{i}\}$ often bears a gap to the true value $\{\theta_{i}\}$, which derives from e.g. environment (aleatoric) uncertainty, model (epistemic) uncertainty, or an attacker trying to manipulate the defender's utility. We use a generic model to formulate this gap:
\begin{equation}\label{linrob}
    \theta_{i} = \hat{\theta}_{i} + y \cdot \tau_{i},
\end{equation}
where $y$ represents the strategy of the nature/attacker, $\tau_{i}$ is the environment factors like weather and transportation conditions, or human subjective factors like the preference of the attacker. 
Such abstraction can represent a wide range of ACCES games, such as facility location covering problems \cite{an2020battery, TIRKOLAEE2020340}, CVRP \cite{vehiclerouting.ch8,dinh2018exact, FLORIO20231081}, security patrolling (OP) \citep{xu2021robust}, and influence maximization problem \cite{kalimeris2019robust}. We describe three instances of ACCES games based on the model (\ref{linrob}).%Based on this model (\ref{linrob}), we focus on three combinatorial optimization problems with attacks or environmental/physical influence.

% \hp{Hard to follow. We should point out what are the two players, what are X, Y, u etc}

\textbf{Adversarial Covering Salesman Problem (ACSP):} In a map of cities, every city $i$ has a coverage $\theta_{i}$. A salesman finds the shortest path such that all cities are visited or covered, with $\theta_{i}$ influenced by physical factors $\tau_i$ and transportation parameters $y$ based on Eq.(\ref{linrob}). The salesman is Player 1 where $X$ consists of the feasible paths of the salesman. Nature is Player 2 with $Y$ = $[0, 1]^K \ni y, K \in \mathbb{N}$. The utility function of Player 1 $u$ is the opposite of the total traveling distance.

\textbf{Adversarial Capacitated Vehicle Routing Problem (ACVRP):} A vehicle with a constrained capacity of goods finds the shortest path under the worst case with the $i_{th}$ customer's demand $\theta_i$ changed by environmental factors $\tau_i$ and weather parameter $y$ on Eq.(\ref{linrob}). The vehicle is Player 1 where $X$ is the set of the feasible path $x$. Nature is Player 2 where $Y$ is $[0, 1]^K \ni y, K \in \mathbb{N}$. The utility function of Player 1  $u$ is the opposite of total delivery distance satisfying all the demands of customers.


\textbf{Patrolling Game (PG):} The patrolling game is described in the introduction.

For all the problem instances, we run our algorithm on two problem sizes: 20 nodes and 50 nodes. The detailed description and problem parameters of the three game instances are in Appendix \ref{app_ex_para_set}.

% Similarly, in the vehicle route problem (VRP), conditions with correlated parameters arouse broad attention from scholars \cite{vehiclerouting.ch8,dinh2018exact,FLORIO20231081}. \cite{dinh2018exact} considered the demand correlation by geographical proximity of nodes, described by some independent random variables in the fractional form. \cite{FLORIO20231081} utilized 'external factors' to stand for unknown covariates affecting all demands and presented a Bayesian model to learn correlations. Further more, about IM problems, \cite{kalimeris2019robust} combined node features and uncertain hyperparameters to fit the influence probability on each edge.

% \subsection{Training CCDO-RL}

% For all the problems, CCDO-RL adopts the REINFORCE algorithm with an attention-based encoder-decoder framework \cite{kool2018attention} (used as an inductive graph representation component) to learn a (generalizable) COP solver for one player (protagonist), and PPO \cite{schulman2017proximal} to train a policy for the other player (adversary) whose strategy space is continuous. CCDO-RL is trained with 50 epochs on a set of 10,000 graphs (with 20 or 50 nodes). The hyperparameters of CCDO-RL are specified in Appendix \ref{app_ex_para_set} (Table \ref{tab_hyper_ccdorl}). Our code is included as supplementary material for ease of reproduction. 
% % \hp{need to specify hyperparas}

\subsection{Performance of CCDO-RL}\label{sub_train_eval}

Two aspects are evaluated for the performance of CCDO-RL, i.e., i) Convergence to NE (Section \ref{sub_per_conver}) exploring whether CCDO-RL can compute the NE, and ii) Protagonist policy's average reward and generalizability (Section \ref{sub_per_rob}). Generalizability refers to the ability of RL models trained on previously seen graphs (problem instances), to perform well on a new set of unseen test graphs. The model’s usability is enhanced by generalizability, rather than focusing solely on the average reward, which is a critical motivation in the literature on RL for COPs \citep{khalil2017learning, kool2018attention}.

For all the problems, CCDO-RL adopts the REINFORCE algorithm with an attention-based encoder-decoder framework \citep{kool2018attention} (used as an inductive graph representation component) to learn a generalizable COP solver for Player 1 (protagonist), and PPO to train a policy for Player 2 (adversary) whose strategy space is continuous. CCDO-RL is trained on a set of 10,000 graphs (with 20 or 50 nodes). The hyperparameters of CCDO-RL are specified in Appendix \ref{app_ex_para_set} (Table \ref{tab_hyper_ccdorl}). Our code is included as supplementary material and will be open-sourced for ease of reproduction. 

% \textbf{Training.} For all the problems, CCDO-RL adopts the REINFORCE algorithm with attention-based encoder-decoder framework \cite{kool2018attention} (used as an inductive graph representation component) to learn a (generalizable) COP solver for one player (protagonist), and PPO \cite{schulman2017proximal} to train a policy for the other player (adversary) whose strategy space is continuous. CCDO-RL is trained with 50 epochs on a set of 10,000 graphs (with 20 or 50 nodes). 

% \hp{We should first present results about convergence as it is mostly aligned with the theory.}

\subsubsection{Convergence to NE} \label{sub_per_conver}

Exploitability is a common metric to describe the closeness to true NE by calculating the sum of performance distances between each new best response and subgame NE, i.e. $\sum_{i=1,2} U(\pi_{i,k}^{br}, \sigma_{-i,k}) - U(\sigma)$ in the general two-player game. Since our game is zero-sum, the calculation is as follows:
\begin{equation*}
   \text{Exploitability}(\sigma) = \max_{\pi_1 \in \Sigma_1} U(\pi_1, \sigma_{2}) - \min_{\pi_2 \in \Sigma_2} U(\sigma_1, \pi_2).
\end{equation*}
From Figure \ref{fig_exploit_20}, we can see that CCDO-RL can converge to approximate NE in 25 iterations or less (in the PG setting), reaching 0.05 in ACSP, 0.10 in ACVRP, and 0.03 in PG with 20 nodes. Similar results are observed in problems with 50 nodes (see Figure \ref{fig_exploit_50} in Appendix \ref{app_exp}). These results validate the effectiveness of CCDO-RL in finding the NE for various types of games.

%Similarly, the exploitability of three COPs in 50 nodes is provided in the appendix \ref{app_exp}.
\vspace{-\baselineskip}
\begin{figure}[htbp]
	\centering
    \subfigure[ACSP20]{
    \label{csp20_nashconv}
    \includegraphics[scale=0.20]{Figures/nashconv_log_csp20_sm_7.eps}
    }
    \subfigure[ACVRP20]{
    \label{cvrp20_nashconv}%文中引用该图片代号
    \includegraphics[scale=0.20]{Figures/nashconv_log_svrp20_sm_7.eps}
    }
    \subfigure[PG20]{
    \label{opsa20_nashconv}
    \includegraphics[scale=0.20]{Figures/nashconv_log_pg20_sm_7.eps}
    }
    \caption{Exploitability curve of CCDO-RL on three games of 20 nodes}
    \label{fig_exploit_20}
\end{figure}
\vspace{-\baselineskip}
\subsubsection{Average reward and Generalizability of Combinatorial player} \label{sub_per_rob}
% \subsubsection{Robustness and Generalizability of Protagonist Policy} \label{sub_per_rob}
%\hp{CCDO-RL being better in these following metrics is only kind of a by-product.}

% \textbf{Evaluation.} The learned policies are then tested on 200 graphs, where 100 of them are randomly selected from the 10,000 training graphs, and the other 100 are unseen graphs. 
% We use two metrics to evaluate the performance of different policies for the protagonist player: \textbf{Average proportional loss} $R-$ describes the policy overfitting degree \citep{lanctot2017unified}; \textbf{Reward} evaluates the performance of the protagonist with the adversary under three COPs.  
% \begin{eqnarray}
%         &R- = (\hat{D} - \hat{O}) / \hat{D}.
% \end{eqnarray}
% in which $\hat{D}$ is the mean value of the diagonals and $\hat{O}$ is the mean value of the off-diagonals in the payoff matrix provided in the Appendix \ref{app_exp}.

% Because the protagonist policy is trained against a powerful adversary under our ACCES game setting, the obtained policy is naturally robust against adversarial perturbations. This subsection sheds a bit of light on this perspective and quantifies the extent of robustness of CCDO-RL as well as the ability of RL to generalize to unseen test graphs.

\textbf{Evaluation.} The learned policies are tested on 200 graphs, with 100 being randomly selected from the 10,000 training graphs (to show the average reward), and the other 100 being unseen graphs (to test policy generalization). We evaluate the performance of the protagonist with the adversary under three COPs. For each COP, the performance is considered both on the 20-node and 50-node map.
% We use two metrics to evaluate the performance of different policies for the protagonist player: \textbf{Average proportional loss} $R-$ describes the policy overfitting degree \citep{lanctot2017unified}; \textbf{Reward} evaluates the performance of the protagonist with the adversary under three COPs.

\textbf{Baselines.} There are heuristic algorithms for each game instance (Heuristic in Table \ref{tab_aver} and \ref{tab_gene}) and a single-player RL algorithm. For ACVRP, we adopt the Tabu Search algorithm (Tabu) \citep{li2020improved} as the heuristic algorithm, which is widely applied in the routing problem. For ACSP, the common benchmark local search algorithm, LS2 \citep{golden2012generalized}, is used. For PG, we choose the greedy algorithm as the baseline. The "RL against Stoc" algorithm in Tables \ref{tab_aver} and \ref{tab_gene} is identical to the protagonist model in CCDO-RL but trained in environments with stochastic adversarial perturbations.

% \textbf{Baselines.} There are a heuristic algorithms for each game instance {\color{red} (Heuristic mentioned in the Table \ref{tab_aver} and \ref{tab_gene})} and a single-player RL algorithm. For ACVRP, we adopt the Clarke-Wright (CW) algorithm \citep{pichpibul2013heuristic} and the Tabu Search algorithm (Tabu) \citep{li2020improved} as heuristics, which are applied widely in the routing problem. For ACSP, two common benchmark local search algorithms, LS1 and LS2 \citep{golden2012generalized}, are used. For PG, we choose a local search algorithm \citep{vansteenwegen2009iterated} and the greedy algorithm as the heuristic baselines. {\color{red} The "RL  against Stoc" algorithm referred to Tables \ref{tab_aver} and \ref{tab_gene}} is identical to the protagonist model in CCDO-RL {\color{red} but trained on environments with stochastic adversarial perturbations.} 

\textbf{Average Reward.}  As illustrated in Table \ref{tab_aver}, our algorithm achieves a better average reward than baselines (10.08\% improvement on average of all settings against two baselines), regardless of CO instance or problem size, when confronting the adversary trained by CCDO-RL. In the setting of CSP-20 nodes, the average reward is improved by 46.98\% compared to the heuristic and by 7.14\% compared with the RL against Stoc. For the 50-node setting, the improvements are 45.91\% and 5.28\% respectively. Similarly, the improvements in contrast to Heuristic and RL against Stoc are as follows: 1.72\% and 3.01\%  for CVRP-20 nodes, 0.75\% and 4.46\% for CVRP-50 nodes, 4.17\% and 1.48\% for PG-20 nodes, and 10.60\% and 4.38\% for PG-50 nodes.

\textbf{Generalizability.} From Table \ref{tab_gene}, CCDO-RL continues to achieve a better average reward when facing the adversary, demonstrating that the learned RL policies generalize well to unseen graphs. Even though the non-RL baselines do have access to the graph structures and other problem information of the unseen problem instances, CCDO-RL can obtain comparable performances without re-training on the new problem instances. The improvements versus Heuristic and RL against Stoc are 46.61\% and 7.02\% for CSP-20 nodes, 42.24\% and 3.94\% for CSP-50 nodes, 1.12\% and 1.56\% for CVRP-20 nodes, 0.90\% and 5.05\% for CVRP-50 nodes, 5.35\% and 2.40\% for PG-20 nodes, and 12.17\% and 10.33\% for PG-50 nodes. Even when confronting the stochastic adversary, CCDO shows superior generalizability compared to two baselines across three COPs, with average improvements of 6.31\%, 3.42\%, and 3.95\% respectively. Detailed results are provided in Appendix \ref{app_exp} (Tables \ref{tab_csp_full_20} - \ref{tab_op_full_50}). 
% The model’s usability is enhanced by the ability to generalize rather than focusing solely on the average reward, which is a critical motivation of the RL for combinatorial optimization literature \citep{khalil2017learning, kool2018attention}.  

\begin{remark}
    In CO problems (or more broadly, operations research and economics), it is known that achieving solution quality improvements against strong baselines (e.g., the RL methods trained with a stochastic adversary) is very challenging, and the margins are usually small \citep{kool2018attention}, sometimes even less than 1\%. However, these “tiny” marginal improvements in profits keep small business owners in the real world alive. Last, the improvement depends a lot on the problem settings, and we show that sometimes the improvement can be much more significant.
\end{remark}
\vspace{-\baselineskip}
% \textbf{Performance analysis.} The robustness results of CCDO-RL for ACSP are shown in Table \ref{tab_csp}. We have the following observations: 1) On both of the 100 seen/unseen graphs, single-player RL performs better than heuristic algorithms no matter whether attacked or not. (2) When confronting the adversary trained by CCDO-RL, CCDO-RL exceeds RL by 0.25 and 0.24 on the training set, and by 0.25 and 0.18 on the test set, respectively under the 20-node and 50-node graphs. This demonstrates the robustness of CCDO-RL. 3) Compared to the performance of the training set with that of the test set, we can see that RL and CCDO-RL both maintain a certain degree of generalization. Similar results for ACVRP (Table \ref{tab_cvrp}) and SPG (Table \ref{tab_op}) are provided in Appendix \ref{app_exp}. 

\begin{table}[ht]
  \caption{Average reward against CCDO-RL's adversary (on seen graphs)}
  \vspace{\baselineskip}
  \label{tab_aver}
  \centering
  \small
  \begin{tabular}{lllllll}
    \toprule
    \multirow{2}{*}{method} & \multicolumn{2}{c}{ACSP (Mean$\pm$Std)} & \multicolumn{2}{c}{ACVRP (Mean$\pm$Std)} & \multicolumn{2}{c}{PG (Mean$\pm$Std)} \\
    \cmidrule(r){2-3} \cmidrule{4-5} \cmidrule(r){6-7}
                            & 20 nodes & 50 nodes & 20 nodes & 50 nodes & 20 nodes & 50 nodes\\
    \midrule
    Heuristic & 6.13$\pm$1.20 & 7.55$\pm$1.42 & 7.65$\pm$1.23  & 13.38$\pm$1.70 & 2.64$\pm$1.03 & 4.53$\pm$1.84   \\
    RL against Stoc    & 3.50$\pm$0.47  & 4.55$\pm$0.62  & 7.55$\pm$1.16  & 13.90$\pm$1.63 & 2.71$\pm$0.90 & 4.80$\pm$2.18   \\
    CCDO-RL   & $\pmb{3.25}$$\pm$0.42 & $\pmb{4.31}$$\pm$0.51  & $\pmb{7.42}$$\pm$1.21  & $\pmb{13.28}$$\pm$1.52 &  $\pmb{2.75}$$\pm$0.87 & $\pmb{5.01}$$\pm$1.91  \\
    \bottomrule
  \end{tabular}
\end{table}
\vspace{-\baselineskip}

\begin{table}[htp]
  \caption{Generalizability against CCDO-RL's adversary (on unseen graphs)}
  \vspace{\baselineskip}
  \label{tab_gene}
  \centering
  \small
  \begin{threeparttable}
  \begin{tabular}{lllllll}
    \toprule
    \multirow{2}{*}{method} & \multicolumn{2}{c}{ACSP (Mean$\pm$Std)} & \multicolumn{2}{c}{ACVRP (Mean$\pm$Std)} & \multicolumn{2}{c}{PG (Mean$\pm$Std)} \\
    \cmidrule(r){2-3} \cmidrule{4-5} \cmidrule(r){6-7}
                            & 20 nodes & 50 nodes & 20 nodes & 50 nodes & 20 nodes & 50 nodes\\
    \midrule
    Heuristic & 6.20$\pm$1.33 & 7.60$\pm$1.37   & 7.64$\pm$1.30  & 13.27$\pm$1.87 & 2.43$\pm$0.98 & 4.19$\pm$1.69    \\
    RL against Stoc  & 3.56$\pm$0.37  & 4.57$\pm$0.58  & 7.67$\pm$1.30  & 13.85$\pm$1.53 &  2.50$\pm$0.95 & 4.26$\pm$2.17 \\
    CCDO-RL   & $\pmb{3.31}$$\pm$0.35 & $\pmb{4.39}$$\pm$0.52  & $\pmb{7.55}$$\pm$1.28  & $\pmb{13.15}$$\pm$1.59 & $\pmb{2.56}$$\pm$0.92 & $\pmb{4.70}$$\pm$1.94\\

    \bottomrule
  \end{tabular}
  \begin{tablenotes}
      \footnotesize
      \item[1] For the average reward of ACSP and ACVRP, smaller is better while for that of PG larger is better.
  \end{tablenotes}
  \end{threeparttable}
\end{table}
\vspace{-\baselineskip}
% two heuristics and one RL
% \begin{table}[ht]
%   \caption{{\color{red} Average reward of CCDO-RL (on seen graphs). For the value of CSP and CVRP, larger is better while for that of PG smaller is better.}}
%   \label{tab_aver}
%   \centering
%   \small
%   \begin{tabular}{lllllll}
%     \toprule
%     \multirow{2}{*}{method} & \multicolumn{2}{c}{CSP (Mean$\pm$Std)} & \multicolumn{2}{c}{CVRP (Mean$\pm$Std)} & \multicolumn{2}{c}{PG (Mean$\pm$Std)} \\
%     \cmidrule(r){2-3} \cmidrule{4-5} \cmidrule(r){6-7}
%                             & 20 nodes & 50 nodes & 20 nodes & 50 nodes & 20 nodes & 50 nodes\\
%     \midrule
%     Baseline 1 & 4.52$\pm$0.71  & 5.98$\pm$0.94 & 7.64$\pm$1.56  & 13.49$\pm$2.10 & 2.71$\pm$1.10 & 1.82$\pm$1.40   \\
%     Baseline 2 & 6.13$\pm$1.20 & 7.55$\pm$1.42   & 7.65$\pm$1.23  & 13.38$\pm$1.70 & 2.64$\pm$1.03 & 1.47$\pm$0.99  \\
%     RL {\color{red}against Stoc}    & 3.50$\pm$0.47  & 4.55$\pm$0.62  & 7.55$\pm$1.16  & 13.90$\pm$1.63 & 2.71$\pm$0.90 & 1.54$\pm$1.03   \\
%     CCDO-RL   & $\pmb{3.25}$$\pm$0.42 & $\pmb{4.31}$$\pm$0.51  & $\pmb{7.42}$$\pm$1.21  & $\pmb{13.28}$$\pm$1.52 &  $\pmb{2.75}$$\pm$0.87 & $\pmb{1.87}$$\pm$1.22  \\
%     \bottomrule
%   \end{tabular}
% \end{table}


% \begin{table}[htp]
%   \caption{{\color{red}Generalizability of CCDO-RL (on unseen graphs)}}
%   \label{tab_gene}
%   \centering
%   \small
%   \begin{threeparttable}
%   \begin{tabular}{lllllll}
%     \toprule
%     \multirow{2}{*}{method} & \multicolumn{2}{c}{CSP (Mean$\pm$Std)} & \multicolumn{2}{c}{CVRP (Mean$\pm$Std)} & \multicolumn{2}{c}{PG (Mean$\pm$Std)} \\
%     \cmidrule(r){2-3} \cmidrule{4-5} \cmidrule(r){6-7}
%                             & 20 nodes & 50 nodes & 20 nodes & 50 nodes & 20 nodes & 50 nodes\\
%     \midrule
%     Baseline 1 & 4.53$\pm$0.79  & 5.95$\pm$0.96 & 7.55$\pm$1.39  & 13.35$\pm$2.04 & 2.52$\pm$1.08 & $\pmb{1.86}$$\pm$1.44  \\
%     Baseline 2 & 6.20$\pm$1.33 & 7.60$\pm$1.37   & 7.64$\pm$1.3  & 13.27$\pm$1.87 & 2.43$\pm$0.98 & 1.52$\pm$1.20    \\
%     RL {\color{red}against Stoc}  & 3.56$\pm$0.37  & 4.57$\pm$0.58  & 7.67$\pm$1.30  & 13.85$\pm$1.53 &  2.50$\pm$0.95 & 1.03$\pm$5.05 \\
%     CCDO-RL   & $\pmb{3.31}$$\pm$0.35 & $\pmb{4.39}$$\pm$0.52  & $\pmb{7.55}$$\pm$1.28  & $\pmb{13.15}$$\pm$1.59 & $\pmb{2.56}$$\pm$0.92 & 1.35$\pm$5.09\\

%     \bottomrule
%   \end{tabular}
%   \begin{tablenotes}
%       \footnotesize
%       \item[1] For the value of CSP and CVRP, larger is better while for that of PG smaller is better.
%   \end{tablenotes}
%   \end{threeparttable}
% \end{table}

%
\scaleSection
\section{Physical Characterization of the Testbed}\label{sec:characterization}
\scaleSectionBelow
\section{Characterizing LLM Inference}
\label{sec:characteristics}
This section characterizes LLM inference performance in on-server and on-device paradigms, which informs our design.

We evaluate four commercial streaming LLM APIs: OpenAI's GPT-4o-mini~\citep{gpt-4o-mini}, DeepSeek's DeepSeek-V2.5~\citep{deepseek-v2_5}, Cohere's Command~\citep{command}, and Hyperbolic-hosted LLaMA-3-70b-Instruct~\citep{llama3-70b}. For on-device analysis, we deploy Qwen-2.5-7B-Instruct~\citep{qwen2_5} and Llama-3.1-8B-Instruct~\citep{grattafiori2024llama3herdmodels} on both server-grade (NVIDIA A40, 48GB) and consumer-grade (dual NVIDIA RTX 3080, denoted as 3080x2) GPUs. 
We sample 1,000 requests from the Alpaca dataset~\citep{alpaca}, following a Poisson distribution with a mean request arrival interval of 30 seconds.

\begin{figure}
    \subfigure[On-Server TTFTs.]{\includegraphics[width=0.5\columnwidth]{figs/ttft_analysis_server.pdf}}\hfill
    \subfigure[On-Device TTFTs.]{\includegraphics[width=0.5\columnwidth]{figs/ttft_analysis_device.pdf}}
    \caption{On-device TTFT performance is more stable.}
    \label{fig:ttft_repeat}
\end{figure}

\paragraph{TTFT characteristics.} 
Our measurements reveal contrasting TTFT patterns between on-device and on-server inference. As shown in Figure~\ref{fig:ttft_repeat}, on-device inference exhibits stable TTFTs when processing identical prompts at 60-second intervals, primarily reflecting the prefill duration due to dedicated local hardware resources. In contrast, on-server inference experiences high variations and significant tail latency, attributed to network delays, request queuing, and resource contention.

We summarize the TTFT performance of 1,000 requests in Table~\ref{tab:correlation-analysis}. We observe that on-device TTFT scales linearly with prompt length due to hardware constraints~\citep{edgebenchmark}, while on-server TTFT shows minimal prompt-length sensitivity through advanced resource scaling~\citep{distserve,splitwise,memserve}.

\begin{table}[t]
    \centering
    \footnotesize
    \begin{tabularx}{\linewidth}{lcc}
    \toprule
    \textbf{Model} & \textbf{Deployment} & \textbf{Pearson Coef.} \\
    \midrule 
    Command & Server & 0.0142 \\
    GPT-4o-mini & Server & 0.0236 \\
    DeepSeek-V2.5 & Server & -0.0273 \\
    LLaMA-3-70b-Instruct & Server & 0.0402 \\
    \hline
    LLaMA-3.1-8b-Instruct & Device & 0.8424 \\
    \bottomrule
    \end{tabularx}
    \caption{Pearson coefficient between prompt length and TTFT in on-server deployment is weak.}
    \vskip -0.1in
    \label{tab:correlation-analysis}
\end{table}

\paragraph{TBT characteristics.}
TBT characterizes the I/O-bound decode stage latency. Analysis of temporal samples and distributions across six setups (Figure~\ref{fig:tbt_analysis}) reveals higher TBT variability in on-server inference compared to on-device execution. More importantly, both deployment approaches achieve generation speeds exceeding user consumption rates~(\S \ref{sec:llm_applications}), making cooperative serving practical.

\begin{figure}[t]
    \centering
    \includegraphics[width=0.49\textwidth]{figs/tbt_analysis.pdf}
    \vskip -0.1in
    \caption{On-device TBT performance is more stable. \footnotemark[1]{}
    }
    \vskip -0.1in
    \label{fig:tbt_analysis}
\end{figure}

\footnotetext[1]{On-server inference, such as in GPT, streams tokens with each packet containing multiple tokens, resulting in near-zero perceived TBTs.}

\paragraph{Opportunities and challenges.}
Our studies further reveal that as on-device models continue to improve---often fine-tuned for specific tasks~\citep{appleintelligence,liu2024mobilellm}---their performance increasingly matches that of on-server models in popular applications like instruction-following and translation (detailed in \S\ref{sec:evaluation} and Appendix~\ref{appendix:accuracy-eval}). However, deploying these models on-device introduces challenges such as long prefilling latency and startup overhead.

On the other hand, our real-world studies of conversational workloads highlight key opportunities: (i) on-server TTFT is largely unpredictable and shows minimal correlation with prompt length, whereas on-device TTFT scales nearly linearly with prompt length and is highly predictable; and (ii) both paradigms achieve token generation speeds that exceed typical user consumption rates.

Taking these findings together---particularly the predictable performance of on-device inference and the elastic scaling capabilities of server-based inference---we observe opportunities for optimization in cost-constrained device-server cooperative serving. Dynamic request migration between server and device endpoints during response generation can yield significant cost savings. 

%
\scaleSection
\section{Communication Scheme}\label{sec:communication}
\scaleSectionBelow
This section introduces the proposed communication scheme, which includes modulation, synchronization, and detection.
%
\scaleSubsection
\subsection{Modulation and Reception}\label{Sec:Modulation}
\scaleSubsectionBelow
%
In \Sections{subsubsec:tx}{subsubsec:rx}, we discussed modulation and reception qualitatively, respectively. This section provides a formal description of the transmission parameters, their effects on the \ac{MC} signal, and the detection at the \ac{RX}.

To transmit a symbol $\symVar[k] \in \{0, 1, \ldots, \modOrder - 1\}$, the \ac{TX} is turned on with a corresponding light intensity $\txInten_\symVar$ for irradiation duration $\Ti$. Here, $\symIdx$ and $\modOrder = 2^\eta$ with $\eta \in \mathbb{N}$, where $\mathbb{N}$ is the set of positive integers, denote the symbol index and the modulation order, respectively. $\Ti$ is followed by the guard interval of duration $\Tg$, which results in $\Ts = \Ti + \Tg$, cf. \Section{subsubsec:tx}, and a data rate $R$ of
\begin{equation}
    R = \frac{\log_2 \modOrder}{\Ts} \;.
\end{equation}
During the irradiation process, the \ac{GFPD} in the tube section of the \ac{TX} undergoes a state transition (from ON to OFF) when hit by the emitted photons, reducing the local fluorescence. Increasing light intensity $\txInten$ increases the hit probability, allowing control over the local fluorescence. This enables higher-order modulation with distinct fluorescence drops for different symbols $\symVar$. 
%
In particular, in this work, we modulate the light intensity ratio $\ratioInten_\symVar$ for symbol $\symVar$ as
\begin{equation}\label{eq:power_ratio}
    \ratioInten_\symVar = \frac{\txInten_\symVar}{\maxTxInten} = \frac{3 \symVar}{4(M-1)} + \frac{1}{4},
\end{equation}
i.e., $\ratioInten_\symVar$ is determined as the ratio of $\txInten_\symVar$ to the maximum available light intensity $\maxTxInten$. Eq.~\Equation{eq:power_ratio} shows that $\ratioInten_\symVar$ ranges from $1/4$ to $1$, i.e., we do not assign intensity ratio zero to any symbol. This guarantees that the transmission of all symbols is distinct from the idle channel state, i.e., a turned off \ac{TX}. This is necessary for symbol-by-symbol synchronization, as described in \Section{subsec:sync}.
%
On the \ac{RX} side, the spectrometer periodically measures the local fluorescence by sampling with time interval $\sampleInt$. After isolating the part of the signal close to the wavelength of interest, i.e., $\lambda_\mathrm{E} = 529 \, \si{\nm}$, and normalizing\footnote{The \textit{ThorSpectra software} internally scales the received fluorescence intensity in a non-transparent manner. Comparing absolute fluorescence values from different experiments is therefore not possible. Instead, it is useful to compare relative values, i.e., trends of the measured fluorescence intensities, which is guaranteed by the proposed normalization.} the maximum signal value to $1$, the discrete-time received signal $\recSig(\tn) \in [0,1]$ is obtained, where $\tn = n\sampleInt$, with $n \in \mathbb{N}_0$. Here, $\mathbb{N}_0$ denotes the set of non-negative integers.
%
\scaleSubsection
\subsection{Synchronization}\label{subsec:sync}
\scaleSubsectionBelow
%
Synchronization is an integral part of any communication system and is required for demodulation. While in stationary environments it can be sufficient to synchronize once at the beginning of the transmission of a data packet, the task of synchronization becomes critical -- and more challenging -- when dealing with non-stationary communication channels. Sources of this non-stationarity can be, for example, closed-loop \ac{ISI} effects, \ac{TX} and \ac{RX} movement, or a time-varying flow velocity. In this section, we propose synchronization schemes consisting of a transmission start detection and symbol-by-symbol synchronization. Hence, they address the aforementioned synchronization challenges.
%
\scaleSubsubsection
\subsubsection{Transmission Start Detection}\label{subsec:transmission}
\scaleSubsubsectionBelow
%
Prior to the initiation of demodulation, the \ac{RX} must detect that data transmission has started. To this end, a simple threshold-based trigger scheme is used, where transmission is assumed to have started when $\recSig(\tn)$ drops below a threshold $\trainThresh$. For this, $\recSig(\tn)$, which prior the to first transmission contains only noise, is used to estimate the noise statistics. Then, $\trainThresh$ is determined based on the noise statistic. We assume that the noise arises from many small, independent sources of randomness, e.g., thermal noise in the spectrometer, photon noise, i.e., the distribution of photons emitted by coherent light \cite{mandel1959fluctuations} from the \ac{RX} \ac{LED}, etc. Hence, applying the central limit theorem, we model the noise as Gaussian distributed.

Formally, before transmission has started, we collect a set of samples $\trainSet[\trainIdx] \triangleq \{\recSig(\tn)|\n \in \{\trainIdx \trainLength, \trainIdx \trainLength + 1, \hdots, (\trainIdx+1) \trainLength -1 \}\}$ of size $\trainLength$ for $\trainIdx \in \mathbb{N}_0$. Next, $\trainSet[\trainIdx]$ is used to fit a Normal distribution with mean $\trainMean[\trainIdx] = \frac{1}{\trainLength}\sum_{\recSig(\tn) \in \trainSet[\trainIdx]}\recSig(\tn)$ and variance $\trainVariance[\trainIdx] = \frac{1}{\trainLength-1}\sum_{\recSig(\tn) \in \trainSet[\trainIdx]} (\recSig(\tn) - \trainMean[\trainIdx])^2$ assuming independent and identically distributed samples, i.e., $\recSig(\tn)\,\sim\,\mathcal{N}(\trainMean[\trainIdx], \trainVariance[\trainIdx])$. Finally, we obtain the threshold value $\trainThresh[\trainIdx] = \Phi^{-1}(\pFA; \trainMean[\trainIdx], \trainVariance[\trainIdx])$ for which a desired false alarm probability $\pFA$, i.e., the residual risk of an incorrect decision that transmission has started, is achieved. Here, $\Phi^{-1}(\cdot)$ denotes the inverse of the Gaussian cumulative distribution function.

The obtained $\trainThresh[\trainIdx]$ is applied to the next sample set $\trainSet[\trainIdx+1]$. The transmission start is detected based on
\begin{equation}
    \tts = \min_{\forall \trainIdx}\{\tn|\recSig(\tn) \leq \trainThresh[\trainIdx],\, \recSig(\tn) \in \trainSet[\trainIdx+1]\}\;,
\end{equation}
where $\tts$ denotes the time at which transmission has started. Since $\recSig(\tn)$ may decreases (slowly) over time due to photobleaching, this process is continuously repeated to ensure that $\trainThresh$ is periodically adjusted until a transmission start has been detected\footnote{Note that detecting the end of transmission is not needed, as we assume fixed-length messages.}.
%
\scaleSubsubsection
\subsubsection{Symbol Synchronization}
\scaleSubsubsectionBelow
Synchronization guarantees a temporal alignment of the \ac{RX} with the \ac{TX}. For the testbed, we use a symbol-by-symbol synchronization approach \cite{jamali2017symbol}, which correlates the processed received signal\footnote{In particular, we use $\Tilde{\recSig}(\tn) \in \{1 - \recSig(\tn), \recSig(t_{\n+1}) - \recSig(\tn)\}$, as explained in detail in the next paragraph.} $\Tilde{\recSig}(\tn)$ with a receive template filter $g(\tn)$ to obtain the synchronization metric
\begin{equation}
  \syncMetric (\tn) = \vec{\filter}^\top \Tilde{\vec{\recSig}} (\tn) \;,
\label{sync_metric}
\end{equation}
based on which the symbol start time $\tsEst[\symIdx]$ is estimated\footnote{Note that $\tsEst[k]$ does not correspond to the actual symbol start time, as the channel-related propagation delay is not known at the \ac{RX}. Instead, it represents the \ac{RX}'s estimate of when a new symbol starts based on the selected template signal/filter.}.
Here, $\vec{\filter} = \big[\, g(t_{0}) \quad g(t_{1}) \quad \hdots$\\$ \quad g(t_{\filterLength-1})\,\big]^\top$ and $\Tilde{\vec{\recSig}}(\tn) = \big[\,\Tilde{\recSig}(\tn) \quad \Tilde{\recSig}(t_{n+1}) \quad \hdots \quad \Tilde{\recSig}(t_{n+\filterLength-1})\,\big]^\top$ are the vector representations of $g(\tn)$ and $\Tilde{\recSig}(\tn)$, respectively, where $\filterLength$ is the filter length and $[\cdot]^\top$ denotes the transpose operator.
Two synchronization schemes are employed: \Ac{CS} and \ac{DCS}.
\begin{itemize}
    \item \textit{Correlation-based Synchronization (CS)}: The \ac{CS} scheme utilizes the cross-correlation between the processed received signal $\Tilde{\recSig}(\tn) = 1 - \recSig(\tn)$, which captures the deviation from maximum fluorescence, and a corresponding receive filter $g(\tn) \in \{g^{\mathrm{D}}_{\mathrm{C}}(\tn), g^{\mathrm{B}}_{\mathrm{C}}(\tn)\}$. Details of how the data-based receive filter $g^{\mathrm{D}}_{\mathrm{C}}(\tn)$, referred to as \ac{SCF}, and the blind receive filter $g^{\mathrm{B}}_{\mathrm{C}}(\tn)$, referred to as \ac{BCF}, are obtained, respectively, are provided in \Section{matched_filter}.
    \item \textit{Differential Correlation-based Synchronization (DCS)}: The \ac{DCS} scheme is designed to mitigate the effects of slowly time-varying processes, such as offset-\ac{ISI}, and utilizes the cross-correlation between the \textit{differential} received signal $\Tilde{\recSig}(\tn) = \diffRecSig(\tn) = \recSig(t_{\n+1}) - \recSig(\tn)$ and a corresponding \textit{differential} receive filter $g(\tn) \in \{g^{\mathrm{D}}_{\mathrm{D}}(\tn), g^{\mathrm{B}}_{\mathrm{D}}(\tn)\}$. Note that in this case the corresponding vectors $\vec{\filter}$ and $\Tilde{\vec{\recSig}} (\tn)$ in \Equation{sync_metric} have dimension $\filterLength-1$, as the forward difference cannot be computed for the last vector entry. Details of how the data-based differential receive filter $g^{\mathrm{D}}_{\mathrm{D}}(\tn)$, referred to as \ac{SDCF}, and the blind differential receive filter $g^{\mathrm{B}}_{\mathrm{D}}(\tn)$, referred to as \ac{BDCF}, are obtained, respectively, are provided in \Section{matched_filter}.
  \end{itemize}
%
\paragraph{Symbol-by-symbol synchronization}
In the proposed symbol-by-symbol scheme, synchronization for the symbol in symbol interval $k$ relies on the last estimated symbol start time $\tsEst[k-1]$. A good initial guess for the start of the current symbol is given by $\tsInit[k] = \tsEst[k-1] + \Ts$. Note that $\tsInit[k]$ would also be the best estimate, if the previous estimate was accurate and the channel had not changed. Next, we define a search interval around $\tsInit[k]$ as $\searchInt[k] \triangleq [\tsInit[k] - \searchRadius \Ts, \tsInit[k] + \searchRadius \Ts]$. Here, $\searchRadius \in [0,1]$ denotes the normalized search radius, which can be adjusted based on the channel statistics, i.e., the stationarity of the channel.
The start time $\tsEst[k]$ of the current symbol interval $k$ is obtained as
\begin{equation}
  \tsEst[k] = \argmax_{\tn \in \searchInt[\symIdx]} \syncMetric(\tn)\;.
  \label{eq:estimation_symbol_start}
\end{equation}
This process is carried out for each symbol interval until the end of the transmission. The proposed synchronization method can adjust to fluctuations of the propagation delays in the channel within a single symbol interval as long as the true symbol start time is within the search interval.
%
\paragraph{Synchronization initialization}
Since for the first symbol interval there is no previous estimate $\tsEst[-1]$, we utilize the detected transmission start time as initial estimate, $\tsInit[0] = \tts$, and set the search radius to $\searchRadius=0.5$ -- resulting in search interval $\searchInt[0] = [\tsInit[0] - \frac{\Ts}{2}, \tsInit[0] + \frac{\Ts}{2}]$. Even if $\tts$ does not correspond to a false alarm, it may still be a suboptimal estimate for the first symbol start. Therefore, choosing search radius $\searchRadius=0.5$ ensures that the best estimate for the first symbol start time is within the search interval. Synchronization then proceeds as described above.
%
\scaleSubsection
\subsection{Receive Filters} \label{matched_filter}
\scaleSubsectionBelow
\begin{figure}[!tbp]
\vspace*{-1cm}
    \centering
    \begin{subfigure}[b]{0.49\textwidth}
        \caption{}
        \vspace{-3pt}
        \includegraphics[width=\textwidth]{fig/blind_matched_filter_new.pdf}
        \label{fig:BlindCorrFilter}
        \vspace*{-5mm}
    \end{subfigure}
    \begin{subfigure}[b]{0.49\textwidth}
        \caption{}
        \vspace{-3pt}
        \includegraphics[width=\textwidth]{fig/blind_diff_matched_filter_new.pdf}
        \label{fig:BlindDiffFilter}
        \vspace*{-5mm}
    \end{subfigure}
    \vspace*{0mm}
    \begin{subfigure}[b]{\textwidth}
        \includegraphics[width=\textwidth]{fig/filter_legend.pdf}
    \end{subfigure}
    \vspace*{-9mm}
    \caption{Data-based receive filters and blind receive filters in comparison for two different configurations: $\Ti=\SI{3}{\second}$, $\Ts=\SI{5}{\second}$ (on the left) and $\Ti=\SI{10}{\second}$, $\Ts=\SI{15}{\second}$ (on the right) for the \acs{CS} (a) and the \acs{DCS} scheme (b), respectively.}
    \label{fig:ReceiveFilter}
    \vspace*{-8mm}
\end{figure}
Two different approaches are used to obtain receive filters for the \ac{CS} and the \ac{DCS} schemes, namely a data-based approach and a blind filter approach. Although these filters are akin to a matched filter, we do no refer to them as such, as we cannot guarantee their optimality \ac{wrt} the achieved signal-to-noise ratio.
%
\scaleSubsubsection
\subsubsection{Data-Based Receive Filters}\label{data_filter}
\scaleSubsubsectionBelow
For each $\Ti$ considered in this work, a separate experiment was performed to obtain the typical shape of the received signal for that irradiation duration, on the basis of which the data-based filter was determined. Each of these experiments consisted of a binary transmission of a single bit 1 repeated 11 times at 60 second intervals. The 11 individual received signals obtained in this way are referred to as \acp{SR} $\SR(\tn)$. To derive the data-based filter, the first \ac{SR} is discarded and the remaining \acp{SR} are averaged and further smoothed using locally weighted polynomial regression \cite{cleveland1988locally} to obtain $\avgSR(\tn)$. Ignoring the first \ac{SR} is necessary because in the beginning almost all \acp{GFPD} are in the equilibrium state, cf. \Section{subsec:GFPD}, and therefore the fluorescence decay for the first \ac{SR} is higher than for the following \acp{SR}.
Then, the \ac{SCF}, $g^{\mathrm{D}}_{\mathrm{C}}(\tn)  = 1 - \avgSR(\tn)$, $n \in \{0, 1, \ldots, \filterLength-1\}$, and the \ac{SDCF}, $g^{\mathrm{D}}_{\mathrm{D}}(\tn)  = \avgSR(t_{\n+1}) - \avgSR(\tn)$, $n \in \{0, 1, \ldots, \filterLength-2\}$, are obtained by limiting them to an $\filterLength$ and $\filterLength-1$ samples long signal, respectively, i.e., limiting them such that their lengths match the lengths of the corresponding $\Tilde{\vec{\recSig}}(\tn)$, cf. \Equation{sync_metric}. The resulting receive filters $g^{\mathrm{D}}_{\mathrm{C}}(\tn)$ and $g^{\mathrm{D}}_{\mathrm{D}}(\tn)$ are shown in blue in \Figure{fig:BlindCorrFilter} and \Figure{fig:BlindDiffFilter}, respectively, exemplarily for two different symbol durations $\Ts$. \Figure{fig:BlindCorrFilter} shows that $g^{\mathrm{D}}_{\mathrm{C}}(\tn)$ increases over the irradiation durations $\Ti=\SI{3}{\second}$ (left subplot in \Figure{fig:BlindCorrFilter}) and $\Ti=\SI{10}{\second}$ (right subplot in \Figure{fig:BlindCorrFilter}), and decreases during the subsequent guard interval. For $\Ts=\SI{5}{\second}$ (left subplot in \Figure{fig:BlindCorrFilter}), $g^{\mathrm{D}}_{\mathrm{C}}(\tn)$ is bell-shaped, while for the longer symbol duration, $\Ts=\SI{15}{\second}$ (right subplot in \Figure{fig:BlindCorrFilter}), we see that $g^{\mathrm{D}}_{\mathrm{C}}(\tn)$ reaches a value of $0.25$, before decreasing after $\tn = \Ti = \SI{10}{\second}$. In \Figure{fig:BlindDiffFilter}, we see that the shape of $g^{\mathrm{D}}_{\mathrm{D}}(\tn)$ follows the derivative of $- g^{\mathrm{D}}_{\mathrm{C}}(\tn)$. Hence, it has a negative and a positive dip. For $\Ts=\SI{15}{\second}$ (right subplot in \Figure{fig:BlindDiffFilter}), $g^{\mathrm{D}}_{\mathrm{D}}(\tn)$ shows some fluctuations around $\tn=\SI{6}{\second}$, caused by noise that is amplified by the differentiation operation.
%
\scaleSubsubsection
\subsubsection{Blind Receive Filter}\label{blind_filter}
\scaleSubsubsectionBelow
In cases where the data-based filter cannot be obtained in advance, we propose blind filters as surrogates for both the \ac{CS} and \ac{DCS} schemes, respectively. For these blind filters, we aim to capture roughly the shape of the data-based filters, with the goal of achieving a solution that is applicable across a broad range of unknown parameters without significant performance loss. For the sake of a simple and practical implementation, we limit the modeling of the blind filters to piecewise polynomials. Hence, the impulse responses of the proposed blind filters \ac{BCF}, $g^{\mathrm{B}}_{\mathrm{C}}(\tn)$, and \ac{BDCF}, $g^{\mathrm{B}}_{\mathrm{D}}(\tn)$, are given as follows,
\begin{equation}\label{eq:blind_corr_filter}
\blindCorrFilter(\tn) =
\begin{cases}
     - \frac{\tn^2}{2\Ti} + \tn, & 0\leq \tn \leq \Ti\\[-0.2cm]
     \frac{0.5\Ti}{(\Ts-\Ti)^2}(\tn^2 -2\Ts \tn + \Ts^2), &  \Ti < \tn \leq  \Ts\\[-0.2cm]
     0, & \text{otherwise}
\end{cases}
\end{equation}
and
\begin{equation}\label{eq:blind_diff_corr_filter}
\blindDiffCorrFilter(\tn) =
    \begin{cases}
        \frac{\tn}{\Ti} - 1, & 0\leq \tn \leq \Ti\\[-0.2cm]
        \frac{\Ts - \tn}{\Ts-\Ti}, &  \Ti < \tn \leq  \Ts\\[-0.2cm]
        0, & \mathrm{otherwise}
    \end{cases}\,,
\end{equation}
respectively. $g^{\mathrm{B}}_{\mathrm{C}}(\tn)$ and $g^{\mathrm{B}}_{\mathrm{D}}(\tn)$ are plotted in green in \Figure{fig:BlindCorrFilter} and \Figure{fig:BlindDiffFilter}, respectively, exemplarily for two different symbol durations $\Ts$. \Figure{fig:BlindCorrFilter} shows that the shapes of the data-based filter $g^{\mathrm{D}}_{\mathrm{C}}(\tn)$ and the blind filter $g^{\mathrm{B}}_{\mathrm{C}}(\tn)$ are quite similar, while those of the differential filters $g^{\mathrm{D}}_{\mathrm{D}}(\tn)$ and $g^{\mathrm{B}}_{\mathrm{D}}(\tn)$ in \Figure{fig:BlindDiffFilter} are more different. We will show in \Section{sssec:effect_filter_and_detection} that despite these visible differences, the use of blind filters only leads to a small loss in performance compared to the use of data-based filters. In addition, blind filters do not require data collection, which justifies their use. Note that the blind filters are defined independent of each other as this leads to better results, i.e., $g^{\mathrm{B}}_{\mathrm{D}}(\tn)$ is not necessarily the sectional derivative of $g^{\mathrm{B}}_{\mathrm{C}}(\tn)$. Furthermore, note that the scaling of the data-based filters and the blind filters are different. However, for a given synchronization scheme, adjusting the scaling of any receive filter has no impact on the decisions derived from \Equation{sync_metric} and \Equation{eq:estimation_symbol_start}, since the operation in \Equation{sync_metric} is linear. Therefore, no scaling adjustment is needed.
%
\scaleSubsection
\subsection{Symbol Detection}\label{subsec:symbol_detection}
\scaleSubsectionBelow
%
For symbol detection, we use a single detection sample $\detecSample[\symIdx]$. Here, the detection sample is chosen to be the filtered received signal for the estimated symbol start time $\tsEst[\symIdx]$, which was determined by the proposed synchronization scheme, i.e., 
\begin{equation}
\detecSample[\symIdx]=\vec{\filter}^\top \Tilde{\vec{\recSig}} (\tsEst[\symIdx]) \;. \label{eq:detection_sample}
\end{equation}
In \Equation{eq:detection_sample}, correlation-based detection and differential correlation-based detection are employed, if \ac{CS} and \ac{DCS} was used for synchronization, respectively. Hence, when \ac{CS} and \ac{DCS} are mentioned in the following, this therefore also refers to the detection scheme used.

For detection, we employ an adaptive (multi-)threshold detector that periodically adjusts its thresholds based on previously detected symbols to account for the gradual reduction of the fluorescence over time due to photobleaching. Let $\detecThreshSet[\detecThreshSetIdx] \triangleq \{\detecThresh_0, \detecThresh_1, \hdots, \detecThresh_{\modOrder-2}\}$ denote the set of detection thresholds, where $\detecThresh_{\detecThreshIdx}$, $\detecThreshIdx \in \{0, 1, \hdots, \modOrder-2\}$, and $\detecThreshSetIdx$ denote the individual threshold and the threshold set index, respectively. Additionally, we assume \ac{wlog} that $\detecThreshSet[\detecThreshSetIdx]$ is ordered in ascending order, i.e., $\detecThresh_\detecThreshIdx < \detecThresh_{\detecThreshIdx+1}$. For the employed single-sample detection, the transmitted symbol $\estSym[\symIdx]$ is determined as follows
\begin{equation}
\estSym[\symIdx] = \max\{\detecThreshIdx | \detecThreshIdx \in \{0, 1, \hdots, \modOrder-2\} \wedge \detecSample[\symIdx] \geq \detecThresh_\detecThreshIdx\} \;,
\end{equation}
i.e., the symbol decision is based on $\detecSample[\symIdx]$ and $\detecThreshSet[\detecThreshSetIdx]$ only.
%
\scaleSubsubsection
\subsubsection{Detection Initialization}
\scaleSubsubsectionBelow
%
To determine the initial set of thresholds, $\detecThreshSet[0]$, each transmission starts with a random sequence of $\nSkip$ symbols.
This sequence is used to allow the system to settle\footnote{The experiment is initially in a settling phase caused by the offset \ac{ISI}, which develops slowly over successive transmissions and remains relatively constant after several symbols. \Section{sssec:effect_modulation_order} presents experimental results that illustrate the transient phase of the testbed.} and is not considered for detection. Next, $\nPilots$ pilot symbols are transmitted, where each $\symVar \in \{0, 1, \ldots, \modOrder - 1\}$ is sent at least once. The corresponding detection samples $\detecSample[k]$ form the initial sets $\symbolSampleSet_{\symVar}[\detecThreshSetIdx = 0] \triangleq \{d[\mu]|\mu \in \{\nSkip, \nSkip+1, \ldots, \nSkip+\nPilots-1\} \wedge \symVar[\mu]=i\}$, $\symVar \in \{0, 1, \ldots, \modOrder - 1\}$. The average of each set can be computed as follows
\begin{equation}
    \Bar{S}_\symVar[\detecThreshSetIdx] = \frac{1}{|\symbolSampleSet_{\symVar}[\detecThreshSetIdx]|}\sum_{s\in \symbolSampleSet_{\symVar}[\detecThreshSetIdx]} s \;.
    \label{eq:average_set}
\end{equation}
Here, $|\cdot|$ denotes the cardinality of a set. Hence, we use $\detecThreshSetIdx = 0$ in \Equation{eq:average_set} to determine the averages $\Bar{S}_\symVar[\detecThreshSetIdx = 0]$ of the initial sets $\symbolSampleSet_{\symVar}[\detecThreshSetIdx = 0]$. Finally, the threshold values are computed as the means of the $\Bar{S}_\symVar[l]$ of adjacent sets:
%
\begin{equation}\label{eq:nt}
    \detecThresh_\detecThreshIdx[\detecThreshSetIdx] = \frac{\avgSymbolSampleSet_{\detecThreshIdx+1}[\detecThreshSetIdx] + \avgSymbolSampleSet_{\detecThreshIdx}[\detecThreshSetIdx]}{2}\,,
\end{equation}
for $\detecThreshIdx \in \{0, 1, \hdots, \modOrder-2\}$. Thus, to obtain the initial set of thresholds, $\detecThreshSet[0]$, we use $\detecThreshSetIdx = 0$ in \Equation{eq:nt}.
%
\scaleSubsubsection
\subsubsection{Adaptation Algorithm}
\scaleSubsubsectionBelow
%
The duration for which the threshold set is valid has an upper limit that is determined by the coherence time of the channel. Therefore, a new set of thresholds must be determined after a certain number of detected symbols $\nCoherence$. In contrast to the initial set of thresholds $\detecThreshSet[0]$, which is determined based on the $\nPilots$ pilot symbols, the \textit{update} of the set of thresholds requires only previously detected \textit{data symbols}, i.e., no pilot symbols are used to update the threshold values. The corresponding update algorithm is described next.

For this, we relate the symbol index $\symIdx$ to the currently valid threshold set index $\detecThreshSetIdx$ via $\detecThreshSetIdx = \lfloor \frac{\symIdx-\nSkip-\nPilots}{\nCoherence}\rfloor$, where $\lfloor \cdot \rfloor$ denotes the floor function. Note that $\detecThreshSetIdx$ is not defined for $\symIdx < \nPilots + \nSkip$ because the first thresholds are computed only after the $\nPilots$ pilot symbols have been received.

Similar to the initial sets $\symbolSampleSet_\symVar[\detecThreshSetIdx = 0]$, we define sets
\begin{equation}
    \symbolSampleSet_\symVar[\detecThreshSetIdx]\triangleq\{\detecSample[\mu]|\mu \in \{\detecThreshSetIdx\nCoherence+\nSkip+\nPilots-\nWindow,..., \detecThreshSetIdx\nCoherence+\nSkip+\nPilots-1\} \wedge \estSym[\mu]=\symVar\}\;,
    \label{eq:sampleSets}
\end{equation}
for $\detecThreshSetIdx>0$ and $\nCoherence+\nPilots-\nWindow \geq 0$. Here, $\nWindow$ denotes the window width that determines the number of past samples $\detecSample[k]$ used for reevaluation. Finally, using \Equation{eq:average_set} and \Equation{eq:nt}, we obtain the new set of thresholds $\detecThreshSet[\detecThreshSetIdx]$.

In contrast to the pilot symbol sequence, now it is not guaranteed that all symbol values will be present in the $\nWindow$ previously received symbols, i.e., some sets may remain empty ($|\symbolSampleSet_\symVar[\detecThreshSetIdx]| = 0$). For this special case, the corresponding new thresholds cannot be determined directly using \Equation{eq:average_set} and \Equation{eq:nt}. We determine the differences between the new thresholds and the previous thresholds for the subset of thresholds for which this is possible, i.e., $\threshDiff_q[\detecThreshSetIdx] =\detecThresh_q[\detecThreshSetIdx] - \detecThresh_q[\detecThreshSetIdx-1]$ for $q \in \mathcal{Q}\triangleq\{\detecThreshIdx \,|\,|\symbolSampleSet_{\detecThreshIdx+1}[\detecThreshSetIdx]|\neq0 \wedge |\symbolSampleSet_{\detecThreshIdx}[\detecThreshSetIdx]|\neq 0\}$ with $\detecThreshIdx \in \{0, 1, \hdots, \modOrder-2\}$. The thresholds corresponding to the empty sets are subsequently obtained by adjusting the previous threshold value according to the \textit{average} change across all thresholds $\avgThreshDiff[\detecThreshSetIdx] = \frac{1}{|\mathcal{Q}|}\sum_{q \in \mathcal{Q}}\threshDiff_q[\detecThreshSetIdx]$ as $\detecThresh_\detecThreshIdx[\detecThreshSetIdx] = \detecThresh_\detecThreshIdx[\detecThreshSetIdx-1] + \avgThreshDiff[\detecThreshSetIdx]$ for $\detecThreshIdx \notin \mathcal{Q}$. For the rare case, for which no new threshold can be determined, i.e., $|\symbolSampleSet_{\detecThreshIdx}[\detecThreshSetIdx]|=0 \,\vee\, |\symbolSampleSet_{\detecThreshIdx+1}[\detecThreshSetIdx]|=0,\,\forall \detecThreshIdx$, the thresholds retain their previous values, i.e., $\detecThreshSet[\detecThreshSetIdx]=\detecThreshSet[\detecThreshSetIdx-1]$.
%
\scaleSection
\section{Performance Metrics}\label{sec:metric}
\scaleSectionBelow
In this section, we describe the quantitative and qualitative communication-specific metrics employed to evaluate \ac{MC} performance in the proposed testbed.
\vspace{0.1cm}
\scaleSubsection
\subsection{Bit Error Rate and Data Rate}\label{sec:ber_and_data_rate}
\scaleSubsectionBelow
%
We define an \textit{empirical} \ac{BER} for our system. The \ac{BER} is a standard metric for quantifying the performance of any communication scheme. For the bit-to-symbol mapping, we use Gray mapping \cite[p. 100]{proakis2008digital} $G: \{0,1\}^{\log_{2}(M)\nSymb} \rightarrow \{0, \ldots, M - 1\}^{\nSymb}$. Hence, the bit sequences corresponding to symbol sequences $\mathbf{i} \in \{0, 1, \ldots , M - 1\}^{\nSymb}$ and $\hat{\mathbf{i}} \in \{0, 1, \ldots , M - 1\}^{\nSymb}$, i.e., the transmitted and estimated symbol sequences, are obtained as $\mathbf{i}_{G} = G^{-1}(\mathbf{i})$ and $\hat{\mathbf{i}}_G = G^{-1}(\hat{\mathbf{i}})$, respectively. In our case, since the use of pilot symbols is required for calibration of all our detection techniques, the \ac{BER} is obtained as
%
\begin{equation}
    \mathrm{BER} = \frac{\sum_{b = \log_2(M)\nPilots}^{\log_{2}(M)\nSymb - 1} \left|\mathbf{i}_{G}[b] - \hat{\mathbf{i}}_G[b]\right|}{(\nSymb - \nPilots)\log_2(M)}\;.
    \label{eq:BER}
\end{equation}
Note that this equals the Hamming distance between $\mathbf{i}_{G}$ and $\hat{\mathbf{i}}_G$ normalized to the bit sequence length excluding the sequence part corresponding to the pilot symbols $\nPilots$. Thus, the empirically defined \ac{BER} is the number of erroneously demodulated bits scaled by the number of all transmitted bits.
\vspace{0.1cm}
\scaleSubsection
\subsection{Absolute Mean Euclidean Distance}
\scaleSubsectionBelow
%
In some cases the \ac{BER} is not insightful, e.g., if the transmission sequence length is too short for an accurate estimation of the \ac{BER} or if the \ac{MC} system is in the low \ac{BER} regime, where irrespectively of the chosen settings, for the evaluated transmission sequence length, $\textnormal{BER}=0$ is obtained.
For such cases, we introduce the \ac{AMED} metric as follows
\begin{equation}
   \AMED = \min_{i' \neq i''} |\Bar{S}_{i'}[l] - \Bar{S}_{i''}[l]| \;; \qquad \forall i', i'' \in \{0, 1, \ldots, \modOrder - 1\} \;,
    \label{eq:amed}
\end{equation}
where we employ \Equation{eq:average_set} and \Equation{eq:sampleSets}.

As can be seen from \eqref{eq:amed}, the \ac{AMED} is the minimum Euclidean distance between the means of the samples $\detecSample[\symIdx]$ received over $\nWindow$ symbol intervals. Consequently, the \ac{AMED} reflects the extent to which the received samples of the various symbols differ. Compared to the \ac{BER}, the \ac{AMED} provides reliable values even for small numbers of samples, i.e., small $\nWindow$, and can therefore serve as an initial indicator of performance for cases where a \ac{BER} evaluation is not meaningful or not possible.
\vspace{0.1cm}
\scaleSubsection
\subsection{Eye Diagram}
\scaleSubsectionBelow
Eye diagrams are a well-established qualitative method to evaluate a communication system's performance \cite[p. 603]{proakis2008digital}. The use of eye diagrams to evaluate \ac{MC} systems has been introduced in \cite{farsad2017novel}. Plotting the received signal modulo the symbol duration generates the eye pattern. Effects such as \ac{ISI}, various system parameters, including the use of a guard interval, and the effects of varying the modulation order affect the eye pattern and the eye opening, which can provide insights for system design and performance.


%
\scaleSection
\section{Experimental Results and Communication Performance Evaluation}\label{sec:results}
\scaleSectionBelow
\begin{table}[ht!]
\centering
\caption{\textbf{Super Resolution Performance Results.} Our proposed WGAN EEG Spatial Upsampling method significantly outperforms a baseline of Bicubic Interpolation commonly used in EEG upsampling pipelines.}
\label{tab:results}
\resizebox{0.8\linewidth}{!}{%
\begin{tabular}{@{}cccccc@{}}
\toprule
\multirow{2}{*}{\textbf{Dataset}} & \multirow{2}{*}{\textbf{Scale}} & \multicolumn{2}{c}{\textbf{Bicubic}} & \multicolumn{2}{c}{\textbf{WGAN}} \\ \cmidrule(l){3-6} 
                      &   & \textbf{MSE} & \textbf{MAE} & \textbf{MSE}    & \textbf{MAE}   \\
\toprule
\multirow{2}{*}{Val}  & 2 & 3.71E7       & 3.89E3       & \textbf{2.01E3} & \textbf{24.38} \\
                      & 4 & 7.23E7       & 6.42E3       & \textbf{8.53E3} & \textbf{63.83} \\
\midrule
\multirow{2}{*}{Test} & 2 & 3.75E7       & 3.91E3       & \textbf{2.06E3} & \textbf{24.66} \\
                      & 4 & 7.30E7       & 6.45E3       & \textbf{8.68E3} & \textbf{64.39} \\
\bottomrule
\end{tabular}%
}
\end{table}
%
\scaleSection
\section{Testbed Comparison and Data Sharing}\label{sec:comparison_Testbeds}
\scaleSectionBelow
In this section, we compare our testbed to other fluid-based testbeds, and provide information regarding the sharing of our data and code.
%
\scaleSubsection
\subsection{Comparison with Other Testbeds}\label{ssec:comparison}
\scaleSubsectionBelow
%
\begin{table}[!tbp]
    \caption{Overview of Selected Fluid-Based \ac{MC} Testbeds}
    \centering
    \resizebox{1\columnwidth}{!}{
        \begin{tabular}{cccccc}
            \hline\hline
             Signaling Molecule (Biocompatibility) & Data rate (bit/s) & \ac{BER} (\# transmitted bits) & Detection Method & Reference\\
             \hline\hline
             Green Fluorescent Protein Dreiklang (GFPD) (\cmark) & 0.6 & 0 (5370) & Differential Signal + Adaptive Threshold & This paper\\
             \hline
             Superparamagnetic Iron Oxide & & & & \\[-0.2cm]
             Nanoparticles (SPIONs) (\cmark) & 10 & 0.0843 (600) & Convolutional Neural Network &  \cite{bartunik2023development}\\ % transmission efficiency is 160 bit/Ml M = 8 (2000 symbols were transmitted, but only 200 = 600bits were used for testing the detection
             & &  & $\mu$ Link Decoder with Channel Estimation& & \\[-0.2cm]
             Sodium chloride (NaCl) (\cmark) & 5 & 0.002 (100 x 100) & and Sequence Detection & \cite{wang2020understanding} \\
             Colored Ink (\xmark) & 1.5 & 1/312 (312) & Threshold & \cite{wietfeld2024evaluation}\\ % three colors simultaneously 3xlog(2) = 3 bit/symbol, data rate = 1.5 bps, 1/312 BER (3 color molecules) Threshold detector on individual colors
             Acid/Base (\xmark) & 2.63 & 0 (1080) & Recurrent Neural Network & \cite{farsad2017novel}\\ % SVM/RNN based detection
             Sodium Hydroxide (NaOH) (\xmark) & 1/375 & 0.04 (100) & Threshold &  \cite{walter2023real}\\ % threshold detection, 6.25*60, supplementary file
             Hydrogen Chloride (HCl) (\xmark) & 1/27 & 0.022 (1000) & Adaptive Threshold & \cite{khaloopour2019experimental}\\ % 10 cm^3 solution per bit (?) , adaptive threshold detection
             Transfer DNA (tDNA) (\cmark) & 1/120 & 0 (20) & Differential Signal + Threshold & \cite{kuscu2021fabrication}\\ %differential detection with threshold detector
             Glucose ($\textrm{C}_6 \textrm{H}_{12} \textrm{0}_6$) (\cmark) & 2 & 0.05 (654) & Machine Learning & \cite{koo2020deep}\\ % Module based machine learning, some of the measurements had to be discarded as the transmitter was running out of molecules earlier than expected
             \hline\hline
        \end{tabular}
    }
    \label{tab:experiments}
     \vspace{-1.0cm}
\end{table}
Our comparison is focused on the experimentally determined \acp{BER}, the detection schemes used, and the efficiency of data transmission. \Table{tab:experiments} provides a selected overview of fluid-based \ac{MC} testbeds, listing for each testbed the signaling molecule used, the lowest reported \ac{BER}, the data rate achieving this \ac{BER}, the length of the bit sequence transmitted, and the employed detection method.

Providing a fair and still meaningful comparison between existing \ac{MC} testbeds and the proposed testbed is challenging due to some fundamental differences.
In particular, none of the existing works has considered a closed-loop and self-contained topology. As a result, only our testbed is confronted with the challenge of resolving inter-loop \ac{ISI}, offset \ac{ISI}, and permanent \ac{ISI}. In addition, all other testbeds require and use a much larger number of signaling molecules due to their open topology. The resulting low molecule efficiency, which in \cite{bartunik2021increasing} has been defined as the number of bit transmitted per volume of dissolved signaling molecule solution, is a significant challenge in existing testbeds. The issue of low efficiency has been recognized and discussed \cite{bartunik2023development, koo2020deep, lin2024testbed}: In \cite{bartunik2023development}, an efficiency of $160 \,\si{\bit \per \milli\liter}$ was achieved, which is still low compared to our testbed with $90,000/9 \, \si{\bit \per \milli\liter} = 10,000 \,\si{\bit \per \milli\liter}$; in \cite{koo2020deep}, the authors reported that some measurements had to be discarded because the \ac{TX} ran out of molecules earlier than expected; and in \cite{lin2024testbed}, high costs were associated with the use of the testbed, resulting in a maximum evaluated bit sequence length of $8$ bit\footnote{The authors in \cite{lin2024testbed} also reported that for such a short sequence a meaningful estimation of the \ac{BER} was not possible. We believe that this is an example where the discussed alternative performance metrics, such as \ac{AMED} and eye diagrams, are particularly useful.}.

Despite the different topologies and the challenges our testbed has to overcome, we achieve a data rate that is only an order of magnitude lower than the largest data rates reported in the \ac{MC} literature, which are from \cite{bartunik2023development}\footnote{For the sake of completeness, we note that, to the best of our knowledge, the highest \textit{achievable} data rate was reported in \cite{huang2024non}.}. At the same time, our testbed provides very reliable communication, i.e., very low \acp{BER}, while employing a low-complexity detection scheme that involves only differentiation of the received signal and an adaptive threshold detector. Quantifying the computational complexity of the detection schemes listed here is beyond the scope of this paper. However, methods based on machine learning, such as Convolutional Neural Networks \cite{bartunik2023development} and Recurrent Neural Networks \cite{farsad2017novel}, as well as methods that explicitly take into account the memory effects caused by \ac{ISI} \cite{wang2020understanding}, even if efficiently implemented using the Viterbi algorithm, are believed to have comparatively higher computational complexities.
%
\scaleSubsection
\subsection{Data Handling and Code Sharing}\label{ssec:datasharing}
\scaleSubsectionBelow
%
To promote transparency and enable experimental evaluation of new communication algorithms developed by the \ac{MC} community, we publish our experimental data and the Python code for synchronization and detection in Zenodo and a Git repository under the CC BY and the MIT licenses, respectively. When used, the data and/or code can and should be cited using the corresponding Zenodo \ac{DOI} \cite{scherer2025Zenodo}. The link to the Git repository can be found on Zenodo.

%
\scaleSection
\section{Conclusion}\label{sec:conclusion}
\scaleSectionBelow
\section*{Conclusion}
This paper aims to enhance our understanding of the computational complexity of computing various Shapley value variants. We found that for various ML models --- including decision trees, regression tree ensembles, weighted automata, and linear regression --- both local and global interventional and baseline SHAP can be computed in polynomial time under HMM modeled distributions. This extends popular algorithms, such as TreeSHAP, beyond their empirical distributional scope. We also establish strict complexity gaps between the various SHAP variants (baseline, interventional, and conditional) and prove the intractability of computing SHAP for tree ensembles and neural networks in simplified scenarios. Overall, we present SHAP as a versatile framework whose complexity depends on four key factors: \begin{inparaenum}[(i)] \item model type, \item SHAP variant, \item distribution modeling approach, \item and local vs. global explanations\end{inparaenum}. We believe this perspective provides deeper insight into the computational complexity of SHAP, paving the way for future work.




%We believe that our framework provides a more intricate understanding of SHAP computation complexity across different models, distributions, and variants, paving the way for further research.

Our work opens promising directions for future research. First, expanding our computational analysis to other SHAP-related metrics, such as asymmetric SHAP~\citep{frye20} and SAGE~\citep{covert2020understanding}, would be valuable. Additionally, we aim to explore more expressive distribution classes and relaxed assumptions beyond those in Section \ref{sec:tractable} while maintaining tractable SHAP computation. Finally, when exact computation is intractable (Section \ref{sec:intractable}), investigating the approximability of SHAP metrics through approximation and parameterized complexity theory~\citep{downey2012parameterized} is an important direction.

%Our work opens several promising avenues for future research on the computational properties of explainable AI methods, with a particular focus on SHAP. First, it would be interesting to broaden the computational analysis conducted in this work to include other popular SHAP-related metrics in the literature, such as asymmetric SHAP \cite{frye20} and SAGE \cite{covert2020understanding}. Also, in the future, we aim to explore more expressive distribution classes and relaxed distributional assumptions—extending beyond those examined in Section \ref{sec:tractable} —that still yield tractable SHAP computation. Finally, when exact computation proves intractable (Section \ref{sec:intractable}), it is worthwhile to theoretically investigate the question of the approximability of computing the SHAP metrics across various configurations, through the lens of approximation and parametrized complexity theory \cite{arora2009computational}.

%This paper aims to deepen our understanding of the computational complexity involved in obtaining different Shapley value variants. We found that for a variety of ML models, including decision trees, tree ensembles for regression, weighted automata, and linear regression models — computing both local and global interventional and baseline SHAP can be done in polynomial time when distributions are modeled by HMMs. This extends the distributional scope of popular algorithms like TreeSHAP, which is limited to empirical distributions. Additionally, we demonstrate a strict complexity gap between SHAP variants, showing that interventional and baseline SHAP can be strictly easier to compute than conditional SHAP. Despite these positive results, we uncovered intractability for various SHAP variants in neural networks and tree ensembles. Finally, we provided generalized complexity relations across SHAP variants. We believe that our framework offers a deeper understanding of the complexity involved in computing SHAP across various variants, models, distributions, as well as in both local and global computations, laying the groundwork for future research.
%
\vspace*{-0.8cm}
\renewcommand{\baselinestretch}{1.05}
\bibliographystyle{IEEEtran}
\bibliography{literature}
\renewcommand{\baselinestretch}{1.25}
\scaleSubsection
\section*{Acknowledgment}
\scaleSectionBelow
{\small
We thank Prof. Stefan Jakobs (Max Planck Institute for Biophysical Chemistry, Göttingen, Germany) for providing a plasmid encoding Dreiklang. This work was supported by the German Research Foundation (DFG) under Project 290825040 and 509922606.
}

\end{document}