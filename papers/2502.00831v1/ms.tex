\documentclass[journal, 12pt, onecolumn,draftclsnofoot, hidelinks, xcolor=dvipsnames]{IEEEtran}

\pdfoutput=1
\IEEEoverridecommandlockouts

\usepackage{amsmath}
\usepackage[english]{babel}
\usepackage{cite}
\usepackage[shortlabels,inline]{enumitem}
\usepackage{graphicx}
\usepackage{hyperref}
\usepackage{bookmark}
\usepackage{ifthen}
\usepackage[utf8]{inputenc}
\usepackage[locale=US,group-separator={,}]{siunitx}
\usepackage[dvipsnames]{xcolor}
\usepackage[T1]{fontenc}
\usepackage[subnum]{cases}
\usepackage[toc,symbols]{glossaries}
\usepackage{bm}
\usepackage{array}
\usepackage{tabularray}
\usepackage{acro}
\usepackage{microtype}
\usepackage{pifont}


%Maike
\usepackage[utopia, cal=cmcal]{mathdesign} % Teena: Option cal=cmcal
\DeclareSIUnit{\bitsperminute}{\text{bit} \, \text{min}^{-1}}
\newcommand{\xmark}{\ding{55}}
\newcommand{\cmark}{\ding{51}}%

%Louis
\usepackage{subcaption}

\captionsetup[subfigure]{font={small}, skip=1pt, singlelinecheck=false}
\allowdisplaybreaks

\newtheorem{corollary}{Corollary}
\newtheorem{remark}{Remark}
\newtheorem{theo}{Theorem}
\newtheorem{lem}{Lemma}
\newtheorem{defin}{Definition}
\newtheorem{prop}{Proposition}
\newtheorem{assump}{Assumption}
\newtheorem{cond}{Condition}

% macros for referencing figures, tables, equations and so on

\newcommand{\Figure}[1]{Fig.~\ref{#1}}
\newcommand{\Figures}[2]{Figs.~\ref{#1} and~\ref{#2}}
\newcommand{\FigureList}[2]{Figs.~\ref{#1} --~\ref{#2}}

\newcommand{\SubFigure}[2]{Fig.~\ref{#1}~\protect\subref{#2}}
\newcommand{\SubFigures}[3]{Figs.~\ref{#1}~\protect\subref{#2} and \protect\subref{#3}}
\newcommand{\SubFigureRange}[3]{Figure~\ref{#1}(#2)--(#3)}

\newcommand{\FigureStart}[1]{Figure~\ref{#1}}
\newcommand{\SubFigureStart}[2]{Figure~\ref{#1}(#2)}
\newcommand{\SubFiguresStart}[3]{Figures~\ref{#1}(#2) and \ref{#1}(#3)}

\newcommand{\Equation}[1]{\eqref{#1}}
\newcommand{\Equations}[2]{\eqref{#1} and~\eqref{#2}}
\newcommand{\EquationsList}[2]{\eqref{#1} --~\eqref{#2}}
\newcommand{\Table}[1]{Table~\ref{#1}}
\newcommand{\Tables}[2]{Tables~\ref{#1}~and~\ref{#2}}
\newcommand{\Section}[1]{Section~\ref{#1}}
\newcommand{\Subsection}[1]{Subsection~\ref{#1}}
\newcommand{\Subsubsection}[1]{Subsubsection~\ref{#1}}
\newcommand{\Sections}[2]{Sections~\ref{#1}~and~\ref{#2}}
\newcommand{\SectionsThree}[3]{Sections~\ref{#1}~,~\ref{#2},~and~\ref{#3}}
\newcommand{\Sectionss}[2]{Sections~\ref{#1} --\ref{#2}}
\newcommand{\Chapter}[1]{Chapter~\ref{#1}}
\newcommand{\Chapters}[2]{Chapters~\ref{#1}~and~\ref{#2}}
\newcommand{\Appendix}[1]{Appendix~\ref{#1}}
\newcommand{\Algorithm}[1]{Algorithm~\ref{#1}}
\newcommand{\Proposition}[1]{Proposition~\ref{#1}}
\newcommand{\Assumption}[1]{\textit{Assumption~\ref{#1}}}

%%%%%%%%%%%%%%
% Operations %
%%%%%%%%%%%%%%
\renewcommand{\vec}[1]{\mathbf{#1}}

%%%%%%%%%%%%%%%%%%%%%
% System Parameters %
%%%%%%%%%%%%%%%%%%%%%
\newcommand{\radius}[0]{r_{\mathrm{T}}} % tube radius
\newcommand{\con}[0]{C_{\mathrm{GFPD}}} % concentration of GFPD
\newcommand{\volTube}[0]{V_{\mathrm{T}}} % tube volume
\newcommand{\restime}[0]{V_{\mathrm{R}}} % Residence time

\newcommand{\integrationTime}[0]{T_{\mathrm{RX,Int}}}

%%%%%%%%%%%%%
% Variables %
%%%%%%%%%%%%%
\newcommand{\cir}[1]{h(#1)}
\newcommand{\systemResponse}[1]{g(#1)}
\newcommand{\systemSpecificResponse}[1]{f(#1)}
\newcommand{\Nobs}[0]{\bar{N}_{\mathrm{obs}}(d, t)} % avg molecule number after instantaneous release
\newcommand{\pswitch}[0]{p_{\mathrm{S}}^{\mathrm{TX}}} % switching probability at the TX
\newcommand{\Tg}[0]{T_{\mathrm{G}}} % guard interval duration
\newcommand{\volRX}[0]{V_{\mathrm{RX}}} % volume of RX
\newcommand{\veff}[0]{v_{\mathrm{eff}}} % effective velocity
\newcommand{\D}[0]{D} % diffusion coefficient
\newcommand{\rect}[2]{\mathrm{rect}_{#1}(#2)} % rect signal
\newcommand{\Tleff}{L_{\mathrm{T, eff}}}% effective tube length
\newcommand{\Tres}{\bar{T}_{\mathrm{res}}}% average reservoir time
\newcommand{\scalingFactor}{\bar{\rho}_{\mathrm{res}}}% scaling factor of what portion of switched molecules re-enters the system before having switched back to the equilibrium state
\newcommand{\bitest}[1]{\hat{b}[#1]} % estimated bit at time #1
\newcommand{\symest}[1]{\hat{s}_{m[#1]}} % estimated bit at time #1
\newcommand{\sym}[1]{s_{m[#1]}} % symbol at time #1
\newcommand{\bit}[1]{b[#1]} % bit at time #1
\newcommand{\recDiff}[1]{d'[#1]} % received signal at time #1
\newcommand{\coh}[0]{n_{\mathrm{c}}} % coherence time
\newcommand{\nTX}[0]{N_{1}} % num of switched molecules
\newcommand{\nRX}[0]{N_{0}} % typical number of molecules at the RX
\newcommand{\volTX}[0]{V_{\mathrm{TX}}} % volume of TX
\newcommand{\disTX}[0]{d_{\mathrm{TX,RX}}}
\newcommand{\wei}[0]{m_{\mathrm{GFPD}}} % weight of GFPD

\newcommand{\Nobstx}[0]{\bar{N}^{\mathrm{TX}}_{\mathrm{obs}}(d, t)} % avg number of switched off molecules at RX
\newcommand{\switching}[0]{a_{\mathrm{D}}} % switching function
\newcommand{\switchingc}[0]{\kappa} % switching constant

\newcommand{\minED}[0]{d_{\mathrm{min}}}
\newcommand{\minSED}[0]{d^2_{\mathrm{min}}}


% Detection scheme
\newcommand{\threshR}[0]{\xi_{\mathrm{R}}} % regular threshold
\newcommand{\threshD}[0]{\xi_{\mathrm{D}}} % differential based threshold
\newcommand{\threshA}[0]{\xi_{\mathrm{A}}} % adaptive threshold
\newcommand{\nSymb}[0]{N_{\mathrm{Sym}}} % number of symbols
\newcommand{\nBits}[0]{N_{\mathrm{Bit}}} % number of bits
\newcommand{\idxPilots}[0]{\mathcal{K}^{\mathrm{P}}} % set of indices belonging to pilot symbols
\newcommand{\idxSym}[1]{\mathcal{K}^{\mathrm{P}}_{#1}} % set of bit #1s in pilot sequence

%%%%%%%%%
% Units %
%%%%%%%%%
\newcommand{\s}[0]{\text{s}} % seconds

%%%%%%%%%%%%%%%%%%%%%%%%%%%%%%%%%%%%%%%%%%%%%%%%%%%%%%%%%%%%%%%%%%%%%%%%%%%%%%%%%%%%%%%%%%%%%%%%%
%%%%%%%%%%%%%
% Operators %
%%%%%%%%%%%%%
\newcommand{\transpose}[1]{#1^{\mathrm{T}}}
\DeclareMathOperator*{\argmax}{\mathrm{arg\,max}}
\DeclareMathOperator*{\argmin}{\mathrm{arg\,min}}

%%%%%%%%%%%%%
% Indices %
%%%%%%%%%%%%%

%General
\newcommand{\n}[0]{n} % time index
\newcommand{\tn}[0]{t_n} % time index
\newcommand{\altTimeIdx}{\kappa} % alternative time index
\newcommand{\symIdx}[0]{k} % symbol index

%Transmission start detection
\newcommand{\trainIdx}[0]{\tau}

%Detection
\newcommand{\detecThreshIdx}[0]{j}
\newcommand{\detecThreshSetIdx}[0]{l}


%%%%%%%%%%%%%%%%%%%
% Variables %
%%%%%%%%%%%%%%%%%%%

%General
\newcommand{\symVar}[0]{i}
\newcommand{\Ts}[0]{T_{\mathrm{S}}} % symbol duration
\newcommand{\Ti}[0]{T_{\mathrm{I}}} % impulse duration
\newcommand{\modOrder}[0]{M} % order of modulation
\newcommand{\SIR}[0]{s}
\newcommand{\SR}[0]{s_{\mathrm{r}}}
\newcommand{\avgSR}[0]{\bar{s}_{\mathrm{r}}}
\newcommand{\recSig}[0]{r}
\newcommand{\sampleInt}[0]{\Delta t}


%Modulation
\newcommand{\ratioInten}[0]{\rho}
\newcommand{\txInten}[0]{I^{\mathrm{TX}}} % TX intensity
\newcommand{\maxTxInten}[0]{I^{\mathrm{TX}}_{\mathrm{max}}} % max TX intensity

%Transmission start detection
\newcommand{\trainLength}[0]{N_\mathrm{T}}
\newcommand{\trainSet}[0]{\mathcal{K}}
\newcommand{\trainMean}[0] {\hat{\mu}}
\newcommand{\trainVariance}[0]{\hat{\sigma}^2}
\newcommand{\trainThresh}[0]{\xi_\mathrm{T}}
\newcommand{\pFA}[0]{p_\mathrm{FA}} % False probability
\newcommand{\tts}[0]{t_\mathrm{TS}} % Time of detected transmission start

%Synchronization
\newcommand{\ts}[0]{t_\mathrm{s}}
\newcommand{\tsEst}[0]{\hat{t}_\mathrm{s}}
\newcommand{\tsInit}[0]{\Tilde{t}_\mathrm{s}}
\newcommand{\avgSIR}[0]{\bar{\SIR}}
\newcommand{\searchInt}[0]{\mathcal{T}^{\mathrm{search}}}
\newcommand{\searchRadius}[0]{r}
\newcommand{\hypothesisSet}[0]{\mathcal{H}}
\newcommand{\syncMetric}[0]{\Lambda}
\newcommand{\filterLength}[0]{\mathrm{N}}

\newcommand{\corrMarker}[0]{\mathrm{C}}
\newcommand{\diffMarker}[0]{\mathrm{D}}
\newcommand{\blindMarker}[0]{\mathrm{B}}
\newcommand{\corr}[0]{\mathrm{corr}}
\newcommand{\dcorr}[0]{\mathrm{dcorr}}

\newcommand{\modRecSig}[0]{\recSig'}
\newcommand{\filter}[0]{g}
\newcommand{\corrFilter}[0]{\filter_\corrMarker}
\newcommand{\vecModRecSig}[0]{\vec{\recSig}'}
\newcommand{\vecCorrFilter}[0]{\vec{\filter}_\corrMarker}
\newcommand{\corrMetric}[0]{\syncMetric^\corr}
\newcommand{\corrFunc}[0]{\varphi}
\newcommand{\tsCorr}[0]{\tsEst^\corr}

\newcommand{\diffRecSig}[0]{\recSig_{\diffMarker}}
\newcommand{\vecDiffRecSig}[0]{\vec{\recSig}_{\diffMarker}}
\newcommand{\diffCorrFilter}[0]{\filter_\diffMarker}
\newcommand{\vecDiffCorrFilter}[0]{\vec{\filter}_\diffMarker}
\newcommand{\dcorrMetric}[0]{\syncMetric^\dcorr}
\newcommand{\dcorrFunc}[0]{\varphi_\diffMarker}
\newcommand{\tsDcorr}[0]{\tsEst^\dcorr}

%Blind filters
\newcommand{\blindCorrFilter}[0]{g_\corrMarker^{\blindMarker}}
\newcommand{\blindDiffCorrFilter}[0]{\filter_\diffMarker^\blindMarker}
\newcommand{\vecBlindCorrFilter}[0]{\vecCorrFilter^\blindMarker}
\newcommand{\vecBlindDiffCorrFilter}[0]{\vecDiffCorrFilter^\blindMarker}

%Detection
\newcommand{\detecSample}[0]{d}
\newcommand{\estSym}[0]{\hat{{\symVar}}}
\newcommand{\detecThresh}[0]{\xi}
\newcommand{\detecThreshSet}[0]{\Xi}
\newcommand{\nPilots}[0]{P} % number of pilot symbols
\newcommand{\nCoherence}[0]{F} % number of symbols the threshold stays fixed
\newcommand{\nWindow}[0]{W} % number of symbols used for reevaluation of thresholds
\newcommand{\nSkip}[0]{\chi} % number of symbols used for reevaluation of thresholds


\newcommand{\symbolSampleSet}[0]{\mathcal{S}}
\newcommand{\avgSymbolSampleSet}[0]{\bar{S}}
\newcommand{\threshDiff}[0]{\delta}
\newcommand{\avgThreshDiff}[0]{\Delta}

\newcommand{\AMED}[0]{\mathrm{AMED}}

\newacronym{rl}{RL}{Reinforcement Learning}
\newacronym{drl}{DRL}{Deep Reinforcement Learning}
\newacronym{mdp}{MDP}{Markov Decision Process}
\newacronym{ppo}{PPO}{Proximal Policy Optimization}
\newacronym{sac}{SAC}{Soft Actor-Critic}
\newacronym{epvf}{EPVF}{Explicit Policy-conditioned Value Function}
\newacronym{unf}{UNF}{Universal Neural Functional}

\makeatletter
\long\def\@makecaption#1#2{\ifx\@captype\@IEEEtablestring%
    \footnotesize\begin{center}{\normalfont\footnotesize #1}\\
        {\normalfont\footnotesize\scshape #2}\end{center}%
    \@IEEEtablecaptionsepspace
    \else
    \@IEEEfigurecaptionsepspace
    \setbox\@tempboxa\hbox{\normalfont\footnotesize {#1.}~~ #2}%
    \ifdim \wd\@tempboxa >\hsize%
    \setbox\@tempboxa\hbox{\normalfont\footnotesize {#1.}~~ }%
    \parbox[t]{\hsize}{\normalfont\footnotesize \noindent\unhbox\@tempboxa#2}%
    \else
    \hbox to\hsize{\normalfont\footnotesize\hfil\box\@tempboxa\hfil}\fi\fi}
\makeatother

\renewcommand{\baselinestretch}{1.25}
\setlength{\skip\footins}{7pt}
\addtolength{\topmargin}{+0.07cm}
\addtolength{\textwidth}{0.22cm}
\addtolength{\textheight}{0.22cm}

% Increase row spacing in tables
\renewcommand{\arraystretch}{1}
\captionsetup{belowskip=0pt,aboveskip=1pt}

\newcommand{\scaleSection}{\vspace*{-0.35cm}}
\newcommand{\scaleSubsection}{\vspace*{-0.52cm}}
\newcommand{\scaleSubsubsection}{\vspace*{-0.01cm}}
\newcommand{\scaleSectionBelow}{\vspace*{-0.1cm}}
\newcommand{\scaleSubsectionBelow}{\vspace*{-0.02cm}}
\newcommand{\scaleSubsubsectionBelow}{\vspace*{-0.01cm}}
\newcommand{\scaleAlign}{\vspace*{-0.1cm}}


\newcommand\todo[1]{\textcolor{red}{#1}}
%\newcommand\todo[1]{\textcolor{red}{}}

\begin{document}
%\tableofcontents
\bstctlcite{IEEEexample:BSTcontrol}
\vspace{-0.4cm}
\title{Closed-Loop Long-Term Experimental Molecular Communication System}
\author{\vspace{-0.3cm}
\IEEEauthorblockN{Maike Scherer$^\star$, Lukas Brand$^\star$, Louis Wolf, Teena tom Dieck, Maximilian Schäfer, Sebastian Lotter, Andreas Burkovski, Heinrich Sticht, Robert Schober, and Kathrin Castiglione\\
\thanks{$^\star$ Co-first authors.\\This paper was presented in part at the IEEE International Conference on Communications, 2024 \cite{brand2024closed}.}}\IEEEauthorblockA{\small Friedrich-Alexander-Universit\"at Erlangen-N\"urnberg, Erlangen, Germany}}
%
\maketitle
\vspace{-1.9cm}
\begin{abstract}
\vspace*{-0.3cm}
\begin{abstract}  
Test time scaling is currently one of the most active research areas that shows promise after training time scaling has reached its limits.
Deep-thinking (DT) models are a class of recurrent models that can perform easy-to-hard generalization by assigning more compute to harder test samples.
However, due to their inability to determine the complexity of a test sample, DT models have to use a large amount of computation for both easy and hard test samples.
Excessive test time computation is wasteful and can cause the ``overthinking'' problem where more test time computation leads to worse results.
In this paper, we introduce a test time training method for determining the optimal amount of computation needed for each sample during test time.
We also propose Conv-LiGRU, a novel recurrent architecture for efficient and robust visual reasoning. 
Extensive experiments demonstrate that Conv-LiGRU is more stable than DT, effectively mitigates the ``overthinking'' phenomenon, and achieves superior accuracy.
\end{abstract}  
\end{abstract}
\setlength{\belowdisplayskip}{2pt}
\setlength{\belowdisplayshortskip}{2pt}
\vspace*{-0.4cm}
\acresetall
%
\scaleSection
\section{Introduction}\label{sec:intro}
\scaleSectionBelow
\section{Introduction}
\label{sec:introduction}
The business processes of organizations are experiencing ever-increasing complexity due to the large amount of data, high number of users, and high-tech devices involved \cite{martin2021pmopportunitieschallenges, beerepoot2023biggestbpmproblems}. This complexity may cause business processes to deviate from normal control flow due to unforeseen and disruptive anomalies \cite{adams2023proceddsriftdetection}. These control-flow anomalies manifest as unknown, skipped, and wrongly-ordered activities in the traces of event logs monitored from the execution of business processes \cite{ko2023adsystematicreview}. For the sake of clarity, let us consider an illustrative example of such anomalies. Figure \ref{FP_ANOMALIES} shows a so-called event log footprint, which captures the control flow relations of four activities of a hypothetical event log. In particular, this footprint captures the control-flow relations between activities \texttt{a}, \texttt{b}, \texttt{c} and \texttt{d}. These are the causal ($\rightarrow$) relation, concurrent ($\parallel$) relation, and other ($\#$) relations such as exclusivity or non-local dependency \cite{aalst2022pmhandbook}. In addition, on the right are six traces, of which five exhibit skipped, wrongly-ordered and unknown control-flow anomalies. For example, $\langle$\texttt{a b d}$\rangle$ has a skipped activity, which is \texttt{c}. Because of this skipped activity, the control-flow relation \texttt{b}$\,\#\,$\texttt{d} is violated, since \texttt{d} directly follows \texttt{b} in the anomalous trace.
\begin{figure}[!t]
\centering
\includegraphics[width=0.9\columnwidth]{images/FP_ANOMALIES.png}
\caption{An example event log footprint with six traces, of which five exhibit control-flow anomalies.}
\label{FP_ANOMALIES}
\end{figure}

\subsection{Control-flow anomaly detection}
Control-flow anomaly detection techniques aim to characterize the normal control flow from event logs and verify whether these deviations occur in new event logs \cite{ko2023adsystematicreview}. To develop control-flow anomaly detection techniques, \revision{process mining} has seen widespread adoption owing to process discovery and \revision{conformance checking}. On the one hand, process discovery is a set of algorithms that encode control-flow relations as a set of model elements and constraints according to a given modeling formalism \cite{aalst2022pmhandbook}; hereafter, we refer to the Petri net, a widespread modeling formalism. On the other hand, \revision{conformance checking} is an explainable set of algorithms that allows linking any deviations with the reference Petri net and providing the fitness measure, namely a measure of how much the Petri net fits the new event log \cite{aalst2022pmhandbook}. Many control-flow anomaly detection techniques based on \revision{conformance checking} (hereafter, \revision{conformance checking}-based techniques) use the fitness measure to determine whether an event log is anomalous \cite{bezerra2009pmad, bezerra2013adlogspais, myers2018icsadpm, pecchia2020applicationfailuresanalysispm}. 

The scientific literature also includes many \revision{conformance checking}-independent techniques for control-flow anomaly detection that combine specific types of trace encodings with machine/deep learning \cite{ko2023adsystematicreview, tavares2023pmtraceencoding}. Whereas these techniques are very effective, their explainability is challenging due to both the type of trace encoding employed and the machine/deep learning model used \cite{rawal2022trustworthyaiadvances,li2023explainablead}. Hence, in the following, we focus on the shortcomings of \revision{conformance checking}-based techniques to investigate whether it is possible to support the development of competitive control-flow anomaly detection techniques while maintaining the explainable nature of \revision{conformance checking}.
\begin{figure}[!t]
\centering
\includegraphics[width=\columnwidth]{images/HIGH_LEVEL_VIEW.png}
\caption{A high-level view of the proposed framework for combining \revision{process mining}-based feature extraction with dimensionality reduction for control-flow anomaly detection.}
\label{HIGH_LEVEL_VIEW}
\end{figure}

\subsection{Shortcomings of \revision{conformance checking}-based techniques}
Unfortunately, the detection effectiveness of \revision{conformance checking}-based techniques is affected by noisy data and low-quality Petri nets, which may be due to human errors in the modeling process or representational bias of process discovery algorithms \cite{bezerra2013adlogspais, pecchia2020applicationfailuresanalysispm, aalst2016pm}. Specifically, on the one hand, noisy data may introduce infrequent and deceptive control-flow relations that may result in inconsistent fitness measures, whereas, on the other hand, checking event logs against a low-quality Petri net could lead to an unreliable distribution of fitness measures. Nonetheless, such Petri nets can still be used as references to obtain insightful information for \revision{process mining}-based feature extraction, supporting the development of competitive and explainable \revision{conformance checking}-based techniques for control-flow anomaly detection despite the problems above. For example, a few works outline that token-based \revision{conformance checking} can be used for \revision{process mining}-based feature extraction to build tabular data and develop effective \revision{conformance checking}-based techniques for control-flow anomaly detection \cite{singh2022lapmsh, debenedictis2023dtadiiot}. However, to the best of our knowledge, the scientific literature lacks a structured proposal for \revision{process mining}-based feature extraction using the state-of-the-art \revision{conformance checking} variant, namely alignment-based \revision{conformance checking}.

\subsection{Contributions}
We propose a novel \revision{process mining}-based feature extraction approach with alignment-based \revision{conformance checking}. This variant aligns the deviating control flow with a reference Petri net; the resulting alignment can be inspected to extract additional statistics such as the number of times a given activity caused mismatches \cite{aalst2022pmhandbook}. We integrate this approach into a flexible and explainable framework for developing techniques for control-flow anomaly detection. The framework combines \revision{process mining}-based feature extraction and dimensionality reduction to handle high-dimensional feature sets, achieve detection effectiveness, and support explainability. Notably, in addition to our proposed \revision{process mining}-based feature extraction approach, the framework allows employing other approaches, enabling a fair comparison of multiple \revision{conformance checking}-based and \revision{conformance checking}-independent techniques for control-flow anomaly detection. Figure \ref{HIGH_LEVEL_VIEW} shows a high-level view of the framework. Business processes are monitored, and event logs obtained from the database of information systems. Subsequently, \revision{process mining}-based feature extraction is applied to these event logs and tabular data input to dimensionality reduction to identify control-flow anomalies. We apply several \revision{conformance checking}-based and \revision{conformance checking}-independent framework techniques to publicly available datasets, simulated data of a case study from railways, and real-world data of a case study from healthcare. We show that the framework techniques implementing our approach outperform the baseline \revision{conformance checking}-based techniques while maintaining the explainable nature of \revision{conformance checking}.

In summary, the contributions of this paper are as follows.
\begin{itemize}
    \item{
        A novel \revision{process mining}-based feature extraction approach to support the development of competitive and explainable \revision{conformance checking}-based techniques for control-flow anomaly detection.
    }
    \item{
        A flexible and explainable framework for developing techniques for control-flow anomaly detection using \revision{process mining}-based feature extraction and dimensionality reduction.
    }
    \item{
        Application to synthetic and real-world datasets of several \revision{conformance checking}-based and \revision{conformance checking}-independent framework techniques, evaluating their detection effectiveness and explainability.
    }
\end{itemize}

The rest of the paper is organized as follows.
\begin{itemize}
    \item Section \ref{sec:related_work} reviews the existing techniques for control-flow anomaly detection, categorizing them into \revision{conformance checking}-based and \revision{conformance checking}-independent techniques.
    \item Section \ref{sec:abccfe} provides the preliminaries of \revision{process mining} to establish the notation used throughout the paper, and delves into the details of the proposed \revision{process mining}-based feature extraction approach with alignment-based \revision{conformance checking}.
    \item Section \ref{sec:framework} describes the framework for developing \revision{conformance checking}-based and \revision{conformance checking}-independent techniques for control-flow anomaly detection that combine \revision{process mining}-based feature extraction and dimensionality reduction.
    \item Section \ref{sec:evaluation} presents the experiments conducted with multiple framework and baseline techniques using data from publicly available datasets and case studies.
    \item Section \ref{sec:conclusions} draws the conclusions and presents future work.
\end{itemize}
%
\scaleSection
\section{Testbed Overview}\label{sec:experiment}
\scaleSectionBelow
\section{Experiments}
\label{sec:exp}
Following the settings in Section \ref{sec:existing}, we evaluate \textit{NovelSum}'s correlation with the fine-tuned model performance across 53 IT datasets and compare it with previous diversity metrics. Additionally, we conduct a correlation analysis using Qwen-2.5-7B \cite{yang2024qwen2} as the backbone model, alongside previous LLaMA-3-8B experiments, to further demonstrate the metric's effectiveness across different scenarios. Qwen is used for both instruction tuning and deriving semantic embeddings. Due to resource constraints, we run each strategy on Qwen for two rounds, resulting in 25 datasets. 

\subsection{Main Results}

\begin{table*}[!t]
    \centering
    \resizebox{\linewidth}{!}{
    \begin{tabular}{lcccccccccc}
    \toprule
    \multirow{3}*{\textbf{Diversity Metrics}} & \multicolumn{10}{c}{\textbf{Data Selection Strategies}} \\
    \cmidrule(lr){2-11}
    & \multirow{2}*{\textbf{K-means}} & \multirow{2}*{\vtop{\hbox{\textbf{K-Center}}\vspace{1mm}\hbox{\textbf{-Greedy}}}}  & \multirow{2}*{\textbf{QDIT}} & \multirow{2}*{\vtop{\hbox{\textbf{Repr}}\vspace{1mm}\hbox{\textbf{Filter}}}} & \multicolumn{5}{c}{\textbf{Random}} & \multirow{2}{*}{\textbf{Duplicate}} \\ 
    \cmidrule(lr){6-10}
    & & & & & \textbf{$\mathcal{X}^{all}$} & ShareGPT & WizardLM & Alpaca & Dolly &  \\
    \midrule
    \rowcolor{gray!15} \multicolumn{11}{c}{\textit{LLaMA-3-8B}} \\
    Facility Loc. $_{\times10^5}$ & \cellcolor{BLUE!40} 2.99 & \cellcolor{ORANGE!10} 2.73 & \cellcolor{BLUE!40} 2.99 & \cellcolor{BLUE!20} 2.86 & \cellcolor{BLUE!40} 2.99 & \cellcolor{BLUE!0} 2.83 & \cellcolor{BLUE!30} 2.88 & \cellcolor{BLUE!0} 2.83 & \cellcolor{ORANGE!20} 2.59 & \cellcolor{ORANGE!30} 2.52 \\    
    DistSum$_{cosine}$  & \cellcolor{BLUE!30} 0.648 & \cellcolor{BLUE!60} 0.746 & \cellcolor{BLUE!0} 0.629 & \cellcolor{BLUE!50} 0.703 & \cellcolor{BLUE!10} 0.634 & \cellcolor{BLUE!40} 0.656 & \cellcolor{ORANGE!30} 0.578 & \cellcolor{ORANGE!10} 0.605 & \cellcolor{ORANGE!20} 0.603 & \cellcolor{BLUE!10} 0.634 \\
    Vendi Score $_{\times10^7}$ & \cellcolor{BLUE!30} 1.70 & \cellcolor{BLUE!60} 2.53 & \cellcolor{BLUE!10} 1.59 & \cellcolor{BLUE!50} 2.23 & \cellcolor{BLUE!20} 1.61 & \cellcolor{BLUE!30} 1.70 & \cellcolor{ORANGE!10} 1.44 & \cellcolor{ORANGE!20} 1.32 & \cellcolor{ORANGE!10} 1.44 & \cellcolor{ORANGE!30} 0.05 \\
    \textbf{NovelSum (Ours)} & \cellcolor{BLUE!60} 0.693 & \cellcolor{BLUE!50} 0.687 & \cellcolor{BLUE!30} 0.673 & \cellcolor{BLUE!20} 0.671 & \cellcolor{BLUE!40} 0.675 & \cellcolor{BLUE!10} 0.628 & \cellcolor{BLUE!0} 0.591 & \cellcolor{ORANGE!10} 0.572 & \cellcolor{ORANGE!20} 0.50 & \cellcolor{ORANGE!30} 0.461 \\
    \midrule    
    \textbf{Model Performance} & \cellcolor{BLUE!60}1.32 & \cellcolor{BLUE!50}1.31 & \cellcolor{BLUE!40}1.25 & \cellcolor{BLUE!30}1.05 & \cellcolor{BLUE!20}1.20 & \cellcolor{BLUE!10}0.83 & \cellcolor{BLUE!0}0.72 & \cellcolor{ORANGE!10}0.07 & \cellcolor{ORANGE!20}-0.14 & \cellcolor{ORANGE!30}-1.35 \\
    \midrule
    \midrule
    \rowcolor{gray!15} \multicolumn{11}{c}{\textit{Qwen-2.5-7B}} \\
    Facility Loc. $_{\times10^5}$ & \cellcolor{BLUE!40} 3.54 & \cellcolor{ORANGE!30} 3.42 & \cellcolor{BLUE!40} 3.54 & \cellcolor{ORANGE!20} 3.46 & \cellcolor{BLUE!40} 3.54 & \cellcolor{BLUE!30} 3.51 & \cellcolor{BLUE!10} 3.50 & \cellcolor{BLUE!10} 3.50 & \cellcolor{ORANGE!20} 3.46 & \cellcolor{BLUE!0} 3.48 \\ 
    DistSum$_{cosine}$ & \cellcolor{BLUE!30} 0.260 & \cellcolor{BLUE!60} 0.440 & \cellcolor{BLUE!0} 0.223 & \cellcolor{BLUE!50} 0.421 & \cellcolor{BLUE!10} 0.230 & \cellcolor{BLUE!40} 0.285 & \cellcolor{ORANGE!20} 0.211 & \cellcolor{ORANGE!30} 0.189 & \cellcolor{ORANGE!10} 0.221 & \cellcolor{BLUE!20} 0.243 \\
    Vendi Score $_{\times10^6}$ & \cellcolor{ORANGE!10} 1.60 & \cellcolor{BLUE!40} 3.09 & \cellcolor{BLUE!10} 2.60 & \cellcolor{BLUE!60} 7.15 & \cellcolor{ORANGE!20} 1.41 & \cellcolor{BLUE!50} 3.36 & \cellcolor{BLUE!20} 2.65 & \cellcolor{BLUE!0} 1.89 & \cellcolor{BLUE!30} 3.04 & \cellcolor{ORANGE!30} 0.20 \\
    \textbf{NovelSum (Ours)}  & \cellcolor{BLUE!40} 0.440 & \cellcolor{BLUE!60} 0.505 & \cellcolor{BLUE!20} 0.403 & \cellcolor{BLUE!50} 0.495 & \cellcolor{BLUE!30} 0.408 & \cellcolor{BLUE!10} 0.392 & \cellcolor{BLUE!0} 0.349 & \cellcolor{ORANGE!10} 0.336 & \cellcolor{ORANGE!20} 0.320 & \cellcolor{ORANGE!30} 0.309 \\
    \midrule
    \textbf{Model Performance} & \cellcolor{BLUE!30} 1.06 & \cellcolor{BLUE!60} 1.45 & \cellcolor{BLUE!40} 1.23 & \cellcolor{BLUE!50} 1.35 & \cellcolor{BLUE!20} 0.87 & \cellcolor{BLUE!10} 0.07 & \cellcolor{BLUE!0} -0.08 & \cellcolor{ORANGE!10} -0.38 & \cellcolor{ORANGE!30} -0.49 & \cellcolor{ORANGE!20} -0.43 \\
    \bottomrule
    \end{tabular}
    }
    \caption{Measuring the diversity of datasets selected by different strategies using \textit{NovelSum} and baseline metrics. Fine-tuned model performances (Eq. \ref{eq:perf}), based on MT-bench and AlpacaEval, are also included for cross reference. Darker \colorbox{BLUE!60}{blue} shades indicate higher values for each metric, while darker \colorbox{ORANGE!30}{orange} shades indicate lower values. While data selection strategies vary in performance on LLaMA-3-8B and Qwen-2.5-7B, \textit{NovelSum} consistently shows a stronger correlation with model performance than other metrics. More results are provided in Appendix \ref{app:results}.}
    \label{tbl:main}
    \vspace{-4mm}
\end{table*}


\begin{table}[t!]
\centering
\resizebox{\linewidth}{!}{
\begin{tabular}{lcccc}
\toprule
\multirow{2}*{\textbf{Diversity Metrics}} & \multicolumn{3}{c}{\textbf{LLaMA}} & \textbf{Qwen}\\
\cmidrule(lr){2-4} \cmidrule(lr){5-5} 
& \textbf{Pearson} & \textbf{Spearman} & \textbf{Avg.} & \textbf{Avg.} \\
\midrule
TTR & -0.38 & -0.16 & -0.27 & -0.30 \\
vocd-D & -0.43 & -0.17 & -0.30 & -0.31 \\
\midrule
Facility Loc. & 0.86 & 0.69 & 0.77 & 0.08 \\
Entropy & 0.93 & 0.80 & 0.86 & 0.63 \\
\midrule
LDD & 0.61 & 0.75 & 0.68 & 0.60 \\
KNN Distance & 0.59 & 0.80 & 0.70 & 0.67 \\
DistSum$_{cosine}$ & 0.85 & 0.67 & 0.76 & 0.51 \\
Vendi Score & 0.70 & 0.85 & 0.78 & 0.60 \\
DistSum$_{L2}$ & 0.86 & 0.76 & 0.81 & 0.51 \\
Cluster Inertia & 0.81 & 0.85 & 0.83 & 0.76 \\
Radius & 0.87 & 0.81 & 0.84 & 0.48 \\
\midrule
NovelSum & \textbf{0.98} & \textbf{0.95} & \textbf{0.97} & \textbf{0.90} \\
\bottomrule
\end{tabular}
}
\caption{Correlations between different metrics and model performance on LLaMA-3-8B and Qwen-2.5-7B.  “Avg.” denotes the average correlation (Eq. \ref{eq:cor}).}
\label{tbl:correlations}
\vspace{-2mm}
\end{table}

\paragraph{\textit{NovelSum} consistently achieves state-of-the-art correlation with model performance across various data selection strategies, backbone LLMs, and correlation measures.}
Table \ref{tbl:main} presents diversity measurement results on datasets constructed by mainstream data selection methods (based on $\mathcal{X}^{all}$), random selection from various sources, and duplicated samples (with only $m=100$ unique samples). 
Results from multiple runs are averaged for each strategy.
Although these strategies yield varying performance rankings across base models, \textit{NovelSum} consistently tracks changes in IT performance by accurately measuring dataset diversity. For instance, K-means achieves the best performance on LLaMA with the highest NovelSum score, while K-Center-Greedy excels on Qwen, also correlating with the highest NovelSum. Table \ref{tbl:correlations} shows the correlation coefficients between various metrics and model performance for both LLaMA and Qwen experiments, where \textit{NovelSum} achieves state-of-the-art correlation across different models and measures.

\paragraph{\textit{NovelSum} can provide valuable guidance for data engineering practices.}
As a reliable indicator of data diversity, \textit{NovelSum} can assess diversity at both the dataset and sample levels, directly guiding data selection and construction decisions. For example, Table \ref{tbl:main} shows that the combined data source $\mathcal{X}^{all}$ is a better choice for sampling diverse IT data than other sources. Moreover, \textit{NovelSum} can offer insights through comparative analyses, such as: (1) ShareGPT, which collects data from real internet users, exhibits greater diversity than Dolly, which relies on company employees, suggesting that IT samples from diverse sources enhance dataset diversity \cite{wang2024diversity-logD}; (2) In LLaMA experiments, random selection can outperform some mainstream strategies, aligning with prior work \cite{xia2024rethinking,diddee2024chasing}, highlighting gaps in current data selection methods for optimizing diversity.



\subsection{Ablation Study}


\textit{NovelSum} involves several flexible hyperparameters and variations. In our main experiments, \textit{NovelSum} uses cosine distance to compute $d(x_i, x_j)$ in Eq. \ref{eq:dad}. We set $\alpha = 1$, $\beta = 0.5$, and $K = 10$ nearest neighbors in Eq. \ref{eq:pws} and \ref{eq:dad}. Here, we conduct an ablation study to investigate the impact of these settings based on LLaMA-3-8B.

\begin{table}[ht!]
\centering
\resizebox{\linewidth}{!}{
\begin{tabular}{lccc}
\toprule
\textbf{Variants} & \textbf{Pearson} & \textbf{Spearman} & \textbf{Avg.} \\
\midrule
NovelSum & 0.98 & 0.96 & 0.97 \\
\midrule
\hspace{0.10cm} - Use $L2$ distance & 0.97 & 0.83 & 0.90\textsubscript{↓ 0.08} \\
\hspace{0.10cm} - $K=20$ & 0.98 & 0.96 & 0.97\textsubscript{↓ 0.00} \\
\hspace{0.10cm} - $\alpha=0$ (w/o proximity) & 0.79 & 0.31 & 0.55\textsubscript{↓ 0.42} \\
\hspace{0.10cm} - $\alpha=2$ & 0.73 & 0.88 & 0.81\textsubscript{↓ 0.16} \\
\hspace{0.10cm} - $\beta=0$ (w/o density) & 0.92 & 0.89 & 0.91\textsubscript{↓ 0.07} \\
\hspace{0.10cm} - $\beta=1$ & 0.90 & 0.62 & 0.76\textsubscript{↓ 0.21} \\
\bottomrule
\end{tabular}
}
\caption{Ablation Study for \textit{NovelSum}.}
\label{tbl:ablation}
\vspace{-2mm}
\end{table}

In Table \ref{tbl:ablation}, $\alpha=0$ removes the proximity weights, and $\beta=0$ eliminates the density multiplier. We observe that both $\alpha=0$ and $\beta=0$ significantly weaken the correlation, validating the benefits of the proximity-weighted sum and density-aware distance. Additionally, improper values for $\alpha$ and $\beta$ greatly reduce the metric's reliability, highlighting that \textit{NovelSum} strikes a delicate balance between distances and distribution. Replacing cosine distance with Euclidean distance and using more neighbors for density approximation have minimal impact, particularly on Pearson's correlation, demonstrating \textit{NovelSum}'s robustness to different distance measures.






%
\scaleSection
\section{Physical Characterization of the Testbed}\label{sec:characterization}
\scaleSectionBelow
In this section, we discuss the physical effects observed in the testbed, such as the involved fluid mechanics and different forms of \ac{ISI} present including those that occur exclusively in closed-loop systems.
%
\scaleSubsection
\subsection{Signaling Molecule Propagation}\label{subsec:propagation}
\scaleSubsectionBelow
%
In our testbed, the propagation of the \ac{GFPD} signaling molecules is mainly affected by advection, facilitated by the bulk background flow of the buffer liquid. In the following, we analyze the flow characteristics and its parameters in the testbed.

When a fluid of kinematic viscosity $\nu$ flows with average flow speed $v_{\mathrm{eff}}$ through a pipe of radius $r_\mathrm{T}$, either laminar flow or turbulent flow occurs. 
Which type of flow is dominant can be predicted by the Reynolds number $\text{Re} = \frac{2 r_\mathrm{T} \veff}{\nu}$ \cite[p. 14]{darby2017chemical}. For increasing Reynolds numbers, the transition from the laminar to the turbulent flow regime occurs around $\text{Re} \approx 2300$ \cite[p. 12]{schlichting2016boundary}. The viscosity of the buffer solution in our testbed equals the viscosity of water \cite{hink2000structural}, i.e., $\nu = 1.037\, \times 10^{-6} \,\si{ \meter \squared \per \second}$ (at $21\, \si{\celsius}$). 
Here, for the parameter regime in which the testbed is operated, we obtain $\text{Re} = 167$, i.e., laminar flow is dominant. This value of $\text{Re}$ corresponds to a medium-sized artery, where the Reynolds number is typically between 100 and 1000 \cite{caro2012mechanics}.
Furthermore, we determine the relative influence of diffusion on the fluid transport, which can be characterized by the dimensionless Péclet number $\text{Pe} = \frac{r_\mathrm{T} \veff}{\D}$ \cite[eq. (4.44)]{tabeling2023introduction}. Here, utilizing the diffusion coefficient $ \D = 1\, \times 10^{-10} \,\si{\meter\squared \per \second}$ of \ac{GFPD} \cite{Junghans2016DiffusionGFPD}, $\text{Pe} = 1.11\, \times 10^{6} \gg 1$ follows. Therefore, flow dominates over diffusion in the system, and the latter can be neglected as a result.
%
\scaleSubsection
\subsection{ISI Caused by Closed-Loop Operation}\label{SubSec:ISI}
\scaleSubsectionBelow
%
The closed-loop character of our testbed gives rise to four distinct forms of \ac{ISI}: \textit{channel \ac{ISI}}, \textit{inter-loop \ac{ISI}}, \textit{offset \ac{ISI}}, and \textit{permanent \ac{ISI}}. The latter three forms can only be observed in closed-loop systems. 
All forms of \ac{ISI} are discussed in detail in the following. 
%
\scaleSubsubsection
\subsubsection{Channel ISI}\label{subsubsec:channel_isi}
\scaleSubsubsectionBelow
Within the channel, the propagation of OFF state \ac{GFPD} molecules from \ac{TX} to \ac{RX} may result in the overlap of consecutive transmitted symbols, leading to \ac{ISI}.
The existence of channel \ac{ISI} has been demonstrated with existing \ac{MC} testbeds, e.g., \cite{grebenstein2018biological, wang2020understanding}.
Channel \ac{ISI} can be mitigated by introducing a guard interval $T_\mathrm{G}$, during which the \ac{TX} is inactive, where the effectiveness of \ac{ISI} mitigation is contingent on the duration of the guard interval. However, increasing $T_\mathrm{G}$ also results in a reduction of the achievable data rate.
\scaleSubsubsection
\subsubsection{Inter-Loop ISI}\label{subsubsec:loop_isi}
\scaleSubsubsectionBelow
Due to the closed-loop design of the testbed, inter-loop \ac{ISI} occurs, as the OFF state \ac{GFPD} molecules remain in the system and reenter the \ac{RX} after completing one loop. Hence, these molecules interfere not only with neighboring symbols (as is the case for channel \ac{ISI}) but also with symbols transmitted much later in time. In particular, OFF state \ac{GFPD} molecules, which are switched
to the OFF state by the \ac{TX}, may travel several times through the loop before they spontaneously switch back to the ON state. Thus, these signaling molecules are likely to affect the received signal multiple times, while the intensity, timing, and number of recurrences of inter-loop \ac{ISI} depend on the length of the loop and the half-life $T_{1/2}$ of \ac{GFPD}. The intensity of inter-loop \ac{ISI} can be reduced by using the \ac{EX}.
%
\scaleSubsubsection
\subsubsection{Offset ISI}\label{subsubsec:offset_isi}
\scaleSubsubsectionBelow
As \ac{GFPD} molecules in the OFF state disperse in the system over multiple loops, the inter-loop \ac{ISI}, characterized by distinct drops in fluorescence intensity, undergoes a gradual transformation to a temporal offset in the observed fluorescence intensity, which we refer to as offset \ac{ISI}. The intensity of offset \ac{ISI} depends on the irradiation intensity used at the \ac{TX}, symbol duration $\Ti$, and half-life $T_{1/2}$ of \ac{GFPD}. The intensity of offset \ac{ISI} can be reduced by using the \ac{EX}.
%
\scaleSubsubsection
\subsubsection{Permanent ISI}\label{subsubsec:permanent_isi}
\scaleSubsubsectionBelow
After prolonged exposure to light, \ac{GFPD} molecules may undergo an irreversible degradation as a consequence of photobleaching effects (cf. \Section{par:photobleaching}). While the decrease in fluorescence intensity resulting from offset \ac{ISI} is reversible, photobleaching is not. Thus, photobleaching causes an irreversible reduction of the number of functional \ac{GFPD} molecules available for modulation. Consequently, this leads to an overall decrease in the received fluorescence signal intensity over time. As a result, we observe a deterioration of the communication performance.
Since \ac{GFPD} molecules are irradiated by the \ac{EX}, the \ac{RX}, and the \ac{TX} in the testbed, all three light sources contribute to photobleaching. Therefore, we refer to the fractions caused by the \ac{TX}, \ac{EX}, and \ac{RX} as permanent \ac{ISI}, \ac{EX} bleaching, and measurement bleaching, respectively. Note that only photobleaching caused by the \ac{TX} is interpreted as \ac{ISI}, since it can be varied by varying the transmitted signal, whereas the photobleaching caused by \ac{EX} and \ac{RX} is independent of the transmitted sequence.
%
\scaleSection
\section{Communication Scheme}\label{sec:communication}
\scaleSectionBelow
\pgfplotsset{compat=newest,
    width=\linewidth, height=3.5cm,
    every axis/.append style={
     semithick,
     ymajorgrids=true,
    },
    legend style={
     draw=none,
     fill=white,
     anchor=south,
     at={(0.5, 1)},
     font=\small,
    },
	{cycle list/Set1},
    cycle list name=Set1,
    cycle multiindex* list={
    	Set1\nextlist
    	{dashed,solid,dotted,dashdotted,densely dashdotdotted}\nextlist
    	mark list*\nextlist
    },
    legend cell align=left,
    xtick align=outside,
    xtick pos=bottom,
    major tick length=1mm,
    /pgfplots/short line legend/.style={
     legend image code/.code={
         \draw[mark repeat=2,mark phase=2,##1]
             plot coordinates {
                 (0cm,0cm)
                     (0.2cm,0cm)
                     (0.4cm,0cm)
             };%
     }
    },
    /pgfplots/ybar legend/.style={
    /pgfplots/legend image code/.code={%
         \draw[##1,/tikz/.cd,yshift=-0.25em]
             (0cm,0cm) rectangle (4pt,6pt);},
    },
    /pgfplots/xbar legend/.style={
    /pgfplots/legend image code/.code={%
         \draw[##1,/tikz/.cd,yshift=-0.25em]
             (0cm,0cm) rectangle (4pt,6pt);},
    },
    unit code/.code={\si{#1}},
    tick label style = {font=\sansmath\sffamily},
    every axis label = {font=\sffamily},
    legend style = {font=\sffamily},
    label style = {font=\sffamily},
    select coords between index/.style 2 args={
    	x filter/.code={
    		\ifnum\coordindex<#1\def\pgfmathresult{}\fi
    		\ifnum\coordindex>#2\def\pgfmathresult{}\fi
    }},
	discard if not/.style 2 args={
		x filter/.code={
			\edef\tempa{\thisrow{#1}}
			\edef\tempb{#2}
			\ifx\tempa\tempb
			\else
			\def\pgfmathresult{inf}
			\fi
	}}
}



\noindent%
\begin{tikzpicture}
\pgfplotsset{cycle list/Set1-5}
\begin{groupplot}[
	group style = {
		group size = 3 by 1,
		horizontal sep=1mm,
		vertical sep=17mm,
	},
	colormap/Dark2,
	colormap/Paired,
	colormap/Set2,
	title style={at={(0.5,0.9)}},
	ymin=0
]

\nextgroupplot[
	width={137pt},
	xshift={7mm},
	title={\mirage (top 10)},
	ybar=0.5pt, bar width=4.5pt,
	legend columns=-1,
	legend style={fill=none, at={(1.1,1.15)}},
	xlabel={Question set},
	xlabel style={at={(1,-0.5)}},
	ylabel={Queries ($\times 10^4$)},
	ylabel style={at={(-0.1,0.5)}},
	symbolic x coords={{pubmedqa},{medqa},{bioasq},{medmcqa},{mmlu}},
	xtick=data,
	xticklabels={{bioasq},{medmcqa},{medqa},{mmlu-med},{pubmedqa}},
	every axis plot/.append style={fill,draw=none,no markers},
	cycle list name=Dark2,
	enlarge x limits=0.12,
	x tick label style={rotate=30,anchor=east,yshift=-2mm,xshift=0.5mm},
]

\foreach \qa in {{Naive},{Relevant}}{
	\addplot+[%
	]
	table[	col sep=comma, x={Benchmark}, y expr={\thisrow{\qa}/10^4}, 
	]
	{raw_data//communication_reduction_top_10.csv};
}

\foreach \qa in {{Predicted}}{
	\addplot+[index of colormap={4 of Paired}, ]
	table[	col sep=comma, x={Benchmark}, y expr={\thisrow{\qa}/10^4}, 
	]
	{raw_data//communication_reduction_top_10.csv};
}




\nextgroupplot[
	width={137pt},
	title={\mirage (top 32)},
	ybar=0.5pt, bar width=4.5pt, 
	legend columns=-1,
	legend style={fill=none,at={(0.25,1.2)}},
	symbolic x coords={{pubmedqa},{medqa},{bioasq},{medmcqa},{mmlu}},
	xtick=data,
	xticklabels={{bioasq},{medmcqa},{medqa},{mmlu-med},{pubmedqa}},
	yticklabels={\empty},
	every axis plot/.append style={fill,draw=none,no markers},
	cycle list name=Dark2,
	enlarge x limits=0.12,
	x tick label style={rotate=30,anchor=east,yshift=-2mm,xshift=0.5mm},
]

\foreach \qa in {{Naive},{Relevant}}{
	\addplot+[%
	]
	table[	col sep=comma, x={Benchmark}, y expr={\thisrow{\qa}/10^4}, 
	]
	{raw_data/communication_reduction_top_32.csv};
}

\foreach \qa in {{Predicted}}{
	\addplot+[index of colormap={4 of Paired}, ]
	table[	col sep=comma, x={Benchmark}, y expr={\thisrow{\qa}/10^4}, 
	]
	{raw_data/communication_reduction_top_32.csv};
}

\legend{{Naive},{Relevant},{Predicted by \textbf{\sys}}}


\nextgroupplot[
	xshift={3.5mm},
	title={\mmlu},
	width={63.58pt},
	ybar=0.5pt, bar width=4.5pt, 
	enlarge x limits=0,
	symbolic x coords={{wiki}},
	xtick=data,
	every axis plot/.append style={fill,draw=none,no markers},
	cycle list name=Dark2,
	enlarge x limits=0.15,
	x tick label style={rotate=30,anchor=east,yshift=-2mm,xshift=0.5mm},
]

\foreach \qa in {{Naive},{Relevant}}{
	\addplot+[%
	]
	table[	col sep=comma, x={Benchmark}, y expr={\thisrow{\qa}/10^4}, 
	]
	{raw_data/communication_reduction_wiki.csv};
}

\foreach \qa in {{Predicted}}{
	\addplot+[index of colormap={4 of Paired}, ]
	table[	col sep=comma, x={Benchmark}, y expr={\thisrow{\qa}/10^4}, 
	]
	{raw_data/communication_reduction_wiki.csv};
}

\end{groupplot}


\end{tikzpicture}

%
\scaleSection
\section{Performance Metrics}\label{sec:metric}
\scaleSectionBelow
In this section, we describe the quantitative and qualitative communication-specific metrics employed to evaluate \ac{MC} performance in the proposed testbed.
\vspace{0.1cm}
\scaleSubsection
\subsection{Bit Error Rate and Data Rate}\label{sec:ber_and_data_rate}
\scaleSubsectionBelow
%
We define an \textit{empirical} \ac{BER} for our system. The \ac{BER} is a standard metric for quantifying the performance of any communication scheme. For the bit-to-symbol mapping, we use Gray mapping \cite[p. 100]{proakis2008digital} $G: \{0,1\}^{\log_{2}(M)\nSymb} \rightarrow \{0, \ldots, M - 1\}^{\nSymb}$. Hence, the bit sequences corresponding to symbol sequences $\mathbf{i} \in \{0, 1, \ldots , M - 1\}^{\nSymb}$ and $\hat{\mathbf{i}} \in \{0, 1, \ldots , M - 1\}^{\nSymb}$, i.e., the transmitted and estimated symbol sequences, are obtained as $\mathbf{i}_{G} = G^{-1}(\mathbf{i})$ and $\hat{\mathbf{i}}_G = G^{-1}(\hat{\mathbf{i}})$, respectively. In our case, since the use of pilot symbols is required for calibration of all our detection techniques, the \ac{BER} is obtained as
%
\begin{equation}
    \mathrm{BER} = \frac{\sum_{b = \log_2(M)\nPilots}^{\log_{2}(M)\nSymb - 1} \left|\mathbf{i}_{G}[b] - \hat{\mathbf{i}}_G[b]\right|}{(\nSymb - \nPilots)\log_2(M)}\;.
    \label{eq:BER}
\end{equation}
Note that this equals the Hamming distance between $\mathbf{i}_{G}$ and $\hat{\mathbf{i}}_G$ normalized to the bit sequence length excluding the sequence part corresponding to the pilot symbols $\nPilots$. Thus, the empirically defined \ac{BER} is the number of erroneously demodulated bits scaled by the number of all transmitted bits.
\vspace{0.1cm}
\scaleSubsection
\subsection{Absolute Mean Euclidean Distance}
\scaleSubsectionBelow
%
In some cases the \ac{BER} is not insightful, e.g., if the transmission sequence length is too short for an accurate estimation of the \ac{BER} or if the \ac{MC} system is in the low \ac{BER} regime, where irrespectively of the chosen settings, for the evaluated transmission sequence length, $\textnormal{BER}=0$ is obtained.
For such cases, we introduce the \ac{AMED} metric as follows
\begin{equation}
   \AMED = \min_{i' \neq i''} |\Bar{S}_{i'}[l] - \Bar{S}_{i''}[l]| \;; \qquad \forall i', i'' \in \{0, 1, \ldots, \modOrder - 1\} \;,
    \label{eq:amed}
\end{equation}
where we employ \Equation{eq:average_set} and \Equation{eq:sampleSets}.

As can be seen from \eqref{eq:amed}, the \ac{AMED} is the minimum Euclidean distance between the means of the samples $\detecSample[\symIdx]$ received over $\nWindow$ symbol intervals. Consequently, the \ac{AMED} reflects the extent to which the received samples of the various symbols differ. Compared to the \ac{BER}, the \ac{AMED} provides reliable values even for small numbers of samples, i.e., small $\nWindow$, and can therefore serve as an initial indicator of performance for cases where a \ac{BER} evaluation is not meaningful or not possible.
\vspace{0.1cm}
\scaleSubsection
\subsection{Eye Diagram}
\scaleSubsectionBelow
Eye diagrams are a well-established qualitative method to evaluate a communication system's performance \cite[p. 603]{proakis2008digital}. The use of eye diagrams to evaluate \ac{MC} systems has been introduced in \cite{farsad2017novel}. Plotting the received signal modulo the symbol duration generates the eye pattern. Effects such as \ac{ISI}, various system parameters, including the use of a guard interval, and the effects of varying the modulation order affect the eye pattern and the eye opening, which can provide insights for system design and performance.


%
\scaleSection
\section{Experimental Results and Communication Performance Evaluation}\label{sec:results}
\scaleSectionBelow

\begin{table*}[t]
\centering
\fontsize{11pt}{11pt}\selectfont
\begin{tabular}{lllllllllllll}
\toprule
\multicolumn{1}{c}{\textbf{task}} & \multicolumn{2}{c}{\textbf{Mir}} & \multicolumn{2}{c}{\textbf{Lai}} & \multicolumn{2}{c}{\textbf{Ziegen.}} & \multicolumn{2}{c}{\textbf{Cao}} & \multicolumn{2}{c}{\textbf{Alva-Man.}} & \multicolumn{1}{c}{\textbf{avg.}} & \textbf{\begin{tabular}[c]{@{}l@{}}avg.\\ rank\end{tabular}} \\
\multicolumn{1}{c}{\textbf{metrics}} & \multicolumn{1}{c}{\textbf{cor.}} & \multicolumn{1}{c}{\textbf{p-v.}} & \multicolumn{1}{c}{\textbf{cor.}} & \multicolumn{1}{c}{\textbf{p-v.}} & \multicolumn{1}{c}{\textbf{cor.}} & \multicolumn{1}{c}{\textbf{p-v.}} & \multicolumn{1}{c}{\textbf{cor.}} & \multicolumn{1}{c}{\textbf{p-v.}} & \multicolumn{1}{c}{\textbf{cor.}} & \multicolumn{1}{c}{\textbf{p-v.}} &  &  \\ \midrule
\textbf{S-Bleu} & 0.50 & 0.0 & 0.47 & 0.0 & 0.59 & 0.0 & 0.58 & 0.0 & 0.68 & 0.0 & 0.57 & 5.8 \\
\textbf{R-Bleu} & -- & -- & 0.27 & 0.0 & 0.30 & 0.0 & -- & -- & -- & -- & - &  \\
\textbf{S-Meteor} & 0.49 & 0.0 & 0.48 & 0.0 & 0.61 & 0.0 & 0.57 & 0.0 & 0.64 & 0.0 & 0.56 & 6.1 \\
\textbf{R-Meteor} & -- & -- & 0.34 & 0.0 & 0.26 & 0.0 & -- & -- & -- & -- & - &  \\
\textbf{S-Bertscore} & \textbf{0.53} & 0.0 & {\ul 0.80} & 0.0 & \textbf{0.70} & 0.0 & {\ul 0.66} & 0.0 & {\ul0.78} & 0.0 & \textbf{0.69} & \textbf{1.7} \\
\textbf{R-Bertscore} & -- & -- & 0.51 & 0.0 & 0.38 & 0.0 & -- & -- & -- & -- & - &  \\
\textbf{S-Bleurt} & {\ul 0.52} & 0.0 & {\ul 0.80} & 0.0 & 0.60 & 0.0 & \textbf{0.70} & 0.0 & \textbf{0.80} & 0.0 & {\ul 0.68} & {\ul 2.3} \\
\textbf{R-Bleurt} & -- & -- & 0.59 & 0.0 & -0.05 & 0.13 & -- & -- & -- & -- & - &  \\
\textbf{S-Cosine} & 0.51 & 0.0 & 0.69 & 0.0 & {\ul 0.62} & 0.0 & 0.61 & 0.0 & 0.65 & 0.0 & 0.62 & 4.4 \\
\textbf{R-Cosine} & -- & -- & 0.40 & 0.0 & 0.29 & 0.0 & -- & -- & -- & -- & - & \\ \midrule
\textbf{QuestEval} & 0.23 & 0.0 & 0.25 & 0.0 & 0.49 & 0.0 & 0.47 & 0.0 & 0.62 & 0.0 & 0.41 & 9.0 \\
\textbf{LLaMa3} & 0.36 & 0.0 & \textbf{0.84} & 0.0 & {\ul{0.62}} & 0.0 & 0.61 & 0.0 &  0.76 & 0.0 & 0.64 & 3.6 \\
\textbf{our (3b)} & 0.49 & 0.0 & 0.73 & 0.0 & 0.54 & 0.0 & 0.53 & 0.0 & 0.7 & 0.0 & 0.60 & 5.8 \\
\textbf{our (8b)} & 0.48 & 0.0 & 0.73 & 0.0 & 0.52 & 0.0 & 0.53 & 0.0 & 0.7 & 0.0 & 0.59 & 6.3 \\  \bottomrule
\end{tabular}
\caption{Pearson correlation on human evaluation on system output. `R-': reference-based. `S-': source-based.}
\label{tab:sys}
\end{table*}



\begin{table}%[]
\centering
\fontsize{11pt}{11pt}\selectfont
\begin{tabular}{llllll}
\toprule
\multicolumn{1}{c}{\textbf{task}} & \multicolumn{1}{c}{\textbf{Lai}} & \multicolumn{1}{c}{\textbf{Zei.}} & \multicolumn{1}{c}{\textbf{Scia.}} & \textbf{} & \textbf{} \\ 
\multicolumn{1}{c}{\textbf{metrics}} & \multicolumn{1}{c}{\textbf{cor.}} & \multicolumn{1}{c}{\textbf{cor.}} & \multicolumn{1}{c}{\textbf{cor.}} & \textbf{avg.} & \textbf{\begin{tabular}[c]{@{}l@{}}avg.\\ rank\end{tabular}} \\ \midrule
\textbf{S-Bleu} & 0.40 & 0.40 & 0.19* & 0.33 & 7.67 \\
\textbf{S-Meteor} & 0.41 & 0.42 & 0.16* & 0.33 & 7.33 \\
\textbf{S-BertS.} & {\ul0.58} & 0.47 & 0.31 & 0.45 & 3.67 \\
\textbf{S-Bleurt} & 0.45 & {\ul 0.54} & {\ul 0.37} & 0.45 & {\ul 3.33} \\
\textbf{S-Cosine} & 0.56 & 0.52 & 0.3 & {\ul 0.46} & {\ul 3.33} \\ \midrule
\textbf{QuestE.} & 0.27 & 0.35 & 0.06* & 0.23 & 9.00 \\
\textbf{LlaMA3} & \textbf{0.6} & \textbf{0.67} & \textbf{0.51} & \textbf{0.59} & \textbf{1.0} \\
\textbf{Our (3b)} & 0.51 & 0.49 & 0.23* & 0.39 & 4.83 \\
\textbf{Our (8b)} & 0.52 & 0.49 & 0.22* & 0.43 & 4.83 \\ \bottomrule
\end{tabular}
\caption{Pearson correlation on human ratings on reference output. *not significant; we cannot reject the null hypothesis of zero correlation}
\label{tab:ref}
\end{table}


\begin{table*}%[]
\centering
\fontsize{11pt}{11pt}\selectfont
\begin{tabular}{lllllllll}
\toprule
\textbf{task} & \multicolumn{1}{c}{\textbf{ALL}} & \multicolumn{1}{c}{\textbf{sentiment}} & \multicolumn{1}{c}{\textbf{detoxify}} & \multicolumn{1}{c}{\textbf{catchy}} & \multicolumn{1}{c}{\textbf{polite}} & \multicolumn{1}{c}{\textbf{persuasive}} & \multicolumn{1}{c}{\textbf{formal}} & \textbf{\begin{tabular}[c]{@{}l@{}}avg. \\ rank\end{tabular}} \\
\textbf{metrics} & \multicolumn{1}{c}{\textbf{cor.}} & \multicolumn{1}{c}{\textbf{cor.}} & \multicolumn{1}{c}{\textbf{cor.}} & \multicolumn{1}{c}{\textbf{cor.}} & \multicolumn{1}{c}{\textbf{cor.}} & \multicolumn{1}{c}{\textbf{cor.}} & \multicolumn{1}{c}{\textbf{cor.}} &  \\ \midrule
\textbf{S-Bleu} & -0.17 & -0.82 & -0.45 & -0.12* & -0.1* & -0.05 & -0.21 & 8.42 \\
\textbf{R-Bleu} & - & -0.5 & -0.45 &  &  &  &  &  \\
\textbf{S-Meteor} & -0.07* & -0.55 & -0.4 & -0.01* & 0.1* & -0.16 & -0.04* & 7.67 \\
\textbf{R-Meteor} & - & -0.17* & -0.39 & - & - & - & - & - \\
\textbf{S-BertScore} & 0.11 & -0.38 & -0.07* & -0.17* & 0.28 & 0.12 & 0.25 & 6.0 \\
\textbf{R-BertScore} & - & -0.02* & -0.21* & - & - & - & - & - \\
\textbf{S-Bleurt} & 0.29 & 0.05* & 0.45 & 0.06* & 0.29 & 0.23 & 0.46 & 4.2 \\
\textbf{R-Bleurt} & - &  0.21 & 0.38 & - & - & - & - & - \\
\textbf{S-Cosine} & 0.01* & -0.5 & -0.13* & -0.19* & 0.05* & -0.05* & 0.15* & 7.42 \\
\textbf{R-Cosine} & - & -0.11* & -0.16* & - & - & - & - & - \\ \midrule
\textbf{QuestEval} & 0.21 & {\ul{0.29}} & 0.23 & 0.37 & 0.19* & 0.35 & 0.14* & 4.67 \\
\textbf{LlaMA3} & \textbf{0.82} & \textbf{0.80} & \textbf{0.72} & \textbf{0.84} & \textbf{0.84} & \textbf{0.90} & \textbf{0.88} & \textbf{1.00} \\
\textbf{Our (3b)} & 0.47 & -0.11* & 0.37 & 0.61 & 0.53 & 0.54 & 0.66 & 3.5 \\
\textbf{Our (8b)} & {\ul{0.57}} & 0.09* & {\ul 0.49} & {\ul 0.72} & {\ul 0.64} & {\ul 0.62} & {\ul 0.67} & {\ul 2.17} \\ \bottomrule
\end{tabular}
\caption{Pearson correlation on human ratings on our constructed test set. 'R-': reference-based. 'S-': source-based. *not significant; we cannot reject the null hypothesis of zero correlation}
\label{tab:con}
\end{table*}

\section{Results}
We benchmark the different metrics on the different datasets using correlation to human judgement. For content preservation, we show results split on data with system output, reference output and our constructed test set: we show that the data source for evaluation leads to different conclusions on the metrics. In addition, we examine whether the metrics can rank style transfer systems similar to humans. On style strength, we likewise show correlations between human judgment and zero-shot evaluation approaches. When applicable, we summarize results by reporting the average correlation. And the average ranking of the metric per dataset (by ranking which metric obtains the highest correlation to human judgement per dataset). 

\subsection{Content preservation}
\paragraph{How do data sources affect the conclusion on best metric?}
The conclusions about the metrics' performance change radically depending on whether we use system output data, reference output, or our constructed test set. Ideally, a good metric correlates highly with humans on any data source. Ideally, for meta-evaluation, a metric should correlate consistently across all data sources, but the following shows that the correlations indicate different things, and the conclusion on the best metric should be drawn carefully.

Looking at the metrics correlations with humans on the data source with system output (Table~\ref{tab:sys}), we see a relatively high correlation for many of the metrics on many tasks. The overall best metrics are S-BertScore and S-BLEURT (avg+avg rank). We see no notable difference in our method of using the 3B or 8B model as the backbone.

Examining the average correlations based on data with reference output (Table~\ref{tab:ref}), now the zero-shoot prompting with LlaMA3 70B is the best-performing approach ($0.59$ avg). Tied for second place are source-based cosine embedding ($0.46$ avg), BLEURT ($0.45$ avg) and BertScore ($0.45$ avg). Our method follows on a 5. place: here, the 8b version (($0.43$ avg)) shows a bit stronger results than 3b ($0.39$ avg). The fact that the conclusions change, whether looking at reference or system output, confirms the observations made by \citet{scialom-etal-2021-questeval} on simplicity transfer.   

Now consider the results on our test set (Table~\ref{tab:con}): Several metrics show low or no correlation; we even see a significantly negative correlation for some metrics on ALL (BLEU) and for specific subparts of our test set for BLEU, Meteor, BertScore, Cosine. On the other end, LlaMA3 70B is again performing best, showing strong results ($0.82$ in ALL). The runner-up is now our 8B method, with a gap to the 3B version ($0.57$ vs $0.47$ in ALL). Note our method still shows zero correlation for the sentiment task. After, ranks BLEURT ($0.29$), QuestEval ($0.21$), BertScore ($0.11$), Cosine ($0.01$).  

On our test set, we find that some metrics that correlate relatively well on the other datasets, now exhibit low correlation. Hence, with our test set, we can now support the logical reasoning with data evidence: Evaluation of content preservation for style transfer needs to take the style shift into account. This conclusion could not be drawn using the existing data sources: We hypothesise that for the data with system-based output, successful output happens to be very similar to the source sentence and vice versa, and reference-based output might not contain server mistakes as they are gold references. Thus, none of the existing data sources tests the limits of the metrics.  


\paragraph{How do reference-based metrics compare to source-based ones?} Reference-based metrics show a lower correlation than the source-based counterpart for all metrics on both datasets with ratings on references (Table~\ref{tab:sys}). As discussed previously, reference-based metrics for style transfer have the drawback that many different good solutions on a rewrite might exist and not only one similar to a reference.


\paragraph{How well can the metrics rank the performance of style transfer methods?}
We compare the metrics' ability to judge the best style transfer methods w.r.t. the human annotations: Several of the data sources contain samples from different style transfer systems. In order to use metrics to assess the quality of the style transfer system, metrics should correctly find the best-performing system. Hence, we evaluate whether the metrics for content preservation provide the same system ranking as human evaluators. We take the mean of the score for every output on each system and the mean of the human annotations; we compare the systems using the Kendall's Tau correlation. 

We find only the evaluation using the dataset Mir, Lai, and Ziegen to result in significant correlations, probably because of sparsity in a number of system tests (App.~\ref{app:dataset}). Our method (8b) is the only metric providing a perfect ranking of the style transfer system on the Lai data, and Llama3 70B the only one on the Ziegen data. Results in App.~\ref{app:results}. 


\subsection{Style strength results}
%Evaluating style strengths is a challenging task. 
Llama3 70B shows better overall results than our method. However, our method scores higher than Llama3 70B on 2 out of 6 datasets, but it also exhibits zero correlation on one task (Table~\ref{tab:styleresults}).%More work i s needed on evaluating style strengths. 
 
\begin{table}%[]
\fontsize{11pt}{11pt}\selectfont
\begin{tabular}{lccc}
\toprule
\multicolumn{1}{c}{\textbf{}} & \textbf{LlaMA3} & \textbf{Our (3b)} & \textbf{Our (8b)} \\ \midrule
\textbf{Mir} & 0.46 & 0.54 & \textbf{0.57} \\
\textbf{Lai} & \textbf{0.57} & 0.18 & 0.19 \\
\textbf{Ziegen.} & 0.25 & 0.27 & \textbf{0.32} \\
\textbf{Alva-M.} & \textbf{0.59} & 0.03* & 0.02* \\
\textbf{Scialom} & \textbf{0.62} & 0.45 & 0.44 \\
\textbf{\begin{tabular}[c]{@{}l@{}}Our Test\end{tabular}} & \textbf{0.63} & 0.46 & 0.48 \\ \bottomrule
\end{tabular}
\caption{Style strength: Pearson correlation to human ratings. *not significant; we cannot reject the null hypothesis of zero corelation}
\label{tab:styleresults}
\end{table}

\subsection{Ablation}
We conduct several runs of the methods using LLMs with variations in instructions/prompts (App.~\ref{app:method}). We observe that the lower the correlation on a task, the higher the variation between the different runs. For our method, we only observe low variance between the runs.
None of the variations leads to different conclusions of the meta-evaluation. Results in App.~\ref{app:results}.
%
\scaleSection
\section{Testbed Comparison and Data Sharing}\label{sec:comparison_Testbeds}
\scaleSectionBelow
In this section, we compare our testbed to other fluid-based testbeds, and provide information regarding the sharing of our data and code.
%
\scaleSubsection
\subsection{Comparison with Other Testbeds}\label{ssec:comparison}
\scaleSubsectionBelow
%
\begin{table}[!tbp]
    \caption{Overview of Selected Fluid-Based \ac{MC} Testbeds}
    \centering
    \resizebox{1\columnwidth}{!}{
        \begin{tabular}{cccccc}
            \hline\hline
             Signaling Molecule (Biocompatibility) & Data rate (bit/s) & \ac{BER} (\# transmitted bits) & Detection Method & Reference\\
             \hline\hline
             Green Fluorescent Protein Dreiklang (GFPD) (\cmark) & 0.6 & 0 (5370) & Differential Signal + Adaptive Threshold & This paper\\
             \hline
             Superparamagnetic Iron Oxide & & & & \\[-0.2cm]
             Nanoparticles (SPIONs) (\cmark) & 10 & 0.0843 (600) & Convolutional Neural Network &  \cite{bartunik2023development}\\ % transmission efficiency is 160 bit/Ml M = 8 (2000 symbols were transmitted, but only 200 = 600bits were used for testing the detection
             & &  & $\mu$ Link Decoder with Channel Estimation& & \\[-0.2cm]
             Sodium chloride (NaCl) (\cmark) & 5 & 0.002 (100 x 100) & and Sequence Detection & \cite{wang2020understanding} \\
             Colored Ink (\xmark) & 1.5 & 1/312 (312) & Threshold & \cite{wietfeld2024evaluation}\\ % three colors simultaneously 3xlog(2) = 3 bit/symbol, data rate = 1.5 bps, 1/312 BER (3 color molecules) Threshold detector on individual colors
             Acid/Base (\xmark) & 2.63 & 0 (1080) & Recurrent Neural Network & \cite{farsad2017novel}\\ % SVM/RNN based detection
             Sodium Hydroxide (NaOH) (\xmark) & 1/375 & 0.04 (100) & Threshold &  \cite{walter2023real}\\ % threshold detection, 6.25*60, supplementary file
             Hydrogen Chloride (HCl) (\xmark) & 1/27 & 0.022 (1000) & Adaptive Threshold & \cite{khaloopour2019experimental}\\ % 10 cm^3 solution per bit (?) , adaptive threshold detection
             Transfer DNA (tDNA) (\cmark) & 1/120 & 0 (20) & Differential Signal + Threshold & \cite{kuscu2021fabrication}\\ %differential detection with threshold detector
             Glucose ($\textrm{C}_6 \textrm{H}_{12} \textrm{0}_6$) (\cmark) & 2 & 0.05 (654) & Machine Learning & \cite{koo2020deep}\\ % Module based machine learning, some of the measurements had to be discarded as the transmitter was running out of molecules earlier than expected
             \hline\hline
        \end{tabular}
    }
    \label{tab:experiments}
     \vspace{-1.0cm}
\end{table}
Our comparison is focused on the experimentally determined \acp{BER}, the detection schemes used, and the efficiency of data transmission. \Table{tab:experiments} provides a selected overview of fluid-based \ac{MC} testbeds, listing for each testbed the signaling molecule used, the lowest reported \ac{BER}, the data rate achieving this \ac{BER}, the length of the bit sequence transmitted, and the employed detection method.

Providing a fair and still meaningful comparison between existing \ac{MC} testbeds and the proposed testbed is challenging due to some fundamental differences.
In particular, none of the existing works has considered a closed-loop and self-contained topology. As a result, only our testbed is confronted with the challenge of resolving inter-loop \ac{ISI}, offset \ac{ISI}, and permanent \ac{ISI}. In addition, all other testbeds require and use a much larger number of signaling molecules due to their open topology. The resulting low molecule efficiency, which in \cite{bartunik2021increasing} has been defined as the number of bit transmitted per volume of dissolved signaling molecule solution, is a significant challenge in existing testbeds. The issue of low efficiency has been recognized and discussed \cite{bartunik2023development, koo2020deep, lin2024testbed}: In \cite{bartunik2023development}, an efficiency of $160 \,\si{\bit \per \milli\liter}$ was achieved, which is still low compared to our testbed with $90,000/9 \, \si{\bit \per \milli\liter} = 10,000 \,\si{\bit \per \milli\liter}$; in \cite{koo2020deep}, the authors reported that some measurements had to be discarded because the \ac{TX} ran out of molecules earlier than expected; and in \cite{lin2024testbed}, high costs were associated with the use of the testbed, resulting in a maximum evaluated bit sequence length of $8$ bit\footnote{The authors in \cite{lin2024testbed} also reported that for such a short sequence a meaningful estimation of the \ac{BER} was not possible. We believe that this is an example where the discussed alternative performance metrics, such as \ac{AMED} and eye diagrams, are particularly useful.}.

Despite the different topologies and the challenges our testbed has to overcome, we achieve a data rate that is only an order of magnitude lower than the largest data rates reported in the \ac{MC} literature, which are from \cite{bartunik2023development}\footnote{For the sake of completeness, we note that, to the best of our knowledge, the highest \textit{achievable} data rate was reported in \cite{huang2024non}.}. At the same time, our testbed provides very reliable communication, i.e., very low \acp{BER}, while employing a low-complexity detection scheme that involves only differentiation of the received signal and an adaptive threshold detector. Quantifying the computational complexity of the detection schemes listed here is beyond the scope of this paper. However, methods based on machine learning, such as Convolutional Neural Networks \cite{bartunik2023development} and Recurrent Neural Networks \cite{farsad2017novel}, as well as methods that explicitly take into account the memory effects caused by \ac{ISI} \cite{wang2020understanding}, even if efficiently implemented using the Viterbi algorithm, are believed to have comparatively higher computational complexities.
%
\scaleSubsection
\subsection{Data Handling and Code Sharing}\label{ssec:datasharing}
\scaleSubsectionBelow
%
To promote transparency and enable experimental evaluation of new communication algorithms developed by the \ac{MC} community, we publish our experimental data and the Python code for synchronization and detection in Zenodo and a Git repository under the CC BY and the MIT licenses, respectively. When used, the data and/or code can and should be cited using the corresponding Zenodo \ac{DOI} \cite{scherer2025Zenodo}. The link to the Git repository can be found on Zenodo.

%
\scaleSection
\section{Conclusion}\label{sec:conclusion}
\scaleSectionBelow
\section{Conclusion}
In this work, we propose a simple yet effective approach, called SMILE, for graph few-shot learning with fewer tasks. Specifically, we introduce a novel dual-level mixup strategy, including within-task and across-task mixup, for enriching the diversity of nodes within each task and the diversity of tasks. Also, we incorporate the degree-based prior information to learn expressive node embeddings. Theoretically, we prove that SMILE effectively enhances the model's generalization performance. Empirically, we conduct extensive experiments on multiple benchmarks and the results suggest that SMILE significantly outperforms other baselines, including both in-domain and cross-domain few-shot settings.
%
\vspace*{-0.8cm}
\renewcommand{\baselinestretch}{1.05}
\bibliographystyle{IEEEtran}
\bibliography{literature}
\renewcommand{\baselinestretch}{1.25}
\scaleSubsection
\section*{Acknowledgment}
\scaleSectionBelow
{\small
We thank Prof. Stefan Jakobs (Max Planck Institute for Biophysical Chemistry, Göttingen, Germany) for providing a plasmid encoding Dreiklang. This work was supported by the German Research Foundation (DFG) under Project 290825040 and 509922606.
}

\end{document}