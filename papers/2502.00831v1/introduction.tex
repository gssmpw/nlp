Synthetic \ac{MC} is an emerging research field at the intersection of biology, nanotechnology, and communications engineering. Inspired by natural processes, \ac{MC} offers innovative methods to transmit information by encoding it into chemical signals \cite{nakano2013molecular}. This approach has the potential to enable transformative applications in medicine \cite{felicetti2016applications,akyildiz2015internet}, nanotechnology \cite{Akan2012nanonetworks,soldner2020survey}, agriculture \cite{Dixon1990agricultural}, and environmental monitoring \cite{nakano2012molecular} by providing biocompatible and energy-efficient communication solutions, particularly in environments where conventional communication methods are impractical \cite{farsad2016comprehensive}.

%From theory to applications
So far, most work in \ac{MC} has focused on theoretical research, resulting in the development of theoretical models and communication schemes \cite{jamali2019channel,kuscu2019transmitter,kuran2020survey, farsad2016comprehensive}.
However, in order to advance from theory to the envisioned applications of \ac{MC}, it is crucial to develop experimental testbeds to bridge the gap between theoretical concepts and their practical application \cite{lotter2023experimental}. 
In recent years, there has been a notable increase in experimental \ac{MC} research and the number of experimental testbeds. A comprehensive overview on experimental \ac{MC} is provided in \cite{lotter2023experimental, Lotter2023testbedII}. 

A significant portion of \ac{MC} research has focused on medical applications, including in-body communication, health monitoring, and targeted drug delivery \cite{felicetti2016applications}. Consequently, suitable \ac{MC} testbeds aim to model these application scenarios, e.g., \ac{MC} within the human cardiovascular system, which directly influences their design. Therefore, many existing \ac{MC} testbeds \cite{bartunik2023development,lin2024testbed,wang2020understanding,wietfeld2024evaluation,angerbauer2023salinity,farsad2017novel,walter2023real} try to emulate real-world conditions, e.g., by employing biocompatible signaling molecules and by mimicking relevant aspects of the closed-loop human cardiovascular system using tube-based propagation channels with a background fluid flow \cite{Lotter2023testbedII}.
%
However, most existing testbeds are only partially successful in this regard. On the one hand, the testbeds presented in \cite{bartunik2023development,lin2024testbed,wang2020understanding} use biocompatible signaling molecules -- specifically, \acp{SPION}, the cyanine dye indocyanine green, and the salt sodium chloride, respectively, that have the potential to be used in future \textit{in vivo} applications. On the other hand, all previously mentioned testbeds \cite{bartunik2023development,lin2024testbed,wang2020understanding,wietfeld2024evaluation,angerbauer2023salinity,farsad2017novel,walter2023real} share the limitation of considering only simple topologies -- primarily a single straight duct as the propagation channel -- and lack self-containment, i.e., they operate as open-loop systems. Here, open-loop refers to systems where the background fluid used in the experiment can enter or leave the system during operation. As a result, in these testbeds, the signaling molecules are added to the tube system at one point, e.g., via an injection, are used \textit{once} for information transmission, and then are collected as waste at the end of the tube. \textbf{Such systems have two major drawbacks}: First, long-term transmission experiments generate a lot of waste. Second, many of the intended applications target closed-loop environments, e.g., the human cardiovascular system, and cannot be accurately emulated by open-loop topologies.

Therefore, experimental \ac{MC} systems are required where the communication system operates within a closed-loop tube system. Currently, there are only two \ac{MC} testbeds operating in a closed-loop topology \cite{tuccitto2017fluorescent,schafer2024chorioallantoic}. In \cite{tuccitto2017fluorescent}, a fluid is pumped in a closed-loop system while fluorescent particles are injected, detected, and then diluted. The testbed presented in \cite{schafer2024chorioallantoic} constitutes the first \textit{in vivo} \ac{MC} testbed and is based on the chorioallantoic membrane of fertilized chicken eggs, where the fluorescent dye indocyanine green is injected into a closed-loop vascular system. However, the repeated dilution of fluorescent particles in \cite{tuccitto2017fluorescent} reduces their detectability in long-term experiments, and in \cite{schafer2024chorioallantoic}, only a single injection is considered due to the strong soiling of the vascular system by indocyanine green molecules. Moreover, in both testbeds, the signaling molecules can be used only for a single transmission, which is not resource efficient. Therefore, although the testbeds in \cite{tuccitto2017fluorescent, schafer2024chorioallantoic} consider a closed loop, their applicability for long-term experiments is limited.

In this paper, we propose the first experimental closed-loop \ac{MC} testbed in which reversibly switchable signaling molecules are reused multiple times, enabled by molecular \textit{media modulation} \cite{Brand2022MediaModulation}. In media modulation, signaling molecules with specific switching properties \cite{brand2023switchable} are employed and, unlike in conventional \ac{MC} systems, they are injected only once into the system. Then, the state of the signaling molecules is repeatedly switched to transmit information, i.e., no additional molecules are injected into or removed from the system during operation. Compared to common injection-based release mechanisms \cite{lotter2023experimental}, in media modulation, the \ac{TX} can be placed outside the \ac{MC} channel and does not affect the propagation environment.  
%
As signaling molecule, we adopt the biocompatible \ac{GFPD} \cite{richards2003safety, brakemann2011reversibly}. \acp{GFPD} can be reversibly switched between a bright fluorescent ON state and a less fluorescent OFF state via light stimuli of different wavelengths \cite{brakemann2011reversibly}. This allows for writing and erasing information employing an optical \ac{TX} and \ac{EX}, respectively. The state of the \acp{GFPD}, which conveys the transmitted information, is read out via fluorescence detection at the \ac{RX}.
In summary, the proposed \ac{MC} testbed facilitates resource-efficient communication and long-term experiments, and significantly reduces channel soling. By considering a closed-loop propagation environment, the testbed mimics an important property of the cardiovascular system, and the use of the biocompatible signaling molecule \ac{GFPD} ensures relevance for future applications. 

In this paper, we provide a comprehensive description and analysis of the proposed closed-loop media modulation testbed, its construction, and its components. Moreover, we discuss the chemical properties of the biocompatible signaling molecule \ac{GFPD} and their impact on communication performance. We identify three forms of \ac{ISI}, i.e., inter-loop \ac{ISI}, offset \ac{ISI}, and permanent \ac{ISI}, which occur particularly in closed-loop systems and during long-term experiments based on media modulation, and propose methods for their mitigation. 
%
In addition, we develop and analyze a suitable communication scheme for the considered experimental \ac{MC} system. Several communication schemes, including methods for modulation \cite{kuran2020survey,wietfeld2024evaluation}, synchronization \cite{lin2016time, jamali2017symbol, Xuewen2024synchro,Debus2024synchro}, as well as detection \cite{khaloopour2019experimental,kuscu2021fabrication,walter2023real}, have been developed for \ac{MC}. In this paper, we desgin a communication scheme suitable for the proposed closed-loop media modulation testbed, building on the existing literature. In particular, the developed scheme comprises a noise estimation-based wake-up method, higher-order modulation, blind and data-based synchronization, and adaptive threshold-based detection.
Although these methods are developed with a focus on the considered media modulation testbed, they can be generalized to other experimental \ac{MC} testbeds, as they are deliberately introduced in a general form.
Furthermore, we discuss relevant qualitative and quantitative performance metrics for experimental \ac{MC} systems, including \ac{BER}, eye diagram, and \ac{AMED}, and apply these metrics to evaluate the performance of the developed communication scheme.
%
Furthermore, we present the first long-term \ac{MC} experiment, achieving error-free transmission of around $90,000 \, \si{bit}$ sent over more than five days. This demonstrates the reliability of the proposed testbed and highlights the potential of media modulation-based \ac{MC} as a promising approach for future applications, such as long-term health monitoring.
Finally, we compare the proposed media modulation testbed to existing \ac{MC} testbeds, in terms of communication-theoretical metrics such as the \ac{BER}, molecule efficiency, and detection methods. 

% Main contributions
The main contributions of this paper can be summarized as follows:
\begin{itemize}
    \item We utilize light-based media modulation to transmit information in a closed-loop tube system using the states of the biocompatible and reusable signaling molecule \ac{GFPD}. This form of media modulation enables the operation of \ac{TX}, \ac{EX}, and \ac{RX} outside of the tube. As a result, they do not interfere with the propagation of the signaling molecules inside the tube; a feature that is beneficial, e.g., for healthcare applications. By considering a closed-loop topology, we ensure that the proposed testbed resembles and allows to study an important property of the cardiovascular system.
    \item Because of the closed-loop and long experiments, new forms of \ac{ISI} occur, which become apparent on different time scales. We characterize these forms of \ac{ISI} and implement suitable \ac{ISI} mitigation schemes. 
    \item We develop a communication scheme for the considered system that can also be used in other experimental \ac{MC} testbeds. The scheme includes a noise-estimation based wake-up method and data-based and blind synchronization. Furthermore, for detection, we propose an adaptive threshold detector. The transmission characteristics of the testbed are evaluated comprehensively - specifically by metrics such as \ac{AMED}, eye diagrams, and \ac{BER}.
    \item We experimentally demonstrate the error-free data transmission of $5,370 \, \si{bit}$ at a data rate of $36 \, \si{\bitsperminute}$, where 8-ary modulation is used. In addition, to the best of our knowledge, we report the longest MC experiment to date: Error-free transmission of around $90,000 \, \si{bit}$ over a period of more than $5$ consecutive days ($125 \, \si{\hour}$) at a data rate of $12 \, \si{\bitsperminute}$. No molecules were injected into or extracted from the system during this time, i.e., only $9 \, \si{\milli \liter}$ of \ac{GFPD} solution with a concentration of $\con = 0.3 \,\si{\milli\gram \per\milli\liter}$ were required for the entire experiment. By this, we  demonstrate the superiority of media modulation for long \ac{MC} experiments compared to the repeated injection of molecules employed in existing \ac{MC} testbeds.
    \item All our experimental data (more than 250 kbit transmitted across different experimental settings) as well as the Python code for evaluation have been released in open access \cite{scherer2025Zenodo}. This large amount of experimental data enables other researchers to develop and evaluate alternative processing methods for the received signals, e.g., machine learning based algorithms.
\end{itemize}
%
We note that a preliminary and simplified version of the testbed was briefly presented as part of \cite{brand2023switchable}. In \cite{brand2023switchable}, the experiment duration was only $3 \,\si{\minute}$, which did not allow to i) evaluate the resource efficiency of the testbed, ii) characterize the \ac{ISI}, and iii) experimentally determine the \ac{BER} for different detection schemes, which are new contributions of this work. 
%
Moreover, this paper significantly extends its conference version \cite{brand2024closed}: We extend the communication schemes developed in \cite{brand2024closed} by methods for synchronization, blind transmission start estimation, and adaptive threshold detection. Moreover, in this paper, we consider higher order modulation schemes, i.e., 8-ary modulation, and employ the \ac{AMED} and eye diagrams as performance metrics to evaluate the transmission characteristics of the testbed. While in \cite{brand2024closed}  the error free transmission of $500\, \si{bit}$ for an experiment of duration $250 \,\si{\minute}$ was shown, in this paper, we consider the transmission of $90,000\,\si{bit}$ for over $5$ days. 

The remainder of this paper is organized as follows. In \Section{sec:experiment}, we provide an overview of the proposed closed-loop testbed, including a detailed description of the signaling molecule \ac{GFPD}, the testbed components, and the working principle of the testbed. We discuss the physical properties of the testbed and their influence on communication performance in \Section{sec:characterization}. In \Section{sec:communication}, we describe the proposed communication scheme in detail, including modulation, symbol synchronization, and detection. The proposed performance metrics are presented in \Section{sec:metric}, and our results and evaluation are provided in \Section{sec:results}, which we compare with results achieved with other testbeds in \Section{sec:comparison_Testbeds}. Finally, \Section{sec:conclusion} concludes the paper and outlines topics for future work.