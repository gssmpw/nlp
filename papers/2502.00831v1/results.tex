In this section, we evaluate our testbed based on the previously proposed metrics and determine the impact of different system parameters. To obtain these results, in total more than 250 kbit of data were transmitted via \ac{MC} in the testbed.
%
\scaleSubsection
\subsection{Illustration of the Different Types of ISI}\label{subsubsec:concret_char_ISI}
\scaleSubsectionBelow
%
\begin{figure}[!tbp]
\centering
\vspace*{-0.5cm}
  \includegraphics[width=\columnwidth]{fig/SIR.pdf}
  \vspace*{-8mm}
  \caption{$\recSig(\tn)$ of single symbol transmissions with $\symVar \in \{0, 1, 2, 3\}$ for different irradiation durations $\Ti = 10 \,\si{\second}$ and $\Ti = 20 \,\si{\second}$, and \ac{TX}-\ac{RX} distances $d_{\mathrm{TX},\mathrm{RX}} = 6 \,\si{\centi\meter}$ and $d_{\mathrm{TX},\mathrm{RX}} = 35 \,\si{\centi\meter}$. Three forms of \ac{ISI} are visualized: channel \ac{ISI}, inter-loop \ac{ISI}, and offset \ac{ISI}.}
  \label{fig:SIR}
  \vspace*{-0.9cm}
\end{figure}
%
The following experiments were conducted to illustrate three different types of \ac{ISI} that occur in our testbed. The results are summarized in \Figure{fig:SIR}. In particular, individual symbols $\symVar \in \{0, 1, 2, 3\}$, indicated by different colors, are sent. Furthermore, for $\symVar = 3$, we show results for two illumination durations, $\Ti= 10 \,\si{\second}$ (solid line) and $\Ti = 20 \,\si{\second}$ (dotted line), and two different transmission distances, $d_{\mathrm{TX},\mathrm{RX}} = 6 \, \si{\centi\meter}$ (no marker) and $d_{\mathrm{TX},\mathrm{RX}} = 35 \, \si{\centi\meter}$ (rectangular markers). \Figure{fig:SIR} shows the corresponding received signals $\recSig(\tn)$. In these experiments, the \ac{EX} was turned off to increase the visibility of the \ac{ISI} effects.

For $\Ti= 10 \,\si{\second}$, \Figure{fig:SIR} shows two drops in $\recSig(\tn)$. The first drop around $10\, \si{\second}$ corresponds to the desired pulse sent by the \ac{TX}, after which the signal returns back to its original level at around $30\, \si{\second}$. The second drop, visible around $60\, \si{\second}$, corresponds to inter-loop \ac{ISI}, while the offset \ac{ISI} is visible in the inset at $90 \, \si{\second}$.
Both the inter-loop \ac{ISI} and the offset \ac{ISI} increase with increasing $\symVar$. 
Since permanent \ac{ISI} caused by photobleaching occurs after many transmission cycles only, it is not visible in \Figure{fig:SIR}.
We observe from \Figure{fig:SIR} that $\recSig(\tn)$ returns back close to its initial value $\recSig(\tn) = 1.0$ at a later time when the irradiation time is increased from $\Ti = 10 \, \si{\second}$ (solid lines) to $\Ti = 20 \, \si{\second}$ (dotted line), consequently increasing the channel \ac{ISI}. Furthermore, we observe that the depth of the fluorescence intensity drop of the first pulse remains the same, while the inter-loop \ac{ISI} intensity increases significantly. From this, we infer that both the irradiation time $\Ti$ and which symbol $\symVar$ is transmitted have an influence on the inter-loop \ac{ISI}. 
Moreover, \Figure{fig:SIR} shows that the irradiation time $\Ti$ and which symbol $\symVar$ is transmitted also have an impact on the offset \ac{ISI}.
%
These observations suggest that there exists an optimal irradiation duration $\Ti$. In particular, there exists a trade-off as $\Ti$ should be chosen large enough to produce a detectable signal, but small enough to prevent the undesired accumulation of \ac{ISI}, as all forms of \ac{ISI} increase when $\Ti$ is increased.

Finally, we observe from \Figure{fig:SIR} that increasing the transmission distance from $d_{\mathrm{TX},\mathrm{RX}} = 6 \, \si{\centi\meter}$ (no marker) to $d_{\mathrm{TX},\mathrm{RX}} = 35 \, \si{\centi\meter}$ (rectangular markers) results in a later arrival of the first pulse at the \ac{RX}, which is expected as the signal molecules have to propagate a longer distance. Furthermore, we observe that the pulse shape remains the same. Hence, we conclude that even a transmission distance of $d_{\mathrm{TX},\mathrm{RX}} = 35 \, \si{\centi\meter}$ is too small for diffusion to have an impact.
\scaleSubsection
\subsection{Transmission and Error-Free Detection of 90 kbit}\label{sec:error_free_80kbit}
\scaleSubsectionBelow
%
\begin{figure}[!tbp]
    \vspace{-1cm}
    \centering
    \begin{subfigure}[b]{1\textwidth}
        \vspace{-0.cm}
        \includegraphics[width=\textwidth]{fig/RX_Signal_Long_Experiments.png}
    \end{subfigure}\\
    \vspace{-6.7cm}
    \begin{subfigure}[b]{1\textwidth} %a)
        \caption{}
        \label{fig:long_term_time}
    \end{subfigure}\\
    \vspace{5.5cm}
    \begin{subfigure}[b]{1\textwidth}
        \caption{}
        \vspace{-0.cm}
        \includegraphics[width=\textwidth]{fig/Detection_Long_Experiments.png}
        \label{fig:long_term_correlation}
        \vspace*{-8mm}
    \end{subfigure}\\
    \vspace{-0.4cm}
    \begin{subfigure}[b]{1\textwidth}
        \caption{}
        \vspace{-0.cm}
        \includegraphics[width=\textwidth]{fig/AMED_Long_Experiments.png}
        \label{fig:long_term_med}
        \vspace*{-8mm}
    \end{subfigure}\\
    \vspace{-0.4cm}
    \begin{subfigure}[b]{1\textwidth}
        \caption{}
        \vspace{-0.1cm}
        \includegraphics[width=\textwidth]{fig/BER_Long_Experiments.png}
        \label{fig:long_term_ber}
        \vspace*{-8mm}
    \end{subfigure}
    \vspace{-0.8cm}
    \caption{Evaluation of a long-term binary transmission with $\Ts = \SI{5}{\s}$, $\nBits = \num{90000}$, $\pFA = 10 \, \times \, 10^{-10}$, $\Delta t = \SI{0.1}{\s} $, $\filterLength = \lfloor \frac{\Ts}{\Delta t} \rfloor$, $\trainLength = 50$, $r = 0.04$, $W = 50$, $F = 1$, $\nSkip = 80$, and $P = 130$ with (blue curves) and without (orange curves) \ac{EX}. (a) Received signal over time. Shaded areas in the insets correspond to bit 1 transmissions. (b) Detection samples for different detection schemes using data-based receive filters and the corresponding thresholds around $\SI{27}{\hour}$ after starting data transmission (left panel) and at the end of transmission (right panel). (c) \ac{AMED} between bit 1 and bit 0 transmissions over time. (d) Moving average \ac{BER} (computed over the most recent $\num{2000}$ bit) for the different detection schemes. }
    \label{fig:longterm}
    \vspace*{-5mm}
\end{figure}
%
In this section, we evaluate the performance of the testbed for long-term transmissions. The results are provided in \Figure{fig:longterm}. To this end, two experiments, one with \ac{EX} and one without \ac{EX}, were conducted over a period of 125 hours each, during which, in each case, $\nBits = \num{90000}$ were transmitted using binary modulation at a data rate of $\SI{12}{\bitsperminute}$. The $\num{90000}$ bit were generated using a random $\num{2000}$ bit sequence, which was repeatedly transmitted $45$ times. These experiments aim to demonstrate the key advantage of the proposed waste-free, self-contained, and closed-loop testbed, namely enabling the long-term study of tube-based MC systems. During each experiment, the testbed was not subjected to any alterations. Specifically, neither new \ac{GFPD} molecules were introduced nor were any substances extracted from the testbed. Instead, the same $9 \,\si{\milli\liter}$ of \ac{GFPD} solution was employed throughout each experiment.
%
% a)
\scaleSubsubsection
\subsubsection{Received Fluorescence Signal}\label{sssec:time_signal}
\scaleSubsubsectionBelow
%
\Figure{fig:long_term_time} shows the received signal $r(\tn)$ over time for both experiments with (blue curve) and without (orange curve) \ac{EX}. The inset plots show the signals at three particularly interesting time instants: the beginning (left), the time of an (undesired and unexpected) fluorescence jump in one of the received signals (center), and the end of the transmission (right), alongside with the corresponding \ac{TX} signal (green). Here, gray shaded areas in the inset plots indicate bit "1" transmissions. The inset plots of \Figure{fig:long_term_time} show that, as expected, $r(\tn)$ exhibits large drops in fluorescence intensity for bit 1 transmissions and small drops in fluorescence intensity for bit 0 transmissions. This indicates that reliable communication is possible. We further observe from \Figure{fig:long_term_time} that the general fluorescence level of the system decreases over time. This is due to offset \ac{ISI} and photobleaching effects, i.e., the prevalence of temporarily and permanently off-switched \acp{GFPD}, respectively. We observe that, as expected, in the beginning ($t < \SI{60000}{\s}$), the average fluorescence level with \ac{EX} is higher than without \ac{EX}. However, later, the fluorescence for the case with \ac{EX} clearly drops below that without \ac{EX}. This indicates that the \ac{EX} successfully mitigates offset \ac{ISI}, which is the predominant immediate cause of a fluorescence decrease. At the same time, the \ac{EX} enhances photobleaching as its use increases the number of photons that hit the \ac{GFPD} molecules. The signal depicted in blue in the right inset plot reveals the increased photobleaching with \ac{EX}, which results in a performance degradation. In particular, in this case, the number of \ac{GFPD} molecules in the ON state that are available for off-switching at the \ac{TX} is very low. The modulation at the \ac{TX} thus exhibits fluorescence drops that cannot be discerned from measurement noise. This is not the case for the scenario without \ac{EX}. 

One peculiar feature of the considered experiment is the rapid fluorescence decrease ($t \approx \SI{60000}{\s}$) and subsequent jump ($t \approx \SI{110000}{\s}$) for the scenario with \ac{EX}. While it is difficult to prove, we believe that this behavior is caused by an agglomeration of degraded \ac{GFPD} molecules in the flow cell, i.e., the \ac{RX}. Since degradation is caused by photobleaching, we expect this agglomeration will be more likely to occur in scenarios with \ac{EX} than without \ac{EX}. An agglomeration can significantly alter the measured fluorescence signal, as it acts as an undesirable physical light filter. This hypothesis matches the observed behavior, as a spontaneous detachment of the agglomeration would explain the re-establishment of the previous fluorescence level at $t \approx \SI{110000}{\s}$.
%
% b)
\scaleSubsubsection
\subsubsection{Detection Samples}\label{sssec:detection_samples}
\scaleSubsubsectionBelow
%
\noindent \Figure{fig:long_term_correlation} shows the detection samples, cf. \Equation{eq:detection_sample}, obtained using the \ac{CS} (top panels) and \ac{DCS} schemes (bottom panels) alongside the respective adaptive thresholds. We focus on two particularly interesting time intervals: The left panels focus on the time around the fluorescence jump, while the right panels show the samples at the end of the transmission. Orange and blue dots correspond to the detection samples, cf. \Equation{eq:detection_sample}, for the setting without and with \ac{EX}, respectively. Comparing the top and bottom panels, we observe that the threshold for the \ac{DCS} scheme (bottom) quickly adapts to the fluorescence jump, while errors are caused for the \ac{CS} scheme. The fluorescence jump, which predominantly results in an offset of the fluorescence, exerts a lower impact on the differential signal. Moreover, \Figure{fig:long_term_correlation} confirms our prior observation that the experiment without \ac{EX} exhibits a more stable long-term behavior. For instance, we observe from the right-hand panels in \Figure{fig:long_term_correlation} that the bit 1 and bit 0 samples are still clearly separated by the threshold in the scenario without \ac{EX} even after $\SI{400000}{\s}$ ($\approx$ 111h) of continuous transmission, while for the scenario with \ac{EX}, the samples cannot be separated by either detection scheme. Interestingly, the advantage of the \ac{DCS} scheme vanishes at the end of the transmission for the experiment with \ac{EX}. The reason for this behavior is that the signal at this point is similar in magnitude to the noise. As the differentiation operation amplifies the noise, it starts to dominate the signal, which degrades the communication performance.
%
% c)
\scaleSubsubsection
\subsubsection{AMED Analysis}\label{sssec:amed_long}
\scaleSubsubsectionBelow
%
\noindent \Figure{fig:long_term_med} shows the \ac{AMED}, cf. \Equation{eq:amed}, for the scenarios with and without \ac{EX} over time. The metric is computed continuously over windows of $\nWindow = 500$ bit. As expected, the \ac{AMED} generally decreases over time, regardless of whether the \ac{EX} is used or not. Initially, when the offset \ac{ISI} is the dominant impairment, the use of the \ac{EX} leads to a higher \ac{AMED}. However, as photobleaching caused by the \ac{EX} becomes pronounced, the \ac{AMED} drops below that of the scenario without \ac{EX}. In addition, the effect of the fast fluorescence decrease and subsequent jump is clearly visible in \Figure{fig:long_term_med}. This is consistent with our previous observations for the fluorescence signal and the detection samples. Based on the \ac{AMED}, the benefits of the \ac{EX} for short transmissions become apparent, but they disappear for long sequences.
%
% d)
\scaleSubsubsection
\subsubsection{BER Analysis}\label{sssec:ber_long}
\scaleSubsubsectionBelow
%
Finally, \Figure{fig:long_term_ber} shows the empirical moving average \ac{BER}, i.e., the \ac{BER} calculated sequentially over the most recent $\num{2000}$ symbols using \Equation{eq:BER}, as a function of time for the different considered receive filters and the setting with and without \ac{EX}, respectively. Plotting the \ac{BER} this way enables us to analyze the causes of detection errors and whether the detectors allow for an adaptation to avoid subsequent errors or not. In \Figure{fig:long_term_ber}, the solid and dotted lines correspond to the use of the \ac{CS} and \ac{DCS} schemes, respectively, while the color indicates whether or not the \ac{EX} was active. The diamond-shaped and round markers specify whether data-based or blind filters were used. We observe that, while the \ac{EX} is in principle beneficial for short transmission times, for long transmission times, the \ac{BER} is larger for the case with \ac{EX}. This is, on the one hand, caused by the abnormal fast fluorescence decrease and fluorescence jump, and on the other hand, by the more severe photobleaching and the resulting low fluorescence level leading to a low signal-to-noise ratio. The former manifests itself in the sudden jump in the \ac{BER} to $0.5$ for the \ac{CS} scheme around the occurrence of the fluorescence jump. Detection errors end when the fluorescence is back roughly to the level at which it was before the jump, such that the threshold becomes valid again. The decreasing signal-to-noise ratio manifests itself in a gradual increase of the \ac{BER} over time. Here, the \ac{BER} rises first for the \ac{DCS} scheme, which can be attributed to the noise enhancement caused by differentiation. In contrast, without \ac{EX}, transmission remains error-free for more than $\SI{300000}{\s}$ for both considered detection schemes. In fact, error-free transmission of around $\nBits = \num{90000}$ bit\footnote{To be precise, with $\nSkip = 80$ and $\nPilots = 130$, $\nBits = 89790$ bit were detected error-free.} is possible with the \ac{DCS} scheme and the data-based receive filter $g^{\mathrm{D}}_{\mathrm{D}}(\tn)$.
%
\scaleSubsection
{\sisetup{detect-all} % Activate style detection only here
\subsection{BER vs. Data Rate: 8-ary Modulation with an Achievable Rate of \textit{\SI{36}{\bit \per \minute}}}\label{sec:higher_order_modulation_rocks}}
\scaleSubsectionBelow
%
\begin{figure}[!tbp]
    \centering
        \vspace{-1cm}
    \includegraphics[width =1\columnwidth]{fig/BER_over_Data_Rate.pdf}\vspace*{-1mm}
    \caption{\acp{BER} for 18 individual experiments. Here, $\pFA = 10 \, \times \, 10^{-8}$, $\Delta t = \SI{0.1}{\s} $, $\filterLength = \lfloor \frac{\Ts}{\Delta t} \rfloor$, $\trainLength = 50$, $W = 50$, $F = 1$, $\nSkip = 80$, $P = 130$, and different search radius lengths $r \in [0.03, 0.12]$ are used. The red circles mark results obtained for $d_{\mathrm{TX},\mathrm{RX}} = 35 \,\si{\centi\meter}$.}\vspace*{-1.1cm}
    \label{fig:ber}
\end{figure}
\vspace{1mm}
%
We evaluate our testbed in terms of the achievable \ac{BER} for various different data rates. Specifically, we experimentally determined the \ac{BER} for 18 individual experiments, in each of which $\num{2000}$ symbols (corresponding to $\num{2000}$, $\num{4000}$, or $\num{6000}$ bit, depending on the modulation order) were transmitted. For each single experiment, a fresh \ac{GFPD} sample was used in order to prevent dependencies between the experiments. For all results shown in this section, the \ac{EX} was used. As we use $\nSkip = 80$ and $P = 130$, the \ac{BER} values were determined based on $1,790$ symbols.

\Figure{fig:ber} shows the achieved \acp{BER} for data rates between $\SI{8}{\bitsperminute}$ and $\SI{60}{\bitsperminute}$. In particular, modulation orders $M = 2$ (yellow), $M = 4$ (green), and $M = 8$ (blue), and symbol durations between $\Ts = \SI{2}{\s}$ and $\Ts = \SI{15}{\s}$ are used. The individual experiments thus correspond to transmission durations between $\SI{67}{\minute}$ and $\SI{500}{\minute}$. Furthermore, we present results for a transmission channel length of $d_{\mathrm{TX},\mathrm{RX}} = 35 \,\si{\centi\meter}$ (marked by red circles) to provide insight into the effect of the channel length and to prove that reliable communication is still feasible for longer channels. Experiments for which no errors were observed are displayed at the bottom of \Figure{fig:ber} after the axis break.
%
\scaleSubsubsection
\subsubsection{Effect of Modulation Order}\label{sssec:effect_modulation_order}
\scaleSubsubsectionBelow
%
\Figure{fig:ber} shows that, for all considered modulation orders examined, error-free transmission was experimentally achieved within a certain data rate region (for $M=2$ from $\SI{12}{\bitsperminute}$ up to $\SI{20}{\bitsperminute}$, for $M=4$ from $\SI{8}{\bitsperminute}$ up to $\SI{30}{\bitsperminute}$, and for $M=8$ from $\SI{24}{\bitsperminute}$ up to $\SI{36}{\bitsperminute}$).
For $M=2$ and $M=4$, we observe that the \ac{BER}, as expected, increases with increasing data rate. For $M=8$, when comparing the \acp{BER} for $\SI{24}{\bitsperminute}$ and $\SI{12}{\bitsperminute}$, we see that the \ac{BER} is larger at $\SI{12}{\bitsperminute}$, which is not intuitive. However, when analyzing the received signal for this setting\footnote{Unfortunately, the mentioned received signal cannot be shown in this paper due to space constraints. However, of course, the data for this received signal is included in the dataset that we have made publicly available on Zenodo, cf. \Section{ssec:datasharing}.}, we observed a gradual decrease in fluorescence intensity, similar to that described in \Section{sssec:time_signal}. We attribute this to the severe photobleaching when using $\Ts = \SI{15}{\s}$ with irradiation duration $\Ti = \SI{10}{\s}$. Therefore, the resulting weak received signal does not support error-free transmission for 8-ary modulation for this data rate.

Note that, for a given modulation order $M$, we increase the data rate by reducing the symbol duration $\Ts$. However, $\Ts$ cannot be reduced arbitrarily, as this leads to an increase in channel \ac{ISI}, which in turn results in an increased \ac{BER}. Specifically, \Figure{fig:ber} reveals that only $M=8$ can achieve error-free transmission at rates greater than $\SI{30}{\bitsperminute}$. Hence, for error-free transmission at high data rates, an increase in modulation order is necessary. For such a setting, where $M=8$ and $\Ts = \SI{5}{\s}$ are used, details are provided next.

\begin{figure}[!tbp]
\centering
    \vspace{-0.5cm}
  \includegraphics[width = \columnwidth]{fig/higher_mod.pdf}\vspace*{-6mm}
  \caption{Detection samples as defined in \Equation{eq:detection_sample} and adaptive thresholds over time. Here, $\modOrder = 8$, $\Ts = \SI{5}{\s}$, $\Delta t = \SI{0.1}{\s}$, $\filterLength = \lfloor \frac{\Ts}{\Delta t} \rfloor$, $\trainLength = 50$, $W = 50$, $F = 1$, $\nSkip = 80$, $P = 130$, and $r = 0.08$ are used. The detection samples are color-coded based on the corresponding transmit symbol $\symVar[k] \in \{0, \ldots, M - 1\}$. Inset plot: Corresponding received signal and \ac{TX} intensity.}
  \label{fig:M8_detection}
  \vspace*{-0.8cm}
\end{figure}
%
\Figure{fig:M8_detection} shows the detection samples $\detecSample[\symIdx]$ for the \ac{DCS} scheme using \ac{SDCF} as receive filter. In \Figure{fig:M8_detection}, the different colors indicate the transmitted symbols $\symVar[k] \in \{0,\ldots,\modOrder - 1\}$, while the dash-dotted lines in the red monochrome color palette correspond to the adaptive thresholds $\modOrder - 1$ used for symbol detection. Additionally, the inset in \Figure{fig:M8_detection} shows the \ac{TX} intensity and the received signal for the symbols transmitted in the interval $t \in [2800, 3100] \, \si{\s}$.

From the inset in \Figure{fig:M8_detection}, we observe that the \ac{TX} intensity is reflected in visible changes of the received signal, which shows distinct peaks. We further observe from \Figure{fig:M8_detection} that, even for 8-ary modulation, the samples for the \ac{DCS} scheme are well distinguishable \ac{wrt} the adaptive threshold values. Moreover, \Figure{fig:M8_detection} shows that the experiment is in a transient phase at the beginning, which leads to a high variance of the sample values. The transient phase is caused by offset \ac{ISI}, which develops slowly over successive transmissions and remains relatively constant after about 50 to 100 symbols. Therefore, the omission of the first $\nSkip = 80$ symbols is crucial to ensure that the thresholds $\detecThresh_\detecThreshIdx$ can be adequately initialized afterwards. 
%
\scaleSubsubsection
\subsubsection{Effect of Receive Filter and Detection Method}\label{sssec:effect_filter_and_detection}
\scaleSubsubsectionBelow
%
In \Figure{fig:ber}, we can also compare the \acp{BER} for the four filters proposed for detection: \ac{SCF} (solid, leftwards pointing marker), \ac{BCF} (dashed, upwards pointing marker), \ac{SDCF} (dotted, rightwards pointing marker), and \ac{BDCF} (dash-dotted, downwards pointing marker). Detection based on the differential signal, i.e., \ac{SDCF} and \ac{BDCF}, in many cases leads to a significantly lower \ac{BER} in direct comparison to \ac{SCF} and \ac{BCF}. Furthermore, \Figure{fig:ber} shows that, as expected, the data-based filters (\ac{SCF} and \ac{SDCF}) are superior to the blind filters (\ac{BCF} and \ac{BDCF}), as evidenced by the lower \ac{BER}.

We conclude that the \ac{DCS} scheme is well suited for detection of higher-order modulation when used in combination with \ac{EX}. However, the \ac{DCS} scheme can be disadvantageous in long experiments, where the signal eventually becomes weak due to photobleaching, because of the noise enhancement introduced by the differentiation operation. In this case, the \ac{CS} scheme is preferable, cf. \Section{sec:error_free_80kbit}.
%
\scaleSubsubsection
\subsubsection{Effect of Transmission Distance}\label{sssec:effect_distance}
\scaleSubsubsectionBelow
%
Finally, \Figure{fig:ber} shows the effects of the elongation of the channel from $d_{\mathrm{TX},\mathrm{RX}} = 6 \,\si{\centi\meter}$ to $d_{\mathrm{TX},\mathrm{RX}} = 35 \,\si{\centi\meter}$. Three distinct settings have been analyzed: $M=4$ with data rate $\SI{8}{\bitsperminute}$, $M=2$ with data rate $\SI{12}{\bitsperminute}$, and $M=4$ with data rate $\SI{24}{\bitsperminute}$. For these settings, error-free transmission is still possible even for $d_{\mathrm{TX},\mathrm{RX}} = 35 \,\si{\centi\meter}$ when using the \ac{SDCF} as receive filter\footnote{Note that only the markers for $\textnormal{BER}>0$ are shown with the red circle. Error-free results for a distance of $35 \,\si{\centi\meter}$ are not indicated by a marker to avoid overloading the figure around $\textnormal{BER}=0$.}. Moreover, we observe from \Figure{fig:ber} that for the detection schemes, where detection errors occur, the corresponding errors are larger for distance $d_{\mathrm{TX},\mathrm{RX}} = 35 \,\si{\centi\meter}$ compared to $d_{\mathrm{TX},\mathrm{RX}} = 6 \,\si{\centi\meter}$, which was expected. For example, for $M=4$ and the \ac{CS} scheme, the \ac{BER} increases by one order of magnitude.
%
\scaleSubsection
\subsection{EX-Based ISI Mitigation in Short-Term Transmissions}\label{ssec:eye_res}
\scaleSubsectionBelow
%
\begin{figure}[!tbp]
    \centering
        \vspace{-1cm}
    \begin{tabular}{@{} c @{\hspace{1em}} c @{\hspace{1em}} c @{}}
        & \small{$\disTX = \SI{6}{\cm}$} & \small{$\disTX = \SI{35}{\cm}$} \\[-0.05cm]
        \raisebox{1.75\height}{\rotatebox{90}{\small{w \ac{EX}}}} &
        \includegraphics[width=0.45\textwidth]{fig/eye_6cm_wEX_update.pdf} &
        \includegraphics[width=0.45\textwidth]{fig/eye_35cm_wEX_update.pdf} \\[-0.5cm]
        \raisebox{1.25\height}{\rotatebox{90}{\small{w/o \ac{EX}}}} &
        \includegraphics[width=0.45\textwidth]{fig/eye_6cm_woEX_update.pdf} &
        \includegraphics[width=0.45\textwidth]{fig/eye_35cm_woEX_update.pdf}
    \end{tabular}\vspace*{-0.2mm}
    \caption{Eye diagrams. Here, $\modOrder=4$, $\Ts = \SI{5}{\s}$, $\Ti = \SI{3}{\second}$, and $\Delta t = \SI{0.1}{\s}$ are used.}
      \vspace*{-1.1cm}
    \label{fig:eye_diagram}
\end{figure}
%
In this section, we show that the \ac{EX} is capable of \ac{ISI} mitigation in short-term transmission. \Figure{fig:eye_diagram} shows eye diagrams for $\modOrder = 4$ and different experimental settings, including varying channel lengths and \ac{EX} settings. The curves correspond to symbols $\symVar[k]$ for $k \in  \{100, 101, \ldots, 299\}$ out of a $\nSymb = \num{2000}$ symbol long sequence being transmitted. Synchronization was achieved using the \ac{CS} scheme with \ac{SCF} as receive filter for all scenarios. We horizontally shifted the received signal $r(\tn)$ for each transmit symbol to have its maximum at 1 for better visibility. In \Figure{fig:eye_diagram}, different transmit symbols $\symVar$ are denoted by different colors.

When the \ac{EX} is used (top panels in \Figure{fig:eye_diagram}), we observe three eyes, indicating that the different transmit symbols can be distinguished easily. This is especially evident for the short channel length of $\disTX = \SI{6}{\centi\meter}$ (top left panel). Elongating $\disTX$ results in a slightly smaller eye opening. Thus, the larger $\disTX$ has only a small effect, which shows that, as analyzed theoretically in \Section{subsec:propagation}, diffusion affecting the TX-RX link can be neglected for the $\disTX$ considered in this work.

If the \ac{EX} is turned off, we see that the eyes are closed (bottom panels in \Figure{fig:eye_diagram}), i.e., the received signals for the different transmit symbols $\symVar$ are largely overlapping.
In particular, we observe that the received signals for the same symbol $\symVar$ vary more without \ac{EX} compared to when the \ac{EX} is used, which we attribute to larger inter-loop and offset \ac{ISI}. We observe from \Figure{fig:eye_diagram} that for the considered settings, reliable communication using 4-ary modulation without \ac{EX} can be difficult and may require sophisticated equalization techniques at the \ac{RX}.

In summary, employing an \ac{EX} has some benefits (\ac{ISI}-mitigation, cf. \Section{ssec:eye_res}), but also undesired side-effects (photobleaching of \ac{GFPD}, cf. \Section{sec:error_free_80kbit}).
Based on our results, we conclude that the use of an \ac{EX} is advisable for short-term transmission, high modulation orders, and short symbol durations $\Ts$.
