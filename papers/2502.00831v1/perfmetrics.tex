In this section, we describe the quantitative and qualitative communication-specific metrics employed to evaluate \ac{MC} performance in the proposed testbed.
\vspace{0.1cm}
\scaleSubsection
\subsection{Bit Error Rate and Data Rate}\label{sec:ber_and_data_rate}
\scaleSubsectionBelow
%
We define an \textit{empirical} \ac{BER} for our system. The \ac{BER} is a standard metric for quantifying the performance of any communication scheme. For the bit-to-symbol mapping, we use Gray mapping \cite[p. 100]{proakis2008digital} $G: \{0,1\}^{\log_{2}(M)\nSymb} \rightarrow \{0, \ldots, M - 1\}^{\nSymb}$. Hence, the bit sequences corresponding to symbol sequences $\mathbf{i} \in \{0, 1, \ldots , M - 1\}^{\nSymb}$ and $\hat{\mathbf{i}} \in \{0, 1, \ldots , M - 1\}^{\nSymb}$, i.e., the transmitted and estimated symbol sequences, are obtained as $\mathbf{i}_{G} = G^{-1}(\mathbf{i})$ and $\hat{\mathbf{i}}_G = G^{-1}(\hat{\mathbf{i}})$, respectively. In our case, since the use of pilot symbols is required for calibration of all our detection techniques, the \ac{BER} is obtained as
%
\begin{equation}
    \mathrm{BER} = \frac{\sum_{b = \log_2(M)\nPilots}^{\log_{2}(M)\nSymb - 1} \left|\mathbf{i}_{G}[b] - \hat{\mathbf{i}}_G[b]\right|}{(\nSymb - \nPilots)\log_2(M)}\;.
    \label{eq:BER}
\end{equation}
Note that this equals the Hamming distance between $\mathbf{i}_{G}$ and $\hat{\mathbf{i}}_G$ normalized to the bit sequence length excluding the sequence part corresponding to the pilot symbols $\nPilots$. Thus, the empirically defined \ac{BER} is the number of erroneously demodulated bits scaled by the number of all transmitted bits.
\vspace{0.1cm}
\scaleSubsection
\subsection{Absolute Mean Euclidean Distance}
\scaleSubsectionBelow
%
In some cases the \ac{BER} is not insightful, e.g., if the transmission sequence length is too short for an accurate estimation of the \ac{BER} or if the \ac{MC} system is in the low \ac{BER} regime, where irrespectively of the chosen settings, for the evaluated transmission sequence length, $\textnormal{BER}=0$ is obtained.
For such cases, we introduce the \ac{AMED} metric as follows
\begin{equation}
   \AMED = \min_{i' \neq i''} |\Bar{S}_{i'}[l] - \Bar{S}_{i''}[l]| \;; \qquad \forall i', i'' \in \{0, 1, \ldots, \modOrder - 1\} \;,
    \label{eq:amed}
\end{equation}
where we employ \Equation{eq:average_set} and \Equation{eq:sampleSets}.

As can be seen from \eqref{eq:amed}, the \ac{AMED} is the minimum Euclidean distance between the means of the samples $\detecSample[\symIdx]$ received over $\nWindow$ symbol intervals. Consequently, the \ac{AMED} reflects the extent to which the received samples of the various symbols differ. Compared to the \ac{BER}, the \ac{AMED} provides reliable values even for small numbers of samples, i.e., small $\nWindow$, and can therefore serve as an initial indicator of performance for cases where a \ac{BER} evaluation is not meaningful or not possible.
\vspace{0.1cm}
\scaleSubsection
\subsection{Eye Diagram}
\scaleSubsectionBelow
Eye diagrams are a well-established qualitative method to evaluate a communication system's performance \cite[p. 603]{proakis2008digital}. The use of eye diagrams to evaluate \ac{MC} systems has been introduced in \cite{farsad2017novel}. Plotting the received signal modulo the symbol duration generates the eye pattern. Effects such as \ac{ISI}, various system parameters, including the use of a guard interval, and the effects of varying the modulation order affect the eye pattern and the eye opening, which can provide insights for system design and performance.

