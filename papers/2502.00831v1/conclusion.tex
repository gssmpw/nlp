In this paper, we presented the first self-contained, closed-loop experimental \ac{MC} system using media modulation. As the testbed leverages the reusable and biocompatible signaling molecule \ac{GFPD}, long-term experiments without the need for repeated injection or removal of molecules were possible.

We developed a communication scheme which features higher order modulation, a noise-based wake-up method, blind and data-based synchronization, and adaptive threshold detection. Utilizing key performance metrics, such as \ac{AMED} and \ac{BER}, the quality and reliability of the testbed were demonstrated. Notably, we achieved error-free transmission of $\num{5790}$ bit at $36 \, \si{\bitsperminute}$. Moreover, we conducted the longest \ac{MC} experiment to date, both \ac{wrt} the number of bits transmitted as well as the duration of the transmission. In particular, $\num{90000} \, \si{bit}$ were transmitted error-free over a period exceeding $5$ days ($125 \, \si{\hour}$) at a data rate of $12 \, \si{\bitsperminute}$, thereby setting a novel benchmark for long-term MC experiments. In total, only $9 \, \si{\milli \liter}$ of \ac{GFPD} solution were required for this long experiment, which serves to illustrate the efficacy of the media modulation approach.
In order to encourage further research, we have made the experimental data (> 250 kbits) and the corresponding evaluation code available via open access on Zenodo and Github, respectively.
This enables researchers to explore advanced methods (e.g., based on machine learning) for improving synchronization and detection performance, which are interesting future research directions.
Moreover, the testbed can easily be expanded due to its modular and flexible design. For example, the incorporation of a branched tube network, which could emulate real biological environments, is of high practical interest.