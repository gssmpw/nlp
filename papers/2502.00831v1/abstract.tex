We present a fluid-based experimental molecular communication (MC) testbed which uses media modulation. Motivated by the natural human cardiovascular system, the testbed operates in a closed-loop tube system.
The proposed system is designed to be biocompatible, resource-efficient, and controllable from outside the tube. As signaling molecule, the testbed employs the green fluorescent protein variant "Dreiklang" (GFPD). GFPDs can be reversibly switched via light of different wavelengths between a bright fluorescent state and a less fluorescent state. GFPDs in solution are filled into the testbed prior to the start of information transmission and remain there for an entire experiment. For information transmission, an optical transmitter (TX) and an optical eraser (EX), which are located outside the tube, are used to write and erase the information encoded in the state of the GFPDs, respectively. At the receiver (RX), the state of the GFPDs is read out by fluorescence detection.
In our testbed, due to the closed-loop setup and the long experiment durations of up to 125 hours, we observe new forms of inter-symbol interferences (ISI), which do not occur in short experiments and open-loop systems. In particular, up to four different forms of ISI, namely channel ISI, inter-loop ISI, offset ISI, and permanent ISI, occur in the considered system.
For the testbed, we developed a communication scheme, which includes blind transmission start detection, symbol-by-symbol synchronization, and adaptive threshold detection, that supports higher order modulation.
We comprehensively analyze our MC experiments using the absolute mean Euclidean distance (AMED), eye diagram, and bit error rate (BER) as performance metrics.
Furthermore, we experimentally demonstrate the error-free transmission of $\num{5370}$ bit at a data rate of 36 $\textrm{bit}\, \textrm{min}^{\boldsymbol{-1}}$ using 8-ary modulation and the error-free binary transmission of around $\num{90000}$ bit at a data rate of 12 $\textrm{bit}\, \textrm{min}^{\boldsymbol{-1}}$. For the latter experiment, data was transmitted continuously for a period of more than five days (125 hours) during which no signaling molecules were injected into or removed from the system.
All signals recorded during the experiments, representing more than 250 kbit of data transmitted via MC, and parts of the evaluation code are publicly available on Zenodo and Github, respectively.