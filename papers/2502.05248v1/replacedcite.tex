\section{Literature Review: }
Notable connections of LLM with psychology began with the advent of models such as GPT-3 ____. Various perspectives of psychology are infused to understand behaviours from multiple dimensions such as emotion, cognition____, Theory-of-Mind, and morality ____. Identifying personality traits has been one of the main emphasises in the field of study ____. Psychometric tools such as IPIP-NEO-120____, Big Five Inventory (BFI) ____, are predominantly used in the literature to assess the five dimensions of personality mentioned above.

Variations in prompts and the use of role-playing agents have been impersonated in several studies to study different personality behaviours in LLMs____. The major aim behind these studies was to assess whether the behaviour is consistent or changing across different simulated situations. Context-sensitive variations were observed for several scenarios____. Jiang et al. ____ used personality prompting methods to induce and tailor the personality of LLMs according to the dynamic needs of the tests. This also draws attention to the need to enforce ethical factors for LLMs to create a safe and moderated environment for users. 

A limited number of LLMs and personality traits sharing the same domain in the literature restrict the scope of analysis. The small sample size at this exploratory stage could impact the validity of results across models. A significant concern is the use of these tests in their original form, raising the risk of training data contamination and potential bias due to the sequential nature of questions within the same Big Five Personality dimension ____.

\begin{table}[h!]
\centering
\caption{Personality Trait Inventories used  denoting the number of questions, Strengths and Weaknesses.}
\small
\begin{tabular}{@{}llp{4.5cm}@{}}
\toprule
\textbf{Test Name} &  \textbf{Questions }& \textbf{Strengths ($\uparrow$) and Weaknesses($\downarrow$)} \\
\midrule
BFI____ & 44(8-10 each) & \textbf{$\uparrow$:} Reliable, widely validated \newline \textbf{$\downarrow$:} Depth, time-consuming \\
\hline
HEXACO____ & 100(10 each) & \textbf{$\uparrow$:} cross-cultural validity. \newline \textbf{$\downarrow$:} Lengthy, complex  \\
\hline
TIPI____ & 10(2 each) & \textbf{$\uparrow$:}  quick, easy to administer. \newline \textbf{$\downarrow$:} medium reliability \\
\hline
MINI-IPIP____ & 20(4 each) & \textbf{$\uparrow$}  balances brevity and validity \newline \textbf{$\downarrow$:} Limited depth \\
\hline
NEO-PI-R____ & 60(12 each) & \textbf{$\uparrow$} More depth, reliable. \newline \textbf{$\downarrow$:} No facet analysis. \\
\bottomrule
\end{tabular}
\label{tab:personality_tests}
\end{table}