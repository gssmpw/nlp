
\begin{abstract}
When reading a document, glancing at the spatial layout of a document is an initial step to understand it roughly. Traditional document layout analysis (DLA) methods, however, offer only a superficial parsing of documents, focusing on basic instance detection and often failing to capture the nuanced spatial and logical relations between instances. 
These limitations hinder DLA-based models from achieving a gradually deeper comprehension akin to human reading. In this work, we propose a novel graph-based Document Structure Analysis (\textbf{gDSA}) task. This task requires that model not only detects document elements but also generates spatial and logical relations in form of a graph structure, allowing to understand documents in a holistic and intuitive manner. For this new task, we construct a relation graph-based document structure analysis dataset (\textbf{GraphDoc}) with 80K document images and 4.13M relation annotations, enabling training models to complete multiple tasks like reading order, hierarchical structures analysis, and complex inter-element relation inference. Furthermore, a document relation graph generator (\textbf{DRGG}) is proposed to address the gDSA task, which achieves performance with \textbf{57.6\%} at mAP$_g$@$0.5$ for a strong benchmark baseline on this novel task and dataset. We hope this graphical representation of document structure can mark an innovative advancement in document structure analysis and understanding. The new dataset and code will be made publicly available at \href{https://yufanchen96.github.io/projects/GraphDoc}{GraphDoc}.
\end{abstract}