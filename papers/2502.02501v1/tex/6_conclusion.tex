\vspace{-12px}
\section{Conclution}
In this paper, we introduced the GraphDoc dataset and proposed a novel graph-based document structure analysis (gDSA) task. By capturing spatial and logical relations among document layouts, we significantly enhanced the understanding of document structures beyond traditional layout analysis methods. Furthermore, we developed the DRGG, an end-to-end architecture that effectively generated relational graphs reflecting the complex interplay of document layouts. As an auxiliary module, DRGG leveraged both spatial and logical relations to improve document structure analysis tasks. We conducted extensive experiments, and the results demonstrated that DRGG achieved superior performance on the gDSA task, attaining an mR$_g$@$0.5$ of $30.7\%$ and mAP$_g$@$0.5$, $0.75$, and $0.95$ scores of $57.6\%$, $56.3\%$, and $46.5\%$, respectively. This performance enhanced the effectiveness of combining document layout analysis with relation prediction to capture document structures.

\noindent \textbf{Limitations}. Our model structure focused only on visual modality input without multi-modality input consideration, which may have influenced the performance of complex document structure analysis. Future work should explore this integration to enhance the model's performance on relational graph prediction. Additionally, our dataset and approach were primarily designed for single-page documents, and extending them to effectively include multi-page documents posed a challenge that remained unaddressed. We acknowledged these limitations and believed that addressing them would be essential for making significant strides toward achieving a human-like understanding of documents, paving the way for intelligent document processing systems.