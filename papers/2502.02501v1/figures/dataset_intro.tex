
\vspace{-10px}
\begin{center}
    \centering
    \captionsetup{type=figure}
    \centering
    \begin{subfigure}[t]{0.6\textwidth}
        \centering
        \adjustimage{width=\textwidth}{figures/GraphDoc.pdf}
        \caption{Document graph structure of layouts with different relations.} \label{fig1-a}
    \end{subfigure}%\hfill
    \begin{subfigure}[t]{0.4\textwidth}
        \centering
        \includegraphics[width=\textwidth]{figures/performance_radar.pdf}
        \caption{Performance of our proposed method.} \label{fig1-b}
    \end{subfigure}%\hfill
    \setcounter{figure}{0} % start from 0
    %\vskip -0.5ex
    \captionof{figure}{ \textbf{GraphDoc Dataset Overview}. Figure~\ref{fig1-a} illustrates the threefold considerations, including (i) the inclusion of spatial and logical relations, (ii) support for multiple relations between layouts pairs, (iii) and the integration of non-textual elements. Figure~\ref{fig1-b} demonstrates the state-of-the-art performance of our proposed method, showcasing $\text{mAP}$ results for the Document Layout Analysis (DLA) task, as well as $mR_g$ and $mAP_g$ results for the graph-based Document Structure Analysis (gDSA) task on the GraphDoc dataset. 
    }
    \label{fig1:banner}
\end{center}%
% }]