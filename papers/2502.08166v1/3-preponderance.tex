\section{Identifying Discrimination by Modeling Preponderance}
\label{sec:interpretation} 
A major challenge of assessing potentially-differential rates of harm across subgroups using only reporting data is to relate the event that someone submits a report to the event that they experienced harm. That is, if someone did experience a negative outcome, how likely is it for them to have reported it, and conversely, if someone submitted a report, how likely is it to reflect ``true'' harm? Moreover, as is known from prior work, reporting rates themselves can vary across subgroups. 

Our central proposal is to conduct a hypothesis test for each group to determine whether it is overrepresented by a factor of $\beta$ among reports. That is, for each $G \in \Groups$, we test the following hypotheses:
\begin{align}
\label{eq:htest}
    \NullH: \mu_G < \beta\Basegroup && \mathcal{H}_1^G: \mu_G > \beta\Basegroup.
\end{align}
In Section \ref{sec:algs}, we will discuss concrete algorithms for conducting this test sequentially and their corresponding theoretical guarantees. 
Before doing so, we first argue that testing for preponderance among reports, i.e., tracking $\mu_G$ in this way, can be a meaningful way to identify discrimination, even when exact reporting behavior is unknown. 
In Sections \ref{subsec:rr} and \ref{subsec:ir}, we describe two distinct ways that this particular test can be interpreted; in Appendix \ref{subsec:practical}, we discuss some practical considerations for the modeling task.

\subsection{Preponderance as relative risk} 
\label{subsec:rr}
The first interpretation of our test allows us to make inferences about relative risk, the ratio between the rate of harm experienced by group $G$ and on average over the population. 
In this interpretation, the key quantity is the \textit{report-to-incidence ratio}.
\begin{definition}[Report-to-incidence ratio]
\label{def:rir}
We define the \emph{report-to-incidence-ratio (RIR)} as $\rho := \tfrac{\Pr[R = 1]}{\Pr[Y = 1]}$, and the group-conditional analogue as $\rho_G := \tfrac{\Pr[R = 1 \mid G ]}{\Pr[Y = 1 \mid G]}.$
\end{definition}

In Proposition \ref{prop:relativerisk-conversion}, we show that if the group-conditional RIR of some group $G$ is at most some constant multiple of the population-wide RIR, then we can convert a lower bound on report preponderance into a lower bound on true relative risk\iftoggle{icml}{ (see Appendix \ref{app:seq-proofs} for proof)}{}.
\begin{restatable}{proposition}{propRR}
\label{prop:relativerisk-conversion}
Define the relative risk of group $G$ to be $\RR_G := \frac{\Pr[Y = 1 \mid G]}{\Pr[Y = 1]}$. 
Suppose that for some group $G$ we have $\rho_G \leq b \cdot \rho$. Suppose that we determine that $\mu_G \geq \beta\Basegroup$ for some $\beta > 1$. Then, the true relative risk experienced by $G$ is at least $\RR_G \geq \nicefrac{\beta}{b}$.
\end{restatable} 
\iftoggle{icml}{}{
\begin{proof}
First, note that by definition of $\rho$, $\rho_G$, and $\RR_G$, we have 
\[
\rho_G \leq b \cdot \rho \iff \frac{\Pr[R = 1 \mid G ]}{\Pr[Y = 1 \mid G]} \leq b \cdot \frac{\Pr[R = 1]}{\Pr[Y = 1]} \iff \RR_G \geq \frac{\Pr[R = 1 \mid G]}{\Pr[R = 1]} \cdot \frac{1}{b}. 
\]
By Bayes' rule, $\frac{\Pr[R = 1 \mid G]}{\Pr[R = 1]} = \frac{\Pr[ G \mid R = 1]}{\Pr[G]} = \frac{\mu_G}{\Basegroup}$; furthermore, by assumption, we have $\frac{\mu_G}{\Basegroup} \geq \beta$. 
The result follows from combining with the previous display. 
\end{proof}}
Suppose we take $\max_G \nicefrac{\rho_G}{\rho} \leq b = 1.25$, i.e., no group over-reports 25\% more often than the population average. Then, if a test identifies a group $G$ for which $\mu_G \geq 1.75 \cdot \Basegroup$, this implies that the true relative risk for group $G$ is at least $\RR_G \geq 1.4$---that is, $G$ experiences harm 40\% more frequently relative to the population average. 

\subsection{Preponderance as true incidence rate}
\label{subsec:ir}
We now discuss an alternate way to convert a lower bound on preponderance into a guarantee on real-world harm. In this case, we can infer the true incidence rate of harm (that is, no longer relative to the average) if we are able to estimate---or willing to make assumptions on---true and false reporting behavior in groups.
Moreover, assumptions (or estimations) of these reporting rates need only be made in relation to the population average reporting rate $\Pr[R]$. 

\begin{definition}[Reporting rates]
Let $r := \Pr[R]$ be the average reporting rate over the full population.
    Let $\Truereportrate := \frac
    1r\Pr[R_i = 1 \mid \Badevent_i =1, X_i \in G]$, 
    $\Falsereportrate := \frac1r\Pr[R_i = 1 \mid \Badevent_i = 0, X_i \in G]$.
    Finally, let
    $\Actualgroup := \Pr[\Badevent \mid G]$ represent the true incidence rate, i.e. the likelihood that an individual in $G$ experiences $Y$.
\end{definition}
Note that $r \cdot \Truereportrate$ represents the (possibly group-conditional) rate at which individuals $X_i \in G$ who experience $\Badevent$ actually report, while $r \cdot \Falsereportrate$ represents the rate that individuals $X_i \in G$ who do not experience $\Badevent$ report.
Thus, $\Truereportrate$ and $\Falsereportrate$ represent how much more (or less) a particular group $G$ makes true or false reports relative to how much the whole population reports on average (which includes both true and false reports). 
The following proposition makes the relationships between $\Truereportrate$, $\Falsereportrate$, and our quantity of interest $\Actualgroup$, more precise.
\begin{restatable}{proposition}{propIR}\label{prop:reporting-conversion} 
Suppose that, for some $G$, it is determined that $\mu_G \geq \beta\Basegroup$ for some $\beta > 1$. As long as $\Truereportrate > \Falsereportrate$ for every $G \in \Groups$, 
% \[
$\Actualgroup \geq \frac{\beta - \Falsereportrate}{\Truereportrate - \Falsereportrate}.$
% \]
\end{restatable}
\iftoggle{icml}{See Appendix \ref{app:seq-proofs} for the (short) proof.}{
\begin{proof}[Proof of Proposition \ref{prop:reporting-conversion}]
Recall that we have defined $\mu_G = \Pr[G \mid R]$, and $\Basegroup = \Pr[G]$ is known by Assumption \ref{assn:ref}.
By Bayes' rule, we have
$    \mu_G = \Pr[G \mid R] = \frac{\Pr[G]\Pr[R \mid G]}{\Pr[R]} =  \Basegroup\frac{\Pr[R \mid G]}{r},$
Now, let us decompose $\Pr[R \mid G]$ by ``true'' reports ($\Badevent = 1$) and ``false'' reports ($\Badevent = 0$). 
By the law of total probability,
$
    \Pr[R \mid G] 
    = r \cdot \left(\Truereportrate \Actualgroup + \Falsereportrate(1-\Actualgroup )\right)
$; more precisely, 
\begin{align*}
    \frac1r\Pr[R \mid G] &= \Pr[R \mid G, \Badevent = 1]\Pr[\Badevent \mid G]  + \Pr[R \mid G,  \Badevent =0](1-\Pr[\Badevent \mid G] )
    \\&= \Truereportrate \Actualgroup + \Falsereportrate(1-\Actualgroup )
    \\&= \Falsereportrate + \Actualgroup (\Truereportrate-\Falsereportrate);
\end{align*} 
combining this with the Bayes' rule computation, cancelling the $\frac1r$ factor, gives us $
    \Actualgroup  = \frac{\frac{\mu_G}{\Basegroup} - \Falsereportrate}{\Truereportrate-\Falsereportrate}.
$
The result follows from the assumption that $\nicefrac{\mu_G}{\Basegroup} \geq \beta.$
\end{proof}
}
Proposition \ref{prop:reporting-conversion} shows that the exact computation of $\Actualgroup$ depends on reporting rates $\Truereportrate$ and $\Falsereportrate$. While these quantities are not directly estimable from reporting data---in fact, estimating reporting rates is itself a distinct research challenge (see, e.g., \citet{liu2024quantifying})---these results can nevertheless guide qualitative interpretation of how severe $\Actualgroup$ is. 

For example, suppose a test is run for $\beta = 1.5$. 
Suppose $G$ overreports relative to the population average, with $\Falsereportrate = 1$, and $\Truereportrate = 2$. 
\iftoggle{icml}{}{That is, $G$ \textit{falsely reports} at the same rate as the population reports on average (which includes both true and false reports), and submits true reports at twice the population average rate.} Under these (generous) assumptions, we will have $\Actualgroup = 0.5$, an extremely high incidence rate for any application---regardless of incidence rates for other groups. 

Alternatively, suppose reporting rates did not vary by group (i.e., $\Truereportrate = \gamma^{\text{TR}}$ and $\Falsereportrate = \gamma^{\text{FR}}$ for all $G$). Then, we can lower bound the disparities between true incidence rates across groups: 
if $G$ is flagged at $\beta > 1$, there must be some other group $G'$ with $\Actualgroup - \mathrm{IR}_{G'} \geq \frac{\beta - 1}{\gamma^{\text{TR}}-\gamma^{\text{FR}}}$. If it is further assumed that $\gamma^{\text{FR}}=0$, then $\Actualgroup - \mathrm{IR}_{G'} \geq \beta - 1$.
