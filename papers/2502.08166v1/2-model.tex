\section{Model, Notation, and Preliminaries}
\label{sec:model}
The goal of constructing an incident database is to determine whether some system that individuals interact with---for example, an (algorithmic) loan decision system, or a medical treatment---results in disproportionate harm to some meaningful subgroups. 
For the incident database associated with a particular system, we will use $\Badevent \in \{0,1\}$ as an indicator variable that denotes the undesirable event corresponding to that system. For example, in loan decisions, this could correspond to the event that a highly-qualified individual was denied a loan; in the medical setting, this may be an adverse physical side effect due to the treatment. 

\paragraph{Subgroup definitions.}
Individuals are characterized with feature vectors $X \in \X$, and we index individuals as 
$X_i$ (``features of individual $i$'') or 
$X_t$ (``features of the individual who reports at time $t$'').
Every individual $X_i$ ``belongs to'' at least one group $G$, and we will denote the event that $X_i$ belongs to $G$ as $\{X_i \in G\}$; we will use $\Groups$ to denote the set of all possible groups. 
This set of possible groups $\Groups$ can be defined arbitrarily as long as all groups can be determined as a function of covariates $\X$. We allow for groups to be overlapping---that is, we allow each individual $X_i$ to be in multiple groups so that $|\{G' \in \Groups: X_i \in G'\}| \geq 1$. 
\iftoggle{icml}{}{For example, it is possible to set $\Groups := 2^{\X}$ as in \citet{hebert2018multicalibration}. }

\paragraph{Reference population.}
The system for which the database is constructed naturally has a corresponding reference population of eligible individuals. For example, this could be everyone who has applied for a loan, or everyone who has been prescribed a certain medication. Thus, given a set of groups $\mathcal G$, we assume that it is possible to compute the composition of the reference population. 
\begin{assumption}[Reference population]
\label{assn:ref}
   For every $G \in \Groups$, the quantity $\Basegroup := \Pr[X \in G]$ is known. Throughout this work, we refer to the set $\{\Basegroup\}_{G \in \Groups}$ as \emph{base preponderances}.
\end{assumption}

\paragraph{Probabilistic model of reporting.}
As the database administrator, the high-level goal is to determine whether there exists some subgroup $G \in \Groups$ where $\Pr[\Badevent\mid X \in G]$ is abnormally high. 
Crucially, the database does not have access to information about every individual who interacts with the system; instead, individuals \textit{may} report to the database if they believe that they experienced bad event $\Badevent$. 
We thus let $R_i$ be a random variable representing whether individual $i$ decides to report (with $R_i = 0$ indicating no report). 

Each report $X_t$ is received sequentially, and assumed to be sampled i.i.d. from some underlying reporting distribution.\iftoggle{icml}{}{\footnote{Note that the events $\{X_t \in G\}$ and $\{X_t \in G'\}$ are correlated for any $G, G' \in \Groups$, i.e. the independence does not hold across groups. The key point in our case is independence across time.}}
Given a group $G$, we denote its corresponding mean among reports $\Pr[X_t \in G \mid R_t = 1]$ as  $\mu_G$.
We will sometimes refer to $\{\mu_G\}_{G \in \Groups}$ as (reporting) preponderances, as they represent the proportion of \textit{reports} that each $G$ comprises. A central claim of this paper is that comparing $\mu_G$ to $\Basegroup$---i.e., the extent to which group $G$ is (over)represented within the reporting database---can be a useful signal for $\Pr[Y \mid G]$ in a wide class of applications.\footnote
{Because we allow groups to overlap, we cannot enforce $\sum_G \Basegroup = 1$ or $\sum_G\mu_G = 1$.} \iftoggle{icml}{}{The i.i.d. model of course simplifies the analysis and exposition, but itself is not intrinsic to modeling the incident reporting problem as a sequential hypothesis test. As we discuss in Appendix \ref{subsec:practical}, the explicit i.i.d. assumption can be relaxed; more generally, any probabilistic model for sequential testing can be adapted to incident reporting. 
}