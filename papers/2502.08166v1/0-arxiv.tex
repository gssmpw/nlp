\documentclass[twoside]{article}

\usepackage{amsmath, amsfonts, amsthm}
\usepackage{nicefrac}
\usepackage{graphicx}
\usepackage{hyperref}
\usepackage{booktabs}
\usepackage{multirow}
\usepackage{siunitx}
\usepackage{natbib}
\usepackage[margin=1in]{geometry}

\usepackage{etoolbox}
\newtoggle{icml}
\togglefalse{icml}

\usepackage{comment}

\usepackage{wrapfig}
\usepackage{tabularray}
\usepackage[shortlabels]{enumitem}

\usepackage{thmtools}
\usepackage{thm-restate}
\declaretheorem[name=Theorem,numberwithin=section]{thm}
\declaretheorem[name=Proposition,numberwithin=section]{prop}
\newtheorem{theorem}{Theorem}
\numberwithin{theorem}{section}
\newtheorem{corollary}[theorem]{Corollary}
\newtheorem{lemma}[theorem]{Lemma}
\newtheorem{claim}[theorem]{Claim}
\newtheorem{proposition}[theorem]{Proposition}
\newtheorem{definition}[theorem]{Definition}
\newtheorem{assumption}[theorem]{Assumption}
\newtheorem{remark}[theorem]{Remark}
\newtheorem{example}[theorem]{Example}
\newtheorem{conjecture}[theorem]{Conjecture}
\usepackage{mathtools}

% macros for this paper
\newcommand{\E}{\mathbb{E}}
\newcommand{\D}{D}
\newcommand{\X}{\mathcal{X}}
\newcommand{\1}{\mathbf{1}}
\renewcommand{\O}{\mathcal{O}}

% notation
\newcommand{\Badevent}{Y}
\newcommand{\Logwealth}{\omega}
\newcommand{\Truereportrate}{{\gamma_G^{\text{TR}}}}
\newcommand{\Falsereportrate}{{\gamma_G^{\text{FR}}}}
\newcommand{\Select}{\text{Select}}
\newcommand{\Test}{\text{Test}}
\newcommand{\Basegroup}{{\mu_G^0}}
\newcommand{\Actualgroup}{{\mathrm{IR}_G}}
\newcommand{\GSelected}{{\widehat G^\star}}
\newcommand{\CandG}{{\mathcal{G}_E^{\text{candidate}}}}
\newcommand{\Groups}{{\mathcal{G}}}
\newcommand{\FlagG}{{\mathcal {G}^{\text{Flag}}}}
\newcommand{\NflagG}{{\mathcal {G}^{\text{Null}}}}
\newcommand{\atilde}{\widetilde{\alpha}}
\newcommand{\bigmid}{\, \bigg| \,}
\newcommand{\NullH}{{\mathcal{H}_0^G}}
\newcommand{\AltH}{{\mathcal{H}_1}}
\newcommand{\nicebonf}{{\nicefrac{\alpha}{|\Groups|}}}
\newcommand{\nicebonfinv}{{\nicefrac{|\Groups|}{\alpha}}}

% baseline algorithms
\newcommand{\SeqNoOns}{{\texttt{Seq-No-ONS}} }
\newcommand{\SeqNoDP}{{\texttt{Seq-No-DP}} }
\newcommand{\SeqBonfer}{{\texttt{Seq-Bonf}} }
\newcommand{\BatchPropz}{{\texttt{Batch-Prop-Ztest}} }

\newcommand{\Esel}{{E^t_{\text{sel}}}}
\newcommand{\Epass}{{E^t_{\text{pass}}}}
\newcommand{\Dcopy}{{\widetilde{\D}}}
\newcommand{\Range}{{\mathrm{Range}}}

\newcommand{\Gstar}{{G^\star}}
\newcommand{\Optwealth}{{\Logwealth_\star^{G^\star}}}
\newcommand{\Basestar}{{\mu_{\Gstar}^0}}

\newcommand{\Gmax}{{G^{\text{max}}}}
\newcommand{\Gmaxt}{{G^{\text{max}}_t}}

\newcommand{\RR}{\mathrm{RR}}

\usepackage[dvipsnames]{xcolor}   

\usepackage{hyperref}
\hypersetup{
colorlinks=true,
linkcolor=DarkOrchid,
filecolor=magenta,
urlcolor=Orchid,
citecolor=CadetBlue
}

\usepackage[algo2e,ruled,noend]{algorithm2e}
\usepackage[font=small,tableposition=top]{caption}


\title{From Individual Experience to Collective Evidence: \\
A Reporting-Based Framework for Identifying Systemic Harms 
}
\author{Jessica Dai, Paula Gradu, Inioluwa Deborah Raji, Benjamin Recht\\{\textit{University of California, Berkeley}}
}
\date{February 2025}

\begin{document}

\maketitle

\thispagestyle{empty}
\begin{abstract}
When an individual reports  a negative interaction with some system, how can their personal experience be contextualized within broader patterns of system behavior?
We study the \textit{incident database} problem, where individual reports of adverse events arrive sequentially, and are aggregated over time.
In this work, our goal is to identify whether there are subgroups---defined by any combination of relevant features---that are disproportionately likely to experience harmful interactions with the system.
We formalize this problem as a sequential hypothesis test, and identify conditions on reporting behavior that are sufficient for making inferences about disparities in true rates of harm across subgroups.
We show that algorithms for sequential hypothesis tests can be applied to this problem with a standard multiple testing correction.
We then demonstrate our method on real-world datasets, including mortgage decisions and vaccine side effects; on each, our method (re-)identifies subgroups known to experience disproportionate harm using only a fraction of the data that was initially used to discover them.
\end{abstract}

\section{Introduction}
\label{sec:intro}

Foundational models (FMs)~\cite{zhang2024data, zhou2023comprehensive} have shown remarkable progress in the healthcare domain, enabling professional-like assessment of disease diagnosis, treatment decision-making, and monitoring~\cite{zhang2023text, wang2022medclip, lu2023mi-zero}. 
Examples include LLaVA-Med~\cite{li2023llava}, Med-PaLM Multimodal~\cite{tu2024towards}, and Med-Flamingo~\cite{moor2023med}, have demonstrated their capacity on question answering, medical image analysis, and report generation.
These studies follow a predominant top-down model development strategy that requires upstream developers to collect data and train models for downstream tasks. 
Consequently, the developed model capabilities are heavily dependent on the training data, limiting their generalization performance in diverse clinical scenarios. 
For instance, Med-Gemini~\cite{yang2024advancing} reveals promising general capabilities in report generation while it lags behind state-of-the-art (SoTA) models on classification tasks, especially for out-of-domain applications. 
This indicates that while the generalizability of the foundation model is promising, more solutions are expected to meet the various specialized clinical needs.

To address these challenges, multi-center data centralization becomes essential to enhance model capacity and robustness across varied clinical scenarios~\cite{rajpurkar2022ai}. 
Centralizing distributed data can significantly improve model training and inference performance.
However, the process of medical data storage, transfer, and aggregation among centers requires extra efforts to ensure data security and system interoperability~\cite{bradford2020international}.
Moreover, a growing concern for patient privacy makes large-scale multi-center data sharing particularly challenging. 
While efforts like federated learning~\cite{wen2023survey, li2020review} can achieve good model performance on local data, the need for synchronized system coordination presents significant challenges, as clients are unable to update asynchronously. This limitation greatly restricts the practical capability of such approaches.
As a result, without a flexible collaboration, medical community still struggles to fully utilize the isolated data and local computation resources for comprehensive medical AI model development. 
To address this dilemma, open-source platforms encourage public data sharing and knowledge integration~\cite{markiewicz2021openneuro, zenodo}.
However, these platforms focus solely on raw data sharing while seldom providing collaborative model training or cooperation between different institutions.
Recently, collaborative learning has emerged as a viable approach for enhancing multi-model robustness~\cite{boulemtafes2020review}. 
For instance, software-like model development~\cite{raffel2023building} mimics software engineering practices by introducing structured workflows, enabling merging, version control, and continuous model integration.
Under this design, model ability can be strengthened with incremental knowledge updates similar to the version updating in software development. 

Although collaborative learning provides a multi-model collaboration, two key challenges remain in the leakage of raw data during collaboration~\cite{huang2023lorahub} and the synchronization of multiple collaborators~\cite{mcmahan2017communication} in the medical AI community. It is still challenging to integrate decentralized, privacy-sensitive data across institutions, leading to under-utilized insights and fragmented knowledge sharing~\cite{kaissis2020secure, rajpurkar2022ai, abdullah2021ethics}.
 To address these challenges, inspired by the collaborative software development, we propose \textbf{Med}ical \textbf{Fo}undation Models Me\textbf{rg}ing (\textbf{MedForge}), a cooperative workflow enabling continuously community-driven foundation model (FM) development.
MedForge enables a lightweight manner for individual centers to share their knowledge among multiple centers, minimizing the burden of data transmission and integration while enhancing model robustness.
Meanwhile, MedForge facilitates asynchronous and flexible collaboration, allowing individual centers to continuously update and improve medical FMs without the need for real-time synchronization.
Similar to open-source software development, MedForge incrementally updates medical knowledge and follows a sustainable model development scheme. 
This key design emphasizes a bottom-up construction of a multi-task medical FM, allowing downstream users to collaboratively build, refine, and update the upstream model according to their local resources. Our major contributions of MedForge are as below: 
\begin{enumerate}
    \item[$\bullet$] We introduce a collaborative workflow to promote the merging scheme of open-source software development. Our proposed MedForge allows distributed clinical centers to asynchronously contribute to comprehensive medical model construction while reducing transmitting costs among centers and avoiding the leakage of raw data, thus enhancing the utilization of private resources in the healthcare system. 
    \item[$\bullet$] We propose two effective knowledge-merging strategies for the asynchronous branch contribution. The MedForge-Fusion strategy updates the plugin module parameters of the main model during the merging phase, whereas the MedForge-Mixture strategy integrates the output of the plugin module by memorizing each contributor's coefficient. These strategies make MedForge more flexible and versatile. MedForge-Fusion is friendly to implement, while the MedForge-Mixture offers better performance and robustness.
    \item[$\bullet$]  We comprehensively evaluate model merging strategies to accumulate medical knowledge among multiple branch plugin modules. MedForge yields superior performance on medical classification tasks compared to other collaborative baselines across multiple datasets. We demonstrate the robustness of MedForge by shuffling the task order and evaluating various configurations of plugin modules and dataset distillation methods.
\end{enumerate}



\section{The Model}\label{sec:model}

A seller has one indivisible good to sell, and there are $n$ potential buyers. Each buyer $i$ has a private value $v_i$, which is drawn from a cumulative distribution $F_i$. The distributions $F_i$ are assumed to satisfy all of the assumptions as in \citet{myerson1981optimal}: they admit densities $f_i$, which are strictly positive everywhere within a support $[a_i,b_i]$. We will also assume the value distributions are regular.  

\begin{assumption}[Regularity]\label{ass:regular} For every $i$, the virtual value function $v-(1-F_i(v))/f_i(v)$ is assumed to be non-decreasing over the support of buyer $i$'s valuation. 
\end{assumption}

The seller has access to a value prediction technology, which generates a signal $s_i$ for each $i$. The signal is a hallucination with probability $\gamma_i \in (0,1)$. If the signal is a hallucination, then $s_i = w_i$, where $w_i$ is a random variable also drawn from distribution $F_i$ that is independent of buyer $i$'s value $v_i$ (we discuss the case where $w_i$ drawn from a different distribution than $v_i$ in  \Cref{sec:failure_myerson}). If the signal is not a hallucination, then the signal is assumed to be accurate: $s_i = v_i$. The seller is assumed to know the values $\gamma_i$, but not whether a given realization is a hallucination or not. We assume that the realizations of hallucinations, $v_i$ and $w_i$ are independent across buyers. 

We will use $\bm{\gamma}$ and $\bm{s}$ to represent the vectors of hallucination probabilities and signals, respectively. Given a signal, the seller can perform a Bayesian update to obtain what we call the posterior distribution of a buyer's value. We will denote by $\bm{F}_{\bm{\gamma},\bm{s}}$ the posterior distribution of the buyers' values and by $\bm{F}_{\bm{\gamma},\bm{s},-i}$ the posterior distribution of the buyers' values excluding the $i^{th}$ buyer.
 
The question we aim to address in this paper is what is the seller's revenue-maximizing mechanism in the presence of this value prediction technology. A mechanism is defined by a pair $(\bm{x},\bm{p})$, where $\bm{x}$ (resp. $\bm{p}$) is an allocation (resp. payment) function which takes as input the vector of reported types $\bm{\theta}$ and the vector of observed signals $\bm{s}$ and outputs the vector of probability of allocation (resp. of payment) for each buyer. We assume that all agents have quasi-linear utilities. For a given vector of signals $\bm{s}$, we will explore the following problem:
\begin{subequations}
\label{eq:optimal_mechanism}
\begin{alignat}{2}
&\!\sup_{(\bm{x},\bm{p})} &\;& \mathbb{E}_{\bm{\theta} \sim \bm{F_{\gamma,s}}} \left[  \sum_{i=1}^n  p_i(\bm{\theta},\bm{s}) \right]    \\
&\text{s.t.} &      &  \mathbb{E}_{\bm{\theta_{-i}} \sim \bm{F}_{\bm{\gamma},\bm{s},-i}} \left[ \theta_i \cdot x_i(\theta_i,\bm{\theta}_{-i},\bm{s}) - p_i(\theta_i,\bm{\theta}_{-i},\bm{s}) \right] \nonumber \\ 
 &  &  & \qquad \geq \mathbb{E}_{\bm{\theta_{-i}} \sim \bm{F}_{\bm{\gamma},\bm{s},-i}} \left[ \theta_i \cdot x_i(\theta'_i,\bm{\theta}_{-i},\bm{s}) - p_i(\theta'_i,\bm{\theta}_{-i},\bm{s}) \right] \quad \text{for every $i, \theta_i, \theta'_i$,} \label{eq:IC} \\
 &  &  &\mathbb{E}_{\bm{\theta_{-i}} \sim \bm{F}_{\bm{\gamma},\bm{s},-i}} \left[ \theta_i \cdot x_i(\theta_i,\bm{\theta}_{-i},\bm{s}) - p_i(\theta_i,\bm{\theta}_{-i},\bm{s}) \right] \geq 0 \quad \text{for every $i, \theta_i$,} \label{eq:IR}\\
& & & \sum_{i=1}^n x_i(\bm{\theta}) \leq 1 \quad \text{for every $\bm{\theta}$.}
\end{alignat}
\end{subequations}

 \noindent \textbf{Signal-revealing direct mechanisms.} Problem \eqref{eq:optimal_mechanism} specifies the problem of finding the optimal signal-revealing direct mechanism. A direct mechanism is one where the seller chooses an incentive-compatible allocation and payment scheme, and asks the buyers to reveal their types. In standard mechanism design, restricting to direct mechanisms is without loss of optimality \citep{myerson1981optimal}. We define a signal-revealing mechanism to be one where the seller shares the signals alongside the allocation and payment rules. Exploring non-signal-revealing mechanisms is a potentially difficult problem, as the choice of allocation and payment rule will reveal the signals unless the seller explicitly pools signals (i.e., chooses a mechanism that is at least partially non-responsive to signals). We leave the question of whether restricting attention to signal-revealing mechanisms is without loss of optimality open for future work. Note that by assuming the mechanism is signal-revealing we made the formulation relatively straightforward: both the objective and the IC and IR constraints use the posterior distributions given signals rather than the priors.\\

\noindent \textbf{On the correctness of non-hallucinatory signals.}
A natural question regarding this model is why we assume that, when a signal is not a hallucination, it equals the buyer's private value. In reality, errors from deep neural network models are likely a combination of hallucinations and classical Gaussian noise. We analyze pure hallucination in this paper in order to achieve a clean characterization. If we added a Gaussian noise on top of the hallucination, the answer would not be as crisp as the near-decomposition obtained in \Cref{thm:main}. This strict separation between hallucinations and Gaussian noise also allows for a sharp comparison of their respective implications (\Cref{fig:single_buyer}).
 
 

\section{Identifying Discrimination by Modeling Preponderance}
\label{sec:interpretation} 
A major challenge of assessing potentially-differential rates of harm across subgroups using only reporting data is to relate the event that someone submits a report to the event that they experienced harm. That is, if someone did experience a negative outcome, how likely is it for them to have reported it, and conversely, if someone submitted a report, how likely is it to reflect ``true'' harm? Moreover, as is known from prior work, reporting rates themselves can vary across subgroups. 

Our central proposal is to conduct a hypothesis test for each group to determine whether it is overrepresented by a factor of $\beta$ among reports. That is, for each $G \in \Groups$, we test the following hypotheses:
\begin{align}
\label{eq:htest}
    \NullH: \mu_G < \beta\Basegroup && \mathcal{H}_1^G: \mu_G > \beta\Basegroup.
\end{align}
In Section \ref{sec:algs}, we will discuss concrete algorithms for conducting this test sequentially and their corresponding theoretical guarantees. 
Before doing so, we first argue that testing for preponderance among reports, i.e., tracking $\mu_G$ in this way, can be a meaningful way to identify discrimination, even when exact reporting behavior is unknown. 
In Sections \ref{subsec:rr} and \ref{subsec:ir}, we describe two distinct ways that this particular test can be interpreted; in Appendix \ref{subsec:practical}, we discuss some practical considerations for the modeling task.

\subsection{Preponderance as relative risk} 
\label{subsec:rr}
The first interpretation of our test allows us to make inferences about relative risk, the ratio between the rate of harm experienced by group $G$ and on average over the population. 
In this interpretation, the key quantity is the \textit{report-to-incidence ratio}.
\begin{definition}[Report-to-incidence ratio]
\label{def:rir}
We define the \emph{report-to-incidence-ratio (RIR)} as $\rho := \tfrac{\Pr[R = 1]}{\Pr[Y = 1]}$, and the group-conditional analogue as $\rho_G := \tfrac{\Pr[R = 1 \mid G ]}{\Pr[Y = 1 \mid G]}.$
\end{definition}

In Proposition \ref{prop:relativerisk-conversion}, we show that if the group-conditional RIR of some group $G$ is at most some constant multiple of the population-wide RIR, then we can convert a lower bound on report preponderance into a lower bound on true relative risk\iftoggle{icml}{ (see Appendix \ref{app:seq-proofs} for proof)}{}.
\begin{restatable}{proposition}{propRR}
\label{prop:relativerisk-conversion}
Define the relative risk of group $G$ to be $\RR_G := \frac{\Pr[Y = 1 \mid G]}{\Pr[Y = 1]}$. 
Suppose that for some group $G$ we have $\rho_G \leq b \cdot \rho$. Suppose that we determine that $\mu_G \geq \beta\Basegroup$ for some $\beta > 1$. Then, the true relative risk experienced by $G$ is at least $\RR_G \geq \nicefrac{\beta}{b}$.
\end{restatable} 
\iftoggle{icml}{}{
\begin{proof}
First, note that by definition of $\rho$, $\rho_G$, and $\RR_G$, we have 
\[
\rho_G \leq b \cdot \rho \iff \frac{\Pr[R = 1 \mid G ]}{\Pr[Y = 1 \mid G]} \leq b \cdot \frac{\Pr[R = 1]}{\Pr[Y = 1]} \iff \RR_G \geq \frac{\Pr[R = 1 \mid G]}{\Pr[R = 1]} \cdot \frac{1}{b}. 
\]
By Bayes' rule, $\frac{\Pr[R = 1 \mid G]}{\Pr[R = 1]} = \frac{\Pr[ G \mid R = 1]}{\Pr[G]} = \frac{\mu_G}{\Basegroup}$; furthermore, by assumption, we have $\frac{\mu_G}{\Basegroup} \geq \beta$. 
The result follows from combining with the previous display. 
\end{proof}}
Suppose we take $\max_G \nicefrac{\rho_G}{\rho} \leq b = 1.25$, i.e., no group over-reports 25\% more often than the population average. Then, if a test identifies a group $G$ for which $\mu_G \geq 1.75 \cdot \Basegroup$, this implies that the true relative risk for group $G$ is at least $\RR_G \geq 1.4$---that is, $G$ experiences harm 40\% more frequently relative to the population average. 

\subsection{Preponderance as true incidence rate}
\label{subsec:ir}
We now discuss an alternate way to convert a lower bound on preponderance into a guarantee on real-world harm. In this case, we can infer the true incidence rate of harm (that is, no longer relative to the average) if we are able to estimate---or willing to make assumptions on---true and false reporting behavior in groups.
Moreover, assumptions (or estimations) of these reporting rates need only be made in relation to the population average reporting rate $\Pr[R]$. 

\begin{definition}[Reporting rates]
Let $r := \Pr[R]$ be the average reporting rate over the full population.
    Let $\Truereportrate := \frac
    1r\Pr[R_i = 1 \mid \Badevent_i =1, X_i \in G]$, 
    $\Falsereportrate := \frac1r\Pr[R_i = 1 \mid \Badevent_i = 0, X_i \in G]$.
    Finally, let
    $\Actualgroup := \Pr[\Badevent \mid G]$ represent the true incidence rate, i.e. the likelihood that an individual in $G$ experiences $Y$.
\end{definition}
Note that $r \cdot \Truereportrate$ represents the (possibly group-conditional) rate at which individuals $X_i \in G$ who experience $\Badevent$ actually report, while $r \cdot \Falsereportrate$ represents the rate that individuals $X_i \in G$ who do not experience $\Badevent$ report.
Thus, $\Truereportrate$ and $\Falsereportrate$ represent how much more (or less) a particular group $G$ makes true or false reports relative to how much the whole population reports on average (which includes both true and false reports). 
The following proposition makes the relationships between $\Truereportrate$, $\Falsereportrate$, and our quantity of interest $\Actualgroup$, more precise.
\begin{restatable}{proposition}{propIR}\label{prop:reporting-conversion} 
Suppose that, for some $G$, it is determined that $\mu_G \geq \beta\Basegroup$ for some $\beta > 1$. As long as $\Truereportrate > \Falsereportrate$ for every $G \in \Groups$, 
% \[
$\Actualgroup \geq \frac{\beta - \Falsereportrate}{\Truereportrate - \Falsereportrate}.$
% \]
\end{restatable}
\iftoggle{icml}{See Appendix \ref{app:seq-proofs} for the (short) proof.}{
\begin{proof}[Proof of Proposition \ref{prop:reporting-conversion}]
Recall that we have defined $\mu_G = \Pr[G \mid R]$, and $\Basegroup = \Pr[G]$ is known by Assumption \ref{assn:ref}.
By Bayes' rule, we have
$    \mu_G = \Pr[G \mid R] = \frac{\Pr[G]\Pr[R \mid G]}{\Pr[R]} =  \Basegroup\frac{\Pr[R \mid G]}{r},$
Now, let us decompose $\Pr[R \mid G]$ by ``true'' reports ($\Badevent = 1$) and ``false'' reports ($\Badevent = 0$). 
By the law of total probability,
$
    \Pr[R \mid G] 
    = r \cdot \left(\Truereportrate \Actualgroup + \Falsereportrate(1-\Actualgroup )\right)
$; more precisely, 
\begin{align*}
    \frac1r\Pr[R \mid G] &= \Pr[R \mid G, \Badevent = 1]\Pr[\Badevent \mid G]  + \Pr[R \mid G,  \Badevent =0](1-\Pr[\Badevent \mid G] )
    \\&= \Truereportrate \Actualgroup + \Falsereportrate(1-\Actualgroup )
    \\&= \Falsereportrate + \Actualgroup (\Truereportrate-\Falsereportrate);
\end{align*} 
combining this with the Bayes' rule computation, cancelling the $\frac1r$ factor, gives us $
    \Actualgroup  = \frac{\frac{\mu_G}{\Basegroup} - \Falsereportrate}{\Truereportrate-\Falsereportrate}.
$
The result follows from the assumption that $\nicefrac{\mu_G}{\Basegroup} \geq \beta.$
\end{proof}
}
Proposition \ref{prop:reporting-conversion} shows that the exact computation of $\Actualgroup$ depends on reporting rates $\Truereportrate$ and $\Falsereportrate$. While these quantities are not directly estimable from reporting data---in fact, estimating reporting rates is itself a distinct research challenge (see, e.g., \citet{liu2024quantifying})---these results can nevertheless guide qualitative interpretation of how severe $\Actualgroup$ is. 

For example, suppose a test is run for $\beta = 1.5$. 
Suppose $G$ overreports relative to the population average, with $\Falsereportrate = 1$, and $\Truereportrate = 2$. 
\iftoggle{icml}{}{That is, $G$ \textit{falsely reports} at the same rate as the population reports on average (which includes both true and false reports), and submits true reports at twice the population average rate.} Under these (generous) assumptions, we will have $\Actualgroup = 0.5$, an extremely high incidence rate for any application---regardless of incidence rates for other groups. 

Alternatively, suppose reporting rates did not vary by group (i.e., $\Truereportrate = \gamma^{\text{TR}}$ and $\Falsereportrate = \gamma^{\text{FR}}$ for all $G$). Then, we can lower bound the disparities between true incidence rates across groups: 
if $G$ is flagged at $\beta > 1$, there must be some other group $G'$ with $\Actualgroup - \mathrm{IR}_{G'} \geq \frac{\beta - 1}{\gamma^{\text{TR}}-\gamma^{\text{FR}}}$. If it is further assumed that $\gamma^{\text{FR}}=0$, then $\Actualgroup - \mathrm{IR}_{G'} \geq \beta - 1$.

\section{Identifying Subgroups with High Reporting Overrepresentation}
\label{sec:algs}

How might the test proposed in Equation \eqref{eq:htest} be carried out in practice, with reports arriving over time, and what properties might we want for such a test? In this section, we provide two ways to instantiate this sequential hypothesis test. For each, we provide two types of guarantees. The first is (sequential) $\alpha$-validity, which, roughly speaking, guarantees correctness of groups identified in $\FlagG$. More formally, we say that a sequential test is valid for a single group $G$ at level $\alpha$ if $\Pr[\exists t: \NullH \; \mathrm{ rejected}] \leq \alpha$ when $\NullH$ is true. Because we are testing for all groups in $\Groups$ simultaneously, we say that a sequential test is valid with respect to all groups $\Groups$ if $\Pr[\exists t, \exists G: \NullH \; \mathrm{erroneously} \; \mathrm{rejected}] \leq \alpha$.

The second type of guarantee is power, which guarantees that the test will identify a harmed group, if one exists. In particular, we are interested in the \textit{stopping time} $T$ of the test, which is the number of samples required for the test to reject the first null, i.e. to raise an alarm for any group. 


At a high level, our algorithms for 
conducting this test
follow the protocol outlined in Algorithm \ref{alg:abstract}. 
\iftoggle{icml}{}{Every report $X_t$ can be considered a binary vector indexed by the groups in $\Groups$; the $G$ component of this vector is equal to $\1[X_t \in G]$. If there was only one group, we could run a sequential hypothesis test to determine whether $\mu_G$ was unacceptably large. With multiple groups, we can run $|\Groups|$ separate sequential hypothesis tests in parallel, one for each group, and correct the confidence levels for multiple hypothesis testing.}
For each group $G$, we maintain a test statistic $\Logwealth_t^G$ that is updated as reports $X_t$ are received over time. At each time $t$, each of these test statistics are compared to a threshold $\theta_t(\alpha)$, which depends on the test level $\alpha$; the null hypothesis $\NullH$ for group $G$ is rejected if $\Logwealth_t^G > \theta_t(\alpha)$. 
For ease of exposition, Algorithm \ref{alg:abstract} is written so that groups corresponding to rejected nulls are collected in a set $\FlagG$; in practice, a database administrator may choose to stop the test entirely as soon as one harmed group has been found.  

Correcting for multiple hypothesis testing across groups is handled by a simple Bonferroni correction---that is, given a particular test level $\alpha$, we test each individual group $G$ at level $\nicebonf$ rather than level $\alpha$.
% \footnote{The Bonferroni correction is also convenient for dealing with the correlations across hypotheses.} 
Though Bonferroni corrections often seem onerous in non-sequential settings, we show that, for sequential problems, the Bonferroni correction incurs only a modest increase in stopping time. 

In Section \ref{subsec:ztest}, we give a simple sequential Z-test-inspired approach which leverages a finite-time Law of the Iterated Logarithm. Section \ref{subsec:e-val-alg} presents a more complicated algorithm that leverages recent developments in anytime-valid inference.
The main differences in each algorithm lie in how they implement Lines 1 and 6 of Algorithm \ref{alg:abstract}---that is, how test statistics and thresholds are computed. 
For each instantiation of Algorithm \ref{alg:abstract}, we show validity and power guarantees.
Omitted proofs are given in Appendix \ref{app:seq-proofs}.


\begin{algorithm2e}
\caption{General protocol for testing overrepresentation}\label{alg:abstract}
\LinesNumbered
\KwIn{Set of groups $\Groups$; base preponderances $\{\Basegroup\}_{G \in \Groups}$; test level $\alpha$; relative strength $\beta$}
Initialize test statistic $\Logwealth_0^G$ for every $G \in \Groups$ and set threshold $\theta_0(\alpha)$\;
Initialize set of rejected nulls (flagged groups) $\FlagG := \emptyset$\;
\For{$t = 1, 2, \dots$}
{
See report $X_t$\;
\For{$G \in \Groups$}
{   Update test statistic $\Logwealth_t^G$ and compute threshold $\theta_t(\alpha)$\;
    \If{$\Logwealth_t^{G} \geq \theta_t(\alpha)$}{
    Add $G$ to $\FlagG$ and take requisite action for $G$, if applicable.}
    }
}
\end{algorithm2e}

\subsection{Sequential Z-test}
\label{subsec:ztest}
One simple observation that arises from the model presented in Section \ref{sec:model} is that if each report $X_t$ is drawn i.i.d. from some underlying distribution, then one might expect to be able to use concentration as a tool to conduct this test, since as time passes, the fraction of reports within the database from group $G$ should converge to the true mean $\mu_G$. We refer to this style of approach as a sequential Z-test, as it relies on measuring deviation from the mean. 

\paragraph{Updating the test statistic $\Logwealth_t^G$.}
Given this intuition, the test statistic is a simple count of the number of times a report from each group has been seen, i.e. (with $\Logwealth_0^G = 0$),
\begin{equation}
    \label{eq:ztest_update}
    \Logwealth_t^G \leftarrow \Logwealth_{t-1}^G +\1[X_t\in G]. 
\end{equation}
\paragraph{Setting the threshold $\theta_t(\alpha)$.}
Given the way that $\Logwealth_t^G$ accumulates evidence, one natural way to construct the threshold at each $t$ is to use the mean under the alternative, plus a correction term for both sample complexity and repeated testing over time. With $C$ set to either $\sqrt{\beta\Basegroup(1 - \beta\Basegroup)}$ or $\nicefrac12$, the threshold (including a Bonferroni correction) is 
\begin{equation}
    \label{eq:ztest_thresh}
    \theta_t(\alpha) := t \cdot \beta\Basegroup + C \sqrt{
    2.07 t
    \ln\left(|\Groups| \frac{(2 + \log_2(t))^2}{\alpha}\right)
    }.
\end{equation}

\paragraph{Theoretical guarantees.}
Our first guarantee is a bound on the probability that any group is incorrectly flagged. 
\begin{restatable}[Validity]{theorem}{ztestvalidity}
\label{thm:validity_ztest}
Running Algorithm~\ref{alg:abstract} with $\theta_t(\alpha)$ as in Equation \eqref{eq:ztest_thresh}, setting $C = \nicefrac
12$, and $\Logwealth_t^G$ updated as in Equation~\eqref{eq:ztest_update}, guarantees that 
the probability that $\FlagG$ will ever contain a group $G$ where $\NullH$ is true is at most $\alpha$, i.e. 
\[
\Pr\left[\exists t: \exists G \in \FlagG \text{ s.t. } \NullH \text{ holds}\right] \leq \alpha.
\]
\end{restatable}

The choice of $C$ affects the nature of the guarantee: the true, finite-sample anytime-validity guarantee requires $C = \nicefrac12$. If instead $C = \sqrt{\beta\Basegroup(1 - \beta\Basegroup)}$, then, strictly speaking, the guarantee holds only asymptotically. However, a higher value of $C$ affects stopping time unfavorably, so the asymptotic approximation can be useful practically. In this case, care must be taken to ensure that the algorithm does not erroneously reject too early due to noise; one way to implement this is to mandate a minimum stopping time. 

Finally, we give a stopping time guarantee for this test.
\begin{restatable}[Power]{theorem}{ztestpower}
\label{thm:power_ztest}
Let $T$ be the stopping time of 
Algorithm~\ref{alg:abstract} with $\theta_t(\alpha)$ as in Equation \eqref{eq:ztest_thresh}, $C = \nicefrac
12$, and $\Logwealth_t^G$ as in Equation~\eqref{eq:ztest_update}.
Let $\Delta_{\max} = \max_{G \in \Groups} \mu_G - \beta\Basegroup.$
If $\Delta_{\max} > 0$, then $\Pr[T < \infty] = 1$. 
Furthermore, with probability $1 - \nicebonf$, we have $T  \leq \widetilde{\mathcal{O}} \left( \frac{\ln(|\Groups|) + \ln(1/\alpha)}{\Delta_{\max}^2}\right)$, and for any $\delta \in (0, \nicebonf)$, we have with probability at least $1 - \delta$ that
$T  \leq \widetilde{\mathcal{O}} \left( \frac{\ln(1/\delta)}{\Delta_{\max}^2}\right)$.
\end{restatable}
\subsection{Betting-style approach}
\label{subsec:e-val-alg}
We refer to our second algorithm as a \textit{betting-style} approach, due to the way we construct our test statistics \citep{shafer2021testing, waudby2024estimating, vovk2021values, chugg2024auditing}; one way to interpret this approach is that the test ``bets against'' the null hypothesis $\NullH$ being true. We direct the reader to these references for more detailed technical exposition\iftoggle{icml}{.}{ and connection with literature on martingales, gambling, and finance. For us, these methods provide an adaptive algorithm which find a middle ground between the two approaches in the previous section: the betting-style approach achieves finite-sample validity but empirically terminates quickly when the null is false.}

\paragraph{Updating the test statistic $\Logwealth_t^G$.}
As in the previous approach, we let $\Logwealth_t^G$ represent some accumulated amount of evidence against the null hypothesis $\NullH$ by time $t$, with a higher value of $\Logwealth_t^G$ corresponding to greater level of evidence.\footnote{The quantity $\exp(\Logwealth_t^G)$ can also be referred to as an \textit{e-value} \citep{vovk2021values}, a measure of evidence against a null hypothesis similar to a p-value.} We initialize $\Logwealth_0^G = 0$, and use the update rule 
\begin{equation}\label{eq:wealth_update}
\Logwealth_t^G \leftarrow \Logwealth_{t-1}^G + \ln\left(1 + \lambda_t^G (\mathbf{1}_{X_t\in G} - \beta \mu_G^0)\right),
\end{equation}
with $\lambda_1^G, \ldots, \lambda_t^G \in [0, 1]$. 
\iftoggle{icml}{}{Note that this expression is similar to the running sum used in Section~\ref{subsec:ztest}.}
Here, the algorithm accumulates a nonlinear function, with an adaptive parameter $\lambda_t^G$ that weights the influence of each new sample. 
Our setting of $\lambda_t$ is motivated by the goal of minimizing stopping time under the alternative, and thus to maximize $\Logwealth_t^G$. 
\iftoggle{icml}{}{Taking $\lambda_{t} = 0$ means $\Logwealth_t^G$ remains the same regardless of what new information is received at time $t$. On the other hand, $\lambda_{t} = \nicefrac{1}{\beta\mu_G^0}$ means that if we receive evidence in accordance with $\NullH$ then $\Logwealth_t^G$ will decrease substantially; but, if we instead receive evidence \textit{against} the null, i.e. $X_{t}\in G$, we maximally increase $\Logwealth_{t}^G$.}
Drawing from the well-studied problem of portfolio optimization in the online learning literature \citep{cover1991universal,
zinkevich2003online, hazan2016introduction}, we use Online Newton Step \citep{hazan2007logarithmic, cutkosky2018black} to ensure that $\Logwealth_t^G$ is not too far from the best achievable in hindsight. This results in the following update for $\{\lambda_t\}_{t \geq 1}$:
\begin{equation}\label{eq:bet_update}
\lambda_{t+1}^G \gets \mathop{\text{Proj}}\limits_{\left[0, 1\right]}\left(\lambda_t^G + \tfrac{2}{2- \ln(3)}\cdot\tfrac{z_t}{1 + \sum_{s \in [t]}z_{s}^2}
   \right),
\end{equation}
where $z_t = \frac{\1[X_t\in G ]- \beta \Basegroup}{1 + \lambda_t^G(\1[X_t\in G]- \beta \Basegroup)}$, and $\lambda_0 = 0$.\footnote{The constant $\frac
{2}{2 - \ln(3)}$ is due to \citet{cutkosky2018black}, who give a tighter version of ONS than in \citet{hazan2007logarithmic}.}

\paragraph{Setting the threshold $\theta_t(\alpha)$.}
Unlike the sequential Z-test, we use the same threshold for all timesteps. Including a Bonferroni correction, we use $\theta_t(\alpha) :=   \ln(\nicefrac{|\Groups|}{\alpha})$ for all $t$; the motivation for this setting will become clear in our discussion of Theorem \ref{thm:validity_evals}.

\paragraph{Theoretical guarantees.}
We first give a validity guarantee that is essentially identical to the Sequential Z-test. 

\begin{restatable}[Validity]{theorem}{evalsvalidity}\label{thm:validity_evals} Running Algorithm~\ref{alg:abstract} with $\theta_t(\alpha) = \ln{(\nicefrac{|\Groups|}{\alpha})}$ and $\Logwealth_t^G$ updated as per Equations~\eqref{eq:wealth_update} and \eqref{eq:bet_update} guarantees that 
the probability that $\FlagG$ will ever contains a group $G$ where $\NullH$ is true is at most $\alpha$, i.e. 
\[
\Pr\left[\exists t: \exists G \in \FlagG \text{ s.t. } \NullH \text{ holds}\right] \leq \alpha.
\]
\end{restatable}
This result follows directly from the prior work referenced at the beginning of this section. At a high level, every sequence $\{\exp(\Logwealth_t^G)\}_{t \geq 1}$ is a non-negative super-martingale under $\NullH$; informally, this means that under the null hypothesis, the sequence $\{\exp(\Logwealth_t^G)\}_{t \geq 1}$ should be non-increasing, in expectation. 
This allows us to apply Ville's inequality, which guarantees that it is unlikely that $\exp(\Logwealth_t^G)$ ever becomes too large under $\NullH$. More specifically, for any $\alpha \in (0,1)$, under the null, $\Pr[\exists t: \exp(\Logwealth_t^G) > 1/\alpha] \leq \alpha $. Thus, maintaining a threshold of $\theta_t(\alpha) = \ln(\nicefrac{|\Groups|}{\alpha})$ is sufficient to provide a per-hypothesis $\nicebonf$-validity guarantee, and thus $\alpha$-validity overall. 

We also provide the following bound on stopping time; see Appendix \ref{app:eval} for additional discussion of the $\Logwealth_\star$ notion of gap. 
\begin{restatable}[Power]{theorem}{evalspower} 
\label{thm:power_evals}
Let $T$ be the stopping time of Algorithm~\ref{alg:abstract} with $\theta_t(\alpha) = \ln{(|\Groups|/\alpha)}$ and $\Logwealth_t^G$ updated as per Equations~\eqref{eq:wealth_update} and \eqref{eq:bet_update}. If $\max_{G \in \Groups} \mu_G - \beta\Basegroup > 0$, then, we have that $\Pr[T < \infty] = 1$. Furthermore, 
\[
\E[T] \leq \mathcal{O}\left(\frac{1}{\Logwealth_\star^2} + \frac{\ln(|\Groups|) + \ln(1/\alpha)}{\Logwealth_\star}\right)
\]
where $\Logwealth_\star := \max_{G\in\mathcal{G}, \lambda\in[0,1]}\E[\ln(1+ \lambda(\mathbf{1}_{X_t\in G} -\beta\Basegroup))]$ is the maximal expected one-step increase in $\omega_t^G$ over all groups and choices of $\lambda$.
\end{restatable}
We conclude this section with two further remarks on Theorems \ref{thm:power_ztest} and \ref{thm:power_evals} in the context of our test. First, our modeling in Section~\ref{sec:interpretation} measures severity of harm via a \textit{multiplicative} factor of overrepresentation.
However, both notions of gap in Theorems \ref{thm:power_ztest} and \ref{thm:power_evals} also on the absolute size of the group $\mu_G$.
Thus, for two groups $G$ and $G'$ with identical multiplicative gaps, i.e. $\nicefrac{\mu_G}{\Basegroup} = \nicefrac{\mu_{G'}}{\mu_{G'}^0}$, the test would stop faster in expectation for $G$ if and only if $\Basegroup > \mu_{G'}^0$. That is, if two groups are ``harmed'' to the same extent, both algorithms will identify the larger one first. 

Second, for both tests, the Bonferroni correction results in only an additive factor ($\nicefrac{\ln(|\Groups|)}{\Delta_{\max}^2}$ in Theorem \ref{thm:validity_ztest}, and $\nicefrac{\ln(|\Groups|)}{\omega_\star}$ in Theorem \ref{thm:validity_evals}) in stopping time over the scenario where we had only been testing the one group with the largest gap. 
This means that, in terms of worst-case guarantee on stopping time, the contribution of the Bonferroni correction is small relative to the contribution of the test level $\alpha$ and, especially, to the gap. In fact, the impact of Bonferroni on real-world data appears to be much smaller even than this additive term.

\section{Experiments and Results}
\subsection{Experiment Settings}
% \begin{table*}[h]
%     \centering
%     \begin{tabular}{cl|ccccc|ccccc}
%      \multirow{3}{*}{\textbf{LLM}}  & \multirow{3}{*}{\textbf{Method}} &  \multicolumn{5}{c|}{\textbf{CCNews}} & \multicolumn{5}{c}{\textbf{Wikipedia}} \\ \cmidrule(lr){3-7}  \cmidrule(lr){8-12}
%       &  & PPL & Loss & Ref & min-k & \multicolumn{1}{c|}{zlib} & PPL & Loss & Ref & min-k & zlib \\ \midrule
%       \multirow{4}{*}{GPT2} & \textit{Base} & \textit{29.442} & \textit{0.505} & \textit{0.498} & \textit{0.520} & \textit{0.500} & \textit{34.429} & \textit{0.473} & \textit{0.513} & \textit{0.446} & \textit{0.497} \\ 
%       \multirow{4}{*}{124M} & FT & \textbf{21.861} & 0.607 & 0.855 & 0.549 & 0.569 & \textbf{12.729} & 0.577 & 0.967 & 0.489 & 0.544 \\
%       & Goldfish & 21.902 & 0.608 & 0.855 & 0.547 & 0.570 & 12.853 & 0.565 & 0.954 & 0.486 & 0.537 \\
%       & DPSGD & 26.022 & 0.507 & 0.513 & \textbf{0.521} & 0.502 & 18.523 & 0.463 & 0.536 & \textbf{0.448} & 0.491 \\
%       & \methodname & 23.733 & \textbf{0.502} & \textbf{0.495} & 0.529 & \textbf{0.499} & 13.628 & \textbf{0.454} & \textbf{0.463} & 0.470 & \textbf{0.485} \\ \midrule
      
%       \multirow{4}{*}{Pythia} & \textit{Base} & \textit{13.973} & \textit{0.507} & \textit{0.512} & \textit{0.528} & \textit{0.501} & \textit{10.287} & \textit{0.466} & \textit{0.503} & \textit{0.464} & \textit{0.489}\\ 
%       \multirow{4}{*}{1.4B} & FT & 11.922 & 0.602 & 0.857 & 0.541 & 0.574 & \textbf{6.439} & 0.578 & 0.985 & 0.484 & 0.557 \\
%       & Goldfish & \textbf{11.903} & 0.609 & 0.862 & 0.543 & 0.579 & 6.465 & 0.564 & 0.981 & 0.482 & 0.546 \\
%       & DPSGD & 13.286 & 0.512 & 0.531 & 0.528 & 0.503 & 7.751 & 0.469 & 0.524 & \textbf{0.462} & 0.488 \\
%       & \methodname & 12.670 & \textbf{0.501} & \textbf{0.460} & \textbf{0.524} & \textbf{0.499} & 6.553 & \textbf{0.468} & \textbf{0.485} & 0.472 & \textbf{0.485} \\ \midrule
      
%       \multirow{4}{*}{Llama-2} & \textit{Base} & \textit{9.364} & \textit{0.505} & \textit{0.495} & \textit{0.516} & \textit{0.503} & \textit{7.014} & \textit{0.458} & \textit{0.491} & \textit{0.476} & \textit{0.488} \\ 
%       \multirow{4}{*}{7B} & FT & \textbf{6.261} & 0.559 & 0.798 & 0.536 & 0.548 & \textbf{3.830} & 0.524 & 0.936 & 0.494 & 0.530 \\
%       & Goldfish & 6.280 & 0.552 & 0.780 & 0.533 & 0.541 & 3.839 & 0.518 & 0.929 & 0.492 & 0.525 \\
%       & DPSGD & 6.777 & 0.509 & 0.538 & 0.523 & 0.504 & 4.490 & 0.466 & 0.516 & \textbf{0.470} & 0.487 \\
%       & \methodname & 6.395 & \textbf{0.507} & \textbf{0.482} & \textbf{0.518} & \textbf{0.500} & 4.006 & \textbf{0.458} & \textbf{0.440} & 0.473 & \textbf{0.480} \\ 
%     \end{tabular}
%     \caption{Caption}
%     \label{tab:main_result}
% \end{table*}


\begin{table*}[h]
  \centering
  \resizebox{0.9\textwidth}{!}{\begin{tabular}{cl|ccccc|ccccc}
  \toprule[1pt]
   \multirow{3}{*}{\textbf{LLM}}  & \multirow{3}{*}{\textbf{Method}} &  \multicolumn{5}{c|}{\textbf{Wikipedia}} & \multicolumn{5}{c}{\textbf{CC-news}} \\ \cmidrule(lr){3-7}  \cmidrule(lr){8-12}
    &  & PPL & Loss & Ref & Min-k & \multicolumn{1}{c|}{Zlib} & PPL & Loss & Ref & Min-k & Zlib \\ \midrule
    \multirow{4}{*}{GPT2} & \textit{Base} & \textit{34.429} & \textit{0.473} & \textit{0.513} & \textit{0.446} & \textit{0.497} & \textit{29.442} & \textit{0.505} & \textit{0.498} & \textit{0.520} & \textit{0.500} \\ 
    \multirow{4}{*}{124M} & FT & \textbf{12.729} & 0.577 & 0.967 & 0.489 & 0.544 & \textbf{21.861} & 0.607 & 0.855 & 0.549 & 0.569 \\
    & Goldfish & 12.853 & 0.565 & 0.954 & 0.486 & 0.537 & 21.902 & 0.608 & 0.855 & 0.547 & 0.570 \\
    & DPSGD & 18.523 & 0.463 & 0.536 & \textbf{0.448} & 0.491 & 26.022 & 0.507 & 0.513 & \textbf{0.521} & 0.502 \\
    & \methodname & 13.628 & \textbf{0.454} & \textbf{0.463} & 0.470 & \textbf{0.485} & 23.733 & \textbf{0.502} & \textbf{0.495} & 0.529 & \textbf{0.499} \\ \midrule
    
    \multirow{4}{*}{Pythia} & \textit{Base} & \textit{10.287} & \textit{0.466} & \textit{0.503} & \textit{0.464} & \textit{0.489} & \textit{13.973} & \textit{0.507} & \textit{0.512} & \textit{0.528} & \textit{0.501}\\ 
    \multirow{4}{*}{1.4B} & FT & \textbf{6.439} & 0.578 & 0.985 & 0.484 & 0.557 & 11.922 & 0.602 & 0.857 & 0.541 & 0.574 \\
    & Goldfish & 6.465 & 0.564 & 0.981 & 0.482 & 0.546 & \textbf{11.903} & 0.609 & 0.862 & 0.543 & 0.579 \\
    & DPSGD & 7.751 & 0.469 & 0.524 & \textbf{0.462} & 0.488 & 13.286 & 0.512 & 0.531 & 0.528 & 0.503 \\
    & \methodname & 6.553 & \textbf{0.468} & \textbf{0.485} & 0.472 & \textbf{0.485} & 12.670 & \textbf{0.501} & \textbf{0.460} & \textbf{0.524} & \textbf{0.499} \\ \midrule
    
    \multirow{4}{*}{Llama-2} & \textit{Base} & \textit{7.014} & \textit{0.458} & \textit{0.491} & \textit{0.476} & \textit{0.488} & \textit{9.364} & \textit{0.505} & \textit{0.495} & \textit{0.516} & \textit{0.503} \\ 
    \multirow{4}{*}{7B} & FT & \textbf{3.830} & 0.524 & 0.936 & 0.494 & 0.530 & \textbf{6.261} & 0.559 & 0.798 & 0.536 & 0.548 \\
    & Goldfish & 3.839 & 0.518 & 0.929 & 0.492 & 0.525 & 6.280 & 0.552 & 0.780 & 0.533 & 0.541 \\
    & DPSGD & 4.490 & 0.466 & 0.516 & \textbf{0.470} & 0.487 & 6.777 & 0.509 & 0.538 & 0.523 & 0.504 \\
    & \methodname & 4.006 & \textbf{0.458} & \textbf{0.440} & 0.473 & \textbf{0.480} & 6.395 & \textbf{0.507} & \textbf{0.482} & \textbf{0.518} & \textbf{0.500} \\
    \bottomrule[1pt]
  \end{tabular}}
  \caption{Overall Evaluation: Perplexity (PPL) and AUC scores of the MIAs with different signals (Loss/Ref/Min-k/Zlib). For all metrics, the lower the value, the better the result. \textit{Base} in the method column indicates the pretrained LLMs without fine-tuning, thus it indicates lower bound for both utility and privacy risk.}
  \label{tab:main_result}
\end{table*}

% \begin{table*}[h]
%   \centering
%   \begin{tabular}{cl|ccccc|ccccc}
%   \multirow{3}{*}{\textbf{LLM}} & \multirow{3}{*}{\textbf{Method}} & \multicolumn{5}{c|}{\textbf{Wikipedia}} & \multicolumn{5}{c}{\textbf{CCNews}} \\
%   \cmidrule(lr){3-7} \cmidrule(lr){8-12}
%   & & PPL & Loss & Ref & min-k & \multicolumn{1}{c|}{zlib} & PPL & Loss & Ref & min-k & zlib \\
%   \midrule
%   \multirow{4}{*}{GPT2} & \textit{Base} & \textit{34.429} & \textit{0.473} & \textit{0.513} & \textit{0.446} & \textit{0.497} & \textit{29.442} & \textit{0.505} & \textit{0.498} & \textit{0.520} & \textit{0.500} \\
%   \multirow{4}{*}{124M} & FT & \textbf{12.729} & 0.577 & 0.967 & 0.489 & 0.544 & \textbf{21.861} & 0.607 & 0.855 & 0.549 & 0.569 \\
%   & Goldfish & 12.853 & 0.565 & 0.954 & 0.486 & 0.537 & 21.902 & 0.608 & 0.855 & 0.547 & 0.570 \\
%   & DPSGD & 18.523 & 0.463 & 0.536 & \textbf{0.448} & 0.491 & 26.022 & 0.507 & 0.513 & \textbf{0.521} & 0.502 \\
%   & \methodname & 13.628 & \textbf{0.454} & \textbf{0.463} & 0.470 & \textbf{0.485} & 23.733 & \textbf{0.502} & \textbf{0.495} & 0.529 & \textbf{0.499} \\
%   \midrule
%   \multirow{4}{*}{Pythia} & \textit{Base} & \textit{10.287} & \textit{0.466} & \textit{0.503} & \textit{0.464} & \textit{0.489} & \textit{13.973} & \textit{0.507} & \textit{0.512} & \textit{0.528} & \textit{0.501} \\
%   \multirow{4}{*}{1.4B} & FT & \textbf{6.439} & 0.578 & 0.985 & 0.484 & 0.557 & 11.922 & 0.602 & 0.857 & 0.541 & 0.574 \\
%   & Goldfish & 6.465 & 0.564 & 0.981 & 0.482 & 0.546 & \textbf{11.903} & 0.609 & 0.862 & 0.543 & 0.579 \\
%   & DPSGD & 7.751 & 0.469 & 0.524 & \textbf{0.462} & 0.488 & 13.286 & 0.512 & 0.531 & 0.528 & 0.503 \\
%   & \methodname & 6.553 & \textbf{0.468} & \textbf{0.485} & 0.472 & \textbf{0.485} & 12.670 & \textbf{0.501} & \textbf{0.460} & \textbf{0.524} & \textbf{0.499} \\
%   \midrule
%   \multirow{4}{*}{Llama-2} & \textit{Base} & \textit{7.014} & \textit{0.458} & \textit{0.491} & \textit{0.476} & \textit{0.488} & \textit{9.364} & \textit{0.505} & \textit{0.495} & \textit{0.516} & \textit{0.503} \\
%   \multirow{4}{*}{7B} & FT & \textbf{3.830} & 0.524 & 0.936 & 0.494 & 0.530 & \textbf{6.261} & 0.559 & 0.798 & 0.536 & 0.548 \\
%   & Goldfish & 3.839 & 0.518 & 0.929 & 0.492 & 0.525 & 6.280 & 0.552 & 0.780 & 0.533 & 0.541 \\
%   & DPSGD & 4.490 & 0.466 & 0.516 & \textbf{0.470} & 0.487 & 6.777 & 0.509 & 0.538 & 0.523 & 0.504 \\
%   & \methodname & 4.006 & \textbf{0.458} & \textbf{0.440} & 0.473 & \textbf{0.480} & 6.395 & \textbf{0.507} & \textbf{0.482} & \textbf{0.518} & \textbf{0.500} \\
%   \end{tabular}
%   \caption{Caption}
%   \label{tab:main_result}
%   \end{table*}
  

\textbf{Datasets}. We conduct experiments on two datasets: CC-news\footnote{\href{https://huggingface.co/datasets/vblagoje/cc_news}{Huggingface: vblagoje/cc\_news}} and Wikipedia\footnote{\href{https://huggingface.co/datasets/legacy-datasets/wikipedia}{Huggingface: legacy-datasets/Wikipedia}}. CC-news is a large collection of news articles which includes diverse topics and reflects real-world temporal events. Meanwhile, Wikipedia covers general knowledge across a wide range of disciplines, such as history, science, arts, and popular culture.\\
\textbf{LLMs}: We experiment on three models including \gpt~(124M)~\cite{gpt2radford}, \pythia~(1.4B)~\cite{pythia}, and \llama~(7B)~\cite{llama2touvron2023}. This selection of models ensures a wide range of model sizes from small to large that allows us to analyze scaling effects and generalizability across different capacities. \\
\textbf{Evaluation Metrics}. For evaluating language modeling performance, we measure perplexity (PPL), as it reflects the overall effectiveness of the model and is often correlated with improvements in other downstream tasks~\cite{kaplan2020scalinglaws, lmsfewshot}. For defense effectiveness, we consider the attack area under the curve (AUC) value and True Positive Rate (TPR) at low False Positive Rate (FPR). In total, we perform 4 MIAs with different MIA signals. Given the sample $x$, the MIA signal function $f$ is formulated as follows: \\
$\bullet$ Loss~\cite{8429311} utilizes the negative cross entropy loss as the MIA signal. 
    \[f_\text{Loss}(x) = \mathcal{L}_\text{CE}(\theta; x) \]
$\bullet$ Ref-Loss~\cite{Carlini2020ExtractingTD} considers the loss differences between the target model and the attack reference model. To enhance the generality, our experiments ensure there is no data contamination between the training data of the target, reference, and attack models.
    \[f_\text{Ref}(x) = \mathcal{L}_\text{CE}(\theta; x) - \mathcal{L}_\text{CE}(\theta_\text{attack}; x) \]
$\bullet$ Min-K~\cite{shi2024detecting} leverages top K tokens that have the lowest loss values.
    \[f_\text{min-K}(x) = \frac{1}{|\text{min-K(x)}|} \sum_{t_i \in \text{min-K(x)}} -\log(P(t_i|t_{<i};\theta) \]
$\bullet$ Zlib~\cite{Carlini2020ExtractingTD} calibrates the loss signal with the zlib compression size.
    \[ f_\text{zlib}(x) = \mathcal{L}_\text{CE}(\theta; x) / \text{zlib}(x) \]

\noindent \textbf{Baselines}. We present the results of four baselines. \textit{Base} refers to the pretrained LLM without fine tuning. \textit{FT} represents the standard causal language modeling without protection. \textit{Goldfish}~\cite{hans2024be} implements a masking mechanism. \textit{DPSGD}~\cite{abadi2016deep, yu2022differentially} applies gradient clipping and injects noise to achieve  sample-level differential privacy.

\noindent \textbf{Implementation}. We conduct full fine-tuning for \gpt and \pythia. For computing efficiency, \llama fine-tuning is implemented using Low-Rank Adaptation (LoRA)~\cite{hu2022lora} which leads to \textasciitilde4.2M trainable parameters. Additionally, we use subsets of 3K samples to fine-tune the LLMs. We present other implementation details in Appendix~\ref{sec:app-implementation}.

\subsection{Overall Evaluation}
Table~\ref{tab:main_result} provides the overall evaluation compared to several baselines across large language model architectures and datasets. Among these two datasets, CCNews is more challenging, which  leads to higher perplexity  for all LLMs and fine-tuning methods. Additionally, the reference-model-based attack performs the best and demonstrates high privacy risks with attack AUC on the conventional fine-tuned models at 0.95 and 0.85 for Wikipedia and CCNews, respectively. Goldfish achieves similar PPL to the conventional FT method but fails to defend against MIAs. This aligns with the reported results by \citet{hans2024be} that Goldfish resists exact match attacks but only marginally affects MIAs. DPSGD provides a very strong protection in all settings (AUC < 0.55) but with a significant PPL tradeoff. Our proposed \methodname guarantees a robust protection, even slightly better than DPSGD, but with a notably smaller tradeoff on language modeling performance. For example, on the Wikipedia dataset, \methodname delivers perplexity reduction by 15\% to 27\%. Moreover, Table~\ref{tab:tpr} (Appendix~\ref{sec:app-add-res}) provides the TPR at 1\% FPR. Both DPSGD and \methodname successfully reduce the TPR to $\sim$0.02 for all LLMs and datasets. \textit{Overall, across multiple LLM architectures and datasets, \methodname consistently offers ideal privacy protection with  little trade-off in language modeling performance.}

\noindent \textbf{Privacy-Utility Trade-off.}
To investigate the privacy-utility trade-off of the methods, we vary the hyper-parameters of the fine-tuning methods. Particularly, for DPSGD, we adjust the privacy budget $\epsilon$ from (8, 1e-5)-DP to (100, 1e-5)-DP. We modify the masking percentage of Goldfish from 25\% to 50\%. Additionally, we vary the loss weight $\alpha$ from 0.2 to 0.8 for \methodname. Figure~\ref{fig:priv-ult-tradeoff} depicts the privacy-utility trade-off for GPT2 on the CCNews dataset. Goldfish, with very large masking rate (50\%), can slightly reduce the risk of the reference attack but can increase the risks of other attacks. By varying the weight $\alpha$, \methodname offers an adjustable trade-off between privacy protection and language modeling performance. \methodname largely dominates DPSGD and improves the language modeling performance by around 10\% with the ideal privacy protection against MIAs.

\begin{figure}[h]
    \centering
    \includegraphics[width=\linewidth]{figs/privacy-ultility-tradeoff.pdf}
    \caption{Privacy-utility trade-off of the methods while varying hyper-parameters. The Goldfish masking rate is set to 25\%, 33\%, and 50\%. The privacy budget $\epsilon$ of DPSGD is evaluated at 8, 16, 50, and 100. The weight $\alpha$ of \methodname is configured at 0.2, 0.5, and 0.8.}
    \label{fig:priv-ult-tradeoff}
\end{figure}


\subsection{Ablation Study}
\textbf{\methodname without reference models.} To study the impact of the reference model, we adapt \methodname to a non-reference version which directly uses the loss of the current training model (i.e., $s(t_i) = \mathcal{L}_{CE}(\theta; t_i)$) to select the learning and unlearning tokens. This means the unlearning tokens are the tokens that have smallest loss values. Figure~\ref{fig:ppl-auc-noref} presents the training loss and testing perplexity. There is an inconsistent trend of the training loss and testing perplexity. Although the training loss decreases overtime, the test perplexity increases. This result indicates that identifying appropriate unlearning tokens  without a reference model is challenging and conducting unlearning on an incorrect set hurts the language modeling performance.

\begin{figure}[htp]
    \centering
    \includegraphics[width=0.35\textwidth]{figs/train_loss_ppl_noref.pdf}
    \caption{Training Loss and Test Perplexity of \methodname without a reference model.
    % (\lrx{If time permits, it would be better to compare with our training curve here)}
    }
    \label{fig:ppl-auc-noref}
\end{figure}

\noindent \textbf{\methodname with out-of-domain reference models.} To examine the influence of the distribution gap in the reference model, we replace the in-domain trained reference model with the original pretrained base model. 
Figure~\ref{fig:ppl-auc-base-woasc} depicts the language modeling performance and privacy risks in this study. \methodname with an out-of-domain reference model can reduce the privacy risks but yield a significant gap in language modeling performance compared to \methodname using an in-domain reference model.

\noindent \textbf{\methodname without Unlearning.} To study the effects of unlearning tokens, we implement \methodname which use the first term of the loss only ({$\mathcal{L}_{\theta} = \mathcal{L}_{CE}(\theta; \mathcal{T}_h)$}). Figure~\ref{fig:ppl-auc-base-woasc} provides the perplexity and MIA AUC scores in this setting. Generally, without gradient ascent, \methodname can marginally reduce membership inference risks while slightly improving the language modeling performance. The token selection serves as a regularizer that helps to improve the language modeling performance. Additionally, tokens that are learned well in previous epochs may not be selected in the next epochs. This slightly helps to not amplify the memorization on these tokens over epochs.

\begin{figure}[htp]
    \centering
    \includegraphics[width=0.28\textwidth]{figs/auc_vs_ppl_base_woasc.pdf}
    \caption{Privacy-utility trade-off of \methodname with different settings: in-domain reference model, out-domain reference model, and without unlearning}
    \label{fig:ppl-auc-base-woasc}
\end{figure}


\subsection{Training Dynamics}
\textbf{Memorization and Generalization Dynamics}. Figure~\ref{fig:training-dynamics} (left) illustrates the training dynamics of conventional fine tuning and \methodname, while Figure~\ref{fig:training-dynamics} (middle) depicts the membership inference risks. Generally, the gap between training and testing loss of conventional fine-tuning steadily increases overtime, leading to model overfitting and high privacy risks. In contrast, \methodname maintains a stable equilibrium where the gap remains more than 10 times smaller. This equilibrium arises from the dual-purpose loss, which balances learning on hard tokens while actively unlearning memorized tokens. By preventing excessive memorization, \methodname mitigates membership inference risks and enhances generalization.

\begin{figure*}[htp]
    \centering
    \includegraphics[width=0.29\linewidth]{figs/loss_vs_steps_ft_duolearn.pdf}
    \includegraphics[width=0.29\linewidth]{figs/auc_vs_steps_ft_duolearn.pdf}
    \includegraphics[width=0.316\linewidth]{figs/cosine.pdf}
    \caption{Training dynamics of \methodname and the conventional fine-tuning approach. The left and middle figures provide the training-testing gap and membership inference risks, respectively. The testing~$\mathcal{L}_{CE}$ of FT and training~$\mathcal{L}_{CE}$ of \methodname are significantly overlapping, we provide the breakdown in Figure~\ref{fig:add-overlap-breakdown} in Appendix~\ref{sec:app-add-res}. The right figure depicts the cosine similarity of the learning and unlearning gradients of \methodname. Cosine similarity of 1 means entire alignment, 0 indicates orthogonality, and -1 presents full conflict.}
    \label{fig:training-dynamics}
\end{figure*}

\noindent \textbf{Gradient Conflicts}. To study the conflict between the learning and unlearning objectives in our dual-purpose loss function, we compute the gradient for each objective separately. We then calculate the cosine similarity of these two gradients. Figure~\ref{fig:training-dynamics} (right) provides the cosine similarity between two gradients over time. During training, the cosine similarity typically ranges from -0.15 to 0.15. This indicates a mix of mild conflicts and near-orthogonal updates. On average, it decreases from 0.05 to -0.1. This trend reflects increasing gradient misalignment. Early in training, the model may not have strongly learned or memorized specific tokens, so the conflicts are weaker. Overtime, as the model learns more and memorization grows, the divergence between hard and memorized tokens increases, making the gradients less aligned. This gradient conflict is the root of the small degradation of language modeling performance of \methodname compared to the conventional fine tuning approach.

\noindent \textbf{Token Selection Dynamics}. Figure~\ref{fig:token-selection} illustrates the token selection dynamics of \methodname during training. The figure shows that the token selection process is dynamic and changes over epochs. In particular, some tokens are selected as an unlearning from the beginning to the end of the training. This indicates that a token, even without being selected as a learning token initially, can be learned and memorized through the connections with other tokens. This also confirms that simple masking as in Goldfish is not sufficient to protect against MIAs. Additionally, there are a significant number of tokens that are selected for learning in the early epochs but unlearned in the later epochs. This indicates that the model learned tokens and then memorized them over epochs, and the during-training unlearning process is essential to mitigate the memorization risks.

\begin{figure}[htp]
    \centering
    \includegraphics[width=0.7\linewidth]{figs/token-selection-dynamics.pdf}
    \caption{Token Selection Dynamics of \methodname}
    \label{fig:token-selection}
    \vspace{-4mm}
\end{figure}

\subsection{Privacy Backdoor}
To study the worst case of privacy attacks and defense effectiveness under the state-of-the-art MIA, we perform a privacy backdoor -- Precurious~\cite{precurious}. In this setup, the target model undergoes continual fine-tuning from a warm-up model. The attacker then applies a reference-based MIA that leverages the warm-up model as the attack's reference. Table~\ref{tab:backdoor} shows the language modeling and MIA performance on CCNews with GPT-2. Precurious increases the MIA AUC score by 5\%. Goldfish achieves the lowest PPL, aligning with~\citet{hans2024be}, where the Goldfish masking mechanism acts as a regularizer that potentially enhances generalization. Both DPSGD and \methodname provide strong privacy protection, with \methodname offering slightly better defense while maintaining lower perplexity than DPSGD.

% \begin{table}[h]
%     \centering
%     \begin{tabular}{c|cc|cc}
%        \multirow{2}{*}{\textbf{Method}}  & \multicolumn{2}{c}{\textbf{CCNews}} & \multicolumn{2}{c}{\textbf{Wikipedia}} \\ 
%        & \textbf{PPL} & \textbf{AUC} & \textbf{PPL} & \textbf{AUC} \\ \hline
%        \textbf{FT}        & 21.593 & 0.911 \\
%        \textbf{Goldfish}  & \textbf{21.074} & 0.886 \\
%        \textbf{DPSGD}     & 23.279 & 0.533 \\
%        \textbf{DuoLearn}  & 22.296 & \textbf{0.499} \\
%     \end{tabular}
%     \caption{Caption}
%     \label{tab:my_label}
% \end{table}

\begin{table}[h]
    \centering
    \resizebox{\columnwidth}{!}{\begin{tabular}{c|cccccc}
        \textbf{Metric} & \textbf{WU} & \textbf{FT} & \textbf{GF} & \textbf{DP} & \textbf{DuoL} \\ \hline
        \textbf{PPL} & \textit{23.318} & 21.593 & \textbf{21.074} & 23.279 & 22.296  \\
        \textbf{AUC} & \textit{0.500} & 0.911 & 0.886 & 0.533 & \textbf{0.499} \\
    \end{tabular}}
    \caption{Experimental results of privacy backdoor for GPT2 on the CC-news dataset. WU stands for the warm-up model leveraged by Precurious. GF, DP, and DuoL are abbreviations of Goldfish, DPSGD, and \methodname}
    \label{tab:backdoor}
\end{table}

% \subsubsection{Hyperparameter Study}

% \subsubsection{Full fine-tuning versus Parameter efficent fine tuning}

% \subsubsection{Extending to Vision Language Models}



\section{Discussion}
\label{sec:discussion}
In this work, we propose to leverage few-shot learning to enable users to self-define personal undesirable actions for personalized intervention on smartwatches.
We developed a three-stage pipeline that began with a self-supervised, pre-trained IMU model for robust feature extraction, then fine-tuned it for accurate human activity recognition, and finally enhanced it with data augmentation and synthesis that enabled rapid customization of new user-defined actions using only a small number of examples. 
We implemented this pipeline on a smartwatch as a real-time intervention system, \projectname, and demonstrated its effectiveness and advantages over the rule-based method through a multi-hour user study.
In this section, we discuss some interesting takeaways from our study, together with our vision of how \projectname can be generally applied to other health domains. We also briefly summarize the limitations of our work.


\subsection{Distorted Perception with AI-powered Intervention}
\label{sub:discussion:distorted}
During the study, we observed an interesting phenomenon where some participants developed a distorted perception towards their own actions or the intervention (see Sec.~\ref{sub:intervention_evaluation:qualitative_results}).
For instance, several participants felt \projectname's vibrations were stronger than the baseline (yet the actual strength of vibration remained constant), and some felt they did the target actions more frequently with \projectname (yet the objective data indicated otherwise).
There are several potential interpretations of such interesting observations.
The distorted perception might be caused by participants' heightened awareness of the AI-guided interventions: because \projectname more accurately and promptly caught the target actions, users started to pay extra and prolonged attention to any intervention. This could leave a stronger impression on them, and subsequently, they found it stronger or more frequent.
Another potential explanation is that the participants, often associating their personal and idiosyncratic undesirable actions with ``wrong-doing'' and thus responding with negative emotions, might have subconsciously perceived their undesirable actions as being more frequent due to the \projectname's more precise and timely feedback eliciting stronger negative emotions. This, combined with an emotional interpretation of being 'corrected', may have amplified their perception of the intervention's intensity (vibration strength) and created the mistaken impression of performing these actions excessively.

Meanwhile, it is an interesting open question of how long such perception will last from a longitudinal intervention perspective. Depending on the cases, the growing self-awareness and/or reliability of AI could potentially assist users in building a long-term habit to reduce the target action, or on the contrary, the effect may fade away due to the AI intervention method no longer being novel or enticing.
Future work can explore the lasting effect of the intervention, the corresponding perception, as well as user engagement in a long-term, field-based intervention study.~\cite{middleton2013long, short2018measuring, wei2020design}.


\subsection{Towards Human-AI Collaborative Interventions}
\label{sub:discussion:collaboration}
Users' mental models of \projectname varied significantly. Some viewed it as a passive watchdog, and some viewed it as a playful interactive system, while others sought to take greater agency in the moment of intervention delivery.
Our findings show the potential for and benefit of developing a collaborative relationship between humans and AI for behavioral intervention.
An AI system can provide appropriate support to users and help them achieve effective intervention outcomes.
Such collaboration is closely relevant to the vision of just-in-time adaptive interventions (JITAIs)~\cite{nahum-shani_translating_2021, nahum2018just}, where the delivery timing and methods of intervention are designed to be dynamically adapting to an individual's internal state and surrounding context.

For instance, for users who see the system as a passive monitor, the system can provide the option for them to configure the frequency and style of intervention (\eg higher/lower-intensity vibrations or consolidated notifications), ensuring the AI remains in the background but still provides supportive nudges.
Taking one step further, the AI system may analyze user behavior over time and suggest new setups or goals for users with transparency (\eg transitioning from nail-biting to managing stress). Users can accept, modify, or reject these suggestions, creating a dialogue where AI acts as a coach or collaborator rather than a rigid enforcer of predefined behaviors.
Meanwhile, for those who see AI as a proactive system, one promising avenue is to incorporate user feedback into the AI's learning process~\cite{orzikulova2024time2stop}. Users can label the AI's predictions as accurate or not, which could serve as input for the model to further adapt to the user and improve performance over time (\eg through reinforcement learning).
Combined with contextual information that can potentially be inferred from sensors~\cite{xu2023globem}, such feedback can enable more precise, context-sensitive and personalized JITIs.
In addition, the system would periodically prompt users to reassess their goals and update intervention targets, ensuring long-term relevance and efficacy.

It is noteworthy that such a human-AI collaboration paradigm needs to follow the principles of transparency and ethical design.
Other than the options mentioned above, namely custom configurations and continuous feedback, users should have visibility into the system's functionality and action logic regardless of the specific collaboration setup. This is important to provide users with agency and build their trust in the system.

\subsection{Beyond Smartwatch and Broader Customization}
In this work, our real-time intervention was implemented on a smartwatch. However, our proposed idea of empowering users to define any personal action and achieve a personalized intervention system can be more broadly applied to other domains.
Instead of relying solely on a watch-based IMU, we can explore other body-based sensor arrays (\eg headbands, rings, or joint sensors) to capture a more diverse range of behaviors in real time.
This would enable the system to accommodate a wide variety of undesirable actions or habits, such as posture corrections and fidgeting management.
In addition, beyond physical interventions, future customization can also delve into psychological or mental health support.
For instance, individuals dealing with obsessive-compulsive disorder (OCD) or other habitual thought/action patterns could define personal triggers (\eg a particular repetitive motion or behavioral cue) and receive timely AI-driven interventions.
Such holistic approaches highlight the flexibility and scalability of our pipeline.
By enabling user-defined actions, we open up possibilities for long-term and effective management of both physical and psychological well-being using a multitude of wearable and sensor-based platforms.

\subsection{Limitations}

Despite \projectname's positive outcome and the promising insights generated, we recognize some limitations in our study design.
As mentioned above, our current model relies solely on accelerometer data for action recognition, which may limit its ability to capture the full range of motion characteristics or other physiology. Future work can explore additional sensing modalities, such as gyroscope, photoplethysmography (PPG), joint locations, to enhance the accuracy and robustness of action recognition. 
Besides, the study was conducted with a relatively small number of participants and a limited set of actions, which may not fully capture the variability and diversity of human activities in real-world scenarios \cite{trapp2015individual, narayanan2013behavioral}.
Additionally, although we tried to simulate real-life scenarios, our intervention study was conducted over a limited duration and in controlled experimental settings, which may not fully reflect the complexities and dynamics of real-life environments. 
Real-world contexts introduce factors such as environmental noise, varying sensor placements, and user behavior changes over time \cite{trapp2015individual,truong2015deployment,mejia2023enhancing,mills2022development}, which were not thoroughly simulated in this study. Future research should conduct longitudinal field experiments with real-world deployment of the system.





\newpage
\section*{Acknowledgements}
We are grateful to Ian Waudby-Smith, Kevin Jamieson, and Robert Nowak for helpful discussions in developing this work. 

JD is supported in part by the National Science Foundation Graduate Research Fellowship Program under Grant No. DGE 2146752. Any opinions, findings, and conclusions or recommendations expressed in this material are those of the author(s) and do not necessarily reflect the views of the National Science Foundation. JD also thanks the AI Policy Hub at UC Berkeley for funding support in the 2023-2024 academic year. BR is generously supported in part by NSF CIF award 2326498, NSF IIS Award 2331881, and ONR Award N00014-24-1-2531. IDR is supported by the Mozilla Foundation and MacArthur Foundation. 

\bibliographystyle{plainnat}
\bibliography{z-biblio}

\newpage
\appendix
\section{Omitted Proofs}
\label{app:seq-proofs}
\iftoggle{icml}{
\subsection{Omitted proofs from Section \ref{sec:model}}
We prove Proposition \ref{prop:relativerisk-conversion}, restated below. 
\propRR*
\begin{proof}[Proof of Proposition \ref{prop:relativerisk-conversion}]
First, note that by definition of $\rho$, $\rho_G$, and $\RR_G$, we have 
\[
\rho_G \leq b \cdot \rho \iff \frac{\Pr[R = 1 \mid G ]}{\Pr[Y = 1 \mid G]} \leq b \cdot \frac{\Pr[R = 1]}{\Pr[Y = 1]} \iff \RR_G \geq \frac{\Pr[R = 1 \mid G]}{\Pr[R = 1]} \cdot \frac{1}{b}. 
\]
By Bayes' rule, $\frac{\Pr[R = 1 \mid G]}{\Pr[R = 1]} = \frac{\Pr[ G \mid R = 1]}{\Pr[G]} = \frac{\mu_G}{\Basegroup}$; furthermore, by assumption, we have $\frac{\mu_G}{\Basegroup} \geq \beta$. 
The result follows from combining with the previous display. 
\end{proof}
We prove Proposition \ref{prop:reporting-conversion}, restated below. 
\propIR*
\begin{proof}[Proof of Proposition \ref{prop:reporting-conversion}]
Recall that we have defined $\mu_G = \Pr[G \mid R]$, and $\Basegroup = \Pr[G]$ is known by Assumption \ref{assn:ref}.
By Bayes' rule, we have
$    \mu_G = \Pr[G \mid R] = \frac{\Pr[G]\Pr[R \mid G]}{\Pr[R]} =  \Basegroup\frac{\Pr[R \mid G]}{r},$
Now, let us decompose $\Pr[R \mid G]$ by ``true'' reports ($\Badevent = 1$) and ``false'' reports ($\Badevent = 0$). 
By the law of total probability,
$
    \Pr[R \mid G] 
    = r \cdot \left(\Truereportrate \Actualgroup + \Falsereportrate(1-\Actualgroup )\right)
$; more precisely, 
\begin{align*}
    \frac1r\Pr[R \mid G] &= \Pr[R \mid G, \Badevent = 1]\Pr[\Badevent \mid G]  + \Pr[R \mid G,  \Badevent =0](1-\Pr[\Badevent \mid G] )
    \\&= \Truereportrate \Actualgroup + \Falsereportrate(1-\Actualgroup )
    \\&= \Falsereportrate + \Actualgroup (\Truereportrate-\Falsereportrate);
\end{align*} 
combining this with the Bayes' rule computation, cancelling the $\frac1r$ factor, gives us $
    \Actualgroup  = \frac{\frac{\mu_G}{\Basegroup} - \Falsereportrate}{\Truereportrate-\Falsereportrate}.
$
The result follows from the assumption that $\nicefrac{\mu_G}{\Basegroup} \geq \beta.$
\end{proof}
}{}

\subsection{Omitted proofs for Sequential Z-test}
We prove Theorem \ref{thm:validity_ztest}, restated below. 
\ztestvalidity*

To prove this result, we will use a foundational result known as Ville's inequality \citep{ville1939etude}.
\begin{theorem}[Ville's inequality]
\label{thm:ville}
    Let $\{M_t\}_{t \in \mathbb{N}^+}$ be a non-negative supermartingale, i.e. for all $t$, $M_t \geq 0$, and $\E[M_{t+1} \mid \mathcal{F}_t] \leq M_t$, where $\mathcal{F}_t$ is the filtration (history) of all realizations of randomness up to and including time $t$. Then, for any $x \in (0,1)$, we have $\Pr[\exists t: M_t > \nicefrac{\E[M_0]}{x}] \leq x$.
\end{theorem}

The central thrust of our proof of Theorem \ref{thm:validity_ztest} is due to \citet{koolen2017quick} (which itself draws from \citet{balsubramani2014sharp}, and is a refinement of \citet{jamieson2014lil}); we reproduce the argument in the context of our work below, though we emphasize that we do not claim the proof technique as ours.
\begin{proof}[Proof of Theorem \ref{thm:validity_ztest}]
It is sufficient to show that for any group $G$ where $\NullH$ holds, we have $\Pr[ \exists t: G \in\FlagG] \leq \nicefrac{\alpha}{|\Groups|}$; the statement of the theorem follows from the Bonferroni correction over all $|\Groups|$ hypotheses. 

Ville's inequality (Theorem \ref{thm:ville}) appears similar in form to the statement we hope to prove; we therefore seek to transform our test statistic $\Logwealth_t^G = \sum_{s \in [t]}\1[X_s \in G]$ into a quantity that can be interpreted as a (non-negative) supermartingale. Although $\{\Logwealth_t^G\}_{t \in \mathbb{N}^+}$ is by itself clearly not a non-negative supermartingale, each $\Logwealth_t^G$ is the sum of $t$ Bernoulli trials with mean $\mu_G$, and Bernoulli random variables are sub-Gaussian with variance parameter $\nicefrac14$. Each $\Logwealth_t^G$ therefore satisfies the property that $\E[\exp(\eta(\Logwealth_t^G - \E[\Logwealth_t^G])] \leq \exp(\eta^2/8)$. 

This holds for any $\eta$, so we will construct a distribution $\phi$ on $\eta$ and use it to construct a martingale $M_t$. In particular, note that under $\NullH$, $\E[\Logwealth_t^G] < t \cdot \beta\Basegroup$. Thus, we 
let $S_t \coloneqq \Logwealth_t^G - \E[\Logwealth_t^G] =  \Logwealth_t^G - t \beta\Basegroup$. 
We will let 
$M_t = \int \phi(\eta) \exp(\eta S_t - t\eta^2/8) d\eta$.
Then, for any distribution $\phi$, $\{M_t\}_{t \in \mathbb{N}^+}$ is a non-negative supermartingale with respect to the randomness in realizations of reports $X_t$.
To see this, we have 
\begin{align*}
    \E[M_{t+1} \mid \mathcal{F}_t] &= \E\left[\int \phi(\eta)\exp\left(\eta(S_t + \1[X_{t+1} \in G] - \beta\Basegroup) - \tfrac{(t+1)\eta^2}{8}\right) d\eta \bigmid \mathcal{F}_t\right]
    \\&= \int \phi(\eta) \exp\left(\eta S_t - \tfrac{t\eta^2}{8}\right) \E\left[\exp\left(\eta(\1[X_{t+1} \in G] - \beta\Basegroup) - \tfrac{(t+1)\eta^2}{8}\right) \bigmid \mathcal{F}_t\right] d\eta 
    \\&\leq \int \phi(\eta) \exp\left(\eta S_t - \tfrac{t\eta^2}{8}\right) d\eta 
    \\&= M_t,
\end{align*}
where the inequality is due to $\frac1t\E[\Logwealth_t^G] \leq \beta\Basegroup$ and subgaussianity. It thus remains to use this martingale to compute an appropriate threshold $\theta_t(\alpha)$ on $\Logwealth_t^G$. 

$M_t$ will satisfy the conditions of Theorem \ref{thm:ville} for any choice of $\phi$, including one which puts point mass of 1 on $\eta = \eta'$ and 0 elsewhere, i.e. $\phi(\eta') = 1$ and $\phi(\eta) = 0$ for any $\eta \neq \eta'$. One path towards establishing the threshold $\theta_t(\alpha)$ is to simply pick one value of $\eta$; however, such an $\eta$ cannot depend on $t$ and would thus result in a suboptimal threshold. 
Instead, we will construct $\phi$ such that it is a discrete distribution, indexed by $i \in \mathbb{N}^+$, so that $\eta$ takes values $\eta_1, \dots, \eta_i$ with probability $\phi_1, \dots, \phi_i$; this allows each $\eta_i$ to depend on $t$ and therefore more finer-grained optimization of the threshold. Before committing to the exact distribution $\phi$, we first illustrate how $\phi_i$ and $\eta_i$ will be used in the threshold. 

Note that $M_t = \sum_{i \in \mathbb{N}^+} \phi_i\exp(\eta_iS_t - t\eta_i^2/8) \geq \max_i  \phi_i\exp(\eta_iS_t - t\eta_i^2/8)$, so for any $\delta$, we have 
\[
\{M_t \geq 1/\delta\} \supseteq  \{\max_i  \phi_i\exp(\eta_iS_t - t\eta_i^2/8) > 1/\delta \} = \left\{S_t \geq \min_i \left(\frac{t \eta_i}{8} + \frac{1}{\eta_i}\ln \frac{1}{\phi_i\delta}\right)\right\}, 
\]
and thus, picking $\theta_t(\alpha) = t \beta\Basegroup + \min_i \left(\frac{t \eta_i}{8} + \frac{1}{\eta_i}\ln \frac{1}{\phi_i\nicebonf}\right)$ would guarantee that $\Pr[\exists t: \Logwealth_t^G >  \theta_t(\alpha)] \leq \nicebonf$. 

Finally, we must commit to $\phi_i$, $\eta_i$, then optimize for $i$. Let $\phi_i = \frac{1}{i(i+1)}$ (note that $\sum_i \phi_i = 1$, so this is a valid distribution), $\eta_i = 2\sqrt{\frac{2\ln(1/\phi_i(\nicebonf))}{2^i } }$, and $i = \lfloor \log_2(t) \rceil$. For $i = \log_2(t)$ (without rounding), this would have yielded $\eta_i = 2 \sqrt{\frac{2 \ln((\log_2(t) + 1)(\log_2(t))/(\nicebonf))}{t}}$ and $\theta_t(\alpha) = \frac12\sqrt{2t \ln((\log_2t)(\log_2t + 1)/\nicebonf)}$. 
The statement follows from handling the numerical impact of rounding. 
\end{proof}
\begin{remark}
    A key constant in the proof of the version of the algorithm that is valid in finite samples is the subgaussian variance parameter, for which we used $\nicefrac14$ (and which propagates to a multiplicative factor of $\sqrt{1/4} = 1/2$ on the threshold). This is because the variance \textit{any} Bernoulli is at most $\nicefrac14$; however, this also motivates the choice of constant for the asymptotically-valid version of the test, which instead uses the variance parameter $\beta\Basegroup(1-\beta\Basegroup)$. 
\end{remark}

We now prove the power result. 
\ztestpower*
\begin{proof}
Let $\Gstar \coloneqq \arg\max_{G \in\Groups} \mu_G - \beta\Basegroup$ and let $\Delta \coloneqq \mu_{\Gstar} - \beta\Basestar$. Without loss of generality, we can consider only the test corresponding to $\Gstar$ (while still testing at level $\nicefrac{\alpha}{|\Groups|}$). 
Recall that for this instantiation of Algorithm \ref{alg:abstract}, the test statistic $\Logwealth_t^\Gstar = \sum_{s \in [t]} \1[X_s \in \Gstar]$ is simply the number of all reports belonging to $\Gstar$ by time $t$, and that stopping time $T$ is the first time where $\Logwealth_t^\Gstar$ surpasses the threshold $\theta_t(\alpha)$, i.e., $T \coloneqq \inf_{t \in {\mathbb{N}}^+} \Logwealth_t^\Gstar > t \beta\Basestar + \tfrac12 \sqrt{2.06 t \ln \left(|\Groups| \frac{(2 + \log_2(t))^2}{\alpha}\right)}$. For ease of notation, we will denote $C_1 \coloneqq \tfrac12\sqrt{2.06} = 0.718$ within this proof. 

For the first claim, it is sufficient to show $\liminf_{t \to \infty} \Pr[T > t] = 0$.\footnote{For a simple proof of this fact, see the solution to Problem 1.13 in \citet{bertsekas2008introduction}.} 
Recall that, by our modeling, we can consider $\Logwealth_t^\Gstar$ to be the sum of $t$ i.i.d. Bernoulli trials with parameter $\mu_\Gstar$.
Applying Hoeffding's inequality to this sum yields for any $t$ that
\begin{align*}
\Pr[T > t] &= 
\Pr\left[\Logwealth_t^\Gstar < t\beta\Basestar + C_1 \sqrt{t \cdot \ln \left(|\Groups| \frac{(2 + \log_2(t))^2}{\alpha}\right)}\right]
% \\&= \Pr\left[\E[\Logwealth_t^\Gstar] - \Logwealth_t^\Gstar > \Delta t + C_1 \sqrt{t \cdot \ln \left(|\Groups| \frac{(2 + \log_2(t))^2}{\alpha}\right)} \right]
% \\&\leq \exp\left(-\frac{2(t^2\Delta^2 + t C_1^2\ln(\frac{(2 + \log_2(t))^2}{\alpha}) - 2t\Delta C_1\sqrt{t \cdot \ln\left(|\Groups| \frac{(2 + \log_2(t))^2}{\alpha}\right)})}{t} \right)
\\&\leq \exp\left(-2\left(\Delta^2t - 2\Delta C_1 \sqrt{t \cdot \ln \left(|\Groups| \frac{(2 + \log_2(t))^2}{\alpha}\right)}\right) \right).
\end{align*}
Note that $\frac{\sqrt{t}\ln(\log_2(t))}{t} \to 0$; it can thus be seen that $\lim_{t \to\infty} \Pr[T > t] = \lim_{t \to \infty} \exp(-t) = 1$.

For the second claim, we apply Hoeffding's inequality again to see that for all $t$,  
\[
\Pr\left[\Logwealth_t^\Gstar \leq \E[\Logwealth_t^\Gstar] - C_1  \sqrt{ t \ln\left(\frac{(2 + \log_2(t))^2}{\delta}\right)}\right] \leq \Pr\left[\Logwealth_t^\Gstar \leq \E[\Logwealth_t^\Gstar] - \sqrt{\frac t2 \ln\left(\frac{ 1}{\delta}\right)}\right] \leq \delta.
\]
Thus, with probability at least $1-\delta$, for all $t$ simultaneously, $\Logwealth_t^\Gstar > t\mu_\Gstar - C_1  \sqrt{ t \ln\left(\frac{(2 + \log_2(t))^2}{\delta}\right)}$. 
The algorithm stops at time $t$ if and only if 
\[
t \mu_\Gstar - C_1  \sqrt{ t \ln\left(\frac{(2 + \log_2(t))^2}{\delta}\right)} > t \beta\Basestar + C_1  \sqrt{ t \ln\left(\frac{(2 + \log_2(t))^2}{\nicebonf}\right)}. 
\]
Rearranging, we have 
\[
\frac{t}{\left(\sqrt{\ln\left(\frac{(2 + \log_2(t))^2}{\nicebonf}\right)} + \sqrt{\ln\left(\frac{(2 + \log_2(t))^2}{\delta}\right)}\right)^2} \geq \frac{C_1}{\Delta^2}.
\]
Note that we can can upper bound the denominator of the left hand side as 
% \begin{align*}
$    \left(\sqrt{\ln\left(\frac{(2 + \log_2(t))^2}{\nicebonf}\right)} + \sqrt{\ln\left(\frac{(2 + \log_2(t))^2}{\delta}\right)}\right)^2 \leq 4 \ln\left(\frac{(2 + \log_2(t))^2}{\min(\nicebonf, \delta)}\right). $
% \end{align*}
Setting $\frac{t}{4 \ln\left(\frac{(2 + \log_2(t))^2}{\min(\nicebonf, \delta)}\right)} \geq \frac{C_1}{\Delta^2}$ and rearranging gives 
\begin{align}
\label{eq:tildeT}
\frac{t}{1 + \ln((2 + \log_2(t))^2)} \geq \frac{4C_1\ln(\max(\nicebonf, 1/\delta)}{\Delta^2}  
\end{align}
Thus, with probability $1-\delta$, the algorithm terminates at the smallest $t$ satisfying Equation \eqref{eq:tildeT}. The statement of the theorem follows from separating the two cases for $\delta < \nicebonf$ and $\delta \geq \nicebonf$, and noting that $\widetilde{O}$ notation suppresses the (negligible) log-log factor. 
\end{proof}

\subsection{Omitted proofs for betting-style algorithm}
\label{app:eval}

We first prove Theorem~\ref{thm:validity_evals}, restated for the sake of presentation.
\evalsvalidity*

\begin{proof} First note that for any $G$ for which $\NullH$ holds, the sequence $\{\exp(\Logwealth_t^G)\}_{t\geq 0}$ is a non-negative super-martingale. The non-negative property follows directly from the quantity being an exponential of a real (albeit possibly negative) number, while the fact that it is a super-martingale follows from the computations below:
\begin{align*}
    \E[\exp(\Logwealth_t^G)|\mathcal{F}_t] 
    &= \E[\exp(\Logwealth_{t-1}^G + \ln(1+\lambda_t^G(\1_{X_t\in G} - \beta \Basegroup)))|\mathcal{F}_t]
    \\&= \exp(\Logwealth_{t-1}^G) \cdot (1+\lambda_t^G(\E[\1_{X_t\in G}|\mathcal{F}_t] - \beta \Basegroup)) 
    \\&= \exp(\Logwealth_{t-1}^G) \cdot (1+\lambda_t^G(\mu_G - \beta \Basegroup) )
    \\&\leq \exp(\Logwealth_{t-1}^G) \cdot (1+\lambda_t^G(\beta \Basegroup - \beta \Basegroup) )
    \\&= \exp(\Logwealth_{t-1}^G),
\end{align*}
where the first equality follows by Eq.~\ref{eq:wealth_update}, the second by re-arranging and noting that all quantities except $\mathbf{1}_{X_t\in G}$ are completely determined by $\mathcal{F}_t$\footnote{In particular, it is imposed that $\lambda_t^G$ be 'predictable' which precisely implies that it is fixed given $\mathcal{F}_t$.}, and the third by definition (see Section~\ref{sec:model}). Finally, the inequality follows because $\mu_G \leq \beta\Basegroup$ under $\NullH$ and $\lambda_t^G \geq 0$.

Next, for any group $G$ such that $\NullH$ holds, we can apply Ville's inequality (Theorem~\ref{thm:ville}), plugging in the super-martingale $\{\exp(\Logwealth_t^G)\}_{t\geq 0}$ and taking $x$ to be $\theta_t(\alpha) = \log{(|\Groups|/\alpha)}$. This yields the following guarantee: 
\begin{align*}
\Pr[\exists t: \Logwealth_t^{G} > \log(|\Groups|/\alpha)] &= \Pr[\exists t: \exp(\Logwealth_t^{G}) > |\Groups|/\alpha] 
\\&\leq \E[\exp(\Logwealth_0^{G})] \cdot \alpha/|\Groups| 
\\&= \alpha/|\Groups|,
\end{align*}
where the final line follows because $\Logwealth_0^{G}$ is initialized as $0$ and hence $\exp(\Logwealth_0^{G})$ is equal to $1$.

Finally, by union bound we get the desired guarantee:
\begin{align*}
\Pr[\exists t: \exists G \in \FlagG \text{ s.t. } \NullH \text{ holds}] &\leq \sum_{G \text{ s.t. }\NullH \text{ holds}} \Pr[\exists t: \Logwealth_t^G > \log{(|\Groups|/\alpha)}] \\
&\leq |G \text{ s.t. }\NullH \text{ holds}| \cdot \alpha / |\Groups|\\
&\leq \alpha.
\end{align*}
\end{proof}

Before proving Theorem~\ref{thm:power_evals}, we first state and prove some helper results. 

\begin{claim}
\label{claim:logwealth-regret} For any $T\geq 4$ and group $G$, we have that the expected value over the randomness in the realizations of each $X_t$ of $\Logwealth_T^G$  defined as per Equations~\eqref{eq:wealth_update} and \eqref{eq:bet_update} can be lower bounded as
\[
\E[\Logwealth_T^G] \geq \E\left[\max_{\lambda\in[0, 1]} \Logwealth_T(\lambda)\right] - 2 \ln T.
\]
where we define $\Logwealth_T^G(\lambda)$ to be the quantity obtained by applying Equation~\eqref{eq:wealth_update} with $\lambda_t^G \coloneqq \lambda$ for all $t\in[T]$.
\end{claim}
\begin{proof} By the definition of regret we have that 
$\max_{\lambda\in[0, 1]} \Logwealth_T^G(\lambda) - \Logwealth_T^G \leq R_T$.
Rearranging and taking expectations, we have 
\[ \E[\Logwealth_T^G] \geq \E\left[\max_{\lambda\in[0, 1]} \Logwealth_T^G(\lambda)\right] -  \E[R_T].\]
Next, it can be verified that Equation~\eqref{eq:bet_update} implements the Online Newton Step algorithm for $\ln(1 + \lambda_t^G(\1_{X_t \in G} - \beta\Basegroup))$ (see Appendix C of \citet{cutkosky2018black}). We therefore have that $R_T \leq \frac{1}{2-\ln(3)}\ln(T  + 1)$ in general, and $R_T \leq 2\ln(T)$ for $T \geq 4.$ 
The statement of the claim plugging this into the expression above.
\end{proof}

\begin{lemma}\label{lem:lambda_opt} 
For each group $G$, taking $\lambda^\star_G = \mathrm{Proj}_{[0,1]}\left[\dfrac{\mu_G - \beta \mu_G^0}{\beta\mu_G^0(1-\beta\mu_G^0)}\right]$ maximizes expected log-wealth (at every step $t$). The resulting expected log-wealth at time $T$ (had $\lambda_G^\star$ been used at every time $t$) is equal to
\[
\E\left[\Logwealth_T^G(\lambda^\star_G)\right] = T\cdot \Logwealth_\star^G 
\]
where we denote $\Logwealth_\star^G \coloneqq \E[\ln(1+\lambda_G^\star(\mathbf{1}_{X_t\in G} - \beta\Basegroup)]$ the expected one-step wealth change under the bet $\lambda^\star_G$.
\end{lemma}

\begin{proof}
For a fixed $\lambda$, the log-wealth at time $T$ is given by 
\begin{equation*}
\Logwealth_T^G(\lambda) = N_T \ln{(1+\lambda(1-\beta\mu_G^0))} + (T-N_T) \ln{(1-\lambda\beta\mu_G^0)},
\end{equation*}
where $N_T = \sum_{t=1}^T \mathbf{1}_{X_t\in G}$. Taking expectations, we have that $\E[N_T] = T\cdot \mu_G$ and therefore 
\begin{equation}\label{eq:expected_logwealth}
\E\left[\Logwealth_T^G(\lambda)\right] = T\cdot\left[\mu_G \ln{(1+\lambda(1-\beta\mu_G^0))} + (1-\mu_G) \ln{(1-\lambda\beta\mu_G^0)}\right]. 
\end{equation}
To maximize \eqref{eq:expected_logwealth}, we only need to find $\lambda_G^\star\in[0,1]$ that maximizes the expressions in the square brackets. Taking the derivative we see that the function is concave,
and, therefore, we can solve for $\lambda_G^\star$ by setting the derivative to $0$ and then projecting the resulting value to $[0,1]$. This yields
$\lambda^\star_G = \text{Proj}_{[0,1]}\left[\frac{\mu_G - \beta \mu_G^0}{\beta\mu_G^0(1-\beta\mu_G^0)}\right].$ Plugging this back into \eqref{eq:expected_logwealth} we get
\begin{align*}
\E\left[\Logwealth_T^G(\lambda_G^\star)\right] &= T\cdot \left[\mu_G \ln{(1+\lambda_G^\star(1-\beta\mu_G^0))} + (1-\mu_G) \ln{(1-\lambda_G^\star\beta\mu_G^0)}\right] \\
&= T\cdot \E[\ln(1+\lambda_G^\star(\mathbf{1}_{X_t\in G} - \beta\Basegroup)] \\
&\coloneqq T\cdot \Logwealth_\star^G.
\end{align*}
\end{proof}

\begin{remark}
    Note that we can explicitly compute 
\[\Logwealth_\star^G = \mu_G \ln\left(    1 + \tfrac{\Delta_G}{\beta\Basegroup(1 - \mu_G )}\right) + \ln\left(1 - \tfrac{\Delta_G}{1-\beta\Basegroup}\right),\]
where $\Delta_G = \mu_G - \beta\Basegroup$, but this quantity is difficult to analyze, and it is not clear that $\Logwealth_\star^G$ can be explicitly lower bounded as $O(\Delta_G)$. 
\end{remark}

We now prove Theorem~\ref{thm:power_evals}, which we restate below.

\evalspower*

\begin{proof}
Let $\Gstar \coloneqq \arg\max_G \Logwealth_\star^G$ and denote the corresponding one-step wealth change $\Logwealth_\star = \Logwealth_\star^\Gstar$. Note that under the alternative this will correspond to a strictly positive quantity and is equivalent to the definition in the theorem statement. We can analyze the likelihood that its null has not been rejected by time $t$ as follows:
\begin{align*}
\Pr\left[\Logwealth_t^{\Gstar} < \ln(\nicebonfinv)\right] 
&=  \Pr\left[\Logwealth_t^{\Gstar} - \E[\Logwealth_t^{\Gstar}]< \ln(\nicebonfinv) - \E[\Logwealth_t^{\Gstar}]\right] \\ 
 &\leq \Pr\left[\Logwealth_t^{\Gstar} - \E[\Logwealth_t^{\Gstar}]< \ln(\nicebonfinv) - (t\cdot \Logwealth_\star - 2\ln t)\right], 
 \end{align*}
 where the inequality follows by Claim \ref{claim:logwealth-regret} and Lemma~\ref{lem:lambda_opt}, and the fact that $\E[\max_{\lambda\in[0, 1]} \Logwealth_t^\Gstar(\lambda)] \geq \E[\Logwealth_t^\Gstar(\lambda_\star^\Gstar)] = t \cdot \Logwealth_\star$. Whenever $t$ is large enough such that $\frac{\ln(t)}{t} \leq \frac{\Logwealth_\star}{4}$, we have 
\begin{align}
\label{eq:prnostop}
    \Pr[\Logwealth_t^\Gstar < \ln(\nicebonfinv)] \leq \Pr\left[\Logwealth_t^{\Gstar} - \E[\Logwealth_t^{\Gstar}]< \ln(\nicebonfinv) - \tfrac{3}{4}(t\cdot \Logwealth_\star)\right].
\end{align}
Since $\sqrt{t} \geq \ln{t}$ for all $t\in \mathbb{N}^\star$, this is satisfied in particular by taking $t\geq \frac{2^4}{\Logwealth_\star^2}$. Further, note that $\ln(\nicebonfinv) \leq \frac{t\cdot \Logwealth_\star}{4}$ whenever $t\geq \frac{2^2\cdot \ln{(\nicebonfinv)}}{\Logwealth_\star}$. So, for $t\geq \max\{\frac{2^4}{\Logwealth_\star^2}, \frac{2^2\cdot \ln{(\nicebonfinv)}}{\Logwealth_\star}\}$, we have
\[\Pr[\Logwealth_t^\Gstar < \ln(\nicebonfinv)] \leq \Pr\left[\Logwealth_t^{\Gstar} - \E[\Logwealth_t^{\Gstar}]< - \tfrac{1}{2}(t\cdot \Logwealth_\star)\right].\]
Now, note that since $\lambda_t^G \in [0,1]$, we have that each $\ln(1 + \lambda_t^G(\1_{X_t \in G} - \beta\Basegroup))$ lies in $[\ln{(1-\beta\Basegroup)}, \ln{(2 - \beta\Basegroup)}]$ and is therefore sub-Gaussian with parameter $\sigma = \frac{1}{2}\ln{\left(1+ \frac{1}{1-\beta\Basegroup}\right)}$; then, Hoeffding's inequality gives 
\begin{align*}
\Pr\left[\Logwealth_t^{\Gstar} - \E[\Logwealth_t^{\Gstar}]< - \tfrac{1}{2}(t\cdot \Logwealth_\star)\right]
&= 
\Pr\left[\sum_{i \in [t]} \ln(1 + \lambda_t^\Gstar(\1_{X_t \in \Gstar} - \beta\mu_\Gstar^0)) - \E[\Logwealth_t^{\Gstar}] \leq -\frac{1}{2} t \cdot \Logwealth_\star \right] \\
&\leq \exp\left(-\frac{(\frac{1}{2} t \cdot \Logwealth_\star)^2}{\frac{1}{2}t \ln^2(1+ \frac{1}{1-\beta\mu_\Gstar^0})}\right)  
\\&= \exp\left(-\frac{\Logwealth_\star^2}{2\ln^2(1+ \frac{1}{1-\beta\mu_\Gstar^0})} \cdot t \right) \\
&\leq  \exp\left(- \frac{(1-\beta\mu_\Gstar^0)^2}{2} \cdot \Logwealth_\star^2 \cdot t \right). 
\end{align*}
where for the last inequality we used $\ln{(1+x)} \leq x$. Now we are ready to analyze the stopping time $T$ of Algorithm~\ref{alg:abstract}. 

\paragraph{Test of power one.} Let $E_t$ be the event that we stop at time $t$, i.e. $E_t = \{\exists G$ such that $\Logwealth_t^G \geq \nicebonfinv\}$. We have that
\begin{align*}
\Pr[T = \infty] &= \Pr\left[\lim_{t \to \infty} \cap_{s \leq t} \neg E_t\right] \\
&= \lim_{t \to \infty} \Pr[ \cap_{s \leq t} \neg E_t] \\
&\leq \lim_{t \to \infty} \Pr[ \neg E_t] \\
&= \lim_{t \to \infty} \Pr[\forall G, \, \Logwealth_t^G < \ln(\nicebonfinv)] \\
&\leq \lim_{t \to \infty}\Pr\left[\Logwealth_t^{\Gstar} < \ln(\nicebonfinv)\right] \\
&\leq \lim_{t \to \infty}\exp\left(- \frac{(1-\beta\Basegroup)^2}{2} \cdot \Logwealth_\star^2 \cdot t \right) \\
&= 0.
\end{align*}

\paragraph{Expected Stopping Time.} Since $T$ is a positive integer, we can express the expected stopping time as
\begin{align}
\E[T] &= \sum_{t=1}^\infty \Pr[T>t] \notag \\
&= \sum_{t=1}^\infty \Pr[\neg E_1 \land \ldots \land \neg E_t] \notag\\
&\leq \sum_{t=1}^\infty \Pr[\neg E_t] \notag\\
&= \sum_{t=1}^\infty \Pr[\forall G, \, \Logwealth_t^G < \ln(\nicebonfinv)]\notag \\
&\leq \sum_{t=1}^\infty \Pr\left[\Logwealth_t^{\Gstar} < \ln(\nicebonfinv)\right] \notag\\
&\leq \max\bigg\{\frac{2^4}{\Logwealth_\star^2}, \frac{2^2 \cdot \ln(\nicebonfinv)}{\Logwealth_\star}\bigg\} + \sum_{t=1}^\infty \exp\left(- \frac{(1-\beta\mu_\Gstar^0)^2}{2} \cdot \Logwealth_\star^2 \cdot t \right) \label{eq:our_up}\\
&= \max\bigg\{\frac{2^4}{\Logwealth_\star^2}, \frac{2^2 \cdot \ln(\nicebonfinv)}{\Logwealth_\star}\bigg\} + \frac{1}{\exp{((1-\beta\mu_\Gstar^0)^2 \Logwealth_\star^2 /2)} - 1} \notag\\
&\leq \max\bigg\{\frac{2^4}{\Logwealth_\star^2}, \frac{2^2 \cdot \ln(\nicebonfinv)}{\Logwealth_\star}\bigg\} + \frac{2}{(1-\beta\mu_\Gstar^0)^2 \Logwealth_\star^2} \label{eq:exp_ineq}\\
&\leq \mathcal{O}\left(\frac{1}{\Logwealth_\star^2} + \frac{\ln(\nicebonfinv)}{\Logwealth_\star}\right) \notag
\end{align}
where \eqref{eq:our_up} follows from the upper bound on $\Pr\left[\Logwealth_t^{\Gstar} < \ln(\nicebonfinv)\right]$ for $t \geq \max\left\{\frac{2^4}{\Logwealth_\star^2}, \frac{2^2 \cdot \ln(\nicebonfinv)}{\Logwealth_\star}\right\}$ derived in \eqref{eq:prnostop}, and \eqref{eq:exp_ineq} follows from $\exp(x) \geq 1+x$. 
\end{proof}

\section{Further practical considerations}
\label{subsec:practical}
\paragraph{Choosing $\Groups$.}
In our experiments in Section \ref{sec:experiments}, we choose to define subgroups as all possible combinations of available demographic characteristics.
That said, a practitioner may seek to define $\Groups$ more carefully in accordance with their application. For instance, if the goal is to illustrate discrimination in a legal sense, $\Groups$ should be defined with respect to (protected) demographic features, rather than arbitrary combinations of covariates. On the other hand, groups need not be solely demographic, which allows our approach to test for safety rather than solely fairness. For example, $\Groups$ could include which batch of a medication an individual received; our tests could then help identify whether some batches were improperly manufactured.  

\paragraph{Baseline rates $\{\Basegroup\}_{G \in \Groups}$.} A natural question that arises from the modeling in this section is how $\{\Basegroup\}_{G \in \Groups}$ can be determined, or if Assumption \ref{assn:ref} is strictly necessary. 
Practically speaking, these base preponderances may be estimated, possibly with some amount of noise; however, the estimation problem can be addressed with standard techniques and is not core to our contribution. 
Similarly, in practice these baseline preponderances may change over time (e.g. if some subgroups increased uptake of a vaccine, or applied for loans more frequently, over time); however, such situations are relatively straightforward to handle under our algorithmic frameworks (see, e.g., the variants discussed in \citet{chugg2024auditing}). 
We therefore focus on the case where we have access to the true, underlying values of $\{\Basegroup\}_{G \in \Groups}$ for ease and clarity of exposition. 

Note that testing against base preponderances of the reference population (i.e., to compare $\mu_G$ to $\Basegroup$) is a new test proposed by this work, and the analysis in Sections \ref{subsec:rr} and \ref{subsec:ir} is specific to this test. Existing approaches to monitoring in incident databases compare to different baselines, most commonly the historical overall incidence rate for the specific symptom, sometimes by subgroup
\citep{shimabukuro2015safety, kulldorff2011maximized, oster2022myocarditis}. 
These baselines could, in principle, be plugged into the algorithms in Section \ref{sec:algs}, but new analysis for (possibly group-varying) reporting rates would be necessary to draw inferences about analogous quantities of interest (e.g., $\RR$ or $\mathrm{IR}$), as current approaches do not generally consider reporting behavior.
In contrast, our modeling allows us to identify what quantities may affect the true incidence rate even if they may be unmeasurable.

\paragraph{Setting $\beta$.} Finally, to run the test proposed in Equation \eqref{eq:htest}, it is necessary to determine how to set the value of $\beta$. As we will see in Section \ref{sec:algs}, when $\beta$ is set too high, then the test may never identify problematic groups, or identify them more slowly; on the other hand, as is clear from the previous subsections, rejecting the null hypothesis for a smaller $\beta$ requires more stringent assumptions on reporting behavior. Thus, we suggest a procedure to set $\beta$ as follows: (1) choose a relative risk or incidence rate threshold where it would be problematic for any group if $\RR_G$ or $\mathrm{IR}_G$ surpassed that threshold; (2) make the corresponding assumptions about reporting behavior; (3) use these quantities to compute a reasonable $\beta$.
We give some example computations in Section \ref{sec:experiments}. 
Due to an equivalence between hypothesis testing and confidence intervals, it is statistically valid to rerun tests with different $\beta$s once data collection has begun. Thus, it may be prudent to begin by setting the lowest $\beta$ that reporting assumptions would allow; then, if the tests appear to be stopping very quickly, to re-run them at higher $\beta$s, which would allow a practitioner to get a better sense of the severity of the harm.

\end{document}
