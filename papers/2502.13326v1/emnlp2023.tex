% This must be in the first 5 lines to tell arXiv to use pdfLaTeX, which is strongly recommended.
%EDIT: https://www.overleaf.com/8654555173qyhjbvdzdkkb#cfe513
\pdfoutput=1
% In particular, the hyperref package requires pdfLaTeX in order to break URLs across lines.

\documentclass[11pt]{article}

% Remove the "review" option to generate the final version.
\usepackage[]{EMNLP2023}
\usepackage{enumitem}
% Standard package includes
\usepackage{times}

\usepackage{latexsym}
\usepackage{graphicx}
\usepackage{booktabs}


% For proper rendering and hyphenation of words containing Latin characters (including in bib files)
\usepackage[T1]{fontenc}
% For Vietnamese characters
% \usepackage[T5]{fontenc}
% See https://www.latex-project.org/help/documentation/encguide.pdf for other character sets

% This assumes your files are encoded as UTF8
\usepackage[utf8]{inputenc}

% This is not strictly necessary and may be commented out.
% However, it will improve the layout of the manuscript,
% and will typically save some space.
\usepackage{microtype}

% This is also not strictly necessary and may be commented out.
% However, it will improve the aesthetics of text in
% the typewriter font.
\usepackage{inconsolata}
%\usepackage[table,xcdraw]{xcolor}
\usepackage{colortbl} 
%\usepackage[table]{}
\newcommand{\vase}[1]{\textbf{\textcolor{orange}{$\rhd$ Vasudha: #1}}}
\newcommand{\ryan}[1]{\textbf{\textcolor{blue}{$\rhd$ Ryan: #1}}}
\newcommand{\andy}[1]{\textbf{\textcolor{blue}{$\rhd$ Andy: #1}}}
\newcommand{\syeda}[1]{\textbf{\textcolor{blue}{$\rhd$ Syeda: #1}}}

\usepackage{amsmath}
% If the title and author information does not fit in the area allocated, uncomment the following
%
%\setlength\titlebox{<dim>}
%
% and set <dim> to something 5cm or larger.

\title{Can Language capture Cognitive Dissonance?}
\title{\textit{I want to but I can't}: Can Language capture Cognitive Coherence in Decision-Making?}
\title{\textit{I want to but I can't}: Validating Cognitive Processes in Language}%\\ A Case of Decision-Making}
\title{\texttt{CoPLa: }Validating Cognitive Processes with Language \\ Cognitive Coherence in Decision-Making}
\title{\textit{I want to but I can't}: Validating Cognitive Coherence with Language}
\title{ Can Language capture Cognitive Processes?\\ The Case of Decision-Making}

\title{\textit{I want to but I can't}: Towards Capturing Cognitive Styles with \\Language in the context of Decision Making}
\title{Capturing Human Cognitive Styles with Language: \\Towards an Experimental Evaluation Paradigm 
}


% Author information can be set in various styles:
% For several authors from the same institution:
% \author{Author 1 \and ... \and Author n \\
% Address line \\ ... \\ Address line}
% if the names do not fit well on one line use
% Author 1 \\ {\bf Author 2} \\ ... \\ {\bf Author n} \\
% For authors from different institutions:
% \author{Author 1 \\ Address line \\ ... \\ Address line
% \And ... \And
% Author n \\ Address line \\ ... \\ Address line}
% To start a separate ``row'' of authors use \AND, as in
% \author{Author 1 \\ Address line \\ ... \\ Address line
% \AND
% Author 2 \\ Address line \\ ... \\ Address line \And
% Author 3 \\ Address line \\ ... \\ Address line}

%\author{Vasudha Varadarajan$^\dagger$$^$, Syeda Mahwish$^\dagger$, Xiaoran Liu$^\spadesuit$, Julia Buffolino$^\dagger$\\ \textbf{Christian C. Luhmann$^\spadesuit$ \textbf{Ryan L. Boyd$^\dagger$ \and H. Andrew Schwartz$^\dagger$} \\
%$^\dagger$Department of Computer Science, $^\spadesuit$Department of Psychology\\ Stony Brook University \\
%{\tt \small \{vvaradarajan, smahwish,has\}@cs.stonybrook.edu} \\
%{\tt \small \{christian.luhmann,xiaoran.liu\}@stonybrook.edu}
% \\}
% }

\author{
Vasudha Varadarajan$^\dagger$, Syeda Mahwish$^\dagger$, Xiaoran Liu$^\spadesuit$, Julia Buffolino$^\dagger$\\
\textbf{Christian C. Luhmann$^\spadesuit$, Ryan L. Boyd$^\clubsuit$, H. Andrew Schwartz$^\dagger$} \\
$^\dagger$Department of Computer Science, Stony Brook University \\
$^\spadesuit$Department of Psychology, Stony Brook University \\
$^\clubsuit$Department of Psychology, University of Texas at Dallas \\
{\tt  \{vvaradarajan, has\}@cs.stonybrook.edu} 
%{\tt \small \{christian.luhmann, xiaoran.liu\}@stonybrook.edu}, 
%{\tt \small boyd@utdallas.edu}
}

 

\begin{document}
\maketitle
\begin{abstract}

While NLP models often seek to capture cognitive states via language, the validity of predicted states is determined by comparing them to annotations created without access the cognitive states of the authors.
%judged labels from annotators who do not have direct access to the cognitive states of the authors. 
%whether cognitive states are captured is typically done with annotators judging the cognitive state of the authors.
% language can provide an insight into human cognitive states, 
% %there has not been much focus on language 
% little is known about the about the connection of underlying cognitive styles with language. 
% In this paper,
%andy: CUT "and advocate for a more stringent" advocate = language for a position paper; this is still not a position paper: 
In behavioral sciences, cognitive states are instead measured via experiments. 
Here, we introduce an experiment-based framework 
for evaluating language-based cognitive style models against human behavior. 
%to allow collection of linguistic data pertaining to cognitive styles using appropriate psychological, construct-validated experiments that capture behavior the subjects. 
We explore the phenomenon of decision making, and its relationship to the linguistic style of an individual talking about a recent decision they made. 
The participants then follow a classical decision-making experiment that captures their cognitive style, determined by how preferences change during a decision exercise. 
%the choice-induced shift in preferences and other decision outcomes. 
We find that language features, intended to capture cognitive style, can %indeed 
predict participants' decision style with moderate-to-high accuracy (AUC $\sim$ 0.8), demonstrating that cognitive style can be partly captured and revealed by discourse patterns.

\end{abstract}

\section{Introduction}

%\subsection{Motivation}


%\ryan{Language predicts some type of cognitive process. Experimental paradigm -- shared underlying cognitive process. In theory the way they talk about their decisions in the past should be similar to the ongoing process as well -- we want to capture some type of cognitive behavior that is relevant / in the same domain (similar to attitudes research-- can attitude predict behavior? -- specificity -- ask how they their attitude is in that specific situation) Parallel of language about decision + decision making process is meaningful, traits or self report is commonplace -- we are predicting something mechanistic/ person-level/ person is unaware about. }

%\ryan{Top-level narrative : the results are an instantiation of a larger narrative about the experimental framework itself.}

%\ryan{Guns blazing! Don't worry to much about the nuances!}

%% NEXT TWO LINES CAN PROBABLY BE MOVED TO ANOTHER PARAGRAPH OR OMITTED:
%Although language has been studied in the NLP community as structured constructions of words or phrases \cite{naseem2021comprehensive} or their semantic roles \cite{li2022survey}, language is fundamentally a cognitive process, which is used to convey personal thoughts, emotions\cite{?}, argumentation\cite{?} and narrative storytelling\cite{?}. 
%Recent developments in LLMs have paved the way for automated processing of cognitive states and traits of humans that interact with them, capturing the \textit{cognitive styles} of the authors, which refers to sets of thoughts patterns pertaining to various cognitive phenomena \footnote{In this work, we specifically focus on decision making, and \textit{cognitive style} specifically refers to decision making style of individuals.}.


%Standard evaluation paradigm has focused on annotations.


\begin{figure}[!ht]
 \centering
 %link to figure: https://docs.google.com/drawings/d/1IaxdUPeQRMDh2ahI36Xln-u2cr6zmlD93Oqu0uMCvkw/edit?usp=sharing
 \includegraphics[width=0.9\columnwidth]{diagrams/validation_paper_spirit_diagram.pdf}
 \caption{An alternate evaluation framework for validation of cognitive processes with language: The participants are first prompted to write about their experiences, eliciting their thought process. Then they are subjected to an experiment that would measure their behavior. The behavior is a \textit{ground truth} measure of their cognitive style that can be tied to the expressed language. %This framework is used for the task of measuring cognitive style as shown in Fig \ref{fig:simons_exp}.
 %\vase{How to validate cognitive style in general, maybe move the }
 }
 \label{fig:spirit}
\end{figure}

%As language models get increasingly bigger and better there is a need to measure and validate their progress in terms of understanding humans' personal and social cues for improved conversational interactions, and other tasks to help humans (sth like that). Validation traditionally has relied on \textit{annotations} vast amounts of data pertaining to various constructs \cite{sentiment, emotion, anxiety, intent, norms, other SOCKET constructs as well etc.}. Such an approach has allowed the field to progress by producing models of increasing predictive power \cite{socialite llama}. Some other works seek to partially address the issue by focused on self-report values on a questionnaire as a proxy to validate the measure in language. However, there are annotator biases and disagreements that dictate the final outcomes, and self-report biases when reporting a construct on their own. For example, study of empathy in NLP has been focusing on either annotations of Bateson's self report of empathy \cite{}. This doesn't captures the general empathic cues from a person in an actual setting of a siatuation that evokes feeling of empathy, instead it is focused on the presence of linguistic cues that make the reader think the writer might be empathic, making the task very contained to language (empathy is not just about expressing it) or a general measure of empathy based on self-concept that is decontextualized from feeling empathy in a social situation (neither is it a trait). This necessitates this turn towards judgment and decision making where experimental outcomes are designed to capture cognitive constructs better than any observer or the person themselves. These experiments measure constructs by measuring the the state of a person before and after experiment to observe changes.
%%CHANGE THE PARAGRAPH TO MAKE THE ANNOTATION PROB MORE PROMINENT

%As language models become increasingly larger and more advanced, there is growth in the evaluation of their abilities to understand the cognitive states of people behind the language in a similar way that people do in social situations~\cite{choi-etal-2023-llms, dey2024socialite}. 
%Such evaluations largely rely on annotating vast amounts of data related to various constructs, such as sentiment and emotions \cite{rosenthal2019semeval,mohammad2018semeval}, empathy \cite{sharma2020computational}, politeness \cite{hayati-etal-2021-bert}, humor \cite{meaney2021semeval}, dissonance \cite{varadarajan-etal-2023}, reasoning abilities \cite{alhamzeh2022s} etc. 
%However, the utility of annotation-based evaluations for tasks that seek to capture author cognitive states, have come under concern only evaluating against proxies for such states~\cite{sandri-etal-2023-dont, sap2021annotators}.
%Annotations are just perceptions of states while fields studying cognition tend to use experiments to more objectively capture states.  
%Instead, for example, annotation-based labels of empathy draw attention to linguistic cues that \textit{look like} empathetic responses~\cite{}. 

%NLP MODELS HAVE RELIED ON ANNOTATION
While language models grow in sophistication, NLP tasks increasingly focus on understanding the people behind the language~\cite{choi-etal-2023-llms, dey2024socialite}.
%researchers are increasingly evaluating models' ability to interpret the cognitive states behind language, similar to human interactions in social settings~\cite{choi-etal-2023-llms, dey2024socialite}. 
Such social and psychological NLP studies still rely primarily on \textbf{annotations} for evaluation. 
%rather than understanding people. 
For example, recent social tasks have depended on annotated datasets for, e.g.,  emotions~\cite{rosenthal2019semeval,mohammad2018semeval}, empathy~\cite{sharma2020computational}, politeness~\cite{hayati-etal-2021-bert}, humor~\cite{meaney2021semeval}, dissonance~\cite{varadarajan-etal-2023}, and reasoning abilities~\cite{alhamzeh2022s}. 
%However, this annotation paradigm emerged when the field was more focused on syntax or non-social semantics. 
%DRAWBACKS OF ANNOTATION BASED METHODS
However, while annotation-based work has pushed NLP towards capturing cognitive states of the language generators (i.e. people),
% annotation-based evaluations are inherently only proxies for the underlying cognitive states
% However the effectiveness of annotation-based evaluations for tasks that aim to determine cognitive states of authors is questionable, as they often assess only proxies for these states
it falls short of offering \textit{ground truth} of psychological processes %that an individual undergoes 
because annotations reflect \textit{perception} of another person's state.~\cite{sandri-etal-2023-dont, sap2021annotators}.
For example, annotations of empathy point to linguistic cues that \textit{appear} empathetic to observers but do not always reflect the actual human experience of empathy~\cite{lahnala-etal-2022-critical}. 
%Outside of NLP, fields like psychology, human cognitive states are often studied directly through experimental paradigms.
Behavioral sciences, on the other hand, often emphasize the importance of \textit{direct} assessment through experimental paradigms for the purpose of understanding constructs of interest.
%IF TIME and good citations, include 
%Fields like psychology turn to experiments to infer human and cognitive states~\cite{others,luhmann2013}.

%OUR WORK'S CONTRIBUTIONS
%In this work, we introduce an experimental evaluative framework where by linguistic data is collected in tandem with cognitive phenomenon that an individual experiences, which is induced using appropriate psychological experiments (Figure \ref{fig:spirit}).
%To demonstrate this framework, we specifically evaluate discourse modeling approaches intended to capture cognitive styles associated with decision making. 
%Discourse analysis approaches are applied to writing about a recent decision in life from participants who also participated separately in an experiment designed to elicit cognitive style. 

% to quantify how much language use signals one's cognitive style. 
%we explore one such implicit phenomenon: cognitive dissonance.


%Associating linguistic patterns with specific cognitive phenomena can be a powerful tool in understanding the unique \textit{cognitive styles} of individuals -- their implicit behavioral traits such as a person's judgment and decision making abilities, reasoning skills, and empathy~\cite{}. %However it is less widely studied due to the scarcity and high cost of procuring linguistic data pertaining to specific cognitive processes. 
%Adding in the use of an experimental test, one could evaluate whether language captures meaningful outcomes associated with cognitive style.
%to quantify how much language use signals one's cognitive style. 
%When faced with multiple choices, individuals tend to weigh the trade-offs based on a satisfaction criteria and make a decision rather than discrete logical rules~\cite{}
%A large part of the complex cognitive phenomena involved in decision-making is unseen to the outside, and the subject could be unaware of them~\cite{campitelli2010herbert}. 
%Usually, the final decision is the only the observable outcome.
%This has lead modern studies of decision making to focus on empirical experimental evaluations of behavior~\cite{}
%Our study seeks to abide by modern psychological experimental designs in order to quantify how much language use signals one's cognitive style.
%In reality, decision making involves a complex set of cognitive processes \cite{}. 
%We are advocating for better, stringent evaluation framework.

We introduce an experimental framework that collects linguistic data alongside induced cognitive phenomena to evaluate the feasibility of discourse modeling approaches for capturing cognitive styles in decision making. By associating linguistic patterns with specific cognitive phenomena, we aim to understand individuals' unique cognitive styles, which are largely unseen and often only observable through the final decision~\cite{campitelli2010herbert}. Our study follows modern psychology--experimental designs to quantify how language use signals Cognitive Styles, or habitual patterns of thought related to various cognitive phenomena. %, involve not just linguistic expressions but also context-dependent changes in mental state in response to social situations. 
%This necessitates a shift towards construct-validated psychological experiments that capture cognitive styles more accurately than external annotators or the individuals themselves by measuring a person's mental state before and after an experiment to observe changes.
 

Our key contributions include: (1) An experiment-based evaluation framework to validate cognitive styles in language;
(2) Exploring discourse and other linguistic features for modeling decision-making cognitive styles;
(3) Finding that language can be indicative of a person's cognitive styles even in the more stringent evaluation framework;
(4) \textit{Decisions} dataset for the language of decision-making cognitive styles.\footnote{For dataset and code: \hyperlink{https://github.com/humanlab/cog_style_validation}{https://github.com/humanlab/cog\_style\_validation}}

%back to motivated based on related work (bring up self-report as another alternative...)
%\ryan{These are all good points that you are raising. An easy couple of citations would to include that get at some of the major pitfalls of self-report and observer-report methods would be:}

%\ryan{Paulhus, D. L., & Vazire, S. (2007). The self-report method. In Handbook of research methods in personality psychology (pp. 224–239). Guilford.}

%\ryan{Vazire, S. (2010). Who knows what about a person? The self–other knowledge asymmetry (SOKA) model. Journal of Personality and Social Psychology, 98(2), 281–300. https://doi.org/10.1037/a0017908}
%THANKS RYAN!

%The current methods fall short of offering a \textit{ground truth} of psychological processes that an individual undergoes, which is paramount to making inferences about how language correlates to such mechanisms. 

\section{Related Work}
%The field of Psychology %, however,
%relies heavily on carefully designed experimental paradigms to both elicit and capture cognitive state more objectively.
%Cognitive Styles, or sets of thought patterns related to specific cognitive phenomena, involve more than just linguistic expressions; they also involve a context-dependent change in mental state in response to social situations. 
%This highlights the need for a shift towards carefully designed construct-validated judgment and decision-making experiments designed to capture cognitive styles more accurately than both external annotators or the individuals themselves-- assessing styles by measuring a person's mental state before and after an experiment to observe changes.
While NLP for social science often relies on labels from annotators or questionnaires, behavioral science theory suggests carefully designed experiments can more objectively elicit and capture cognitive states.
% better than relying on judgments (of either annotators or self-reported in questionnaires). 
For instance, \citet{saxbe2013embodiment} investigated emotional responses using an experimental design in which participants' brain activity was imaged as they listened to narratives eliciting different emotions. These methods can offer a more objective foundation for understanding the psychological processes at play~\citep{brook2015linking}. In our study, we focus on cognitive styles in decision making -- reflecting one's tendency to maintain consistency and resolve dissonance~\citep{harmon2007cognitive,mcgrath2017dealing}. %, specifically processes such as constraint satisfaction and dissonance resolution
%This occurs very frequently --   when one has to make difficult decisions, such that their beliefs are challenged in some way
Since people show little awareness of their decision-making~\cite{nisbett1977halo}, 
%pyszczynski1993, 
 cognitive styles are measured experimentally by observing shifts in preferences after decision~\citep{simon2004construction,aguilar2022cognitive}. 
%We pick~\citet{simon2004construction} job-offer experiment to capture preference shifts.
%When making decisions, people seem to exhibit little to no awareness of the processes that lead to the decision. 
%The human tendency to be consistent in one's beliefs~\citep{tedeschi1971cognitive} is also a factor that leads to resolution of dissonance without awareness\cite{mcgrath2017dealing}. 
%This resolution can be 

%\citet{saxbe2013embodiment} studied emotions by eliciting language describing the emotions felt on watching or listening to emotional narratives, offering a more objective ground truth for underlying psychological processes that is most relevant for the produced language~\citep{brook2015linking}. \vase{more examples/constructs}

%\ryan{I would recommend including a concrete example of what this looks like from the literature. An easy example would be something like emotion elicitation paradigms where they induce some kind of affective state, then have people talk, e.g.,}

%\ryan{https://academic.oup.com/scan/article/8/7/806/1656575}

%\ryan{https://doi.org/10.1080/19312458.2014.999751}

%\ryan{Essentially, then you can say something like 'see, this is how you get a more objective ground truth for which psychological processes are most salient/operant at the time of language production'}


%TODO: define cognitive styles better






%The evoked Cognitive Styles, which are sets of thought patterns related to specific cognitive phenomena\footnote{in this paper, we specifically refer to decision-making \vase{maybe say this elsewhere when we talk specifically about decision making?}}, such as intented humor or offensiveness, cognitive dissonance, reasoning skills or empathy can be best measured by capturing the mental states of the subjects rather than crowd-sourcing the perceived qualities based on the author's text, which is prone to disagreements and errors \cite{sandri-etal-2023-dont, sap2021annotators, varadarajan-etal-2023}. 




%TODO: MAKE THIS POINT STAND OUT \ryan{An important distinction to raise here, for example, is that these approaches still fall somewhat short because we don't have a ground truth on what psychological processes are in operation. A person may utter two statements that are logically inconsistent, but we still need some semi-objective ground truth on what is actually happening in the person's head if we want to make sound inferences about how verbal behavior relates to or reflects such mechanisms.}

% ADD THE CITATIONS TO THE FIRST PARAGRAPH: Opinion, affect and stance shifts, cognitive dissonance and decision-making has heavily relied on annotations \cite{hovy2015sentiment, sakketou2022investigating, varadarajan2022disso, varadarajan-etal-2023}. Empathy has often relied on annotations as well \cite{lahnala-etal-2024-appraisal-framework}, sometimes including a self-reported empathic concern and distress score \cite{barriere-etal-2022-wassa}. 


% \ryan{My sense is that the following paragraphs are really the most important, and I would recommend getting to them as quickly as possible. Above, you're making a strong case that 1) we rely too heavily on annotator reports, which have established pitfalls, 2) other methods of ground truth such as self-report have pitfalls as well, especially when it comes to psychological phenomena that a person might lack insight into (e.g., by their very nature, people aren't aware that they are making cognitive distortions, etc.), and 3) that the field of psychology has long worked under a more classical scientific/empirical paradigm of conducting research to establish a more objective or well-defined ground truth that minimizes or eliminates subjective biases, particularly in the domain of psychological phenomena that cannot be directly observed. I think that you can probably condense that down to maybe 3 terse paragraphs, leaning heavily on citations to make your points, and with an illustrative example for point #3 (psychology uses experimental paradigms). This then sets you up to say 'so, what we propose here is the adoption of experimental cognitive science for NLP evaluation and inference tasks.'}



% DISCUSS THIS: \ryan{For the above paragraph, I would go even bigger. Sure, this stuff is valuable as a 'language-based assessment' type of task, but (I think) that there is more to it here that will strike a stronger chord with an NLP audience. What you're effectively doing here is showing a paradigm for discovering the particulars of *how* verbal behavior can reveal deeper cognitive processes. When the mind is doing X, it has Y effect on a person's language that is — critically — able to be observed empirically through statistical/computational modeling. This might be a bigger point for the discussion, but I'd imagine that people interested in the LLM space would be interested in thinking about applying this idea to bigger models — this is why people are doing so much comparative work in psychology right now between humans and ChatGPT — if we understand how 'heuristic processing' works in humans, we can get ground truth information from humans using an experimental paradigm, then see how GPT stacks up, etc. etc. Put another way: for the folks in pure/theoretical NLP who don't actually care that much about 'human factors' like empathy or social connectedness, I'd be willing to bet that they still have a deep, vested interest in understanding how the cognitive underpinnings of humans impact language use.}


%\vase{Longer intro: Problems with annotation, partially addressed with self report but there are still biases and JDM turn to experiments evoking CS and observation rather than SF and annot.}

%Although language has been studied in the NLP community as structured constructions of words or phrases \cite{naseem2021comprehensive} or their semantic roles \cite{li2022survey}, language is fundamentally a cognitive process, which is used to express explicit cognitive phenomena such as conveying emotion\cite{?}, argumentation\cite{?} and narrative storytelling\cite{?}. The multitude of implicit cognitive phenomena, such as cognitive dissonance and its resolution are much rarer to be expressed in language. 






%\section{Related Work}

%YET TO REFINE:

%Our study builds on work in NLP that validates author state measurements via annotations or self-report questionnaires. 
%For example, some have measured affect versus self-report for mental health using annotations of self-disclosures~\citep{zirikly-etal-2019-clpsych,valizadeh-etal-2021-identifying}.
%Some studies have attempted to capture specific patterns of cognitive styles using discourse frameworks~\citep{juhng2023discourse,sharma-etal-2023-cognitive}, causal and counterfactual thinking~\cite{son-etal-2017-recognizing,son-etal-2018-causal}, dating back to the Rhetorical Structure Theory for capturing rhetorical thinking~\citep{taboada2006rhetorical}.
%Still annotation and self-report can be biased by differences in perception -- such as in the case of dialog evaluations~\citep{liang2020beyond} or when capturing humor, empathy, or offensiveness~\citep{yang2021choral, paulhus2007self,buechel-etal-2018-modeling}.


%Building on this, 
Our study draws from previous NLP research that validates author state measurements through annotations or self-report questionnaires. For example, past work has compared affective states with self-reported mental health by analyzing self-disclosures~\citep{zirikly-etal-2019-clpsych,valizadeh-etal-2021-identifying}, while others have examined cognitive styles in the context of discourse~\citep{sharma-etal-2023-cognitive, juhng2023discourse, varadarajan2022disso,varadarajan-etal-2023}. However, annotations and self-reports are subject to perceptual biases, such as those observed in dialogue evaluations~\citep{liang2020beyond} or when assessing constructs such as humor, empathy, or offensiveness~\citep{yang2021choral, paulhus2007self,buechel-etal-2018-modeling, lahnala-etal-2024-appraisal-framework}. To address these limitations, we adopt an experimental approach that aims to objectively capture cognitive states, focusing on how individuals manage dissonance and consistency in their decision-making.

%Discourse structures are cognitively mediated, forming an essential interface between discourse and social understanding~\cite{van2014discourse}. The cognitive model theory underlying discourse provides the crucial link between cognitive structures and discourse patterns~\cite{van1990social}, offering a theoretical foundation for examining how different explanatory styles manifest in communication. 
Discourse structures provide a theoretically grounded link between cognitive processes and communication patterns, serving as a window into how individuals construct and convey explanations~\cite{van1990social, van2014discourse}.
Research in psychology has established strong connections between linguistic patterns and cognitive styles, particularly in how individuals process and communicate information~\cite{buchanan2013explanatory}. The analysis of discourse relations is especially valuable because they capture both explicit and implicit connections between text segments, revealing deeper patterns in explanatory styles such as reasoning~\cite{son-etal-2017-recognizing,son-etal-2018-causal} and rhetorical structures~\citep{taboada2006rhetorical} that may not be apparent from lexical-level features alone~\cite{juhng2023discourse, varadarajan-etal-2024-archetypes}. It serves as a powerful indicator of explanatory and rhetorical patterns in text, offering insights into how ideas are connected and presented~\cite{knaebel2023discourse}. In this work, we explore discourse features as well as state-of-the-art LLMs to model the outcomes of the cognitive experiment.
 %Such linguistic styles has been demonstrated to be related to cognitive styles in the field of psychology~\cite{buchanan2013explanatory}, which is the reason for exploring discourse features. 
 
%, self-report could suffer from noise in reporting due to the discrepancy between behavior and self-concept 
%However, self-report biases, much like annotator biases, can undermine the construct validity of the outcomes. Individual differences in perception could affect the magnitude of self-report values from person to person, rendering them difficult to compare -- such as in the case of dialog evaluations \cite{liang2020beyond}, and in case of constructs dependent on social cues such as humor, empathy or offensiveness, self-reported measures % of individual cognitive styles 
%could suffer from noise in reporting due to the discrepancy between behavior and self-concept \cite{paulhus2007self, baumeister2007psychology, vazire2010knows}. 
%important related work to be contrasted -- such work could benefit from  experimental evaluations. 


%\vase{include a paragraph about the work on cognitive distortions and how they are a certain kind of cognitive style that has more recently been explored in the context of studying mental health with language more specifically}
% Decision making involves complex processes like constraint satisfaction and cognitive dissonance
% %This occurs very frequently --   when one has to make difficult decisions, such that their beliefs are challenged in some way
% ~\cite{harmon2007cognitive}.
% People often show little awareness of their decision-making processes~\cite{pyszczynski1993, nisbett1977halo}, and the tendency to maintain consistency can lead to unconscious resolution of dissonance~\citep{mcgrath2017dealing}. 
% %When making decisions, people seem to exhibit little to no awareness of the processes that lead to the decision. 
% %The human tendency to be consistent in one's beliefs~\citep{tedeschi1971cognitive} is also a factor that leads to resolution of dissonance without awareness\cite{mcgrath2017dealing}. 
% This resolution can be measured with shifts in preferences before and after decision, decision-making time, and other behaviors~\citep{aguilar2022cognitive}. We pick~\citet{simon2004construction} job-offer experiment to capture preference shifts.


%\section{Experiment and Data}
\section{Experiment}


A total of 514 participants were recruited in person for the study; 12 were excluded due to incomplete or invalid responses, resulting in a final dataset of 502 participants. Data collection was performed in 2 stages (see Figure~\ref{fig:spirit}). The questionnaire has been described in detail in Appendix~\ref{sec:joboffer_desc}.
 
 \paragraph{Writing Task } Participants received 2 writing prompts to elicit language relevant to their decision-making cognitive style:
 %\begin{enumerate}[leftmargin=11pt]
 %\item 
 1) ``Please describe a recent important and difficult decision that you have made'' (20-100 words), and 
 %\item 
 2) ``What were the considerations that you thought about while making the decision? When answering, please consider all of the circumstances and details that went into the difficult decision'' (100-300 words).
% \end{enumerate}
These questions were chosen to elicit detailed descriptions of a recent decision-making process, encouraging participants to discuss options and explain their reasoning. %In the following paragraphs, we describe the semantic description of the corpus collected, which we 
The elicited essays to the two questions were concatenated for all further analysis.
We henceforth call the  collection of essays from the participants the \textit{Decisions} dataset.

 \paragraph{Constraint Satisfaction Experiment}


 \label{subsec:job_offer}
We replicated the experiment from \citet{simon2004construction}, modifying the preference score calculation to quantify overall preference changes rather than single attribute fluctuations as described below.
 \paragraph{1. Pre-Decision Preferences} 
Participants answered questions assessing preferences on a 6-point scale (-5 to 5, interval of 2) for four attributes: 
 \newcommand{\f}{\mkern-2mu f\mkern-3mu}
 Commute ($com$), Vacation ($vac$), Office space ($o\f\f$) and Salary ($sal$).  
 %Commute (C), Vacation (V), Office space (O) and Salary (S). 
 
% Each of the attributes had a positive ($com_{+}$) question and a negative ($com_{-}$) question each to establish the lower and upper bounds of their preferences. 
 Each attribute had positive (+) and negative (-) questions for preference bounds. 
 For example: 
 $com_{+}$ (commute): ``%The commute to work will take you about 18 minutes each way. - 
 Please select how desirable the 18 minute commute is to you.'' 
 $com_{-}$: ``%The commute to work will take you about 40 minutes each way. - 
 Please select how desirable the 40 minute commute is to you.'' 
 Participants rated each attribute's relative weight ($\texttt{W}$) on a 1-8 scale. 
 Final preference ($\rho_{\scriptsize\texttt{com}}$) for each attribute was calculated as:\\
%The participants were further asked to rate the relative weight ($\texttt{W}$) of each attribute on a scale of 1-8. The final preference for each attribute was calculated by weighing the difference between the stated preferences for positive and negative items by the importance:
%\begin{equation*}
\hspace*{2em} $\rho_{\scriptsize\texttt{com}} = (com^{+} - com^{-})\ \texttt{x}\ \texttt{W}_{com}$
\\Note that each $\texttt{$\rho$}$ is a value between -80 to +80.
%\end{equation*}
\paragraph{2. Job Offers}
Two choices were offered to the participants, such that in choosing either of the jobs, they would likely make compromises on at least two attributes. The two options were: \\
Company A: $com_{+}$, $vac_{+}$, $o{\f\f}_{-}$, $sal_{-}$; and\\ Company B: $com_{-}$, $vac_{-}$, $o{\f\f}_{+}$, $sal_{+}$ \\where $_{+},_{-}$ (in subscript) refer to the favorable and unfavorable conditions for each of the four attributes.
Therefore, the pre-decision preference score $\psi$ for company A and B can be calculated as:
%\begin{equation*}
 $   \texttt{$\psi$}_{\texttt{A}}  = + \texttt{$\rho$}_{com} + \texttt{$\rho$}_{vac} -\texttt{$\rho$}_{o\f\f} - \texttt{$\rho$}_{sal} $\\
%\end{equation*}
%\begin{equation*}
 $\texttt{$\psi$}_{\texttt{B}} = -\texttt{$\rho$}_{com} - \texttt{$\rho$}_{vac} +\texttt{$\rho$}_{o\f\f} + \texttt{$\rho$}_{sal} $\\
%\end{equation*}
%The participants pick one of the two choices, usually consistent with their initial preference scores. However, we randomly introduce a location ($loc$) that describes the job as being fun and located near a mall versus being located in a dull, dusty construction site, to further influence them to make a contrarian decision inconsistent with initial preferences.
Each $\texttt{$\psi$}$ thus has a value between -320 and +320. Participants choose between two options, typically aligning with their initial preferences. %However, we randomly introduce an influencing factor, location (loc) describing the job as either near a fun mall or in a dull, dusty construction site, potentially influencing contrarian decisions inconsistent with initial preferences.
We randomly introduce an influencing factor, location ($loc$), describing the job as either near a fun mall or in a dull construction site. This aims to induce dissonance, compelling participants to compromise and potentially make contrarian decisions, inconsistent with their initial preferences.
\paragraph{3. Post-Decision Preferences}
%After picking one job, the participants are asked to answer the same questions as the pre-decision questionnaire again. 
After selecting a job, participants answer the same questions from the pre-decision questionnaire again.

\paragraph{Decision-Making Outcomes}
%and Time to Decide aka (TD). 
We define each construct and describe their measurement from the experiment below:

%\subsection{Cognitive Style Outcomes}
\label{subsec:dm_outcomes}
\paragraph{1. Choice-Induced Shift (CIS)} The change in preference is captured by subtracting the pre-experiment scores from post-experiment scores.
%If the job choice is A:\\
%\begin{equation*}
$CIS = \texttt{$\psi$}^{post}_\texttt{A} - \texttt{$\psi$}^{pre}_\texttt{A}; \texttt{choice = A}$\\
%\end{equation*}
%\begin{equation*}\\
$CIS = \texttt{$\psi$}^{post}_\texttt{B} - \texttt{$\psi$}^{pre}_\texttt{B};  \texttt{choice = B}$\\
%\end{equation*}
Here, we model binarized CIS which captures the direction of the preference change towards the job choice.

\paragraph{2. Influenced or Not (Inf)} 
The job offer is further influenced by introducing a confounding attribute $loc$ (location). Many participants choose the job influenced by the description of the location attribute, however not all of them change their minds. The change is a cognitive signal that measures if someone's choice was influenced by confounding attribute. This indicates that their initial preferences were not strong enough to begin with. This is captured as a binary variable: making the choice in the direction of the influenced variable or not.
%\andy{define CIS\_Inf (preferably as a function).}
%\vase{should we include histograms and other stats for each outcome? could include a small figure}


\begin{figure}[ht]
 \centering
 \includegraphics[width=\linewidth]{diagrams/validation_topic_clouds.pdf}
  \caption{Randomly selected topics emerging from LDA on the participant writing describing a recent decision, depicting the types of content evoked. 
 %\andy{expand to 10 or 12 topics and make a two column figure. Can we add any meta information about the topics?}
 }
  \label{fig:topics}
\end{figure}
\paragraph{Description of the \textit{Decisions} Dataset}
\label{sec:topics}
%In order to describe the main themes of the corpus , 
We employed topic modeling to describe the main themes of the \textit{Decisions} dataset while preserving privacy of our study participants. Figure~\ref{fig:topics} shows that respondents writing about their difficult decisions frequently mention topics related to college education, career goals, finances, mental health, friendships, family relationships, and vacation plans. These subjects are largely connected to common decision-making aspects of student life. The average length of essays is 186.28 words (min: 120, max: 508 words). The average Choice-Induced Shift (CIS) is 25.6  ($\sigma$: 38.4, min: -102.4, max: 140.8) 
-- this is consistent with the \citet{simon2004construction} paper that shows 
that more people tend to change their preferences towards the decision they make, i.e. CIS skews positive. Finally, out of the 502 participants, 417 (83\%) were influenced in the direction of the confounding attribute ($loc$) whereas 85 (17\%) remained uninfluenced.




%\andy{Move from here forward in section 3 to end of methods instead; it's more methods than the experiment or data}

 \section{Methods}


We explored the following theoretically relevant \textbf{Discourse Relations}:%\vase{Discourse relations are meant to capture the explanatory and rhetorical style in the text [4]. Such patterns in language has been demonstrated to be related to cognitive styles in the field of psychology [5], which is the reason for exploring discourse features. }

% \paragraph{Theoretically Relevant Lexical Measures} We hypothesize that personality and latent human traits can be key to understanding the cognitive style in decision-making of the individual. To this end, we use the Big Five Personality Lexica (OCEAN) from  \citet{park2015automatic} and the Behavioral Linguistic Traits (BehavLingTraits) from \citet{kulkarni2018latent} to measure the latent traits in the writings. Other than that, we also use lexical anxiety~\cite{guntuku2019twitter}, stress~\cite{guntuku2019understanding}, loneliness~\cite{guntuku2019studying} and empathic concern~\cite{giorgi2023human}, since decision making is heavily affected by these factors -- feeling stressed or anxious, or having concern for how others are affected by the decision can have an impact on the decision-making cognitive style~\cite{??}


% \paragraph{Discourse Relations} 
% We included discourse relations that are theoretically relevant to decision-making in our analysis.
 %\vase{start with dissonance and conso -- don't start with counterfactuals}

\paragraph{1. Causal explanations} %We extract the reasoning that an individual provides for their behaviors in the decision making. Using a causal explanation model trained on social media posts \cite{son2018causal}, we infer the proportion of the messages that contain causal explanation, written by the individual. 
We extract individual reasoning behind decision-making behaviors using a causal explanation detection model trained on social media posts with a F1-macro of 0.85~\cite{son2018causal}. We infer the proportion of messages containing causal explanations provided by the individual.
\paragraph{2. Counterfactuals} These are statements of alternate reality; of what could have happened instead of actual events. We used the counterfactual relation recognition model based on a social media dataset with an F1-macro of 0.77 \cite{son-etal-2017-recognizing} to calculate the proportion of the messages from each individual that contains counterfactual statements.

\paragraph{3. Dissonance and Consonance} %Since decision-making is directly related to dissonance~\cite{}, we extract dissonance and consonance using a dissonance detection model trained on social media posts , we compute the the average probability of dissonance for each message among pairs of consecutive phrases predicted as dissonance (or consonance).
We extracted linguistic dissonance and consonance using a model trained on social media posts (AUC = 0.75) introduced in~\citet{varadarajan-etal-2023}, which captures signals of cognitive dissonance exhibited through language. We then calculated the average probability of dissonance for consecutive phrases predicted as dissonant or consonant.
% over all the pairs of consecutive phrases to compute the scores.

\paragraph{4. Discourse Relation Embeddings} To capture \textit{other} discourse-level information, we use discourse relation embeddings that is extracted from pairs of consecutive discourse arguments~\cite{son-etal-2022-discourse}, aggregated by averaging at a message level.
%\vase{ motivate discourse features--  write a paragraph about each relation and the model information -- go deeper and aim for a longer paper }



 %\subsection{Latent Contextual Representations}
 %We also extract two latent embedding models: (a) RoBERTA-base reduced to 16 dimensions since we operate in a low-resource regime \cite{ganesan2021empirical}; and (b) a discourse relation embedding model \cite{son-etal-2022-discourse} to embed the relationships between phrases to encode the explanatory style rather than use explicit discourse-level measures.
 
Further, we explored common baseline models to capture the decision-making cognitive styles: a random baseline, zero- and four-shot prompting on both Llama3.1-8B-chat and Gemma-7B-Instruct%\vase{add a description of what was prompted --  what were the scores and how was it supposed to be inferred}
\footnote{The LLMs were prompted with the definitions of CIS and Inf variables. For the prompts, please check \S\ref{tab:prompts}.}, and finally, a predictive model from averaged embeddings of the text from L23 of RoBERTa-large.  

\paragraph{Predictive Models for Decision Making}
% (as shown in Fig \ref{fig:CISInf}).  
%\begin{figure}[!ht]
% \centering
% \includegraphics[width=0.5\columnwidth]{diagrams/validation_inf_diagram_1.pdf}
% \caption{A breakdown of participants with various categories of CIS\_Inf decision outcome \ref{subsec:dm_outcomes}. This forms the space of outcomes from the experiment against which discourse models are evaluated.  }
% \label{fig:CISInf}
%\end{figure}
We model 2 outcomes together:  Choice-Induced Shift (CIS) and Influence (Inf). \textbf{CIS} and \textbf{Inf} variables capture the magnitude and direction of the tendency of a person to vacillate when exposed to conflict-inducing information. We combine them into a single variable \textbf{CIS\_Inf} for modeling four distinct cognitive styles for decision making: (a) Negative CIS,
Not Influenced (↓CIS↓Inf, 6\%), (b)
Negative CIS,
Influenced (↓CIS↑Inf, 17\%), (c) Positive CIS,
Not Influenced (↑CIS↓Inf, 11\%) and (d) Positive CIS,
Influenced (↑CIS↑Inf, 66\%).

We use a logistic regression model for 4-way classification with the features listed in Table~\ref{tab:CIS_modeling_results}, where we calculate stratified 5-fold cross-validation accuracies
%\vase{AUC not accuracy - need to change this to be more specific} 
using DLATK~\citep{schwartz2017dlatk}. %\vase{expand this a bit}
%\paragraph{Time to Decision (TD)} 
%The time taken to make the job offer decision was measured for 120 participants and is a meaningful cognitive signal that measures the time duration to make decision, which has been previously shown to be related to cognitive decision making styles~\cite{}.




%remove terms that are in the LIWC2022 func lexicon and then use the resulting 1gram table as the feature table. 


 
%\begin{table}[tb]
%\small
% \centering
 
 %\begin{tabular}{lccc}
 %\toprule
 %\textbf{Discourse measure} &\textbf{ CIS} & \textbf{Inf} & \textbf{TD} \\
 %\midrule

 %Causal explanations &0.01 &-0.01 & -0.09\\
 %Counterfactual &0.00 &0.02 &0.02 \\
 %Consonance &0.03 & 0.05 & 0.06\\
 %Dissonance &0.00 & -0.04& -0.04\\
 %\bottomrule
 %\end{tabular}
 %\caption{\vase{include other relations} Pearson r of discourse-level features measures in language. \textbf{Bold} indicates significant correlations with $\dagger$: $p<.05$; $\ddagger$:$p<.001$. None of the theorized discourse-level features are correlated with the outcomes, which is surprising.}
 %\label{tab:discourse}
%\end{table}

 

 

%Table 1 shows that language can indeed capture signals of CIS\_Inf. CIS\_Inf captures two different variables (Fig 3) -- how much a person's preference shifts before and after the experiment and if they were manipulated in making the decision.
%With AUC $\sim$0.8 for most linguistic features, language shows promise in capturing cognitive styles of individuals. 
%We find that dimension reduced Roberta and BehavLingTraits are particularly good at distinguishing the four classes of CIS\_Inf variable.
%This is further supported by the results in Table~\ref{tab: correlations}, where we find that the four classes are very distinguishable along lexical-based measures, and especially so with the five factors of BehavLingTraits~\cite{kulkarni2018latent}, the highest of which (BLT3) has a correlation of 0.35 with the class ↑CIS↓Inf.
%Discourse relations, both in the form of explicit inferred relations as well as learned representations seem to have very low predictive power towards the cognitive styles of individuals pertaining to actual decision-making behavior. 
%This indicates that decision-making cognitive styles can be gleaned from language.




%\andy{I see how it would have been ideal if discourse features worked better than lexical but this sounds like an overclaim given discre was not significantly worse than the best RoBerta features}
%While discourse relations were originally intended to capture cognitive states through coherence and rhetorical structures, our annotative method for inferring these relations reveals limited correlations. This suggests that regular contextual embeddings might be more adept at learning cognitive styles than models based on discourse relations. 

%These findings underscore the potential of harnessing language to enhance our understanding and modeling of human cognition, thereby improving engagement and utility in NLP applications. The four classes of CIS\_Inf are distinctly correlated with specific traits and linguistic discourse styles, highlighting distinct cognitive styles.

%\andy{restate motivation: We first examine results for our primary application of the \textit{experimental validation framework}: do discourse relation models, intended to capture cognitive style, predict the cognitive style of a decision one makes? }
\begin{table}[!ht]

\small
\begin{tabular}{l|r||l|lr}
\toprule
  
 %\multicolumn{2}{c}{\textbf{CIS\_highlow05 (lr)}} & 
% \multicolumn{2}{c}{\textbf{CIS\_Inf (lr)}}  & \\

\multicolumn{1}{l}{\textbf{Baselines}} &
%\multicolumn{1}{l}{\textbf{AUC}} & \multicolumn{1}{c}{\textbf{F1 wei}} &
\textbf{AUC}% & \multicolumn{1}{l}{\textbf{F1}} 
%& \textbf{k}
&\multicolumn{1}{l}{\textbf{Discourse feats}}  &
%\multicolumn{1}{l}{\textbf{AUC}} & \multicolumn{1}{c}{\textbf{F1 wei}} &
\multicolumn{1}{c}{\textbf{AUC}} % & \multicolumn{1}{l}{\textbf{F1}} 
& \textbf{k}\\
\midrule
 

Random & 0.50 &Causal %& 0.337 & 0.451 
& \textbf{0.81} 
%& 0.527 %& \textbf{0.199} 
 & 1\\%&0.526\\
%&  -  \\
%Llama2-8B chat (0-shot) & 0.577 & -\\ 
Llama3.1 (0-sh)  & 0.56 &Counterfactual %& 0.337 & 0.451 
& 0.80 
%& 0.661 %& \textbf{0.199} 
 & 1\\%&- \\
Gemma (0-sh) &0.56 &Consonance  %& 0.337 & 0.451 
& \textbf{0.81}
%& 0.661 %& \textbf{0.199} 
 & 1\\%& -\\
%Llama2-8B chat (4-shot) & 0.603 & -\\ 
Llama3.1 (4-sh)  &0.64&Dissonance  %& 0.337 & 0.451 
& 0.80
%& 0.661 %& \textbf{0.199} 
 & 1\\% &- \\
Gemma (4-sh) & \textbf{0.79}&\hspace{0mm}DiscRE (full) %& 0.423 & 0.451 
& 0.76 
%& 0.560 %& \textbf{0.311} 
 &845\\%& -\\
%Roberta-base (L11) & 0.693& 768\\
RoBERTa-L23 & 0.69&\hspace{0mm}DiscRE (16-D) %& 0.423 & 0.451 
& 0.79
%& 0.657 %& \textbf{0.198} 
 &16\\%& 1024\\

\bottomrule
%Emotions & 0.545 & 0.451 & \textbf{0.811} & 0.527 %& 0.199
%& \textbf{9.00} & \multicolumn{1}{r}{0.056} \\
%Anxiety & 0.438 & 0.451 & 0.808 & 0.526 %& 0.199 
%& 9.04 & - \\
%Stress & 0.449 & 0.451 & 0.800 & 0.527 %& 0.199 
%& 9.02 & - \\
%Demographics & 0.384 & 0.451 & 0.805 & 0.527% & 0.199
%& 9.05 & - \\
%Loneliness & 0.566 & 0.451 & 0.805 & 0.527 %& 0.199 
%& \textbf{9.00} & - \\
%Empathy & 0.437 & 0.451 & 0.802 & 0.527% & 0.199 
%& 9.02 & - \\

%1to2grams & \textbf{0.541} & \textbf{0.527} & 0.725 & 0.500 & \textbf{0.239} & 9.03 & \multicolumn{1}{r}{0.024} \\
%Lexica-based measures\\
%\hspace{2mm}LIWC22 %& 0.488 & 0.411 
%& 0.760 & 0.517 %& 0.206 
% &16\\
%\hspace{2mm}FB topics %& 0.469 & 0.496 
%& 0.641 & 0.443 %& \textbf{0.217} 
%&2000\\
%\hspace{2mm}BehavLingTraits
%\cite{kulkarni2018latent} %& 0.428 & 0.451 
%& \textbf{0.808} & 0.526 %& 0.199 
 %& 5 \\
%\hspace{2mm}OCEAN %& 0.442 & 0.451 
%& 0.794 & 0.527 %& 0.199 
%&  5 \\ \midrule
%Discourse relations \\
%  \hspace{2mm}Causal %& 0.337 & 0.451 
%& \textbf{0.809} 
%& 0.527 %& \textbf{0.199} 
% & 1\\
% \hspace{2mm}Counterfactual %& 0.337 & 0.451 
%& 0.800 
%& 0.661 %& \textbf{0.199} 
% & 1\\
% \hspace{2mm}Consonance  %& 0.337 & 0.451 
%& \textbf{0.811}
%& 0.661 %& \textbf{0.199} 
% & 1\\
% \hspace{2mm}Dissonance  %& 0.337 & 0.451 
%& 0.804
%& 0.661 %& \textbf{0.199} 
% & 1\\

%RoBERTa-b (16) & \textbf{0.613} & \textbf{0.462} & 0.807 & \textbf{0.529} & \textbf{0.204} & 9.02 & - \\ 

%\hspace{2mm}PDTB %& 0.337 & 0.451 
%& - & -%& \textbf{0.199} 
% & -\\

%\midrule
%Discourse Relation Representations \\
%\hspace{0mm}DiscRE (full) %& 0.423 & 0.451 
%& 0.758 
%& 0.560 %& \textbf{0.311} 
% &845\\ 
% \hspace{2mm}RoBERTa-b (full) %& \textbf{0.613} & \textbf{0.462} 
%& 0.717 & 0.521 %& \textbf{0.204} 
% &768\\ 
%\hspace{0mm}DiscRE (16-dim) %& 0.423 & 0.451 
%& 0.792 
%& 0.657 %& \textbf{0.198} 
 %&16\\
%\hspace{2mm}RoBERTa-b (16-dim) %& \textbf{0.613} & \textbf{0.462} 
%& \textbf{0.807} & \textbf{0.529} %& \textbf{0.204} 
 %&16\\ 
 
%\bottomrule
\end{tabular}
\caption{Performance of various feature sets  over the CIS\_Inf outcome (\textbf{AUC}: mean Area Under the ROC Curve; \textbf{k}: number of input features). %Discourse-level representations of participant's pre-experiment were able to predict experimental outcomes with moderate accuracy, nearly matching the accuracy of low dimensional representation from Roberta which outperformed full Roberta. 
Linguistic measures from the participants' pre-experiment writing can predict CIS\_Inf with moderate-high, non-trivial accuracy.  %\vase{first and foremost: discourse relations can capture CogStyles --don't do this in bake-off style}
% Andy: the null hypothesis should be that lexical are stronger -- that is where most of our representational development in NLP has been, so we can almost think of  think one will expect lexical to work well "some lexical outcomes have a better predictive power over discourse.  %\andy{could move TD results *below* ridgecv.}
}
\label{tab:CIS_modeling_results}
\end{table}
%\vspace{-5mm}
\section{Results}
%\vspace{-3mm}
We explore results for our primary application of the \textit{experimental validation framework}: do discourse relation models, which capture explanatory styles and coherence in language of individuals, predict the cognitive style of a decision that an individual makes?
Table \ref{tab:CIS_modeling_results} shows that cognitive styles, represented by CIS\_Inf, have predictive correlates in language. CIS\_Inf captures two different variables (Fig 3) -- how much a person's preference shifts before and after the experiment and whether they were influenced in making the decision.
While discourse relation embeddings themselves seem to have low predictive power, specific relevant relations such as \textbf{Causal} and \textbf{Consonance} have high predictive power towards the cognitive styles of individuals pertaining to actual decision-making. 
With discourse relation features achieving an AUC of $\sim$0.8, language shows promise in capturing cognitive styles of individuals that are exhibited through their behavior. While few-shot prompting achieves comparable performance to discourse features, the latter's success is particularly noteworthy given their significantly lower parameter count compared to large language models. The effectiveness of these interpretable discourse features reinforces our finding that linguistic patterns reflect underlying cognitive styles.


%\andy{Move to later: We also find that dimension-reduced RoBERTa and BehavLingTraits are good at distinguishing the four classes of CIS\_Inf variable.}


%\andy{Would expand writing about table 1 to 2 paragraphs: first covering motivation and factually what we saw. The second discussing what this means. with one paragraph covering to make sure to cover interesting points from all tables and figures}


%\andy{transition sentence needed: before discussing, why are we now looking at lexical features?}

\begin{table}[!ht]
\small
 \centering
 
 \begin{tabular}{p{0.4\linewidth}|p{0.09\linewidth}p{0.09\linewidth}p{0.09\linewidth}p{0.09\linewidth}}
 
 \toprule
 % & \multicolumn{4}{|c}{CIS\_Inf Category} & 
\textbf{Theoretical Features} &\scriptsize{\textbf{$\downarrow$CIS$\downarrow$Inf}} & \scriptsize{\textbf{$\downarrow$CIS$\uparrow$Inf}} & \scriptsize
{\textbf{$\uparrow$CIS$\downarrow$Inf}}& \scriptsize{\textbf{$\uparrow$CIS$\uparrow$Inf}} \\

 
% \toprule
% \textbf{Theoretical Features} &\textbf{ CIS} & \textbf{Inf} & \textbf{TD} \\
 \midrule
OCEAN%~\citep{park2015automatic}  
& \multicolumn{1}{l}{}          & \multicolumn{1}{l}{}          & \multicolumn{1}{l}{}          & \multicolumn{1}{l}{}          \\
 \hspace{5pt}Openness        & \cellcolor[HTML]{F6D2CF}-0.14 & \cellcolor[HTML]{B4E1CB}0.15  & \cellcolor[HTML]{F4C5C1}-0.18 & \cellcolor[HTML]{EAF7F1}0.03  \\
  \hspace{5pt}Conscientiousness & \cellcolor[HTML]{F6D2CF}-0.14 & \cellcolor[HTML]{F3C3BF}-0.18 & \cellcolor[HTML]{9ED8BB}0.20  & \cellcolor[HTML]{E1F3EA}0.05  \\
   \hspace{5pt}Extraversion      & \cellcolor[HTML]{FEFBFB}-0.03 & \cellcolor[HTML]{FFFFFF}-0.02 & \cellcolor[HTML]{E7F6EE}0.03  & \cellcolor[HTML]{F9FDFB}-0.01 \\
    \hspace{5pt}Agreeableness     & \cellcolor[HTML]{FCF3F3}-0.05 & \cellcolor[HTML]{B7E2CD}0.14  & \cellcolor[HTML]{E0F3EA}0.05  & \cellcolor[HTML]{FAE5E4}-0.09 \\
 \hspace{5pt}Emotional Stability         & \multicolumn{1}{c}{\cellcolor[HTML]{F6FCF9}{\color[HTML]{000000}0.00} } & \cellcolor[HTML]{F6FCF9}0.00  & \cellcolor[HTML]{FCF3F3}-0.05 & \cellcolor[HTML]{EFF9F4}0.02  \\

\midrule
%BehavLingTraits  & \multicolumn{1}{l}{}          & \multicolumn{1}{l}{}          & \multicolumn{1}{l}{}          & \multicolumn{1}{l}{}          \\
% \hspace{5pt}BLT1                & \cellcolor[HTML]{C1E6D4}0.12  & \cellcolor[HTML]{C1E6D4}0.12  & \cellcolor[HTML]{C9E9D9}0.10  & \cellcolor[HTML]{F6D0CC}-0.15 \\
% \hspace{5pt}BLT2                & \cellcolor[HTML]{63C093}0.33  & \cellcolor[HTML]{FCF3F3}-0.05 & \cellcolor[HTML]{FEFAFA}-0.03 & \cellcolor[HTML]{FEFBFB}-0.03 \\
% \hspace{5pt}BLT3              & \cellcolor[HTML]{ABDDC5}0.17  & \cellcolor[HTML]{FEFCFC}-0.03 & \cellcolor[HTML]{57BB8A}0.35  & \cellcolor[HTML]{F3C0BC}-0.19 \\
% \hspace{5pt}BLT4               & \cellcolor[HTML]{FAE5E4}-0.09 & \cellcolor[HTML]{95D4B5}0.22  & \cellcolor[HTML]{CBEADB}0.10  & \cellcolor[HTML]{F6D0CC}-0.15 \\
% \hspace{5pt}BLT5                & \cellcolor[HTML]{FDF5F4}-0.05 & \cellcolor[HTML]{FDF4F3}-0.05 & \cellcolor[HTML]{FEFCFC}-0.03 & \cellcolor[HTML]{DEF2E8}0.05  \\
% \midrule
%Emotion           & \multicolumn{1}{l}{}          & \multicolumn{1}{l}{}          & \multicolumn{1}{l}{}          & \multicolumn{1}{l}{}          \\
 %\hspace{5pt}Joy               & \cellcolor[HTML]{FBECEB}-0.07 & \cellcolor[HTML]{5CBD8D}0.34  & \cellcolor[HTML]{E67C73}-0.37 & \cellcolor[HTML]{FBEEED}-0.07 \\
 %\hspace{5pt}Surprise          & \cellcolor[HTML]{FFFFFF}-0.02 & \cellcolor[HTML]{6CC499}0.31  & \cellcolor[HTML]{D3EEE1}0.08  & \cellcolor[HTML]{F1B6B1}-0.22 \\
 %\hspace{5pt}Disgust           & \cellcolor[HTML]{FCF3F3}-0.05 & \cellcolor[HTML]{B9E3CF}0.14  & \cellcolor[HTML]{F0B5B0}-0.22 & \cellcolor[HTML]{EBF7F1}0.03  \\
 %\hspace{5pt}Trust             & \cellcolor[HTML]{A5DBC0}0.18  & \cellcolor[HTML]{E3F4EB}0.04  & \cellcolor[HTML]{F8DAD7}-0.12 & \cellcolor[HTML]{FFFFFF}-0.02 \\
 %\hspace{5pt}Sadness           & \cellcolor[HTML]{F5CCC9}-0.16 & \cellcolor[HTML]{FEFAFA}-0.03 & \cellcolor[HTML]{D5EEE2}0.07  & \cellcolor[HTML]{EDF8F3}0.02  \\
 %\hspace{5pt}Anger             & \cellcolor[HTML]{FAE5E3}-0.09 & \cellcolor[HTML]{EEF9F4}0.02  & \cellcolor[HTML]{EC9F98}-0.28 & \cellcolor[HTML]{BBE4D0}0.13  \\
 %\hspace{5pt}Fear              & \cellcolor[HTML]{FAE5E3}-0.09 & \cellcolor[HTML]{FBECEB}-0.07 & \cellcolor[HTML]{FAE8E7}-0.08 & \cellcolor[HTML]{C8E9D9}0.10  \\
 %\hspace{5pt}Anticipation      & \cellcolor[HTML]{FCF3F3}-0.05 & \cellcolor[HTML]{FDF5F4}-0.05 & \cellcolor[HTML]{FEFDFD}-0.03 & \cellcolor[HTML]{DEF2E8}0.05  \\
Anxiety%\scriptsize{~\citep{guntuku2019twitter}}         
& \cellcolor[HTML]{D5EEE2}0.07  & \cellcolor[HTML]{F9E1DE}-0.10 & \cellcolor[HTML]{88CFAC}0.24  & \cellcolor[HTML]{FCF0EF}-0.06 \\
Stress            & \cellcolor[HTML]{F4C9C5}-0.17 & \cellcolor[HTML]{DAF0E5}0.06  & \cellcolor[HTML]{D3EDE0}0.08  & \cellcolor[HTML]{FDF9F9}-0.04 \\
 Loneliness        & \cellcolor[HTML]{F3C1BD}-0.19 & \cellcolor[HTML]{F4C7C3}-0.17 & \cellcolor[HTML]{D4EEE1}0.08  & \cellcolor[HTML]{BFE5D3}0.12  \\
Empathic Concern     & \cellcolor[HTML]{AEDFC7}0.16  & \cellcolor[HTML]{C9E9D9}0.10  & \cellcolor[HTML]{F5CECB}-0.15 & \cellcolor[HTML]{FEFBFB}-0.03 \\
\midrule
Discourse  relations       & \multicolumn{1}{l}{}          & \multicolumn{1}{l}{}          & \multicolumn{1}{l}{}          & \multicolumn{1}{l}{}          \\
 \hspace{5pt}Causal            & \cellcolor[HTML]{73C79E}0.29  & \cellcolor[HTML]{F7D6D3}-0.13 & \cellcolor[HTML]{F5CECB}-0.15 & \cellcolor[HTML]{DBF1E6}0.06  \\
 \hspace{5pt}Counterfactual    & \cellcolor[HTML]{F8DCDA}-0.11 & \cellcolor[HTML]{F1F9F5}0.01  & \cellcolor[HTML]{BAE3CF}0.13  & \cellcolor[HTML]{FDF7F7}-0.04 \\
 \hspace{5pt}Consonance        & \cellcolor[HTML]{E3F4EC}0.04  & \cellcolor[HTML]{F6D0CC}-0.15 & \cellcolor[HTML]{F0B4AF}-0.22 & \cellcolor[HTML]{A4DBC0}0.18  \\
 \hspace{5pt}Dissonance        & \cellcolor[HTML]{A5DBC1}0.18  & \cellcolor[HTML]{F5CCC8}-0.16 & \cellcolor[HTML]{E7F5EE}0.04  & \cellcolor[HTML]{E8F6EF}0.03 \\
% Big 5 Traits &&& \\
% \hspace{5pt}Emotional Stability & 0.00& 0.01&-0.06 \\

 
% \hspace{5pt}Extraversion &-0.03 &0.00 & 0.03\\
% \hspace{5pt}Conscientiousness & -0.14& -0.03&0.03 \\ 
% \hspace{5pt}Agreeableness &-0.05 & 0.00&-0.02 \\
% \hspace{5pt}Openness &-0.14 &0.06 &\textbf{0.14}$^\dagger$ \\
% Age & .07* & -.03 & -.04\\
% Gender & -.09** & .03 & .06*\\
% BehavLingTraits&&& \\
% \hspace{5pt}F0 &-0.07 & -0.04&0.01 \\
% \hspace{5pt}F1 &0.01 & -0.03& 0.02\\
% \hspace{5pt}F2 &-0.02 & \textbf{-0.12}$^\dagger$& 0.03\\ 
% \hspace{5pt}F3 & -0.04&-0.01 & 0.04\\
% \hspace{5pt}F4 &0.09 & 0.01& 0.01\\ 
% Emotion &&& \\
% \hspace{5pt}Joy& -0.10& 0.09&-0.12 \\
% \hspace{5pt}Surprise & -0.05&-0.02 & 0.06\\
% \hspace{5pt}Disgust & -0.03& 0.06& -0.02\\ 
% \hspace{5pt}Trust & -0.01& 0.00& 0.02\\
% \hspace{5pt} Sadness& 0.00&0.00 &-0.04 \\ 
% \hspace{5pt} Anger& 0.02&0.09 & 0.00\\
% \hspace{5pt} Fear& 0.05& 0.03& 0.02\\
% \hspace{5pt} Anticipation& 0.07 &0.01&0.06 \\
% Anxiety &-0.01 &-0.07 & 0.00\\
% Stress &-0.03 &0.00 &-0.04 \\
% Loneliness &\textbf{0.10}$^\dagger$ & 0.01 & 0.05\\
% Empathic concern &\textbf{-0.11}$^\dagger$ & 0.02& -0.06\\
% Discourse relations\\
% \hspace{5pt}  Causal explanations &0.01 &-0.01 & -0.09\\
% \hspace{5pt} Counterfactual &0.00 &0.02 &0.02 \\
% \hspace{5pt} Consonance &0.03 & 0.05 & 0.06\\
% \hspace{5pt} Dissonance &0.00 & -0.04& -0.04\\
 \bottomrule
 \end{tabular}
 \caption{Cohen's \textit{d} for theoretical features against cognitive style outcomes of CIS\_Inf. %\vase{for TD as well in main paper}
 %\textbf{Bold} indicates significant correlations with $\dagger$: $p<.05$; $\ddagger$:$p<.001$. 
 %\andy{change to cohen's d on high (tercile) versus low (tercile) of each attribute, leaving out middle tercile. make the columns just CIS\_Inf and TD. Move to supplement.}
 }

 \label{tab: correlations}
\end{table}

To explore language-specific patterns that relate to each type of cognitive style, we also extracted %\vase{correlations? Cohen's d} 
for theoretically-relevant lexical and discourse relation features in predicting each class of CIS\_Inf. Results are presented in Table~\ref{tab: correlations}, where we find that the four classes are highly differentiable along lexical-based measures for personality~\citep{park2015automatic}, anxiety~\citep{mangalik2024robust}, stress~\citep{guntuku2019understanding}, loneliness~\citep{guntuku2019studying} and empathic concern~\citep{giorgi2023human}. Discourse relations, especially the Causal relation has a Cohen's \textit{d} of 0.29 with the class ↓CIS↓Inf. % and the five factors of BehavLingTraits~\citep{kulkarni2018latent}, the highest of which (BLT3) has a correlation of 0.35 with the class ↑CIS↓Inf. 
We find that individuals who use more causal explanations and dissonant statements in their description of a recent past decision are less likely to change their minds about a decision due to external influence, and are less likely to change their preferences after making a decision, whereas, individuals who use less consonant statements in describing their decisions are more likely to switch their preferences after making a decision in the experiment. 

Interestingly, higher linguistic dissonance is associated with less change in preferences / tendency to be influenced, which may signal difficulty in resolving dissonance surrounding one's decision.
Higher change in preferences with low tendency to be influenced also seems to be signaled by linguistic anxiety, and each of the cognitive styles have a distinct signature across personality and well-being dimensions. 
This indicates that individual decision-making cognitive styles derived from simulated real-life experiments can be gleaned from personal discourse and the explanatory style of the person.%, pointing to the efficacy of discourse relation modeling to understand cognitive styles.


%\andy{Would expand to make sure to cover interesting points from all tables and figures}

\paragraph{Recommendations:}
%\vase{As an initial step towards this evaluation framework, we make some recommendations for collecting data while shifting from traditional annotation-based approaches to incorporate direct behavioral measurements through controlled experiments. 
%The two-stage data collection process involves initially collecting natural language data from participants, then conducting behavioral experiments to assess cognitive or psychological constructs. 
%The experimental framework should ideally utilize established psychological paradigms, and maintain ecological validity by collecting language data \textit{before} the psychological experiment to avoid any influence of the experiment on the participant. 
%Researchers should implement multiple measurement approaches, including behavioral metrics like questionnaire completion time and click-through rates, alongside tracking changes in participant responses during the experiment. By moving beyond simple annotations to capture actual cognitive states and behavioral outcomes, this methodology provides more reliable data and enables direct validation of the relationship between language use and psychological constructs.}

As an initial step in developing this evaluation framework, we recommend incorporating direct behavioral measurements into linguistic analyses, moving beyond traditional annotation-based methods. While annotations provide useful approximations of cognitive states, they rely on external judgments rather than direct psychological evidence. In contrast, experimental paradigms—widely used in psychology—allow researchers to systematically measure cognition and behavior under controlled conditions, offering a more reliable way to validate language-based models. To ensure ecological validity, language data should be collected before the experiment to prevent unintended influence on participants' responses.
To capture a fuller picture of cognitive processes, researchers should combine linguistic features with behavioral metrics such as response times (e.g., questionnaire completion speed), click-through rates, and dynamic shifts in participant responses. This multimodal approach provides stronger evidence for the relationship between language and cognition, allowing NLP models to be evaluated against real psychological processes rather than relying solely on subjective annotations. By integrating experimental methods, this framework strengthens the scientific grounding of language-based models and enhances their validity for applications in cognitive science, decision-making research, and human-computer interaction.

\section{Conclusion}

We demonstrated that experimentally-evoked cognitive styles can indeed be captured by language, offering a more solid ``ground truth'' compared to annotations of perceived behavior, which often fail to reflect a person's true state.
%This experimental paradigm marks a significant step towards validating language-based cognitive models by directly pairing linguistic analysis with experimentally induced psychological states. 
%Unlike traditional NLP approaches that rely on potentially unreliable annotated or self-reported data, 
This framework emphasizes methodological rigor through controlled psychological experiments, enabling researchers to establish robust connections between language patterns and realistic estimates of cognitive states. 
Our framework's effectiveness is demonstrated with language-based features having strong predictive power for objective cognitive styles, especially discourse features successfully capturing experimentally measured cognitive styles. 
This approach not only enhances statistical validity but also has practical applications in the use of LLMs for mental health therapy, agent engagement systems, and cognitive science. 
By moving beyond the limitations of annotation-based or questionnaire-based labels, this paradigm represents a crucial step toward more rigorous evaluation in NLP, suggesting promising directions for future research in understanding the relationship between language and cognition.
%%Experiment-based validation is a way to move NLP evaluations beyond the limitations of annotation- or questionnaire-based labels. 
%%Here, we found that discourse-based cognitive language features were able to predict the cognitive style measured from a classic experiment. 
%Methods to capture cognitive style can help researchers better understand human cognition as well as lead to improved applications such as those for mental health where cognitive style is important in therapy or for more helpful agent engagement in general. 

%%Methods for capturing cognitive style can enhance our understanding of human cognition and improve applications in mental health therapy and agent engagement by tailoring approaches to individual thought processes. 


%By leveraging language to gain insights into cognitive styles, we can develop more responsive and personalized NLP applications, enhancing user interaction and satisfaction.

% \andy{alternative paragraph 1:
% We introduce and apply an evaluation framework to allow collection of lin
% guistic data pertaining to cognitive styles using006
% appropriate psychological, construct-validated007
% experiments that capture behavior the subjects.008
% To demonstrate the framework, we specifically009
% explore the phenomenon of decision making,010
% and its relationship to the linguistic style of011
% an individual talking about decision-making.012
% The experiment is conducted by collecting lan-013
% guage pertaining to a recent decision-making014
% exercise in their lives, followed by replicating015
% a classical decision-making experiment on the016
% participants that captures their cognitive coher-017
% ence, determined by the choice-induced shift in018
% preferences and other decision outcomes. We019
% find that such a stringent setup can still yield020
% decision outcomes with significant correlates021
% (AUC ∼ 0.8) in language, showing that cogni-022
% tive style can be partly captured and revealed023
% by linguistic patterns. }

\section*{Limitations}
While our experiment aims to capture cognitive dissonance through language in tandem with the replication of \citet{simon2004construction}, our study does not include direct questions in the writing prompts that explicitly prompt participants to discuss their decision-making process within the experiment itself. Despite the indirect writing prompt, we were able to capture promising cognitive style of individuals irrespective of the experimental outcome.  Further, the experiment offers a simulated job offer scenario, and the outcomes could be different in real-life. That said, our work is an initial step towards exploring associations of explicit linguistic structures and language modeling with observable psychological constructs, through the inclusion of psychological experiments in data collection. Therefore we chose a simpler abstraction of a real-world decision making problem as is usually done in the field of social psychology. However, this creates limitations in directly predicting participants' actual decision-making behaviors.

While discourse relations were originally intended to capture cognitive states through coherence and rhetorical structures, our predictive model-based method for inferring these relations offers only a small boost to the correlations when compared to lexical measures and contextual representations. This suggests that regular contextual embeddings might contain enough information to pick up cognitive styles and human behavior from language.

Our study population introduces several limitations that should be noted. The experiment uses undergraduate students at a public university which may limit the generalizability of the findings to other populations or age groups. While the study's focus on job decisions was particularly relevant to undergraduate students, who are often navigating a transitional phase focused career personal development, their decision-making processes may vary considerably from those of individuals in diverse life stages or professional environments. Furthermore, the linguistic outcomes were constrained by the small number of participants limited to the university. Therefore, the effect size was influenced by the restricted diversity in the population and the size of the participants.





\section*{Ethics Statement}

This study included an experiment with human subjects. The experiment followed closely to what that has been well replicated with no known risks in the past. The experiments were approved by ethical Institutional Review Board (IRB) who conducted a full review granting their approval. 

All participants provided informed consent prior to their participation. Participants were informed that they have the right to withdraw from the study at any time without any repercussions. Participants were also informed about how their data would be used and the measures taken to protect their privacy. Additionally participants confidentiality and privacy have been maintained throughout the research and analysis process. Any identifiable information collected during the study has been securely stored on a password-protected server, ensuring that only authorized personnel could access the information. All data were anonymized, any identifying details were removed or coded so that individuals could not be readily identified from the dataset. These steps ensured that the study upheld the highest ethical standards, prioritizing the privacy and well-being of all participants. The participants were paid USD 25 for completing the questionnaire after being recruited through the university. 

We run all of our experiments on an NVIDIA-RTX-A6000 with 50 GB of memory in an internal server, on open-sourced models. The LLMs were used for inferences rather than training for zero- and few-shot settings, with resource usage of about 15-20 hours on a single GPU. 

This work is part of a growing initiative to improve NLP for the human context. The models produced are not intended for any clinical or industrial application, and in particular not for targeted marketing or in use case where one's language is assessed for individual targeted information without individual awareness. 
The primary aim is to enhance the way cognitive processes are understood, ensuring that technology serves to augment psychological processes and measures. 

\section*{Acknowledgements}
%\vase{add ack}
This work was supported in part by a grant from the NIH-NIAAA (R01 AA028032) and a DARPA Young Faculty Award grant \#W911NF-20-1-0306 awarded to H. Andrew Schwartz at Stony Brook University.
The conclusions contained herein are those of the
authors and should not be interpreted as necessarily representing the official policies, either expressed or implied, of DARPA, NIH, any other
government organization, or the U.S. Government.


% Entries for the entire Anthology, followed by custom entries
\bibliography{anthology,custom}
\bibliographystyle{acl_natbib}


\clearpage
\appendix
\section*{Appendix}
\counterwithin{figure}{section}
\counterwithin{table}{section}
\label{sec:appendix}
%\section{Semantic Description of the \textit{Decisions} Corpus}
%\label{sec:topics}
%In order to describe the main themes of the corpus while preserving privacy of our study participants, we employed topic modeling. The results show that respondents writing about their difficult decisions frequently mention topics related to college education, career goals, finances, mental health, friendships, family relationships, and vacation plans. These subjects are largely connected to common decision-making aspects of student life.

%\begin{figure}[!ht]
% \centering
% \includegraphics[width=\linewidth]{diagrams/validation_topic_clouds.pdf}
%  \caption{Randomly selected topics emerging from LDA on the participant writing describing a recent decision, depicting the types of content evoked. 
 %%\andy{expand to 10 or 12 topics and make a two column figure. Can we add any meta information about the topics?}
 %}
  %\label{fig:topics}
%\end{figure}
  
\begin{figure}
 \centering
 \includegraphics[width=\columnwidth]{diagrams/validation_paper_method.pdf}

 \caption{
 %\andy{Would drop the picture of person with squiggles for implied dissoance. ACL paper style is figures are technical after the initial one. For the job offers boxes be more explicit "more money", "less money" rather than using solar signs.} 
After participants wrote about recent decisions that they had made (Step 1 in Figure 1), 
they completed a decision-making experiment wherein they encountered a simulated a job offer setting % wherein they made choices in real-time. 
 (See \S\ref{subsec:job_offer}). If the participant picks the job with higher salary and longer commute (marked in green), their preferences are expected to change in the direction of preferring high salary more, and less in the direction of preferring short commute times. %for more details. 
 %\andy{make horizontal and less busy to save space}
 }
 \label{fig:simons_exp}
\end{figure}

\section{Job Offer Questions}
A schematic diagram to demonstrate how the preference change is measured is shown in Figure~\ref{fig:simons_exp}. The detailed questionnaire administered to the participants is shown in Table \ref{tab:job_offer_qus}.
\label{sec:joboffer_desc}
% Please add the following required packages to your document preamble:
% \usepackage{graphicx}
\begin{table*}[]
\small
%\resizebox{\textwidth}{!}{%
\begin{tabular}{p{0.8\linewidth}p{0.1\linewidth}}
\toprule
\textbf{Questions} & \textbf{Response type} \\
\midrule
\textbf{Writing about a recent difficult decision }\\
\hspace{1mm} 1. Please describe a recent important and difficult decision that you have made (20-150 words) & text \\
\hspace{1mm} 2. What were the considerations that you thought about while making the decision? When answering, please consider all of the circumstances and details that went into the difficult decision (100-300 words) & text \\
\midrule
\textbf{Background:}\\Imagine that you have just graduated from college and have decided to look for a job. You have had interviews with a few companies, and are hoping to receive some job offers. In this experiment you will be asked to state how you feel about an assortment of aspects that might be included in job offers. 
Specifically, you will be asked to state how desirable or undesirable you find each aspect. 
There are no right or wrong answers to these questions. Please state how you personally feel about these aspects as if you were evaluating them in the context of making a real decision about your future career. You are not expected to have any special knowledge.
You might find that the information given to you is less complete than you would like to have; nonetheless, respond as best as you can given the available information.
The issues are unrelated, so simply consider each one independently.
\\
 \midrule
\hspace{1mm} 1. A company maintains a national training center in Jackstown, Tennessee. Every employee must spend 3 weeks of training at that center every year. Most employees describe the training as boring and the life in Jackstown as gloomy. - Please select how desirable participating in the training sessions at Jackstown is to you. & -5 to 5 \\
\hspace{1mm} 2. The commute to work will take you about 18 minutes each way. - Please select how desirable the 18 minute commute is to you. & -5 to 5 \\
\hspace{1mm} 3. The average annual salary for the position you are considering is \$60,000. The salary you are being offered is \$61,200. - Please select how desirable it is to you to receive \$1200 above the average salary. & -5 to 5 \\
\hspace{1mm} 4. You will be given a cubicle, which is located in a pretty noisy area. - Please select how desirable it is to work in a cubicle. & -5 to 5 \\
\hspace{1mm} 5. Given your credentials, you should be considered for promotion within a year or two. Being promoted will mean that you will have more independence, but it also means that you will have many more responsibilities. Some veterans maintain that in this type of profession, it is best to gain more experience before being promoted. - Please select how desirable a promotion is to you. & -5 to 5 \\
\hspace{1mm}6. All companies give their employees at least two weeks of vacation a year.  Some companies give additional vacation benefits.  A company offers you only the minimum two-week vacation. - Please select how desirable it is to receive only the minimum two-week vacation. & -5 to 5 \\
\hspace{1mm}7. The commute to work will take you about 40 minutes each way. - Please select how desirable the 40 minute commute is to you. & -5 to 5 \\
\hspace{1mm} 8. You are offered an office to yourself.  The office is pretty small, though adequate. - Please select how desirable the private office is to you. & -5 to 5 \\
\hspace{1mm}9. The average annual salary for the position you are considering is \$60,000. A company offers you \$59,100. - Please select how desirable it is to you to receive \$900 below the average salary. & -5 to 5 \\
\hspace{1mm}10. In addition to the standard two-week annual vacation, a company takes its employees and their families to a week-long retreat in San Diego.  The retreat consists of work-related lectures and workshops, but it is usually quite a lot of fun. - Please select how desirable the retreat in San Diego is to you. & -5 to 5 \\
\hspace{1mm}11. A company has a policy of encouraging personnel mobility among its numerous branches located throughout the country and across Europe.  Every employee is entitled to spend up to 3 months every 2 years working at any one of the company's branches. - Please select how desirable this mobility is to you. & -5 to 5 \\
\midrule
1. Please state the relative weight you would assign each of the aspects in the overall context of choosing a job (using the slider).  You are encouraged to use the full range of the scale: - 1. The office & 1 to 8 \\
2. Please state the relative weight you would assign each of the aspects in the overall context of choosing a job (using the slider).  You are encouraged to use the full range of the scale: - 2. The commute & 1 to 8 \\
3. Please state the relative weight you would assign each of the aspects in the overall context of choosing a job (using the slider).  You are encouraged to use the full range of the scale: - 3. The salary & 1 to 8 \\
4. Please state the relative weight you would assign each of the aspects in the overall context of choosing a job (using the slider).  You are encouraged to use the full range of the scale: - 4. The vacation package & 1 to 8 \\
\bottomrule
\end{tabular}

\end{table*}
\begin{table*}
\small
\begin{tabular}{p{0.98\linewidth}}
%\toprule
%\textbf{Questions} & \textbf{Response type} \\

%Please state the relative weight you would assign each of the aspects in the overall context of choosing a job (using the slider).  You are encouraged to use the full range of the scale: - 1. The office & 1 to 8 \\
%Please state the relative weight you would assign each of the aspects in the overall context of choosing a job (using the slider).  You are encouraged to use the full range of the scale: - 2. The commute & 1 to 8 \\
%Please state the relative weight you would assign each of the aspects in the overall context of choosing a job (using the slider).  You are encouraged to use the full range of the scale: - 3. The salary & 1 to 8 \\
%Please state the relative weight you would assign each of the aspects in the overall context of choosing a job (using the slider).  You are encouraged to use the full range of the scale: - 4. The vacation package & 1 to 8 \\
\midrule 
\textbf{[DISTRACTION] Synonyms task: Match the synonyms for 20 moderately difficult English words}   \\
\midrule
\textbf{Background:}\\
In this experiment you will be asked to play the role of a person who has just graduated from college. You are currently looking for a job in the field of marketing.You have just received interesting job offers from two large department store chains, Splendor and Bonnie’s Best. The two companies are similar in terms of their size, reputation and stability, and your prospects for promotion seem the same with both companies. You have already spent a couple of days at each of their offices, and have been interviewed by the key personnel. You found both companies to be stimulating and pleasant. After receiving more information about the two job offers, you will be asked to decide which one to accept.\\
\bottomrule
\end{tabular}

\textbf{Participants randomly get one of the two configurations (one with Splendor in a positive $loc$ condition and the other with Bonnie's Best in a positive $loc$ condition):}\\
\begin{tabular}{p{0.48\linewidth}||p{0.48\linewidth}}
\toprule

\textbf{Option A: Splendor (positive $loc$ condition)}& \textbf{Option A: Bonnie’s Best (positive $loc$ condition)}\\
Splendor is located in a fun part of town, next door to a new mall. There are many food joints, clothing stores, and cinemas close by. Most of the employees there go out to lunch in groups and eat at different places every day. They also do some convenient shopping on their way home from work. The average annual salary of a person at your position is \$60,000. The salary you are being offered by Splendor is \$59,100. At Splendor, you are offered an office to yourself. The office is pretty small, though adequate.The commute to the offices of Splendor takes about 18 minutes each way. Splendor offers its employees two weeks of vacation a year.&   Bonnie's Best is located in a fun part of town, next door to a new mall. There are many food joints, clothing stores, and cinemas close by. Most of the employees there go out to lunch in groups and eat at different places every day. They also do some convenient shopping on their way home from work. The average annual salary of a person at your position is \$60,000. The salary you are being offered by Bonnie’s Best is \$61,200. At Bonnie’s Best, you will be given a cubicle, which is located in a pretty noisy area. The commute to the offices of Bonnie’s Best takes about 40 minutes each way. In addition to the standard two-week annual vacation, every summer Bonnie’s Best takes its employees and their families to a retreat in San Diego. The retreat consists of work-related lectures and workshops, but it is usually quite a lot of fun.
\vspace{5mm}\\ \textbf{Option B: Bonnie’s Best} & \textbf{Option B: Splendor}\\Bonnie’s Best is located in a dull, sparsely populated industrial area. There is only one mediocre cafeteria nearby. Most employees bring their own sandwiches and eat on their own, or spend much of their lunch break driving to eateries that are a fair distance away. The average annual salary of a person at your position is \$60,000. The salary you are being offered by Bonnie’s Best is \$61,200. At Bonnie’s Best, you will be given a cubicle, which is located in a pretty noisy area. The commute to the offices of Bonnie’s Best takes about 40 minutes each way. In addition to the standard two-week annual vacation, every summer Bonnie’s Best takes its employees and their families to a retreat in San Diego. The retreat consists of work-related lectures and workshops, but it is usually quite a lot of fun.&Splendor is located in a dull, sparsely populated industrial area. There is only one mediocre cafeteria nearby. Most employees bring their own sandwiches and eat on their own, or spend much of their lunch break driving to eateries that are a fair distance away. The average annual salary of a person at your position is \$60,000. The salary you are being offered by Splendor is \$59,100. At Splendor, you are offered an office to yourself. The office is pretty small, though adequate. The commute to the offices of Splendor takes about 18 minutes each way. Splendor offers its employees two weeks of vacation a year. \\
\bottomrule
\end{tabular}
\end{table*}
\begin{table*}
\small
\begin{tabular}{p{0.8\linewidth}p{0.1\linewidth}}
\toprule
\textbf{Questions} & \textbf{Response type} \\
\midrule
At this point you have all the available information, and you are now asked to make your decision. Take your time and feel free to look back at the information provided. Please consider all pros and cons of both job offers carefully. Try to make this decision as if you were really in the described situation, and were facing a choice that will strongly influence your future career. When you have made your decision, please choose one of the two options. I accept the job offer of: & Bonnie's Best /  Splendor \\
\midrule
You will now be requested to state your preferences towards the aspects of the job offers made by Splendor and Bonnie’s Best. Specifically, you are requested to state how desirable or undesirable you find each of these aspects. There are no right or wrong answers to these questions. Please state your subjective preferences.
You are requested to answer the following questions using the provided scales. You are encouraged to use the full range of the scale:
\\\\
\hspace{1mm} 1.The commute to the offices of Splendor takes about 18 minutes each way. - Please select how desirable the 18 minute commute is to you. & -5 to 5 \\
\hspace{1mm} 2. Splendor does not offer any vacation benefits above the minimum two-week vacation a year. - Please select how desirable it is to receive only the minimum two-week vacation. & -5 to 5 \\
\hspace{1mm} 3.The salary you are being offered by Bonnie’s Best is \$1,200 above the average salary in the field. - Please select how desirable it is to you to receive \$1200 above the average salary. & -5 to 5 \\
\hspace{1mm} 4. At Splendor, you are offered an office to yourself.  The office is pretty small, though adequate. - Please select how desirable the private office is to you. & -5 to 5 \\
\hspace{1mm} 5. At Bonnie’s Best, you will be given a cubicle, which is located in a pretty noisy area. - Please select how desirable it is to work in a cubicle. & -5 to 5 \\
\hspace{1mm} 6. In addition to the standard two-week annual vacation, every summer Bonnie’s Best takes its employees and their families to a retreat in San Diego.  The retreat consists of work-related lectures and workshops, but it is usually quite a lot of fun. - Please select how desirable the San Diego retreat is to you. & -5 to 5 \\
\hspace{1mm} 7. The commute to the offices of Bonnie’s Best takes about 40 minutes each way. - Please select how desirable the 40 minute commute is to you. & -5 to 5 \\
\hspace{1mm} 8. The salary you are being offered by Splendor is \$900 below the average salary in the field. - Please select how desirable it is to you to receive \$900 below the average salary. & -5 to 5 \\
\bottomrule
\end{tabular}
\begin{tabular}{p{0.8\linewidth}p{0.1\linewidth}}
\toprule
1. Please state the relative weight you would assign each of the aspects in the overall context of choosing a job (using the slider).  You are encouraged to use the full range of the scale: - 1. The office & 1 to 8 \\
2. Please state the relative weight you would assign each of the aspects in the overall context of choosing a job (using the slider).  You are encouraged to use the full range of the scale: - 2. The commute & 1 to 8 \\
3. Please state the relative weight you would assign each of the aspects in the overall context of choosing a job (using the slider).  You are encouraged to use the full range of the scale: - 3. The salary & 1 to 8 \\
4. Please state the relative weight you would assign each of the aspects in the overall context of choosing a job (using the slider).  You are encouraged to use the full range of the scale: - 4. The vacation package & 1 to 8 \\
\bottomrule

\end{tabular}%
%}
\caption{Detailed description of the job offer questionnaire that the participants were administered.}
\label{tab:job_offer_qus}
\end{table*}

%\label{Time to Decide}
%\begin{table}[!ht]
%\centering
%\small
%\begin{tabular}{l|rr}
%\toprule
 
 %\multicolumn{2}{c}{\textbf{CIS\_highlow05 (lr)}} & 
%\textbf{Time to Decision (TD) } 
%\textbf{Features} &
%\multicolumn{1}{l}{\textbf{AUC}} & \multicolumn{1}{c}{\textbf{F1 wei}} &
%\textbf{MAE} $\downarrow$\\
%\midrule

%Most Freq Class & && -  \\
%\midrule
%Emotions & 0.545 & 0.451 & \textbf{0.811} & 0.527 %& 0.199
%& \textbf{9.00} & \multicolumn{1}{r}{0.056} \\
%Anxiety & 0.438 & 0.451 & 0.808 & 0.526 %& 0.199 
%& 9.04 & - \\
%Stress & 0.449 & 0.451 & 0.800 & 0.527 %& 0.199 
%& 9.02 & - \\
%Demographics & 0.384 & 0.451 & 0.805 & 0.527% & 0.199
%& 9.05 & - \\
%Loneliness & 0.566 & 0.451 & 0.805 & 0.527 %& 0.199 
%& \textbf{9.00} & - \\
%Empathy & 0.437 & 0.451 & 0.802 & 0.527% & 0.199 
%& 9.02 & - \\

%1to2grams & \textbf{0.541} & \textbf{0.527} & 0.725 & 0.500 & \textbf{0.239} & 9.03 & \multicolumn{1}{r}{0.024} \\
%BehavLingTraits %& 0.428 & 0.451 
%& 0.199 
%& \multicolumn{1}{c}{\textbf{9.02}}  \\
%OCEAN %& 0.442 & 0.451 
 %& 0.199 
%& \multicolumn{1}{c}{9.03} \\ \midrule
%Discourse relations %& 0.337 & 0.451 
%& \textbf{0.199} 
%& \multicolumn{1}{c}{9.08}\\
%RoBERTa-b (16) & \textbf{0.613} & \textbf{0.462} & 0.807 & \textbf{0.529} & \textbf{0.204} & 9.02 & - \\ 

%LIWC22 %& 0.488 & 0.411 
%& 0.206 
%& \multicolumn{1}{c}{9.07}\\
%FB topics %& 0.469 & 0.496 
%& \textbf{0.217} 
%& \multicolumn{1}{c}{9.22} \\
%\midrule
%DiscRE (16) %& 0.423 & 0.451 
%& \textbf{0.199} 
%& \multicolumn{1}{c}{\textbf{9.02}} \\
%RoBERTa-b (16) %& \textbf{0.613} & \textbf{0.462} 
 %& \textbf{0.204} 
%& \multicolumn{1}{c}{\textbf{9.02}} \\ 
%\bottomrule
%\end{tabular}
%\caption{
%We also find evidence of language being able to predict behavior -- the time taken to make a decision in the experiment.
%\vase{move table 2 to supplement and add corr}
%}
%\label{tab:TD_modeling_results}
%\end{table}

\section{Prompts}
The zero-shot and few-shot prompts for eliciting the CIS\_Inf scores are shown in Table~\ref{tab:prompts}.
\begin{table*}[]
\small
\begin{tabular}{p{0.1\linewidth}p{0.8\linewidth}}
\toprule
\textbf{Shot} & \textbf{Prompt}\\
\midrule
0-shot & You are an expert social and cognitive psychologist analyzing decision-making patterns from the 2004 study "Construction of Preferences by Constraint Satisfaction".   You are tasked with evaluating how preferences change when participants choose between two job offers with multiple attributes,              in a simulated setting. This experiment measured preferences before and after making a decision,              revealing "coherence shifts" where preferences aligned more closely with the chosen job offer,              occurring both with and without influencing attributes in the job description.              Your goal is to estimate two scores:              (1) the score of a coherence shift towards preferring the chosen job offer, expressed as a value between 0 and 1, where 0 indicates an increased preference for the rejected offer and 1 indicates a strong preference for the chosen offer; and             (2) the score that the decision is influenced by the job descriptions, also on a scale from 0 to 1, where 0 signifies no influence and a rigid preference, and 1 signifies being easily swayed by minor incentives.              Base your assessment on text provided by the user about a recent personal decision that need not be related to the job offer scenario.              Consider the cognitive styles and patterns of decision making evident in their narrative.              Present your findings in this format: "The score of a coherence shift towards the chosen job offer is: \textless{}score\textgreater and the score of being influenced by minor incentives is: \textless{}score\textgreater{},"              with each score ranging between 0 and 1. \\
\midrule
4-shot & You are an expert social and cognitive psychologist analyzing decision-making patterns from the 2004 study "Construction of Preferences by Constraint Satisfaction".              You are tasked with evaluating how preferences change when participants choose between two job offers with multiple attributes,              in a simulated setting. This experiment measured preferences before and after making a decision,              revealing "coherence shifts" where preferences aligned more closely with the chosen job offer,              occurring both with and without influencing attributes in the job description.              Your goal is to estimate two scores based on user-provided text:              (1) the score of a coherence shift towards preferring the chosen job offer, expressed as a value between 0 and 1, where 0 indicates an increased preference for the rejected offer and 1 indicates a strong preference for the chosen offer; and             (2) the score that the decision is influenced by the job descriptions, also on a scale from 0 to 1, where 0 signifies no influence and a rigid preference, and 1 signifies being easily swayed by minor incentives.              Here are four different examples of participants' narratives about recent personal decisions and with a score towards 1 if they had a coherence shift towards the chosen job offer, 0 if coherence shift is towards the rejected offer. Similarly, there is also a score for if being influenced by minor incentives (1 if influenced, 0 if not influenced):                          Example 1:             User's Narrative: "I recently had to decide whether to buy a new car or keep my old one. The new car had better fuel efficiency and more features, but I was attached to my old car due to sentimental reasons. After considering the costs and benefits, I decided to go with the new car."             Output: "The score of a coherence shift towards the chosen job offer is: 0.8 and the score of being influenced by minor incentives is: 0.6."              Example 2:             User's Narrative: "I was choosing between two vacation destinations: a beach resort and a mountain cabin. I love both settings, but ultimately chose the beach resort because it was more affordable and had better amenities."             Output: "The score of a coherence shift towards the chosen job offer is: 0.7 and the score of being influenced by minor incentives is: 0.5."              Example 3:             User's Narrative: "I had to decide whether to take an online course or attend in-person classes for my professional development. The online course was more flexible, but I prefer face-to-face interaction. I chose the online course because it fit better with my schedule."             Output: "The score of a coherence shift towards the chosen job offer is: 0.9 and the score of being influenced by minor incentives is: 0.4."               Similarly, for the following user input text, estimate the scores. Base your assessment on text provided by the user about a recent personal decision that need not be related to the job offer scenario.              Consider the cognitive styles and patterns of decision making evident in their narrative.              Present your findings in this format: "The score of a coherence shift towards the chosen job offer is: \textless{}score\textgreater and the score of being influenced by minor incentives is: \textless{}score\textgreater{},"              with each score ranging between 0 and 1 as a continuous value.\\
\bottomrule
\end{tabular}%
\caption{Zero- and 4-shot prompts for both Llama3.1 and Gemma models.}
\label{tab:prompts}
\end{table*}

\end{document}
