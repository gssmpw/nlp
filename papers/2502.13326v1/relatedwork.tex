\section{Related Work}
%The field of Psychology %, however,
%relies heavily on carefully designed experimental paradigms to both elicit and capture cognitive state more objectively.
%Cognitive Styles, or sets of thought patterns related to specific cognitive phenomena, involve more than just linguistic expressions; they also involve a context-dependent change in mental state in response to social situations. 
%This highlights the need for a shift towards carefully designed construct-validated judgment and decision-making experiments designed to capture cognitive styles more accurately than both external annotators or the individuals themselves-- assessing styles by measuring a person's mental state before and after an experiment to observe changes.
While NLP for social science often relies on labels from annotators or questionnaires, behavioral science theory suggests carefully designed experiments can more objectively elicit and capture cognitive states.
% better than relying on judgments (of either annotators or self-reported in questionnaires). 
For instance, \citet{saxbe2013embodiment} investigated emotional responses using an experimental design in which participants' brain activity was imaged as they listened to narratives eliciting different emotions. These methods can offer a more objective foundation for understanding the psychological processes at play~\citep{brook2015linking}. In our study, we focus on cognitive styles in decision making -- reflecting one's tendency to maintain consistency and resolve dissonance~\citep{harmon2007cognitive,mcgrath2017dealing}. %, specifically processes such as constraint satisfaction and dissonance resolution
%This occurs very frequently --   when one has to make difficult decisions, such that their beliefs are challenged in some way
Since people show little awareness of their decision-making~\cite{nisbett1977halo}, 
%pyszczynski1993, 
 cognitive styles are measured experimentally by observing shifts in preferences after decision~\citep{simon2004construction,aguilar2022cognitive}. 
%We pick~\citet{simon2004construction} job-offer experiment to capture preference shifts.
%When making decisions, people seem to exhibit little to no awareness of the processes that lead to the decision. 
%The human tendency to be consistent in one's beliefs~\citep{tedeschi1971cognitive} is also a factor that leads to resolution of dissonance without awareness\cite{mcgrath2017dealing}. 
%This resolution can be 

%\citet{saxbe2013embodiment} studied emotions by eliciting language describing the emotions felt on watching or listening to emotional narratives, offering a more objective ground truth for underlying psychological processes that is most relevant for the produced language~\citep{brook2015linking}. \vase{more examples/constructs}

%\ryan{I would recommend including a concrete example of what this looks like from the literature. An easy example would be something like emotion elicitation paradigms where they induce some kind of affective state, then have people talk, e.g.,}

%\ryan{https://academic.oup.com/scan/article/8/7/806/1656575}

%\ryan{https://doi.org/10.1080/19312458.2014.999751}

%\ryan{Essentially, then you can say something like 'see, this is how you get a more objective ground truth for which psychological processes are most salient/operant at the time of language production'}


%TODO: define cognitive styles better






%The evoked Cognitive Styles, which are sets of thought patterns related to specific cognitive phenomena\footnote{in this paper, we specifically refer to decision-making \vase{maybe say this elsewhere when we talk specifically about decision making?}}, such as intented humor or offensiveness, cognitive dissonance, reasoning skills or empathy can be best measured by capturing the mental states of the subjects rather than crowd-sourcing the perceived qualities based on the author's text, which is prone to disagreements and errors \cite{sandri-etal-2023-dont, sap2021annotators, varadarajan-etal-2023}. 




%TODO: MAKE THIS POINT STAND OUT \ryan{An important distinction to raise here, for example, is that these approaches still fall somewhat short because we don't have a ground truth on what psychological processes are in operation. A person may utter two statements that are logically inconsistent, but we still need some semi-objective ground truth on what is actually happening in the person's head if we want to make sound inferences about how verbal behavior relates to or reflects such mechanisms.}

% ADD THE CITATIONS TO THE FIRST PARAGRAPH: Opinion, affect and stance shifts, cognitive dissonance and decision-making has heavily relied on annotations \cite{hovy2015sentiment, sakketou2022investigating, varadarajan2022disso, varadarajan-etal-2023}. Empathy has often relied on annotations as well \cite{lahnala-etal-2024-appraisal-framework}, sometimes including a self-reported empathic concern and distress score \cite{barriere-etal-2022-wassa}. 


% \ryan{My sense is that the following paragraphs are really the most important, and I would recommend getting to them as quickly as possible. Above, you're making a strong case that 1) we rely too heavily on annotator reports, which have established pitfalls, 2) other methods of ground truth such as self-report have pitfalls as well, especially when it comes to psychological phenomena that a person might lack insight into (e.g., by their very nature, people aren't aware that they are making cognitive distortions, etc.), and 3) that the field of psychology has long worked under a more classical scientific/empirical paradigm of conducting research to establish a more objective or well-defined ground truth that minimizes or eliminates subjective biases, particularly in the domain of psychological phenomena that cannot be directly observed. I think that you can probably condense that down to maybe 3 terse paragraphs, leaning heavily on citations to make your points, and with an illustrative example for point #3 (psychology uses experimental paradigms). This then sets you up to say 'so, what we propose here is the adoption of experimental cognitive science for NLP evaluation and inference tasks.'}



% DISCUSS THIS: \ryan{For the above paragraph, I would go even bigger. Sure, this stuff is valuable as a 'language-based assessment' type of task, but (I think) that there is more to it here that will strike a stronger chord with an NLP audience. What you're effectively doing here is showing a paradigm for discovering the particulars of *how* verbal behavior can reveal deeper cognitive processes. When the mind is doing X, it has Y effect on a person's language that is — critically — able to be observed empirically through statistical/computational modeling. This might be a bigger point for the discussion, but I'd imagine that people interested in the LLM space would be interested in thinking about applying this idea to bigger models — this is why people are doing so much comparative work in psychology right now between humans and ChatGPT — if we understand how 'heuristic processing' works in humans, we can get ground truth information from humans using an experimental paradigm, then see how GPT stacks up, etc. etc. Put another way: for the folks in pure/theoretical NLP who don't actually care that much about 'human factors' like empathy or social connectedness, I'd be willing to bet that they still have a deep, vested interest in understanding how the cognitive underpinnings of humans impact language use.}


%\vase{Longer intro: Problems with annotation, partially addressed with self report but there are still biases and JDM turn to experiments evoking CS and observation rather than SF and annot.}

%Although language has been studied in the NLP community as structured constructions of words or phrases \cite{naseem2021comprehensive} or their semantic roles \cite{li2022survey}, language is fundamentally a cognitive process, which is used to express explicit cognitive phenomena such as conveying emotion\cite{?}, argumentation\cite{?} and narrative storytelling\cite{?}. The multitude of implicit cognitive phenomena, such as cognitive dissonance and its resolution are much rarer to be expressed in language. 






%