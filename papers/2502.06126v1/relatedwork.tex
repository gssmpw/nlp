\section{Related Works}
\label{append:related_works}

\paragraph{Graph Neural Networks and Different Types of Graphs}
GNNs were originally proposed to resolve the challenge of data point dependencies via the traditional convolution neural networks, which, in general, treat every input data point independently of each other \citep{kipf2016semi,defferrard2016convolutional}. By considering the so-called adjacency information stored in the graph, GNNs propagate graph node features by aggregating its neighboring information \citep{wu2020comprehensive}. Such propagation paradigm has made GNNs one of the most successful tools in generating predictions via various types of graphs, such as citation networks \citep{wu2020comprehensive}, social networks \citep{sharma2024survey}, molecules (as well as protein and ligands) \citep{zhang2022graph}, traffic networks \citep{jiang2022graph}, to name a few. These graphs vary from different levels of measurements of the data, e.g., scientific papers in citation networks \citep{yang2016revisiting}, atoms, and amino acids in protein and ligand graphs \citep{wu2018moleculenet,GilmerSchoenholzRiley2017} and nodes are connected with different types of attributes (e.g., citations and chemical bonds). In this work, we provide a \textbf{novel graph dataset} (known as JR5558, more details see Section \ref{sec:preliminaries}) in which graphs are formed by the genetic pathways (as nodes) and pathway similarities (as edges), serving as the profiles of the experimental objects (e.g., mice). In addition, we also label these graphs with mice's lesion severity scores (as a continuous variable) obtained from fundus photographs of these mice, where severity was quantified by measuring subretinal lesion size.We train two different types of GNNs to capture the patterns between genetic pathways and mice lesion severity scores: one that does not consider temporal information, and another designed to account for temporal information from the estimated disease progression in the mice.



\paragraph{Pseudotime Analysis and Stochastic Differential Equations}
Pseudotime analysis (PA) was originally developed in single-cell transcriptomics to reconstruct cell differentiation trajectories from static snapshots of gene expression profiles \citep{trapnell2014dynamics,haghverdi2016diffusion,wei2021dtflow}. Since time-resolved measurements of individual cells are often infeasible, pseudotime methods infer an intrinsic ordering (e.g., trajectories) of cells based on their transcriptional similarities, providing insights into dynamic biological processes. While it is a powerful tool widely applied in biology and medical science, its adoption in the machine learning community has only gained significant attention in recent years. For example, recent work has explored using diffusion models to infer pseudotime from single-cell transcriptomic data \citep{mosspseudotime}. In this work, we apply PA to the embeddings of graphs constructed from the genetic pathways of the mice and analyze the estimated trajectories via neural stochastic differential equations \citep{li2020scalable} (see Section \ref{sec:nsde} for more details), which allow us to further exploit pathway stability and disease bifurcation points. This paradigm paves the path of incorporating advanced machine learning approaches to the exploration of the fundamental problems in complex biological systems.














%