\section{Related Works}
\subsection{Lightweight Visual Tracking}

Lightweight visual tracking has advanced significantly with the rise of Siamese-based methods. Early methods, such as SiamFC ____ and ECO ____, achieve real-time performance on edge devices. However, their tracking accuracy remains inferior compared to state-of-the-art trackers. Recently, more efficient trackers have been introduced. Yan \emph{et al.} ____ employ neural architecture search (NAS) to design efficient and compact models for resource-constrained hardware. Borsuk \emph{et al.} ____ introduce a dual-template-based lightweight tracker, combining static and dynamic templates to handle appearance changes. Gopal \emph{et al.} ____ use Mobile Vision Transformer to integrate features from both the template and search regions, improving localization and classification for tracking. These advancements underline the focus on creating robust, real-time tracking solutions suitable for practical applications on mobile platforms. However, these methods often compromise on discriminative ability due to their simplified architectures.

\subsection{Contrastive Learning}
Contrastive learning is a self-supervised technique that aims to distinguish between positive pairs (augmentations of the same instance) and negative pairs (augmentations of different instances). It has shown promise in various computer vision tasks by learning effective feature representations ____. In the filed of visual tracking, contrastive learning has also been employed to improve instance and category-level representation. Pi \emph{et al.} ____ leverage instance-aware and category-aware modules combined with video-level contrastive learning to enhance inter-instance separability and intra-instance compactness. Wu \emph{et al.} ____ employ contrastive learning in an unsupervised framework, using noise-robust temporal mining strategies to generate positive and negative pairs, thereby improving temporal consistency. While these methods improve robustness, they lack a focus on lightweight design, which is crucial for real-time applications. Dan \emph{et al.} ____ specifically targets UAV tracking by utilizing contrastive learning within video frames to enhance feature discrimination against occlusion. However, the contrastive module is not used during inference, limiting its ability to distinguish targets from distractors in real-world tracking scenarios. In this paper, we address these gaps by incorporating a contrastive learning-based feature matching module that operates during both training and inference within a lightweight Siamese framework, ensuring accurate and efficient tracking even in resource-constrained applications.