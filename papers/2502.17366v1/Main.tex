\documentclass[twocolumn,10pt]{IEEEtran}
% \usepackage{hyperref}
\usepackage{graphicx}
\usepackage{bm,cite,float,amsmath,amssymb,amsthm}
\usepackage{multirow}
\usepackage{algorithm}
\usepackage[noend]{algpseudocode}
% \usepackage{indentfirst}
\usepackage{xcolor}
\usepackage{makecell}
\usepackage{color}
\usepackage{bbding}
\usepackage{lipsum}
\usepackage{mathtools}
\usepackage{subfigure}
\usepackage{cuted}
\usepackage[caption=false,font=footnotesize]{subfig}
% \usepackage{epstopdf}
% \usepackage{epsfig}	
\usepackage{amsfonts,balance}
\usepackage{bbm}
\usepackage{balance}
\usepackage[utf8]{inputenc}
\usepackage{adjustbox}
\usepackage{verbatim}

% \IEEEoverridecommandlockouts
\begin{document}

\title{Distributed Coordination for Heterogeneous Non-Terrestrial Networks}


\author{Jikang~Deng,~\IEEEmembership{Student Member,~IEEE,} Hui~Zhou,~\IEEEmembership{Member,~IEEE,} and Mohamed-Slim~Alouini,~\IEEEmembership{Fellow,~IEEE}
		
		\thanks{Jikang~Deng, and Mohamed-Slim~Alouini are with CEMSE Division, King Abdullah University of Science and Technology (KAUST), Thuwal, 23955-6900, Kingdom of Saudi Arabia (KSA) (email: {jikang.deng@kaust.edu.sa}; {slim.alouini@kaust.edu.sa})}
		\thanks{Hui Zhou is with Centre for Future Transport and Cities, Coventry University, U.K. (email:{hui.zhou@coventry.ac.uk}). This work was done while he was working at King Abdullah University of Science and Technology.}
	}% <-this % stops a space

\maketitle

\begin{abstract}
To guarantee global coverage and ubiquitous connectivity, the Non-terrestrial Network (NTN) technology has been regarded as a key enabling technology in the Six Generation (6G) network, which consists of the unmanned aerial vehicle (UAV), high-altitude platform (HAP), and satellite. It is noted that the unique characteristics of various NTN platforms directly impact the design and implementation of NTNs, which results in highly dynamic and heterogeneous networks. Even within the same tier, such as the space tier, the NTNs are developed based on different platforms including Low Earth Orbit (LEO), Medium Earth Orbit (MEO), and Geostationary Earth Orbit (GEO). Therefore, distributed coordination among heterogeneous NTNs remains an important challenge. Although distributed learning framework finds a wide range of applications by leveraging rich distributed data and computation resources. The explicit and systematic analysis of the individual layers' challenges, and corresponding distributed coordination solutions in heterogeneous NTNs has not been proposed yet. In this article, we first summarize the unique characteristics of each NTN platform, and analyze the corresponding impact on the design and implementation of the NTN. We then identify the communication challenges of heterogeneous NTNs in individual layers, where the potential coordinated solutions are identified. We further
illustrate the multi-agent deep reinforcement learning (MADRL) algorithms tailored for coordinated solutions in heterogeneous NTNs. Last but not least, we present a case study of the user scheduling optimization problem in heterogeneous UAVs-based cellular networks, where the multi-agent deep deterministic policy gradient (MADDPG) technique is developed to validate the effectiveness of distributed coordination in heterogeneous NTNs.
\end{abstract}

\begin{IEEEkeywords}
Distributed Coordination, Distributed Learning, Heterogeneous Network, Non-terrestrial Network 
\end{IEEEkeywords}
\IEEEpeerreviewmaketitle

\section{Introduction}
Due to the improved coverage and resilience against terrestrial infrastructure failures, the Non-terrestrial Network (NTN) has been identified as one of the emerging usage scenarios in International Mobile Telecommunications 2030 (IMT-2030) for the Six Generation (6G) network, which has the potential to provide connections in post-disaster scenarios and remote areas \cite{Xiao2024}. Currently, the research in the NTN community focuses on three platforms including the unmanned aerial vehicle (UAV), high-altitude platform (HAP), and satellites. It is noted that the unique characteristics of each platform (e.g., power supply and loading capability) play a critical role in the design and implementation of the NTN (e.g., transmission power and number of antennas) \cite{Toka2024}. Therefore, the coordination of heterogeneous NTNs remains an important challenge to solve.

Different from traditional terrestrial networks with fixed infrastructure, the NTN is characterized by dynamic network topology, which imposes a heavy burden on the centralized coordination among NTNs within stringent latency. In \cite{Tuzi2023}, the authors exploited the distributed homogeneous satellites to form a sparse phased array for direct-to-cell connectivity. Distributed learning has been regarded as a promising solution to enhance coordination among NTNs by capturing their dynamic topology, and several existing works mainly focus on the general distributed deep reinforcement learning (DRL) framework. In \cite{Xu2024}, the authors presented and analyzed buffer-aided NTN based on the decentralized DRL algorithm. In \cite{Cao2024}, the authors proposed generalized multi-tier DRL architectures to enhance the cooperation among space-tier, air-tier, and ground-tier stations. However, a comprehensive study of unique challenges and coordinated solutions in individual layers with distributed learning frameworks has never been carried out. 

\begin{table*}[!htb]
    \caption{Typical Characteristics of Non-terrestrial Network platform}
    \begin{center}
	\begin{tabular}{|c|c|c|c|c|c|c|}
		\hline
		\textbf{Specification}  &Tethered UAV&Untethered UAV&HAP&LEO&MEO&GEO\\
		\hline			
		Loading Capability &20 Kg&2.7 Kg& 140 Kg&-&-&-\\
		\hline
		Power Supply &30 KWh&0.3 KWh&20 KWh&-&-&-\\
		\hline
        Latency  &1 ms&1 ms&1 ms&30-50 ms&150 ms&600 ms\\
		\hline
        Endurance &24 hours&30 mins& 2 months&5-7 years&10-15 years& 15-20 years\\
		\hline
        Deployment Cost&Low&Low& Medium&High&High&High\\
        \hline
        Network Type &1 Macro &1 Micro/Pico &7 Multiple Macro &61 Multiple Macro &61 Multiple Macro &61 Multiple Macro \\
        \hline
        Payload Option &\makecell[l]{$\bullet$ RU\\$\bullet$RU+DU\\$\bullet$RU+DU+CU}& $\bullet$RU+DU+CU& \makecell[l]{$\bullet$RU+DU\\$\bullet$RU+DU+CU} &\makecell[l]{$\bullet$RU+DU\\$\bullet$RU+DU+CU}&\makecell[l]{$\bullet$RU+DU\\$\bullet$RU+DU+CU}&\makecell[l]{$\bullet$RU+DU\\$\bullet$RU+DU+CU}\\
        \hline
	\end{tabular}
	\label{aerial_BS_summarization}
    \end{center}
\end{table*}

Motivated by this, in this paper, we provide a concrete vision of coordination among heterogeneous NTNs with a focus on the corresponding distributed framework. The main contributions of this article are:
\begin{enumerate}
\item We first summarize the unique characteristics of various NTN platforms, including UAV, HAP, and satellites, compared to the traditional terrestrial infrastructure, and analyze the corresponding impact on the design and implementation of NTNs in Section~II.
\item Based on the unique characteristics of each NTN platform, we then provide a comprehensive analysis of challenges and coordinated solutions in heterogeneous NTNs, where the physical layer, MAC layer, network layer, and application layer are considered in Section~III.
\item To solve the challenges in Section~III in a distributed manner, we further present emerging multi-agent deep reinforcement learning (MADRL) algorithms tailored for coordinated heterogeneous NTNs in Section~IV, which are divided into delay-tolerant and delay-sensitive functions, respectively.
\item To demonstrate the effectiveness of our proposed distributed learning for heterogeneous NTNs, we present a case study of user scheduling optimization over heterogeneous UAV networks and analyze the results, specifically with the multi-agent deep deterministic policy gradient (MADDPG) algorithm in Section~V. Finally, we conclude the article in Section~VI.
\end{enumerate}
 

\section{Unique Characteristics of Non-terrestrial Network Platforms}
In this section, we present and analyze the characteristics of different aerial platforms, which include the UAV, HAP, and satellite. More importantly, we analyze how their unique characteristics impact the design and implementation of the cellular network as summarized in Table~\ref{aerial_BS_summarization}. 


Fig. \ref{aerial_BS_system} shows the typical terrestrial 5G network consisting of a Radio Unit (RU), a Distributed Unit (DU), a Centralized Unit (CU), and the core network, where O-RAN CatA is adopted for the fronthaul split \cite{Larsen2019}. To achieve the tradeoff between fronthaul bandwidth and coordination, the beamforming procedure, inverse fast Fourier transform (iFFT), and cyclic prefix processing are integrated within the RU, where the user data symbols and precoding weights are transmitted over the fronthaul. In terms of the Size, Weight and Power (SWaP),  it is noted that the typical weights of macro massive MIMO RU, and Baseband Unit (BBU) are over 10Kg and 5Kg, respectively, where BBU consists of DU and CU. The average power consumption of RU and CU can reach 1.175KWh and 0.325KWh under a full workload, which can even reach 3.8KWh with the three-sector setting.

\begin{figure}[!ht]
  \centering
  \includegraphics[width = 8.8cm]{magazine_CU_DU_RU.pdf}
  \caption{A typical structure of terrestrial cellular network}
  \label{aerial_BS_system}
\end{figure}

\subsection{Unmanned Aerial Vehicle}
\subsubsection{Untethered UAV} Taking DJI as an example, the flagship UAV platform Matrice 350 RTK achieves a loading capability of 2.7Kg with 31 minutes of flight time, which cannot carry the macro BS equipment discussed above. Although DJI has customized a delivery UAV platform with the 40Kg loading capability (i.e., DJI Flycart30). The flight time is limited to 8 minutes, which fails to guarantee continuous connections to the users. Apart from that, the battery is originally designed to support the UAV flight with a typical capacity 0.3KWh (e.g., TB65 from DJI). Therefore the commercial untethered UAV can only be utilized to implement a single micro or pico network with RU, DU, and CU integrated as a complete payload. 

\subsubsection{Tethered UAV}
To overcome the limitations above, the tethered UAV has been regarded as a promising solution, which has been utilized for emergency communication in post-disaster areas. Based on the power supply provided by the tethered system, the tethered UAV achieves much stronger loading capability and endurance time, which is suitable to implement a single macro network, e.g., DG-M30 tethered UAV can carry 20Kg payloads hovering at 200m for 24 hours.  More importantly, the optical fiber provided by the tethered system enables flexible communication payload options, including RU, RU/DU, and RU/DU/CU. However, the tethered system limits the mobility of the UAV, which may lead to coverage degradation.

\begin{figure*}[!ht]
  \centering
  \includegraphics[scale = 0.72]{Figures/System_Model.pdf}
  \caption{Individual-layer challenges in heterogeneous Non-terrestrial Network.}
  \label{System_model}
\end{figure*}

\subsection{High Altitude Platform}
Operated in the stratosphere e.g., 20km, the HAP is capable of flying for several months continuously, and circles at a radius of a few kilometers or less. Taking the HAP from Stratospheric Platforms Ltd as an example, with a wingspan of 60m, it can support a maximum of 140Kg payload and can provide a 20kW power supply. As indicated in 3GPP, the HAP can support the operation of 7 macro networks with a coverage of 100Km, where the communication payloa can be RU/DU or RU/DU/CU \cite{3gppHAP}. It is noted that 3GPP supports a maximum cell range up to 100km, which means the HAP can achieve direct connectivity to the regular user terminal without modifying the protocols. More importantly, The relative stationary position of the HAP makes it a promising solution to implement wireless backhaul or midhaul transmission based on point-to-point (P2P) LoS communication. However, It is noted that the environmental temperature of the HAP can be as low as -50 to -70 degrees Celsius. Therefore, an additional heating system should also be considered to guarantee the device's operation in a suitable temperature range. 


\subsection{Satellite}
According to the operation altitude, the satellite platform can be divided into low earth orbit (LEO) satellite, medium earth orbit (MEO), and geostationary orbit (GEO) satellite. Although the GEO satellite can provide continuous connections due to its stationary position relative to the Earth, the large signal path loss and delay remain important challenges to be solved, especially for the uplink transmission. Apart from that, the GEO is exactly above the equator, which means the ground users in high-latitude areas will experience low LoS probability due to low elevation. Another common challenge for satellite communication is space weather including solar flares and geomagnetic storms, which may disrupt satellite systems and degrade the reliability of the cellular network significantly. For example, the 38 satellites of Starlink were destroyed due to the solar storm in 2022. In general, due to the advantages of less gravity and extensive sunlight, the loading capability and power supply of the satellite can guarantee the operation of 61 macro networks for years, where the Effective Isotropic Radiated Power (EIRP) reaches 59dBW/MHz for each cell \cite{3gppNTN}. 

\section{Challenges and Coordinated Solutions in Heterogeneous Non-Terrestrial Networks}According to the characteristics analysis of various NTN platforms, in this section, we further present communication challenges and potential coordinated solutions of individual layers in heterogeneous NTNs as shown in Fig.~\ref{System_model}, which are summarized in Table.~\ref{challenges_solutions}.

\begin{table*}[!htb]
    \caption{Challenges and Coordinated Solutions in Heterogeneous Non-terrestrial Networks}
    \begin{center}
	\begin{tabular}{|c|c|c|c|c|}
		\hline
		\textbf{Individual Layer}&Function  &Heterogeneous Challenges&Coordinated Solutions&Delay-tolerant\\
		\hline			
		\multirow{5}{*}{\makecell[l]{Physical Layer}} &Initial Access& Cell Selection &Cell Shaping&\checkmark\\
		\cline{2-5}
		 &Channel Estimation&Pilot Interference& Pilot Resource Allocation&\checkmark\\
		\cline{2-5}
         &Precoding&Inter-cell Interference&Distributed MIMO&\texttimes\\
		\cline{2-5}
         &\multirow{2}{*}{Power Allocation}&Cross-Link Interference&\multirow{2}{*}{UL/DL Power Allocation}&\multirow{2}{*}{\texttimes}\\
		\cline{3-3}
        &&Remote Interference&&\\
        \hline
        \multirow{2}{*}{MAC Layer}&\multirow{2}{*}{Scheduling}&\multirow{2}{*}{Inter-cell Interference}&\multirow{2}{*}{ Coordinated Scheduling}&\multirow{2}{*}{\texttimes}\\
        &&&&\\
        \hline
        Network Layer&Routing&Dynamic Backhaul Topology&Multi-hop IAB&\checkmark\\
        \hline
        Application Layer&Trajectory Control& Distinct Control Period&Multi-timescale Control&\checkmark\\
        \hline
	\end{tabular}
	\label{challenges_solutions}
    \end{center}
\end{table*}

\subsection{Physical Layer}
\subsubsection{Initial Access}
The existing initial access procedures are based on the downlink Reference Signal Received Power (RSRP) of the Synchronization Signal Block (SSB), where the UE connects to the cell with the highest RSRP \cite{Bernab2024}. In traditional terrestrial networks, most of the cellular cells are homogeneous, where the downlink RSRP can be regarded as an effective indicator of transmission performance for cell selection. However, in the heterogeneous NTNs, cells in each tier have distinct communication capabilities, including the number of antennas, bandwidth, traffic loading, etc. Even for the NTNs in the same tier, the BSs may still have different characteristics, e.g., the tethered UAV and untethered UAV, and the visibility time length of different satellites. 

More importantly, the uplink transmission performance should also be considered during cell selection, where the UE may experience poor uplink performance to certain cells due to limited transmission power. To solve the problem, a hierarchical coordination framework should be developed for cell selection, where the NTNs perform coordinated intelligent cell shaping using optimal SSB beams, and then each UE selects the cell based on its local characteristics including transmission power. One typical example is the intelligent cell shaping developed for terrestrial networks by Ericsson, where multiple BSs coordinate to optimize the cell coverage area based on reinforcement learning. It is noted that the cell selection procedure is delay-tolerant over the Telecom operator's level. 

\subsubsection{Channel Estimation}
Uplink channel estimation has been regarded as an important step in guiding downlink user data transmission, including scheduling in the MAC layer and precoding in the physical layer, where each UE is required to transmit the uplink pilot signal to the NTNs, i.e., Sounding Reference Signal (SRS). If the UE is associated with a higher tier NTN, higher transmission power is usually needed to compensate for the larger path loss. Therefore, in the vertical heterogeneous NTNs, UEs associated with lower-tier NTNs will experience serious pilot interference from the UEs associated with the higher-tier NTNs. Considering the much larger beam footprint in the higher-tier NTNs and limited beamforming capability at the UE, pilot interference will be more serious than that in traditional terrestrial networks.

To solve this challenge, a coordinated pilot resource allocation scheme should be developed to enhance uplink channel estimation accuracy based on the distribution of terrestrial UEs in each cell. It is noted that the periodicity of pilot transmission can be configured to be one of [2,5,10,20,40,80,160,320] ms, which is generally delay-tolerant in most cases.

\subsubsection{Precoding}
Distributed MIMO (DMIMO) technique has been regarded as a promising solution to reduce inter-cell interference based on coherent joint transmission (CJT), which has been proven to benefit terrestrial cell-edge users significantly \cite{Haliloglu2023}. In the CJT scheme, a central server is required to obtain the complete wireless channel (i.e., the wireless channel between one UE and multiple cells), and then perform the precoding algorithm to calculate the precoding matrix, such as weighted minimum mean-square error (WMMSE).  It is noted that the precoding procedure is delay-sensitive and executed in every slot, e.g., 0.5 ms with 30KHz subcarrier spacing (SCS).

However, the performance improvement of cell-edge users usually comes with the performance degradation of coordinated cells in the CJT scheme, which becomes more complicated and challenging to quantify in heterogeneous NTNs. Therefore, evaluating the overall gain of performing DMIMO and selecting the optimal coordinated cells remains an important challenge. Apart from that, the heterogeneous NTNs adopt distinct single-antenna power constraints. Hence, the challenge of guaranteeing a coherent phase with high power utilization efficiency also needs to be solved.

\subsubsection{Power Allocation}
Power allocation optimization has been regarded as an effective solution to maximize the channel capacity both in single-cell and multiple-cell scenarios. In the heterogeneous NTNs, cross-link interference (CLI) and remote interference (RI) emerge to be two important challenges due to the LoS-dominated aerial channel, where CLI happens between adjacent cells with different TDD frame structures and RI happens between very distant cells even with synchronized TDD frame structures. It is noted that the upper-tier NTNs usually adopt a much higher downlink transmission power than the lower-tier NTNs, and will cause strong CLI and RI to the lower-tier NTNs. To solve the CLI and RI in heterogeneous NTNs, especially for the downlink (DL) to uplink (UL) direction, a coordinated UL-DL power allocation scheme should be developed to enhance the UL signal strength and mitigate DL interference. Similar to precoding, the power allocation procedure is delay-sensitive and needs to be executed in each time slot. 

\subsection{MAC Layer}
\subsubsection{Scheduling}
Different from the DMIMO technique to mitigate inter-cell interference in the physical layer, the scheduling optimization has the potential to enhance the performance of cell-edge users by allocating the same resource block to UEs with near-orthogonal channel characteristics. In the heterogeneous NTNs, the multi-tier NTNs will lead to high coverage overlap, where a highly efficient coordinated scheduling algorithm is needed to satisfy latency requirements in each slot. It is noted that the coordinated scheduling algorithm is also important for the CJT scheme in DMIMO, where the participating cells must transmit the user data on the same resource block. Otherwise, the coordinated cells will sacrifice the power consumption without improving the cell-edge UEs in the primary cell. One possible solution is to pre-schedule the UEs in the primary cell, and share the information with coordinated cells for precoding calculation in the next slot.

\subsection{Network Layer}
\subsubsection{Routing}
Due to the mobility in heterogeneous NTNs, it is difficult to maintain stable point-to-point LoS backhaul transmission between the NTN platforms and the core network on the ground. Therefore, the integrated access and backhaul (IAB) technology has been regarded as a promising solution for more robust backhaul transmission \cite{Lin2024}. More importantly, the Backhaul Adaptation Protocol (BAP) layer is introduced in DU for routing, which makes it possible to achieve multi-hop backhaul transmission. However, the dynamic topology and heterogeneous communication capabilities make it challenging to find the optimal routing path. Therefore, the coordinated routing algorithm should be developed by considering both the local cell's status and the neighboring cells' status. Another challenge in the backhaul routing via IAB is the channel capacity of the aerial communication. Different from traditional access links, the IAB link transmits the cell-level traffic between two NTNs via an access link, which has a much higher capacity requirement. However, the propagation path is usually dominated by LoS in aerial links, which limits the application of spatial multiplexing to enhance channel capacity. One promising solution is the multi-connectivity technology to enable distributed intelligent backhaul traffic allocation on multiple routing paths. The routing table optimization is classified as a delay-tolerant function, which aims to keep the traffic relatively stable over a period of time and collect enough observations before making a new decision.

\subsection{Application Layer}
\subsubsection{Trajectory Control}
Different from the terrestrial network, heterogeneous NTNs are required to be controlled to optimize the trajectories for collaboratively providing optimal coverage in large-scale areas. Although the satellites have fixed orbits based on their altitude, the satellites still have the capability to change the trajectory with a longer response time at a higher cost, e.g., in emergency communications. Therefore, due to the distinct mobility characteristics of heterogeneous NTNs, the multiple-timescale trajectory optimization algorithm should be developed to enable coordination among NTNs for optimal coverage. More importantly, the received quality of control data over wireless backhaul transmission significantly impacts the trajectory update (e.g., latency of control signal), and then degrades the performance of access links to the terrestrial UEs. Therefore, it remains an important challenge to investigate the closely coupled control signal transmission and user data transmission.

\section{Distributed Learning for Coordinated Heterogeneous Non-Terrestrial Networks}
It is noted that centralized coordination is challenging for heterogeneous NTNs due to dynamic topology and large delay. Therefore, in this section, we further illustrate the emerging MADRL to facilitate distributed coordination above, where the centralized training and decentralized execution (CTDE) framework is adopted due to higher convergence performance \cite{Chen2021}. Based on the delay requirements in Table.~\ref{challenges_solutions}, the NTN-based MADRL solutions tailored for the delay-tolerant distributed coordination and delay-sensitive distributed coordination as follows, respectively:

\subsection{Delay-tolerant Distributed Coordination}
According to the O-RAN structure, the near real-time (near-RT) RAN intelligent controller (RIC) deployed in the CU can typically support the RAN control with latency from 10ms to 1s, and the non-RT RIC deployed in the core network can support the RAN control with latency over 1s. Therefore, for delay-tolerant distributed coordination, the centralized server during training can be deployed in CU either onboard or on the ground, or even in the core network. 
\begin{itemize}
    \item Cell Shaping: To cooperatively provide optimal coverage in initial access, the MADRL algorithm should be developed to select the optimal SSB codebook for each cell considering the UEs distribution and the capabilities of neighboring cells. One potential solution is to integrate the Graph Neural Network (GNN), which models heterogeneous NTNs as nodes with various features. Apart from that, the UEs associated with heterogeneous NTNs have distant transmission power capabilities, which limits their connections to upper-tier NTNs. Therefore,  a lightweight model should also be deployed at the UE to make the association decision based on local characteristics including transmission power and battery. %\textcolor{red}{Same question as the physical layer: do we need to change the description of cell shaping to be initial access?} Updated
    \item Pilot Resource Allocation: To mitigate the uplink pilot interference in the heterogenous NTNs, the UE-level coordinated pilot resource allocation problem should be solved by the MADRL, where multiple cells learn to cooperatively minimize the channel correlation of UEs sharing the same pilot resource. However, UE-level allocation may increase the observation and action space significantly in MADRL, especially for higher-tier NTNs. Therefore, hierarchical MADRL can be utilized to divide the coverage into several low-correlated smaller areas first, and then focus on the pilot resource allocation within each smaller area. To reduce the communication overhead during centralized training, channel charting can be a potential solution by extracting the low-dimensional features from the statistical channels \cite{Ferrand2021}.
     \item Multi-hop IAB: Although the training of the routing optimization algorithm is delay-tolerant, the user data transmission may have stringent latency requirements. Therefore, multi-hop transmission of duplicate user data leads to increased transmission latency and channel capacity demands. To solve the problem, the hierarchical goal-oriented semantic communication can be integrated with the MADRL to exploit the context of data and its importance to the task along with the latest latency, which extracts the most important data for transmission in each hop \cite{zhou2024}. 
     \item Multi-timescale Control: In the multi-timescale trajectory optimization problem, the NTNs in different tiers experience different timescales of observation, action, and reward in MADRL. Hence, a potential solution is to decouple the formulated MADRL problem based on the timescales, and then optimize the subproblems iteratively \cite{Liu2024}. It is noted that goal-oriented semantic communication can also be utilized to quantify the importance of control signal transmission, which is closely coupled with the performance of the access links.
\end{itemize}

\subsection{Delay-sensitive Distributed Coordination}
It is noted that the cellular network has stringent numerology and frame structure, where most of the physical layer and MAC layer procedures need to be completed within every slot length. To satisfy the stringent latency requirement during coordination, the central server during training should be deployed in the DU, and a single DU should connect to multiple RUs for coordination. In this case, more functions in the physical layer and MAC layer are required to be moved to the RU.
\begin{itemize}
    \item UL/DL Power Allocation: In this case, each NTN is required to deploy two heterogeneous agents to optimize the UL and DL power allocation, respectively. Specifically, the first agent detects the occurrence of CLI or RI based on the received signal and increases the uplink transmission power correspondingly. Then, the second agent optimizes the downlink transmission power based on the positions of neighboring cells to mitigate causing CLI or RI. It is noted that two agents of a single NTN work sequentially in the TDD frame structure, and same type of agents in different heterogeneous NTNs may work in an asynchronous way due to dynamic TDD. 
     \item Coordinated Scheduling: To reduce the inter-cell interference in the MAC layer, each agent should learn to cooperatively schedule the UEs with minimal channel correlation on the same resource block. In this case, the reward in MADRL can be designed as the sum capacity in the scenario. However, to further support CJT scheme in DMIMO, the MADRL algorithm will incorporate the distributed scheduling consensus constraint, where the coordinated cells should schedule the cell-edge UEs on the same resource block for precoding.
     \item Distributed MIMO: Since the heterogeneous NTNs are equipped with a distinct number of antennas, the observation space (e.g., dimensions of the estimated channel) and action (e.g., dimensions of the precoding matrix) space of different agents vary from each other, the distributed MADRL algorithm with heterogeneous agents including MADDPG should be investigated. Considering the channel sharing overhead, to fully enable DMIMO precoding without channel sharing, one possible solution is to perform channel charting for cell-edge UEs via multiple cells in advance. Then, each NTN reconstructs the complete channel for cell-edge UEs based on the local estimated channel and channel charting map.%\textcolor{red}{(This paragraph has something in common with the challenges about precoding. Maybe we can keep one of them?)}updated
\end{itemize}

\section{Case Study}
In this section, to validate the effectiveness of our proposed MADRL solutions in dealing with heterogeneous NTN challenges, we present the MADDPG method tailored for the user scheduling task in heterogeneous UAV networks in an emergency communication scenario. Specifically, we consider one tethered UAV (i.e., IAB donor) and four untethered UAVs (i.e., IAB node) are deployed in the disaster areas with a radius of 500m for downlink transmission, where the untethered UAVs achieve the backhaul transmission based on IAB  via tethered UAV. Without loss of generality, we assume the traffic of each terrestrial user follows a Poisson process with parameter $\lambda = 4$. 

To maximize the downlink transmission throughput via distributed scheduling coordination, we propose a MADDPG algorithm for user scheduling based on the CTDE framework, where each UAV agent learns the optimal scheduling policy with local observations. It is noted that the observation of each agent consists of buffer status, historical SINR status towards its associated users, and the action space is the index of the associated users.

\begin{figure}[!ht]
    \centering
    \subfigure[Average downlink throughput during training.]
    {\includegraphics[scale=0.5]{Figures/each_UAV_new.pdf}} 
    \centering
    \subfigure[Downlink throughput of each UAV.]
    {\includegraphics[scale=0.5]{Figures/each_UAV.pdf}}
    \caption{Average downlink transmission throughput during training and throughput of each UAV.}
    \label{MADDPG_performance}
\end{figure}

Fig.~\ref{MADDPG_performance} presents the average downlink transmission throughput during training and throughput UAV. In Fig. \ref{MADDPG_performance} (a), we can observe that the deployment of untethered UAVs significantly improves the performance due to the enhancements to the cell-edge users. We can also see that the proposed MADDPG algorithm further achieves 37\% performance gain compared with the round-robin scheduling policy. This is because the IAB links will get higher priority to be scheduled by the tethered UAV, which serves as the backhaul transmission of untethered UAVs. In Fig. \ref{MADDPG_performance} (b), we can observe that the downlink throughput of tethered UAV degrades around 16\%, and the downlink throughput of each untethered UAV increases significantly, which validates the feasibility of MADRL for solving the challenges in heterogeneous NTNs.
	
\section{Conclusion}
In this article, we first summarize the unique characteristics of different NTN platforms, and analyze their impact on the implementation of NTNs. We then focus on investigating
the communication challenges for heterogeneous NTNs in individual layers, along
with their potential coordinated solutions. Based on the emerging distributed algorithm,
we further investigate the distributed MADRL algorithms tailed for each potential coordinated solution in heterogeneous NTNs, and analyze the CTDE deployment options according to their latency requirement.
Importantly, we utilize MADDPG to solve the user scheduling optimization problem in heterogeneous UAVs-based wireless networks, which validates the effectiveness of distributed learning for heterogeneous NTNs.
	%This work serves as an attempt toward customizing TOSA communication solutions
	%that consider both the semantic level information and effectiveness-aware performance metrics in future network architecture.
	
	
	
	%\section*{Publications}
	%\begin{enumerate}
	%	\item $ \mathbf{H.\ Zhou} $, F. Hu, M. Juras, A. Mehta, and Y. Deng, "Real-time Video Streaming and Control of Cellular-connected UAV Testbed," submitted to IEEE Globecom, 2020. Under review
	%\end{enumerate}
	% if have a single appendix:
	%\appendix[Proof of the Zonklar Equations]
	% or
	%\appendix  % for no appendix heading
	% do not use \section anymore after \appendix, only \section*
	% is possibly needed
	% use appendices with more than one appendix
	% then use \section to start each appendix
	% you must declare a \section before using any
	% \subsection or using \label (\appendices by itself
	% starts a section numbered zero.)
	%
	
	
	%%%%%\appendices
	
	
	% you can choose not to have a title for an appendix
	% if you want by leaving the argument blank
	%\section{}
	
	
	
	% use section* for acknowledgement
	%\section*{Acknowledgment}
	%{\color{red}This work was supported by the US National Science Foundation (NSF) through grant MCB-1449014, and the NSF Nebraska EPSCoR through the First Award grant EPS-1004094.}
	
	%The authors would like to thank...
	
	
	% Can use something like this to put references on a page
	% by themselves when using endfloat and the captionsoff option.
	\ifCLASSOPTIONcaptionsoff
	\newpage
	\fi
	
	
	
	% trigger a \newpage just before the given reference
	% number - used to balance the columns on the last page
	% adjust value as needed - may need to be readjusted if
	% the document is modified later
	%\IEEEtriggeratref{8}
	% The "triggered" command can be changed if desired:
	%\IEEEtriggercmd{\enlargethispage{-5in}}
	
	
	
	
	% references section
	
	% can use a bibliography generated by BibTeX as a .bbl file
	% BibTeX documentation can be easily obtained at:
	% http://www.ctan.org/tex-archive/biblio/bibtex/contrib/doc/
	% The IEEEtran BibTeX style support page is at:
	% http://www.michaelshell.org/tex/ieeetran/bibtex/
	% \bibliographystyle{IEEEtran}
        \bibliographystyle{IEEEtran}
	\bibliography{IEEEabrv,ref}
	%	
	% \vskip -2\baselineskip plus -1fil
	
	% %	\begin{IEEEbiographynophoto}{Hui Zhou}
	% 	%		is currently a Ph.D. student in the Center for Telecommunications  Research (CTR), King's College London.
	% 	%	\end{IEEEbiographynophoto}
	
	
	
	% \vskip -2\baselineskip plus -1fil
	
	%	\begin{IEEEbiographynophoto}{Yansha Deng} 
		%		is currently a  senior lecturer in the  CTR, King’s College London. Her research interests include machine learning for 5G/B5G wireless networks and molecular communications. 
		%	\end{IEEEbiographynophoto}
	%	
	
	%
	% <OR> manually copy in the resultant .bbl file
	% set second argument of \begin to the number of references
	% (used to reserve space for the reference number labels box)
	%
	%\begin{comment}
	%	\begin{thebibliography}{1}
		
		%	\bibitem{IEEEhowto:kopka}
		%	H.~Kopka and P.~W. Daly, \emph{A Guide to \LaTeX}, 3rd~ed.\hskip 1em plus
		%	0.5em minus 0.4em\relax Harlow, England: Addison-Wesley, 1999.
		
		%	\end{thebibliography}
	%\end{comment}
\end{document}





