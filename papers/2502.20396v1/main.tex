\documentclass[conference]{IEEEtran}
\usepackage[T1]{fontenc} % Add this line
\usepackage[utf8]{inputenc} % Ensure UTF-8 input encoding

\usepackage{times}

% numbers option provides compact numerical references in the text. 
\usepackage[numbers]{natbib}
\usepackage{multicol}
\usepackage[bookmarks=true]{hyperref}
\usepackage{graphicx}
\usepackage{caption}
\usepackage{capt-of}


\usepackage{amsmath} % assumes amsmath package installed
\usepackage{amssymb}  % assumes amsmath package installed
\usepackage{balance}
\usepackage{times}
\usepackage{dsfont}
\usepackage{siunitx}
\usepackage{cleveref}
\usepackage{algorithm}% http://ctan.org/pkg/algorithms
\usepackage{algpseudocode}% http://ctan.org/pkg/algorithmicx
\usepackage{algorithmicx}
\usepackage{xspace}
\usepackage{xurl}
\usepackage[usenames,dvipsnames,table,xcdraw]{xcolor}
\definecolor{lightgray}{gray}{0.9}
\usepackage{booktabs}  % For better-looking horizontal lines
\usepackage{float} % Include this in the preamble
% \usepackage[noadjust]{cite}
\usepackage[flushleft]{threeparttable}
\usepackage{amsmath,amssymb,amsfonts}
\usepackage{arydshln} % Add this to the preamble

\usepackage{stfloats}

\definecolor{mygreen}{RGB}{0 205 0}
\definecolor{mybrown}{RGB}{139 69 19}

\hypersetup{
	colorlinks=true,
	linkcolor=blue,
	urlcolor=mybrown,
	citecolor=mygreen,
}

\newcommand{\paragraphc}[1]{\vspace{0.2em}\noindent\textbf{#1}}
\newcommand{\paragrapht}[1]{\vspace{0.2em}\noindent\textit{#1}}


\pdfinfo{
   /Author (Toru Lin)
   /Title  (Sim-to-Real Reinforcement Learning for Vision-Based Dexterous Manipulation on Humanoids)
   /CreationDate (D:20101201120000)
   /Subject (Robots)
   /Keywords (Reinforcement Learning; Dexterous Manipulation; Humanoids)
}

\begin{document}

\let\oldtwocolumn\twocolumn
\renewcommand\twocolumn[1][]{%
\oldtwocolumn[{#1}{
\vspace{-1em}
\begin{center}
    \includegraphics[width=\linewidth]{figures/fig1.pdf}
    \captionof{figure}{\textbf{Overview.} We train a humanoid robot with two multi-fingered hands to perform a range of contact-rich dexterous manipulation tasks on various objects. Observations are obtained from a third-view camera, an egocentric camera, and robot proprioception. The deployed reinforcement learning policies can adapt to a variety of unseen real-world objects that have varying physical properties (e.g., shape, size, color, material, mass) and remain robust against force disturbances.
    }
    \label{fig:teaser}
\end{center}
}]
}

\title{{\Huge Sim-to-Real Reinforcement Learning for \\ Vision-Based\hspace{-3pt} Dexterous\hspace{-3pt} Manipulation\hspace{-3pt} on\hspace{-3pt} Humanoids}}

\author{
\authorblockN{Toru Lin${}^{1,2}$, Kartik Sachdev${}^{2}$, Linxi ``Jim'' Fan${}^{2}$, Jitendra Malik${}^{1}$, Yuke Zhu${}^{2,3}$}
\vspace{0.1em}
\authorblockA{UC Berkeley${}^{1}$ \quad NVIDIA${}^{2}$ \quad UT Austin${}^{3}$\\
}
\vspace{0.1em}
{\texttt{\url{https://toruowo.github.io/recipe}}}
\vspace{-1em}
}

\maketitle

\begin{abstract}
Reinforcement learning has delivered promising results in achieving human- or even superhuman-level capabilities across diverse problem domains, but success in dexterous robot manipulation remains limited. This work investigates the key challenges in applying reinforcement learning to solve a collection of contact-rich manipulation tasks on a humanoid embodiment. We introduce novel techniques to overcome the identified challenges with empirical validation. Our main contributions include an automated real-to-sim tuning module that brings the simulated environment closer to the real world, a generalized reward design scheme that simplifies reward engineering for long-horizon contact-rich manipulation tasks, a divide-and-conquer distillation process that improves the sample efficiency of hard-exploration problems while maintaining sim-to-real performance, and a mixture of sparse and dense object representations to bridge the sim-to-real perception gap. We show promising results on three humanoid dexterous manipulation tasks, with ablation studies on each technique. Our work presents a successful approach to learning humanoid dexterous manipulation using sim-to-real reinforcement learning, achieving robust generalization and high performance without the need for human demonstration.
\end{abstract}

\IEEEpeerreviewmaketitle

\section{Introduction}
\label{sec:intro}
Deep reinforcement learning (RL) has delivered a number of impressive results during recent years, covering a diverse range of application domains: classical board games \cite{silver2017mastering}, competitive multiplayer online games \cite{berner2019dota,vinyals2019grandmaster}, large language models \cite{openai2024,deepseekai2025deepseekr1incentivizingreasoningcapability}, real-world robotic locomotion \cite{hwangbo2019learning,lee2020learning}, autonomous drone racing \cite{kaufmann2023champion} --- to name a few. These accomplishments have not only showcased RL's potential to achieve or surpass human-level performance across various tasks but also highlighted its distinctive ability to scale and generalize via autonomous exploration. Such inherent characteristics position RL as a performant and long-term approach to tackling tasks that are difficult to solve with human priors or demonstrations.

Motivated by its potential, we explore RL to address challenging dexterous manipulation tasks from vision. The successes that deep RL has produced in this problem domain remain limited so far. Previous works have demonstrated highly dexterous manipulation capabilities that could not be simply programmed or teleoperated by humans \cite{akkaya2019solving, handa2023dextreme, lin2024twisting}. However, these approaches are often tailored to a single manipulation skill, limiting their broad applicability.

What prevents RL from being more generally applicable to vision-based dexterous manipulation? We first investigate this by identifying the inherent properties of dexterous manipulation that differentiate this application domain from others. Then, we examine how these properties contribute to challenges in applying RL algorithms and develop a collection of novel techniques to address the challenges. Putting together our experiences and techniques, we outline a recipe for applying sim-to-real RL to vision-based humanoid manipulation tasks and show promising results. Below, we articulate the key challenges and our strategies to tackle them.

\paragraphc{Challenge in environment modeling.} The first challenge in applying RL to dexterous manipulation lies in the difficulty (or impossibility) of matching a simulated environment with the real environment. While real-world RL circumvents this problem, training with physical hardware is highly demanding regarding hardware quality, maintenance support, controller robustness, and safety. With a system as high-dimensional as a humanoid with multi-fingered hands, real-world exploration becomes even less tractable. In contrast, simulations offer unlimited chances of trial and error in a virtual sandbox, motivating the development of sim-to-real RL approaches. While previous successes in RL-based locomotion~\cite{haarnoja2024learning,hwangbo2019learning,lee2020learning,RealHumanoid2023} are encouraging, we observe that previous successes in dexterous manipulation involve much more laborious real-to-sim engineering efforts that are task-specific or hardware-specific ~\cite{akkaya2019solving,handa2023dextreme,lin2024twisting}. To better align simulation with the real world, we propose an automated real-to-sim tuning module that substantially reduces the engineering effort required for the environment modeling gap.

\paragraphc{Challenge in reward design.} While the reward function is commonly used as a general interface for specifying a task to train RL policies, it is notoriously hard to design generalizable rewards for manipulation tasks, especially for those that are contact-rich or long-horizon. Prior work often resorts to hand engineering based on the knowledge of human experts \cite{petrenko2023dexpbt,zakka2023robopianist}, which has limited scalability in the long run. This challenge differentiates manipulation from locomotion, where many tasks of interest can be specified with variations of the reward for a single ``walking'' task. We propose a general principle to design rewards for dexterous manipulation tasks: disentangle a full task into intermediate ``contact goals'' and ``object goals''. We use a novel keypoint-based state representation to specify contact goals. Following our reward design techniques, a task as long-horizon and contact-rich as bimanual handover can be learned with RL \textit{tabula rasa}.

\paragraphc{Challenge in policy learning.} A well-defined reward function does not guarantee the successful learning of RL policies due to the sample complexity and reward sparsity of exploring in a high-dimensional space. The variety and complexity of contact patterns in dexterous manipulation with multi-fingered hands further exacerbate the problem. Although unsupervised methods \cite{burda2018exploration,lin2024mimex,pathak2017curiosity} have been proposed to encourage exploration by favoring novel state visitations, they do not fundamentally reduce the difficulty of hard-exploration problems. We tackle this challenge by introducing two practical techniques: (1) initializing tasks with task-aware hand poses; (2) breaking down hard exploration problems into sub-tasks with much-reduced dimensionality, training expert policies on the sub-tasks, then distilling them into a generalist policy for the full task. We experimentally verify that these techniques improve sample efficiency of learning and study how different divide-and-conquer schemes vary in effectiveness.

\paragraphc{Challenge in object perception.} Compared to other robotic tasks, object perception is particularly important for manipulation because the task is inevitably coupled with interaction with objects. Object perception is a long-standing challenge because the variety of objects is uncountable in shapes, sizes, masses, colors, textures, and many other properties. Research in applying sim-to-real RL to dexterous manipulation is bottlenecked by this dilemma --- while object representations that are more expressive and information-dense can improve dexterity and capability of the learned policy, they also present a larger sim-to-real gap. To overcome this challenge, we propose to use a mixture of low-dimensional and high-dimensional object representations, with modality-specific data augmentation on the high-dimensional features to reduce the sim-to-real perceptual gap. We systematically study how this combination could help achieve a good balance between learning dexterous manipulation policy and reliably transferring the policy onto real robot hardware.

The strategies we outline above form a complete recipe of sim-to-real RL for vision-based dexterous manipulation. We show successful results of learning a collection of three dexterous manipulation tasks on humanoids and conduct systematic ablation studies.

\begin{figure*}[t]
\begin{center}
\includegraphics[width=\textwidth]{figures/fig2.pdf}
\end{center}
\caption{\textbf{A sim-to-real RL recipe for vision-based dexterous manipulation.} We close the environment modeling gap between simulation and the real world through an automated real-to-sim tuning module, design generalizable task rewards by disentangling each manipulation task into contact states and object states, improve sample efficiency of dexterous manipulation policy training by using task-aware hand poses and divide-and-conquer distillation, and transfer vision-based policies to the real world with a mixture of sparse and dense object representations.}
\label{fig:overview}
\vspace{-1em}
\end{figure*}


\section{Background}

\subsection{Deep Reinforcement Learning Applications to Robotics}

The successes of deep RL across a wide range of applications~\cite{openai2024,berner2019dota,deepseekai2025deepseekr1incentivizingreasoningcapability,hwangbo2019learning,kaufmann2023champion,lee2020learning,silver2017mastering,vinyals2019grandmaster} have sparked lots of excitement in recent years. However, works over the years have identified brittleness with this paradigm, most notably the sensitivity to hyperparameters~\cite{henderson2018deep} and questionable reproducbility~\cite{islam2017reproducibility} due to the high variance intrinsic to RL algorithms.

Among the open problems in RL, the most important and long-standing is exploration. In supervised learning, it is often assumed that data is given. In RL, however, agents need to collect their own data and update their policy based on the collected data. The problem of \textit{how} data is collected is known as the exploration problem.
Real-world robotics, with high-dimensional observations and dynamics and often sparse rewards, present a particularly challenging set of hard exploration problems for RL. While there have been works that algorithmically scale exploration to high-dimensional inputs by encouraging visitation to novel states~\cite{bellemare2016unifying,burda2018exploration,lin2024mimex,ostrovski2017count,pathak2017curiosity,stadie2015incentivizing, tang2017exploration}, they do not fundamentally resolve the exploration challenge.
Additionally, applying RL to solve real-world robotics also reveals important challenges that standard benchmarks in RL~\cite{bellemare2013arcade,tassa2018deepmind} fail to capture: (1) the lack of fully or accurately modeled environments; (2) the lack of well-defined reward functions for tasks of interest.

Past works in the intersection of robotics and RL have proposed various practical techniques to alleviate these problems, such as learning from human motion data or teleoperated demonstrations~\cite{chen2024object,rajeswaran2017learning,yin2025dexteritygen,zhu2018reinforcement}, real-to-sim techniques to model object and visual environments~\cite{akkaya2019solving,haarnoja2024learning,handa2023dextreme,lin2024twisting,torne2024rialto} and more principled ways to design rewards~\cite{memmel2024asid,zhang2024wococo}. While some of them overfit to specific tasks and settings, they point to promising directions upon which this work builds.

\subsection{Vision-Based Dexterous Manipulation on Humanoids}

\paragraphc{Imitation learning and classical approaches.}
Innovations in teleoperation~\cite{cheng2024open,lin2024learning,wu2023gello,zhao2023learning} and learning from demonstrations~\cite{chi2023diffusion,li2024planning} have brought about many recent advances in vision-based dexterous manipulation~\cite{cheng2024open,li2024planning,lin2024learning,zhao2024aloha}. However, in practice, onboarding teleoperators to collect high-quality dexterous manipulation data remains costly, and the performance scaling with data purely collected from real-world teleoperation~\cite{levine2018learning,lin2024data,zhao2024aloha} suggests that the cost to reach human-level performance could be prohibitively large.

\paragraphc{Reinforcement learning approaches.} A number of existing works have successfully applied RL to solve dexterous manipulation problems with multi-fingered hands, but either assume a single-hand setup~\cite{akkaya2019solving,chen2023sequential,handa2023dextreme,lum2024dextrah,qi2023general,singh2024dextrah,wang2024lessons} or do not use pixel inputs as object representation~\cite{chen2024object,huang2023dynamic,lin2024twisting}. Moreover, most of the existing works focus on a single manipulation skill, including in-hand reorientation~\cite{akkaya2019solving,handa2023dextreme,qi2023general,wang2024lessons}, grasping~\cite{lum2024dextrah,singh2024dextrah}, twisting~\cite{lin2024twisting}, and dynamic handover~\cite{huang2023dynamic}. The closest to our work is Chen et al.~\cite{chen2024object}, but their method relies on human hand motion capture data to learn a wrist controller rather than learning the full hand-arm joint control from scratch.
In addition, existing works often focus on hardware whose models in physics simulation have been more extensively tested. Our work is the first to show successful sim-to-real RL transfer of vision-based dexterous manipulation policies to a novel humanoid hardware with multi-fingered hands.

\section{Challenges and Approaches}

In Section~\ref{sec:intro}, we identify four areas of challenges in applying sim-to-real RL to dexterous manipulation and briefly describe our strategies to tackle the challenge in each area. Below, we describe our specific approaches in detail. Figure~\ref{fig:overview} shows an overview of the challenges and approaches.


\subsection{Real-to-Sim Modeling}
\label{sec:realsim}

Simulators offer unlimited trial-and-error chances to perform the exploration necessary for RL. However, whether policies learned in simulation can be reliably transferred to the real world heavily depends on the faithfulness of modeling --- both the robot itself and the environment. When applying sim-to-real RL to solve dexterous manipulation, this real-to-sim modeling problem is further exacerbated by the necessity to model objects, which have great variability and whose full physical properties cannot be easily quantified. Even when one assumes that the ground-truth physical parameters are known, quantitatively matching the simulation with the real world is hard: due to the limitations of physics engines, the same values for physical constants in simulation and the real world do not necessarily correspond to identical kinematic and dynamic relationships.

\paragraphc{Autotuned robot modeling.} While robot manufacturers are often able to provide proprietary model files for their robot hardwares, the models mostly serve a starting reference for robot real-to-sim effort rather than ground truth models that can be used without modifications. Empirical solutions to increase modeling accuracy range from hand-tuning the robot model constants and simulatable physical parameters~\cite{akkaya2019solving} to re-formulating specific kinematic structures (e.g., four-bar linkage) in the simulator of choice~\cite{radosavovic2024real}. This is a laborious process as there is no ``ground truth'' pairing between the real world and the simulated world.
We propose a practical technique to speed up this real-to-sim modeling process via an ``autotune'' module. The autotune module enables rapid calibration of simulator parameters to match real robot behavior by automatically searching the parameter space to identify optimal values for both simulator physics and robot model constants in under four minutes (or 2000 simulated steps in \SI{10}{\Hz}).
We illustrate the module in Figure~\ref{fig:overview}A and Algorithm~\ref{alg:autotune}.
The module operates on two parameter types: simulator physics parameters affecting kinematics and dynamics, as well as robot model constants from the URDF file (including link inertia values, joint limits, and joint/link poses). The calibration process begins by initializing multiple simulated environments using randomly sampled parameter combinations from the parameter space, bootstrapped from the manufacturer's robot model file. It then executes $N$ calibration sequences consisting of joint position targets on both the real robot hardware (single run) and all simulated environments in parallel. By comparing the tracking error between each simulated environment and the real robot when following the same joint targets, the module selects the parameter set that minimizes the mean squared error in tracking performance. This approach eliminates iterative manual tuning by requiring only one set of calibration runs on the real robot, automatically optimizing traditionally hard-to-tune URDF parameters, and supporting parallel evaluation of multiple parameter combinations. The method's generality allows it to tune any exposed simulator or robot model parameter that affects kinematic behavior.

\paragraphc{Approximate object modeling.} As demonstrated in previous works~\cite{lin2024twisting,qi2023hand}, modeling objects as primitive shapes like cylinders with randomized parameters is sufficient for sim-to-real transferrable dexterous manipulation policies to be learned. Our recipe adopts this approach and finds it effective.

\begin{algorithm}[t]
\caption{Real-to-Sim Autotune Module}
\begin{algorithmic}[1]
\Require
    \State $E$ : Set of environment parameters to tune
    \State $N$ : Number of calibration action sequences
    \State $R$ : Real robot hardware environment
    \State $M$ : Initial robot model file
    
\Procedure{Autotune}{$E, N, R, M$}
    \State $P \gets \text{InitializeParameterSpace}(E, M)$ \Comment{Initialize from model}
    \State $S \gets \{\}$ \Comment{Set of simulated environments}
    
    \For{$i \gets 1$ to $K$}  \Comment{$K$ is population size}
        \State $p_i \gets \text{RandomSample}(P)$
        \State $S_i \gets \text{CreateSimEnvironment}(p_i)$
        \State $S \gets S \cup \{S_i\}$
    \EndFor

    \State $J \gets \text{GenerateJointTargets}(N)$ \Comment{Joint target sequences}
    \State $R_{track} \gets \text{GetTrackingErrors}(R, J)$ \Comment{Real tracking}

    \State $best\_params \gets \text{null}$
    \State $min\_error \gets \infty$
    
    \For{$S_i \in S$}
        \State $S_{track} \gets \text{GetTrackingErrors}(S_i, J)$
        \State $error \gets \text{ComputeMSE}(S_{track}, R_{track})$
        \If{$error < min\_error$}
            \State $min\_error \gets error$
            \State $best\_params \gets \text{GetParameters}(S_i)$
        \EndIf
    \EndFor
    
    \Return $best\_params$
\EndProcedure
\end{algorithmic}
\label{alg:autotune}
\end{algorithm}


\subsection{Generalizable Reward Design}
\label{sec:reward}

In the standard formulation of RL~\cite{sutton1998introduction}, the reward function is a crucial element within the paradigm because it is solely responsible for defining the agent’s behavior. Nevertheless, the mainstream of RL research has been preoccupied with the development and analysis of learning algorithms, treating reward signals as given and not subject to change~\cite{eschmann2021}. As tasks of interest become more general, designing reward mechanisms to elicit desired behaviors becomes more important and more difficult~\cite{dewey2014reinforcement} --- as is the case of applications to robotics. When it comes to dexterous manipulation with multi-fingered hands, reward design becomes even more difficult due to the variety of contact patterns and object geometries.

\paragraphc{Manipulation as contact and object goals.} From a wide variety of human manipulation activities~\cite{grauman2024ego}, we observe a general pattern in dexterous manipulation: each motion sequence to execute a task can be defined as a combination of hand-object contact and object states. Building on this intuition, we propose a general reward design scheme for even long-horizon contact-rich manipulation tasks. For each task of interest, we first break it down into an interleaving sequence of contact states and object states. For example, the handover task can be broken down into the following steps: (1) one hand contacting the object; (2) the object being lifted to a position near the other hand; (3) the other hand contacting the object; (4) object being transferred to the final goal position. The reward can then be defined based on solely the ``contact goals'' and ``object goals'': each contact goal can be specified by penalizing the distance from fingers to desirable contact points or simply the object's center-of-mass position; each object goal can be specified by penalizing the distance from its current state (e.g., current \textit{xyz} position) to its target state (e.g., target \textit{xyz} position). To reduce the difficulty of specifying contact goals, we propose a novel keypoint-based technique as follows: for each simulated asset, we procedurally generate a set of ``contact stickers'' attaching to the surface of the object, where each sticker represents a potentially desirable contact point. The contact goal, in terms of reward, can then be specified as 

\begin{equation}
    r_{\mathrm{contact}} = \sum_{i}\left[\frac{1}{1+\alpha d(\mathbf{X}^L, \mathbf{F}^L_i)} + \frac{1}{1+\beta d(\mathbf{X}^R, \mathbf{F}^R_i)}\right],
\end{equation}
where $\mathbf{X}^L\in\mathbb{R}^{n\times 3}$ and $\mathbf{X}^R\in\mathbb{R}^{m\times 3}$ are the positions of contact markers specified for left and right hands, $\mathbf{F}^L\in\mathbb{R}^{4\times 3}$ and $\mathbf{F}^R\in\mathbb{R}^{4\times 3}$ are the position of left and right fingertips, $\alpha$ and $\beta$ are scaling hyperparameters, and $d$ is a distance function defined as
\begin{equation}
    d(\mathbf{A}, \mathbf{x}) = \min_i \Vert\mathbf{A}_i - \mathbf{x}\Vert_2.
\end{equation}
We show a visualization of contact markers in Figure~\ref{fig:overview}B, and experimental results on their effectiveness in Section~\ref{sec:exp}.

\subsection{Sample Efficient Policy Learning}
\label{sec:policy}

Due to the sample complexity and reward sparsity in exploring a high-dimensional space --- especially on a humanoid embodiment with multi-fingered hands --- policy learning can take a prohibitively long time, even with a well-defined reward function. We propose two techniques that more effectively improve the sample efficiency of policy learning: (1) initializing tasks with task-aware hand poses; (2) dividing a challenging task into easier sub-tasks, then distilling the sub-task specialists into a generalist policy.

\paragraphc{Task-aware hand poses for initialization.} We reduce the exploration challenge by collecting task-aware hand pose data from humans. This can be done by connecting any teleoperation system for bimanual multi-fingered hands to the simulator of choice. The collected states, including object poses and robot joint positions, are then randomly sampled as task initialization states in simulation. Distinct from prior works that require full demonstration trajectories \cite{bauza2024demostart}, we find that teleoperators need not accomplish the task and only need to ``play around'' with the task goal in mind while environmental states are collected. The approach massively reduces the time needed for teleoperation since human operators do not need to spend time ``ramping up'' to collect high-quality data. In our experiments, each task requires less than 30 seconds for sufficient amount of hand pose data to be collected.

\paragraphc{Divide-and-conquer distillation.} Previous methods to improve sample efficiency of policy learning mostly focus on exploring the state space more efficiently \cite{burda2018exploration,lin2024mimex,pathak2017curiosity,taiga2019benchmarking}. However, these methods do not reduce the difficulty of the exploration problem fundamentally: the probability of receiving learning signals from exploring the ``right'' states remains the same. Following this observation, we reason that an easier way to overcome the exploration problem in sparse reward settings is to break down the explorable state space itself. For example, a multi-object manipulation task can be divided into multiple single-object manipulation tasks. After dividing a complex task into easier sub-tasks, we can train specialized policies for each sub-task and distill them into a generalist policy. Another benefit of this approach is that we can flexibly filter out trajectory data from the sub-task policies based on their optimality and only retain high-quality samples for training. This effectively brings RL closer to learning from demonstrations, where the sub-task policies act as teleoperators for task data collection in the simulation environment, and the generalist policy acts as a centralized model trained from curated data.

\subsection{Vision-Based Sim-to-Real Transfer}
\label{sec:vis}

Transferring a policy learned in simulation to the real world is challenging because of the sim-to-real gap. In the case of vision-based dexterous manipulation, the sim-to-real gap stems from both dynamics and visual perception --- both are challenging open research problems to solve. We outline two key techniques we employ to reduce the gap.

\paragraphc{Mixing object representations.} Object perception is crucial for dexterous manipulation because the task is inevitably coupled with object interaction. Prior works that show successful sim-to-real transfer of manipulation policies have explored a wide range of object representations, including (in order of increasing dimensionality and complexity) 3D object position \cite{lin2024twisting}, 6D object pose \cite{akkaya2019solving}, depth \cite{lum2024dextrah,qi2023general}, point cloud \cite{liu2024visual}, and RGB images \cite{handa2023dextreme}. There is a delicate trade-off between using these different object representations: while higher-dimensional representations encode richer information about the object, the larger sim-to-real gap in those data modalities makes the learned policy harder to be transferred; on the other hand, it is harder to learn optimal policies with lower-dimensional object representations because of the limited amount of information. We, therefore, propose a combination of both types of object representations to balance the trade-offs: a low-dimensional 3D object position and a high-dimensional depth image. Importantly, the 3D object position is obtained from a third-view camera to ensure the object is also in camera view and its noisy position can be consistently tracked. The depth image complements with information on object geometry. We provide more empirical validation on the effectiveness of this approach in Section~\ref{sec:exp}.

\paragraphc{Domain randomization for dynamics and perception.} We apply a wide range of domain randomizations to ensure reliable sim-to-real transfer. We list the details in Appendix~\ref{sec:dr}.

\section{Experiments}
\label{sec:exp}

Our proposed approaches form a general recipe that allows for the practical application of RL to solve dexterous manipulation with humanoids. In this section, we show experimental results of task capabilities and ablation studies of each proposed technique. Videos can be found on project website.

\subsection{Real-World and Simulator Setup}
We use a Fourier GR1 humanoid robot with two arms and two multi-fingered hands. Each arm has 7 degrees of freedom (DoF). For most experiments, we use the Fourier hands, each of which has six actuated DoFs and five underactuated DoFs. To show cross-embodiment generalization, we include results on the Inspire hands, each with 6 actuated DoFs and six underactuated DoFs. The hardware has substantially different mass properties, surface frictions, finger and palm morphologies, and thumb actuations. Figure~\ref{fig:teaser} shows a visualization of both robot hands.
We use the NVIDIA Isaac Gym simulator~\cite{makoviychuk2021isaac}.

\paragraphc{Perception.} As outlined in Section~\ref{sec:vis}, we use a combination of dense and sparse object features for policy learning in both simulation and real-world transfer. In the real world, we set up an egocentric-view camera by attaching a RealSense D435 depth camera onto the head of a humanoid robot and a third-view camera by putting another RealSense D435 depth camera on a tripod in front of the robot (illustrated in Figure~\ref{fig:teaser}). In simulation, we similarly set up an egocentric-view camera and a third-view camera by calibrating the camera poses against the real camera poses. The \textit{dense} object feature is obtained by directly reading depth observations from the egocentric-view camera. The \textit{sparse} feature is obtained by approximating the object's center-of-mass from the third-view camera, using a similar technique as in Lin et al.~\cite{lin2024twisting}. As illustrated in Figure~\ref{fig:overview}, we utilize the Segment Anything Model 2 (SAM2)~\cite{ravi2024sam} to generate a segmentation mask for the object of interest at the first frame of each trajectory sequence and leverage the tracking capabilities of SAM2 to track the mask throughout all remaining frames. To approximate the 3D center-of-mass coordinates of the object, we calculate the center position of their masks in the image plane, then obtain noisy depth readings from a depth camera to recover a corresponding 3D position. The perception pipeline runs at \SI{5}{\Hz} to match the neural network policy's control frequency.

\subsection{Task Definition}
\label{sec:task}

\paragraphc{Grasp-and-reach.} In this task, the robot needs to use one hand to grasp a tabletop object, lift it up, and move it to a goal position. At task initialization, a scripted visual module determines which hand is closer to the object and instructs the robot to use that hand. The test objects have varying geometric shapes, masses, volumes, surface frictions, colors, and textures; a visualization of all objects can be found in Figure~\ref{fig:teaser}. For each trial, we vary the initial position and orientation of objects, as well as the goal position.

\paragraphc{Box lift.} In this task, the robot needs to lift a box that is too large to be grasped with a single hand. Box colors, sizes, and masses are varied. For each trial, we randomize the boxes' initial position and orientation about the vertical axis.

\paragraphc{Bimanual handover.} In this task, the robot needs to grasp a block object from one side of the table with one hand --- that is too far to reach for the other hand --- and hand the object over to the other hand. Objects include blocks of varying colors, dimensions, and masses. We vary the initial position and orientation of blocks in each trial.


\begin{figure}[!t]
    \centering
    \vspace{-1em}
    \includegraphics[width=\linewidth]{figures/simtraj.pdf}
    \caption{\textbf{Policies learned in simulation.} Left: grasp; middle: box lift; right: bimanual handover (right-to-left, left-to-right).}
    \label{fig:sim}
\end{figure}

\subsection{Evaluation of Real-to-Sim Modeling}

\paragraphc{Effectiveness of autotuned robot modeling.}
We apply the autotune module outline in Section~\ref{sec:realsim} to find the optimal parameter set for robot modeling. To evaluate its effectiveness, we compare the sim-to-real transfer success rates of three sets of policy checkpoints. All policies are trained with identical task and RL settings, but each set of policies uses robot modeling parameters that achieve different MSE from autotune, ranging from lowest (i.e., smallest real-to-sim gap) to highest (i.e., highest real-to-sim gap). The quantitative results in Table~\ref{table:autotune} indicate that autotuned robot modeling improves sim-to-real transfer. Additionally, in our video, we show qualitative results of successful sim-to-real transfer of \texttt{grasp-and-reach} policies to the Inspire hands, demonstrating the generalizability of our autotune module.
\begin{algorithm}[t]
\caption{Real-to-Sim Autotune Module}
\begin{algorithmic}[1]
\Require
    \State $E$ : Set of environment parameters to tune
    \State $N$ : Number of calibration action sequences
    \State $R$ : Real robot hardware environment
    \State $M$ : Initial robot model file
    
\Procedure{Autotune}{$E, N, R, M$}
    \State $P \gets \text{InitializeParameterSpace}(E, M)$ \Comment{Initialize from model}
    \State $S \gets \{\}$ \Comment{Set of simulated environments}
    
    \For{$i \gets 1$ to $K$}  \Comment{$K$ is population size}
        \State $p_i \gets \text{RandomSample}(P)$
        \State $S_i \gets \text{CreateSimEnvironment}(p_i)$
        \State $S \gets S \cup \{S_i\}$
    \EndFor

    \State $J \gets \text{GenerateJointTargets}(N)$ \Comment{Joint target sequences}
    \State $R_{track} \gets \text{GetTrackingErrors}(R, J)$ \Comment{Real tracking}

    \State $best\_params \gets \text{null}$
    \State $min\_error \gets \infty$
    
    \For{$S_i \in S$}
        \State $S_{track} \gets \text{GetTrackingErrors}(S_i, J)$
        \State $error \gets \text{ComputeMSE}(S_{track}, R_{track})$
        \If{$error < min\_error$}
            \State $min\_error \gets error$
            \State $best\_params \gets \text{GetParameters}(S_i)$
        \EndIf
    \EndFor
    
    \Return $best\_params$
\EndProcedure
\end{algorithmic}
\label{alg:autotune}
\end{algorithm}

\paragraphc{Effectiveness of approximate object modeling.} Empirically, we find that modeling objects as primitive geometric shapes (cylinders, cubes, and spheres) strikes a good balance between training efficiency and sim-to-real transferability. In Figure~\ref{fig:objexp} (left), we compare the training curves of \texttt{grasp-and-reach} policies with primitive shapes against those with complex shapes, and the former setting is more sample efficient.
More importantly, policies trained with randomized primitive shapes can also generalize to a variety of unseen objects, as shown in our video.

\begin{figure}[!t]
    \centering
    % \vspace{-1em}
    \includegraphics[width=\linewidth]{figures/objexp.png}
    \caption{\textbf{Training \texttt{grasp-and-reach} policy with different object sets.} Each curve is computed from the statistics of 10 training runs with different random seeds. Left: training with complex objects v.s. simple geometric primitive objects. Right: training with differently grouped geometric objects.}
    \vspace{-0.1em}
    \label{fig:objexp}
\end{figure}

\subsection{Evaluation of Reward Design}

\paragraphc{Task capabilities.} With our proposed reward design principle, a wide range of long-horizon contact-rich tasks can be accomplished with pure reinforcement learning, as shown in Figure~\ref{fig:sim} and our video. The learned policies exhibit a high degree of dexterity and robustness against random force disturbances.

\paragraphc{Effectiveness of contact-based rewards.} In Figure~\ref{fig:contactexp}, we visualize different contact behaviors that emerged from different placements or contact markers, using the \texttt{box lift} task as an example. We find that contact behaviors align closely with the contact positions specified, showing the effectiveness of using contact markers to specify contact goals.


\begin{figure}[!t]
    \centering
    \vspace{-1em}
    \includegraphics[width=0.95\linewidth]{figures/contactexp.pdf}
    \caption{\textbf{Different contact patterns emerge from different placements of contact markers.} Top: contact markers on the left and right side centers; middle: markers on the top and bottom side centers; bottom: markers on the bottom side edges.}
    \vspace{-1.3em}
    \label{fig:contactexp}
\end{figure}

\subsection{Evaluation of Policy Learning}

% sample efficiency

\paragraphc{Effectiveness of task-aware hand pose initialization.} In Table~\ref{table:humaninit}, we compare the percentage of successfully trained policies for each task with and without task-aware hand pose initialization. The empirical results show that having human priors upon initialization can greatly improve the exploration efficiency of hard RL tasks.

\paragraphc{Divide-and-conquer distillation.} We evaluate our divide-and-conquer distillation approach through two ablation studies. First, we study the effect of divide-and-conquer granularity on training efficiency. Specifically, we experiment with dividing a multi-object \texttt{grasp-and-reach} task into sub-tasks that handle different numbers of objects. Starting from a total of 10 objects, we experiment with three task designs: (1) training with all objects in one policy (\textit{all}); (2) training with three groups of similarly shaped objects in three policies (\textit{shape}); (3) training with three groups of differently shaped objects in three policies (\textit{mix}); (4) training with ten single-object policies (\textit{single}). As shown in Figure~\ref{fig:objexp}, sample efficiency is highest for \textit{single}, followed by \textit{shape}, \textit{all}, and \textit{mix}. There is also a noticeable difference in the average success rates of each task, which could be explained as an indicator of task difficulty. Interestingly, while training with a reduced object set all reaches the same performance, training all objects in one policy shows a consistently lower performance. Second, we study the sim-to-real transfer success rate of each type of policy. Over 30 trials of each policy on an in-distribution object, we find that sim-to-real performance of the \texttt{mix} policy is the highest (90.0\%), followed by \texttt{shape} (63.3\%), \texttt{single} (40.0\%), and \texttt{all} (23.3\%). Based on the qualitative behavior of policies, we hypothesize that the low success rates of \texttt{mix} and \texttt{single} policies stem from overfitting to specific geometries, and that of \texttt{all} policy correlates with its worse performance during RL training. These results suggest that divide-and-conquer distillation helps achieve a good balance between policy training and sim-to-real transfer performance.


\begin{table}[t]
\centering
\setlength{\tabcolsep}{7pt}
\renewcommand{\arraystretch}{1.3}
\resizebox{0.85\linewidth}{!}{%
\begin{tabular}{rrrrr}
\toprule
{\% Success} & {Grasping} & {Lifting} & {Handover} \\
\cmidrule(r){1-1}
\cmidrule(l){2-4}
\cmidrule(lr){1-1} \cmidrule(lr){2-4} 
with Human Init & 80\% & 90\% & 30\% \\
w/o Human Init & 60\% & 90\% & 0\% \\
\bottomrule
\end{tabular}
}
\caption{\textbf{Initializing with human data.} Correlation between the percentage of successfully learned task policies and whether human play data is used for initialization. We define \textit{successfully learned} policies as those that achieve over 60\% episodic success during evaluation. For each task and each initialization setting, we test with 10 random seeds.}
\label{table:humaninit}
\vspace{-1em}
\end{table}

\subsection{Evaluation of Vision-Based Sim-to-Real Transfer}

\paragraphc{Effectiveness of mixing object representations.} We investigate the effect of using different object representations and show the sim-to-real transfer comparisons in Table~\ref{table:objrep}. These results suggest combining dense object representation (segmented depth image) and sparse object representation (3D object center-of-mass position) improves sim-to-real transfer. Notably, the gap between the success rate of the depth-and-position policy and that of the depth-only policy increases as knowledge of the full object geometry becomes more crucial to task success.

\begin{table}[!t]
\centering
\setlength{\tabcolsep}{7pt}
\renewcommand{\arraystretch}{1.3}
\resizebox{\linewidth}{!}{%
\begin{tabular}{rrrrr}
\toprule
{Task} & {Grasping} & {Lifting} & {HandoverA} & {HandoverB} \\
\cmidrule(r){1-1}
\cmidrule(l){2-5}

\multicolumn{5}{l}{\textbf{Depth + Pos}} \\
\cmidrule(lr){1-1} \cmidrule(lr){2-5} 
Pickup & 10 / 10 & 10 / 10 & 10 / 10 & 10 / 10 \\
Task Success & 10 / 10 & 10 / 10 & 9 / 10 & 5 / 10 \\

\cmidrule(lr){1-1} \cmidrule(lr){2-5} 
\multicolumn{5}{l}{\textbf{Depth Only}} \\
\cmidrule(lr){1-1} \cmidrule(lr){2-5} 
Pickup & 2 / 10 & 0 / 10 & 0 / 10 & 0 / 10 \\
Task Success & 2 / 10 & 0 / 10 & 0 / 10 & 0 / 10 \\

\bottomrule
\end{tabular}
}
\caption{\textbf{Comparison of sim-to-real transfer performance between depth-and-position policy and depth-only policy.} We separate the \texttt{bimanual handover} task into two columns due to its longer horizon. The pickup success is an intermediate success metric that measures how often the hands successfully pick up the object of interest. We find that combining low-dimensional representation (3D object position) with depth enables easier sim-to-real transfer.}
\label{table:objrep}
\vspace{-1em}
\end{table}


\subsection{System Capabilities}

\paragraphc{Task performance, generalization, robustness.} We evaluate the overall capabilities of our system via task success rates of the best-performing task policy. For each task, we conduct 10 trials for each test object and report the average success rate across all objects. We report a 62.3\% success rate for the \texttt{grasp-and-reach} task, 80\% for the \texttt{box lift} task, and 52.5\% for the \texttt{bimanual handover} task. We test our \texttt{grasp-and-reach} policy's ability to generalize to out-of-distribution objects and report qualitative results of successful zero-shot generalization in our video. We also show the robustness against force perturbations of our learned policies for all tasks in Figure~\ref{fig:robust} and video. More details on the object set for each task are reported in Figure~\ref{fig:teaser}.

\begin{figure}[!t]
    \centering
    \vspace{-1em}
    \includegraphics[width=\linewidth]{figures/robustness.pdf}
    \caption{\textbf{Policy robustness.} Our learned policies remain robust under different force perturbations, including knock (top left), pull (top right), push (bottom left), and drag (bottom right).}
    \vspace{-1em}
    \label{fig:robust}
\end{figure}


\paragraphc{Extension to a more capable system.}
The learned RL policies can be flexibly chained with finite state machines, teleoperation, etc., to perform longer-horizon tasks while maintaining dexterity, robustness, and generalization. As an example, we present in our video a general pick-and-drop system that can be scripted by utilizing the \texttt{grasp-and-reach} policy.

\section{Limitations} 
\label{sec:limitations}

In this work, we investigate the key challenges in applying RL to robot manipulation and introduce practical and principled techniques to overcome the hurdles. Based on the techniques proposed, we build a sim-to-real RL pipeline that demonstrates a feasible path to solve robot manipulation, with evidence on generalizability, robustness, and dexterity.

However, the capabilities achieved in this work are still far from the kind of ``general-purpose'' manipulation that humans are capable of. Much work remains to be done to improve each individual component of this pipeline and unlock the full potential of sim-to-real RL.
For example, the reward design could be improved by integrating even stronger human priors, such as task demonstrations collected from teleoperation.

There are also important open problems that our work does not address. For example, our work uses no novel technique to reduce the sim-to-real gap in dynamics other than applying naive domain randomization. We hypothesize that this could be a reason for the low success rate on \texttt{bimanual handover} task, which is the most dynamic among our collection of tasks.

Lastly, we find ourselves heavily constrained by the lack of reliable hardware for dexterous manipulation. While we use multi-fingered robot hands, the dexterity of these hands is far from that of human hands in terms of the active degrees of freedom.
We believe the dexterity of our learned policies is not limited by the approach, and we hope to extend our framework to robot hands with more sophisticated designs in the future.

\section{Conclusion} 
\label{sec:conclusion}

We present a comprehensive recipe for applying sim-to-real RL to vision-based dexterous manipulation on humanoids. By addressing key challenges in environment modeling, reward design, policy learning, and sim-to-real transfer, we show that RL can be a powerful tool for learning highly useful manipulation skills without the need for extensive human demonstrations. Our learned policies exhibit strong generalization to unseen objects, robustness against force disturbances, and the ability to perform long-horizon contact-rich tasks.

\clearpage

\section*{Acknowledgments}

We thank members of NVIDIA GEAR lab for help with hardware infrastructure, in particular Zhenjia Xu, Yizhou Zhao, and Zu Wang. This work was partially conducted during TL's internship at NVIDIA. TL is supported by NVIDIA and the National Science Foundation fellowship.

%% Use plainnat to work nicely with natbib. 
\bibliographystyle{plainnat}
\bibliography{references}
% \documentclass{MITstyle}

%\usepackage[table]{xcolor}
\usepackage{chngcntr}
\usepackage{hyperref}
\usepackage{microtype}

\title{A Lightweight and Extensible Cell Segmentation and Classification Model for Whole Slide Images}

\author{Nikita Shvetsov~$^{1, }$\footnote{Correspondence e-mail: nikita.shvetsov@uit.no}, Thomas K. Kilvaer~$^{2, 3}$, Masoud Tafavvoghi~$^{4}$, Anders Sildnes~$^{1}$, \\ Kajsa Møllersen~$^{4}$, Lill-Tove Rasmussen Busund~$^{5, 6}$, Lars Ailo Bongo~$^{1}$ \\
%
\vspace{1em} % Space between authors and afilliations
%
\normalfont{\small $^{1}$Department of Computer Science, UiT The Arctic University of Norway}\\
\normalfont{\small $^{2}$Department of Oncology, University Hospital of North Norway}\\
\normalfont{\small $^{3}$Department of Clinical Medicine, UiT The Arctic University of Norway}\\
\normalfont{\small $^{4}$Department of Community Medicine, UiT The Arctic University of Norway}\\
\normalfont{\small $^{5}$Department of Medical Biology, UiT The Arctic University of Norway} \\
\normalfont{\small $^{6}$Department of Clinical Pathology, University Hospital of North Norway} %\vspace{2em}
}

\begin{document}
\maketitle

\section*{Abstract}

% \begin{abstract}
% Developing clinically useful cell-level analysis tools in digital pathology remains challenging due to limitations in dataset granularity, inconsistent annotations, computational demands of advanced models, and difficulties in integrating new technologies into clinical workflows. To address these challenges, we propose a multi-faceted solution that enhances data quality, model performance, and usability to create a lightweight and extensible cell segmentation and classification model.

% First, we update data labels by employing a cross-relabeling process that refines the labels of two existing datasets, PanNuke and MoNuSAC, to create a new unified dataset with enhanced granularity, encompassing seven distinct cell types. Second, we leverage the H-Optimus foundation model as a fixed encoder to improve feature representation for simultaneous cell segmentation and classification tasks. Third, to address the computational demands of foundation models, we employ knowledge distillation to reduce model size and complexity while maintaining comparable performance. Finally, to facilitate integration into clinical workflows, we integrate the distilled model into the QuPath software, a widely used open-source platform in digital pathology.

% Our results demonstrate improvements in cell segmentation and classification performance using the H‑Optimus-based model compared to a CNN-based model. Specifically, the average $R^2$ improved from 0.575 to 0.871, and the average $PQ$ score improved from 0.450 to 0.492, indicating better alignment with actual cell counts and enhanced segmentation and classification quality. Furthermore, the distilled student model maintains performance comparable to the larger foundation model while reducing the parameter count by a factor of 48.
% Overall, by reducing computational complexity and integrating it into existing workflows, the proposed approach may significantly impact diagnostic processes, reduce the workload of pathologists, and contribute to improved patient outcomes. Though our approach shows potential enhancements in efficiency and usability of cell segmentation and classification models in digital pathology, extensive validation is needed to deploy these models in clinical practice.
% \end{abstract}

%%% shortened abstract
\begin{abstract}
Developing clinically useful cell-level analysis tools in digital pathology remains challenging due to limitations in dataset granularity, inconsistent annotations, high computational demands, and difficulties integrating new technologies into workflows. To address these issues, we propose a solution that enhances data quality, model performance, and usability by creating a lightweight, extensible cell segmentation and classification model. 

First, we update data labels through cross-relabeling to refine annotations of PanNuke and MoNuSAC, producing a unified dataset with seven distinct cell types. Second, we leverage the H-Optimus foundation model as a fixed encoder to improve feature representation for simultaneous segmentation and classification tasks. Third, to address foundation models' computational demands, we distill knowledge to reduce model size and complexity while maintaining comparable performance. Finally, we integrate the distilled model into QuPath, a widely used open-source digital pathology platform. 

Results demonstrate improved segmentation and classification performance using the H-Optimus-based model compared to a CNN-based model. Specifically, average $R^2$ improved from 0.575 to 0.871, and average $PQ$ score improved from 0.450 to 0.492, indicating better alignment with actual cell counts and enhanced segmentation quality. The distilled model maintains comparable performance while reducing parameter count by a factor of 48. By reducing computational complexity and integrating into workflows, this approach may significantly impact diagnostics, reduce pathologist workload, and improve outcomes. Although the method shows promise, extensive validation is necessary prior to clinical deployment.
\end{abstract}
\clearpage

\section{Introduction}
In digital pathology, accurate segmentation and classification of cells are crucial for many diagnostic, prognostic, and predictive analyses \cite{Jaber_Beziaeva_etal._2019,Lin_Pan_etal._2022,Park_Ock_etal._2022,Shen_Choi_etal._2024}. Nowadays, developments in computational pathology offer multiple solutions \cite{H._Qu_P._Wu_etal._2020,Javed_Mahmood_etal._2020} to utilize cell-level datasets to train machine learning models that solve these problems. The quality and specificity of training datasets are critical for robust and accurate models. Adhering to the principle of "garbage in, garbage out", it is essential to ensure that these datasets are extensively and accurately labeled with distinct classes that reflect the diverse biological characteristics of different cell types. Unfortunately, the number of open-source datasets comprising such high-quality annotations is limited. Existing cell segmentation datasets \cite{Gamper_Koohbanani_etal._2019,Graham_Vu_etal._2019,Verma_Kumar_etal._2021} may offer extensive annotations for certain cell types while providing more general labels for others. For example, in PanNuke, which is one of the largest open-source datasets comprising labeled cells, various types of morphologically and functionally different inflammatory cells like macrophages and lymphocytes are clustered in a broad "inflammatory" class. Consequently, these classes are frequently omitted from analyses or aggregated into broader meta-classes \cite{Gamper_Koohbanani_etal._2020} and likely interfere with other cell classes included in the dataset. This and similar inconsistencies in annotation granularity limit the ability of machine learning models to learn the comprehensive and nuanced features necessary for accurate cell segmentation and classification. To address these challenges, methods for refining and standardizing dataset annotations are essential to enhance the quality of training data.

A complementary approach to mitigate the absence of high-quality training data is the use of foundation models. Foundation models as encoders are defined as large-scale, versatile networks pre-trained on vast, diverse datasets using self-supervised learning, contrasting with convolutional neural network (CNN) pre-trained encoders that rely on supervised learning with labeled data. In practice, foundation models leverage enormous amounts of weakly or unlabeled data from millions of whole slide images (WSIs) and employ self-attention mechanisms to capture long-range dependencies and global context \cite{Chen_Ding_etal._2024,Saillard_Jenatton_etal._2024,Vorontsov_Bozkurt_etal._2024,Xu_Usuyama_etal._2024}. As a consequence, foundation models are able to produce transferable feature representations across different cell types and tissue environments. The feature representations can be leveraged by decoder networks to produce segmentation masks and pixel-level classifications. Because foundation models have comprehensive feature representations, they can be effectively fine-tuned using much smaller amounts of cell-level data compared to the large datasets needed to train models from scratch. Furthermore, foundation models incorporate adversarial training elements or contrastive learning \cite{Chen_Ding_etal._2024,Xu_Usuyama_etal._2024}, enhancing their resilience and adaptability by exposing them to challenging and varied scenarios during training. This may result in more generalizable models, often making them well-suited for diverse and complex tasks in digital pathology.

Despite the inherent advantages of foundation models, their deployment for practical use faces its own obstacles. In particular, they require substantial computational power, financial investments and rigorous testing to ensure reliability and efficacy for a given task \cite{Akkus_Dangott_etal._2022,Dragomir_Cocuz_etal._2022,Go_2022,Jafri_Farooqui_etal._2024}. Moreover, while foundation models enhance feature representation and performance, they depend on the quality of available annotations for decoder fine-tuning and, like any other model, cannot resolve existing inconsistencies or ambiguities in data labels. Therefore, there remains a critical need for solutions that address both data quality and practical deployment considerations.
Further, integrating new technologies into existing clinical workflows often encounters resistance, as it necessitates adjustments to established diagnostic processes. So, there is a need to develop solutions that could be integrated into current practices, minimizing the burden on medical professionals to adopt new tools \cite{King_Williams_etal._2023}.

Existing solutions \cite{Goldsborough_Philps_etal._2024,Hörst_Rempe_etal._2024}, while addressing some aspects of these challenges, fall short in providing a comprehensive approach. To address the data quality and clinical deployment issues, we propose a multi-faceted solution that encompasses data refinement, model optimization, and integration with existing pathology tools (\hyperref[fig:fig1]{Figure 1}). The outcome is a lightweight cell segmentation and classification model that can be integrated into digital pathology workflows for practical clinical use.

\begin{figure}[h!]
    \centering
    \includegraphics[width=\textwidth, height=0.82\textheight, keepaspectratio]{images/Figure_1.pdf}
    \caption{Overview of the proposed solution, including 1) Data refinement using cross-relabeling, 2) Teacher model development and fine tuning, 3) Student model optimization with knowledge distillation and 4) Student model and QuPath integration}
    \label{fig:fig1}
\end{figure}
\clearpage

Our approach begins with preparing the data for the fine-tuning and training of the machine learning models. We create a refined dataset, acquired via cross-relabeling two cell-level datasets, enhancing annotation specificity and consistency of the labeled data. Subsequently, we create a cell segmentation and classification model based on the foundation model. We leverage the foundation model as a fixed encoder and fine-tune a decoder using the refined dataset to improve generalization across diverse tissue- and cell types.
To ensure that the model remains lightweight and deployable in a possibly resource-constrained environment, we employ knowledge distillation to approximate the functionality of the foundation model. Finally, to facilitate the practical application of our model in digital pathology workflows, we integrate it with the QuPath \cite{Bankhead_Loughrey_etal._2017} application. Each methodological component contributes to the overarching goal of enhancing model performance, generalizability, and usability in clinical settings.

The primary contributions of this paper are:
\begin{enumerate}
    \item \textit{Data labels refinement through cross-relabeling:}
    
    We propose a new method for refining labels of cell-level datasets through cross-relabeling. This method employs classification models to re-label broad and ambiguous instances, resulting in a more diverse dataset. Our evaluation demonstrates that these classification models achieve high accuracy on test subsets, indicating the reliability of the method for label refinement.

    \item \textit{Enhanced model performance via foundation models:}
    
    We employ a foundation model as a feature extractor for the cell segmentation and classification task. In comparison with training a CNN model from scratch, the foundation model backbone only needs fine-tuning, which significantly reduces training time, computational resources and data requirements. We show that using a foundation model encoder leads to better performance in cell segmentation and classification networks than using a CNN-based encoder. This improvement may enable the model to generalize more effectively across various tissue types and imaging methods.
    
    \item \textit{Model optimization through knowledge distillation:}
    
    We show that a smaller student model trained using knowledge distillation on the refined dataset obtained via our cross-relabeling approach from a foundation model achieves comparable performance in cell segmentation and quantification tasks. As a result, this model is more suitable for deployment in environments without high-performance computing resources.
    
    \item \textit{Integration with QuPath:}
    
    We integrate the distilled cell segmentation and classification model into QuPath, a widely used open-source digital pathology platform, to accelerate clinical adaptation by enabling pathologists to more easily incorporate advanced computational tools into their existing workflows.
\end{enumerate}

Through these methodological steps, we aim to bridge the gap between advanced machine learning techniques and practical clinical applications, making accurate and efficient digital pathology accessible in a broader range of healthcare settings.

\section{Refining Existing Datasets Using Cross-Relabeling}
To address the limitations of sparse and ambiguous labeling of cell-level datasets, we propose a generalizable cross-relabeling strategy that can be applied to any dataset containing broadly categorized or imprecisely labeled cell types. This approach involves training and subsequently leveraging classification models to refine broad categories into more specific or biologically relevant classes.
When applied to cell-level data, the methodology includes extracting individual cell images from the dataset patches, preprocessing these images to standardize the size and accommodate partial cells, and then training deep learning classifiers capable of distinguishing between the finer cell subtypes within the coarser categories. 
To illustrate our approach, we focus on the PanNuke \cite{Gamper_Koohbanani_etal._2020, Gamper_Koohbanani_etal._2019} and MoNuSAC \cite{Verma_Kumar_etal._2021} datasets that we have used to train models for cell quantification in our previous works \cite{Shvetsov_Grønnesby_etal._2022,Shvetsov_Sildnes_etal._2024}. We find that for better cell differentiation we have to introduce more granular labels. PanNuke includes a broad classification of "inflammatory" cells, encompassing lymphocytes, macrophages, and neutrophils. Each cell type differs significantly in structure, function, and clinical relevance. Conversely, MoNuSAC uses the label "epithelial" for a class that comprises both benign epithelial cells and malignant neoplastic cells. This practice makes it challenging to differentiate between benign and malignant epithelial cells in the dataset, which is a critical distinction when identifying tumor areas within tissue samples. To address these issues, we implement a cross-relabeling strategy as shown in \hyperref[fig:fig2]{Figure 2}. The key components are two classification models: one is trained on singular cell images from PanNuke data to classify the epithelial meta-class into epithelial and neoplastic classes. The other is trained on MoNuSAC to refine the inflammatory class into lymphocytes, neutrophils, and macrophages.

\begin{figure}[h!]
    \centering
    \includegraphics[width=\textwidth]{images/Figure_2.pdf}
    \caption{Refined dataset generation via cross relabeling}
    \label{fig:fig2}
\end{figure}

The refining approach consists of three consecutive steps. The first is the preprocessing step, in which we extract individual cells from both datasets (\hyperref[fig:fig3]{Figure 3}). The specifics of PanNuke and MoNuSAC patch preparation before cell preprocessing are provided in \hyperref[chap:S1]{Appendix S1}.

\begin{figure}[h!]
    \centering
    \includegraphics[width=\textwidth]{images/Figure_3.pdf}
    \caption{Cell instances preprocessing including (1) cell map extraction, (2) bounding box delineation, (3) adjusting cell boxes and (4) cropping and resizing of cell images}
    \label{fig:fig3}
\end{figure}

During preprocessing, we extract cell type maps from the ground truth label mask and calculate bounding boxes around each cell instance. To accommodate partial cells at patch borders, a common issue in cropped patch images, we employ mirror padding and extend the field of view of the cell label by 15 pixels to capture adjacent cells. We then crop and resize the identified regions to $64 \times 64$ pixels using bicubic interpolation.

The preprocessed PanNuke dataset comprises 68,031 neoplastic and 23,207 epithelial cell images, while MoNuSAC comprises  33,104 lymphocytes, 1,252 neutrophils, and 1,695 macrophages, which we subsequently use in training cell classification models and classifying the cell image data \hyperref[fig:S2]{Appendix Figure S2 (1)}. 

The next step is to train two distinct ResNet50-based classifiers tailored to address the specific labeling challenges inherent in each dataset. We use ResNet50 for classification models due to its proven effectiveness for image classification tasks in histopathology \cite{pan2022reviewmachinelearningapproaches}, and its compatibility with small images. For the PanNuke dataset, we design the classifier, trained on MoNuSAC data, to disaggregate the heterogeneous "inflammatory" cell category into distinct subtypes: lymphocytes, macrophages, and neutrophils. Similarly, for the MoNuSAC dataset, the classifier is trained on PanNuke data and distinguishes between benign and malignant epithelial cells within the overarching "epithelial" label. By applying these targeted classifiers to their respective datasets, we assign more specific labels to individual cell instances, thus enabling us to create a unified dataset.
To ensure a balanced representation of classes, we train both models on datasets that had been equalized to match the size of the least represented class. Thus, we obtain datasets comprising 23,207 samples per class for PanNuke and 1,252 samples per class for MoNuSAC data. Next, we partition both of them into training (70\%), validation (20\%), and testing (10\%) subsets. To mitigate the risk of overfitting, we use a single dropout layer with a rate of p=0.5 in both models and data augmentation using randomized color perturbations, rotation, and horizontal and vertical flipping. We employ AdamW optimizer and the cross-entropy loss function for the training criterion.

To evaluate the two trained models, we measure the classification accuracy on the respective test subsets. The accuracies on the test subset for both classifiers are presented in \hyperref[tab:1]{Table 1}. The PanNuke model achieves an average accuracy of 93.57\%, with higher accuracy for neoplastic cells (96.06\%) compared to epithelial cells (86.26\%). The confusion matrix in Figure A3.1 shows that the model predominantly distinguishes accurately between epithelial and neoplastic tissues, with a substantial number of correct classifications and relatively few misclassifications. The MoNuSAC model demonstrates an average accuracy of 98.92\%, excelling in classifying lymphocytes (99.67\%) and macrophages (94.12\%), with lower performance for neutrophils (85.71\%). The confusion matrix in Figure A3.2 shows that the model identifies lymphocytes and performs reasonably well with macrophages and neutrophils.

\begin{table}[h!]
\renewcommand{\arraystretch}{1.5}
  \centering
  \caption{Cell classification results for PanNuke and MoNuSAC trained models (CI 95\%).}
  \label{tab:1}
  \begin{tabular}{|l|c|c|}
   \hline
   %\rowcolor{gray!30}
    Accuracy               & PanNuke model              & MoNuSAC model              \\
    \hline
    Average      & 0.936 (0.931--0.941)         & 0.989 (0.986--0.993)        \\
    \hline
    Neoplastic   & 0.961 (0.956--0.965)         & -                          \\
    \hline
    Epithelial   & 0.863 (0.849--0.877)         & -                          \\
    \hline
    Lymphocytes  & -                          & 0.997 (0.995--0.999)        \\
    \hline
    Neutrophils  & -                          & 0.857 (0.796--0.918)        \\
    \hline
    Macrophages  & -                          & 0.941 (0.906--0.976)        \\
    \hline
  \end{tabular}
\end{table}

Finally, during the last step, we use the model trained on PanNuke data for epithelial cells in MoNuSAC and the model trained on MoNuSAC for the inflammatory cells class in PanNuke. Specifically, we use classifier models to relabel epithelial cells in MoNuSAC and inflammatory cells in PanNuke data. Then we combine cells with refined labels and the rest of the cells in both datasets to create a refined dataset (\hyperref[fig:S2]{Appendix Figure S2 (2)}). The process of relabeling cells and visualizing them on a patch is shown in \hyperref[fig:fig4]{Figure 4}. The cell counts in the refined dataset are provided in \hyperref[tab:S4]{Appendix Table S4}.

\begin{figure}[h!]
    \centering
    \includegraphics[width=\textwidth, height=0.42\textheight, keepaspectratio]{images/Figure_4.pdf}
    \caption{Cell relabeling procedure for epithelial and inflammatory cell classes}
    \label{fig:fig4}
\end{figure}

%\hfill

Relabeling and combining datasets have been explored in a prior study \cite{Parulekar_Kanwat_etal._2023}, where consecutive fine-tuning on multiple datasets was employed to account for hierarchical class label structures. While the method presented in \cite{Parulekar_Kanwat_etal._2023} is intuitive, it often lacks consistency and requires multiple fine-tuning runs, which can be cumbersome and time-consuming. 
In contrast, cross-relabeling simplifies this process by using specialized classification models tailored to each dataset's specific labeling challenges. This approach provides better transparency and produces a unified dataset encompassing seven distinct cell types across multiple tissue samples, enhancing data diversity for further model training or fine-tuning.

Despite these improvements, cross-relabeling does not entirely resolve issues related to poor labeling quality or the amount of labeled data. Specifically, our results show lower accuracies persist for underrepresented classes, such as macrophages, which may stem from a limited sample availability and intrinsic challenges in distinguishing these cells based solely on H\&E staining. Furthermore, while our method enhances label specificity, it relies on the initial quality of the broad labels; thus, any fundamental inaccuracies in the original annotations can propagate through the relabeling process. Addressing the overall problem of limited data labels may require integrating additional data sources or utilizing complementary immunohistochemical staining methods.
Although the reported performance metrics are obtained from evaluations on the native test sets of each dataset, it is important to note that the primary application of these classifiers is to perform cross-relabeling, where a model trained on one dataset (e.g., PanNuke) is applied to another (e.g., MoNuSAC) and vice versa. We acknowledge that a more systematic evaluation of cross-dataset generalization is needed and could be performed in future work.

Overall, the refined dataset produced by our approach can enhance the supervised training or fine-tuning of cell segmentation and classification models, especially those that utilize pre-trained foundation models to improve feature extraction robustness. In addition, these models can detect nuanced classes that enable researchers to conduct more detailed analyses of biological processes in computational pathology.

\section{Foundation models for robust cell segmentation and classification}

Accurate cell segmentation and classification in digital pathology are hindered by limited labeled data and the fact that conventional CNNs are unable to capture global contextual information due to their local receptive field constraints \cite{Gheflati_Rivaz_2022,Yang_Marcus_etal.}. Traditional approaches in cell quantification have predominantly relied on CNN encoders, such as ResNet50, given their proven effectiveness in semantic segmentation tasks \cite{Deshmane_2023,Graham_Vu_etal._2019,Mukasheva_Koishiyeva_etal._2024,Stringer_Wang_etal._2021}. However, approaches that include fine-tuning of pretrained CNNs, data augmentation, and stain normalization to partially increase data variability and address staining differences often fail to achieve the necessary generalization and robustness across diverse tissue types and staining conditions \cite{G._Wang_W._Li_etal._2018,Gao_Bagci_etal._2018,Karim_El_Khoury_Martin_Fockedey_etal._2021}.

To overcome these challenges, we leverage an encoder-decoder network that uses a foundation model as the encoder and a CNN upsampling decoder (\hyperref[fig:fig5]{Figure 5}) for simultaneous cell segmentation and classification in 2D patches extracted from WSIs. Foundation models with transformer-based architectures are viable alternatives to CNN-based encoders \cite{Shamshad_Khan_etal._2023,Sourget_2023}. They enable the creation of more advanced architectures that can decode or transform learned features more effectively \cite{Chen_Duan_etal._2023,Cheng_Misra_etal._2022,Xie_Wang_etal._2021}.

\begin{figure}[h!]
    \centering
    \includegraphics[width=\textwidth]{images/Figure_5.pdf}
    \caption{UNETR-like model with foundational model as backbone}
    \label{fig:fig5}
\end{figure}

By utilizing a transformer-based encoder, we incorporate global contextual information into the feature extraction process, which is a key advantage of such architectures \cite{Chen_Lu_etal._2021}. This foundation model integration facilitates accurate pixel-wise segmentation and classification without the need for extensive encoder training, thereby potentially improving generalization across varied cellular structures and tissue types.
In our implementation, we employ a modified UNETR \cite{Hatamizadeh_Tang_etal._2021} architecture that combines a vision transformer (ViT) \cite{Dosovitskiy_Beyer_etal._2021} encoder with a CNN-based decoder. The encoder utilizes the pretrained H-Optimus foundation model, which contains 1.1 billion parameters and is trained on over 500,000 H\&E stained WSIs \cite{Saillard_Jenatton_etal._2024}. We extract outputs from four evenly spaced transformer blocks $Z_i$, where $i \in [1, 14, 26, 38]$, to serve as residual connections for the CNN decoder. We select these blocks based on our observation that features from non-adjacent levels of the encoder lead to better overall performance on the test subset.

The CNN decoder upsamples the feature representations, acquired from the transformer blocks, to generate an intermediate vector that is handled by two task-specific layers that generate cell segmentation and classification masks. The first task-specific layer is the ‘Cellpose head’,  which is used to delineate cell instances. The layer generates horizontal and vertical gradient maps to form vector fields that are refined through gradient tracking in a post-processing step using the Cellpose algorithm \cite{Stringer_Wang_etal._2021}, known for its efficacy in cell segmentation tasks and generalizability across multiple domains \cite{Pachitariu_Stringer_2022,Stringer_Pachitariu_2024}. The second task-specific layer is the "Cell type head", which assigns labels to individual pixels. In the post-processing step, we determine the output classification label of each segmented cell instance by majority voting over the labeled pixels that comprise the cell in the segmentation map.

To evaluate model performance and measure the impact of adding a foundation model as backbone, we compare it to a ResNet50-based model. ResNet50 is a widely used solution for encoders in segmentation architectures in the medical domain \cite{Deshmane_2023,Graham_Vu_etal._2019,Mukasheva_Koishiyeva_etal._2024,Stringer_Wang_etal._2021}. For the H-Optimus-based model, we utilize frozen weights for the encoder and only fine-tune the decoder to take advantage of the extensive pre-training of the foundation model. For the ResNet50-based model we start with ImageNet \cite{Deng_Dong_etal.} weights and train both encoder and decoder parts. Hyperparameters for the training step are set to be identical, where possible, for comparable evaluation. 
For this evaluation, we deliberately use the PanNuke dataset to provide a standardized and controlled comparison between the H‑Optimus and ResNet50-based models (\hyperref[fig:S2]{Appendix Figure S2 (3)}). Specifically, we use two of the default PanNuke dataset splits (66\%) for training and validation, and reserve the third split (33\%) for testing.

To address the challenge of cell class imbalance in the PanNuke dataset, which is a common characteristic in most cell-level H\&E patch datasets, both models’ training processes employ a weighted loss function comprising cross-entropy and focal loss \cite{Lin_Goyal_etal._2018}. The focal loss component is adjusted with coefficients derived from each cell class' instance frequency, emphasizing learning from underrepresented classes and enhancing the model's sensitivity to rare but significant cellular patterns. The cross-entropy loss is augmented with spectral decoupling regularization \cite{Pezeshki_Kaba_etal._2021,Pohjonen_Stürenberg_etal._2022} and spatially varying label smoothing \cite{Islam_Glocker_2021}, which potentially stabilizes training and improves generalization in case of complex tissue morphologies. For optimization, we employ the \textit{AdamW} \cite{Loshchilov_Hutter_2019} to counter unbalanced class scenarios, with cosine annealing learning rate scheduler.

We utilize the scikit-learn library \cite{Van_der_Walt_Schönberger_etal._2014} and HoVer-Net \cite{Graham_Vu_etal._2019} implementations of $R^2$ (the coefficient of determination) and $PQ$ (panoptic quality) to evaluate our experiments. Complete mathematical formulations and detailed explanations of these metrics are provided in \hyperref[chap:S5]{Appendix S5}. To compute confidence intervals, we use nonparametric bootstrapping, where after calculating the metric on the full sample, we generated 1000 bootstrap replicates by resampling with replacement and then determined the 95\% confidence intervals as the 2.5th and 97.5th percentiles of the resulting empirical distribution.

%\hfill

The model comparisons are summarized in \hyperref[tab:2]{Table 2}. The H‑Optimus-based model achieves higher $R^2$ across all cell classes compared to the ResNet50-based model, which means that its predictions are more closely aligned with the PanNuke cell counts, indicating a stronger correlation with the observed data. Notably, the improvement of $R^2_{dead}$ may be an indicator of better global contextual representations provided by the foundation model backbone. In terms of segmentation and classification quality combined, measured by the PQ score, the H‑Optimus-based model demonstrates notable improvements across most cell classes. Overall, the average $R^2$ improved from 0.575 to 0.871, while the average $PQ$ score improved from 0.450 to 0.492, demonstrating better performance of the H-Optimus-based model.

\begin{table}[h!]
\renewcommand{\arraystretch}{1.5}
  \centering
  \caption{Cell quantification metrics for baseline and proposed models (CI 95\%).}
  \label{tab:2}
  \begin{tabular}{|l|c|c|}
    \hline
    %\rowcolor{gray!30}
    Metric             & Resnet50-based            & H-optimus-based              \\
    \hline
    $R^2_{neoplastic}$    & 0.681 (0.576--0.769)       & \textbf{0.941 (0.917--0.960)} \\
    \hline
    $R^2_{inflammatory}$  & 0.863 (0.778--0.903)       & \textbf{0.949 (0.918--0.966)} \\
    \hline
    $R^2_{connective}$    & 0.600 (0.488--0.698)       & 0.609 (0.436--0.772)          \\
    \hline
    $R^2_{dead}$          & 0.097 (-11.389--0.669)     & 0.925 (0.404--0.982)          \\
    \hline
    $R^2_{epithelial}$    & 0.635 (0.490--0.747)       & \textbf{0.930 (0.886--0.964)} \\
    \hline
    $PQ_{neoplastic}$       & 0.517 (0.499--0.535)       & \textbf{0.589 (0.575--0.604)} \\
    \hline
    $PQ_{inflammatory}$     & 0.455 (0.429--0.482)       & \textbf{0.528 (0.507--0.549)} \\
    \hline
    $PQ_{connective}$       & 0.416 (0.400--0.431)       & \textbf{0.451 (0.436--0.465)} \\
    \hline
    $PQ_{dead}$             & 0.374 (0.342--0.408)       & 0.292 (0.209--0.365)          \\
    \hline
    $PQ_{epithelial}$       & 0.488 (0.460--0.519)       & \textbf{0.599 (0.579--0.618)} \\
    \hline
  \end{tabular}
\end{table}

Our results  show that integrating the H‑Optimus foundation model within the UNETR architecture enhances the model's ability to segment and classify cells across diverse tissues from PanNuke data. The pretrained transformer encoder provides robust feature representations, resulting in higher average $R^2$ and $PQ$ scores compared to the CNN-based model. This leads to more reliable cell quantification and more accurate downstream analysis. Additionally, the streamlined fine-tuning process reduces computational overhead and training time, making the model more adaptable for new data.

Despite these advancements, the foundation model-based approach does not fully resolve all challenges related to cell segmentation and classification. We observe lower metric scores for underrepresented classes in the training data. Furthermore, foundation models typically encompass billions of parameters, resulting in substantial computational and memory requirements. It therefore poses challenges for deployment in resource-constrained environments, limiting their practical applicability in certain clinical settings.

\section{Model optimization via Knowledge Distillation}

To address the limitations posed by the extensive size of foundation models, we implement knowledge distillation — a model compression technique that leverages the teacher-student paradigm \cite{Hinton_Vinyals_etal._2015}. By training a smaller, more efficient student model to replicate the output of a larger, pre-trained teacher model, we retain performance while significantly reducing the model's complexity and resource requirements (\hyperref[fig:fig6]{Figure 6}).

\begin{figure}[h!]
    \centering
    \includegraphics[width=\textwidth, height=0.45\textheight, keepaspectratio]{images/Figure_6.pdf}
    \caption{Knowledge distillation framework for training a student model using a pre-trained teacher}
    \label{fig:fig6}
\end{figure}

We employ knowledge distillation to compress the H‑Optimus-based teacher model into a more efficient student model. The teacher model is the modified UNETR architecture with the H‑Optimus foundation model described in the previous chapter. The student model is based on a UNet architecture augmented with residual connections and incorporates a smaller ViT encoder with 9 million parameters \cite{Steiner_Kolesnikov_etal._2022,Wightman_2019}. 

First, we fine-tune the teacher model using the refined dataset from the cross-relabeling procedure (Section 2). Initially we train the decoder of the teacher model while keeping the encoder weights frozen. We split the refined dataset into train (70\%), validation (20\%) and test (10\%) subsets (\hyperref[fig:S2]{Appendix Figure S2 (4)}). During fine-tuning, we use the train and validation subsets, while leaving the test subset for model evaluation. We set the training procedure and model hyperparameters to be identical to those that were used to demonstrate the utility of foundation models for the simultaneous cell segmentation and classification task.

Next, we perform knowledge distillation from teacher to student using the refined dataset used to fine-tune the teacher model. The student model is trained to replicate the teacher model's outputs. We utilize a specialized loss function that aligns the student's predicted probability distribution with the teacher's, incorporating the teacher's class probability distribution derived from the output. Following the methodology of Hinton et al. \cite{Hinton_Vinyals_etal._2015}, we experiment with various hyperparameter settings for the temperature ($T$) and the balancing coefficients ($\alpha$ and $\beta$) in the loss function. We vary $T$ from 1 to 20 and adjust $\alpha$ and $\beta$ to balance the distillation and student losses. Through iterative tuning and evaluation, we identify that setting $T=14$, $\alpha=0.3$, and $\beta=0.7$ yields a configuration that converges and closely approximates the teacher model's performance during training.

Finally, we assess the performance of both models using the $R^2$ and $PQ$ (defined in \hyperref[chap:S5]{Appendix S5}) on the test set of the refined dataset (\hyperref[tab:3]{Table 3}). We observe that the 95\% confidence intervals overlap for most cell types, so we cannot claim statistically significant performance differences between the teacher and student models. One exception appears in the neoplastic class. The teacher model produces an $R^2$ of 0.919, while the student model shows an $R^2$ of 0.852. In addition, the student model achieves higher $PQ$ values for the neoplastic and connective classes, though the confidence intervals show overlap.

\begin{table}[h!]
\renewcommand{\arraystretch}{1.5}
  \centering
  \caption{Cell quantification metrics for teacher and distilled student models (CI 95\%).}
  \label{tab:3}
  \begin{tabular}{|l|c|c|}
    \hline
    %\rowcolor{gray!30}
    Metric & Teacher & Student \\
    \hline
    $R^2_{neoplastic}$    & \textbf{0.919} (0.898--0.939) & 0.852 (0.800--0.891) \\
    \hline
    $R^2_{lymphocyte}$    & 0.969 (0.956--0.977)         & 0.969 (0.956--0.978) \\
    \hline
    $R^2_{connective}$    & 0.694 (0.548--0.809)         & 0.618 (0.469--0.741) \\
    \hline
    $R^2_{dead}$          & 0.755 (0.400--0.908)         & 0.424 (0.100--0.731) \\
    \hline
    $R^2_{epithelial}$    & 0.922 (0.870--0.958)         & 0.843 (0.738--0.917) \\
    \hline
    $R^2_{macrophage}$    & 0.384 (-0.369--0.724)        & 0.704 (0.352--0.859) \\
    \hline
    $R^2_{neutrofil}$     & 0.854 (0.578--0.929)         & 0.833 (0.502--0.925) \\
    \hline
    $PQ_{neoplastic}$       & 0.581 (0.569--0.593)         & 0.601 (0.588--0.613) \\
    \hline
    $PQ_{lymphocyte}$       & 0.536 (0.520--0.553)         & 0.563 (0.544--0.579) \\
    \hline
    $PQ_{connective}$       & 0.436 (0.421--0.451)         & 0.457 (0.441--0.474) \\
    \hline
    $PQ_{dead}$             & 0.272 (0.235--0.315)         & 0.279 (0.201--0.369) \\
    \hline
    $PQ_{epithelial}$       & 0.522 (0.500--0.545)         & 0.530 (0.506--0.555) \\
    \hline
    $PQ_{macrophage}$       & 0.524 (0.459--0.588)         & 0.474 (0.405--0.543) \\
    \hline
    $PQ_{neutrofil}$        & 0.541 (0.490--0.592)         & 0.565 (0.522--0.607) \\
    \hline
  \end{tabular}
\end{table}


We further decompose the $PQ$ metric into its $SQ$ and $DQ$ components (\hyperref[tab:S6]{Appendix Table S6}). Both models produce nearly identical $SQ$ values, which indicates that they predict instance boundaries with similar precision. Although the student model shows some improvement in $DQ$ scores for certain classes, the confidence intervals overlap and do not confirm a statistically significant difference.

We observe that the student and teacher models yield comparable detection performance despite the student model using a much smaller and simpler architecture. A model with fewer parameters reduces the risk of overfitting when training data are scarce relative to the model’s complexity \cite{Farias_Ludermir_etal._2022}. The knowledge distillation process also encourages the student model to focus on the most generalizable detection features learned from the teacher. These factors enable the student model to achieve similar detection performance across different cell types.

Additionally, considering the model sizes reported in \hyperref[tab:4]{Table 4}, the distilled model achieves a significant reduction compared to the teacher model, with a 48-fold decrease in parameter count and a 5.5-fold reduction in on-disk size. In inference mode, the teacher model requires 16 GB of VRAM for a batch size of 32, while the distilled model only needs 3 GB of VRAM for the same batch size. These reductions make the distilled model significantly more practical for fine-tuning and deployment in resource-constrained environments.

\begin{table}[h!]
\renewcommand{\arraystretch}{1.5}
  \centering
  \caption{Parameter counts and size of teacher and distilled model}
  \label{tab:4}
  \adjustbox{max width=\textwidth}{%
  \begin{tabular}{|l|c|c|c|}
    \hline
    %\rowcolor{gray!30}
    Metric & H-optimus-based (Teacher) & mobileViT-based (Student) & Magnitude of difference \\
    \hline
    Parameters count       & 1,158,917,906   & \textbf{24,093,393}   & \textbf{48x}  \\
    \hline
    Estimated Total Size (MB) & 87,912       & \textbf{15,935}    & \textbf{5.5x} \\
    \hline
  \end{tabular}%
}
\end{table}

%\hfill

With recent advancements in complex network architectures and the use of pretrained encoders to achieve state-of-the-art performance \cite{Baumann_Dislich_etal._2024,Hörst_Rempe_etal._2024} in cell segmentation and classification tasks, model size, computational complexity, and processing times have increased. This limits the scalability and accessibility of these models. As we demonstrate, this may be mitigated using knowledge distillation. Studies in the field of natural language processing have demonstrated the efficacy of knowledge distillation in retaining the capabilities of the teacher model while achieving significant reductions in size and complexity \cite{Huangpu_Gao_2024,Sun_Yu_etal.}. 

We demonstrate the feasibility of knowledge distillation in digital pathology, specifically for cell segmentation and classification tasks. Moreover, we achieve this performance while also significantly reducing the parameter count. In addressing the challenge of knowledge transfer, we found that distillation from a transformer-based model to a smaller transformer is more straightforward than attempting to map transformer features to CNN blocks. In our experiments, using a CNN-based network as a student results in worse cell quantification performance due to the structural constraints of CNN feature space dimensions. 

Although our primary approach relies on a transformer-based student model that performs well, it can be further optimized to incorporate advantages from CNN architectures. For example, employing alternative techniques such as using ViT adapters \cite{Chen_Duan_etal._2023} or $1 \times 1$ convolutions to adjust feature map sizes may be beneficial for harnessing CNN advantages like enhanced local feature extraction. Moreover, if additional performance improvements are desired, the process can be further enhanced by applying supplementary knowledge distillation techniques, such as self-distillation \cite{Zhang_Song_etal._2019} or online distillation \cite{Houyon_Cioppa_etal._2023}.

Despite these promising results, further validation on independent datasets is necessary to fully understand the model's limitations. Underrepresented classes may pose challenges when addressing complex cases. Pathologists need to validate these models to adopt them in clinical settings. While the distilled models are smaller and more deployable, a technological gap persists because pathologists traditionally rely on established methods for inspecting WSIs and diagnosing diseases. Addressing the complexities involved in deploying models for inference and supporting pathologists in adopting new tools is essential for integrating these models into clinical workflows.

\section{Model integration with QuPath}
Digital pathology tools with graphical user interfaces are essential for visualizing and analyzing WSIs. To make our student model useful in clinical pathology workflows, it needs to be integrated into a tool that enables inspecting regions, creating annotations, and providing quantitative analyses of biomarkers. Therefore, we integrate the trained student model from the previous chapter into the QuPath open‑source platform \cite{Bankhead_Loughrey_etal._2017}. QuPath provides the required annotation, visualization, and analysis tools to interpret complex histological data, including workflows for cell segmentation, classification, and quantification (\hyperref[fig:fig7]{Figure 7}). 

\begin{figure}[h!]
    \centering
    \includegraphics[width=\textwidth]{images/Figure_7.pdf}
    \caption{Visualization of model-generated cell quantification annotations (left) and the corresponding unannotated slide (right) in QuPath}
    \label{fig:fig7}
\end{figure}

To identify the regions in a WSI critical for prognosticating tumor development, such as specific tumor areas or border regions without overlapping healthy tissue, the pathologist uses QuPath to outline these regions. Then, the pathologist initiates a cell segmentation and classification script through the QuPath interface for the selected regions. The resulting annotations and quantified cell information are then directly overlaid onto the WSI in the QuPath interface. Additional design and implementation details are in \hyperref[chap:S7]{Appendix S7}. 

Two common approaches for integrating deep learning models into QuPath are Java‑based native QuPath extensions \cite{Goldsborough_Philps_etal._2024} and the execution of RESTful API requests to a model server coupled with handling the response via an extension, as demonstrated in the application of cell segmentation models applied to immunofluorescence images \cite{Sugawara_2023}. While the community is actively working on these integration strategies, there is currently no universal solution that fully addresses all integration and performance requirements.

Extensions may offer better integration with QuPath, allowing slightly improved performance and more widespread usage of the built-in QuPath models, but they lack the flexibility to customize models and modify their behavior. For example, the newest version of QuPath includes models such as StarDist \cite{Weigert_Schmidt} and InstanSeg \cite{Goldsborough_Philps_etal._2024} that can perform cell segmentation. Both models pose limitations when applied to simultaneous cell segmentation and classification. StarDist performs well only on convex, round shapes by design, whereas some neoplastic, inflammatory, and connective cells exhibit complex and non-convex shapes. InstanSeg provides only semantic segmentation without assigning classes to the segmented cells.

%\hfill

In contrast, our approach offers an alternative integration strategy. It utilizes the paquo library to directly interact with QuPath’s internal application programming interface from within Python. This enables data exchange and processing without the need for intermediate conversion steps and provides greater control over model customization, retraining, and the incorporation of custom processing steps.

The integration of our custom model with QuPath underscores its potential to significantly enhance the diagnostic process by reducing the time burden on pathologists and enabling them to focus on more complex interpretative tasks using familiar software. Leveraging a tool that is already well-established among pathologists increases the likelihood of its adoption into daily clinical workflows. The quantitative data generated through the automated workflow is critical for both clinical decision-making and research, facilitating more accurate biomarker analysis, enabling robust statistical evaluations, and supporting hypothesis generation and testing. Additionally, by streamlining cell segmentation and classification, the tool enhances the scalability and reproducibility of pathological assessments, ultimately contributing to improved diagnostic accuracy and patient outcomes.

\section{Conclusion and future work}

In this study, we address critical challenges in digital pathology and tackle the usability and deployment issues of the developed models in standard computing environments without the need for high-performance computing systems. Our multi-faceted approach encompasses data refinement through cross-relabeling, leveraging foundation models for robust cell segmentation and classification, optimizing model performance via knowledge distillation, and integrating the optimized model into the QuPath software for practical application. This approach is used to construct a capable, versatile, and adjustable model for cell segmentation and classification, with enhanced performance and usability.

\begin{sloppypar}
While our approach shows potential in the field of computational pathology, certain limitations persist. 
For example, our implementation currently exhibits lower performance in detecting macrophages. 
This serves as an instance of the broader challenge of accurately identifying complex cell types. In order to address this issue, extending our approach to incorporate additional data sources, exploring alternative modeling approaches, and integrating other imaging modalities such as immunohistochemical staining may help improve detection accuracy. Moreover, although the distilled model reduces computational demands, integrating advanced deep learning models into clinical practice requires addressing technological gaps and potential resistance to adopting new tools within established diagnostic processes.
\end{sloppypar}

Future work could focus on several key areas to refine the proposed approach and facilitate its adoption in clinical environments. Enhancing the cell-relabeling process with additional datasets \cite{Graham_Jahanifar_etal._2021} could improve the representation of underrepresented cell types and enhance overall model performance. Also, incorporating additional data sources, such as multi-modal imaging or complementary staining methods, may address limitations related to cell type differentiation and class imbalance. Exploring other foundation models \cite{Vorontsov_Bozkurt_etal._2024,Zimmermann_Vorontsov_etal._2024} or introducing additional modalities \cite{Ding_Wagner_etal._2024,Vaidya_Zhang_etal._2025} may provide alternative architectures better suited to specific tasks or offer improved efficiency. Implementing more complex knowledge distillation techniques \cite{Houyon_Cioppa_etal._2023,Zhang_Song_etal._2019} could further optimize the model's performance and adaptability. Additionally, deeper integration with QuPath or other digital pathology software could provide pathologists more control over cell quantification analysis directly within the QuPath interface, thereby increasing accessibility and usability. Such enhancements would not only refine model performance but also ensure greater adaptability and scalability within various clinical environments. Finally, extensive validation of the model by pathologists and benchmarking against independent datasets are essential steps toward establishing the model's reliability and fostering confidence in its clinical utility.

\section*{Acknowledgments} 
This work was funded in part by the Research Council of Norway grant no. 309439 SFI Visual Intelligence, and the North Norwegian Health Authority grant no. HNF1521-20.

\bibliographystyle{IEEEtran}
\begin{sloppypar}
\begin{thebibliography}{99}

\bibitem{chaplot2020neural} Chaplot, Devendra Singh, et al. "Neural topological slam for visual navigation." Proceedings of the IEEE/CVF conference on computer vision and pattern recognition. 2020.

\bibitem{maksymets2021thda} Maksymets, Oleksandr, et al. "Thda: Treasure hunt data augmentation for semantic navigation." Proceedings of the IEEE/CVF International Conference on Computer Vision. 2021.

\bibitem{mezghan2022memory} Mezghan, Lina, et al. "Memory-augmented reinforcement learning for image-goal navigation." 2022 IEEE/RSJ International Conference on Intelligent Robots and Systems (IROS). IEEE, 2022.

\bibitem{al2022zero} Al-Halah, Ziad, Santhosh Kumar Ramakrishnan, and Kristen Grauman. "Zero experience required: Plug \& play modular transfer learning for semantic visual navigation." Proceedings of the IEEE/CVF Conference on Computer Vision and Pattern Recognition. 2022.

\bibitem{ye2021auxiliary} Ye, Joel, et al. "Auxiliary tasks and exploration enable objectgoal navigation." Proceedings of the IEEE/CVF international conference on computer vision. 2021.

\bibitem{chaplot2020object} Chaplot, Devendra Singh, et al. "Object goal navigation using goal-oriented semantic exploration." Advances in Neural Information Processing Systems 33 (2020)

\bibitem{ramakrishnan2022poni} Ramakrishnan, Santhosh Kumar, et al. "Poni: Potential functions for objectgoal navigation with interaction-free learning." Proceedings of the IEEE/CVF Conference on Computer Vision and Pattern Recognition. 2022.

\bibitem{ramrakhya2022habitat} Ramrakhya, Ram, et al. "Habitat-web: Learning embodied object-search strategies from human demonstrations at scale." Proceedings of the IEEE/CVF Conference on Computer Vision and Pattern Recognition. 2022.

\bibitem{mousavian2019visual} Mousavian, Arsalan, et al. "Visual representations for semantic target driven navigation." 2019 International Conference on Robotics and Automation (ICRA). IEEE, 2019.

\bibitem{dhariwal2021diffusion} Dhariwal, Prafulla, and Alexander Nichol. "Diffusion models beat gans on image synthesis." Advances in neural information processing systems 34 (2021)

\bibitem{ho2022classifier} Ho, Jonathan, and Tim Salimans. "Classifier-free diffusion guidance." arXiv preprint arXiv:2207.12598 (2022).

\bibitem{nichol2021glide} Nichol, Alex, et al. "Glide: Towards photorealistic image generation and editing with text-guided diffusion models." arXiv preprint arXiv:2112.10741 (2021)

\bibitem{brooks2023instructpix2pix} Brooks, Tim, Aleksander Holynski, and Alexei A. Efros. "Instructpix2pix: Learning to follow image editing instructions." Proceedings of the IEEE/CVF Conference on Computer Vision and Pattern Recognition. 2023.

\bibitem{fu2023guiding} Fu, Tsu-Jui, et al. "Guiding instruction-based image editing via multimodal large language models." arXiv preprint arXiv:2309.17102 (2023).

\bibitem{geng2024instructdiffusion} Geng, Zigang, et al. "Instructdiffusion: A generalist modeling interface for vision tasks." Proceedings of the IEEE/CVF Conference on Computer Vision and Pattern Recognition. 2024.

\bibitem{zhou2024minedreamer} Zhou, Enshen, et al. "Minedreamer: Learning to follow instructions via chain-of-imagination for simulated-world control." arXiv preprint arXiv:2403.12037 (2024).

\bibitem{zhou2023esc} Zhou, Kaiwen, et al. "Esc: Exploration with soft commonsense constraints for zero-shot object navigation." International Conference on Machine Learning. PMLR, 2023.

\bibitem{yu2023l3mvn} Yu, Bangguo, Hamidreza Kasaei, and Ming Cao. "L3mvn: Leveraging large language models for visual target navigation." 2023 IEEE/RSJ International Conference on Intelligent Robots and Systems (IROS). IEEE, 2023.

\bibitem{gadre2023cows} Gadre, Samir Yitzhak, et al. "Cows on pasture: Baselines and benchmarks for language-driven zero-shot object navigation." Proceedings of the IEEE/CVF Conference on Computer Vision and Pattern Recognition. 2023.

\bibitem{shah2023navigation} Shah, Dhruv, et al. "Navigation with large language models: Semantic guesswork as a heuristic for planning." Conference on Robot Learning. PMLR, 2023.

\bibitem{cai2024bridging} Cai, Wenzhe, et al. "Bridging zero-shot object navigation and foundation models through pixel-guided navigation skill." 2024 IEEE International Conference on Robotics and Automation (ICRA). IEEE, 2024.

\bibitem{yu2023co} Yu, Bangguo, Hamidreza Kasaei, and Ming Cao. "Co-NavGPT: Multi-robot cooperative visual semantic navigation using large language models." arXiv preprint arXiv:2310.07937 (2023).

\bibitem{wu2024voronav} Wu, Pengying, et al. "Voronav: Voronoi-based zero-shot object navigation with large language model." arXiv preprint arXiv:2401.02695 (2024).

\bibitem{qin2023mp5} Qin, Yiran, et al. "Mp5: A multi-modal open-ended embodied system in minecraft via active perception." arXiv preprint arXiv:2312.07472 (2023).

\bibitem{du2024learning} Du, Yilun, et al. "Learning universal policies via text-guided video generation." Advances in Neural Information Processing Systems 36 (2024).

\bibitem{ajay2024compositional} Ajay, Anurag, et al. "Compositional foundation models for hierarchical planning." Advances in Neural Information Processing Systems 36 (2024).

\bibitem{liang2024skilldiffuser} Liang, Zhixuan, et al. "Skilldiffuser: Interpretable hierarchical planning via skill abstractions in diffusion-based task execution." Proceedings of the IEEE/CVF Conference on Computer Vision and Pattern Recognition. 2024.

\bibitem{heusel2017gans} Heusel, Martin, et al. "Gans trained by a two time-scale update rule converge to a local nash equilibrium." Advances in neural information processing systems 30 (2017).

\bibitem{zhang2018unreasonable} Zhang, Richard, et al. "The unreasonable effectiveness of deep features as a perceptual metric." Proceedings of the IEEE conference on computer vision and pattern recognition. 2018.

\bibitem{brown2020language} Brown, Tom B. "Language models are few-shot learners." arXiv preprint arXiv:2005.14165 (2020).

\bibitem{podell2023sdxl} Podell, Dustin, et al. "Sdxl: Improving latent diffusion models for high-resolution image synthesis." arXiv preprint arXiv:2307.01952 (2023).

\bibitem{brohan2022rt} Brohan, Anthony, et al. "Rt-1: Robotics transformer for real-world control at scale." arXiv preprint arXiv:2212.06817 (2022).

\bibitem{brohan2023rt} Brohan, Anthony, et al. "Rt-2: Vision-language-action models transfer web knowledge to robotic control." arXiv preprint arXiv:2307.15818 (2023).

\bibitem{li2024manipllm} Li, Xiaoqi, et al. "Manipllm: Embodied multimodal large language model for object-centric robotic manipulation." Proceedings of the IEEE/CVF Conference on Computer Vision and Pattern Recognition. 2024.

\bibitem{shah2023vint} Shah, Dhruv, et al. "ViNT: A foundation model for visual navigation." arXiv preprint arXiv:2306.14846 (2023).

\bibitem{liu2024visual} Liu, Haotian, et al. "Visual instruction tuning." Advances in neural information processing systems 36 (2024).

\bibitem{hu2021lora} Hu, Edward J., et al. "Lora: Low-rank adaptation of large language models." arXiv preprint arXiv:2106.09685 (2021).

\bibitem{qin2023supfusion} Qin, Yiran, et al. "SupFusion: Supervised LiDAR-camera fusion for 3D object detection." Proceedings of the IEEE/CVF International Conference on Computer Vision. 2023.

\bibitem{qin2024worldsimbench} Qin, Yiran, et al. "Worldsimbench: Towards video generation models as world simulators." arXiv preprint arXiv:2410.18072 (2024).

\bibitem{yu2025gamefactory} Yu, Jiwen, et al. "GameFactory: Creating New Games with Generative Interactive Videos." arXiv preprint arXiv:2501.08325 (2025).

\bibitem{zhou2024code} Zhou, Enshen, et al. "Code-as-Monitor: Constraint-aware Visual Programming for Reactive and Proactive Robotic Failure Detection." arXiv preprint arXiv:2412.04455 (2024).

\bibitem{zhang2024ad} Zhang, Zaibin, et al. "AD-H: Autonomous Driving with Hierarchical Agents." arXiv preprint arXiv:2406.03474 (2024).

\bibitem{wang2024toward} Wang, Chaoqun, et al. "Toward Accurate Camera-based 3D Object Detection via Cascade Depth Estimation and Calibration." arXiv preprint arXiv:2402.04883 (2024).

\bibitem{huang2024story3d} Huang, Yuzhou, et al. "Story3d-agent: Exploring 3d storytelling visualization with large language models." arXiv preprint arXiv:2408.11801 (2024).

\bibitem{savinov2018semi} Savinov, Nikolay, Alexey Dosovitskiy, and Vladlen Koltun. "Semi-parametric topological memory for navigation." arXiv preprint arXiv:1803.00653 (2018).

\bibitem{majumdar2022zson} Majumdar, Arjun, et al. "Zson: Zero-shot object-goal navigation using multimodal goal embeddings." Advances in Neural Information Processing Systems 35 (2022): 32340-32352.

\bibitem{yadav2023offline} Yadav, Karmesh, et al. "Offline visual representation learning for embodied navigation." Workshop on Reincarnating Reinforcement Learning at ICLR 2023. 2023.

\bibitem{yadav2023ovrl} Yadav, Karmesh, et al. "Ovrl-v2: A simple state-of-art baseline for imagenav and objectnav." arXiv preprint arXiv:2303.07798 (2023).

\bibitem{sun2024fgprompt} Sun, Xinyu, et al. "FGPrompt: fine-grained goal prompting for image-goal navigation." Advances in Neural Information Processing Systems 36 (2024).

\bibitem{zhu2017target} Zhu, Yuke, et al. "Target-driven visual navigation in indoor scenes using deep reinforcement learning." 2017 IEEE international conference on robotics and automation (ICRA). IEEE, 2017.

\bibitem{koh2024generating} Koh, Jing Yu, Daniel Fried, and Russ R. Salakhutdinov. "Generating images with multimodal language models." Advances in Neural Information Processing Systems 36 (2024).

\bibitem{krantz2022instance} Krantz, Jacob, et al. "Instance-specific image goal navigation: Training embodied agents to find object instances." arXiv preprint arXiv:2211.15876 (2022).

\bibitem{schulman2017proximal} Schulman, John, et al. "Proximal policy optimization algorithms." arXiv preprint arXiv:1707.06347 (2017).

\bibitem{anderson2018evaluation} Anderson, Peter, et al. "On evaluation of embodied navigation agents." arXiv preprint arXiv:1807.06757 (2018).

\bibitem{lin2024navcot} Lin, Bingqian, et al. "NavCoT: Boosting LLM-Based Vision-and-Language Navigation via Learning Disentangled Reasoning." arXiv preprint arXiv:2403.07376 (2024).

\bibitem{NavGPT} Zhou, Gengze, Yicong Hong, and Qi Wu. "Navgpt: Explicit reasoning in vision-and-language navigation with large language models." Proceedings of the AAAI Conference on Artificial Intelligence.

\bibitem{hahn2021no} Hahn, Meera, et al. "No rl, no simulation: Learning to navigate without navigating." Advances in Neural Information Processing Systems 34 (2021): 26661-26673.

\bibitem{li2025t2isafety} Li, Lijun, et al. "T2ISafety: Benchmark for Assessing Fairness, Toxicity, and Privacy in Image Generation." arXiv preprint arXiv:2501.12612 (2025).

\bibitem{an2024agfsync} An, Jingkun, et al. "AGFSync: Leveraging AI-Generated Feedback for Preference Optimization in Text-to-Image Generation." arXiv preprint arXiv:2403.13352 (2024).


\end{thebibliography}
\end{sloppypar}

\clearpage
\beginsupplement
\section*{Appendix}
\renewcommand{\thesubsection}{S\arabic{subsection}}

\subsection{\label{chap:S1}PanNuke and MoNuSAC preprocessing}
The PanNuke dataset comprises a set of 7,901 RGB patches, each with dimensions of $256 \times 256$ pixels, which we set as the standard patch size for our analysis. In contrast, the MoNuSAC dataset encompasses 294 images of heterogeneous dimensions. To standardize the MoNuSAC images with our experiments, we implement a standardization protocol. Specifically, for images exceeding the dimensions of $256 \times 256$ pixels, we segment them into equal-sized patches and apply mirror padding to the remaining portions to avoid information loss at the peripherals. Patches with dimensions less than $128 \times 128$ pixels are excluded from the dataset due to the insufficient resolution to capture relevant cellular details. For patches where either dimension falls between 128 and 256 pixels, we employ upsampling to achieve the standard patch size. As a result, we obtain a total of 2,823 RGB patches derived from the MoNuSAC dataset for subsequent analysis. For additional details on the MoNuSAC data preparation process, refer to the source code \cite{Shvetsov_2025a}.
\clearpage

\subsection{\label{chap:S2}Data usage for the methodology}

\counterwithin{figure}{subsection}
\renewcommand{\thefigure}{S\arabic{subsection}}

\begin{figure}[h!]
    \centering
    \includegraphics[width=\textwidth, height=0.85\textheight, keepaspectratio]{images/A2.pdf}
    \caption{Overview of the methodology for cross-labeling, dataset refinement, and model comparison. (1) Cross-relabeling - training and testing cell classification models, (2) Cross-relabeling - using cell classification models to create refined dataset, (3) Fine-tuning and training models for comparison, (4) Student knowledge distillation with refined dataset}
    \label{fig:S2}
\end{figure}
\clearpage

\subsection{\label{chap:S3}Confusion matrices for classification models}
\counterwithin{figure}{subsection}
\renewcommand{\thefigure}{S\arabic{subsection}.\arabic{figure}}

\begin{figure}[h!]
    \centering
    \includegraphics[width=\textwidth, height=0.4\textheight, keepaspectratio]{images/A3_1.pdf}
    \caption{Confusion matrix for PanNuke trained model}
    \label{fig:S3.1}
\end{figure}

\begin{figure}[h!]
    \centering
    \includegraphics[width=\textwidth, height=0.4\textheight, keepaspectratio]{images/A3_2.pdf}
    \caption{Confusion matrix for MoNuSAC trained model}
    \label{fig:S3.2}
\end{figure}

\clearpage

\subsection{\label{chap:S4}Datasets cell counts}

\counterwithin{table}{subsection}
\renewcommand{\thetable}{S\arabic{subsection}}

\begin{table}[h!]
\renewcommand{\arraystretch}{2.0}
\centering
\caption{\label{tab:S4}Cell counts for PanNuke, MoNuSAC and refined datasets. Numbers in parentheses indicate preprocessed cell counts for cell classifier models training and testing.}
%\adjustbox{max width=\textwidth}{%
\begin{tabular}{|l|c|c|c|}
\hline
%\rowcolor{gray!30}
Cell type & PanNuke & MoNuSAC & Refined \\
\hline
Neoplastic & 77,403 (68,031) & - & 105,451 \\
\hline
Epithelial & 26,572 (23,207) & - & 29,926 \\
\hline
Epithelial (benign and malignant) & - & 31,402 & - \\
\hline
Inflammatory & 32,276 & - & - \\
\hline
Lymphocytes & - & 37,045 (33,104) & 65,275 \\
\hline
Neutrophils & - & 1,355 (1,252) & 3,833 \\
\hline
Macrophage & - & 1,842 (1,695) & 3,410 \\
\hline
Dead & 2,908 & - & 2,908 \\
\hline
Connective & 50,585 & - & 50,585 \\
\hline
\end{tabular}
%
%}
\end{table}



\clearpage

\subsection{\label{chap:S5}Definition of validation metrics}
\counterwithin{equation}{subsection}
\renewcommand{\theequation}{\arabic{equation}}

\subsubsection{\label{chap:S5.1}R\textsuperscript{2}}
The coefficient of determination, denoted as $R^2$, is a statistical measure that represents the proportion of variance in the dependent variable that is predictable from the independent variables. In the context of cell quantification in pathology, $R^2$ is used to assess how well the predicted quantities of different cell types in a patch align with the actual quantities observed in the ground truth data, with higher values representing more accurate quantification. $R^2$ is defined as
\begin{equation*}
R^2 = 1 - \frac{\sum_{i=1}^n (y_i - \hat{y}_i)^2}{\sum_{i=1}^n (y_i - \bar{y})^2},
\end{equation*}
where $y_i$ represents the actual number of cells of a specific type in the $i$-th image, $\hat{y}_i$ represents the predicted number of cells of that type in the $i$-th image, $\bar{y}$ is the mean of the actual numbers across all images, and $n$ is the total number of images in the dataset.

The $R^2$ metric has a range of $(-\infty, 1]$. An $R^2$ of 1 indicates perfect prediction, where all predicted values exactly match the actual values. An $R^2$ of 0 suggests that the model explains none of the variability of the response data around its mean. If $R^2$ is negative, it indicates that the model performs worse than a model that simply predicts the mean of the actual values for all observations.

\subsubsection{\label{chap:S5.2}PQ}
Panoptic Quality ($PQ$) is a comprehensive metric used to evaluate the performance of segmentation models in tasks that require both instance segmentation and classification. $PQ$ provides a single score that encapsulates both the detection accuracy (i.e., how many objects were correctly identified) and the segmentation quality (i.e., how accurately the objects' boundaries were delineated). This metric is particularly useful in multiclass scenarios where each pixel is classified into distinct categories, such as different cell types in pathology images.

$PQ$ is calculated as the product of two terms: Detection Quality ($DQ$) and Segmentation Quality ($SQ$). It can be expressed as
\begin{equation*}
PQ = DQ \cdot SQ,
\end{equation*}
where
\begin{equation*}
DQ = \frac{TP}{TP + 0.5\, FP + 0.5\, FN},
\end{equation*}
\begin{equation*}
SQ = \frac{\sum_{(p, g) \in \mathcal{M}} IoU(p, g)}{TP}.
\end{equation*}
In these formulas, $TP$ denotes the number of correctly matched instances between ground truth and prediction, $FP$ denotes the predicted instances that have no corresponding ground truth, $FN$ denotes the ground truth instances that were not detected, $IoU(p, g)$ is the Intersection over Union for a pair of matched instances $p$ (prediction) and $g$ (ground truth), and $\mathcal{M}$ is the set of matched pairs.

The $PQ$ metric is calculated for each class and is averaged across classes to provide a global performance measure.

The $PQ$ score has a range of $[0, 1.0]$, where a higher score indicates better performance in both detecting and segmenting the instances correctly. A $PQ$ of 1 signifies perfect identification and segmentation of all instances, whereas a $PQ$ of 0 indicates that no instances were correctly identified and segmented.

\clearpage

\subsection{\label{chap:S6}Segmentation and Detection quality metrics for teacher and student models}

\begin{table}[h!]
\renewcommand{\arraystretch}{2.0}
\centering
\caption{Segmentation and detection quality for student and teacher models (CI 95\%)}
\label{tab:S6}
%\adjustbox{max width=\textwidth}{%
\begin{tabular}{|l|c|c|}
\hline
%\rowcolor{gray!30}
Metric & Teacher & Student \\
\hline
$SQ_{neoplastic}$ & 0.819 (0.815--0.823) & 0.824 (0.819--0.828) \\
\hline
$SQ_{lymphocyte}$ & 0.795 (0.788--0.802) & 0.790 (0.783--0.796) \\
\hline
$SQ_{connective}$ & 0.770 (0.762--0.776) & 0.780 (0.772--0.786) \\
\hline
$SQ_{dead}$ & 0.659 (0.623--0.688) & 0.657 (0.624--0.695) \\
\hline
$SQ_{epithelial}$ & 0.780 (0.770--0.790) & 0.788 (0.779--0.797) \\
\hline
$SQ_{macrophage}$ & 0.788 (0.760--0.810) & 0.757 (0.730--0.783) \\
\hline
$SQ_{neutrofil}$ & 0.782 (0.761--0.801) & 0.775 (0.759--0.792) \\
\hline
$DQ_{neoplastic}$ & 0.706 (0.692--0.719) & 0.727 (0.712--0.741) \\
\hline
$DQ_{lymphocyte}$ & 0.675 (0.656--0.698) & 0.713 (0.691--0.734) \\
\hline
$DQ_{connective}$ & 0.566 (0.546--0.584) & 0.583 (0.565--0.602) \\
\hline
$DQ_{dead}$ & 0.410 (0.361--0.465) & 0.435 (0.306--0.561) \\
\hline
$DQ_{epithelial}$ & 0.668 (0.639--0.694) & 0.673 (0.644--0.702) \\
\hline
$DQ_{macrophage}$ & 0.657 (0.583--0.727) & 0.615 (0.531--0.703) \\
\hline
$DQ_{neutrofil}$ & 0.691 (0.625--0.753) & 0.729 (0.679--0.778) \\
\hline
\end{tabular}
%
%}
\end{table}

\clearpage

\subsection{\label{chap:S7}QuPath integration method}
We adopt an integration strategy leveraging the paquo \cite{Bayer_AG} library, a Python package that enables direct interaction with QuPath’s internal API, thereby facilitating seamless data exchange without intermediate conversion steps. The data processing pipeline (\hyperref[fig:S7]{Appendix Figure S7}) begins with the acquisition of WSIs and their associated annotations from QuPath, which are represented as Shapely \cite{Gillies_Wel_etal._2024} polygons. Utilizing paquo, we directly read, create, and modify these annotations and detections within a QuPath project in the Python environment. Images are then cropped using these polygons and processed by cell segmentation and classification models employing standard vision processing toolkits such as OpenCV, pyvips, and PyTorch. Additionally, QuPath employs Groovy scripts to initiate a Python process that starts the entire pipeline from QuPath graphical interface: fetching polygons, extracting images from them, and running deep learning model inference on the cropped images. 
The results are returned to QuPath, leveraging paquo's Python bindings to manipulate QuPath data while minimizing the computational overhead typically associated with cross-environment communication.

\counterwithin{figure}{subsection}
\renewcommand{\thefigure}{S\arabic{subsection}}

\begin{figure}[h!]
    \centering
    \includegraphics[width=\textwidth]{images/A7.pdf}
    \caption{QuPath integration workflow using Python environment}
    \label{fig:S7}
\end{figure}

Compared to traditional workflows that involve exporting annotations as GeoJSON, classifying them in Python, and reimporting them into QuPath, our approach offers several advantages. We eliminate the need to switch between programming languages, providing a cohesive and streamlined development process entirely within QuPath software and removing the necessity to use other tools. Meanwhile, we avoid storing annotations as intermediate JSON files unless required for external use or archiving. By conducting the entire inference and post-processing workflow within the Python environment, we leverage the power and flexibility of Python libraries for image processing and machine learning. This approach also enables adjustments to any set of labels and models, thereby improving its applicability.

%\hfill

The distilled model and QuPath integration code are packaged into a Docker container, enabling streamlined execution with the Docker engine. Detailed integration code and deployment instructions can be found in the GitHub repository \cite{Shvetsov_2025b}.

Despite these benefits, we acknowledge that the paquo library is a proof‑of‑concept project in its early development stage and has not been tested across all versions of QuPath.

\clearpage

\subsection{\label{chap:S8}Data and code availability statement}
All datasets, models, and code used in this study are publicly available and can be obtained from the repositories listed below. 
The PanNuke \cite{Gamper_Koohbanani_etal._2019} and MoNuSAC \cite{Verma_Kumar_etal._2021} datasets are publicly accessible, and download information along with detailed descriptions can be found in their respective articles. Preprocessing scripts for PanNuke and MoNuSAC data, as well as individual cell extraction scripts, are available on GitHub \cite{Shvetsov_2025a}. The H-Optimus foundation model used in our experiments can be downloaded from the HuggingFace repository \cite{hoptimus2024}, and model information is available on GitHub \cite{Saillard_Jenatton_etal._2024}. In addition, the integration code for QuPath and the distilled model packaged in a Docker container are provided in the repository \cite{Shvetsov_2025b}, and paquo Python library is available from the authors GitHub repository \cite{Bayer_AG}.
\clearpage

\end{document}


\clearpage

\appendix

\subsection{Environment Modeling Details}

\paragraphc{Modeling underactuated joints.}
Since modeling underactuated joint structure is not directly supported, we approximate the relationship between each pair of actuated and underactuated joints by fitting a linear function $q_{u} = k \cdot q_{a} + b$, where $q_u$ is the underactuated joint angle and $q_a$ is the actuated joint angle. Note that parameters $k, b$ are included as tunable parameters to search over using our autotune module detailed in Section~\ref{sec:realsim}.

\subsection{Reward Design Details}

We design generalizable rewards based on the principle outlined in Section~\ref{sec:reward} and list task reward details below.

Both \textbf{grasp} and \textbf{lift} tasks can be defined with the following goal states: (1) finger contact with the object; (2) the object being lifted up to a goal position. Our reward design can, therefore, follow by combining the contact goal reward and the object goal reward terms:
\begin{equation}
r(s_h,s_o) = r_{contact}(s_h, s_o) + r_{goal}(s_o)
\end{equation}
where $s_h$ includes fingertip positions, $s_o$ includes object center-of-mass position, and all contact marker positions (if any).

Similarly, the \textbf{handover} task can be defined with the following goal states: (1) one hand's finger contact with the object; (2) object being transferred to an intermediate goal position while still in contact with the first hand; (3) the second hand's finger contact with the object; (4) object being transferred to the final goal position. Due to the hand switching, we introduce a stage variable $a \in \{0,1\}$ and design the reward as follows:
\begin{equation}
\begin{split}
r(s_h,s_o) & = (1 - a) \cdot ( r_{contact}(s_{h_A}, s_{o_A}) + r_{goal}(s_{o_A})) \\
& + a \cdot (r_{contact}(s_{h_B}, s_{o_B}) + r_{goal}(s_{o_B}))
\end{split}
\end{equation}
where $s_{h_A}, s_{h_B}$ denote fingertip positions of the engaged hand at each stage, $s_o$ denote object center-of-mass position and desirable contact marker positions (if any) at each stage. At completion of each stage, we also reward policy with a bonus whose scale increases as stage progresses.

\subsection{Policy Training Details}

\paragraphc{RL implementation.}
To learn the specialist policies, the observation space includes object position and robot joint position at each time step, and the action space is robot joint angles. We use Proximal Policy Optimization~\cite{schulman2017proximal} with asymmetric actor-critic as the RL algorithm. In addition to the policy inputs, we provide the following privilege state inputs to the asymmetric critic: arm joint velocities, hand joint velocities, all fingertip positions, object orientation, object velocity, object angular velocity, object mass randomization scale, object friction randomization scale, and object shape randomization scale. Both the actor and critic networks are 3-layer MLPs with units $(512,512,512)$.

\paragraphc{Domain randomization.}
\label{sec:dr}
Physical randomization includes the randomization of object friction, mass, and scale. We also apply random forces to the object to simulate the physical effects that are not implemented by the simulator. Non-physical randomization models the noise in observation~(e.g., joint position measurement and detected object positions) and action. A summary of our randomization attributes and parameters is shown in Table~\ref{table:dr}.
 
\begin{table}[!t]
\renewcommand\arraystretch{1.05}
\caption{Domain Randomization Setup.}
\centering
\begin{tabular*}{\linewidth}{l@{\extracolsep{\fill}}c}
\toprule
Object: Mass~(kg)             & [0.03, 0.1]    \\
Object: Friction              & [0.5, 1.5]     \\
Object: Shape                 & $\times\mathcal{U}(0.95, 1.05)$     \\
Object: Initial Position~(cm) &  $+\mathcal{U}(-0.02, 0.02)$ \\
Object: Initial $z$-orientation & $+\mathcal{U}(-0.75, 0.75)$ \\
Hand: Friction                & [0.5, 1.5]    \\
\midrule
PD Controller: P Gain         &  $\times\mathcal{U}(0.8, 1.1)$      \\
PD Controller: D Gain         &  $\times\mathcal{U}(0.7, 1.2)$     \\
\midrule
Random Force: Scale           & 2.0       \\
Random Force: Probability     & 0.2    \\
Random Force: Decay Coeff. and Interval & 0.99 every 0.1s     \\ 
\midrule
Object Pos Observation: Noise & 0.02      \\
Joint Observation Noise.      & $+\mathcal{N}(0, 0.4)$     \\
Action Noise.                 & $+\mathcal{N}(0, 0.1)$   \\
\midrule
Frame Lag Probability         & 0.1 \\
Action Lag Probability        & 0.1 \\
\midrule
Depth: Camera Pos Noise~(cm)      & 0.005  \\
Depth: Camera Rot Noise~(deg)      & 5.0  \\
Depth: Camera Field-of-View~(deg)  & 5.0  \\
\bottomrule
\end{tabular*}
\label{table:dr}
\end{table}

\subsection{Distillation Details}

To learn the generalist policy, we reduce the choices of observation inputs to the robot joint states and selective object states, including 3D object position and egocentric depth view, since privileged information is unavailable for sim-to-real transfer. To more efficiently utilize the trajectory data and improve training stability, for each sub-task specialist policy, we evaluate for 5000 steps over 100 environments, saving trajectories filtered by success at episode reset on the hard disk. We then treat the saved data as ``demonstrations'' and learn a generalist policy for each task with Diffusion Policies~\cite{chi2023diffusion}.

The proprioception and object position states are concatenated and passed through a three-layer network with ELU activation, hidden sizes of $(512,512,512)$, and an output feature size of $64$. For depth observations, we use the ResNet-18 architecture~\cite{he2016deep} and replace all the BatchNorm~\cite{ioffe2015batch} in the network with GroupNorm~\cite{wu2018group}, following~\cite{chi2023diffusion}. All the encoded features are then concatenated as the input to a diffusion model. We use the same noise schedule (square cosine schedule) and the same number of diffusion steps (100) for training as in \cite{chi2023diffusion}.
The diffusion output from the model is the normalized 7 DoF absolute desired joint positions of each humanoid arm and the 6 DoF normalized ($0$ to $1$) desired joint positions of each humanoid hand. We use the AdamW optimizer~\cite{kingma2014adam, loshchilov2017decoupled} with a learning rate of $0.0001$, weight decay of $0.00001$, and a batch size of 128. Following \cite{chi2023diffusion}, we maintain an exponential weighted average of the model weights and use it during evaluation/deployment.

\end{document}


