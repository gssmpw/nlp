\begin{table}[!t]
\centering
\setlength{\tabcolsep}{7pt}
\renewcommand{\arraystretch}{1.3}
\resizebox{\linewidth}{!}{%
\begin{tabular}{rrrrr}
\toprule
{Task} & {Grasping} & {Lifting} & {HandoverA} & {HandoverB} \\
\cmidrule(r){1-1}
\cmidrule(l){2-5}

\multicolumn{5}{l}{\textbf{Depth + Pos}} \\
\cmidrule(lr){1-1} \cmidrule(lr){2-5} 
Pickup & 10 / 10 & 10 / 10 & 10 / 10 & 10 / 10 \\
Task Success & 10 / 10 & 10 / 10 & 9 / 10 & 5 / 10 \\

\cmidrule(lr){1-1} \cmidrule(lr){2-5} 
\multicolumn{5}{l}{\textbf{Depth Only}} \\
\cmidrule(lr){1-1} \cmidrule(lr){2-5} 
Pickup & 2 / 10 & 0 / 10 & 0 / 10 & 0 / 10 \\
Task Success & 2 / 10 & 0 / 10 & 0 / 10 & 0 / 10 \\

\bottomrule
\end{tabular}
}
\caption{\textbf{Comparison of sim-to-real transfer performance between depth-and-position policy and depth-only policy.} We separate the \texttt{bimanual handover} task into two columns due to its longer horizon. The pickup success is an intermediate success metric that measures how often the hands successfully pick up the object of interest. We find that combining low-dimensional representation (3D object position) with depth enables easier sim-to-real transfer.}
\label{table:objrep}
\vspace{-1em}
\end{table}