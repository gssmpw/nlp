\section{Related Work\protect\footnote{This section was originally written in
  1995 but got revised with new references in 2025.}
}
\label{sec:related}

The technical literature on metaheuristics is vast, e.g., see
\cite{AhErOr02,LoMaSt02,MaPaRe18,PiRo10,SoSeGl18,Wi20} for
overviews.

Variable depth search was proposed by Kernighan and Lin and used in
\cite{KeLi70} for graph partitioning and in \cite{LiKe73} for
traveling salesman problem. In the latter reference, the variable
depth search is introduced as ``a general approach to heuristics that
we believe to be of wide applicability,'' and its ``considerable
success'' with graph partitioning, citing the former reference, is
also celebrated. It is interesting that it took almost two decades for
variable depth search mechanism to be used for problems other than
these two.

The complexity of finding good solutions using local search
techniques, including the KL and LK algorithms, is explored in
\cite{JoPaYa88,PaScYa90} under a new complexity class called
``polynomial local search'' (PLS).

Early references recognizing the benefits of variable depth search
include \cite{AhErOr02} and \cite{PiRo10} as a part of very-large
neighborhood search techniques, \cite{LoMaSt01} as part of the
iterated local search metaheuristic, \cite{ThOr89} as a special case
of the cyclic transfers metaheuristic, \cite{Gl96} as a special case
of the ejection chains metaheuristic, \cite{Gr03} and \cite{MeFr02} to
improve the performance of genetic algorithms.

Papers related to this work include the short survey of applications
in \cite{AhErOr02}, the applications in \cite{Gr03} and \cite{MeFr02}
to improve the performance of genetic algorithms, and the applications
to the vehicle routing problems in \cite{VaLeSc90,VaLeSc93}, the
quadratic assignment problem in \cite{Sk90} and more recently in
\cite{GoGo12,MeFr02,Pa09}, the maximum clique problem in
\cite{KaSaNa07}, and nurse rostering in \cite{BuCuQu07}.

Given this paper was originally published in 1995, it may also be
considered as an early reference for the variable depth search
mechanism and its general applications.

For further study, interested readers should consult these references
together with a huge number of references available on Arxiv, Google
Scholar, and generic search engines. A search query containing the
phrase ``variable depth search'' can help filter irrelevant results.