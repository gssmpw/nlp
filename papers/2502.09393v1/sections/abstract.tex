\begin{abstract}
Real-time autonomous systems utilize multi-layer computational frameworks to perform critical tasks such as perception, goal finding, and path planning. Traditional methods implement perception using occupancy grid mapping (OGM), segmenting the environment into discretized cells with probabilistic information. This classical approach is well-established and provides a structured input for downstream processes like goal finding and path planning algorithms. Recent approaches leverage a biologically inspired mathematical framework known as vector symbolic architectures (VSA), commonly known as hyperdimensional computing, to perform probabilistic OGM in hyperdimensional space. This approach, VSA-OGM, provides native compatibility with spiking neural networks, positioning VSA-OGM as a potential neuromorphic alternative to conventional OGM. However, for large-scale integration, it is essential to assess the performance implications of VSA-OGM on downstream tasks compared to established OGM methods. This study examines the efficacy of VSA-OGM against a traditional OGM approach, Bayesian Hilbert Maps (BHM), within reinforcement learning based goal finding and path planning frameworks, across a controlled exploration environment and an autonomous driving scenario inspired by the F1-Tenth challenge. Our results demonstrate that VSA-OGM maintains comparable learning performance across single and multi-scenario training configurations while improving performance on unseen environments by approximately 47\%. These findings highlight the increased generalizability of policy networks trained with VSA-OGM over BHM, reinforcing its potential for real-world deployment in diverse environments.
% Our results demonstrate that VSA-OGM maintains comparable learning performance while improving computational efficiency by two orders of magnitude. Lastly, we also perform mult-map training and evaluation where VSA-OGM presents increased generalization across all evaluation maps compared to BHM.
% Furthermore, we investigate an intermediate approach where the hyperdimensional representation of the environment directly informs a linear policy network, leveraging the inherit orthogonality and linear separability within hyperdimensional space and therefore removing the necessity of conventional decoding operations compared to traditional OGM methods.
\end{abstract}