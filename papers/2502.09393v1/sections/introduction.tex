\section{Introduction}
\label{sec:introduction}

% \hl{MP: for Index Terms we can also add brain-inspired learning to match our special session :D}

\noindent
% Today's robotic systems are defined by a multi-level autonomy framework segmenting the task into multiple interconnected sub-units. Individually, these sub-units perform tasks such mapping, localization, goal finding, and path planning~\cite{robotics13010012}. Once connected together, these systems allow robotic systems to achieve features culminating in high-level performance in tasks such as autonomous driving~\cite{REDA2024104630} and unknown environment exploration~\cite{zhao2024exploration}.
Contemporary robotic systems are structured around a multi-level autonomy framework, where complex tasks are divided into interconnected sub-units. Each sub-unit focuses on specific functions, such as mapping, localization, goal identification, or path planning~\cite{robotics13010012}. Together, these sub-systems enable high-level capabilities, such as autonomous driving~\cite{REDA2024104630} and unknown environment exploration~\cite{zhao2024exploration}.

% Localization and mapping are the most well studied pieces of the autonomy stack with the advent of simultaneous localization and mapping (SLAM) with the overarching goal of giving agents a information rich representation that can be utilized by downstream goal-finding and path planning modules to arrive at some ultimate objective. Traditional approaches to solve the mapping portion of SLAM use a method known as Occupancy Grid Mapping (OGM) that creates a continuous or discretized representation of the free space modeling the state of individual regions. Each region is typically defined with a probability value representing the likelihood of occupancy such that an obstacle is present within the region.
Localization and mapping have long been central to the multi-level autonomy framework, with simultaneous localization and mapping (SLAM) being a key technique for enabling autonomous agents to navigate complex environments~\cite{macario2022comprehensive}. SLAM provides agents with an information-rich representation of their surroundings, which can be leveraged by downstream goal-finding and path planning modules.
% , to achieve specific objectives.
% A traditional approach to the mapping component is Occupancy Grid Mapping (OGM), which constructs probabilistic environmental representation of the environment specifying the state of individual regions, with each region assigned a real value quantifying the likelihood of occupancy, indicating the presence of obstacles in the area.
A traditional approach to the mapping component is Occupancy Grid Mapping (OGM)~\cite{elfes1989using}. These algorithms construct probabilistic representations of the environment, where each region is assigned a real value quantifying the likelihood of occupancy.
% indicating the presence of obstacles in the area.

OGM literature is well-established, providing a structured framework for downstream processes such as goal finding and path planning algorithms at the expense of exponential Big O complexity~\cite{wilson2022convolutional}. Recently, biologically inspired mathematical frameworks, specifically vector symbolic architectures (VSA), have been applied to probabilistic OGM in hyperdimensional space~\cite{snyder2024brain}. This novel approach, known as Vector Symbolic Architectures for Occupancy Grid Mapping (VSA-OGM)~\cite{snyder2024brain}, supports integration with spiking neural networks, making it a promising neuromorphic alternative to conventional OGM techniques.
% More importantly, VSA-OGM reduces the aforementioned exponential Big O complexity to constant values while maintaining similar performance characteristics.
However, to facilitate large-scale integration with existing systems, it is crucial to evaluate VSA-OGM's performance on downstream tasks relative to traditional methods.

\footnotetext{DISTRIBUTION STATEMENT A. Approved for public release; distribution is unlimited. OPSEC \# 9329 approved for Release.}
% \footnotetext{The source code for this work will be available for the camera ready version.}
This study investigates the efficacy of VSA-OGM compared to a well-established OGM technique, Bayesian Hilbert Maps (BHM)~\cite{senanayake2017bayesian}, within reinforcement learning (RL) frameworks for goal finding and path planning. Experiments are conducted across controlled environments for unknown environment exploration~\cite{Koutras2021MarsExplorer} and autonomous driving inspired by the F1-Tenth challenge~\cite{Brunnbauer_racecar_gym}. Our results indicate that VSA-OGM not only maintains robust learning performance but also significantly enhances policy generalizability to different starting conditions and diverse unseen environments.
% Additionally, we explore an intermediate approach where hyperdimensional representations directly inform a linear policy network, leveraging the inherent orthogonality and linear separability of hyperdimensional space. This method eliminates the need for conventional decoding operations, further reducing computational complexity and distinguishing VSA-OGM from traditional OGM methods.
In summary, the major contributions of this paper are as follows:
\begin{itemize}
    % \item \hl{We evaluate the performance characteristics of a hyperdimensional OGM technique (VSA-OGM}~\cite{snyder2024brain}) compared to an existing probabilistic approach (BHM~\cite{senanayake2017bayesian}).
    \item We design environment wrappers for two distinct reinforcement learning scenarios: one simulating a rover exploring unknown regions~\cite{Koutras2021MarsExplorer}, and another inspired by the F1-Tenth driving challenge~\cite{Brunnbauer_racecar_gym}. These wrappers enable seamless integration with OGM methods.
    % \item Our experiments are conducted using various parameter combinations and environmental configurations to evaluate performance across a diverse range of scenarios.
    % \item We also highlight a unique capability of VSA-OGM where the learned hyperdimensional memories can be directly leveraged by downstream linear policy networks and avoid unnecessary decoding operations.
    \item Our results, conducted using various parameter combinations and environmental configurations to evaluate performance across a diverse range of scenarios, highlight that VSA-OGM achieves comparable performance to traditional approaches while improving policy generalization on unseen environments and different starting conditions by approximately 47\%.
    \item We perform multi-map training and evaluation across both environments with both OGM methods and VSA-OGM presents increased generalization, on average, across all evaluation maps and unseen scenarios.
\end{itemize}