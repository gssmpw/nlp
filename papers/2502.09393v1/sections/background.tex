\section{Background}
\label{sec:background}

% -------------------------------------
% Background on Occupancy Grid Mapping
% -------------------------------------
\noindent
OGM techniques have been developed to incorporate semantically labeled spatial information and give autonomous agents critical information to accomplish diverse tasks such as unknown environment exploration~\cite{robotics13010012} and path planning~\cite{zhao2024exploration}. The field has diverged into three primary sub-fields with each being defined by unique computational attributes.

\textit{Traditional methods} are the most studied from the foundational work of Elfes~\cite{elfes1989using} to more recent approaches like BHM~\cite{senanayake2017bayesian}. These approaches perform dense probabilistic calculations to learn from semantically-labeled spatial information at the cost of cubic computational complexity. BHM is a flexible framework that allows users to dictate the retention of the entire covariance matrix or only the variance. This enables BHM to scale between offline mapping scenarios with cubic complexity or real-time edge applications sub-cubic complexity, respectively.  
A newer version of BHM, known as Fast Bayesian Hilbert Maps (Fast-BHM)~\cite{zhi2019continuous}, reduces computational complexity to sub-quadratic levels by assuming independence between voxels and not utilizing a full covariance matrix. We were unable to include Fast-BHM in this study because their code is not publically available.

With the advent in AI, \textit{neural methods} have emerged with the goal of compressing probabilistic information into the black-box latent space of weights and biases~\cite{evilog}. While these approaches are effective for OGM, the black-box nature of deep neural networks makes them difficult to validate in safety critical environments such as autonomous driving~\cite{REDA2024104630}. Moreover, these methods must be trained on each operating domain to avoid performance degradation from domain shift. We limit the scope of our comparison to only traditional methods because we are evaluating the generalizability of OGMs in unknown environments which would require domain specific retraining for \textit{neural methods}.
% \hl{MP: Any comments on their real-time performance? I would add a bit more details on why you do not consider this method at all, real-time, pre-training, and the fact that you specifically looking at generalizability of OGMs in unknown environment which makes this method naturally not suitable due to pre-training requirement.}

A third subfield, \textit{neuro-symbolic methods} have emerged. These approaches, Convolutional Bayesian Kernel Inference (ConvBKI)~\cite{wilson2022convolutional} and VSA-OGM~\cite{snyder2024brain} combine benefits of both prior methods. Although ConvBKI and VSA-OGM have shown improvements in algorithmic complexity with respect to environment size and density, their efficacy for downstream path planning and environmental exploration remains a critical concern because they have not been compared against traditional OGM methods.
% \hl{We limit the scope our evaluation to VSA-OGM because ConvBKI was not designed for the 2D environments we utilize in this study. In future works, where a suitable 3D hyperdimensional mapping method is available, we would like to extend this study to include ConvBKI.}

VSA-OGM leverages VSAs, a framework also known as hyperdimensional computing. VSAs provide a way to represent and manipulate information using high-dimensional vectors, drawing parallels to how the brain encodes and processes data. This biologically motivated approach serves as a mathematical framework that approximates higher-order cognitive functions in neural networks~\cite{10.1016/0004-3702(90)90007-M, eliasmith2013build}. A specific VSA architecture, known as Spatial Semantic Pointers (SSPs)~\cite{eliasmith2013build}, enables probabilistic inference over continuous representations in hyperdimensional space through a process known as fractional binding~\cite{komer2020biologically}.

SSPs operate on high dimensional vectors that are unit length and generated by performing the inverse discrete Fourier transform on uniformly distributed phasors between $-\pi$ and $\pi$. SSPs utilize two primary operations: binding 
and bundling. Binding, analogous to multiplication, combines two or more input vectors into a final invertible
representation that is distinct from each input. Bundling, analogous to addition, combines multiple input
vectors into their superposition. SSPs follow the same vector operations described in Holographic Reduced Representations (HRR)~\cite{plate1991holographic} where binding is implemented as circular convolution and bundling is implemented as element-wise addition. SSPs support encoding continuous values by exponentiating an axis vector $\phi_x$ in the complex domain by the desired axis values $x$ through a process known as fractional binding:

\begin{equation}
    \phi(x) = \mathcal{F}^{-1}(\mathcal{F}(\phi)^{x/l}),
\end{equation}

where $l$ is the length-scale parameter~\cite{komer2020biologically}. Probabilistic inference is performed with SSPs through Hadamard products between vectors. This process returns a quasi-kernel density estimator over the axis approaching a sinc function as the vector dimensionality approaches infinity~\cite{furlong2022fractional}. The length scale parameter in the fractional binding operation controls the width of this kernel and can be used to adjust the resolution and noise resilience of the SSPs.

SSPs are the basis of VSA-OGM and enable probabilistic modelling of occupancy in hyperdimensional space. More information on the underlying mathematics of VSAs and HRRs can be found in~\cite{10.1016/0004-3702(90)90007-M, plate1991holographic, plate1994distributed, gayler2004vectorsymbolicarchitecturesanswer, eliasmith2013build}. More specific details on the extension of SSPs to create VSA-OGM can be found in~\cite{snyder2024brain}. Given this drastically different approach to probabilistic computation compared to traditional methods, a major question remains around the ability of downstream path planning algorithms to leverage the quasi-probabilistic properties of VSA-OGM. This stands as the central question of this work where we compare the efficacy of VSA-OGM against BHM for reinforcement learning based path planning and environmental exploration with diverse environments, scenarios, and parameter combinations.