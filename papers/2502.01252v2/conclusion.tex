\section{Conclusion \& Limitations}
Drawing insights from existing literature and real-world applications, we define a new class of games called ACCES games. We prove the existence of NE for ACCES games, providing a fundamental basis for solution algorithms. Two NE solvers are introduced, namely CCDO and its practical version CCDO-RL, along with original theoretical analysis and ABRs' impact on convergence.
%guarantees. %Due to the effectiveness and widespread usage of RL on COPs, the deployable algorithm based on CCDOA, CCDO-RL, is put forward to solve ACCES games on COPs. 
Empirical results show that CCDO-RL can converge to approximate NE in a small number of iterations. The protagonist policy obtained via CCDO-RL has better average rewards against adversarial perturbations and shows great generalizability on unseen graphs. A potential limitation of our method is scalability -- our experiments mainly focus on small COPs (20 and 50 nodes). While scalability is not the focus of this study, it does remain unexplored and deserves more investigation. Our work also opens up a new area of research centering on ACCES games, and more broadly asymmetric games such as the uniqueness of NE, as well as more efficient and practical algorithms.
\newpage