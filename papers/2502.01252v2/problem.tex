\section{Preliminaries}
% 写入所有背景知识,主要是game theory的,要concise and specific.

\subsection{Two-Player Asymmetric Combinatorial-Continuous zEro-Sum (ACCES) Games}

Most two-player zero-sum strategic games are described as a payoff matrix $\Pi_{n\times m}$ where the rows and columns represent pure strategies for the two players. This does not hold for ACCES games because the strategy space for the continuous player is infinite. Hence, we provide the first formal formulation of ACCES games.

%But for combinatorial-continuous zero-sum games, the general payoff can not be depicted as a matrix because the strategy space for the continuous player is infinite. 

Formally, we represent a two-player ACCES game a tuple $\{X, Y, u\}$, where $X$ is the combinatorial but finite space, and $Y$ is a compact and infinite metric space, as the pure strategy space for players 1 and 2 respectively. $u$ is the utility function mapping the joint strategy space $X \times Y$ to a scalar $\mathbb{R}$, with the continuity on $Y$ when fixing $x \in X$. The utility function of Player 1 is $u$, and for Player 2 is $-u$. For the security patrolling game exemplified in the introduction, $X$ should be all routes that satisfy the distance constraint, $Y$ is the real vector interval $[0, 1]$ for the attack probability $p_i$ on each target $i=1,...,N$, and $u$ is the expectation of successfully attacked target negative values. 

%The utility function of Player 2 is continuous $-u: \mathcal{X} \times \mathcal{Y} \rightarrow \mathbb{R}$ when fixing $x \in \mathcal{X}$. 

% \textcolor{red}{The \textit{pure strategy} of Player 1 is ... The pure strategy of Player 2 is ...}  
The mixed strategy in the combinatorial-continuous game is defined separately because two players own entirely different forms of strategy spaces. For Player 1, the set of \textit{mixed strategies} can be written as $\bigtriangleup_{X} \triangleq \{p = [p(x_1), ..., p(x_{|X|})] | \sum_{i=1}^{|X|} p(x_i) = 1, p(x_i) \geq 0 \}$, where $p(x_i)$ is regarded as the chosen probability of the pure strategy $x_i \in X$. For Player 2, a mixed strategy is a Borel probability measure $q$ on $Y$ which can be seen as a probability distribution function $q:\mathcal{F} \rightarrow [0, 1]$, where $\mathcal{F}$ is $\sigma$- algebra of $Y$. The set of mixed strategies of Player 2 is denoted by $\bigtriangleup_{Y}$. Every mixed strategy in $\bigtriangleup_{X}$ corresponds to a distribution on all feasible routes in the security patrolling game, and that in $\bigtriangleup_{Y}$ passes as a cumulative distribution function defined on $[0, 1]^N$.

Due to the infiniteness of strategy space $\mathcal{Y}$, the support of $q$ may be infinite. Given a mixed strategy $(p, q) \in \bigtriangleup_{X} \times \bigtriangleup_{Y} \triangleq \bigtriangleup$, the \textit{expected utility function} of Player 1 can be defined as 
\begin{eqnarray}
    \begin{aligned}
        U(p, q) = \sum_{x \in X} \int_{y \in Y} p(x) u(x,y) dq.
    \end{aligned}
\end{eqnarray}
Correspondingly, the expected utility function of Player 2 is $-U(p, q)$.

\subsection{Nash Equilibrium in Two-Player ACCES Games}
% 基本NE定义,\epsilon-NE定义,best response定义以及\epsilon-br定义。
In two-player ACCES games, a mixed strategy pair $(p^*, q^*)$ is \textit{Nash equilibrium (NE)} if and only if
\begin{eqnarray}
    \begin{aligned}
        U(p, q^*) \leq U(p^*, q^*) \leq U(p^*, q), \forall p \in \bigtriangleup_{X}, q \in \bigtriangleup_{Y}.
    \end{aligned}
\end{eqnarray}

%In matrix games, NE can be solved directly using linear programming (LP). %But in ACCES games, it is infeasible to use the LP method because of the infiniteness of the continuous strategy space for Player 2. Besides, possibly uncountable supports in the game make this problem more difficult which means that we can not find NE with acceptable time complexity. Hence, it is effective enough to find \textbf{$\epsilon$- NE} $(p^*, q^*)$ which is denoted by
Additionally, we denote \textit{$\epsilon$- NE} as a mixed strategy pair $(p^*, q^*)$ which satisfies
\begin{eqnarray}
    \begin{aligned}
        U(p, q^*) - \epsilon \leq U(p^*, q^*) \leq U(p^*, q) + \epsilon, \forall p \in \bigtriangleup_{X}, q \in \bigtriangleup_{Y}.
    \end{aligned}
\end{eqnarray}

\textit{Best response} $\mathbb{BR}_i(\pi_{-i})$ defines the best pure strategy for Player $i$ for a fixed mixed strategy $\pi_{-i}$ of the other player $-i$. In ACCES games, the set of best responses for the two players are:
\begin{eqnarray}
    \begin{aligned}
        \mathbb{BR}_1(q) = \{x \in X| U(x, q) = \max_{x' \in X} U(x', q) \},
        \mathbb{BR}_2(p) = \{y \in Y| U(p, y) = \min_{y' \in Y} U(p, y') \}.
    \end{aligned}
\end{eqnarray}
In many situations, finding the best response is inherently difficult, especially in most combinatorial optimization problems which are $NP$-hard. %, there still lacks the polynomial algorithm to solve these problems exactly unless $P=NP$. 
Approximate or heuristic algorithms are often used to sacrifice solution accuracy for faster computation. We use \textit{$\epsilon$- best response} $\mathbb{BR}_i^{\epsilon}(\pi_{-i})$ to define the solution that is no worse than the ground truth best response by $\epsilon$:
\begin{eqnarray}
    \begin{aligned}
        \mathbb{BR}_1^{\epsilon}(q) = \{x \in X| U(x, q) \geq \max_{x' \in X} U(x', q) - \epsilon \}, 
        \mathbb{BR}_2^{\epsilon}(p) = \{y \in Y| U(p, y) \leq \min_{y' \in Y} U(p, y') +\epsilon \}.
    \end{aligned}
\end{eqnarray}
