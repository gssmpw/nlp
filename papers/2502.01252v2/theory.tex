
\section{The Existence of Nash Equilibrium}

We first study the existence of Nash equilibrium in ACCES games, which is a critical step before designing any actual solutions. For a two-player ACCES game $\mathcal{G}=\{X, Y, u\}$, the strategy set $X$ of Player 1 is finite consisting of certain permutations/combinations of nodes, although the number of the strategy set is possibly exponentially large. In contrast, the strategy set $Y$ of Player 2 is an infinite and compact set. Although the utility function of Player 2 are continuous on $Y$ when fixing $x \in X$, but its strategy's infiniteness disqualifies the finite condition of matrix games and makes the convergence to NE less straightforward. Meanwhile, the discreteness of $X$ destroys the continuity of the utility function on $X \times Y$.
% For a two-player ACCES game $\mathcal{G}=\{X, Y, u\}$, the strategy set $X$ of Player 1 is finite in theory consisting of certain permutations/combinations of nodes, although the number of the strategy set is possibly exponentially large. In contrast, the strategy set $Y$ of Player 2 has some good properties such as the continuity of the utility function on $Y$ and compactness of its strategy space, but its infiniteness disqualifies the finite condition of matrix games and makes the convergence to NE less straightforward. Meanwhile, the discreteness of $X$ destroys the continuity of the utility function $u(x,y)$ on $X \times Y$. %Hence we reanalyze the property of the joint strategy space and prove the existence of mixed equilibrium.
To see whether the existence of mixed strategy NE still holds in ACCES games, we must better understand the game structure. Our thought flow is as follows.

On a high level, we first prove Proposition \ref{prop1} of weakly sequential compactness in the mixed strategy product space of ACCES games. Then, the continuity of the expected utility function on the product space, which contributes to the existence proof and the following convergence of algorithms in Section 5, is proven in Proposition \ref{prop2}. Note that these are two key technical novelties that not only are critical intermediate steps for the proof of the existence of NE, but also build the foundation of the analysis of convergence to NE of our proposed algorithms in Section \ref{sec_alg}.

%introduce the definition of weak convergence, common in the space of probability measures (Definition \ref{def3} in the Appendix). We use it to derive the equivalence condition of the weakly sequential compactness of separable product spaces (Lemma \ref{lem1} in the Appendix), which further

%On a high level, we first introduce the definition of weak convergence, common in the space of probability measures (Definition \ref{def3} in the Appendix). We use it to derive the equivalence condition of the weakly sequential compactness of separable product spaces (Lemma \ref{lem1} in the Appendix), which further leads to Proposition \ref{prop1} of weakly sequential compactness in the mixed strategy product space of ACCES games. Then, the continuity of the expected utility function on the product space, which contributes to the existence proof and the following convergence of algorithms in Section 5, is proven in Proposition \ref{prop2}. \textcolor{blue}{It should be noted that Propositions \ref{prop1} and \ref{prop2} are the two key technical novelties in this section, and not only are critical intermediate steps for the proof of the existence of NE, but also build the foundation of the analysis of convergence to NE of our proposed algorithms in Section \ref{sec_alg}.}

% of this section because they assure the basic properties of combinatorial-continuous games, i.e. weakly sequential compactness of the product space and continuity of the expected utility function, which are critical in the proof of the algorithm convergence. \hp{Are Props 1-2 for existence or convergence? Are Props 3-4 not important?}

% \begin{definition}\label{def3} {\normalfont[Weakly Convergence.]}
%     Suppose $S$ is a space, probability measures $P_n$ weakly converges to $P$, written by $P_n \Rightarrow P$, if for every bounded continuous function $f$, $$\lim_{n \rightarrow \infty} \int_S f dP_n \rightarrow \int_S f dP.$$
% \end{definition} 
% If any sequence in a set has a weakly convergent subsequence, the set is weakly sequentially compact.
% \begin{lemma}\label{lem1}
%     $\mathcal{S'},\mathcal{S''}$ are uncorrelated general metric spaces and $P', P''$ are the probability measure on $\mathcal{S'},\mathcal{S''}$ respectively. Define $\mathcal{T} \triangleq S' \times S''$ as the product space of $S'$ and $S''$. if $\mathcal{T}$ is separable, then $P_n' \times P_n'' \Rightarrow P' \times P''$ if and only if $P_n' \Rightarrow P'$ and $P_n'' \Rightarrow P''$. \rm{(\citep{Myerson1991GameT}, Theorem 2.8)}
% \end{lemma}
\begin{proposition} \label{prop1} {\normalfont[Weakly Sequential Compactness.]}
    Set the ACCES game is $\mathcal{G} = (X, Y,u)$, where $X$ is finite, $Y$ is a nonempty compact metric space, and the utility function $u$ is continuous on $Y$ fixing $x \in X$. Then the joint mixed strategy space $\bigtriangleup \triangleq \bigtriangleup_X \times \bigtriangleup_Y$ is weakly sequentially compact. 
\end{proposition}
\begin{proof-sketch}
    To prove the product space $X \times Y$ is weakly sequential compact, we just need to prove two parts, weakly sequential compactness and separability of $X, Y$ based on Lemma \ref{lem1}. See the full proof in Appendix \ref{Appendix_A1}.
\end{proof-sketch}
% \hp{For each proposition/lemma/theorem that we derived, add a proof sketch to summarize what is the key idea of the proof and say something like Please see Appendix xxx for the detailed proof.}

% According to the proof in the appendix \ref{Appendix_A1}, the product space $X \times Y$ is compact too. We re-emphasize that the proofs of Propositions \ref{prop1} and \ref{prop2} are two key technical contributions of this section, and they also serve as the basis for the proofs of both the existence of NE (this section) and convergence to NE using our designed algorithms (next section).
\begin{proposition} \label{prop2} {\normalfont[Continuity of Expected Utility Function.]}
    The expected utility function $U(p,q) \triangleq \sum_{x\in X}\int_{y\in Y} p(x)u(x,y)dq$ is continuous on the joint mixed strategy space $\bigtriangleup$, $\forall p \in \bigtriangleup_X, q \in \bigtriangleup_Y$.
\end{proposition}
%\yh{This is self-created proof too. In a continuous game it is obviously true.}
\begin{proof-sketch}
    We prove the continuity of the expected utility function by definition. First, define the metric distance on mixed strategy sets $\bigtriangleup_{X}, \bigtriangleup_{Y}$ and their product space $\bigtriangleup_{X} \times \bigtriangleup_{Y}$. Following this, the distance between two mixed strategy pairs $(p, q)$ and $(p', q')$ can be scaled to the distance sum between $p, p'$ and $q, q'$ because of the compactness of $Y$, the continuity of utility function on $Y$, and Proposition \ref{prop1}. The full proof is provided in Appendix \ref{Appendix_A1}.
\end{proof-sketch}

Via Proposition \ref{prop2} and the continuity of $U$ on $Y$, the following two statements hold:
\begin{itemize}
    \item When $p_n \Rightarrow p$ in $\bigtriangleup_X$, $q_n \Rightarrow q$ in $\bigtriangleup_Y$, $U(p_n, q_n) \rightarrow U(p, q).$
    \item When $p_n \Rightarrow p$ in $\bigtriangleup_X$, $y_n \rightarrow y$ in $Y$, $U(p_n, y_n) \rightarrow U(p, y).$
\end{itemize}

% {\color{red} We re-emphasize that the proofs of Propositions \ref{prop1} and \ref{prop2} are two key technical contributions of this section, and they also serve as the basis for the proofs of both the existence of NE (this section) and convergence to NE using our designed algorithms (next section).}

Secondly, for the proof of equilibrium existence, we build on the idea in \citep{Myerson1991GameT} which approximates the strategy spaces by finite grids. To describe the approximation and the feasibility of approximation by finite games, we first introduce definitions of $\alpha$-approximate games and essentially finite games. Based on these definitions, we establish Propositions \ref{epsilon_alpha}, and \ref{ess_appro_exist}, where the proofs are provided in Appendix \ref{Appendix_A1}.
\begin{definition}{\normalfont[$\alpha$-Approximate Game.]}
    Assume there exist two strategic games $\mathcal{G}=\langle X, Y, u \rangle$, $\mathcal{G}'= \langle  X, Y,  u' \rangle$ in which $u$ and $u'$ are bounded and measurable utility functions. If every joint strategy $(x, y) \in X \times Y$, $|u(x, y)-u'(x, y)| \leq \alpha$, then $\mathcal{G}'$ is an $\alpha$-approximation of $\mathcal{G}$. 
    % Assume two strategic games $\mathcal{G} = \langle N, (X_n), (u_n) \rangle$, $\mathcal{G}' = \langle N, (X_n), (u_n') \rangle$, of which utility functions $u_n$ and $u_n'$ are bounded and measurable. If for $\forall n \in N, \forall x \in \bigtimes_{n=1}^N X_n$, $|u_n(x)-u_n'(x)| \leq \alpha$, then $\mathcal{G}'$ is an $\alpha$-approximation of $\mathcal{G}$.
\end{definition}
% \yh{The content of Proposition 3, 4, and 5 are identical to that in continuous games which can be seen in  \cite{Myerson1991GameT} Chapter 3 Theorem 3.3, 3.4, and 3.5, but to prove these in our game setting I improve some processes of proof.}
\begin{definition}{\normalfont[Essentially Finite Game.]}
    The game $\mathcal{G} = \langle X, Y, u \rangle$ is essentially finite, if and only if there exists some finite strategic game $\hat{\mathcal{G}} = \langle \hat{X}, \hat{Y},  \hat{u} \rangle$ and measurable functions $f_1: X \rightarrow \hat{X}$, $f_2: Y \rightarrow \hat{Y}$ s.t. $u(x, y) = \hat{u}(f_1(x), f_2(y)), \forall (x, y) \in X \times Y.$

    % The game $\mathcal{G} = \langle N, (X_n), (u_n) \rangle$ is essentially finite, if and only if there exists some finite strategic game $\hat{\mathcal{G}} = \langle N, (Y_n), (v_n) \rangle$ and measurable functions $f_n: X_n \rightarrow Y_n, n \in N$, s.t. $u_n(x) = v_n(f_1(x_1),..., f_N(x_N)), \forall n\in N, \forall x = (x_1,..., x_N) \in \bigtimes_{n=1}^N X_n.$
\end{definition}

% Combining these two definitions with $\epsilon$-equilibrium, we can establish Propositions \ref{epsilon_alpha}, and \ref{ess_appro_exist}, where the proofs are provided in Appendix \ref{Appendix_A1}.

\begin{proposition}\label{epsilon_alpha} {\normalfont[Approximate NE of ACCES.]}
    $\mathcal{G}'=\langle X, Y, \Tilde{u} \rangle$ is $\alpha$-approximation of $\mathcal{G}=\langle X, Y, u \rangle$, where $\mathcal{G}$ is an ACCES game. $(p^*, q^*)$ is an $\epsilon$-equilibrium of $\mathcal{G}'$, then $(p^*, q^*)$ is an $(\epsilon + 2\alpha)$-equilibrium of $\mathcal{G}$.
\end{proposition}

\begin{proposition}\label{ess_appro_exist} {\normalfont[Essentially Finite of ACCES.]}
    For an ACCES $\mathcal{G}$, $\forall \alpha >0$, there exists an essentially finite strategic game $\hat{\mathcal{G}}=\langle X, \hat{Y}, \hat{u}\rangle$, s.t. $\hat{\mathcal{G}}$ is $\alpha$-approximation of $\mathcal{G}$.
\end{proposition}

\begin{proposition}\label{epsi_equil_converge} {\normalfont[Convergence of Approximate ACCES NE.]}
    $\mathcal{G}$ is an ACCES game, for each $n$, $(p_n, q_n)$ is $\epsilon_n$-NE of $\mathcal{G}$, $(p_n, q_n) \Rightarrow (p^*, q^*)$, $\epsilon_n \rightarrow \epsilon$, then $(p^*, q^*)$ is an $\epsilon$-equilibrium of $\mathcal{G}$.
\end{proposition}

Based on Chapter 3 in \cite{Myerson1991GameT} and Proposition \ref{prop2}, Proposition \ref{epsi_equil_converge} holds naturally. On account of Proposition \ref{epsilon_alpha}, \ref{ess_appro_exist}, and \ref{epsi_equil_converge}, the existence of equilibrium can be obtained. We have provided a further discussion of the existence of NE for $N$-player ACCES games in Appendix \ref{Appendix_A2}.

\begin{theorem}\label{existence} \textbf{\emph{[Existence of NE]}}
    $\mathcal{G}=\langle X, Y, u \rangle$, where $X$ is finite space, $Y$ is nonempty compact metric space, $u: X \times Y \rightarrow \mathbb{R}$ is a continuous utility function on $Y$ when fixing $x \in X$. Game $\mathcal{G}$ has a mixed strategy Nash equilibrium.
\end{theorem}
\begin{proof-sketch}
    For any sequence $\{\alpha_k\} \rightarrow 0$, there exists an essentially finite game sequence $\{\mathcal{G}_k\}$ (Prop \ref{ess_appro_exist}) such that its NE $\{(p_k^*, q_k^*)\}$ is $2\alpha_k$- NE of the initial game $\mathcal{G}$ (Prop\ref{epsilon_alpha}). We can prove that the sequence $\{(p_k^*, q_k^*)\}$ converges and its convergent point $(p^*, q^*)$ is the NE of the game $\mathcal{G}$ (Prop \ref{epsi_equil_converge}).
\end{proof-sketch}

\begin{remark}
    The proving idea of approximating by finite games is one feasible and concise way to prove Theorem \ref{existence}. After analyzing the basic properties in Proposition \ref{prop1} and \ref{prop2}, the existence of NE can also be proved by the fixed point theorem in \citep{glicksberg1952further} while going a little bit of a detour to fit the problem into its proof framework. 
\end{remark}
