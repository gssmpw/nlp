\section{Literature Review}
J. Jacobs, author of \textit{The Death and Life of Great American Cities} \cite{jacobs_death_1993}, was among the first to focus on urban vitality, highlighting elements such as urban density, land use diversity, and urban planning. Jacobs criticizes modern city planning for large-scale developments that prioritize structural expansion over the needs and scale of human interaction. In her methodology, she prioritizes empirical observation over large theoretical models. Also, J. E. Drewes and M. Van Aswegen \cite{drewes_determining_2010}, and X. Li and al. \cite{li_data_2020}, define vitality as the ability of an entity to survive or function effectively. Thus, an urban Vitality Index represents the capacity of an urban center to remain viable, meet community needs, and enhance residents' quality of life. Drewes and Van Aswegen’s index is constructed using normative scores for welfare, satisfaction, social dynamics, and spatial distribution, calculated based on the average quartiles of the associated variables. Meanwhile, Dessureault et al. \cite{dessureault_unsupervised_2021} include eight variables, such as material and social deprivation indices, housing condition, property values, vacancy rates, renovation permit values, and rents. They use a genetic algorithm and k-means to cluster 135 areas into 10 clusters, while also using RF for weighting features. On the other hand, Haynes, Hook, Chiodi Grensing, and Ecklund \cite{haynes_analysis_2018} propose a Metropolitan Vitality Index (MVI) with 24 indicators, including the number of small businesses and physical activity spaces. The studied areas were ranked from 1 to 381 for each indicator, and those rankings were then summed. Finally, the MVI values were recalculated within a range of 80 to 120, with higher scores indicating better performance. K. Scott \cite{scott_katherine_katherine_2010} suggests a Community Vitality Index based on the direction of data trends over time for variables such as property crimes, volunteerism, and violent crimes. Analogously, Liu, Gou, and Xiong \cite{liu_vital_2022} examine growth, diversity, and mobility indicators, such as youth proportion, economic and racial diversity, and service proximity. MinMax standardization is applied to the indicators, and feature weighting is performed using the entropy method, along with scores provided by urban planning scholars. Additionally, correlations between different dimensions of urban vitality are analyzed. Herath and Mittal \cite{herath_adoption_2022} provide a comprehensive review of smart cities and artificial intelligence (AI), citing Singapore and Zurich are among the top ten smart cities. Eighty percent of studies on urbanization and machine learning use clustering, as detailed by Wang and Biljecki \cite{wang_unsupervised_2022}, with popular algorithms such as k-means, self-organizing maps, and DBSCAN. Unsupervised algorithms are also used, as demonstrated by Dessureault \cite{dessureault_unsupervised_2021} and Li \cite{li_data_2020}. To evaluate the effectiveness of clustering, P.J. Rousseeuw introduced the Silhouette Score \cite{rousseeuw_silhouettes_1987}, a metric that measures how well points are grouped within their respective clusters. The k-means algorithm, used for the classification of areas, was originally developed by S. Lloyd \cite{lloyd_least_1982}. This unsupervised clustering method is renowned for its simplicity, efficiency, and speed, as it minimizes the sum of squared distances between data points and their respective cluster centers, called 'centroids'. Additionally, data imputation is performed using K-Nearest Neighbors, introduced by Thomas Cover and Peter Hart in 1967 \cite{cover_nearest_1967}. Furthermore, feature weighting is done using the RF algorithm, introduced by Leo Breiman in 2001 \cite{breiman_random_2001}.