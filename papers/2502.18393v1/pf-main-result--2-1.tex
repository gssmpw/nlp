\section{Proof of the Main Results}
\label{outline:pf-main-result}

In this section, the main results---\THEOREM \ref{thm:approx-error:sparse} and \COROLLARY \ref{corollary:approx-error:logistic-regression}% %and \ref{corollary:approx-error:probit}
---are proved, contingent on the correctness of the main technical results---\THEOREM \ref{thm:main-technical:sparse} and \COROLLARY \ref{corollary:main-technical:logistic-regression}% %and \ref{corollary:main-technical:probit}
---and some auxiliary results, whose proofs are deferred to \SECTIONS \ref{outline:pf-main-technical-result} and \ref{outline:concentration-ineq}.

%%%%%%%%%%%%%%%%%%%%%%%%%%%%%%%%%%%%%%%%%%%%%%%%%%%%%%%%%%%%%%%%%%%%%%%%%%%%%%%%%%%%%%%%%%%%%%%%%%%%
%%%%%%%%%%%%%%%%%%%%%%%%%%%%%%%%%%%%%%%%%%%%%%%%%%%%%%%%%%%%%%%%%%%%%%%%%%%%%%%%%%%%%%%%%%%%%%%%%%%%>
%%%%%%%%%%%%%%%%%%%%%%%%%%%%%%%%%%%%%%%%%%%%%%%%%%%%%%%%%%%%%%%%%%%%%%%%%%%%%%%%%%%%%%%%%%%%%%%%%%%%

\subsection{Intermediate Results}
\label{outline:pf-main-result|intermediate}

Before \THEOREM \ref{thm:approx-error:sparse} can be proved, two auxiliary results, stated below as \LEMMA \ref{lemma:error:deterministic} and \FACT \ref{fact:recurrence}, are needed.
The first of these intermediate results---whose proof is deferred to \SECTION \ref{outline:pf-main-result|pf-intermediate}---will allow the main technical result, \THEOREM \ref{thm:main-technical:sparse}, as well as its corollaries, to be related to the error of the approximations iteratively produced by BIHT (\ALGORITHM \ref{alg:biht:normalized}).
%whereas the latter result will facilitate the calculation of a close-form bound on the iterative approximation errors.

%|>>|*******************************************************************************************|>>|
%|>>|*******************************************************************************************|>>|
%|>>|*******************************************************************************************|>>|
\begin{lemma}
\label{lemma:error:deterministic}
%
Let
%|>>|:::::::::::::::::::::::::::::::::::::::::::::::::::::::::::::::::::::::::::::::::::::::::::|>>|
%\(  \Vec{\uV}, \Vec{\vV} \in \SparseRealSubspace{\k}{\n}  \),
\(  \Vec{\uV} \in \SparseRealSubspace{\k}{\n}  \) and
\(  \Vec{\vV} \in \R^{\n}  \),
%|<<|:::::::::::::::::::::::::::::::::::::::::::::::::::::::::::::::::::::::::::::::::::::::::::|<<|
and let
%|>>|:::::::::::::::::::::::::::::::::::::::::::::::::::::::::::::::::::::::::::::::::::::::::::|>>|
\(  \JCoords, \JCoordsu, \JCoordsv \subseteq [\n]  \),
%|<<|:::::::::::::::::::::::::::::::::::::::::::::::::::::::::::::::::::::::::::::::::::::::::::|<<|
where
%|>>|:::::::::::::::::::::::::::::::::::::::::::::::::::::::::::::::::::::::::::::::::::::::::::|>>|
\(  | \JCoords | \leq \k  \),
\(  \JCoordsu \defeq \Supp( \Vec{\uV} )  \), and
\(  \JCoordsv \defeq \Supp( \Threshold{\k}( \Vec{\vV} ) )  \).
%|<<|:::::::::::::::::::::::::::::::::::::::::::::::::::::::::::::::::::::::::::::::::::::::::::|<<|
Then,
%|>>|===========================================================================================|>>|
\begin{gather}
  \left\|
    \Vec{\uV}
    -
    \frac
    {\Threshold{\k}(\Vec{\vV})}
    {\| \Threshold{\k}(\Vec{\vV}) \|_{2}}
  \right\|_{2}
  \leq
  3
  \left\|
    \Vec{\uV}
    -
    \frac
    {\ThresholdSet{\JCoords \cup \JCoordsu \cup \JCoordsv}(\Vec{\vV})}
    {\| \ThresholdSet{\JCoords \cup \JCoordsu \cup \JCoordsv}(\Vec{\vV}) \|_{2}}
  \right\|_{2}
.\end{gather}
%|<<|===========================================================================================|<<|
\end{lemma}
%|<<|*******************************************************************************************|<<|
%|<<|*******************************************************************************************|<<|
%|<<|*******************************************************************************************|<<|

The proof of the main theorem will additionally utilize the following fact from \cite{matsumoto2022binary}.
The iterative approximation errors will turn out to be upper bounded by the functions in this fact, and thus, this fact will facilitate the calculation of a close-form bound on the iterative approximation errors, much like the approach in \cite{matsumoto2022binary}.
%much like what is seen in \cite{matsumoto2022binary}.

%|>>|*******************************************************************************************|>>|
%|>>|*******************************************************************************************|>>|
%|>>|*******************************************************************************************|>>|
\begin{fact}[{\cite[{\FACT 4.1}]{matsumoto2022binary}}]
\label{fact:recurrence}
%
Let
%|>>|:::::::::::::::::::::::::::::::::::::::::::::::::::::::::::::::::::::::::::::::::::::::::::|>>|
\(  \ux, \vx, \wx \in \R_{+}  \), where
\(  \ux \defeq \frac{1}{2} ( 1 + \sqrt{1 + 4\wx} )  \) and
\(  1 \leq \ux \leq \sqrt{\frac{2}{\vx}}  \).
%|<<|:::::::::::::::::::::::::::::::::::::::::::::::::::::::::::::::::::::::::::::::::::::::::::|<<|
Let
%|>>|:::::::::::::::::::::::::::::::::::::::::::::::::::::::::::::::::::::::::::::::::::::::::::|>>|
\(  \fx, \fxx : \Z_{\geq 0} \to \R  \)
%|<<|:::::::::::::::::::::::::::::::::::::::::::::::::::::::::::::::::::::::::::::::::::::::::::|<<|
be functions given by
%|>>|===========================================================================================|>>|
\begin{gather*}
  \fx( 0 ) = 2
  ,\\
  \fx( \tx ) = \sqrt{\vx \fx( \tx-1 )} + \vx \wx
  ,\quad \tx \in \Z_{+}
  ,\\
  \fxx( \tx ) = 2^{2^{-\tx}} ( \ux^{2} \vx )^{1-2^{-\tx}}
  ,\quad \tx \in \Z_{\geq 0}
.\end{gather*}
%|<<|===========================================================================================|<<|
Then,
%|>>|===========================================================================================|>>|
\begin{gather*}
  \fx( \tx ) > \fx( \tx' )
  ,\quad \tx < \tx' \in \Z_{\geq 0}
  ,\\
  \fxx( \tx ) > \fxx( \tx' )
  ,\quad \tx < \tx' \in \Z_{\geq 0}
  ,\\
  \fx( \tx ) \leq \fxx( \tx )
  ,\quad \tx \in \Z_{\geq 0}
  ,\\
  \lim_{\tx \to \infty} \fx( \tx ) \leq \lim_{\tx \to \infty} \fxx( \tx ) = \ux^{2} \vx
.\end{gather*}
%|<<|===========================================================================================|<<|
\end{fact}
%|<<|*******************************************************************************************|<<|
%|<<|*******************************************************************************************|<<|
%|<<|*******************************************************************************************|<<|

%%%%%%%%%%%%%%%%%%%%%%%%%%%%%%%%%%%%%%%%%%%%%%%%%%%%%%%%%%%%%%%%%%%%%%%%%%%%%%%%%%%%%%%%%%%%%%%%%%%%
%%%%%%%%%%%%%%%%%%%%%%%%%%%%%%%%%%%%%%%%%%%%%%%%%%%%%%%%%%%%%%%%%%%%%%%%%%%%%%%%%%%%%%%%%%%%%%%%%%%%
%%%%%%%%%%%%%%%%%%%%%%%%%%%%%%%%%%%%%%%%%%%%%%%%%%%%%%%%%%%%%%%%%%%%%%%%%%%%%%%%%%%%%%%%%%%%%%%%%%%%

\subsection{Proof of \THEOREM \ref{thm:approx-error:sparse}}
\label{outline:pf-main-result|pf-main}

\begin{comment}
%|>>|~~~~~~~~~~~~~~~~~~~~~~~~~~~~~~~~~~~~~~~~~~~~~~~~~~~~~~~~~~~~~~~~~~~~~~~~~~~~~~~~~~~~~~~~~~~|>>|
%|>>|~~~~~~~~~~~~~~~~~~~~~~~~~~~~~~~~~~~~~~~~~~~~~~~~~~~~~~~~~~~~~~~~~~~~~~~~~~~~~~~~~~~~~~~~~~~|>>|
%|>>|~~~~~~~~~~~~~~~~~~~~~~~~~~~~~~~~~~~~~~~~~~~~~~~~~~~~~~~~~~~~~~~~~~~~~~~~~~~~~~~~~~~~~~~~~~~|>>|
\begin{proof}
{\THEOREM \ref{thm:approx-error:dense}}
%
Let
%|>>|:::::::::::::::::::::::::::::::::::::::::::::::::::::::::::::::::::::::::::::::::::::::::::|>>|
\(  \ParamSpace = \Sphere{\n}  \),
%|<<|:::::::::::::::::::::::::::::::::::::::::::::::::::::::::::::::::::::::::::::::::::::::::::|<<|
and fix
%|>>|:::::::::::::::::::::::::::::::::::::::::::::::::::::::::::::::::::::::::::::::::::::::::::|>>|
\(  \thetaStar \in \ParamSpace  \)
%|<<|:::::::::::::::::::::::::::::::::::::::::::::::::::::::::::::::::::::::::::::::::::::::::::|<<|
arbitrarily.
First, \EQUATION \eqref{eqn:approx-error:dense:iterative} in \THEOREM \ref{thm:approx-error:dense} will be verified, and subsequently, \EQUATION \eqref{eqn:approx-error:dense:asymptotic} in the theorem will quickly follow.
Before beginning the main argument, note that with the number of measurements, \(  \m  \), set to at least
%|>>|===========================================================================================|>>|
\begin{gather}
\label{eqn:pf:thm:approx-error:dense:m}
  \m
  \geq
  \mEXPR[d][,]
\end{gather}
%|<<|===========================================================================================|<<|
\THEOREM \ref{thm:main-technical:dense} ensures that with probability at least \(  1-\rhoX  \), for every \(  \thetaXX \in \ParamSpace  \), the following inequality holds:
%|>>|===========================================================================================|>>|
\begin{gather}
\label{eqn:pf:thm:approx-error:dense:ic}
  \left\|
    \thetaStar
    -
    \frac
    {\thetaXX + \hfFn( \thetaStar, \thetaXX )}
    {\| \thetaXX + \hfFn( \thetaStar, \thetaXX ) \|_{2}}
  \right\|_{2}
  \leq
  \sqrt{\deltaX \| \thetaStar - \thetaXX \|_{2}} + \deltaX
,\end{gather}
%|<<|===========================================================================================|<<|
where
%|>>|===========================================================================================|>>|
\begin{gather}
\label{eqn:pf:thm:approx-error:dense:delta}
  \deltaX \defeq \frac{\epsilonX}{\frac{1}{2} ( 3+\sqrt{5} )}
.\end{gather}
%|<<|===========================================================================================|<<|
Additionally, adopting notations consistent with \FACT \ref{fact:recurrence}, where in the fact, we take
%|>>|:::::::::::::::::::::::::::::::::::::::::::::::::::::::::::::::::::::::::::::::::::::::::::|>>|
\(  \ux = \frac{1}{2} ( 1+\sqrt{5} )  \),
\(  \vx = \deltaX  \), and
\(  \wx = 1  \),
%|<<|:::::::::::::::::::::::::::::::::::::::::::::::::::::::::::::::::::::::::::::::::::::::::::|<<|
let
%|>>|:::::::::::::::::::::::::::::::::::::::::::::::::::::::::::::::::::::::::::::::::::::::::::|>>|
\(  \fx, \fxx : \Z_{\geq 0} \to \R  \),
%|<<|:::::::::::::::::::::::::::::::::::::::::::::::::::::::::::::::::::::::::::::::::::::::::::|<<|
be functions given by
%|>>|===========================================================================================|>>|
\begin{gather}
  \label{eqn:pf:thm:approx-error:dense:f1(0)}
  \fx( 0 ) = 2
  ,\\ \label{eqn:pf:thm:approx-error:dense:f1(t)}
  \fx( \tx ) = \deltaX + \sqrt{\deltaX \fx( \tx-1 )}
  ,\quad \tx \in \Z_{+}
  ,\\ \label{eqn:pf:thm:approx-error:dense:f2(t)}
  \fxx( \tx )
  = 2^{2^{-\tx}} ( \ux^{2} \deltaX )^{1-2^{-\tx}}
  = 2^{2^{-\tx}} \left( \frac{1}{2} ( 3+\sqrt{5} ) \deltaX \right)^{1-2^{-\tx}}
  = 2^{2^{-\tx}} \epsilonX^{1-2^{-\tx}}
  ,\quad \tx \in \Z_{\geq 0}
.\end{gather}
%|<<|===========================================================================================|<<|
Note that
%|>>|:::::::::::::::::::::::::::::::::::::::::::::::::::::::::::::::::::::::::::::::::::::::::::|>>|
\(  \ux^{2} = \frac{1}{2} ( 3+\sqrt{5} )  \),
%|<<|:::::::::::::::::::::::::::::::::::::::::::::::::::::::::::::::::::::::::::::::::::::::::::|<<|
and thus, as defined in \EQUATION \eqref{eqn:pf:thm:approx-error:dense:delta}, the choice of \(  \deltaX  \) ensures that \(  \ux  \) and \(  \vx  \) satisfy
%|>>|===========================================================================================|>>|
\begin{gather*}
  \sqrt{\frac{2}{\vx}}
  = \sqrt{\frac{2}{\deltaX}}
  = \sqrt{\frac{2\ux^{2}}{\epsilonX}}
  \geq \sqrt{2}\ux
  > \ux
,\end{gather*}
%|<<|===========================================================================================|<<|
%%|>>|:::::::::::::::::::::::::::::::::::::::::::::::::::::::::::::::::::::::::::::::::::::::::::|>>|
%\(  \sqrt{\frac{2}{\vx}} = \sqrt{\frac{2}{\deltaX}} = \sqrt{\frac{2\ux^{2}}{\epsilonX}} \geq \sqrt{2}\ux > \ux  \),
%%|<<|:::::::::::::::::::::::::::::::::::::::::::::::::::::::::::::::::::::::::::::::::::::::::::|<<|
as required by \FACT \ref{fact:recurrence}.
Then, by this fact,
%|>>|===========================================================================================|>>|
\begin{gather}
\label{eqn:pf:thm:approx-error:dense:1}
  \fx( \tx )
  \leq \fxx( \tx )
  = 2^{2^{-\tx}} \epsilonX^{1-2^{-\tx}}
\end{gather}
%|<<|===========================================================================================|<<|
for all \(  \tx \in \Z_{\geq 0}  \), and
%|>>|===========================================================================================|>>|
\begin{gather}
\label{eqn:pf:thm:approx-error:dense:2}
  \lim_{\tx \to \infty} \fx( \tx )
  \leq
  \lim_{\tx \to \infty} \fxx( \tx )
  =
  \epsilonX
.\end{gather}
%|<<|===========================================================================================|<<|
%
%%%%%%%%%%%%%%%%%%%%%%%%%%%%%%%%%%%%%%%%%%%%%%%%%%%%%%%%%%%%%%%%%%%%%%%%%%%%%%%%%%%%%%%%%%%%%%%%%%%%
\par %%%%%%%%%%%%%%%%%%%%%%%%%%%%%%%%%%%%%%%%%%%%%%%%%%%%%%%%%%%%%%%%%%%%%%%%%%%%%%%%%%%%%%%%%%%%%%%
%%%%%%%%%%%%%%%%%%%%%%%%%%%%%%%%%%%%%%%%%%%%%%%%%%%%%%%%%%%%%%%%%%%%%%%%%%%%%%%%%%%%%%%%%%%%%%%%%%%%
%
Moving onto the proof of \EQUATION \eqref{eqn:approx-error:dense:iterative}, we will operate under the assumption that \EQUATION \eqref{eqn:pf:thm:approx-error:dense:ic} holds for all \(  \thetaXX \in \ParamSpace  \), which occurs with probability at least \(  1-\rhoX  \) due to the sufficiently choice of \(  \m  \).
\EQUATION \eqref{eqn:approx-error:dense:iterative} can be established inductively, inducting on the iterations, \(  \Iter = 0, 1, 2, 3, \dots  \), showing that at each \(  \Iter\Th  \) step, the \(  \Iter\Th  \) inductive claim, \(  \InductiveClaim{\Iter}  \), holds, where
%|>>|===========================================================================================|>>|
\begin{gather}
\label{eqn:pf:thm:approx-error:dense:inductive-claim}
  \InductiveClaim{\Iter}
  \defeq
  \text{``}
  \| \thetaStar-\thetaHat[\Iter] \|_{2}
  \leq
  \fx( \Iter )
%  2^{2^{-\Iter}}
%  \epsilonX^{1-2^{-\Iter}}
  .\text{''}
\end{gather}
%|<<|===========================================================================================|<<|
%
%%%%%%%%%%%%%%%%%%%%%%%%%%%%%%%%%%%%%%%%%%%%%%%%%%%%%%%%%%%%%%%%%%%%%%%%%%%%%%%%%%%%%%%%%%%%%%%%%%%%
\par %%%%%%%%%%%%%%%%%%%%%%%%%%%%%%%%%%%%%%%%%%%%%%%%%%%%%%%%%%%%%%%%%%%%%%%%%%%%%%%%%%%%%%%%%%%%%%%
%%%%%%%%%%%%%%%%%%%%%%%%%%%%%%%%%%%%%%%%%%%%%%%%%%%%%%%%%%%%%%%%%%%%%%%%%%%%%%%%%%%%%%%%%%%%%%%%%%%%
%
For the base case, when \(  \Iter=0  \), recall that the algorithm draws the \(  0\Th  \) approximation uniformly at random from the unit \(  \n  \)-sphere:
%|>>|:::::::::::::::::::::::::::::::::::::::::::::::::::::::::::::::::::::::::::::::::::::::::::|>>|
\(  \thetaHat[0] \sim \Sphere{\n}  \).
%|<<|:::::::::::::::::::::::::::::::::::::::::::::::::::::::::::::::::::::::::::::::::::::::::::|<<|
The farthest distance between any two points on a sphere is the diameter of the sphere, which in this case (for a unit sphere), is distance \(  2  \)---that is,
%|>>|===========================================================================================|>>|
\begin{gather*}
  \| \thetaStar - \thetaHat[0] \|_{2}
  \leq 2
  = \fx( 0 )
  %= 2^{2^{-\Iter}} \epsilonX^{1-2^{-\Iter}} |_{\Iter=0}
,\end{gather*}
%|<<|===========================================================================================|<<|
where the \RHS equality is due to \EQUATION \eqref{eqn:pf:thm:approx-error:dense:f1(0)}.
Hence, when \(  \Iter=0  \), the \(  0\Th  \) inductive claim, \(  \InductiveClaim{0}  \), holds.
Now, arbitrarily fixing some \(  \Iter \in \Z_{+}  \), suppose that the \(  \IterX\Th  \) claim, \(  \InductiveClaim{\IterX}  \), holds for all \(  \IterX < \Iter  \).
Then, the aim in the inductive step is to verify the \(  \Iter\Th  \) inductive claim, \(  \InductiveClaim{\Iter}  \).
Towards this, recall that
%|>>|===========================================================================================|>>|
\begin{gather}
\label{eqn:pf:thm:approx-error:dense:3}
  \thetaHatX[\Iter]
  =
  \thetaHat[\Iter-1]
  +
  \frac{\sqrt{2\pi}}{\m}
  \CovM^{\T}
  \frac{1}{2}
  \left( \fFn( \CovM \thetaStar ) - \Sign*( \CovM \thetaHat[\Iter-1] ) \right)
  =
  \thetaHat[\Iter-1] + \hFn( \thetaStar, \thetaHat[\Iter-1] )
  ,\\
\label{eqn:pf:thm:approx-error:dense:4}
  \thetaHat[\Iter]
  =
  \frac{\thetaHatX[\Iter]}{\| \thetaHatX[\Iter] \|_{2}}
,\end{gather}
%|<<|===========================================================================================|<<|
and therefore,
%|>>|===========================================================================================|>>|
\begin{align*}
  \| \thetaStar - \thetaHat[\Iter] \|_{2}
  &=
  \left\| \thetaStar - \frac{\thetaHatX[\Iter]}{\| \thetaHatX[\Iter] \|_{2}} \right\|_{2}
  \\
  &\dCmt{by \EQUATION \eqref{eqn:pf:thm:approx-error:dense:4}}
  \\
  &=
  \left\|
    \thetaStar
    -
    \frac
    {\thetaHat[\Iter-1] + \hFn( \thetaStar, \thetaHat[\Iter-1] )}
    {\| \thetaHat[\Iter-1] + \hFn( \thetaStar, \thetaHat[\Iter-1] ) \|_{2}}
  \right\|_{2}
  \\
  &\dCmt{by \EQUATION \eqref{eqn:pf:thm:approx-error:dense:3}}
  \\
  &\leq
  \deltaX + \sqrt{\deltaX \| \thetaStar - \thetaHat[\Iter-1] \|_{2}}
  \\
  &\dCmt{by \EQUATION \eqref{eqn:pf:thm:approx-error:dense:ic}}
  \\
  &\leq
  \deltaX + \sqrt{\deltaX \fx( \Iter-1 )}
  \\
  &\dCmt{by the inductive hypothesis, i.e., the assumption of correctness of \(  \InductiveClaim{\Iter-1}  \)}
  \\
  &=
  \fx( \Iter )
  .\\
  &\dCmt{by the definition of \(  \fx  \) in \EQUATION \eqref{eqn:pf:thm:approx-error:dense:f1(t)}}
\end{align*}
%|<<|===========================================================================================|<<|
Thus, under the inductive hypothesis, the \(  \Iter\Th  \) inductive claim, \(  \InductiveClaim{\Iter}  \), holds:
%|>>|===========================================================================================|>>|
\begin{gather*}
  \| \thetaStar - \thetaHat[\Iter] \|_{2}
  \leq
  \fx( \Iter )
,\end{gather*}
%|<<|===========================================================================================|<<|
as desired.
By induction, it follows that for all \(  \Iter \in \Z_{\geq 0}  \), the \(  \Iter\Th  \) inductive claim, \(  \InductiveClaim{\Iter}  \), is correct, i.e.,
%|>>|===========================================================================================|>>|
\begin{gather*}
  \| \thetaStar - \thetaHat[\Iter] \|_{2}
  \leq
  \fx( \Iter )
.\end{gather*}
%|<<|===========================================================================================|<<|
Then, as mentioned earlier in \EQUATIONS \eqref{eqn:pf:thm:approx-error:dense:1} and \eqref{eqn:pf:thm:approx-error:dense:2},
%|>>|===========================================================================================|>>|
\begin{gather*}
  \| \thetaStar - \thetaHat[\Iter] \|_{2}
  \leq
  \fx( \tx )
  \leq \fxx( \tx )
  = 2^{2^{-\tx}} \epsilonX^{1-2^{-\tx}}
\end{gather*}
%|<<|===========================================================================================|<<|
for all \(  \tx \in \Z_{\geq 0}  \), and
%|>>|===========================================================================================|>>|
\begin{gather*}
  \lim_{\Iter \to \infty} \| \thetaStar - \thetaHat[\Iter] \|_{2}
  \leq
  \lim_{\Iter \to \infty} \fx( \Iter )
  \leq
  \lim_{\Iter \to \infty} \fxx( \Iter )
  =
  \epsilonX
.\end{gather*}
%|<<|===========================================================================================|<<|
As noted earlier, this holds uniformly with probability at least \(  1-\rhoX  \).
This completes the proof of \THEOREM \ref{thm:approx-error:dense}.
\end{proof}
%|<<|~~~~~~~~~~~~~~~~~~~~~~~~~~~~~~~~~~~~~~~~~~~~~~~~~~~~~~~~~~~~~~~~~~~~~~~~~~~~~~~~~~~~~~~~~~~|<<|
%|<<|~~~~~~~~~~~~~~~~~~~~~~~~~~~~~~~~~~~~~~~~~~~~~~~~~~~~~~~~~~~~~~~~~~~~~~~~~~~~~~~~~~~~~~~~~~~|<<|
%|<<|~~~~~~~~~~~~~~~~~~~~~~~~~~~~~~~~~~~~~~~~~~~~~~~~~~~~~~~~~~~~~~~~~~~~~~~~~~~~~~~~~~~~~~~~~~~|<<|
\end{comment}

With the above results in \SECTION \ref{outline:pf-main-result|intermediate}, the convergence of the BIHT approximations, as stated in the main theorem, can now be proved.

%|>>|~~~~~~~~~~~~~~~~~~~~~~~~~~~~~~~~~~~~~~~~~~~~~~~~~~~~~~~~~~~~~~~~~~~~~~~~~~~~~~~~~~~~~~~~~~~|>>|
%|>>|~~~~~~~~~~~~~~~~~~~~~~~~~~~~~~~~~~~~~~~~~~~~~~~~~~~~~~~~~~~~~~~~~~~~~~~~~~~~~~~~~~~~~~~~~~~|>>|
%|>>|~~~~~~~~~~~~~~~~~~~~~~~~~~~~~~~~~~~~~~~~~~~~~~~~~~~~~~~~~~~~~~~~~~~~~~~~~~~~~~~~~~~~~~~~~~~|>>|
\begin{proof}
{\THEOREM \ref{thm:approx-error:sparse}}
%
\checkoff%
%
Setting
%|>>|===========================================================================================|>>|
\begin{gather}
  \deltaX
  =
  \frac{\epsilonX}{\frac{3}{2} ( 5+\sqrt{21} )}
  =
  \frac{\epsilonX}{9 \left( \frac{1}{2} ( 1+\sqrt{\frac{7}{3}} ) \right)^{2}}
,\end{gather}
%|<<|===========================================================================================|<<|
and taking
%|>>|===========================================================================================|>>|
\begin{gather}
  \m \geq \mEXPR[s][,]
\end{gather}
%|<<|===========================================================================================|<<|
the following bound holds for all \(  \thetaXX \in \ParamSpace  \) and all \(  \JCoords \subseteq[\n]  \), \(  | \JCoords | \leq \k  \), uniformly with probability at least \(  1-\rhoX  \) due to \THEOREM \ref{thm:main-technical:sparse}:
%|>>|===========================================================================================|>>|
\begin{gather}
\label{eqn:pf:thm:approx-error:sparse:ic}
  \left\|
    \thetaStar
    -
    \frac
    {\thetaXX + \hfFn[\JCoords]( \thetaStar, \thetaXX )}
    {\| \thetaXX + \hfFn[\JCoords]( \thetaStar, \thetaXX ) \|_{2}}
  \right\|_{2}
  \leq
  \sqrt{\deltaX \| \thetaStar - \thetaXX \|_{2}} + \deltaX
.\end{gather}
%|<<|===========================================================================================|<<|
The remainder of the proof will assume that the inequality in \EQUATION \eqref{eqn:pf:thm:approx-error:sparse:ic} holds uniformly, which occurs with bounded probability, as just stated.
%and hence is true with probability at least \(  1-\rhoX  \).
Additionally, using the notations in \FACT \ref{fact:recurrence}---wherein the variables are set as
%|>>|:::::::::::::::::::::::::::::::::::::::::::::::::::::::::::::::::::::::::::::::::::::::::::|>>|
\(  \ux \defeq \frac{1}{2} ( 1+\sqrt{\frac{7}{3}} )  \),
\(  \vx \defeq 9 \deltaX  \), and
\(  \wx \defeq \frac{1}{3}  \)
%|<<|:::::::::::::::::::::::::::::::::::::::::::::::::::::::::::::::::::::::::::::::::::::::::::|<<|
and satisfy the fact's requirement, %on the relationship between \(  \ux  \) and \(  \vx  \),
%|>>|===========================================================================================|>>|
\(
  \sqrt{\frac{2}{\vx}}
  =
  \sqrt{\frac{2}{9 \deltaX}}
  =
  \sqrt{\frac{2 \cdot 9 \ux^{2}}{9 \epsilonX}}
  =
  \ux \sqrt{\frac{2}{\epsilonX}}
  >
  \ux 
\)
%|<<|===========================================================================================|<<|
---define the functions
%|>>|:::::::::::::::::::::::::::::::::::::::::::::::::::::::::::::::::::::::::::::::::::::::::::|>>|
\(  \fx, \fxx : \Z_{\geq 0} \to \R  \)
%|<<|:::::::::::::::::::::::::::::::::::::::::::::::::::::::::::::::::::::::::::::::::::::::::::|<<|
by
%|>>|===========================================================================================|>>|
\begin{gather}
  \label{eqn:pf:thm:approx-error:sparse:f1(0)}
  \fx( 0 ) = 2
  ,\\ \label{eqn:pf:thm:approx-error:sparse:f1(t)}
  \fx( \tx )
  = \sqrt{\vx \fx( \tx-1 )} + \wx \vx
  = \sqrt{9 \deltaX \fx( \tx-1 )} + 3 \deltaX
  ,\quad \tx \in \Z_{+}
  ,\\ \label{eqn:pf:thm:approx-error:sparse:f2(t)}
  \fxx( \tx )
  = 2^{2^{-\tx}} ( \ux^{2} \vx )^{1-2^{-\tx}}
  = 2^{2^{-\tx}} \left( \frac{3}{2} ( 5+\sqrt{21} ) \delta \right)^{1-2^{-\tx}}
  = 2^{2^{-\tx}} \epsilonX^{1-2^{-\tx}}
  ,\quad \tx \in \Z_{\geq 0}
.\end{gather}
%|<<|===========================================================================================|<<|
Then, by \FACT \ref{fact:recurrence}, for all \(  \tx \in \Z_{\geq 0}  \),
%|>>|===========================================================================================|>>|
\begin{gather}
\label{eqn:pf:thm:approx-error:sparse:1}
  \fx( \tx )
  \leq \fxx( \tx )
  = 2^{2^{-\tx}} \epsilonX^{1-2^{-\tx}}
,\end{gather}
%|<<|===========================================================================================|<<|
and asymptotically,
%|>>|===========================================================================================|>>|
\begin{gather}
\label{eqn:pf:thm:approx-error:sparse:2}
  \lim_{\tx \to \infty} \fx( \tx )
  \leq
  \lim_{\tx \to \infty} \fxx( \tx )
  =
  \epsilonX
.\end{gather}
%|<<|===========================================================================================|<<|
%
%%%%%%%%%%%%%%%%%%%%%%%%%%%%%%%%%%%%%%%%%%%%%%%%%%%%%%%%%%%%%%%%%%%%%%%%%%%%%%%%%%%%%%%%%%%%%%%%%%%%
\par %%%%%%%%%%%%%%%%%%%%%%%%%%%%%%%%%%%%%%%%%%%%%%%%%%%%%%%%%%%%%%%%%%%%%%%%%%%%%%%%%%%%%%%%%%%%%%%
%%%%%%%%%%%%%%%%%%%%%%%%%%%%%%%%%%%%%%%%%%%%%%%%%%%%%%%%%%%%%%%%%%%%%%%%%%%%%%%%%%%%%%%%%%%%%%%%%%%%
%
With these preliminaries laid out, we are ready to verify \EQUATIONS \eqref{eqn:approx-error:sparse:asymptotic} and \eqref{eqn:approx-error:sparse:iterative} in \THEOREM \ref{thm:approx-error:sparse}, which can be argued inductively.
Inducting on the iterations, \(  \Iter = 0,1,2,3,\dots  \), the following inductive claim will be shown:
%|>>|===========================================================================================|>>|
\begin{gather}
\label{eqn:pf:thm:approx-error:sparse:inductive-claim}
  \InductiveClaim{\Iter}
  \defeq
  \text{``}
  \| \thetaStar-\thetaHat[\Iter] \|_{2}
  \leq
  \fx( \Iter )
%  2^{2^{-\Iter}}
%  \epsilonX^{1-2^{-\Iter}}
  .\text{''}
\end{gather}
%|<<|===========================================================================================|<<|
The base case, when \(  \Iter=0  \), is trivial:
since there is the membership of
%|>>|:::::::::::::::::::::::::::::::::::::::::::::::::::::::::::::::::::::::::::::::::::::::::::|>>|
\(  \thetaStar, \thetaHat[0] \in \SparseSphereSubspace{\k}{\n} \subseteq \Sphere{\n}  \),
%|<<|:::::::::::::::::::::::::::::::::::::::::::::::::::::::::::::::::::::::::::::::::::::::::::|<<|
the Euclidean distance between \(  \thetaStar  \) and \(  \thetaHat[0]  \) cannot exceed the diameter of the unit sphere (distance \(  2  \)), i.e.,
%|>>|===========================================================================================|>>|
\begin{gather*}
  \| \thetaStar - \thetaHat[0] \|_{2} \leq 2 = \fx( 0 )
,\end{gather*}
%|<<|===========================================================================================|<<|
where the rightmost equality is due to the definition of \(  \fx  \) in \EQUATION \eqref{eqn:pf:thm:approx-error:sparse:f1(0)}.
Next, consider some arbitrary choice of \(  \Iter \in \Z_{+}  \), and suppose that for every \(  \IterX < \Iter  \), the \(  \IterX\Th  \) inductive claim, \(  \InductiveClaim{\IterX}  \), holds.
Then, under this inductive assumption, the \(  \Iter\Th  \) inductive claim, \(  \InductiveClaim{\Iter}  \), needs to be verified.
Recall that for \(  \Iter > 0  \), \ALGORITHM \ref{alg:biht:normalized} sets
%|>>|===========================================================================================|>>|
\begin{gather}
\label{eqn:pf:thm:approx-error:sparse:3}
  \thetaHatX[\Iter]
  =
  \thetaHat[\Iter-1]
  +
  \frac{\sqrt{2\pi}}{\m}
  \sep
  \CovM^{\T}
  \sep
  \frac{1}{2}
  \left( \fFn( \CovM \thetaStar ) - \Sign*( \CovM \thetaHat[\Iter-1] ) \right)
  =
  \thetaHat[\Iter-1] + \hfFn( \thetaStar, \thetaHat[\Iter-1] )
  ,\\
\label{eqn:pf:thm:approx-error:sparse:4}
  \thetaHat[\Iter]
  =
  \frac{\Threshold*{\k}( \thetaHatX[\Iter] )}{\| \Threshold*{\k}( \thetaHatX[\Iter] ) \|_{2}}
.\end{gather}
%|<<|===========================================================================================|<<|
%%%%%%%%%%%%%%%%%%%%%%%%%%%%%%%%%%%%%%%%%%%%%%%%%%%%%%%%%%%%%%%%%%%%%%%%%%%%%%%%%%%%%%%%%%%%%%%%%%%%
\renewcommand{\TS}{\Supp( \thetaStar ) \cup \Supp( \thetaHat[\Iter-1] ) \cup \Supp( \thetaHat[\Iter] )}%
%%%%%%%%%%%%%%%%%%%%%%%%%%%%%%%%%%%%%%%%%%%%%%%%%%%%%%%%%%%%%%%%%%%%%%%%%%%%%%%%%%%%%%%%%%%%%%%%%%%%
Additionally, due to \LEMMA \ref{lemma:error:deterministic}---where, in the context of this proof, the sets \(  \JCoords, \JCoordsX, \JCoordsXX \subseteq [\n]  \) in the lemma are taken to be
%|>>|:::::::::::::::::::::::::::::::::::::::::::::::::::::::::::::::::::::::::::::::::::::::::::|>>|
\(  \JCoords \defeq \Supp( \thetaHat[\Iter-1] )  \),
\(  \JCoordsX \defeq \Supp( \thetaStar )  \), and
\(  \JCoordsXX \defeq \Supp( \thetaHat[\Iter] )  \)%
%|<<|:::::::::::::::::::::::::::::::::::::::::::::::::::::::::::::::::::::::::::::::::::::::::::|<<|
---the following holds:
%|>>|===========================================================================================|>>|
\begin{gather}
\label{eqn:pf:thm:approx-error:sparse:5}
  \| \thetaStar - \thetaHat[\Iter] \|_{2}
  =
  \left\|
    \thetaStar
    -
    \frac
    {\Threshold*{\k}( \thetaHatX[\Iter] )}
    {\| \Threshold*{\k}( \thetaHatX[\Iter] ) \|_{2}}
  \right\|_{2}
  \leq
  3
  \left\|
    \thetaStar
    -
    \frac
    {\ThresholdSet*{\TS}( \thetaHatX[\Iter] )}
    {\| \ThresholdSet*{\TS}( \thetaHatX[\Iter] ) \|_{2}}
  \right\|_{2}
.\end{gather}
%|<<|===========================================================================================|<<|
Then, the \(  \Iter\Th  \) inductive claim, \(  \InductiveClaim{\Iter}  \), can now be established:
%|>>|===========================================================================================|>>|
\begin{align*}
  \| \thetaStar - \thetaHat[\Iter] \|_{2}
  &\leq
  3
  \left\|
    \thetaStar
    -
    \frac
    {\ThresholdSet*{\TS}( \thetaHatX[\Iter] )}
    {\| \ThresholdSet*{\TS}( \thetaHatX[\Iter] ) \|_{2}}
  \right\|_{2}
  \\
  &\dCmt{by \EQUATION \eqref{eqn:pf:thm:approx-error:sparse:5}}
  \\
  &=
  3
  \left\|
    \thetaStar
    -
    \frac
    {\ThresholdSet*{\TS}( \thetaHat[\Iter-1] + \hfFn( \thetaStar, \thetaHat[\Iter-1] ) )}
    {\| \ThresholdSet*{\TS}( \thetaHat[\Iter-1] + \hfFn( \thetaStar, \thetaHat[\Iter-1] ) ) \|_{2}}
  \right\|_{2}
  \\
  &\dCmt{by \EQUATION \eqref{eqn:pf:thm:approx-error:sparse:3}}
  \\
  &=
  3
  \left\|
    \thetaStar
    -
    \frac
    {\thetaHat[\Iter-1] + \hfFn[{\Supp( \thetaHat[\Iter] )}]( \thetaStar, \thetaHat[\Iter-1] )}
    {\| \thetaHat[\Iter-1] + \hfFn[{\Supp( \thetaHat[\Iter] )}]( \thetaStar, \thetaHat[\Iter-1] ) \|_{2}}
  \right\|_{2}
  \\
  &\dCmt{by the definitions of the subset thresholding operation and \(  \hfFn[\JCoords]  \) (\(  \JCoords \subseteq [\n]  \))}
  % &\dCmt{because \(  \Supp( \thetaHat[\Iter-1] ), \Supp( \hfFn[{\Supp( \thetaHat[\Iter] )}]( \thetaStar, \thetaHat[\Iter-1] ) ) \subseteq \TS  \)}
  \\
  &\leq
  3
  \left( \sqrt{\deltaX \| \thetaStar - \thetaHat[\Iter-1] \|_{2}} + \deltaX \right)
  \\
  &\dCmt{by \EQUATION \eqref{eqn:pf:thm:approx-error:sparse:ic}}
  \\
  &=
  \sqrt{9 \deltaX \| \thetaStar - \thetaHat[\Iter-1] \|_{2}} + 3 \deltaX
  \\
  &\leq
  \sqrt{9 \deltaX \fx( \Iter-1 )} + 3 \deltaX
  \\
  &\dCmt{by the inductive hypothesis, i.e., the assumed correctness of \(  \InductiveClaim{\Iter-1}  \)}
  \\
  &=
  \fx( \Iter )
  ,\\
  &\dCmt{by the definition of \(  \fx  \) in \EQUATION \eqref{eqn:pf:thm:approx-error:sparse:f1(t)}}
\end{align*}
%|<<|===========================================================================================|<<|
as desired.
%
%%%%%%%%%%%%%%%%%%%%%%%%%%%%%%%%%%%%%%%%%%%%%%%%%%%%%%%%%%%%%%%%%%%%%%%%%%%%%%%%%%%%%%%%%%%%%%%%%%%%
\par %%%%%%%%%%%%%%%%%%%%%%%%%%%%%%%%%%%%%%%%%%%%%%%%%%%%%%%%%%%%%%%%%%%%%%%%%%%%%%%%%%%%%%%%%%%%%%%
%%%%%%%%%%%%%%%%%%%%%%%%%%%%%%%%%%%%%%%%%%%%%%%%%%%%%%%%%%%%%%%%%%%%%%%%%%%%%%%%%%%%%%%%%%%%%%%%%%%%
%
Having verified the \(  \Iter\Th  \) inductive claim, \(  \InductiveClaim{\Iter}  \), under the inductive assumption, it follows by induction that for all \(  \Iter \in \Z_{\geq 0}  \), the \(  \Iter\Th  \) inductive claim, \(  \InductiveClaim{\Iter}  \), holds:
%|>>|===========================================================================================|>>|
\begin{gather*}
  \| \thetaStar - \thetaHat[\Iter] \|_{2}
  \leq
  \fx( \Iter )
.\end{gather*}
%|<<|===========================================================================================|<<|
Therefore, the assumption that \EQUATION \eqref{eqn:pf:thm:approx-error:sparse:ic} holds uniformly---which occurs with probability at least \(  1-\rhoX  \)---and \EQUATIONS \eqref{eqn:pf:thm:approx-error:sparse:1} and \eqref{eqn:pf:thm:approx-error:sparse:2} together imply that
%|>>|===========================================================================================|>>|
\begin{gather*}
  \| \thetaStar - \thetaHat[\Iter] \|_{2}
  \leq
  \fx( \Iter )
  \leq \fxx( \Iter )
  = 2^{2^{-\Iter}} \epsilonX^{1-2^{-\Iter}}
\end{gather*}
%|<<|===========================================================================================|<<|
for every \(  \Iter \in \Z_{\geq 0}  \) and that
%|>>|===========================================================================================|>>|
\begin{gather*}
  \lim_{\Iter \to \infty} \| \thetaStar - \thetaHat[\Iter] \|_{2}
  \leq
  \lim_{\Iter \to \infty} \fx( \Iter )
  \leq
  \lim_{\Iter \to \infty} \fxx( \Iter )
  =
  \epsilonX
,\end{gather*}
%|<<|===========================================================================================|<<|
concluding the theorem's proof.
\end{proof}
%|<<|~~~~~~~~~~~~~~~~~~~~~~~~~~~~~~~~~~~~~~~~~~~~~~~~~~~~~~~~~~~~~~~~~~~~~~~~~~~~~~~~~~~~~~~~~~~|<<|
%|<<|~~~~~~~~~~~~~~~~~~~~~~~~~~~~~~~~~~~~~~~~~~~~~~~~~~~~~~~~~~~~~~~~~~~~~~~~~~~~~~~~~~~~~~~~~~~|<<|
%|<<|~~~~~~~~~~~~~~~~~~~~~~~~~~~~~~~~~~~~~~~~~~~~~~~~~~~~~~~~~~~~~~~~~~~~~~~~~~~~~~~~~~~~~~~~~~~|<<|

%%%%%%%%%%%%%%%%%%%%%%%%%%%%%%%%%%%%%%%%%%%%%%%%%%%%%%%%%%%%%%%%%%%%%%%%%%%%%%%%%%%%%%%%%%%%%%%%%%%%
%%%%%%%%%%%%%%%%%%%%%%%%%%%%%%%%%%%%%%%%%%%%%%%%%%%%%%%%%%%%%%%%%%%%%%%%%%%%%%%%%%%%%%%%%%%%%%%%%%%%
%%%%%%%%%%%%%%%%%%%%%%%%%%%%%%%%%%%%%%%%%%%%%%%%%%%%%%%%%%%%%%%%%%%%%%%%%%%%%%%%%%%%%%%%%%%%%%%%%%%%

\subsection{Proof of \COROLLARY \ref{corollary:approx-error:logistic-regression}} % and \ref{corollary:approx-error:probit}}
\label{outline:pf-main-result|pf-main-corollaries}

%|>>|~~~~~~~~~~~~~~~~~~~~~~~~~~~~~~~~~~~~~~~~~~~~~~~~~~~~~~~~~~~~~~~~~~~~~~~~~~~~~~~~~~~~~~~~~~~|>>|
%|>>|~~~~~~~~~~~~~~~~~~~~~~~~~~~~~~~~~~~~~~~~~~~~~~~~~~~~~~~~~~~~~~~~~~~~~~~~~~~~~~~~~~~~~~~~~~~|>>|
%|>>|~~~~~~~~~~~~~~~~~~~~~~~~~~~~~~~~~~~~~~~~~~~~~~~~~~~~~~~~~~~~~~~~~~~~~~~~~~~~~~~~~~~~~~~~~~~|>>|
\begin{proof}
{\COROLLARY \ref{corollary:approx-error:logistic-regression}} % and \ref{corollary:approx-error:probit}}
%
%\checkoff%
%
%\ToDo{I (probably) need to revise this.}
%Having laid out the arguments for \THEOREMS \ref{thm:approx-error:dense} and \ref{thm:approx-error:sparse} in \SECTION \ref{outline:pf-main-result|pf-main}, and with
Under the presumed correctness of the main technical corollary, \COROLLARY \ref{corollary:main-technical:logistic-regression}, %and \ref{corollary:main-technical:probit}, %the proofs of the main corollaries, 
\COROLLARY \ref{corollary:approx-error:logistic-regression} % and \ref{corollary:approx-error:probit},
for the convergence of BIHT (\ALGORITHM \ref{alg:biht:normalized}) in logistic and probit regressions, %respectively, 
now follow along the same arguments as in the proof of \THEOREM \ref{thm:approx-error:sparse}. %with, therein, the use of \THEOREMS \ref{thm:main-technical:dense} and \ref{thm:main-technical:sparse} replaced by \COROLLARY \ref{corollary:main-technical:logistic-regression} in order to establish \COROLLARY \ref{corollary:approx-error:logistic-regression}, and replaced by \COROLLARY \ref{corollary:main-technical:probit} in order to establish \COROLLARY \ref{corollary:approx-error:probit}.
In this analogous proof, the use of \COROLLARY \ref{corollary:main-technical:logistic-regression} replaces \THEOREM \ref{thm:main-technical:sparse} in order to establish the logistic and probit cases of \COROLLARY \ref{corollary:approx-error:logistic-regression}. %, and likewise with \COROLLARY \ref{corollary:main-technical:probit}
%towards
%to prove the probit case.
%\COROLLARY \ref{corollary:approx-error:probit}.
%and likewise with \COROLLARY \ref{corollary:main-technical:probit} in order to establish \COROLLARY \ref{corollary:approx-error:probit}.
\end{proof}
%|<<|~~~~~~~~~~~~~~~~~~~~~~~~~~~~~~~~~~~~~~~~~~~~~~~~~~~~~~~~~~~~~~~~~~~~~~~~~~~~~~~~~~~~~~~~~~~|<<|
%|<<|~~~~~~~~~~~~~~~~~~~~~~~~~~~~~~~~~~~~~~~~~~~~~~~~~~~~~~~~~~~~~~~~~~~~~~~~~~~~~~~~~~~~~~~~~~~|<<|
%|<<|~~~~~~~~~~~~~~~~~~~~~~~~~~~~~~~~~~~~~~~~~~~~~~~~~~~~~~~~~~~~~~~~~~~~~~~~~~~~~~~~~~~~~~~~~~~|<<|

%%%%%%%%%%%%%%%%%%%%%%%%%%%%%%%%%%%%%%%%%%%%%%%%%%%%%%%%%%%%%%%%%%%%%%%%%%%%%%%%%%%%%%%%%%%%%%%%%%%%
%%%%%%%%%%%%%%%%%%%%%%%%%%%%%%%%%%%%%%%%%%%%%%%%%%%%%%%%%%%%%%%%%%%%%%%%%%%%%%%%%%%%%%%%%%%%%%%%%%%%
%%%%%%%%%%%%%%%%%%%%%%%%%%%%%%%%%%%%%%%%%%%%%%%%%%%%%%%%%%%%%%%%%%%%%%%%%%%%%%%%%%%%%%%%%%%%%%%%%%%%

\subsection{Proof of the Intermediate Result, \LEMMA \ref{lemma:error:deterministic}}
\label{outline:pf-main-result|pf-intermediate}

This section verifies the intermediate result, \LEMMA \ref{lemma:error:deterministic}, which was introduced in \SECTION \ref{outline:pf-main-result|intermediate}.
The proof of \LEMMA \ref{lemma:error:deterministic} will use the following fact.

%|>>|*******************************************************************************************|>>|
%|>>|*******************************************************************************************|>>|
%|>>|*******************************************************************************************|>>|
\begin{fact}
\label{fact:dist-btw-normalized-vectors}
%
Let
%|>>|:::::::::::::::::::::::::::::::::::::::::::::::::::::::::::::::::::::::::::::::::::::::::::|>>|
\(  \Vec{\uV}, \Vec{\vV} \in \R^{\n}  \).
%|<<|:::::::::::::::::::::::::::::::::::::::::::::::::::::::::::::::::::::::::::::::::::::::::::|<<|
Then,
%|>>|===========================================================================================|>>|
\begin{gather}
  \left\| \frac{\Vec{\uV}}{\| \Vec{\uV} \|_{2}} - \frac{\Vec{\vV}}{\| \Vec{\vV} \|_{2}} \right\|_{2}
  \leq
  2
  \min \left\{
    \frac{\| \Vec{\uV} - \Vec{\vV} \|_{2}}{\| \Vec{\uV} \|_{2}},
    \frac{\| \Vec{\uV} - \Vec{\vV} \|_{2}}{\| \Vec{\vV} \|_{2}}
  \right\}
.\end{gather}
%|<<|===========================================================================================|<<|
\end{fact}
%|<<|*******************************************************************************************|<<|
%|<<|*******************************************************************************************|<<|
%|<<|*******************************************************************************************|<<|

%|>>|~~~~~~~~~~~~~~~~~~~~~~~~~~~~~~~~~~~~~~~~~~~~~~~~~~~~~~~~~~~~~~~~~~~~~~~~~~~~~~~~~~~~~~~~~~~|>>|
%|>>|~~~~~~~~~~~~~~~~~~~~~~~~~~~~~~~~~~~~~~~~~~~~~~~~~~~~~~~~~~~~~~~~~~~~~~~~~~~~~~~~~~~~~~~~~~~|>>|
%|>>|~~~~~~~~~~~~~~~~~~~~~~~~~~~~~~~~~~~~~~~~~~~~~~~~~~~~~~~~~~~~~~~~~~~~~~~~~~~~~~~~~~~~~~~~~~~|>>|
\begin{proof}
{\FACT \ref{fact:dist-btw-normalized-vectors}}
%
\checkoff%
%
Before the fact is verified, the following easily verifiable claim is derived. %introduced and proved.
%
%|>>|*******************************************************************************************|>>|
%|>>|*******************************************************************************************|>>|
%|>>|*******************************************************************************************|>>|
\begin{claim}
\label{claim:pf:fact:dist-btw-normalized-vectors:1}
%
Let
%|>>|:::::::::::::::::::::::::::::::::::::::::::::::::::::::::::::::::::::::::::::::::::::::::::|>>|
\(  \zx \geq 0  \).
%|<<|:::::::::::::::::::::::::::::::::::::::::::::::::::::::::::::::::::::::::::::::::::::::::::|<<|
Then,
%|>>|:::::::::::::::::::::::::::::::::::::::::::::::::::::::::::::::::::::::::::::::::::::::::::|>>|
\(  | 1 - | 1 - \zx | | \leq \zx  \).
%|<<|:::::::::::::::::::::::::::::::::::::::::::::::::::::::::::::::::::::::::::::::::::::::::::|<<|
\end{claim}
%|<<|*******************************************************************************************|<<|
%|<<|*******************************************************************************************|<<|
%|<<|*******************************************************************************************|<<|
%
%|>>|~~~~~~~~~~~~~~~~~~~~~~~~~~~~~~~~~~~~~~~~~~~~~~~~~~~~~~~~~~~~~~~~~~~~~~~~~~~~~~~~~~~~~~~~~~~|>>|
%|>>|~~~~~~~~~~~~~~~~~~~~~~~~~~~~~~~~~~~~~~~~~~~~~~~~~~~~~~~~~~~~~~~~~~~~~~~~~~~~~~~~~~~~~~~~~~~|>>|
%|>>|~~~~~~~~~~~~~~~~~~~~~~~~~~~~~~~~~~~~~~~~~~~~~~~~~~~~~~~~~~~~~~~~~~~~~~~~~~~~~~~~~~~~~~~~~~~|>>|
% \begin{proof} %✓
% {\CLAIM \ref{claim:pf:fact:dist-btw-normalized-vectors:1}}
% %
% \checkoff%
% %
% The claim can be verified by separating the analysis into three cases:
% \Enum[\label{enum:pf:claim:pf:fact:dist-btw-normalized-vectors:1:i}]{i} when \(  \zx \in [0,1]  \),
% \Enum[\label{enum:pf:claim:pf:fact:dist-btw-normalized-vectors:1:ii}]{ii} when \(  \zx \in (1,2]  \), and
% \Enum[\label{enum:pf:claim:pf:fact:dist-btw-normalized-vectors:1:iii}]{iii} when \(  \zx > 2  \).
% In \CASE \ref{enum:pf:claim:pf:fact:dist-btw-normalized-vectors:1:i},
% %|>>|:::::::::::::::::::::::::::::::::::::::::::::::::::::::::::::::::::::::::::::::::::::::::::|>>|
% \(  | 1 - | 1 - \zx | | = | 1 - ( 1 - \zx ) | = | \zx | = \zx  \).
% %|<<|:::::::::::::::::::::::::::::::::::::::::::::::::::::::::::::::::::::::::::::::::::::::::::|<<|
% In \CASES \ref{enum:pf:claim:pf:fact:dist-btw-normalized-vectors:1:ii} and \ref{enum:pf:claim:pf:fact:dist-btw-normalized-vectors:1:iii},
% %|>>|:::::::::::::::::::::::::::::::::::::::::::::::::::::::::::::::::::::::::::::::::::::::::::|>>|
% \(  | 1 - | 1 - \zx | | = | 1 - ( \zx - 1 ) | = | 2 - \zx |  \).
% %|<<|:::::::::::::::::::::::::::::::::::::::::::::::::::::::::::::::::::::::::::::::::::::::::::|<<|
% Then, in \CASE \ref{enum:pf:claim:pf:fact:dist-btw-normalized-vectors:1:ii},
% %|>>|:::::::::::::::::::::::::::::::::::::::::::::::::::::::::::::::::::::::::::::::::::::::::::|>>|
% \(  | 1 - | 1 - \zx | | = | 2 - \zx | = 2 - \zx < \zx  \),
% %|<<|:::::::::::::::::::::::::::::::::::::::::::::::::::::::::::::::::::::::::::::::::::::::::::|<<|
% where the inequality holds since \(  \zx > 1  \).
% Lastly, in \CASE \ref{enum:pf:claim:pf:fact:dist-btw-normalized-vectors:1:iii},
% %|>>|:::::::::::::::::::::::::::::::::::::::::::::::::::::::::::::::::::::::::::::::::::::::::::|>>|
% \(  | 1 - | 1 - \zx | | = | 2 - \zx | = \zx - 2 < \zx  \).
% %|<<|:::::::::::::::::::::::::::::::::::::::::::::::::::::::::::::::::::::::::::::::::::::::::::|<<|
% Therefore, for all \(  \zx \geq 0  \),
% %|>>|:::::::::::::::::::::::::::::::::::::::::::::::::::::::::::::::::::::::::::::::::::::::::::|>>|
% \(  | 1 - | 1 - \zx | | \leq \zx  \),
% %|<<|:::::::::::::::::::::::::::::::::::::::::::::::::::::::::::::::::::::::::::::::::::::::::::|<<|
% as claimed.
% \end{proof}
%|<<|~~~~~~~~~~~~~~~~~~~~~~~~~~~~~~~~~~~~~~~~~~~~~~~~~~~~~~~~~~~~~~~~~~~~~~~~~~~~~~~~~~~~~~~~~~~|<<|
%|<<|~~~~~~~~~~~~~~~~~~~~~~~~~~~~~~~~~~~~~~~~~~~~~~~~~~~~~~~~~~~~~~~~~~~~~~~~~~~~~~~~~~~~~~~~~~~|<<|
%|<<|~~~~~~~~~~~~~~~~~~~~~~~~~~~~~~~~~~~~~~~~~~~~~~~~~~~~~~~~~~~~~~~~~~~~~~~~~~~~~~~~~~~~~~~~~~~|<<|
%
Now, returning to the proof of \FACT \ref{fact:dist-btw-normalized-vectors}, fix
%|>>|:::::::::::::::::::::::::::::::::::::::::::::::::::::::::::::::::::::::::::::::::::::::::::|>>|
\(  \Vec{\uV}, \Vec{\vV} \in \R^{\n}  \)
%|<<|:::::::::::::::::::::::::::::::::::::::::::::::::::::::::::::::::::::::::::::::::::::::::::|<<|
arbitrarily.
Observe:
%|>>|===========================================================================================|>>|
\begin{align*}
  \left\| \frac{\Vec{\uV}}{\| \Vec{\uV} \|_{2}} - \frac{\Vec{\vV}}{\| \Vec{\vV} \|_{2}} \right\|_{2}
  &=
  \left\|
    \left(
      \frac{\Vec{\uV}}{\| \Vec{\uV} \|_{2}}
      -
      \frac{\Vec{\vV}}{\| \Vec{\uV} \|_{2}}
    \right)
    +
    \left(
      \frac{\Vec{\vV}}{\| \Vec{\uV} \|_{2}}
      -
      \frac{\Vec{\vV}}{\| \Vec{\vV} \|_{2}}
    \right)
  \right\|_{2}
  %\\
  % &\dCmt{by inserting in \(  \pm \frac{\Vec{\vV}}{\| \Vec{\uV} \|_{2}}  \) terms that cancel each other}
  \\
  &\leq
  \left\|
    \frac{\Vec{\uV}}{\| \Vec{\uV} \|_{2}}
    -
    \frac{\Vec{\vV}}{\| \Vec{\uV} \|_{2}}
  \right\|_{2}
  +
  \left\|
    \frac{\Vec{\vV}}{\| \Vec{\uV} \|_{2}}
    -
    \frac{\Vec{\vV}}{\| \Vec{\vV} \|_{2}}
  \right\|_{2}
  \\
  &\dCmt{by the triangle inequality}
  \\
  % &=
  % \frac{\| \Vec{\uV} - \Vec{\vV} \|_{2}}{\| \Vec{\uV} \|_{2}}
  % +
  % \left\|
  %   \frac{\Vec{\vV}}{\| \Vec{\vV} \|_{2}}
  %   \left(
  %   \frac{\| \Vec{\vV} \|_{2}}{\| \Vec{\uV} \|_{2}}
  %   -
  %   1
  %   \right)
  % \right\|_{2}
  % \\
  % &=
  % \frac{\| \Vec{\uV} - \Vec{\vV} \|_{2}}{\| \Vec{\uV} \|_{2}}
  % +
  % \left|
  % \frac{\| \Vec{\vV} \|_{2}}{\| \Vec{\uV} \|_{2}}
  % -
  % 1
  % \right|
  % \left\|
  %   \frac{\Vec{\vV}}{\| \Vec{\vV} \|_{2}}
  % \right\|_{2}
  % \\
  % &\dCmt{by the homogeneity of the (\(  \lnorm{2}  \)-)norm}
  % \\
  % &=
  % \frac{\| \Vec{\uV} - \Vec{\vV} \|_{2}}{\| \Vec{\uV} \|_{2}}
  % +
  % \left|
  %   1 - \frac{\| \Vec{\vV} \|_{2}}{\| \Vec{\uV} \|_{2}}
  % \right|
  % \\
  &=
  \frac{\| \Vec{\uV} - \Vec{\vV} \|_{2}}{\| \Vec{\uV} \|_{2}}
  +
  \left|
    1 - \frac{\| \Vec{\uV} - ( \Vec{\uV} - \Vec{\vV} ) \|_{2}}{\| \Vec{\uV} \|_{2}}
  \right|
  \\
  &\leq
  \frac{\| \Vec{\uV} - \Vec{\vV} \|_{2}}{\| \Vec{\uV} \|_{2}}
  +
  \max \left\{
    \left|
      1 - \frac{\| \Vec{\uV} \|_{2} + \|  \Vec{\uV} - \Vec{\vV} \|_{2}}{\| \Vec{\uV} \|_{2}}
    \right|
    ,~
    \left|
      1 -
      \left| \frac{\| \Vec{\uV} \|_{2} - \|  \Vec{\uV} - \Vec{\vV} \|_{2}}{\| \Vec{\uV} \|_{2}} \right|
    \right|
  \right\}
  \\
  &\dCmt{since \(  | \| \Vec{\uV} \|_{2} - \| \Vec{\uV} - \Vec{\vV} \|_{2} | \leq \| \Vec{\uV} - ( \Vec{\uV} - \Vec{\vV} ) \|_{2} \leq \| \Vec{\uV} \|_{2} + \| \Vec{\uV} - \Vec{\vV} \|_{2}  \)}
  \\
  &\dCmtIndent\text{by the triangle inequality}
  \\
  &=
  \frac{\| \Vec{\uV} - \Vec{\vV} \|_{2}}{\| \Vec{\uV} \|_{2}}
  +
  \max \left\{
    \left|
      1 - \left( 1 + \frac{\|  \Vec{\uV} - \Vec{\vV} \|_{2}}{\| \Vec{\uV} \|_{2}} \right)
    \right|
    ,~
    \left|
      1 - \left| 1 - \frac{\|  \Vec{\uV} - \Vec{\vV} \|_{2}}{\| \Vec{\uV} \|_{2}} \right|
    \right|
  \right\}
  \\
  % &=
  % \frac{\| \Vec{\uV} - \Vec{\vV} \|_{2}}{\| \Vec{\uV} \|_{2}}
  % +
  % \max \left\{
  %   \frac{\|  \Vec{\uV} - \Vec{\vV} \|_{2}}{\| \Vec{\uV} \|_{2}}
  %   ,~
  %   \left|
  %     1 - \left| 1 - \frac{\|  \Vec{\uV} - \Vec{\vV} \|_{2}}{\| \Vec{\uV} \|_{2}} \right|
  %   \right|
  % \right\}
  % \\ %\TagEqn{\label{pf:fact:dist-btw-normalized-vectors:1}}
  &=
  \frac{\| \Vec{\uV} - \Vec{\vV} \|_{2}}{\| \Vec{\uV} \|_{2}}
  +
  \max \left\{
    \frac{\|  \Vec{\uV} - \Vec{\vV} \|_{2}}{\| \Vec{\uV} \|_{2}}
    ,~
    \frac{\|  \Vec{\uV} - \Vec{\vV} \|_{2}}{\| \Vec{\uV} \|_{2}}
  \right\}
  \\
  &\dCmt{by \CLAIM \ref{claim:pf:fact:dist-btw-normalized-vectors:1}}
  \\
  &=
  \frac{2 \|  \Vec{\uV} - \Vec{\vV} \|_{2}}{\| \Vec{\uV} \|_{2}}
.\end{align*}
%|<<|===========================================================================================|<<|
%%|>>|===========================================================================================|>>|
%\begin{align*}
%  \left\| \frac{\Vec{\uV}}{\| \Vec{\uV} \|_{2}} - \frac{\Vec{\vV}}{\| \Vec{\vV} \|_{2}} \right\|_{2}
%  &=
%  \left\|
%    \left(
%      \frac{\Vec{\uV}}{\| \Vec{\uV} \|_{2}}
%      -
%      \frac{\Vec{\vV}}{\| \Vec{\uV} \|_{2}}
%    \right)
%    +
%    \left(
%      \frac{\Vec{\vV}}{\| \Vec{\uV} \|_{2}}
%      -
%      \frac{\Vec{\vV}}{\| \Vec{\vV} \|_{2}}
%    \right)
%  \right\|_{2}
%  \\
%  &\leq
%  \left\|
%    \frac{\Vec{\uV}}{\| \Vec{\uV} \|_{2}}
%    -
%    \frac{\Vec{\vV}}{\| \Vec{\uV} \|_{2}}
%  \right\|_{2}
%  +
%  \left\|
%    \frac{\Vec{\vV}}{\| \Vec{\uV} \|_{2}}
%    -
%    \frac{\Vec{\vV}}{\| \Vec{\vV} \|_{2}}
%  \right\|_{2}
%  \\
%  &\dCmt{by the triangle inequality}
%  \\
%  &=
%  \frac{\| \Vec{\uV} - \Vec{\vV} \|_{2}}{\| \Vec{\uV} \|_{2}}
%  +
%  \left\|
%    \frac{\Vec{\vV}}{\| \Vec{\vV} \|_{2}}
%    \left(
%    \frac{\| \Vec{\vV} \|_{2}}{\| \Vec{\uV} \|_{2}}
%    -
%    1
%    \right)
%  \right\|_{2}
%  \\
%  &=
%  \frac{\| \Vec{\uV} - \Vec{\vV} \|_{2}}{\| \Vec{\uV} \|_{2}}
%  +
%  \left|
%  \frac{\| \Vec{\vV} \|_{2}}{\| \Vec{\uV} \|_{2}}
%  -
%  1
%  \right|
%  \left\|
%    \frac{\Vec{\vV}}{\| \Vec{\vV} \|_{2}}
%  \right\|_{2}
%  \\
%  &\dCmt{by the homogeneity of the (\(  \lnorm{2}  \)-)norm}
%  \\
%  &=
%  \frac{\| \Vec{\uV} - \Vec{\vV} \|_{2}}{\| \Vec{\uV} \|_{2}}
%  +
%  \left|
%    1 - \frac{\| \Vec{\vV} \|_{2}}{\| \Vec{\uV} \|_{2}}
%  \right|
%  \\
%  &=
%  \frac{\| \Vec{\uV} - \Vec{\vV} \|_{2}}{\| \Vec{\uV} \|_{2}}
%  +
%  \left|
%    1 - \frac{\| \Vec{\uV} - ( \Vec{\uV} - \Vec{\vV} ) \|_{2}}{\| \Vec{\uV} \|_{2}}
%  \right|
%  \\
%  &\leq
%  \max \left\{
%  \begin{array}{c}
%    \frac{\| \Vec{\uV} - \Vec{\vV} \|_{2}}{\| \Vec{\uV} \|_{2}}
%    +
%    \left|
%      1 - \frac{\| \Vec{\uV} \|_{2} + \|  \Vec{\uV} - \Vec{\vV} \|_{2}}{\| \Vec{\uV} \|_{2}}
%    \right|
%    ,\\
%    \frac{\| \Vec{\uV} - \Vec{\vV} \|_{2}}{\| \Vec{\uV} \|_{2}}
%    +
%    \left|
%      1 -
%      \left| \frac{\| \Vec{\uV} \|_{2} - \|  \Vec{\uV} - \Vec{\vV} \|_{2}}{\| \Vec{\uV} \|_{2}} \right|
%    \right|
%  \end{array}
%  \right\}
%  \\
%  &=
%  \max \left\{
%  \begin{array}{c}
%    \frac{\| \Vec{\uV} - \Vec{\vV} \|_{2}}{\| \Vec{\uV} \|_{2}}
%    +
%    \left|
%      1 - \left( 1 + \frac{\|  \Vec{\uV} - \Vec{\vV} \|_{2}}{\| \Vec{\uV} \|_{2}} \right)
%    \right|
%    ,\\
%    \frac{\| \Vec{\uV} - \Vec{\vV} \|_{2}}{\| \Vec{\uV} \|_{2}}
%    +
%    \left|
%      1 - \left| 1 - \frac{\|  \Vec{\uV} - \Vec{\vV} \|_{2}}{\| \Vec{\uV} \|_{2}} \right|
%    \right|
%  \end{array}
%  \right\}
%  \\
%  &=
%  \max \left\{
%  \begin{array}{c}
%    \frac{\| \Vec{\uV} - \Vec{\vV} \|_{2}}{\| \Vec{\uV} \|_{2}}
%    +
%    \frac{\|  \Vec{\uV} - \Vec{\vV} \|_{2}}{\| \Vec{\uV} \|_{2}}
%    ,\\
%    \frac{\| \Vec{\uV} - \Vec{\vV} \|_{2}}{\| \Vec{\uV} \|_{2}}
%    +
%    \left|
%      1 - \left| 1 - \frac{\|  \Vec{\uV} - \Vec{\vV} \|_{2}}{\| \Vec{\uV} \|_{2}} \right|
%    \right|
%  \end{array}
%  \right\}
%  \\ %\TagEqn{\label{pf:fact:dist-btw-normalized-vectors:1}}
%  &=
%  \max \left\{
%  \begin{array}{c}
%    \frac{\| \Vec{\uV} - \Vec{\vV} \|_{2}}{\| \Vec{\uV} \|_{2}}
%    +
%    \frac{\|  \Vec{\uV} - \Vec{\vV} \|_{2}}{\| \Vec{\uV} \|_{2}}
%    ,\\
%    \frac{\| \Vec{\uV} - \Vec{\vV} \|_{2}}{\| \Vec{\uV} \|_{2}}
%    +
%    \frac{\|  \Vec{\uV} - \Vec{\vV} \|_{2}}{\| \Vec{\uV} \|_{2}}
%  \end{array}
%  \right\}
%  \\
%  &\dCmt{by \CLAIM \ref{claim:pf:fact:dist-btw-normalized-vectors:1}}
%  \\
%  &=
%  \frac{2 \|  \Vec{\uV} - \Vec{\vV} \|_{2}}{\| \Vec{\uV} \|_{2}}
%.\end{align*}
%%|<<|===========================================================================================|<<|
A nearly identical derivation obtains
%|>>|===========================================================================================|>>|
\begin{align*}
  \left\| \frac{\Vec{\uV}}{\| \Vec{\uV} \|_{2}} - \frac{\Vec{\vV}}{\| \Vec{\vV} \|_{2}} \right\|_{2}
  \leq
  \frac{2 \|  \Vec{\uV} - \Vec{\vV} \|_{2}}{\| \Vec{\vV} \|_{2}}
.\end{align*}
%|<<|===========================================================================================|<<|
Combining the two acquired bounds implies the fact:
%|>>|===========================================================================================|>>|
\begin{align*}
  \left\| \frac{\Vec{\uV}}{\| \Vec{\uV} \|_{2}} - \frac{\Vec{\vV}}{\| \Vec{\vV} \|_{2}} \right\|_{2}
  \leq
  2
  \min \left\{
    \frac{\|  \Vec{\uV} - \Vec{\vV} \|_{2}}{\| \Vec{\uV} \|_{2}},
    \frac{\|  \Vec{\uV} - \Vec{\vV} \|_{2}}{\| \Vec{\vV} \|_{2}}
  \right\}
,\end{align*}
%|<<|===========================================================================================|<<|
as desired.
\end{proof}
%|<<|~~~~~~~~~~~~~~~~~~~~~~~~~~~~~~~~~~~~~~~~~~~~~~~~~~~~~~~~~~~~~~~~~~~~~~~~~~~~~~~~~~~~~~~~~~~|<<|
%|<<|~~~~~~~~~~~~~~~~~~~~~~~~~~~~~~~~~~~~~~~~~~~~~~~~~~~~~~~~~~~~~~~~~~~~~~~~~~~~~~~~~~~~~~~~~~~|<<|
%|<<|~~~~~~~~~~~~~~~~~~~~~~~~~~~~~~~~~~~~~~~~~~~~~~~~~~~~~~~~~~~~~~~~~~~~~~~~~~~~~~~~~~~~~~~~~~~|<<|

We now proceed to the proof of \LEMMA \ref{lemma:error:deterministic}.

%|>>|~~~~~~~~~~~~~~~~~~~~~~~~~~~~~~~~~~~~~~~~~~~~~~~~~~~~~~~~~~~~~~~~~~~~~~~~~~~~~~~~~~~~~~~~~~~|>>|
%|>>|~~~~~~~~~~~~~~~~~~~~~~~~~~~~~~~~~~~~~~~~~~~~~~~~~~~~~~~~~~~~~~~~~~~~~~~~~~~~~~~~~~~~~~~~~~~|>>|
%|>>|~~~~~~~~~~~~~~~~~~~~~~~~~~~~~~~~~~~~~~~~~~~~~~~~~~~~~~~~~~~~~~~~~~~~~~~~~~~~~~~~~~~~~~~~~~~|>>|
\begin{proof}
{\LEMMA \ref{lemma:error:deterministic}}
%
\checkoff%
%
Consider any \(  \Vec{\uV} \in \SparseRealSubspace{\k}{\n}  \), \(  \Vec{\vV} \in \R^{\n}  \), and \(  \JCoords \subseteq [\n]  \), \(  | \JCoords | \leq \k  \).
Recall the notations of the coordinate subsets
%|>>|:::::::::::::::::::::::::::::::::::::::::::::::::::::::::::::::::::::::::::::::::::::::::::|>>|
\(  \JCoordsu, \JCoordsv \subseteq [\n]  \),
%|<<|:::::::::::::::::::::::::::::::::::::::::::::::::::::::::::::::::::::::::::::::::::::::::::|<<|
where
%|>>|:::::::::::::::::::::::::::::::::::::::::::::::::::::::::::::::::::::::::::::::::::::::::::|>>|
\(  \JCoordsu \defeq \Supp( \Vec{\uV} )  \) and
\(  \JCoordsv \defeq \Supp( \Threshold{\k}( \Vec{\vV} ) )  \).
%|<<|:::::::::::::::::::::::::::::::::::::::::::::::::::::::::::::::::::::::::::::::::::::::::::|<<|
Due to the definition of \(  \JCoordsv  \),
%|>>|===========================================================================================|>>|
\begin{align*}
  \left\|
    \Vec{\uV}
    -
    \frac
    {\Threshold{\k}(\Vec{\vV})}
    {\| \Threshold{\k}(\Vec{\vV}) \|_{2}}
  \right\|_{2}
  =
  \left\|
    \Vec{\uV}
    -
    \frac
    {\ThresholdSet{\JCoordsv}(\Vec{\vV})}
    {\| \ThresholdSet{\JCoordsv}(\Vec{\vV}) \|_{2}}
  \right\|_{2}
.\end{align*}
%|<<|===========================================================================================|<<|
Then,
%|>>|===========================================================================================|>>|
\begin{align*}
  \left\|
    \Vec{\uV}
    -
    \frac
    {\Threshold{\k}(\Vec{\vV})}
    {\| \Threshold{\k}(\Vec{\vV}) \|_{2}}
  \right\|_{2}
  &=
  \left\|
    \Vec{\uV}
    -
    \frac
    {\ThresholdSet{\JCoordsv}(\Vec{\vV})}
    {\| \ThresholdSet{\JCoordsv}(\Vec{\vV}) \|_{2}}
  \right\|_{2}
  \\
  &=
  \left\|
    \left(
    \Vec{\uV}
    -
    \frac
    {\ThresholdSet{\JCoords \cup \JCoordsu \cup \JCoordsv}(\Vec{\vV})}
    {\| \ThresholdSet{\JCoords \cup \JCoordsu \cup \JCoordsv}(\Vec{\vV}) \|_{2}}
    \right)
    +
    \left(
    \frac
    {\ThresholdSet{\JCoords \cup \JCoordsu \cup \JCoordsv}(\Vec{\vV})}
    {\| \ThresholdSet{\JCoords \cup \JCoordsu \cup \JCoordsv}(\Vec{\vV}) \|_{2}}
    -
    \frac
    {\ThresholdSet{\JCoordsv}(\Vec{\vV})}
    {\| \ThresholdSet{\JCoordsv}(\Vec{\vV}) \|_{2}}
    \right)
  \right\|_{2}
  \\
  &\leq
  \left\|
    \Vec{\uV}
    -
    \frac
    {\ThresholdSet{\JCoords \cup \JCoordsu \cup \JCoordsv}(\Vec{\vV})}
    {\| \ThresholdSet{\JCoords \cup \JCoordsu \cup \JCoordsv}(\Vec{\vV}) \|_{2}}
  \right\|_{2}
  +
  \left\|
    \frac
    {\ThresholdSet{\JCoords \cup \JCoordsu \cup \JCoordsv}(\Vec{\vV})}
    {\| \ThresholdSet{\JCoords \cup \JCoordsu \cup \JCoordsv}(\Vec{\vV}) \|_{2}}
    -
    \frac
    {\ThresholdSet{\JCoordsv}(\Vec{\vV})}
    {\| \ThresholdSet{\JCoordsv}(\Vec{\vV}) \|_{2}}
  \right\|_{2}
\TagEqn{\label{eqn:pf:lemma:error:deterministic:5}}
,\end{align*}
%|<<|===========================================================================================|<<|
where the last line follows from the triangle inequality.
Focusing in on the second term in the last line above, it follows from \FACT \ref{fact:dist-btw-normalized-vectors} that
%|>>|===========================================================================================|>>|
\begin{align*}
  \left\|
    \frac
    {\ThresholdSet{\JCoords \cup \JCoordsu \cup \JCoordsv}(\Vec{\vV})}
    {\| \ThresholdSet{\JCoords \cup \JCoordsu \cup \JCoordsv}(\Vec{\vV}) \|_{2}}
    -
    \frac
    {\ThresholdSet{\JCoordsv}(\Vec{\vV})}
    {\| \ThresholdSet{\JCoordsv}(\Vec{\vV}) \|_{2}}
  \right\|_{2}
  \leq
  \frac
  {2 \| \ThresholdSet{\JCoords \cup \JCoordsu \cup \JCoordsv}(\Vec{\vV}) - \ThresholdSet{\JCoordsv}(\Vec{\vV}) \|_{2}}
  {\| \ThresholdSet{\JCoords \cup \JCoordsu \cup \JCoordsv}(\Vec{\vV}) \|_{2}}
.\end{align*}
%|<<|===========================================================================================|<<|
Note that
%|>>|:::::::::::::::::::::::::::::::::::::::::::::::::::::::::::::::::::::::::::::::::::::::::::|>>|
\(  \ThresholdSet{\JCoords \cup \JCoordsu \cup \JCoordsv}(\Vec{\vV}) - \ThresholdSet{\JCoordsv}(\Vec{\vV}) = \ThresholdSet{( \JCoords \cup \JCoordsu ) \setminus \JCoordsv}(\Vec{\vV})  \),
%|<<|:::::::::::::::::::::::::::::::::::::::::::::::::::::::::::::::::::::::::::::::::::::::::::|<<|
and hence,
%|>>|===========================================================================================|>>|
\begin{align}
\label{eqn:pf:lemma:error:deterministic:1}
  \left\|
    \frac
    {\ThresholdSet{\JCoords \cup \JCoordsu \cup \JCoordsv}(\Vec{\vV})}
    {\| \ThresholdSet{\JCoords \cup \JCoordsu \cup \JCoordsv}(\Vec{\vV}) \|_{2}}
    -
    \frac
    {\ThresholdSet{\JCoordsv}(\Vec{\vV})}
    {\| \ThresholdSet{\JCoordsv}(\Vec{\vV}) \|_{2}}
  \right\|_{2}
  \leq
  \frac
  {2 \| \ThresholdSet{\JCoords \cup \JCoordsu \cup \JCoordsv}(\Vec{\vV}) - \ThresholdSet{\JCoordsv}(\Vec{\vV}) \|_{2}}
  {\| \ThresholdSet{\JCoords \cup \JCoordsu \cup \JCoordsv}(\Vec{\vV}) \|_{2}}
  =
  \frac
  {2 \| \ThresholdSet{( \JCoords \cup \JCoordsu ) \setminus \JCoordsv}(\Vec{\vV}) \|_{2}}
  {\| \ThresholdSet{\JCoords \cup \JCoordsu \cup \JCoordsv}(\Vec{\vV}) \|_{2}}
.\end{align}
%|<<|===========================================================================================|<<|
%Towards upper bounding
%%|>>|:::::::::::::::::::::::::::::::::::::::::::::::::::::::::::::::::::::::::::::::::::::::::::|>>|
%\(  \| \ThresholdSet{( \JCoords \cup \JCoordsu ) \setminus \JCoordsv}(\Vec{\vV}) \|_{2}  \),
%%|<<|:::::::::::::::::::::::::::::::::::::::::::::::::::::::::::::::::::::::::::::::::::::::::::|<<|
%notice that
%by a union bound and the definitions of \(  \JCoords  \) and \(  \JCoordsu  \), their union has cardinality at most \(  | \JCoords \cup \JCoordsu | \leq | \JCoords | + | \JCoordsu | \leq 2\k  \), and thus,
Since
%|>>|:::::::::::::::::::::::::::::::::::::::::::::::::::::::::::::::::::::::::::::::::::::::::::|>>|
\(  | \JCoordsu | = | \Supp( \Vec{\uV} ) | \leq \k  \),
%|<<|:::::::::::::::::::::::::::::::::::::::::::::::::::::::::::::::::::::::::::::::::::::::::::|<<|
the definitions of \(  \JCoordsXX  \) and the top-\(  \k  \) thresholding operation imply that
%|>>|:::::::::::::::::::::::::::::::::::::::::::::::::::::::::::::::::::::::::::::::::::::::::::|>>|
\(  | \Supp( \ThresholdSet{\JCoordsu}( \Vec{\vV} ) ) | \leq | \Supp( \ThresholdSet{\JCoordsv}( \Vec{\vV} ) ) |  \),
%|<<|:::::::::::::::::::::::::::::::::::::::::::::::::::::::::::::::::::::::::::::::::::::::::::|<<|
as well as that
%|>>|===========================================================================================|>>|
\begin{gather}
\label{eqn:pf:lemma:error:deterministic:2}
  \| \ThresholdSet{( \JCoords \cup \JCoordsu ) \setminus \JCoordsv}(\Vec{\vV}) \|_{2}
  =
  \| \ThresholdSet{( \JCoords \cup \JCoordsu \cup \JCoordsv ) \setminus \JCoordsv}(\Vec{\vV}) \|_{2}
  \leq
  \| \ThresholdSet{( \JCoords \cup \JCoordsu \cup \JCoordsv ) \setminus \JCoordsu}(\Vec{\vV}) \|_{2}
  =
  \| \ThresholdSet{( \JCoords \cup \JCoordsv ) \setminus \JCoordsu}(\Vec{\vV}) \|_{2}
.\end{gather}
%|<<|===========================================================================================|<<|
%%|>>|===========================================================================================|>>|
%\begin{gather}
%\label{eqn:pf:lemma:error:deterministic:2}
%  \| \ThresholdSet{( \JCoords \cup \JCoordsu ) \setminus \JCoordsv}(\Vec{\vV}) \|_{2}
%  \leq
%  2 \| \ThresholdSet{\JCoordsv \setminus ( \JCoords \cup \JCoordsu )}(\Vec{\vV}) \|_{2}
%.\end{gather}
%%|<<|===========================================================================================|<<|
Additionally, observe:
%|>>|===========================================================================================|>>|
\begin{align*}
  \left\|
    \Vec{\uV}
    -
    \frac
    {\ThresholdSet{\JCoords \cup \JCoordsu \cup \JCoordsv}(\Vec{\vV})}
    {\| \ThresholdSet{\JCoords \cup \JCoordsu \cup \JCoordsv}(\Vec{\vV}) \|_{2}}
  \right\|_{2}^{2}
  &=
  \left\|
    \Vec{\uV}
    -
    \frac
    {\ThresholdSet{\JCoordsu}(\Vec{\vV})}
    {\| \ThresholdSet{\JCoords \cup \JCoordsu \cup \JCoordsv}(\Vec{\vV}) \|_{2}}
    -
    \frac
    {\ThresholdSet{(\JCoords \cup \JCoordsv ) \setminus \JCoordsu}(\Vec{\vV})}
    {\| \ThresholdSet{\JCoords \cup \JCoordsu \cup \JCoordsv}(\Vec{\vV}) \|_{2}}
  \right\|_{2}^{2}
  \\
  &=
  \left\|
    \Vec{\uV}
    -
    \frac
    {\ThresholdSet{\JCoordsu}(\Vec{\vV})}
    {\| \ThresholdSet{\JCoords \cup \JCoordsu \cup \JCoordsv}(\Vec{\vV}) \|_{2}}
  \right\|_{2}^{2}
  +
  \left\|
    \frac
    {\ThresholdSet{(\JCoords \cup \JCoordsv ) \setminus \JCoordsu}(\Vec{\vV})}
    {\| \ThresholdSet{\JCoords \cup \JCoordsu \cup \JCoordsv}(\Vec{\vV}) \|_{2}}
  \right\|_{2}^{2}
\TagEqn{\label{eqn:pf:lemma:error:deterministic:3}}
,\end{align*}
%|<<|===========================================================================================|<<|
where the first equality holds since
%|>>|:::::::::::::::::::::::::::::::::::::::::::::::::::::::::::::::::::::::::::::::::::::::::::|>>|
\(  \ThresholdSet{\JCoords \cup \JCoordsu \cup \JCoordsv}(\Vec{\vV}) = \ThresholdSet{\JCoordsu}(\Vec{\vV}) + \ThresholdSet{( \JCoords \cup \JCoordsu \cup \JCoordsv ) \setminus \JCoordsu}(\Vec{\vV}) = \ThresholdSet{\JCoordsu}(\Vec{\vV}) + \ThresholdSet{( \JCoords \cup \JCoordsu ) \setminus \JCoordsu}(\Vec{\vV})  \),
%|<<|:::::::::::::::::::::::::::::::::::::::::::::::::::::::::::::::::::::::::::::::::::::::::::|<<|
and where the second equality is due to the orthogonality of
%|>>|:::::::::::::::::::::::::::::::::::::::::::::::::::::::::::::::::::::::::::::::::::::::::::|>>|
\(  \Vec{\uV} + w \ThresholdSet{\JCoordsu}(\Vec{\vV})  \)
    %= \ThresholdSet{\JCoordsu}( \Vec{\uV} + w \Vec{\vV} )  \)
%|<<|:::::::::::::::::::::::::::::::::::::::::::::::::::::::::::::::::::::::::::::::::::::::::::|<<|
and
%|>>|:::::::::::::::::::::::::::::::::::::::::::::::::::::::::::::::::::::::::::::::::::::::::::|>>|
\(  \ThresholdSet{( \JCoords \cup \JCoordsv ) \setminus \JCoordsu}(\Vec{\vV})  \)
%|<<|:::::::::::::::::::::::::::::::::::::::::::::::::::::::::::::::::::::::::::::::::::::::::::|<<|
for any scalar \(  w \in \R  \).
This orthogonality is the result of disjoint support sets:
%|>>|:::::::::::::::::::::::::::::::::::::::::::::::::::::::::::::::::::::::::::::::::::::::::::|>>|
\(  \Supp( \Vec{\uV} + w \ThresholdSet{\JCoordsu}(\Vec{\vV}) ) \cap \Supp( \ThresholdSet{( \JCoords \cup \JCoordsv ) \setminus \JCoordsu}(\Vec{\vV}) ) \subseteq {\JCoordsu \cap ( ( \JCoords \cup \JCoordsv ) \setminus \JCoordsu )} = \emptyset  \).
%|<<|:::::::::::::::::::::::::::::::::::::::::::::::::::::::::::::::::::::::::::::::::::::::::::|<<|
Rearranging the terms in \EQUATION \eqref{eqn:pf:lemma:error:deterministic:3} and taking the square root yields:
%|>>|===========================================================================================|>>|
\begin{align}
\label{eqn:pf:lemma:error:deterministic:3b}
  \left\|
    \frac
    {\ThresholdSet{(\JCoords \cup \JCoordsv ) \setminus \JCoordsu}(\Vec{\vV})}
    {\| \ThresholdSet{\JCoords \cup \JCoordsu \cup \JCoordsv}(\Vec{\vV}) \|_{2}}
  \right\|_{2}
  =
  \sqrt{
  \left\|
    \Vec{\uV}
    -
    \frac
    {\ThresholdSet{\JCoords \cup \JCoordsu \cup \JCoordsv}(\Vec{\vV})}
    {\| \ThresholdSet{\JCoords \cup \JCoordsu \cup \JCoordsv}(\Vec{\vV}) \|_{2}}
  \right\|_{2}^{2}
  -
  \left\|
    \Vec{\uV}
    -
    \frac
    {\ThresholdSet{\JCoordsu}(\Vec{\vV})}
    {\| \ThresholdSet{\JCoords \cup \JCoordsu \cup \JCoordsv}(\Vec{\vV}) \|_{2}}
  \right\|_{2}^{2}
  }
.\end{align}
%|<<|===========================================================================================|<<|
From \EQUATION \eqref{eqn:pf:lemma:error:deterministic:3b}, it follows that
%|>>|===========================================================================================|>>|
\begin{align}
  \frac
  {\| \ThresholdSet{( \JCoords \cup \JCoordsv ) \setminus \JCoordsu}(\Vec{\vV}) \|_{2}}
  {\| \ThresholdSet{\JCoords \cup \JCoordsu \cup \JCoordsv}(\Vec{\vV}) \|_{2}}
  &=
  \sqrt{
  \left\|
    \Vec{\uV}
    -
    \frac
    {\ThresholdSet{\JCoords \cup \JCoordsu \cup \JCoordsv}(\Vec{\vV})}
    {\| \ThresholdSet{\JCoords \cup \JCoordsu \cup \JCoordsv}(\Vec{\vV}) \|_{2}}
  \right\|_{2}^{2}
  -
  \left\|
    \Vec{\uV}
    -
    \frac
    {\ThresholdSet{\JCoordsu}(\Vec{\vV})}
    {\| \ThresholdSet{\JCoords \cup \JCoordsu \cup \JCoordsv}(\Vec{\vV}) \|_{2}}
  \right\|_{2}^{2}
  }
  \\
  &\leq
  \left\|
    \Vec{\uV}
    -
    \frac
    {\ThresholdSet{\JCoords \cup \JCoordsu \cup \JCoordsv}(\Vec{\vV})}
    {\| \ThresholdSet{\JCoords \cup \JCoordsu \cup \JCoordsv}(\Vec{\vV}) \|_{2}}
  \right\|_{2}
\label{eqn:pf:lemma:error:deterministic:4}
.\end{align}
%|<<|===========================================================================================|<<|
Combining \EQUATIONS \eqref{eqn:pf:lemma:error:deterministic:1}, \eqref{eqn:pf:lemma:error:deterministic:2}, and \eqref{eqn:pf:lemma:error:deterministic:4},
%|>>|===========================================================================================|>>|
\begin{align*}
  \left\|
    \frac
    {\ThresholdSet{\JCoords \cup \JCoordsu \cup \JCoordsv}(\Vec{\vV})}
    {\| \ThresholdSet{\JCoords \cup \JCoordsu \cup \JCoordsv}(\Vec{\vV}) \|_{2}}
    -
    \frac
    {\ThresholdSet{\JCoordsv}(\Vec{\vV})}
    {\| \ThresholdSet{\JCoordsv}(\Vec{\vV}) \|_{2}}
  \right\|_{2}
  &\leq
  \frac
  {2 \| \ThresholdSet{( \JCoords \cup \JCoordsu ) \setminus \JCoordsv}(\Vec{\vV}) \|_{2}}
  {\| \ThresholdSet{\JCoords \cup \JCoordsu \cup \JCoordsv}(\Vec{\vV}) \|_{2}}
  \\
  &\dCmt{by \EQUATION \eqref{eqn:pf:lemma:error:deterministic:1}}
  \\
  &\leq
  \frac
  {2 \| \ThresholdSet{( \JCoords \cup \JCoordsv ) \setminus \JCoordsu}(\Vec{\vV}) \|_{2}}
  {\| \ThresholdSet{\JCoords \cup \JCoordsu \cup \JCoordsv}(\Vec{\vV}) \|_{2}}
  \\
  &\dCmt{by \EQUATION \eqref{eqn:pf:lemma:error:deterministic:2}}
  \\
  &\leq
  2
  \left\|
    \Vec{\uV}
    -
    \frac
    {\ThresholdSet{\JCoords \cup \JCoordsu \cup \JCoordsv}(\Vec{\vV})}
    {\| \ThresholdSet{\JCoords \cup \JCoordsu \cup \JCoordsv}(\Vec{\vV}) \|_{2}}
  \right\|_{2}
  . \TagEqn{\label{eqn:pf:lemma:error:deterministic:6}} \\
  &\dCmt{by \EQUATION \eqref{eqn:pf:lemma:error:deterministic:4}}
\end{align*}
%|<<|===========================================================================================|<<|
Now, returning to \EQUATION \eqref{eqn:pf:lemma:error:deterministic:5}, the proof is completed as follows:
%|>>|===========================================================================================|>>|
\begin{align*}
  \left\|
    \Vec{\uV}
    -
    \frac
    {\Threshold{\k}(\Vec{\vV})}
    {\| \Threshold{\k}(\Vec{\vV}) \|_{2}}
  \right\|_{2}
  &\leq
  \left\|
    \Vec{\uV}
    -
    \frac
    {\ThresholdSet{\JCoords \cup \JCoordsu \cup \JCoordsv}(\Vec{\vV})}
    {\| \ThresholdSet{\JCoords \cup \JCoordsu \cup \JCoordsv}(\Vec{\vV}) \|_{2}}
  \right\|_{2}
  +
  \left\|
    \frac
    {\ThresholdSet{\JCoords \cup \JCoordsu \cup \JCoordsv}(\Vec{\vV})}
    {\| \ThresholdSet{\JCoords \cup \JCoordsu \cup \JCoordsv}(\Vec{\vV}) \|_{2}}
    -
    \frac
    {\ThresholdSet{\JCoordsv}(\Vec{\vV})}
    {\| \ThresholdSet{\JCoordsv}(\Vec{\vV}) \|_{2}}
  \right\|_{2}
  \\
  &\dCmt{by \EQUATION \eqref{eqn:pf:lemma:error:deterministic:5}}
  \\
  &\leq
  \left\|
    \Vec{\uV}
    -
    \frac
    {\ThresholdSet{\JCoords \cup \JCoordsu \cup \JCoordsv}(\Vec{\vV})}
    {\| \ThresholdSet{\JCoords \cup \JCoordsu \cup \JCoordsv}(\Vec{\vV}) \|_{2}}
  \right\|_{2}
  +
  2
  \left\|
    \Vec{\uV}
    -
    \frac
    {\ThresholdSet{\JCoords \cup \JCoordsu \cup \JCoordsv}(\Vec{\vV})}
    {\| \ThresholdSet{\JCoords \cup \JCoordsu \cup \JCoordsv}(\Vec{\vV}) \|_{2}}
  \right\|_{2}
  \\
  &\dCmt{by \EQUATION \eqref{eqn:pf:lemma:error:deterministic:6}}
  \\
  &=
  3
  \left\|
    \Vec{\uV}
    -
    \frac
    {\ThresholdSet{\JCoords \cup \JCoordsu \cup \JCoordsv}(\Vec{\vV})}
    {\| \ThresholdSet{\JCoords \cup \JCoordsu \cup \JCoordsv}(\Vec{\vV}) \|_{2}}
  \right\|_{2}
,\end{align*}
%|<<|===========================================================================================|<<|
as desired.
\end{proof}
%|<<|~~~~~~~~~~~~~~~~~~~~~~~~~~~~~~~~~~~~~~~~~~~~~~~~~~~~~~~~~~~~~~~~~~~~~~~~~~~~~~~~~~~~~~~~~~~|<<|
%|<<|~~~~~~~~~~~~~~~~~~~~~~~~~~~~~~~~~~~~~~~~~~~~~~~~~~~~~~~~~~~~~~~~~~~~~~~~~~~~~~~~~~~~~~~~~~~|<<|
%|<<|~~~~~~~~~~~~~~~~~~~~~~~~~~~~~~~~~~~~~~~~~~~~~~~~~~~~~~~~~~~~~~~~~~~~~~~~~~~~~~~~~~~~~~~~~~~|<<|

%%%%%%%%%%%%%%%%%%%%%%%%%%%%%%%%%%%%%%%%%%%%%%%%%%%%%%%%%%%%%%%%%%%%%%%%%%%%%%%%%%%%%%%%%%%%%%%%%%%%
%%%%%%%%%%%%%%%%%%%%%%%%%%%%%%%%%%%%%%%%%%%%%%%%%%%%%%%%%%%%%%%%%%%%%%%%%%%%%%%%%%%%%%%%%%%%%%%%%%%%
%%%%%%%%%%%%%%%%%%%%%%%%%%%%%%%%%%%%%%%%%%%%%%%%%%%%%%%%%%%%%%%%%%%%%%%%%%%%%%%%%%%%%%%%%%%%%%%%%%%%