\section{Proof of the Concentration Inequalities, \LEMMA \ref{lemma:concentration-ineq}}
\label{outline:concentration-ineq}

%%%%%%%%%%%%%%%%%%%%%%%%%%%%%%%%%%%%%%%%%%%%%%%%%%%%%%%%%%%%%%%%%%%%%%%%%%%%%%%%%%%%%%%%%%%%%%%%%%%%
%%%%%%%%%%%%%%%%%%%%%%%%%%%%%%%%%%%%%%%%%%%%%%%%%%%%%%%%%%%%%%%%%%%%%%%%%%%%%%%%%%%%%%%%%%%%%%%%%%%%
%%%%%%%%%%%%%%%%%%%%%%%%%%%%%%%%%%%%%%%%%%%%%%%%%%%%%%%%%%%%%%%%%%%%%%%%%%%%%%%%%%%%%%%%%%%%%%%%%%%%

\subsection{Intermediate Results}
\label{outline:concentration-ineq|intermediate}

We return to \LEMMA \ref{lemma:concentration-ineq} to present its proof.
Towards this, the following pair of intermediate lemmas, whose proofs can be found in \SECTIONS \ref{outline:concentration-ineq|pf-noiseless} and \ref{outline:concentration-ineq|pf-noisy}, are provided below.

%|>>|*******************************************************************************************|>>|
%|>>|*******************************************************************************************|>>|
%|>>|*******************************************************************************************|>>|
\begin{lemma}
\label{lemma:concentration-ineq:noiseless}
%
Fix
%|>>|:::::::::::::::::::::::::::::::::::::::::::::::::::::::::::::::::::::::::::::::::::::::::::|>>|
\(  \sXX, \tX, \tauX \in (0,1)  \),
%|<<|:::::::::::::::::::::::::::::::::::::::::::::::::::::::::::::::::::::::::::::::::::::::::::|<<|
and let
%|>>|:::::::::::::::::::::::::::::::::::::::::::::::::::::::::::::::::::::::::::::::::::::::::::|>>|
\(  \thetaStar \in \ParamSpace  \),
\(  \JS \subseteq 2^{[\n]}  \), and
%\(  \ParamCoverX \subseteq \ParamSpace \setminus \Ball{\tauX}( \thetaStar )  \)
\(  \ParamCover, \ParamCoverX \subset \ParamSpace  \),
%|<<|:::::::::::::::::::::::::::::::::::::::::::::::::::::::::::::::::::::::::::::::::::::::::::|<<|
where \(  \JS  \), \(  \ParamCover  \), and \(  \ParamCoverX  \) are finite sets, and where
%|>>|:::::::::::::::::::::::::::::::::::::::::::::::::::::::::::::::::::::::::::::::::::::::::::|>>|
\(  \ParamCoverX \defeq \ParamCover \setminus \Ball{\tauX}( \thetaStar )  \).
%|<<|:::::::::::::::::::::::::::::::::::::::::::::::::::::::::::::::::::::::::::::::::::::::::::|<<|
Let
%|>>|:::::::::::::::::::::::::::::::::::::::::::::::::::::::::::::::::::::::::::::::::::::::::::|>>|
\(  \kO \defeq \kOExpr  \).
%|<<|:::::::::::::::::::::::::::::::::::::::::::::::::::::::::::::::::::::::::::::::::::::::::::|<<|
% Define the random variables \(  X_{1}, X_{2}  \) and random vector \(  \Vec{X}_{3}  \) by
% %|>>|===========================================================================================|>>|
% \begin{gather*}
%   X_{1} \defeq \left\langle \frac{1}{\sqrt{2\pi}} \hFn[\JCoords]( \thetaStar, \thetaX ), \frac{\thetaStar-\thetaX}{\| \thetaStar-\thetaX \|_{2}} \right\rangle
%   \\
%   X_{2} \defeq \left\langle \frac{1}{\sqrt{2\pi}} \hFn[\JCoords]( \thetaStar, \thetaX ), \frac{\thetaStar+\thetaX}{\| \thetaStar+\thetaX \|_{2}} \right\rangle
%   \\
%   \Vec{X}_{3} \defeq \frac{1}{\sqrt{2\pi}} \gFn[\JCoords]( \thetaStar, \thetaX )
% \end{gather*}
% %|<<|===========================================================================================|<<|
Then,
%|>>|===========================================================================================|>>|
\begin{gather}
  {%\textstyle
  \Pr \left(
    \ExistsST{\JCoords \in \JS, \thetaX \in \ParamCoverX}{
    \left| \left\langle \frac{\hFn[\JCoords]( \thetaStar, \thetaX )}{\sqrt{2\pi}}, \frac{\thetaStar-\thetaX}{\| \thetaStar-\thetaX \|_{2}} \right\rangle - \E \left[ \left\langle \frac{\hFn[\JCoords]( \thetaStar, \thetaX )}{\sqrt{2\pi}}, \frac{\thetaStar-\thetaX}{\| \thetaStar-\thetaX \|_{2}} \right\rangle \right] \right|
    >
    \frac{\tX \ADIST}{\pi}
    }
  \right) }
  \nonumber \\
  \leq
  2 | \JS | | \ParamCoverX | e^{-\frac{1}{3\pi} \m \tX^{2} \ADIST}
  \label{eqn:lemma:concentration-ineq:noiseless:pr:1}
  ,\\
  {%\textstyle
  \Pr \left(
    \ExistsST{\JCoords \in \JS, \thetaX \in \ParamCoverX}{
    \left| \left\langle \frac{\hFn[\JCoords]( \thetaStar, \thetaX )}{\sqrt{2\pi}}, \frac{\thetaStar+\thetaX}{\| \thetaStar+\thetaX \|_{2}} \right\rangle - \E \left[ \left\langle \frac{\hFn[\JCoords]( \thetaStar, \thetaX )}{\sqrt{2\pi}}, \frac{\thetaStar+\thetaX}{\| \thetaStar+\thetaX \|_{2}} \right\rangle \right] \right|
    >
    \frac{\tX \ADIST}{\pi}
    }
  \right) }
  \nonumber \\
  \leq
  2 | \JS | | \ParamCoverX | e^{-\frac{1}{3\pi} \m \tX^{2} \ADIST}
  \label{eqn:lemma:concentration-ineq:noiseless:pr:2}
  ,\\
  {%\textstyle
  \Pr \left(
    \ExistsST{\JCoords \in \JS, \thetaX \in \ParamCoverX}{
    \left\| \frac{\gFn[\JCoords]( \thetaStar, \thetaX )}{\sqrt{2\pi}} - \E \left[ \frac{\gFn[\JCoords]( \thetaStar, \thetaX )}{\sqrt{2\pi}} \right] \right\|_{2}
    >
    \sqrt{\frac{2 ( 1+\sXX )( \kO-2 ) \ADIST}{\m}}
    +
    \frac{\tX \ADIST}{\pi}
    }
    % \left\| \frac{1}{\sqrt{2\pi}} \gFn[\JCoords]( \thetaStar, \thetaX ) - \E \left[ \frac{1}{\sqrt{2\pi}} \gFn[\JCoords]( \thetaStar, \thetaX ) \right] \right\|_{2}
    % >
    % \sqrt{\frac{2}{\m} ( 1+\sXX )( \kO-2 ) \ADIST}
    % \right. }
    % \\
    % {\textstyle \left.
    % +
    % \frac{1}{\pi} \tX \ADIST
    % \ \ \exists \JCoords \in \JS, \thetaX \in \ParamCoverX
  \right) }
  \nonumber \\
  \leq
  | \JS | | \ParamCoverX | e^{-\frac{1}{2\pi ( 1+\sXX )} \m \tX^{2} \ADIST}
  +
  | \ParamCoverX | e^{-\frac{1}{3\pi} \m \sXX^{2} \ADIST}
  \label{eqn:lemma:concentration-ineq:noiseless:pr:3}
%  ,\\
%  \Pr \left(
%    \ExistsST{\JCoords \in \JS, \thetaX \in \ParamCover}{
%    \left\| \frac{1}{\sqrt{2\pi}} \gFn[\JCoords]( \thetaStar, \thetaX ) - \E \left[ \frac{1}{\sqrt{2\pi}} \gFn[\JCoords]( \thetaStar, \thetaX ) \right]  \right\|_{2}
%    >
%    \E \left[ \left\| \frac{1}{\sqrt{2\pi}} \gFn[\JCoords]( \thetaStar, \thetaX ) \right\|_{2} \right]
%    +
%    \frac{1}{\pi} \tX \ADIST
%    }
%  \right)
%  \nonumber \\
%  \leq
%  | \JS | | \ParamCover | e^{-\frac{1}{2\pi ( 1+\sXX )} \m \tX^{2} \ADIST}
%  +
%  | \ParamCover | e^{-\frac{1}{3\pi} \m \sXX^{2} \ADIST}
%  \label{eqn:lemma:concentration-ineq:noiseless:pr:3b}
,\end{gather}
%|<<|===========================================================================================|<<|
where for any \(  \JCoords \subseteq [\n]  \) and \(  \thetaX \in \ParamSpace  \),
%|>>|===========================================================================================|>>|
\begin{gather}
  \label{eqn:lemma:concentration-ineq:noiseless:ev:1}
  \E \left[ \left\langle \hFn( \thetaStar, \thetaX ), \frac{\thetaStar-\thetaX}{\| \thetaStar-\thetaX \|_{2}} \right\rangle \right]
  =
  \E \left[ \left\langle \hFn[\JCoords]( \thetaStar, \thetaX ), \frac{\thetaStar-\thetaX}{\| \thetaStar-\thetaX \|_{2}} \right\rangle \right]
  =
  \| \thetaStar-\thetaX \|_{2}
  ,\\ \label{eqn:lemma:concentration-ineq:noiseless:ev:2}
  \E \left[ \left\langle \hFn( \thetaStar, \thetaX ), \frac{\thetaStar+\thetaX}{\| \thetaStar+\thetaX \|_{2}} \right\rangle \right]
  =
  \E \left[ \left\langle \hFn[\JCoords]( \thetaStar, \thetaX ), \frac{\thetaStar+\thetaX}{\| \thetaStar+\thetaX \|_{2}} \right\rangle \right]
  = 0
  ,\\ \label{eqn:lemma:concentration-ineq:noiseless:ev:3}
  \E[ \gFn( \thetaStar, \thetaX ) ]
  =
  \E[ \gFn[\JCoords]( \thetaStar, \thetaX ) ]
  = \Vec{0}
%  ,\\
%  \E \left[ \left\| \gFn[\JCoords]( \thetaStar, \thetaX ) \right\|_{2} \right]
%  \leq
%  \sqrt{\frac{2}{\m} ( 1+\sXX )( \kO-2 ) \ADIST}
.\end{gather}
%|<<|===========================================================================================|<<|
\end{lemma}
%|<<|*******************************************************************************************|<<|
%|<<|*******************************************************************************************|<<|
%|<<|*******************************************************************************************|<<|

\begin{comment}
%|>>|*******************************************************************************************|>>|
%|>>|*******************************************************************************************|>>|
%|>>|*******************************************************************************************|>>|
\begin{lemma}
\label{lemma:}
\label{lemma:concentration-ineq:noiseless:small}
%
Fix
%|>>|:::::::::::::::::::::::::::::::::::::::::::::::::::::::::::::::::::::::::::::::::::::::::::|>>|
\(  \sXX, \tX \in (0,1)  \),
%|<<|:::::::::::::::::::::::::::::::::::::::::::::::::::::::::::::::::::::::::::::::::::::::::::|<<|
and let
%|>>|:::::::::::::::::::::::::::::::::::::::::::::::::::::::::::::::::::::::::::::::::::::::::::|>>|
\(  \JSXX \subseteq 2^{[\n]}  \) and
\(  \ParamCover \subseteq \ParamSpace  \)
%|<<|:::::::::::::::::::::::::::::::::::::::::::::::::::::::::::::::::::::::::::::::::::::::::::|<<|
be finite sets.
Let
%|>>|:::::::::::::::::::::::::::::::::::::::::::::::::::::::::::::::::::::::::::::::::::::::::::|>>|
\(  \kO \defeq \kOExpr  \).
%|<<|:::::::::::::::::::::::::::::::::::::::::::::::::::::::::::::::::::::::::::::::::::::::::::|<<|
Then,
%
%|>>|===========================================================================================|>>|
\begin{gather}
  \Pr \left(
    \ExistsST{\JCoordsXX \in \JSXX, \thetaX \in \ParamCover, \thetaXX \in \BallX{\tauX}( \thetaX )}{
    \left| \left\langle \frac{1}{\sqrt{2\pi}} \hFn[\JCoordsXX]( \thetaX, \thetaXX ), \frac{\thetaX-\thetaXX}{\| \thetaX-\thetaXX \|_{2}} \right\rangle - \E \left[ \left\langle \frac{1}{\sqrt{2\pi}} \hFn[\JCoordsXX]( \thetaX, \thetaXX ), \frac{\thetaX-\thetaXX}{\| \thetaX-\thetaXX \|_{2}} \right\rangle \right] \right|
    >
    %\frac{1}{\pi} \tX \ADISTX
    \left( \sqrt{\frac{2}{\pi}} + \tX \right) \tauX
    }
  \right)
  \nonumber \\
  \leq
  2 | \JSXX | | \ParamCover | e^{-\frac{1}{3\pi} \m \tX^{2} \ADIST}
  \label{eqn:lemma:concentration-ineq:noiseless:small:pr:1}
  ,\\
  \Pr \left(
    \ExistsST{\JCoordsXX \in \JSXX, \thetaX \in \ParamCover, \thetaXX \in \BallX{\tauX}( \thetaX )}{
    \left| \left\langle \frac{1}{\sqrt{2\pi}} \hFn[\JCoords]( \thetaX, \thetaXX ), \frac{\thetaX+\thetaXX}{\| \thetaStar+\thetaX \|_{2}} \right\rangle - \E \left[ \left\langle \frac{1}{\sqrt{2\pi}} \hFn[\JCoords]( \thetaStar, \thetaX ), \frac{\thetaX+\thetaXX}{\| \thetaX+\thetaXX \|_{2}} \right\rangle \right] \right|
    >
    \frac{1}{\pi} \tX \ADISTX
    }
  \right)
  \nonumber \\
  \leq
  2 | \JSXX | | \ParamCover | e^{-\frac{1}{3\pi} \m \tX^{2} \ADIST}
  \label{eqn:lemma:concentration-ineq:noiseless:small:pr:2}
  ,\\
  \Pr \left(
    \ExistsST{\JCoordsXX \in \JSXX, \thetaX \in \ParamCover, \thetaXX \in \BallX{\tauX}( \thetaX )}{
    \left\| \frac{1}{\sqrt{2\pi}} \gFn[\JCoords]( \thetaX, \thetaXX ) - \E \left[ \frac{1}{\sqrt{2\pi}} \gFn[\JCoords]( \thetaX, \thetaXX ) \right] \right\|_{2}
    >
    \sqrt{\frac{1}{\pi \m} ( 1+\sXX )( \kO-2 ) \tauX}
    +
    \frac{1}{\pi} \tX \ADISTX
    }
  \right)
  \nonumber \\
  \leq
  | \JSXX | | \ParamCover | e^{-\frac{1}{2\pi ( 1+\sXX )} \m \tX^{2} \ADISTX}
  +
  | \ParamCoverX | e^{-\frac{1}{3\pi} \m \sXX^{2} \ADISTX}
  \label{eqn:lemma:concentration-ineq:noiseless:small:pr:3}
,\end{gather}
%|<<|===========================================================================================|<<|
\end{lemma}
%|<<|*******************************************************************************************|<<|
%|<<|*******************************************************************************************|<<|
%|<<|*******************************************************************************************|<<|
\end{comment}

%|>>|*******************************************************************************************|>>|
%|>>|*******************************************************************************************|>>|
%|>>|*******************************************************************************************|>>|
\begin{lemma}
\label{lemma:concentration-ineq:noisy}
%
Fix
%|>>|:::::::::::::::::::::::::::::::::::::::::::::::::::::::::::::::::::::::::::::::::::::::::::|>>|
\(  \tX > 0  \), \(  \sXXX \in (0,1)  \), and \(  \deltaX \in (0,1)  \),
%|<<|:::::::::::::::::::::::::::::::::::::::::::::::::::::::::::::::::::::::::::::::::::::::::::|<<|
and let
%|>>|:::::::::::::::::::::::::::::::::::::::::::::::::::::::::::::::::::::::::::::::::::::::::::|>>|
\(  \JSX \subseteq 2^{[\n]}  \).
%|<<|:::::::::::::::::::::::::::::::::::::::::::::::::::::::::::::::::::::::::::::::::::::::::::|<<|
Let
%|>>|:::::::::::::::::::::::::::::::::::::::::::::::::::::::::::::::::::::::::::::::::::::::::::|>>|
\(  \kOX \defeq \kOXExpr  \),
%|<<|:::::::::::::::::::::::::::::::::::::::::::::::::::::::::::::::::::::::::::::::::::::::::::|<<|
and define
%|>>|:::::::::::::::::::::::::::::::::::::::::::::::::::::::::::::::::::::::::::::::::::::::::::|>>|
\(  \alphaO = \alphaO( \deltaX ) = \alphaOExpr  \).
%|<<|:::::::::::::::::::::::::::::::::::::::::::::::::::::::::::::::::::::::::::::::::::::::::::|<<|
Then, for \(  \thetaStar \in \ParamSpace  \),
%|>>|===========================================================================================|>>|
\begin{gather}
  \label{eqn:lemma:concentration-ineq:noisy:pr:1}
  \Pr \left(
    \ExistsST{\JCoordsX \in \JSX}
    {\left| \left\langle \frac{\hfFn[\JCoordsX]( \thetaStar, \thetaStar )}{\sqrt{2\pi}}, \thetaStar \right\rangle - \E \left[ \left\langle \frac{\hfFn[\JCoordsX]( \thetaStar, \thetaStar )}{\sqrt{2\pi}}, \thetaStar \right\rangle \right] \right|
    >
    \alphaX \tX}
  \right)
  \leq
  2 | \JSX | e^{-\frac{1}{3} \alphaX \m \tX^{2}}
  ,\\ \nonumber
  \Pr \left(
    \ExistsST{\JCoordsX \in \JSX}
    {\left\| \frac{\gfFn[\JCoordsX]( \thetaStar, \thetaStar )}{\sqrt{2\pi}} - \E \left[ \frac{\gfFn[\JCoordsX]( \thetaStar, \thetaStar )}{\sqrt{2\pi}} \right]  \right\|_{2}
    >
    \sqrt{\frac{\alphaO ( 1+\sXXX )( \kOX-1 )}{\m}}
    +
    \alphaO \tX}
  \right)
  \\ \label{eqn:lemma:concentration-ineq:noisy:pr:2}
  \leq
  | \JSX | e^{-\frac{1}{2 ( 1+\sXXX )} \alphaO \m \tX^{2}}
  +
  e^{-\frac{1}{3} \alphaO \m \sXXX^{2}}
,\end{gather}
%|<<|===========================================================================================|<<|
where for any \(  \JCoordsX \subseteq [\n]  \),
%|>>|===========================================================================================|>>|
\begin{gather}
  \label{eqn:lemma:concentration-ineq:noisy:ev:1}
  \E \left[ \left\langle \hfFn( \thetaStar, \thetaStar ), \thetaStar \right\rangle \right]
  =
  \E \left[ \left\langle \hfFn[\JCoordsX]( \thetaStar, \thetaStar ), \thetaStar \right\rangle \right]
  = -\left( 1 - \sqrt{\frac{\pi}{2}} \gammaX \right)
  ,\\ \label{eqn:lemma:concentration-ineq:noisy:ev:2}
  \E[ \gFn( \thetaStar, \thetaStar ) ]
  =
  \E[ \gFn[\JCoordsX]( \thetaStar, \thetaStar ) ]
  = \Vec{0}
.\end{gather}
%|<<|===========================================================================================|<<|
\end{lemma}
%|<<|*******************************************************************************************|<<|
%|<<|*******************************************************************************************|<<|
%|<<|*******************************************************************************************|<<|

%%%%%%%%%%%%%%%%%%%%%%%%%%%%%%%%%%%%%%%%%%%%%%%%%%%%%%%%%%%%%%%%%%%%%%%%%%%%%%%%%%%%%%%%%%%%%%%%%%%%
%%%%%%%%%%%%%%%%%%%%%%%%%%%%%%%%%%%%%%%%%%%%%%%%%%%%%%%%%%%%%%%%%%%%%%%%%%%%%%%%%%%%%%%%%%%%%%%%%%%%
%%%%%%%%%%%%%%%%%%%%%%%%%%%%%%%%%%%%%%%%%%%%%%%%%%%%%%%%%%%%%%%%%%%%%%%%%%%%%%%%%%%%%%%%%%%%%%%%%%%%
%%%%%%%%%%%%%%%%%%%%%%%%%%%%%%%%%%%%%%%%%%%%%%%%%%%%%%%%%%%%%%%%%%%%%%%%%%%%%%%%%%%%%%%%%%%%%%%%%%%%
%%%%%%%%%%%%%%%%%%%%%%%%%%%%%%%%%%%%%%%%%%%%%%%%%%%%%%%%%%%%%%%%%%%%%%%%%%%%%%%%%%%%%%%%%%%%%%%%%%%%

\subsection{Proof of \LEMMA \ref{lemma:concentration-ineq}}
\label{outline:concentration-ineq|pf}

We are ready to prove \LEMMA \ref{lemma:concentration-ineq} by means of the intermediate lemmas in \SECTION \ref{outline:concentration-ineq|intermediate}.

%|>>|~~~~~~~~~~~~~~~~~~~~~~~~~~~~~~~~~~~~~~~~~~~~~~~~~~~~~~~~~~~~~~~~~~~~~~~~~~~~~~~~~~~~~~~~~~~|>>|
%|>>|~~~~~~~~~~~~~~~~~~~~~~~~~~~~~~~~~~~~~~~~~~~~~~~~~~~~~~~~~~~~~~~~~~~~~~~~~~~~~~~~~~~~~~~~~~~|>>|
%|>>|~~~~~~~~~~~~~~~~~~~~~~~~~~~~~~~~~~~~~~~~~~~~~~~~~~~~~~~~~~~~~~~~~~~~~~~~~~~~~~~~~~~~~~~~~~~|>>|
\begin{proof}
{\LEMMA \ref{lemma:concentration-ineq}}
%
\checkoff%
%
The proof of the lemma is split across \SECTIONS \ref{outline:concentration-ineq|pf|pr} and \ref{outline:concentration-ineq|pf|ev}, where the former derives \EQUATIONS \eqref{eqn:lemma:concentration-ineq:pr:1} and \eqref{eqn:lemma:concentration-ineq:pr:2} and the latter establishes \EQUATIONS \eqref{eqn:lemma:concentration-ineq:ev:1}--\eqref{eqn:lemma:concentration-ineq:ev:4}.

%%%%%%%%%%%%%%%%%%%%%%%%%%%%%%%%%%%%%%%%%%%%%%%%%%%%%%%%%%%%%%%%%%%%%%%%%%%%%%%%%%%%%%%%%%%%%%%%%%%%
%%%%%%%%%%%%%%%%%%%%%%%%%%%%%%%%%%%%%%%%%%%%%%%%%%%%%%%%%%%%%%%%%%%%%%%%%%%%%%%%%%%%%%%%%%%%%%%%%%%%
%%%%%%%%%%%%%%%%%%%%%%%%%%%%%%%%%%%%%%%%%%%%%%%%%%%%%%%%%%%%%%%%%%%%%%%%%%%%%%%%%%%%%%%%%%%%%%%%%%%%

\subsubsection{Proof of \EQUATIONS \eqref{eqn:lemma:concentration-ineq:pr:1} and \eqref{eqn:lemma:concentration-ineq:pr:2}}
\label{outline:concentration-ineq|pf|pr}

%%%%%%%%%%%%%%%%%%%%%%%%%%%%%%%%%%%%%%%%%%%%%%%%%%%%%%%%%%%%%%%%%%%%%%%%%%%%%%%%%%%%%%%%%%%%%%%%%%%%
\paragraph{Verification of \EQUATION \eqref{eqn:lemma:concentration-ineq:pr:1}} %%%%%%%%%%%%%%%%%%%%
%%%%%%%%%%%%%%%%%%%%%%%%%%%%%%%%%%%%%%%%%%%%%%%%%%%%%%%%%%%%%%%%%%%%%%%%%%%%%%%%%%%%%%%%%%%%%%%%%%%%
%
Fix
%|>>|:::::::::::::::::::::::::::::::::::::::::::::::::::::::::::::::::::::::::::::::::::::::::::|>>|
\(  \thetaStar \in \ParamSpace  \),
\(  \thetaX \in \ParamCoverX  \), and
\(  \JCoords \in \JS  \)
%|<<|:::::::::::::::::::::::::::::::::::::::::::::::::::::::::::::::::::::::::::::::::::::::::::|<<|
arbitrarily.
Towards bounding the concentration of \(  \hFn[\JCoords]( \thetaStar, \thetaX )  \) around its mean, consider the following orthogonal decomposition:
%|>>|===========================================================================================|>>|
\begin{align}
\nonumber
  \hFn[\JCoords]( \thetaStar, \thetaX )
  &=
  \left\langle
    \hFn[\JCoords]( \thetaStar, \thetaX ),
    \frac
    {\thetaStar-\thetaX}
    {\| \thetaStar-\thetaX \|_{2}}
  \right\rangle
  \frac
  {\thetaStar-\thetaX}
  {\| \thetaStar-\thetaX \|_{2}}
  \\
  &\AlignSp+
  \left\langle
    \hFn[\JCoords]( \thetaStar, \thetaX ),
    \frac
    {\thetaStar+\thetaX}
    {\| \thetaStar+\thetaX \|_{2}}
  \right\rangle
  \frac
  {\thetaStar+\thetaX}
  {\| \thetaStar+\thetaX \|_{2}}
  +
  \gFn[\JCoords]( \thetaStar, \thetaX )
\label{eqn:pf:lemma:concentration-ineq:1}
,\end{align}
%|<<|===========================================================================================|<<|
where, recalling \EQUATION \eqref{eqn:notations:gJ:def},
%|>>|===========================================================================================|>>|
\begin{align}
\nonumber
  \gFn[\JCoords]( \thetaStar, \thetaX )
  &=
  \hFn[\JCoords]( \thetaStar, \thetaX )
  -
  \left\langle
    \hFn[\JCoords]( \thetaStar, \thetaX ),
    \frac
    {\thetaStar-\thetaX}
    {\| \thetaStar-\thetaX \|_{2}}
  \right\rangle
  \frac
  {\thetaStar-\thetaX}
  {\| \thetaStar-\thetaX \|_{2}}
  \\
  &\AlignSp-
  \left\langle
    \hFn[\JCoords]( \thetaStar, \thetaX ),
    \frac
    {\thetaStar+\thetaX}
    {\| \thetaStar+\thetaX \|_{2}}
  \right\rangle
  \frac
  {\thetaStar+\thetaX}
  {\| \thetaStar+\thetaX \|_{2}}
\label{eqn:pf:lemma:concentration-ineq:1:b}
.\end{align}
%|<<|===========================================================================================|<<|
Due to \EQUATION \eqref{eqn:pf:lemma:concentration-ineq:1} and the linearity of expectation, the centered random vector
%|>>|:::::::::::::::::::::::::::::::::::::::::::::::::::::::::::::::::::::::::::::::::::::::::::|>>|
\(  \hFn[\JCoords]( \thetaStar, \thetaX ) - \E[ \hFn[\JCoords]( \thetaStar, \thetaX ) ]  \)
%|<<|:::::::::::::::::::::::::::::::::::::::::::::::::::::::::::::::::::::::::::::::::::::::::::|<<|
has the following orthogonal decomposition:
%|>>|===========================================================================================|>>|
\begin{align*}
  &
  \hFn[\JCoords]( \thetaStar, \thetaX ) - \E[ \hFn[\JCoords]( \thetaStar, \thetaX ) ]
  % \\
  % &\AlignIndent =
  % \left\langle
  %   \hFn[\JCoords]( \thetaStar, \thetaX ),
  %   \frac
  %   {\thetaStar-\thetaX}
  %   {\| \thetaStar-\thetaX \|_{2}}
  % \right\rangle
  % \frac
  % {\thetaStar-\thetaX}
  % {\| \thetaStar-\thetaX \|_{2}}
  % +
  % \left\langle
  %   \hFn[\JCoords]( \thetaStar, \thetaX ),
  %   \frac
  %   {\thetaStar+\thetaX}
  %   {\| \thetaStar+\thetaX \|_{2}}
  % \right\rangle
  % \frac
  % {\thetaStar+\thetaX}
  % {\| \thetaStar+\thetaX \|_{2}}
  % +
  % \gFn[\JCoords]( \thetaStar, \thetaX )
  % \\
  % &\AlignIndent \AlignSp-
  % \E \left[
  %   \left\langle
  %     \hFn[\JCoords]( \thetaStar, \thetaX ),
  %     \frac
  %     {\thetaStar-\thetaX}
  %     {\| \thetaStar-\thetaX \|_{2}}
  %   \right\rangle
  %   \frac
  %   {\thetaStar-\thetaX}
  %   {\| \thetaStar-\thetaX \|_{2}}
  %   +
  %   \left\langle
  %     \hFn[\JCoords]( \thetaStar, \thetaX ),
  %     \frac
  %     {\thetaStar+\thetaX}
  %     {\| \thetaStar+\thetaX \|_{2}}
  %   \right\rangle
  %   \frac
  %   {\thetaStar+\thetaX}
  %   {\| \thetaStar+\thetaX \|_{2}}
  %   +
  %   \gFn[\JCoords]( \thetaStar, \thetaX )
  % \right]
  % \\
  % &\AlignIndent
  % \dCmt{by \EQUATION \eqref{eqn:pf:lemma:concentration-ineq:1}}
  % \\
  % &\AlignIndent =
  % \left(
  %   \left\langle
  %     \hFn[\JCoords]( \thetaStar, \thetaX ),
  %     \frac
  %     {\thetaStar-\thetaX}
  %     {\| \thetaStar-\thetaX \|_{2}}
  %   \right\rangle
  %   \frac
  %   {\thetaStar-\thetaX}
  %   {\| \thetaStar-\thetaX \|_{2}}
  %   -
  %   \E \left[
  %     \left\langle
  %       \hFn[\JCoords]( \thetaStar, \thetaX ),
  %       \frac
  %       {\thetaStar-\thetaX}
  %       {\| \thetaStar-\thetaX \|_{2}}
  %     \right\rangle
  %     \frac
  %     {\thetaStar-\thetaX}
  %     {\| \thetaStar-\thetaX \|_{2}}
  %   \right]
  % \right)
  % \\
  % &\AlignIndent \AlignSp+
  % \left(
  %   \left\langle
  %     \hFn[\JCoords]( \thetaStar, \thetaX ),
  %     \frac
  %     {\thetaStar+\thetaX}
  %     {\| \thetaStar+\thetaX \|_{2}}
  %   \right\rangle
  %   \frac
  %   {\thetaStar+\thetaX}
  %   {\| \thetaStar+\thetaX \|_{2}}
  %   -
  %   \E \left[
  %   \left\langle
  %     \hFn[\JCoords]( \thetaStar, \thetaX ),
  %     \frac
  %     {\thetaStar+\thetaX}
  %     {\| \thetaStar+\thetaX \|_{2}}
  %     \right\rangle
  %     \frac
  %     {\thetaStar+\thetaX}
  %     {\| \thetaStar+\thetaX \|_{2}}
  %   \right]
  % \right)
  % \\
  % &\AlignIndent \AlignSp+
  % \left(
  %   \gFn[\JCoords]( \thetaStar, \thetaX )
  %   -
  %   \E[ \gFn[\JCoords]( \thetaStar, \thetaX ) ]
  % \right)
  % \\
  % &\AlignIndent
  % \dCmt{by \EQUATION \eqref{eqn:pf:lemma:concentration-ineq:1} and the linearity of expectation}
  \\
  &\AlignIndent =
  \left(
    \left\langle
      \hFn[\JCoords]( \thetaStar, \thetaX ),
      \frac
      {\thetaStar-\thetaX}
      {\| \thetaStar-\thetaX \|_{2}}
    \right\rangle
    -
    \E \left[
      \left\langle
        \hFn[\JCoords]( \thetaStar, \thetaX ),
        \frac
        {\thetaStar-\thetaX}
        {\| \thetaStar-\thetaX \|_{2}}
      \right\rangle
    \right]
  \right)
  \frac
  {\thetaStar-\thetaX}
  {\| \thetaStar-\thetaX \|_{2}}
  \\
  &\AlignIndent \AlignSp+
  \left(
    \left\langle
      \hFn[\JCoords]( \thetaStar, \thetaX ),
      \frac
      {\thetaStar+\thetaX}
      {\| \thetaStar+\thetaX \|_{2}}
    \right\rangle
    -
    \E \left[
    \left\langle
      \hFn[\JCoords]( \thetaStar, \thetaX ),
      \frac
      {\thetaStar+\thetaX}
      {\| \thetaStar+\thetaX \|_{2}}
      \right\rangle
    \right]
  \right)
  \frac
  {\thetaStar+\thetaX}
  {\| \thetaStar+\thetaX \|_{2}}
  \\
  &\AlignIndent \AlignSp+
  \left(
    \gFn[\JCoords]( \thetaStar, \thetaX )
    -
    \E[ \gFn[\JCoords]( \thetaStar, \thetaX ) ]
  \right)
\TagEqn{\label{lemma:concentration-ineq:pr:1}}
  % .\\
  % &\AlignIndent
  % \dCmt{by distributivity and the nonrandomness of \(  \thetaStar  \) and \(  \thetaX  \)}
\end{align*}
%|<<|===========================================================================================|<<|
due to \EQUATION \eqref{eqn:pf:lemma:concentration-ineq:1} and the linearity of expectation.
Applying the triangle inequality to the \(  \lnorm{2}  \)-norm of the orthogonal decomposition in \EQUATION \eqref{lemma:concentration-ineq:pr:1} and scaling it by a factor of \(  \frac{1}{\sqrt{2\pi}}  \) yields:
%|>>|===========================================================================================|>>|
\begin{align*}
  &%\AlignIndent
  \left\| \frac{1}{\sqrt{2\pi}} \hFn[\JCoords]( \thetaStar, \thetaX ) - \E \left[ \frac{1}{\sqrt{2\pi}} \hFn[\JCoords]( \thetaStar, \thetaX ) \right] \right\|_{2}
  \\
  &\AlignIndent \leq
  % \left\|
  % \left(
  %   \left\langle
  %     \frac{1}{\sqrt{2\pi}}
  %     \hFn[\JCoords]( \thetaStar, \thetaX ),
  %     \frac
  %     {\thetaStar-\thetaX}
  %     {\| \thetaStar-\thetaX \|_{2}}
  %   \right\rangle
  %   -
  %   \E \left[
  %     \left\langle
  %       \frac{1}{\sqrt{2\pi}}
  %       \hFn[\JCoords]( \thetaStar, \thetaX ),
  %       \frac
  %       {\thetaStar-\thetaX}
  %       {\| \thetaStar-\thetaX \|_{2}}
  %     \right\rangle
  %   \right]
  % \right)
  % \frac
  % {\thetaStar-\thetaX}
  % {\| \thetaStar-\thetaX \|_{2}}
  % \right\|_{2}
  % \\
  % &\AlignIndent \AlignSp+
  % \left\|
  % \left(
  %   \left\langle
  %     \frac{1}{\sqrt{2\pi}}
  %     \hFn[\JCoords]( \thetaStar, \thetaX ),
  %     \frac
  %     {\thetaStar+\thetaX}
  %     {\| \thetaStar+\thetaX \|_{2}}
  %   \right\rangle
  %   -
  %   \E \left[
  %   \left\langle
  %     \frac{1}{\sqrt{2\pi}}
  %     \hFn[\JCoords]( \thetaStar, \thetaX ),
  %     \frac
  %     {\thetaStar+\thetaX}
  %     {\| \thetaStar+\thetaX \|_{2}}
  %     \right\rangle
  %   \right]
  % \right)
  % \frac
  % {\thetaStar+\thetaX}
  % {\| \thetaStar+\thetaX \|_{2}}
  % \right\|_{2}
  % \\
  % &\AlignIndent \AlignSp+
  % \left\|
  %   \frac{1}{\sqrt{2\pi}}
  %   \gFn[\JCoords]( \thetaStar, \thetaX )
  %   -
  %   \E \left[ \frac{1}{\sqrt{2\pi}} \gFn[\JCoords]( \thetaStar, \thetaX ) \right]
  % \right\|_{2}
  % \\
  % &\AlignIndent =
  \left|
    \left\langle
      \frac{1}{\sqrt{2\pi}}
      \hFn[\JCoords]( \thetaStar, \thetaX ),
      \frac
      {\thetaStar-\thetaX}
      {\| \thetaStar-\thetaX \|_{2}}
    \right\rangle
    -
    \E \left[
      \left\langle
        \frac{1}{\sqrt{2\pi}}
        \hFn[\JCoords]( \thetaStar, \thetaX ),
        \frac
        {\thetaStar-\thetaX}
        {\| \thetaStar-\thetaX \|_{2}}
      \right\rangle
    \right]
  \right|
  \\
  &\AlignIndent \AlignSp+
  \left|
    \left\langle
      \frac{1}{\sqrt{2\pi}}
      \hFn[\JCoords]( \thetaStar, \thetaX ),
      \frac
      {\thetaStar+\thetaX}
      {\| \thetaStar+\thetaX \|_{2}}
    \right\rangle
    -
    \E \left[
    \left\langle
      \frac{1}{\sqrt{2\pi}}
      \hFn[\JCoords]( \thetaStar, \thetaX ),
      \frac
      {\thetaStar+\thetaX}
      {\| \thetaStar+\thetaX \|_{2}}
      \right\rangle
    \right]
  \right|
  \\
  &\AlignIndent \AlignSp+
  \left\|
    \frac{1}{\sqrt{2\pi}}
    \gFn[\JCoords]( \thetaStar, \thetaX )
    -
    \E \left[ \frac{1}{\sqrt{2\pi}} \gFn[\JCoords]( \thetaStar, \thetaX ) \right]
  \right\|_{2}
\TagEqn{\label{lemma:concentration-ineq:pr:2}}
.\end{align*}
%|<<|===========================================================================================|<<|
Due to \LEMMA \ref{lemma:concentration-ineq:noiseless}, the three terms in the last expression in \EQUATION \eqref{lemma:concentration-ineq:pr:2} are individually controlled with bounded probability as follows:
%|>>|===========================================================================================|>>|
\begin{gather*}
  {%\textstyle
  \Pr \left(
    \ExistsST{\JCoords \in \JS, \thetaX \in \ParamCoverX}{
    \left| \left\langle \frac{\hFn[\JCoords]( \thetaStar, \thetaX )}{\sqrt{2\pi}}, \frac{\thetaStar-\thetaX}{\| \thetaStar-\thetaX \|_{2}} \right\rangle - \E \left[ \left\langle \frac{\hFn[\JCoords]( \thetaStar, \thetaX )}{\sqrt{2\pi}}, \frac{\thetaStar-\thetaX}{\| \thetaStar-\thetaX \|_{2}} \right\rangle \right] \right|
    >
    \frac{\tX \ADIST}{3\pi}
    }
  \right) }
  \\
  \leq
  2 | \JS | | \ParamCoverX | e^{-\frac{1}{27\pi} \m \tX^{2} \ADIST}
  ,\\
  {%\textstyle
  \Pr \left(
    \ExistsST{\JCoords \in \JS, \thetaX \in \ParamCoverX}{
    \left| \left\langle \frac{\hFn[\JCoords]( \thetaStar, \thetaX )}{\sqrt{2\pi}}, \frac{\thetaStar+\thetaX}{\| \thetaStar+\thetaX \|_{2}} \right\rangle - \E \left[ \left\langle \frac{\hFn[\JCoords]( \thetaStar, \thetaX )}{\sqrt{2\pi}}, \frac{\thetaStar+\thetaX}{\| \thetaStar+\thetaX \|_{2}} \right\rangle \right] \right|
    >
    \frac{\tX \ADIST}{3\pi}
    }
  \right) }
  \\
  \leq
  2 | \JS | | \ParamCoverX | e^{-\frac{1}{27\pi} \m \tX^{2} \ADIST}
  ,\\
  {%\textstyle
  \Pr \Bigl(
    \ExistsST{\JCoords \in \JS, \thetaX \in \ParamCoverX}{
    \left\| \frac{\gFn[\JCoords]( \thetaStar, \thetaX )}{\sqrt{2\pi}} - \E \left[ \frac{\gFn[\JCoords]( \thetaStar, \thetaX )}{\sqrt{2\pi}} \right]  \right\|_{2}
    >
    \sqrt{\frac{2 ( 1+\sXX )( \kO-2 ) \ADIST}{\m}}
    +
    \frac{\tX \ADIST}{3\pi}
    }
    % \exists \JCoords \in \JS, \thetaX \in \ParamCoverX
    % \\
    % \textstyle \AlignIndent\AlignIndent\AlignIndent\AlignIndent
    % \left\| \frac{\gFn[\JCoords]( \thetaStar, \thetaX )}{\sqrt{2\pi}} - \E \left[ \frac{\gFn[\JCoords]( \thetaStar, \thetaX )}{\sqrt{2\pi}} \right]  \right\|_{2}
    % >
    % \sqrt{\frac{2}{\m} ( 1+\sXX )( \kO-2 ) \ADIST}
    % +
    % \frac{1}{3\pi} \tX \ADIST
  \Bigr) }
  \\
  \leq
  | \JS | | \ParamCoverX | e^{-\frac{1}{18\pi ( 1+\sXX )} \m \tX^{2} \ADIST}
  +
  | \ParamCoverX | e^{-\frac{1}{3\pi} \m \sXX^{2} \ADIST}
.\end{gather*}
%|<<|===========================================================================================|<<|
Combining the three above concentration inequalities via a union bound and complementing, it follows that with probability at least
%|>>|===========================================================================================|>>|
\begin{align*}
  1
  &-
  2 | \JS | | \ParamCoverX | e^{-\frac{1}{27\pi} \m \tX^{2} \ADIST}
  -
  2 | \JS | | \ParamCoverX | e^{-\frac{1}{27\pi} \m \tX^{2} \ADIST}
  -
  | \JS | | \ParamCoverX | e^{-\frac{1}{18\pi ( 1+\sXX )} \m \tX^{2} \ADIST}
  \\
  &\AlignSp-
  | \ParamCoverX | e^{-\frac{1}{3\pi} \m \sXX^{2} \ADIST}
  \\
  &=
  1
  -
  4 | \JS | | \ParamCoverX | e^{-\frac{1}{27\pi} \m \tX^{2} \ADIST}
  -
  | \JS | | \ParamCoverX | e^{-\frac{1}{18\pi ( 1+\sXX )} \m \tX^{2} \ADIST}
  -
  | \ParamCoverX | e^{-\frac{1}{3\pi} \m \sXX^{2} \ADIST}
,\end{align*}
%|<<|===========================================================================================|<<|
for all \(  \JCoords \in \JS  \) and all \(  \thetaX \in \ParamCoverX  \), the following three inequalities hold simultaneously:
%|>>|===========================================================================================|>>|
\begin{gather*}
  %\textstyle
  \left|
    \left\langle
      \frac{\hFn[\JCoords]( \thetaStar, \thetaX )}{\sqrt{2\pi}}
      ,
      \frac
      {\thetaStar-\thetaX}
      {\| \thetaStar-\thetaX \|_{2}}
    \right\rangle
    -
    \E \left[
      \left\langle
        \frac{\hFn[\JCoords]( \thetaStar, \thetaX )}{\sqrt{2\pi}}
        ,
        \frac
        {\thetaStar-\thetaX}
        {\| \thetaStar-\thetaX \|_{2}}
      \right\rangle
    \right]
  \right|
  \leq
  \frac{\tX \ADIST}{3\pi}
  ,\\ %\textstyle
  \left|
    \left\langle
      \frac{\hFn[\JCoords]( \thetaStar, \thetaX )}{\sqrt{2\pi}}
      ,
      \frac
      {\thetaStar+\thetaX}
      {\| \thetaStar+\thetaX \|_{2}}
    \right\rangle
    -
    \E \left[
    \left\langle
      \frac{\hFn[\JCoords]( \thetaStar, \thetaX )}{\sqrt{2\pi}}
      ,
      \frac
      {\thetaStar+\thetaX}
      {\| \thetaStar+\thetaX \|_{2}}
      \right\rangle
    \right]
  \right|
  \leq
  \frac{\tX \ADIST}{3\pi}
  ,\\ %\textstyle
  \left\|
    \frac{\gFn[\JCoords]( \thetaStar, \thetaX )}{\sqrt{2\pi}}
    -
    \E \left[ \frac{\gFn[\JCoords]( \thetaStar, \thetaX )}{\sqrt{2\pi}} \right]
  \right\|_{2}
  \leq
  \sqrt{\frac{2 ( 1+\sXX )( \kO-2 ) \ADIST}{\m}}
  +
  \frac{\tX \ADIST}{3\pi}
.\end{gather*}
%|<<|===========================================================================================|<<|
Then, taking this with \EQUATION \eqref{lemma:concentration-ineq:pr:2}, this implies that with probability at least
%|>>|===========================================================================================|>>|
\begin{align*}
  1
  -
  4 | \JS | | \ParamCoverX | e^{-\frac{1}{27\pi} \m \tX^{2} \ADIST}
  -
  | \JS | | \ParamCoverX | e^{-\frac{1}{18\pi ( 1+\sXX )} \m \tX^{2} \ADIST}
  -
  | \ParamCoverX | e^{-\frac{1}{3\pi} \m \sXX^{2} \ADIST}
,\end{align*}
%|<<|===========================================================================================|<<|
uniformly for all \(  \JCoords \in \JS  \) and all \(  \thetaX \in \ParamCoverX  \),
%|>>|===========================================================================================|>>|
\begin{align*}
  \left\| \frac{\hFn[\JCoords]( \thetaStar, \thetaX )}{\sqrt{2\pi}} - \E \left[ \frac{\hFn[\JCoords]( \thetaStar, \thetaX )}{\sqrt{2\pi}} \right] \right\|_{2}
  \leq
  \sqrt{\frac{2 ( 1+\sXX )( \kO-2 ) \ADIST}{\m} }
  +
  \frac{\tX \ADIST}{\pi}
,\end{align*}
%|<<|===========================================================================================|<<|
as desired.
This establishes \EQUATION \eqref{eqn:lemma:concentration-ineq:pr:1}.
%
%%%%%%%%%%%%%%%%%%%%%%%%%%%%%%%%%%%%%%%%%%%%%%%%%%%%%%%%%%%%%%%%%%%%%%%%%%%%%%%%%%%%%%%%%%%%%%%%%%%%
\paragraph{Verification of \EQUATION \eqref{eqn:lemma:concentration-ineq:pr:2}} %%%%%%%%%%%%%%%%%%%%
%%%%%%%%%%%%%%%%%%%%%%%%%%%%%%%%%%%%%%%%%%%%%%%%%%%%%%%%%%%%%%%%%%%%%%%%%%%%%%%%%%%%%%%%%%%%%%%%%%%%
%
Next, \EQUATION \eqref{eqn:lemma:concentration-ineq:pr:2} is derived via an analogous technique.
Again, an orthogonal decomposition will facilitate the proof, this time into just two components:
%|>>|===========================================================================================|>>|
\begin{align}
  \hfFn[\JCoords]( \thetaStar, \thetaStar )
  =
  \langle \hfFn[\JCoords]( \thetaStar, \thetaStar ), \thetaStar \rangle
  \thetaStar
  +
  \gfFn[\JCoords]( \thetaStar, \thetaStar )
\label{eqn:pf:lemma:concentration-ineq:2}
,\end{align}
%|<<|===========================================================================================|<<|
where, as defined in \EQUATION \eqref{eqn:notations:gfJ:def},
%|>>|===========================================================================================|>>|
\begin{align}
  \gfFn[\JCoords]( \thetaStar, \thetaStar )
  =
  \hfFn[\JCoords]( \thetaStar, \thetaStar )
  -
  \langle \hfFn[\JCoords]( \thetaStar, \thetaStar ), \thetaStar \rangle
  \thetaStar
\label{eqn:pf:lemma:concentration-ineq:2:b}
.\end{align}
%|<<|===========================================================================================|<<|
By the above orthogonal decomposition in \eqref{eqn:pf:lemma:concentration-ineq:2} and the linearity of expectation,
%|>>|===========================================================================================|>>|
\begin{align*}
  &
  \hfFn[\JCoords]( \thetaStar, \thetaStar ) - \E[ \hfFn[\JCoords]( \thetaStar, \thetaStar ) ]
  \\
  &\AlignIndent=
  \langle \hfFn[\JCoords]( \thetaStar, \thetaStar ), \thetaStar \rangle
  \thetaStar
  +
  \gfFn[\JCoords]( \thetaStar, \thetaStar )
  -
  \E[
    \langle \hfFn[\JCoords]( \thetaStar, \thetaStar ), \thetaStar \rangle
    \thetaStar
    +
    \gfFn[\JCoords]( \thetaStar, \thetaStar )
  ]
  \\
  &\AlignIndent=
  \left(
    \langle \hfFn[\JCoords]( \thetaStar, \thetaStar ), \thetaStar \rangle
    \thetaStar
    -
    \E[
      \langle \hfFn[\JCoords]( \thetaStar, \thetaStar ), \thetaStar \rangle
      \thetaStar
    ]
  \right)
  +
  \left(
    \gfFn[\JCoords]( \thetaStar, \thetaStar )
    -
    \E[ \gfFn[\JCoords]( \thetaStar, \thetaStar ) ]
  \right)
  \\
  &\AlignIndent=
  \left(
    \langle \hfFn[\JCoords]( \thetaStar, \thetaStar ), \thetaStar \rangle
    -
    \E[ \langle \hfFn[\JCoords]( \thetaStar, \thetaStar ), \thetaStar \rangle ]
  \right)
  \thetaStar
  +
  \left(
    \gfFn[\JCoords]( \thetaStar, \thetaStar )
    -
    \E[ \gfFn[\JCoords]( \thetaStar, \thetaStar ) ]
  \right)
.\end{align*}
%|<<|===========================================================================================|<<|
Then, taking the norm of the above expressions and applying the triangle inequality to the last line,
%|>>|===========================================================================================|>>|
\begin{align*}
  &
  \| \hfFn[\JCoords]( \thetaStar, \thetaStar ) - \E[ \hfFn[\JCoords]( \thetaStar, \thetaStar ) ] \|_{2}
  \\
  &\AlignIndent\leq
  \left\|
  \left(
    \langle \hfFn[\JCoords]( \thetaStar, \thetaStar ), \thetaStar \rangle
    -
    \E[
      \langle \hfFn[\JCoords]( \thetaStar, \thetaStar ), \thetaStar \rangle
    ]
  \right)
  \thetaStar
  \right\|_{2}
  +
  \left\|
    \gfFn[\JCoords]( \thetaStar, \thetaStar )
    -
    \E[ \gfFn[\JCoords]( \thetaStar, \thetaStar ) ]
  \right\|_{2}
  \\
  &\AlignIndent=
  \left|
    \langle \hfFn[\JCoords]( \thetaStar, \thetaStar ), \thetaStar \rangle
    -
    \E[
      \langle \hfFn[\JCoords]( \thetaStar, \thetaStar ), \thetaStar \rangle
    ]
  \right|
  +
  \left\|
    \gfFn[\JCoords]( \thetaStar, \thetaStar )
    -
    \E[ \gfFn[\JCoords]( \thetaStar, \thetaStar ) ]
  \right\|_{2}
.\end{align*}
%|<<|===========================================================================================|<<|
Scaling this by a factor of \(  \frac{1}{\sqrt{2\pi}}  \) yields:
%|>>|===========================================================================================|>>|
\begin{align*}
  &
  \left\| \frac{\hfFn[\JCoords]( \thetaStar, \thetaStar )}{\sqrt{2\pi}} - \E \left[ \frac{\hfFn[\JCoords]( \thetaStar, \thetaStar )}{\sqrt{2\pi}} \right] \right\|_{2}
  \\
  &\AlignIndent\leq
  \left|
    \left\langle \frac{\hfFn[\JCoords]( \thetaStar, \thetaStar )}{\sqrt{2\pi}} , \thetaStar \right\rangle
    -
    \E \left[
      \left\langle \frac{\hfFn[\JCoords]( \thetaStar, \thetaStar )}{\sqrt{2\pi}} , \thetaStar \right\rangle
    \right]
  \right|
  \\
  &\AlignIndent\AlignIndent+
  \left\|
    \frac{\gfFn[\JCoords]( \thetaStar, \thetaStar )}{\sqrt{2\pi}}
    -
    \E \left[ \frac{\gfFn[\JCoords]( \thetaStar, \thetaStar )}{\sqrt{2\pi}}  \right]
  \right\|_{2}
.\end{align*}
%|<<|===========================================================================================|<<|
By \LEMMA \ref{lemma:concentration-ineq:noisy}, for \(  \sXXX, \tXX \in (0,1)  \),
%|>>|===========================================================================================|>>|
\begin{align*}
  &\Pr \left(
    \ExistsST{\JCoordsX \in \JSX}
    {\left| \left\langle \frac{\hfFn[\JCoordsX]( \thetaStar, \thetaStar )}{\sqrt{2\pi}} , \thetaStar \right\rangle - \E \left[ \left\langle \frac{\hfFn[\JCoordsX]( \thetaStar, \thetaStar )}{\sqrt{2\pi}} , \thetaStar \right\rangle \right] \right|
    >
    \frac{\alphaO \tXX}{2}}
  \right)
  \\
  &\AlignIndent =
  \Pr \left(
    \ExistsST{\JCoordsX \in \JSX}
    {\left| \left\langle \frac{\hfFn[\JCoordsX]( \thetaStar, \thetaStar )}{\sqrt{2\pi}} , \thetaStar \right\rangle - \E \left[ \left\langle \frac{\hfFn[\JCoordsX]( \thetaStar, \thetaStar )}{\sqrt{2\pi}} , \thetaStar \right\rangle \right] \right|
    >
    \alphaX \left( \frac{\alphaO \tXX}{2 \alphaX} \right)}
  \right)
  % \\
  % &\AlignIndent \leq
  % 2 | \JSX | e^{-\frac{1}{12} \alphaX \frac{\alphaO^{2}}{\alphaX^{2}} \m \tXX^{2}}
  % \\
  % &\AlignIndent
  % \dCmt{due to \LEMMA \ref{lemma:concentration-ineq:noisy}}
  \\
  &\AlignIndent =
  2 | \JSX | e^{-\frac{1}{12} \frac{\alphaO}{\alphaX} \alphaO \m \tXX^{2}}
  \\
  &\AlignIndent
  \dCmt{due to \LEMMA \ref{lemma:concentration-ineq:noisy}}
  \\
  &\AlignIndent \leq
  2 | \JSX | e^{-\frac{1}{12} \alphaO \m \tXX^{2}}
  ,\\
  &\AlignIndent
  \dCmt{\(  \alphaO = \alphaOExpr \geq \alphaX  \) implies \(  \tfrac{\alphaO}{\alphaX} \geq 1  \)}
\end{align*}
%|<<|===========================================================================================|<<|
and
%|>>|===========================================================================================|>>|
\begin{gather*}
%  \Pr \left(
%    \ExistsST{\JCoordsX \in \JSX}
%    {\left| \left\langle \frac{1}{\sqrt{2\pi}} \hfFn[\JCoordsX]( \thetaStar, \thetaStar ), \thetaStar \right\rangle - \E \left[ \left\langle \frac{1}{\sqrt{2\pi}} \hfFn[\JCoordsX]( \thetaStar, \thetaStar ), \thetaStar \right\rangle \right] \right|
%    >
%    \frac{1}{2} \alphaX \tXX}
%  \right)
%  \leq
%  2 | \JSX | e^{-\frac{1}{12} \alphaX \m \tXX^{2}}
%  ,\\
  \Pr \left(
    \ExistsST{\JCoordsX \in \JSX}
    {\left\| \frac{\gfFn[\JCoordsX]( \thetaStar, \thetaStar )}{\sqrt{2\pi}} - \E \left[ \frac{\gfFn[\JCoordsX]( \thetaStar, \thetaStar )}{\sqrt{2\pi}} \right]  \right\|_{2}
    >
    \sqrt{\frac{\alphaO ( 1+\sXXX )( \kOX-1 )}{\m} }
    +
    \frac{\alphaO \tXX}{2} }
  \right)
  \\
  \leq
  | \JSX | e^{-\frac{1}{8 ( 1+\sXXX )} \alphaO \m \tXX^{2}}
  +
  e^{-\frac{1}{3} \alphaO \m \sXXX^{2}}
,\end{gather*}
%|<<|===========================================================================================|<<|
and hence, by a union bound over the above two probabilities, with probability at least
%|>>|===========================================================================================|>>|
\begin{gather*}
  1
  -
  2 | \JSX | e^{-\frac{1}{12} \alphaO \m \tXX^{2}}
  -
  | \JSX | e^{-\frac{1}{8 ( 1+\sXXX )} \alphaO \m \tXX^{2}}
  -
  e^{-\frac{1}{3} \alphaO \m \sXXX^{2}}
,\end{gather*}
%|<<|===========================================================================================|<<|
the norm of the centered random vector
%|>>|:::::::::::::::::::::::::::::::::::::::::::::::::::::::::::::::::::::::::::::::::::::::::::|>>|
\(  \frac{1}{\sqrt{2\pi}} \hfFn[\JCoords]( \thetaStar, \thetaStar ) - \E[ \frac{1}{\sqrt{2\pi}} \hfFn[\JCoords]( \thetaStar, \thetaStar ) ]  \)
%|<<|:::::::::::::::::::::::::::::::::::::::::::::::::::::::::::::::::::::::::::::::::::::::::::|<<|
is bounded from above by
%|>>|===========================================================================================|>>|
\begin{align*}
  % &
  \left\| \frac{\hfFn[\JCoords]( \thetaStar, \thetaStar )}{\sqrt{2\pi}} - \E \left[ \frac{\hfFn[\JCoords]( \thetaStar, \thetaStar )}{\sqrt{2\pi}} \right] \right\|_{2}
  % \\
  % &\AlignIndent\leq
  &\leq
  \left|
    \left\langle \frac{\hfFn[\JCoords]( \thetaStar, \thetaStar )}{\sqrt{2\pi}} , \thetaStar \right\rangle
    -
    \E \left[
      \left\langle \frac{\hfFn[\JCoords]( \thetaStar, \thetaStar )}{\sqrt{2\pi}} , \thetaStar \right\rangle
    \right]
  \right|
  \\
  &\AlignIndent\AlignIndent+
  \left\|
    \frac{\gfFn[\JCoords]( \thetaStar, \thetaStar )}{\sqrt{2\pi}}
    -
    \E \left[ \frac{\gfFn[\JCoords]( \thetaStar, \thetaStar )}{\sqrt{2\pi}}  \right]
  \right\|_{2}
  \\
  &\AlignIndent\leq
  \frac{\alphaO \tXX}{2} 
  +
  \sqrt{\frac{\alphaO ( 1+\sXXX )( \kOX-1 )}{\m} }
  +
  \frac{\alphaO \tXX}{2} 
  \\
  &\AlignIndent=
  \sqrt{\frac{\alphaO ( 1+\sXXX )( \kOX-1 )}{\m} }
  +
  \alphaO \tXX
.\end{align*}
%|<<|===========================================================================================|<<|
Thus, \EQUATION \eqref{eqn:lemma:concentration-ineq:pr:2} holds.

%%%%%%%%%%%%%%%%%%%%%%%%%%%%%%%%%%%%%%%%%%%%%%%%%%%%%%%%%%%%%%%%%%%%%%%%%%%%%%%%%%%%%%%%%%%%%%%%%%%%
%%%%%%%%%%%%%%%%%%%%%%%%%%%%%%%%%%%%%%%%%%%%%%%%%%%%%%%%%%%%%%%%%%%%%%%%%%%%%%%%%%%%%%%%%%%%%%%%%%%%
%%%%%%%%%%%%%%%%%%%%%%%%%%%%%%%%%%%%%%%%%%%%%%%%%%%%%%%%%%%%%%%%%%%%%%%%%%%%%%%%%%%%%%%%%%%%%%%%%%%%

\subsubsection{Proof of \EQUATIONS \eqref{eqn:lemma:concentration-ineq:ev:1}--\eqref{eqn:lemma:concentration-ineq:ev:4}}
\label{outline:concentration-ineq|pf|ev}

Next, the four expectations, \EQUATIONS \eqref{eqn:lemma:concentration-ineq:ev:1}--\eqref{eqn:lemma:concentration-ineq:ev:4}, in \LEMMA \ref{lemma:concentration-ineq:noisy} are verified.
Let
%|>>|:::::::::::::::::::::::::::::::::::::::::::::::::::::::::::::::::::::::::::::::::::::::::::|>>|
\(  \JCoords \subseteq [\n]  \)
%|<<|:::::::::::::::::::::::::::::::::::::::::::::::::::::::::::::::::::::::::::::::::::::::::::|<<|
be an arbitrary coordinate subset.
Note that it suffices to establish the results for \(  \hFn[\JCoords]  \) as those for \(  \hFn  \) immediately follow by taking
%|>>|:::::::::::::::::::::::::::::::::::::::::::::::::::::::::::::::::::::::::::::::::::::::::::|>>|
\(  \JCoords = [\n]  \).
%|<<|:::::::::::::::::::::::::::::::::::::::::::::::::::::::::::::::::::::::::::::::::::::::::::|<<|
%
%%%%%%%%%%%%%%%%%%%%%%%%%%%%%%%%%%%%%%%%%%%%%%%%%%%%%%%%%%%%%%%%%%%%%%%%%%%%%%%%%%%%%%%%%%%%%%%%%%%%
\paragraph{Verification of \EQUATION \eqref{eqn:lemma:concentration-ineq:ev:1}} %%%%%%%%%%%%%%%%%%%%
%%%%%%%%%%%%%%%%%%%%%%%%%%%%%%%%%%%%%%%%%%%%%%%%%%%%%%%%%%%%%%%%%%%%%%%%%%%%%%%%%%%%%%%%%%%%%%%%%%%%
%
Towards establishing the first expectation, \EQUATION \eqref{eqn:lemma:concentration-ineq:ev:1}, recall the orthogonal decomposition in \EQUATION \eqref{eqn:pf:lemma:concentration-ineq:2} in the proof of \EQUATION \eqref{eqn:lemma:concentration-ineq:pr:1}:
%|>>|===========================================================================================|>>|
\begin{align}
  \hFn[\JCoords]( \thetaStar, \thetaX )
  &=
  \left\langle
    \hFn[\JCoords]( \thetaStar, \thetaX ),
    \frac
    {\thetaStar-\thetaX}
    {\| \thetaStar-\thetaX \|_{2}}
  \right\rangle
  \frac
  {\thetaStar-\thetaX}
  {\| \thetaStar-\thetaX \|_{2}}
  +
  \left\langle
    \hFn[\JCoords]( \thetaStar, \thetaX ),
    \frac
    {\thetaStar+\thetaX}
    {\| \thetaStar+\thetaX \|_{2}}
  \right\rangle
  \frac
  {\thetaStar+\thetaX}
  {\| \thetaStar+\thetaX \|_{2}}
  \nonumber\\
  &\AlignSp\AlignSp+
  \gFn[\JCoords]( \thetaStar, \thetaX )
\label{eqn:pf:eqn:lemma:concentration-ineq:ev:1:1}
,\end{align}
%|<<|===========================================================================================|<<|
where \(  \gFn[\JCoords]( \thetaStar, \thetaX )  \) is given in \EQUATION \eqref{eqn:notations:gJ:def} or \eqref{eqn:pf:lemma:concentration-ineq:1:b}.
Hence, in expectation,
%|>>|===========================================================================================|>>|
\begin{align*}
  \E[ \hFn[\JCoords]( \thetaStar, \thetaX ) ]
  % \\
  % &\AlignIndent=
  % \E \left[
  %   \left\langle
  %     \hFn[\JCoords]( \thetaStar, \thetaX ),
  %     \frac
  %     {\thetaStar-\thetaX}
  %     {\| \thetaStar-\thetaX \|_{2}}
  %   \right\rangle
  %   \frac
  %   {\thetaStar-\thetaX}
  %   {\| \thetaStar-\thetaX \|_{2}}
  %   +
  %   \left\langle
  %     \hFn[\JCoords]( \thetaStar, \thetaX ),
  %     \frac
  %     {\thetaStar+\thetaX}
  %     {\| \thetaStar+\thetaX \|_{2}}
  %   \right\rangle
  %   \frac
  %   {\thetaStar+\thetaX}
  %   {\| \thetaStar+\thetaX \|_{2}}
  %   +
  %   \gFn[\JCoords]( \thetaStar, \thetaX )
  % \right]
  % \\
  % &\AlignIndent\dCmt{by \EQUATION \eqref{eqn:pf:eqn:lemma:concentration-ineq:ev:1:1}}
  % \\
  % &\AlignIndent=
  % \E \left[
  %   \left\langle
  %     \hFn[\JCoords]( \thetaStar, \thetaX ),
  %     \frac
  %     {\thetaStar-\thetaX}
  %     {\| \thetaStar-\thetaX \|_{2}}
  %   \right\rangle
  %   \frac
  %   {\thetaStar-\thetaX}
  %   {\| \thetaStar-\thetaX \|_{2}}
  % \right]
  % +
  % \E \left[
  %   \left\langle
  %     \hFn[\JCoords]( \thetaStar, \thetaX ),
  %     \frac
  %     {\thetaStar+\thetaX}
  %     {\| \thetaStar+\thetaX \|_{2}}
  %   \right\rangle
  %   \frac
  %   {\thetaStar+\thetaX}
  %   {\| \thetaStar+\thetaX \|_{2}}
  % \right]
  % \\
  % &\AlignSp\AlignSp+
  % \E \left[
  %   \gFn[\JCoords]( \thetaStar, \thetaX )
  % \right]
  % \\
  % &\AlignIndent\dCmt{by \EQUATION \eqref{eqn:pf:eqn:lemma:concentration-ineq:ev:1:1} and the linearity of expectation}
  &=
  \E \left[
    \left\langle
      \hFn[\JCoords]( \thetaStar, \thetaX ),
      \frac
      {\thetaStar-\thetaX}
      {\| \thetaStar-\thetaX \|_{2}}
    \right\rangle
  \right]
  \frac
  {\thetaStar-\thetaX}
  {\| \thetaStar-\thetaX \|_{2}}
  \\
  &\AlignSp\AlignSp+
  \E \left[
    \left\langle
      \hFn[\JCoords]( \thetaStar, \thetaX ),
      \frac
      {\thetaStar+\thetaX}
      {\| \thetaStar+\thetaX \|_{2}}
    \right\rangle
  \right]
  \frac
  {\thetaStar+\thetaX}
  {\| \thetaStar+\thetaX \|_{2}}
  +
  \E \left[
    \gFn[\JCoords]( \thetaStar, \thetaX )
  \right]
\TagEqn{\label{eqn:pf:eqn:lemma:concentration-ineq:ev:1:2}}
\end{align*}
%|<<|===========================================================================================|<<|
by \EQUATION \eqref{eqn:pf:eqn:lemma:concentration-ineq:ev:1:1} and the linearity of expectation.
Due to \LEMMA \ref{lemma:concentration-ineq:noiseless},
%|>>|===========================================================================================|>>|
\begin{gather}
  \label{eqn:pf:eqn:lemma:concentration-ineq:ev:1:3:1}
  \E \left[ \left\langle \hFn[\JCoords]( \thetaStar, \thetaX ), \frac{\thetaStar-\thetaX}{\| \thetaStar-\thetaX \|_{2}} \right\rangle \right]
  \frac{\thetaStar-\thetaX}{\| \thetaStar-\thetaX \|_{2}}
  =
  \| \thetaStar-\thetaX \|_{2}
  \frac{\thetaStar-\thetaX}{\| \thetaStar-\thetaX \|_{2}}
  =
  \thetaStar-\thetaX
  ,\\ \label{eqn:pf:eqn:lemma:concentration-ineq:ev:1:3:2}
  \E \left[ \left\langle \hFn[\JCoords]( \thetaStar, \thetaX ), \frac{\thetaStar+\thetaX}{\| \thetaStar+\thetaX \|_{2}} \right\rangle \right]
  \frac{\thetaStar+\thetaX}{\| \thetaStar+\thetaX \|_{2}}
  = 0 \cdot \frac{\thetaStar+\thetaX}{\| \thetaStar+\thetaX \|_{2}}
  =
  \Vec{0}
  ,\\ \label{eqn:pf:eqn:lemma:concentration-ineq:ev:1:3:3}
  \E[ \gFn[\JCoords]( \thetaStar, \thetaX ) ]
  = \Vec{0}
.\end{gather}
%|<<|===========================================================================================|<<|
Then, by \EQUATIONS \eqref{eqn:pf:eqn:lemma:concentration-ineq:ev:1:2}--\eqref{eqn:pf:eqn:lemma:concentration-ineq:ev:1:3:3},
%|>>|===========================================================================================|>>|
\begin{align*}
  &
  \E[ \hFn[\JCoords]( \thetaStar, \thetaX ) ]
  \\
  &\AlignIndent=
  \E \left[
    \left\langle
      \hFn[\JCoords]( \thetaStar, \thetaX ),
      \frac
      {\thetaStar-\thetaX}
      {\| \thetaStar-\thetaX \|_{2}}
    \right\rangle
  \right]
  \frac
  {\thetaStar-\thetaX}
  {\| \thetaStar-\thetaX \|_{2}}
  +
  \E \left[
    \left\langle
      \hFn[\JCoords]( \thetaStar, \thetaX ),
      \frac
      {\thetaStar+\thetaX}
      {\| \thetaStar+\thetaX \|_{2}}
    \right\rangle
  \right]
  \frac
  {\thetaStar+\thetaX}
  {\| \thetaStar+\thetaX \|_{2}}
  \\
  &\AlignSp\AlignSp
  +
  \E \left[
    \gFn[\JCoords]( \thetaStar, \thetaX )
  \right]
  \\
  &\AlignIndent\dCmt{by \EQUATION \eqref{eqn:pf:eqn:lemma:concentration-ineq:ev:1:2}}
  \\
  &\AlignIndent=
  \thetaStar-\thetaX
  +
  \Vec{0}
  +
  \Vec{0}
  \\
  &\AlignIndent\dCmt{by \EQUATIONS \eqref{eqn:pf:eqn:lemma:concentration-ineq:ev:1:3:1}--\eqref{eqn:pf:eqn:lemma:concentration-ineq:ev:1:3:3}}
  \\
  &\AlignIndent=
  \thetaStar-\thetaX
,\end{align*}
%|<<|===========================================================================================|<<|
as claimed.
%
%%%%%%%%%%%%%%%%%%%%%%%%%%%%%%%%%%%%%%%%%%%%%%%%%%%%%%%%%%%%%%%%%%%%%%%%%%%%%%%%%%%%%%%%%%%%%%%%%%%%
\paragraph{Verification of \EQUATION \eqref{eqn:lemma:concentration-ineq:ev:2}} %%%%%%%%%%%%%%%%%%%%
%%%%%%%%%%%%%%%%%%%%%%%%%%%%%%%%%%%%%%%%%%%%%%%%%%%%%%%%%%%%%%%%%%%%%%%%%%%%%%%%%%%%%%%%%%%%%%%%%%%%
%
To verify \EQUATION \eqref{eqn:lemma:concentration-ineq:ev:2}, we turn to the orthogonal decomposition in \EQUATION \eqref{eqn:pf:lemma:concentration-ineq:2} used to prove \EQUATION \eqref{eqn:lemma:concentration-ineq:pr:2}:
%|>>|===========================================================================================|>>|
\begin{align*}
  \hfFn[\JCoords]( \thetaStar, \thetaStar )
  =
  \langle \hfFn[\JCoords]( \thetaStar, \thetaStar ), \thetaStar \rangle
  \thetaStar
  +
  \gfFn[\JCoords]( \thetaStar, \thetaStar )
,\end{align*}
%|<<|===========================================================================================|<<|
where \(  \hfFn[\JCoords]( \thetaStar, \thetaStar )  \) is as stated in \EQUATION \eqref{eqn:notations:gfJ:def} or \eqref{eqn:pf:lemma:concentration-ineq:2:b}.
With this decomposition, due to the linearity of expectation,
%|>>|===========================================================================================|>>|
\begin{align*}
  \E[ \hfFn[\JCoords]( \thetaStar, \thetaStar ) ]
\XXX{
  &=
  \E [
    \langle \hfFn[\JCoords]( \thetaStar, \thetaStar ), \thetaStar \rangle
    \thetaStar
    +
    \gfFn[\JCoords]( \thetaStar, \thetaStar )
  ]
  \\
  &=
  \E [ \langle \hfFn[\JCoords]( \thetaStar, \thetaStar ), \thetaStar \rangle \thetaStar ]
  +
  \E[ \gfFn[\JCoords]( \thetaStar, \thetaStar ) ]
  \\
}
  &=
  \E [ \langle \hfFn[\JCoords]( \thetaStar, \thetaStar ), \thetaStar \rangle ]
  \thetaStar
  +
  \E[ \gfFn[\JCoords]( \thetaStar, \thetaStar ) ]
\TagEqn{\label{eqn:pf:eqn:lemma:concentration-ineq:ev:2:1}}
,\end{align*}
%|<<|===========================================================================================|<<|
where the last line uses the fact that \(  \thetaStar  \) is nonrandom.
Recall from \LEMMA \ref{lemma:concentration-ineq:noisy} that
%|>>|===========================================================================================|>>|
\begin{gather*}
  \E [ \langle \hfFn[\JCoords]( \thetaStar, \thetaStar ), \thetaStar \rangle ]
  =
  - \left( 1 - \sqrt{\frac{\pi}{2}} \gammaX \right)
  ,\\
  \E[ \gfFn[\JCoords]( \thetaStar, \thetaStar ) ]
  =
  \Vec{0}
.\end{gather*}
%|<<|===========================================================================================|<<|
Thus, continuing the above derivation in \eqref{eqn:pf:eqn:lemma:concentration-ineq:ev:2:1}, \EQUATION \eqref{eqn:lemma:concentration-ineq:ev:2} follows:
%|>>|===========================================================================================|>>|
\begin{align*}
  \E[ \hfFn[\JCoords]( \thetaStar, \thetaStar ) ]
\XXX{
  &=
  \E [ \langle \hfFn[\JCoords]( \thetaStar, \thetaStar ), \thetaStar \rangle ]
  \thetaStar
  +
  \E[ \gfFn[\JCoords]( \thetaStar, \thetaStar ) ]
  \\
  &=
  -\left( 1 - \sqrt{\frac{\pi}{2}} \gammaX \right) \thetaStar + \Vec{0}
  \\
}
  &=
  -\left( 1 - \sqrt{\frac{\pi}{2}} \gammaX \right) \thetaStar
.\end{align*}
%|<<|===========================================================================================|<<|
%
%%%%%%%%%%%%%%%%%%%%%%%%%%%%%%%%%%%%%%%%%%%%%%%%%%%%%%%%%%%%%%%%%%%%%%%%%%%%%%%%%%%%%%%%%%%%%%%%%%%%
\paragraph{Verification of \EQUATION \eqref{eqn:lemma:concentration-ineq:ev:4}} %%%%%%%%%%%%%%%%%%%%
%%%%%%%%%%%%%%%%%%%%%%%%%%%%%%%%%%%%%%%%%%%%%%%%%%%%%%%%%%%%%%%%%%%%%%%%%%%%%%%%%%%%%%%%%%%%%%%%%%%%
%
Observe:
%|>>|===========================================================================================|>>|
\begin{align*}
  &
  \hfFn[\JCoords]( \thetaStar, \thetaX )
  \\
  &=
  \ThresholdSet{\Supp( \thetaStar ) \cup \Supp( \thetaX ) \cup \JCoords}(
    \frac{\sqrt{2\pi}}{\m}
    \sep \CovM^{\T}
    \sep \frac{1}{2}
    ( \fFn( \CovM \thetaStar ) - \Sign( \CovM \thetaX ) )
  )
  \\
  &\dCmt{by the definition of \(  \hfFn[\JCoords]  \) in \EQUATION \eqref{eqn:notations:hfJ:def}}
  % \\
  % &=
  % \ThresholdSet{\Supp( \thetaStar ) \cup \Supp( \thetaX ) \cup \JCoords}(
  %   \frac{\sqrt{2\pi}}{\m}
  %   \sep \CovM^{\T}
  %   \sep \frac{1}{2}
  %   \bigl(
  %     ( \Sign( \CovM \thetaStar ) - \Sign( \CovM \thetaX ) )
  %     +
  %     ( \fFn( \CovM \thetaStar ) - \Sign( \CovM \thetaStar ) )
  %   \bigr)
  % )
  % \\
  % &\dCmt{the \(  \pm \Sign( \CovM \thetaStar )  \) terms cancel}
  \\
  &=
  \ThresholdSet{\Supp( \thetaStar ) \cup \Supp( \thetaX ) \cup \JCoords}(
    \frac{\sqrt{2\pi}}{\m}
    \sep \CovM^{\T}
    \sep \frac{1}{2}
    ( \Sign( \CovM \thetaStar ) - \Sign( \CovM \thetaX ) )
  )
  \\
  &\AlignSp+
  \ThresholdSet{\Supp( \thetaStar ) \cup \Supp( \thetaX ) \cup \JCoords}(
    \frac{\sqrt{2\pi}}{\m}
    \sep \CovM^{\T}
    \sep \frac{1}{2}
    ( \fFn( \CovM \thetaStar ) - \Sign( \CovM \thetaStar ) )
  )
  \\
  &\dCmt{the subset thresholding operation is a linear transformation (\see \SECTIONREF \ref{outline:notations})}
  \\
  &=
  \hFn[\JCoords]( \thetaStar, \thetaX )
  +
  \hfFn[\Supp( \thetaX ) \cup \JCoords]( \thetaStar, \thetaStar )
  . \TagEqn{\label{eqn:pf:lemma:concentration-ineq:6}} \\
  &\dCmt{by the definitions of \(  \hFn[\JCoords]  \) and \(  \hfFn[\JCoords]  \) in \EQUATIONS \eqref{eqn:notations:hJ:def} and \eqref{eqn:notations:hfJ:def}, respectively}
\end{align*}
%|<<|===========================================================================================|<<|
Thus,
%|>>|===========================================================================================|>>|
\begin{align}
  \E[ \hfFn[\JCoords]( \thetaStar, \thetaX ) ]
  &=
  \E[
    \hFn[\JCoords]( \thetaStar, \thetaX )
    +
    \hfFn[\Supp( \thetaX ) \cup \JCoords]( \thetaStar, \thetaStar )
  ]
  =
  \E[ \hFn[\JCoords]( \thetaStar, \thetaX ) ]
  +
  \E[ \hfFn[\Supp( \thetaX ) \cup \JCoords]( \thetaStar, \thetaStar ) ]
\label{eqn:pf:lemma:concentration-ineq:3}
,\end{align}
%|<<|===========================================================================================|<<|
where the first equality applies \EQUATION \eqref{eqn:pf:lemma:concentration-ineq:6} and the second equality follows from the linearity of expectation.
By \EQUATIONS \eqref{eqn:lemma:concentration-ineq:ev:1} and \eqref{eqn:lemma:concentration-ineq:ev:2}, respectively,
%|>>|===========================================================================================|>>|
\begin{gather}
  \label{eqn:pf:lemma:concentration-ineq:4}
  \E[ \hFn[\JCoords]( \thetaStar, \thetaX ) ]
  =
  \thetaStar-\thetaX
  ,\\ \label{eqn:pf:lemma:concentration-ineq:5}
  \E[ \hfFn[\Supp( \thetaX ) \cup \JCoords]( \thetaStar, \thetaStar ) ]
  =
  -\left( 1 - \sqrt{\frac{\pi}{2}} \gammaX \right) \thetaStar
,\end{gather}
%|<<|===========================================================================================|<<|
and therefore,
%|>>|===========================================================================================|>>|
\begin{align*}
  \E[ \thetaX + \hfFn[\JCoords]( \thetaStar, \thetaX ) ]
  % &=
  % \thetaX + \E[ \hfFn[\JCoords]( \thetaStar, \thetaX ) ]
  % \\
  % &\dCmt{\(  \thetaX  \) is nonrandom}
  % \\
  &=
  \thetaX
  +
  \E[ \hFn[\JCoords]( \thetaStar, \thetaX ) ]
  +
  \E[ \hfFn[\Supp( \thetaX ) \cup \JCoords]( \thetaStar, \thetaStar ) ]
  \\
  &\dCmt{by \EQUATION \eqref{eqn:pf:lemma:concentration-ineq:3}}
  \\
  &=
  \thetaX
  +
  \thetaStar - \thetaX
  -
  \left( 1 - \sqrt{\frac{\pi}{2}} \gammaX \right) \thetaStar
  \\
  &\dCmt{by \EQUATIONS \eqref{eqn:pf:lemma:concentration-ineq:4} and \eqref{eqn:pf:lemma:concentration-ineq:5}}
\XXX{
  \\
  &=
  \thetaStar - \left( 1 - \sqrt{\frac{\pi}{2}} \gammaX \right) \thetaStar
  \\
  &\dCmt{by canceling terms}
}
  \\
  &=
  \sqrt{\frac{\pi}{2}} \gammaX \thetaStar
  \TagEqn{\label{eqn:pf:lemma:concentration-ineq:7}}
  % .\\
  % &\dCmt{by distributivity}
.\end{align*}
%|<<|===========================================================================================|<<|
Then,
%|>>|===========================================================================================|>>|
\begin{gather*}
  \| \E \left[ \thetaX + \hfFn[\JCoords]( \thetaStar, \thetaX ) \right] \|_{2}
  % =
  % \left\| \sqrt{\frac{\pi}{2}} \gammaX \thetaStar \right\|_{2}
  =
  \sqrt{\frac{\pi}{2}} \gammaX \| \thetaStar \|_{2}
  =
  \sqrt{\frac{\pi}{2}} \gammaX
,\end{gather*}
%|<<|===========================================================================================|<<|
where the first equality applies \EQUATION \eqref{eqn:pf:lemma:concentration-ineq:7}, the second follows from the homogeneity of the (\(  \lnorm{2}  \)-)norm, and the third equality recalls that \(  \thetaStar  \) has unit \(  \lnorm{2}  \)-norm.
This completes the proof of \LEMMA \ref{lemma:concentration-ineq}.
\end{proof}
%|<<|~~~~~~~~~~~~~~~~~~~~~~~~~~~~~~~~~~~~~~~~~~~~~~~~~~~~~~~~~~~~~~~~~~~~~~~~~~~~~~~~~~~~~~~~~~~|<<|
%|<<|~~~~~~~~~~~~~~~~~~~~~~~~~~~~~~~~~~~~~~~~~~~~~~~~~~~~~~~~~~~~~~~~~~~~~~~~~~~~~~~~~~~~~~~~~~~|<<|
%|<<|~~~~~~~~~~~~~~~~~~~~~~~~~~~~~~~~~~~~~~~~~~~~~~~~~~~~~~~~~~~~~~~~~~~~~~~~~~~~~~~~~~~~~~~~~~~|<<|

%%%%%%%%%%%%%%%%%%%%%%%%%%%%%%%%%%%%%%%%%%%%%%%%%%%%%%%%%%%%%%%%%%%%%%%%%%%%%%%%%%%%%%%%%%%%%%%%%%%%
%%%%%%%%%%%%%%%%%%%%%%%%%%%%%%%%%%%%%%%%%%%%%%%%%%%%%%%%%%%%%%%%%%%%%%%%%%%%%%%%%%%%%%%%%%%%%%%%%%%%
%%%%%%%%%%%%%%%%%%%%%%%%%%%%%%%%%%%%%%%%%%%%%%%%%%%%%%%%%%%%%%%%%%%%%%%%%%%%%%%%%%%%%%%%%%%%%%%%%%%%
%%%%%%%%%%%%%%%%%%%%%%%%%%%%%%%%%%%%%%%%%%%%%%%%%%%%%%%%%%%%%%%%%%%%%%%%%%%%%%%%%%%%%%%%%%%%%%%%%%%%
%%%%%%%%%%%%%%%%%%%%%%%%%%%%%%%%%%%%%%%%%%%%%%%%%%%%%%%%%%%%%%%%%%%%%%%%%%%%%%%%%%%%%%%%%%%%%%%%%%%%

\subsection{Proof of \LEMMA \ref{lemma:concentration-ineq:noiseless}}
\label{outline:concentration-ineq|pf-noiseless}

%|>>|~~~~~~~~~~~~~~~~~~~~~~~~~~~~~~~~~~~~~~~~~~~~~~~~~~~~~~~~~~~~~~~~~~~~~~~~~~~~~~~~~~~~~~~~~~~|>>|
%|>>|~~~~~~~~~~~~~~~~~~~~~~~~~~~~~~~~~~~~~~~~~~~~~~~~~~~~~~~~~~~~~~~~~~~~~~~~~~~~~~~~~~~~~~~~~~~|>>|
%|>>|~~~~~~~~~~~~~~~~~~~~~~~~~~~~~~~~~~~~~~~~~~~~~~~~~~~~~~~~~~~~~~~~~~~~~~~~~~~~~~~~~~~~~~~~~~~|>>|
\begin{proof}
{\LEMMA \ref{lemma:concentration-ineq:noiseless}}
%
\mostlycheckoff%
%
The proof of the expectations, \EQUATIONS \eqref{eqn:lemma:concentration-ineq:noiseless:ev:1}--\eqref{eqn:lemma:concentration-ineq:noiseless:ev:3}, in \LEMMA \ref{lemma:concentration-ineq:noiseless} are presented in \SECTION \ref{outline:concentration-ineq|pf-noiseless|ev}, while the concentration inequalities in \EQUATIONS \eqref{eqn:lemma:concentration-ineq:noiseless:pr:1}--\eqref{eqn:lemma:concentration-ineq:noiseless:pr:3} are proved in \SECTION \ref{outline:concentration-ineq|pf-noiseless|pr}.

%%%%%%%%%%%%%%%%%%%%%%%%%%%%%%%%%%%%%%%%%%%%%%%%%%%%%%%%%%%%%%%%%%%%%%%%%%%%%%%%%%%%%%%%%%%%%%%%%%%%
%%%%%%%%%%%%%%%%%%%%%%%%%%%%%%%%%%%%%%%%%%%%%%%%%%%%%%%%%%%%%%%%%%%%%%%%%%%%%%%%%%%%%%%%%%%%%%%%%%%%
%%%%%%%%%%%%%%%%%%%%%%%%%%%%%%%%%%%%%%%%%%%%%%%%%%%%%%%%%%%%%%%%%%%%%%%%%%%%%%%%%%%%%%%%%%%%%%%%%%%%

\subsubsection{Proof the Expectations, \EQUATIONS \eqref{eqn:lemma:concentration-ineq:noiseless:ev:1}--\eqref{eqn:lemma:concentration-ineq:noiseless:ev:3}}
\label{outline:concentration-ineq|pf-noiseless|ev}

The expectations, \EQUATIONS \eqref{eqn:lemma:concentration-ineq:noiseless:ev:1}--\eqref{eqn:lemma:concentration-ineq:noiseless:ev:3}, follow largely from work already done by \cite{matsumoto2022binary}, which is summarized below as \LEMMA \ref{lemma:pf:lemma:concentration-ineq:noiseless:ev:cond-ev}.
%
%|>>|*******************************************************************************************|>>|
%|>>|*******************************************************************************************|>>|
%|>>|*******************************************************************************************|>>|
\begin{lemma}[{\dueto \cite[{\APPENDIX B}]{matsumoto2022binary}}]
\label{lemma:pf:lemma:concentration-ineq:noiseless:ev:cond-ev}
%
Fix
%|>>|:::::::::::::::::::::::::::::::::::::::::::::::::::::::::::::::::::::::::::::::::::::::::::|>>|
\(  \thetaStar, \thetaX \in \ParamSpace  \)
%|<<|:::::::::::::::::::::::::::::::::::::::::::::::::::::::::::::::::::::::::::::::::::::::::::|<<|
and
%|>>|:::::::::::::::::::::::::::::::::::::::::::::::::::::::::::::::::::::::::::::::::::::::::::|>>|
\(  \lX \in \ZeroTo{\m}  \).
%|<<|:::::::::::::::::::::::::::::::::::::::::::::::::::::::::::::::::::::::::::::::::::::::::::|<<|
Let
%|>>|:::::::::::::::::::::::::::::::::::::::::::::::::::::::::::::::::::::::::::::::::::::::::::|>>|
\(  \LRV \defeq \| \I( \Sign( \CovM \thetaStar ) \neq \Sign( \CovM \thetaX ) ) \|_{0}  \).
%|<<|:::::::::::::::::::::::::::::::::::::::::::::::::::::::::::::::::::::::::::::::::::::::::::|<<|
Then,
%|>>|===========================================================================================|>>|
\begin{gather}
  \label{eqn:lemma:pf:lemma:concentration-ineq:noiseless:ev:cond-ev:1}
  \E_{\CovM \Mid| \LRV} \left[ \left\langle \hFn[\JCoords]( \thetaStar, \thetaX ), \frac{\thetaStar-\thetaX}{\| \thetaStar-\thetaX \|_{2}} \right\rangle \middle| \LRV=\lX \right]
  =
  \frac{\pi \lX \| \thetaStar-\thetaX \|_{2}}{\m \ADIST}
  ,\\ \label{eqn:lemma:pf:lemma:concentration-ineq:noiseless:ev:cond-ev:2}
  \E_{\CovM \Mid| \LRV} \left[ \left\langle \hFn[\JCoords]( \thetaStar, \thetaX ), \frac{\thetaStar+\thetaX}{\| \thetaStar+\thetaX \|_{2}} \right\rangle \middle| \LRV=\lX \right]
  = 0
  ,\\ \label{eqn:lemma:pf:lemma:concentration-ineq:noiseless:ev:cond-ev:3}
  \E_{\CovM \Mid| \LRV} [ \gFn[\JCoords]( \thetaStar, \thetaX ) \Mid| \LRV=\lX ]
  = \Vec{0}
.\end{gather}
%|<<|===========================================================================================|<<|
\end{lemma}
%|<<|*******************************************************************************************|<<|
%|<<|*******************************************************************************************|<<|
%|<<|*******************************************************************************************|<<|
%
Taking
%|>>|:::::::::::::::::::::::::::::::::::::::::::::::::::::::::::::::::::::::::::::::::::::::::::|>>|
\(  \thetaStar, \thetaX \in \ParamSpace  \)
%|<<|:::::::::::::::::::::::::::::::::::::::::::::::::::::::::::::::::::::::::::::::::::::::::::|<<|
arbitrarily, via the law of total expectation, \EQUATIONS \eqref{eqn:lemma:concentration-ineq:noiseless:ev:2} and \eqref{eqn:lemma:concentration-ineq:noiseless:ev:3} follow from \EQUATIONS \eqref{eqn:lemma:pf:lemma:concentration-ineq:noiseless:ev:cond-ev:2} and \eqref{eqn:lemma:pf:lemma:concentration-ineq:noiseless:ev:cond-ev:3}, respectively:
%|>>|===========================================================================================|>>|
\begin{gather}
  \label{eqn:pf:lemma:concentration-ineq:noiseless:1:1}
  \E_{\CovM} \left[ \left\langle \hFn[\JCoords]( \thetaStar, \thetaX ), \frac{\thetaStar+\thetaX}{\| \thetaStar+\thetaX \|_{2}} \right\rangle \right]
  =
  \E_{\LRV} \left[ \E_{\CovM \Mid| \LRV} \left[ \left\langle \hFn[\JCoords]( \thetaStar, \thetaX ), \frac{\thetaStar+\thetaX}{\| \thetaStar+\thetaX \|_{2}} \right\rangle \middle| \LRV \right] \right]
  =
  \E_{\LRV} [ 0 ]
  =
  0
  ,\\ \label{eqn:pf:lemma:concentration-ineq:noiseless:1:2}
  \E_{\CovM} [ \gFn[\JCoords]( \thetaStar, \thetaX ) ]
  =
  \E_{\LRV} \left[ \E_{\CovM \Mid| \LRV} \left[ \gFn[\JCoords]( \thetaStar, \thetaX ) \middle| \LRV \right] \right]
  =
  \E_{\LRV} [ \Vec{0} ]
  =
  \Vec{0}
.\end{gather}
%|<<|===========================================================================================|<<|
Note that because \(  \JCoords \subseteq [\n]  \) is arbitrary, \EQUATIONS \eqref{eqn:pf:lemma:concentration-ineq:noiseless:1:1} and \eqref{eqn:pf:lemma:concentration-ineq:noiseless:1:2} further imply that
%|>>|===========================================================================================|>>|
\begin{gather*}
  \E_{\CovM} \left[ \left\langle \hFn( \thetaStar, \thetaX ), \frac{\thetaStar+\thetaX}{\| \thetaStar+\thetaX \|_{2}} \right\rangle \right]
  =
  0
  ,\\
  \E_{\CovM} [ \gFn( \thetaStar, \thetaX ) ]
  =
  \Vec{0}
\end{gather*}
%|<<|===========================================================================================|<<|
by taking \(  \JCoords = [\n]  \).
%
%%%%%%%%%%%%%%%%%%%%%%%%%%%%%%%%%%%%%%%%%%%%%%%%%%%%%%%%%%%%%%%%%%%%%%%%%%%%%%%%%%%%%%%%%%%%%%%%%%%%
\par %%%%%%%%%%%%%%%%%%%%%%%%%%%%%%%%%%%%%%%%%%%%%%%%%%%%%%%%%%%%%%%%%%%%%%%%%%%%%%%%%%%%%%%%%%%%%%%
%%%%%%%%%%%%%%%%%%%%%%%%%%%%%%%%%%%%%%%%%%%%%%%%%%%%%%%%%%%%%%%%%%%%%%%%%%%%%%%%%%%%%%%%%%%%%%%%%%%%
%
Proceeding to \EQUATION \eqref{eqn:lemma:concentration-ineq:noiseless:ev:1}, define the random variable
%|>>|:::::::::::::::::::::::::::::::::::::::::::::::::::::::::::::::::::::::::::::::::::::::::::|>>|
\(  \LRV \defeq \| \I( \Sign( \CovM \thetaStar ) \neq \Sign( \CovM \thetaX ) ) \|_{0}  \)
%|<<|:::::::::::::::::::::::::::::::::::::::::::::::::::::::::::::::::::::::::::::::::::::::::::|<<|
as in \LEMMA \ref{lemma:pf:lemma:concentration-ineq:noiseless:ev:cond-ev}.
To derive
%\EQUATION \eqref{eqn:lemma:concentration-ineq:noiseless:ev:1},
the result,
first note that \(  \LRV  \) follows a binomial distribution:
%|>>|:::::::::::::::::::::::::::::::::::::::::::::::::::::::::::::::::::::::::::::::::::::::::::|>>|
\(  \LRV \sim \Binomial( \m, \frac{1}{\pi} \ADIST )  \),
%|<<|:::::::::::::::::::::::::::::::::::::::::::::::::::::::::::::::::::::::::::::::::::::::::::|<<|
where the characterization of this random variable, \(  \LRV  \), is folklore (\seeeg \cite{charikar2002similarity}).
Then, observe:
%|>>|===========================================================================================|>>|
\begin{align*}
  &
  \E \left[ \left\langle \hFn[\JCoords]( \thetaStar, \thetaX ), \frac{\thetaStar-\thetaX}{\| \thetaStar-\thetaX \|_{2}} \right\rangle \right]
  \\
  &\AlignIndent=
  \E_{\LRV} \left[ \E_{\CovM \Mid| \LRV} \left[ \left\langle \hFn[\JCoords]( \thetaStar, \thetaX ), \frac{\thetaStar-\thetaX}{\| \thetaStar-\thetaX \|_{2}} \right\rangle \middle| \LRV \right] \right]
  \\
  &\AlignIndent\dCmt{by the law of total expectation}
  \\
  &\AlignIndent=
  \sum_{\lX=0}^{\m}
  \binom{\m}{\lX}
  \left( \frac{\ADIST}{\pi} \right)^{\lX}
  \left( 1-\frac{\ADIST}{\pi} \right)^{\m-\lX}
  \E_{\CovM \Mid| \LRV} \left[ \left\langle \hFn[\JCoords]( \thetaStar, \thetaX ), \frac{\thetaStar-\thetaX}{\| \thetaStar-\thetaX \|_{2}} \right\rangle \middle| \LRV=\lX \right]
  \\
  &\AlignIndent\dCmt{by the law of the lazy statistician and the mass function of \(  {\textstyle \LRV \sim \Binomial( \m, \frac{1}{\pi} \ADIST )}    \)}
  % \\
  % &\AlignIndent=
  % \sum_{\lX=0}^{\m}
  % \binom{\m}{\lX}
  % \left( \frac{1}{\pi} \ADIST \right)^{\lX}
  % \left( 1-\frac{1}{\pi} \ADIST \right)^{\m-\lX}
  % \frac{\pi \lX \| \thetaStar-\thetaX \|_{2}}{\m \ADIST}
  % \\
  % &\AlignIndent\dCmt{by \EQUATION \eqref{eqn:lemma:pf:lemma:concentration-ineq:noiseless:ev:cond-ev:1}}
  \\
  &\AlignIndent=
  \frac{\pi \| \thetaStar-\thetaX \|_{2}}{\m \ADIST}
  \sum_{\lX=0}^{\m}
  \binom{\m}{\lX}
  \left( \frac{\ADIST}{\pi} \right)^{\lX}
  \left( 1-\frac{\ADIST}{\pi} \right)^{\m-\lX}
  \lX
  \\
  &\AlignIndent\dCmt{by \EQUATION \eqref{eqn:lemma:pf:lemma:concentration-ineq:noiseless:ev:cond-ev:1}}
  % &\AlignIndent\dCmt{\(  {\textstyle \frac{\pi \| \thetaStar-\thetaX \|_{2}}{\m \ADIST}}  \) does not depend on the index of summation, \(  \lX  \)}
  \\
  &\AlignIndent=
  \frac{\pi \| \thetaStar-\thetaX \|_{2}}{\m \ADIST}
  \E[ \LRV ]
  \\
  &\AlignIndent\dCmt{by the definition of expectation and the mass function of a binomial random variable}
  % \\
  % &\AlignIndent=
  % \frac{\pi \| \thetaStar-\thetaX \|_{2}}{\m \ADIST}
  % \cdot
  % \frac{1}{\pi} \m \ADIST
  % \\
  % &\AlignIndent\dCmt{due to the well-known formula for the expectation of a binomial random variable}
  \\
  &\AlignIndent=
  \| \thetaStar-\thetaX \|_{2}
  .\\
  &\AlignIndent\dCmt{by the expectation of a binomial random variable, \(  \E[ \LRV ] = \tfrac{1}{\pi} \m \ADIST  \)}
\end{align*}
%|<<|===========================================================================================|<<|
Again, because \(  \JCoords \subseteq [\n]  \) is arbitrary, it directly follows from the above derivation that
%|>>|===========================================================================================|>>|
\begin{gather*}
  \E \left[ \left\langle \hFn( \thetaStar, \thetaX ), \frac{\thetaStar-\thetaX}{\| \thetaStar-\thetaX \|_{2}} \right\rangle \right]
  =
  \| \thetaStar-\thetaX \|_{2}
.\end{gather*}
%|<<|===========================================================================================|<<|
This verifies \EQUATION \eqref{eqn:lemma:concentration-ineq:noiseless:ev:1}.

%%%%%%%%%%%%%%%%%%%%%%%%%%%%%%%%%%%%%%%%%%%%%%%%%%%%%%%%%%%%%%%%%%%%%%%%%%%%%%%%%%%%%%%%%%%%%%%%%%%%
%%%%%%%%%%%%%%%%%%%%%%%%%%%%%%%%%%%%%%%%%%%%%%%%%%%%%%%%%%%%%%%%%%%%%%%%%%%%%%%%%%%%%%%%%%%%%%%%%%%%
%%%%%%%%%%%%%%%%%%%%%%%%%%%%%%%%%%%%%%%%%%%%%%%%%%%%%%%%%%%%%%%%%%%%%%%%%%%%%%%%%%%%%%%%%%%%%%%%%%%%

\subsubsection{Proof of the Concentration Inequalities, \EQUATIONS \eqref{eqn:lemma:concentration-ineq:noiseless:pr:1}--\eqref{eqn:lemma:concentration-ineq:noiseless:pr:3}}
\label{outline:concentration-ineq|pf-noiseless|pr}

Next, we turn our attention to \EQUATIONS \eqref{eqn:lemma:concentration-ineq:noiseless:pr:1}--\eqref{eqn:lemma:concentration-ineq:noiseless:pr:3}.
We begin with some preliminary analysis that will facilitate the derivations of these equations.
%Fix
%%|>>|:::::::::::::::::::::::::::::::::::::::::::::::::::::::::::::::::::::::::::::::::::::::::::|>>|
%\(  \JCoords \subseteq [\n]  \), \(  | \JCoords | \leq \kJ  \),
%%|<<|:::::::::::::::::::::::::::::::::::::::::::::::::::::::::::::::::::::::::::::::::::::::::::|<<|
%arbitrarily.
Initially, fix
%|>>|:::::::::::::::::::::::::::::::::::::::::::::::::::::::::::::::::::::::::::::::::::::::::::|>>|
%\(  \thetaX \in \ParamCoverX  \) and \(  \JCoords \in \JS  \)
\(  \thetaX \in \ParamCoverX  \) and \(  \JCoords \in \JS  \)
%|<<|:::::::::::::::::::::::::::::::::::::::::::::::::::::::::::::::::::::::::::::::::::::::::::|<<|
arbitrarily, where later \(  \thetaX  \) and \(  \JCoords  \) will be varied over the entire sets \(  \ParamCoverX  \) and \(  \JS  \), respectively, in union bounds.
Write
%|>>|:::::::::::::::::::::::::::::::::::::::::::::::::::::::::::::::::::::::::::::::::::::::::::|>>|
\(  \CovVX\VIx{\iIx} \defeq
    \ThresholdSet{\Supp( \thetaStar ) \cup \Supp( \thetaX ) \cup \JCoords}( \CovV\VIx{\iIx} )
    \in \R^{\n}  \),
%|<<|:::::::::::::::::::::::::::::::::::::::::::::::::::::::::::::::::::::::::::::::::::::::::::|<<|
\(  \iIx \in [\m]  \).
The definition of \(  \frac{1}{\sqrt{2\pi}} \hFn[\JCoords]( \thetaStar, \thetaX )  \) in \EQUATION \eqref{eqn:notations:hJ:def} can be rewritten as follows:
%|>>|===========================================================================================|>>|
\begin{align*}
  \frac{\hFn[\JCoords]( \thetaStar, \thetaX )}{\sqrt{2\pi}}
  &=
  \ThresholdSet{\Supp( \thetaStar ) \cup \Supp( \thetaX ) \cup \JCoords} \left(
    \frac{1}{\m}
    \sum_{\iIx=1}^{\m}
    \CovV\VIx{\iIx}
   \sep
    \frac{1}{2} \left( \Sign( \langle \CovV, \thetaStar \rangle ) - \Sign( \langle \CovV, \thetaX \rangle ) \right)
  \right)
  \\
  &\dCmt{by the definition of \(  \hFn[\JCoords]  \) in \EQUATION \eqref{eqn:notations:hJ:def}}
 %  \\
 %  &=
 %  \frac{1}{\m}
 %  \sum_{\iIx=1}^{\m}
 %  \ThresholdSet{\Supp( \thetaStar ) \cup \Supp( \thetaX ) \cup \JCoords} \left(
 %    \CovV\VIx{\iIx}
 %  \right)
 % \sep
 %  \frac{1}{2} \left( \Sign( \langle \CovV, \thetaStar \rangle ) - \Sign( \langle \CovV, \thetaX \rangle ) \right)
 %  \\
 %  &\dCmt{by the linearity of the map \(  \ThresholdSet{\Supp( \thetaStar ) \cup \Supp( \thetaX ) \cup \JCoords}  \) (\see \SECTIONREF \ref{outline:notations})}
  \\
  &=
  \frac{1}{\m}
  \sum_{\iIx=1}^{\m}
  \CovVX\VIx{\iIx}
 \sep
  \frac{1}{2} \left( \Sign( \langle \CovV\VIx{\iIx}, \thetaStar \rangle ) - \Sign( \langle \CovV\VIx{\iIx}, \thetaX \rangle ) \right)
  \\
  &\dCmt{by the linearity of the map \(  \ThresholdSet{\Supp( \thetaStar ) \cup \Supp( \thetaX ) \cup \JCoords}  \)}
  \\
  &\dCmtx{(\see \SECTIONREF \ref{outline:notations}), and by the definition of \(  \CovVX\VIx{\iIx}  \), \(  \iIx \in [\m]  \)}
  \\
  &=
  \frac{1}{\m}
  \sum_{\iIx=1}^{\m}
  \CovVX\VIx{\iIx}
 \sep
  \frac{1}{2} \left( \Sign( \langle \CovVX\VIx{\iIx}, \thetaStar \rangle ) - \Sign( \langle \CovVX\VIx{\iIx}, \thetaX \rangle ) \right)
  \\
  &\dCmt{\(  \Supp( \thetaStar ), \Supp( \thetaX ) \subseteq \Supp( \thetaStar ) \cup \Supp( \thetaX ) \cup \JCoords  \)}
  \\
  &=
  \frac{1}{\m}
  \sum_{\iIx=1}^{\m}
  \CovVX\VIx{\iIx}
 \sep
  \Sign( \langle \CovVX\VIx{\iIx}, \thetaStar \rangle )
 \sep
  \I( \Sign( \langle \CovVX\VIx{\iIx}, \thetaStar \rangle ) \neq \Sign( \langle \CovVX\VIx{\iIx}, \thetaX \rangle ) )
\TagEqn{\label{eqn:pf:eqn:lemma:concentration-ineq:noiseless:pr:5}}
  .\\
  &\dCmt{\see justification below}
\end{align*}
%|<<|===========================================================================================|<<|
The last line can be verified by checking the value taken by
%|>>|:::::::::::::::::::::::::::::::::::::::::::::::::::::::::::::::::::::::::::::::::::::::::::|>>|
\(  \frac{1}{2} ( \Sign( u ) - \Sign( v ) )  \)
%|<<|:::::::::::::::::::::::::::::::::::::::::::::::::::::::::::::::::::::::::::::::::::::::::::|<<|
at \(  u, v \in \R  \) for each possible pair values of
%|>>|:::::::::::::::::::::::::::::::::::::::::::::::::::::::::::::::::::::::::::::::::::::::::::|>>|
\(  \Sign( u ), \Sign( v ) \in \{ -1,1 \}  \):
%|<<|:::::::::::::::::::::::::::::::::::::::::::::::::::::::::::::::::::::::::::::::::::::::::::|<<|
%|>>|===========================================================================================|>>|
\begin{align*}
  \frac{1}{2} ( \Sign( u ) - \Sign( v ) )
  % &=
  % \begin{cases}
  % \frac{1}{2} ( 1 - 1 ) ,&\cIf \Sign( u )=\Sign( v )=1,\\
  % \frac{1}{2} ( -1 + 1 ) ,&\cIf \Sign( u )=\Sign( v )=-1 ,\\
  % \frac{1}{2} ( 1 + 1 ) ,&\cIf \Sign( u ) = 1 \neq -1 = \Sign( v ) ,\\
  % \frac{1}{2} ( -1 - 1 ) ,&\cIf \Sign( u ) = -1 \neq 1 = \Sign( v ) ,\\
  % \end{cases}
  % \\
  &=
  \begin{cases}
  \+ 0 ,&\cIf \Sign( u )=\Sign( v )=1,\\
  \+ 0 ,&\cIf \Sign( u )=\Sign( v )=-1 ,\\
  \+ 1 ,&\cIf \Sign( u ) = 1 \neq 1 = \Sign( v ) ,\\
  -1 ,&\cIf \Sign( u ) = -1 \neq 1 = \Sign( v ) ,\\
  \end{cases}
  \\
  &=
  \Sign( u ) \sep \I( \Sign( u ) \neq \Sign( v ) )
.\end{align*}
%|<<|===========================================================================================|<<|
Therefore,
%|>>|===========================================================================================|>>|
\begin{align*}
  &
  \left\langle \frac{1}{\sqrt{2\pi}} \hFn[\JCoords]( \thetaStar, \thetaX ), \frac{\thetaStar-\thetaX}{\| \thetaStar-\thetaX \|_{2}} \right\rangle
  \\
  &\AlignIndent=
  \frac{1}{\m}
  \sum_{\iIx=1}^{\m}
  \left\langle \CovVX\VIx{\iIx}, \frac{\thetaStar-\thetaX}{\| \thetaStar-\thetaX \|_{2}} \right\rangle
 \sep
  \Sign( \langle \CovVX\VIx{\iIx}, \thetaStar \rangle )
 \sep
  \I( \Sign( \langle \CovVX\VIx{\iIx}, \thetaStar \rangle ) \neq \Sign( \langle \CovVX\VIx{\iIx}, \thetaX \rangle ) )
  \\
  &\AlignIndent\dCmt{by \EQUATION \eqref{eqn:pf:eqn:lemma:concentration-ineq:noiseless:pr:5} and the linearity of inner products}
  \\ \TagEqn{\label{eqn:pf:eqn:lemma:concentration-ineq:noiseless:pr:4}}
  &\AlignIndent=
  \frac{1}{\m}
  \sum_{\iIx=1}^{\m}
  \left\langle \CovVX\VIx{\iIx}, \frac{\thetaStar-\thetaX}{\| \thetaStar-\thetaX \|_{2}} \right\rangle
 \sep
  \Sign( \left\langle \CovVX\VIx{\iIx}, \frac{\thetaStar-\thetaX}{\| \thetaStar-\thetaX \|_{2}} \right\rangle )
 \sep
  \I( \Sign( \langle \CovVX\VIx{\iIx}, \thetaStar \rangle ) \neq \Sign( \langle \CovVX\VIx{\iIx}, \thetaX \rangle ) )
  \\
  &\AlignIndent\dCmt{\see justification below}
  \\
  &\AlignIndent=
  \frac{1}{\m}
  \sum_{\iIx=1}^{\m}
  \left| \left\langle \CovVX\VIx{\iIx}, \frac{\thetaStar-\thetaX}{\| \thetaStar-\thetaX \|_{2}} \right\rangle \right|
 \sep
  \I( \Sign( \langle \CovVX\VIx{\iIx}, \thetaStar \rangle ) \neq \Sign( \langle \CovVX\VIx{\iIx}, \thetaX \rangle ) )
\TagEqn{\label{eqn:pf:eqn:lemma:concentration-ineq:noiseless:pr:6}}
  ,\\
  &\AlignIndent\dCmt{\(  u \sep \Sign( u ) = | u |  \) for any \(  u \in \R  \)}
\end{align*}
%|<<|===========================================================================================|<<|
where the second to last equality, \eqref{eqn:pf:eqn:lemma:concentration-ineq:noiseless:pr:4}, follows from the observation that either the indicator term takes the value \(  0  \), or otherwise, if
%|>>|:::::::::::::::::::::::::::::::::::::::::::::::::::::::::::::::::::::::::::::::::::::::::::|>>|
\(  \Sign( \langle \CovVX\VIx{\iIx}, \thetaStar \rangle ) \neq \Sign( \langle \CovVX\VIx{\iIx}, \thetaX \rangle )  \),
%|<<|:::::::::::::::::::::::::::::::::::::::::::::::::::::::::::::::::::::::::::::::::::::::::::|<<|
then
%|>>|:::::::::::::::::::::::::::::::::::::::::::::::::::::::::::::::::::::::::::::::::::::::::::|>>|
\(  \Sign( \langle \CovVX\VIx{\iIx}, \thetaX \rangle ) = -\Sign( \langle \CovVX\VIx{\iIx}, \thetaStar \rangle )  \),
%|<<|:::::::::::::::::::::::::::::::::::::::::::::::::::::::::::::::::::::::::::::::::::::::::::|<<|
and hence,
%|>>|===========================================================================================|>>|
\begin{align*}
  \Sign( \left\langle \CovVX\VIx{\iIx}, \frac{\thetaStar-\thetaX}{\| \thetaStar-\thetaX \|_{2}} \right\rangle )
  % &=
  % \Sign( \left\langle \CovVX\VIx{\iIx}, \thetaStar-\thetaX \right\rangle )
  % \\
  &=
  \Sign( \left\langle \CovVX\VIx{\iIx}, \thetaStar \right\rangle - \left\langle \CovVX\VIx{\iIx}, \thetaX \right\rangle )
  \\
  &=
  \Sign( \Sign( \left\langle \CovVX\VIx{\iIx}, \thetaStar \right\rangle ) \left|\left\langle \CovVX\VIx{\iIx}, \thetaStar \right\rangle\right| - \Sign( \left\langle \CovVX\VIx{\iIx}, \thetaX \right\rangle ) \left| \left\langle \CovVX\VIx{\iIx}, \thetaX \right\rangle \right| )
  \\
  &=
  \Sign( \Sign( \left\langle \CovVX\VIx{\iIx}, \thetaStar \right\rangle ) \left|\left\langle \CovVX\VIx{\iIx}, \thetaStar \right\rangle\right| + \Sign( \left\langle \CovVX\VIx{\iIx}, \thetaStar \right\rangle ) \left| \left\langle \CovVX\VIx{\iIx}, \thetaX \right\rangle \right| )
  \\
  &=
  \Sign( \Sign( \left\langle \CovVX\VIx{\iIx}, \thetaStar \right\rangle ) \left( \left|\left\langle \CovVX\VIx{\iIx}, \thetaStar \right\rangle\right| + \left| \left\langle \CovVX\VIx{\iIx}, \thetaX \right\rangle \right| \right) )
  \\
  &=
  \Sign( \left\langle \CovVX\VIx{\iIx}, \thetaStar \right\rangle )
.\end{align*}
%|<<|===========================================================================================|<<|
With the above work out of the way, we are ready to derive \EQUATIONS \eqref{eqn:lemma:concentration-ineq:noiseless:pr:1}--\eqref{eqn:lemma:concentration-ineq:noiseless:pr:3}.
%
%%%%%%%%%%%%%%%%%%%%%%%%%%%%%%%%%%%%%%%%%%%%%%%%%%%%%%%%%%%%%%%%%%%%%%%%%%%%%%%%%%%%%%%%%%%%%%%%%%%%
\paragraph{Verification of \EQUATION \eqref{eqn:lemma:concentration-ineq:noiseless:pr:1}} %%%%%%%%%%
%%%%%%%%%%%%%%%%%%%%%%%%%%%%%%%%%%%%%%%%%%%%%%%%%%%%%%%%%%%%%%%%%%%%%%%%%%%%%%%%%%%%%%%%%%%%%%%%%%%%
%
For \(  \iIx \in [\m]  \), let
% %|>>|:::::::::::::::::::::::::::::::::::::::::::::::::::::::::::::::::::::::::::::::::::::::::::|>>|
% \(  \Ui \defeq | \langle \CovVX\VIx{\iIx}, \frac{\thetaStar-\thetaX}{\| \thetaStar-\thetaX \|_{2}} \rangle |
%  \sep\I( \Sign( \langle \CovVX\VIx{\iIx}, \thetaStar \rangle ) \neq \Sign( \langle \CovVX\VIx{\iIx}, \thetaX \rangle ) )  \),
% %|<<|:::::::::::::::::::::::::::::::::::::::::::::::::::::::::::::::::::::::::::::::::::::::::::|<<|
%|>>|===========================================================================================|>>|
\begin{gather*}
  \Ui \defeq | \langle \CovVX\VIx{\iIx}, \frac{\thetaStar-\thetaX}{\| \thetaStar-\thetaX \|_{2}} \rangle |
 \sep\I( \Sign( \langle \CovVX\VIx{\iIx}, \thetaStar \rangle ) \neq \Sign( \langle \CovVX\VIx{\iIx}, \thetaX \rangle ) )
,\end{gather*}
%|<<|===========================================================================================|<<|
and let
% %|>>|:::::::::::::::::::::::::::::::::::::::::::::::::::::::::::::::::::::::::::::::::::::::::::|>>|
% \(  \Ri \defeq \I( \Sign( \langle \CovVX\VIx{\iIx}, \thetaStar \rangle ) \neq \Sign( \langle \CovVX\VIx{\iIx}, \thetaX \rangle ) )  \).
% %|<<|:::::::::::::::::::::::::::::::::::::::::::::::::::::::::::::::::::::::::::::::::::::::::::|<<|
%|>>|===========================================================================================|>>|
\begin{gather*}
  \Ri \defeq \I( \Sign( \langle \CovVX\VIx{\iIx}, \thetaStar \rangle ) \neq \Sign( \langle \CovVX\VIx{\iIx}, \thetaX \rangle ) )
.\end{gather*}
%|<<|===========================================================================================|<<|
Note that although this definition of \(  \Ri  \) differs slightly from a similar random variable analyzed in \cite[\APPENDIX B]{matsumoto2022binary}, nearly the same arguments as those in \cite{matsumoto2022binary} apply here, and hence we omit the
%analogous
analysis here.
Due to the analysis in \cite[\APPENDIX B.1.1]{matsumoto2022binary}, the mass function of the random variable \(  \Ri  \) is given by
%|>>|===========================================================================================|>>|
\begin{gather}
\label{eqn:pf:eqn:lemma:concentration-ineq:noiseless:pr:1:f_Ri}
  \pdf{\Ri}( \rX )
  =
  \begin{cases}
  1-\frac{1}{\pi} \ADIST ,& \cIf \rX=0, \\
  \frac{1}{\pi} \ADIST   ,& \cIf \rX=1,
  \end{cases}
\end{gather}
%|<<|===========================================================================================|<<|
for \(  \rX \in \{ 0,1 \}  \).
For \(  \zX \in \R  \) and \(  \rX \in \{ 0,1 \}  \), the density function of the conditioned random variable \(  \Ui \Mid| \Ri  \) is given by
%|>>|===========================================================================================|>>|
\begin{gather}
\label{eqn:pf:eqn:lemma:concentration-ineq:noiseless:pr:1:f_Ui|Ri}
  \pdf{\Ui \Mid| \Ri}( \zX \Mid| \rX )
  =
  \begin{cases}
  0 ,& \cIf \rX=0, \zX \neq 0, \\
  1 ,& \cIf \rX=0, \zX=0, \\
  0 ,& \cIf \rX=1, \zX<0, \\
  \frac{\pi}{\ADIST}
  \sqrt{\frac{2}{\pi}}
  e^{-\frac{1}{2} \zX^{2}}
  \frac{1}{\sqrt{2\pi}}
  \int_{\yX=-\zX \tan( \frac{1}{2} \ADIST )}^{\yX=\zX \tan( \frac{1}{2} \ADIST )}
  e^{-\frac{1}{2} \yX^{2}}
  d\yX
  ,& \cIf \rX=1, \zX \geq 0.
  \end{cases}
\end{gather}
%|<<|===========================================================================================|<<|
%while by symmetry, the density function of the negated conditioned random variable \(  -\Ui \Mid| \Ri  \) is given by
%%|>>|===========================================================================================|>>|
%\begin{gather}
%\label{eqn:pf:eqn:lemma:concentration-ineq:noiseless:pr:1:f_-Ui|Ri}
%  \pdf{-\Ui \Mid| \Ri}( \zX \Mid| \rX )
%  =
%  \begin{cases}
%  0 ,& \cIf \rX=0, \zX \neq 0, \\
%  1 ,& \cIf \rX=0, \zX=0, \\
%  0 ,& \cIf \rX=1, \zX>0, \\
%  \frac{\pi}{\ADIST}
%  \sqrt{\frac{2}{\pi}}
%  e^{-\frac{1}{2} \zX^{2}}
%  \frac{1}{\sqrt{2\pi}}
%  \int_{\yX=-|\zX| \tan( \frac{1}{2} \ADIST )}^{\yX=|\zX| \tan( \frac{1}{2} \ADIST )}
%  e^{-\frac{1}{2} \yX^{2}}
%  d\yX
%  ,& \cIf \rX=1, \zX \leq 0.
%  \end{cases}
%\end{gather}
%%|<<|===========================================================================================|<<|
%
%%%%%%%%%%%%%%%%%%%%%%%%%%%%%%%%%%%%%%%%%%%%%%%%%%%%%%%%%%%%%%%%%%%%%%%%%%%%%%%%%%%%%%%%%%%%%%%%%%%%
\par %%%%%%%%%%%%%%%%%%%%%%%%%%%%%%%%%%%%%%%%%%%%%%%%%%%%%%%%%%%%%%%%%%%%%%%%%%%%%%%%%%%%%%%%%%%%%%%
%%%%%%%%%%%%%%%%%%%%%%%%%%%%%%%%%%%%%%%%%%%%%%%%%%%%%%%%%%%%%%%%%%%%%%%%%%%%%%%%%%%%%%%%%%%%%%%%%%%%
%
Having specified the density function of the conditioned random variable \(  \Ui \Mid| \Ri  \), the next step is obtaining the \MGFs of the centered random variables \(  ( \Ui \Mid| \Ri ) - \E[ \Ui \Mid| \Ri ]  \) and \(  ( -\Ui \Mid| \Ri ) - \E[ -\Ui \Mid| \Ri ]  \).
To simplify notation, write \(  \muX[1] \defeq \E[ \Ui \Mid| \Ri=1 ]  \) and \(  \muX[0] \defeq \E[ \Ui \Mid| \Ri=0 ]  \), where in the latter case, %\(  \muX[0] = 0  \) since
%|>>|===========================================================================================|>>|
\begin{align}
  \muX[0]
  &= \E[ \Ui \Mid| \Ri=0 ]
  = \int_{\zX=-\infty}^{\zX=\infty} \zX \pdf{\Ui \Mid| \Ri}( \zX \Mid| 0 ) d\zX
  = 0 \pdf{\Ui \Mid| \Ri}( 0 \Mid| 0 )
  = 0
\label{eqn:pf:eqn:lemma:concentration-ineq:noiseless:pr:7}
.\end{align}
%|<<|===========================================================================================|<<|
Due to \cite[{\APPENDIX B.1}]{matsumoto2022binary}, the \MGF of \(  (\Ui \Mid| \Ri=1)-\E[\Ui \Mid| \Ri=1]  \), denoted by \(  \mgf{(\Ui \Mid| \Ri=1)-\E[\Ui \Mid| \Ri=1]}  \), is given and upper bounded at \(  \sX \in [0,\infty)  \) by
%|>>|===========================================================================================|>>|
\begin{align*}
  &\AlignIndent
  \mgf{(\Ui \Mid| \Ri=1)-\E[\Ui \Mid| \Ri=1]}( \sX )
  \\
  &\AlignIndent=
  %\E \left[ e^{\sX ( (\Ui \Mid| \Ri=1) - \E[ \Ui \Mid| \Ri=1 ] )} \right]
  \E \left[ e^{\sX ( \Ui - \E[\Ui] )} \middle| \Ri=1 \right]
  \\
  &\AlignIndent=
  e^{\frac{1}{2} \sX^{2}}
  e^{-\sX \muX[1]}
  \frac{\pi}{\ADIST}
  \sqrt{\frac{2}{\pi}}
  \int_{\zX=0}^{\zX=\infty}
  e^{-\frac{1}{2} (\zX-\sX)^{2}}
  \frac{1}{\sqrt{2\pi}}
  \int_{\yX=-\zX \tan( \frac{1}{2} \ADIST )}^{\yX=\zX \tan( \frac{1}{2} \ADIST )}
  e^{-\frac{1}{2} \yX^{2}}
  d\yX\,
  d\zX
  \\
  &\AlignIndent\dCmt{by the law of the lazy statistician and \EQUATION \eqref{eqn:pf:eqn:lemma:concentration-ineq:noiseless:pr:1:f_Ui|Ri}}
  \\
  &\AlignIndent\leq
  e^{\frac{1}{2} \sX^{2}}
\TagEqn{\label{eqn:pf:eqn:lemma:concentration-ineq:noiseless:pr:2:a}}
.\end{align*}
%|<<|===========================================================================================|<<|
%
Likewise, the \MGF of \(  (-\Ui \Mid| \Ri=1)-\E[-\Ui \Mid| \Ri=1]  \), denoted by \(  \mgf{(-\Ui \Mid| \Ri=1)-\E[-\Ui \Mid| \Ri=1]}  \), is \XXX{given and} upper bounded at \(  \sX \in [0,\infty)  \) by
%|>>|===========================================================================================|>>|
\begin{align*}
  &
  \mgf{(-\Ui \Mid| \Ri=1)-\E[-\Ui \Mid| \Ri=1]}( \sX )
\XXX{
  \\
  &\AlignIndent=
  %\E \left[ e^{\sX ( (-\Ui \Mid| \Ri=1) - \E[ -\Ui \Mid| \Ri=1 ] )} \right]
  \E \left[ e^{\sX ( -\Ui - \E[-\Ui] )} \middle| \Ri=1 \right]
  \\
  &\AlignIndent=
  \E \left[ e^{-\sX ( \Ui - \E[\Ui] )} \middle| \Ri=1 \right]
  \\
  &\AlignIndent=
  e^{\frac{1}{2} \sX^{2}}
  e^{\sX \muX[1]}
  \frac{\pi}{\ADIST}
  \sqrt{\frac{2}{\pi}}
  \int_{\zX=0}^{\zX=\infty}
  e^{-\frac{1}{2} (\zX+\sX)^{2}}
  \frac{1}{\sqrt{2\pi}}
  \int_{\yX=-\zX \tan( \frac{1}{2} \ADIST )}^{\yX=\zX \tan( \frac{1}{2} \ADIST )}
  e^{-\frac{1}{2} \yX^{2}}
  d\yX\,
  d\zX
  \\
  &\AlignIndent\dCmt{by the law of the lazy statistician and \EQUATION \eqref{eqn:pf:eqn:lemma:concentration-ineq:noiseless:pr:1:f_Ui|Ri}}
  \\
  &\AlignIndent
}
  \leq
  e^{\frac{1}{2} \sX^{2}}
\TagEqn{\label{eqn:pf:eqn:lemma:concentration-ineq:noiseless:pr:2:b}}
.\end{align*}
%|<<|===========================================================================================|<<|
On the other hand, when conditioning on \(  \Ri=0  \), the \MGF of the centered conditioned random variable \(  (\Ui \Mid| \Ri=0)-\E[\Ui \Mid| \Ri=0]  \), written \(  \mgf{(\Ui \Mid| \Ri=0)-\E[\Ui \Mid| \Ri=0]}  \), is given at \(  \sX \in [0,\infty)  \) by
%|>>|===========================================================================================|>>|
\begin{align*}
  \mgf{(\Ui \Mid| \Ri=0)-\E[\Ui \Mid| \Ri=0]}( \sX )
  &=
  %\E \left[ e^{\sX ( (\Ui \Mid| \Ri=0) - \E[ \Ui \Mid| \Ri=0 ] )} \right]
\XXX{
  \E \left[ e^{\sX ( \Ui - \E[\Ui] )} \middle| \Ri=0 \right]
  \\
  &=
}
  \E \left[ e^{\sX \Ui} \middle| \Ri=0 \right]
  \\
  &\dCmt{using \EQUATION \eqref{eqn:pf:eqn:lemma:concentration-ineq:noiseless:pr:7}}
\XXX{
  \\
  &=
  \int_{\zX=-\infty}^{\zX=\infty} e^{\sX \zX} \pdf{\Ui \Mid| \Ri}( \zX \Mid| 0 ) d\zX
  \\
  &\dCmt{by the law of the lazy statistician and \EQUATION \eqref{eqn:pf:eqn:lemma:concentration-ineq:noiseless:pr:1:f_Ui|Ri}}
}
  \\
  &=
  e^{\sX \cdot 0} \pdf{\Ui \Mid| \Ri}( 0 \Mid| 0 )
  \\
  &\dCmt{by the law of the lazy statistician and \EQUATION \eqref{eqn:pf:eqn:lemma:concentration-ineq:noiseless:pr:1:f_Ui|Ri}, and}
  \\
  &\dCmtx{since the mass of \(  \Ui \Mid| \Ri=0  \) is entirely concentrated at \(  0   \)}
  \\
  &=
  1
\TagEqn{\label{eqn:pf:eqn:lemma:concentration-ineq:noiseless:pr:3:a}}
,\end{align*}
%|<<|===========================================================================================|<<|
and the \MGF of the centered conditioned random variable \(  (-\Ui \Mid| \Ri=0)-\E[-\Ui \Mid| \Ri=0]  \), written \(  \mgf{(-\Ui \Mid| \Ri=0)-\E[-\Ui \Mid| \Ri=0]}  \), is similarly given at \(  \sX \in [0,\infty)  \) by
%|>>|===========================================================================================|>>|
\begin{align*}
  \mgf{(-\Ui \Mid| \Ri=0)-\E[-\Ui \Mid| \Ri=0]}( \sX )
  =
  1
\TagEqn{\label{eqn:pf:eqn:lemma:concentration-ineq:noiseless:pr:3:b}}
.\end{align*}
%|<<|===========================================================================================|<<|
%
\XXX{and the \MGF of the centered conditioned random variable \(  (-\Ui \Mid| \Ri=0)-\E[-\Ui \Mid| \Ri=0]  \), written \(  \mgf{(-\Ui \Mid| \Ri=0)-\E[-\Ui \Mid| \Ri=0]}  \), is similarly given at \(  \sX \in [0,\infty)  \) by
%|>>|===========================================================================================|>>|
\begin{align*}
  \mgf{(-\Ui \Mid| \Ri=0)-\E[-\Ui \Mid| \Ri=0]}( \sX )
  &=
  %\E \left[ e^{\sX ( (-\Ui \Mid| \Ri=0) - \E[ -\Ui \Mid| \Ri=0 ] )} \right]
  \E \left[ e^{\sX ( -\Ui - \E[-\Ui] )} \middle| \Ri=0 \right]
  \\
  &=
  \E \left[ e^{-\sX \Ui} \middle| \Ri=0 \right]
  \\
  &\dCmt{by \EQUATION \eqref{eqn:pf:eqn:lemma:concentration-ineq:noiseless:pr:7}}
  \\
  &=
  \int_{\zX=-\infty}^{\zX=\infty} e^{-\sX \zX} \pdf{\Ui \Mid| \Ri}( \zX \Mid| 0 ) d\zX
  \\
  &\dCmt{by the law of the lazy statistician and \EQUATION \eqref{eqn:pf:eqn:lemma:concentration-ineq:noiseless:pr:1:f_Ui|Ri}}
  \\
  &=
  e^{-\sX \cdot 0} \pdf{\Ui \Mid| \Ri}( 0 \Mid| 0 )
  \\
  &=
  1
\TagEqn{\label{eqn:pf:eqn:lemma:concentration-ineq:noiseless:pr:3:b}}
.\end{align*}
%|<<|===========================================================================================|<<|
}
%
Taking together the two cases for \(  \Ri=1  \) and \(  \Ri=0  \), the \MGF of \(  \Ui-\E[ \Ui ]  \), written \(  \mgf{\Ui-\E[\Ui]}  \), is given and bounded from above at \(  \sX \in [0,\infty)  \) by
%|>>|===========================================================================================|>>|
\begin{align*}
  \mgf{\Ui-\E[\Ui]}( \sX )
  &=
  \E \left[ e^{\sX ( \Ui - \E[ \Ui ] )} \right]
  \\
  &=
  \pdf{\Ri}( 1 )
  \E \left[ e^{\sX ( \Ui - \E[ \Ui ] )} \middle| \Ri=1 \right]
  +
  \pdf{\Ri}( 0 )
  \E \left[ e^{\sX ( \Ui - \E[ \Ui ] )} \middle| \Ri=0 \right]
  \\
  &\dCmt{by the law of total expectation}
  \\
  &=
  \pdf{\Ri}( 1 )
  \mgf{( \Ui \Mid| \Ri=1 )-\E[ \Ui \Mid| \Ri=1 ]}( \sX )
  +
  \pdf{\Ri}( 0 )
  \mgf{( \Ui \Mid| \Ri=0 )-\E[ \Ui \Mid| \Ri=0 ]}( \sX )
  \\
  &\dCmt{by the definition of \MGFs}
  \\
  &=
  \frac{1}{\pi} \ADIST
  \mgf{( \Ui \Mid| \Ri=1 )-\E[ \Ui \Mid| \Ri=1 ]}( \sX )
  \\
  &\AlignSp+
  \left( 1-\frac{1}{\pi} \ADIST \right)
  \mgf{( \Ui \Mid| \Ri=0 )-\E[ \Ui \Mid| \Ri=0 ]}( \sX )
  \\
  &\dCmt{by \EQUATION \eqref{eqn:pf:eqn:lemma:concentration-ineq:noiseless:pr:1:f_Ri}}
  \\
  &\leq
  \frac{1}{\pi} \ADIST
  e^{\frac{1}{2} \sX^{2}}
  +
  \left( 1-\frac{1}{\pi} \ADIST \right)
  \\
  &\dCmt{by \EQUATIONS \eqref{eqn:pf:eqn:lemma:concentration-ineq:noiseless:pr:1:f_Ri}, \eqref{eqn:pf:eqn:lemma:concentration-ineq:noiseless:pr:2:a}, and \eqref{eqn:pf:eqn:lemma:concentration-ineq:noiseless:pr:3:a}}
  \\
  &=
  1 + \frac{1}{\pi} \ADIST \left( e^{\frac{1}{2} \sX^{2}} - 1 \right)
\TagEqn{\label{eqn:pf:eqn:lemma:concentration-ineq:noiseless:pr:1:1a}}
,\end{align*}
%|<<|===========================================================================================|<<|
The \MGF of \(  -\Ui-\E[-\Ui]  \), denoted by \(  \mgf{-\Ui-\E[-\Ui]}  \), is similarly bounded from above by
%|>>|===========================================================================================|>>|
\begin{align*}
  \mgf{-\Ui-\E[-\Ui]}( \sX )
  \leq
  1 + \frac{1}{\pi} \ADIST \left( e^{\frac{1}{2} \sX^{2}} - 1 \right)
\TagEqn{\label{eqn:pf:eqn:lemma:concentration-ineq:noiseless:pr:1:1b}}
.\end{align*}
%|<<|===========================================================================================|<<|
%
\XXX{and similarly, the \MGF of \(  -\Ui-\E[-\Ui]  \), denoted by \(  \mgf{-\Ui-\E[-\Ui]}  \), is given and bounded from above by
%|>>|===========================================================================================|>>|
\begin{align*}
  \mgf{-\Ui-\E[-\Ui]}( \sX )
  &=
  \E \left[ e^{\sX ( -\Ui - \E[ -\Ui ] )} \right]
  \\
  &=
  \pdf{\Ri}( 1 )
  \E \left[ e^{\sX ( -\Ui - \E[ -\Ui ] )} \middle| \Ri=1 \right]
  +
  \pdf{\Ri}( 0 )
  \E \left[ e^{\sX ( -\Ui - \E[ -\Ui ] )} \middle| \Ri=0 \right]
  \\
  &\dCmt{by the law of total expectation}
  \\
  &=
  \pdf{\Ri}( 1 )
  \mgf{( -\Ui \Mid| \Ri=1 )-\E[ -\Ui \Mid| \Ri=1 ]}( \sX )
  +
  \pdf{\Ri}( 0 )
  \mgf{( -\Ui \Mid| \Ri=0 )-\E[ -\Ui \Mid| \Ri=0 ]}( \sX )
  \\
  &\dCmt{by the definition of \MGFs}
\XXX{
  \\
  &=
  \frac{1}{\pi} \ADIST
  \mgf{( -\Ui \Mid| \Ri=1 )-\E[ -\Ui \Mid| \Ri=1 ]}( \sX )
  \\
  &\AlignIndent+
  \left( 1-\frac{1}{\pi} \ADIST \right)
  \mgf{( -\Ui \Mid| \Ri=0 )-\E[ -\Ui \Mid| \Ri=0 ]}( \sX )
  \\
  &\dCmt{by \EQUATION \eqref{eqn:pf:eqn:lemma:concentration-ineq:noiseless:pr:1:f_Ri}}
}
  \\
  &\leq
  \frac{1}{\pi} \ADIST
  e^{\frac{1}{2} \sX^{2}}
  +
  \left( 1-\frac{1}{\pi} \ADIST \right)
  \\
  &\dCmt{by \EQUATIONS \eqref{eqn:pf:eqn:lemma:concentration-ineq:noiseless:pr:2:b} and \eqref{eqn:pf:eqn:lemma:concentration-ineq:noiseless:pr:3:b}}
  \\
  &=
  1 + \frac{1}{\pi} \ADIST \left( e^{\frac{1}{2} \sX^{2}} - 1 \right)
\TagEqn{\label{eqn:pf:eqn:lemma:concentration-ineq:noiseless:pr:1:1b}}
.\end{align*}
%|<<|===========================================================================================|<<|
}
%
%%%%%%%%%%%%%%%%%%%%%%%%%%%%%%%%%%%%%%%%%%%%%%%%%%%%%%%%%%%%%%%%%%%%%%%%%%%%%%%%%%%%%%%%%%%%%%%%%%%%
\par %%%%%%%%%%%%%%%%%%%%%%%%%%%%%%%%%%%%%%%%%%%%%%%%%%%%%%%%%%%%%%%%%%%%%%%%%%%%%%%%%%%%%%%%%%%%%%%
%%%%%%%%%%%%%%%%%%%%%%%%%%%%%%%%%%%%%%%%%%%%%%%%%%%%%%%%%%%%%%%%%%%%%%%%%%%%%%%%%%%%%%%%%%%%%%%%%%%%
%
Let
%|>>|:::::::::::::::::::::::::::::::::::::::::::::::::::::::::::::::::::::::::::::::::::::::::::|>>|
\(  \URV \defeq \sum_{\iIx=1}^{\m} \Ui  \).
%|<<|:::::::::::::::::::::::::::::::::::::::::::::::::::::::::::::::::::::::::::::::::::::::::::|<<|
Using the above bound on the \MGFs of \(  \Ui-\E[\Ui]  \), \(  \iIx \in [\m]  \), in \EQUATION \eqref{eqn:pf:eqn:lemma:concentration-ineq:noiseless:pr:1:1a}, the \MGF of \(  \URV-\E[\URV]  \), written \(  \mgf{\URV-\E[\URV]}  \), is given and upper bounded at \(  \sX \in [0,\infty)  \) as follows:
%|>>|===========================================================================================|>>|
\begin{align*}
  \mgf{\URV-\E[\URV]}( \sX )
  &=
  \E \left[ e^{\sX ( \URV - \E[ \URV ] )} \right]
  \\
  &\dCmt{by the definition of the \MGF \(  \mgf{\URV-\E[\URV]}  \)}
  \\
  &=
  \E \left[ e^{\sX \sum_{\iIx=1}^{\m} \Ui - \E[ \Ui ]} \right]
  \\
  &\dCmt{by the definition of \(  \URV  \)}
\XXX{
  \\
  &=
  \E \left[ \prod_{\iIx=1}^{\m} e^{\sX ( \Ui - \E[ \Ui ] )} \right]
  \\
  &\dCmt{by standard facts about exponents}
}
  \\
  &=
  \prod_{\iIx=1}^{\m} \E \left[ e^{\sX ( \Ui - \E[ \Ui ] )} \right]
  \\
  &\dCmt{since \(  \Ui[1], \dots, \Ui[\m]  \) are mutually independent}
\XXX{
  \\
  &=
  \prod_{\iIx=1}^{\m} \mgf{\Ui-\E[\Ui]}( \sX )
  \\
  &\dCmt{by the definition of \(  \mgf{\Ui-\E[\Ui]}  \), \(  \iIx \in [\m]  \)}
}
  \\
  &=
  ( \mgf{\Ui-\E[\Ui]}( \sX ) )^{\m}
  \\
  &\dCmt{for any \(  \iIx \in [\m]  \);}
  \\
  &\dCmt{by the definition of \(  \mgf{\Ui-\E[\Ui]}  \), \(  \iIx \in [\m]  \), and}
  \\
  &\dCmtx{since \(  \Ui[1], \dots, \Ui[\m]  \) are identically distributed}
  \\
  &\leq
  \left( 1 + \frac{1}{\pi} \ADIST \left( e^{\frac{1}{2} \sX^{2}} - 1 \right) \right)^{\m}
  \\
  &\dCmt{by \EQUATION \eqref{eqn:pf:eqn:lemma:concentration-ineq:noiseless:pr:1:1a}}
  \\ \TagEqn{\label{eqn:pf:eqn:lemma:concentration-ineq:noiseless:pr:1:3a}}
  &\leq
  e^{\frac{1}{\pi} \m \ADIST ( e^{\frac{1}{2} \sX^{2}} - 1 )}
  ,\\
  &\dCmt{by a well-known inequality, \(  \log( 1+u ) \leq u  \) for \(  u > -1  \)}
  % \\
  % &\dCmt{\(  {\textstyle \log ( 1 + \frac{1}{\pi} \ADIST ( e^{\frac{1}{2} \sX^{2}} - 1 ) ) \leq \frac{1}{\pi} \ADIST ( e^{\frac{1}{2} \sX^{2}} - 1 )}  \)}
  % \\
  % &\dCmtIndent\text{(by a well-known inequality)}
\end{align*}
%|<<|===========================================================================================|<<|
Likewise, applying \EQUATION \eqref{eqn:pf:eqn:lemma:concentration-ineq:noiseless:pr:1:1b} for the negated random variable, \(  -\URV-\E[-\URV]  \), obtains:
%|>>|===========================================================================================|>>|
\begin{align*}
  \mgf{-\URV-\E[-\URV]}( \sX )
  &\leq
  e^{\frac{1}{\pi} \m \ADIST ( e^{\frac{1}{2} \sX^{2}} - 1 )}
\TagEqn{\label{eqn:pf:eqn:lemma:concentration-ineq:noiseless:pr:1:3b}}
.\end{align*}
%|<<|===========================================================================================|<<|
%
\XXX{and likewise, applying \EQUATION \eqref{eqn:pf:eqn:lemma:concentration-ineq:noiseless:pr:1:1b} for the negated random variable, \(  -\URV-\E[-\URV]  \):
%|>>|===========================================================================================|>>|
\begin{align*}
  \mgf{-\URV-\E[-\URV]}( \sX )
  &=
  \E \left[ e^{\sX ( -\URV - \E[ -\URV ] )} \right]
  \\
  &\dCmt{by the definition of the \MGF \(  \mgf{-\URV-\E[-\URV]}  \)}
  \\
  &=
  \E \left[ e^{\sX \sum_{\iIx=1}^{\m} -\Ui - \E[ -\Ui ]} \right]
  \\
  &\dCmt{by the definition of \(  \URV  \)}
\XXX{
  \\
  &=
  \E \left[ \prod_{\iIx=1}^{\m} e^{\sX ( -\Ui - \E[ -\Ui ] )} \right]
  \\
  &\dCmt{by standard facts about exponents}
}
  \\
  &=
  \prod_{\iIx=1}^{\m} \E \left[ e^{\sX ( -\Ui - \E[ -\Ui ] )} \right]
  \\
  &\dCmt{\(  -\Ui[1], \dots, -\Ui[\m]  \) are mutually independent}
  \\
  &=
  \prod_{\iIx=1}^{\m} \mgf{-\Ui-\E[-\Ui]}( \sX )
  \\
  &\dCmt{by the definition of \(  \mgf{-\Ui-\E[-\Ui]}  \), \(  \iIx \in [\m]  \)}
  \\
  &=
  ( \mgf{-\Ui-\E[-\Ui]}( \sX ) )^{\m}
  \\
  &\dCmt{for any \(  \iIx \in [\m]  \);}
  \\
  &\dCmt{since \(  -\Ui[1], \dots, -\Ui[\m]  \) are identically distributed}
  \\
  &\leq
  \left( 1 + \frac{1}{\pi} \ADIST \left( e^{\frac{1}{2} \sX^{2}} - 1 \right) \right)^{\m}
  \\
  &\dCmt{by \EQUATION \eqref{eqn:pf:eqn:lemma:concentration-ineq:noiseless:pr:1:1b}}
  \\ \TagEqn{\label{eqn:pf:eqn:lemma:concentration-ineq:noiseless:pr:1:3b}}
  &\leq
  e^{\frac{1}{\pi} \m \ADIST ( e^{\frac{1}{2} \sX^{2}} - 1 )}
  .\\
  &\dCmt{by a well-known inequality, \(  \log( 1+u ) \leq u  \) for \(  u > -1  \)}
  % &\dCmt{\(  {\textstyle \log ( 1 + \frac{1}{\pi} \ADIST ( e^{\frac{1}{2} \sX^{2}} - 1 ) ) \leq \frac{1}{\pi} \ADIST ( e^{\frac{1}{2} \sX^{2}} - 1 )}  \)}
  % \\
  % &\dCmtIndent\text{by a well-known inequality}
\end{align*}
%|<<|===========================================================================================|<<|
}
%
Then, due to Bernstein (\see e.g., \cite{vershynin2018high}),
%|>>|===========================================================================================|>>|
\begin{align*}
  \Pr \left( \frac{\URV}{\m} - \E \left[ \frac{\URV}{\m} \right] > \frac{1}{\pi} \tX \ADIST \right)
  &\leq
  \inf_{\sX \geq 0}
  e^{-\frac{1}{\pi} \m \sX \tX \ADIST}
  \mgf{\URV-\E[\URV]}( \sX )
  \\
  &\dCmt{due to Bernstein}
  \\
  &\leq
  \inf_{\sX \geq 0}
  e^{-\frac{1}{\pi} \m \sX \tX \ADIST}
  e^{\frac{1}{\pi} \m \ADIST ( e^{\frac{1}{2} \sX^{2}} - 1 )}
  \\
  &\dCmt{by \EQUATION \eqref{eqn:pf:eqn:lemma:concentration-ineq:noiseless:pr:1:3a}}
  \\
  &=
  \inf_{\sX \geq 0}
  e^{-\frac{1}{\pi} \m \ADIST ( \sX \tX - e^{\frac{1}{2} \sX^{2}} + 1 )}
\TagEqn{\label{eqn:pf:eqn:lemma:concentration-ineq:noiseless:pr:1:2a}}
,\end{align*}
%|<<|===========================================================================================|<<|
and for the concentration on the other side, an analogous derivation gives
%|>>|===========================================================================================|>>|
\begin{align*}
  \Pr \left( \frac{\URV}{\m} - \E \left[ \frac{\URV}{\m} \right] < -\frac{1}{\pi} \tX \ADIST \right)
  \leq
  \inf_{\sX \geq 0}
  e^{-\frac{1}{\pi} \m \ADIST ( \sX \tX - e^{\frac{1}{2} \sX^{2}} + 1 )}
\TagEqn{\label{eqn:pf:eqn:lemma:concentration-ineq:noiseless:pr:1:2b}}
.\end{align*}
%|<<|===========================================================================================|<<|
%
\XXX{and for the concentration on the other side,
%|>>|===========================================================================================|>>|
\begin{align*}
  \Pr \left( \frac{\URV}{\m} - \E \left[ \frac{\URV}{\m} \right] < -\frac{1}{\pi} \tX \ADIST \right)
  &=
  \Pr \left( -\frac{\URV}{\m} - \E \left[ -\frac{\URV}{\m} \right] > \frac{1}{\pi} \tX \ADIST \right)
  \\
  &\dCmt{by negating both sides of the inequality}
  \\
  &\leq
  \inf_{\sX \geq 0}
  e^{-\frac{1}{\pi} \m \sX \tX \ADIST}
  \mgf{-\URV-\E[-\URV]}( \sX )
  \\
  &\dCmt{due to Bernstein}
  \\
  &\leq
  \inf_{\sX \geq 0}
  e^{-\frac{1}{\pi} \m \sX \tX \ADIST}
  e^{\frac{1}{\pi} \m \ADIST ( e^{\frac{1}{2} \sX^{2}} - 1 )}
  \\
  &\dCmt{by \EQUATION \eqref{eqn:pf:eqn:lemma:concentration-ineq:noiseless:pr:1:3b}}
  \\
  &=
  \inf_{\sX \geq 0}
  e^{-\frac{1}{\pi} \m \ADIST ( \sX \tX - e^{\frac{1}{2} \sX^{2}} + 1 )}
\TagEqn{\label{eqn:pf:eqn:lemma:concentration-ineq:noiseless:pr:1:2b}}
.\end{align*}
%|<<|===========================================================================================|<<|
}
%
To minimize the last expressions in \EQUATIONS \eqref{eqn:pf:eqn:lemma:concentration-ineq:noiseless:pr:1:2a} and \eqref{eqn:pf:eqn:lemma:concentration-ineq:noiseless:pr:1:2b}, observe that when \(  \sX, \tX > 0  \) are small,
%|>>|===========================================================================================|>>|
\begin{align*}
  \left. \frac{\partial}{\partial \sX} \sX \tX - e^{\frac{1}{2} \sX^{2}} + 1 \right|_{\sX=\tX}
  =
  \left. \tX - \sX e^{\frac{1}{2} \sX^{2}} \right|_{\sX=\tX}
  \approx
  0
,\end{align*}
%|<<|===========================================================================================|<<|
where
%|>>|===========================================================================================|>>|
\begin{align*}
  \frac{\partial^{2}}{\partial \sX^{2}} \sX \tX - e^{\frac{1}{2} \sX^{2}} + 1
  =
  -( 1+\sX^{2} ) e^{\frac{1}{2} \sX^{2}}
  <
  0
\end{align*}
%|<<|===========================================================================================|<<|
for any \(  \sX \in \R  \).
Hence, for small \(  \sX, \tX \in (0,1]  \), the expression \(  \sX \tX - e^{\frac{1}{2} \sX^{2}} + 1  \) is maximized with respect to \(  \sX  \) at approximately \(  \sX \approx \tX  \) (which minimizes \EQUATIONS \eqref{eqn:pf:eqn:lemma:concentration-ineq:noiseless:pr:1:2a} and \eqref{eqn:pf:eqn:lemma:concentration-ineq:noiseless:pr:1:2b}),
and moreover, when \(  \tX \in (0,1]  \),
%|>>|===========================================================================================|>>|
\begin{align*}
  \left. \sX \tX - e^{\frac{1}{2} \sX^{2}} + 1 \right|_{\sX=\tX}
  =
  \tX^{2} - e^{\frac{1}{2} \tX^{2}} + 1
  \geq
  \frac{\tX^{2}}{3}
.\end{align*}
%|<<|===========================================================================================|<<|
Therefore, taking \(  \sX=\tX  \) in \EQUATIONS \eqref{eqn:pf:eqn:lemma:concentration-ineq:noiseless:pr:1:2a} and \eqref{eqn:pf:eqn:lemma:concentration-ineq:noiseless:pr:1:2b} yields the following concentration inequalities for \(  \tX \in (0,1]  \):
%|>>|===========================================================================================|>>|
\begin{gather*}
  \Pr \left( \frac{\URV}{\m} - \E \left[ \frac{\URV}{\m} \right] > \frac{1}{\pi} \tX \ADIST \right)
  \leq
  e^{-\frac{1}{3\pi} \m \tX^{2} \ADIST}
  ,\\
  \Pr \left( \frac{\URV}{\m} - \E \left[ \frac{\URV}{\m} \right] < -\frac{1}{\pi} \tX \ADIST \right)
  \leq
  e^{-\frac{1}{3\pi} \m \tX^{2} \ADIST}
,\end{gather*}
%|<<|===========================================================================================|<<|
and combining these via a union bound subsequently obtains:
%|>>|===========================================================================================|>>|
\begin{gather*}
  \Pr \left( \left| \frac{\URV}{\m} - \E \left[ \frac{\URV}{\m} \right] \right| > \frac{1}{\pi} \tX \ADIST \right)
  \leq
  2 e^{-\frac{1}{3\pi} \m \tX^{2} \ADIST}
.\end{gather*}
%|<<|===========================================================================================|<<|
This, along with \EQUATION \eqref{eqn:pf:eqn:lemma:concentration-ineq:noiseless:pr:6} and the definition of the random variable \(  \URV  \), immediately implies that
%|>>|===========================================================================================|>>|
\begin{gather*}
  %\textstyle
  \Pr \left(
    \left| \left\langle \frac{\hFn[\JCoords]( \thetaStar, \thetaX )}{\sqrt{2\pi}}, \frac{\thetaStar-\thetaX}{\| \thetaStar-\thetaX \|_{2}} \right\rangle - \E \left[ \left\langle \frac{\hFn[\JCoords]( \thetaStar, \thetaX )}{\sqrt{2\pi}}, \frac{\thetaStar-\thetaX}{\| \thetaStar-\thetaX \|_{2}} \right\rangle \right] \right|
    >
    \frac{\tX \ADIST}{\pi}
  \right)
  \nonumber \\
  \leq
  2 e^{-\frac{1}{3\pi} \m \tX^{2} \ADIST}
.\end{gather*}
%|<<|===========================================================================================|<<|
Then, \EQUATION \eqref{eqn:lemma:concentration-ineq:noiseless:pr:1} follows from union bounds over all \(  \JCoords \in \JS  \) and \(  \thetaX \in \ParamCoverX  \):
%|>>|===========================================================================================|>>|
\begin{gather*}
  %\textstyle
  \Pr \left(
    \ExistsST{\JCoords \in \JS, \thetaX \in \ParamCoverX}{
    \left| \left\langle \frac{\hFn[\JCoords]( \thetaStar, \thetaX )}{\sqrt{2\pi}}, \frac{\thetaStar-\thetaX}{\| \thetaStar-\thetaX \|_{2}} \right\rangle - \E \left[ \left\langle \frac{\hFn[\JCoords]( \thetaStar, \thetaX )}{\sqrt{2\pi}}, \frac{\thetaStar-\thetaX}{\| \thetaStar-\thetaX \|_{2}} \right\rangle \right] \right|
    >
    \frac{\tX \ADIST}{\pi}
    }
  \right)
  \nonumber \\
  \leq
  2 | \JS | | \ParamCoverX | e^{-\frac{1}{3\pi} \m \tX^{2} \ADIST}
,\end{gather*}
%|<<|===========================================================================================|<<|
as desired.
%
%%%%%%%%%%%%%%%%%%%%%%%%%%%%%%%%%%%%%%%%%%%%%%%%%%%%%%%%%%%%%%%%%%%%%%%%%%%%%%%%%%%%%%%%%%%%%%%%%%%%
\paragraph{Verification of \EQUATION \eqref{eqn:lemma:concentration-ineq:noiseless:pr:2}} %%%%%%%%%%
%%%%%%%%%%%%%%%%%%%%%%%%%%%%%%%%%%%%%%%%%%%%%%%%%%%%%%%%%%%%%%%%%%%%%%%%%%%%%%%%%%%%%%%%%%%%%%%%%%%%
%
\let\oldvX\vX%
\let\vX\zX%
%
Next, the concentration inequality in \EQUATION \eqref{eqn:lemma:concentration-ineq:noiseless:pr:2} will be verified.
Again, take any \(  \JCoords \in \JS  \) and \(  \thetaX \in \ParamCoverX  \).
Write the random variables
%|>>|:::::::::::::::::::::::::::::::::::::::::::::::::::::::::::::::::::::::::::::::::::::::::::|>>|
\(  \Vi \defeq \langle \CovVX\VIx{\iIx}, \frac{\thetaStar+\thetaX}{\| \thetaStar+\thetaX \|_{2}} \rangle  \), \(  \iIx \in [\m]  \),
%|<<|:::::::::::::::::::::::::::::::::::::::::::::::::::::::::::::::::::::::::::::::::::::::::::|<<|
and carry over the notation of the random variables
%|>>|:::::::::::::::::::::::::::::::::::::::::::::::::::::::::::::::::::::::::::::::::::::::::::|>>|
\(  \Ri \defeq \I( \Sign( \langle \CovVX\VIx{\iIx}, \thetaStar \rangle ) \neq \Sign( \langle \CovVX\VIx{\iIx}, \thetaX \rangle ) )  \), \(  \iIx \in [\m]  \),
%|<<|:::::::::::::::::::::::::::::::::::::::::::::::::::::::::::::::::::::::::::::::::::::::::::|<<|
from the verification of \EQUATION \eqref{eqn:lemma:concentration-ineq:noiseless:pr:1}, whose mass functions are given in \EQUATION \eqref{eqn:pf:eqn:lemma:concentration-ineq:noiseless:pr:1:f_Ri}.
Due to \cite[\LEMMA B.6]{matsumoto2022binary}, the conditioned random variable \(  \Vi \Mid| \Ri=1  \) is standard Gaussian, \ie
%|>>|:::::::::::::::::::::::::::::::::::::::::::::::::::::::::::::::::::::::::::::::::::::::::::|>>|
\(  ( \Vi \Mid| \Ri=1 ) \sim \N(0,1)  \),
%|<<|:::::::::::::::::::::::::::::::::::::::::::::::::::::::::::::::::::::::::::::::::::::::::::|<<|
and thus, the density function of \(  \Vi \Mid| \Ri  \) is given for \(  \zX \in \R  \) and \(  \rX \in \{ 0,1 \}  \) by:
%|>>|===========================================================================================|>>|
\begin{gather*}
  \pdf{\Vi \Mid| \Ri}( \zX \Mid| \rX )
  =
  \begin{cases}
  0                                             ,& \cIf \rX=0, \zX \neq 0 ,\\
  1                                             ,& \cIf \rX=0, \zX = 0    ,\\
  \frac{1}{\sqrt{2\pi}} e^{-\frac{1}{2} \zX^{2}} ,& \cIf \rX=1.
  \end{cases}
\end{gather*}
%|<<|===========================================================================================|<<|
Since \(  \Vi \Mid| \Ri=1  \) is standard Gaussian, its expectation is
%|>>|:::::::::::::::::::::::::::::::::::::::::::::::::::::::::::::::::::::::::::::::::::::::::::|>>|
\(  \E[ \Vi \Mid| \Ri=1 ] = 0  \),
%|<<|:::::::::::::::::::::::::::::::::::::::::::::::::::::::::::::::::::::::::::::::::::::::::::|<<|
while also,
%|>>|===========================================================================================|>>|
\begin{gather*}
  \E[ \Vi \Mid| \Ri=0 ]
  = \int_{\zX=-\infty}^{\zX=\infty} \zX \pdf{\Vi \Mid| \Ri}( \zX \Mid| 0 ) d\zX
  = 0 \pdf{\Vi \Mid| \Ri}( 0 \Mid| 0 )
  = 0
.\end{gather*}
%|<<|===========================================================================================|<<|
With this, the \MGFs of the (centered) conditioned random variables \(  \Vi \Mid| \Ri  \) and \(  -\Vi \Mid| \Ri  \) are obtained for \(  \sX \in [0,\infty)  \) as follows:
%|>>|===========================================================================================|>>|
\begin{align*}
  \mgf{\Vi \Mid| \Ri=1}( \sX )
  &=
  \E \left[ e^{\sX \Vi} \middle| \Ri=1 \right]
  \\
  &=
  \frac{1}{\sqrt{2\pi}}
  \int_{\zX=-\infty}^{\zX=\infty}
  e^{\sX \zX}
  e^{-\frac{1}{2} \zX^{2}}
  d\zX
  \\
  &=
  e^{\frac{1}{2} \sX^{2}}
  \frac{1}{\sqrt{2\pi}}
  \int_{\zX=-\infty}^{\zX=\infty}
  e^{-\frac{1}{2} (\zX-\sX)^{2}}
  d\zX
  \\ \TagEqn{\label{eqn:pf:eqn:lemma:concentration-ineq:noiseless:pr:2:1a}}
  &=
  e^{\frac{1}{2} \sX^{2}}
  ,\\
  &\dCmt{by evaluating the density of a mean-\(  \sX  \), variance-\(  1  \)}
  \\
  &\dCmtIndent \text{Gaussian random variable over its entire support}
\end{align*}
%|<<|===========================================================================================|<<|
and by an analogous derivation,
%|>>|===========================================================================================|>>|
\begin{align*}
  \mgf{-\Vi \Mid| \Ri=1}( \sX )
  &=
  e^{\frac{1}{2} \sX^{2}}
\TagEqn{\label{eqn:pf:eqn:lemma:concentration-ineq:noiseless:pr:2:1b}}
.\end{align*}
%|<<|===========================================================================================|<<|
%
\XXX{and likewise,
%|>>|===========================================================================================|>>|
\begin{align*}
  \mgf{-\Vi \Mid| \Ri=1}( \sX )
  &=
  \E \left[ e^{-\sX \Vi} \middle| \Ri=1 \right]
  \\
  &=
  \frac{1}{\sqrt{2\pi}}
  \int_{\zX=-\infty}^{\zX=\infty}
  e^{-\sX \zX}
  e^{-\frac{1}{2} \zX^{2}}
  d\zX
  \\
  &=
  e^{\frac{1}{2} \sX^{2}}
  \frac{1}{\sqrt{2\pi}}
  \int_{\zX=-\infty}^{\zX=\infty}
  e^{-\frac{1}{2} (\zX+\sX)^{2}}
  d\zX
  \\ \TagEqn{\label{eqn:pf:eqn:lemma:concentration-ineq:noiseless:pr:2:1b}}
  &=
  e^{\frac{1}{2} \sX^{2}}
  .\\
  &\dCmt{by evaluating the density of a mean-\(  ( -\sX )  \), variance-\(  1  \)}
  \\
  &\dCmtIndent \text{Gaussian random variable over its entire support}
\end{align*}
%|<<|===========================================================================================|<<|
}
%
Additionally,
%|>>|===========================================================================================|>>|
\begin{gather}
  \label{eqn:pf:eqn:lemma:concentration-ineq:noiseless:pr:2:2a}
  \mgf{\Vi \Mid| \Ri=0}( \sX )
  =
  \E \left[ e^{\sX \Vi} \middle| \Ri=0 \right]
  =
  \int_{\vX=-\infty}^{\vX=\infty} e^{\sX \vX} \pdf{\Vi \Mid| \Ri}( \vX \Mid| 0 ) d\vX
  =
  e^{\sX \cdot 0}
  =
  1
  ,\\ \label{eqn:pf:eqn:lemma:concentration-ineq:noiseless:pr:2:2b}
  \mgf{-\Vi \Mid| \Ri=0}( \sX )
  =
  \E \left[ e^{-\sX \Vi} \middle| \Ri=0 \right]
  =
  \int_{\vX=-\infty}^{\vX=\infty} e^{-\sX \vX} \pdf{\Vi \Mid| \Ri}( \vX \Mid| 0 ) d\vX
  =
  e^{-\sX \cdot 0}
  =
  1
.\end{gather}
%|<<|===========================================================================================|<<|
It follows that the \MGF of \(  \Vi  \) is bounded from above by
%|>>|===========================================================================================|>>|
\begin{align*}
  \mgf{\Vi}( \sX )
  &=
  \E \left[ e^{\sX \Vi} \right]
  \\
  &=
  \pdf{\Ri}(1)
  \E \left[ e^{\sX \Vi} \middle| \Ri=1 \right]
  +
  \pdf{\Ri}(0)
  \E \left[ e^{\sX \Vi} \middle| \Ri=0 \right]
  \\
  &\dCmt{by the law of total expectation}
  \\
  &=
  \pdf{\Ri}(1)
  \mgf{\Vi \Mid| \Ri=1}( \sX )
  +
  \pdf{\Ri}(0)
  \mgf{\Vi \Mid| \Ri=0}( \sX )
  \\
  &\dCmt{by the definitions of \(  \mgf{\Vi \Mid| \Ri=1}, \mgf{\Vi \Mid| \Ri=0}  \)}
\XXX{
  \\
  &=
  \frac{1}{\pi} \ADIST
  \mgf{\Vi \Mid| \Ri=1}( \sX )
  +
  \left( 1 - \frac{1}{\pi} \ADIST \right)
  \mgf{\Vi \Mid| \Ri=0}( \sX )
  \\
  &\dCmt{by \EQUATION \eqref{eqn:pf:eqn:lemma:concentration-ineq:noiseless:pr:1:f_Ri}}
}
  \\
  &\leq
  \frac{1}{\pi} \ADIST
  e^{\frac{1}{2} \sX^{2}}
  +
  \left( 1 - \frac{1}{\pi} \ADIST \right)
  \\
  &\dCmt{by \EQUATIONS \eqref{eqn:pf:eqn:lemma:concentration-ineq:noiseless:pr:1:f_Ri}, \eqref{eqn:pf:eqn:lemma:concentration-ineq:noiseless:pr:2:1a}, and \eqref{eqn:pf:eqn:lemma:concentration-ineq:noiseless:pr:2:2a}}
  \\
  &=
  1 + \frac{1}{\pi} \ADIST \left( e^{\frac{1}{2} \sX^{2}} - 1 \right)
\TagEqn{\label{eqn:pf:eqn:lemma:concentration-ineq:noiseless:pr:2:3a}}
,\end{align*}
%|<<|===========================================================================================|<<|
and likewise, the \MGF of \(  -\Vi  \) is bounded by %at \(  \sX \in \R  \) by
%|>>|===========================================================================================|>>|
\begin{align*}
  \mgf{-\Vi}( \sX )
  &\leq
  1 + \frac{1}{\pi} \ADIST \left( e^{\frac{1}{2} \sX^{2}} - 1 \right)
\TagEqn{\label{eqn:pf:eqn:lemma:concentration-ineq:noiseless:pr:2:3b}}
.\end{align*}
%|<<|===========================================================================================|<<|
%
\XXX{and
%|>>|===========================================================================================|>>|
\begin{align*}
  \mgf{-\Vi}( \sX )
  &=
  \E \left[ e^{-\sX \Vi} \right]
  \\
  &=
  \pdf{\Ri}(1)
  \E \left[ e^{-\sX \Vi} \middle| \Ri=1 \right]
  +
  \pdf{\Ri}(0)
  \E \left[ e^{-\sX \Vi} \middle| \Ri=0 \right]
  \\
  &\dCmt{by the law of total expectation}
  \\
  &=
  \pdf{\Ri}(1)
  \mgf{-\Vi \Mid| \Ri=1}( \sX )
  +
  \pdf{\Ri}(0)
  \mgf{-\Vi \Mid| \Ri=0}( \sX )
  \\
  &\dCmt{by the definitions of \(  \mgf{-\Vi \Mid| \Ri=1}, \mgf{-\Vi \Mid| \Ri=0}  \)}
  \\
  &=
  \frac{1}{\pi} \ADIST
  \mgf{-\Vi \Mid| \Ri=1}( \sX )
  +
  \left( 1 - \frac{1}{\pi} \ADIST \right)
  \mgf{-\Vi \Mid| \Ri=0}( \sX )
  \\
  &\dCmt{by \EQUATION \eqref{eqn:pf:eqn:lemma:concentration-ineq:noiseless:pr:1:f_Ri}}
  \\
  &\leq
  \frac{1}{\pi} \ADIST
  e^{\frac{1}{2} \sX^{2}}
  +
  \left( 1 - \frac{1}{\pi} \ADIST \right)
  \\
  &\dCmt{by \EQUATIONS \eqref{eqn:pf:eqn:lemma:concentration-ineq:noiseless:pr:2:1b} and \eqref{eqn:pf:eqn:lemma:concentration-ineq:noiseless:pr:2:2b}}
  \\
  &=
  1 + \frac{1}{\pi} \ADIST \left( e^{\frac{1}{2} \sX^{2}} - 1 \right)
\TagEqn{\label{eqn:pf:eqn:lemma:concentration-ineq:noiseless:pr:2:3b}}
.\end{align*}
%|<<|===========================================================================================|<<|
}
%
Let \(  \VRV \defeq \sum_{\iIx=1}^{\m} \Vi  \), where in expectation,
%|>>|===========================================================================================|>>|
\begin{align*}
  \E[\VRV]
\XXX{
  &=
  \E \left[ \sum_{\iIx=1}^{\m} \Vi \right]
  \\
  &\dCmt{by the definition of \(  \VRV  \)}
  \\
}
  &=
  \sum_{\iIx=1}^{\m} \E[\Vi]
  \\
  &\dCmt{by by the definition of \(  \VRV  \) and the linearity of expectation}
  \\
  &=
  \sum_{\iIx=1}^{\m} \pdf{\Ri}(0) \E[ \Vi \Mid| \Ri=0 ] + \pdf{\Ri}(1) \E[ \Vi \Mid| \Ri=1 ]
  \\
  &\dCmt{by the law of total expectation}
  \\
  &=
  \sum_{\iIx=1}^{\m} \pdf{\Ri}(0) \cdot 0 + \pdf{\Ri}(1) \cdot 0
  \\
  &\dCmt{as argued earlier}
  \\
  &=
  0
.\end{align*}
%|<<|===========================================================================================|<<|
Then, the \MGFs of the (centered) random variables \(  \VRV  \) and \(  -\VRV  \) are given and upper bounded at \(  \sX \in [0,\infty)  \) as follows:
%|>>|===========================================================================================|>>|
\begin{align*}
  \mgf{\VRV}( \sX )
  &=
  % \E \left[ e^{\sX \VRV} \right]
  % \\
  % &\dCmt{by the definition of \MGFs}
  % \\
  % &=
  % \E \left[ e^{\sX \sum_{\iIx=1}^{\m} \Vi} \right]
  % \\
  % &\dCmt{by the definition of \(  \VRV  \)}
  % \\
  % &=
  % \E \left[ \prod_{\iIx=1}^{\m} e^{\sX \Vi} \right]
  % \\
  % &\dCmt{by standard facts about exponents}
  % \\
  % &=
  % \prod_{\iIx=1}^{\m}
  % \E \left[ e^{\sX \Vi} \right]
  % \\
  % &\dCmt{\(  \Vi[1], \dots, \Vi[\m]  \) are mutually independent}
  % \\
  % &=
  % \prod_{\iIx=1}^{\m}
  % \mgf{\Vi}( \sX )
  % \\
  % &\dCmt{by the definition of \(  \mgf{\Vi}  \), \(  \iIx \in [\m]  \)}
  % \\
  %&=
  \left( \mgf{\Vi}( \sX ) \right)^{\m}
  \\
  &\dCmt{for any \(  \iIx \in [\m]  \);}
  \\
  &\dCmt{since \(  \Vi[1], \dots, \Vi[\m]  \) are identically distributed}
  \\
  &\leq
  \left( 1 + \frac{1}{\pi} \ADIST \left( e^{\frac{1}{2} \sX^{2}} - 1 \right) \right)^{\m}
  \\
  &\dCmt{by \EQUATION \eqref{eqn:pf:eqn:lemma:concentration-ineq:noiseless:pr:2:3a}}
  \\ \TagEqn{\label{eqn:pf:eqn:lemma:concentration-ineq:noiseless:pr:2:4a}}
  &\leq
  e^{\frac{1}{\pi} \m \ADIST ( e^{\frac{1}{2} \sX^{2}} - 1 )}
  ,\\
  &\dCmt{by a well-known inequality, \(  \log( 1+u ) \leq u  \) for \(  u > -1  \)}
  % &\dCmt{\(  {\textstyle \log ( 1 + \frac{1}{\pi} \ADIST ( e^{\frac{1}{2} \sX^{2}} - 1 ) ) \leq \frac{1}{\pi} \ADIST ( e^{\frac{1}{2} \sX^{2}} - 1 )}  \)}
  % \\
  % &\dCmtIndent\text{(by a well-known inequality)}
\end{align*}
%|<<|===========================================================================================|<<|
and likewise,
%|>>|===========================================================================================|>>|
\begin{align*}
  \mgf{-\VRV}( \sX )
  % &=
  % \E \left[ e^{-\sX \VRV} \right]
  % \\
  % &\dCmt{by the definition of \MGFs}
  % \\
  % &=
  % \E \left[ e^{\sX \sum_{\iIx=1}^{\m} -\Vi} \right]
  % \\
  % &\dCmt{by the definition of \(  \VRV  \)}
  % \\
  % &=
  % \E \left[ \prod_{\iIx=1}^{\m} e^{-\sX \Vi} \right]
  % \\
  % &\dCmt{by standard facts about exponents}
  % \\
  % &=
  % \prod_{\iIx=1}^{\m}
  % \E \left[ e^{-\sX \Vi} \right]
  % \\
  % &\dCmt{\(  -\Vi[1], \dots, -\Vi[\m]  \) are mutually independent}
  % \\
  % &=
  % \prod_{\iIx=1}^{\m}
  % \mgf{-\Vi}( \sX )
  % \\
  % &\dCmt{by the definition of \(  \mgf{-\Vi}  \), \(  \iIx \in [\m]  \)}
  % \\
\XXX{
  &=
  \left( \mgf{-\Vi}( \sX ) \right)^{\m}
  \\
  &\dCmt{for any \(  \iIx \in [\m]  \);}
  \\
  &\dCmt{since \(  -\Vi[1], \dots, -\Vi[\m]  \) are identically distributed}
  \\
  &\leq
  \left( 1 + \frac{1}{\pi} \ADIST \left( e^{\frac{1}{2} \sX^{2}} - 1 \right) \right)^{\m}
  \\
  &\dCmt{by \EQUATION \eqref{eqn:pf:eqn:lemma:concentration-ineq:noiseless:pr:2:3b}}
  \\
}
  &\leq
  e^{\frac{1}{\pi} \m \ADIST ( e^{\frac{1}{2} \sX^{2}} - 1 )}
  \TagEqn{\label{eqn:pf:eqn:lemma:concentration-ineq:noiseless:pr:2:4b}}
\XXX{  .\\
  &\dCmt{\(  {\textstyle \log ( 1 + \frac{1}{\pi} \ADIST ( e^{\frac{1}{2} \sX^{2}} - 1 ) ) \leq \frac{1}{\pi} \ADIST ( e^{\frac{1}{2} \sX^{2}} - 1 )}  \)}
  \\
  &\dCmtIndent\text{(by a well-known inequality)}
}
.\end{align*}
%|<<|===========================================================================================|<<|
Now, observe:
%|>>|===========================================================================================|>>|
\begin{align*}
  &
  \Pr \left(
    \frac{\VRV}{\m} - \E \left[ \frac{\VRV}{\m} \right] > \frac{1}{\pi} \tX \ADIST
  \middle|
    \Ri=1
  \right)
  \\
  &\AlignSp\leq
  \inf_{\sX \geq 0}
  e^{-\frac{1}{\pi} \m \sX \tX \ADIST}
  \mgf{\VRV}( \sX )
  \\
  &\AlignSp\dCmt{due to Bernstein (\see e.g., \cite{vershynin2018high})}
  \\
  &\AlignSp\leq
  \inf_{\sX \geq 0}
  e^{-\frac{1}{\pi} \m \sX \tX \ADIST}
  e^{\frac{1}{\pi} \m \ADIST ( e^{\frac{1}{2} \sX^{2}} - 1 )}
  \\
  &\AlignSp\dCmt{by \EQUATION \eqref{eqn:pf:eqn:lemma:concentration-ineq:noiseless:pr:2:4a}}
  \\ \TagEqn{\label{eqn:pf:eqn:lemma:concentration-ineq:noiseless:pr:2:5a}}
  &\AlignSp\leq
  e^{\frac{1}{3\pi} \m \tX^{2} \ADIST}
  ,\\
  &\AlignSp\dCmt{as argued earlier in the proof of \EQUATION \eqref{eqn:lemma:concentration-ineq:noiseless:pr:1}}
\end{align*}
%|<<|===========================================================================================|<<|
and on the other side:
%|>>|===========================================================================================|>>|
\begin{align*}
  &
  \Pr \left(
    \frac{\VRV}{\m} - \E \left[ \frac{\VRV}{\m} \right] < -\frac{1}{\pi} \tX \ADIST
  \middle|
    \Ri=1
  \right)
\XXX{
  \\
  &\AlignSp=
  \Pr \left(
    -\frac{\VRV}{\m} - \E \left[ -\frac{\VRV}{\m} \right] > \frac{1}{\pi} \tX \ADIST
  \middle|
    \Ri=1
  \right)
  \\
  &\AlignSp\leq
  \inf_{\sX \geq 0}
  e^{-\frac{1}{\pi} \m \sX \tX \ADIST}
  \mgf{-\VRV}( \sX )
  \\
  &\AlignSp\dCmt{due to Bernstein (\see e.g., \cite{vershynin2018high})}
  \\
  &\AlignSp\leq
  \inf_{\sX \geq 0}
  e^{-\frac{1}{\pi} \m \sX \tX \ADIST}
  e^{\frac{1}{\pi} \m \ADIST ( e^{\frac{1}{2} \sX^{2}} - 1 )}
  \\
  &\AlignSp\dCmt{by \EQUATION \eqref{eqn:pf:eqn:lemma:concentration-ineq:noiseless:pr:2:4b}}
  \\
  &\AlignSp\leq
}
  \leq
  e^{\frac{1}{3\pi} \m \tX^{2} \ADIST}
  \TagEqn{\label{eqn:pf:eqn:lemma:concentration-ineq:noiseless:pr:2:5b}}
\XXX{
  .\\
  &\AlignSp\dCmt{as argued earlier in the proof of \EQUATION \eqref{eqn:lemma:concentration-ineq:noiseless:pr:1}}
}
.\end{align*}
%|<<|===========================================================================================|<<|
By a union bound over the above pair of inequalities in \EQUATIONS \eqref{eqn:pf:eqn:lemma:concentration-ineq:noiseless:pr:2:5a} and \eqref{eqn:pf:eqn:lemma:concentration-ineq:noiseless:pr:2:5b},
%|>>|===========================================================================================|>>|
\begin{gather*}
  \Pr \left( \left| \frac{\VRV}{\m} - \E \left[ \frac{\VRV}{\m} \right] \right| > \frac{1}{\pi} \tX \ADIST \right)
  \leq
  2 e^{-\frac{1}{3\pi} \m \tX^{2} \ADIST}
,\end{gather*}
%|<<|===========================================================================================|<<|
and therefore, recalling the definitions of the random variables \(  \Vi  \), \(  \iIx \in [\m]  \), and their relationship to \(  \hFn[\JCoords]( \thetaStar, \thetaX )  \), the above concentration inequality further implies that
%|>>|===========================================================================================|>>|
\begin{gather*}
  \textstyle
  \Pr \left(
    \left| \left\langle \frac{\hFn[\JCoords]( \thetaStar, \thetaX )}{\sqrt{2\pi}}, \frac{\thetaStar+\thetaX}{\| \thetaStar+\thetaX \|_{2}} \right\rangle - \E \left[ \left\langle \frac{\hFn[\JCoords]( \thetaStar, \thetaX )}{\sqrt{2\pi}}, \frac{\thetaStar+\thetaX}{\| \thetaStar+\thetaX \|_{2}} \right\rangle \right] \right|
    >
    \frac{1}{\pi} \tX \ADIST
  \right)
  \nonumber \\
  \leq
  2 e^{-\frac{1}{3\pi} \m \tX^{2} \ADIST}
.\end{gather*}
%|<<|===========================================================================================|<<|
Lastly, by union bounding over all \(  \JCoords \in \JS  \) and \(  \thetaX \in \ParamCoverX  \), \EQUATION \eqref{eqn:lemma:concentration-ineq:noiseless:pr:2} follows:
%|>>|===========================================================================================|>>|
\begin{gather*}
  \textstyle
  \Pr \left(
    \ExistsST{\JCoords \in \JS, \thetaX \in \ParamCoverX}{
    \left| \left\langle \frac{\hFn[\JCoords]( \thetaStar, \thetaX )}{\sqrt{2\pi}}, \frac{\thetaStar+\thetaX}{\| \thetaStar+\thetaX \|_{2}} \right\rangle - \E \left[ \left\langle \frac{\hFn[\JCoords]( \thetaStar, \thetaX )}{\sqrt{2\pi}}, \frac{\thetaStar+\thetaX}{\| \thetaStar+\thetaX \|_{2}} \right\rangle \right] \right|
    >
    \frac{1}{\pi} \tX \ADIST
    }
  \right)
  \nonumber \\
  \leq
  2 | \JS | | \ParamCoverX | e^{-\frac{1}{3\pi} \m \tX^{2} \ADIST}
.\end{gather*}
%|<<|===========================================================================================|<<|
%
\let\vX\oldvX%
%
%%%%%%%%%%%%%%%%%%%%%%%%%%%%%%%%%%%%%%%%%%%%%%%%%%%%%%%%%%%%%%%%%%%%%%%%%%%%%%%%%%%%%%%%%%%%%%%%%%%%
\paragraph{Verification of \EQUATION \eqref{eqn:lemma:concentration-ineq:noiseless:pr:3}} %%%%%%%%%%
%%%%%%%%%%%%%%%%%%%%%%%%%%%%%%%%%%%%%%%%%%%%%%%%%%%%%%%%%%%%%%%%%%%%%%%%%%%%%%%%%%%%%%%%%%%%%%%%%%%%
%
Towards deriving the third concentration inequality, \EQUATION \eqref{eqn:lemma:concentration-ineq:noiseless:pr:3}, consider an orthonormal basis,
%|>>|:::::::::::::::::::::::::::::::::::::::::::::::::::::::::::::::::::::::::::::::::::::::::::|>>|
\(  \{ \Vec{\vV}\VIx{1}, \dots, \Vec{\vV}\VIx{\kX} \} \subset \R^{\n}  \),
%|<<|:::::::::::::::::::::::::::::::::::::::::::::::::::::::::::::::::::::::::::::::::::::::::::|<<|
for the subspace
%|>>|:::::::::::::::::::::::::::::::::::::::::::::::::::::::::::::::::::::::::::::::::::::::::::|>>|
\(  \Set{V} \defeq \{ \Vec{v} \in \R^{\n} : \Supp( \Vec{v} ) \subseteq \Supp( \thetaStar ) \cup \Supp( \thetaX ) \cup \JCoords \}  \),
%|<<|:::::::::::::::::::::::::::::::::::::::::::::::::::::::::::::::::::::::::::::::::::::::::::|<<|
where
%|>>|:::::::::::::::::::::::::::::::::::::::::::::::::::::::::::::::::::::::::::::::::::::::::::|>>|
\(  \kX \defeq | \Supp( \thetaStar ) \cup \Supp( \thetaX ) \cup \JCoords |  \),
%|<<|:::::::::::::::::::::::::::::::::::::::::::::::::::::::::::::::::::::::::::::::::::::::::::|<<|
and where
%|>>|:::::::::::::::::::::::::::::::::::::::::::::::::::::::::::::::::::::::::::::::::::::::::::|>>|
\(  \Vec{\vV}\VIx{\kX-1} \defeq \frac{\thetaStar-\thetaX}{\| \thetaStar-\thetaX \|_{2}}  \) and
\(  \Vec{\vV}\VIx{\kX}   \defeq \frac{\thetaStar+\thetaX}{\| \thetaStar+\thetaX \|_{2}}  \).
%|<<|:::::::::::::::::::::::::::::::::::::::::::::::::::::::::::::::::::::::::::::::::::::::::::|<<|
Then, the orthogonal decomposition of \(  \frac{1}{\sqrt{2\pi}} \gFn[\JCoords]( \thetaStar, \thetaX )  \) using this basis is given and subsequently rewritten as follows:
%|>>|===========================================================================================|>>|
\begin{align*}
  % &
  \gFn[\JCoords]( \thetaStar, \thetaX )
  % \\
  &=
  \sum_{\jIx=1}^{\kX}
  \left\langle \frac{\gFn[\JCoords]( \thetaStar, \thetaX )}{\sqrt{2\pi}} , \Vec{\vV}\VIx{\jIx} \right\rangle \Vec{\vV}\VIx{\jIx}
  \\
  % &=
  % \sum_{\jIx=1}^{\kX}
  % \left\langle
  %   \frac{1}{\sqrt{2\pi}} \hFn[\JCoords]( \thetaStar, \thetaX )
  %   -
  %   \left\langle \frac{1}{\sqrt{2\pi}} \hFn[\JCoords]( \thetaStar, \thetaX ), \frac{\thetaStar-\thetaX}{\| \thetaStar-\thetaX \|_{2}} \right\rangle
  %   \frac{\thetaStar-\thetaX}{\| \thetaStar-\thetaX \|_{2}}
  %   -
  %   \left\langle \frac{1}{\sqrt{2\pi}} \hFn[\JCoords]( \thetaStar, \thetaX ), \frac{\thetaStar+\thetaX}{\| \thetaStar+\thetaX \|_{2}} \right\rangle
  %   \frac{\thetaStar+\thetaX}{\| \thetaStar+\thetaX \|_{2}}
  %   ,
  %   \Vec{\vV}\VIx{\jIx}
  % \right\rangle
  % \Vec{\vV}\VIx{\jIx}
  % \\
  % &\dCmt{by the definition of \(  \gFn[\JCoords]  \) in \EQUATION \eqref{eqn:notations:gJ:def}}
  % \\
  &=
  \sum_{\jIx=1}^{\kX}
  \left\langle
    \frac{\hFn[\JCoords]( \thetaStar, \thetaX )}{\sqrt{2\pi}} 
    -
    \left\langle \frac{\hFn[\JCoords]( \thetaStar, \thetaX )}{\sqrt{2\pi}} , \Vec{\vV}\VIx{\kX-1} \right\rangle
    \Vec{\vV}\VIx{\kX-1}
    -
    \left\langle \frac{\hFn[\JCoords]( \thetaStar, \thetaX )}{\sqrt{2\pi}} , \Vec{\vV}\VIx{\kX} \right\rangle
    \Vec{\vV}\VIx{\kX}
    ,
    \Vec{\vV}\VIx{\jIx}
  \right\rangle
  \Vec{\vV}\VIx{\jIx}
  \\
  &\dCmt{by the choice of \(  \Vec{\vV}\VIx{\kX-1} = \tfrac{\thetaStar-\thetaX}{\| \thetaStar-\thetaX \|_{2}}, \Vec{\vV}\VIx{\kX}   = \tfrac{\thetaStar+\thetaX}{\| \thetaStar+\thetaX \|_{2}}  \)}
  \\
  % &=
  % \sum_{\jIx=1}^{\kX}
  % \left\langle
  %   \frac{1}{\sqrt{2\pi}} \hFn[\JCoords]( \thetaStar, \thetaX ),
  %   \Vec{\vV}\VIx{\jIx}
  % \right\rangle
  % \Vec{\vV}\VIx{\jIx}
  % -
  % \sum_{\jIx=1}^{\kX}
  % \left\langle \frac{1}{\sqrt{2\pi}} \hFn[\JCoords]( \thetaStar, \thetaX ), \Vec{\vV}\VIx{\kX-1} \right\rangle
  % \left\langle
  %   \Vec{\vV}\VIx{\kX-1},
  %   \Vec{\vV}\VIx{\jIx}
  % \right\rangle
  % \Vec{\vV}\VIx{\jIx}
  % -
  % \sum_{\jIx=1}^{\kX}
  % \left\langle \frac{1}{\sqrt{2\pi}} \hFn[\JCoords]( \thetaStar, \thetaX ), \Vec{\vV}\VIx{\kX} \right\rangle
  % \left\langle
  %   \Vec{\vV}\VIx{\kX},
  %   \Vec{\vV}\VIx{\jIx}
  % \right\rangle
  % \Vec{\vV}\VIx{\jIx}
  % \\
  % &\dCmt{due to the linearity of inner products}
  % \\
  &=
  \sum_{\jIx=1}^{\kX}
  \left\langle
    \frac{\hFn[\JCoords]( \thetaStar, \thetaX )}{\sqrt{2\pi}} ,
    \Vec{\vV}\VIx{\jIx}
  \right\rangle
  \Vec{\vV}\VIx{\jIx}
  -
  \left\langle \frac{\hFn[\JCoords]( \thetaStar, \thetaX )}{\sqrt{2\pi}} , \Vec{\vV}\VIx{\kX-1} \right\rangle
  \Vec{\vV}\VIx{\kX-1}
  -
  \left\langle \frac{\hFn[\JCoords]( \thetaStar, \thetaX )}{\sqrt{2\pi}} , \Vec{\vV}\VIx{\kX} \right\rangle
  \Vec{\vV}\VIx{\kX}
  \\
  &\dCmt{due to the orthogonality of \(  \Vec{\vV}\VIx{1}, \dots, \Vec{\vV}\VIx{\kX}  \)}
  \\
  &=
  \sum_{\jIx=1}^{\kX-2}
  \left\langle
    \frac{\hFn[\JCoords]( \thetaStar, \thetaX )}{\sqrt{2\pi}} ,
    \Vec{\vV}\VIx{\jIx}
  \right\rangle
  \Vec{\vV}\VIx{\jIx}
  % \\
  % &\dCmt{by canceling terms}
  % \\
  % &=
  % \sum_{\jIx=1}^{\kX-2}
  % \left\langle
  %   \frac{1}{\m}
  %   \sum_{\iIx=1}^{\m}
  %   \CovVX\VIx{\iIx}
  %  \sep \Sign( \langle \CovVX\VIx{\iIx}, \thetaStar \rangle )
  %  \sep \I( \Sign( \langle \CovVX\VIx{\iIx}, \thetaStar \rangle ) \neq \Sign( \langle \CovVX\VIx{\iIx}, \thetaX \rangle ) )
  %   ,
  %   \Vec{\vV}\VIx{\jIx}
  % \right\rangle
  % \Vec{\vV}\VIx{\jIx}
  % \\
  % &\dCmt{by \EQUATION \eqref{eqn:pf:eqn:lemma:concentration-ineq:noiseless:pr:5}}
  \\
  &=
  \frac{1}{\m}
  \sum_{\iIx=1}^{\m}
  \sum_{\jIx=1}^{\kX-2}
  \langle \CovVX\VIx{\iIx}, \Vec{\vV}\VIx{\jIx} \rangle
  \Vec{\vV}\VIx{\jIx}
 \sep
  \Sign( \langle \CovVX\VIx{\iIx}, \thetaStar \rangle )
 \sep
  \I( \Sign( \langle \CovVX\VIx{\iIx}, \thetaStar \rangle ) \neq \Sign( \langle \CovVX\VIx{\iIx}, \thetaX \rangle ) )
  .\\
  % &\dCmt{by the linearity of inner products}
  &\dCmt{by \EQUATION \eqref{eqn:pf:eqn:lemma:concentration-ineq:noiseless:pr:5}}
\end{align*}
%|<<|===========================================================================================|<<|
%
%%%%%%%%%%%%%%%%%%%%%%%%%%%%%%%%%%%%%%%%%%%%%%%%%%%%%%%%%%%%%%%%%%%%%%%%%%%%%%%%%%%%%%%%%%%%%%%%%%%%
\par %%%%%%%%%%%%%%%%%%%%%%%%%%%%%%%%%%%%%%%%%%%%%%%%%%%%%%%%%%%%%%%%%%%%%%%%%%%%%%%%%%%%%%%%%%%%%%%
%%%%%%%%%%%%%%%%%%%%%%%%%%%%%%%%%%%%%%%%%%%%%%%%%%%%%%%%%%%%%%%%%%%%%%%%%%%%%%%%%%%%%%%%%%%%%%%%%%%%
%
Note that
%|>>|:::::::::::::::::::::::::::::::::::::::::::::::::::::::::::::::::::::::::::::::::::::::::::|>>|
\(  \thetaStar, \thetaX \in \Span( \{ \Vec{\vV}\VIx{\kX-1}, \Vec{\vV}\VIx{\kX} \} )  \),
%|<<|:::::::::::::::::::::::::::::::::::::::::::::::::::::::::::::::::::::::::::::::::::::::::::|<<|
which implies by the orthogonality of the set
%|>>|:::::::::::::::::::::::::::::::::::::::::::::::::::::::::::::::::::::::::::::::::::::::::::|>>|
\(  \{ \Vec{\vV}\VIx{1}, \dots, \Vec{\vV}\VIx{\kX} \}  \)
%|<<|:::::::::::::::::::::::::::::::::::::::::::::::::::::::::::::::::::::::::::::::::::::::::::|<<|
that
%|>>|:::::::::::::::::::::::::::::::::::::::::::::::::::::::::::::::::::::::::::::::::::::::::::|>>|
\(   \thetaStar, \thetaX \perp \Vec{\vV}\VIx{\jIx}  \)
%|<<|:::::::::::::::::::::::::::::::::::::::::::::::::::::::::::::::::::::::::::::::::::::::::::|<<|
for every \(  \jIx \in [\kX-2]  \).
Thus, applying standard facts about Gaussians, for each \(  \iIx \in [\m]  \), \(  \jIx \in [\kX-2]  \), there is an equivalence in distribution:
%|>>|===========================================================================================|>>|
\begin{gather*}
  \langle \CovVX\VIx{\iIx}, \Vec{\vV}\VIx{\jIx} \rangle
  \sep
  \Sign( \langle \CovVX\VIx{\iIx}, \thetaStar \rangle )
  \sep
  \I( \Sign( \langle \CovVX\VIx{\iIx}, \thetaStar \rangle ) \neq \Sign( \langle \CovVX\VIx{\iIx}, \thetaX \rangle ) )
  \sim
  \Wij
  \defeq
  \Zij \Yi
  %\Sign( \Zij[\iIx][\kX-1] ) \I( \Sign( \Zij[\iIx][\kX-1] ) \neq \Sign( \Zij[\iIx][\kX] ) )
,\end{gather*}
%|<<|===========================================================================================|<<|
where for each \(  \iIx \in [\m]  \), \(  \jIx \in [\kX]  \), the random variable
%|>>|:::::::::::::::::::::::::::::::::::::::::::::::::::::::::::::::::::::::::::::::::::::::::::|>>|
\(  \Zij \sim \N(0,1)  \)
%|<<|:::::::::::::::::::::::::::::::::::::::::::::::::::::::::::::::::::::::::::::::::::::::::::|<<|
is standard Gaussian and
%|>>|:::::::::::::::::::::::::::::::::::::::::::::::::::::::::::::::::::::::::::::::::::::::::::|>>|
% \(  \Yi \defeq \Sign( \Zij[\iIx][\kX-1] ) \sep \I( \Sign( \Zij[\iIx][\kX-1] ) \neq \Sign( \Zij[\iIx][\kX] ) )  \).
\begin{gather*}
  \Yi \defeq \Sign( \Zij[\iIx][\kX-1] ) \sep \I( \Sign( \Zij[\iIx][\kX-1] ) \neq \Sign( \Zij[\iIx][\kX] ) )
.\end{gather*}
%|<<|:::::::::::::::::::::::::::::::::::::::::::::::::::::::::::::::::::::::::::::::::::::::::::|<<|
Notice that the random variables
%|>>|:::::::::::::::::::::::::::::::::::::::::::::::::::::::::::::::::::::::::::::::::::::::::::|>>|
\(  \{ \Zij \}_{\iIx \in [\m], \jIx \in [\kX-2]}  \) %\(  \iIx \in [\m]  \), \(  \jIx \in [\kX-2]  \),
%|<<|:::::::::::::::::::::::::::::::::::::::::::::::::::::::::::::::::::::::::::::::::::::::::::|<<|
are \iid and also independent of
%|>>|:::::::::::::::::::::::::::::::::::::::::::::::::::::::::::::::::::::::::::::::::::::::::::|>>|
\(  \Zij[\iIx][\kX-1]  \), \(  \Zij[\iIx][\kX]  \), and \(  \Yi  \), \(  \iIx \in [\m]  \).
%|<<|:::::::::::::::::::::::::::::::::::::::::::::::::::::::::::::::::::::::::::::::::::::::::::|<<|
Moreover,
%|>>|===========================================================================================|>>|
\begin{gather*}
  \frac{1}{\m}
  \sum_{\iIx=1}^{\m}
  \sum_{\jIx=1}^{\kX-2}
  \langle \CovVX\VIx{\iIx}, \Vec{\vV}\VIx{\jIx} \rangle
  \Vec{\vV}\VIx{\jIx}
  \sep
  \Sign( \langle \CovVX\VIx{\iIx}, \thetaStar \rangle )
  \sep
  \I( \Sign( \langle \CovVX\VIx{\iIx}, \thetaStar \rangle ) \neq \Sign( \langle \CovVX\VIx{\iIx}, \thetaX \rangle ) )
  \sim
%  \frac{1}{\m}
%  \sum_{\iIx=1}^{\m}
%  \sum_{\jIx=1}^{\kX-2}
%  \Zij
%  \Yi
%  \Vec{\vV}\VIx{\jIx}
%  =
  \frac{1}{\m}
  \sum_{\iIx=1}^{\m}
  \sum_{\jIx=1}^{\kX-2}
  \Wij
  \Vec{\vV}\VIx{\jIx}
.\end{gather*}
%|<<|===========================================================================================|<<|
Due to the rotational invariance of Gaussians,
%|>>|===========================================================================================|>>|
\begin{align*}
  &
  \langle \CovVX\VIx{\iIx}, \Vec{\vV}\VIx{\jIx} \rangle
  \sep
  \Sign( \langle \CovVX\VIx{\iIx}, \thetaStar \rangle )
  \sep
  \I( \Sign( \langle \CovVX\VIx{\iIx}, \thetaStar \rangle ) \neq \Sign( \langle \CovVX\VIx{\iIx}, \thetaX \rangle ) )
  \\
  &\AlignIndent
  \sim
  \langle \CovVX\VIx{\iIx}, \ej \rangle
  \sep
  \Sign( \langle \CovVX\VIx{\iIx}, \thetaStar \rangle )
  \sep
  \I( \Sign( \langle \CovVX\VIx{\iIx}, \thetaStar \rangle ) \neq \Sign( \langle \CovVX\VIx{\iIx}, \thetaX \rangle ) )
,\end{align*}
%|<<|===========================================================================================|<<|
where
%|>>|:::::::::::::::::::::::::::::::::::::::::::::::::::::::::::::::::::::::::::::::::::::::::::|>>|
\(  \ej \defeq \BVec{\{ \jIx \}} \in \R^{\n}  \)
%|<<|:::::::::::::::::::::::::::::::::::::::::::::::::::::::::::::::::::::::::::::::::::::::::::|<<|
is the \(  \jIx\Th  \) standard basis vector for \(  \R^{\n}  \) in which the \(  \jIx\Th  \) entry is set to \(  1  \) and all other entries are set to \(  0  \).
Hence, \WLOG, the analysis will proceed under the assumption that the first \(  ( \kX-2 )  \)-many \(  \jIx\Th  \) basis vectors are \(  \Vec{\vV}\VIx{\jIx} = \ej  \), \(  \jIx \in [\kX-2]  \).
Under this assumption, the random vector, \(  \Vec{\URV}  \), which is given by
%|>>|===========================================================================================|>>|
\begin{gather*}
  \Vec{\URV}
  \defeq
  \frac{1}{\m}
  \sum_{\iIx=1}^{\m}
  \sum_{\jIx=1}^{\kX-2}
  \Wij \Vec{\vV}\VIx{\jIx}
  =
  \frac{1}{\m}
  \sum_{\iIx=1}^{\m}
  \sum_{\jIx=1}^{\kX-2}
  \Wij \ej
,\end{gather*}
%|<<|===========================================================================================|<<|
has \(  \jIx\Th  \) entries, \(  \jIx \in [\n]  \),
%|>>|===========================================================================================|>>|
\begin{gather*}
  \Vec*{\URV}\VIx{\jIx}
  =
  \begin{cases}
  0                                    ,& \cIf \jIx \in [\n] \setminus [\kX-2] ,\\
  \frac{1}{\m} \sum_{\iIx=1}^{\m} \Wij ,& \cIf \jIx \in [\kX-2]                .
  \end{cases}
\end{gather*}
%|<<|===========================================================================================|<<|
%
%%%%%%%%%%%%%%%%%%%%%%%%%%%%%%%%%%%%%%%%%%%%%%%%%%%%%%%%%%%%%%%%%%%%%%%%%%%%%%%%%%%%%%%%%%%%%%%%%%%%
\par %%%%%%%%%%%%%%%%%%%%%%%%%%%%%%%%%%%%%%%%%%%%%%%%%%%%%%%%%%%%%%%%%%%%%%%%%%%%%%%%%%%%%%%%%%%%%%%
%%%%%%%%%%%%%%%%%%%%%%%%%%%%%%%%%%%%%%%%%%%%%%%%%%%%%%%%%%%%%%%%%%%%%%%%%%%%%%%%%%%%%%%%%%%%%%%%%%%%
%
For \(  \iIx \in [\m]  \), define the random variable
%|>>|:::::::::::::::::::::::::::::::::::::::::::::::::::::::::::::::::::::::::::::::::::::::::::|>>|
\(  \Ri \defeq \I( \Sign( \Zij[\iIx][\kX-1] ) \neq \Sign( \Zij[\iIx][\kX] ) )  \),
%|<<|:::::::::::::::::::::::::::::::::::::::::::::::::::::::::::::::::::::::::::::::::::::::::::|<<|
whose a mass function given at \(  \rX \in \{ 0,1 \}  \) by
%|>>|===========================================================================================|>>|
\begin{gather*}
  \pdf{\Ri}( \rX )
  =
  \begin{cases}
  1-\frac{1}{\pi} \ADIST ,& \cIf \rX=0, \\
  \frac{1}{\pi} \ADIST   ,& \cIf \rX=1.
  \end{cases}
\end{gather*}
%|<<|===========================================================================================|<<|
As in the verification of \EQUATION \eqref{eqn:lemma:concentration-ineq:noiseless:pr:1}, this mass function can be derived by way of an approach similar to that which appears in \cite[{\APPENDIX B.1.1}]{matsumoto2022binary}.
Additionally, write the random vector
%|>>|:::::::::::::::::::::::::::::::::::::::::::::::::::::::::::::::::::::::::::::::::::::::::::|>>|
\(  \Vec{\RRV} \defeq ( \Ri[1], \dots, \Ri[\m] )  \),
%|<<|:::::::::::::::::::::::::::::::::::::::::::::::::::::::::::::::::::::::::::::::::::::::::::|<<|
whose entries are \iid[,] and let
%|>>|:::::::::::::::::::::::::::::::::::::::::::::::::::::::::::::::::::::::::::::::::::::::::::|>>|
\(  \LRV \defeq \| \Vec{\RRV} \|_{0}  \).
%|<<|:::::::::::::::::::::::::::::::::::::::::::::::::::::::::::::::::::::::::::::::::::::::::::|<<|
Because each random variable \(  \Zij  \), \(  \jIx \in [\kX-2]  \), is independent of \(  \Zij[\iIx][\kX-1]  \) and \(  \Zij[\iIx][\kX]  \), it is also independent of \(  \Sign( \Zij[\iIx][\kX-1] )  \) and \(  \Ri  \),
where \(  \Sign( \Zij[\iIx][\kX-1] )  \) follows a Rademacher distribution.
Since mean-\(  0  \) Gaussian random variables have the same distribution as their negations, there are the following equivalences in distribution: \(  -\Zij \sim \Zij \sim \N(0,1)  \) and \(  \Zij \sep \Sign( \Zij[\iIx][\kX-1] ) \sim \Zij \sim \N(0,1)  \) (\see e.g., \cite[{\APPENDIX B}]{matsumoto2022binary} for a formal argument).
Hence,
%|>>|:::::::::::::::::::::::::::::::::::::::::::::::::::::::::::::::::::::::::::::::::::::::::::|>>|
\(  ( \Wij \Mid| \Ri=1 ) \sim \N(0,1)  \).
%|<<|:::::::::::::::::::::::::::::::::::::::::::::::::::::::::::::::::::::::::::::::::::::::::::|<<|
Since the random variables \(  \Wij[1], \dots, \Wij[\m]  \) are \iid[,] it follows that
%|>>|:::::::::::::::::::::::::::::::::::::::::::::::::::::::::::::::::::::::::::::::::::::::::::|>>|
\(  ( \Vec*{\URV}\VIx{\jIx} \Mid| \LRV=\lX ) \sim ( \Vec*{\URV}\VIx{\jIx} \Mid| \Vec{\RRV}=\Vec{\rX} ) \sim \N( 0, \frac{\lX}{\m^{2}} )  \)
%|<<|:::::::::::::::::::::::::::::::::::::::::::::::::::::::::::::::::::::::::::::::::::::::::::|<<|
for each \(  \jIx \in [\kX-2]  \) and an arbitrary choice of \(  \Vec{\rX} \in \{ 0,1 \}^{\m}  \), and where \(  \lX \defeq \| \Vec{\rX} \|_{0}  \).
(A more rigorous analysis can employ the law of total probability.)
Therefore, \(  \Vec{\URV}  \) is a \(  \frac{\sqrt{\lX}}{\m}  \)-\subgaussian random vector with support of cardinality \(  \| \Vec{\URV} \|_{0} = \kX-2  \).
%
%%%%%%%%%%%%%%%%%%%%%%%%%%%%%%%%%%%%%%%%%%%%%%%%%%%%%%%%%%%%%%%%%%%%%%%%%%%%%%%%%%%%%%%%%%%%%%%%%%%%
\par %%%%%%%%%%%%%%%%%%%%%%%%%%%%%%%%%%%%%%%%%%%%%%%%%%%%%%%%%%%%%%%%%%%%%%%%%%%%%%%%%%%%%%%%%%%%%%%
%%%%%%%%%%%%%%%%%%%%%%%%%%%%%%%%%%%%%%%%%%%%%%%%%%%%%%%%%%%%%%%%%%%%%%%%%%%%%%%%%%%%%%%%%%%%%%%%%%%%
%
Before proceeding, two results are introduced to facilitate the proof.
%
%\ToDo{Check the numbering of this lemma in \cite{matsumoto2022binary} (note: this is from the new version).}
%|>>|*******************************************************************************************|>>|
%|>>|*******************************************************************************************|>>|
%|>>|*******************************************************************************************|>>|
\begin{lemma}[{\LEMMA \cite[\LEMMA A.2]{matsumoto2022binary}}]
\label{lemma:count-mismatch}
%
Fix
%|>>|:::::::::::::::::::::::::::::::::::::::::::::::::::::::::::::::::::::::::::::::::::::::::::|>>|
\(  \sXX \in (0,1)  \).
%|<<|:::::::::::::::::::::::::::::::::::::::::::::::::::::::::::::::::::::::::::::::::::::::::::|<<|
Let
%|>>|:::::::::::::::::::::::::::::::::::::::::::::::::::::::::::::::::::::::::::::::::::::::::::|>>|
\(  \Vec{\ZRVX}\VIx{1}, \dots, \Vec{\ZRVX}\VIx{\m} \sim \N( \Vec{0}, \Id{\n} )  \),
%|<<|:::::::::::::::::::::::::::::::::::::::::::::::::::::::::::::::::::::::::::::::::::::::::::|<<|
and let
%|>>|:::::::::::::::::::::::::::::::::::::::::::::::::::::::::::::::::::::::::::::::::::::::::::|>>|
\(  \Vec{\uV}, \Vec{\vV} \in \Sphere{\n}  \).
%|<<|:::::::::::::::::::::::::::::::::::::::::::::::::::::::::::::::::::::::::::::::::::::::::::|<<|
Define the random variable
%|>>|:::::::::::::::::::::::::::::::::::::::::::::::::::::::::::::::::::::::::::::::::::::::::::|>>|
\(  \LRV \defeq | \{ \iIx \in [\m] : \Sign( \langle \Vec{\ZRVX}\VIx{\iIx}, \Vec{\uV} \rangle ) \neq \Sign( \langle \Vec{\ZRVX}\VIx{\iIx}, \Vec{\vV} \rangle ) \} |  \).
%|<<|:::::::::::::::::::::::::::::::::::::::::::::::::::::::::::::::::::::::::::::::::::::::::::|<<|
Then,
%|>>|===========================================================================================|>>|
\begin{gather}
\label{eqn:lemma:count-mismatch:ev}
  \mu_{\LRV} \defeq \E[ \LRV ] = \frac{\m \arccos( \langle \Vec{\uV}, \Vec{\vV} \rangle )}{\pi}
\end{gather}
%|<<|===========================================================================================|<<|
and
%|>>|===========================================================================================|>>|
\begin{gather}
\label{eqn:lemma:count-mismatch:pr}
  \Pr \left( \LRV > ( 1+\sXX ) \mu_{\LRV} \right) \leq e^{-\frac{1}{3\pi} \m \sXX^{2} \arccos( \langle \Vec{\uV}, \Vec{\vV} \rangle )}
.\end{gather}
%|<<|===========================================================================================|<<|
\end{lemma}
%|<<|*******************************************************************************************|<<|
%|<<|*******************************************************************************************|<<|
%|<<|*******************************************************************************************|<<|

%|>>|*******************************************************************************************|>>|
%|>>|*******************************************************************************************|>>|
%|>>|*******************************************************************************************|>>|
\begin{lemma}
\label{lemma:norm-subgaussian:1}
%
Fix \(  \tXXX, \sigma  > 0  \) and \(  0 < \dX \leq \n  \).
Let
%|>>|:::::::::::::::::::::::::::::::::::::::::::::::::::::::::::::::::::::::::::::::::::::::::::|>>|
\(  \ICoords \subseteq [\n]  \), \(  | \ICoords | = \dX  \),
%|<<|:::::::::::::::::::::::::::::::::::::::::::::::::::::::::::::::::::::::::::::::::::::::::::|<<|
and
%|>>|:::::::::::::::::::::::::::::::::::::::::::::::::::::::::::::::::::::::::::::::::::::::::::|>>|
\(  \Vec{\XRV} \sim \N( \Vec{0}, \sigma^{2} \sum_{\jIx \in \ICoords} \ej \ej^{\T} )  \).
%|<<|:::::::::::::::::::::::::::::::::::::::::::::::::::::::::::::::::::::::::::::::::::::::::::|<<|
Then,
%|>>|===========================================================================================|>>|
\begin{gather}
\label{eqn:lemma:norm-subgaussian:1:1}
  \Pr \left(
    \| \Vec{\XRV} - \E[ \Vec{\XRV} ] \|_{2}
    >
    \sqrt{\dX} \sigma
    +
    \tXXX
  \right)
  \leq
  \Pr \left(
    \| \Vec{\XRV} - \E[ \Vec{\XRV} ] \|_{2}
    >
    \E[ \| \Vec{\XRV} \|_{2} ]
    +
    \tXXX
  \right)
  \leq
  e^{-\frac{1}{2 \sigma^{2}} \tXXX^{2}}
.\end{gather}
%|<<|===========================================================================================|<<|
\end{lemma}
%|<<|*******************************************************************************************|<<|
%|<<|*******************************************************************************************|<<|
%|<<|*******************************************************************************************|<<|
%
%|>>|~~~~~~~~~~~~~~~~~~~~~~~~~~~~~~~~~~~~~~~~~~~~~~~~~~~~~~~~~~~~~~~~~~~~~~~~~~~~~~~~~~~~~~~~~~~|>>|
%|>>|~~~~~~~~~~~~~~~~~~~~~~~~~~~~~~~~~~~~~~~~~~~~~~~~~~~~~~~~~~~~~~~~~~~~~~~~~~~~~~~~~~~~~~~~~~~|>>|
%|>>|~~~~~~~~~~~~~~~~~~~~~~~~~~~~~~~~~~~~~~~~~~~~~~~~~~~~~~~~~~~~~~~~~~~~~~~~~~~~~~~~~~~~~~~~~~~|>>|
\begin{subproof}
{\LEMMA \ref{lemma:norm-subgaussian:1}}
%
Note that
%|>>|:::::::::::::::::::::::::::::::::::::::::::::::::::::::::::::::::::::::::::::::::::::::::::|>>|
\(  \| \Vec{\XRV} - \E[ \Vec{\XRV} ] \|_{2} = \| \Vec{\XRV} \|_{2}  \)
%|<<|:::::::::::::::::::::::::::::::::::::::::::::::::::::::::::::::::::::::::::::::::::::::::::|<<|
due to the lemma's condition that \(  \Vec{\XRV}  \) is zero-mean.
By standard properties of Gaussians,
%(\seeeg \cite[{\LEMMA D.9}]{matsumoto2022binary}),
the expected \(  \lnorm{2}  \)-norm of \(  \Vec{\XRV}  \) is bound from above by
%|>>|:::::::::::::::::::::::::::::::::::::::::::::::::::::::::::::::::::::::::::::::::::::::::::|>>|
\(  \E[ \| \Vec{\XRV} \|_{2} ] \leq \sqrt{\dX} \sigma  \).
%|<<|:::::::::::::::::::::::::::::::::::::::::::::::::::::::::::::::::::::::::::::::::::::::::::|<<|
Due to a well-known concentration inequality for Lipschitz functions of \subgaussian random vectors (\seeeg \cite{wainwright2019high}), and noting that the \(  \lnorm{2}  \)-norm is \(  1  \)-Lipschitz, the claimed inequality holds:
%|>>|===========================================================================================|>>|
\begin{align*}
  \Pr \left( \| \Vec{\XRV} - \E[ \Vec{\XRV} ] \|_{2} > \sqrt{\dX} \sigma + \tXXX \right)
  &=
  \Pr( \| \Vec{\XRV} \|_{2} > \sqrt{\dX} \sigma + \tXXX )
  \\
  &\leq
  \Pr( \| \Vec{\XRV} \|_{2} > \E[ \| \Vec{\XRV} \|_{2} ] + \tXXX )
  \\
  &\leq
  e^{-\frac{1}{2 \sigma^{2}} \tXXX^{2}}
,\end{align*}
%|<<|===========================================================================================|<<|
as desired.
\end{subproof}
%|<<|~~~~~~~~~~~~~~~~~~~~~~~~~~~~~~~~~~~~~~~~~~~~~~~~~~~~~~~~~~~~~~~~~~~~~~~~~~~~~~~~~~~~~~~~~~~|<<|
%|<<|~~~~~~~~~~~~~~~~~~~~~~~~~~~~~~~~~~~~~~~~~~~~~~~~~~~~~~~~~~~~~~~~~~~~~~~~~~~~~~~~~~~~~~~~~~~|<<|
%|<<|~~~~~~~~~~~~~~~~~~~~~~~~~~~~~~~~~~~~~~~~~~~~~~~~~~~~~~~~~~~~~~~~~~~~~~~~~~~~~~~~~~~~~~~~~~~|<<|
%
Fixing \(  \sXX \in (0,1)  \), the random variable \(  \LRV  \) exceeds
%|>>|:::::::::::::::::::::::::::::::::::::::::::::::::::::::::::::::::::::::::::::::::::::::::::|>>|
\(  \LRV > ( 1+\sXX ) \frac{1}{\pi} \m \ADIST  \)
%|<<|:::::::::::::::::::::::::::::::::::::::::::::::::::::::::::::::::::::::::::::::::::::::::::|<<|
with probability at most
%|>>|:::::::::::::::::::::::::::::::::::::::::::::::::::::::::::::::::::::::::::::::::::::::::::|>>|
\(  e^{-\frac{1}{3\pi} \m \sXX^{2} \ADIST}  \)
%|<<|:::::::::::::::::::::::::::::::::::::::::::::::::::::::::::::::::::::::::::::::::::::::::::|<<|
by \LEMMA \ref{lemma:count-mismatch}.
Additionally, by an earlier observation,
%|>>|:::::::::::::::::::::::::::::::::::::::::::::::::::::::::::::::::::::::::::::::::::::::::::|>>|
\(  \E[ \Vec{\URV} \Mid| \LRV=\lX ] = \Vec{0}  \),
%|<<|:::::::::::::::::::::::::::::::::::::::::::::::::::::::::::::::::::::::::::::::::::::::::::|<<|
and thus, due to \LEMMA \ref{lemma:norm-subgaussian:1}, for \(  \tXXX > 0  \),
%|>>|===========================================================================================|>>|
\begin{align}
  \nonumber
  \Pr \left( \| \Vec{\URV} - \E[ \Vec{\URV} ] \|_{2} > \frac{\sqrt{( \kX-2 ) \lX}}{\m} + \tXXX \middle| \LRV \leq \lX \right)
  &\leq
  \Pr \left( \| \Vec{\URV} - \E[ \Vec{\URV} ] \|_{2} > \frac{\sqrt{( \kX-2 ) \lX}}{\m} + \tXXX \middle| \LRV = \lX \right)
  \\ \label{eqn:pf:eqn:lemma:concentration-ineq:noiseless:pr:3:cond-on-L}
  &\leq
  e^{-\frac{\m^{2} \tXXX^{2}}{2 \lX}}
,\end{align}
%|<<|===========================================================================================|<<|
where in particular, taking
%|>>|:::::::::::::::::::::::::::::::::::::::::::::::::::::::::::::::::::::::::::::::::::::::::::|>>|
\(  \lX = ( 1+\sXX ) \frac{1}{\pi} \m \ADIST  \)
%|<<|:::::::::::::::::::::::::::::::::::::::::::::::::::::::::::::::::::::::::::::::::::::::::::|<<|
and
%|>>|:::::::::::::::::::::::::::::::::::::::::::::::::::::::::::::::::::::::::::::::::::::::::::|>>|
\(  \tXXX = \frac{1}{\pi} \tX \ADIST  \)
%|<<|:::::::::::::::::::::::::::::::::::::::::::::::::::::::::::::::::::::::::::::::::::::::::::|<<|
in \EQUATION \eqref{eqn:pf:eqn:lemma:concentration-ineq:noiseless:pr:3:cond-on-L} and noting that
%|>>|:::::::::::::::::::::::::::::::::::::::::::::::::::::::::::::::::::::::::::::::::::::::::::|>>|
%\(  \kX = | \Supp( \thetaStar ) \cup \Supp( \thetaX ) \cup \JCoords | \leq \min \{ | \Supp( \thetaStar ) | + | \Supp( \thetaX ) | + | \JCoords |, \n \} \leq \min \{ 2\k + \max_{\JCoords' \in \JS} | \JCoords' |, \n \} = \kO  \)
%|<<|:::::::::::::::::::::::::::::::::::::::::::::::::::::::::::::::::::::::::::::::::::::::::::|<<|
%|>>|===========================================================================================|>>|
\begin{align}
\label{eqn:pf:eqn:lemma:concentration-ineq:noiseless:pr:3:1}
  \kX
  &= | \Supp( \thetaStar ) \cup \Supp( \thetaX ) \cup \JCoords | \leq \min \{ | \Supp( \thetaStar ) | + | \Supp( \thetaX ) | + | \JCoords |, \n \}
  \nonumber\\
  &\leq \min \{ 2\k + \max_{\JCoords' \in \JS} | \JCoords' |, \n \} = \kO
\end{align}
%|<<|===========================================================================================|<<|
for any \(  \JCoords \in \JS  \),
the following holds:
%|>>|===========================================================================================|>>|
\begin{align*}
  &
  \textstyle
  \Pr \Bigl(
    \| \Vec{\URV} - \E[ \Vec{\URV} ] \|_{2}
    >
    \sqrt{\frac{1}{\pi \m} ( 1+\sXX )( \kO-2 ) \ADIST}
    +
    \frac{1}{\pi} \tX \ADIST
    %\right.
    \\
    &\AlignIndent\AlignIndent \textstyle
    \Bigl| \LRV \leq ( 1+\sXX ) \frac{1}{\pi} \m \ADIST
  \Bigr)
  \\
  & \textstyle \AlignIndent \leq
  \Pr \Bigl(
    \| \Vec{\URV} - \E[ \Vec{\URV} ] \|_{2}
    >
    \sqrt{\frac{1}{\pi \m} ( 1+\sXX )( \kX-2 ) \ADIST}
    +
    \frac{1}{\pi} \tX \ADIST
    %\right.
    \\ & \textstyle \AlignIndent\AlignIndent\AlignIndent\AlignIndent
    \Bigl|
    \LRV \leq ( 1+\sXX ) \frac{1}{\pi} \m \ADIST
  \Bigr)
  \\
  & \AlignIndent \dCmt{by \EQUATION \eqref{eqn:pf:eqn:lemma:concentration-ineq:noiseless:pr:3:1}, \(  \kX \leq \kO  \)}
  \\
  & \AlignIndent \leq
  % e^{-\frac{\m^{2} \tX^{2} \ADIST^{2}}{2 \pi^{2} ( 1+\sXX ) \frac{1}{\pi} \m \ADIST}}
  % \\
  % &\dCmt{by \EQUATION \eqref{eqn:pf:eqn:lemma:concentration-ineq:noiseless:pr:3:cond-on-L}}
  % \\ \TagEqn{\label{eqn:pf:eqn:lemma:concentration-ineq:noiseless:pr:3:2}}
  % & \AlignIndent =
  e^{-\frac{1}{2\pi ( 1+\sXX )} \m \tX^{2} \ADIST}
  .\\
  % & \AlignIndent \dCmt{canceling terms}
  &\AlignIndent\dCmt{by \EQUATION \eqref{eqn:pf:eqn:lemma:concentration-ineq:noiseless:pr:3:cond-on-L}}
  \\ \TagEqn{\label{eqn:pf:eqn:lemma:concentration-ineq:noiseless:pr:3:2}}
\end{align*}
%|<<|===========================================================================================|<<|
Combining the above arguments obtains:
%|>>|===========================================================================================|>>|
\begin{align*}
  &
  \textstyle
  \Pr \Bigl(
    \| \Vec{\URV} - \E[ \Vec{\URV} ] \|_{2}
    >
    \sqrt{\frac{1}{\pi \m} ( 1+\sXX )( \kO-2 ) \ADIST}
    +
    \frac{1}{\pi} \tX \ADIST
  \Bigr)
  \\
  % &=
  % \Pr \left( \LRV \leq ( 1+\sXX ) \frac{1}{\pi} \m \ADIST \right)
  % \\
  % &\AlignSp
  % \Pr \left( \| \Vec{\URV} - \E[ \Vec{\URV} ] \|_{2} > \sqrt{\frac{1}{\pi \m} ( 1+\sXX )( \kO-2 ) \ADIST} + \frac{1}{\pi} \tX \ADIST \middle| \LRV \leq ( 1+\sXX ) \frac{1}{\pi} \m \ADIST \right)
  % \\
  % &\AlignSp+
  % \Pr \left( \LRV > ( 1+\sXX ) \frac{1}{\pi} \m \ADIST \right)
  % \\
  % &\AlignSp\AlignSp
  % \Pr \left( \| \Vec{\URV} - \E[ \Vec{\URV} ] \|_{2} > \sqrt{\frac{1}{\pi \m} ( 1+\sXX )( \kO-2 ) \ADIST} + \frac{1}{\pi} \tX \ADIST \middle| \LRV > ( 1+\sXX ) \frac{1}{\pi} \m \ADIST \right)
  % \\
  % &\dCmt{by the law to total probability and the definition of conditional probabilities}
  % \\
  & \textstyle
  \AlignIndent \leq
  \Pr \Bigl( \| \Vec{\URV} - \E[ \Vec{\URV} ] \|_{2} > \sqrt{\frac{1}{\pi \m} ( 1+\sXX )( \kO-2 ) \ADIST} + \frac{1}{\pi} \tX \ADIST
  \\
  & \AlignIndent\AlignIndent\AlignIndent\AlignIndent \textstyle
  \Bigl| \LRV \leq ( 1+\sXX ) \frac{1}{\pi} \m \ADIST \Bigr)
  \\
  &\AlignIndent\AlignIndent+
  \Pr \left( \LRV > ( 1+\sXX ) \frac{1}{\pi} \m \ADIST \right)
  \\
  & \AlignIndent \leq
  e^{-\frac{1}{2\pi ( 1+\sXX )} \m \tX^{2} \ADIST}
  +
  e^{-\frac{1}{3\pi} \m \sXX^{2} \ADIST}
  .\\
  & \AlignIndent \dCmt{by \EQUATION \eqref{eqn:pf:eqn:lemma:concentration-ineq:noiseless:pr:3:2} and \LEMMA
  \ref{lemma:count-mismatch}}
\end{align*}
%|<<|===========================================================================================|<<|
Recalling the equivalences in distribution described earlier in the proofs, it directly follows that
%|>>|===========================================================================================|>>|
\begin{gather*}
  \textstyle
  \Pr \left( \left\| \frac{\gFn[\JCoords]( \thetaStar, \thetaX )}{\sqrt{2\pi}} - \E \left[ \frac{\gFn[\JCoords]( \thetaStar, \thetaX )}{\sqrt{2\pi}} \right] \right\|_{2} > \sqrt{\frac{1}{\pi \m} ( 1+\sXX )( \kO-2 ) \ADIST} + \frac{1}{\pi} \tX \ADIST \right)
  \\
  \leq
  e^{-\frac{1}{2\pi ( 1+\sXX )} \m \tX^{2} \ADIST}
  +
  e^{-\frac{1}{3\pi} \m \sXX^{2} \ADIST}
%  \\
%  &\leq
%  e^{-\frac{1}{2\pi ( 1+\sXX )} \m \tX^{2} \ADIST}
%  +
%  e^{-\frac{1}{3\pi} \m \sXX^{2} \ADIST}
\end{gather*}
%|<<|===========================================================================================|<<|
for any single choice of
%|>>|:::::::::::::::::::::::::::::::::::::::::::::::::::::::::::::::::::::::::::::::::::::::::::|>>|
\(  \JCoords \in \JS  \) and \(  \thetaX \in \ParamCoverX  \).
%|<<|:::::::::::::::::::::::::::::::::::::::::::::::::::::::::::::::::::::::::::::::::::::::::::|<<|
%where
%%|>>|:::::::::::::::::::::::::::::::::::::::::::::::::::::::::::::::::::::::::::::::::::::::::::|>>|
%\(  \kO = \kOExpr  \).
%|<<|:::::::::::::::::::::::::::::::::::::::::::::::::::::::::::::::::::::::::::::::::::::::::::|<<|
Then, union bounds over \(  \JS  \) and \(  \ParamCoverX  \) yields the concentration inequality in \EQUATION \eqref{eqn:lemma:concentration-ineq:noiseless:pr:3}:
%|>>|===========================================================================================|>>|
\begin{gather*}
  \textstyle
  \Pr \left(
    \ExistsST{\JCoords \in \JS, \thetaX \in \ParamCoverX}{
    \left\| \frac{\gFn[\JCoords]( \thetaStar, \thetaX )}{\sqrt{2\pi}} - \E \left[ \frac{\gFn[\JCoords]( \thetaStar, \thetaX )}{\sqrt{2\pi}} \right] \right\|_{2} > \sqrt{\frac{( 1+\sXX )( \kO-2 ) \ADIST}{\pi \m} } + \frac{\tX \ADIST}{\pi}
    }
  \right)
  \\
  \leq
  | \JS | | \ParamCoverX | e^{-\frac{1}{2\pi ( 1+\sXX )} \m \tX^{2} \ADIST}
  +
  | \ParamCoverX | e^{-\frac{1}{3\pi} \m \sXX^{2} \ADIST}
,\end{gather*}
%|<<|===========================================================================================|<<|
as claimed.
\end{proof}
%|<<|~~~~~~~~~~~~~~~~~~~~~~~~~~~~~~~~~~~~~~~~~~~~~~~~~~~~~~~~~~~~~~~~~~~~~~~~~~~~~~~~~~~~~~~~~~~|<<|
%|<<|~~~~~~~~~~~~~~~~~~~~~~~~~~~~~~~~~~~~~~~~~~~~~~~~~~~~~~~~~~~~~~~~~~~~~~~~~~~~~~~~~~~~~~~~~~~|<<|
%|<<|~~~~~~~~~~~~~~~~~~~~~~~~~~~~~~~~~~~~~~~~~~~~~~~~~~~~~~~~~~~~~~~~~~~~~~~~~~~~~~~~~~~~~~~~~~~|<<|

%%%%%%%%%%%%%%%%%%%%%%%%%%%%%%%%%%%%%%%%%%%%%%%%%%%%%%%%%%%%%%%%%%%%%%%%%%%%%%%%%%%%%%%%%%%%%%%%%%%%
%%%%%%%%%%%%%%%%%%%%%%%%%%%%%%%%%%%%%%%%%%%%%%%%%%%%%%%%%%%%%%%%%%%%%%%%%%%%%%%%%%%%%%%%%%%%%%%%%%%%
%%%%%%%%%%%%%%%%%%%%%%%%%%%%%%%%%%%%%%%%%%%%%%%%%%%%%%%%%%%%%%%%%%%%%%%%%%%%%%%%%%%%%%%%%%%%%%%%%%%%

%\input{concentration-ineq-noiseless-small--2-6}