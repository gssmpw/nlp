%%%%%%%%%%%%%%%%%%%%%%%%%%%%%%%%%%%%%%%%%%%%%%%%%%%%%%%%%%%%%%%%%%%%%%%%%%%%%%%%%%%%%%%%%%%%%%%%%%%%
%%%%%%%%%%%%%%%%%%%%%%%%%%%%%%%%%%%%%%%%%%%%%%%%%%%%%%%%%%%%%%%%%%%%%%%%%%%%%%%%%%%%%%%%%%%%%%%%%%%%
%%%%%%%%%%%%%%%%%%%%%%%%%%%%%%%%%%%%%%%%%%%%%%%%%%%%%%%%%%%%%%%%%%%%%%%%%%%%%%%%%%%%%%%%%%%%%%%%%%%%

\section{Proof of the Main Technical Results}
\label{outline:pf-main-technical-result}

\subsection{Overview of the Proof of \THEOREM \ref{thm:main-technical:sparse}}
\label{outline:main-technical-result|outline-of-pf}

\checkoffbutmayberecheck%
%
The proof of the main technical theorem, \THEOREM \ref{thm:main-technical:sparse}, takes up the majority of the work in this manuscript.
This section provides an overview of the proof.
The proof in full is located in \APPENDIX \ref{outline:pf-main-technical-result} with some auxiliary results therein proved in \APPENDIX \ref{outline:concentration-ineq}.
Before outlining the arguments, recall the  definitions of Equations \eqref{eqn:notations:h:def}--\eqref{eqn:notations:hfJ:def} from \SECTION \ref{outline:main-result|outline-of-pf}. % for
%\(  \Vec{u}, \Vec{v} \in \R^{\n}  \) and \(  \JCoords \subseteq [\n]  \).
%are recalled for convenience:
%for \(  \Vec{u}, \Vec{v} \in \R^{\n}  \) and \(  \JCoords \subseteq [\n]  \), let
%|>>|===========================================================================================|>>|
% \begin{gather*}
%   %\label{eqn:notations:h:def}
%   \hFn( \Vec{u}, \Vec{v} )
%   \defeq
%   \frac{\sqrt{2\pi}}{\m}
%   \sep
%   \CovM^{\T}
%   \sep
%   \frac{1}{2}
%   \left( \Sign( \CovM \Vec{u} ) - \Sign( \CovM \Vec{v} ) \right)
%   ,\\ %\label{eqn:notations:hJ:def}
%   \hFn[\JCoords]( \Vec{u}, \Vec{v} )
%   \defeq
%   \ThresholdSet{\Supp( \Vec{u} ) \cup \Supp( \Vec{v} ) \cup \JCoords}(
%   \hFn( \Vec{u}, \Vec{v} )
% %  \frac{\sqrt{2\pi}}{\m}
% %  \sep
% %  \CovM^{\T}
% %  \sep
% %  \frac{1}{2}
% %  \left( \Sign( \CovM \Vec{u} ) - \Sign( \CovM \Vec{v} ) \right)
%   )
%   ,\\ %\label{eqn:notations:hf:def}
%   \hfFn( \Vec{u}, \Vec{v} )
%   \defeq
%   \frac{\sqrt{2\pi}}{\m}
%   \sep
%   \CovM^{\T}
%   \sep
%   \frac{1}{2}
%   \left( \fFn( \CovM \Vec{u} ) - \Sign( \CovM \Vec{v} ) \right)
%   ,\\ %\label{eqn:notations:hfJ:def}
%   \hfFn[\JCoords]( \Vec{u}, \Vec{v} )
%   \defeq
%   \ThresholdSet{\Supp( \Vec{u} ) \cup \Supp( \Vec{v} ) \cup \JCoords}(
%   \hfFn( \Vec{u}, \Vec{v} )
% %  \frac{\sqrt{2\pi}}{\m}
% %  \sep
% %  \CovM^{\T}
% %  \sep
% %  \frac{1}{2}
% %  \left( \fFn( \CovM \Vec{u} ) - \Sign( \CovM \Vec{v} ) \right)
%   )
% ,\end{gather*}
%|<<|===========================================================================================|<<|
Additionally, define the following related notations for \(  \Vec{u}, \Vec{v} \in \R^{\n}  \) and \(  \JCoords \subseteq [\n]  \):
%|>>|===========================================================================================|>>|
\begin{gather}
  \label{eqn:notations:g:def}
  \gFn( \Vec{u}, \Vec{v} )
  \defeq
  \hFn( \Vec{u}, \Vec{v} )
  -
  \left\langle \hFn( \Vec{u}, \Vec{v} ), \frac{\Vec{u}-\Vec{v}}{\| \Vec{u}-\Vec{v} \|_{2}} \right\rangle \frac{\Vec{u}-\Vec{v}}{\| \Vec{u}-\Vec{v} \|_{2}}
  -
  \left\langle \hFn( \Vec{u}, \Vec{v} ), \frac{\Vec{u}+\Vec{v}}{\| \Vec{u}+\Vec{v} \|_{2}} \right\rangle \frac{\Vec{u}+\Vec{v}}{\| \Vec{u}+\Vec{v} \|_{2}}
  ,\\ \label{eqn:notations:gJ:def}
  \gFn[\JCoords]( \Vec{u}, \Vec{v} )
  \defeq
  \ThresholdSet{\Supp( \Vec{u} ) \cup \Supp( \Vec{v} ) \cup \JCoords}( \gFn( \Vec{u}, \Vec{v} ) )
  ,\\ \label{eqn:notations:gf:def}
  \gfFn( \Vec{u}, \Vec{u} )
  \defeq
  \hfFn( \Vec{u}, \Vec{u} ) - \left\langle \hfFn( \Vec{u}, \Vec{u} ), \Vec{u} \right\rangle \Vec{u}
  ,\\ \label{eqn:notations:gfJ:def}
  \gfFn[\JCoords]( \Vec{u}, \Vec{u} )
  \defeq
  \ThresholdSet{\Supp( \Vec{u} ) \cup \Supp( \Vec{v} ) \cup \JCoords}( \gfFn( \Vec{u}, \Vec{u} ) )
.\end{gather}
%|<<|===========================================================================================|<<|
Note that
%|>>|===========================================================================================|>>|
\begin{gather*}
  \gFn[\JCoords]( \Vec{u}, \Vec{v} )
  =
  \hFn[\JCoords]( \Vec{u}, \Vec{v} )
  -
  \left\langle \hFn[\JCoords]( \Vec{u}, \Vec{v} ), \frac{\Vec{u}-\Vec{v}}{\| \Vec{u}-\Vec{v} \|_{2}} \right\rangle \frac{\Vec{u}-\Vec{v}}{\| \Vec{u}-\Vec{v} \|_{2}}
  -
  \left\langle \hFn[\JCoords]( \Vec{u}, \Vec{v} ), \frac{\Vec{u}+\Vec{v}}{\| \Vec{u}+\Vec{v} \|_{2}} \right\rangle \frac{\Vec{u}+\Vec{v}}{\| \Vec{u}+\Vec{v} \|_{2}}
\end{gather*}
%|<<|===========================================================================================|<<|
and that
%|>>|===========================================================================================|>>|
\begin{gather*}
  \gfFn[\JCoords]( \Vec{u}, \Vec{u} )
  =
  \hfFn[\JCoords]( \Vec{u}, \Vec{u} )
  -
  \left\langle \hfFn[\JCoords]( \Vec{u}, \Vec{u} ), \Vec{u} \right\rangle \Vec{u}
.\end{gather*}
%|<<|===========================================================================================|<<|

%%%%%%%%%%%%%%%%%%%%%%%%%%%%%%%%%%%%%%%%%%%%%%%%%%%%%%%%%%%%%%%%%%%%%%%%%%%%%%%%%%%%%%%%%%%%%%%%%%%%
%%%%%%%%%%%%%%%%%%%%%%%%%%%%%%%%%%%%%%%%%%%%%%%%%%%%%%%%%%%%%%%%%%%%%%%%%%%%%%%%%%%%%%%%%%%%%%%%%%%%

\subsubsection{Key Steps of the Proof}
\label{outline:main-technical-result|outline-of-pf|outline}

The proof of \THEOREM \ref{thm:main-technical:sparse} are sketched as follows.
%
\checkoff%

%|>>|XXXXXXXXXXXXXXXXXXXXXXXXXXXXXXXXXXXXXXXXXXXXXXXXXXXXXXXXXXXXXXXXXXXXXXXXXXXXXXXXXXXXXXXXXXX|>>|
%|>>|XXXXXXXXXXXXXXXXXXXXXXXXXXXXXXXXXXXXXXXXXXXXXXXXXXXXXXXXXXXXXXXXXXXXXXXXXXXXXXXXXXXXXXXXXXX|>>|
%|>>|XXXXXXXXXXXXXXXXXXXXXXXXXXXXXXXXXXXXXXXXXXXXXXXXXXXXXXXXXXXXXXXXXXXXXXXXXXXXXXXXXXXXXXXXXXX|>>|
\begin{enumerate}
%XXXXXXXXXXXXXXXXXXXXXXXXXXXXXXXXXXXXXXXXXXXXXXXXXXXXXXXXXXXXXXXXXXXXXXXXXXXXXXXXXXXXXXXXXXXXXXXXXXX
\item \label{enum:outline-pf-main-technical:1}
Recall that the aim is to bound
%|>>|===========================================================================================|>>|
\begin{gather}
\label{eqn:enum:outline-pf-main-technical:1}
  \left\|
    \thetaStar
    -
    \frac
    {\thetaXX + \hfFn[\JCoords]( \thetaStar, \thetaXX )}
    {\| \thetaXX + \hfFn[\JCoords]( \thetaStar, \thetaXX ) \|_{2}}
  \right\|_{2}
\end{gather}
%|<<|===========================================================================================|<<|
from above with high probability uniformly for all \(  \thetaXX \in \ParamSpace  \) and all \(  \JCoords \subseteq [\n]  \), \(  | \JCoords | \leq \k  \).
%We begin by arbitrarily fixing \(  \thetaStar, \thetaXX \in \ParamSpace  \) and \(  \JCoords \subseteq [\n]  \), \(  | \JCoords | \leq \k  \), where \(  \thetaXX  \) and \(  \JCoords  \) will be varied later on to establish the uniform result.
%XXXXXXXXXXXXXXXXXXXXXXXXXXXXXXXXXXXXXXXXXXXXXXXXXXXXXXXXXXXXXXXXXXXXXXXXXXXXXXXXXXXXXXXXXXXXXXXXXXX
\item \label{enum:outline-pf-main-technical:2}
To obtain a uniform result, a \(  \tauX  \)-net, \(  \ParamCover \subset \ParamSpace  \), over the parameter space, \(  \ParamSpace  \), is constructed with a particular design, the details of which are left to the formal proof of \THEOREM \ref{thm:main-technical:sparse}.
For the purpose of this overview, its suffices to say that, crucially, the design of \(  \ParamCover  \) ensures that for each \(  \thetaXX \in \ParamSpace  \), there exists an element, \(  \thetaX \in \ParamCover  \), such that both
%|>>|:::::::::::::::::::::::::::::::::::::::::::::::::::::::::::::::::::::::::::::::::::::::::::|>>|
\(  \| \thetaX - \thetaXX \|_{2} \leq \tauX  \) and
\(  \Supp( \thetaX ) = \Supp( \thetaXX )  \).
%|<<|:::::::::::::::::::::::::::::::::::::::::::::::::::::::::::::::::::::::::::::::::::::::::::|<<|
This cover, \(  \ParamCover  \), will allow the establishment of a global result for points, \(  \thetaX \in \ParamCover  \), within it, which can subsequently be extended to arbitrary points, \(  \thetaXX \in \ParamSpace  \), in the entire parameter space via a local analysis.
%XXXXXXXXXXXXXXXXXXXXXXXXXXXXXXXXXXXXXXXXXXXXXXXXXXXXXXXXXXXXXXXXXXXXXXXXXXXXXXXXXXXXXXXXXXXXXXXXXXX
\item \label{enum:outline-pf-main-technical:3}
As another preliminary step, it will be shown that for any \(  \thetaXX \in \ParamSpace  \) and \(  \JCoords \subseteq [\n]  \),
%|>>|===========================================================================================|>>|
\begin{gather*}
  \frac
  {\E[ \thetaXX + \hfFn[\JCoords]( \thetaStar, \thetaXX ) ]}
  {\| \E[ \thetaXX + \hfFn[\JCoords]( \thetaStar, \thetaXX ) ] \|_{2}}
  =
  \thetaStar
.\end{gather*}
%|<<|===========================================================================================|<<|
In other words, the quantity in \eqref{eqn:enum:outline-pf-main-technical:1}---which we seek to bound---describes a notion of deviation of
%|>>|:::::::::::::::::::::::::::::::::::::::::::::::::::::::::::::::::::::::::::::::::::::::::::|>>|
\(  \thetaXX + \hfFn[\JCoords]( \thetaStar, \thetaXX )  \)
%|<<|:::::::::::::::::::::::::::::::::::::::::::::::::::::::::::::::::::::::::::::::::::::::::::|<<|
from its mean (after normalization):
%|>>|===========================================================================================|>>|
\begin{gather*}
  \left\|
    \thetaStar
    -
    \frac
    {\thetaXX + \hfFn[\JCoords]( \thetaStar, \thetaXX )}
    {\| \thetaXX + \hfFn[\JCoords]( \thetaStar, \thetaXX ) \|_{2}}
  \right\|_{2}
  =
  \left\|
    \frac
    {\thetaXX + \hfFn[\JCoords]( \thetaStar, \thetaXX )}
    {\| \thetaXX + \hfFn[\JCoords]( \thetaStar, \thetaXX ) \|_{2}}
    -
    \frac
    {\E[ \thetaXX + \hfFn[\JCoords]( \thetaStar, \thetaXX ) ]}
    {\| \E[ \thetaXX + \hfFn[\JCoords]( \thetaStar, \thetaXX ) ] \|_{2}}
  \right\|_{2}
.\end{gather*}
%|<<|===========================================================================================|<<|
In fact, this deviation turns out to roughly scale with the deviation of the random function \(  \hfFn[\JCoords]  \) around its mean:
%|>>|===========================================================================================|>>|
\begin{gather}
\label{eqn:enum:outline-pf-main-technical:1b}
%  \left\|
%    \thetaStar
%    -
%    \frac
%    {\thetaXX + \hfFn[\JCoords]( \thetaStar, \thetaXX )}
%    {\| \thetaXX + \hfFn[\JCoords]( \thetaStar, \thetaXX ) \|_{2}}
%  \right\|_{2}
%  =
  \left\|
    \frac
    {\thetaXX + \hfFn[\JCoords]( \thetaStar, \thetaXX )}
    {\| \thetaXX + \hfFn[\JCoords]( \thetaStar, \thetaXX ) \|_{2}}
    -
    \frac
    {\E[ \thetaXX + \hfFn[\JCoords]( \thetaStar, \thetaXX ) ]}
    {\| \E[ \thetaXX + \hfFn[\JCoords]( \thetaStar, \thetaXX ) ] \|_{2}}
  \right\|_{2}
  \propto
  \| \hfFn[\JCoords]( \thetaStar, \thetaXX ) - \E[ \hfFn[\JCoords]( \thetaStar, \thetaXX ) ] \|_{2}
.\end{gather}
%|<<|===========================================================================================|<<|
Analyzing (a decomposition of) the deviation of \(  \hfFn[\JCoords]  \) will be at the core of the proof.
%XXXXXXXXXXXXXXXXXXXXXXXXXXXXXXXXXXXXXXXXXXXXXXXXXXXXXXXXXXXXXXXXXXXXXXXXXXXXXXXXXXXXXXXXXXXXXXXXXXX
\item \label{enum:outline-pf-main-technical:4}
Letting
%|>>|:::::::::::::::::::::::::::::::::::::::::::::::::::::::::::::::::::::::::::::::::::::::::::|>>|
\(  \thetaStar, \thetaXX \in \ParamSpace  \)
%|<<|:::::::::::::::::::::::::::::::::::::::::::::::::::::::::::::::::::::::::::::::::::::::::::|<<|
be arbitrary, and using the observations in \STEP \ref{enum:outline-pf-main-technical:3}, the triangle inequality, algebraic manipulations, and other standard techniques, the expression in \eqref{eqn:enum:outline-pf-main-technical:1}
%is decomposed into (and bounded by)
is bounded by the sum of
three terms which will admit an easier analysis than directly handling \eqref{eqn:enum:outline-pf-main-technical:1}:
%|>>|===========================================================================================|>>|
\begin{subequations}
\label{eqn:enum:outline-pf-main-technical:2}
\begin{align}
  \left\| \thetaStar - \frac{\thetaXX+\hfFn[\JCoords]( \thetaStar, \thetaXX )}{\| \thetaXX+\hfFn[\JCoords]( \thetaStar, \thetaXX ) \|_{2}} \right\|_{2}
  &\leq
  \label{enum:outline-pf-main-technical:5:i}
  \frac
  {2 \| \hFn[\JCoords]( \thetaStar, \thetaX ) - \E[ \hFn[\JCoords]( \thetaStar, \thetaX ) ] \|_{2}}
  {\DENOM}
  \\ \label{enum:outline-pf-main-technical:5:iii}
  &\AlignSp+
  \frac
  {2 \| \hFn[\Supp( \thetaStar ) \cup \JCoords]( \thetaX, \thetaXX ) - \E[ \hFn[\Supp( \thetaStar ) \cup \JCoords]( \thetaX, \thetaXX ) ] \|_{2}}
  {\DENOM}
  \\ \label{enum:outline-pf-main-technical:5:ii}
  &\AlignSp+
  \frac
  {2 \| \hfFn[\Supp( \thetaX ) \cup \JCoords]( \thetaStar, \thetaStar ) - \E[ \hfFn[\Supp( \thetaX ) \cup \JCoords]( \thetaStar, \thetaStar ) ] \|_{2}}
  {\DENOM}
%\TagEqn{\label{eqn:enum:outline-pf-main-technical:2}}
,\end{align}
\end{subequations}
%|<<|===========================================================================================|<<|
where \(  \JCoords \subseteq [\n]  \), \(  | \JCoords | \leq \k  \), is arbitrary,
and where
%|>>|:::::::::::::::::::::::::::::::::::::::::::::::::::::::::::::::::::::::::::::::::::::::::::|>>|
\(  \thetaX \in \ParamCover \setminus \Ball{\tauX}( \thetaStar )  \)
%|<<|:::::::::::::::::::::::::::::::::::::::::::::::::::::::::::::::::::::::::::::::::::::::::::|<<|
such that
%|>>|:::::::::::::::::::::::::::::::::::::::::::::::::::::::::::::::::::::::::::::::::::::::::::|>>|
\(  \| \thetaX - \thetaXX \|_{2} \leq 2\tauX  \) and
\(  \Supp( \thetaX ) \cup \JCoords = \Supp( \thetaXX ) \cup \JCoords  \)
%|<<|:::::::::::::::::::::::::::::::::::::::::::::::::::::::::::::::::::::::::::::::::::::::::::|<<|
(\see \LEMMA \ref{lemma:combine}).
Per the design of the \(  \tauX  \)-net, \(  \ParamCover \subset \ParamSpace  \), in \STEP \ref{enum:outline-pf-main-technical:2}, such a point \(  \thetaX \in \ParamCover  \) exists for any choice of \(  \thetaXX \in \ParamSpace  \).
%preserve some notion of deviations.
%XXXXXXXXXXXXXXXXXXXXXXXXXXXXXXXXXXXXXXXXXXXXXXXXXXXXXXXXXXXXXXXXXXXXXXXXXXXXXXXXXXXXXXXXXXXXXXXXXXX
\item \label{enum:outline-pf-main-technical:5}
The three terms on the \RHS of \EQUATION \eqref{eqn:enum:outline-pf-main-technical:2} can be viewed as bounding \eqref{eqn:enum:outline-pf-main-technical:1} by relating it (with appropriate scaling) to the deviation of \(  \hfFn  \) specified on the \RHS of \eqref{eqn:enum:outline-pf-main-technical:1b}, and then controlling the \RHS of \eqref{eqn:enum:outline-pf-main-technical:1b} by
%breaking it up into,
decomposing the deviation of \(  \hfFn  \) into three components of deviation,
in order:
%|>>|×××××××××××××××××××××××××××××××××××××××××××××××××××××××××××××××××××××××××××××××××××××××××××|>>|
%\Enum[{\label{enum:outline-pf-main-technical:5:i}}]{i}
\eqref{enum:outline-pf-main-technical:5:i},
a component handling points in the cover, \(  \ParamCover  \), over \(  \ParamSpace  \) that are sufficiently far from \(  \thetaStar  \)---a ``global'' result;
%×××××××××××××××××××××××××××××××××××××××××××××××××××××××××××××××××××××××××××××××××××××××××××××××××××
%\Enum[{\label{enum:outline-pf-main-technical:5:iii}}]{ii}
\eqref{enum:outline-pf-main-technical:5:iii},
a component reconciling the discrepancy between the original point, \(  \thetaXX \in \ParamSpace  \), which may be outside the cover, and a nearby neighbor in the cover, \(  \thetaX \in \ParamCover \setminus \Ball{\tauX}( \thetaStar )  \)---a ``local'' result; and
%×××××××××××××××××××××××××××××××××××××××××××××××××××××××××××××××××××××××××××××××××××××××××××××××××××
%\Enum[{\label{enum:outline-pf-main-technical:5:ii}}]{iii}
\eqref{enum:outline-pf-main-technical:5:ii},
a component handling the ``noise'' introduced into the GLM through the randomness of \(  \fFn  \).
%|<<|×××××××××××××××××××××××××××××××××××××××××××××××××××××××××××××××××××××××××××××××××××××××××××|<<|
%Hence, these three terms in \eqref{eqn:enum:outline-pf-main-technical:2} in effect decompose the original deviation captured by \eqref{eqn:enum:outline-pf-main-technical:1} into three components of deviation. %each itself capturing the deviation of particular a random vector.
%%The first component, \eqref{enum:outline-pf-main-technical:5:i}, captures the aforementioned ``global'' result, while the second component, \eqref{enum:outline-pf-main-technical:5:iii}, corresponds with the ``local'' result.
%This enumeration of the terms, \eqref{enum:outline-pf-main-technical:5:i}--\eqref{enum:outline-pf-main-technical:5:ii}, in \EQUATION \eqref{eqn:enum:outline-pf-main-technical:2} will be carried forward in the remaining overview.
%XXXXXXXXXXXXXXXXXXXXXXXXXXXXXXXXXXXXXXXXXXXXXXXXXXXXXXXXXXXXXXXXXXXXXXXXXXXXXXXXXXXXXXXXXXXXXXXXXXX
\item \label{enum:outline-pf-main-technical:6}
The technical work in this manuscript then lies largely with bounding the three terms on the \RHS of \EQUATION \eqref{eqn:enum:outline-pf-main-technical:2}.
While most of the details of this analysis are left to the formal proofs (\see \APPENDICES \ref{outline:pf-main-technical-result|pf-intermediate}--\ref{outline:concentration-ineq}), a few salient ideas in the approach are mentioned here.
%The bulk of the work is then to
%Using concentration inequalities, which are ,
%For each of the three terms on the \RHS of \EQUATION \eqref{eqn:enum:outline-pf-main-technical:2},
%XXXXXXXXXXXXXXXXXXXXXXXXXXXXXXXXXXXXXXXXXXXXXXXXXXXXXXXXXXXXXXXXXXXXXXXXXXXXXXXXXXXXXXXXXXXXXXXXXXX
\item \label{enum:outline-pf-main-technical:7}
For the three terms, \eqref{enum:outline-pf-main-technical:5:i}--\eqref{enum:outline-pf-main-technical:5:ii}, the (shared) denominator can be calculated directly.
%XXXXXXXXXXXXXXXXXXXXXXXXXXXXXXXXXXXXXXXXXXXXXXXXXXXXXXXXXXXXXXXXXXXXXXXXXXXXXXXXXXXXXXXXXXXXXXXXXXX
\item \label{enum:outline-pf-main-technical:8}
On the other hand, the numerators in \eqref{enum:outline-pf-main-technical:5:i}--\eqref{enum:outline-pf-main-technical:5:ii} are upper bounded with bounded probability
%, rather than exactly calculated,
through concentration inequalities derived with standard techniques.
To do so, each numerator is orthogonally decomposed into two to three components for which derivations of concentration inequalities are easier.
%for which concentration inequalities can be derived more easily.
Subsequently, for each numerator, the concentration inequalities for its associated components are combined via the triangle inequality.
Then, these are extended into uniform results by appropriate union bounds.
%XXXXXXXXXXXXXXXXXXXXXXXXXXXXXXXXXXXXXXXXXXXXXXXXXXXXXXXXXXXXXXXXXXXXXXXXXXXXXXXXXXXXXXXXXXXXXXXXXXX
\item \label{enum:outline-pf-main-technical:9}
For the first and last terms, \eqref{enum:outline-pf-main-technical:5:i} and \eqref{enum:outline-pf-main-technical:5:ii}, the union bounds are straightforward: simply taken over the coordinate subsets of cardinality at most \(  \k  \), as well as, in the case of \eqref{enum:outline-pf-main-technical:5:i}, over the cover, \(  \ParamCover  \).
%XXXXXXXXXXXXXXXXXXXXXXXXXXXXXXXXXXXXXXXXXXXXXXXXXXXXXXXXXXXXXXXXXXXXXXXXXXXXXXXXXXXXXXXXXXXXXXXXXXX
\item \label{enum:outline-pf-main-technical:10}
In contrast, the second term, \eqref{enum:outline-pf-main-technical:5:iii},  requires a more careful---and somewhat indirect---argument.
In this case, the union bound is taken over the set
%|>>|:::::::::::::::::::::::::::::::::::::::::::::::::::::::::::::::::::::::::::::::::::::::::::|>>|
%\(  \{ \hFn[\Supp( \thetaStar ) \cup \JCoords]( \thetaX, \thetaXX ) : \thetaXX \in \BallX{2\tauX}( \thetaX ), \Supp( \thetaXX ) = \Supp( \thetaX ) \}  \),
\(  \{ \hFn[\Supp( \thetaStar ) \cup \JCoords]( \thetaX, \thetaXX ) : \thetaXX \in \BallX{2\tauX}( \thetaX ) \}  \),
%|<<|:::::::::::::::::::::::::::::::::::::::::::::::::::::::::::::::::::::::::::::::::::::::::::|<<|
which has a sufficiently small cardinality due to the local binary embeddings of \cite[{\COROLLARY 3.3}]{oymak2015near}.
%\ToDo{Revise this.}
%XXXXXXXXXXXXXXXXXXXXXXXXXXXXXXXXXXXXXXXXXXXXXXXXXXXXXXXXXXXXXXXXXXXXXXXXXXXXXXXXXXXXXXXXXXXXXXXXXXX
\item \label{enum:outline-pf-main-technical:11}
Using the uniform results obtained in \STEPS \ref{enum:outline-pf-main-technical:7}--\ref{enum:outline-pf-main-technical:10}, the number of covariates, \(  \m  \), can then be determined such that desired bounds on the terms \eqref{enum:outline-pf-main-technical:5:i}--\eqref{enum:outline-pf-main-technical:5:ii},
%on the \RHS \eqref{eqn:enum:outline-pf-main-technical:2},
and hence also the desired bound on \eqref{eqn:enum:outline-pf-main-technical:1}, hold uniformly with high probability.
This will establish the invertibility condition for Gaussian covariate matrices claimed in \THEOREM \ref{thm:main-technical:sparse}.
%Using \EQUATION \eqref{eqn:enum:outline-pf-main-technical:2} and the bounds on \eqref{enum:outline-pf-main-technical:5:i}--\eqref{enum:outline-pf-main-technical:5:ii} obtained in \STEPS \ref{enum:outline-pf-main-technical:7}--\ref{enum:outline-pf-main-technical:10}, the number of covariates, \(  \m  \), can then be determined such that a desired bound on \eqref{eqn:enum:outline-pf-main-technical:1} hold uniformly with high probability.
%XXXXXXXXXXXXXXXXXXXXXXXXXXXXXXXXXXXXXXXXXXXXXXXXXXXXXXXXXXXXXXXXXXXXXXXXXXXXXXXXXXXXXXXXXXXXXXXXXXX
%\item \label{enum:outline-pf-main-technical:12}
%%XXXXXXXXXXXXXXXXXXXXXXXXXXXXXXXXXXXXXXXXXXXXXXXXXXXXXXXXXXXXXXXXXXXXXXXXXXXXXXXXXXXXXXXXXXXXXXXXXXX
\end{enumerate}
%|<<|XXXXXXXXXXXXXXXXXXXXXXXXXXXXXXXXXXXXXXXXXXXXXXXXXXXXXXXXXXXXXXXXXXXXXXXXXXXXXXXXXXXXXXXXXXX|<<|
%|<<|XXXXXXXXXXXXXXXXXXXXXXXXXXXXXXXXXXXXXXXXXXXXXXXXXXXXXXXXXXXXXXXXXXXXXXXXXXXXXXXXXXXXXXXXXXX|<<|
%|<<|XXXXXXXXXXXXXXXXXXXXXXXXXXXXXXXXXXXXXXXXXXXXXXXXXXXXXXXXXXXXXXXXXXXXXXXXXXXXXXXXXXXXXXXXXXX|<<|

\subsection{Detailed Proofs}
Several constants will  appear throughout this section.
For convenient reference later, they are specified in the following definition.
%|>>|###########################################################################################|>>|
%|>>|###########################################################################################|>>|
%|>>|###########################################################################################|>>|
\begin{definition}
\label{def:univ-const}
Let
%|>>|:::::::::::::::::::::::::::::::::::::::::::::::::::::::::::::::::::::::::::::::::::::::::::|>>|
\(  \ConstC, \Constb, \Constc, \ConstbLD, \ConstbSD > 0  \)
%|<<|:::::::::::::::::::::::::::::::::::::::::::::::::::::::::::::::::::::::::::::::::::::::::::|<<|
be (absolute) constants such that
%|>>|:::::::::::::::::::::::::::::::::::::::::::::::::::::::::::::::::::::::::::::::::::::::::::|>>|
\(  \ConstC \defeq \frac{\Constb}{\Constc^{2}}  \geq \frac{1}{50}  \),
\(  \sqrt{8} \Constb < \frac{\ConstbSD}{2}  \),
\(  \Constc \defeq \frac{\ConstbSD}{2}-\sqrt{8} \Constb = \BigOmega(1)  \),
\(  \ConstbLD < 1 - \sqrt{\frac{2\Constb}{\ConstdSD}}  \), and
\(  \ConstbSD \leq 1 - \ConstbLD - \sqrt{\frac{2\Constb}{\ConstdSD}}  \).
%\ToDo{I need to correct \(  \Constb  \), etc.}
%|<<|:::::::::::::::::::::::::::::::::::::::::::::::::::::::::::::::::::::::::::::::::::::::::::|<<|
%\?{@Arya -- Could you please check the value of the constant \(  \ConstdSD  \)? (It should be chosen according to \cite{oymak2015near}.)}
Additionally, let
%|>>|:::::::::::::::::::::::::::::::::::::::::::::::::::::::::::::::::::::::::::::::::::::::::::|>>|
\(  \ConstA, \ConstB, \ConstCTwo, \ConstCThree, \ConstCFour, \ConstCFive > 0  \)
%|<<|:::::::::::::::::::::::::::::::::::::::::::::::::::::::::::::::::::::::::::::::::::::::::::|<<|
be the (absolute) constants given by:
%|>>|:::::::::::::::::::::::::::::::::::::::::::::::::::::::::::::::::::::::::::::::::::::::::::|>>|
\(  \ConstA = \ConstAValue  \),
\(  \ConstB = \ConstBValue  \),
%\(  \ConstCOne = \ConstCOneValue  \),
\(  \ConstCTwo = \ConstCTwoValue  \),
\(  \ConstCThree = \ConstCThreeValue  \),
\(  \ConstCFour = \ConstCFourValue  \), and
\(  \ConstCFive = \ConstCFiveValue  \).
%|<<|:::::::::::::::::::::::::::::::::::::::::::::::::::::::::::::::::::::::::::::::::::::::::::|<<|
\end{definition}

%%%%%%%%%%%%%%%%%%%%%%%%%%%%%%%%%%%%%%%%%%%%%%%%%%%%%%%%%%%%%%%%%%%%%%%%%%%%%%%%%%%%%%%%%%%%%%%%%%%%
%%%%%%%%%%%%%%%%%%%%%%%%%%%%%%%%%%%%%%%%%%%%%%%%%%%%%%%%%%%%%%%%%%%%%%%%%%%%%%%%%%%%%%%%%%%%%%%%%%%%

Two additional notations will be used in this manuscript, which are introduced in \DEFINITION \ref{def:nu-and-tau}, below.
%
%|>>|###########################################################################################|>>|
%|>>|###########################################################################################|>>|
%|>>|###########################################################################################|>>|
\begin{definition}
\label{def:nu-and-tau}
For \(  \deltaX > 0  \), let \(  \nuX( \deltaX ), \tauX( \deltaX ) > 0  \) be given by
%|>>|===========================================================================================|>>|
\begin{gather}
\label{eqn:def:nu-and-tau:nu}
  \nuX( \deltaX )
  =
  \nuXEXPR[\deltaX]
,\end{gather}
%|<<|===========================================================================================|<<|
and
%|>>|===========================================================================================|>>|
\begin{gather}
\label{eqn:def:nu-and-tau:tau}
  \tauX( \deltaX ) \defeq \tauXEXPR[\deltaX]
,\end{gather}
%|<<|===========================================================================================|<<|
where \(  \ConstCFive > 0  \) is given in \DEFINITION \ref{def:univ-const}.
To condense notation, the explicit parameterization by \(  \deltaX  \) will in general be dropped and left implicit in this manuscript, \ie
%|>>|:::::::::::::::::::::::::::::::::::::::::::::::::::::::::::::::::::::::::::::::::::::::::::|>>|
\(  \nuX = \nuX( \deltaX )  \) and \(  \tauX = \tauX( \deltaX )  \),
%|<<|:::::::::::::::::::::::::::::::::::::::::::::::::::::::::::::::::::::::::::::::::::::::::::|<<|
where the specific choice of \(  \deltaX  \) may vary but will be clear from the context.
\end{definition}
%|<<|###########################################################################################|<<|
%|<<|###########################################################################################|<<|
%|<<|###########################################################################################|<<|
%
%%%%%%%%%%%%%%%%%%%%%%%%%%%%%%%%%%%%%%%%%%%%%%%%%%%%%%%%%%%%%%%%%%%%%%%%%%%%%%%%%%%%%%%%%%%%%%%%%%%%

We restate \THEOREM \ref{thm:main-technical:sparse} below with the sample complexity more specific with the above-defined constants.

\begin{theorem}[Restatement of \THEOREM \ref{thm:main-technical:sparse}]
\label{thm:main-technical:sparse:detail}
%
Let
%|>>|:::::::::::::::::::::::::::::::::::::::::::::::::::::::::::::::::::::::::::::::::::::::::::|>>|
\(  \ConstA, \ConstB, \ConstCOne, \ConstCTwo, \ConstCThree, \ConstCFour, \ConstCFive > 0  \)
%|<<|:::::::::::::::::::::::::::::::::::::::::::::::::::::::::::::::::::::::::::::::::::::::::::|<<|
be absolute constants as specified in \DEFINITION \ref{def:univ-const}, and fix
%|>>|:::::::::::::::::::::::::::::::::::::::::::::::::::::::::::::::::::::::::::::::::::::::::::|>>|
\(  \n, \k, \m \in \Z_{+}  \), \(  \k \leq \n  \), and \(  \rhoX, \deltaX \in (0,1)  \) where
\begin{gather}
  \deltaX \defeq \frac{\epsilonX}{\frac{3}{2} ( 5+\sqrt{21} )}
  %\frac{\epsilonX}{\frac{1}{2} ( 3+\sqrt{5} )}
.\end{gather}
%|<<|:::::::::::::::::::::::::::::::::::::::::::::::::::::::::::::::::::::::::::::::::::::::::::|<<|
Set \(  \nuX = \nuX( \deltaX ) > 0  \)
% such that
% %|>>|===========================================================================================|>>|
% \begin{gather}
% \label{eqn:main-technical:sparse:nu}
%   \nuX
%   =
%   \nuXEXPR
% ,\end{gather}
%|<<|===========================================================================================|<<|
and let \(  \tauX = \tauX( \deltaX ) > 0  \)
%be given by
%|>>|===========================================================================================|>>|
% \begin{gather}
% \label{eqn:main-technical:sparse:tau}
%   \tauX \defeq \tauXEXPR
% ,\end{gather}
%|<<|===========================================================================================|<<|
as in \DEFINITION \ref{def:nu-and-tau}.
Write
%|>>|:::::::::::::::::::::::::::::::::::::::::::::::::::::::::::::::::::::::::::::::::::::::::::|>>|
\(  \alphaO = \alphaO( \deltaX ) \defeq \alphaOExpr[\deltaX]  \)
%|<<|:::::::::::::::::::::::::::::::::::::::::::::::::::::::::::::::::::::::::::::::::::::::::::|<<|
as in \EQUATION \eqref{eqn:notations:alpha_0:def}.
Let
%|>>|:::::::::::::::::::::::::::::::::::::::::::::::::::::::::::::::::::::::::::::::::::::::::::|>>|
\(  \ParamSpace = \SparseSphereSubspace{\k}{\n}  \),
%|<<|:::::::::::::::::::::::::::::::::::::::::::::::::::::::::::::::::::::::::::::::::::::::::::|<<|
and fix
%|>>|:::::::::::::::::::::::::::::::::::::::::::::::::::::::::::::::::::::::::::::::::::::::::::|>>|
\(  \thetaStar \in \ParamSpace  \).
%|<<|:::::::::::::::::::::::::::::::::::::::::::::::::::::::::::::::::::::::::::::::::::::::::::|<<|
Under \ASSUMPTION \ref{assumption:p}, if
%|>>|===========================================================================================|>>|
\begin{align}
\nonumber
  \m
  &\geq
  \mEXPR[s]
  \\ \nonumber
  &=
  \mOEXPRS{\deltaX}[,]
  \\
%\label{eqn:main-technical:sparse:m}
\end{align}
%|<<|===========================================================================================|<<|
then with probability at least \(  1-\rhoX  \), uniformly for all \(  \thetaXX \in \ParamSpace  \) and all \(  \JCoords \subseteq [\n]  \), \(  | \JCoords | \leq \k  \),
%|>>|===========================================================================================|>>|
\begin{gather}
%\label{eqn:main-technical:sparse:1}
  \left\|
    \thetaStar
    -
    \frac
    {\thetaXX + \hfFn[\JCoords]( \thetaStar, \thetaXX )}
    {\| \thetaXX + \hfFn[\JCoords]( \thetaStar, \thetaXX ) \|_{2}}
  \right\|_{2}
  \leq
  \sqrt{\deltaX \| \thetaStar-\thetaXX \|_{2}}
  +
  \deltaX
.\end{gather}
%|<<|===========================================================================================|<<|
\end{theorem}
%|

\subsection{Intermediate Results for the Proof of \THEOREM \ref{thm:main-technical:sparse}}
\label{outline:pf-main-technical-result|intermediate}

\checkoff

Lemmas  \ref{lemma:combine}--\ref{lemma:large-dist:2}, stated below in this section, lay the groundwork for proving the main technical theorem, \THEOREM \ref{thm:main-technical:sparse}.
The proofs of these intermediate results can be found in \SECTION \ref{outline:pf-main-technical-result|pf-intermediate}.
Recall that the ultimate goal is to uniformly bound
%|>>|===========================================================================================|>>|
\begin{gather}
\label{eqn:pf-main-technical-result:intermediate:discussion:1}
  \left\| \thetaStar - \frac{\thetaXX+\hfFn[\JCoords]( \thetaStar, \thetaXX )}{\| \thetaXX+\hfFn[\JCoords]( \thetaStar, \thetaXX ) \|_{2}} \right\|_{2}
\end{gather}
%|<<|===========================================================================================|<<|
from above.
%, where in the dense regime, \(  \JCoords = [\n]  \).
\LEMMA \ref{lemma:combine} starts off by upper bounding \eqref{eqn:pf-main-technical-result:intermediate:discussion:1} by the sum of three terms (with some scaling), each of which describes how much
%a particular random vector
the functions \(  \hFn  \) and \(  \hfFn  \) (with thresholding)
deviate from their means.
Subsequently, \LEMMAS \ref{lemma:large-dist:1}--\ref{lemma:large-dist:2} provide bounds on these deviations.

%|>>|*******************************************************************************************|>>|
%|>>|*******************************************************************************************|>>|
%|>>|*******************************************************************************************|>>|
\begin{lemma}
\label{lemma:combine}
%
%Let \(  \JS \subseteq 2^{[\n]}  \), and let \(  \ParamCover \subset \ParamSpace  \) be a \(  \tauX  \)-net over \(  \ParamSpace  \) such that \(  \thetaStar \in \ParamCover  \).
%For every \(  \thetaXX \in \ParamSpace  \), there exists \(  \thetaX \in \ParamCover  \) such that
Let
%|>>|:::::::::::::::::::::::::::::::::::::::::::::::::::::::::::::::::::::::::::::::::::::::::::|>>|
\(  \JCoords \subseteq [\n]  \),
%|<<|:::::::::::::::::::::::::::::::::::::::::::::::::::::::::::::::::::::::::::::::::::::::::::|<<|
and fix
%|>>|:::::::::::::::::::::::::::::::::::::::::::::::::::::::::::::::::::::::::::::::::::::::::::|>>|
\(  \thetaStar, \thetaX, \thetaXX \in \ParamSpace  \)
%|<<|:::::::::::::::::::::::::::::::::::::::::::::::::::::::::::::::::::::::::::::::::::::::::::|<<|
such that
%|>>|:::::::::::::::::::::::::::::::::::::::::::::::::::::::::::::::::::::::::::::::::::::::::::|>>|
\(  \Supp( \thetaXX ) \cup \JCoords = \Supp( \thetaX ) \cup \JCoords  \).
%|<<|:::::::::::::::::::::::::::::::::::::::::::::::::::::::::::::::::::::::::::::::::::::::::::|<<|
Then,
%|>>|===========================================================================================|>>|
\begin{align}
  \left\| \thetaStar - \frac{\thetaXX+\hfFn[\JCoords]( \thetaStar, \thetaXX )}{\| \thetaXX+\hfFn[\JCoords]( \thetaStar, \thetaXX ) \|_{2}} \right\|_{2}
  &\leq
  \frac
  {2 \| \hFn[\JCoords]( \thetaStar, \thetaX ) - \E[ \hFn[\JCoords]( \thetaStar, \thetaX ) ] \|_{2}}
  {\DENOM}
  \nonumber \\
  &\AlignSp+
  \frac
  {2 \| \hFn[\Supp( \thetaStar ) \cup \JCoords]( \thetaX, \thetaXX ) - \E[ \hFn[\Supp( \thetaStar ) \cup \JCoords]( \thetaX, \thetaXX ) ] \|_{2}}
  {\DENOM}
  \nonumber \\
  &\AlignSp+
  \frac
  {2 \| \hfFn[\Supp( \thetaX ) \cup \JCoords]( \thetaStar, \thetaStar ) - \E[ \hfFn[\Supp( \thetaX ) \cup \JCoords]( \thetaStar, \thetaStar ) ] \|_{2}}
  {\DENOM}
\label{eqn:lemma:combine:1}
.\end{align}
%|<<|===========================================================================================|<<|
\end{lemma}
%|<<|*******************************************************************************************|<<|
%|<<|*******************************************************************************************|<<|
%|<<|*******************************************************************************************|<<|

\LEMMA \ref{lemma:combine} motivates three additional results, presented next in \LEMMAS \ref{lemma:large-dist:1}--\ref{lemma:large-dist:2}.
%Note that whereas \LEMMAS \ref{lemma:large-dist:1}--\ref{lemma:large-dist:2} are probabilistic results, \LEMMA \ref{lemma:combine} holds deterministically.
Note that whereas \LEMMA \ref{lemma:combine} holds deterministically, \LEMMAS \ref{lemma:large-dist:1}--\ref{lemma:large-dist:2} are probabilistic results.

%|>>|*******************************************************************************************|>>|
%|>>|*******************************************************************************************|>>|
%|>>|*******************************************************************************************|>>|
\begin{lemma}
\label{lemma:large-dist:1}
%
Let
%|>>|:::::::::::::::::::::::::::::::::::::::::::::::::::::::::::::::::::::::::::::::::::::::::::|>>|
\(  \rhoLDX, \deltaX \in (0,1)  \),
%|<<|:::::::::::::::::::::::::::::::::::::::::::::::::::::::::::::::::::::::::::::::::::::::::::|<<|
and define
%|>>|:::::::::::::::::::::::::::::::::::::::::::::::::::::::::::::::::::::::::::::::::::::::::::|>>|
\(  \tauX = \tauX( \deltaX )  \)
%|<<|:::::::::::::::::::::::::::::::::::::::::::::::::::::::::::::::::::::::::::::::::::::::::::|<<|
according to \DEFINITION \ref{def:nu-and-tau}.
Fix
%|>>|:::::::::::::::::::::::::::::::::::::::::::::::::::::::::::::::::::::::::::::::::::::::::::|>>|
\(  \thetaStar \in \ParamSpace  \),
%|<<|:::::::::::::::::::::::::::::::::::::::::::::::::::::::::::::::::::::::::::::::::::::::::::|<<|
and let \(  \JS \subseteq 2^{[\n]}  \) and \(  \ParamCover \subset \ParamSpace  \) be finite sets.
Define \(  \kO \defeq \kOExpr  \).
If
%|>>|===========================================================================================|>>|
\begin{gather}
\label{eqn:lemma:large-dist:1:m}
  \m
  \geq
  \frac{16}{\GAMMAX^{2} \deltaX}
  \max \left\{
    27\pi \log \left( \frac{12}{\rhoLDX} | \JS | | \ParamCover | \right)
    ,
    4 ( \kO-2 )
  \right\}
%
% \frac{8}{\pi \GAMMAX^{2} \deltaX}
%   \max \left\{
%     54\pi^{2} \log \left( \frac{12}{\rhoLDX} | \JS | | \ParamCover | \right)
%     ,
%     8\pi ( \kO-2 )
%   \right\}
%
%
%  +
%  \frac{4 \ConstbLD^{2}}{\GAMMAX^{2} \deltaX^{2}}
%  \max \left\{
%    24\pi \alphaX \log \left( \frac{12}{\rhoLDX} | \JS | | \ParamCover | \right)
%    ,
%    16\pi \alphaX ( \kOX-1 )
%  \right\}
,\end{gather}
%|<<|===========================================================================================|<<|
then with probability at least \(  1-\rhoLDX  \), uniformly for all \(  \JCoords \in \JS  \) and all \(  \thetaX \in \ParamCover \setminus \Ball{\tauX}( \thetaStar )  \),
%|>>|===========================================================================================|>>|
\begin{gather}
\label{eqn:lemma:large-dist:1:ub}
  %\left\| \thetaStar - \frac{\thetaX+\hfFn[\JCoords]( \thetaStar, \thetaX )}{\| \thetaX+\hfFn[\JCoords]( \thetaStar, \thetaX ) \|_{2}} \right\|_{2}
  \frac
  {2 \| \hFn[\JCoords]( \thetaStar, \thetaX ) - \E[ \hFn[\JCoords]( \thetaStar, \thetaX ) ] \|_{2}}
  {\DENOM}
  \leq
  \sqrt{\deltaX \EDIST} %+ \ConstbLD \deltaX
  %3 \sqrt{\frac{\pi}{\2}} \| \thetaStar - \thetaX \|_{2}
.\end{gather}
%|<<|===========================================================================================|<<|
\end{lemma}
%|<<|*******************************************************************************************|<<|
%|<<|*******************************************************************************************|<<|
%|<<|*******************************************************************************************|<<|

%|>>|*******************************************************************************************|>>|
%|>>|*******************************************************************************************|>>|
%|>>|*******************************************************************************************|>>|
\begin{lemma}%[{\cf \cite[{\LEMMA A.4 \ToDo{Check the reference's numbering.}}]{matsumoto2022binary}}]
\label{lemma:small-dist}
%
Let
%|>>|:::::::::::::::::::::::::::::::::::::::::::::::::::::::::::::::::::::::::::::::::::::::::::|>>|
\(  \ConstbSD, \ConstdSD > 0  \)
%|<<|:::::::::::::::::::::::::::::::::::::::::::::::::::::::::::::::::::::::::::::::::::::::::::|<<|
be constants specified in \DEFINITION \ref{def:univ-const}.
Let
%|>>|:::::::::::::::::::::::::::::::::::::::::::::::::::::::::::::::::::::::::::::::::::::::::::|>>|
\(  \rhoSD, \deltaX \in (0,1)  \),
%|<<|:::::::::::::::::::::::::::::::::::::::::::::::::::::::::::::::::::::::::::::::::::::::::::|<<|
and define
%|>>|:::::::::::::::::::::::::::::::::::::::::::::::::::::::::::::::::::::::::::::::::::::::::::|>>|
\(  \nuX = \nuX( \deltaX )  \) and \(  \tauX = \tauX( \deltaX )  \)
%|<<|:::::::::::::::::::::::::::::::::::::::::::::::::::::::::::::::::::::::::::::::::::::::::::|<<|
according to \DEFINITION \ref{def:nu-and-tau}.
Let
%%|>>|:::::::::::::::::::::::::::::::::::::::::::::::::::::::::::::::::::::::::::::::::::::::::::|>>|
%\(  \ParamSpace \subseteq \Sphere{\n}  \),
%%|<<|:::::::::::::::::::::::::::::::::::::::::::::::::::::::::::::::::::::::::::::::::::::::::::|<<|
%and let
%|>>|:::::::::::::::::::::::::::::::::::::::::::::::::::::::::::::::::::::::::::::::::::::::::::|>>|
\(  \ParamCover \subset \ParamSpace  \)
%|<<|:::::::::::::::::::::::::::::::::::::::::::::::::::::::::::::::::::::::::::::::::::::::::::|<<|
be a finite set, %of the parameter space.
and fix
%|>>|:::::::::::::::::::::::::::::::::::::::::::::::::::::::::::::::::::::::::::::::::::::::::::|>>|
\(  \thetaStar \in \ParamSpace  \).
%|<<|:::::::::::::::::::::::::::::::::::::::::::::::::::::::::::::::::::::::::::::::::::::::::::|<<|
Let \(  \JS, \JSXX \subseteq 2^{[\n]}  \), where
%|>>|:::::::::::::::::::::::::::::::::::::::::::::::::::::::::::::::::::::::::::::::::::::::::::|>>|
\(  \JSXX \defeq \{ \Supp( \thetaStar ) \cup \JCoords : \JCoords \in \JS \}  \).
%|<<|:::::::::::::::::::::::::::::::::::::::::::::::::::::::::::::::::::::::::::::::::::::::::::|<<|
Set
%|>>|:::::::::::::::::::::::::::::::::::::::::::::::::::::::::::::::::::::::::::::::::::::::::::|>>|
%\(  \nO \defeq \max_{\thetaX \in \ParamSpace} \| \thetaX \|_{0}  \) and
\(  \kOXX \defeq \kOXXExpr  \).
%|<<|:::::::::::::::::::::::::::::::::::::::::::::::::::::::::::::::::::::::::::::::::::::::::::|<<|
%such that
%%|>>|:::::::::::::::::::::::::::::::::::::::::::::::::::::::::::::::::::::::::::::::::::::::::::|>>|
%\(  \thetaStar \in \ParamCover  \).
%%|<<|:::::::::::::::::::::::::::::::::::::::::::::::::::::::::::::::::::::::::::::::::::::::::::|<<|
%\ToDo{Check if the condition \(  \thetaStar \in \ParamCover  \) can be removed here and elsewhere (this should be possible, but keeping it should be fine, too).}
If
%|>>|===========================================================================================|>>|
\begin{gather}
\label{eqn:lemma:small-dist:m}
  \m
  \geq
  \max \left\{
  \frac{200 \nuX \log \left( \frac{6}{\rhoSD} | \JS | | \ParamCover | \right)}{\left( \sqrt{\frac{\pi}{8}} \GAMMAX \ConstbSD \deltaX - \nuX \sqrt{8 \log \left( \frac{e}{\nuX} \right)} \right)^{2}}
  ,
  \frac{200 \nuX \kOXX}{\pi \GAMMAX^{2} \ConstbSD^{2} \deltaX^{2}}
%  ,
%  \frac{64}{\nuX} \log \left( \frac{6}{\rhoSD} | \ParamCover | \right)
  ,
  \frac{64}{\nuX} \log \left( \frac{6}{\rhoSD} \binom{\n}{\nO} \right)
  ,
  \frac{\ConstdSD \nO}{\nuX} \log \left( \frac{1}{\nuX} \right)
  \right\}
%  \max \!\!\begin{array}[t]{l} \displaystyle \Biggl\{
%  \frac{4}{\GAMMAX^{2} \ConstbSD^{2} \deltaX^{2}}
%  \max \!\!\begin{array}[t]{l} \displaystyle \Biggl\{
%    200 \nuX \log \left( \frac{6}{\rhoSD} | \JS | | \ParamCover | \qXExpr \right)
%    ,
%    25 \nuX \kOXX
%  \Biggr\}, \end{array}
%  \\ \displaystyle \phantom{\Bigg\{}
%  \frac{64}{\nuX} \log \left( \frac{6}{\rhoSD} | \ParamCover | \right)
%  ,
%  \frac{\ConstdSD \nO}{\nuX} \log \left( \frac{1}{\nuX} \right)
%  \Biggr\}, \end{array}
,\end{gather}
%|<<|===========================================================================================|<<|
then with probability at least \(  1-\rhoSD  \), uniformly for all
\(  \JCoordsXX \in \JSXX  \),
\(  \thetaX \in \ParamCover \setminus \Ball{\tauX}( \thetaStar )  \), and
\(  \thetaXX \in \BallX{2\tauX}( \thetaX )  \),
%|>>|===========================================================================================|>>|
\begin{gather}
\label{eqn:lemma:small-dist:ub}
  \frac
  {2 \| \hFn[\JCoordsXX]( \thetaX, \thetaXX ) - \E[ \hFn[\JCoordsXX]( \thetaX, \thetaXX ) ] \|_{2}}
  {\DENOM}
  \leq
  \ConstbSD \deltaX
.\end{gather}
%|<<|===========================================================================================|<<|
% \COMMENTOUT{%
% \?{I am pretty sure we can/should delete this.}
% Alternatively, taking
% %|>>|:::::::::::::::::::::::::::::::::::::::::::::::::::::::::::::::::::::::::::::::::::::::::::|>>|
% \(  \JS = \JSXX = \{ [\n] \}  \),
% %|<<|:::::::::::::::::::::::::::::::::::::::::::::::::::::::::::::::::::::::::::::::::::::::::::|<<|
% and setting
% %|>>|:::::::::::::::::::::::::::::::::::::::::::::::::::::::::::::::::::::::::::::::::::::::::::|>>|
% \(  \nO \defeq \n  \),
% %|<<|:::::::::::::::::::::::::::::::::::::::::::::::::::::::::::::::::::::::::::::::::::::::::::|<<|
% with probability at least \(  1 - \rhoSD  \), for all
% \(  \thetaX \in \ParamCover \setminus \Ball{\tauX}( \thetaStar )  \) and
% \(  \thetaXX \in \Ball{2\tauX}( \thetaX )  \),
% %|>>|===========================================================================================|>>|
% \begin{gather}
% \label{eqn:lemma:small-dist:ub:dense}
%   \frac
%   {2 \| \hFn( \thetaX, \thetaXX ) - \E[ \hFn( \thetaX, \thetaXX ) ] \|_{2}}
%   {\DENOM}
%   \leq
%   \ConstbSD \deltaX
% .\end{gather}
% %|<<|===========================================================================================|<<|
% }
\end{lemma}
%|<<|*******************************************************************************************|<<|
%|<<|*******************************************************************************************|<<|
%|<<|*******************************************************************************************|<<|

%|>>|*******************************************************************************************|>>|
%|>>|*******************************************************************************************|>>|
%|>>|*******************************************************************************************|>>|
\begin{lemma}
\label{lemma:large-dist:2}
%
Let
%|>>|:::::::::::::::::::::::::::::::::::::::::::::::::::::::::::::::::::::::::::::::::::::::::::|>>|
\(  \ConstbLD > 0  \)
%|<<|:::::::::::::::::::::::::::::::::::::::::::::::::::::::::::::::::::::::::::::::::::::::::::|<<|
be a constant specified in \DEFINITION \ref{def:univ-const}.
Let
%|>>|:::::::::::::::::::::::::::::::::::::::::::::::::::::::::::::::::::::::::::::::::::::::::::|>>|
\(  \rhoLDXX, \deltaX \in (0,1)  \),
%|<<|:::::::::::::::::::::::::::::::::::::::::::::::::::::::::::::::::::::::::::::::::::::::::::|<<|
and define
%|>>|:::::::::::::::::::::::::::::::::::::::::::::::::::::::::::::::::::::::::::::::::::::::::::|>>|
\(  \alphaO = \alphaO( \deltaX ) \defeq \alphaOExpr  \).
%|<<|:::::::::::::::::::::::::::::::::::::::::::::::::::::::::::::::::::::::::::::::::::::::::::|<<|
Fix
%|>>|:::::::::::::::::::::::::::::::::::::::::::::::::::::::::::::::::::::::::::::::::::::::::::|>>|
\(  \thetaStar \in \ParamSpace  \).
%|<<|:::::::::::::::::::::::::::::::::::::::::::::::::::::::::::::::::::::::::::::::::::::::::::|<<|
Let \(  \JS \subseteq 2^{[\n]}  \) and \(  \ParamCover \subset \ParamSpace  \) be finite sets,
and let
%|>>|:::::::::::::::::::::::::::::::::::::::::::::::::::::::::::::::::::::::::::::::::::::::::::|>>|
\(  \JSX \defeq \{ \Supp( \thetaX ) \cup \JCoords : \thetaX \in \ParamCover, \JCoords \in \JS \}  \).
%|<<|:::::::::::::::::::::::::::::::::::::::::::::::::::::::::::::::::::::::::::::::::::::::::::|<<|
Define \(  \kOX \defeq \kOXExpr  \).
If
%|>>|===========================================================================================|>>|
\begin{gather}
\label{eqn:lemma:large-dist:2:m}
  \m
  \geq
%%  \frac{4}{\GAMMAX^{2} \deltaX}
%%  \max \left\{
%%    54\pi^{2} \log \left( \frac{24}{\rhoLDX} | \JS | | \ParamCover | \right)
%%    ,
%%    8\pi ( \kO-2 )
%%  \right\}
%%  +
%  \frac{4}{\GAMMAX^{2} \ConstbLD^{2} \deltaX^{2}}
%  \max \left\{
%    24\pi \alphaX \log \left( \frac{6}{\rhoLDXX} | \JS | | \ParamCover | \right)
%    ,
%    16\pi \alphaX ( \kOX-1 )
%  \right\}
%
  \max \left\{
  \frac{64 \alphaO}{\GAMMAX^{2} \ConstbLD^{2} \deltaX^{2}}
  \max \left\{
    3 \log \left( \frac{6}{\rhoLDXX} | \JS | | \ParamCover | \right)
    ,
    2 ( \kOX-1 )
  \right\}
  ,
  \frac{4}{\alphaO} \log \left( \frac{6}{\rhoLDXX} | \JS | | \ParamCover | \right)
  \right\}
%
  % \max \left\{
  % \frac{4}{\GAMMAX^{2} \ConstbLD^{2} \deltaX^{2}}
  % \max \left\{
  %   24\pi \alphaO \log \left( \frac{6}{\rhoLDXX} | \JS | | \ParamCover | \right)
  %   ,
  %   16\pi \alphaO ( \kOX-1 )
  % \right\}
  % ,
  % \frac{4}{\alphaO} \log \left( \frac{6}{\rhoLDXX} | \JS | | \ParamCover | \right)
  % \right\}
,\end{gather}
%|<<|===========================================================================================|<<|
then with probability at least \(  1-\rhoLDXX  \), uniformly for all \(  \JCoordsX \in \JSX  \),
%|>>|===========================================================================================|>>|
\begin{gather}
\label{eqn:lemma:large-dist:2:ub}
  %\left\| \thetaStar - \frac{\thetaX+\hfFn[\JCoords]( \thetaStar, \thetaX )}{\| \thetaX+\hfFn[\JCoords]( \thetaStar, \thetaX ) \|_{2}} \right\|_{2}
  \frac
  {2 \| \hfFn[\JCoordsX]( \thetaStar, \thetaStar ) - \E[ \hfFn[\JCoordsX]( \thetaStar, \thetaStar ) ] \|_{2}}
  {\DENOM}
  \leq
  %\sqrt{\deltaX \EDIST} +
  \ConstbLD \deltaX
  %3 \sqrt{\frac{\pi}{\2}} \| \thetaStar - \thetaX \|_{2}
,\end{gather}
%|<<|===========================================================================================|<<|
for every \(  \thetaX \in \ParamSpace  \).
\end{lemma}
%|<<|*******************************************************************************************|<<|
%|<<|*******************************************************************************************|<<|
%|<<|*******************************************************************************************|<<|

%%%%%%%%%%%%%%%%%%%%%%%%%%%%%%%%%%%%%%%%%%%%%%%%%%%%%%%%%%%%%%%%%%%%%%%%%%%%%%%%%%%%%%%%%%%%%%%%%%%%
%%%%%%%%%%%%%%%%%%%%%%%%%%%%%%%%%%%%%%%%%%%%%%%%%%%%%%%%%%%%%%%%%%%%%%%%%%%%%%%%%%%%%%%%%%%%%%%%%%%%
%%%%%%%%%%%%%%%%%%%%%%%%%%%%%%%%%%%%%%%%%%%%%%%%%%%%%%%%%%%%%%%%%%%%%%%%%%%%%%%%%%%%%%%%%%%%%%%%%%%%

\subsection{Proof of \THEOREM  \ref{thm:main-technical:sparse}}
\label{outline:pf-main-technical-result|pf}

%\ToDo{Divide this up into two scenarios: (1) when \(  \thetaXX \in \Ball{\tauX}( \thetaStar )  \), and (2) when \(  \thetaXX \notin \Ball{\tauX}( \thetaStar )  \).}

%Construct a \(  \tauX  \)-net, \(  \ParamCover \subset \ParamSpace  \), over \(  \ParamSpace  \), such that \(  \thetaStar \in \ParamCover  \).

%|>>|~~~~~~~~~~~~~~~~~~~~~~~~~~~~~~~~~~~~~~~~~~~~~~~~~~~~~~~~~~~~~~~~~~~~~~~~~~~~~~~~~~~~~~~~~~~|>>|
%|>>|~~~~~~~~~~~~~~~~~~~~~~~~~~~~~~~~~~~~~~~~~~~~~~~~~~~~~~~~~~~~~~~~~~~~~~~~~~~~~~~~~~~~~~~~~~~|>>|
%|>>|~~~~~~~~~~~~~~~~~~~~~~~~~~~~~~~~~~~~~~~~~~~~~~~~~~~~~~~~~~~~~~~~~~~~~~~~~~~~~~~~~~~~~~~~~~~|>>|
An important (and standard) construct for the analysis in this work is a \(  \tauX  \)-net. %which is formalized in the following definition.
%
%|>>|###########################################################################################|>>|
%|>>|###########################################################################################|>>|
%|>>|###########################################################################################|>>|
% \begin{definition}
% \label{def:tau-net}
Fix \(  \tauX > 0  \).
Let \(  ( \SX, \dSX )  \) be a metric space.
A subset,
%|>>|:::::::::::::::::::::::::::::::::::::::::::::::::::::::::::::::::::::::::::::::::::::::::::|>>|
\(  \SXX \subseteq \SX  \),
%|<<|:::::::::::::::::::::::::::::::::::::::::::::::::::::::::::::::::::::::::::::::::::::::::::|<<|
is a \emph{\(  \tauX  \)-net} over \(  \SX  \) if
%|>>|:::::::::::::::::::::::::::::::::::::::::::::::::::::::::::::::::::::::::::::::::::::::::::|>>|
\(  \inf_{\sSXX \in \SXX} \dSX( \sSX, \sSXX ) \leq \tauX  \)
%|<<|:::::::::::::::::::::::::::::::::::::::::::::::::::::::::::::::::::::::::::::::::::::::::::|<<|
for all \(  \sSX \in \SX  \).
%\end{definition}
%|<<|###########################################################################################|<<|
%|<<|###########################################################################################|<<|
%|<<|###########################################################################################|<<|
%
%Later on, the following upper bound on the minimal cardinality of a \(  \tauX  \)-net over a hypersphere sphere or a union of hyperspheres will be useful.
We will use the following upper bound on the minimal cardinality of a \(  \tauX  \)-net of a sphere.
%
%|>>|*******************************************************************************************|>>|
%|>>|*******************************************************************************************|>>|
%|>>|*******************************************************************************************|>>|
\begin{lemma}[{\see \eg \cite{vershynin2018high}}]
\label{lemma:tau-net-cardinality}
%
Fix \(  \tauX > 0  \), and
%Let \(  \dDim, \n \in \Z_{+}  \), \(  \dDim \leq \n  \).
let \(  \dDim \in \Z_{+}  \).
%The cardinality of a \(  \tauX  \)-net, \(  \SXX \subset \Sphere{\n}  \), over \(  \Sphere{\n}  \) need not exceed
There exists an $\ell_2$ \(  \tauX  \)-net, \(  \SXX \subset \Sphere{\dDim}  \), over \(  \Sphere{\dDim}  \) of cardinality not exceeding
%|>>|:::::::::::::::::::::::::::::::::::::::::::::::::::::::::::::::::::::::::::::::::::::::::::|>>|
\(  | \SXX | \leq ( \frac{3}{\tauX} )^{\dDim}  \).
%|<<|:::::::::::::::::::::::::::::::::::::::::::::::::::::::::::::::::::::::::::::::::::::::::::|<<|
%Moreover, there exists a union of \(  \tauX  \)-nets, \(  \bigcup_{\iIx=1}^{\binom{\n}{\k}}  \), over \(  \SparseSphereSubspace{\dDim}{\n}  \)
\end{lemma}
%|<<|*******************************************************************************************|<<|
%|<<|*******************************************************************************************|<<|
%|
\begin{proof}{\THEOREM \ref{thm:main-technical:sparse}}
%
\mostlycheckoff%
%
Fix
%|>>|:::::::::::::::::::::::::::::::::::::::::::::::::::::::::::::::::::::::::::::::::::::::::::|>>|
\(  \thetaStar \in \ParamSpace  \).
%|<<|:::::::::::::::::::::::::::::::::::::::::::::::::::::::::::::::::::::::::::::::::::::::::::|<<|
%Let
%%|>>|:::::::::::::::::::::::::::::::::::::::::::::::::::::::::::::::::::::::::::::::::::::::::::|>>|
%\(  \ParamSpace = \SparseSphereSubspace{\k}{\n}  \),
%%|<<|:::::::::::::::::::::::::::::::::::::::::::::::::::::::::::::::::::::::::::::::::::::::::::|<<|
%and let
Let
%|>>|:::::::::::::::::::::::::::::::::::::::::::::::::::::::::::::::::::::::::::::::::::::::::::|>>|
\(  \JS \subseteq 2^{[\n]}  \)
%|<<|:::::::::::::::::::::::::::::::::::::::::::::::::::::::::::::::::::::::::::::::::::::::::::|<<|
be the set of coordinate subsets with cardinality at most \(  \k  \)---that is, the set given by
%|>>|===========================================================================================|>>|
\(
  \JS
  \defeq
  {\{ \JCoords \subseteq [\n] : | \JCoords | \leq \k \}}
\).
%|<<|===========================================================================================|<<|
Construct a \(  \tauX  \)-net, \(  \ParamCover \subset \ParamSpace  \), over \(  \ParamSpace  \) with the following design.
%such that \(  \thetaStar \in \ParamCover  \), and with the following construction.
\checkthis[Done]%
For each \(  \JCoords \in \JS  \), let \(  \ParamCoverJ \subset \ParamSpace  \) be a \(  \tauX  \)-net over the set of points in \(  \ParamSpace  \) whose support is a subset of \(  \JCoords  \)---formally, over the set
%|>>|:::::::::::::::::::::::::::::::::::::::::::::::::::::::::::::::::::::::::::::::::::::::::::|>>|
\(  \{ \Vec{v} \in \ParamSpace : \Supp( \Vec{v} ) \subseteq  \JCoords \}  \)%
%|<<|:::::::::::::::::::::::::::::::::::::::::::::::::::::::::::::::::::::::::::::::::::::::::::|<<|
---such that each vector in the cover, \(  \ParamCoverJ  \), has support exactly \(  \JCoords  \), \ie
%|>>|:::::::::::::::::::::::::::::::::::::::::::::::::::::::::::::::::::::::::::::::::::::::::::|>>|
\(  \Supp( \Vec{v} ) = \JCoords  \) for all \(  \Vec{v} \in \ParamCoverJ  \).
%|<<|:::::::::::::::::::::::::::::::::::::::::::::::::::::::::::::::::::::::::::::::::::::::::::|<<|
(This last condition on the support of elements in the \(  \tauX  \)-net is possible through a rotation.)
Then, let
%|>>|:::::::::::::::::::::::::::::::::::::::::::::::::::::::::::::::::::::::::::::::::::::::::::|>>|
\(  \ParamCover \defeq \bigcup_{\JCoords \in \JS} \ParamCoverJ  \).
%|<<|:::::::::::::::::::::::::::::::::::::::::::::::::::::::::::::::::::::::::::::::::::::::::::|<<|
%Write
%%|>>|:::::::::::::::::::::::::::::::::::::::::::::::::::::::::::::::::::::::::::::::::::::::::::|>>|
%\(  \ParamCoverX \defeq \ParamCover \setminus \Ball{\tauX}( \thetaStar )  \).
%%|<<|:::::::::::::::::::::::::::::::::::::::::::::::::::::::::::::::::::::::::::::::::::::::::::|<<|
Note that this construction ensures that for every point in the parameter space, \(  \ParamSpace  \), the cover, \(  \ParamCover  \), contains at least one point within distance \(  \tauX  \) of it and with precisely the same support.
Additionally, the cardinalities of \(  \JS  \) and \(  \ParamCover  \) satisfy
%|>>|:::::::::::::::::::::::::::::::::::::::::::::::::::::::::::::::::::::::::::::::::::::::::::|>>|
\(  | \JS | = \JSSIZES  \) and
\(  | \ParamCover | \leq \sum_{\JCoords \in \JS} ( \frac{3}{\tauX} )^{| \JCoords |} = \PCSIZES  \),
%|<<|:::::::::::::::::::::::::::::::::::::::::::::::::::::::::::::::::::::::::::::::::::::::::::|<<|
where the bound on \(  | \ParamCover |  \) is due to \LEMMA \ref{lemma:tau-net-cardinality} combined with a union bound.
%
%%%%%%%%%%%%%%%%%%%%%%%%%%%%%%%%%%%%%%%%%%%%%%%%%%%%%%%%%%%%%%%%%%%%%%%%%%%%%%%%%%%%%%%%%%%%%%%%%%%%
\par %%%%%%%%%%%%%%%%%%%%%%%%%%%%%%%%%%%%%%%%%%%%%%%%%%%%%%%%%%%%%%%%%%%%%%%%%%%%%%%%%%%%%%%%%%%%%%%
%%%%%%%%%%%%%%%%%%%%%%%%%%%%%%%%%%%%%%%%%%%%%%%%%%%%%%%%%%%%%%%%%%%%%%%%%%%%%%%%%%%%%%%%%%%%%%%%%%%%
%
Consider an arbitrary choice of
%|>>|:::::::::::::::::::::::::::::::::::::::::::::::::::::::::::::::::::::::::::::::::::::::::::|>>|
\(  \thetaXX \in \ParamSpace  \),
%|<<|:::::::::::::::::::::::::::::::::::::::::::::::::::::::::::::::::::::::::::::::::::::::::::|<<|
to later be varied over the entire parameter space, \(  \ParamSpace  \), and let
%|>>|:::::::::::::::::::::::::::::::::::::::::::::::::::::::::::::::::::::::::::::::::::::::::::|>>|
\(  \thetaX \in \ParamCover  \)
%|<<|:::::::::::::::::::::::::::::::::::::::::::::::::::::::::::::::::::::::::::::::::::::::::::|<<|
satisfy
%|>>|:::::::::::::::::::::::::::::::::::::::::::::::::::::::::::::::::::::::::::::::::::::::::::|>>|
%\(  \| \thetaStar-\thetaX \|_{2} \geq \tauX  \) and
\(  \thetaX \notin \Ball{\tauX}( \thetaStar )  \) and
%\(  \| \thetaX-\thetaXX \|_{2} \leq 2\tauX  \), and
\(  \thetaXX \in \BallXX{2\tauX}( \thetaX )  \),
%|<<|:::::::::::::::::::::::::::::::::::::::::::::::::::::::::::::::::::::::::::::::::::::::::::|<<|
%for all \(  \JCoords \in \JS  \),
where such a point, \(  \thetaX  \), exists in the \(  \tauX  \)-net, \(  \ParamCover  \), by its design.
Note that this ensures that
%|>>|:::::::::::::::::::::::::::::::::::::::::::::::::::::::::::::::::::::::::::::::::::::::::::|>>|
\(  \Supp( \thetaX ) \cup \JCoords = \Supp( \thetaXX ) \cup \JCoords  \)
%|<<|:::::::::::::::::::::::::::::::::::::::::::::::::::::::::::::::::::::::::::::::::::::::::::|<<|
for all \(  \JCoords \in \JS  \), and hence,
%Note that
%%|>>|:::::::::::::::::::::::::::::::::::::::::::::::::::::::::::::::::::::::::::::::::::::::::::|>>|
%\(  \thetaXX \in \BallX{2\tauX}( \thetaX )  \).
%%|<<|:::::::::::::::::::::::::::::::::::::::::::::::::::::::::::::::::::::::::::::::::::::::::::|<<|
by \LEMMA \ref{lemma:combine},
%|>>|===========================================================================================|>>|
\begin{align*}
  \left\| \thetaStar - \frac{\thetaXX+\hfFn[\JCoords]( \thetaStar, \thetaXX )}{\| \thetaXX+\hfFn[\JCoords]( \thetaStar, \thetaXX ) \|_{2}} \right\|_{2}
  &\leq
  \frac
  {2 \| \hFn[\JCoords]( \thetaStar, \thetaX ) - \E[ \hFn[\JCoords]( \thetaStar, \thetaX ) ] \|_{2}}
  {\DENOM}
  \\
  &\AlignSp+
  \frac
  {2 \| \hFn[\Supp( \thetaStar ) \cup \JCoords]( \thetaX, \thetaXX ) - \E[ \hFn[\Supp( \thetaStar ) \cup \JCoords]( \thetaX, \thetaXX ) ] \|_{2}}
  {\DENOM}
  \\
  &\AlignSp+
  \frac
  {2 \| \hfFn[\Supp( \thetaX ) \cup \JCoords]( \thetaStar, \thetaStar ) - \E[ \hfFn[\Supp( \thetaX ) \cup \JCoords]( \thetaStar, \thetaStar ) ] \|_{2}}
  {\DENOM}
\TagEqn{\label{eqn:pf:thm:main-technical:1}}
,\end{align*}
%|<<|===========================================================================================|<<|
where this bound holds deterministically.
%Note that in the dense regime, where \(  \JS = \{ [\n] \}  \),
%%|>>|:::::::::::::::::::::::::::::::::::::::::::::::::::::::::::::::::::::::::::::::::::::::::::|>>|
%\(  \hFn[\Supp( \thetaStar ) \cup \JCoords] = \hFn[{[\n]}] = \hFn  \).
%%|<<|:::::::::::::::::::::::::::::::::::::::::::::::::::::::::::::::::::::::::::::::::::::::::::|<<|
Now, suppose
%|>>|===========================================================================================|>>|
\begin{align*}
  \m
  &\geq
  \mEXPR[s][.]
%  \\
%  &\geq
%  \max \!\!\begin{array}[t]{l} \displaystyle \Biggl\{
%    \frac{216\pi^{2}}{\GAMMAX^{2} \deltaX}
%    \log \left( \frac{24}{\rhoX} | \JS | | \ParamCover |  \right)
%    ,\\ \displaystyle \phantom{\Bigg\{}
%    \frac{96\pi \alphaX}{\GAMMAX^{2} \ConstbLD^{2} \deltaX^{2}}
%    \log \left( \frac{24}{\rhoX} | \JS | | \ParamCover | \right)
%    ,\\ \displaystyle \phantom{\Bigg\{}
%    \frac{400}{\GAMMAX^{2} \Constb \ConstbSD^{2} \deltaX \sqrt{\log \left( \frac{4e}{\nuX} \right)}}
%    \log \left( \frac{24}{\rhoX} | \JS | | \ParamCover | \qXExpr \right)
%    ,\\ \displaystyle \phantom{\Bigg\{}
%    \frac{64}{\nuX} \log \left( \frac{24}{\rhoX} | \ParamCover | \right)
%    ,\\ \displaystyle \phantom{\Bigg\{}
%    \frac{\ConstdSD \nO}{\nuX} \log \left( \frac{1}{\nuX} \right)
%  \Biggr\} \end{array}
%  \\
%  &=
%  \max \!\!\begin{array}[t]{l} \displaystyle \Biggl\{
%    \frac{216\pi^{2}}{\GAMMAX^{2} \deltaX}
%    \log \left( \frac{24}{\rhoX} | \JS | | \ParamCover | \right)
%    ,\\ \displaystyle \phantom{\Bigg\{}
%    \frac{96\pi \alphaX}{\GAMMAX^{2} \ConstbLD^{2} \deltaX^{2}}
%    \log \left( \frac{24}{\rhoX} | \JS | | \ParamCover | \right)
%    ,\\ \displaystyle \phantom{\Bigg\{}
%    \frac{800 \nuX}{\GAMMAX^{2} \ConstbSD^{2} \deltaX^{2}}
%    \log \left( \frac{24}{\rhoX} | \JS | | \ParamCover | \qXExpr \right)
%    ,\\ \displaystyle \phantom{\Bigg\{}
%    \frac{64}{\nuX} \log \left( \frac{24}{\rhoX} | \ParamCover | \right)
%    ,\\ \displaystyle \phantom{\Bigg\{}
%    \frac{\ConstdSD \nO}{\nuX} \log \left( \frac{1}{\nuX} \right)
%  \Biggr\}. \end{array}
\end{align*}
%|<<|===========================================================================================|<<|
This choice of \(  \m  \) is sufficiently large so that taking
%|>>|:::::::::::::::::::::::::::::::::::::::::::::::::::::::::::::::::::::::::::::::::::::::::::|>>|
\(  \rhoLDX = \frac{\rhoX}{2}  \),
\(  \rhoSD = \frac{\rhoX}{4}  \), and
\(  \rhoLDXX = \frac{\rhoX}{4}  \)%
%|<<|:::::::::::::::::::::::::::::::::::::::::::::::::::::::::::::::::::::::::::::::::::::::::::|<<|
---such that
%|>>|:::::::::::::::::::::::::::::::::::::::::::::::::::::::::::::::::::::::::::::::::::::::::::|>>|
\(  \rhoLDX + \rhoSD + \rhoLDXX = \rhoX  \)%
%|<<|:::::::::::::::::::::::::::::::::::::::::::::::::::::::::::::::::::::::::::::::::::::::::::|<<|
---in \LEMMAS \ref{lemma:large-dist:1}--\ref{lemma:large-dist:2}, respectively, and then combining the bounds in these lemmas with a union bound, the three terms on the \RHS of the inequality in \EQUATION \eqref{eqn:pf:thm:main-technical:1} are simultaneously bounded from above with probability at least
%|>>|:::::::::::::::::::::::::::::::::::::::::::::::::::::::::::::::::::::::::::::::::::::::::::|>>|
\(  1 - \rhoLDX - \rhoSD - \rhoLDXX = 1 - \rhoX  \)
%|<<|:::::::::::::::::::::::::::::::::::::::::::::::::::::::::::::::::::::::::::::::::::::::::::|<<|
by
first,
%|>>|===========================================================================================|>>|
\begin{align*}
  \sup_{\substack{\JCoords \in \JS ,\\
                  \thetaX \in \ParamCover \setminus \Ball{\tauX}( \thetaStar )}}
  \frac
  {2 \| \hFn[\JCoords]( \thetaStar, \thetaX ) - \E[ \hFn[\JCoords]( \thetaStar, \thetaX ) ] \|_{2}}
  {\DENOM}
  \leq
  \sqrt{\deltaX \EDIST}
,\end{align*}
%|<<|===========================================================================================|<<|
second,
%|>>|===========================================================================================|>>|
\begin{align*}
  &\negphantom{\AlignSp}
  \sup_{\substack{\JCoords \in \JS ,\\
                  \thetaX \in \ParamCover \setminus \Ball{\tauX}( \thetaStar ) ,\\
                  \thetaXX \in \BallXX{2\tauX}( \thetaX )}}
  \frac
  {2 \| \hFn[\Supp( \thetaStar ) \cup \JCoords]( \thetaX, \thetaXX ) - \E[ \hFn[\Supp( \thetaStar ) \cup \JCoords]( \thetaX, \thetaXX ) ] \|_{2}}
  {\DENOM}
  \\
  &\leq
  \sup_{\substack{\JCoordsXX \in \JSXX ,\\
                  \thetaX \in \ParamCover \setminus \Ball{\tauX}( \thetaStar ) ,\\
                  \thetaXX \in \BallXX{2\tauX}( \thetaX )}}
  \frac
  {2 \| \hFn[\JCoordsXX]( \thetaX, \thetaXX ) - \E[ \hFn[\JCoordsXX]( \thetaX, \thetaXX ) ] \|_{2}}
  {\DENOM}
  \\
  &\leq
  \ConstbSD \deltaX
  \\
  &\leq
  \left( 1-\ConstbLD-\sqrt{\frac{2\Constb}{\ConstdSD}} \right) \deltaX
  \\
  &\dCmt{by \DEFINITION \ref{def:univ-const}}
  \\
  &\leq
  \left( 1-\ConstbLD-\sqrt{\frac{2\tauX}{\deltaX}} \right) \deltaX
  ,\\
  &\dCmt{by the definitions of \(  \deltaX, \tauX  \)}
\end{align*}
%|<<|===========================================================================================|<<|
and
third,
%|>>|===========================================================================================|>>|
\begin{align*}
  \sup_{\substack{\JCoords \in \JS ,\\
                  \thetaX \in \ParamCover \setminus \Ball{\tauX}( \thetaStar )}}
  \frac
  {2 \| \hfFn[\Supp( \thetaX ) \cup \JCoords]( \thetaStar, \thetaStar ) - \E[ \hfFn[\Supp( \thetaX ) \cup \JCoords]( \thetaStar, \thetaStar ) ] \|_{2}}
  {\DENOM}
  &\leq
  \sup_{\JCoordsX \in \JSX}
  \frac
  {2 \| \hfFn[\JCoordsX]( \thetaStar, \thetaStar ) - \E[ \hfFn[\JCoordsX]( \thetaStar, \thetaStar ) ] \|_{2}}
  {\DENOM}
  \leq
  \ConstbLD \deltaX
,\end{align*}
%|<<|===========================================================================================|<<|
where
%|>>|:::::::::::::::::::::::::::::::::::::::::::::::::::::::::::::::::::::::::::::::::::::::::::|>>|
\(  \JSXX \defeq \{ \Supp( \thetaStar ) \cup \JCoords : \JCoords \in \JS \}  \)
%|<<|:::::::::::::::::::::::::::::::::::::::::::::::::::::::::::::::::::::::::::::::::::::::::::|<<|
and
%|>>|:::::::::::::::::::::::::::::::::::::::::::::::::::::::::::::::::::::::::::::::::::::::::::|>>|
\(  \JSX \defeq \{ \Supp( \thetaX ) \cup \JCoords : \thetaX \in \ParamCover, \JCoords \in \JS \}  \).
%|<<|:::::::::::::::::::::::::::::::::::::::::::::::::::::::::::::::::::::::::::::::::::::::::::|<<|
It follows that under the stated condition on \(  \m  \), with probability at least \(  1 - \rhoX  \), for all \(  \thetaXX \in \ParamSpace  \) and \(  \JCoords \in \JS  \),
%|>>|===========================================================================================|>>|
\begin{align*}
  \left\| \thetaStar - \frac{\thetaXX+\hfFn[\JCoords]( \thetaStar, \thetaXX )}{\| \thetaXX+\hfFn[\JCoords]( \thetaStar, \thetaXX ) \|_{2}} \right\|_{2}
  &\leq
  \sqrt{\deltaX \EDIST}
  +
  \left( 1-\ConstbLD-\sqrt{\frac{2\tauX}{\deltaX}} \right) \deltaX
  +
  \ConstbLD \deltaX
  \\
  &
  \dCmt{for some \(  \thetaX \in ( \ParamCover \cap \BallXX{2\tauX}( \thetaXX ) ) \setminus \Ball{\tauX}( \thetaStar )  \)}
  \\
  &=
  \sqrt{\deltaX \| ( \thetaStar-\thetaXX ) - ( \thetaX-\thetaXX ) \|_{2}}
  +
  \left( 1-\ConstbLD-\sqrt{\frac{2\tauX}{\deltaX}} \right) \deltaX
  +
  \ConstbLD \deltaX
  \\
  % &\leq
  % \sqrt{\deltaX \| \thetaStar-\thetaXX \|_{2} + \deltaX \| \thetaX-\thetaXX \|_{2}}
  % +
  % \left( 1-\ConstbLD-\sqrt{\frac{2\tauX}{\deltaX}} \right) \deltaX
  % +
  % \ConstbLD \deltaX
  % \\
  % &\dCmt{by the triangle inequality}
  % \\
  &\leq
  \sqrt{\deltaX \| \thetaStar-\thetaXX \|_{2}}
  +
  \sqrt{\deltaX \| \thetaX-\thetaXX \|_{2}}
  +
  \left( 1-\ConstbLD-\sqrt{\frac{2\tauX}{\deltaX}} \right) \deltaX
  +
  \ConstbLD \deltaX
  \\
  &\dCmt{by the triangle inequality}
  \\
  &\leq
  \sqrt{\deltaX \| \thetaStar-\thetaXX \|_{2}}
  +
  \sqrt{2 \deltaX \tauX}
  +
  \left( 1-\ConstbLD-\sqrt{\frac{2\tauX}{\deltaX}} \right) \deltaX
  +
  \ConstbLD \deltaX
  \\
  &\dCmt{\(  \because \thetaX \in \BallXX{2\tauX}( \thetaXX )  \)}
  \\
  &=
  \sqrt{\deltaX \| \thetaStar-\thetaXX \|_{2}}
  +
  \sqrt{\frac{2\tauX}{\deltaX}} \delta
  +
  \left( 1-\ConstbLD-\sqrt{\frac{2\tauX}{\deltaX}} \right) \deltaX
  +
  \ConstbLD \deltaX
  \\
  &=
  \sqrt{\deltaX \| \thetaStar-\thetaXX \|_{2}}
  +
  \deltaX
,\end{align*}
%|<<|===========================================================================================|<<|
as claimed.
\end{proof}
%|<<|~~~~~~~~~~~~~~~~~~~~~~~~~~~~~~~~~~~~~~~~~~~~~~~~~~~~~~~~~~~~~~~~~~~~~~~~~~~~~~~~~~~~~~~~~~~|<<|
%|<<|~~~~~~~~~~~~~~~~~~~~~~~~~~~~~~~~~~~~~~~~~~~~~~~~~~~~~~~~~~~~~~~~~~~~~~~~~~~~~~~~~~~~~~~~~~~|<<|
%|<<|~~~~~~~~~~~~~~~~~~~~~~~~~~~~~~~~~~~~~~~~~~~~~~~~~~~~~~~~~~~~~~~~~~~~~~~~~~~~~~~~~~~~~~~~~~~|<<|

%%%%%%%%%%%%%%%%%%%%%%%%%%%%%%%%%%%%%%%%%%%%%%%%%%%%%%%%%%%%%%%%%%%%%%%%%%%%%%%%%%%%%%%%%%%%%%%%%%%%
%%%%%%%%%%%%%%%%%%%%%%%%%%%%%%%%%%%%%%%%%%%%%%%%%%%%%%%%%%%%%%%%%%%%%%%%%%%%%%%%%%%%%%%%%%%%%%%%%%%%
%%%%%%%%%%%%%%%%%%%%%%%%%%%%%%%%%%%%%%%%%%%%%%%%%%%%%%%%%%%%%%%%%%%%%%%%%%%%%%%%%%%%%%%%%%%%%%%%%%%%

\subsection{Proof of \COROLLARY \ref{corollary:main-technical:logistic-regression}}
\label{outline:pf-main-technical-result|pf-main-corollaries}

%\ToDo{Maybe revert this notation and keep \(  \betaX  \) implicit in \(  \pFn  \).}
%
%Recall that in \COROLLARIES \ref{corollary:main-technical:logistic-regression} and \ref{corollary:main-technical:probit}, the function \(  \pFn  \) is parameterized by the \betaXname, \(  \betaX \GTR 0  \), as per \DEFINITIONS \ref{def:p:logistic-regression} and \ref{def:p:probit}, respectively.
%In the proofs of these two corollaries, specifying this parameter, \(  \betaX  \), explicitly will increase clarity, and hence, for \(  z \in \R  \) and \(  \betaX \GTR 0  \), write
%%|>>|===========================================================================================|>>|
%\begin{gather*}
%  \pbetaFn{z}
%  =
%  \frac{1}{1+e^{-\betaX z}}
%\end{gather*}
%%|<<|===========================================================================================|<<|
%in the case of logistic regression, and in the case of probit regression, write
%%|>>|===========================================================================================|>>|
%\begin{gather*}
%  \pbetaFn{z}
%  =
%  \frac{1}{\sqrt{2\pi}}
%  \int_{w=-\infty}^{w=\betaX z}
%  e^{-\frac{1}{2} w^{2}}
%  dw
%.\end{gather*}
%%|<<|===========================================================================================|<<|

%|>>|~~~~~~~~~~~~~~~~~~~~~~~~~~~~~~~~~~~~~~~~~~~~~~~~~~~~~~~~~~~~~~~~~~~~~~~~~~~~~~~~~~~~~~~~~~~|>>|
%|>>|~~~~~~~~~~~~~~~~~~~~~~~~~~~~~~~~~~~~~~~~~~~~~~~~~~~~~~~~~~~~~~~~~~~~~~~~~~~~~~~~~~~~~~~~~~~|>>|
%|>>|~~~~~~~~~~~~~~~~~~~~~~~~~~~~~~~~~~~~~~~~~~~~~~~~~~~~~~~~~~~~~~~~~~~~~~~~~~~~~~~~~~~~~~~~~~~|>>|
\begin{proof}
{\COROLLARY \ref{corollary:main-technical:logistic-regression}}
%
\mostlycheckoff%
%
The specialization of the main technical result to logistic regression in \COROLLARY \ref{corollary:main-technical:logistic-regression} requires two arguments:
\Enum[{\label{enum:pf:corollary:main-technical:logistic-regression:a}}]{a}
\ASSUMPTION \ref{assumption:p} needs to be shown to hold for logistic regression, i.e., when \(  \pFn  \) is the logistic function with \betaXnamelr \(  \betaX \GTR 0  \), as in \DEFINITION \ref{def:p:logistic-regression};
and
\Enum[{\label{enum:pf:corollary:main-technical:logistic-regression:b}}]{b}
explicit forms for the variables \(  \alphaX  \) (and \(  \alphaO  \)) and \(  \gammaX  \) need specification.
Once these are achieved, the corollary will follow from combining the bounds on \(  \alphaX  \) and \(  \gammaX  \) obtained from \TASK \ref{enum:pf:corollary:main-technical:logistic-regression:b} with \THEOREM \ref{thm:main-technical:sparse}.
Throughout this proof, \(  \pFn  \) is taken to be the logistic function, parameterized by the \betaXnamelr, \(  \betaX \GTR 0  \), per \DEFINITION \ref{def:p:logistic-regression}, which is recalled for convenience:
%|>>|===========================================================================================|>>|
\begin{gather*}
  \pbetaFn{z}
  =
  \frac{1}{1+e^{-\betaX z}}
.\end{gather*}
%|<<|===========================================================================================|<<|
%
%%%%%%%%%%%%%%%%%%%%%%%%%%%%%%%%%%%%%%%%%%%%%%%%%%%%%%%%%%%%%%%%%%%%%%%%%%%%%%%%%%%%%%%%%%%%%%%%%%%%
\par %%%%%%%%%%%%%%%%%%%%%%%%%%%%%%%%%%%%%%%%%%%%%%%%%%%%%%%%%%%%%%%%%%%%%%%%%%%%%%%%%%%%%%%%%%%%%%%
%%%%%%%%%%%%%%%%%%%%%%%%%%%%%%%%%%%%%%%%%%%%%%%%%%%%%%%%%%%%%%%%%%%%%%%%%%%%%%%%%%%%%%%%%%%%%%%%%%%%
%
For \TASK \ref{enum:pf:corollary:main-technical:logistic-regression:a}, recall that \ASSUMPTION \ref{assumption:p} imposes two conditions: \ref{condition:assumption:p:i} that \(  \pFn  \) is nondecreasing over the entire real line, and \ref{condition:assumption:p:ii} that
%|>>|:::::::::::::::::::::::::::::::::::::::::::::::::::::::::::::::::::::::::::::::::::::::::::|>>|
\(  \frac{\partial}{\partial \zX} \frac{\pExpr*{\zX+\wX \betaXParam}}{\pExpr{\zX \betaXParam}} \leq 0  \).
%|<<|:::::::::::::::::::::::::::::::::::::::::::::::::::::::::::::::::::::::::::::::::::::::::::|<<|
%Note that for the latter condition, scaling by a factor of \(  \betaX > 0  \) yields an equivalent condition for the assumption, while the assumption's condition always holds when \(  \betaX=0  \).
Let
%|>>|:::::::::::::::::::::::::::::::::::::::::::::::::::::::::::::::::::::::::::::::::::::::::::|>>|
\(  \zX < \zXX \in \R  \).
%|<<|:::::::::::::::::::::::::::::::::::::::::::::::::::::::::::::::::::::::::::::::::::::::::::|<<|
Then, clearly,
%|>>|===========================================================================================|>>|
\begin{gather*}
  \pbetaFn{ \zX }
  =
  \frac{1}{1+e^{-\betaX \zX}}
  <
  \frac{1}{1+e^{-\betaX \zXX}}
  =
  \pbetaFn{ \zXX }
,\end{gather*}
%|<<|===========================================================================================|<<|
and thus, \CONDITION \ref{condition:assumption:p:i} holds.
On the other hand, \CONDITION \ref{condition:assumption:p:ii} can be established via basis calculus.
First, note that for any \(  \zX \in \R  \),
%|>>|===========================================================================================|>>|
\begin{align}
\label{eqn:pf:corollary:main-technical:logistic-regression:7}
  1-\pbetaFn{ \zX }
  =
  1 - \frac{1}{1+e^{-\betaX \zX}}
  =
  \frac{e^{-\betaX \zX}}{1+e^{-\betaX \zX}}
  =
  \frac{1}{1+e^{\betaX \zX}}
  =
  \pbetaFn{ -\zX }
,\end{align}
%|<<|===========================================================================================|<<|
and hence, for \(  \wX, \zX \in \R  \), \(  \wX > 0  \),
%|>>|===========================================================================================|>>|
\begin{align*}
  \frac{\pExpr*{\zX+\wX \betaXParam}}{\pExpr{\zX \betaXParam}}
  =
  \frac{2 \pbetaFn{ -( \zX+\wX } ) }{2 \pbetaFn{ -\zX }}
  =
  \frac{\pbetaFn{ -( \zX+\wX } ) }{\pbetaFn{ -\zX }}
  =
  \frac{1+e^{\betaX \zX}}{1+e^{\betaX ( \zX+\wX )}}
.\end{align*}
%|<<|===========================================================================================|<<|
Then,
%|>>|===========================================================================================|>>|
\begin{align*}
  \frac{\partial}{\partial \zX} \frac{\pExpr*{\zX+\wX}}{\pExpr{\zX}}
  &=
  \frac{\partial}{\partial \zX} \frac{1+e^{\betaX \zX}}{1+e^{\betaX ( \zX+\wX )}}
  =
  -\frac
  {\betaX e^{\betaX \zX} ( e^{\betaX \wX}-1 )}
  {( 1+e^{\betaX ( \zX+\wX )} )^{2}}
  \leq 0
,\end{align*}
%|<<|===========================================================================================|<<|
as desired.
Thus, \CONDITION \ref{condition:assumption:p:ii} also holds when \(  \pFn  \) is the logistic function.
This complete \TASK \ref{enum:pf:corollary:main-technical:logistic-regression:a}.
%
%%%%%%%%%%%%%%%%%%%%%%%%%%%%%%%%%%%%%%%%%%%%%%%%%%%%%%%%%%%%%%%%%%%%%%%%%%%%%%%%%%%%%%%%%%%%%%%%%%%%
\par %%%%%%%%%%%%%%%%%%%%%%%%%%%%%%%%%%%%%%%%%%%%%%%%%%%%%%%%%%%%%%%%%%%%%%%%%%%%%%%%%%%%%%%%%%%%%%%
%%%%%%%%%%%%%%%%%%%%%%%%%%%%%%%%%%%%%%%%%%%%%%%%%%%%%%%%%%%%%%%%%%%%%%%%%%%%%%%%%%%%%%%%%%%%%%%%%%%%
%
Proceeding to \TASK \ref{enum:pf:corollary:main-technical:logistic-regression:b}, the aim now is to derive closed-form bounds on \(  \alphaX   \) and \(  \gammaX  \).
Looking first at \(  \alphaX  \), recall its definition from \EQUATION \eqref{eqn:notations:alpha:def}:
%|>>|===========================================================================================|>>|
\begin{gather*}
  \alphaX
  \defeq
  \Pr_{Z \sim \N(0,1)} (
    \fFn( Z ) \neq \Sign( Z )
  )
  =
  \frac{1}{\sqrt{2\pi}}
  \int_{\zX=0}^{\zX=\infty}
  e^{-\frac{1}{2} \zX^{2}}
  (\pExpr{\zX})
  d\zX
.\end{gather*}
%|<<|===========================================================================================|<<|
By the earlier observation in \eqref{eqn:pf:corollary:main-technical:logistic-regression:7}, when \(  \pFn  \) is the logistic function,
%|>>|===========================================================================================|>>|
\begin{align*}
  \alphaX
  &=
  \frac{1}{\sqrt{2\pi}}
  \int_{\zX=0}^{\zX=\infty}
  e^{-\frac{1}{2} \zX^{2}}
  (\pExpr{\zX})
  d\zX
  \\
  &\dCmt{by \EQUATION \eqref{eqn:notations:alpha:def}}
  \\
  &=
  \sqrt{\frac{2}{\pi}}
  \int_{\zX=0}^{\zX=\infty}
  e^{-\frac{1}{2} \zX^{2}}
  \pbetaFn{ -\zX }
  d\zX
  \\
  &\dCmt{by \EQUATION \eqref{eqn:pf:corollary:main-technical:logistic-regression:7}}
  \\
  &=
  \E \left[ \pbetaFn{ -| \ZRV | } \right]
%\TagEqn{\label{eqn:pf:corollary:main-technical:logistic-regression:2}}
  \\
  &\dCmt{by the law of the lazy statistician and the density function}
  \\
  &\dCmtIndent \text{for standard half-normal random variables}
  \\
  &=
  \E \left[ \frac{1}{1 + e^{\betaX | \ZRV |}} \right]
\TagEqn{\label{eqn:pf:corollary:main-technical:logistic-regression:1}}
,\end{align*}
%|<<|===========================================================================================|<<|
where \(  \ZRV \sim \N(0,1)  \) is a standard univariate Gaussian random variable.
Note that
%|>>|===========================================================================================|>>|
\begin{gather}
\label{eqn:pf:corollary:main-technical:logistic-regression:3}
  \pbetaFn{ 0 }
  =
  \frac{1}{1+e^{0}}
  =
  \frac{1}{2}
.\end{gather}
%|<<|===========================================================================================|<<|
Hence, when \(  \betaX=0  \), \EQUATION \eqref{eqn:pf:corollary:main-technical:logistic-regression:1} trivially evaluates to
%|>>|:::::::::::::::::::::::::::::::::::::::::::::::::::::::::::::::::::::::::::::::::::::::::::|>>|
\(  \alphaX = \frac{1}{2}  \).
%|<<|:::::::::::::::::::::::::::::::::::::::::::::::::::::::::::::::::::::::::::::::::::::::::::|<<|
To bound \EQUATION \eqref{eqn:pf:corollary:main-technical:logistic-regression:1} when \(  \betaX > 0  \), we can directly apply the following result from \cite{hsu2024sample}.
%
%|>>|*******************************************************************************************|>>|
%|>>|*******************************************************************************************|>>|
%|>>|*******************************************************************************************|>>|
\begin{lemma}[{\cite[{\LEMMA 13}]{hsu2024sample}}]
\label{lemma:gaussian-integral:E[p]}
%
Fix \(  \betaXX > 0  \), and let \(  \ZRV \sim \N(0,1)  \) be a standard univariate Gaussian random variable.
Then,
%|>>|===========================================================================================|>>|
\begin{gather}
  \E \left[ \frac{1}{1 + e^{\betaXX | \ZRV |}} \right]
  \leq
  \min
  \left\{
    \frac{1}{2},
    \frac{1}{2} - \sqrt{\frac{2}{\pi}} \left( 1-\frac{\betaXX^{2}}{6} \right) \frac{\betaXX}{4},
    \sqrt{\frac{2}{\pi}} \frac{1}{\betaXX}
  \right\}
.\end{gather}
%|<<|===========================================================================================|<<|
\end{lemma}
%|<<|*******************************************************************************************|<<|
%|<<|*******************************************************************************************|<<|
%|<<|*******************************************************************************************|<<|
%
It immediately follows from \EQUATION \eqref{eqn:pf:corollary:main-technical:logistic-regression:1} and \LEMMA \ref{lemma:gaussian-integral:E[p]} that for \(  \betaX > 0  \),
%|>>|===========================================================================================|>>|
\begin{align}
\label{eqn:pf:corollary:main-technical:logistic-regression:alpha:ub}
  \alphaX
  &=
  \E \left[ \frac{1}{1 + e^{\betaX | \ZRV |}} \right]
  \leq
  \min
  \left\{
    \frac{1}{2},
    \frac{1}{2} - \sqrt{\frac{2}{\pi}} \left( 1-\frac{\betaX^{2}}{6} \right) \frac{\betaX}{4},
    \sqrt{\frac{2}{\pi}} \frac{1}{\betaX}
  \right\}
.\end{align}
%|<<|===========================================================================================|<<|
\checkthis%
Using the above bound on \(  \alphaX  \) in \EQUATION \eqref{eqn:pf:corollary:main-technical:logistic-regression:alpha:ub}, an upper bound on \(  \alphaO  \) can also be obtained.
%For the purposes of this proof, assume for simplicity that
%%|>>|:::::::::::::::::::::::::::::::::::::::::::::::::::::::::::::::::::::::::::::::::::::::::::|>>|
%\(  \deltaX \leq \frac{1}{2}  \).
%%|<<|:::::::::::::::::::::::::::::::::::::::::::::::::::::::::::::::::::::::::::::::::::::::::::|<<|
%Note that the analysis here can easily be extended to the case when
%%|>>|:::::::::::::::::::::::::::::::::::::::::::::::::::::::::::::::::::::::::::::::::::::::::::|>>|
%\(  \deltaX \in (\frac{1}{2},1)  \),
%%|<<|:::::::::::::::::::::::::::::::::::::::::::::::::::::::::::::::::::::::::::::::::::::::::::|<<|
%%though handling these larger values of \(  \deltaX  \) is not necessary for the way in which this corollary is ultimately utilized to establish the main result of this work for logistic regression (\ie \COROLLARY \ref{corollary:approx-error:logistic-regression}).
%though handling these larger values of \(  \deltaX  \) is not necessary for the ultimate utilization of this corollary to establish the main result for logistic regression (\ie for \COROLLARY \ref{corollary:approx-error:logistic-regression}).
%Under this assumption and letting
Noting that
%|>>|:::::::::::::::::::::::::::::::::::::::::::::::::::::::::::::::::::::::::::::::::::::::::::|>>|
\(  \deltaX \leq \frac{1}{2}  \),
%|<<|:::::::::::::::::::::::::::::::::::::::::::::::::::::::::::::::::::::::::::::::::::::::::::|<<|
and letting
%|>>|:::::::::::::::::::::::::::::::::::::::::::::::::::::::::::::::::::::::::::::::::::::::::::|>>|
\(  \ConstbetaXThrsholdLR \defeq \ConstbetaXThrsholdValueLR  \),
%|<<|:::::::::::::::::::::::::::::::::::::::::::::::::::::::::::::::::::::::::::::::::::::::::::|<<|
if
%|>>|:::::::::::::::::::::::::::::::::::::::::::::::::::::::::::::::::::::::::::::::::::::::::::|>>|
\(  \betaX < \betaXThresholdLRTwo = \ConstbetaXThrsholdValuedeltaLR \frac{1}{\deltaX}  \),
%|<<|:::::::::::::::::::::::::::::::::::::::::::::::::::::::::::::::::::::::::::::::::::::::::::|<<|
then
%|>>|:::::::::::::::::::::::::::::::::::::::::::::::::::::::::::::::::::::::::::::::::::::::::::|>>|
\(  \alphaO = \max \{ \alphaX, \deltaX \} \leq \min \{ \frac{1}{2}, \sqrt{\frac{2}{\pi}} \frac{1}{\betaX} \}  \),
%|<<|:::::::::::::::::::::::::::::::::::::::::::::::::::::::::::::::::::::::::::::::::::::::::::|<<|
whereas if
%|>>|:::::::::::::::::::::::::::::::::::::::::::::::::::::::::::::::::::::::::::::::::::::::::::|>>|
\(  \betaX \geq \betaXThresholdLRTwo = \ConstbetaXThrsholdValuedeltaLR \frac{1}{\deltaX}  \),
%|<<|:::::::::::::::::::::::::::::::::::::::::::::::::::::::::::::::::::::::::::::::::::::::::::|<<|
then
%|>>|:::::::::::::::::::::::::::::::::::::::::::::::::::::::::::::::::::::::::::::::::::::::::::|>>|
\(  \alphaO = \max \{ \alphaX, \deltaX \} = \deltaX  \).
%|<<|:::::::::::::::::::::::::::::::::::::::::::::::::::::::::::::::::::::::::::::::::::::::::::|<<|
%
%%%%%%%%%%%%%%%%%%%%%%%%%%%%%%%%%%%%%%%%%%%%%%%%%%%%%%%%%%%%%%%%%%%%%%%%%%%%%%%%%%%%%%%%%%%%%%%%%%%%
\par %%%%%%%%%%%%%%%%%%%%%%%%%%%%%%%%%%%%%%%%%%%%%%%%%%%%%%%%%%%%%%%%%%%%%%%%%%%%%%%%%%%%%%%%%%%%%%%
%%%%%%%%%%%%%%%%%%%%%%%%%%%%%%%%%%%%%%%%%%%%%%%%%%%%%%%%%%%%%%%%%%%%%%%%%%%%%%%%%%%%%%%%%%%%%%%%%%%%
%
Next, an explicit form for an lower  bound on \(  \gammaX  \) will be derived.
This will largely hinge on showing that
%|>>|:::::::::::::::::::::::::::::::::::::::::::::::::::::::::::::::::::::::::::::::::::::::::::|>>|
\(  \gammaX \geq \sqrt{\frac{2}{\pi}} ( 1-2 \alphaX )  \),
%|<<|:::::::::::::::::::::::::::::::::::::::::::::::::::::::::::::::::::::::::::::::::::::::::::|<<|
from where the above bound on \(  \alphaX  \) can subsequently provide a closed-form bound on \(  \gammaX  \).
Towards this, define
%\(  \zetaX \defeq \frac{\sqrt{\hfrac{2}{\pi}} - \gammaX}{2}  \).
\(  \zetaX \defeq 1 - \sqrt{\frac{\pi}{2}} \gammaX  \).
Then, the inequality
\(  \gammaX \geq \sqrt{\frac{2}{\pi}} ( 1-2 \alphaX )  \)
is equivalent to
%precisely the same as
\(  \zetaX \leq 2\alphaX  \),
the latter of which will be our focus.
Note that \(  \zetaX  \) can be calculated by the following expression:
%|>>|===========================================================================================|>>|
\begin{align*}
  \zetaX
  =
  \int_{\zX=0}^{\zX=\infty}
  \zX
  e^{-\frac{1}{2} \zX^{2}}
  (\pExpr{\zX})
  d\zX
.\end{align*}
%|<<|===========================================================================================|<<|
It is convenient to view \(  \alphaX  \) and \(  \zetaX  \) as being parameterized by \(  \betaX  \), and hence, the following argument will use the notations: \(  \alphaX( \betaX )  \) and \(  \zetaX( \betaX )  \).
%The definition of \(  \gammaX  \) in \EQUATION \eqref{eqn:notations:gamma:def} is recalled:
%where, again, \(  \ZRV \sim \N(0,1)  \) is a standard univariate Gaussian random variable.
Note that the \betaXnamelr, \(  \betaX \GTR 0  \), is left as %hidden
implicit in \(  \pFn  \) to simplify the notation.
%Let \(  \zetaX( \betaX ) \defeq \zetaX = 1 - \sqrt{\frac{\pi}{2}} \gammaX  \).
Then, when \(  \pFn  \) is taken as in \DEFINITION \ref{def:p:logistic-regression} for logistic regression, \(  \zetaX( \betaX )  \) has the form:
%|>>|===========================================================================================|>>|
\begin{align*}
  \zetaX( \betaX )
%  =
%  \zetaX
  % &=
  % \E[ Z \fFn( Z ) ]
  % =
  % \sqrt{\frac{2}{\pi}}
  % \int_{\zX=0}^{\zX=\infty}
  % \zX
  % e^{-\frac{1}{2} \zX^{2}}
  % ( \pFn( \zX ) - \pFn( -\zX ) )
  % d\zX
  % \\
  % &=
  % \sqrt{\frac{2}{\pi}}
  % \int_{\zX=0}^{\zX=\infty}
  % \zX
  % e^{-\frac{1}{2} \zX^{2}}
  % ( 2 \pFn( \zX ) - 1 )
  % d\zX
  % \\
  % &\ToDo{}
  %
  %
  %
  %
  =
  \int_{\zX=0}^{\zX=\infty}
  \zX
  e^{-\frac{1}{2} \zX^{2}}
  (\pExpr{\zX})
  d\zX
  =
  2
  \int_{\zX=0}^{\zX=\infty}
  \zX
  e^{-\frac{1}{2} \zX^{2}}
  \pbetaFn{ -\zX }
  d\zX
,\end{align*}
%|<<|===========================================================================================|<<|
where the second equality applies \EQUATION \eqref{eqn:pf:corollary:main-technical:logistic-regression:7}.
Notice that due to \EQUATION \eqref{eqn:pf:corollary:main-technical:logistic-regression:3}, when \(  \betaX = 0  \),
%\ToDo{change this with the new \(  \gammaX  \)}
%|>>|===========================================================================================|>>|
\begin{gather*}
  \zetaX( 0 )
  =
  2
  \int_{\zX=0}^{\zX=\infty}
  \zX
  e^{-\frac{1}{2} \zX^{2}}
  \pbetaFn{ 0 }
  d\zX
  =
  \int_{\zX=0}^{\zX=\infty}
  \zX
  e^{-\frac{1}{2} \zX^{2}}
  d\zX
  =
  1
,\end{gather*}
%|<<|===========================================================================================|<<|
where the last equality is obtained by scaling the expected value of a standard half-normal random variable by \(  \sqrt{\frac{\pi}{2}}  \).
Thus, in this scenario,
%|>>|:::::::::::::::::::::::::::::::::::::::::::::::::::::::::::::::::::::::::::::::::::::::::::|>>|
\(  \zetaX(0) = 1 = 2 \cdot \frac{1}{2} = 2 \alphaX(0)  \),
%|<<|:::::::::::::::::::::::::::::::::::::::::::::::::::::::::::::::::::::::::::::::::::::::::::|<<|
where the last equality follows from the earlier observation
%in \eqref{eqn:pf:corollary:main-technical:logistic-regression:3}
that \(  \alphaX(0) = \frac{1}{2}  \).
Given this, it suffices to show that the ratio \(  \frac{\zetaX( \betaX )}{\alphaX( \betaX )}  \) is maximized when \(  \betaX=0  \), i.e., that
%|>>|:::::::::::::::::::::::::::::::::::::::::::::::::::::::::::::::::::::::::::::::::::::::::::|>>|
\(  \sup_{\betaX \GTR 0} \frac{\zetaX( \betaX )}{\alphaX( \betaX )} = \frac{\zetaX( 0 )}{\alphaX( 0 )} = 2  \).
%|<<|:::::::::::::::::::::::::::::::::::::::::::::::::::::::::::::::::::::::::::::::::::::::::::|<<|
This would indeed be true if
%|>>|:::::::::::::::::::::::::::::::::::::::::::::::::::::::::::::::::::::::::::::::::::::::::::|>>|
\(  \frac{\partial}{\partial \betaX} \frac{\zetaX( \betaX )}{\alphaX( \betaX )} \leq 0  \)
%|<<|:::::::::::::::::::::::::::::::::::::::::::::::::::::::::::::::::::::::::::::::::::::::::::|<<|
for all \(  \betaX \GTR 0  \).
As scaling this by a positive constant will not affect the inequality, it will be more convenient to establish that
%|>>|:::::::::::::::::::::::::::::::::::::::::::::::::::::::::::::::::::::::::::::::::::::::::::|>>|
\(  \frac{\partial}{\partial \betaX} \frac{\zetaX( \betaX )}{\sqrt{2\pi} \alphaX( \betaX )} \leq 0  \),
%|<<|:::::::::::::::::::::::::::::::::::::::::::::::::::::::::::::::::::::::::::::::::::::::::::|<<|
which will be done next.
%
%%%%%%%%%%%%%%%%%%%%%%%%%%%%%%%%%%%%%%%%%%%%%%%%%%%%%%%%%%%%%%%%%%%%%%%%%%%%%%%%%%%%%%%%%%%%%%%%%%%%
\par %%%%%%%%%%%%%%%%%%%%%%%%%%%%%%%%%%%%%%%%%%%%%%%%%%%%%%%%%%%%%%%%%%%%%%%%%%%%%%%%%%%%%%%%%%%%%%%
%%%%%%%%%%%%%%%%%%%%%%%%%%%%%%%%%%%%%%%%%%%%%%%%%%%%%%%%%%%%%%%%%%%%%%%%%%%%%%%%%%%%%%%%%%%%%%%%%%%%
%
To begin, note the following partial derivatives.
%\ToDo{Fix the derivative of \(  \gammaX  \):}
%|>>|===========================================================================================|>>|
\begin{gather}
\label{eqn:pf:corollary:main-technical:logistic-regression:4:alpha}
  \frac{\partial}{\partial \betaX} \sqrt{2\pi} \alphaX( \betaX )
  =
  \frac{\partial}{\partial \betaX}
  2
  \int_{\zX=0}^{\zX=\infty}
  e^{-\frac{1}{2} \zX^{2}}
  \pbetaFn{ -\zX }
  d\zX
  =
  2
  \int_{\zX=0}^{\zX=\infty}
  e^{-\frac{1}{2} \zX^{2}}
  \frac{\partial}{\partial \betaX} \pbetaFn{ -\zX }
  d\zX
,\\ \label{eqn:pf:corollary:main-technical:logistic-regression:4:gamma}
  \frac{\partial}{\partial \betaX} \zetaX( \betaX )
  =
  \frac{\partial}{\partial \betaX}
  2
  \int_{\zX=0}^{\zX=\infty}
  \zX
  e^{-\frac{1}{2} \zX^{2}}
  \pbetaFn{ -\zX }
  d\zX
  =
  2
  \int_{\zX=0}^{\zX=\infty}
  \zX
  e^{-\frac{1}{2} \zX^{2}}
  \frac{\partial}{\partial \betaX} \pbetaFn{ -\zX }
  d\zX
,\end{gather}
%|<<|===========================================================================================|<<|
where due to the quotient rule,
%|>>|===========================================================================================|>>|
\begin{align*}
  \frac{\partial}{\partial \betaX} \pbetaFn{ -\zX }
  &=
  \frac{\partial}{\partial \betaX} \frac{1}{1+e^{\betaX \zX}}
  =
  -\frac{\zX e^{\betaX \zX}}{( 1+e^{\betaX \zX} )^{2}}
  =
  -\frac{\zX}{( 1+e^{-\betaX \zX} ) ( 1+e^{\betaX \zX} )}
%  =
%  -\frac{\zX}{( 1+e^{-\betaX \zX} ) ( 1+e^{\betaX \zX} )}
  =
  -\zX \pbetaFn{ \zX } \pbetaFn{ -\zX }
\TagEqn{\label{eqn:pf:corollary:main-technical:logistic-regression:4:p}}
.\end{align*}
%|<<|===========================================================================================|<<|
Plugging \eqref{eqn:pf:corollary:main-technical:logistic-regression:4:p} into \eqref{eqn:pf:corollary:main-technical:logistic-regression:4:alpha} and \eqref{eqn:pf:corollary:main-technical:logistic-regression:4:gamma} and scaling by a factor of \(  \frac{1}{2}  \) yields
%|>>|===========================================================================================|>>|
\begin{gather}
\label{eqn:pf:corollary:main-technical:logistic-regression:4:alpha:b}
  \frac{1}{2}
  \frac{\partial}{\partial \betaX} \sqrt{2\pi} \alphaX( \betaX )
  =
  \int_{\zX=0}^{\zX=\infty}
  e^{-\frac{1}{2} \zX^{2}}
  \frac{\partial}{\partial \betaX} \pbetaFn{ -\zX }
  d\zX
  =
  -
  \int_{\zX=0}^{\zX=\infty}
  \zX
  e^{-\frac{1}{2} \zX^{2}}
  \pbetaFn{ \zX } \pbetaFn{ -\zX }
  d\zX
,\\ \label{eqn:pf:corollary:main-technical:logistic-regression:4:gamma:b}
  \frac{1}{2}
  \frac{\partial}{\partial \betaX} \zetaX( \betaX )
  =
  \int_{\zX=0}^{\zX=\infty}
  \zX
  e^{-\frac{1}{2} \zX^{2}}
  \frac{\partial}{\partial \betaX} \pbetaFn{ -\zX }
  d\zX
  =
  -
  \int_{\zX=0}^{\zX=\infty}
  \zX^{2}
  e^{-\frac{1}{2} \zX^{2}}
  \pbetaFn{ \zX } \pbetaFn{ -\zX }
  d\zX
.\end{gather}
%|<<|===========================================================================================|<<|
Then, by applying the quotient rule and plugging in \eqref{eqn:pf:corollary:main-technical:logistic-regression:4:alpha:b} and \eqref{eqn:pf:corollary:main-technical:logistic-regression:4:gamma:b},
%|>>|===========================================================================================|>>|
\begin{align*}
  &
  \frac{\partial}{\partial \betaX} \frac{\zetaX( \betaX )}{\sqrt{2\pi} \alphaX( \betaX )}
  \\
  &=
  \frac
  {
    \sqrt{2\pi} \alphaX( \betaX ) \frac{\partial}{\partial \betaX} \zetaX( \betaX )
    -
    \zetaX( \betaX ) \frac{\partial}{\partial \betaX} \sqrt{2\pi} \alphaX( \betaX )
  }
  {2\pi \alphaX( \betaX )^{2}}
  \\
  &=
  \frac
  {
    ( -\zetaX( \betaX ) \frac{\partial}{\partial \betaX} \sqrt{2\pi} \alphaX( \betaX ) )
    -
    ( -\sqrt{2\pi} \alphaX( \betaX ) \frac{\partial}{\partial \betaX} \zetaX( \betaX ) )
  }
  {2\pi \alphaX( \betaX )^{2}}
  \\
  &=
  \frac
  {
    ( -\frac{1}{2} \zetaX( \betaX ) \frac{1}{2} \frac{\partial}{\partial \betaX} \sqrt{2\pi} \alphaX( \betaX ) )
    -
    ( -\frac{1}{2} \sqrt{2\pi} \alphaX( \betaX ) \frac{1}{2} \frac{\partial}{\partial \betaX} \zetaX( \betaX ) )
  }
  {2\pi ( \frac{1}{2} \alphaX( \betaX ) )^{2}}
%  \\
%  &=
%  \frac
%  {
%    \left( 2 \int_{\zX=0}^{\zX=\infty} e^{-\frac{1}{2} \zX^{2}} \pbetaFn{ -\zX } d\zX \right)
%    \left( 2 \int_{\zX=0}^{\zX=\infty} \zX^{2} e^{-\frac{1}{2} \zX^{2}} \pbetaFn{ \zX } \pbetaFn{ -\zX } d\zX \right)
%    -
%    \left( 2 \int_{\zX=0}^{\zX=\infty} \zX e^{-\frac{1}{2} \zX^{2}} \pbetaFn{ -\zX } d\zX \right)
%    \left( 2 \int_{\zX=0}^{\zX=\infty} \zX e^{-\frac{1}{2} \zX^{2}} \pbetaFn{ \zX } \pbetaFn{ -\zX } d\zX \right)
%  }
%  {\left( 2 \int_{\zX=0}^{\zX=\infty} e^{-\frac{1}{2} \zX^{2}} \pbetaFn{ -\zX } d\zX \right)^{2}}
  \\
  &=
  \tfrac
  {
    \left( \int_{\zX=0}^{\zX=\infty} \zX e^{-\frac{1}{2} \zX^{2}} \pbetaFn{ -\zX } d\zX \right)
    \left( \int_{\zX=0}^{\zX=\infty} \zX e^{-\frac{1}{2} \zX^{2}} \pbetaFn{ \zX } \pbetaFn{ -\zX } d\zX \right)
    -
    \left( \int_{\zX=0}^{\zX=\infty} e^{-\frac{1}{2} \zX^{2}} \pbetaFn{ -\zX } d\zX \right)
    \left( \int_{\zX=0}^{\zX=\infty} \zX^{2} e^{-\frac{1}{2} \zX^{2}} \pbetaFn{ \zX } \pbetaFn{ -\zX } d\zX \right)
  }
  {\left( \int_{\zX=0}^{\zX=\infty} e^{-\frac{1}{2} \zX^{2}} \pbetaFn{ -\zX } d\zX \right)^{2}}
.\end{align*}
%|<<|===========================================================================================|<<|
In the last line, the numerator clearly determines the sign of \(  \frac{\partial}{\partial \betaX} \frac{\zetaX( \betaX )}{\sqrt{2\pi} \alphaX( \betaX )}  \).
Focusing in on this expression, the following claim provides an upper bound.
Its verification is deferred to the end of this proof of the corollary.
%
%|>>|*******************************************************************************************|>>|
%|>>|*******************************************************************************************|>>|
%|>>|*******************************************************************************************|>>|
\begin{claim}
\label{claim:pf:corollary:main-technical:logistic-regression:1}
%
Using the notations of this proof,
%|>>|===========================================================================================|>>|
\begin{align*}
  &
  \left( \int_{\zX=0}^{\zX=\infty} \zX e^{-\frac{1}{2} \zX^{2}} \pbetaFn{ -\zX } d\zX \right)
  \left( \int_{\zX=0}^{\zX=\infty} \zX e^{-\frac{1}{2} \zX^{2}} \pbetaFn{ \zX } \pbetaFn{ -\zX } d\zX \right)
  \\
  &-
  \left( \int_{\zX=0}^{\zX=\infty} e^{-\frac{1}{2} \zX^{2}} \pbetaFn{ -\zX } d\zX \right)
  \left( \int_{\zX=0}^{\zX=\infty} \zX^{2} e^{-\frac{1}{2} \zX^{2}} \pbetaFn{ \zX } \pbetaFn{ -\zX } d\zX \right)
  \\
  &\AlignSp \leq
  \left( \int_{\zX=0}^{\zX=\infty} e^{-\frac{1}{2} \zX^{2}} \pbetaFn{ -\zX } d\zX \right)
  \left( \int_{\zX=0}^{\zX=\infty} e^{-\frac{1}{2} \zX^{2}} \pbetaFn{ \zX } \pbetaFn{ -\zX } d\zX \right)
  \\
  &\AlignSp\AlignSp
  \left(
    \left(
      \frac
      {\int_{\zX=0}^{\zX=\infty} \zX e^{-\frac{1}{2} \zX^{2}} \pbetaFn{ \zX } \pbetaFn{ -\zX } d\zX}
      {\int_{\zX=0}^{\zX=\infty} e^{-\frac{1}{2} \zX^{2}} \pbetaFn{ \zX } \pbetaFn{ -\zX } d\zX}
    \right)^{2}
    -
    \frac
    {\int_{\zX=0}^{\zX=\infty} \zX^{2} e^{-\frac{1}{2} \zX^{2}} \pbetaFn{ \zX } \pbetaFn{ -\zX } d\zX}
    {\int_{\zX=0}^{\zX=\infty} e^{-\frac{1}{2} \zX^{2}} \pbetaFn{ \zX } \pbetaFn{ -\zX } d\zX}
  \right)
\TagEqn{\label{eqn:claim:pf:corollary:main-technical:logistic-regression:1}}
.\end{align*}
%|<<|===========================================================================================|<<|
\end{claim}
%|<<|*******************************************************************************************|<<|
%|<<|*******************************************************************************************|<<|
%|<<|*******************************************************************************************|<<|
%
Under the assumed correctness of \CLAIM \ref{claim:pf:corollary:main-technical:logistic-regression:1}, the proof of \COROLLARY \ref{corollary:main-technical:logistic-regression} can be completed.
The \RHS of the inequality in \EQUATION \eqref{eqn:claim:pf:corollary:main-technical:logistic-regression:1} has the same sign as
\newcommand{\EXPRVARIABLE}{y}
%|>>|===========================================================================================|>>|
\begin{gather}
  \EXPRVARIABLE
  \defeq
  \left(
    \frac
    {\int_{\zX=0}^{\zX=\infty} \zX e^{-\frac{1}{2} \zX^{2}} \pbetaFn{ \zX } \pbetaFn{ -\zX } d\zX}
    {\int_{\zX=0}^{\zX=\infty} e^{-\frac{1}{2} \zX^{2}} \pbetaFn{ \zX } \pbetaFn{ -\zX } d\zX}
  \right)^{2}
  -
  \frac
  {\int_{\zX=0}^{\zX=\infty} \zX^{2} e^{-\frac{1}{2} \zX^{2}} \pbetaFn{ \zX } \pbetaFn{ -\zX } d\zX}
  {\int_{\zX=0}^{\zX=\infty} e^{-\frac{1}{2} \zX^{2}} \pbetaFn{ \zX } \pbetaFn{ -\zX } d\zX}
\end{gather}
%|<<|===========================================================================================|<<|
since the product of the first two integrals is positive, \ie
%|>>|===========================================================================================|>>|
\begin{gather*}
  \left( \int_{\zX=0}^{\zX=\infty} e^{-\frac{1}{2} \zX^{2}} \pbetaFn{ -\zX } d\zX \right)
  \left( \int_{\zX=0}^{\zX=\infty} e^{-\frac{1}{2} \zX^{2}} \pbetaFn{ \zX } \pbetaFn{ -\zX } d\zX \right)
  >
  0
.\end{gather*}
%|<<|===========================================================================================|<<|
Hence, if \(  \EXPRVARIABLE \leq 0  \), then due to \CLAIM \ref{claim:pf:corollary:main-technical:logistic-regression:1} and the earlier discussion, it must also happen that \(  \frac{\partial}{\partial \zetaX} \frac{\zetaX( \betaX )}{\sqrt{2\pi} \alphaX( \betaX )} \leq 0  \).
%
%%%%%%%%%%%%%%%%%%%%%%%%%%%%%%%%%%%%%%%%%%%%%%%%%%%%%%%%%%%%%%%%%%%%%%%%%%%%%%%%%%%%%%%%%%%%%%%%%%%%
\par %%%%%%%%%%%%%%%%%%%%%%%%%%%%%%%%%%%%%%%%%%%%%%%%%%%%%%%%%%%%%%%%%%%%%%%%%%%%%%%%%%%%%%%%%%%%%%%
%%%%%%%%%%%%%%%%%%%%%%%%%%%%%%%%%%%%%%%%%%%%%%%%%%%%%%%%%%%%%%%%%%%%%%%%%%%%%%%%%%%%%%%%%%%%%%%%%%%%
%
\let\oldwX\wX%
\let\wX\zX%
\let\oldWRV\WRV%
\let\WRV\ZRV%
%
To establish the nonpositivity of \(  \EXPRVARIABLE  \), consider a univariate standard Gaussian random variable, \(  \WRV \sim \N(0,1)  \), and a random variable \(  \URV  \) which takes values in \(  \{ 0,1 \}  \) such that for \(  \wX \geq 0  \),
%|>>|===========================================================================================|>>|
\begin{gather*}
  ( \URV \Mid| | \WRV |=\wX )
  =
  \begin{cases}
    0 ,& \cWP  1 - \pbetaFn{ \wX } \pbetaFn{ -\wX }, \\
    1 ,& \cWP \pbetaFn{ \wX } \pbetaFn{ -\wX }.
  \end{cases}
\end{gather*}
%|<<|===========================================================================================|<<|
The mass function of this conditioned random variable, \(  \URV \Mid| | \WRV |  \), is given for \(  \wX \geq 0  \) by
%|>>|===========================================================================================|>>|
\begin{gather*}
  \pdf{\URV \Mid| | \WRV |}( \uX \Mid| \wX )
  =
  \begin{cases}
  1 - \pbetaFn{ \wX } \pbetaFn{ -\wX } ,& \cIf \uX = 0 ,\\
  \pbetaFn{ \wX } \pbetaFn{ -\wX }     ,& \cIf \uX = 1 .
  \end{cases}
\end{gather*}
%|<<|===========================================================================================|<<|
In addition, by the law of the total probability and the definition of conditional probabilities,
%|>>|===========================================================================================|>>|
\begin{gather*}
  \pdf{\URV}( 1 )
  =
  \int_{\zX=-\infty}^{\zX=\infty} \pdf{\URV \Mid| | \WRV |}( 1 \Mid| \zX ) \pdf{| \WRV |}( \zX ) d\zX
  =
  \sqrt{\frac{2}{\pi}} \int_{\zX=0}^{\zX=\infty} e^{-\frac{1}{2} \zX^{2}} \pbetaFn{ \zX } \pbetaFn{ -\zX } d\zX
.\end{gather*}
%|<<|===========================================================================================|<<|
Then, by Bayes' theorem, the density function of the conditioned random variable \(  | \WRV | \Mid| \URV = 1  \) is given for \(  \wX \geq 0  \) by
%|>>|===========================================================================================|>>|
\begin{gather*}
  \pdf{| \WRV | \Mid| \URV} ( \wX \Mid| 1 )
  =
  \frac
  {\pdf{\URV \Mid| | \WRV |}( 1 \Mid| \wX ) \pdf{| \WRV |}( \wX )}
  {\pdf{\URV}( 1 )}
  =
  \frac
  {\sqrt{\frac{2}{\pi}} e^{-\frac{1}{2} \zX^{2}} \pbetaFn{ \zX } \pbetaFn{ -\zX } d\zX}
  {\sqrt{\frac{2}{\pi}} \int_{\zXX=0}^{\zXX=\infty} e^{-\frac{1}{2} \zXX^{2}} \pbetaFn{ \zXX } \pbetaFn{ -\zXX } d\zXX}
,\end{gather*}
%|<<|===========================================================================================|<<|
while for \(  \wX < 0  \),
%|>>|:::::::::::::::::::::::::::::::::::::::::::::::::::::::::::::::::::::::::::::::::::::::::::|>>|
\(  \pdf{| \WRV | \Mid| \URV} ( \wX \Mid| 1 ) = 0  \).
%|<<|:::::::::::::::::::::::::::::::::::::::::::::::::::::::::::::::::::::::::::::::::::::::::::|<<|
The variance of \(  | \WRV | \Mid| \URV=1  \) is therefore:
%|>>|===========================================================================================|>>|
\begin{align*}
  \Var( | \WRV | \Mid| \URV=1 )
  &=
  \E[ | \WRV |^{2} \Mid| \URV=1 ]
  -
  \E[ | \WRV | \Mid| \URV=1 ]^{2}
  \\
  &=
  \frac
  {\sqrt{\frac{2}{\pi}} \int_{\wX=0}^{\wX \infty} \zX^{2} e^{-\frac{1}{2} \zX^{2}} \pbetaFn{ \zX } \pbetaFn{ -\zX } d\zX}
  {\sqrt{\frac{2}{\pi}} \int_{\zX=0}^{\zX=\infty} e^{-\frac{1}{2} \zX^{2}} \pbetaFn{ \zX } \pbetaFn{ -\zX } d\zX}
  -
  \left(
    \frac
    {\sqrt{\frac{2}{\pi}} \int_{\wX=0}^{\wX \infty} \zX e^{-\frac{1}{2} \zX^{2}} \pbetaFn{ \zX } \pbetaFn{ -\zX } d\zX}
    {\sqrt{\frac{2}{\pi}} \int_{\zX=0}^{\zX=\infty} e^{-\frac{1}{2} \zX^{2}} \pbetaFn{ \zX } \pbetaFn{ -\zX } d\zX}
  \right)^{2}
  \\
  &=
  \frac
  {\int_{\wX=0}^{\wX \infty} \zX^{2} e^{-\frac{1}{2} \zX^{2}} \pbetaFn{ \zX } \pbetaFn{ -\zX } d\zX}
  {\int_{\zX=0}^{\zX=\infty} e^{-\frac{1}{2} \zX^{2}} \pbetaFn{ \zX } \pbetaFn{ -\zX } d\zX}
  -
  \left(
    \frac
    {\int_{\wX=0}^{\wX \infty} \zX e^{-\frac{1}{2} \zX^{2}} \pbetaFn{ \zX } \pbetaFn{ -\zX } d\zX}
    {\int_{\zX=0}^{\zX=\infty} e^{-\frac{1}{2} \zX^{2}} \pbetaFn{ \zX } \pbetaFn{ -\zX } d\zX}
  \right)^{2}
  \\
  &=
  -\EXPRVARIABLE
.\end{align*}
%|<<|===========================================================================================|<<|
The variance of a random variable is always nonnegative, which implies that
%|>>|===========================================================================================|>>|
\begin{gather*}
  \EXPRVARIABLE = -\Var( | \WRV | \Mid| \URV=1 ) \leq 0
,\end{gather*}
%|<<|===========================================================================================|<<|
and thus, combining this with some previous remarks, it follows that \(  \frac{\partial}{\partial \betaX} \frac{\zetaX( \betaX )}{\sqrt{2\pi} \alphaX( \betaX )} \leq 0  \) when \(  \betaX \GTR 0  \), as claimed.
%
%%%%%%%%%%%%%%%%%%%%%%%%%%%%%%%%%%%%%%%%%%%%%%%%%%%%%%%%%%%%%%%%%%%%%%%%%%%%%%%%%%%%%%%%%%%%%%%%%%%%
\par %%%%%%%%%%%%%%%%%%%%%%%%%%%%%%%%%%%%%%%%%%%%%%%%%%%%%%%%%%%%%%%%%%%%%%%%%%%%%%%%%%%%%%%%%%%%%%%
%%%%%%%%%%%%%%%%%%%%%%%%%%%%%%%%%%%%%%%%%%%%%%%%%%%%%%%%%%%%%%%%%%%%%%%%%%%%%%%%%%%%%%%%%%%%%%%%%%%%
%
As noted earlier, this further implies that
%|>>|===========================================================================================|>>|
\begin{gather*}
  \sup_{\betaX \GTR 0} \frac{\zetaX( \betaX )}{\alphaX( \betaX )}
  =
  \frac{\zetaX( 0 )}{\alphaX( 0 )}
  =
  2
.\end{gather*}
%|<<|===========================================================================================|<<|
Then, by \LEMMA \ref{lemma:gaussian-integral:E[p]},
%|>>|===========================================================================================|>>|
\begin{gather}
\label{eqn:pf:corollary:main-technical:logistic-regression:zeta:ub}
  \zetaX
  \leq
  2\alphaX
  \leq
  \min
  \left\{
    1,
    1 - \sqrt{\frac{2}{\pi}} \left( 1-\frac{\betaX^{2}}{6} \right) \frac{\betaX}{2},
    \sqrt{\frac{2}{\pi}} \frac{2}{\betaX}
  \right\}
.\end{gather}
%|<<|===========================================================================================|<<|
This upper bound on \(  \zetaX  \) now gives the following lower bound on \(  \gammaX  \):
%|>>|===========================================================================================|>>|
\begin{gather}
\label{eqn:pf:corollary:main-technical:logistic-regression:gamma:ub}
  \gammaX
  =
  \sqrt{\frac{2}{\pi}} ( 1-\zetaX )
  \geq
  \sqrt{\frac{2}{\pi}} ( 1-2\alphaX )
  \geq
  \sqrt{\frac{2}{\pi}}
  \left(
  1 -
  \min
  \left\{
    1,
    1 - \sqrt{\frac{2}{\pi}} \left( 1-\frac{\betaX^{2}}{6} \right) \frac{\betaX}{2},
    \sqrt{\frac{2}{\pi}} \frac{2}{\betaX}
  \right\}
  \right)
.\end{gather}
%|<<|===========================================================================================|<<|
This completes \TASK \ref{enum:pf:corollary:main-technical:logistic-regression:b}.
%
%%%%%%%%%%%%%%%%%%%%%%%%%%%%%%%%%%%%%%%%%%%%%%%%%%%%%%%%%%%%%%%%%%%%%%%%%%%%%%%%%%%%%%%%%%%%%%%%%%%%
\par %%%%%%%%%%%%%%%%%%%%%%%%%%%%%%%%%%%%%%%%%%%%%%%%%%%%%%%%%%%%%%%%%%%%%%%%%%%%%%%%%%%%%%%%%%%%%%%
%%%%%%%%%%%%%%%%%%%%%%%%%%%%%%%%%%%%%%%%%%%%%%%%%%%%%%%%%%%%%%%%%%%%%%%%%%%%%%%%%%%%%%%%%%%%%%%%%%%%
%
The above work sets up the realization of the corollary's proof.
With the explicit bounds on \(  \alphaX  \) and \(  \gammaX  \), note the following:
%|>>|===========================================================================================|>>|
\begin{gather}
\label{eqn:pf:corollary:main-technical:logistic-regression:5:a}
  \gammaX
  \geq
  \begin{cases}
  \left( 1 - \frac{\betaX^{2}}{6} \right) \frac{\betaX}{{\pi}}
  ,& \cIf \betaX \in (0, \betaXThresholdLROne) ,\\
  \sqrt{\frac{2}{\pi}} \left( 1 - \sqrt{\frac{2}{\pi}} \frac{2}{\betaX} \right)
  ,& \cIf \betaX \in [\betaXThresholdLROne, \betaXThresholdLRTwo) ,\\
  \sqrt{\frac{2}{\pi}} \left( 1 - \sqrt{\frac{2}{\pi}} \frac{2}{\betaX} \right)
  ,& \cIf \betaX \in [\betaXThresholdLRTwo, \infty) ,
  \end{cases}
  \geq
  \begin{cases}
  \left( 1 - \frac{\cO^{2}}{6} \right) \frac{\betaX}{{\pi}}
  ,& \cIf \betaX \in (0, \betaXThresholdLROne) ,\\
  \bO
  ,& \cIf \betaX \in [\betaXThresholdLROne, \betaXThresholdLRTwo) ,\\
  \bO
  ,& \cIf \betaX \in [\betaXThresholdLRTwo, \infty) .
  \end{cases}
%   \\
% \label{eqn:pf:corollary:main-technical:logistic-regression:5:b}
%   \gammaX^{2}
%   \geq
%   \begin{cases}
%   \left( 1 - \frac{\betaX^{2}}{6} \right)^{2} \frac{\betaX^{2}}{\pi^{2}}
%   ,& \cIf \betaX \in (0, \betaXThresholdLROne) ,\\
%   \frac{2}{\pi} \left( 1 - \sqrt{\frac{2}{\pi}} \frac{2}{\betaX} \right)^{2}
%   ,& \cIf \betaX \in [\betaXThresholdLROne, \betaXThresholdLRTwo) ,\\
%   \frac{2}{\pi} \left( 1 - \sqrt{\frac{2}{\pi}} \frac{2}{\betaX} \right)^{2}
%   ,& \cIf \betaX \in [\betaXThresholdLRTwo, \infty) ,
%   \end{cases}
%   \geq
%   \begin{cases}
%   \left( 1 - \frac{\cO^{2}}{6} \right)^{2} \frac{\betaX^{2}}{\pi^{2}}
%   ,& \cIf \betaX \in (0, \betaXThresholdLROne) ,\\
%   \bO^{2}
%   ,& \cIf \betaX \in [\betaXThresholdLROne, \betaXThresholdLRTwo) ,\\
%   \bO^{2}
%   ,& \cIf \betaX \in [\betaXThresholdLRTwo, \infty) ,
%   \end{cases}
\end{gather}
%|<<|===========================================================================================|<<|
%\ToDo{Update these constant.}
where
%|>>|:::::::::::::::::::::::::::::::::::::::::::::::::::::::::::::::::::::::::::::::::::::::::::|>>|
\(  \bO, \cO, \ConstbetaXThrsholdLR > 0  \)
%|<<|:::::::::::::::::::::::::::::::::::::::::::::::::::::::::::::::::::::::::::::::::::::::::::|<<|
are constants such that
%|>>|:::::::::::::::::::::::::::::::::::::::::::::::::::::::::::::::::::::::::::::::::::::::::::|>>|
\(  \cO = \frac{\sqrt{\hfrac{8}{\pi}}}{1+\bO} \geq 1  \) and
\(  \ConstbetaXThrsholdLR \defeq \ConstbetaXThrsholdValueLR  \).
%|<<|:::::::::::::::::::::::::::::::::::::::::::::::::::::::::::::::::::::::::::::::::::::::::::|<<|
%Additionally, if
%%|>>|:::::::::::::::::::::::::::::::::::::::::::::::::::::::::::::::::::::::::::::::::::::::::::|>>|
%\(  \betaX < \frac{\ConstbetaXThrsholdLR}{\deltaX}  \),
%%|<<|:::::::::::::::::::::::::::::::::::::::::::::::::::::::::::::::::::::::::::::::::::::::::::|<<|
%then
%%|>>|:::::::::::::::::::::::::::::::::::::::::::::::::::::::::::::::::::::::::::::::::::::::::::|>>|
%\(  \alphaO = \max \{ \alphaX, \deltaX \} = \alphaX  \),
%%|<<|:::::::::::::::::::::::::::::::::::::::::::::::::::::::::::::::::::::::::::::::::::::::::::|<<|
%whereas if
%%|>>|:::::::::::::::::::::::::::::::::::::::::::::::::::::::::::::::::::::::::::::::::::::::::::|>>|
%\(  \betaX \geq \frac{\ConstbetaXThrsholdLR}{\deltaX}  \),
%%|<<|:::::::::::::::::::::::::::::::::::::::::::::::::::::::::::::::::::::::::::::::::::::::::::|<<|
%then
%%|>>|:::::::::::::::::::::::::::::::::::::::::::::::::::::::::::::::::::::::::::::::::::::::::::|>>|
%\(  \alphaO = \max \{ \alphaX, \deltaX \} = \deltaX  \).
%%|<<|:::::::::::::::::::::::::::::::::::::::::::::::::::::::::::::::::::::::::::::::::::::::::::|<<|
Recall from an earlier discussion that
%|>>|===========================================================================================|>>|
\begin{gather*}
  \alphaO
  \leq
  \begin{cases}
  \min \left\{ \frac{1}{2}, \sqrt{\frac{2}{\pi}} \frac{1}{\betaX} \right\} ,&\cIf \betaX < \frac{\ConstbetaXThrsholdLR}{\deltaX} ,\\
  \deltaX ,&\cIf \betaX \geq \frac{\ConstbetaXThrsholdLR}{\deltaX}.
  \end{cases}
\end{gather*}
%|<<|===========================================================================================|<<|
As a result,
%|>>|===========================================================================================|>>|
\begin{gather}
\label{eqn:pf:corollary:main-technical:logistic-regression:6}
  \frac{\alphaO}{\GAMMAX^{2}}
  \leq
  \begin{cases}
  \frac{\pi^{2}}{2 \bigl( 1 - \frac{\betaX^{2}}{6} \bigr)^{2} \betaX^{2}}
  ,& \cIf \betaX \in (0, \betaXThresholdLROne) ,\\
  \frac{\sqrt{\hfrac{\pi}{2}}}{\bigl( 1 - \sqrt{\frac{2}{\pi}} \frac{2}{\betaX} \bigr)^{2} \betaX}
  ,& \cIf \betaX \in [\betaXThresholdLROne, \betaXThresholdLRTwo) ,\\
  \frac{\pi \deltaX}{2 \bigl( 1 - \sqrt{\frac{2}{\pi}} \frac{2}{\betaX} \bigr)^{2}}
  ,& \cIf \betaX \in [\betaXThresholdLRTwo, \infty) ,
  \end{cases}
  \leq
  \begin{cases}
  \frac{\pi^{2}}{2 \bigl( 1 - \frac{\cO^{2}}{6} \bigr)^{2} \betaX^{2}}
  ,& \cIf \betaX \in (0, \betaXThresholdLROne) ,\\
  \frac{\sqrt{\hfrac{\pi}{2}}}{\bO^{2} \betaX}
  ,& \cIf \betaX \in [\betaXThresholdLROne, \betaXThresholdLRTwo) ,\\
  \frac{\pi \deltaX}{2 \bO^{2}}
  ,& \cIf \betaX \in [\betaXThresholdLRTwo, \infty) .
  \end{cases}
\end{gather}
%|<<|===========================================================================================|<<|
%\ToDo{Double check the incorporation of \(  \alphaO  \).}
Now, the corollary's result for logitic regression can be established by substituting the bounds in \EQUATIONS \eqref{eqn:pf:corollary:main-technical:logistic-regression:5:a}--\eqref{eqn:pf:corollary:main-technical:logistic-regression:6} into \EQUATION \eqref{eqn:main-technical:sparse:m} of \THEOREM \ref{thm:main-technical:sparse} and the definitions of \(  \nuX( \deltaX )  \) and \(  \tauX( \deltaX )  \).
%, as well as plugging in the bounds in \EQUATION \eqref{eqn:pf:corollary:main-technical:logistic-regression:5:a} into
%\EQUATION \eqref{eqn:main-technical:sparse:nu} of \THEOREM \ref{thm:main-technical:sparse} and then calculating \EQUATION \eqref{eqn:main-technical:sparse:tau} accordingly.
Take \(  \m  \) to be at least
%|>>|===========================================================================================|>>|
\begin{align*}
%\TagEqn{\label{eqn:pf:corollary:main-technical:logistic-regression:n:sparse}}
  \m
  &\geq
  \max \{ \mOneS, \mTwoS, \mThreeS, \mFourS, \mFiveS \}
  =
  \mOEXPRLR[s]{\epsilonX}[,]
%  \\
%  &\geq
%  \mEXPR[s]
\end{align*}
%|<<|===========================================================================================|<<|
where
\checkthis[Done]%
%|>>|===========================================================================================|>>|
\begin{gather*}
%\label{eqn:pf:corollary:main-technical:logistic-regression:n:1:sparse}
  \mOneS = \mOneEXPR[s]
  \\
%\label{eqn:pf:corollary:main-technical:logistic-regression:n:2:sparse}
  \mTwoS = \mTwoEXPR[s]
  \\
%\label{eqn:pf:corollary:main-technical:logistic-regression:n:3:sparse}
  \mThreeS = \mThreeEXPR[s]
  \\
%\label{eqn:pf:corollary:main-technical:logistic-regression:n:4:sparse}
  \mFourS = \mFourEXPR[s]
  \\
%\label{eqn:pf:corollary:main-technical:logistic-regression:n:5:sparse}
  \mFiveS = \mFiveEXPR[s]
\end{gather*}
%|<<|===========================================================================================|<<|
and where, as a result of \EQUATION \eqref{eqn:pf:corollary:main-technical:logistic-regression:5:a}, \(  \nuX  \) is bounded from below by
\checkthis%
%|>>|===========================================================================================|>>|
\begin{gather*}
%\label{eqn:pf:corollary:main-technical:logistic-regression:nu:sparse}
  \nuX
  =
  \nuXEXPR
  \geq
  \begin{cases}
  \nuXEXPRLROne
  ,& \cIf \betaX \in (0, \betaXThresholdLROne) ,\\
  \nuXEXPRLRTwo
  ,& \cIf \betaX \in [\betaXThresholdLROne, \betaXThresholdLRTwo) ,\\
  \nuXEXPRLRThree
  ,& \cIf \betaX \in [\betaXThresholdLRTwo, \infty) .
  \end{cases}
\end{gather*}
%|<<|===========================================================================================|<<|
Then, due to \EQUATIONS \eqref{eqn:pf:corollary:main-technical:logistic-regression:5:a} and \eqref{eqn:pf:corollary:main-technical:logistic-regression:6},
%|>>|===========================================================================================|>>|
\begin{gather*}
  \m \geq \mEXPR[s][.]
\end{gather*}
%|<<|===========================================================================================|<<|
Therefore, by \THEOREM \ref{thm:main-technical:sparse} and this choice of \(  \m  \), with probability at least \(  1-\rho  \), uniformly for all \(  \thetaXX \in \ParamSpace  \) and \(  \JCoords \subseteq [\n]  \), \(  | \JCoords | \leq \k  \), \EQUATION \eqref{eqn:main-technical:sparse:1} holds:
%|>>|===========================================================================================|>>|
\begin{gather*}
  \left\|
    \thetaStar
    -
    \frac
    {\thetaXX + \hfFn[\JCoords]( \thetaStar, \thetaXX )}
    {\| \thetaXX + \hfFn[\JCoords]( \thetaStar, \thetaXX ) \|_{2}}
  \right\|_{2}
  \leq
  \sqrt{\deltaX \| \thetaStar-\thetaXX \|_{2}}
  +
  \deltaX
.\end{gather*}
%|<<|===========================================================================================|<<|
%
%%%%%%%%%%%%%%%%%%%%%%%%%%%%%%%%%%%%%%%%%%%%%%%%%%%%%%%%%%%%%%%%%%%%%%%%%%%%%%%%%%%%%%%%%%%%%%%%%%%%
\par %%%%%%%%%%%%%%%%%%%%%%%%%%%%%%%%%%%%%%%%%%%%%%%%%%%%%%%%%%%%%%%%%%%%%%%%%%%%%%%%%%%%%%%%%%%%%%%
%%%%%%%%%%%%%%%%%%%%%%%%%%%%%%%%%%%%%%%%%%%%%%%%%%%%%%%%%%%%%%%%%%%%%%%%%%%%%%%%%%%%%%%%%%%%%%%%%%%%
%
%The sample complexity provided across \EQUATIONS \eqref{eqn:pf:corollary:main-technical:logistic-regression:n:sparse}--\eqref{eqn:pf:corollary:main-technical:logistic-regression:nu:sparse} can be
%%written more concisely as
%consolidated into
%the following \orderwise expression:
%%|>>|===========================================================================================|>>|
%\begin{gather*}
%  \m = \mOEXPRLR[s]{\epsilonX}[,]
%\end{gather*}
%%|<<|===========================================================================================|<<|
%as claimed in the corollary.
Barring the verification of \CLAIM \ref{claim:pf:corollary:main-technical:logistic-regression:1}, this concludes
%the arguments for
the proof of \COROLLARY \ref{corollary:main-technical:logistic-regression} for logistic regression.
The last remaining task is returning to and proving \CLAIM \ref{claim:pf:corollary:main-technical:logistic-regression:1}.
%
%|>>|~~~~~~~~~~~~~~~~~~~~~~~~~~~~~~~~~~~~~~~~~~~~~~~~~~~~~~~~~~~~~~~~~~~~~~~~~~~~~~~~~~~~~~~~~~~|>>|
%|>>|~~~~~~~~~~~~~~~~~~~~~~~~~~~~~~~~~~~~~~~~~~~~~~~~~~~~~~~~~~~~~~~~~~~~~~~~~~~~~~~~~~~~~~~~~~~|>>|
%|>>|~~~~~~~~~~~~~~~~~~~~~~~~~~~~~~~~~~~~~~~~~~~~~~~~~~~~~~~~~~~~~~~~~~~~~~~~~~~~~~~~~~~~~~~~~~~|>>|
\begin{subproof}
{\CLAIM \ref{claim:pf:corollary:main-technical:logistic-regression:1}}
%
Looking at the \LHS of \EQUATION \eqref{eqn:claim:pf:corollary:main-technical:logistic-regression:1}, observe:
%|>>|===========================================================================================|>>|
\begin{align*}
  &
  \left( \int_{\zX=0}^{\zX=\infty} \zX e^{-\frac{1}{2} \zX^{2}} \pbetaFn{ -\zX } d\zX \right)
  \left( \int_{\zX=0}^{\zX=\infty} \zX e^{-\frac{1}{2} \zX^{2}} \pbetaFn{ \zX } \pbetaFn{ -\zX } d\zX \right)
  \\
  &-
  \left( \int_{\zX=0}^{\zX=\infty} e^{-\frac{1}{2} \zX^{2}} \pbetaFn{ -\zX } d\zX \right)
  \left( \int_{\zX=0}^{\zX=\infty} \zX^{2} e^{-\frac{1}{2} \zX^{2}} \pbetaFn{ \zX } \pbetaFn{ -\zX } d\zX \right)
  \\
  &\AlignSp=
  \left( \int_{\zX=0}^{\zX=\infty} e^{-\frac{1}{2} \zX^{2}} \pbetaFn{ -\zX } d\zX \right)
  \left( \int_{\zX=0}^{\zX=\infty} e^{-\frac{1}{2} \zX^{2}} \pbetaFn{ \zX } \pbetaFn{ -\zX } d\zX \right)
  \\
  &\AlignSp\AlignSp \left(
  \left( \tfrac{\int_{\zX=0}^{\zX=\infty} \zX e^{-\frac{1}{2} \zX^{2}} \pbetaFn{ -\zX } d\zX}{\int_{\zX=0}^{\zX=\infty} e^{-\frac{1}{2} \zX^{2}} \pbetaFn{ -\zX } d\zX} \right)
  \left( \tfrac{\int_{\zX=0}^{\zX=\infty} \zX e^{-\frac{1}{2} \zX^{2}} \pbetaFn{ \zX } \pbetaFn{ -\zX } d\zX}{\int_{\zX=0}^{\zX=\infty} e^{-\frac{1}{2} \zX^{2}} \pbetaFn{ \zX } \pbetaFn{ -\zX } d\zX} \right)
  -
  \left( \tfrac{\int_{\zX=0}^{\zX=\infty} \zX^{2} e^{-\frac{1}{2} \zX^{2}} \pbetaFn{ \zX } \pbetaFn{ -\zX } d\zX}{\int_{\zX=0}^{\zX=\infty} e^{-\frac{1}{2} \zX^{2}} \pbetaFn{ \zX } \pbetaFn{ -\zX } d\zX} \right)
  \right)
\TagEqn{\label{eqn:pf:claim:pf:corollary:main-technical:logistic-regression:1:1}}
.\end{align*}
%|<<|===========================================================================================|<<|
In the above equation, \eqref{eqn:pf:claim:pf:corollary:main-technical:logistic-regression:1:1}, the term
%|>>|===========================================================================================|>>|
\begin{gather*}
  \frac
  {\int_{\zX=0}^{\zX=\infty} \zX e^{-\frac{1}{2} \zX^{2}} \pbetaFn{ -\zX } d\zX}
  {\int_{\zX=0}^{\zX=\infty} e^{-\frac{1}{2} \zX^{2}} \pbetaFn{ -\zX } d\zX}
\end{gather*}
%|<<|===========================================================================================|<<|
can be bounded from above as follows.
Similarly to earlier in the proof, let \(  \WRV \sim \N(0,1)  \) be a univariate standard Gaussian random variable such that \(  | \WRV |  \) is a standard half-normal random variable with density
%|>>|===========================================================================================|>>|
\begin{gather}
\label{eqn:pf:claim:pf:corollary:main-technical:logistic-regression:2}
  \pdf{\WRV}( \wX )
  =
  \begin{cases}
  0 ,& \cIf \wX < 0 ,\\
  \sqrt{\frac{2}{\pi}} e^{-\frac{1}{2} \wX^{2}} ,& \cIf \wX \geq 0,
  \end{cases}
\end{gather}
%|<<|===========================================================================================|<<|
and let \(  \URV  \) and \(  \VRV  \) be random variables taking values in \(  \{ 0,1 \}  \), where for \(  \wX \geq 0  \),
%|>>|===========================================================================================|>>|
\begin{gather*}
  ( \URV \Mid| | \WRV |=\wX )
  =
  \begin{cases}
    0 ,& \cWP  1 - \pbetaFn{ \wX } \pbetaFn{ -\wX }, \\
    1 ,& \cWP \pbetaFn{ \wX } \pbetaFn{ -\wX },
  \end{cases}
  \\
  ( \VRV \Mid| | \WRV |=\wX )
  =
  \begin{cases}
    0 ,& \cWP  1 - \pbetaFn{ -\wX }, \\
    1 ,& \cWP \pbetaFn{ -\wX }.
  \end{cases}
\end{gather*}
%|<<|===========================================================================================|<<|
These conditioned random variables, \(  \URV \Mid| | \WRV |  \) and \(  \VRV \Mid| | \WRV |  \), have mass functions:
%|>>|===========================================================================================|>>|
\begin{gather}
\label{eqn:pf:claim:pf:corollary:main-technical:logistic-regression:3}
  \pdf{\URV \Mid| | \WRV |}( \uX \Mid| \wX )
  =
  \begin{cases}
  1 - \pbetaFn{ \wX } \pbetaFn{ -\wX } ,& \cIf \uX=0 ,\\
  \pbetaFn{ \wX } \pbetaFn{ -\wX }     ,& \cIf \uX=1 ,
  \end{cases}
  \\
\label{eqn:pf:claim:pf:corollary:main-technical:logistic-regression:4}
  \pdf{\VRV \Mid| | \WRV |}( \vX \Mid| \wX )
  =
  \begin{cases}
  1 - \pbetaFn{ -\wX } ,& \cIf \vX=0 ,\\
  \pbetaFn{ -\wX }     ,& \cIf \vX=1 ,
  \end{cases}
\end{gather}
%|<<|===========================================================================================|<<|
for \(  \uX, \vX \in \{ 0,1 \}  \) and \(  \wX \geq 0  \).
Applying, in order twice, the law of total probability, the definition of conditional probabilities, and \EQUATIONS \eqref{eqn:pf:claim:pf:corollary:main-technical:logistic-regression:2}-\eqref{eqn:pf:claim:pf:corollary:main-technical:logistic-regression:4} obtains:
%|>>|===========================================================================================|>>|
\begin{gather*}
  \pdf{\URV}( 1 )
  =
  \int_{\zX=-\infty}^{\zX=\infty} \pdf{\URV, | \WRV |}( 1, \wX ) d\zX
  =
  \int_{\zX=0}^{\zX=\infty} \pdf{\URV \Mid| | \WRV |}( 1 \Mid| \wX ) \pdf{| \WRV |}( \wX ) d\zX
  =
  \sqrt{\frac{2}{\pi}} \int_{\zX=0}^{\zX=\infty} e^{-\frac{1}{2} \zX^{2}} \pbetaFn{ \zX } \pbetaFn{ -\zX } d\zX
  ,\\
  \pdf{\VRV}( 1 )
  =
  \int_{\zX=-\infty}^{\zX=\infty} \pdf{\VRV, | \WRV |}( 1, \wX ) d\zX
  =
  \int_{\zX=0}^{\zX=\infty} \pdf{\VRV \Mid| | \WRV |}( 1 \Mid| \wX ) \pdf{| \WRV |}( \wX ) d\zX
  =
  \sqrt{\frac{2}{\pi}} \int_{\zX=0}^{\zX=\infty} e^{-\frac{1}{2} \zX^{2}} \pbetaFn{ -\wX } d\zX
.\end{gather*}
%|<<|===========================================================================================|<<|
Using Bayes' theorem, for \(  \wX \geq 0  \), the density functions of the conditioned random variables \(  | \WRV | \Mid| \URV = 1  \) and \(  | \WRV | \Mid| \VRV = 1  \) are respectively given by
%|>>|===========================================================================================|>>|
\begin{gather*}
  \pdf{| \WRV | \Mid| \URV} ( \wX \Mid| 1 )
  =
  \frac
  {\pdf{\URV \Mid| | \WRV |}( 1 \Mid| \wX ) \pdf{| \WRV |}( \wX )}
  {\pdf{\URV}( 1 )}
  =
  \frac
  {\sqrt{\frac{2}{\pi}} e^{-\frac{1}{2} \wX^{2}} \pbetaFn{ \wX } \pbetaFn{ -\wX } d\zX}
  {\sqrt{\frac{2}{\pi}} \int_{\zXX=0}^{\zXX=\infty} e^{-\frac{1}{2} \zXX^{2}} \pbetaFn{ \zXX } \pbetaFn{ -\zXX } d\zXX}
  ,\\
  \pdf{| \WRV | \Mid| \VRV} ( \wX \Mid| 1 )
  =
  \frac
  {\pdf{\VRV \Mid| | \WRV |}( 1 \Mid| \wX ) \pdf{| \WRV |}( \wX )}
  {\pdf{\VRV}( 1 )}
  =
  \frac
  {\sqrt{\frac{2}{\pi}} e^{-\frac{1}{2} \wX^{2}} \pbetaFn{ -\wX } d\zX}
  {\sqrt{\frac{2}{\pi}} \int_{\zXX=0}^{\zXX=\infty} e^{-\frac{1}{2} \zXX^{2}} \pbetaFn{ -\zXX } d\zXX}
,\end{gather*}
%|<<|===========================================================================================|<<|
while for \(  \wX < 0  \),
%|>>|:::::::::::::::::::::::::::::::::::::::::::::::::::::::::::::::::::::::::::::::::::::::::::|>>|
\(  \pdf{| \WRV | \Mid| \URV} ( \wX \Mid| 1 ) = 0  \) and
\(  \pdf{| \WRV | \Mid| \VRV} ( \wX \Mid| 1 ) = 0  \).
%|<<|:::::::::::::::::::::::::::::::::::::::::::::::::::::::::::::::::::::::::::::::::::::::::::|<<|
Then, in expectation,
%|>>|===========================================================================================|>>|
\begin{gather*}
  \E[ | \WRV | \Mid| \URV=1 ]
  =
  \frac
  {\sqrt{\frac{2}{\pi}} \int_{\zX=0}^{\zX=\infty} \zX e^{-\frac{1}{2} \zX^{2}} \pbetaFn{ \zX } \pbetaFn{ -\zX } d\zX}
  {\sqrt{\frac{2}{\pi}} \int_{\zX=0}^{\zX=\infty} e^{-\frac{1}{2} \zX^{2}} \pbetaFn{ \zX } \pbetaFn{ -\zX } d\zX}
  =
  \frac
  {\int_{\zX=0}^{\zX=\infty} \zX e^{-\frac{1}{2} \zX^{2}} \pbetaFn{ \zX } \pbetaFn{ -\zX } d\zX}
  {\int_{\zX=0}^{\zX=\infty} e^{-\frac{1}{2} \zX^{2}} \pbetaFn{ \zX } \pbetaFn{ -\zX } d\zX}
  ,\\
  \E[ | \WRV | \Mid| \VRV=1 ]
  =
  \frac
  {\sqrt{\frac{2}{\pi}} \int_{\zX=0}^{\zX=\infty} \zX e^{-\frac{1}{2} \zX^{2}} \pbetaFn{ -\zX } d\zX}
  {\sqrt{\frac{2}{\pi}} \int_{\zX=0}^{\zX=\infty} e^{-\frac{1}{2} \zX^{2}} \pbetaFn{ -\zX } d\zX}
  =
  \frac
  {\int_{\zX=0}^{\zX=\infty} \zX e^{-\frac{1}{2} \zX^{2}} \pbetaFn{ -\zX } d\zX}
  {\int_{\zX=0}^{\zX=\infty} e^{-\frac{1}{2} \zX^{2}} \pbetaFn{ -\zX } d\zX}
.\end{gather*}
%|<<|===========================================================================================|<<|
Since \(  \pbetaFn{ \zX } = \frac{1}{1+e^{-\betaX \zX}}  \) is a nondecreasing function and the support of \(  \pdf{| \WRV | \Mid| \URV=1}  \) and \(  \pdf{| \WRV | \Mid| \VRV=1}  \) lies on the nonnegative real line, it follows that
%|>>|:::::::::::::::::::::::::::::::::::::::::::::::::::::::::::::::::::::::::::::::::::::::::::|>>|
\(  \E[ | \WRV | \Mid| \URV=1 ] \geq \E[ | \WRV | \Mid| \VRV=1 ]  \),
%|<<|:::::::::::::::::::::::::::::::::::::::::::::::::::::::::::::::::::::::::::::::::::::::::::|<<|
and hence, by the above pair of equations,
%|>>|===========================================================================================|>>|
\begin{gather}
\label{eqn:pf:claim:pf:corollary:main-technical:logistic-regression:1:2}
  \frac
  {\int_{\zX=0}^{\zX=\infty} \zX e^{-\frac{1}{2} \zX^{2}} \pbetaFn{ -\zX } d\zX}
  {\int_{\zX=0}^{\zX=\infty} e^{-\frac{1}{2} \zX^{2}} \pbetaFn{ -\zX } d\zX}
  =
  \E[ | \WRV | \Mid| \VRV=1 ]
  \leq
  \E[ | \WRV | \Mid| \URV=1 ]
  =
  \frac
  {\int_{\zX=0}^{\zX=\infty} \zX e^{-\frac{1}{2} \zX^{2}} \pbetaFn{ \zX } \pbetaFn{ -\zX } d\zX}
  {\int_{\zX=0}^{\zX=\infty} e^{-\frac{1}{2} \zX^{2}} \pbetaFn{ \zX } \pbetaFn{ -\zX } d\zX}
.\end{gather}
%|<<|===========================================================================================|<<|
Therefore, returning to \EQUATION \eqref{eqn:pf:claim:pf:corollary:main-technical:logistic-regression:1:1} and applying the above inequality in \EQUATION \eqref{eqn:pf:claim:pf:corollary:main-technical:logistic-regression:1:2}, \CLAIM \ref{claim:pf:corollary:main-technical:logistic-regression:1} follows:
%|>>|===========================================================================================|>>|
\begin{align*}
  &
  \left( \int_{\zX=0}^{\zX=\infty} \zX e^{-\frac{1}{2} \zX^{2}} \pbetaFn{ -\zX } d\zX \right)
  \left( \int_{\zX=0}^{\zX=\infty} \zX e^{-\frac{1}{2} \zX^{2}} \pbetaFn{ \zX } \pbetaFn{ -\zX } d\zX \right)
  \\
  &-
  \left( \int_{\zX=0}^{\zX=\infty} e^{-\frac{1}{2} \zX^{2}} \pbetaFn{ -\zX } d\zX \right)
  \left( \int_{\zX=0}^{\zX=\infty} \zX^{2} e^{-\frac{1}{2} \zX^{2}} \pbetaFn{ \zX } \pbetaFn{ -\zX } d\zX \right)
  \\
  &\AlignSp=
  \left( \int_{\zX=0}^{\zX=\infty} e^{-\frac{1}{2} \zX^{2}} \pbetaFn{ -\zX } d\zX \right)
  \left( \int_{\zX=0}^{\zX=\infty} e^{-\frac{1}{2} \zX^{2}} \pbetaFn{ \zX } \pbetaFn{ -\zX } d\zX \right)
  \\
  &\AlignSp\AlignSp \left(
  \left( \tfrac{\int_{\zX=0}^{\zX=\infty} \zX e^{-\frac{1}{2} \zX^{2}} \pbetaFn{ -\zX } d\zX}{\int_{\zX=0}^{\zX=\infty} e^{-\frac{1}{2} \zX^{2}} \pbetaFn{ -\zX } d\zX} \right)
  \left( \tfrac{\int_{\zX=0}^{\zX=\infty} \zX e^{-\frac{1}{2} \zX^{2}} \pbetaFn{ \zX } \pbetaFn{ -\zX } d\zX}{\int_{\zX=0}^{\zX=\infty} e^{-\frac{1}{2} \zX^{2}} \pbetaFn{ \zX } \pbetaFn{ -\zX } d\zX} \right)
  -
  \left( \tfrac{\int_{\zX=0}^{\zX=\infty} \zX^{2} e^{-\frac{1}{2} \zX^{2}} \pbetaFn{ \zX } \pbetaFn{ -\zX } d\zX}{\int_{\zX=0}^{\zX=\infty} e^{-\frac{1}{2} \zX^{2}} \pbetaFn{ \zX } \pbetaFn{ -\zX } d\zX} \right)
  \right)
  \\
  &\AlignSp \leq
  \left( \int_{\zX=0}^{\zX=\infty} e^{-\frac{1}{2} \zX^{2}} \pbetaFn{ -\zX } d\zX \right)
  \left( \int_{\zX=0}^{\zX=\infty} e^{-\frac{1}{2} \zX^{2}} \pbetaFn{ \zX } \pbetaFn{ -\zX } d\zX \right)
  \\
  &\AlignSp\AlignSp
  \left(
    \left(
      \tfrac
      {\int_{\zX=0}^{\zX=\infty} \zX e^{-\frac{1}{2} \zX^{2}} \pbetaFn{ \zX } \pbetaFn{ -\zX } d\zX}
      {\int_{\zX=0}^{\zX=\infty} e^{-\frac{1}{2} \zX^{2}} \pbetaFn{ \zX } \pbetaFn{ -\zX } d\zX}
    \right)^{2}
    -
    \tfrac
    {\int_{\zX=0}^{\zX=\infty} \zX^{2} e^{-\frac{1}{2} \zX^{2}} \pbetaFn{ \zX } \pbetaFn{ -\zX } d\zX}
    {\int_{\zX=0}^{\zX=\infty} e^{-\frac{1}{2} \zX^{2}} \pbetaFn{ \zX } \pbetaFn{ -\zX } d\zX}
  \right)
,\end{align*}
%|<<|===========================================================================================|<<|
as desired.
\end{subproof}
%|<<|~~~~~~~~~~~~~~~~~~~~~~~~~~~~~~~~~~~~~~~~~~~~~~~~~~~~~~~~~~~~~~~~~~~~~~~~~~~~~~~~~~~~~~~~~~~|<<|
%|<<|~~~~~~~~~~~~~~~~~~~~~~~~~~~~~~~~~~~~~~~~~~~~~~~~~~~~~~~~~~~~~~~~~~~~~~~~~~~~~~~~~~~~~~~~~~~|<<|
%|<<|~~~~~~~~~~~~~~~~~~~~~~~~~~~~~~~~~~~~~~~~~~~~~~~~~~~~~~~~~~~~~~~~~~~~~~~~~~~~~~~~~~~~~~~~~~~|<<|
Having established \CLAIM \ref{claim:pf:corollary:main-technical:logistic-regression:1}, \COROLLARY \ref{corollary:main-technical:logistic-regression} for logistic regression is thus proved.
%
\let\wX\oldwX%
\let\WRV\oldWRV%
%\end{proof}
%|<<|~~~~~~~~~~~~~~~~~~~~~~~~~~~~~~~~~~~~~~~~~~~~~~~~~~~~~~~~~~~~~~~~~~~~~~~~~~~~~~~~~~~~~~~~~~~|<<|
%|<<|~~~~~~~~~~~~~~~~~~~~~~~~~~~~~~~~~~~~~~~~~~~~~~~~~~~~~~~~~~~~~~~~~~~~~~~~~~~~~~~~~~~~~~~~~~~|<<|
%|<<|~~~~~~~~~~~~~~~~~~~~~~~~~~~~~~~~~~~~~~~~~~~~~~~~~~~~~~~~~~~~~~~~~~~~~~~~~~~~~~~~~~~~~~~~~~~|<<|

Next, the specialization to \textbf{probit regression} in \COROLLARY \ref{corollary:main-technical:logistic-regression} is addressed.

%|>>|~~~~~~~~~~~~~~~~~~~~~~~~~~~~~~~~~~~~~~~~~~~~~~~~~~~~~~~~~~~~~~~~~~~~~~~~~~~~~~~~~~~~~~~~~~~|>>|
%|>>|~~~~~~~~~~~~~~~~~~~~~~~~~~~~~~~~~~~~~~~~~~~~~~~~~~~~~~~~~~~~~~~~~~~~~~~~~~~~~~~~~~~~~~~~~~~|>>|
%|>>|~~~~~~~~~~~~~~~~~~~~~~~~~~~~~~~~~~~~~~~~~~~~~~~~~~~~~~~~~~~~~~~~~~~~~~~~~~~~~~~~~~~~~~~~~~~|>>|
% \begin{proof}
% {\COROLLARY \ref{corollary:main-technical:probit}}
%
\mostlycheckoff%
%
As in the proof of %\COROLLARY \ref{corollary:main-technical:logistic-regression}, the corollary 
the logistic case, the probit case follows from a two-step argument---the first being \STEP \Enum[{\label{enum:pf:corollary:main-technical:probit:a}}]{a}, verifying that \ASSUMPTION \ref{assumption:p} holds when \(  \pFn  \) is parameterized by the \betaXnamepr, \(  \betaX \GTR 0  \), and defined at \(  \zX \in \R  \) as
%|>>|===========================================================================================|>>|
\begin{gather}
\label{eqn:pf:corollary:main-technical:probit:p}
  \pbetaFn{ \zX } = \frac{1}{\sqrt{2\pi}} \int_{\uX=-\infty}^{\uX=\betaX \zX} e^{-\frac{1}{2} \uX^{2}} d\uX
\end{gather}
%|<<|===========================================================================================|<<|
for probit regression, as per \DEFINITION \ref{def:p:probit}; and the second being \STEP \Enum[{\label{enum:pf:corollary:main-technical:probit:b}}]{b}, deriving \(  \alphaX  \) and \(  \gammaX  \).
%and establishing upper bounds on \(  \alphaX  \) and \(  \gammaX  \).
%the corollary will immediately follow from \THEOREMS \ref{thm:main-technical:dense} and \ref{thm:main-technical:sparse}.
%
%%%%%%%%%%%%%%%%%%%%%%%%%%%%%%%%%%%%%%%%%%%%%%%%%%%%%%%%%%%%%%%%%%%%%%%%%%%%%%%%%%%%%%%%%%%%%%%%%%%%
\par %%%%%%%%%%%%%%%%%%%%%%%%%%%%%%%%%%%%%%%%%%%%%%%%%%%%%%%%%%%%%%%%%%%%%%%%%%%%%%%%%%%%%%%%%%%%%%%
%%%%%%%%%%%%%%%%%%%%%%%%%%%%%%%%%%%%%%%%%%%%%%%%%%%%%%%%%%%%%%%%%%%%%%%%%%%%%%%%%%%%%%%%%%%%%%%%%%%%
%
For  \ref{enum:pf:corollary:main-technical:probit:a},
beginning with \CONDITION \ref{condition:assumption:p:i} of \ASSUMPTION \ref{assumption:p}---that \(  \pFn  \) nondecreasing over the real line---notice that when \(  \betaX = 0  \), the function \(  \pFn  \) is constant:
%|>>|:::::::::::::::::::::::::::::::::::::::::::::::::::::::::::::::::::::::::::::::::::::::::::|>>|
\(  \pFn( \zX ) = \frac{1}{2}  \)
%|<<|:::::::::::::::::::::::::::::::::::::::::::::::::::::::::::::::::::::::::::::::::::::::::::|<<|
for all \(  \zX \in \R  \), and therefore, in this scenario, \(  \pFn  \) is nondecreasing.
Otherwise, when \(  \betaX > 0  \), \(  \pFn  \) can be rewritten as follows for \(  \zX \in \R  \):
%|>>|===========================================================================================|>>|
\begin{gather*}
  \pbetaFn{ \zX }
  =
  \frac{1}{\sqrt{2\pi}} \int_{\uX=-\infty}^{\uX=\betaX \zX} e^{-\frac{1}{2} \uX^{2}} d\uX
  =
  \frac{\betaX}{\sqrt{2\pi}} \int_{\uX=-\infty}^{\uX=\zX} e^{-\frac{1}{2} \betaX^{2} \uX^{2}} d\uX
,\end{gather*}
%|<<|===========================================================================================|<<|
where the second equality applies a change of variables.
Hence, \(  \pFn  \) is the distribution function of a mean\nobreakdash-\(  0  \), variance\nobreakdash-\(  \frac{1}{\betaX^{2}}  \) Gaussian random variable.
Since distribution functions are nondecreasing, clearly the first condition of \ASSUMPTION \ref{assumption:p} holds under this model.
Proceeding to the second requirement, \CONDITION \ref{condition:assumption:p:ii}, of \ASSUMPTION \ref{assumption:p}---that
%|>>|===========================================================================================|>>|
\begin{gather}
\label{eqn:pf:corollary:main-technical:probit:1}
  \frac{\partial}{\partial \zX}
  \frac{\pExpr*{\zX+\wX \betaXParam}}{\pExpr{\zX \betaXParam}}
  \leq 0
\end{gather}
%|<<|===========================================================================================|<<|
for all
%|>>|:::::::::::::::::::::::::::::::::::::::::::::::::::::::::::::::::::::::::::::::::::::::::::|>>|
\(  \zX \in \R  \) and \(  \wX > 0  \)%
%|<<|:::::::::::::::::::::::::::::::::::::::::::::::::::::::::::::::::::::::::::::::::::::::::::|<<|
%(recall the remark in the proof of \COROLLARY \ref{corollary:main-technical:logistic-regression} justifying the scaling by a factor of \(  \betaX \GTR 0  \))%
---note the following properties of \(  \pFn  \):
%|>>|===========================================================================================|>>|
\begin{gather}
\label{eqn:pf:corollary:main-technical:probit:6}
  1 - \pbetaFn{ \zX }
  =
  1 - \frac{1}{\sqrt{2\pi}} \int_{\uX=-\infty}^{\uX=\betaX \zX} e^{-\frac{1}{2} \uX^{2}} d\uX
  =
  \frac{1}{\sqrt{2\pi}} \int_{\uX=\betaX \zX}^{\uX=\infty} e^{-\frac{1}{2} \uX^{2}} d\uX
  =
  \frac{1}{\sqrt{2\pi}} \int_{\uX=-\infty}^{\uX=-\betaX \zX} e^{-\frac{1}{2} \uX^{2}} d\uX
  =
  \pbetaFn{ -\zX }
  ,\\
\label{eqn:pf:corollary:main-technical:probit:7}
  1 - \pbetaFn{ \zX }
  =
  1 - \frac{1}{\sqrt{2\pi}} \int_{\uX=-\infty}^{\uX=\betaX \zX} e^{-\frac{1}{2} \uX^{2}} d\uX
  =
  \frac{1}{\sqrt{2\pi}} \int_{\uX=\betaX \zX}^{\uX=\infty} e^{-\frac{1}{2} \uX^{2}} d\uX
  =
  \frac{1}{\sqrt{2\pi}} \int_{\uX=0}^{\uX=\infty} e^{-\frac{1}{2} ( \uX+\betaX \zX )^{2}} d\uX
,\end{gather}
%|<<|===========================================================================================|<<|
where
%|>>|:::::::::::::::::::::::::::::::::::::::::::::::::::::::::::::::::::::::::::::::::::::::::::|>>|
\(  \zX \in \R  \).
%|<<|:::::::::::::::::::::::::::::::::::::::::::::::::::::::::::::::::::::::::::::::::::::::::::|<<|
%Additionally, as discussed in \REMARK \ref{remark:condition:assumption:p:ii}, it suffices to verify \ASSUMPTION \ref{condition:assumption:p:ii} for \(  \betaX = 1  \).
Additionally, it suffices to verify \ASSUMPTION \ref{condition:assumption:p:ii} for \(  \betaX = 1  \).
Thus, for \(  \wX > 0  \),
%|>>|===========================================================================================|>>|
\begin{align*}
  \frac{\pExpr*{\zX+\wX}}{\pExpr{\zX}}
  &=
  \frac{2( 1-\pbetaFn{ \zX+\wX })}{2( 1-\pbetaFn{ \zX } )}
  \\
  &=
  \frac{1-\pbetaFn{ \zX+\wX }}{1-\pbetaFn{ \zX }}
  \\
  &=
  \frac
  {\frac{1}{\sqrt{2\pi}} \int_{\uX=0}^{\uX=\infty} e^{-\frac{1}{2} ( \uX + \zX+\wX )^{2}} d\uX}
  {\frac{1}{\sqrt{2\pi}} \int_{\uX=0}^{\uX=\infty} e^{-\frac{1}{2} ( \uX + \zX )^{2}} d\uX}
  \\
  &=
  \frac
  {\int_{\uX=0}^{\uX=\infty} e^{-\frac{1}{2} ( \uX + \zX+\wX )^{2}} d\uX}
  {\int_{\uX=0}^{\uX=\infty} e^{-\frac{1}{2} ( \uX + \zX )^{2}} d\uX}
,\end{align*}
%|<<|===========================================================================================|<<|
and hence, the desired inequality in \EQUATION \eqref{eqn:pf:corollary:main-technical:probit:1} is equivalently stated as:
%|>>|===========================================================================================|>>|
\begin{align*}
  \frac{\partial}{\partial \zX}
  \frac{\pExpr*{\zX+\wX}}{\pExpr{\zX}}
  &=
  \frac{\partial}{\partial \zX}
  \frac
  {\int_{\uX=0}^{\uX=\infty} e^{-\frac{1}{2} ( \uX + \zX+\wX )^{2}} d\uX}
  {\int_{\uX=0}^{\uX=\infty} e^{-\frac{1}{2} ( \uX + \zX )^{2}} d\uX}
  \leq 0
.\end{align*}
%|<<|===========================================================================================|<<|
To evaluate this partial derivative, observe:
%|>>|===========================================================================================|>>|
\begin{align*}
  &\AlignSp
  \frac{\partial}{\partial \zX}
  \frac{\pExpr*{\zX+\wX}}{\pExpr{\zX}}
  \\
  &\AlignSp=
  \frac{\partial}{\partial \zX}
  \frac
  {\int_{\uX=0}^{\uX=\infty} e^{-\frac{1}{2} ( \uX + \zX+\wX )^{2}} d\uX}
  {\int_{\uX=0}^{\uX=\infty} e^{-\frac{1}{2} ( \uX + \zX )^{2}} d\uX}
  % \\
  % &\AlignSp=
  % \frac
  % {\left( \int_{\uX=0}^{\uX=\infty} e^{-\frac{1}{2} ( \uX + \zX )^{2}} d\uX \right)
  %  \left( \frac{\partial}{\partial \zX} \int_{\uX=0}^{\uX=\infty} e^{-\frac{1}{2} ( \uX + \zX+\wX )^{2}} d\uX \right)
  %  -
  %  \left( \int_{\uX=0}^{\uX=\infty} e^{-\frac{1}{2} ( \uX + \zX+\wX )^{2}} d\uX \right)
  %  \left( \frac{\partial}{\partial \zX} \int_{\uX=0}^{\uX=\infty} e^{-\frac{1}{2} ( \uX + \zX )^{2}} d\uX \right)}
  % {\left( \int_{\uX=0}^{\uX=\infty} e^{-\frac{1}{2} ( \uX + \zX )^{2}} d\uX \right)^{2}}
  \\
  &\AlignSp=
  \frac
  {\left( \int_{\uX=0}^{\uX=\infty} e^{-\frac{1}{2} ( \uX + \zX )^{2}} d\uX \right)
   \left( \frac{\partial}{\partial \zX} \int_{\uX=0}^{\uX=\infty} e^{-\frac{1}{2} ( \uX + \zX+\wX )^{2}} d\uX \right)}
   {\left( \int_{\uX=0}^{\uX=\infty} e^{-\frac{1}{2} ( \uX + \zX )^{2}} d\uX \right)^{2}}
   \\
   &\AlignSp\AlignSp-
   \frac{
   \left( \int_{\uX=0}^{\uX=\infty} e^{-\frac{1}{2} ( \uX + \zX+\wX )^{2}} d\uX \right)
   \left( \frac{\partial}{\partial \zX} \int_{\uX=0}^{\uX=\infty} e^{-\frac{1}{2} ( \uX + \zX )^{2}} d\uX \right)}
  {\left( \int_{\uX=0}^{\uX=\infty} e^{-\frac{1}{2} ( \uX + \zX )^{2}} d\uX \right)^{2}}
  \\
  &\dCmt{by the quotient rule}
  \\
  &\AlignSp=
  \frac{1}{\left( \int_{\uX=0}^{\uX=\infty} e^{-\frac{1}{2} ( \uX + \zX )^{2}} d\uX \right)^{2}}
  \left(
  \left( \int_{\uX=0}^{\uX=\infty} e^{-\frac{1}{2} ( \uX + \zX )^{2}} d\uX \right)
   \left( \frac{\partial}{\partial \zX} \int_{\uX=0}^{\uX=\infty} e^{-\frac{1}{2} ( \uX + \zX+\wX )^{2}} d\uX \right)
   \right.
   \\
   &\phantom{\AlignSp \displaystyle \frac{1}{\displaystyle \left( \int_{\uX=0}^{\uX=\infty} e^{-\frac{1}{2} ( \uX + \zX )^{2}} d\uX \right)^{2}} \Biggl(} \AlignSp
   \left.
   -
   \left( \int_{\uX=0}^{\uX=\infty} e^{-\frac{1}{2} ( \uX + \zX+\wX )^{2}} d\uX \right)
   \left( \frac{\partial}{\partial \zX} \int_{\uX=0}^{\uX=\infty} e^{-\frac{1}{2} ( \uX + \zX )^{2}} d\uX \right)
   \right)
  \\
  &\AlignSp=
  \frac{1}{\left( \int_{\uX=0}^{\uX=\infty} e^{-\frac{1}{2} ( \uX + \zX )^{2}} d\uX \right)^{2}}
  \left(
  \left( \int_{\uX=0}^{\uX=\infty} e^{-\frac{1}{2} ( \uX + \zX )^{2}} d\uX \right)
  \left( \int_{\uX=0}^{\uX=\infty} \frac{\partial}{\partial \zX} e^{-\frac{1}{2} ( \uX + \zX+\wX )^{2}} d\uX \right)
  \right.
  \\
  &\phantom{\AlignSp \displaystyle \frac{1}{\displaystyle \left( \int_{\uX=0}^{\uX=\infty} e^{-\frac{1}{2} ( \uX + \zX )^{2}} d\uX \right)^{2}} \Biggl(} \AlignSp
  \left.
  -
  \left( \int_{\uX=0}^{\uX=\infty} e^{-\frac{1}{2} ( \uX + \zX+\wX )^{2}} d\uX \right)
  \left( \int_{\uX=0}^{\uX=\infty} \frac{\partial}{\partial \zX} e^{-\frac{1}{2} ( \uX + \zX )^{2}} d\uX \right)
  \right)
  \\
  &\dCmt{\(  \zX  \) does not depend on the variable of integration}
  \\
  &\AlignSp=
  \frac{1}{\left( \int_{\uX=0}^{\uX=\infty} e^{-\frac{1}{2} ( \uX + \zX )^{2}} d\uX \right)^{2}}
  \left(
  \left( \int_{\uX=0}^{\uX=\infty} e^{-\frac{1}{2} ( \uX + \zX )^{2}} d\uX \right)
  \left( -\int_{\uX=0}^{\uX=\infty} ( \uX + \zX+\wX ) e^{-\frac{1}{2} ( \uX + \zX+\wX )^{2}} d\uX \right)
  \right.
  \\
  &\phantom{\AlignSp \displaystyle \frac{1}{\displaystyle \left( \int_{\uX=0}^{\uX=\infty} e^{-\frac{1}{2} ( \uX + \zX )^{2}} d\uX \right)^{2}} \Biggl(} \AlignSp
  \left.
  -
  \left( \int_{\uX=0}^{\uX=\infty} e^{-\frac{1}{2} ( \uX + \zX+\wX )^{2}} d\uX \right)
  \left( -\int_{\uX=0}^{\uX=\infty} ( \uX + \zX ) e^{-\frac{1}{2} ( \uX + \zX )^{2}} d\uX \right)
  \right)
  \\
  &\dCmt{via the chain rule}
  \\
  &\AlignSp=
  \frac
  {
   \left( \int_{\uX=0}^{\uX=\infty} e^{-\frac{1}{2} ( \uX + \zX )^{2}} d\uX \right)
   \left( \int_{\uX=0}^{\uX=\infty} e^{-\frac{1}{2} ( \uX + \zX+\wX )^{2}} d\uX \right)}
  {\left( \int_{\uX=0}^{\uX=\infty} e^{-\frac{1}{2} ( \uX + \zX )^{2}} d\uX \right)^{2}}
  \\
  &\AlignSp\AlignSp
  \left(
  \frac
  {\int_{\uX=0}^{\uX=\infty} ( \uX + \zX ) e^{-\frac{1}{2} ( \uX + \zX )^{2}} d\uX}
  {\int_{\uX=0}^{\uX=\infty} e^{-\frac{1}{2} ( \uX + \zX )^{2}} d\uX}
  -
  \frac
  {\int_{\uX=0}^{\uX=\infty} ( \uX + \zX+\wX ) e^{-\frac{1}{2} ( \uX + \zX+\wX )^{2}} d\uX}
  {\int_{\uX=0}^{\uX=\infty} e^{-\frac{1}{2} ( \uX + \zX+\wX )^{2}} d\uX}
  \right)
  \\
  &\dCmt{by distributivity and commutativity}
  \\
  &\AlignSp=
  \frac
  {
   \left( \int_{\uX=\zX}^{\uX=\infty} e^{-\frac{1}{2} \uX^{2}} d\uX \right)
   \left( \int_{\uX=\zX+\wX}^{\uX=\infty} e^{-\frac{1}{2} \uX^{2}} d\uX \right)}
  {\left( \int_{\uX=\zX}^{\uX=\infty} e^{-\frac{1}{2} \uX^{2}} d\uX \right)^{2}}
  \left(
  \frac
  {\int_{\uX=\zX}^{\uX=\infty} \uX e^{-\frac{1}{2} \uX^{2}} d\uX}
  {\int_{\uX=\zX}^{\uX=\infty} e^{-\frac{1}{2} \uX^{2}} d\uX}
  -
  \frac
  {\int_{\uX=\zX+\wX}^{\uX=\infty} \uX e^{-\frac{1}{2} \uX^{2}} d\uX}
  {\int_{\uX=\zX+\wX}^{\uX=\infty} e^{-\frac{1}{2} \uX^{2}} d\uX}
  \right)
  .\\
  &\dCmt{by a change of variables (applied to each of the four integrals)}
\end{align*}
%|<<|===========================================================================================|<<|
Notice in the last line that the sign of
%|>>|:::::::::::::::::::::::::::::::::::::::::::::::::::::::::::::::::::::::::::::::::::::::::::|>>|
\(  \frac{\partial}{\partial \zX} \frac{\pExpr*{\zX+\wX}}{\pExpr{\zX}}  \)
%|<<|:::::::::::::::::::::::::::::::::::::::::::::::::::::::::::::::::::::::::::::::::::::::::::|<<|
is entirely determined by the sign of the rightmost multiplicand,
%|>>|===========================================================================================|>>|
\begin{gather*}
  \frac
  {\int_{\uX=\zX}^{\uX=\infty} \uX e^{-\frac{1}{2} \uX^{2}} d\uX}
  {\int_{\uX=\zX}^{\uX=\infty} e^{-\frac{1}{2} \uX^{2}} d\uX}
  -
  \frac
  {\int_{\uX=\zX+\wX}^{\uX=\infty} \uX e^{-\frac{1}{2} \uX^{2}} d\uX}
  {\int_{\uX=\zX+\wX}^{\uX=\infty} e^{-\frac{1}{2} \uX^{2}} d\uX}
,\end{gather*}
%|<<|===========================================================================================|<<|
and hence, it suffices to show nonpositivity of this term, \ie that
%|>>|===========================================================================================|>>|
\begin{gather}
\label{eqn:pf:corollary:main-technical:probit:2}
  \frac
  {\int_{\uX=\zX}^{\uX=\infty} \uX e^{-\frac{1}{2} \uX^{2}} d\uX}
  {\int_{\uX=\zX}^{\uX=\infty} e^{-\frac{1}{2} \uX^{2}} d\uX}
  -
  \frac
  {\int_{\uX=\zX+\wX}^{\uX=\infty} \uX e^{-\frac{1}{2} \uX^{2}} d\uX}
  {\int_{\uX=\zX+\wX}^{\uX=\infty} e^{-\frac{1}{2} \uX^{2}} d\uX}
  \leq 0
.\end{gather}
%|<<|===========================================================================================|<<|
Towards verifying this last inequality, \eqref{eqn:pf:corollary:main-technical:probit:2}, let
%|>>|:::::::::::::::::::::::::::::::::::::::::::::::::::::::::::::::::::::::::::::::::::::::::::|>>|
\(  \URV \sim \N(0,1)  \)
%|<<|:::::::::::::::::::::::::::::::::::::::::::::::::::::::::::::::::::::::::::::::::::::::::::|<<|
be a standard univariate Gaussian random variable, and let \(  \VRV  \) and \(  \WRV  \) be indicator random variables given by
%|>>|:::::::::::::::::::::::::::::::::::::::::::::::::::::::::::::::::::::::::::::::::::::::::::|>>|
\(  \VRV \defeq \I( | \URV | \geq \zX )  \) and
\(  \WRV \defeq \I( | \URV | \geq \zX+\wX )  \),
%|<<|:::::::::::::::::::::::::::::::::::::::::::::::::::::::::::::::::::::::::::::::::::::::::::|<<|
respectively.
The random variable \(  | \URV |  \) is standard half-normal with density
%|>>|===========================================================================================|>>|
\begin{gather}
\label{eqn:pf:corollary:main-technical:probit:5}
  \pdf{| \URV |}( \uX )
  =
  \sqrt{\frac{2}{\pi}} e^{-\frac{1}{2} \uX^{2}}
.\end{gather}
%|<<|===========================================================================================|<<|
The masses of the conditioned random variables \(  \VRV=1 \Mid| | \URV |  \) and \(  \WRV=1 \Mid| | \URV |  \) are given by
%|>>|===========================================================================================|>>|
\begin{gather}
\label{eqn:pf:corollary:main-technical:probit:3}
  \pdf{\VRV \Mid| | \URV |}( 1 \Mid| \uX )
  =
  \begin{cases}
  0 ,& \cIf \uX    < \zX,\\
  1 ,& \cIf \uX \geq \zX,
  \end{cases}
  \\
\label{eqn:pf:corollary:main-technical:probit:4}
  \pdf{\WRV \Mid| | \URV |}( 1 \Mid| \uX )
  =
  \begin{cases}
  0 ,& \cIf \uX    < \zX+\wX,\\
  1 ,& \cIf \uX \geq \zX+\wX.
  \end{cases}
\end{gather}
%|<<|===========================================================================================|<<|
Observe:
%|>>|===========================================================================================|>>|
\begin{align*}
  \pdf{\VRV}( 1 )
  % &=
  % \int_{\uX=-\infty}^{\uX=\infty}
  % \pdf{\VRV, | \URV |}( 1, \uX )
  % d\uX
  % \\
  % &\dCmt{by the law of total probability}
  % \\
  % &=
  % \int_{\uX=-\infty}^{\uX=\infty}
  % \pdf{| \URV |}( \uX )
  % \pdf{\VRV \Mid| | \URV |}( 1 \Mid| \uX )
  % d\uX
  % \\
  % &\dCmt{by the law of total probability and the definition of conditional probabilities}
  % \\
  &=
  \int_{\uX=0}^{\uX=\infty}
  \pdf{| \URV |}( \uX )
  \pdf{\VRV \Mid| | \URV |}( 1 \Mid| \uX )
  d\uX
  \\
  &\dCmt{by the law of total probability, definition of}
  \\
  &\dCmtx{conditional probabilities, and support of \(  \pdf{| \URV |}  \)}
  \\
  &=
  \int_{\uX=0}^{\uX=\zX}
  \pdf{| \URV |}( \uX )
  \cdot 0
  d\uX
  +
  \int_{\uX=\zX}^{\uX=\infty}
  \pdf{| \URV |}( \uX )
  \cdot 1
  d\uX
  \\
  % &\dCmt{by the linearity of integration and \EQUATION \eqref{eqn:pf:corollary:main-technical:probit:3}}
  &\dCmt{by \EQUATION \eqref{eqn:pf:corollary:main-technical:probit:3}}
  % \\
  % &=
  % \int_{\uX=\zX}^{\uX=\infty}
  % \pdf{| \URV |}( \uX )
  % d\uX
  \\
  &=
  \sqrt{\frac{2}{\pi}}
  \int_{\uX=\zX}^{\uX=\infty}
  e^{-\frac{1}{2} \uX^{2}}
  d\uX
  ,\\
  &\dCmt{by \EQUATION \eqref{eqn:pf:corollary:main-technical:probit:5}}
\end{align*}
%|<<|===========================================================================================|<<|
and likewise,
%|>>|===========================================================================================|>>|
\begin{align*}
  \pdf{\WRV}( 1 )
  % &=
  % \int_{\uX=-\infty}^{\uX=\infty}
  % \pdf{\WRV, | \URV |}( 1, \uX )
  % d\uX
  % \\
  % &\dCmt{by the law of total probability}
  % \\
  % &=
  % \int_{\uX=-\infty}^{\uX=\infty}
  % \pdf{| \URV |}( \uX )
  % \pdf{\WRV \Mid| | \URV |}( 1 \Mid| \uX )
  % d\uX
  % \\
  % &\dCmt{by the definition of conditional probabilities}
  % \\
  &=
  \int_{\uX=0}^{\uX=\infty}
  \pdf{| \URV |}( \uX )
  \pdf{\WRV \Mid| | \URV |}( 1 \Mid| \uX )
  d\uX
  \\
  &\dCmt{by the law of total probability, definition of}
  \\
  &\dCmtx{conditional probabilities, and support of \(  \pdf{| \URV |}  \)}
  \\
  &=
  \int_{\uX=0}^{\uX=\zX+\wX}
  \pdf{| \URV |}( \uX )
  \cdot 0
  d\uX
  +
  \int_{\uX=\zX+\wX}^{\uX=\infty}
  \pdf{| \URV |}( \uX )
  \cdot 1
  d\uX
  \\
  % &\dCmt{by the linearity of integration and \EQUATION \eqref{eqn:pf:corollary:main-technical:probit:4}}
  &\dCmt{by \EQUATION \eqref{eqn:pf:corollary:main-technical:probit:4}}
  % \\
  % &=
  % \int_{\uX=\zX+\wX}^{\uX=\infty}
  % \pdf{| \URV |}( \uX )
  % d\uX
  \\
  &=
  \sqrt{\frac{2}{\pi}}
  \int_{\uX=\zX+\wX}^{\uX=\infty}
  e^{-\frac{1}{2} \uX^{2}}
  d\uX
  ,\\
  &\dCmt{by \EQUATION \eqref{eqn:pf:corollary:main-technical:probit:5}}
\end{align*}
%|<<|===========================================================================================|<<|
%%|>>|===========================================================================================|>>|
%\begin{align*}
%  \pdf{\WRV}( 1 )
%  &=
%  \int_{\uX=0}^{\uX=\infty}
%  \pdf{| \URV |}( \uX )
%  \pdf{\WRV \Mid| | \URV |}( 1 \Mid| \uX )
%  d\uX
%  =
%  \int_{\uX=0}^{\uX=\zX+\wX}
%  \pdf{| \URV |}( \uX )
%  \cdot 0
%  d\uX
%  +
%  \int_{\uX=\zX+\wX}^{\uX=\infty}
%  \pdf{| \URV |}( \uX )
%  \cdot 1
%  d\uX
%  \\
%  &=
%  \int_{\uX=\zX+\wX}^{\uX=\infty}
%  \pdf{| \URV |}( \uX )
%  d\uX
%  =
%  \sqrt{\frac{2}{\pi}}
%  \int_{\uX=\zX+\wX}^{\uX=\infty}
%  e^{-\frac{1}{2} \uX^{2}}
%  d\uX
%.\end{align*}
%%|<<|===========================================================================================|<<|
By Bayes' theorem,
%|>>|===========================================================================================|>>|
\begin{align*}
  \pdf{| \URV | \Mid| \VRV}( \uX \Mid| 1 )
  &=
  \frac
  {\pdf{| \URV |}( \uX ) \pdf{\VRV \Mid| | \URV |}( 1 \Mid| \uX )}
  {\pdf{\VRV}( 1 )}
  \\
  &=
  \begin{cases}
  0 ,& \cIf \uX < \zX ,\\
  \frac
  {\sqrt{\frac{2}{\pi}} e^{-\frac{1}{2} \uX^{2}} d\uX}
  {\sqrt{\frac{2}{\pi}} \int_{\yX=\zX}^{\yX=\infty} e^{-\frac{1}{2} \yX^{2}} d\yX}
  ,& \cIf \uX \geq \zX,
  \end{cases}
  \\
  &=
  \begin{cases}
  0 ,& \cIf \uX < \zX ,\\
  \frac
  {e^{-\frac{1}{2} \uX^{2}} d\uX}
  {\int_{\yX=\zX}^{\yX=\infty} e^{-\frac{1}{2} \yX^{2}} d\yX}
  ,& \cIf \uX \geq \zX,
  \end{cases}
\end{align*}
%|<<|===========================================================================================|<<|
and
%|>>|===========================================================================================|>>|
\begin{align*}
  \pdf{| \URV | \Mid| \WRV}( \uX \Mid| 1 )
  &=
  \frac
  {\pdf{| \URV |}( \uX ) \pdf{\WRV \Mid| | \URV |}( 1 \Mid| \uX )}
  {\pdf{\WRV}( 1 )}
  \\
  &=
  \begin{cases}
  0 ,& \cIf \uX < \zX+\wX ,\\
  \frac
  {\sqrt{\frac{2}{\pi}} e^{-\frac{1}{2} \uX^{2}} d\uX}
  {\sqrt{\frac{2}{\pi}} \int_{\yX=\zX+\wX}^{\yX=\infty} e^{-\frac{1}{2} \yX^{2}} d\yX}
  ,& \cIf \uX \geq \zX+\wX,
  \end{cases}
  \\
  &=
  \begin{cases}
  0 ,& \cIf \uX < \zX+\wX ,\\
  \frac
  {e^{-\frac{1}{2} \uX^{2}} d\uX}
  {\int_{\yX=\zX+\wX}^{\yX=\infty} e^{-\frac{1}{2} \yX^{2}} d\yX}
  ,& \cIf \uX \geq \zX+\wX.
  \end{cases}
\end{align*}
%|<<|===========================================================================================|<<|
Therefore, in expectation,
%|>>|===========================================================================================|>>|
\begin{gather*}
  \E[ | \URV | \Mid| | \URV | \geq \zX ]
  =
  \E[ | \URV | \Mid| \VRV=1 ]
  =
  \frac
  {\int_{\uX=\zX}^{\uX=\infty} \uX e^{-\frac{1}{2} \uX^{2}} d\uX}
  {\int_{\uX=\zX}^{\uX=\infty} e^{-\frac{1}{2} \uX^{2}} d\uX}
  ,\\
  \E[ | \URV | \Mid| | \URV | \geq \zX+\wX ]
  =
  \E[ | \URV | \Mid| \WRV=1 ]
  =
  \frac
  {\int_{\uX=\zX+\wX}^{\uX=\infty} \uX e^{-\frac{1}{2} \uX^{2}} d\uX}
  {\int_{\uX=\zX+\wX}^{\uX=\infty} e^{-\frac{1}{2} \uX^{2}} d\uX}
.\end{gather*}
%|<<|===========================================================================================|<<|
Note that
%|>>|:::::::::::::::::::::::::::::::::::::::::::::::::::::::::::::::::::::::::::::::::::::::::::|>>|
\(  \E[ | \URV | \Mid| | \URV | \geq \zX ] \leq \E[ | \URV | \Mid| | \URV | \geq \zX+\wX ]  \)
%|<<|:::::::::::::::::::::::::::::::::::::::::::::::::::::::::::::::::::::::::::::::::::::::::::|<<|
when \(  \wX > 0  \), implying that
%|>>|===========================================================================================|>>|
\begin{gather*}
  \frac
  {\int_{\uX=\zX}^{\uX=\infty} \uX e^{-\frac{1}{2} \uX^{2}} d\uX}
  {\int_{\uX=\zX}^{\uX=\infty} e^{-\frac{1}{2} \uX^{2}} d\uX}
  =
  \E[ | \URV | \Mid| | \URV | \geq \zX ]
  \leq
  \E[ | \URV | \Mid| | \URV | \geq \zX+\wX ]
  =
  \frac
  {\int_{\uX=\zX+\wX}^{\uX=\infty} \uX e^{-\frac{1}{2} \uX^{2}} d\uX}
  {\int_{\uX=\zX+\wX}^{\uX=\infty} e^{-\frac{1}{2} \uX^{2}} d\uX}
,\end{gather*}
%|<<|===========================================================================================|<<|
and hence also that
%|>>|===========================================================================================|>>|
\begin{gather*}
  \frac
  {\int_{\uX=\zX}^{\uX=\infty} \uX e^{-\frac{1}{2} \uX^{2}} d\uX}
  {\int_{\uX=\zX}^{\uX=\infty} e^{-\frac{1}{2} \uX^{2}} d\uX}
  -
  \frac
  {\int_{\uX=\zX+\wX}^{\uX=\infty} \uX e^{-\frac{1}{2} \uX^{2}} d\uX}
  {\int_{\uX=\zX+\wX}^{\uX=\infty} e^{-\frac{1}{2} \uX^{2}} d\uX}
  \leq 0
,\end{gather*}
%|<<|===========================================================================================|<<|
as desired.
It follows from this and the earlier discussion that
%|>>|===========================================================================================|>>|
\begin{align*}
  \frac{\partial}{\partial \zX}
  \frac{\pExpr*{\zX+\wX}}{\pExpr{\zX}}
  &=
  \frac{\partial}{\partial \zX}
  \frac
  {\int_{\uX=0}^{\uX=\infty} e^{-\frac{1}{2} ( \uX + \zX+\wX )^{2}} d\uX}
  {\int_{\uX=0}^{\uX=\infty} e^{-\frac{1}{2} ( \uX + \zX )^{2}} d\uX}
  \leq 0
,\end{align*}
%|<<|===========================================================================================|<<|
which verifies that \CONDITION \ref{condition:assumption:p:ii} of \ASSUMPTION \ref{assumption:p} is upheld when \(  \pFn  \) is as defined for probit regression.
Having shown that both conditions of \ASSUMPTION \ref{assumption:p} are satisfied for probit regression, \STEP \ref{enum:pf:corollary:main-technical:probit:a} is completed.
%
%%%%%%%%%%%%%%%%%%%%%%%%%%%%%%%%%%%%%%%%%%%%%%%%%%%%%%%%%%%%%%%%%%%%%%%%%%%%%%%%%%%%%%%%%%%%%%%%%%%%
\par %%%%%%%%%%%%%%%%%%%%%%%%%%%%%%%%%%%%%%%%%%%%%%%%%%%%%%%%%%%%%%%%%%%%%%%%%%%%%%%%%%%%%%%%%%%%%%%
%%%%%%%%%%%%%%%%%%%%%%%%%%%%%%%%%%%%%%%%%%%%%%%%%%%%%%%%%%%%%%%%%%%%%%%%%%%%%%%%%%%%%%%%%%%%%%%%%%%%
%
%Having shown that both conditions of \ASSUMPTION \ref{assumption:p} are satisfied for probit regression, the corollary now follows directly from the proofs of \THEOREMS \ref{thm:main-technical:dense} and \ref{thm:main-technical:sparse}.
%
Moving ahead with \STEP \ref{enum:pf:corollary:main-technical:probit:b}, recall that the aim here is to derive \(  \alphaX  \) and \(  \gammaX  \).
For \(  \alphaX  \), observe:
%|>>|===========================================================================================|>>|
\begin{align*}
  \alphaX
  &=
  \frac{1}{\sqrt{2\pi}}
  \int_{\zX=0}^{\zX=\infty}
  e^{-\frac{1}{2} \zX^{2}}
  ( 1 - \pFn( \zX ) + \pFn( -\zX ) )
  d\zX
  \\
  &=
  \frac{2}{\sqrt{2\pi}}
  \int_{\zX=0}^{\zX=\infty}
  e^{-\frac{1}{2} \zX^{2}}
  \pFn( -\zX )
  d\zX
  \\
  &=
  \frac{2}{\sqrt{2\pi}}
  \int_{\zX=0}^{\zX=\infty}
  e^{-\frac{1}{2} \zX^{2}}
  \frac{1}{\sqrt{2\pi}}
  \int_{\uX=-\infty}^{\uX=-\betaX \zX}
  e^{-\frac{1}{2} \uX^{2}}
  d\uX
  d\zX
  \\
  &=
  \frac{2}{\sqrt{2\pi}}
  \int_{\zX=0}^{\zX=\infty}
  e^{-\frac{1}{2} \zX^{2}}
  \left(
    \frac{1}{\sqrt{2\pi}}
    \int_{\uX=-\infty}^{\uX=0}
    e^{-\frac{1}{2} \uX^{2}}
    d\uX
    -
    \frac{1}{\sqrt{2\pi}}
    \int_{\uX=-\betaX \zX}^{\uX=0}
    e^{-\frac{1}{2} \uX^{2}}
    d\uX
  \right)
  d\zX
  \\
  &=
  \frac{2}{\sqrt{2\pi}}
  \int_{\zX=0}^{\zX=\infty}
  e^{-\frac{1}{2} \zX^{2}}
  \left(
    \frac{1}{2}
    -
    \frac{1}{\sqrt{2\pi}}
    \int_{\uX=-\betaX \zX}^{\uX=0}
    e^{-\frac{1}{2} \uX^{2}}
    d\uX
  \right)
  d\zX
  % \\
  % &=
  % \frac{2}{\sqrt{2\pi}}
  % \int_{\zX=0}^{\zX=\infty}
  % e^{-\frac{1}{2} \zX^{2}}
  % \left(
  %   \frac{1}{2}
  %   -
  %   \frac{1}{2}
  %   \frac{1}{\sqrt{2\pi}}
  %   \int_{\uX=-\betaX \zX}^{\uX=\betaX \zX}
  %   e^{-\frac{1}{2} \uX^{2}}
  %   d\uX
  % \right)
  % d\zX
  % \\
  % &=
  % \frac{2}{\sqrt{2\pi}}
  % \int_{\zX=0}^{\zX=\infty}
  % e^{-\frac{1}{2} \zX^{2}}
  % \frac{1}{2}
  % \left(
  %   1
  %   -
  %   \frac{1}{\sqrt{2\pi}}
  %   \int_{\uX=-\betaX \zX}^{\uX=\betaX \zX}
  %   e^{-\frac{1}{2} \uX^{2}}
  %   d\uX
  % \right)
  % d\zX
  \\
  &=
  \frac{1}{\sqrt{2\pi}}
  \int_{\zX=0}^{\zX=\infty}
  e^{-\frac{1}{2} \zX^{2}}
  \left(
    1
    -
    \frac{1}{\sqrt{2\pi}}
    \int_{\uX=-\betaX \zX}^{\uX=\betaX \zX}
    e^{-\frac{1}{2} \uX^{2}}
    d\uX
  \right)
  d\zX
  \\
  &=
  \frac{1}{\sqrt{2\pi}}
  \int_{\zX=0}^{\zX=\infty}
  e^{-\frac{1}{2} \zX^{2}}
  d\zX
  -
  \frac{1}{\sqrt{2\pi}}
  \int_{\zX=0}^{\zX=\infty}
  e^{-\frac{1}{2} \zX^{2}}
  \frac{1}{\sqrt{2\pi}}
  \int_{\uX=-\betaX \zX}^{\uX=\betaX \zX}
  e^{-\frac{1}{2} \uX^{2}}
  d\uX
  d\zX
  \\
  &=
  \frac{1}{2}
  -
  \frac{1}{\pi}
  \arctan( \betaX )
  \\
  &=
  \frac{1}{\pi}
  \arctan \left( \frac{1}{\betaX} \right)
,\end{align*}
%|<<|===========================================================================================|<<|
where the last equality holds for \(  \betaX > 0  \).
Towards deriving a closed form expression for \(  \gammaX  \), define \(  \zetaX = 1 - \sqrt{\frac{\pi}{2}} \gammaX  \).
Then, \(  \zetaX  \) is similarly obtained as follows:
%|>>|===========================================================================================|>>|
\begin{align*}
  \zetaX
  &=
  \int_{\zX=0}^{\zX=\infty}
  \zX
  e^{-\frac{1}{2} \zX^{2}}
  ( 1 - \pFn( \zX ) + \pFn( -\zX ) )
  d\zX
  \\
  &=
  2
  \int_{\zX=0}^{\zX=\infty}
  \zX
  e^{-\frac{1}{2} \zX^{2}}
  \pFn( -\zX )
  d\zX
  \\
  &=
  2
  \int_{\zX=0}^{\zX=\infty}
  \zX
  e^{-\frac{1}{2} \zX^{2}}
  \frac{1}{\sqrt{2\pi}}
  \int_{\uX=-\infty}^{\uX=-\betaX \zX}
  e^{-\frac{1}{2} \uX^{2}}
  d\uX
  d\zX
  \\
  &=
  2
  \int_{\zX=0}^{\zX=\infty}
  \zX
  e^{-\frac{1}{2} \zX^{2}}
  \left(
    \frac{1}{\sqrt{2\pi}}
    \int_{\uX=-\infty}^{\uX=0}
    e^{-\frac{1}{2} \uX^{2}}
    d\uX
    -
    \frac{1}{\sqrt{2\pi}}
    \int_{\uX=-\betaX \zX}^{\uX=0}
    e^{-\frac{1}{2} \uX^{2}}
    d\uX
  \right)
  d\zX
  \\
  &=
  2
  \int_{\zX=0}^{\zX=\infty}
  \zX
  e^{-\frac{1}{2} \zX^{2}}
  \left(
    \frac{1}{2}
    -
    \frac{1}{\sqrt{2\pi}}
    \int_{\uX=-\betaX \zX}^{\uX=0}
    e^{-\frac{1}{2} \uX^{2}}
    d\uX
  \right)
  d\zX
  % \\
  % &=
  % 2
  % \int_{\zX=0}^{\zX=\infty}
  % \zX
  % e^{-\frac{1}{2} \zX^{2}}
  % \left(
  %   \frac{1}{2}
  %   -
  %   \frac{1}{2}
  %   \frac{1}{\sqrt{2\pi}}
  %   \int_{\uX=-\betaX \zX}^{\uX=\betaX \zX}
  %   e^{-\frac{1}{2} \uX^{2}}
  %   d\uX
  % \right)
  % d\zX
  % \\
  % &=
  % 2
  % \int_{\zX=0}^{\zX=\infty}
  % \zX
  % e^{-\frac{1}{2} \zX^{2}}
  % \frac{1}{2}
  % \left(
  %   1
  %   -
  %   \frac{1}{\sqrt{2\pi}}
  %   \int_{\uX=-\betaX \zX}^{\uX=\betaX \zX}
  %   e^{-\frac{1}{2} \uX^{2}}
  %   d\uX
  % \right)
  % d\zX
  \\
  &=
  \int_{\zX=0}^{\zX=\infty}
  \zX
  e^{-\frac{1}{2} \zX^{2}}
  \left(
    1
    -
    \frac{1}{\sqrt{2\pi}}
    \int_{\uX=-\betaX \zX}^{\uX=\betaX \zX}
    e^{-\frac{1}{2} \uX^{2}}
    d\uX
  \right)
  d\zX
  \\
  &=
  \int_{\zX=0}^{\zX=\infty}
  \zX
  e^{-\frac{1}{2} \zX^{2}}
  d\zX
  -
  \int_{\zX=0}^{\zX=\infty}
  \zX
  e^{-\frac{1}{2} \zX^{2}}
  \frac{1}{\sqrt{2\pi}}
  \int_{\uX=-\betaX \zX}^{\uX=\betaX \zX}
  e^{-\frac{1}{2} \uX^{2}}
  d\uX
  d\zX
  \\
  &=
  1
  -
  \sin( \arctan( \betaX ) )
  \\
  &=
  1
  -
  \frac{\betaX}{\sqrt{\betaX^{2} + 1}}
.\end{align*}
%|<<|===========================================================================================|<<|
It follows that
%|>>|===========================================================================================|>>|
\(
  1-\zetaX = \frac{\betaX}{\sqrt{\betaX^{2} + 1}}
,\)
%|<<|===========================================================================================|<<|
and hence,
%|>>|===========================================================================================|>>|
\(
  \gammaX = \frac{\sqrt{2/\pi}\betaX}{\sqrt{\betaX^{2} + 1}}
.\)
%|<<|===========================================================================================|<<|
% and hence,
Thus,
%|>>|===========================================================================================|>>|
\begin{gather*}
  \frac{1}{\GAMMAX}
  =
  \sqrt{\frac{\pi}{2} \left( 1 + \frac{1}{\betaX^{2}} \right)}
  =
  \begin{cases}
  \displaystyle
  \BigO( \frac{1}{\betaX} ) ,&\cIf \betaX \in (0,\betaXThresholdPROne),\\
  \displaystyle
  \BigO( 1 )                ,&\cIf \betaX \in [\betaXThresholdPROne, \infty) ,\\
  \end{cases}
  ,\\
  \frac{1}{\GAMMAX^{2}}
  =
  \frac{\pi}{2}
  \left( 1 + \frac{1}{\betaX^{2}} \right)
  =
  \begin{cases}
  \displaystyle
  \BigO( \frac{1}{\betaX^{2}} ) ,&\cIf \betaX \in (0,\betaXThresholdPROne) ,\\
  \displaystyle
  \BigO( 1 )                    ,&\cIf \betaX \in [\betaXThresholdPROne, \infty) ,
  \end{cases}
\end{gather*}
%|<<|===========================================================================================|<<|
where
%|>>|:::::::::::::::::::::::::::::::::::::::::::::::::::::::::::::::::::::::::::::::::::::::::::|>>|
\(  \ConstbetaXThresholdPROne \defeq \ConstbetaXThresholdPROneValue  \).
%|<<|:::::::::::::::::::::::::::::::::::::::::::::::::::::::::::::::::::::::::::::::::::::::::::|<<|
%As these expressions for \(  \alphaX  \) and \(  \gammaX  \) are exact and hold for all \(  \betaX > 0  \), their insertion into the sample complexity in \EQUATION \eqref{eqn:main-technical:sparse:m} of \THEOREM \ref{thm:main-technical:sparse} are left to the reader.
%However, the asymptotic expressions for the sample complexity stated in \COROLLARIES \ref{corollary:main-technical:probit} will be briefly justified.
Additionally, notice that
%|>>|===========================================================================================|>>|
\(
  \alphaX
  =
  \frac{1}{\pi}
  \arctan \left( \frac{1}{\betaX} \right)
  \leq
  \min \left\{ \frac{1}{2}, \frac{1}{\pi \betaX} \right\}
,\)
%|<<|===========================================================================================|<<|
and therefore,
%|>>|===========================================================================================|>>|
\begin{align*}
  \alphaO
  \leq
  \max \left\{ \min \left\{ \frac{1}{2}, \frac{1}{\pi \betaX} \right\}, \deltaX \right\}
  =
  \begin{cases}
  \displaystyle
  \BigO( 1 )                ,&\cIf \betaX \in (0,\betaXThresholdPROne) ,\\
  %\displaystyle
  \BigO( \frac{1}{\betaX} ) ,&\cIf \betaX \in [\betaXThresholdPROne, \betaXThresholdPRTwo) ,\\
  %\displaystyle
  \BigO( \epsilonX )          ,&\cIf \betaX \in [\betaXThresholdPRTwo, \infty) ,
  \end{cases}
\end{align*}
%|<<|===========================================================================================|<<|
where
%|>>|:::::::::::::::::::::::::::::::::::::::::::::::::::::::::::::::::::::::::::::::::::::::::::|>>|
\(  \ConstbetaXThresholdPRTwo \defeq \ConstbetaXThresholdPRTwoValue  \).
%|<<|:::::::::::::::::::::::::::::::::::::::::::::::::::::::::::::::::::::::::::::::::::::::::::|<<|
Incorporating these expressions for \(  \alphaO  \) and \(  \gammaX  \) into the sample complexity in \EQUATION \eqref{eqn:main-technical:sparse:m} of \THEOREM \ref{thm:main-technical:sparse}, and into the definitions of \(  \nuX=\nuX( \deltaX )  \) and \(  \tauX=\tauX( \deltaX )  \),
%in \EQUATIONS \eqref{eqn:main-technical:sparse:nu} and \eqref{eqn:main-technical:sparse:tau},
the corollary's bound holds due to \THEOREM \ref{thm:main-technical:sparse}:
%|>>|===========================================================================================|>>|
\begin{gather*}
  \left\|
    \thetaStar
    -
    \frac
    {\thetaXX + \hfFn[\JCoords]( \thetaStar, \thetaXX )}
    {\| \thetaXX + \hfFn[\JCoords]( \thetaStar, \thetaXX ) \|_{2}}
  \right\|_{2}
  \leq
  \sqrt{\deltaX \| \thetaStar-\thetaXX \|_{2}}
  +
  \deltaX
,\end{gather*}
%|<<|===========================================================================================|<<|
for every \(  \thetaXX \in \ParamSpace  \) and every \(  \JCoords \subseteq [\n]  \), \(  | \JCoords | \leq \k  \), uniformly with probability at least \(  1-\rho  \) as long as
\checkthis%
%|>>|===========================================================================================|>>|
\begin{align*}
  \m
  &\geq
  \max \!\!\begin{array}[t]{l} \displaystyle \Biggl\{
    \frac{108\pi^{3} ( 1+\frac{1}{\betaX^{2}} )}{\deltaX}
    \log \left( \frac{24}{\rhoX} \JSPCSIZES \right)
    ,\\ \displaystyle \phantom{\Bigg\{}
    \frac{48\pi^{2} \alphaO ( 1+\frac{1}{\betaX^{2}} )}{\ConstbLD^{2} \deltaX^{2}}
    \log \left( \frac{24}{\rhoX} \JSPCSIZES \right)
    ,\\ \displaystyle \phantom{\Bigg\{}
    \frac{200 \Constb \sqrt{\frac{\pi}{2} \left( 1+\frac{1}{\betaX^{2}} \right)}}{\Constc^{2} \deltaX}
    \frac{\log \left( \frac{24}{\rhoX} \JSPCSIZES \right)}{\sqrt{\log \left( \frac{4e}{\nuX} \right)}}
    ,\\ \displaystyle \phantom{\Bigg\{}
    \frac{64}{\Constb \deltaX} \sqrt{\frac{\pi}{2}  \left(1 + \frac{1}{\betaX^{2}} \right) \log \left( \frac{4e}{\nuX} \right)} \log \left( \frac{24}{\rhoX} \right)
    ,\\ \displaystyle \phantom{\Bigg\{}
    \frac{\ConstdSD \k}{\Constb \deltaX} \sqrt{\frac{\pi}{2} \left(1 + \frac{1}{\betaX^{2}} \right) \log \left( \frac{4e}{\nuX} \right)} \log \left( \frac{1}{\nuX} \right)
  \Biggr\}  \end{array}
  \\
  &=
  \begin{cases}
  \displaystyle
  \BigO'( \frac{\k}{\betaX^{2} \epsilonX^{2}} ) ,&\cIf \betaX \in (0,\betaXThresholdPROne) ,\\
  \displaystyle
  \BigO'( \frac{\k}{\betaX \epsilonX^{2}} )     ,&\cIf \betaX \in [\betaXThresholdPROne,\betaXThresholdPRTwo) ,\\
  \displaystyle
  \BigO'( \frac{\k}{\epsilonX} )     ,&\cIf \betaX \in [\betaXThresholdPRTwo, \infty) ,
  \end{cases}
\end{align*}
%|<<|===========================================================================================|<<|
where
%|>>|===========================================================================================|>>|
\(
  \nuX = \nuX( \deltaX ) = \nuXEXPRPR
.\)
%|<<|===========================================================================================|<<|
\begin{comment}
\BEGINGRAYOUT
It will turn out that these are upper bounded by their counterparts under the logistic model.
For clarity, some notations are introduced to differentiate between the functions and parameters in these two models:
%|>>|===========================================================================================|>>|
\begin{gather*}
  \pFnL( \zX )
  \defeq
  \frac{1}{1+e^{-\betaX \zX}}
  ,\\
  \pFnP( \zX )
  \defeq
  \frac{1}{\sqrt{2\pi}}
  \int_{\uX=-\infty}^{\uX=\betaX \zX}
  e^{-\frac{1}{2} \uX^{2}}
  d\uX
  ,\\
  \alphaXL
  \defeq
  \frac{1}{\sqrt{2\pi}}
  \int_{\zX=0}^{\zX=\infty}
  e^{-\frac{1}{2} \zX^{2}}
  ( 1 - \pFnL( \zX ) + \pFnL( -\zX ) )
  d\zX
  ,\\
  \alphaXP
  \defeq
  \frac{1}{\sqrt{2\pi}}
  \int_{\zX=0}^{\zX=\infty}
  e^{-\frac{1}{2} \zX^{2}}
  ( 1 - \pFnP( \zX ) + \pFnP( -\zX ) )
  d\zX
  ,\\
  \gammaXL
  \defeq
  \int_{\zX=0}^{\zX=\infty}
  \zX
  e^{-\frac{1}{2} \zX^{2}}
  ( 1 - \pFnL( \zX ) + \pFnL( -\zX ) )
  d\zX
  ,\\
  \gammaXP
  \defeq
  \int_{\zX=0}^{\zX=\infty}
  \zX
  e^{-\frac{1}{2} \zX^{2}}
  ( 1 - \pFnP( \zX ) + \pFnP( -\zX ) )
  d\zX
.\end{gather*}
%|<<|===========================================================================================|<<|
Note that the analysis for this step returns to taking the parameter \(  \betaX \GTR 0  \) as arbitrary.
The goal will be to show that
%|>>|:::::::::::::::::::::::::::::::::::::::::::::::::::::::::::::::::::::::::::::::::::::::::::|>>|
\(  \alphaXP \leq \alphaXL  \) and \(  \gammaXP \leq \gammaXL  \)
%|<<|:::::::::::::::::::::::::::::::::::::::::::::::::::::::::::::::::::::::::::::::::::::::::::|<<|
for all
%|>>|:::::::::::::::::::::::::::::::::::::::::::::::::::::::::::::::::::::::::::::::::::::::::::|>>|
\(  \betaX \GTR 0  \),
%|<<|:::::::::::::::::::::::::::::::::::::::::::::::::::::::::::::::::::::::::::::::::::::::::::|<<|
from which the desired bounds on \(  \alphaXP  \) and \(  \gammaXP  \) will follow by applying the bounds on their counterparts in \COROLLARY \ref{corollary:main-technical:logistic-regression} and its proof.
To start off, recall from \EQUATIONS \eqref{eqn:pf:corollary:main-technical:logistic-regression:7} and \eqref{eqn:pf:corollary:main-technical:probit:6} that
%|>>|===========================================================================================|>>|
\begin{gather*}
  1-\pFnL( \zX ) = \pFnL( -\zX )
  ,\\
  1-\pFnP( \zX ) = \pFnP( -\zX )
,\end{gather*}
%|<<|===========================================================================================|<<|
and therefore,
%|>>|===========================================================================================|>>|
\begin{gather}
\label{eqn:pf:corollary:main-technical:probit:8:1}
  \alphaXL
  =
  \sqrt{\frac{2}{\pi}}
  \int_{\zX=0}^{\zX=\infty}
  e^{-\frac{1}{2} \zX^{2}}
  \pFnL( -\zX )
  d\zX
  ,\\
\label{eqn:pf:corollary:main-technical:probit:8:2}
  \alphaXP
  =
  \sqrt{\frac{2}{\pi}}
  \int_{\zX=0}^{\zX=\infty}
  e^{-\frac{1}{2} \zX^{2}}
  \pFnP( -\zX )
  d\zX
  ,\\
\label{eqn:pf:corollary:main-technical:probit:8:3}
  \gammaXL
  =
  2
  \int_{\zX=0}^{\zX=\infty}
  \zX
  e^{-\frac{1}{2} \zX^{2}}
  \pFnL( -\zX )
  d\zX
  ,\\
\label{eqn:pf:corollary:main-technical:probit:8:4}
  \gammaXP
  =
  2
  \int_{\zX=0}^{\zX=\infty}
  \zX
  e^{-\frac{1}{2} \zX^{2}}
  \pFnP( -\zX )
  d\zX
.\end{gather}
%|<<|===========================================================================================|<<|
Clearly, the integrand in each of the four integrals in \EQUATIONS \eqref{eqn:pf:corollary:main-technical:probit:8:1}--\eqref{eqn:pf:corollary:main-technical:probit:8:4} is nonnegative at every \(  \zX \geq 0  \).
Therefore, it suffices to show that
%|>>|:::::::::::::::::::::::::::::::::::::::::::::::::::::::::::::::::::::::::::::::::::::::::::|>>|
\(  \pFnP( -\zX ) \leq \pFnL( -\zX )  \)
%|<<|:::::::::::::::::::::::::::::::::::::::::::::::::::::::::::::::::::::::::::::::::::::::::::|<<|
for all \(  \zX \geq 0  \) since this, together with the stated nonnegativity, will imply that
%|>>|:::::::::::::::::::::::::::::::::::::::::::::::::::::::::::::::::::::::::::::::::::::::::::|>>|
\(  \alphaXP \leq \alphaXL  \) and \(  \gammaXP \leq \gammaXL  \).
%|<<|:::::::::::::::::::::::::::::::::::::::::::::::::::::::::::::::::::::::::::::::::::::::::::|<<|
For ease of reference later, this comparison of \(  \pFnP  \) and \(  \pFnL  \) is formalized in the following claim.
%
%|>>|*******************************************************************************************|>>|
%|>>|*******************************************************************************************|>>|
%|>>|*******************************************************************************************|>>|
\begin{claim}
\label{claim:pf:corollary:main-technical:probit:p_P<p_L}
%
Let \(  \zX \geq 0  \).Then,
%|>>|:::::::::::::::::::::::::::::::::::::::::::::::::::::::::::::::::::::::::::::::::::::::::::|>>|
\(  \pFnP( -\zX ) \leq \pFnL( -\zX )  \).
%|<<|:::::::::::::::::::::::::::::::::::::::::::::::::::::::::::::::::::::::::::::::::::::::::::|<<|
\end{claim}
%|<<|*******************************************************************************************|<<|
%|<<|*******************************************************************************************|<<|
%|<<|*******************************************************************************************|<<|
%
The proof of \CLAIM \ref{claim:pf:corollary:main-technical:probit:p_P<p_L} is deferred momentarily so that, contingent on the verification of the claim, the proof of \COROLLARY \ref{corollary:main-technical:probit} can first be deduced.
Due to the assumed correctness of \CLAIM \ref{claim:pf:corollary:main-technical:probit:p_P<p_L} and the earlier remarks,
%|>>|:::::::::::::::::::::::::::::::::::::::::::::::::::::::::::::::::::::::::::::::::::::::::::|>>|
\(  \alphaXP \leq \alphaXL  \) and \(  \gammaXP \leq \gammaXL  \),
%|<<|:::::::::::::::::::::::::::::::::::::::::::::::::::::::::::::::::::::::::::::::::::::::::::|<<|
and therefore, recalling \EQUATIONS \eqref{eqn:pf:corollary:main-technical:logistic-regression:5:a}--\eqref{eqn:pf:corollary:main-technical:logistic-regression:6} from the proof of \COROLLARY \ref{corollary:main-technical:logistic-regression},
%|>>|===========================================================================================|>>|
\begin{gather}
\label{eqn:pf:corollary:main-technical:probit:9:a}
  1-\gammaXP
  \geq
  1-\gammaXL
  \geq
  \begin{cases}
  \left( 1 - \frac{\betaX^{2}}{6} \right) \frac{\betaX}{\sqrt{2\pi}}
  ,& \cIf \betaX \in (0, \betaXThresholdLROne) ,\\
  1 - \sqrt{\frac{2}{\pi}} \frac{2}{\betaX}
  ,& \cIf \betaX \in [\betaXThresholdLROne, \betaXThresholdLRTwo) ,\\
  1 - \sqrt{\frac{2}{\pi}} \frac{2}{\betaX}
  ,& \cIf \betaX \in [\betaXThresholdLRTwo, \infty) ,
  \end{cases}
  \geq
  \begin{cases}
  \left( 1 - \frac{\cO^{2}}{6} \right) \frac{\betaX}{\sqrt{2\pi}}
  ,& \cIf \betaX \in (0, \betaXThresholdLROne) ,\\
  \bO
  ,& \cIf \betaX \in [\betaXThresholdLROne, \betaXThresholdLRTwo) ,\\
  \bO
  ,& \cIf \betaX \in [\betaXThresholdLRTwo, \infty) ,
  \end{cases}
  \\
\label{eqn:pf:corollary:main-technical:probit:9:b}
  (1-\gammaXP)^{2}
  \geq
  (1-\gammaXL)^{2}
  \geq
  \begin{cases}
  \left( 1 - \frac{\betaX^{2}}{6} \right)^{2} \frac{\betaX^{2}}{2\pi}
  ,& \cIf \betaX \in (0, \betaXThresholdLROne) ,\\
  \left( 1 - \sqrt{\frac{2}{\pi}} \frac{2}{\betaX} \right)^{2}
  ,& \cIf \betaX \in [\betaXThresholdLROne, \betaXThresholdLRTwo) ,\\
  \left( 1 - \sqrt{\frac{2}{\pi}} \frac{2}{\betaX} \right)^{2}
  ,& \cIf \betaX \in [\betaXThresholdLRTwo, \infty) ,
  \end{cases}
  \geq
  \begin{cases}
  \left( 1 - \frac{\cO^{2}}{6} \right)^{2} \frac{\betaX^{2}}{2\pi}
  ,& \cIf \betaX \in (0, \betaXThresholdLROne) ,\\
  \bO^{2}
  ,& \cIf \betaX \in [\betaXThresholdLROne, \betaXThresholdLRTwo) ,\\
  \bO^{2}
  ,& \cIf \betaX \in [\betaXThresholdLRTwo, \infty) ,
  \end{cases}
  \\
\label{eqn:pf:corollary:main-technical:probit:9:c}
  \frac{\alphaXP}{(1-\gammaXP)^{2}}
  \leq
  \frac{\alphaXL}{(1-\gammaXL)^{2}}
  \leq
  \begin{cases}
  \frac{\pi}{\bigl( 1 - \frac{\betaX^{2}}{6} \bigr)^{2} \betaX^{2}}
  ,& \cIf \betaX \in (0, \betaXThresholdLROne) ,\\
  \frac{\sqrt{\hfrac{2}{\pi}}}{\bigl( 1 - \sqrt{\frac{2}{\pi}} \frac{2}{\betaX} \bigr)^{2} \betaX}
  ,& \cIf \betaX \in [\betaXThresholdLROne, \betaXThresholdLRTwo) ,\\
  \frac{\sqrt{\hfrac{2}{\pi}}}{\bigl( 1 - \sqrt{\frac{2}{\pi}} \frac{2}{\betaX} \bigr)^{2} \betaX}
  ,& \cIf \betaX \in [\betaXThresholdLRTwo, \infty) ,
  \end{cases}
  \leq
  \begin{cases}
  \frac{\pi}{\bigl( 1 - \frac{\cO^{2}}{6} \bigr)^{2} \betaX^{2}}
  ,& \cIf \betaX \in (0, \betaXThresholdLROne) ,\\
  \frac{\sqrt{\hfrac{2}{\pi}}}{\bO^{2} \betaX}
  ,& \cIf \betaX \in [\betaXThresholdLROne, \betaXThresholdLRTwo) ,\\
  \frac{\sqrt{\hfrac{2}{\pi}} \deltaX}{\bO^{2}}
  ,& \cIf \betaX \in [\betaXThresholdLRTwo, \infty) ,
  \end{cases}
\end{gather}
%|<<|===========================================================================================|<<|
where
%|>>|:::::::::::::::::::::::::::::::::::::::::::::::::::::::::::::::::::::::::::::::::::::::::::|>>|
\(  \bO, \cO > 0  \)
%|<<|:::::::::::::::::::::::::::::::::::::::::::::::::::::::::::::::::::::::::::::::::::::::::::|<<|
are constants which satisfy
%|>>|:::::::::::::::::::::::::::::::::::::::::::::::::::::::::::::::::::::::::::::::::::::::::::|>>|
\(  \cO = \frac{\sqrt{\hfrac{8}{\pi}}}{1+\bO} \geq 1  \).
%|<<|:::::::::::::::::::::::::::::::::::::::::::::::::::::::::::::::::::::::::::::::::::::::::::|<<|
%
%%%%%%%%%%%%%%%%%%%%%%%%%%%%%%%%%%%%%%%%%%%%%%%%%%%%%%%%%%%%%%%%%%%%%%%%%%%%%%%%%%%%%%%%%%%%%%%%%%%%
\par %%%%%%%%%%%%%%%%%%%%%%%%%%%%%%%%%%%%%%%%%%%%%%%%%%%%%%%%%%%%%%%%%%%%%%%%%%%%%%%%%%%%%%%%%%%%%%%
%%%%%%%%%%%%%%%%%%%%%%%%%%%%%%%%%%%%%%%%%%%%%%%%%%%%%%%%%%%%%%%%%%%%%%%%%%%%%%%%%%%%%%%%%%%%%%%%%%%%
%
Having obtained bounds for \(  \alphaXP  \) and \(  \gammaXP  \), we will return to the notations of
%|>>|:::::::::::::::::::::::::::::::::::::::::::::::::::::::::::::::::::::::::::::::::::::::::::|>>|
\(  \alphaX = \alphaXP  \) and \(  \gammaX = \gammaXP  \)
%|<<|:::::::::::::::::::::::::::::::::::::::::::::::::::::::::::::::::::::::::::::::::::::::::::|<<|
for the rest of the proof (until the deferred proof of \CLAIM \ref{claim:pf:corollary:main-technical:probit:p_P<p_L}).
%By the same arguments as in the proof of \COROLLARY \ref{corollary:main-technical:logistic-regression}, it follows that if
Due to \EQUATIONS \eqref{eqn:pf:corollary:main-technical:probit:9:a}\nobreakdash--\eqref{eqn:pf:corollary:main-technical:probit:9:c}, if
%|>>|===========================================================================================|>>|
\begin{align}
\label{eqn:pf:corollary:main-technical:probit:10}
  \m
  &\geq
  \max \{ \mOneS, \mTwoS, \mThreeS, \mFourS, \mFiveS \}
%  \\
%  &\geq
%  \mEXPR[s]
%  \\
  =
  \mOEXPRLR[s]{\epsilonX}[,]
\end{align}
%|<<|===========================================================================================|<<|
where %due to \EQUATIONS \eqref{eqn:pf:corollary:main-technical:probit:9:a}--\eqref{eqn:pf:corollary:main-technical:probit:9:c},
%|>>|===========================================================================================|>>|
\begin{gather*}
  \mOneS = \mOneEXPR[s]
  ,\\
  \mTwoS = \mTwoEXPR[s]
  ,\\
  \mThreeS = \mThreeEXPR[s]
  ,\\
  \mFourS = \mFourEXPR[s]
  ,\\
  \mFiveS = \mFiveEXPR[s]
,\end{gather*}
%|<<|===========================================================================================|<<|
and where %due to \EQUATION \eqref{eqn:pf:corollary:main-technical:probit:9:a},
%|>>|===========================================================================================|>>|
\begin{gather*}
  \nuX
  =
  \nuXEXPR
  \geq
  \begin{cases}
  \nuXEXPRLROne
  ,& \cIf \betaX \in (0, \betaXThresholdLROne) ,\\
  \nuXEXPRLRTwo
  ,& \cIf \betaX \in [\betaXThresholdLROne, \betaXThresholdLRTwo) ,\\
  \nuXEXPRLRThree
  ,& \cIf \betaX \in [\betaXThresholdLRTwo, \infty) ,
  \end{cases}
\end{gather*}
%|<<|===========================================================================================|<<|
then,
%|>>|===========================================================================================|>>|
\begin{gather*}
  \m \geq \mEXPR[s][.]
\end{gather*}
%|<<|===========================================================================================|<<|
Therefore, taking \(  \m  \) to satisfy \EQUATION \eqref{eqn:pf:corollary:main-technical:probit:10}, by \THEOREM \ref{thm:main-technical:sparse},
%|>>|===========================================================================================|>>|
\begin{gather*}
  \left\|
    \thetaStar
    -
    \frac
    {\thetaXX + \hfFn[\JCoords]( \thetaStar, \thetaXX )}
    {\| \thetaXX + \hfFn[\JCoords]( \thetaStar, \thetaXX ) \|_{2}}
  \right\|_{2}
  \leq
  \sqrt{\deltaX \| \thetaStar-\thetaXX \|_{2}}
  +
  \deltaX
,\end{gather*}
%|<<|===========================================================================================|<<|
for every \(  \thetaXX \in \ParamSpace  \) and every \(  \JCoords \subseteq [\n]  \), \(  | \JCoords | \leq \k  \), uniformly with probability at least \(  1-\rho  \).
As long as \CLAIM \ref{claim:pf:corollary:main-technical:probit:p_P<p_L} holds, this completes the proof of \COROLLARY \ref{corollary:main-technical:probit}.
The final step in the corollary's proof is establishing \CLAIM \ref{claim:pf:corollary:main-technical:probit:p_P<p_L}.
%
%|>>|~~~~~~~~~~~~~~~~~~~~~~~~~~~~~~~~~~~~~~~~~~~~~~~~~~~~~~~~~~~~~~~~~~~~~~~~~~~~~~~~~~~~~~~~~~~|>>|
%|>>|~~~~~~~~~~~~~~~~~~~~~~~~~~~~~~~~~~~~~~~~~~~~~~~~~~~~~~~~~~~~~~~~~~~~~~~~~~~~~~~~~~~~~~~~~~~|>>|
%|>>|~~~~~~~~~~~~~~~~~~~~~~~~~~~~~~~~~~~~~~~~~~~~~~~~~~~~~~~~~~~~~~~~~~~~~~~~~~~~~~~~~~~~~~~~~~~|>>|
\begin{proof}
{\CLAIM \ref{claim:pf:corollary:main-technical:probit:p_P<p_L}}
%
The case when \(  \betaX = 0  \) is trivial since
%|>>|:::::::::::::::::::::::::::::::::::::::::::::::::::::::::::::::::::::::::::::::::::::::::::|>>|
\(  \pFnP( \zX ) = \pFnL( \zX ) = \frac{1}{2}  \)
%|<<|:::::::::::::::::::::::::::::::::::::::::::::::::::::::::::::::::::::::::::::::::::::::::::|<<|
for all \(  \zX \in \R  \).
Otherwise, for \(  \betaX > 0  \), it suffices to prove the claim when the \betaXnamepr fixed at \(  \betaX = 1  \), as will be done throughout the remainder of this claim's proof.
Observe that, using the chain rule,
%|>>|===========================================================================================|>>|
\begin{gather*}
  \frac{d}{d\uX} \frac{1}{1+e^{-\uX}}
  =
  \frac{e^{-\uX}}{( 1+e^{-\uX} )^{2}}
  =
  \frac{1}{( 1+e^{-\uX} ) ( 1+e^{\uX} )}
\end{gather*}
%|<<|===========================================================================================|<<|
and that
%|>>|===========================================================================================|>>|
\begin{gather*}
  \lim_{\uX \to -\infty}
  %\frac{d}{d\uX}
  \frac{1}{1+e^{-\uX}}
  %=
  %\lim_{\uX \to -\infty}
  %\frac{1}{( 1+e^{-\uX} ) ( 1+e^{\uX} )}
  =
  0
.\end{gather*}
%|<<|===========================================================================================|<<|
Thus, for \(  \zX \in \R  \), and with \(  \betaX = 1  \),
%|>>|===========================================================================================|>>|
\begin{gather}
\label{eqn:pf:claim:pf:corollary:main-technical:probit:p_P<p_L:1}
  \pFnL( \zX )
  =
  \frac{1}{1+e^{-\zX}}
  =
  \int_{\uX=-\infty}^{\uX=\zX}
  \frac{1}{( 1+e^{-\uX} ) ( 1+e^{\uX} )}
  d\uX
.\end{gather}
%|<<|===========================================================================================|<<|
Additionally, recall that
%|>>|===========================================================================================|>>|
\begin{gather}
\label{eqn:pf:claim:pf:corollary:main-technical:probit:p_P<p_L:2}
  \pFnP( \zX )
  =
  \frac{1}{\sqrt{2\pi}}
  \int_{\uX=-\infty}^{\uX=\zX}
  e^{-\frac{1}{2} \uX^{2}}
  d\uX
.\end{gather}
%|<<|===========================================================================================|<<|
For convenience later on, write the integrands in \EQUATIONS \eqref{eqn:pf:claim:pf:corollary:main-technical:probit:p_P<p_L:1} and \eqref{eqn:pf:claim:pf:corollary:main-technical:probit:p_P<p_L:2} as the functions
%|>>|:::::::::::::::::::::::::::::::::::::::::::::::::::::::::::::::::::::::::::::::::::::::::::|>>|
\(  \integrandL, \integrandP : \R \to \R  \),
%|<<|:::::::::::::::::::::::::::::::::::::::::::::::::::::::::::::::::::::::::::::::::::::::::::|<<|
where
%|>>|===========================================================================================|>>|
\begin{gather}
\label{eqn:pf:claim:pf:corollary:main-technical:probit:p_P<p_L:3}
  \integrandL( \uX ) = \frac{1}{( 1+e^{-\uX} ) ( 1+e^{\uX} )}
  ,\\
\label{eqn:pf:claim:pf:corollary:main-technical:probit:p_P<p_L:4}
  \integrandP( \uX ) = \frac{1}{\sqrt{2\pi}} e^{-\frac{1}{2} \uX^{2}}
,\end{gather}
%|<<|===========================================================================================|<<|
for \(  \uX \in \R  \).
%
%%%%%%%%%%%%%%%%%%%%%%%%%%%%%%%%%%%%%%%%%%%%%%%%%%%%%%%%%%%%%%%%%%%%%%%%%%%%%%%%%%%%%%%%%%%%%%%%%%%%
\par %%%%%%%%%%%%%%%%%%%%%%%%%%%%%%%%%%%%%%%%%%%%%%%%%%%%%%%%%%%%%%%%%%%%%%%%%%%%%%%%%%%%%%%%%%%%%%%
%%%%%%%%%%%%%%%%%%%%%%%%%%%%%%%%%%%%%%%%%%%%%%%%%%%%%%%%%%%%%%%%%%%%%%%%%%%%%%%%%%%%%%%%%%%%%%%%%%%%
%
The argument will proceed with the goal of showing that
%|>>|:::::::::::::::::::::::::::::::::::::::::::::::::::::::::::::::::::::::::::::::::::::::::::|>>|
\(  \pFnP( -\zX ) \leq \pFnL( -\zX )  \)
%|<<|:::::::::::::::::::::::::::::::::::::::::::::::::::::::::::::::::::::::::::::::::::::::::::|<<|
for any
%|>>|:::::::::::::::::::::::::::::::::::::::::::::::::::::::::::::::::::::::::::::::::::::::::::|>>|
\(  \zX \geq 0  \).
%|<<|:::::::::::::::::::::::::::::::::::::::::::::::::::::::::::::::::::::::::::::::::::::::::::|<<|
Observe that for \(  \uX \leq 0  \),
%|>>|===========================================================================================|>>|
\begin{gather*}
  \frac{1}{( 1+e^{-\uX} ) ( 1+e^{\uX} )}
  =
  \frac{1}{( 1+e^{\uX} )^{2} e^{-\uX}}
  \geq
  \frac{1}{( 1+e^{0} )^{2} e^{-\uX}}
  =
  \frac{1}{4 e^{-\uX}}
.\end{gather*}
%|<<|===========================================================================================|<<|
Therefore, for \(  \uX \leq 0  \),
%|>>|===========================================================================================|>>|
\begin{gather*}
  \frac{\integrandL( \uX )}{\integrandP( \uX )}
  =
  \frac{\sqrt{2\pi} e^{\frac{1}{2} \uX^{2}}}{( 1+e^{-\uX} ) ( 1+e^{\uX} )}
  \geq
  \frac{\sqrt{2\pi} e^{\frac{1}{2} \uX^{2}}}{4 e^{-\uX}}
  =
  \sqrt{\frac{\pi}{8}} e^{\frac{1}{2} \uX^{2} + \uX}
,\end{gather*}
%|<<|===========================================================================================|<<|
where there exists some
%|>>|:::::::::::::::::::::::::::::::::::::::::::::::::::::::::::::::::::::::::::::::::::::::::::|>>|
\(  \zO \geq 0  \)
%|<<|:::::::::::::::::::::::::::::::::::::::::::::::::::::::::::::::::::::::::::::::::::::::::::|<<|
such that for every \(  \uX \leq -\zO  \),
%|>>|===========================================================================================|>>|
%|>>|===========================================================================================|>>|
\begin{gather*}
  \frac{\integrandL( \uX )}{\integrandP( \uX )}
  =
  \frac{\sqrt{2\pi} e^{\frac{1}{2} \uX^{2}}}{( 1+e^{-\uX} ) ( 1+e^{\uX} )}
  \geq
  \sqrt{\frac{\pi}{8}} e^{\frac{1}{2} \uX^{2} + \uX}
  \geq
  1
\end{gather*}
%|<<|===========================================================================================|<<|
and hence also such that
%%|>>|===========================================================================================|>>|
\begin{gather*}
  \integrandP( \uX )
  =
  \frac{1}{\sqrt{2\pi}}
  e^{-\frac{1}{2} \uX^{2}}
  \leq
  \frac{1}{( 1+e^{-\uX} ) ( 1+e^{\uX} )}
  =
  \integrandL( \uX )
\end{gather*}
%|<<|===========================================================================================|<<|
for all \(  \uX \leq -\zO  \).
%%|>>|===========================================================================================|>>|
%\begin{gather*}
%  \lim_{\uX \to -\infty}
%  \frac{\sqrt{2\pi} e^{\frac{1}{2} \uX^{2}}}{( 1+e^{-\uX} ) ( 1+e^{\uX} )}
%  \geq
%  \lim_{\uX \to -\infty}
%  \frac{\sqrt{2\pi} e^{\frac{1}{2} \uX^{2}}}{4 e^{-\uX}}
%  =
%  \lim_{\uX \to -\infty}
%  \sqrt{\frac{\pi}{8}} e^{\frac{1}{2} \uX^{2} + \uX}
%  \geq
%  1
%,\end{gather*}
%%|<<|===========================================================================================|<<|
%which implies that there exists some
%%|>>|:::::::::::::::::::::::::::::::::::::::::::::::::::::::::::::::::::::::::::::::::::::::::::|>>|
%\(  \zO \geq 0  \)
%%|<<|:::::::::::::::::::::::::::::::::::::::::::::::::::::::::::::::::::::::::::::::::::::::::::|<<|
%such that for all \(  \uX \leq -\zO  \),
%%|>>|===========================================================================================|>>|
%\begin{gather*}
%  \integrandP( \uX )
%  =
%  \frac{1}{\sqrt{2\pi}}
%  e^{-\frac{1}{2} \uX^{2}}
%  \leq
%  \frac{1}{( 1+e^{-\uX} ) ( 1+e^{\uX} )}
%  =
%  \integrandL( \uX )
%.\end{gather*}
%%|<<|===========================================================================================|<<|
%
%%%%%%%%%%%%%%%%%%%%%%%%%%%%%%%%%%%%%%%%%%%%%%%%%%%%%%%%%%%%%%%%%%%%%%%%%%%%%%%%%%%%%%%%%%%%%%%%%%%%
\par %%%%%%%%%%%%%%%%%%%%%%%%%%%%%%%%%%%%%%%%%%%%%%%%%%%%%%%%%%%%%%%%%%%%%%%%%%%%%%%%%%%%%%%%%%%%%%%
%%%%%%%%%%%%%%%%%%%%%%%%%%%%%%%%%%%%%%%%%%%%%%%%%%%%%%%%%%%%%%%%%%%%%%%%%%%%%%%%%%%%%%%%%%%%%%%%%%%%
%
This motivates the division of the analysis into two cases:
\Enum[{\label{enum:pf:claim:pf:corollary:main-technical:probit:p_P<p_L:1:i}}]{i}
when \(  \zX \geq \zO  \), and
\Enum[{\label{enum:pf:claim:pf:corollary:main-technical:probit:p_P<p_L:1:ii}}]{ii}
when \(  \zX \in [0,\zO)  \),
where \(  \zO \in [0,\infty)  \) is taken to be
%|>>|===========================================================================================|>>|
\begin{gather}
\label{eqn:pf:claim:pf:corollary:main-technical:probit:p_P<p_L:5}
  \zO
  \defeq
  \min_{\zX' \in [0,\infty)}
  \zX'
  \quad
  \SubjectTo
  \quad
%  \frac{1}{\sqrt{2\pi}}
%  e^{-\frac{1}{2} \uX^{2}}
%  \leq
%  \frac{1}{( 1+e^{-\uX} ) ( 1+e^{\uX} )}
  \integrandP( \uX ) \leq \integrandL( \uX )
  ,~
  \forall \uX \leq -\zX'
.\end{gather}
%|<<|===========================================================================================|<<|
(Note that, per the above definition in \EQUATION \eqref{eqn:pf:claim:pf:corollary:main-technical:probit:p_P<p_L:5}, \(  \zO \approx 1.3245  \), though this specification is not necessary for the proof as the argument will not require assumptions about or knowledge of the precise value.)
Since the integrands in \EQUATIONS \eqref{eqn:pf:claim:pf:corollary:main-technical:probit:p_P<p_L:1} and \eqref{eqn:pf:claim:pf:corollary:main-technical:probit:p_P<p_L:2} are nonnegative at all \(  \uX \in \R  \), in the former case, \ref{enum:pf:claim:pf:corollary:main-technical:probit:p_P<p_L:1:i}, the claim follows immediately from the property that
%|>>|:::::::::::::::::::::::::::::::::::::::::::::::::::::::::::::::::::::::::::::::::::::::::::|>>|
\(  \integrandP( \uX ) \leq \integrandL( \uX )  \)
%|<<|:::::::::::::::::::::::::::::::::::::::::::::::::::::::::::::::::::::::::::::::::::::::::::|<<|
for \(  \uX \leq -\zO  \):
%|>>|===========================================================================================|>>|
\begin{gather*}
  \pFnP( -\zX )
  =
%  \frac{1}{\sqrt{2\pi}}
%  \int_{\uX=-\infty}^{\uX=-\zX}
%  e^{-\frac{1}{2} \uX^{2}}
%  d\uX
  \int_{\uX=-\infty}^{\uX=-\zX}
  \integrandP( \uX )
  d\uX
  \leq
%  \int_{\uX=-\infty}^{\uX=-\zX}
%  \frac{1}{( 1+e^{-\uX} ) ( 1+e^{\uX} )}
%  d\uX
  \int_{\uX=-\infty}^{\uX=-\zX}
  \integrandL( \uX )
  d\uX
  =
  \frac{1}{1+e^{\zX}}
  =
  \pFnL( -\zX )
.\end{gather*}
%|<<|===========================================================================================|<<|
%
%%%%%%%%%%%%%%%%%%%%%%%%%%%%%%%%%%%%%%%%%%%%%%%%%%%%%%%%%%%%%%%%%%%%%%%%%%%%%%%%%%%%%%%%%%%%%%%%%%%%
\par %%%%%%%%%%%%%%%%%%%%%%%%%%%%%%%%%%%%%%%%%%%%%%%%%%%%%%%%%%%%%%%%%%%%%%%%%%%%%%%%%%%%%%%%%%%%%%%
%%%%%%%%%%%%%%%%%%%%%%%%%%%%%%%%%%%%%%%%%%%%%%%%%%%%%%%%%%%%%%%%%%%%%%%%%%%%%%%%%%%%%%%%%%%%%%%%%%%%
%
If \(  \zO = 0  \), then we are done since there is no \CASE \ref{enum:pf:claim:pf:corollary:main-technical:probit:p_P<p_L:1:ii}.
On the other hand, if \(  \zO > 0  \) (it is), the latter case, \ref{enum:pf:claim:pf:corollary:main-technical:probit:p_P<p_L:1:ii}, can be argued by showing that
%|>>|:::::::::::::::::::::::::::::::::::::::::::::::::::::::::::::::::::::::::::::::::::::::::::|>>|
\(  \sup_{\zX \in [0,\zO)} \pFnP( -\zX ) - \pFnL( -\zX ) = \pFnP( 0 ) - \pFnL( 0 ) = 0  \)
%|<<|:::::::::::::::::::::::::::::::::::::::::::::::::::::::::::::::::::::::::::::::::::::::::::|<<|
%(the relevance of which is justified later).
because this implies that
%|>>|:::::::::::::::::::::::::::::::::::::::::::::::::::::::::::::::::::::::::::::::::::::::::::|>>|
\(  \pFnP( -\zX ) - \pFnL( -\zX ) \leq 0  \)---or equivalently, \(  \pFnP( -\zX ) \leq \pFnL( -\zX )  \)---%
%|<<|:::::::::::::::::::::::::::::::::::::::::::::::::::::::::::::::::::::::::::::::::::::::::::|<<|
for all \(  \zX \in [0,\zO)  \).
This would indeed be true if
%|>>|:::::::::::::::::::::::::::::::::::::::::::::::::::::::::::::::::::::::::::::::::::::::::::|>>|
\(  \integrandP( \uX ) \geq \integrandL( \uX )  \)
%|<<|:::::::::::::::::::::::::::::::::::::::::::::::::::::::::::::::::::::::::::::::::::::::::::|<<|
for all \(  \uX \in (-\zO,0]  \).
In turn, the inequality
%|>>|:::::::::::::::::::::::::::::::::::::::::::::::::::::::::::::::::::::::::::::::::::::::::::|>>|
\(  \integrandP( \uX ) \geq \integrandL( \uX )  \)
%|<<|:::::::::::::::::::::::::::::::::::::::::::::::::::::::::::::::::::::::::::::::::::::::::::|<<|
would hold for all \(  \uX \in (-\zO,0]  \) if \(  \integrandP  \) and \(  \integrandL  \) do not intersect on the interval \(  (-\zO,0]  \) since this, along with the choice of \(  \zO  \) in \EQUATION \eqref{eqn:pf:claim:pf:corollary:main-technical:probit:p_P<p_L:5}, would imply that
%|>>|:::::::::::::::::::::::::::::::::::::::::::::::::::::::::::::::::::::::::::::::::::::::::::|>>|
\(  \integrandP( \uX ) > \integrandL( \uX )  \)
%|<<|:::::::::::::::::::::::::::::::::::::::::::::::::::::::::::::::::::::::::::::::::::::::::::|<<|
for all \(  \uX \in (-\zO,0]  \).
%
%%%%%%%%%%%%%%%%%%%%%%%%%%%%%%%%%%%%%%%%%%%%%%%%%%%%%%%%%%%%%%%%%%%%%%%%%%%%%%%%%%%%%%%%%%%%%%%%%%%%
\par %%%%%%%%%%%%%%%%%%%%%%%%%%%%%%%%%%%%%%%%%%%%%%%%%%%%%%%%%%%%%%%%%%%%%%%%%%%%%%%%%%%%%%%%%%%%%%%
%%%%%%%%%%%%%%%%%%%%%%%%%%%%%%%%%%%%%%%%%%%%%%%%%%%%%%%%%%%%%%%%%%%%%%%%%%%%%%%%%%%%%%%%%%%%%%%%%%%%
%
%This will be established in the following argument.
The fact that
%|>>|:::::::::::::::::::::::::::::::::::::::::::::::::::::::::::::::::::::::::::::::::::::::::::|>>|
\(  \integrandP[ (-\zO,0] ] \cap \integrandL[ (-\zO,0] ] = \emptyset  \)
%|<<|:::::::::::::::::::::::::::::::::::::::::::::::::::::::::::::::::::::::::::::::::::::::::::|<<|
will be established in the following argument.
First, by \EQUATION \eqref{eqn:pf:claim:pf:corollary:main-technical:probit:p_P<p_L:5} and the condition that
%|>>|:::::::::::::::::::::::::::::::::::::::::::::::::::::::::::::::::::::::::::::::::::::::::::|>>|
\(  \zO > 0  \),
%|<<|:::::::::::::::::::::::::::::::::::::::::::::::::::::::::::::::::::::::::::::::::::::::::::|<<|
there is an equality:
%|>>|:::::::::::::::::::::::::::::::::::::::::::::::::::::::::::::::::::::::::::::::::::::::::::|>>|
\(  \integrandP( -\zO ) = \integrandL( -\zO )  \).
%|<<|:::::::::::::::::::::::::::::::::::::::::::::::::::::::::::::::::::::::::::::::::::::::::::|<<|
Because \(  \integrandP  \) and \(  \integrandL  \) are even functions, it suffices to show that the two functions do not intersect on the interval \(  [0,\zO)  \), which is preferred for the slightly simpler notation.
This symmetry across the \(  y  \)-axis also implies that
%|>>|:::::::::::::::::::::::::::::::::::::::::::::::::::::::::::::::::::::::::::::::::::::::::::|>>|
\(  \integrandP( \zO ) = \integrandL( \zO )  \)
%|<<|:::::::::::::::::::::::::::::::::::::::::::::::::::::::::::::::::::::::::::::::::::::::::::|<<|
since, as noted earlier,
%|>>|:::::::::::::::::::::::::::::::::::::::::::::::::::::::::::::::::::::::::::::::::::::::::::|>>|
\(  \integrandP( -\zO ) = \integrandL( -\zO )  \).
%|<<|:::::::::::::::::::::::::::::::::::::::::::::::::::::::::::::::::::::::::::::::::::::::::::|<<|
Because \(  -\log( \cdot )  \) is defined over both \(  \integrandP[ \R ]  \) and \(  \integrandL[ \R ]  \), if \(  -\log \circ \integrandP  \) and \(  -\log \circ \integrandL  \) do not intersect on the interval \(  [0,\zO)  \), then the analog is true for \(  \integrandP  \) and \(  \integrandL  \).
Moreover,
%|>>|:::::::::::::::::::::::::::::::::::::::::::::::::::::::::::::::::::::::::::::::::::::::::::|>>|
\(  \integrandP( \zO ) = \integrandL( \zO )  \)
%|<<|:::::::::::::::::::::::::::::::::::::::::::::::::::::::::::::::::::::::::::::::::::::::::::|<<|
implies that
%|>>|:::::::::::::::::::::::::::::::::::::::::::::::::::::::::::::::::::::::::::::::::::::::::::|>>|
\(  -\log \circ \integrandP( \zO ) = -\log \circ \integrandL( \zO )  \).
%|<<|:::::::::::::::::::::::::::::::::::::::::::::::::::::::::::::::::::::::::::::::::::::::::::|<<|
Observe that for \(  \uX \in \R  \),
%|>>|===========================================================================================|>>|
\begin{gather*}
  -\log \circ \integrandP( \uX )
  =
  -\log \left( \frac{1}{\sqrt{2\pi}} e^{-\frac{1}{2} \uX^{2}} \right)
  =
  \frac{1}{2} \uX^{2} + \log( \sqrt{2\pi} )
  ,\\
  -\log \circ \integrandL( \uX )
  =
  -\log \left( \frac{1}{( 1+e^{-\uX} ) ( 1+e^{\uX} )} \right)
  =
  \log( 1+e^{-\uX} ) + \log( 1+e^{\uX} )
,\end{gather*}
%|<<|===========================================================================================|<<|
and
%|>>|===========================================================================================|>>|
\begin{gather*}
  \frac{d}{d\uX} (  -\log \circ \integrandP( \uX ) )
  =
  \frac{d}{d\uX} \left( \frac{1}{2} \uX^{2} + \log( \sqrt{2\pi} ) \right)
  =
  \uX
  ,\\
  \frac{d}{d\uX} ( -\log \circ \integrandL( \uX ) )
  =
  \frac{d}{d\uX} \left( \log( 1+e^{-\uX} ) + \log( 1+e^{\uX} ) \right)
  =
  \frac{e^{\uX}-1}{e^{\uX}+1}
  =
  \tanh \left( \frac{\uX}{2} \right)
.\end{gather*}
%|<<|===========================================================================================|<<|
Consider two cases:
\Enum[{\label{enum:pf:claim:pf:corollary:main-technical:probit:p_P<p_L:2:a}}]{a}
when \(  \uX \geq 1  \), and
\Enum[{\label{enum:pf:claim:pf:corollary:main-technical:probit:p_P<p_L:2:b}}]{b}
when \(  \uX \in [0,1)  \).
In \CASE \ref{enum:pf:claim:pf:corollary:main-technical:probit:p_P<p_L:2:a},
%|>>|===========================================================================================|>>|
\begin{gather*}
  \frac{d}{d\uX} ( -\log \circ \integrandL( \uX ) )
  =
  \frac{e^{\uX}-1}{e^{\uX}+1}
  <
  1
  \leq
  \uX
  =
  \frac{d}{d\uX} ( -\log \circ \integrandP( \uX ) )
.\end{gather*}
%|<<|===========================================================================================|<<|
Otherwise, in \CASE \ref{enum:pf:claim:pf:corollary:main-technical:probit:p_P<p_L:2:b}, it follows from the (truncated) Taylor expansion of the hyperbolic tangent function %at \(  \frac{\uX}{2} \in [0,\frac{1}{2})  \)
at the origin
that
%|>>|===========================================================================================|>>|
\begin{gather*}
  \frac{d}{d\uX} ( -\log \circ \integrandL( \uX ) )
  =
  \tanh \left( \frac{\uX}{2} \right)
  \leq
  \frac{\uX}{2}
  \leq
  \uX
  =
  \frac{d}{d\uX} ( -\log \circ \integrandP( \uX ) )
,\end{gather*}
%|<<|===========================================================================================|<<|
where equality holds if and only if \(  \uX = 0  \).
Therefore, combining \CASES \ref{enum:pf:claim:pf:corollary:main-technical:probit:p_P<p_L:2:a} and \ref{enum:pf:claim:pf:corollary:main-technical:probit:p_P<p_L:2:b},
%|>>|:::::::::::::::::::::::::::::::::::::::::::::::::::::::::::::::::::::::::::::::::::::::::::|>>|
\(  \frac{d}{d\uX} ( -\log \circ \integrandL( \uX ) ) \leq \frac{d}{d\uX} ( -\log \circ \integrandP( \uX ) )  \)
%|<<|:::::::::::::::::::::::::::::::::::::::::::::::::::::::::::::::::::::::::::::::::::::::::::|<<|
for all \(  \uX \geq 0  \), where again, equality holds if and only if \(  \uX = 0 < \zO  \).
Recall that in addition,
%|>>|:::::::::::::::::::::::::::::::::::::::::::::::::::::::::::::::::::::::::::::::::::::::::::|>>|
\(  -\log \circ \integrandP( \zO ) = -\log \circ \integrandL( \zO )  \)
%|<<|:::::::::::::::::::::::::::::::::::::::::::::::::::::::::::::::::::::::::::::::::::::::::::|<<|
and that \(  \zO > 0  \).
Together, using basic calculus, these properties imply that \(  -\log \circ \integrandP  \) and \(  -\log \circ \integrandL  \) do not intersect on the interval \(  [0,\zO)  \).
It follows that
%|>>|:::::::::::::::::::::::::::::::::::::::::::::::::::::::::::::::::::::::::::::::::::::::::::|>>|
\(  \integrandP[ [0,\zO) ] \cap \integrandL[ [0,\zO) ] = \emptyset  \),
%|<<|:::::::::::::::::::::::::::::::::::::::::::::::::::::::::::::::::::::::::::::::::::::::::::|<<|
and hence also that
%|>>|:::::::::::::::::::::::::::::::::::::::::::::::::::::::::::::::::::::::::::::::::::::::::::|>>|
\(  \integrandP[ (-\zO,0] ] \cap \integrandL[ (-\zO,0] ] = \emptyset  \),
%|<<|:::::::::::::::::::::::::::::::::::::::::::::::::::::::::::::::::::::::::::::::::::::::::::|<<|
as claimed.
%
%%%%%%%%%%%%%%%%%%%%%%%%%%%%%%%%%%%%%%%%%%%%%%%%%%%%%%%%%%%%%%%%%%%%%%%%%%%%%%%%%%%%%%%%%%%%%%%%%%%%
\par %%%%%%%%%%%%%%%%%%%%%%%%%%%%%%%%%%%%%%%%%%%%%%%%%%%%%%%%%%%%%%%%%%%%%%%%%%%%%%%%%%%%%%%%%%%%%%%
%%%%%%%%%%%%%%%%%%%%%%%%%%%%%%%%%%%%%%%%%%%%%%%%%%%%%%%%%%%%%%%%%%%%%%%%%%%%%%%%%%%%%%%%%%%%%%%%%%%%
%
Having shown that \(  \integrandP  \) and \(  \integrandL  \) do not intersect on the interval \(  (-\zO,0]  \), it can be deduced from the earlier line of argument that
%|>>|:::::::::::::::::::::::::::::::::::::::::::::::::::::::::::::::::::::::::::::::::::::::::::|>>|
\(  \sup_{\zX \in [0,\zO)} \pFnP( -\zX ) - \pFnL( -\zX ) = \pFnP( 0 ) - \pFnL( 0 ) = 0  \).
%|<<|:::::::::::::::::::::::::::::::::::::::::::::::::::::::::::::::::::::::::::::::::::::::::::|<<|
Thus,
%|>>|:::::::::::::::::::::::::::::::::::::::::::::::::::::::::::::::::::::::::::::::::::::::::::|>>|
\(  \pFnP( -\zX ) - \pFnL( -\zX ) \leq 0  \)
%|<<|:::::::::::::::::::::::::::::::::::::::::::::::::::::::::::::::::::::::::::::::::::::::::::|<<|
for every \(  \zX \in [0,\zO)  \), or equivalently,
%|>>|:::::::::::::::::::::::::::::::::::::::::::::::::::::::::::::::::::::::::::::::::::::::::::|>>|
\(  \pFnP( -\zX ) \leq \pFnL( -\zX )  \)
%|<<|:::::::::::::::::::::::::::::::::::::::::::::::::::::::::::::::::::::::::::::::::::::::::::|<<|
on this interval, completing the analysis for \CASE \ref{enum:pf:claim:pf:corollary:main-technical:probit:p_P<p_L:1:ii}.
%
%%%%%%%%%%%%%%%%%%%%%%%%%%%%%%%%%%%%%%%%%%%%%%%%%%%%%%%%%%%%%%%%%%%%%%%%%%%%%%%%%%%%%%%%%%%%%%%%%%%%
\par %%%%%%%%%%%%%%%%%%%%%%%%%%%%%%%%%%%%%%%%%%%%%%%%%%%%%%%%%%%%%%%%%%%%%%%%%%%%%%%%%%%%%%%%%%%%%%%
%%%%%%%%%%%%%%%%%%%%%%%%%%%%%%%%%%%%%%%%%%%%%%%%%%%%%%%%%%%%%%%%%%%%%%%%%%%%%%%%%%%%%%%%%%%%%%%%%%%%
%
Taken together, the arguments for \CASES \ref{enum:pf:claim:pf:corollary:main-technical:probit:p_P<p_L:1:i} and \ref{enum:pf:claim:pf:corollary:main-technical:probit:p_P<p_L:1:ii} establish that
%|>>|:::::::::::::::::::::::::::::::::::::::::::::::::::::::::::::::::::::::::::::::::::::::::::|>>|
\(  \pFnP( -\zX ) \leq \pFnL( -\zX )  \)
%|<<|:::::::::::::::::::::::::::::::::::::::::::::::::::::::::::::::::::::::::::::::::::::::::::|<<|
for all \(  \zX \geq 0  \).
\CLAIM \ref{claim:pf:corollary:main-technical:probit:p_P<p_L} follows.
\end{proof}
%|<<|~~~~~~~~~~~~~~~~~~~~~~~~~~~~~~~~~~~~~~~~~~~~~~~~~~~~~~~~~~~~~~~~~~~~~~~~~~~~~~~~~~~~~~~~~~~|<<|
%|<<|~~~~~~~~~~~~~~~~~~~~~~~~~~~~~~~~~~~~~~~~~~~~~~~~~~~~~~~~~~~~~~~~~~~~~~~~~~~~~~~~~~~~~~~~~~~|<<|
%|<<|~~~~~~~~~~~~~~~~~~~~~~~~~~~~~~~~~~~~~~~~~~~~~~~~~~~~~~~~~~~~~~~~~~~~~~~~~~~~~~~~~~~~~~~~~~~|<<|
With \CLAIM \ref{claim:pf:corollary:main-technical:probit:p_P<p_L} now proved, \COROLLARY \ref{corollary:main-technical:probit} holds.
\ENDGRAYOUT
\end{comment}
\end{proof}
%|<<|~~~~~~~~~~~~~~~~~~~~~~~~~~~~~~~~~~~~~~~~~~~~~~~~~~~~~~~~~~~~~~~~~~~~~~~~~~~~~~~~~~~~~~~~~~~|<<|
%|<<|~~~~~~~~~~~~~~~~~~~~~~~~~~~~~~~~~~~~~~~~~~~~~~~~~~~~~~~~~~~~~~~~~~~~~~~~~~~~~~~~~~~~~~~~~~~|<<|
%|<<|~~~~~~~~~~~~~~~~~~~~~~~~~~~~~~~~~~~~~~~~~~~~~~~~~~~~~~~~~~~~~~~~~~~~~~~~~~~~~~~~~~~~~~~~~~~|<<|

%%%%%%%%%%%%%%%%%%%%%%%%%%%%%%%%%%%%%%%%%%%%%%%%%%%%%%%%%%%%%%%%%%%%%%%%%%%%%%%%%%%%%%%%%%%%%%%%%%%%
%%%%%%%%%%%%%%%%%%%%%%%%%%%%%%%%%%%%%%%%%%%%%%%%%%%%%%%%%%%%%%%%%%%%%%%%%%%%%%%%%%%%%%%%%%%%%%%%%%%%
%%%%%%%%%%%%%%%%%%%%%%%%%%%%%%%%%%%%%%%%%%%%%%%%%%%%%%%%%%%%%%%%%%%%%%%%%%%%%%%%%%%%%%%%%%%%%%%%%%%%

\subsection{Proof of the Intermediate Results}
\label{outline:pf-main-technical-result|pf-intermediate}

Having completed the proofs of the main technical results, \THEOREM \ref{thm:main-technical:sparse} and \COROLLARY \ref{corollary:main-technical:logistic-regression},
%and \ref{corollary:main-technical:probit},
the auxiliary results used therein, \LEMMAS \ref{lemma:combine}--\ref{lemma:large-dist:2}---which were introduced in \SECTION \ref{outline:pf-main-technical-result|intermediate}---are proved next.

%%%%%%%%%%%%%%%%%%%%%%%%%%%%%%%%%%%%%%%%%%%%%%%%%%%%%%%%%%%%%%%%%%%%%%%%%%%%%%%%%%%%%%%%%%%%%%%%%%%%
%%%%%%%%%%%%%%%%%%%%%%%%%%%%%%%%%%%%%%%%%%%%%%%%%%%%%%%%%%%%%%%%%%%%%%%%%%%%%%%%%%%%%%%%%%%%%%%%%%%%
%%%%%%%%%%%%%%%%%%%%%%%%%%%%%%%%%%%%%%%%%%%%%%%%%%%%%%%%%%%%%%%%%%%%%%%%%%%%%%%%%%%%%%%%%%%%%%%%%%%%

\subsubsection{Concentration Inequalities, Expectations, and a Deterministic Bound}
\label{outline:pf-main-technical-result|pf-intermediate|concentration-ineq}

The concentration inequalities and expectations in \LEMMA \ref{lemma:concentration-ineq}, below, will be crucial to the proofs of the intermediate results, \LEMMAS \ref{lemma:combine}--\ref{lemma:large-dist:2}.
The proof this lemma is deferred to \SECTION \ref{outline:concentration-ineq}.

%|>>|*******************************************************************************************|>>|
%|>>|*******************************************************************************************|>>|
%|>>|*******************************************************************************************|>>|
\begin{lemma}
\label{lemma:concentration-ineq}
%
Fix
%|>>|:::::::::::::::::::::::::::::::::::::::::::::::::::::::::::::::::::::::::::::::::::::::::::|>>|
\(  \sXX, \sXXX, \tX, \tXX, \deltaX \in (0,1)  \).
%|<<|:::::::::::::::::::::::::::::::::::::::::::::::::::::::::::::::::::::::::::::::::::::::::::|<<|
Fix
%|>>|:::::::::::::::::::::::::::::::::::::::::::::::::::::::::::::::::::::::::::::::::::::::::::|>>|
\(  \thetaStar \in \ParamSpace  \),
%|<<|:::::::::::::::::::::::::::::::::::::::::::::::::::::::::::::::::::::::::::::::::::::::::::|<<|
and let
%|>>|:::::::::::::::::::::::::::::::::::::::::::::::::::::::::::::::::::::::::::::::::::::::::::|>>|
\(  \JS, \JSX \subseteq 2^{[\n]}  \) and
\(  \ParamCover \subset \ParamSpace  \)
%|<<|:::::::::::::::::::::::::::::::::::::::::::::::::::::::::::::::::::::::::::::::::::::::::::|<<|
be finite sets, and define
%|>>|:::::::::::::::::::::::::::::::::::::::::::::::::::::::::::::::::::::::::::::::::::::::::::|>>|
\(  \ParamCoverX \defeq \ParamCover \setminus \Ball{\tauX}( \thetaStar )  \).
%|<<|:::::::::::::::::::::::::::::::::::::::::::::::::::::::::::::::::::::::::::::::::::::::::::|<<|
Let
%|>>|:::::::::::::::::::::::::::::::::::::::::::::::::::::::::::::::::::::::::::::::::::::::::::|>>|
\(  \kO \defeq \kOExpr  \) and
\(  \kOX \defeq \kOXExpr  \),
%|<<|:::::::::::::::::::::::::::::::::::::::::::::::::::::::::::::::::::::::::::::::::::::::::::|<<|
and let
%|>>|:::::::::::::::::::::::::::::::::::::::::::::::::::::::::::::::::::::::::::::::::::::::::::|>>|
\(  \alphaO = \alphaO( \deltaX ) = \alphaOExpr  \).
%|<<|:::::::::::::::::::::::::::::::::::::::::::::::::::::::::::::::::::::::::::::::::::::::::::|<<|
Then,
%|>>|===========================================================================================|>>|
\begin{gather}
  \Pr \left( {%\textstyle
    \Forall{\JCoords \in \JS, \thetaX \in \ParamCoverX}{
    \left\| \frac{\hFn[\JCoords]( \thetaStar, \thetaX )}{\sqrt{2\pi}} - \E \left[ \frac{\hFn[\JCoords]( \thetaStar, \thetaX )}{\sqrt{2\pi}} \right] \right\|_{2}
    \leq
    \sqrt{\frac{( 1+\sXX )( \kO-2 ) \ADIST}{\pi \m}}
    +
    \frac{\tX \ADIST}{\pi}
    }
  } \right)
  \nonumber \\
  \geq
  1
  -
  4 | \JS | | \ParamCoverX | e^{-\frac{1}{27\pi} \m \tX^{2} \ADIST}
  -
  | \JS | | \ParamCoverX | e^{-\frac{1}{18\pi ( 1+\sXX )} \m \tX^{2} \ADIST}
  -
  | \ParamCoverX | e^{-\frac{1}{3\pi} \m \sXX^{2} \ADIST}
  \label{eqn:lemma:concentration-ineq:pr:1}
  ,\\
  \Pr \left( {%\textstyle
    \Forall{\JCoordsX \in \JSX}{
    \left\| \frac{\hfFn[\JCoordsX]( \thetaStar, \thetaStar )}{\sqrt{2\pi}} - \E \left[ \frac{\hfFn[\JCoordsX]( \thetaStar, \thetaStar )}{\sqrt{2\pi}} \right] \right\|_{2}
    \leq
    \sqrt{\frac{\alphaO ( 1+\sXXX )( \kOX-1 )}{\m}}
    +
    \alphaO \tXX
    }
  } \right)
  \nonumber \\
  \geq
  1
  -
  2 | \JSX | e^{-\frac{1}{12} \alphaO \m \tXX^{2}}
  -
  | \JSX | e^{-\frac{1}{8} \frac{\alphaO \m \tXX^{2}}{1+\sXXX}}
  -
  e^{-\frac{1}{3} \alphaO \m \sXXX^{2}}
  \label{eqn:lemma:concentration-ineq:pr:2}
,\end{gather}
%|<<|===========================================================================================|<<|
where in expectation, for any \(  \JCoords \subseteq [\n]  \) and \(  \thetaX \in \ParamSpace  \),
%|>>|===========================================================================================|>>|
\begin{gather}
  \label{eqn:lemma:concentration-ineq:ev:1}
  \E \left[ \hFn[\JCoords]( \thetaStar, \thetaX ) \right]
  =
  \thetaStar - \thetaX
  ,\\ \label{eqn:lemma:concentration-ineq:ev:2}
  \E \left[ \hfFn[\JCoords]( \thetaStar, \thetaStar ) \right]
  =
  \E \left[ \langle \hfFn[\JCoords]( \thetaStar, \thetaStar ), \thetaStar \rangle \right] \thetaStar
  =
  -\left( 1 - \sqrt{\frac{\pi}{2}} \gammaX \right) \thetaStar
%  ,\\ \label{eqn:lemma:concentration-ineq:ev:3}
%  \E \left[ \langle \hfFn[\JCoords]( \thetaStar, \thetaStar ), \thetaStar \rangle \right]
%  =
%  -\gammaX
  ,\\ \label{eqn:lemma:concentration-ineq:ev:4}
  \DENOM
  =
  \| \E \left[ \thetaX + \hfFn[\JCoords]( \thetaStar, \thetaX ) \right] \|_{2}
  =
  \sqrt{\frac{\pi}{2}}\gammaX
.\end{gather}
%|<<|===========================================================================================|<<|
\end{lemma}
%|<<|*******************************************************************************************|<<|
%|<<|*******************************************************************************************|<<|
%|<<|*******************************************************************************************|<<|

In addition to the above lemma, the following fact will facilitate the analysis in this section.

%|>>|*******************************************************************************************|>>|
%|>>|*******************************************************************************************|>>|
%|>>|*******************************************************************************************|>>|
\begin{fact}
\label{fact:d_angular-d_S}
%
Let
%|>>|:::::::::::::::::::::::::::::::::::::::::::::::::::::::::::::::::::::::::::::::::::::::::::|>>|
\(  \Vec{\uV}, \Vec{\vV} \in \Sphere{\n}  \).
%|<<|:::::::::::::::::::::::::::::::::::::::::::::::::::::::::::::::::::::::::::::::::::::::::::|<<|
Then,
%|>>|===========================================================================================|>>|
\begin{gather}
\label{eqn:fact:d_angular-d_S:1}
  \| \Vec{\uV} - \Vec{\vV} \|_{2}
  \leq
  \arccos( \langle \Vec{\uV}, \Vec{\vV} \rangle )
  \leq
  \frac{\pi}{2} \| \Vec{\uV} - \Vec{\vV} \|_{2}
.\end{gather}
%|<<|===========================================================================================|<<|
\end{fact}
%|<<|*******************************************************************************************|<<|
%|<<|*******************************************************************************************|<<|
%|<<|*******************************************************************************************|<<|

%|>>|~~~~~~~~~~~~~~~~~~~~~~~~~~~~~~~~~~~~~~~~~~~~~~~~~~~~~~~~~~~~~~~~~~~~~~~~~~~~~~~~~~~~~~~~~~~|>>|
%|>>|~~~~~~~~~~~~~~~~~~~~~~~~~~~~~~~~~~~~~~~~~~~~~~~~~~~~~~~~~~~~~~~~~~~~~~~~~~~~~~~~~~~~~~~~~~~|>>|
%|>>|~~~~~~~~~~~~~~~~~~~~~~~~~~~~~~~~~~~~~~~~~~~~~~~~~~~~~~~~~~~~~~~~~~~~~~~~~~~~~~~~~~~~~~~~~~~|>>|
\begin{proof}
{\FACT \ref{fact:d_angular-d_S}}
%
\checkoff%
%
Fix \(  \Vec{\uV}, \Vec{\vV} \in \Sphere{\n}  \) arbitrarily.
To verify the first inequality in \EQUATION \eqref{eqn:fact:d_angular-d_S:1}---that
%|>>|:::::::::::::::::::::::::::::::::::::::::::::::::::::::::::::::::::::::::::::::::::::::::::|>>|
\({  \| \Vec{\uV} - \Vec{\vV} \|_{2} \leq \arccos( \langle \Vec{\uV}, \Vec{\vV} \rangle )  }\)%
%|<<|:::::::::::::::::::::::::::::::::::::::::::::::::::::::::::::::::::::::::::::::::::::::::::|<<|
---observe:
%|>>|===========================================================================================|>>|
\begin{align*}
  \| \Vec{\uV} - \Vec{\vV} \|_{2}
  % &=
  % \sqrt{\| \Vec{\uV} - \Vec{\vV} \|_{2}^{2}}
  % \\
  &=
  \sqrt{2 - 2 \langle \Vec{\uV}, \Vec{\vV} \rangle}
  \\
  % &=
  % \sqrt{2 ( 1 - \langle \Vec{\uV}, \Vec{\vV} \rangle )}
  % \\
  % &=
  % \sqrt{2 ( 1 - \cos( \arccos( \langle \Vec{\uV}, \Vec{\vV} \rangle ) ) )}
  % \\
  &\leq
  \sqrt{2 \left( 1 - \left( 1 - \frac{\arccos^{2}( \langle \Vec{\uV}, \Vec{\vV} \rangle )}{2} \right) \right)}
  \\
  &\dCmt{by the Taylor series for the cosine function, \(  \cos(x) \geq 1 - \frac{x^{2}}{2}  \), \(  x \in \R  \)}
  % \\
  % &=
  % \sqrt{2 \frac{\arccos^{2}( \langle \Vec{\uV}, \Vec{\vV} \rangle )}{2}}
  \\
  &=
  \arccos( \langle \Vec{\uV}, \Vec{\vV} \rangle )
,\end{align*}
%|<<|===========================================================================================|<<|
as desired.
For the second inequality in \EQUATION \eqref{eqn:fact:d_angular-d_S:1}---that
%|>>|:::::::::::::::::::::::::::::::::::::::::::::::::::::::::::::::::::::::::::::::::::::::::::|>>|
\(  \arccos( \langle \Vec{\uV}, \Vec{\vV} \rangle ) \leq \frac{\pi}{2} \| \Vec{\uV} - \Vec{\vV} \|_{2}  \)%
%|<<|:::::::::::::::::::::::::::::::::::::::::::::::::::::::::::::::::::::::::::::::::::::::::::|<<|
---note that by standard trigonometric properties,
%|>>|:::::::::::::::::::::::::::::::::::::::::::::::::::::::::::::::::::::::::::::::::::::::::::|>>|
\(  \| \Vec{\uV} - \Vec{\vV} \|_{2} = 2 \sin( \frac{\arccos( \langle \Vec{\uV}, \Vec{\vV} \rangle )}{2} )  \).
%|<<|:::::::::::::::::::::::::::::::::::::::::::::::::::::::::::::::::::::::::::::::::::::::::::|<<|
To proceed, some basic calculus is needed to examine the function \(  \frac{\sin(x)}{x}  \) on the interval \(  x \in (0,\frac{\pi}{2}]  \).
Using the quotient rule,
%|>>|===========================================================================================|>>|
\begin{align*}
  \frac{d}{dx} \frac{\sin(x)}{x} = \frac{x \cos(x) - \sin(x)}{x^{2}}
,\end{align*}
%|<<|===========================================================================================|<<|
where the numerator determines the sign of the above expression and has a
%Maclaurin
Taylor
series given by
%|>>|===========================================================================================|>>|
\begin{align*}
  x \cos(x) - \sin(x)
  &=
  \sum_{z=0}^{\infty}
  \frac{(-1)^{z} x^{2z+1}}{(2z)!}
  -
  \sum_{z=0}^{\infty}
  \frac{(-1)^{z} x^{2z+1}}{(2z+1)!}
%  =
%  \sum_{z=1}^{\infty}
%  \frac{(-1)^{z} x^{2z+1}}{(2z)!}
%  -
%  \sum_{z=1}^{\infty}
%  \frac{(-1)^{z} x^{2z+1}}{(2z+1)!}
  =
  \sum_{z=1}^{\infty}
  \frac{(-1)^{z} x^{2z+1}}{(2z)!}
  \left( 1 - \frac{1}{2z+1} \right)
.\end{align*}
%|<<|===========================================================================================|<<|
Now, it can be seen that for \(  x \in (0, \frac{\pi}{2}]  \),
%|>>|===========================================================================================|>>|
\begin{align*}
  \frac{d}{dx} \frac{\sin(x)}{x}
  =
  \frac{x \cos(x) - \sin(x)}{x^{2}}
  =
  \frac{1}{x^{2}}
  \sum_{z=1}^{\infty}
  \frac{(-1)^{z} x^{2z+1}}{(2z)!}
  \left( 1 - \frac{1}{2z+1} \right)
  <
  0
,\end{align*}
%|<<|===========================================================================================|<<|
which implies that \(  \frac{\sin(x)}{x}  \) decreases over the interval \(  x \in (0, \frac{\pi}{2}]  \).
Hence,
%|>>|===========================================================================================|>>|
\begin{gather*}
  \inf_{x \in (0,\frac{\pi}{2}]} \frac{\sin(x)}{x}
  =
  \left. \frac{\sin(x)}{x} \right|_{x = \frac{\pi}{2}}
  =
  \frac{\sin \left( \frac{\pi}{2} \right)}{\frac{\pi}{2}}
  =
  \frac{2}{\pi}
.\end{gather*}
%|<<|===========================================================================================|<<|
Then,
%|>>|===========================================================================================|>>|
\begin{align*}
  \frac{\| \Vec{\uV} - \Vec{\vV} \|_{2}}{\arccos( \langle \Vec{\uV}, \Vec{\vV} \rangle )}
  =
  \frac{2 \sin \left( \frac{\arccos( \langle \Vec{\uV}, \Vec{\vV} \rangle )}{2} \right)}{\arccos( \langle \Vec{\uV}, \Vec{\vV} \rangle )}
  % =
  % \frac{\sin \left( \frac{\arccos( \langle \Vec{\uV}, \Vec{\vV} \rangle )}{2} \right)}{\frac{\arccos( \langle \Vec{\uV}, \Vec{\vV} \rangle )}{2}}
  \geq
  \frac{2}{\pi}
,\end{align*}
%|<<|===========================================================================================|<<|
implying that
%|>>|===========================================================================================|>>|
\begin{gather*}
  \arccos( \langle \Vec{\uV}, \Vec{\vV} \rangle )
  \leq
  \frac{\pi}{2} \| \Vec{\uV} - \Vec{\vV} \|_{2}
,\end{gather*}
%|<<|===========================================================================================|<<|
as claimed.
\end{proof}
%|<<|~~~~~~~~~~~~~~~~~~~~~~~~~~~~~~~~~~~~~~~~~~~~~~~~~~~~~~~~~~~~~~~~~~~~~~~~~~~~~~~~~~~~~~~~~~~|<<|
%|<<|~~~~~~~~~~~~~~~~~~~~~~~~~~~~~~~~~~~~~~~~~~~~~~~~~~~~~~~~~~~~~~~~~~~~~~~~~~~~~~~~~~~~~~~~~~~|<<|
%|<<|~~~~~~~~~~~~~~~~~~~~~~~~~~~~~~~~~~~~~~~~~~~~~~~~~~~~~~~~~~~~~~~~~~~~~~~~~~~~~~~~~~~~~~~~~~~|<<|

%%%%%%%%%%%%%%%%%%%%%%%%%%%%%%%%%%%%%%%%%%%%%%%%%%%%%%%%%%%%%%%%%%%%%%%%%%%%%%%%%%%%%%%%%%%%%%%%%%%%
%%%%%%%%%%%%%%%%%%%%%%%%%%%%%%%%%%%%%%%%%%%%%%%%%%%%%%%%%%%%%%%%%%%%%%%%%%%%%%%%%%%%%%%%%%%%%%%%%%%%

\subsubsection{Proofs of \LEMMAS \ref{lemma:combine}--\ref{lemma:large-dist:2}}
\label{outline:pf-main-technical-result|pf-intermediate|pf}

With the above auxiliary results, \LEMMAS \ref{lemma:combine}--\ref{lemma:large-dist:2} can now be established.
We begin with the proof of \LEMMA \ref{lemma:combine}.

%|>>|~~~~~~~~~~~~~~~~~~~~~~~~~~~~~~~~~~~~~~~~~~~~~~~~~~~~~~~~~~~~~~~~~~~~~~~~~~~~~~~~~~~~~~~~~~~|>>|
%|>>|~~~~~~~~~~~~~~~~~~~~~~~~~~~~~~~~~~~~~~~~~~~~~~~~~~~~~~~~~~~~~~~~~~~~~~~~~~~~~~~~~~~~~~~~~~~|>>|
%|>>|~~~~~~~~~~~~~~~~~~~~~~~~~~~~~~~~~~~~~~~~~~~~~~~~~~~~~~~~~~~~~~~~~~~~~~~~~~~~~~~~~~~~~~~~~~~|>>|
\begin{proof}
{\LEMMA \ref{lemma:combine}}
%
\mostlycheckoff%
%
The first step towards proving the lemma will be showing that
%|>>|===========================================================================================|>>|
\begin{align*}
  \frac
  {\E [ \thetaXX+\hfFn[\JCoords]( \thetaStar, \thetaXX ) ]}
  {\| \E [ \thetaXX+\hfFn[\JCoords]( \thetaStar, \thetaXX ) ] \|_{2}}
  =
  \thetaStar
,\end{align*}
%|<<|===========================================================================================|<<|
where
%|>>|:::::::::::::::::::::::::::::::::::::::::::::::::::::::::::::::::::::::::::::::::::::::::::|>>|
\(  \thetaStar, \thetaXX \in \ParamSpace  \) and \(  \JCoords \subseteq [\n]  \)
%|<<|:::::::::::::::::::::::::::::::::::::::::::::::::::::::::::::::::::::::::::::::::::::::::::|<<|
are arbitrary.
Notice that for any \(  \Vec{\uV}, \Vec{\vV}, \Vec{\wV} \in \R^{\n}  \) such that
%|>>|:::::::::::::::::::::::::::::::::::::::::::::::::::::::::::::::::::::::::::::::::::::::::::|>>|
\(  \Supp( \Vec{\wV}  ) \cup \JCoords = \Supp( \Vec{\vV} ) \cup \JCoords  \),
%|<<|:::::::::::::::::::::::::::::::::::::::::::::::::::::::::::::::::::::::::::::::::::::::::::|<<|
the following pair of equations holds:
%|>>|===========================================================================================|>>|
\begin{gather}
  \label{eqn:pf:lemma:combine:1a}
  \hfFn[\JCoords]( \Vec{\uV}, \Vec{\vV} )
  =
  \hFn[\JCoords]( \Vec{\uV}, \Vec{\vV} )
  +
  \hfFn[\Supp( \Vec{\vV} ) \cup \JCoords]( \Vec{\uV}, \Vec{\uV} )
  =
  \hFn[\JCoords]( \Vec{\uV}, \Vec{\vV} )
  +
  \hfFn[\Supp( \Vec{\wV} ) \cup \JCoords]( \Vec{\uV}, \Vec{\uV} )
  ,\\ \label{eqn:pf:lemma:combine:1b}
  \hFn[\JCoords]( \Vec{\uV}, \Vec{\vV} )
  =
  \hFn[\JCoords]( \Vec{\uV}, \Vec{\wV} )
  +
  \hFn[\Supp( \Vec{\uV} ) \cup \JCoords]( \Vec{\wV}, \Vec{\vV} )
.\end{gather}
%|<<|===========================================================================================|<<|
To justify these equations, observe:
%\checkthis%
%|>>|===========================================================================================|>>|
\begin{align*}
  % &\negphantom{\AlignSp}
  \hfFn[\JCoords]( \Vec{\uV}, \Vec{\vV} )
  % \\
  &=
  \ThresholdSet{\Supp( \Vec{\uV} ) \cup \Supp( \Vec{\vV} ) \cup \JCoords} (
  \frac{\sqrt{2\pi}}{\m}
  \sep
  \CovM^{\T}
  \sep
  \frac{1}{2}
  \left( \fFn( \CovM \Vec{\uV} ) - \Sign( \CovM \Vec{\vV} ) \right)
  )
  % \\
  % &=
  % \ThresholdSet{\Supp( \Vec{\uV} ) \cup \Supp( \Vec{\vV} ) \cup \JCoords} (
  % \frac{\sqrt{2\pi}}{\m}
  % \sep
  % \CovM^{\T}
  % \sep
  % \frac{1}{2}
  % \left( \Sign( \CovM \Vec{\uV} ) - \Sign( \CovM \Vec{\vV} ) + \fFn( \CovM \Vec{\uV} ) - \Sign( \CovM \Vec{\uV} ) \right)
  % )
  \\
  &=
  \ThresholdSet{\Supp( \Vec{\uV} ) \cup \Supp( \Vec{\vV} ) \cup \JCoords} (
  \frac{\sqrt{2\pi}}{\m}
  \sep
  \CovM^{\T}
  \sep
  \frac{1}{2}
  \left( \Sign( \CovM \Vec{\uV} ) - \Sign( \CovM \Vec{\vV} ) \right)
  )
  \\ &\AlignSp
  +
  \ThresholdSet{\Supp( \Vec{\uV} ) \cup ( \Supp( \Vec{\vV} ) \cup \JCoords )} (
  \frac{\sqrt{2\pi}}{\m}
  \sep
  \CovM^{\T}
  \sep
  \frac{1}{2}
  \left( \fFn( \CovM \Vec{\uV} ) - \Sign( \CovM \Vec{\uV} ) \right)
  )
  \\
  &=
  \hFn[\JCoords]( \Vec{\uV}, \Vec{\vV} )
  +
  \hfFn[\Supp( \Vec{\vV} ) \cup \JCoords]( \Vec{\uV}, \Vec{\uV} )
  \\
  &=
  \hFn[\JCoords]( \Vec{\uV}, \Vec{\vV} )
  +
  \hfFn[\Supp( \Vec{\wV} ) \cup \JCoords]( \Vec{\uV}, \Vec{\uV} )
\end{align*}
%|<<|===========================================================================================|<<|
and
%|>>|===========================================================================================|>>|
\begin{align*}
  % &\negphantom{\AlignSp}
  \hFn[\JCoords]( \Vec{\uV}, \Vec{\vV} )
  % \\
  &=
  \ThresholdSet{\Supp( \Vec{\uV} ) \cup \Supp( \Vec{\vV} ) \cup \JCoords} (
  \frac{\sqrt{2\pi}}{\m}
  \sep
  \CovM^{\T}
  \sep
  \frac{1}{2}
  \left( \Sign( \CovM \Vec{\uV} ) - \Sign( \CovM \Vec{\vV} ) \right)
  )
  % \\
  % &=
  % \ThresholdSet{\Supp( \Vec{\uV} ) \cup \Supp( \Vec{\vV} ) \cup \JCoords} (
  % \frac{\sqrt{2\pi}}{\m}
  % \sep
  % \CovM^{\T}
  % \sep
  % \frac{1}{2}
  % \left( \Sign( \CovM \Vec{\uV} ) - \Sign( \CovM \Vec{\wV} ) + \Sign( \CovM \Vec{\wV} ) - \Sign( \CovM \Vec{\vV} ) \right)
  % )
  \\
  &=
  \ThresholdSet{\Supp( \Vec{\uV} ) \cup \Supp( \Vec{\vV} ) \cup \JCoords} (
  \frac{\sqrt{2\pi}}{\m}
  \sep
  \CovM^{\T}
  \sep
  \frac{1}{2}
  \left( \Sign( \CovM \Vec{\uV} ) - \Sign( \CovM \Vec{\wV} ) \right)
  )
  \\ &\AlignSp
  +
  \ThresholdSet{\Supp( \Vec{\uV} ) \cup \Supp( \Vec{\vV} ) \cup \JCoords} (
  \frac{\sqrt{2\pi}}{\m}
  \sep
  \CovM^{\T}
  \sep
  \frac{1}{2}
  \left( \Sign( \CovM \Vec{\wV} ) - \Sign( \CovM \Vec{\vV} ) \right)
  )
  \\
  &=
  \ThresholdSet{\Supp( \Vec{\uV} ) \cup \Supp( \Vec{\wV} ) \cup \JCoords} (
  \frac{\sqrt{2\pi}}{\m}
  \sep
  \CovM^{\T}
  \sep
  \frac{1}{2}
  \left( \Sign( \CovM \Vec{\uV} ) - \Sign( \CovM \Vec{\wV} ) \right)
  )
  \\ &\AlignSp
  +
  \ThresholdSet{\Supp( \Vec{\wV} ) \cup \Supp( \Vec{\vV} ) \cup ( \Supp( \Vec{\uV} ) \cup \JCoords )} (
  \frac{\sqrt{2\pi}}{\m}
  \sep
  \CovM^{\T}
  \sep
  \frac{1}{2}
  \left( \Sign( \CovM \Vec{\wV} ) - \Sign( \CovM \Vec{\vV} ) \right)
  )
  \\
  &=
  \hFn[\JCoords]( \Vec{\uV}, \Vec{\wV} )
  +
  \hFn[\Supp( \Vec{\uV} ) \cup \JCoords]( \Vec{\wV}, \Vec{\vV} )
.\end{align*}
%|<<|===========================================================================================|<<|
Additionally, by \LEMMA \ref{lemma:concentration-ineq},
%|>>|===========================================================================================|>>|
\begin{gather}
  \label{eqn:pf:lemma:combine:2}
  \E[ \hFn[\JCoords]( \Vec{\uV}, \Vec{\vV} ) ]
  =
  \Vec{\uV} - \Vec{\vV}
  ,\\ \label{eqn:pf:lemma:combine:3}
  \E[ \hfFn[\JCoords]( \Vec{\uV}, \Vec{\uV} ) ]
  =
  -\left( 1 - \sqrt{\frac{\pi}{2}} \gammaX \right) \Vec{\uV}
.\end{gather}
%|<<|===========================================================================================|<<|
Thus,
%|>>|===========================================================================================|>>|
\begin{align*}
  \E [ \thetaXX+\hfFn[\JCoords]( \thetaStar, \thetaXX ) ]
  &=
  \thetaXX + \E [ \hfFn[\JCoords]( \thetaStar, \thetaXX ) ]
  \\
  &=
  \thetaXX + \E[ \hFn[\JCoords]( \thetaStar, \thetaXX ) ] + \E[ \hfFn[\JCoords]( \thetaStar, \thetaStar ) ]
  \\
  &\dCmt{by \EQUATION \eqref{eqn:pf:lemma:combine:1a} and the linearity of expectation}
  \\
  &=
  \thetaXX + ( \thetaStar-\thetaXX ) - \left( 1 - \sqrt{\frac{\pi}{2}} \gammaX \right) \thetaStar
  \\
  &\dCmt{by \EQUATIONS \eqref{eqn:pf:lemma:combine:2} and \eqref{eqn:pf:lemma:combine:3}}
  % \\
  % &=
  % \thetaStar - \left( 1 - \sqrt{\frac{\pi}{2}} \gammaX \right) \thetaStar
  \\
  &=
  \sqrt{\frac{\pi}{2}} \gammaX \thetaStar
\TagEqn{\label{eqn:pf:lemma:combine:5}}
.\end{align*}
%|<<|===========================================================================================|<<|
It follows that
%|>>|===========================================================================================|>>|
\begin{align}
\label{eqn:pf:lemma:combine:4}
  \frac
  {\E [ \thetaXX+\hfFn[\JCoords]( \thetaStar, \thetaXX ) ]}
  {\| \E [ \thetaXX+\hfFn[\JCoords]( \thetaStar, \thetaXX ) ] \|_{2}}
  =
  \frac{\sqrt{\hfrac{\pi}{2}} \gammaX \thetaStar}{\sqrt{\hfrac{\pi}{2}} \gammaX}
  =
  \thetaStar
,\end{align}
%|<<|===========================================================================================|<<|
as claimed.
%Moreover, it is clear from \EQUATION \eqref{eqn:pf:lemma:combine:5} that
%%|>>|===========================================================================================|>>|
%\begin{gather}
%\label{eqn:pf:lemma:combine:6}
%  \E \left[ \thetaX+\hfFn[\JCoords]( \thetaStar, \thetaX ) \right]
%  =
%  \E \left[ \thetaXX+\hfFn[\JCoords]( \thetaStar, \thetaXX ) \right]
%.\end{gather}
%%|<<|===========================================================================================|<<|
%Due to \EQUATION \eqref{eqn:pf:lemma:combine:4},
%%|>>|===========================================================================================|>>|
%\begin{align*}
%  \left\| \thetaStar - \frac{\thetaXX+\hfFn[\JCoords]( \thetaStar, \thetaXX )}{\| \thetaXX+\hfFn[\JCoords]( \thetaStar, \thetaXX ) \|_{2}} \right\|_{2}
%  =
%  \left\|
%    \frac{\thetaXX+\hfFn[\JCoords]( \thetaStar, \thetaXX )}{\| \thetaXX+\hfFn[\JCoords]( \thetaStar, \thetaXX ) \|_{2}}
%    -
%    \frac{\E[ \thetaXX+\hfFn[\JCoords]( \thetaStar, \thetaXX ) ]}{\E[ \| \thetaXX+\hfFn[\JCoords]( \thetaStar, \thetaXX ) \|_{2} ]}
%  \right\|_{2}
%.\end{align*}
%%|<<|===========================================================================================|<<|
%Then, applying \FACT \ref{fact:dist-btw-normalized-vectors},
%
Having achieved the first task, the \LHS of the inequality in \EQUATION \eqref{eqn:pf:thm:main-technical:1} can now be bounded from above as follows:
%|>>|===========================================================================================|>>|
\begin{align*}
  &\negphantom{\AlignSp}
  \left\| \thetaStar - \frac{\thetaXX+\hfFn[\JCoords]( \thetaStar, \thetaXX )}{\| \thetaXX+\hfFn[\JCoords]( \thetaStar, \thetaXX ) \|_{2}} \right\|_{2}
  \\
  % &=
  % \left\| \frac{\thetaXX+\hfFn[\JCoords]( \thetaStar, \thetaXX )}{\| \thetaXX+\hfFn[\JCoords]( \thetaStar, \thetaXX ) \|_{2}} - \thetaStar \right\|_{2}
  % \\
  % &\dCmt{negation does not change the norm}
  % \\
  &=
  \left\|
    \frac{\thetaXX+\hfFn[\JCoords]( \thetaStar, \thetaXX )}{\| \thetaXX+\hfFn[\JCoords]( \thetaStar, \thetaXX ) \|_{2}}
    -
    \frac{\E[ \thetaXX+\hfFn[\JCoords]( \thetaStar, \thetaXX ) ]}{\E[ \| \thetaXX+\hfFn[\JCoords]( \thetaStar, \thetaXX ) \|_{2} ]}
  \right\|_{2}
  \\
  &\dCmt{by \EQUATION \eqref{eqn:pf:lemma:combine:4}}
  \\
  &\leq
  \frac{2 \| \thetaXX+\hfFn[\JCoords]( \thetaStar, \thetaXX ) - \E[ \thetaXX+\hfFn[\JCoords]( \thetaStar, \thetaXX ) ] \|_{2}}{\| \E[ \thetaXX+\hfFn[\JCoords]( \thetaStar, \thetaXX ) ] \|_{2}}
  \\
  &\dCmt{by \FACT \ref{fact:dist-btw-normalized-vectors}}
  \\
  % &=
  % \frac{2 \| \thetaXX+\hfFn[\JCoords]( \thetaStar, \thetaXX ) - \thetaXX - \E[ \hfFn[\JCoords]( \thetaStar, \thetaXX ) ] \|_{2}}{\| \E[ \thetaXX+\hfFn[\JCoords]( \thetaStar, \thetaXX ) ] \|_{2}}
  % \\
  % &\dCmt{\(  \thetaXX  \) is nonrandom}
  % \\
  &=
  \frac{2 \| \hfFn[\JCoords]( \thetaStar, \thetaXX ) - \E[ \hfFn[\JCoords]( \thetaStar, \thetaXX ) ] \|_{2}}{\| \E[ \thetaXX+\hfFn[\JCoords]( \thetaStar, \thetaXX ) ] \|_{2}}
  % \\
  % &\dCmt{canceling terms}
  \\
  &=
  \frac
  {2 \| \hFn[\JCoords]( \thetaStar, \thetaXX ) + \hfFn[\Supp( \thetaX ) \cup \JCoords]( \thetaStar, \thetaStar ) - \E[ \hFn[\JCoords]( \thetaStar, \thetaXX ) + \hfFn[\Supp( \thetaX ) \cup \JCoords]( \thetaStar, \thetaStar ) ] \|_{2}}
  {\| \E[ \thetaXX+\hfFn[\JCoords]( \thetaStar, \thetaXX ) ] \|_{2}}
  \\
  &\dCmt{by \EQUATION \eqref{eqn:pf:lemma:combine:1a} and the lemma's condition} % that \(  \Supp( \thetaXX ) \cup \JCoords = \Supp( \thetaX ) \cup \JCoords \) }
  % \\
  % &=
  % \frac
  % {2 \| \hFn[\JCoords]( \thetaStar, \thetaXX ) - \E[ \hFn[\JCoords]( \thetaStar, \thetaXX ) ] + \hfFn[\Supp( \thetaX ) \cup \JCoords]( \thetaStar, \thetaStar ) - \E[ \hfFn[\Supp( \thetaX ) \cup \JCoords]( \thetaStar, \thetaStar ) ] \|_{2}}
  % {\| \E[ \thetaXX+\hfFn[\JCoords]( \thetaStar, \thetaXX ) ] \|_{2}}
  % \\
  % &\dCmt{by the linearity of expectation}
  \\
%   &=
%   \frac
%   {2 \| \hFn[\JCoords]( \thetaStar, \thetaX ) + \hFn[\Supp( \thetaStar ) \cup \JCoords]( \thetaX, \thetaXX ) - \E[ \hFn[\JCoords]( \thetaStar, \thetaX ) + \hFn[\Supp( \thetaStar ) \cup \JCoords]( \thetaX, \thetaXX ) ] + \hfFn[\Supp( \thetaX ) \cup \JCoords]( \thetaStar, \thetaStar ) - \E[ \hfFn[\Supp( \thetaX ) \cup \JCoords]( \thetaStar, \thetaStar ) ] \|_{2}}
%   {\| \E[ \thetaXX+\hfFn[\JCoords]( \thetaStar, \thetaXX ) ] \|_{2}}
%   \\
%   &\dCmt{by \EQUATION \eqref{eqn:pf:lemma:combine:1b} and the lemma's condition} 
% %  \(  \Supp( \thetaXX ) \cup \JCoords = \Supp( \thetaX ) \cup \JCoords  \)}
%   \\
  % &=
  % \frac
  % {2 \| \hFn[\JCoords]( \thetaStar, \thetaX ) - \E[ \hFn[\JCoords]( \thetaStar, \thetaX ) ] + \hFn[\Supp( \thetaStar ) \cup \JCoords]( \thetaX, \thetaXX ) - \E[ \hFn[\Supp( \thetaStar ) \cup \JCoords]( \thetaX, \thetaXX ) ] + \hfFn[\Supp( \thetaX ) \cup \JCoords]( \thetaStar, \thetaStar ) - \E[ \hfFn[\Supp( \thetaX ) \cup \JCoords]( \thetaStar, \thetaStar ) ] \|_{2}}
  % {\| \E[ \thetaXX+\hfFn[\JCoords]( \thetaStar, \thetaXX ) ] \|_{2}}
  % \\
  % &\dCmt{by the linearity of expectation}
  % \\
  &=
  \frac
  {2 \| \hFn[\JCoords]( \thetaStar, \thetaX ) - \E[ \hFn[\JCoords]( \thetaStar, \thetaX ) ] \|_{2}}
  {\| \E[ \thetaXX+\hfFn[\JCoords]( \thetaStar, \thetaXX ) ] \|_{2}}
  +
  \frac
  {2\| \hFn[\Supp( \thetaStar ) \cup \JCoords]( \thetaX, \thetaXX ) - \E[ \hFn[\Supp( \thetaStar ) \cup \JCoords]( \thetaX, \thetaXX ) ] \|_{2}}
  {\| \E[ \thetaXX+\hfFn[\JCoords]( \thetaStar, \thetaXX ) ] \|_{2}}
  \\
  &\AlignSp+
  \frac
  {2 \| \hfFn[\Supp( \thetaX ) \cup \JCoords]( \thetaStar, \thetaStar ) - \E[ \hfFn[\Supp( \thetaX ) \cup \JCoords]( \thetaStar, \thetaStar ) ] \|_{2}}
  {\| \E[ \thetaXX+\hfFn[\JCoords]( \thetaStar, \thetaXX ) ] \|_{2}}
  \\
  &\dCmt{by the linearity of expectation and the triangle inequality}
  \\
  &=
  \frac
  {2 \| \hFn[\JCoords]( \thetaStar, \thetaX ) - \E[ \hFn[\JCoords]( \thetaStar, \thetaX ) ] \|_{2}}
  {\DENOM}
  +
  \frac
  {2 \| \hFn[\Supp( \thetaStar ) \cup \JCoords]( \thetaX, \thetaXX ) - \E[ \hFn[\Supp( \thetaStar ) \cup \JCoords]( \thetaX, \thetaXX ) ] \|_{2}}
  {\DENOM}
  \\
  &\AlignSp+
  \frac
  {2 \| \hfFn[\Supp( \thetaX ) \cup \JCoords]( \thetaStar, \thetaStar ) - \E[ \hfFn[\Supp( \thetaX ) \cup \JCoords]( \thetaStar, \thetaStar ) ] \|_{2}}
  {\DENOM}
  .\\
  &\dCmt{by the first equality in \EQUATION \eqref{eqn:lemma:concentration-ineq:ev:4}}
%  \\
%  &=
%  \frac
%  {2 \| \hFn[\JCoords]( \thetaStar, \thetaX ) - \E[ \hFn[\JCoords]( \thetaStar, \thetaX ) ] \|_{2}}
%  {\DENOM}
%  +
%  \frac
%  {2\| \hfFn[\Supp( \thetaX ) \cup \JCoords]( \thetaStar, \thetaStar ) - \E[ \hfFn[\Supp( \thetaX ) \cup \JCoords]( \thetaStar, \thetaStar ) ] \|_{2}}
%  {\DENOM}
%  \\
%  &\AlignSp+
%  \frac
%  {2 \| \hFn[\Supp( \thetaStar ) \cup \JCoords]( \thetaX, \thetaXX ) - \E[ \hFn[\Supp( \thetaStar ) \cup \JCoords]( \thetaX, \thetaXX ) ] \|_{2}}
%  {\DENOM}
%  .\\
%  &\dCmt{by commutativity}
\end{align*}
%|<<|===========================================================================================|<<|
This completes the proof of \LEMMA \ref{lemma:combine}.
\end{proof}
%|<<|~~~~~~~~~~~~~~~~~~~~~~~~~~~~~~~~~~~~~~~~~~~~~~~~~~~~~~~~~~~~~~~~~~~~~~~~~~~~~~~~~~~~~~~~~~~|<<|
%|<<|~~~~~~~~~~~~~~~~~~~~~~~~~~~~~~~~~~~~~~~~~~~~~~~~~~~~~~~~~~~~~~~~~~~~~~~~~~~~~~~~~~~~~~~~~~~|<<|
%|<<|~~~~~~~~~~~~~~~~~~~~~~~~~~~~~~~~~~~~~~~~~~~~~~~~~~~~~~~~~~~~~~~~~~~~~~~~~~~~~~~~~~~~~~~~~~~|<<|

%|>>|~~~~~~~~~~~~~~~~~~~~~~~~~~~~~~~~~~~~~~~~~~~~~~~~~~~~~~~~~~~~~~~~~~~~~~~~~~~~~~~~~~~~~~~~~~~|>>|
%|>>|~~~~~~~~~~~~~~~~~~~~~~~~~~~~~~~~~~~~~~~~~~~~~~~~~~~~~~~~~~~~~~~~~~~~~~~~~~~~~~~~~~~~~~~~~~~|>>|
%|>>|~~~~~~~~~~~~~~~~~~~~~~~~~~~~~~~~~~~~~~~~~~~~~~~~~~~~~~~~~~~~~~~~~~~~~~~~~~~~~~~~~~~~~~~~~~~|>>|
\begin{proof}
{\LEMMA \ref{lemma:large-dist:1}}
%
\checkoff%
%
Let
%|>>|:::::::::::::::::::::::::::::::::::::::::::::::::::::::::::::::::::::::::::::::::::::::::::|>>|
\(  \thetaStar \in \ParamSpace  \)
%|<<|:::::::::::::::::::::::::::::::::::::::::::::::::::::::::::::::::::::::::::::::::::::::::::|<<|
be arbitrary.
Write
%|>>|:::::::::::::::::::::::::::::::::::::::::::::::::::::::::::::::::::::::::::::::::::::::::::|>>|
\(  \ParamCoverX \defeq \ParamCover \setminus \Ball{\tauX}( \thetaStar )  \).
%|<<|:::::::::::::::::::::::::::::::::::::::::::::::::::::::::::::::::::::::::::::::::::::::::::|<<|
%Initially, fix
For the time being, fix
%|>>|:::::::::::::::::::::::::::::::::::::::::::::::::::::::::::::::::::::::::::::::::::::::::::|>>|
\(  \JCoords \in \JS  \) and \(  \thetaX \in \ParamCoverX  \)
%|<<|:::::::::::::::::::::::::::::::::::::::::::::::::::::::::::::::::::::::::::::::::::::::::::|<<|
arbitrarily---to be varied over all possible choices later---and let
%|>>|:::::::::::::::::::::::::::::::::::::::::::::::::::::::::::::::::::::::::::::::::::::::::::|>>|
\(  \JCoordsX \defeq \Supp( \thetaX ) \cup \JCoords  \).
%|<<|:::::::::::::::::::::::::::::::::::::::::::::::::::::::::::::::::::::::::::::::::::::::::::|<<|
By \EQUATION \eqref{eqn:lemma:concentration-ineq:ev:4} in \LEMMA \ref{lemma:concentration-ineq},
%|>>|===========================================================================================|>>|
\begin{gather}
\label{eqn:pf:lemma:large-dist:1:2}
  \DENOM
  =
  \sqrt{\frac{\pi}{2}} \gammaX
,\end{gather}
%|<<|===========================================================================================|<<|
and thus, substituting \EQUATION \eqref{eqn:pf:lemma:large-dist:1:2} into the \RHS of \EQUATION \eqref{eqn:lemma:large-dist:1:ub} in \LEMMA \ref{lemma:large-dist:1} yields
%|>>|===========================================================================================|>>|
\begin{gather}
\label{eqn:pf:lemma:large-dist:1:3}
  \frac
  {2 \| \hFn[\JCoords]( \thetaStar, \thetaX ) - \E[ \hFn[\JCoords]( \thetaStar, \thetaX ) ] \|_{2}}
  {\DENOM}
  =
  \frac
  {2 \| \hFn[\JCoords]( \thetaStar, \thetaX ) - \E[ \hFn[\JCoords]( \thetaStar, \thetaX ) ] \|_{2}}
  {\sqrt{\hfrac{\pi}{2}} \gammaX}
.\end{gather}
%|<<|===========================================================================================|<<|
Towards bounding the term
%|>>|:::::::::::::::::::::::::::::::::::::::::::::::::::::::::::::::::::::::::::::::::::::::::::|>>|
\({  \| \hFn[\JCoords]( \thetaStar, \thetaX ) - \E[ \hFn[\JCoords]( \thetaStar, \thetaX ) ] \|_{2}  }\)
%|<<|:::::::::::::::::::::::::::::::::::::::::::::::::::::::::::::::::::::::::::::::::::::::::::|<<|
in the numerator on the \RHS of \EQUATION \eqref{eqn:pf:lemma:large-dist:1:3},
consider \EQUATION \eqref{eqn:lemma:concentration-ineq:pr:1} in \LEMMA \ref{lemma:concentration-ineq}, where \(  \sXX, \tX \in (0,1)  \) are taken to be
%|>>|===========================================================================================|>>|
\begin{gather}
\label{eqn:pf:lemma:large-dist:1:s}
  \sXX
  \defeq
  \sqrt{
    \frac{
      3\pi \log \left( \frac{3}{\rhoLDX} | \ParamCover | \right)
    }{
      \m \EDIST
    }
  }
,\end{gather}
%|<<|===========================================================================================|<<|
and
%|>>|===========================================================================================|>>|
\begin{align*}
  \tX
  &\defeq
%  \max \left\{
%  \sqrt{
%    \frac{27\pi \log \left( \frac{12}{\rhoLDX} | \JS | | \ParamCover | \right)}
%         {\m \EDIST}}
%  ,
%  \sqrt{
%    \frac{2\pi ( 1+\sXX ) \log \left( \frac{3}{\rhoLDX} | \JS | | \ParamCover | \right)}
%         {\m \EDIST}
%  }
%  \right\}
%  \\
%  &\leq
%  \max \left\{
%  \sqrt{
%    \frac{27\pi \log \left( \frac{12}{\rhoLDX} | \JS | | \ParamCover | \right)}
%         {\m \EDIST}}
%  ,
%  \sqrt{
%    \frac{4\pi \log \left( \frac{3}{\rhoLDX} | \JS | | \ParamCover | \right)}
%         {\m \EDIST}
%  }
%  \right\}
%  \\
%  &=
  \sqrt{
    \frac{27\pi \log \left( \frac{12}{\rhoLDX} | \JS | | \ParamCover | \right)}
         {\m \EDIST}}
\TagEqn{\label{eqn:pf:lemma:large-dist:1:t}}
,\end{align*}
%|<<|===========================================================================================|<<|
and where \(  \m  \) is at least
%|>>|===========================================================================================|>>|
\begin{align}
\label{eqn:pf:lemma:large-dist:1:m}
  \m
  \geq
  \frac{16}{\GAMMAX^{2} \deltaX}
  \max \left\{
    27\pi \log \left( \frac{12}{\rhoLDX} | \JS | | \ParamCover | \right)
    ,
    4 ( \kO-2 )
  \right\}
  % \frac{8}{\pi \GAMMAX^{2} \deltaX}
  % \max \left\{
  %   54\pi^{2} \log \left( \frac{12}{\rhoLDX} | \JS | | \ParamCover | \right)
  %   ,
  %   8\pi ( \kO-2 )
  % \right\}
.\end{align}
%|<<|===========================================================================================|<<|
This bound on \(  \m  \) suffices to ensure that under the lemma's condition that
%|>>|:::::::::::::::::::::::::::::::::::::::::::::::::::::::::::::::::::::::::::::::::::::::::::|>>|
\(  \thetaX \in \ParamCoverX = \ParamCover \setminus \Ball{\tauX}( \thetaStar )  \),
%|<<|:::::::::::::::::::::::::::::::::::::::::::::::::::::::::::::::::::::::::::::::::::::::::::|<<|
the variable \(  \sXX  \) satisfies the requirement that
%|>>|:::::::::::::::::::::::::::::::::::::::::::::::::::::::::::::::::::::::::::::::::::::::::::|>>|
\(  \sXX < 1  \),
%|<<|:::::::::::::::::::::::::::::::::::::::::::::::::::::::::::::::::::::::::::::::::::::::::::|<<|
which further implies that
%|>>|:::::::::::::::::::::::::::::::::::::::::::::::::::::::::::::::::::::::::::::::::::::::::::|>>|
\(  1+\sXX < 2  \).
%|<<|:::::::::::::::::::::::::::::::::::::::::::::::::::::::::::::::::::::::::::::::::::::::::::|<<|
Hence also,
%|>>|===========================================================================================|>>|
\begin{align*}
  \tX
  &=
  \sqrt{
    \frac{27\pi \log \left( \frac{12}{\rhoLDX} | \JS | | \ParamCover | \right)}
         {\m \EDIST}}
  \\
  &=
  \max \left\{
  \sqrt{
    \frac{27\pi \log \left( \frac{12}{\rhoLDX} | \JS | | \ParamCover | \right)}
         {\m \EDIST}}
  ,
  \sqrt{
    \frac{4\pi \log \left( \frac{3}{\rhoLDX} | \JS | | \ParamCover | \right)}
         {\m \EDIST}
  }
  \right\}
  \\
  &=
  \max \left\{
  \sqrt{
    \frac{27\pi \log \left( \frac{12}{\rhoLDX} | \JS | | \ParamCover | \right)}
         {\m \EDIST}}
  ,
  \sqrt{
    \frac{2\pi ( 1+\sXX ) \log \left( \frac{3}{\rhoLDX} | \JS | | \ParamCover | \right)}
         {\m \EDIST}
  }
  \right\}
\TagEqn{\label{eqn:pf:lemma:large-dist:1:t:b}}
.\end{align*}
%|<<|===========================================================================================|<<|
Moreover, with these choices,
%|>>|===========================================================================================|>>|
\begin{align*}
  & \negphantom{\AlignSp}
  \sqrt{\frac{\pi ( 1+\sXX )( \kO-2 ) \EDIST}{\m}}
  +
  \sqrt{\frac{\pi}{2}} \tX \EDIST
  \\
  &\leq
  \frac{1}{2} \cdot \sqrt{\frac{\pi}{8}} \gammaX \sqrt{\deltaX \EDIST}
  +
  \frac{1}{2} \cdot \sqrt{\frac{\pi}{8}} \gammaX \sqrt{\deltaX \EDIST}
  \\
  &=
  \sqrt{\frac{\pi}{8}} \gammaX \sqrt{\deltaX \EDIST}
\TagEqn{\label{eqn:pf:lemma:large-dist:1:4}}
.\end{align*}
%|<<|===========================================================================================|<<|
%
%%%%%%%%%%%%%%%%%%%%%%%%%%%%%%%%%%%%%%%%%%%%%%%%%%%%%%%%%%%%%%%%%%%%%%%%%%%%%%%%%%%%%%%%%%%%%%%%%%%%
\par %%%%%%%%%%%%%%%%%%%%%%%%%%%%%%%%%%%%%%%%%%%%%%%%%%%%%%%%%%%%%%%%%%%%%%%%%%%%%%%%%%%%%%%%%%%%%%%
%%%%%%%%%%%%%%%%%%%%%%%%%%%%%%%%%%%%%%%%%%%%%%%%%%%%%%%%%%%%%%%%%%%%%%%%%%%%%%%%%%%%%%%%%%%%%%%%%%%%
%
For any random variable \(  U  \) taking values in \(  \Set{S} \subseteq \R  \), and for values \(  u \leq u' \in \Set{S}  \), the event that \(  U \leq u  \) implies \(  U \leq u'  \), and therefore,
%|>>|===========================================================================================|>>|
\begin{gather}
\label{eqn:pf:lemma:large-dist:1:5}
  \Pr( U \leq u ) \leq \Pr( U \leq u' )
  ,\quad
  u \leq u'
.\end{gather}
%|<<|===========================================================================================|<<|
Combining these observations with \EQUATION \eqref{eqn:lemma:concentration-ineq:pr:1} in \LEMMA \ref{lemma:concentration-ineq} yields:
%|>>|===========================================================================================|>>|
\begin{align*}
  & \textstyle
  \Pr \left(
    \Forall{\JCoords \in \JS, \thetaX \in \ParamCoverX}{
    \left\| \hFn[\JCoords]( \thetaStar, \thetaX ) - \E \left[ \hFn[\JCoords]( \thetaStar, \thetaX ) \right] \right\|_{2}
    \leq
    \sqrt{\frac{\pi}{8}} \gammaX \sqrt{\deltaX \EDIST}
    }
  \right)
  \\
  & \textstyle \geq
  \Pr \left(
    \Forall{\JCoords \in \JS, \thetaX \in \ParamCoverX}{
    \left\| \hFn[\JCoords]( \thetaStar, \thetaX ) - \E \left[ \hFn[\JCoords]( \thetaStar, \thetaX ) \right] \right\|_{2}
    \leq
    \sqrt{\frac{\pi ( 1+\sXX )( \kO-2 ) \EDIST}{\m}}
    +
    \sqrt{\frac{\pi}{2}} \tX \EDIST
    }
  \right)
  \\
  &\dCmt{by \EQUATIONS \eqref{eqn:pf:lemma:large-dist:1:4} and \eqref{eqn:pf:lemma:large-dist:1:5}}
  \\
  & \textstyle \geq
  \Pr \left(
    \Forall{\JCoords \in \JS, \thetaX \in \ParamCoverX}{
    \left\| \hFn[\JCoords]( \thetaStar, \thetaX ) - \E \left[ \hFn[\JCoords]( \thetaStar, \thetaX ) \right] \right\|_{2}
    \leq
    \sqrt{\frac{2 ( 1+\sXX )( \kO-2 ) \ADIST}{\m}}
    +
    \sqrt{\frac{2}{\pi}} \tX \ADIST
    }
  \right)
  \\
  &\dCmt{\(  \ADIST \leq \frac{\pi}{2} \EDIST  \) by \FACT \ref{fact:d_angular-d_S}}
  \\
  &\geq
  1
  -
  4 | \JS | | \ParamCoverX | e^{-\frac{1}{27\pi} \m \tX^{2} \ADIST}
  -
  | \JS | | \ParamCoverX | e^{-\frac{1}{2\pi ( 1+\sXX )} \m \tX^{2} \ADIST}
  -
  | \ParamCoverX | e^{-\frac{1}{3\pi} \m \sXX^{2} \ADIST}
  \\
  &\dCmt{by \EQUATION \eqref{eqn:lemma:concentration-ineq:pr:1}}
  \\
  &\geq
  1
  -
  4 | \JS | | \ParamCoverX | e^{-\frac{1}{27\pi} \m \tX^{2} \EDIST}
  -
  | \JS | | \ParamCoverX | e^{-\frac{1}{2\pi ( 1+\sXX )} \m \tX^{2} \EDIST}
  -
  | \ParamCoverX | e^{-\frac{1}{3\pi} \m \sXX^{2} \EDIST}
  \\
  &\dCmt{\(  \ADIST \geq \EDIST  \) due to \FACT \ref{fact:d_angular-d_S}}
  \\
  &\geq
  1 - \frac{\rhoLDX}{3} - \frac{\rhoLDX}{3} - \frac{\rhoLDX}{3}
  \\
  &\dCmt{by the choice of \(  \sX, \tX  \) in \EQUATIONS \eqref{eqn:pf:lemma:large-dist:1:s} and \eqref{eqn:pf:lemma:large-dist:1:t}, respectively, and by \EQUATION \eqref{eqn:pf:lemma:large-dist:1:t:b}}
  \\
  &=
  1 - \rhoLDX
%  \\
%  &\geq
%  1
%  -
%  4 | \JS | | \ParamCoverX | e^{-\frac{1}{27\pi} \tauX \m \tX^{2}}
%  -
%  2 | \JS | | \ParamCoverX | e^{-\frac{\tauX \m \tX^{2}}{144 \pi ( 1+\sXX ) ( \kO-2 )}}
%  -
%  | \ParamCoverX | e^{-\frac{1}{3\pi} \tauX \m \sXX^{2}}
%  \\
%  &\dCmt{\(  \because \thetaX \in \ParamCoverX = \ParamCover \setminus \Ball{\tauX}( \thetaStar )  \) implies \(  \EDIST \geq \tauX  \)}
\TagEqn{\label{eqn:pf:lemma:large-dist:1:7}}
.\end{align*}
%|<<|===========================================================================================|<<|
%\ToDo{Check if the constant inside the log for the expression for \(  \sX  \) should be 3 instead of 6. (I changed it from 6 to 3, but need to double-check that that is correct.)}
Returning to \EQUATION \eqref{eqn:pf:lemma:large-dist:1:3} and inserting \EQUATION \eqref{eqn:pf:lemma:large-dist:1:7}, it follows that if \(  \m  \) satisfies \EQUATION \eqref{eqn:pf:lemma:large-dist:1:m}, then with probability at least \(  1-\rhoLDX  \), for all \(  \JCoords \in \JS  \) and \(  \thetaX \in \ParamCoverX  \),
%|>>|===========================================================================================|>>|
\begin{align*}
  \frac
  {2 \| \hFn[\JCoords]( \thetaStar, \thetaX ) - \E[ \hFn[\JCoords]( \thetaStar, \thetaX ) ] \|_{2}}
  {\DENOM}
  &=
  \frac
  {2 \| \hFn[\JCoords]( \thetaStar, \thetaX ) - \E[ \hFn[\JCoords]( \thetaStar, \thetaX ) ] \|_{2}}
  {\sqrt{\hfrac{\pi}{2}} \gammaX}
  \\
  &\leq
  \sqrt{\frac{8}{\pi}} \frac{1}{\gammaX}
  \sqrt{\frac{\pi}{8}} \gammaX
  \sqrt{\deltaX \EDIST}
  \\
  &=
  \sqrt{\deltaX \EDIST}
,\end{align*}
%|<<|===========================================================================================|<<|
as desired.
\end{proof}
%|<<|~~~~~~~~~~~~~~~~~~~~~~~~~~~~~~~~~~~~~~~~~~~~~~~~~~~~~~~~~~~~~~~~~~~~~~~~~~~~~~~~~~~~~~~~~~~|<<|
%|<<|~~~~~~~~~~~~~~~~~~~~~~~~~~~~~~~~~~~~~~~~~~~~~~~~~~~~~~~~~~~~~~~~~~~~~~~~~~~~~~~~~~~~~~~~~~~|<<|
%|<<|~~~~~~~~~~~~~~~~~~~~~~~~~~~~~~~~~~~~~~~~~~~~~~~~~~~~~~~~~~~~~~~~~~~~~~~~~~~~~~~~~~~~~~~~~~~|<<|
%|>>|~~~~~~~~~~~~~~~~~~~~~~~~~~~~~~~~~~~~~~~~~~~~~~~~~~~~~~~~~~~~~~~~~~~~~~~~~~~~~~~~~~~~~~~~~~~|>>|
%|>>|~~~~~~~~~~~~~~~~~~~~~~~~~~~~~~~~~~~~~~~~~~~~~~~~~~~~~~~~~~~~~~~~~~~~~~~~~~~~~~~~~~~~~~~~~~~|>>|
%|>>|~~~~~~~~~~~~~~~~~~~~~~~~~~~~~~~~~~~~~~~~~~~~~~~~~~~~~~~~~~~~~~~~~~~~~~~~~~~~~~~~~~~~~~~~~~~|>>|
\begin{proof}
{\LEMMA \ref{lemma:small-dist}}
%
\mostlycheckoff%
%
Take any \(  \thetaStar \in \ParamSpace  \), and write
%|>>|:::::::::::::::::::::::::::::::::::::::::::::::::::::::::::::::::::::::::::::::::::::::::::|>>|
\(  \ParamCoverX \defeq \ParamCover \setminus \Ball{\tauX}( \thetaStar )  \) and
\(  \JSXX \defeq \{ \Supp( \thetaStar ) \cup \JCoords : \JCoords \in \JS \}  \),
%|<<|:::::::::::::::::::::::::::::::::::::::::::::::::::::::::::::::::::::::::::::::::::::::::::|<<|
where
%|>>|:::::::::::::::::::::::::::::::::::::::::::::::::::::::::::::::::::::::::::::::::::::::::::|>>|
\(  \JS \subseteq 2^{[\n]}  \)
%|<<|:::::::::::::::::::::::::::::::::::::::::::::::::::::::::::::::::::::::::::::::::::::::::::|<<|
is arbitrary.
Let
%|>>|:::::::::::::::::::::::::::::::::::::::::::::::::::::::::::::::::::::::::::::::::::::::::::|>>|
\(  \thetaX \in \ParamCoverX  \),
\(  \thetaXX \in \BallXX{2\tauX}( \thetaX )  \), and
\(  \JCoordsXX \in \JSXX  \)
%|<<|:::::::::::::::::::::::::::::::::::::::::::::::::::::::::::::::::::::::::::::::::::::::::::|<<|
be arbitrary.
%Similarly to the proof of \LEMMA \ref{lemma:concentration-ineq:noiseless}, write
Write
%|>>|:::::::::::::::::::::::::::::::::::::::::::::::::::::::::::::::::::::::::::::::::::::::::::|>>|
\(  \CovVX\VIx{\iIx} \defeq
    \ThresholdSet{\Supp( \thetaX ) \cup \JCoordsXX}( \CovV\VIx{\iIx} ) \in \R^{\n}  \),
%|<<|:::::::::::::::::::::::::::::::::::::::::::::::::::::::::::::::::::::::::::::::::::::::::::|<<|
\(  \iIx \in [\m]  \), and let
%|>>|:::::::::::::::::::::::::::::::::::::::::::::::::::::::::::::::::::::::::::::::::::::::::::|>>|
\(  \CovMX \defeq ( \CovVX\VIx{1} \,\cdots\, \CovVX\VIx{\m} )^{\T} \in \R^{\m \times \n}  \).
%|<<|:::::::::::::::::::::::::::::::::::::::::::::::::::::::::::::::::::::::::::::::::::::::::::|<<|
Using these notations, \(  \frac{1}{\sqrt{2\pi}} \hFn[\JCoordsXX]( \thetaX, \thetaXX )  \) can be expressed as follows:
%|>>|===========================================================================================|>>|
\begin{align*}
  \frac{1}{\sqrt{2\pi}} \hFn[\JCoordsXX]( \thetaX, \thetaXX )
  &=
  \ThresholdSet{\Supp( \thetaX ) \cup \JCoordsXX} \left(
    \frac{1}{m}
    \sum_{\iIx=1}^{\m}
    \CovV\VIx{\iIx}
    \sep \frac{1}{2} \left( \Sign*( \langle \CovV\VIx{\iIx}, \thetaX \rangle ) - \Sign*( \langle \CovV\VIx{\iIx}, \thetaXX \rangle ) \right)
  \right)
  \\
  &=
  \frac{1}{m}
  \sum_{\iIx=1}^{\m}
  \CovVX\VIx{\iIx}
  \sep \frac{1}{2} \left( \Sign*( \langle \CovVX\VIx{\iIx}, \thetaX \rangle ) - \Sign*( \langle \CovVX\VIx{\iIx}, \thetaXX \rangle ) \right)
  \\
  &=
  \frac{1}{m}
  \sum_{\iIx=1}^{\m}
  \CovVX\VIx{\iIx}
  \sep \Sign*( \langle \CovVX\VIx{\iIx}, \thetaX \rangle )
  \sep \I( \Sign*( \langle \CovVX\VIx{\iIx}, \thetaX \rangle ) \neq \Sign*( \langle \CovVX\VIx{\iIx}, \thetaXX \rangle ) )
  \\
  &=
  \frac{1}{m}
  \CovMX^{\T}
  \Diag( \Sign*( \CovMX \thetaX ) )
  \sep \I( \Sign*( \CovMX \thetaX ) \neq \Sign*( \CovMX \thetaXX ) )
.\end{align*}
%|<<|===========================================================================================|<<|
It is clear from the last line that after fixing the covariates, \(  \CovVX\VIx{\iIx}  \), \(  \iIx \in [\m]  \), the function \(  \hFn[\JCoordsXX]  \) can only take finitely many values.
Moreover, upon additionally fixing \(  \thetaX \in \ParamSpace  \), the finitely many values that can be taken by the function \(  \hFn[\JCoordsXX]( \thetaX, \cdot )  \) is determined by the number of values that can be taken by
%|>>|:::::::::::::::::::::::::::::::::::::::::::::::::::::::::::::::::::::::::::::::::::::::::::|>>|
\(  \I( \Sign*( \CovMX \thetaX ) \neq \Sign*( \CovMX \thetaXX ) ) \in \{ 0,1 \}^{\m}  \)
%|<<|:::::::::::::::::::::::::::::::::::::::::::::::::::::::::::::::::::::::::::::::::::::::::::|<<|
over all choices of \(  \thetaXX \in \BallXX{2\tauX}( \thetaX )  \).
As such, write
%|>>|:::::::::::::::::::::::::::::::::::::::::::::::::::::::::::::::::::::::::::::::::::::::::::|>>|
\(  \WS{\thetaX} \defeq \{ \I( \Sign*( \CovMX \thetaX ) \neq \Sign*( \CovMX \thetaXX ) ) : \thetaXX \in \BallXX{2\tauX}( \thetaX ) \}  \)
%|<<|:::::::::::::::::::::::::::::::::::::::::::::::::::::::::::::::::::::::::::::::::::::::::::|<<|
for \(  \thetaX \in \ParamSpace  \).
In addition, for \(  \thetaX, \thetaXX \in \ParamSpace  \), define
%|>>|:::::::::::::::::::::::::::::::::::::::::::::::::::::::::::::::::::::::::::::::::::::::::::|>>|
\(  \LRVXX{\thetaX}{\thetaXX} \defeq \| \I( \Sign( \CovM \thetaX ) \neq \Sign( \CovM \thetaXX ) ) \|_{0}  \),
%\(  \LRVXX{\thetaX}{\thetaXX} \defeq | \{ \iIx \in [\m] : \Sign*( \langle \CovV\VIx{\iIx}, \thetaX \rangle ) \neq \Sign*( \langle \CovV\VIx{\iIx}, \thetaXX \rangle ) \} |  \),
%|<<|:::::::::::::::::::::::::::::::::::::::::::::::::::::::::::::::::::::::::::::::::::::::::::|<<|
and let
%|>>|:::::::::::::::::::::::::::::::::::::::::::::::::::::::::::::::::::::::::::::::::::::::::::|>>|
\(  \LRVX{\thetaX} \defeq \sup_{\thetaX' \in \BallXX{2\tauX}( \thetaX )} \LRVXX{\thetaX}{\thetaX'}  \).
%|<<|:::::::::::::::::::::::::::::::::::::::::::::::::::::::::::::::::::::::::::::::::::::::::::|<<|
While there is a na\"{i}ve upper bound of
%|>>|:::::::::::::::::::::::::::::::::::::::::::::::::::::::::::::::::::::::::::::::::::::::::::|>>|
\(  | \WS{\thetaX} | \leq 2^{\m}  \),
%|<<|:::::::::::::::::::::::::::::::::::::::::::::::::::::::::::::::::::::::::::::::::::::::::::|<<|
it turns out that a little more nuance admits a tighter bound on the cardinality of \(  \WS{\thetaX}  \) by means of the random variable \(  \LRVX{\thetaX}  \) and the following lemma.
%and that
%%|>>|===========================================================================================|>>|
%\begin{align*}
%  \left\langle \frac{1}{\sqrt{2\pi}} \hFn[\JCoordsXX]( \thetaX, \thetaXX ), \frac{\thetaX-\thetaXX}{\| \thetaX-\thetaXX \|_{2}} \right\rangle
%  &=
%  \frac{1}{m}
%  \sum_{\iIx=1}^{\m}
%  \left\langle \CovVX\VIx{\iIx}, \frac{\thetaX-\thetaXX}{\| \thetaX-\thetaXX \|_{2}} \right\rangle
%  \Sign*( \langle \CovVX\VIx{\iIx}, \thetaX \rangle )
%  \I( \Sign*( \langle \CovVX\VIx{\iIx}, \thetaX \rangle ) \neq \Sign*( \langle \CovVX\VIx{\iIx}, \thetaXX \rangle ) )
%  \\
%  &=
%  \frac{1}{m}
%  \sum_{\iIx=1}^{\m}
%  \left| \left\langle \CovVX\VIx{\iIx}, \frac{\thetaX-\thetaXX}{\| \thetaX-\thetaXX \|_{2}} \right\rangle \right|
%  \I( \Sign*( \langle \CovVX\VIx{\iIx}, \thetaX \rangle ) \neq \Sign*( \langle \CovVX\VIx{\iIx}, \thetaXX \rangle ) )
%.\end{align*}
%%|<<|===========================================================================================|<<|
%%Let
%%%|>>|:::::::::::::::::::::::::::::::::::::::::::::::::::::::::::::::::::::::::::::::::::::::::::|>>|
%%\(  \Zi[1], \dots, \Zi[\m] \sim \N(0,1)  \)
%%%|<<|:::::::::::::::::::::::::::::::::::::::::::::::::::::::::::::::::::::::::::::::::::::::::::|<<|
%%be \iid standard univariate Gaussians, and let
%%%|>>|:::::::::::::::::::::::::::::::::::::::::::::::::::::::::::::::::::::::::::::::::::::::::::|>>|
%%\(  \ZXi[1], \dots, \ZXi[\m] \sim \N(0,1)  \)
%%%|<<|:::::::::::::::::::::::::::::::::::::::::::::::::::::::::::::::::::::::::::::::::::::::::::|<<|
%%also be \iid standard univariate Gaussians, where for each \(  \iIx \in [\m]  \),
%%%|>>|===========================================================================================|>>|
%%\begin{gather*}
%%  \begin{pmatrix}
%%    \Zi \\
%%    \ZXi
%%  \end{pmatrix}
%%  \sim
%%  \N \left(
%%    \begin{pmatrix}
%%      0 \\
%%      0
%%    \end{pmatrix},
%%    \begin{pmatrix}
%%      1 & \langle \thetaX, \thetaXX \rangle \\
%%      \langle \thetaX, \thetaXX \rangle & 1
%%    \end{pmatrix}
%%  \right)
%%.\end{gather*}
%%%|<<|===========================================================================================|<<|
%%For \(  \iIx \in [\m]  \), let
%%%|>>|:::::::::::::::::::::::::::::::::::::::::::::::::::::::::::::::::::::::::::::::::::::::::::|>>|
%%\(  \Ui \defeq \left| \Zi \right| \I( \Sign*( \Zi ) \neq \Sign*( \ZXi ) )  \),
%%%|<<|:::::::::::::::::::::::::::::::::::::::::::::::::::::::::::::::::::::::::::::::::::::::::::|<<|
%%and let
%%%|>>|:::::::::::::::::::::::::::::::::::::::::::::::::::::::::::::::::::::::::::::::::::::::::::|>>|
%%\(  \URV \defeq \frac{1}{\m} \sum_{\iIx=1}^{\m} \Ui  \).
%%%|<<|:::::::::::::::::::::::::::::::::::::::::::::::::::::::::::::::::::::::::::::::::::::::::::|<<|
%%Additionally, define
%%%|>>|:::::::::::::::::::::::::::::::::::::::::::::::::::::::::::::::::::::::::::::::::::::::::::|>>|
%%\(  \Ri \defeq \I( \Sign*( \Zi ) \neq \Sign*( \ZXi ) )  \)
%%%|<<|:::::::::::::::::::::::::::::::::::::::::::::::::::::::::::::::::::::::::::::::::::::::::::|<<|
%%and
%%%|>>|:::::::::::::::::::::::::::::::::::::::::::::::::::::::::::::::::::::::::::::::::::::::::::|>>|
%%\(  \Vec{\RRV} \defeq ( \Ri[1], \dots, \Ri[\m] )  \).
%%%|<<|:::::::::::::::::::::::::::::::::::::::::::::::::::::::::::::::::::::::::::::::::::::::::::|<<|
%%Then,
%%%|>>|===========================================================================================|>>|
%%\begin{gather*}
%%  \left\langle \frac{1}{\sqrt{2\pi}} \hFn[\JCoordsXX]( \thetaX, \thetaXX ), \thetaX \right\rangle
%%  \sim
%%  \URV
%%,\end{gather*}
%%%|<<|===========================================================================================|<<|
%%and hence, the following argument will focus on the random variables \(  \URV  \) and \(  \Ui  \), \(  \iIx \in [\m]  \).
%For \(  \iIx \in [\m]  \), let
%%|>>|:::::::::::::::::::::::::::::::::::::::::::::::::::::::::::::::::::::::::::::::::::::::::::|>>|
%\(  \Ui \defeq \left| \left\langle \CovVX\VIx{\iIx}, \frac{\thetaX-\thetaXX}{\| \thetaX-\thetaXX \|_{2}} \right\rangle \right| \I( \Sign*( \langle \CovVX\VIx{\iIx}, \thetaX \rangle ) \neq \Sign*( \langle \CovVX\VIx{\iIx}, \thetaXX \rangle ) )  \),
%%|<<|:::::::::::::::::::::::::::::::::::::::::::::::::::::::::::::::::::::::::::::::::::::::::::|<<|
%and let
%%|>>|:::::::::::::::::::::::::::::::::::::::::::::::::::::::::::::::::::::::::::::::::::::::::::|>>|
%\(  \URV \defeq \frac{1}{\m} \sum_{\iIx=1}^{\m} \Ui  \).
%%|<<|:::::::::::::::::::::::::::::::::::::::::::::::::::::::::::::::::::::::::::::::::::::::::::|<<|
%Additionally, define
%%|>>|:::::::::::::::::::::::::::::::::::::::::::::::::::::::::::::::::::::::::::::::::::::::::::|>>|
%\(  \Ri \defeq \I( \Sign*( \langle \CovVX\VIx{\iIx}, \thetaX \rangle ) \neq \Sign*( \langle \CovVX\VIx{\iIx}, \thetaXX \rangle ) )  \)
%%|<<|:::::::::::::::::::::::::::::::::::::::::::::::::::::::::::::::::::::::::::::::::::::::::::|<<|
%and
%%|>>|:::::::::::::::::::::::::::::::::::::::::::::::::::::::::::::::::::::::::::::::::::::::::::|>>|
%\(  \Vec{\RRV} \defeq ( \Ri[1], \dots, \Ri[\m] )  \).
%%|<<|:::::::::::::::::::::::::::::::::::::::::::::::::::::::::::::::::::::::::::::::::::::::::::|<<|
%%
%%|>>|*******************************************************************************************|>>|
%%|>>|*******************************************************************************************|>>|
%%|>>|*******************************************************************************************|>>|
%\begin{lemma}[{\COROLLARY to \cite[{\COROLLARY 3.3}]{oymak2015near}}]
%\label{claim:pf:lemma:concentration-ineq:noiseless:small:mismatches:dense}
%%
%Suppose
%%|>>|:::::::::::::::::::::::::::::::::::::::::::::::::::::::::::::::::::::::::::::::::::::::::::|>>|
%\(  \ParamSpace = \Sphere{\n}  \).
%%|<<|:::::::::::::::::::::::::::::::::::::::::::::::::::::::::::::::::::::::::::::::::::::::::::|<<|
%Let
%%|>>|:::::::::::::::::::::::::::::::::::::::::::::::::::::::::::::::::::::::::::::::::::::::::::|>>|
%\(  \LRVX{\thetaX} \defeq \sup_{\thetaX' \in \ParamSpace} | \{ \iIx \in [\m] : \exists \thetaX'' \in \Ball{\tauX}( \thetaX' ) \ \st \ \Sign*( \langle \CovV\VIx{\iIx}, \thetaX' \rangle ) \neq \Sign*( \langle \CovV\VIx{\iIx}, \thetaX'' \rangle ) \} |  \).
%%|<<|:::::::::::::::::::::::::::::::::::::::::::::::::::::::::::::::::::::::::::::::::::::::::::|<<|
%If
%%|>>|===========================================================================================|>>|
%\begin{gather}
%  \m
%  \geq
%  \ToDo{}
%,\end{gather}
%%|<<|===========================================================================================|<<|
%then with probability at least \(  1 - \ToDo{}  \),
%%|>>|:::::::::::::::::::::::::::::::::::::::::::::::::::::::::::::::::::::::::::::::::::::::::::|>>|
%\(  \LRVX{\thetaX} \leq \ToDo{\(  \nuX \m  \)}  \).
%%|<<|:::::::::::::::::::::::::::::::::::::::::::::::::::::::::::::::::::::::::::::::::::::::::::|<<|
%\end{lemma}
%%|<<|*******************************************************************************************|<<|
%%|<<|*******************************************************************************************|<<|
%%|<<|*******************************************************************************************|<<|
%%
%%|>>|*******************************************************************************************|>>|
%%|>>|*******************************************************************************************|>>|
%%|>>|*******************************************************************************************|>>|
%\begin{lemma}[{\COROLLARY to \cite[{\COROLLARY 3.3}]{oymak2015near}}]
%\label{claim:pf:lemma:concentration-ineq:noiseless:small:mismatches:sparse}
%%
%Suppose
%%|>>|:::::::::::::::::::::::::::::::::::::::::::::::::::::::::::::::::::::::::::::::::::::::::::|>>|
%\(  \ParamSpace = \SparseSphereSubspace{\k}{\n}  \).
%%|<<|:::::::::::::::::::::::::::::::::::::::::::::::::::::::::::::::::::::::::::::::::::::::::::|<<|
%Let
%%|>>|:::::::::::::::::::::::::::::::::::::::::::::::::::::::::::::::::::::::::::::::::::::::::::|>>|
%\(  \LRVX{\thetaX} \defeq \sup_{\thetaX' \in \ParamSpace} | \{ \iIx \in [\m] : \exists \thetaX'' \in \BallX{\tauX}( \thetaX' ) \ \st \ \Sign*( \langle \CovV\VIx{\iIx}, \thetaX' \rangle ) \neq \Sign*( \langle \CovV\VIx{\iIx}, \thetaX'' \rangle ) \} |  \).
%%|<<|:::::::::::::::::::::::::::::::::::::::::::::::::::::::::::::::::::::::::::::::::::::::::::|<<|
%If
%%|>>|===========================================================================================|>>|
%\begin{gather}
%  \m
%  \geq
%  \ToDo{Fill this in. Remember to union bound over all \(  \k  \)-subsets of support.}
%,\end{gather}
%%|<<|===========================================================================================|<<|
%then with probability at least \(  1 - \ToDo{}  \),
%%|>>|:::::::::::::::::::::::::::::::::::::::::::::::::::::::::::::::::::::::::::::::::::::::::::|>>|
%\(  \LRVX{\thetaX} \leq \ToDo{\(  \nuX \m  \)}  \).
%%|<<|:::::::::::::::::::::::::::::::::::::::::::::::::::::::::::::::::::::::::::::::::::::::::::|<<|
%\end{lemma}
%%|<<|*******************************************************************************************|<<|
%%|<<|*******************************************************************************************|<<|
%%|<<|*******************************************************************************************|<<|
%
%|>>|*******************************************************************************************|>>|
%|>>|*******************************************************************************************|>>|
%|>>|*******************************************************************************************|>>|
\begin{lemma}[{\COROLLARY to \cite[{\COROLLARY 3.3}]{oymak2015near}}]
\label{lemma:pf:lemma:concentration-ineq:noiseless:small:mismatches}
%
Let
%|>>|:::::::::::::::::::::::::::::::::::::::::::::::::::::::::::::::::::::::::::::::::::::::::::|>>|
\(  \ConstdSD > 0  \)
%|<<|:::::::::::::::::::::::::::::::::::::::::::::::::::::::::::::::::::::::::::::::::::::::::::|<<|
be the constant defined in \DEFINITION \ref{def:univ-const}.
%Let
%%|>>|:::::::::::::::::::::::::::::::::::::::::::::::::::::::::::::::::::::::::::::::::::::::::::|>>|
%\(  \JSXXX \defeq \{ \JCoordsXXX \subseteq [\n] : | \JCoordsXXX | = \nO \}  \),
%%|<<|:::::::::::::::::::::::::::::::::::::::::::::::::::::::::::::::::::::::::::::::::::::::::::|<<|
%where
%%|>>|:::::::::::::::::::::::::::::::::::::::::::::::::::::::::::::::::::::::::::::::::::::::::::|>>|
%\(  | \JSXXX | = \binom{\n}{\nO}  \).
%%|<<|:::::::::::::::::::::::::::::::::::::::::::::::::::::::::::::::::::::::::::::::::::::::::::|<<|
If
%|>>|===========================================================================================|>>|
\begin{gather}
  \m
  \geq
  \frac{\ConstdSD \nO}{\nuX} \log \left( \frac{1}{\nuX} \right)
,\end{gather}
%|<<|===========================================================================================|<<|
then with probability at least \(  1 - \binom{\n}{\nO} e^{-\frac{1}{64} \nuX \m}  \), the random variable \(  \LRVX{\thetaX}  \) is bounded from above by
%|>>|:::::::::::::::::::::::::::::::::::::::::::::::::::::::::::::::::::::::::::::::::::::::::::|>>|
\(  \LRVX{\thetaX} \leq \nuX \m  \)
%|<<|:::::::::::::::::::::::::::::::::::::::::::::::::::::::::::::::::::::::::::::::::::::::::::|<<|
uniformly for all
%|>>|:::::::::::::::::::::::::::::::::::::::::::::::::::::::::::::::::::::::::::::::::::::::::::|>>|
\(  \thetaX \in \SparseSphereSubspace{\nO}{\n}  \).
%|<<|:::::::::::::::::::::::::::::::::::::::::::::::::::::::::::::::::::::::::::::::::::::::::::|<<|
\end{lemma}
%|<<|*******************************************************************************************|<<|
%|<<|*******************************************************************************************|<<|
%|<<|*******************************************************************************************|<<|
%
%|>>|~~~~~~~~~~~~~~~~~~~~~~~~~~~~~~~~~~~~~~~~~~~~~~~~~~~~~~~~~~~~~~~~~~~~~~~~~~~~~~~~~~~~~~~~~~~|>>|
%|>>|~~~~~~~~~~~~~~~~~~~~~~~~~~~~~~~~~~~~~~~~~~~~~~~~~~~~~~~~~~~~~~~~~~~~~~~~~~~~~~~~~~~~~~~~~~~|>>|
%|>>|~~~~~~~~~~~~~~~~~~~~~~~~~~~~~~~~~~~~~~~~~~~~~~~~~~~~~~~~~~~~~~~~~~~~~~~~~~~~~~~~~~~~~~~~~~~|>>|
\begin{subproof}
{\LEMMA \ref{lemma:pf:lemma:concentration-ineq:noiseless:small:mismatches}}
%
\LEMMA \ref{lemma:pf:lemma:concentration-ineq:noiseless:small:mismatches} is a corollary to \cite[{\COROLLARY 3.3}]{oymak2015near}, which is presented below as \LEMMA \ref{lemma:oymak2015near:corollary3.3}.
%
%|>>|*******************************************************************************************|>>|
%|>>|*******************************************************************************************|>>|
%|>>|*******************************************************************************************|>>|
\begin{lemma}[{(part of) \cite[{\COROLLARY 3.3}]{oymak2015near}}]
\label{lemma:oymak2015near:corollary3.3}
%
Let \(  \Set{U} \subseteq \R^{\n}  \).
If the set
%|>>|:::::::::::::::::::::::::::::::::::::::::::::::::::::::::::::::::::::::::::::::::::::::::::|>>|
\(  \Set{\hat{U}} \defeq \{ w \Vec{\uV} : \Vec{\uV} \in \Set{U}, w \in \R \}  \)
%|<<|:::::::::::::::::::::::::::::::::::::::::::::::::::::::::::::::::::::::::::::::::::::::::::|<<|
is a subspace with dimension
%|>>|:::::::::::::::::::::::::::::::::::::::::::::::::::::::::::::::::::::::::::::::::::::::::::|>>|
\(  \Dim \, \Set{\hat{U}} = t  \)
%|<<|:::::::::::::::::::::::::::::::::::::::::::::::::::::::::::::::::::::::::::::::::::::::::::|<<|
and
%|>>|===========================================================================================|>>|
\begin{gather}
  \m
  \geq
  \frac{\ConstdSD t}{\nuX} \log \left( \frac{1}{\nuX} \right)
,\end{gather}
%|<<|===========================================================================================|<<|
then
%|>>|:::::::::::::::::::::::::::::::::::::::::::::::::::::::::::::::::::::::::::::::::::::::::::|>>|
\(  \LRVXX{\Vec{\uV}}{\Vec{\vV}} \leq \nuX \m  \)
%\(  \| \I( \Sign( \CovM \Vec{\uV} ) \neq \Sign( \CovM \Vec{\vV} ) ) \|_{0} \leq \nuX \m  \)
%|<<|:::::::::::::::::::::::::::::::::::::::::::::::::::::::::::::::::::::::::::::::::::::::::::|<<|
for each pair
%|>>|:::::::::::::::::::::::::::::::::::::::::::::::::::::::::::::::::::::::::::::::::::::::::::|>>|
\(  \Vec{\uV}, \Vec{\vV} \in \Set{U}  \)
%|<<|:::::::::::::::::::::::::::::::::::::::::::::::::::::::::::::::::::::::::::::::::::::::::::|<<|
such that
%|>>|:::::::::::::::::::::::::::::::::::::::::::::::::::::::::::::::::::::::::::::::::::::::::::|>>|
\(  \| \Vec{\uV} - \Vec{\vV} \|_{2} \leq \smash[b]{\frac{\nuX}{\ConstdSD \sqrt{\log \left( \frac{1}{\nuX} \right)}}}  \),
%|<<|:::::::::::::::::::::::::::::::::::::::::::::::::::::::::::::::::::::::::::::::::::::::::::|<<|
uniformly with probability at least
%|>>|:::::::::::::::::::::::::::::::::::::::::::::::::::::::::::::::::::::::::::::::::::::::::::|>>|
\(  1 - e^{-\frac{1}{64} \nuX \m}  \).
%|<<|:::::::::::::::::::::::::::::::::::::::::::::::::::::::::::::::::::::::::::::::::::::::::::|<<|
\end{lemma}
%|<<|*******************************************************************************************|<<|
%|<<|*******************************************************************************************|<<|
%|<<|*******************************************************************************************|<<|
%
Resuming the verification of \LEMMA \ref{lemma:pf:lemma:concentration-ineq:noiseless:small:mismatches},
let
%|>>|:::::::::::::::::::::::::::::::::::::::::::::::::::::::::::::::::::::::::::::::::::::::::::|>>|
\(  \JSXXX \defeq \{ \JCoordsXXX \subseteq [\n] : | \JCoordsXXX | = \nO \}  \),
%|<<|:::::::::::::::::::::::::::::::::::::::::::::::::::::::::::::::::::::::::::::::::::::::::::|<<|
where
%|>>|:::::::::::::::::::::::::::::::::::::::::::::::::::::::::::::::::::::::::::::::::::::::::::|>>|
\(  | \JSXXX | = \binom{\n}{\nO}  \).
%|<<|:::::::::::::::::::::::::::::::::::::::::::::::::::::::::::::::::::::::::::::::::::::::::::|<<|
Note that
%|>>|:::::::::::::::::::::::::::::::::::::::::::::::::::::::::::::::::::::::::::::::::::::::::::|>>|
\(  \bigcup_{\JCoordsXXX \in \JSXXX} \{ \Vec{\uV} \in \Sphere{\n} : \Supp( \Vec{\uV} ) \subseteq \JCoordsXXX \} = \SparseSphereSubspace{\k}{\n}  \).
%|<<|:::::::::::::::::::::::::::::::::::::::::::::::::::::::::::::::::::::::::::::::::::::::::::|<<|
Fixing
%|>>|:::::::::::::::::::::::::::::::::::::::::::::::::::::::::::::::::::::::::::::::::::::::::::|>>|
\(  \JCoordsXXX \in \JSXXX  \)
%|<<|:::::::::::::::::::::::::::::::::::::::::::::::::::::::::::::::::::::::::::::::::::::::::::|<<|
arbitrarily, and writing
%|>>|:::::::::::::::::::::::::::::::::::::::::::::::::::::::::::::::::::::::::::::::::::::::::::|>>|
\(  \Set{U} \defeq \{ \Vec{\uV} \in \Sphere{\n} : \Supp( \Vec{\uV} ) \subseteq \JCoordsXXX \}  \) and
\(  \Set{\hat{U}} \defeq \{ w \Vec{\uV} : \Vec{\uV} \in \Set{U}, w \in \R \}  \)%
%|<<|:::::::::::::::::::::::::::::::::::::::::::::::::::::::::::::::::::::::::::::::::::::::::::|<<|
---where \(  \Set{\hat{U}}  \) has dimension \(  \Dim \, \Set{\hat{U}} = \nO  \)---consider any
%|>>|:::::::::::::::::::::::::::::::::::::::::::::::::::::::::::::::::::::::::::::::::::::::::::|>>|
\(  \thetaX \in \Set{U}  \).
%|<<|:::::::::::::::::::::::::::::::::::::::::::::::::::::::::::::::::::::::::::::::::::::::::::|<<|
Notice that because
%|>>|:::::::::::::::::::::::::::::::::::::::::::::::::::::::::::::::::::::::::::::::::::::::::::|>>|
\(  \BallXX{2\tauX}( \thetaX ) \subseteq \Set{U}  \),
%|<<|:::::::::::::::::::::::::::::::::::::::::::::::::::::::::::::::::::::::::::::::::::::::::::|<<|
it happens that
%|>>|:::::::::::::::::::::::::::::::::::::::::::::::::::::::::::::::::::::::::::::::::::::::::::|>>|
\(  \LRVX{\thetaX} = \sup_{\thetaXX \in \BallXX{2\tauX}( \thetaX )} \LRVXX{\thetaX}{\thetaXX} \leq \sup_{\thetaXX \in \Set{U}} \LRVXX{\thetaX}{\thetaXX} \leq \sup_{\thetaX', \thetaXX \in \Set{U}} \LRVXX{\thetaX'}{\thetaXX}  \).
%|<<|:::::::::::::::::::::::::::::::::::::::::::::::::::::::::::::::::::::::::::::::::::::::::::|<<|
Hence, it immediately follows from \LEMMA \ref{lemma:oymak2015near:corollary3.3} that with probability at least
%|>>|:::::::::::::::::::::::::::::::::::::::::::::::::::::::::::::::::::::::::::::::::::::::::::|>>|
\(  1 - e^{-\frac{1}{64} \nuX \m}  \),
%|<<|:::::::::::::::::::::::::::::::::::::::::::::::::::::::::::::::::::::::::::::::::::::::::::|<<|
the desired upper bound holds:
%|>>|:::::::::::::::::::::::::::::::::::::::::::::::::::::::::::::::::::::::::::::::::::::::::::|>>|
\(  \LRVX{\thetaX} \leq \sup_{\thetaX', \thetaXX \in \Set{U}} \LRVXX{\thetaX'}{\thetaXX} \leq \nuX \m  \).
%|<<|:::::::::::::::::::::::::::::::::::::::::::::::::::::::::::::::::::::::::::::::::::::::::::|<<|
By a union bound over \(  \JSXXX  \), this bound on \(  \LRVX{\thetaX}  \) holds uniformly over all
%|>>|:::::::::::::::::::::::::::::::::::::::::::::::::::::::::::::::::::::::::::::::::::::::::::|>>|
\(  \thetaX \in \SparseSphereSubspace{\nO}{\n}  \)
%|<<|:::::::::::::::::::::::::::::::::::::::::::::::::::::::::::::::::::::::::::::::::::::::::::|<<|
with probability at least
%|>>|:::::::::::::::::::::::::::::::::::::::::::::::::::::::::::::::::::::::::::::::::::::::::::|>>|
\(  1 - | \JSXXX | e^{-\frac{1}{64} \nuX \m} = 1 - \binom{\n}{\nO} e^{-\frac{1}{64} \nuX \m}  \),
%|<<|:::::::::::::::::::::::::::::::::::::::::::::::::::::::::::::::::::::::::::::::::::::::::::|<<|
as claimed.
\end{subproof}
%|<<|~~~~~~~~~~~~~~~~~~~~~~~~~~~~~~~~~~~~~~~~~~~~~~~~~~~~~~~~~~~~~~~~~~~~~~~~~~~~~~~~~~~~~~~~~~~|<<|
%|<<|~~~~~~~~~~~~~~~~~~~~~~~~~~~~~~~~~~~~~~~~~~~~~~~~~~~~~~~~~~~~~~~~~~~~~~~~~~~~~~~~~~~~~~~~~~~|<<|
%|<<|~~~~~~~~~~~~~~~~~~~~~~~~~~~~~~~~~~~~~~~~~~~~~~~~~~~~~~~~~~~~~~~~~~~~~~~~~~~~~~~~~~~~~~~~~~~|<<|
%
%%|>>|*******************************************************************************************|>>|
%%|>>|*******************************************************************************************|>>|
%%|>>|*******************************************************************************************|>>|
%\begin{lemma}[{\cite[{\LEMMAS A.2 and A.3}]{matsumoto2024robust}}]
%\label{lemma:pf:lemma:concentration-ineq:noiseless:small:concentration-ineq}
%%
%%|>>|===========================================================================================|>>|
%\begin{gather}
%  \Pr \left(
%    %
%  \right)
%\end{gather}
%%|<<|===========================================================================================|<<|
%\end{lemma}
%%|<<|*******************************************************************************************|<<|
%%|<<|*******************************************************************************************|<<|
%%|<<|*******************************************************************************************|<<|
%
Returning to the proof of \LEMMA \ref{lemma:small-dist}, recall that
%|>>|:::::::::::::::::::::::::::::::::::::::::::::::::::::::::::::::::::::::::::::::::::::::::::|>>|
\(  \ParamSpace = \SparseSphereSubspace{\nO}{\n}  \).
%|<<|:::::::::::::::::::::::::::::::::::::::::::::::::::::::::::::::::::::::::::::::::::::::::::|<<|
Thus, due to \LEMMA \ref{lemma:pf:lemma:concentration-ineq:noiseless:small:mismatches} and the sufficiently large choice of
%|>>|===========================================================================================|>>|
\begin{gather}
\label{eqn:pf:lemma:concentration-ineq:noiseless:small:m}
  \m
  \geq
  \frac{\ConstdSD \nO}{\nuX} \log \left( \frac{1}{\nuX} \right)
\end{gather}
%|<<|===========================================================================================|<<|
in \LEMMA \ref{lemma:small-dist},
%with high probability (as specified earlier),
with probability no less than
%|>>|:::::::::::::::::::::::::::::::::::::::::::::::::::::::::::::::::::::::::::::::::::::::::::|>>|
\(  1 - \binom{\n}{\nO} e^{-\frac{1}{64} \nuX \m}  \),
%|<<|:::::::::::::::::::::::::::::::::::::::::::::::::::::::::::::::::::::::::::::::::::::::::::|<<|
for every
%|>>|:::::::::::::::::::::::::::::::::::::::::::::::::::::::::::::::::::::::::::::::::::::::::::|>>|
\(  \thetaX \in \SparseSphereSubspace{\nO}{\n} = \ParamSpace  \) and every \(  \thetaXX \in \BallXX{2\tauX}( \thetaX )  \),
%|<<|:::::::::::::::::::::::::::::::::::::::::::::::::::::::::::::::::::::::::::::::::::::::::::|<<|
the indicator random vector
%|>>|:::::::::::::::::::::::::::::::::::::::::::::::::::::::::::::::::::::::::::::::::::::::::::|>>|
\(  \I( \Sign*( \CovMX \thetaX ) \neq \Sign*( \CovMX \thetaXX ) ) \in \{ 0,1 \}^{\m}  \)
%|<<|:::::::::::::::::::::::::::::::::::::::::::::::::::::::::::::::::::::::::::::::::::::::::::|<<|
contains at most \(  \nuX \m  \)-many nonzero entries.
%, with all remaining entries set to \(  0  \).
Therefore, with probability at least
%|>>|:::::::::::::::::::::::::::::::::::::::::::::::::::::::::::::::::::::::::::::::::::::::::::|>>|
\(  1 - \binom{\n}{\nO} e^{-\frac{1}{64} \nuX \m}  \),
%|<<|:::::::::::::::::::::::::::::::::::::::::::::::::::::::::::::::::::::::::::::::::::::::::::|<<|
for every \(  \thetaX \in \ParamSpace  \),
%|>>|===========================================================================================|>>|
\begin{gather}
\label{eqn:pf:lemma:concentration-ineq:noiseless:small:1}
  | \WS{\thetaX} |
  \leq
  \sum_{\lIx=0}^{\nuX \m}
  \binom{\m}{\lIx}
  \leq
  \left( \frac{e \m}{\nuX \m} \right)^{\nuX \m}
  =
  \left( \frac{e}{\nuX} \right)^{\nuX \m}
,\end{gather}
%|<<|===========================================================================================|<<|
where the second inequality is due to a well-known bound for sums of binomial coefficients.
In light of this, for each \(  \thetaX \in \ParamSpace  \), construct the following cover, \(  \BallXCover{\thetaX} \subset \BallXX{2\tauX}( \thetaX )  \), over \(  \BallXX{2\tauX}( \thetaX )  \):
for each \(  \wS \in \WS{\thetaX}  \), insert into \(  \BallXCover{\thetaX}  \) exactly one \(  \thetaXX \in \BallXX{2\tauX}( \thetaX )  \) for which
%|>>|:::::::::::::::::::::::::::::::::::::::::::::::::::::::::::::::::::::::::::::::::::::::::::|>>|
\(  \I( \Sign*( \CovMX \thetaX ) \neq \Sign*( \CovMX \thetaXX ) ) = \wS  \).
%|<<|:::::::::::::::::::::::::::::::::::::::::::::::::::::::::::::::::::::::::::::::::::::::::::|<<|
Note that
%|>>|:::::::::::::::::::::::::::::::::::::::::::::::::::::::::::::::::::::::::::::::::::::::::::|>>|
\(  | \BallXCover{\thetaX} | = | \WS{\thetaX} |  \).
%|<<|:::::::::::::::::::::::::::::::::::::::::::::::::::::::::::::::::::::::::::::::::::::::::::|<<|
Define the random variable
%|>>|:::::::::::::::::::::::::::::::::::::::::::::::::::::::::::::::::::::::::::::::::::::::::::|>>|
\(  \MaxBallXCoverSize \defeq \max_{\thetaX' \in \ParamSpace} | \BallXCover{\thetaX'} | \equiv \max_{\thetaX' \in \ParamSpace} | \WS{\thetaX'} |  \),
%|<<|:::::::::::::::::::::::::::::::::::::::::::::::::::::::::::::::::::::::::::::::::::::::::::|<<|
and write
%|>>|:::::::::::::::::::::::::::::::::::::::::::::::::::::::::::::::::::::::::::::::::::::::::::|>>|
\(  \qX \defeq \sum_{\lIx=0}^{\nuX \m} \binom{\m}{\lIx} \leq ( \frac{e}{\nuX} )^{\nuX \m}  \).
%|<<|:::::::::::::::::::::::::::::::::::::::::::::::::::::::::::::::::::::::::::::::::::::::::::|<<|
Due to \LEMMA \ref{lemma:pf:lemma:concentration-ineq:noiseless:small:mismatches} and \EQUATION \eqref{eqn:pf:lemma:concentration-ineq:noiseless:small:1}, as well as the sufficient choice of \(  \m  \) in \EQUATION \eqref{eqn:pf:lemma:concentration-ineq:noiseless:small:m},
the random variable \(  \MaxBallXCoverSize  \) is bounded from above by
%|>>|:::::::::::::::::::::::::::::::::::::::::::::::::::::::::::::::::::::::::::::::::::::::::::|>>|
\(  \MaxBallXCoverSize \leq \qX  \)
%|<<|:::::::::::::::::::::::::::::::::::::::::::::::::::::::::::::::::::::::::::::::::::::::::::|<<|
with, once again, probability at least
%|>>|:::::::::::::::::::::::::::::::::::::::::::::::::::::::::::::::::::::::::::::::::::::::::::|>>|
\(  1 - \binom{\n}{\nO} e^{-\frac{1}{64} \nuX \m}  \).
%|<<|:::::::::::::::::::::::::::::::::::::::::::::::::::::::::::::::::::::::::::::::::::::::::::|<<|
%
%%%%%%%%%%%%%%%%%%%%%%%%%%%%%%%%%%%%%%%%%%%%%%%%%%%%%%%%%%%%%%%%%%%%%%%%%%%%%%%%%%%%%%%%%%%%%%%%%%%%
\par %%%%%%%%%%%%%%%%%%%%%%%%%%%%%%%%%%%%%%%%%%%%%%%%%%%%%%%%%%%%%%%%%%%%%%%%%%%%%%%%%%%%%%%%%%%%%%%
%%%%%%%%%%%%%%%%%%%%%%%%%%%%%%%%%%%%%%%%%%%%%%%%%%%%%%%%%%%%%%%%%%%%%%%%%%%%%%%%%%%%%%%%%%%%%%%%%%%%
%
An additional helpful technique is an orthogonal decomposition of \(  \hFn[\JCoordsXX]  \)---in this case:
%|>>|===========================================================================================|>>|
\begin{align}
\label{eqn:pf:lemma:concentration-ineq:noiseless:small:6}
  \hFn[\JCoordsXX]( \thetaX, \thetaXX )
  &=
  \left\langle \hFn[\JCoordsXX]( \thetaX, \thetaXX ), \frac{\thetaX-\thetaXX}{\| \thetaX-\thetaXX \|_{2}} \right\rangle \frac{\thetaX-\thetaXX}{\| \thetaX-\thetaXX \|_{2}}
  +
  \left\langle \hFn[\JCoordsXX]( \thetaX, \thetaXX ), \frac{\thetaX+\thetaXX}{\| \thetaX+\thetaXX \|_{2}} \right\rangle \frac{\thetaX+\thetaXX}{\| \thetaX+\thetaXX \|_{2}}
  +
  \gFn[\JCoordsXX]( \thetaX, \thetaXX )
,\end{align}
%|<<|===========================================================================================|<<|
where
%|>>|===========================================================================================|>>|
\begin{align*}
  \gFn[\JCoordsXX]( \thetaX, \thetaXX )
  =
  \hFn[\JCoordsXX]( \thetaX, \thetaXX )
  -
  \left\langle \hFn[\JCoordsXX]( \thetaX, \thetaXX ), \frac{\thetaX-\thetaXX}{\| \thetaX-\thetaXX \|_{2}} \right\rangle \frac{\thetaX-\thetaXX}{\| \thetaX-\thetaXX \|_{2}}
  -
  \left\langle \hFn[\JCoordsXX]( \thetaX, \thetaXX ), \frac{\thetaX+\thetaXX}{\| \thetaX+\thetaXX \|_{2}} \right\rangle \frac{\thetaX+\thetaXX}{\| \thetaX+\thetaXX \|_{2}}
\end{align*}
%|<<|===========================================================================================|<<|
per \EQUATION \eqref{eqn:notations:gJ:def}.
Note that similar orthogonal decompositions appear in, \eg \cite{plan2017high,friedlander2021nbiht,matsumoto2022binary,matsumoto2024robust}.
Due to \EQUATION \eqref{eqn:pf:lemma:concentration-ineq:noiseless:small:6} in combination with the linearity of expectation,
%|>>|===========================================================================================|>>|
\begin{align*}
  & \negphantom{\AlignSp}
  \hFn[\JCoordsXX]( \thetaX, \thetaXX ) - \E[ \hFn[\JCoordsXX]( \thetaX, \thetaXX ) ]
  \\
  &=
  \left(
    \left\langle \hFn[\JCoordsXX]( \thetaX, \thetaXX ), \frac{\thetaX-\thetaXX}{\| \thetaX-\thetaXX \|_{2}} \right\rangle
    -
    \E \left[ \left\langle \hFn[\JCoordsXX]( \thetaX, \thetaXX ), \frac{\thetaX-\thetaXX}{\| \thetaX-\thetaXX \|_{2}} \right\rangle \right]
  \right)
  \frac{\thetaX-\thetaXX}{\| \thetaX-\thetaXX \|_{2}}
  \\
  &\AlignSp+
  \left(
    \left\langle \hFn[\JCoordsXX]( \thetaX, \thetaXX ), \frac{\thetaX+\thetaXX}{\| \thetaX+\thetaXX \|_{2}} \right\rangle
    -
    \E \left[ \left\langle \hFn[\JCoordsXX]( \thetaX, \thetaXX ), \frac{\thetaX+\thetaXX}{\| \thetaX+\thetaXX \|_{2}} \right\rangle \right]
  \right)
  \frac{\thetaX+\thetaXX}{\| \thetaX+\thetaXX \|_{2}}
  \\
  &\AlignSp+
  (
    \gFn[\JCoordsXX]( \thetaX, \thetaXX )
    -
    \E[ \gFn[\JCoordsXX]( \thetaX, \thetaXX ) ]
  )
.\end{align*}
%|<<|===========================================================================================|<<|
Combining this orthogonal decomposition with the triangle inequality yields the following upper bound on the \(  \lnorm{2}  \)-distance of \(  \hFn[\JCoordsXX]( \thetaX, \thetaXX )  \) from its mean:
%|>>|===========================================================================================|>>|
\begin{align*}
  \| \hFn[\JCoordsXX]( \thetaX, \thetaXX ) - \E[ \hFn[\JCoordsXX]( \thetaX, \thetaXX ) ] \|_{2}
  &\leq
  \left|
    \left\langle \hFn[\JCoordsXX]( \thetaX, \thetaXX ), \frac{\thetaX-\thetaXX}{\| \thetaX-\thetaXX \|_{2}} \right\rangle
    -
    \E \left[ \left\langle \hFn[\JCoordsXX]( \thetaX, \thetaXX ), \frac{\thetaX-\thetaXX}{\| \thetaX-\thetaXX \|_{2}} \right\rangle \right]
  \right|
  \\
  &\AlignSp+
  \left|
    \left\langle \hFn[\JCoordsXX]( \thetaX, \thetaXX ), \frac{\thetaX+\thetaXX}{\| \thetaX+\thetaXX \|_{2}} \right\rangle
    -
    \E \left[ \left\langle \hFn[\JCoordsXX]( \thetaX, \thetaXX ), \frac{\thetaX+\thetaXX}{\| \thetaX+\thetaXX \|_{2}} \right\rangle \right]
  \right|
  \\
  &\AlignSp+
  \|
    \gFn[\JCoordsXX]( \thetaX, \thetaXX )
    -
    \E[ \gFn[\JCoordsXX]( \thetaX, \thetaXX ) ]
  \|_{2}
\TagEqn{\label{eqn:pf:lemma:concentration-ineq:noiseless:small:2:1}}
.\end{align*}
%|<<|===========================================================================================|<<|
%
%%%%%%%%%%%%%%%%%%%%%%%%%%%%%%%%%%%%%%%%%%%%%%%%%%%%%%%%%%%%%%%%%%%%%%%%%%%%%%%%%%%%%%%%%%%%%%%%%%%%
\par %%%%%%%%%%%%%%%%%%%%%%%%%%%%%%%%%%%%%%%%%%%%%%%%%%%%%%%%%%%%%%%%%%%%%%%%%%%%%%%%%%%%%%%%%%%%%%%
%%%%%%%%%%%%%%%%%%%%%%%%%%%%%%%%%%%%%%%%%%%%%%%%%%%%%%%%%%%%%%%%%%%%%%%%%%%%%%%%%%%%%%%%%%%%%%%%%%%%
%
Most of the remaining arguments in this proof are towards bounding the three terms on the \RHS of \EQUATION \eqref{eqn:pf:lemma:concentration-ineq:noiseless:small:2:1}.
The following lemma from \cite{matsumoto2022binary} will facilitate the bound on \EQUATION \eqref{eqn:pf:lemma:concentration-ineq:noiseless:small:2:1}.
Note that \LEMMA \ref{lemma:pf:lemma:concentration-ineq:noiseless:small:condition-L} is not an exact restatement of, but is implied by, \cite[{\LEMMA A.1}]{matsumoto2022binary} and its proof.
%
%|>>|*******************************************************************************************|>>|
%|>>|*******************************************************************************************|>>|
%|>>|*******************************************************************************************|>>|
\begin{lemma}[{\dueto \cite[{\LEMMA A.1}]{matsumoto2022binary}}]
\label{lemma:pf:lemma:concentration-ineq:noiseless:small:condition-L}
%
Let
%|>>|:::::::::::::::::::::::::::::::::::::::::::::::::::::::::::::::::::::::::::::::::::::::::::|>>|
\(  \tX > 0  \) and \(  \uX \in (0,1)  \),
%|<<|:::::::::::::::::::::::::::::::::::::::::::::::::::::::::::::::::::::::::::::::::::::::::::|<<|
and let
%|>>|:::::::::::::::::::::::::::::::::::::::::::::::::::::::::::::::::::::::::::::::::::::::::::|>>|
\(  \thetaX, \thetaXX \in \SparseSphereSubspace{\k}{\n}  \) and \(  \JCoordsXX \in \JSXX  \).
%|<<|:::::::::::::::::::::::::::::::::::::::::::::::::::::::::::::::::::::::::::::::::::::::::::|<<|
Write
%|>>|:::::::::::::::::::::::::::::::::::::::::::::::::::::::::::::::::::::::::::::::::::::::::::|>>|
\(  \kOXX \defeq \kOXXExpr \geq \kOXXExprX  \).
%|<<|:::::::::::::::::::::::::::::::::::::::::::::::::::::::::::::::::::::::::::::::::::::::::::|<<|
Then,
%|>>|===========================================================================================|>>|
\begin{gather}
  \label{eqn:lemma:pf:lemma:concentration-ineq:noiseless:small:condition-L:1}
  \Pr \left( {\textstyle
    \left| \left\langle \frac{\hFn[\JCoordsXX]( \thetaX, \thetaXX )}{\sqrt{2\pi}} , \frac{\thetaX-\thetaXX}{\| \thetaX-\thetaXX \|_{2}} \right\rangle - \E \left[ \left\langle \frac{\hFn[\JCoordsXX]( \thetaX, \thetaXX )}{\sqrt{2\pi}} , \frac{\thetaX-\thetaXX}{\| \thetaX-\thetaXX \|_{2}} \right\rangle \right] \right|
    >
    \ux \tX
  } \middle| {\textstyle
    \LRVXX{\thetaX}{\thetaXX} \leq \uX \m
  } \right)
  \leq
  2 e^{-\frac{1}{2} \uX \m \tX^{2}}
  ,\\
  \label{eqn:lemma:pf:lemma:concentration-ineq:noiseless:small:condition-L:2}
  \textstyle{ \Pr \left(
    \left| \left\langle \frac{\hFn[\JCoordsXX]( \thetaX, \thetaXX )}{\sqrt{2\pi}} , \frac{\thetaX+\thetaXX}{\| \thetaX+\thetaXX \|_{2}} \right\rangle - \E \left[ \left\langle \frac{\hFn[\JCoordsXX]( \thetaX, \thetaXX )}{\sqrt{2\pi}} , \frac{\thetaX+\thetaXX}{\| \thetaX+\thetaXX \|_{2}} \right\rangle \right] \right|
    >
    \ux \tX
  \middle|
    \LRVXX{\thetaX}{\thetaXX} \leq \uX \m
  \right) }
  \leq
  2 e^{-\frac{1}{2} \uX \m \tX^{2}}
  ,\\
  \label{eqn:lemma:pf:lemma:concentration-ineq:noiseless:small:condition-L:3}
  {\textstyle \Pr \left(
    \left\| \frac{\gFn[\JCoordsXX]( \thetaX, \thetaXX )}{\sqrt{2\pi}}  - \E \left[ \frac{\gFn[\JCoordsXX]( \thetaX, \thetaXX )}{\sqrt{2\pi}}  \right] \right\|_{2}
    >
    2 \sqrt{\frac{\kOXX \uX}{\m}}
    +
    \ux \tX
  \middle|
    \LRVXX{\thetaX}{\thetaXX} \leq \uX \m
  \right) }
  \leq
  e^{-\frac{1}{8} \uX \m \tX^{2}}
.\end{gather}
%|<<|===========================================================================================|<<|
%Let
%%|>>|:::::::::::::::::::::::::::::::::::::::::::::::::::::::::::::::::::::::::::::::::::::::::::|>>|
%\(  \ISet \subseteq [\m]  \),
%\(  \EllX \defeq | \ISet |  \).
%%|<<|:::::::::::::::::::::::::::::::::::::::::::::::::::::::::::::::::::::::::::::::::::::::::::|<<|
%Write
%%|>>|:::::::::::::::::::::::::::::::::::::::::::::::::::::::::::::::::::::::::::::::::::::::::::|>>|
%\(  \hXFn*[\JCoordsXX], \gXFn*[\JCoordsXX] : \R^{\n} \to \R^{\n}  \),
%%|<<|:::::::::::::::::::::::::::::::::::::::::::::::::::::::::::::::::::::::::::::::::::::::::::|<<|
%%|>>|===========================================================================================|>>|
%\begin{gather}
%  \hXFn[\JCoordsXX]{\ISet}{\thetaX}
%  =
%  \ThresholdSet{\Supp( \thetaX ) \cup \JCoordsXX} \left(
%  \frac{\sqrt{2\pi}}{\m}
%  \sum_{\iIx \in \ISet}
%  \CovV\VIx{\iIx} \Sign.( \langle \CovV\VIx{\iIx}, \thetaX \rangle )
%  \right)
%  ,\\
%  \gXFn[\JCoordsXX]{\ISet}{\thetaX}
%  =
%  \hXFn[\JCoordsXX]{\ISet}{\thetaX}
%  -
%  \langle \hXFn[\JCoordsXX]{\ISet}{\thetaX}, \thetaX \rangle \thetaX
%.\end{gather}
%%|<<|===========================================================================================|<<|
%Then,
%%|>>|===========================================================================================|>>|
%\begin{gather}
%  \Pr \left(
%    \left|
%     \left\langle \frac{1}{\sqrt{2\pi}} \hXFn[\JCoordsXX]{\ISet}{\thetaX}, \thetaX \right\rangle
%      -
%      \E \left[ \left\langle \frac{1}{\sqrt{2\pi}} \hXFn[\JCoordsXX]{\ISet}{\thetaX}, \thetaX \right\rangle \right]
%    \right|
%    >
%    \frac{\EllX \tX}{\m}
%  \right)
%  \leq
%  2 e^{-\frac{1}{2} \EllX \tX^{2}}
%  ,\\
%  \Pr \left(
%    \left\|
%     \frac{1}{\sqrt{2\pi}} \gXFn[\JCoordsXX]{\ISet}{\thetaX}
%      -
%      \E \left[ \frac{1}{\sqrt{2\pi}} \gXFn[\JCoordsXX]{\ISet}{\thetaX} \right]
%    \right\|_{2}
%    >
%    \frac{\sqrt{\frac{1}{2} ( \kOXX-1 ) \EllX}}{\m}
%    +
%    \frac{\EllX \tX}{\m}
%  \right)
%  \leq
%  2 e^{-\frac{1}{2} \EllX \tX^{2}}
%.\end{gather}
%%|<<|===========================================================================================|<<|
\end{lemma}
%|<<|*******************************************************************************************|<<|
%|<<|*******************************************************************************************|<<|
%|<<|*******************************************************************************************|<<|
%
\newcommand{\COND}{\Forall{\thetaX' \in \ParamSpace}{\LRVX{\thetaX'} \leq \uX \m}}
\newcommand{\NEGCOND}{\ExistsST{\thetaX' \in \ParamSpace}{\LRVX{\thetaX'} > \uX \m}}
By symmetry, this implies that for any
%|>>|:::::::::::::::::::::::::::::::::::::::::::::::::::::::::::::::::::::::::::::::::::::::::::|>>|
\(  \thetaX \in \ParamSpace  \) and \(  \thetaXX \in \BallXX{2\tauX}( \thetaX )  \),
%|<<|:::::::::::::::::::::::::::::::::::::::::::::::::::::::::::::::::::::::::::::::::::::::::::|<<|
%|>>|===========================================================================================|>>|
\begin{gather}
  {\textstyle
  \Pr \left(
    \left| \left\langle \frac{\hFn[\JCoordsXX]( \thetaX, \thetaXX )}{\sqrt{2\pi}}, \frac{\thetaX-\thetaXX}{\| \thetaX-\thetaXX \|_{2}} \right\rangle - \E \left[ \left\langle \frac{\hFn[\JCoordsXX]( \thetaX, \thetaXX )}{\sqrt{2\pi}}, \frac{\thetaX-\thetaXX}{\| \thetaX-\thetaXX \|_{2}} \right\rangle \right] \right|
    >
    \ux \tX
  \middle|
    \COND
  \right) }
  % \nonumber \\ 
  \leq
  2 e^{-\frac{1}{2} \uX \m \tX^{2}}
\label{eqn:lemma:pf:lemma:concentration-ineq:noiseless:small:condition-L:1:b}
  ,\\
  {\textstyle
  \Pr \left(
    \left| \left\langle \frac{\hFn[\JCoordsXX]( \thetaX, \thetaXX )}{\sqrt{2\pi}}, \frac{\thetaX+\thetaXX}{\| \thetaX+\thetaXX \|_{2}} \right\rangle - \E \left[ \left\langle \frac{\hFn[\JCoordsXX]( \thetaX, \thetaXX )}{\sqrt{2\pi}}, \frac{\thetaX+\thetaXX}{\| \thetaX+\thetaXX \|_{2}} \right\rangle \right] \right|
    >
    \ux \tX
  \middle|
    \COND
  \right) }
  % \nonumber \\ 
  \leq
  2 e^{-\frac{1}{2} \uX \m \tX^{2}}
\label{eqn:lemma:pf:lemma:concentration-ineq:noiseless:small:condition-L:2:b}
  ,\\
  {\textstyle
  \Pr \left(
    \left\| \frac{\gFn[\JCoordsXX]( \thetaX, \thetaXX )}{\sqrt{2\pi}} - \E \left[ \frac{\gFn[\JCoordsXX]( \thetaX, \thetaXX )}{\sqrt{2\pi}} \right] \right\|_{2}
    >
    2 \sqrt{\frac{\kOXX \uX}{\m}}
    +
    \ux \tX
  \middle|
    \COND
  \right) }
  % \nonumber \\ 
  \leq
  e^{-\frac{1}{8} \uX \m \tX^{2}}
\label{eqn:lemma:pf:lemma:concentration-ineq:noiseless:small:condition-L:3:b}
,\end{gather}
%|<<|===========================================================================================|<<|
where notations are taken from the above lemma.
%
%%%%%%%%%%%%%%%%%%%%%%%%%%%%%%%%%%%%%%%%%%%%%%%%%%%%%%%%%%%%%%%%%%%%%%%%%%%%%%%%%%%%%%%%%%%%%%%%%%%%
\par %%%%%%%%%%%%%%%%%%%%%%%%%%%%%%%%%%%%%%%%%%%%%%%%%%%%%%%%%%%%%%%%%%%%%%%%%%%%%%%%%%%%%%%%%%%%%%%
%%%%%%%%%%%%%%%%%%%%%%%%%%%%%%%%%%%%%%%%%%%%%%%%%%%%%%%%%%%%%%%%%%%%%%%%%%%%%%%%%%%%%%%%%%%%%%%%%%%%
%
To help condense notations in the upcoming analysis, define the following indicator random variables:
%|>>|===========================================================================================|>>|
\begin{gather*}
  \textstyle
  \ARV{1}{\thetaX}{\thetaXX}
  \defeq
  \I \left(
    \left| \left\langle \frac{\hFn[\JCoordsXX]( \thetaX, \thetaXX )}{\sqrt{2\pi}}, \frac{\thetaX-\thetaXX}{\| \thetaX-\thetaXX \|_{2}} \right\rangle - \E \left[ \left\langle \frac{\hFn[\JCoordsXX]( \thetaX, \thetaXX )}{\sqrt{2\pi}}, \frac{\thetaX-\thetaXX}{\| \thetaX-\thetaXX \|_{2}} \right\rangle \right] \right|
    >
    \nuX \tX
  \right)
  ,\\ \textstyle
  \ARV{2}{\thetaX}{\thetaXX}
  \defeq
  \I \left(
    \left| \left\langle \frac{\hFn[\JCoordsXX]( \thetaX, \thetaXX )}{\sqrt{2\pi}}, \frac{\thetaX+\thetaXX}{\| \thetaX+\thetaXX \|_{2}} \right\rangle - \E \left[ \left\langle \frac{\hFn[\JCoordsXX]( \thetaX, \thetaXX )}{\sqrt{2\pi}}, \frac{\thetaX+\thetaXX}{\| \thetaX+\thetaXX \|_{2}} \right\rangle \right] \right|
    >
    \nuX \tX
  \right)
  ,\\ \textstyle
  \ARV{3}{\thetaX}{\thetaXX}
  \defeq
  \I \left(
    \left\| \frac{\gFn[\JCoordsXX]( \thetaX, \thetaXX )}{\sqrt{2\pi}} - \E \left[ \frac{\gFn[\JCoordsXX]( \thetaX, \thetaXX )}{\sqrt{2\pi}} \right] \right\|_{2}
    >
    2 \sqrt{\frac{\kOXX \nuX}{\m}}
    +
    \nuX \tX
  \right)
.\end{gather*}
%|<<|===========================================================================================|<<|
%%|>>|===========================================================================================|>>|
%\begin{gather*}
%  \ARV{1}{\thetaX}{\thetaXX}
%  \defeq
%  \I \left(
%    \left| \left\langle \frac{1}{\sqrt{2\pi}} \hXFn[\JCoordsXX]{\ISet}{\thetaX}, \thetaX \right\rangle - \E \left[ \left\langle \frac{1}{\sqrt{2\pi}} \hXFn[\JCoordsXX]{\ISet}{\thetaX}, \thetaX \right\rangle \right] \right|
%    >
%    \nuX \tX
%  \right)
%  ,\\
%  \ARV{2}{\thetaX}{\thetaXX}
%  \defeq
%  \I \left(
%    \left\| \frac{1}{\sqrt{2\pi}} \gXFn[\JCoordsXX]{\ISet}{\thetaX} - \E \left[ \frac{1}{\sqrt{2\pi}} \gXFn[\JCoordsXX]{\ISet}{\thetaX} \right] \right\|_{2}
%    >
%    \sqrt{\frac{( \kOXX-1 ) \nuX}{2 \m}}
%    +
%    \nuX \tX
%  \right)
%.\end{gather*}
%%|<<|===========================================================================================|<<|
%Before proceeding with applying the lemma, note that by definition, for
%%|>>|:::::::::::::::::::::::::::::::::::::::::::::::::::::::::::::::::::::::::::::::::::::::::::|>>|
%\(  \thetaX \in \ParamSpace  \) and \(  \thetaXX \in \BallXX{2\tauX}( \thetaX )  \),
%%|<<|:::::::::::::::::::::::::::::::::::::::::::::::::::::::::::::::::::::::::::::::::::::::::::|<<|
%the random variable \(  \LRVX{\thetaX}  \) always upper bounds the random variable \(  \LRVXX{\thetaX}{\thetaXX}  \), i.e., \(  \LRVXX{\thetaX}{\thetaXX} \leq \LRVX{\thetaX}  \).
%Hence, \(  \LRVXX{\thetaX}{\thetaXX} > \nuX \m  \) implies \(  \LRVX{\thetaX} > \nuX \m  \), and therefore, \(  \Pr( \LRVXX{\thetaX}{\thetaXX} > \nuX \m ) \leq \Pr( \LRVX{\thetaX} > \nuX \m ) \).
%%%%%%%%%%%%%%%%%%%%%%%%%%%%%%%%%%%%%%%%%%%%%%%%%%%%%%%%%%%%%%%%%%%%%%%%%%%%%%%%%%%%%%%%%%%%%%%%%%%%
\renewcommand{\COND}{\Forall{\thetaX' \in \ParamSpace}{\LRVX{\thetaX'} \leq \nuX \m}}%
\renewcommand{\NEGCOND}{\ExistsST{\thetaX' \in \ParamSpace}{\LRVX{\thetaX'} > \nuX \m}}%
%%%%%%%%%%%%%%%%%%%%%%%%%%%%%%%%%%%%%%%%%%%%%%%%%%%%%%%%%%%%%%%%%%%%%%%%%%%%%%%%%%%%%%%%%%%%%%%%%%%%
Now, observe:
%|>>|===========================================================================================|>>|
\begin{align*}
  &
  \Pr( \ARV{1}{\thetaX}{\thetaXX}=1 \VEE \ARV{2}{\thetaX}{\thetaXX}=1 \VEE \ARV{3}{\thetaX}{\thetaXX}=1 \Mid| \COND )
  \\
  &\AlignIndent \leq
  \Pr( \ARV{1}{\thetaX}{\thetaXX}=1 \Mid| \COND )
  \\
  &\AlignIndent \AlignSp
  +
  \Pr( \ARV{2}{\thetaX}{\thetaXX}=1 \Mid| \COND )
  \\
  &\AlignIndent \AlignSp
  +
  \Pr( \ARV{3}{\thetaX}{\thetaXX}=1 \Mid| \COND )
  \\
  &\AlignIndent \dCmt{by a union bound}
  \\
  &\AlignIndent \leq
  2 e^{-\frac{1}{2} \nuX \m \tX^{2}}
  +
  2 e^{-\frac{1}{2} \nuX \m \tX^{2}}
  +
  e^{-\frac{1}{8} \nuX \m \tX^{2}}
  \\
  &\AlignIndent \dCmt{by \EQUATIONS \eqref{eqn:lemma:pf:lemma:concentration-ineq:noiseless:small:condition-L:1:b}--\eqref{eqn:lemma:pf:lemma:concentration-ineq:noiseless:small:condition-L:3:b}}
  \\
  &\AlignIndent \leq
  5 e^{-\frac{1}{8} \nuX \m \tX^{2}}
\TagEqn{\label{eqn:pf:lemma:concentration-ineq:noiseless:small:5}}
.\end{align*}
%|<<|===========================================================================================|<<|
%%|>>|===========================================================================================|>>|
%\begin{align*}
%  &
%  \Pr( \ARV{1}{\thetaX}{\thetaXX}=1 \VEE \ARV{2}{\thetaX}{\thetaXX}=1 \VEE \ARV{3}{\thetaX}{\thetaXX}=1 )
%  \\
%  &=
%  \Pr( \ARV{1}{\thetaX}{\thetaXX}=1 \VEE \ARV{2}{\thetaX}{\thetaXX}=1 \VEE \ARV{3}{\thetaX}{\thetaXX}=1, \LRVXX{\thetaX}{\thetaXX} \leq \nuX \m )
%  \\
%  &\AlignSp+
%  \Pr( \ARV{1}{\thetaX}{\thetaXX}=1 \VEE \ARV{2}{\thetaX}{\thetaXX}=1 \VEE \ARV{3}{\thetaX}{\thetaXX}=1, \LRVXX{\thetaX}{\thetaXX} > \nuX \m )
%  \\
%  &\dCmt{by the law of total probability}
%  \\
%  &=
%  \Pr( \ARV{1}{\thetaX}{\thetaXX}=1 \VEE \ARV{2}{\thetaX}{\thetaXX}=1 \VEE \ARV{3}{\thetaX}{\thetaXX}=1 \Mid| \LRVXX{\thetaX}{\thetaXX} \leq \nuX \m )
%  \Pr( \LRVXX{\thetaX}{\thetaXX} \leq \nuX \m )
%  \\
%  &\AlignSp+
%  \Pr( \ARV{1}{\thetaX}{\thetaXX}=1 \VEE \ARV{2}{\thetaX}{\thetaXX}=1 \VEE \ARV{3}{\thetaX}{\thetaXX}=1 \Mid| \LRVXX{\thetaX}{\thetaXX} > \nuX \m )
%  \Pr( \LRVXX{\thetaX}{\thetaXX} > \nuX \m )
%  \\
%  &\dCmt{by the definition of conditional probabilities}
%  \\
%  &\leq
%  \Pr( \ARV{1}{\thetaX}{\thetaXX}=1 \VEE \ARV{2}{\thetaX}{\thetaXX}=1 \VEE \ARV{3}{\thetaX}{\thetaXX}=1 \Mid| \LRVXX{\thetaX}{\thetaXX} \leq \nuX \m )
%  +
%  \Pr( \LRVXX{\thetaX}{\thetaXX} > \nuX \m )
%  \\
%  &\leq
%  \Pr( \ARV{1}{\thetaX}{\thetaXX}=1  \Mid| \LRVXX{\thetaX}{\thetaXX} \leq \nuX \m )
%  +
%  \Pr( \ARV{2}{\thetaX}{\thetaXX}=1  \Mid| \LRVXX{\thetaX}{\thetaXX} \leq \nuX \m )
%  +
%  \Pr( \ARV{3}{\thetaX}{\thetaXX}=1 \Mid| \LRVXX{\thetaX}{\thetaXX} \leq \nuX \m )
%  \\
%  &\AlignSp+
%  \Pr( \LRVXX{\thetaX}{\thetaXX} > \nuX \m )
%  \\
%  &\dCmt{by a union bound}
%  \\
%  &\leq
%  \Pr( \ARV{1}{\thetaX}{\thetaXX}=1  \Mid| \LRVXX{\thetaX}{\thetaXX} \leq \nuX \m )
%  +
%  \Pr( \ARV{2}{\thetaX}{\thetaXX}=1  \Mid| \LRVXX{\thetaX}{\thetaXX} \leq \nuX \m )
%  +
%  \Pr( \ARV{3}{\thetaX}{\thetaXX}=1 \Mid| \LRVXX{\thetaX}{\thetaXX} \leq \nuX \m )
%  \\
%  &\AlignSp+
%  \Pr( \LRVX{\thetaX} > \nuX \m )
%  \\
%  &\dCmt{as noted earlier}
%  \\
%  &\leq
%  2 e^{-\frac{1}{2} \nuX \m \tX^{2}}
%  +
%  2 e^{-\frac{1}{2} \nuX \m \tX^{2}}
%  +
%  e^{-\frac{1}{8} \nuX \m \tX^{2}}
%  +
%  e^{-\frac{1}{64} \nuX \m}
%  \\
%  &\dCmt{by \LEMMAS \ref{lemma:pf:lemma:concentration-ineq:noiseless:small:mismatches} and \ref{lemma:pf:lemma:concentration-ineq:noiseless:small:condition-L}}
%  \\
%  &\leq
%  5 e^{-\frac{1}{8} \nuX \m \tX^{2}}
%  +
%  e^{-\frac{1}{64} \nuX \m}
%  \\
%  &\leq
%  6 \max \left\{
%    e^{-\frac{1}{8} \nuX \m \tX^{2}},
%    e^{-\frac{1}{64} \nuX \m}
%  \right\}
%.\end{align*}
%%|<<|===========================================================================================|<<|
%By a union bound over all \(  \JCoordsXX \in \JSXX  \), \(  \thetaX \in \ParamCoverX  \), and \(  \thetaXX \in \BallXCover{\thetaX}  \),
A uniform bound over all \(  \JCoordsXX \in \JSXX  \), \(  \thetaX \in \ParamCoverX  \), and \(  \thetaXX \in \BallXCover{\thetaX}  \) is then obtained as follows:
%|>>|===========================================================================================|>>|
\begin{align*}
  &
  \Pr( \ExistsST{\JCoordsXX \in \JSXX, \thetaX \in \ParamCoverX, \thetaXX \in \BallXCover{\thetaX}}{\ARV{1}{\thetaX}{\thetaXX}=1 \VEE \ARV{2}{\thetaX}{\thetaXX}=1 \VEE \ARV{3}{\thetaX}{\thetaXX}=1} )
  % \\
  % &\AlignIndent
  % =
  % \Pr \!\!\!\! \begin{array}[t]{l} \displaystyle (
  %   \ExistsST
  %   {\JCoordsXX \in \JSXX, \thetaX \in \ParamCoverX, \thetaXX \in \BallXCover{\thetaX}}
  %   {\ARV{1}{\thetaX}{\thetaXX}=1 \VEE \ARV{2}{\thetaX}{\thetaXX}=1 \VEE \ARV{3}{\thetaX}{\thetaXX}=1},
  %   \\ \displaystyle \phantom{(}
  %   \text{and }
  %   \COND
  % ) \end{array}
  % \\
  % &\AlignIndent \AlignSp
  % +
  % \Pr \!\!\!\! \begin{array}[t]{l} \displaystyle (
  %   \ExistsST
  %   {\JCoordsXX \in \JSXX, \thetaX \in \ParamCoverX, \thetaXX \in \BallXCover{\thetaX}}
  %   {\ARV{1}{\thetaX}{\thetaXX}=1 \VEE \ARV{2}{\thetaX}{\thetaXX}=1 \VEE \ARV{3}{\thetaX}{\thetaXX}=1},
  %   \\ \displaystyle \phantom{(}
  %   \text{and }
  %   \NEGCOND
  % ) \end{array}
  % \\
  % &\AlignIndent
  % \dCmt{by the law of total probability}
  \\
  &\AlignIndent
  =
  \Pr \!\!\!\! \begin{array}[t]{l} \displaystyle (
    \ExistsST
    {\JCoordsXX \in \JSXX, \thetaX \in \ParamCoverX, \thetaXX \in \BallXCover{\thetaX}}
    {\ARV{1}{\thetaX}{\thetaXX}=1 \VEE \ARV{2}{\thetaX}{\thetaXX}=1 \VEE \ARV{3}{\thetaX}{\thetaXX}=1}
    \\ \displaystyle \phantom{(}
    \Mid|
    \COND
  ) \end{array}
  \\
  &\AlignIndent \AlignSp
  \cdot
  \Pr(
    \COND
  )
  \\
  &\AlignIndent \AlignSp
  +
  \Pr \!\!\!\! \begin{array}[t]{l} \displaystyle (
    \ExistsST
    {\JCoordsXX \in \JSXX, \thetaX \in \ParamCoverX, \thetaXX \in \BallXCover{\thetaX}}
    {\ARV{1}{\thetaX}{\thetaXX}=1 \VEE \ARV{2}{\thetaX}{\thetaXX}=1 \VEE \ARV{3}{\thetaX}{\thetaXX}=1}
    \\ \displaystyle \phantom{(}
    \Mid|
    \NEGCOND
  ) \end{array}
  \\
  &\AlignIndent \AlignSp \phantom{+}
  \cdot
  \Pr(
    \NEGCOND
  )
  \\
  &\AlignIndent
  \dCmt{by the law of total probability and the definition of conditional probabilities}
  \\
  &\AlignIndent
  \leq
  \Pr \!\!\!\! \begin{array}[t]{l} \displaystyle (
    \ExistsST
    {\JCoordsXX \in \JSXX, \thetaX \in \ParamCoverX, \thetaXX \in \BallXCover{\thetaX}}
    {\ARV{1}{\thetaX}{\thetaXX}=1 \VEE \ARV{2}{\thetaX}{\thetaXX}=1 \VEE \ARV{3}{\thetaX}{\thetaXX}=1}
    \\ \displaystyle \phantom{(}
    \Mid|
    \COND
  ) \end{array}
  \\
  &\AlignIndent \AlignSp
  +
  \Pr(
    \NEGCOND
  )
  \\
  &\AlignIndent
  \leq
  | \JSXX | | \ParamCoverX | \qX
  \Pr (
    \ARV{1}{\thetaX}{\thetaXX}=1 \VEE \ARV{2}{\thetaX}{\thetaXX}=1 \VEE \ARV{3}{\thetaX}{\thetaXX}=1
    \Mid| \COND
  )
  \\
  &\AlignIndent \AlignSp
  +
  \Pr(
    \NEGCOND
  )
  \\
  &\AlignIndent
  \dCmt{for an arbitrary choice of \(  \JCoordsXX \in \JSXX, \thetaX \in \ParamCoverX, \thetaXX \in \BallXCover{\thetaX}  \);}
  \\
  &\AlignIndent
  \dCmt{by a union bound and an earlier discussion about the cardinality of \(  \BallXCover{\cdot}  \)}
  \\
  &\AlignIndent
  \leq
  5 | \JSXX | | \ParamCoverX | \qX e^{-\frac{1}{8} \nuX \m \tX^{2}}
  +
  \binom{\n}{\nO} e^{-\frac{1}{64} \nuX \m}
  \\
  &\AlignIndent
  \dCmt{by \EQUATION \eqref{eqn:pf:lemma:concentration-ineq:noiseless:small:5} and \LEMMA \ref{lemma:pf:lemma:concentration-ineq:noiseless:small:mismatches}}
  \\
  &\AlignIndent
  \leq
  5 | \JS | | \ParamCover | \qX e^{-\frac{1}{8} \nuX \m \tX^{2}}
  +
  \binom{\n}{\nO} e^{-\frac{1}{64} \nuX \m}
  .\\
  &\AlignIndent
  \dCmt{\(  \because | \JSXX | \leq | \JS |  \) and \(  | \ParamCoverX | \leq | \ParamCover |  \)}
%\TagEqn{\label{eqn:pf:lemma:concentration-ineq:noiseless:small:3}}
\end{align*}
%|<<|===========================================================================================|<<|
Note that the last line follows from recalling the definition of \(  \JSXX  \):
%|>>|:::::::::::::::::::::::::::::::::::::::::::::::::::::::::::::::::::::::::::::::::::::::::::|>>|
\(  \JSXX \defeq \{ \JCoords \cup \Supp( \thetaStar ) : \JCoords \in \JS \}  \),
%|<<|:::::::::::::::::::::::::::::::::::::::::::::::::::::::::::::::::::::::::::::::::::::::::::|<<|
which inserts at most one coordinate subset into \(  \JSXX  \) for each coordinate subset in \(  \JS  \).
%
%%%%%%%%%%%%%%%%%%%%%%%%%%%%%%%%%%%%%%%%%%%%%%%%%%%%%%%%%%%%%%%%%%%%%%%%%%%%%%%%%%%%%%%%%%%%%%%%%%%%
\par %%%%%%%%%%%%%%%%%%%%%%%%%%%%%%%%%%%%%%%%%%%%%%%%%%%%%%%%%%%%%%%%%%%%%%%%%%%%%%%%%%%%%%%%%%%%%%%
%%%%%%%%%%%%%%%%%%%%%%%%%%%%%%%%%%%%%%%%%%%%%%%%%%%%%%%%%%%%%%%%%%%%%%%%%%%%%%%%%%%%%%%%%%%%%%%%%%%%
%
Returning to \EQUATION \eqref{eqn:pf:lemma:concentration-ineq:noiseless:small:2:1} and now applying the results just derived above, the following bound holds for all \(  \JCoordsXX \in \JSXX  \), \(  \thetaX \in \ParamCoverX  \), and \(  \thetaXX \in \BallXCover{\thetaX}  \) with probability at least
%|>>|:::::::::::::::::::::::::::::::::::::::::::::::::::::::::::::::::::::::::::::::::::::::::::|>>|
\(  1 -  5 | \JS | | \ParamCover | \qX e^{-\frac{1}{8} \nuX \m \tX^{2}} - \binom{\n}{\nO} e^{-\frac{1}{64} \nuX \m}  \):
%|<<|:::::::::::::::::::::::::::::::::::::::::::::::::::::::::::::::::::::::::::::::::::::::::::|<<|
%as stated in \EQUATION \eqref{eqn:pf:lemma:concentration-ineq:noiseless:small:3}:
%|>>|===========================================================================================|>>|
\begin{align*}
  \| \hFn[\JCoordsXX]( \thetaX, \thetaXX ) - \E[ \hFn[\JCoordsXX]( \thetaX, \thetaXX ) ] \|_{2}
  &\leq
  \left|
    \left\langle \hFn[\JCoordsXX]( \thetaX, \thetaXX ), \frac{\thetaX-\thetaXX}{\| \thetaX-\thetaXX \|_{2}} \right\rangle
    -
    \E \left[ \left\langle \hFn[\JCoordsXX]( \thetaX, \thetaXX ), \frac{\thetaX-\thetaXX}{\| \thetaX-\thetaXX \|_{2}} \right\rangle \right]
  \right|
  \\
  &\AlignSp+
  \left|
    \left\langle \hFn[\JCoordsXX]( \thetaX, \thetaXX ), \frac{\thetaX+\thetaXX}{\| \thetaX+\thetaXX \|_{2}} \right\rangle
    -
    \E \left[ \left\langle \hFn[\JCoordsXX]( \thetaX, \thetaXX ), \frac{\thetaX+\thetaXX}{\| \thetaX+\thetaXX \|_{2}} \right\rangle \right]
  \right|
  \\
  &\AlignSp+
  \|
    \gFn[\JCoordsXX]( \thetaX, \thetaXX )
    -
    \E[ \gFn[\JCoordsXX]( \thetaX, \thetaXX ) ]
  \|_{2}
  \\
  &\leq
  2\sqrt{\frac{\kOXX \nuX}{\m}} + 3 \nuX \tX
  \\
  &\leq
  5 \max \left\{
    \sqrt{\frac{\kOXX \nuX}{\m}},
    \nuX \tX
  \right\}
\TagEqn{\label{eqn:pf:lemma:concentration-ineq:noiseless:small:2}}
.\end{align*}
%|<<|===========================================================================================|<<|
Set
%|>>|===========================================================================================|>>|
\begin{align*}
  \tX
  &=
  \sqrt{\frac{8 \log \left( \frac{6}{\rhoSD} | \JS | | \ParamCover | \qX \right)}{\nuX \m}}
  \\
  % &=
  % \sqrt{
  %   \frac{8 \log \left( \frac{6}{\rhoSD} | \JS | | \ParamCover | \right)}{\nuX \m}
  %   +
  %   \frac{8 \log \left( \qX \right)}{\nuX \m}
  % }
  % \\
  &=
  \sqrt{
    \frac{8 \log \left( \frac{6}{\rhoSD} | \JS | | \ParamCover | \right)}{\nuX \m}
    +
    \frac{8 \log \left( \qXExpr \right)}{\nuX \m}
  }
  \\
  &\leq
  % \sqrt{
  %   \frac{8 \log \left( \frac{6}{\rhoSD} | \JS | | \ParamCover | \right)}{\nuX \m}
  %   +
  %   \frac{8 \nuX \m \log \left( \frac{e}{\nuX} \right)}{\nuX \m}
  % }
  % \\
  % &\dCmt{by an earlier remark}
  % \\
  % &=
  \sqrt{
    \frac{8 \log \left( \frac{6}{\rhoSD} | \JS | | \ParamCover | \right)}{\nuX \m}
    +
    8 \log \left( \frac{e}{\nuX} \right)
  }
  \\
  &\dCmt{by an earlier remark}
  \\
  &\leq
  \sqrt{\frac{8 \log \left( \frac{6}{\rhoSD} | \JS | | \ParamCover | \right)}{\nuX \m}}
  +
  \sqrt{8 \log \left( \frac{e}{\nuX} \right)}
\TagEqn{\label{eqn:pf:lemma:concentration-ineq:noiseless:small:3}}
  .\\
  &\dCmt{by the triangle inequality (in one dimension)}
\end{align*}
%|<<|===========================================================================================|<<|
In accordance with \EQUATION \eqref{eqn:lemma:small-dist:m} of \LEMMA \ref{lemma:small-dist}, let
%|>>|===========================================================================================|>>|
\begin{align*}
  \m
  &\geq
  \max \left\{
  \frac{200 \nuX \log \left( \frac{6}{\rhoSD} | \JS | | \ParamCover | \right)}{\left( \sqrt{\frac{\pi}{8}} \GAMMAX \ConstbSD \deltaX - \nuX \sqrt{8 \log \left( \frac{e}{\nuX} \right)} \right)^{2}}
  ,
  \frac{200 \nuX \kOXX}{\pi \GAMMAX^{2} \ConstbSD^{2} \deltaX^{2}}
%  ,
%  \frac{64}{\nuX} \log \left( \frac{6}{\rhoSD} | \ParamCover | \right)
  ,
  \frac{64}{\nuX} \log \left( \frac{6}{\rhoSD} \binom{\n}{\nO} \right)
  ,
  \frac{\ConstdSD \nO}{\nuX} \log \left( \frac{1}{\nuX} \right)
  \right\}
%  \\
%  &\geq
%  \max \left\{
%  \frac{200 \nuX \log \left( \frac{6}{\rhoSD} | \JSXX | | \ParamCover | \right)}{\left( \frac{\GAMMAX \ConstbSD \deltaX}{2} - \nuX \sqrt{8 \log \left( \frac{e}{\nuX} \right)} \right)^{2}}
%  ,
%  \frac{100 \nuX \kOXX}{\GAMMAX^{2} \ConstbSD^{2} \deltaX^{2}}
%%  ,
%%  \frac{64}{\nuX} \log \left( \frac{6}{\rhoSD} | \ParamCover | \right)
%  ,
%  \frac{64}{\nuX} \log \left( \frac{6}{\rhoSD} \binom{\n}{\nO} \right)
%  ,
%  \frac{\ConstdSD \nO}{\nuX} \log \left( \frac{1}{\nuX} \right)
%  \right\}
\TagEqn{\label{eqn:pf:lemma:concentration-ineq:noiseless:small:4}}
,\end{align*}
%|<<|===========================================================================================|<<|
where \(  \ConstbSD, \ConstdSD > 0  \) are constants as per \DEFINITION \ref{def:univ-const} and \(  \nuX  \) is as defined in \DEFINITION \ref{def:nu-and-tau}.
Then, with probability at least
%|>>|===========================================================================================|>>|
\begin{align*}
  &
  1
  -
  5 | \JS | | \ParamCover | \qX e^{-\frac{1}{8} \nuX \m \tX^{2}}
  -
  \binom{\n}{\nO} e^{-\frac{1}{64} \nuX \m}
  \\
  &\geq
  1
  -
  5 | \JS | | \ParamCover | \qX
  e^{-\frac{1}{8} \nuX \m \cdot \frac{8}{\nuX \m} \log ( \frac{6}{\rhoSD} | \JS | | \ParamCover | \qX )}
  -
  \binom{\n}{\nO}
  e^{-\frac{1}{64} \nuX \cdot \frac{64}{\nuX} \log ( \frac{6}{\rhoSD} \binom{\n}{\nO} )}
  \\
  &\dCmt{by the choices of \(  \tX, \m  \) in \EQUATIONS \eqref{eqn:pf:lemma:concentration-ineq:noiseless:small:3} and \eqref{eqn:pf:lemma:concentration-ineq:noiseless:small:4}}
  \\
  &\geq
  1 - \rhoSD
,\end{align*}
%|<<|===========================================================================================|<<|
for all \(  \JCoordsXX \in \JSXX  \), \(  \thetaX \in \ParamCoverX  \), and \(  \thetaXX \in \BallXCover{\thetaX}  \),
%|>>|===========================================================================================|>>|
\begin{align*}
  \| \hFn[\JCoordsXX]( \thetaX, \thetaXX ) - \E[ \hFn[\JCoordsXX]( \thetaX, \thetaXX ) ] \|_{2}
  &\leq
  5 \max \left\{
    \sqrt{\frac{\kOXX \nuX}{\m}},
    \nuX \tX
  \right\}
  \\
  &\dCmt{by \EQUATION \eqref{eqn:pf:lemma:concentration-ineq:noiseless:small:2}}
  \\
  &=
  5 \max \left\{
    \sqrt{\frac{\kOXX \nuX}{\m}},
    \sqrt{\frac{8 \nuX \log \left( \frac{6}{\rhoSD} | \JS | | \ParamCover | \right)}{\m}}
    +
    \nuX \sqrt{8 \log \left( \frac{e}{\nuX} \right)}
  \right\}
  \\
  &\dCmt{by the choice of \(  \tX  \) specified in \EQUATION \eqref{eqn:pf:lemma:concentration-ineq:noiseless:small:3}}
  \\
  &\leq
  \sqrt{\frac{\pi}{8}} \gammaX \ConstbSD \deltaX
  .\\
  &\dCmt{by the choice of \(  \m  \) specified in \EQUATION \eqref{eqn:pf:lemma:concentration-ineq:noiseless:small:4}}
  \\
  &\dCmtx{and the definition of \(  \nuX  \) in \EQUATION \eqref{eqn:def:nu-and-tau:nu}}
\end{align*}
%|<<|===========================================================================================|<<|
The bound in \EQUATION \eqref{eqn:lemma:small-dist:ub} of \LEMMA \ref{lemma:small-dist} now follows: with probability at least \(  1-\rhoSD  \), uniformly for every \(  \JCoordsXX \in \JSXX  \), \(  \thetaX \in \ParamCoverX  \), and \(  \thetaXX \in \BallXCover{\thetaX}  \),
%|>>|===========================================================================================|>>|
\begin{align*}
  \frac
  {2 \| \hFn[\JCoordsXX]( \thetaX, \thetaXX ) - \E[ \hFn[\JCoordsXX]( \thetaX, \thetaXX ) ] \|_{2}}
  {\DENOM}
  % =
  % \frac
  % {2 \| \hFn[\JCoordsXX]( \thetaX, \thetaXX ) - \E[ \hFn[\JCoordsXX]( \thetaX, \thetaXX ) ] \|_{2}}
  % {\sqrt{\hfrac{\pi}{2}}}
  \leq
  \sqrt{\frac{8}{\pi}}\frac{1}{\gammaX} \sqrt{\frac{\pi}{8}} \gammaX \ConstbSD \deltaX
  =
  \ConstbSD \deltaX
.\end{align*}
%|<<|===========================================================================================|<<|
\end{proof}
%|<<|~~~~~~~~~~~~~~~~~~~~~~~~~~~~~~~~~~~~~~~~~~~~~~~~~~~~~~~~~~~~~~~~~~~~~~~~~~~~~~~~~~~~~~~~~~~|<<|
%|<<|~~~~~~~~~~~~~~~~~~~~~~~~~~~~~~~~~~~~~~~~~~~~~~~~~~~~~~~~~~~~~~~~~~~~~~~~~~~~~~~~~~~~~~~~~~~|<<|
%|<<|~~~~~~~~~~~~~~~~~~~~~~~~~~~~~~~~~~~~~~~~~~~~~~~~~~~~~~~~~~~~~~~~~~~~~~~~~~~~~~~~~~~~~~~~~~~|<<|
%|>>|~~~~~~~~~~~~~~~~~~~~~~~~~~~~~~~~~~~~~~~~~~~~~~~~~~~~~~~~~~~~~~~~~~~~~~~~~~~~~~~~~~~~~~~~~~~|>>|
%|>>|~~~~~~~~~~~~~~~~~~~~~~~~~~~~~~~~~~~~~~~~~~~~~~~~~~~~~~~~~~~~~~~~~~~~~~~~~~~~~~~~~~~~~~~~~~~|>>|
%|>>|~~~~~~~~~~~~~~~~~~~~~~~~~~~~~~~~~~~~~~~~~~~~~~~~~~~~~~~~~~~~~~~~~~~~~~~~~~~~~~~~~~~~~~~~~~~|>>|
\begin{proof}
{\LEMMA \ref{lemma:large-dist:2}}
%
\checkoffbutmayberecheck%
%
Fix any \(  \thetaStar \in \ParamSpace  \).
Let
%|>>|:::::::::::::::::::::::::::::::::::::::::::::::::::::::::::::::::::::::::::::::::::::::::::|>>|
\(  \JCoordsX \in \JSX  \)
%|<<|:::::::::::::::::::::::::::::::::::::::::::::::::::::::::::::::::::::::::::::::::::::::::::|<<|
be arbitrary.
%
%Note that
%%|>>|:::::::::::::::::::::::::::::::::::::::::::::::::::::::::::::::::::::::::::::::::::::::::::|>>|
%\(  | \JSX | \leq | \JS | | \ParamCover |  \).
%%|<<|:::::::::::::::::::::::::::::::::::::::::::::::::::::::::::::::::::::::::::::::::::::::::::|<<|
%and let
%%|>>|:::::::::::::::::::::::::::::::::::::::::::::::::::::::::::::::::::::::::::::::::::::::::::|>>|
%\(  \JCoordsX \defeq \Supp( \thetaX ) \cup \JCoords  \).
%%|<<|:::::::::::::::::::::::::::::::::::::::::::::::::::::::::::::::::::::::::::::::::::::::::::|<<|
%Define
%%|>>|:::::::::::::::::::::::::::::::::::::::::::::::::::::::::::::::::::::::::::::::::::::::::::|>>|
%\(  \JSX \defeq \{ \Supp( \Vec{\theta}' ) \cup \JCoords'' \subseteq [\n] : \Vec{\theta}' \in \ParamCoverX, \JCoords'' \in \JS \}  \),
%%|<<|:::::::::::::::::::::::::::::::::::::::::::::::::::::::::::::::::::::::::::::::::::::::::::|<<|
%and note that \(  \JCoordsX \in \JSX  \), and that
%%|>>|:::::::::::::::::::::::::::::::::::::::::::::::::::::::::::::::::::::::::::::::::::::::::::|>>|
%\(  | \JSX | \leq | \JS | | \ParamCover |  \).
%%|<<|:::::::::::::::::::::::::::::::::::::::::::::::::::::::::::::::::::::::::::::::::::::::::::|<<|
Once again, due to \EQUATION \eqref{eqn:lemma:concentration-ineq:ev:4} in \LEMMA \ref{lemma:concentration-ineq},
%|>>|===========================================================================================|>>|
\begin{gather}
\label{eqn:pf:lemma:large-dist:2:2}
  \| \E[ \thetaX + \hfFn[\JCoordsX]( \thetaStar, \thetaX ) ] \|_{2}
  =
  \sqrt{\frac{\pi}{2}} \gammaX
.\end{gather}
%|<<|===========================================================================================|<<|
Then, inserting \eqref{eqn:pf:lemma:large-dist:2:2} into the \RHS \EQUATION \eqref{eqn:lemma:large-dist:2:ub} in \LEMMA \ref{lemma:large-dist:2},
%|>>|===========================================================================================|>>|
\begin{gather}
\label{eqn:pf:lemma:large-dist:2:3}
  \frac
  {2 \| \hfFn[\JCoordsX]( \thetaStar, \thetaStar ) - \E[ \hfFn[\JCoordsX]( \thetaStar, \thetaStar ) ] \|_{2}}
  {\DENOM}
  =
  \frac
  {2 \| \hfFn[\JCoordsX]( \thetaStar, \thetaStar ) - \E[ \hfFn[\JCoordsX]( \thetaStar, \thetaStar ) ] \|_{2}}
  {\sqrt{\hfrac{\pi}{2}} \gammaX}
.\end{gather}
%|<<|===========================================================================================|<<|
The \(  \lnorm{2}  \)-distance between \(  \hfFn[\JCoordsX]( \thetaStar, \thetaStar )  \) and its mean, \ie the term
%|>>|:::::::::::::::::::::::::::::::::::::::::::::::::::::::::::::::::::::::::::::::::::::::::::|>>|
\(  \| \hfFn[\JCoordsX]( \thetaStar, \thetaStar ) - \E[ \hfFn[\JCoordsX]( \thetaStar, \thetaStar ) ] \|_{2}  \)
%|<<|:::::::::::::::::::::::::::::::::::::::::::::::::::::::::::::::::::::::::::::::::::::::::::|<<|
%in the numerator
on the \RHS of \EQUATION \eqref{eqn:pf:lemma:large-dist:2:3},
is bound from above as follows.
Take \(  \sXXX, \tXX > 0  \) in \EQUATION \eqref{eqn:lemma:concentration-ineq:pr:2} in \LEMMA \ref{lemma:concentration-ineq} to be
%|>>|===========================================================================================|>>|
\begin{gather}
\label{eqn:pf:lemma:large-dist:2:s'}
  \sXXX \defeq \sqrt{\frac{3 \log \left( \frac{3}{\rhoLDXX} \right)}{\alphaO \m}}
\end{gather}
%|<<|===========================================================================================|<<|
and
%|>>|===========================================================================================|>>|
\begin{align}
\label{eqn:pf:lemma:large-dist:2:t'}
  \tXX
  &\defeq
  \max \left\{
    \sqrt{\frac{3 \log \left( \frac{6}{\rhoLDXX} | \JSX | \right)}{\alphaO \m}}
    ,
    \sqrt{\frac{2 ( 1+\sXXX ) \log \left( \frac{3}{\rhoLDXX} | \JSX | \right)}{\alphaO \m}}
  \right\}
,\end{align}
%|<<|===========================================================================================|<<|
and take \(  \m  \) to be bounded from below by
%|>>|===========================================================================================|>>|
\begin{align}
\label{eqn:pf:lemma:large-dist:2:m:2}
  \m
  &\geq
  \max \left\{
  \frac{64 \alphaO}{\GAMMAX^{2} \ConstbLD^{2} \deltaX^{2}}
  \max \left\{
    3 \log \left( \frac{6}{\rhoLDXX} | \JS | | \ParamCover | \right)
    ,
    2 ( \kOX-1 )
  \right\}
  ,
  \frac{4}{\alphaO} \log \left( \frac{6}{\rhoLDXX} | \JS | | \ParamCover | \right)
  \right\}
%
% \max \left\{
  % \frac{4}{\GAMMAX^{2} \ConstbLD^{2} \deltaX^{2}}
  % \max \left\{
  %   24\pi \alphaO \log \left( \frac{6}{\rhoLDXX} | \JS | | \ParamCover | \right)
  %   ,
  %   16\pi \alphaO ( \kOX-1 )
  % \right\}
  % ,
  % \frac{4}{\alphaO} \log \left( \frac{6}{\rhoLDXX} | \JS | | \ParamCover | \right)
  % \right\}
,\end{align}
%|<<|===========================================================================================|<<|
where this choice of \(  \m  \) satisfies
%|>>|===========================================================================================|>>|
\begin{align*}
  \m
%  &\geq
%  \frac{4 \ConstbLD^{2}}{\GAMMAX^{2} \deltaX^{2}}
%  \max \left\{
%    24\pi \alphaX \log \left( \frac{12}{\rhoLDX} | \JS | | \ParamCover | \right)
%    ,
%    8\pi \alphaX ( 1+\sXXX )( \kOX-1 )
%  \right\}
%  \\
  &\geq
  \max \left\{
  \frac{64 \alphaO}{\GAMMAX^{2} \ConstbLD^{2} \deltaX^{2}}
  \max \left\{
    3 \log \left( \frac{6}{\rhoLDXX} | \JSX | \right)
    ,
    ( 1+\sXXX )( \kOX-1 )
  \right\}
  ,
  \frac{4}{\alphaO} \log \left( \frac{6}{\rhoLDXX} | \JSX | \right)
  \right\}
%
  % \max \left\{
  % \frac{4}{\GAMMAX^{2} \ConstbLD^{2} \deltaX^{2}}
  % \max \left\{
  %   24\pi \alphaO \log \left( \frac{6}{\rhoLDXX} | \JSX | \right)
  %   ,
  %   8\pi \alphaO ( 1+\sXXX )( \kOX-1 )
  % \right\}
  % ,
  % \frac{4}{\alphaO} \log \left( \frac{6}{\rhoLDXX} | \JSX | \right)
  % \right\}
\end{align*}
%|<<|===========================================================================================|<<|
because \(  | \JSX | \leq | \JS | | \ParamCover |  \).
This condition on \(  \m  \) also ensures that \(  \sXXX, \tXX < 1  \), as required to utilize \LEMMA \ref{lemma:concentration-ineq}.
Then, observe:
%|>>|===========================================================================================|>>|
\begin{align*}
  \sqrt{\frac{2\pi \alphaO ( 1+\sXXX )( \kOX-1 )}{\m}}
  +
  \sqrt{2\pi} \alphaO \tXX
  &\leq
  \sqrt{\frac{2\pi \alphaO ( 1+\sXXX )( \kOX-1 )}{\m}}
  +
  \sqrt{\frac{6\pi \alphaO \log \left( \frac{6}{\rhoLDXX} | \JSX | \right)}{\m}}
  \\
  &\leq
  \frac{1}{2} \cdot \sqrt{\frac{\pi}{8}} \gammaX \ConstbLD \delta
  +
  \frac{1}{2} \cdot \sqrt{\frac{\pi}{8}} \gammaX \ConstbLD \delta
  \\
  &=
  \sqrt{\frac{\pi}{8}} \gammaX \ConstbLD \delta
\TagEqn{\label{eqn:pf:lemma:large-dist:2:6}}
,\end{align*}
%|<<|===========================================================================================|<<|
where \(  \ConstbLD > 0  \) is a constant as per \DEFINITION \ref{def:univ-const}.
Together with \EQUATION \eqref{eqn:pf:lemma:large-dist:1:5} from the proof of \LEMMA \ref{lemma:large-dist:1}, \EQUATION \eqref{eqn:pf:lemma:large-dist:2:6} gives way to the following bound:
%|>>|===========================================================================================|>>|
\begin{align*}
  &
  \Pr \left(
    \Forall{\JCoordsX \in \JSX}{
    \left\| \hFn[\JCoordsX]( \thetaStar, \thetaStar ) - \E \left[ \hFn[\JCoordsX]( \thetaStar, \thetaStar ) \right] \right\|_{2}
    \leq
    \sqrt{\frac{\pi}{8}} \gammaX \ConstbLD \deltaX
    }
  \right)
  \\
  &\geq
  \Pr \left(
    \Forall{\JCoordsX \in \JSX}{
    \left\| \hFn[\JCoordsX]( \thetaStar, \thetaStar ) - \E \left[ \hFn[\JCoordsX]( \thetaStar, \thetaStar ) \right] \right\|_{2}
    \leq
  \sqrt{\frac{2\pi \alphaO ( 1+\sXXX )( \kOX-1 )}{\m}}
  +
  \sqrt{2\pi} \alphaO \tXX
    }
  \right)
  \\
  &\dCmt{due to \EQUATIONS \eqref{eqn:pf:lemma:large-dist:1:5} and \eqref{eqn:pf:lemma:large-dist:2:6}}
  \\
  &\geq
  1
  -
  2 | \JSX | e^{-\frac{1}{3} \alphaO \m \tXX^{2}}
  -
  | \JSX | e^{-\frac{1}{2} \frac{\alphaO \m \tXX^{2}}{1+\sXXX}}
  -
  e^{-\frac{1}{3} \alphaO \m \sXXX^{2}}
  \\
  &\dCmt{by \EQUATION \eqref{eqn:lemma:concentration-ineq:pr:2} in \LEMMA \ref{lemma:concentration-ineq}}
  \\
  &\geq
  1 - \frac{\rhoLDXX}{3} - \frac{\rhoLDXX}{3} - \frac{\rhoLDXX}{3}
  \\
  &\dCmt{by the choice of \(  \sXXX, \tXX  \) in \EQUATIONS \eqref{eqn:pf:lemma:large-dist:2:s'} and \eqref{eqn:pf:lemma:large-dist:2:t'}}
  \\
  &=
  1 - \rhoLDXX
\TagEqn{\label{eqn:pf:lemma:large-dist:2:8}}
.\end{align*}
%|<<|===========================================================================================|<<|
Therefore, taken together, \EQUATIONS \eqref{eqn:pf:lemma:large-dist:2:2} and \eqref{eqn:pf:lemma:large-dist:2:8} imply that if \(  \m  \) is bounded from below as in \EQUATION \eqref{eqn:pf:lemma:large-dist:2:m:2}, then with probability at least \(  1-\rhoLDXX  \), for all \(  \JCoordsX \in \JSX  \),
%|>>|===========================================================================================|>>|
\begin{align*}
  \frac
  {2 \| \hfFn[\JCoordsX]( \thetaStar, \thetaStar ) - \E[ \hfFn[\JCoordsX]( \thetaStar, \thetaStar ) ] \|_{2}}
  {\DENOM}
  &=
  \frac
  {2 \| \hfFn[\JCoordsX]( \thetaStar, \thetaStar ) - \E[ \hfFn[\JCoordsX]( \thetaStar, \thetaStar ) ] \|_{2}}
  {\sqrt{\hfrac{\pi}{2}} \gammaX}
  \\
  &\leq
  \sqrt{\frac{8}{\pi}} \frac{1}{\gammaX}
  \sqrt{\frac{\pi}{8}} \gammaX
  \ConstbLD \deltaX
  \\
  &=
  \ConstbLD \deltaX
,\end{align*}
%|<<|===========================================================================================|<<|
thus establishing \LEMMA \ref{lemma:large-dist:2}.
\end{proof}
%|<<|~~~~~~~~~~~~~~~~~~~~~~~~~~~~~~~~~~~~~~~~~~~~~~~~~~~~~~~~~~~~~~~~~~~~~~~~~~~~~~~~~~~~~~~~~~~|<<|
%|<<|~~~~~~~~~~~~~~~~~~~~~~~~~~~~~~~~~~~~~~~~~~~~~~~~~~~~~~~~~~~~~~~~~~~~~~~~~~~~~~~~~~~~~~~~~~~|<<|
%|<<|~~~~~~~~~~~~~~~~~~~~~~~~~~~~~~~~~~~~~~~~~~~~~~~~~~~~~~~~~~~~~~~~~~~~~~~~~~~~~~~~~~~~~~~~~~~|<<|

%%%%%%%%%%%%%%%%%%%%%%%%%%%%%%%%%%%%%%%%%%%%%%%%%%%%%%%%%%%%%%%%%%%%%%%%%%%%%%%%%%%%%%%%%%%%%%%%%%%%
%%%%%%%%%%%%%%%%%%%%%%%%%%%%%%%%%%%%%%%%%%%%%%%%%%%%%%%%%%%%%%%%%%%%%%%%%%%%%%%%%%%%%%%%%%%%%%%%%%%%