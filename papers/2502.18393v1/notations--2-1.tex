\section{Notations and Key Properties of GLMs}
\label{outline:notations}

%The set of all real numbers is denoted by \(  \R  \), and the set of all
%integers is written as \(  \Z  \).
For a set of real numbers, \(  \Set{S} \subseteq \R  \), let \(  \Set{S}_{\geq 0}, \Set{S}_{+} \subseteq \Set{S}  \) denote the sets of nonnegative elements and, respectively, positive elements in \(  \Set{S}  \)---formally,
%|>>|:::::::::::::::::::::::::::::::::::::::::::::::::::::::::::::::::::::::::::::::::::::::::::|>>|
\(  \Set{S}_{\geq 0} \defeq \{ s \in \Set{S} : s \geq 0 \}  \) and
\(  \Set{S}_{+} \defeq \{ s \in \Set{S} : s > 0 \}  \).
%|<<|:::::::::::::::::::::::::::::::::::::::::::::::::::::::::::::::::::::::::::::::::::::::::::|<<|
%Write the power set of the set \(  \Set{S}  \) as \(  2^{\Set{S}}  \).
For \(  t \in \Z_{+}  \), let \(  [t] \defeq \{ 1, \dots, t \}  \).
Vectors and matrices are denoted in lowercase and uppercase bold typeface, respectively, e.g.,
%|>>|:::::::::::::::::::::::::::::::::::::::::::::::::::::::::::::::::::::::::::::::::::::::::::|>>|
\(  \Vec{v} \in \R^{n}  \) and \(  \Mat{M} \in \R^{m \times n}  \),
%|<<|:::::::::::::::::::::::::::::::::::::::::::::::::::::::::::::::::::::::::::::::::::::::::::|<<|
with their entries in italic font such that, e.g., \(  \Vec{v} = ( \Vec*{v}_{j} )_{j \in [\n]}  \) and \(  \Mat{M} = ( \Mat*{M}_{i,j} )_{(i,j) \in [\m] \times [\n]}  \).
The all-zeros vector is written in boldface: \(  \Vec{0} = ( 0, \dots, 0 )  \).
For a coordinate subset \(  J \subseteq [\n]  \), the vector with entries indexed by \(  J  \) taking value \(  1  \) and with all other entries set to \(  0  \) is written as \(  \BVec{J} \in \R^{\n}  \).
%The \(  \n \times \n  \) identity matrix (the matrix with all ones along the diagonal and all zeros off the diagonal) is written as
%|>>|:::::::::::::::::::::::::::::::::::::::::::::::::::::::::::::::::::::::::::::::::::::::::::|>>|
% \(  \Id{\n} \in \R^{\n \times \n}  \).
%|<<|:::::::::::::::::::::::::::::::::::::::::::::::::::::::::::::::::::::::::::::::::::::::::::|<<|
% The \(  \Ell \times \Ell  \) diagonal matrix, \(  \Ell \in \Z_{+}  \), whose diagonal entries are specified by a vector \(  \Vec{v} \in \R^{\Ell}  \) (with all off-diagonal entries set to  \(  0  \)) is denoted by \(  \Diag( \Vec{v} ) \in \R^{\Ell \times \Ell}  \).
%For \realvalued \(  p \geq 1  \), the \(  \lnorm{p}  \)-norm of a vector \(  \Vec{v} \in \R^{\n}  \) is written as \(  \| \Vec{v} \|_{p} \defeq ( \smash{\sum_{\jIx=1}^{\n} | \Vec*{v}_{\jIx} |^{p}} )^{1/p}  \).
%Additionally, 
%The \(  \lnorm{0}  \)-``norm'' is denoted by
%|>>|:::::::::::::::::::::::::::::::::::::::::::::::::::::::::::::::::::::::::::::::::::::::::::|>>|
%\(  \| \Vec{v} \|_{0} \defeq | \Supp( \Vec{v} ) |  \),
%|<<|:::::::::::::::::::::::::::::::::::::::::::::::::::::::::::::::::::::::::::::::::::::::::::|<<|
%where
%|>>|:::::::::::::::::::::::::::::::::::::::::::::::::::::::::::::::::::::::::::::::::::::::::::|>>|
%\(  \Supp( \Vec{v} ) \defeq \{ \jIx \in [\n] : \Vec*{v}_{\jIx} \neq 0 \}  \)
%|<<|:::::::::::::::::::::::::::::::::::::::::::::::::::::::::::::::::::::::::::::::::::::::::::|<<|
%is the \emph{support}---or nonzero entries---of the vector \(  \Vec{v}  \).
%The origin-centered, unit \(  ( \n-1 )  \)-sphere (the hypersphere in \(  \R^{\n}  \)) is written as
%|>>|:::::::::::::::::::::::::::::::::::::::::::::::::::::::::::::::::::::::::::::::::::::::::::|>>|
%\(  \Sphere{\n} \defeq \{ \Vec{v} \in \R^{\n} : \| \Vec{v} \|_{2} = 1 \}  \),
%|<<|:::::::::::::::::::::::::::::::::::::::::::::::::::::::::::::::::::::::::::::::::::::::::::|<<|
%and 
The set of all \(  s  \)-sparse, real-valued vectors is denoted by
%|>>|:::::::::::::::::::::::::::::::::::::::::::::::::::::::::::::::::::::::::::::::::::::::::::|>>|
\(  \SparseSubspace{s}{\n} \defeq \{ \Vec{v} \in \R^{\n} : \| \Vec{v} \|_{0} \leq s \}  \).
%|<<|:::::::::::::::::::::::::::::::::::::::::::::::::::::::::::::::::::::::::::::::::::::::::::|<<|
For \(  r > 0  \) and \(  \Vec{v} \in \R^{\n}  \), specify the radius-\(  r  \) ball around \(  \Vec{v}  \) by
%|>>|:::::::::::::::::::::::::::::::::::::::::::::::::::::::::::::::::::::::::::::::::::::::::::|>>|
\(  \Ball{r}( \Vec{v} ) \defeq \{ \Vec{u} \in \R^{\n} : \| \Vec{u} - \Vec{v} \|_{2} \leq r \}  \),
%|<<|:::::::::::::::::::::::::::::::::::::::::::::::::::::::::::::::::::::::::::::::::::::::::::|<<|
and let
%|>>|:::::::::::::::::::::::::::::::::::::::::::::::::::::::::::::::::::::::::::::::::::::::::::|>>|
\(  \BallX{r}( \Vec{v} ) \defeq \Ball{r}( \Vec{v} ) \cap \Sphere{\n} \cap \{ \Vec{u} \in \R^{\n} : \Supp( \Vec{u} ) = \Supp( \Vec{v} ) \}  \).
%|<<|:::::::::::::::::::::::::::::::::::::::::::::::::::::::::::::::::::::::::::::::::::::::::::|<<|
Let \(  \Set{X}  \), \(  \Set{Y}  \), and \(  \Set{U} \subseteq \Set{X}  \) be sets, and let \(  f : \Set{X} \to \Set{Y}  \).
The image of \(  \Set{U}  \) under \(  f  \) is denoted by
%|>>|:::::::::::::::::::::::::::::::::::::::::::::::::::::::::::::::::::::::::::::::::::::::::::|>>|
\(  f[ \Set{U} ] \subseteq \Set{Y}  \).
%|<<|:::::::::::::::::::::::::::::::::::::::::::::::::::::::::::::::::::::::::::::::::::::::::::|<<|
%Additionally, let \(  \Set{X}  \), \(  \Set{Y}  \), and \(  \Set{Z}  \) be sets, and let \(  f : \Set{Y} \to \Set{Z}  \) and \(  g : \Set{X} \to \Set{Y}  \).
%The composition of \(  f  \) and \(  g  \) is written as \(  f \circ g : \Set{X} \to \Set{Z}  \), where \(  f \circ g ( x ) = f( g( x ) )  \) for \(  x \in \Set{X}  \).
The natural logarithm is denoted by \(  \log : \R \to \R  \).
The indicator function, \(  \I  \), is given for a true/false condition, \(  C  \), by \(\I( C )
  = 0\) if \(C\) is false and \(\I( C )
  = 1\) if \(C\) is true,
%|>>|===========================================================================================|>>|
% \begin{gather*}
%   \I( C )
%   =
%   \begin{cases}
%   0 ,&\cIf C \text{ is false} ,\\
%   1 ,&\cIf C \text{ is true} ,\\
%   \end{cases}
% \end{gather*}
%|<<|===========================================================================================|<<|
where this notation extends to vectors by applying it \entrywise.
The sign function, \(  \SignOp : \R \to \{ -1, 1 \}  \), simply returns the \(  \pm  \) sign of the input, i.e., \(\Sign( a )
  = +1\) if and only if \(a\geq 0\),
%|>>|===========================================================================================|>>|
% \begin{gather*}
%   \Sign( a )
%   =
%   \begin{cases}
%    -1 ,&\cIf a    < 0, \\
%   \+1 ,&\cIf a \geq 0.
%   \end{cases}
% \end{gather*}
% %|<<|===========================================================================================|<<|
 for \(  a \in \R  \) with this notation extending to vectors by applying it \entrywise.
For \(  s \in \Z_{+}  \), \emph{the \topk[s] thresholding operation}, written \(  \Threshold{s} : \R^{\n} \to \R^{\n}  \), maps \(  \Vec{v} \mapsto \Threshold{s}( \Vec{v} )  \), where \(  \Threshold{s}( \Vec{v} )  \) retains the largest magnitude entries in \(  \Vec{v}  \) and sets all other entries to \(  0  \) with ties broken arbitrarily.
Similarly, for a set \(  J \subseteq [\n]  \), define the \emph{subset thresholding operation}, denoted by \(  \ThresholdSet{J} : \R^{\n} \to \R^{\n}  \), to be the map which takes a vector \(  \Vec{v} \in \R^{\n}  \) to a vector with \(  j\Th  \) entries \(  \ThresholdSet{J}( \Vec{v} )_{j} = \Vec*{v}_{j} \I( j \in J )  \), \(  j \in [\n]  \).
Note that the latter thresholding operation is a linear transformation given by
%|>>|:::::::::::::::::::::::::::::::::::::::::::::::::::::::::::::::::::::::::::::::::::::::::::|>>|
\(  \ThresholdSet{J}( \Vec{v} ) = \Diag( \BVec{J} ) \Vec{v}  \)
%|<<|:::::::::::::::::::::::::::::::::::::::::::::::::::::::::::::::::::::::::::::::::::::::::::|<<|
for \(  \Vec{v} \in \R^{\n}  \).
%
%%%%%%%%%%%%%%%%%%%%%%%%%%%%%%%%%%%%%%%%%%%%%%%%%%%%%%%%%%%%%%%%%%%%%%%%%%%%%%%%%%%%%%%%%%%%%%%%%%%%
\par %%%%%%%%%%%%%%%%%%%%%%%%%%%%%%%%%%%%%%%%%%%%%%%%%%%%%%%%%%%%%%%%%%%%%%%%%%%%%%%%%%%%%%%%%%%%%%%
%%%%%%%%%%%%%%%%%%%%%%%%%%%%%%%%%%%%%%%%%%%%%%%%%%%%%%%%%%%%%%%%%%%%%%%%%%%%%%%%%%%%%%%%%%%%%%%%%%%%
%
% An important (and standard) construct for the analysis in this work is a \(  \tauX  \)-net. %which is formalized in the following definition.
% %
% %|>>|###########################################################################################|>>|
% %|>>|###########################################################################################|>>|
% %|>>|###########################################################################################|>>|
% % \begin{definition}
% % \label{def:tau-net}
% Fix \(  \tauX > 0  \).
% Let \(  ( \SX, \dSX )  \) be a metric space.
% A subset,
% %|>>|:::::::::::::::::::::::::::::::::::::::::::::::::::::::::::::::::::::::::::::::::::::::::::|>>|
% \(  \SXX \subseteq \SX  \),
% %|<<|:::::::::::::::::::::::::::::::::::::::::::::::::::::::::::::::::::::::::::::::::::::::::::|<<|
% is a \emph{\(  \tauX  \)-net} over \(  \SX  \) if
% %|>>|:::::::::::::::::::::::::::::::::::::::::::::::::::::::::::::::::::::::::::::::::::::::::::|>>|
% \(  \inf_{\sSXX \in \SXX} \dSX( \sSX, \sSXX ) \leq \tauX  \)
% %|<<|:::::::::::::::::::::::::::::::::::::::::::::::::::::::::::::::::::::::::::::::::::::::::::|<<|
% for all \(  \sSX \in \SX  \).
% %\end{definition}
% %|<<|###########################################################################################|<<|
% %|<<|###########################################################################################|<<|
% %|<<|###########################################################################################|<<|
% %
% %Later on, the following upper bound on the minimal cardinality of a \(  \tauX  \)-net over a hypersphere sphere or a union of hyperspheres will be useful.
% We will use the following upper bound on the minimal cardinality of a \(  \tauX  \)-net of a sphere.
% %
% %|>>|*******************************************************************************************|>>|
% %|>>|*******************************************************************************************|>>|
% %|>>|*******************************************************************************************|>>|
% \begin{lemma}[{\see \eg \cite{vershynin2018high}}]
% \label{lemma:tau-net-cardinality}
% %
% Fix \(  \tauX > 0  \), and
% %Let \(  \dDim, \n \in \Z_{+}  \), \(  \dDim \leq \n  \).
% let \(  \dDim \in \Z_{+}  \).
% %The cardinality of a \(  \tauX  \)-net, \(  \SXX \subset \Sphere{\n}  \), over \(  \Sphere{\n}  \) need not exceed
% There exists an $\ell_2$ \(  \tauX  \)-net, \(  \SXX \subset \Sphere{\dDim}  \), over \(  \Sphere{\dDim}  \) of cardinality not exceeding
% %|>>|:::::::::::::::::::::::::::::::::::::::::::::::::::::::::::::::::::::::::::::::::::::::::::|>>|
% \(  | \SXX | \leq ( \frac{3}{\tauX} )^{\dDim}  \).
% %|<<|:::::::::::::::::::::::::::::::::::::::::::::::::::::::::::::::::::::::::::::::::::::::::::|<<|
% %Moreover, there exists a union of \(  \tauX  \)-nets, \(  \bigcup_{\iIx=1}^{\binom{\n}{\k}}  \), over \(  \SparseSphereSubspace{\dDim}{\n}  \)
% \end{lemma}
% %|<<|*******************************************************************************************|<<|
% %|<<|*******************************************************************************************|<<|
% %|<<|*******************************************************************************************|<<|
%
%%%%%%%%%%%%%%%%%%%%%%%%%%%%%%%%%%%%%%%%%%%%%%%%%%%%%%%%%%%%%%%%%%%%%%%%%%%%%%%%%%%%%%%%%%%%%%%%%%%%
\par %%%%%%%%%%%%%%%%%%%%%%%%%%%%%%%%%%%%%%%%%%%%%%%%%%%%%%%%%%%%%%%%%%%%%%%%%%%%%%%%%%%%%%%%%%%%%%%
%%%%%%%%%%%%%%%%%%%%%%%%%%%%%%%%%%%%%%%%%%%%%%%%%%%%%%%%%%%%%%%%%%%%%%%%%%%%%%%%%%%%%%%%%%%%%%%%%%%%
%
Denote by \(  X \sim \Distr{D}  \) a random variable \(  X  \), which follows a distribution \(  \Distr{D}  \). If \(\Set{S}\) is a set then  \(  X \sim \Set{S}  \) means \(X\) follows the uniform distribution over  \(  \Set{S}  \).
Additionally, the density function and \MGF* (\MGF) (when well-defined) of a random variable, \(  X  \), are written \(  \pdf{X}  \) and \(  \mgf{X}  \), respectively.
%The mean-\(  \bm{\mu}  \), covariance-\(  \Mat{\Sigma}  \) multivariate Gaussian distribution is denoted by
%|>>|:::::::::::::::::::::::::::::::::::::::::::::::::::::::::::::::::::::::::::::::::::::::::::|>>|
% \(  \N( \bm{\mu}, \Mat{\Sigma} )  \),
%|<<|:::::::::::::::::::::::::::::::::::::::::::::::::::::::::::::::::::::::::::::::::::::::::::|<<|
% while the mean-\(  \mu  \), variance-\(  \sigma^{2}  \) univariate Gaussian distribution is simply written as
%|>>|:::::::::::::::::::::::::::::::::::::::::::::::::::::::::::::::::::::::::::::::::::::::::::|>>|
% \(  \N( \mu, \sigma^{2} )  \).
%|<<|:::::::::::::::::::::::::::::::::::::::::::::::::::::::::::::::::::::::::::::::::::::::::::|<<|
%
%%%%%%%%%%%%%%%%%%%%%%%%%%%%%%%%%%%%%%%%%%%%%%%%%%%%%%%%%%%%%%%%%%%%%%%%%%%%%%%%%%%%%%%%%%%%%%%%%%%%
\par %%%%%%%%%%%%%%%%%%%%%%%%%%%%%%%%%%%%%%%%%%%%%%%%%%%%%%%%%%%%%%%%%%%%%%%%%%%%%%%%%%%%%%%%%%%%%%%
%%%%%%%%%%%%%%%%%%%%%%%%%%%%%%%%%%%%%%%%%%%%%%%%%%%%%%%%%%%%%%%%%%%%%%%%%%%%%%%%%%%%%%%%%%%%%%%%%%%%
%
\paragraph{\bf Binary GLMs} 
Throughout this manuscript,
%|>>|:::::::::::::::::::::::::::::::::::::::::::::::::::::::::::::::::::::::::::::::::::::::::::|>>|
\(  \n, \k, \m \in \Z_{+}  \)
%|<<|:::::::::::::::::::::::::::::::::::::::::::::::::::::::::::::::::::::::::::::::::::::::::::|<<|
denote, in order, the dimension of the parameter space, the sparsity, and the number of samples (or measurements), and
the \errorrate is denoted by
%|>>|:::::::::::::::::::::::::::::::::::::::::::::::::::::::::::::::::::::::::::::::::::::::::::|>>|
\(  \epsilonX \in (0,1)  \).
%|<<|:::::::::::::::::::::::::::::::::::::::::::::::::::::::::::::::::::::::::::::::::::::::::::|<<|
%and the \betaXname is written as
%%|>>|:::::::::::::::::::::::::::::::::::::::::::::::::::::::::::::::::::::::::::::::::::::::::::|>>|
%\(  \betaX \in \R_{\geq 0}  \).
%%|<<|:::::::::::::::::::::::::::::::::::::::::::::::::::::::::::::::::::::::::::::::::::::::::::|<<|
The parameter space is written as
%|>>|:::::::::::::::::::::::::::::::::::::::::::::::::::::::::::::::::::::::::::::::::::::::::::|>>|
\(  \ParamSpace = \SparseSphereSubspace{\k}{\n} \subseteq \R^{\n}  \).
%|<<|:::::::::::::::::::::::::::::::::::::::::::::::::::::::::::::::::::::::::::::::::::::::::::|<<|
Note that the results in this manuscript extend to the dense regime by taking
%|>>|:::::::::::::::::::::::::::::::::::::::::::::::::::::::::::::::::::::::::::::::::::::::::::|>>|
\(  \k = \n  \) and \(  \ParamSpace = \Sphere{\n}  \).
%|<<|:::::::::::::::::::::::::::::::::::::::::::::::::::::::::::::::::::::::::::::::::::::::::::|<<|
The covariates are \(  \n  \)-variate \iid standard Gaussian random vectors, written as
%|>>|:::::::::::::::::::::::::::::::::::::::::::::::::::::::::::::::::::::::::::::::::::::::::::|>>|
\(  \CovV\VIx{1}, \dots, \CovV\VIx{\m} \sim \N( \Vec{0}, \Id{\n} )  \),
%|<<|:::::::::::::::::::::::::::::::::::::::::::::::::::::::::::::::::::::::::::::::::::::::::::|<<|
which are stacked up into the covariate matrix,
%|>>|:::::::::::::::::::::::::::::::::::::::::::::::::::::::::::::::::::::::::::::::::::::::::::|>>|
\(  \CovM = ( \CovV\VIx{1} \,\cdots\, \CovV\VIx{\m} )^{\T} \in \R^{\m \times \n}  \).
%|<<|:::::::::::::::::::::::::::::::::::::::::::::::::::::::::::::::::::::::::::::::::::::::::::|<<|
The unknown parameter vector which is being estimated is denoted by
%|>>|:::::::::::::::::::::::::::::::::::::::::::::::::::::::::::::::::::::::::::::::::::::::::::|>>|
\(  \thetaStar \in \ParamSpace  \),
%|<<|:::::::::::::::::::::::::::::::::::::::::::::::::::::::::::::::::::::::::::::::::::::::::::|<<|
and the \(  \Iter\Th  \) approximations produced by the \(  \Iter\Th  \) iteration of the recovery algorithms are written as
%|>>|:::::::::::::::::::::::::::::::::::::::::::::::::::::::::::::::::::::::::::::::::::::::::::|>>|
\(  \thetaHat[\Iter] \in \ParamSpace  \),
%|<<|:::::::::::::::::::::::::::::::::::::::::::::::::::::::::::::::::::::::::::::::::::::::::::|<<|
where \(  \Iter \in \Z_{\geq 0}  \).
There is assumed access to the covariates, \(  \CovV\VIx{\iIx}  \), \(  \iIx \in [\m]  \), as well as \(  \m  \) binary measurement responses specified as the vector
%|>>|:::::::::::::::::::::::::::::::::::::::::::::::::::::::::::::::::::::::::::::::::::::::::::|>>|
%\(  \RespV = \fFn( \CovM \thetaStar )  \).
\(  \RespV \in \{ -1,1 \}^{\m}  \).
%|<<|:::::::::::::::::::::::::::::::::::::::::::::::::::::::::::::::::::::::::::::::::::::::::::|<<|
%


For a function,
%|>>|:::::::::::::::::::::::::::::::::::::::::::::::::::::::::::::::::::::::::::::::::::::::::::|>>|
\(  \pFn : \R \to [0,1]  \),
%|<<|:::::::::::::::::::::::::::::::::::::::::::::::::::::::::::::::::::::::::::::::::::::::::::|<<|
the \(  \iIx\Th  \) measurement responses, \(  \RespV*\VIx{\iIx} \in \{ -1,1 \}  \), \(  \iIx \in [\m]  \), are obtained through a random function \(  \fFn : \R \to \{ -1,1 \}  \), given by
%|>>|===========================================================================================|>>|
\begin{gather}
\label{eqn:notations:f:def}
  \fFn( z )
  =
  \begin{cases}
  -1 ,&\cWP 1 - \pFn( z )  ,\\
  \+1 ,&\cWP \pFn( z ) ,
  \end{cases}
\end{gather}
%|<<|===========================================================================================|<<|
for \(  z \in \R  \),
such that
%|>>|:::::::::::::::::::::::::::::::::::::::::::::::::::::::::::::::::::::::::::::::::::::::::::|>>|
\(  \RespV*\VIx{\iIx} = \fFn( \langle \CovV\VIx{\iIx}, \thetaStar \rangle )  \),
%|<<|:::::::::::::::::::::::::::::::::::::::::::::::::::::::::::::::::::::::::::::::::::::::::::|<<|
where the notation of \(  \fFn  \) extends to vectors, i.e., \(  \fFn : \R^{\m} \to \{ -1,1 \}^{\m}  \), by applying it \entrywise independently so that the response vector is given concisely by
%|>>|:::::::::::::::::::::::::::::::::::::::::::::::::::::::::::::::::::::::::::::::::::::::::::|>>|
\(  \RespV = \fFn( \CovM \thetaStar )  \).
%|<<|:::::::::::::::::::::::::::::::::::::::::::::::::::::::::::::::::::::::::::::::::::::::::::|<<|
%Notice that in the limit as \(  \betaX \to \infty  \), the function, \(  \fFn  \), converges to the sign function, defined earlier.

%|<<|:::::::::::::::::::::::::::::::::::::::::::::::::::::::::::::::::::::::::::::::::::::::::::|<<|
%notions and aspects
%|>>|===========================================================================================|>>|
\begin{definition}[``Noise'' and ``Slope'']
    There are two important quantities related to the function \(  \pFn  \), which are concisely represented as the variables
%|>>|:::::::::::::::::::::::::::::::::::::::::::::::::::::::::::::::::::::::::::::::::::::::::::|>>|
\(  \alphaX >0\) that measures the amount of ``noise'' in the random function $f$ compared to the $\Sign$ function; and \( \gammaX > 0  \) that measures the average ``slope'' of the function:
\begin{gather}
\label{eqn:notations:alpha:def}
  \alphaX
  \defeq
  \Pr(
    \fFn( Z ) \neq \Sign( Z )
  )
%  =
%  \frac{1}{\sqrt{2\pi}}
  %\int_{\zX=0}^{\zX=\infty}
%  e^{-\frac{1}{2} \zX^{2}}
%  (\pExpr{\zX})
%  d\zX
  ,\\
\label{eqn:notations:gamma:def}
  \gammaX
  \defeq
  \E[ Z \fFn( Z ) ]
  % =
  % \sqrt{\frac{2}{\pi}}
  % \int_{\zX=0}^{\zX=\infty}
  % \zX
  % e^{-\frac{1}{2} \zX^{2}}
  % ( p( \zX ) - p( -\zX ) )
  % d\zX
%  \int_{\zX=0}^{\zX=\infty}
%  \zX
%  e^{-\frac{1}{2} \zX^{2}}
%  (\pExpr{\zX})
%  d\zX
,\end{gather}
%|<<|===========================================================================================|<<|
where \(  Z \sim \N(0,1)  \) is a standard univariate Gaussian random variable and the probabilities and expectations are with respect to $Z$ and the randomness of $f$. Note that, \(\gammaX
  \leq
  \E[|Z|] = \sqrt{\frac{2}{\pi}} \). From Stein's lemma, if the function $p$ is differentiable then \[\gamma = 2\E[p'(Z)], \]
where the expectation is now with respect to $Z$.
\end{definition}

In addition, for \(  \epsilon > 0  \), define
%|>>|===========================================================================================|>>|
\begin{gather}
\label{eqn:notations:alpha_0:def}
  \alphaO \defeq \max \{ \alphaX, \frac{\epsilonX}{\frac{3}2(5+\sqrt{21})} \}
.\end{gather}
%|<<|===========================================================================================|<<|
%When the parameter \(  \epsilonX  \) is understood from context, this notation will be condensed to
%|>>|:::::::::::::::::::::::::::::::::::::::::::::::::::::::::::::::::::::::::::::::::::::::::::|>>|
%\(  \alphaO = \alphaO( \epsilonX )  \).
%|<<|:::::::::::::::::::::::::::::::::::::::::::::::::::::::::::::::::::::::::::::::::::::::::::|<<|
%
%|>>|♢♢♢♢♢♢♢♢♢♢♢♢♢♢♢♢♢♢♢♢♢♢♢♢♢♢♢♢♢♢♢♢♢♢♢♢♢♢♢♢♢♢♢♢♢♢♢♢♢♢♢♢♢♢♢♢♢♢♢♢♢♢♢♢♢♢♢♢♢♢♢♢♢♢♢♢♢♢♢♢♢♢♢♢♢♢♢♢♢♢♢|>>|
%|>>|♢♢♢♢♢♢♢♢♢♢♢♢♢♢♢♢♢♢♢♢♢♢♢♢♢♢♢♢♢♢♢♢♢♢♢♢♢♢♢♢♢♢♢♢♢♢♢♢♢♢♢♢♢♢♢♢♢♢♢♢♢♢♢♢♢♢♢♢♢♢♢♢♢♢♢♢♢♢♢♢♢♢♢♢♢♢♢♢♢♢♢|>>|
%|>>|♢♢♢♢♢♢♢♢♢♢♢♢♢♢♢♢♢♢♢♢♢♢♢♢♢♢♢♢♢♢♢♢♢♢♢♢♢♢♢♢♢♢♢♢♢♢♢♢♢♢♢♢♢♢♢♢♢♢♢♢♢♢♢♢♢♢♢♢♢♢♢♢♢♢♢♢♢♢♢♢♢♢♢♢♢♢♢♢♢♢♢|>>|
% \begin{remark}
% \label{remark:alpha}
% %
%The variable \(  \alphaX  \) is calculated as follows, where \(  Z \sim \N(0,1)  \) is a standard univariate Gaussian random variable:
%|>>|===========================================================================================|>>|
% \begin{align*}
%   \alphaX
%   &=
%   \Pr(
%     \fFn( Z ) \neq \Sign( Z )
%   )
%   \\
%   % &=
%   % \Pr(
%   %   \fFn( Z ) \neq \Sign( Z ),
%   %   \Sign( Z ) = 1
%   % )
%   % +
%   % \Pr(
%   %   \fFn( Z ) \neq \Sign( Z ),
%   %   \Sign( Z ) = -1
%   % )
%   % \\
%   % &\dCmt{by the law of total probability}
%   % \\
%   &=
%   \Pr(
%     \fFn( Z ) = -1,
%     \Sign( Z ) = 1
%   )
%   +
%   \Pr(
%     \fFn( Z ) = 1,
%     \Sign( Z ) = -1
%   )
%   \\
%   &=
%   \frac{1}{\sqrt{2\pi}}
%   \int_{\zX=0}^{\zX=\infty}
%   e^{-\frac{1}{2} \zX^{2}}
%   ( 1-\pFn( \zX ) )
%   d\zX
%   +
%   \frac{1}{\sqrt{2\pi}}
%   \int_{\zX=-\infty}^{\zX=0}
%   e^{-\frac{1}{2} \zX^{2}}
%   \pFn( \zX )
%   d\zX
%   \\
%   &\dCmt{by \EQUATION \eqref{eqn:notations:f:def} and the density function for standard Gaussians}
%   \\
%   &=
%   \frac{1}{\sqrt{2\pi}}
%   \int_{\zX=0}^{\zX=\infty}
%   e^{-\frac{1}{2} \zX^{2}}
%   ( 1-\pFn( \zX ) )
%   d\zX
%   +
%   \frac{1}{\sqrt{2\pi}}
%   \int_{\zX=0}^{\zX=\infty}
%   e^{-\frac{1}{2} \zX^{2}}
%   \pFn( -\zX )
%   d\zX
%   \\
%   &=
%   \frac{1}{\sqrt{2\pi}}
%   \int_{\zX=0}^{\zX=\infty}
%   e^{-\frac{1}{2} \zX^{2}}
%   (\pExpr{\zX})
%   d\zX
% .\end{align*}
% %|<<|===========================================================================================|<<|
Note that when \(  \pFn(-z) = 1-\pFn(z)  \)---as is the case in logistic and probit regression (\see the formal definitions of these models in \DEFINITIONS \ref{def:p:logistic-regression} and \ref{def:p:probit} below)---the expression for \(  \alphaX  \) simplifies to
%|>>|===========================================================================================|>>|
\begin{gather*}
  \alphaX
  =
  \frac{1}{\sqrt{2\pi}}
  \int_{\zX=0}^{\zX=\infty}
  e^{-\frac{1}{2} \zX^{2}}
  (\pExpr{\zX})
  d\zX
  =
  \sqrt{\frac{2}{\pi}}
  \int_{\zX=0}^{\zX=\infty}
  e^{-\frac{1}{2} \zX^{2}}
  \pFn( -\zX )
  d\zX
.\end{gather*}
%|<<|===========================================================================================|<<|
%%|>>|===========================================================================================|>>|
%\begin{align*}
%  \alphaX
%  &=
%  \frac{1}{\sqrt{2\pi}}
%  \int_{\zX=0}^{\zX=\infty}
%  e^{-\frac{1}{2} \zX^{2}}
%  (\pExpr{\zX})
%  d\zX
%  =
%  \sqrt{\frac{2}{\pi}}
%  \int_{\zX=0}^{\zX=\infty}
%  e^{-\frac{1}{2} \zX^{2}}
%  ( 1-\pFn( \zX ) )
%  d\zX
%  \\
%  &=
%  \sqrt{\frac{2}{\pi}}
%  \int_{\zX=0}^{\zX=\infty}
%  e^{-\frac{1}{2} \zX^{2}}
%  d\zX
%  -
%  \sqrt{\frac{2}{\pi}}
%  \int_{\zX=0}^{\zX=\infty}
%  e^{-\frac{1}{2} \zX^{2}}
%  \pFn( \zX )
%  d\zX
%  =
%  1
%  -
%  \sqrt{\frac{2}{\pi}}
%  \int_{\zX=0}^{\zX=\infty}
%  e^{-\frac{1}{2} \zX^{2}}
%  \pFn( \zX )
%  d\zX
%.\end{align*}
%%|<<|===========================================================================================|<<|
%\end{remark}
%|<<|♢♢♢♢♢♢♢♢♢♢♢♢♢♢♢♢♢♢♢♢♢♢♢♢♢♢♢♢♢♢♢♢♢♢♢♢♢♢♢♢♢♢♢♢♢♢♢♢♢♢♢♢♢♢♢♢♢♢♢♢♢♢♢♢♢♢♢♢♢♢♢♢♢♢♢♢♢♢♢♢♢♢♢♢♢♢♢♢♢♢♢|<<|
%|<<|♢♢♢♢♢♢♢♢♢♢♢♢♢♢♢♢♢♢♢♢♢♢♢♢♢♢♢♢♢♢♢♢♢♢♢♢♢♢♢♢♢♢♢♢♢♢♢♢♢♢♢♢♢♢♢♢♢♢♢♢♢♢♢♢♢♢♢♢♢♢♢♢♢♢♢♢♢♢♢♢♢♢♢♢♢♢♢♢♢♢♢|<<|
%|<<|♢♢♢♢♢♢♢♢♢♢♢♢♢♢♢♢♢♢♢♢♢♢♢♢♢♢♢♢♢♢♢♢♢♢♢♢♢♢♢♢♢♢♢♢♢♢♢♢♢♢♢♢♢♢♢♢♢♢♢♢♢♢♢♢♢♢♢♢♢♢♢♢♢♢♢♢♢♢♢♢♢♢♢♢♢♢♢♢♢♢♢|<<|
%
In addition, the function \(  \pFn  \) must satisfy two assumptions, stated together in \ASSUMPTION \ref{assumption:p}, below.
%
%|>>|*******************************************************************************************|>>|
%|>>|*******************************************************************************************|>>|
%|>>|*******************************************************************************************|>>|
\begin{assumption}
\label{assumption:p}
%
The following conditions are enforced on \(  \pFn  \):
\Enum[\label{condition:assumption:p:i}]{i} \(  \pFn  \) monotonically increases over the real line; and
\Enum[\label{condition:assumption:p:ii}]{ii} let $\nu(z) \equiv \pExpr{\zX}$; the function \begin{gather}
\label{eqn:assumption:p:ii}
\frac{\nu(\zX+\wX)}{\nu(\zX)}
 \end{gather}
% \[\frac{P(f(\zX+\wX)=-1)+P(f(-(\zX+\wX))=1)}{P(f(\zX)=-1)+P(f(-\zX)=1)}\]
is non-increasing in $\zX \geq 0,$ for any \(  \wX > 0  \).
\end{assumption}
%The second condition means that for any \(  \wX > 0  \) and \(  \zX, \zX' \geq 0  \), such that \(  \zX \leq \zX'  \), \(  \pFn  \) satisfies
%|>>|===========================================================================================|>>|
% \begin{gather}
% \label{eqn:assumption:p:ii}
%   \frac{\pExpr*{\zX+\wX}}{\pExpr{\zX}} \geq \frac{\pExpr*{\zX'+\wX}}{\pExpr{\zX'}}
% .\end{gather}
%|<<|===========================================================================================|<<|
Intuitively, the second condition means that the ``noise'' of the GLM defined above increases at a slower rate away from ``margin'' ($z=0)$. Indeed, $\nu(z) = P(f(\zX)=-1)+P(f(-\zX)=1)$ for $\zX \ge 0$ can be thought of as a proxy for the noise with respect to the $\Sign$ function, and the ratio in \eqref{eqn:assumption:p:ii} quantifies the growth-rate of the function.

% Note that if \(  \pFn  \) is continuously differentiable over the
% %nonnegative
% real line, then \CONDITION \ref{condition:assumption:p:ii} is equivalent to
% %|>>|===========================================================================================|>>|
% \begin{gather}
% \label{eqn:assumption:p:ii:differentiable}
%   \frac{\partial}{\partial \zX}
%   \frac{\pExpr*{\zX+\wX}}{\pExpr{\zX}} \leq 0
% \end{gather}
% %|<<|===========================================================================================|<<|
% for \(  \wX > 0  \) and \(  \zX \geq 0  \).

%|<<|*******************************************************************************************|<<|
%|<<|*******************************************************************************************|<<|
%|<<|*******************************************************************************************|<<|
%
\begin{comment}
%|>>|♢♢♢♢♢♢♢♢♢♢♢♢♢♢♢♢♢♢♢♢♢♢♢♢♢♢♢♢♢♢♢♢♢♢♢♢♢♢♢♢♢♢♢♢♢♢♢♢♢♢♢♢♢♢♢♢♢♢♢♢♢♢♢♢♢♢♢♢♢♢♢♢♢♢♢♢♢♢♢♢♢♢♢♢♢♢♢♢♢♢♢|>>|
%|>>|♢♢♢♢♢♢♢♢♢♢♢♢♢♢♢♢♢♢♢♢♢♢♢♢♢♢♢♢♢♢♢♢♢♢♢♢♢♢♢♢♢♢♢♢♢♢♢♢♢♢♢♢♢♢♢♢♢♢♢♢♢♢♢♢♢♢♢♢♢♢♢♢♢♢♢♢♢♢♢♢♢♢♢♢♢♢♢♢♢♢♢|>>|
%|>>|♢♢♢♢♢♢♢♢♢♢♢♢♢♢♢♢♢♢♢♢♢♢♢♢♢♢♢♢♢♢♢♢♢♢♢♢♢♢♢♢♢♢♢♢♢♢♢♢♢♢♢♢♢♢♢♢♢♢♢♢♢♢♢♢♢♢♢♢♢♢♢♢♢♢♢♢♢♢♢♢♢♢♢♢♢♢♢♢♢♢♢|>>|
\begin{remark}
\label{remark:condition:assumption:p:ii}
%
Note that for any nonnegative scalar, \(  r \geq 0  \), \CONDITION \ref{condition:assumption:p:ii} of \ASSUMPTION \ref{assumption:p} is equivalently stated with \EQUATION \eqref{eqn:assumption:p:ii} replaced by
%|>>|===========================================================================================|>>|
\begin{gather}
  \frac{\pExpr*{r( \zX+\wX )}}{\pExpr{r \zX}} \geq \frac{\pExpr*{r( \zX'+\wX )}}{\pExpr{r \zX'}}
\end{gather}
%|<<|===========================================================================================|<<|
for \(  \wX > 0  \) and \(  0 \leq \zX \leq \zX'  \),
or, when \(  \pFn  \) is continuously differentiable, with \EQUATION \eqref{eqn:assumption:p:ii:differentiable} replaced by
%|>>|===========================================================================================|>>|
\begin{gather}
  \frac{\partial}{\partial \zX}
  \frac{\pExpr*{r( \zX+\wX )}}{\pExpr{r \zX}} \leq 0
,\end{gather}
%|<<|===========================================================================================|<<|
where \(  \wX > 0  \) and \(  \zX \geq 0  \).
\end{remark}
%|<<|♢♢♢♢♢♢♢♢♢♢♢♢♢♢♢♢♢♢♢♢♢♢♢♢♢♢♢♢♢♢♢♢♢♢♢♢♢♢♢♢♢♢♢♢♢♢♢♢♢♢♢♢♢♢♢♢♢♢♢♢♢♢♢♢♢♢♢♢♢♢♢♢♢♢♢♢♢♢♢♢♢♢♢♢♢♢♢♢♢♢♢|<<|
%|<<|♢♢♢♢♢♢♢♢♢♢♢♢♢♢♢♢♢♢♢♢♢♢♢♢♢♢♢♢♢♢♢♢♢♢♢♢♢♢♢♢♢♢♢♢♢♢♢♢♢♢♢♢♢♢♢♢♢♢♢♢♢♢♢♢♢♢♢♢♢♢♢♢♢♢♢♢♢♢♢♢♢♢♢♢♢♢♢♢♢♢♢|<<|
%|<<|♢♢♢♢♢♢♢♢♢♢♢♢♢♢♢♢♢♢♢♢♢♢♢♢♢♢♢♢♢♢♢♢♢♢♢♢♢♢♢♢♢♢♢♢♢♢♢♢♢♢♢♢♢♢♢♢♢♢♢♢♢♢♢♢♢♢♢♢♢♢♢♢♢♢♢♢♢♢♢♢♢♢♢♢♢♢♢♢♢♢♢|<<|
\end{comment}
%

%Note that, the left hand side of \eqref{eqn:assumption:p:ii} simply is 

Here, it is worth noting---and later, it will be proved---that \ASSUMPTION \ref{assumption:p} is satisfied by two ubiquitous models in binary classification and statistical modeling with binary outcomes: logistic and probit regression.
As these two models will be studied later on, the functions, \(  \pFn  \), corresponding with these models are formally defined below in \DEFINITION \ref{def:p:logistic-regression} and \ref{def:p:probit}.
%In order to provide a more general result,
%To promote
To provide greater generality with these models, these definitions introduce an addition parameter: \(  \betaX \GTR 0  \), which denotes the \betaXnamelr and signal-to-noise ratio (\betaXnamepr) in logistic and probit regression, respectively.
%
%|>>|###########################################################################################|>>|
%|>>|###########################################################################################|>>|
%|>>|###########################################################################################|>>|
\begin{definition}
\label{def:p:logistic-regression}
%
For {\bf logistic regression} with \betaXnamelr \(  \betaX \GTR 0  \), the function
%|>>|:::::::::::::::::::::::::::::::::::::::::::::::::::::::::::::::::::::::::::::::::::::::::::|>>|
\(  \pFn : \R \to [0,1]  \)
%|<<|:::::::::::::::::::::::::::::::::::::::::::::::::::::::::::::::::::::::::::::::::::::::::::|<<|
is given at \(  z \in \R  \) by
%|>>|===========================================================================================|>>|
\begin{gather}
  \pFn( z )
  =
  \frac{1}{1+e^{-\betaX z}}
.\end{gather}
%|<<|===========================================================================================|<<|
\end{definition}
%|<<|###########################################################################################|<<|
%|<<|###########################################################################################|<<|
%|<<|###########################################################################################|<<|
%
%|>>|###########################################################################################|>>|
%|>>|###########################################################################################|>>|
%|>>|###########################################################################################|>>|
\begin{definition}
\label{def:p:probit}
%
For the {\bf probit model} with signal-to-noise ratio (SNR) \(  \betaX \GTR 0  \), the function
%|>>|:::::::::::::::::::::::::::::::::::::::::::::::::::::::::::::::::::::::::::::::::::::::::::|>>|
\(  \pFn : \R \to [0,1]  \)
%|<<|:::::::::::::::::::::::::::::::::::::::::::::::::::::::::::::::::::::::::::::::::::::::::::|<<|
is given at \(  z \in \R  \) by
%|>>|===========================================================================================|>>|
\begin{gather}
  \pFn( z )
  =
  \frac{1}{\sqrt{2\pi}}
  \int_{u=-\infty}^{u=\betaX z}
  e^{-\frac{1}{2} u^{2}}
  du
.\end{gather}
%|<<|===========================================================================================|<<|
Note that, equivalently, \(  \pFn  \) is simply the distribution function of a standard Gaussian random variable composed with multiplication by \(  \betaX  \).
\end{definition}
%|<<|###########################################################################################|<<|
%|<<|###########################################################################################|<<|
%|<<|###########################################################################################|<<|

%%%%%%%%%%%%%%%%%%%%%%%%%%%%%%%%%%%%%%%%%%%%%%%%%%%%%%%%%%%%%%%%%%%%%%%%%%%%%%%%%%%%%%%%%%%%%%%%%%%%
%%%%%%%%%%%%%%%%%%%%%%%%%%%%%%%%%%%%%%%%%%%%%%%%%%%%%%%%%%%%%%%%%%%%%%%%%%%%%%%%%%%%%%%%%%%%%%%%%%%%
%%%%%%%%%%%%%%%%%%%%%%%%%%%%%%%%%%%%%%%%%%%%%%%%%%%%%%%%%%%%%%%%%%%%%%%%%%%%%%%%%%%%%%%%%%%%%%%%%%%%

