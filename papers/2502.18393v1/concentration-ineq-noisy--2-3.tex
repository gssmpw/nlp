%%%%%%%%%%%%%%%%%%%%%%%%%%%%%%%%%%%%%%%%%%%%%%%%%%%%%%%%%%%%%%%%%%%%%%%%%%%%%%%%%%%%%%%%%%%%%%%%%%%%
%%%%%%%%%%%%%%%%%%%%%%%%%%%%%%%%%%%%%%%%%%%%%%%%%%%%%%%%%%%%%%%%%%%%%%%%%%%%%%%%%%%%%%%%%%%%%%%%%%%%
%%%%%%%%%%%%%%%%%%%%%%%%%%%%%%%%%%%%%%%%%%%%%%%%%%%%%%%%%%%%%%%%%%%%%%%%%%%%%%%%%%%%%%%%%%%%%%%%%%%%

\subsection{Proof of \LEMMA \ref{lemma:concentration-ineq:noisy}}
\label{outline:concentration-ineq|pf-noisy}

%|>>|~~~~~~~~~~~~~~~~~~~~~~~~~~~~~~~~~~~~~~~~~~~~~~~~~~~~~~~~~~~~~~~~~~~~~~~~~~~~~~~~~~~~~~~~~~~|>>|
%|>>|~~~~~~~~~~~~~~~~~~~~~~~~~~~~~~~~~~~~~~~~~~~~~~~~~~~~~~~~~~~~~~~~~~~~~~~~~~~~~~~~~~~~~~~~~~~|>>|
%|>>|~~~~~~~~~~~~~~~~~~~~~~~~~~~~~~~~~~~~~~~~~~~~~~~~~~~~~~~~~~~~~~~~~~~~~~~~~~~~~~~~~~~~~~~~~~~|>>|
\begin{proof}
{\LEMMA \ref{lemma:concentration-ineq:noisy}}
%
\checkoff%
%
The proof of the lemma is split across a few subsections within this section, \SECTION \ref{outline:concentration-ineq|pf-noisy}:
\SECTION \ref{outline:concentration-ineq|pf-noisy|1} is devoted to \EQUATIONS \eqref{eqn:lemma:concentration-ineq:noisy:ev:1} and \eqref{eqn:lemma:concentration-ineq:noisy:pr:1}, while \SECTION \ref{outline:concentration-ineq|pf-noisy|2} derives \EQUATIONS \eqref{eqn:lemma:concentration-ineq:noisy:ev:2} and \eqref{eqn:lemma:concentration-ineq:noisy:pr:2}.
Lastly, \SECTION \ref{outline:concentration-ineq|pf-noisy|pf-f1,f2} proves an intermediate result.

%%%%%%%%%%%%%%%%%%%%%%%%%%%%%%%%%%%%%%%%%%%%%%%%%%%%%%%%%%%%%%%%%%%%%%%%%%%%%%%%%%%%%%%%%%%%%%%%%%%%
%%%%%%%%%%%%%%%%%%%%%%%%%%%%%%%%%%%%%%%%%%%%%%%%%%%%%%%%%%%%%%%%%%%%%%%%%%%%%%%%%%%%%%%%%%%%%%%%%%%%

\subsubsection{Proof of \EQUATIONS \eqref{eqn:lemma:concentration-ineq:noisy:pr:1} and \eqref{eqn:lemma:concentration-ineq:noisy:ev:1}}
\label{outline:concentration-ineq|pf-noisy|1}

%%|>>|~~~~~~~~~~~~~~~~~~~~~~~~~~~~~~~~~~~~~~~~~~~~~~~~~~~~~~~~~~~~~~~~~~~~~~~~~~~~~~~~~~~~~~~~~~~|>>|
%%|>>|~~~~~~~~~~~~~~~~~~~~~~~~~~~~~~~~~~~~~~~~~~~~~~~~~~~~~~~~~~~~~~~~~~~~~~~~~~~~~~~~~~~~~~~~~~~|>>|
%%|>>|~~~~~~~~~~~~~~~~~~~~~~~~~~~~~~~~~~~~~~~~~~~~~~~~~~~~~~~~~~~~~~~~~~~~~~~~~~~~~~~~~~~~~~~~~~~|>>|
%\begin{proof}
%{\EQUATIONS \eqref{eqn:lemma:concentration-ineq:noisy:ev:1} and \eqref{eqn:lemma:concentration-ineq:noisy:pr:1}}
%
Fix
%|>>|:::::::::::::::::::::::::::::::::::::::::::::::::::::::::::::::::::::::::::::::::::::::::::|>>|
\(  \thetaStar \in \ParamSpace  \) and \(  \JCoordsX \in \JSX  \)
%|<<|:::::::::::::::::::::::::::::::::::::::::::::::::::::::::::::::::::::::::::::::::::::::::::|<<|
arbitrarily.
Write
%|>>|:::::::::::::::::::::::::::::::::::::::::::::::::::::::::::::::::::::::::::::::::::::::::::|>>|
\(  \CovVX\VIx{\iIx} \defeq \ThresholdSet{\Supp( \thetaStar ) \cup \JCoordsX}( \CovV\VIx{\iIx} )  \),
%|<<|:::::::::::::::::::::::::::::::::::::::::::::::::::::::::::::::::::::::::::::::::::::::::::|<<|
\(  \iIx \in [\m]  \).
As similarly seen earlier, \(  \frac{1}{\sqrt{2\pi}} \hfFn[\JCoordsX]( \thetaStar, \thetaStar )  \) can be written as follows:
%|>>|===========================================================================================|>>|
\begin{align}
  \frac{1}{\sqrt{2\pi}} \hfFn[\JCoordsX]( \thetaStar, \thetaStar )
  &=
  \ThresholdSet{\Supp( \thetaStar ) \cup \JCoordsX} \left(
    \frac{1}{\m}
    \sum_{\iIx=1}^{\m}
    \CovV\VIx{\iIx}
    \sep \frac{1}{2} \left( \fFn( \langle \CovV, \thetaStar \rangle ) - \Sign( \langle \CovV, \thetaStar \rangle ) \right)
  \right)
  \\
  &\dCmt{by the definition of \(  \hfFn[\JCoords]  \) in \EQUATION \eqref{eqn:notations:hfJ:def}}
  \\
  &=
  \frac{1}{\m}
  \sum_{\iIx=1}^{\m}
  \ThresholdSet{\Supp( \thetaStar ) \cup \JCoordsX}( \CovV\VIx{\iIx} )
  \sep \frac{1}{2} \left( \fFn( \langle \CovV, \thetaStar \rangle ) - \Sign( \langle \CovV, \thetaStar \rangle ) \right)
  \\
  &\dCmt{by the linearity of the subset thresholding operation (\see \SECTIONREF \ref{outline:notations})}
  \\
  &=
  \frac{1}{\m}
  \sum_{\iIx=1}^{\m}
  \CovVX\VIx{\iIx}
  \sep \frac{1}{2} \left( \fFn( \langle \CovVX\VIx{\iIx}, \thetaStar \rangle ) - \Sign( \langle \CovVX\VIx{\iIx}, \thetaStar \rangle ) \right)
  \\
  &\dCmt{by the definition of \(  \CovVX\VIx{\iIx}  \), \(  \iIx \in [\m]  \)}
  \\
  &=
  -\frac{1}{\m}
  \sum_{\iIx=1}^{\m}
  \CovVX\VIx{\iIx}
  \sep \Sign( \langle \CovVX\VIx{\iIx}, \thetaStar \rangle )
  \sep \I( \fFn( \langle \CovVX\VIx{\iIx}, \thetaStar \rangle ) \neq \Sign( \langle \CovVX\VIx{\iIx}, \thetaStar \rangle ) )
\TagEqn{\label{eqn:pf:lemma:concentration-ineq:noisy:16}}
,\end{align}
%|<<|===========================================================================================|<<|
and thus,
%|>>|===========================================================================================|>>|
\begin{align*}
  \left\langle \frac{1}{\sqrt{2\pi}} \hfFn[\JCoordsX]( \thetaStar, \thetaStar ), \thetaStar \right\rangle
  &=
  -\frac{1}{\m}
  \sum_{\iIx=1}^{\m}
  \langle \CovVX\VIx{\iIx}, \thetaStar \rangle
  \sep
  \Sign( \langle \CovVX\VIx{\iIx}, \thetaStar \rangle )
  \sep
  \I( \fFn( \langle \CovVX\VIx{\iIx}, \thetaStar \rangle ) \neq \Sign( \langle \CovVX\VIx{\iIx}, \thetaStar \rangle ) )
  \\
  &=
  -\frac{1}{\m}
  \sum_{\iIx=1}^{\m}
  | \langle \CovVX\VIx{\iIx}, \thetaStar \rangle |
  \sep
  \I( \fFn( \langle \CovVX\VIx{\iIx}, \thetaStar \rangle ) \neq \Sign( \langle \CovVX\VIx{\iIx}, \thetaStar \rangle ) )
\TagEqn{\label{eqn:pf:lemma:concentration-ineq:noisy:1}}
.\end{align*}
%|<<|===========================================================================================|<<|
Note that justifications for some of the steps taken above can be obtained by extending those appearing in the proof of \LEMMA \ref{lemma:concentration-ineq:noiseless}.
%
%%%%%%%%%%%%%%%%%%%%%%%%%%%%%%%%%%%%%%%%%%%%%%%%%%%%%%%%%%%%%%%%%%%%%%%%%%%%%%%%%%%%%%%%%%%%%%%%%%%%
\par %%%%%%%%%%%%%%%%%%%%%%%%%%%%%%%%%%%%%%%%%%%%%%%%%%%%%%%%%%%%%%%%%%%%%%%%%%%%%%%%%%%%%%%%%%%%%%%
%%%%%%%%%%%%%%%%%%%%%%%%%%%%%%%%%%%%%%%%%%%%%%%%%%%%%%%%%%%%%%%%%%%%%%%%%%%%%%%%%%%%%%%%%%%%%%%%%%%%
%
The first step towards deriving \EQUATIONS \eqref{eqn:lemma:concentration-ineq:noisy:pr:1} and \eqref{eqn:lemma:concentration-ineq:noisy:ev:1} is characterizing the distribution of each \(  \iIx\Th  \) summand, \(  \iIx \in [\m]  \), in \EQUATION \eqref{eqn:pf:lemma:concentration-ineq:noisy:1}.
Let
%|>>|:::::::::::::::::::::::::::::::::::::::::::::::::::::::::::::::::::::::::::::::::::::::::::|>>|
\(  \Zi \sim \N(0,1)  \) and \(  \Ri \defeq \I( \fFn( \Zi ) \neq \Sign( \Zi ) )  \),
\(  \iIx \in [\m]  \).
%|<<|:::::::::::::::::::::::::::::::::::::::::::::::::::::::::::::::::::::::::::::::::::::::::::|<<|
Then, each \(  \iIx\Th  \) summand, \(  \iIx \in [\m]  \), follows the same distribution as
%|>>|===========================================================================================|>>|
\begin{align*}
  | \langle \CovVX\VIx{\iIx}, \thetaStar \rangle |
  \sep \I( \fFn( \langle \CovVX\VIx{\iIx}, \thetaStar \rangle ) \neq \Sign( \langle \CovVX\VIx{\iIx}, \thetaStar \rangle ) )
  \sim
  \Wi
  \defeq
  | \Zi | \sep \I( \fFn( \Zi ) \neq \Sign( \Zi ) )
  =
  | \Zi | \Ri
.\end{align*}
%|<<|===========================================================================================|<<|
The density and mass functions of \(  | \Zi |  \) and \(  \Ri  \), respectively, are given by
%|>>|===========================================================================================|>>|
\begin{gather}
\label{eqn:pf:lemma:concentration-ineq:noisy:4}
  \pdf{| \Zi |}( \zX )
  =
  \begin{cases}
  \sqrt{\frac{2}{\pi}} e^{-\frac{1}{2} \zX^{2}} ,& \cIf \zX \geq 0, \\
  0                                             ,& \cIf \zX = 0,
  \end{cases}
  \\
\label{eqn:pf:lemma:concentration-ineq:noisy:5}
  \pdf{\Ri}( \rX )
  =
  \begin{cases}
  1-\alphaX ,& \cIf \rX = 0, \\
  \alphaX   ,& \cIf \rX = 1.
  \end{cases}
\end{gather}
%|<<|===========================================================================================|<<|
Additionally, the mass function of the conditioned random variable \(  \Ri=1 \Mid| \Zi  \) is given by
%|>>|===========================================================================================|>>|
\begin{gather}
\label{eqn:pf:lemma:concentration-ineq:noisy:2}
  \pdf{\Ri \Mid| \Zi}( 1 \Mid| \zX )
  =
  \begin{cases}
  \pFn( \zX )   ,& \cIf \zX < 0,   \\
  1-\pFn( \zX ) ,& \cIf \zX \geq 0,
  \end{cases}
\end{gather}
%|<<|===========================================================================================|<<|
and the mass function of the conditioned random variable \(  \Ri=1 \Mid| |\Zi|   \) is given by
%|>>|===========================================================================================|>>|
\begin{gather}
\label{eqn:pf:lemma:concentration-ineq:noisy:8}
  \pdf{\Ri \Mid| |\Zi|}( 1 \Mid| \zX )
  =
  \begin{cases}
  0                         ,& \cIf \zX < 0,   \\
  \frac{1}{2} (\pExpr{\zX}) ,& \cIf \zX \geq 0,
  \end{cases}
\end{gather}
%|<<|===========================================================================================|<<|
where the latter case---when \(  \zX \geq 0  \)---is obtained as follows: %using the law of total probability and the observation that \(  \Zi  \) completely determines \(  |\Zi|  \), which implies \(  ( \Ri \Mid| |\Zi|, \Zi ) \sim ( \Ri \Mid| \Zi )  \):
%|>>|===========================================================================================|>>|
\begin{align*}
  \pdf{\Ri \Mid| |\Zi|}( 1 \Mid| \zX )
  &=
  \pdf{\Ri \Mid| |\Zi|, \Zi}( 1 \Mid| \zX, \zX )
  \pdf{\Zi \Mid| |\Zi|}( \zX | \zX )
  +
  \pdf{\Ri \Mid| |\Zi|, \Zi}( 1 \Mid| \zX, -\zX )
  \pdf{\Zi \Mid| |\Zi|}( \zX | -\zX )
  \\
  &\dCmt{by the law of total probability and the definition of conditional probabilities}
  \\
  &\dCmtx{(and the observation that \(  \pdf{\Zi \Mid| |\Zi|}( \zX' | \zX ) = 0  \) whenever \(  | \zX' | \neq \zX  \), \(  \zX' \in \R, \zX \geq 0  \))}
  \\
  &=
  \frac{1}{2}
  \pdf{\Ri \Mid| |\Zi|, \Zi}( 1 \Mid| \zX, \zX )
  +
  \frac{1}{2}
  \pdf{\Ri \Mid| |\Zi|, \Zi}( 1 \Mid| \zX, -\zX )
  \\
  &\dCmt{by symmetry}
  \\
  &=
  \frac{1}{2}
  \pdf{\Ri \Mid| \Zi}( 1 \Mid| \zX )
  +
  \frac{1}{2}
  \pdf{\Ri \Mid| \Zi}( 1 \Mid| -\zX )
  \\
  &\dCmt{because \(  \Zi  \) completely determines \(  |\Zi|  \), which implies \(  ( \Ri \Mid| |\Zi|, \Zi ) \sim ( \Ri \Mid| \Zi )  \)}
  \\
  &=
  \frac{1}{2}
  ( 1-\pFn( \zX ) )
  +
  \frac{1}{2}
  \pFn( -\zX )
  \\
  &\dCmt{by \EQUATION \eqref{eqn:pf:lemma:concentration-ineq:noisy:2}}
  \\
  &=
  \frac{1}{2} (\pExpr{\zX})
.\end{align*}
%|<<|===========================================================================================|<<|
Note that
%|>>|:::::::::::::::::::::::::::::::::::::::::::::::::::::::::::::::::::::::::::::::::::::::::::|>>|
\(  ( \Wi \Mid| \Ri=1 ) \sim ( |\Zi|\Ri \Mid| \Ri=1 ) \sim ( |\Zi| \Mid| \Ri=1 )  \).
%|<<|:::::::::::::::::::::::::::::::::::::::::::::::::::::::::::::::::::::::::::::::::::::::::::|<<|
Thus, via Bayes' theorem, for \(  \zX \in \R  \),
%|>>|===========================================================================================|>>|
\begin{align*}
  \pdf{\Wi \Mid| \Ri}( \zX \Mid| 1 )
  &=
  \pdf{|\Zi| \Mid| \Ri}( \zX \Mid| 1 )
  \\
  &\dCmt{by the above remark}
  \\
  &=
  \frac{\pdf{|\Zi|}( \zX ) \pdf{\Ri \Mid| |\Zi|}( 1 \Mid| \zX )}{\pdf{\Ri}(1)}
  \\
  &\dCmt{by Bayes' theorem}
  \\
  &=
  \frac{\sqrt{\frac{2}{\pi}} e^{-\frac{1}{2} \zX^{2}} \frac{1}{2} (\pExpr{\zX})}{\alphaX}
  \\
  &\dCmt{by \EQUATIONS \eqref{eqn:pf:lemma:concentration-ineq:noisy:4}, \eqref{eqn:pf:lemma:concentration-ineq:noisy:5}, and \eqref{eqn:pf:lemma:concentration-ineq:noisy:8}}
  \\
  &=
  \frac{1}{\sqrt{2\pi} \alphaX} e^{-\frac{1}{2} \zX^{2}} (\pExpr{\zX})
,\end{align*}
%|<<|===========================================================================================|<<|
and therefore, taking together the above work, the density of the conditioned random variable \(  \Wi \Mid| \Ri  \) is given for \(  \rX \in \{ 0,1 \}  \) and \(  \zX \in \R  \) by
%|>>|===========================================================================================|>>|
\begin{gather}
\label{eqn:pf:lemma:concentration-ineq:noisy:3}
  \pdf{\Wi \Mid| \Ri}( \zX \Mid| \rX )
  =
  \begin{cases}
  0                                                                    ,& \cIf \rX=0, \zX \neq 0, \\
  1                                                                    ,& \cIf \rX=0, \zX = 0, \\
  0                                                                    ,& \cIf \rX=1, \zX < 0, \\
  \frac{1}{\sqrt{2\pi} \alphaX} e^{-\frac{1}{2} \zX^{2}} (\pExpr{\zX}) ,& \cIf \rX=1, \zX \geq 0.
  \end{cases}
\end{gather}
%|<<|===========================================================================================|<<|
%By the law of total probability,
%%|>>|===========================================================================================|>>|
%\begin{align*}
%  \pdf{\Wi}( \zX )
%  &=
%  \pdf{\Wi \Mid| \Ri}( \zX \Mid| 0 ) \pdf{\Ri}( 0 )
%  +
%  \pdf{\Wi \Mid| \Ri}( \zX \Mid| 1 ) \pdf{\Ri}( 1 )
%  \\
%  &=
%  \begin{cases}
%  1-\alphaX ,& \cIf \zX=0
%  \end{cases}
%  ( 1-\alphaX ) + \frac{1}{\sqrt{2\pi}} e^{-\frac{1}{2} \zX^{2}} (\pExpr{\zX})
%\end{align*}
%%|<<|===========================================================================================|<<|
In expectation, when conditioning on \(  \Ri=0  \),
%|>>|===========================================================================================|>>|
\begin{align*}
  \E[ \Wi \Mid| \Ri=0 ]
  &=
  \int_{\zX=-\infty}^{\zX=\infty} \zX \pdf{\Wi \Mid| \Ri}( \zX \Mid| 0 ) d\zX
  \\
  &=
  0 \pdf{\Wi \Mid| \Ri}( 0 \Mid| 0 )
  \\
  &\dCmt{due to \EQUATION \eqref{eqn:pf:lemma:concentration-ineq:noisy:3}}
  \\
  &=
  0
\TagEqn{\label{eqn:pf:lemma:concentration-ineq:noisy:6}}
,\end{align*}
%|<<|===========================================================================================|<<|
and when conditioning on \(  \Ri=1  \),
%|>>|===========================================================================================|>>|
\begin{align*}
  \E[ \Wi \Mid| \Ri=1 ]
  &=
  \int_{\zX=-\infty}^{\zX=\infty} \zX \pdf{\Wi \Mid| \Ri}( \zX \Mid| 1 ) d\zX
  \\
  &=
  \int_{\zX=0}^{\zX=\infty}
  \frac{1}{\sqrt{2\pi} \alphaX} \zX e^{-\frac{1}{2} \zX^{2}} (\pExpr{\zX})
  d\zX
  \\
  &\dCmt{by \EQUATION \eqref{eqn:pf:lemma:concentration-ineq:noisy:3}}
  \\
  &=
  \int_{\zX=0}^{\zX=\infty}
  \frac{1}{\sqrt{2\pi} \alphaX} \zX e^{-\frac{1}{2} \zX^{2}}
  d\zX
  -
  \int_{\zX=0}^{\zX=\infty}
  \frac{1}{\sqrt{2\pi} \alphaX} \zX e^{-\frac{1}{2} \zX^{2}} (\pFn( \zX ) - \pFn( -\zX ))
  d\zX
  \\
  &=
  \frac{1}{\sqrt{2\pi} \alphaX}
  -
  \frac{\gammaX}{2\alphaX}
  \\
  &\dCmt{by the definition of \(  \gammaX  \)}
  \\ \TagEqn{\label{eqn:pf:lemma:concentration-ineq:noisy:7}}
  &=
  %\frac{\sqrt{\hfrac{2}{\pi}}-\gammaX}{2\alphaX}
  \EWRValue
.\end{align*}
%|<<|===========================================================================================|<<|

With this preliminary work completed, we now proceed to the derivations of \EQUATIONS \eqref{eqn:lemma:concentration-ineq:noisy:pr:1} and \eqref{eqn:lemma:concentration-ineq:noisy:ev:1}, starting with the former.
%
%%%%%%%%%%%%%%%%%%%%%%%%%%%%%%%%%%%%%%%%%%%%%%%%%%%%%%%%%%%%%%%%%%%%%%%%%%%%%%%%%%%%%%%%%%%%%%%%%%%%
\paragraph{Verification of \EQUATION \eqref{eqn:lemma:concentration-ineq:noisy:pr:1}} %%%%%%%%%%%%%%
%%%%%%%%%%%%%%%%%%%%%%%%%%%%%%%%%%%%%%%%%%%%%%%%%%%%%%%%%%%%%%%%%%%%%%%%%%%%%%%%%%%%%%%%%%%%%%%%%%%%
%
Having obtained the density function and expectations for the conditioned random variable \(  \Wi \Mid| \Ri  \), the expectation of the random variable \(  | \langle \CovVX\VIx{\iIx}, \thetaStar \rangle | \sep \I( \fFn( \langle \CovVX\VIx{\iIx}, \thetaStar \rangle ) \neq \Sign( \langle \CovVX\VIx{\iIx}, \thetaStar \rangle ) )  \) is now calculated as follows: %via the law of total expectation:
%|>>|===========================================================================================|>>|
\begin{align*}
  \E[ | \langle \CovVX\VIx{\iIx}, \thetaStar \rangle | \sep \I( \fFn( \langle \CovVX\VIx{\iIx}, \thetaStar \rangle ) \neq \Sign( \langle \CovVX\VIx{\iIx}, \thetaStar \rangle ) ) ]
  &=
  \E[ \Wi ]
  \\
  &=
  \pdf{\Ri}( 0 ) \E[ \Wi \Mid| \Ri=0 ]
  +
  \pdf{\Ri}( 1 ) \E[ \Wi \Mid| \Ri=1 ]
  \\
  &\dCmt{by the law of total expectation}
  \\
  &=
  ( 1-\alphaX )
  0
  +
  \alphaX
  %\frac{\gammaX}{\sqrt{2\pi} \alphaX}
  \EWRValue
  \\
  &\dCmt{by \EQUATIONS \eqref{eqn:pf:lemma:concentration-ineq:noisy:5}, \eqref{eqn:pf:lemma:concentration-ineq:noisy:6}, and \eqref{eqn:pf:lemma:concentration-ineq:noisy:7}}
  \\
  &=
  %\frac{\gammaX}{\sqrt{2\pi}}
  \frac{\sqrt{\hfrac{2}{\pi}}-\gammaX}{2}
.\end{align*}
%|<<|===========================================================================================|<<|
By the linearity of expectation, it follows that
%|>>|===========================================================================================|>>|
\begin{align*}
  \E \left[ \left\langle \hfFn[\JCoordsX]( \thetaStar, \thetaStar ), \thetaStar \right\rangle \right]
  &=
  \E \left[
    -\frac{\sqrt{2\pi}}{\m}
    \sum_{\iIx=1}^{\m}
    | \langle \CovVX\VIx{\iIx}, \thetaStar \rangle |
    \sep \I( \fFn( \langle \CovVX\VIx{\iIx}, \thetaStar \rangle ) \neq \Sign( \langle \CovVX\VIx{\iIx}, \thetaStar \rangle ) )
  \right]
  \\
  &=
  -\frac{\sqrt{2\pi}}{\m}
  \sum_{\iIx=1}^{\m}
  \E \left[
    | \langle \CovVX\VIx{\iIx}, \thetaStar \rangle |
    \sep \I( \fFn( \langle \CovVX\VIx{\iIx}, \thetaStar \rangle ) \neq \Sign( \langle \CovVX\VIx{\iIx}, \thetaStar \rangle ) )
  \right]
  \\
  &=
  -\frac{\sqrt{2\pi}}{\m}
  \sum_{\iIx=1}^{\m}
  %\frac{\gammaX}{\sqrt{2\pi}}
  \frac{\sqrt{\hfrac{2}{\pi}}-\gammaX}{2}
  \\
  &=
  -\left( 1 - \sqrt{\frac{\pi}{2}} \gammaX \right)
,\end{align*}
%|<<|===========================================================================================|<<|
as claimed.
This completes the derivation of \EQUATION \eqref{eqn:lemma:concentration-ineq:noisy:pr:1}.
%
%%%%%%%%%%%%%%%%%%%%%%%%%%%%%%%%%%%%%%%%%%%%%%%%%%%%%%%%%%%%%%%%%%%%%%%%%%%%%%%%%%%%%%%%%%%%%%%%%%%%
\paragraph{Verification of \EQUATION \eqref{eqn:lemma:concentration-ineq:noisy:ev:1}} %%%%%%%%%%%%%%
%%%%%%%%%%%%%%%%%%%%%%%%%%%%%%%%%%%%%%%%%%%%%%%%%%%%%%%%%%%%%%%%%%%%%%%%%%%%%%%%%%%%%%%%%%%%%%%%%%%%
%
Next, \EQUATION \eqref{eqn:lemma:concentration-ineq:noisy:ev:1} is derived.
%
This derivation is based on the \MGFs of the centered conditioned random variables \(  ( \Wi \Mid| \Ri ) - \E[ \Wi \Mid| \Ri ]  \) and \(  ( -\Wi \Mid| \Ri ) - \E[ -\Wi \Mid| \Ri ]  \), which are denoted by \(  \mgf{( \Wi \Mid| \Ri ) - \E[ \Wi \Mid| \Ri ]}  \) and \(  \mgf{( -\Wi \Mid| \Ri ) - \E[ -\Wi \Mid| \Ri ]}  \), respectively.
Write
%|>>|:::::::::::::::::::::::::::::::::::::::::::::::::::::::::::::::::::::::::::::::::::::::::::|>>|
\(  \muX[0] \defeq \E[ \Wi \Mid| \Ri=0 ] = 0  \) and
\(  \muX[1] \defeq \E[ \Wi \Mid| \Ri=1 ] = \frac{\sqrt{\hfrac{2}{\pi}}-\gammaX}{2\alphaX} \),
%|<<|:::::::::::::::::::::::::::::::::::::::::::::::::::::::::::::::::::::::::::::::::::::::::::|<<|
where these expectations were calculated previously in \EQUATIONS \eqref{eqn:pf:lemma:concentration-ineq:noisy:6} and \eqref{eqn:pf:lemma:concentration-ineq:noisy:7}.
Conditioned on \(  \Ri=1  \), the \MGFs are given at \(  \sX \in [0,\infty)  \) by
%|>>|===========================================================================================|>>|
\begin{align*}
  \mgf{(\Wi \Mid| \Ri=1)-\E[\Wi \Mid| \Ri=1]}(\sX)
  &=
  \E \left[
    e^{\sX ( \Wi-\E[\Wi] )}
  \middle|
    \Ri=1
  \right]
  \\
  &=
  \int_{\zX=0}^{\zX=\infty}
  \frac{1}{\sqrt{2\pi} \alphaX} e^{\sX( \zX-\muX[1] )} e^{-\frac{1}{2} \zX^{2}} (\pExpr{\zX})
  d\zX
  \\
  &=
  \frac{1}{\alphaX}
  e^{\frac{1}{2} \sX^{2}}
  \frac{1}{\sqrt{2\pi}}
  e^{-\sX \muX[1]}
  \int_{\zX=0}^{\zX=\infty}
  e^{-\frac{1}{2} (\zX-\sX)^{2}} (\pExpr{\zX})
  d\zX
  \\
  &=
  \frac{1}{\alphaX}
  e^{\frac{1}{2} \sX^{2}}
  \fFnX( \sX )
,\end{align*}
%|<<|===========================================================================================|<<|
and
%|>>|===========================================================================================|>>|
\begin{align*}
  \mgf{(-\Wi \Mid| \Ri=1)-\E[-\Wi \Mid| \Ri=1]}(\sX)
  &=
  \E \left[
    e^{\sX ( -\Wi-\E[-\Wi] )}
  \middle|
    \Ri=1
  \right]
  \\
  &=
  \int_{\zX=0}^{\zX=\infty}
  \frac{1}{\sqrt{2\pi} \alphaX} e^{\sX( \zX+\muX[1] )} e^{-\frac{1}{2} \zX^{2}} (\pExpr{\zX})
  d\zX
  \\
  &=
  \frac{1}{\alphaX}
  e^{\frac{1}{2} \sX^{2}}
  \frac{1}{\sqrt{2\pi}}
  e^{\sX \muX[1]}
  \int_{\zX=0}^{\zX=\infty}
  e^{-\frac{1}{2} (\zX+\sX)^{2}} (\pExpr{\zX})
  d\zX
  \\
  &=
  \frac{1}{\alphaX}
  e^{\frac{1}{2} \sX^{2}}
  \fFnXX( \sX )
,\end{align*}
%|<<|===========================================================================================|<<|
where
%|>>|===========================================================================================|>>|
\begin{gather}
\label{eqn:pf:lemma:concentration-ineq:noisy:f1}
  \fFnX( \sX )
  \defeq
  \frac{1}{\sqrt{2\pi}}
  e^{-\sX \muX[1]}
  \int_{\zX=0}^{\zX=\infty}
  e^{-\frac{1}{2} (\zX-\sX)^{2}} (\pExpr{\zX})
  d\zX
  ,\\
\label{eqn:pf:lemma:concentration-ineq:noisy:f2}
  \fFnXX( \sX )
  \defeq
  \frac{1}{\sqrt{2\pi}}
  e^{\sX \muX[1]}
  \int_{\zX=0}^{\zX=\infty}
  e^{-\frac{1}{2} (\zX+\sX)^{2}} (\pExpr{\zX})
  d\zX
.\end{gather}
%|<<|===========================================================================================|<<|
%
Before proceeding, the following lemma is introduced to facilitate upper bounds on the \MGFs, \(  \mgf{(\Wi \Mid| \Ri=1)-\E[\Wi \Mid| \Ri=1]}  \) and \(  \mgf{(-\Wi \Mid| \Ri=1)-\E[-\Wi \Mid| \Ri=1]}  \).
Its proof is deferred to \SECTION \ref{outline:concentration-ineq|pf-noisy|pf-f1,f2}.
%
%|>>|*******************************************************************************************|>>|
%|>>|*******************************************************************************************|>>|
%|>>|*******************************************************************************************|>>|
\begin{lemma}
\label{lemma:pf:lemma:concentration-ineq:noisy:f1,f2}
%
Let \(  \fFnX, \fFnXX : \R \to \R  \) be the functions defined in \EQUATIONS \eqref{eqn:pf:lemma:concentration-ineq:noisy:f1} and \eqref{eqn:pf:lemma:concentration-ineq:noisy:f2}.
Then,
%|>>|===========================================================================================|>>|
\begin{gather}
\label{eqn:pf:lemma:concentration-ineq:noisy:f1:ub}
  \sup_{\sX \geq 0} \fFnX( \sX ) = \fFnX( 0 )
  ,\\
\label{eqn:pf:lemma:concentration-ineq:noisy:f2:ub}
  \sup_{\sX \geq 0} \fFnXX( \sX ) = \fFnXX( 0 )
,\end{gather}
%|<<|===========================================================================================|<<|
where
%|>>|===========================================================================================|>>|
\begin{gather}
\label{eqn:pf:lemma:concentration-ineq:noisy:f1(0)}
  \fFnX( 0 ) = \alphaX
  ,\\
\label{eqn:pf:lemma:concentration-ineq:noisy:f2(0)}
  \fFnXX( 0 ) = \alphaX
.\end{gather}
%|<<|===========================================================================================|<<|
\end{lemma}
%|<<|*******************************************************************************************|<<|
%|<<|*******************************************************************************************|<<|
%|<<|*******************************************************************************************|<<|
%
Due to \EQUATIONS \eqref{eqn:pf:lemma:concentration-ineq:noisy:f1:ub}--\eqref{eqn:pf:lemma:concentration-ineq:noisy:f2(0)} in \LEMMA \ref{lemma:pf:lemma:concentration-ineq:noisy:f1,f2}, the \MGFs of \(  {( \Wi \Mid| \Ri=1 )} - {\E[ \Wi \Mid| \Ri=1 ]}  \) and \(  {( -\Wi \Mid| \Ri=1 )} - {\E[ -\Wi \Mid| \Ri=1 ]}  \) are now upper bounded by
%|>>|===========================================================================================|>>|
\begin{gather}
\label{eqn:pf:lemma:concentration-ineq:noisy:11}
  \mgf{(\Wi \Mid| \Ri=1)-\E[\Wi \Mid| \Ri=1]}(\sX)
  =
  \frac{1}{\alphaX}
  e^{\frac{1}{2} \sX^{2}}
  \fFnX( \sX )
  \leq
  e^{\frac{1}{2} \sX^{2}}
  ,\\
\label{eqn:pf:lemma:concentration-ineq:noisy:12}
  \mgf{(-\Wi \Mid| \Ri=1)-\E[-\Wi \Mid| \Ri=1]}(\sX)
  =
  \frac{1}{\alphaX}
  e^{\frac{1}{2} \sX^{2}}
  \fFnXX( \sX )
  \leq
  e^{\frac{1}{2} \sX^{2}}
.\end{gather}
%|<<|===========================================================================================|<<|
%
%%%%%%%%%%%%%%%%%%%%%%%%%%%%%%%%%%%%%%%%%%%%%%%%%%%%%%%%%%%%%%%%%%%%%%%%%%%%%%%%%%%%%%%%%%%%%%%%%%%%
\par %%%%%%%%%%%%%%%%%%%%%%%%%%%%%%%%%%%%%%%%%%%%%%%%%%%%%%%%%%%%%%%%%%%%%%%%%%%%%%%%%%%%%%%%%%%%%%%
%%%%%%%%%%%%%%%%%%%%%%%%%%%%%%%%%%%%%%%%%%%%%%%%%%%%%%%%%%%%%%%%%%%%%%%%%%%%%%%%%%%%%%%%%%%%%%%%%%%%
%
On the other hand, when conditioned on \(  \Ri=0  \), the \MGFs of these random variables \(  {( \Wi \Mid| \Ri=0 )} - {\E[ \Wi \Mid| \Ri=0 ]}  \) and \(  ( -\Wi \Mid| \Ri=0 ) - \E[ -\Wi \Mid| \Ri=0 ]  \), written as \(  \mgf{(\Wi \Mid| \Ri=0)-\E[\Wi \Mid| \Ri=0]}  \) and \(  \mgf{(-\Wi \Mid| \Ri=0)-\E[-\Wi \Mid| \Ri=0]}  \), are obtained as follows for \(  \sX \in [0,\infty)  \):
%|>>|===========================================================================================|>>|
\begin{align*}
  \mgf{(\Wi \Mid| \Ri=0)-\E[\Wi \Mid| \Ri=0]}(\sX)
  &=
  \E \left[
    e^{\sX( \Wi-\muX[0] )}
  \middle|
    \Ri=0
  \right]
  \\
  &=
  \E \left[
    e^{\sX\Wi}
  \middle|
    \Ri=0
  \right]
  \\
  &=
  e^{\sX \cdot 0} \pdf{\Wi \Mid| \Ri}( 0 \Mid| 0 )
  \\
  &=
  1
\TagEqn{\label{eqn:pf:lemma:concentration-ineq:noisy:13}}
,\end{align*}
%|<<|===========================================================================================|<<|
and likewise,
%|>>|===========================================================================================|>>|
\begin{align*}
  \mgf{(-\Wi \Mid| \Ri=0)-\E[-\Wi \Mid| \Ri=0]}(\sX)
\XXX{
  &=
  \E \left[
    e^{\sX( -\Wi+\muX[0] )}
  \middle|
    \Ri=0
  \right]
  \\
  &=
  \E \left[
    e^{-\sX\Wi}
  \middle|
    \Ri=0
  \right]
  \\
  &=
  e^{-\sX \cdot 0} \pdf{\Wi \Mid| \Ri}( 0 \Mid| 0 )
  \\
}
  &=
  1
\TagEqn{\label{eqn:pf:lemma:concentration-ineq:noisy:14}}
.\end{align*}
%|<<|===========================================================================================|<<|
%
%%%%%%%%%%%%%%%%%%%%%%%%%%%%%%%%%%%%%%%%%%%%%%%%%%%%%%%%%%%%%%%%%%%%%%%%%%%%%%%%%%%%%%%%%%%%%%%%%%%%
\par %%%%%%%%%%%%%%%%%%%%%%%%%%%%%%%%%%%%%%%%%%%%%%%%%%%%%%%%%%%%%%%%%%%%%%%%%%%%%%%%%%%%%%%%%%%%%%%
%%%%%%%%%%%%%%%%%%%%%%%%%%%%%%%%%%%%%%%%%%%%%%%%%%%%%%%%%%%%%%%%%%%%%%%%%%%%%%%%%%%%%%%%%%%%%%%%%%%%
%
Now, consider the sum of the random variables \(  \Wi  \), \(  \iIx \in [\m]  \).
Write
%|>>|:::::::::::::::::::::::::::::::::::::::::::::::::::::::::::::::::::::::::::::::::::::::::::|>>|
\(  \WRV \defeq \sum_{\iIx=1}^{\m} \Wi  \),
%|<<|:::::::::::::::::::::::::::::::::::::::::::::::::::::::::::::::::::::::::::::::::::::::::::|<<|
and let
%|>>|:::::::::::::::::::::::::::::::::::::::::::::::::::::::::::::::::::::::::::::::::::::::::::|>>|
\(  \Vec{\RRV} \defeq ( \Ri[1], \dots, \Ri[\m] )  \).
%|<<|:::::::::::::::::::::::::::::::::::::::::::::::::::::::::::::::::::::::::::::::::::::::::::|<<|
Fixing
%|>>|:::::::::::::::::::::::::::::::::::::::::::::::::::::::::::::::::::::::::::::::::::::::::::|>>|
\(  \Vec{\rX} \in \{ 0,1 \}^{\m}  \),
%|<<|:::::::::::::::::::::::::::::::::::::::::::::::::::::::::::::::::::::::::::::::::::::::::::|<<|
the \MGF of \(  ( \WRV \Mid| \Vec{\RRV}=\Vec{\rX} ) - \E[ \WRV \Mid| \Vec{\RRV}=\Vec{\rX} ]  \), written \(  \mgf{( \WRV \Mid| \Vec{\RRV}=\Vec{\rX} ) - \E[ \WRV \Mid| \Vec{\RRV}=\Vec{\rX} ]}( \sX )  \), is given and upper bounded at \(  \sX \in [0,\infty)  \) by
%|>>|===========================================================================================|>>|
\begin{align*}
  \mgf{( \WRV \Mid| \Vec{\RRV}=\Vec{\rX} ) - \E[ \WRV \Mid| \Vec{\RRV}=\Vec{\rX} ]}( \sX )
\XXX{
  &=
  \E \left[
    e^{\sX( \WRV - \E[ \WRV ] )}
  \middle|
    \Vec{\RRV}=\Vec{\rX}
  \right]
  \\
  &\dCmt{by the definition of \MGFs}
  \\
}
  &=
  \E \left[
    e^{\sX \sum_{\iIx=1}^{\m} \Wi - \E[ \Wi ]}
  \middle|
    \Vec{\RRV}=\Vec{\rX}
  \right]
  \\
  &\dCmt{by the definition of \MGFs and by the definition of \(  \WRV  \)} %and the linearity of expectation}
  \\
  % &=
  % \E \left[
  %   \prod_{\iIx=1}^{\m}
  %   e^{\sX( \Wi - \E[ \Wi ] )}
  % \middle|
  %   \Vec{\RRV}=\Vec{\rX}
  % \right]
  % \\
  % &\dCmt{by standard facts about exponents}
  % \\
\XXX{  &=
  \E \left[
    \prod_{\iIx=1}^{\m}
    \left( e^{\sX( \Wi - \E[ \Wi ] )} \middle| \Ri=\rX[\iIx] \right)
  \right]
  \\
  &\dCmt{each random variable \(  \Wi  \), \(  \iIx \in [\m]  \), is independent of}
  \\
  &\dCmtIndent\text{the random variables \(  \{ \Ri[\iIx'] \}_{\iIx' \neq \iIx}  \)}
  \\
}
  &=
  \prod_{\iIx=1}^{\m}
  \E \left[
    e^{\sX( \Wi - \E[ \Wi ] )}
  \middle|
    \Ri=\rX[\iIx]
  \right]
  \\
  &\dCmt{each \(  \Wi  \), \(  \iIx \in [\m]  \), is independent of \(  \{ \Ri[\iIx'] \}_{\iIx' \neq \iIx}  \); and}
  \\
  &\dCmt{\(  \{ ( \Wi[1] \Mid| \Ri[1] ), \dots, ( \Wi[\m] \Mid| \Ri[\m] ) \}  \) are mutually independent}
\XXX{
  \\
  &=
  \prod_{\substack{\iIx=1 :\\ \rX[\iIx]=1}}^{\m}
  \E \left[
    e^{\sX( \Wi - \E[ \Wi ] )}
  \middle|
    \Ri=1
  \right]
  \prod_{\substack{\iIx=1 :\\ \rX[\iIx]=0}}^{\m}
  \E \left[
    e^{\sX( \Wi - \E[ \Wi ] )}
  \middle|
    \Ri=0
  \right]
  \\
  &\dCmt{by partitioning the values of the index of multiplication}
}
  \\
  &=
  \prod_{\substack{\iIx=1 :\\ \rX[\iIx]=1}}^{\m}
  \mgf{( \Wi \Mid| \Ri=1 ) - \E[ \Wi \Mid| \Ri=1 ]}( \sX )
  \prod_{\substack{\iIx=1 :\\ \rX[\iIx]=0}}^{\m}
  \mgf{( \Wi \Mid| \Ri=0 ) - \E[ \Wi \Mid| \Ri=0 ]}( \sX )
  \\
  &\dCmt{by partitioning the values of the index of multiplication}
  \\
  &\dCmtx{and by the definition of \(  \mgf{( \Wi \Mid| \Ri ) - \E[ \Wi \Mid| \Ri ]}  \), \(  \iIx \in [\m]  \)}
  \\
  &\leq
  \prod_{\substack{\iIx=1 :\\ \rX[\iIx]=1}}^{\m}
  e^{\frac{1}{2} \sX^{2}}
  \prod_{\substack{\iIx=1 :\\ \rX[\iIx]=0}}^{\m}
  1
  \\
  &\dCmt{by \EQUATIONS \eqref{eqn:pf:lemma:concentration-ineq:noisy:11} and \eqref{eqn:pf:lemma:concentration-ineq:noisy:13}}
  \\
  &=
  e^{\frac{1}{2} \| \Vec{\rX} \|_{0} \sX^{2}}
\TagEqn{\label{eqn:pf:lemma:concentration-ineq:noisy:15}}
.\end{align*}
%|<<|===========================================================================================|<<|
\begin{comment}
Likewise,
%|>>|===========================================================================================|>>|
\begin{align*}
  \mgf{( -\WRV \Mid| \Vec{\RRV}=\Vec{\rX} ) - \E[ -\WRV \Mid| \Vec{\RRV}=\Vec{\rX} ]}( \sX )
  &=
  \E \left[
    e^{\sX( -\WRV - \E[ -\WRV ] )}
  \middle|
    \Vec{\RRV}=\Vec{\rX}
  \right]
  \\
  &\dCmt{by the definition of \MGFs}
  \\
  &=
  \E \left[
    e^{\sX \sum_{\iIx=1}^{\m} -\Wi - \E[ -\Wi ]}
  \middle|
    \Vec{\RRV}=\Vec{\rX}
  \right]
  \\
  &\dCmt{by the definition of \(  \WRV  \) and the linearity of expectation}
  \\
  &=
  \E \left[
    \prod_{\iIx=1}^{\m}
    e^{\sX( -\Wi - \E[ -\Wi ] )}
  \middle|
    \Vec{\RRV}=\Vec{\rX}
  \right]
  \\
  &\dCmt{by standard facts about exponents}
  \\
  &=
  \E \left[
    \prod_{\iIx=1}^{\m}
    \left( e^{\sX( -\Wi - \E[ -\Wi ] )} \middle| \Ri=\rX[\iIx] \right)
  \right]
  \\
  &\dCmt{each random variable \(  -\Wi  \), \(  \iIx \in [\m]  \), is independent of}
  \\
  &\dCmtIndent\text{the random variables \(  \{ \Ri[\iIx'] \}_{\iIx' \neq \iIx}  \)}
  \\
  &=
  \prod_{\iIx=1}^{\m}
  \E \left[
    e^{\sX( -\Wi - \E[ -\Wi ] )}
  \middle|
    \Ri=\rX[\iIx]
  \right]
  \\
  &\dCmt{the random variables \(  ( -\Wi[1] \Mid| \Ri[1] ), \dots, ( -\Wi[\m] \Mid| \Ri[\m] )  \)}
  \\
  &\dCmtIndent\text{are mutually independent}
  \\
  &=
  \prod_{\substack{\iIx=1 :\\ \rX[\iIx]=1}}^{\m}
  \E \left[
    e^{\sX( -\Wi - \E[ -\Wi ] )}
  \middle|
    \Ri=1
  \right]
  \prod_{\substack{\iIx=1 :\\ \rX[\iIx]=0}}^{\m}
  \E \left[
    e^{\sX( -\Wi - \E[ -\Wi ] )}
  \middle|
    \Ri=0
  \right]
  \\
  &\dCmt{by partitioning the values of the index of summation}
  \\
  &=
  \prod_{\substack{\iIx=1 :\\ \rX[\iIx]=1}}^{\m}
  \mgf{( -\Wi \Mid| \Ri=1 ) - \E[ -\Wi \Mid| \Ri=1 ]}( \sX )
  \prod_{\substack{\iIx=1 :\\ \rX[\iIx]=0}}^{\m}
  \mgf{( -\Wi \Mid| \Ri=0 ) - \E[ -\Wi \Mid| \Ri=0 ]}( \sX )
  \\
  &\dCmt{by the definition of \(  \mgf{( \Wi \Mid| \Ri ) - \E[ \Wi \Mid| \Ri ]}  \), \(  \iIx \in [\m]  \)}
  \\
  &\leq
  \prod_{\substack{\iIx=1 :\\ \rX[\iIx]=1}}^{\m}
  e^{\frac{1}{2} \sX^{2}}
  \prod_{\substack{\iIx=1 :\\ \rX[\iIx]=0}}^{\m}
  1
  \\
  &\dCmt{by \EQUATIONS \eqref{eqn:pf:lemma:concentration-ineq:noisy:12} and \eqref{eqn:pf:lemma:concentration-ineq:noisy:14}}
  \\
  &=
  e^{\frac{1}{2} \| \Vec{\rX} \|_{0} \sX^{2}}
.\end{align*}
%|<<|===========================================================================================|<<|
\end{comment}
It follows that
%|>>|===========================================================================================|>>|
\begin{align*}
  \Pr \left(
    \frac{\WRV}{\m} - \E \left[ \frac{\WRV}{\m} \right] > \alphaX \tX
  \middle|
    \Vec{\RRV}=\Vec{\rX}
  \right)
  &\leq
  \inf_{\sX \geq 0}
  e^{-\alphaX \m \sX \tX}
  \mgf{( \WRV \Mid| \Vec{\RRV}=\Vec{\rX} ) - \E[ \WRV \Mid| \Vec{\RRV}=\Vec{\rX} ]}( \sX )
  \\
  &\dCmt{due to Bernstein (\seeeg \cite{vershynin2018high})}
  \\
  \TagEqn{\label{eqn:pf:lemma:concentration-ineq:noisy:9}}
  &\leq
  \inf_{\sX \geq 0}
  e^{-\alphaX \m \sX \tX}
  e^{\frac{1}{2} \| \Vec{\rX} \|_{0} \sX^{2}}
  .\\
  &\dCmt{by \EQUATION \eqref{eqn:pf:lemma:concentration-ineq:noisy:15}}
\end{align*}
%|<<|===========================================================================================|<<|
Additionally, note that
%|>>|===========================================================================================|>>|
\begin{align}
\label{eqn:pf:lemma:concentration-ineq:noisy:10}
  \pdf{\Vec{\RRV}}( \Vec{\rX} ) = \alphaX^{\| \Vec{\rX} \|_{0}} (1-\alphaX)^{\m-\| \Vec{\rX} \|_{0}}
.\end{align}
%|<<|===========================================================================================|<<|
Then,
%|>>|===========================================================================================|>>|
\begin{align*}
  \Pr \left(
    \frac{\WRV}{\m} - \E \left[ \frac{\WRV}{\m} \right] > \alphaX \tX
  \right)
  &=
  \sum_{\Vec{\rX} \in \{ 0,1 \}^{\m}}
  \pdf{\Vec{\RRV}}( \Vec{\rX} )
  \Pr \left(
    \frac{\WRV}{\m} - \E \left[ \frac{\WRV}{\m} \right] > \alphaX \tX
  \middle|
    \Vec{\RRV}=\Vec{\rX}
  \right)
%  \Pr \left(
%    \WRV - \E[ \WRV ] > \alphaX \tX
%  \middle|
%    \Vec{\RRV}=\Vec{\rX}
%  \right)
  \\
  &\dCmt{by the law of total expectation}
  \\
  &\leq
  \sum_{\Vec{\rX} \in \{ 0,1 \}^{\m}}
  \alphaX^{\| \Vec{\rX} \|_{0}} ( 1-\alphaX )^{\m-\| \Vec{\rX} \|_{0}}
  \inf_{\sX \geq 0}
  e^{-\alphaX \m \sX \tX}
  e^{\frac{1}{2} \| \Vec{\rX} \|_{0} \sX^{2}}
  \\
  &\dCmt{by \EQUATIONS \eqref{eqn:pf:lemma:concentration-ineq:noisy:9} and \eqref{eqn:pf:lemma:concentration-ineq:noisy:10}}
  \\
  &\leq
  \inf_{\sX \geq 0}
  e^{-\alphaX \m \sX \tX}
  \sum_{\Vec{\rX} \in \{ 0,1 \}^{\m}}
  ( \alphaX e^{\frac{1}{2} \sX^{2}} )^{\| \Vec{\rX} \|_{0}}
  ( 1-\alphaX )^{\m-\| \Vec{\rX} \|_{0}}
\XXX{  \\
  &=
  \inf_{\sX \geq 0}
  e^{-\alphaX \m \sX \tX}
  \sum_{\Ell=0}^{\m}
  \sum_{\substack{\Vec{\rX} \in \{ 0,1 \}^{\m} :\\ \| \Vec{\rX} \|_{0} = \Ell}}
  ( \alphaX e^{\frac{1}{2} \sX^{2}} )^{\Ell}
  ( 1-\alphaX )^{\m-\Ell}
  \\
  &\dCmt{by partitioning the values of the index of summation}
}
  \\
  &=
  \inf_{\sX \geq 0}
  e^{-\alphaX \m \sX \tX}
  \sum_{\Ell=0}^{\m}
  \binom{\m}{\Ell}
  ( \alphaX e^{\frac{1}{2} \sX^{2}} )^{\Ell}
  ( 1-\alphaX )^{\m-\Ell}
  \\
  &\dCmt{by partitioning the values of the index of}
  \\
  &\dCmtx{summation according to \(  \Ell = \| \Vec{\rX} \|_{0}  \)}
\XXX{
  \\
  &=
  \inf_{\sX \geq 0}
  e^{-\alphaX \m \sX \tX}
  \left( \alphaX e^{\frac{1}{2} \sX^{2}} + 1 - \alphaX \right)^{\m}
  \\
  &\dCmt{by the binomial theorem}
}
  \\
  &=
  \inf_{\sX \geq 0}
  e^{-\alphaX \m \sX \tX}
  \left( 1 + \alphaX ( e^{\frac{1}{2} \sX^{2}} - 1 ) \right)^{\m}
  \\
  &\dCmt{by the binomial theorem}
\XXX{
  \\
  &\leq
  \inf_{\sX \geq 0}
  e^{-\alphaX \m \sX \tX}
  e^{\alphaX \m ( e^{\frac{1}{2} \sX^{2}} - 1 )}
  \\
  &\dCmt{\(  {\textstyle 1 + \alphaX ( e^{\frac{1}{2} \sX^{2}} - 1 ) = e^{\log( 1 + \alphaX ( e^{\frac{1}{2} \sX^{2}} - 1 ) )} \leq e^{\alphaX ( e^{\frac{1}{2} \sX^{2}} - 1 )}}  \),}
  \\
  &\dCmtIndent\text{where the inequality is due to a well-known result}
}
  \\
  &\leq
  \inf_{\sX \geq 0}
  e^{-\alphaX \m ( \sX \tX - e^{\frac{1}{2} \sX^{2}} + 1 )}
  .\\
  &\dCmt{by a well-known inequality, \(  \log( 1+u ) \leq u  \) for \(  u > -1  \)} %, \(  {\textstyle 1 + \alphaX ( e^{\frac{1}{2} \sX^{2}} - 1 ) \leq e^{\alphaX ( e^{\frac{1}{2} \sX^{2}} - 1 )}}  \)}
\end{align*}
%|<<|===========================================================================================|<<|
As discussed earlier in the proof of \LEMMA \ref{lemma:concentration-ineq:noiseless}, \(  \sX \tX - e^{\frac{1}{2} \sX^{2}} + 1  \) is maximized with respect to \(  \sX  \) for \(  \sX, \tX \in (0,1)  \) when roughly \(  \sX \approx \tX  \) because
%|>>|===========================================================================================|>>|
\begin{gather*}
  \left. \frac{\partial}{\partial \sX} \sX \tX - e^{\frac{1}{2} \sX^{2}} + 1 \right|_{\sX=\tX}
  =
  \left. \tX - \sX e^{\frac{1}{2} \sX^{2}} \right|_{\sX=\tX}
  \approx
  0
\end{gather*}
%|<<|===========================================================================================|<<|
for small \(  \sX, \tX > 0  \), and because for all \(  \sX \in \R  \),
%|>>|===========================================================================================|>>|
\begin{align*}
  \frac{\partial^{2}}{\partial \sX^{2}} \sX \tX - e^{\frac{1}{2} \sX^{2}} + 1
  =
  -( 1+\sX^{2} ) e^{\frac{1}{2} \sX^{2}}
  <
  0
.\end{align*}
%|<<|===========================================================================================|<<|
Hence, we will take \(  \sX=\tX  \).
In addition, recall that
%|>>|===========================================================================================|>>|
\begin{gather*}
  \left. \sX \tX - e^{\frac{1}{2} \sX^{2}} + 1 \right|_{\sX=\tX}
  \geq
  \frac{\tX^{2}}{3}
.\end{gather*}
%|<<|===========================================================================================|<<|
It follows that
%|>>|===========================================================================================|>>|
\begin{gather*}
  \Pr \left(
    \WRV - \E[ \WRV ] > \alphaX \tX
  \right)
  \leq
  e^{-\frac{1}{3} \alphaX \m \tX^{2}}
,\end{gather*}
%|<<|===========================================================================================|<<|
and therefore, due to the design of \(  \WRV  \),
%|>>|===========================================================================================|>>|
\begin{gather*}
  \Pr \left(
    \left\langle \frac{\hfFn[\JCoordsX]( \thetaStar, \thetaStar )}{\sqrt{2\pi}} , \thetaStar \right\rangle
    -
    \E \left[ \left\langle \frac{\hfFn[\JCoordsX]( \thetaStar, \thetaStar )}{\sqrt{2\pi}} , \thetaStar \right\rangle \right]
    >
    \alphaX \tX
  \right)
  \leq
  e^{-\frac{1}{3} \alphaX \m \tX^{2}}
.\end{gather*}
%|<<|===========================================================================================|<<|
Moreover, by a nearly identical argument (omitted here), the other side of the bound is obtained:
%|>>|===========================================================================================|>>|
\begin{gather*}
  \Pr \left(
    \WRV - \E[ \WRV ] < -\alphaX \tX
  \right)
  =
  \Pr \left(
    -\WRV - \E[ -\WRV ] > \alphaX \tX
  \right)
  \leq
  e^{-\frac{1}{3} \alphaX \m \tX^{2}}
,\end{gather*}
%|<<|===========================================================================================|<<|
and thus,
%|>>|===========================================================================================|>>|
\begin{gather*}
  \Pr \left(
    \left\langle \frac{\hfFn[\JCoordsX]( \thetaStar, \thetaStar )}{\sqrt{2\pi}} , \thetaStar \right\rangle
    -
    \E \left[ \left\langle \frac{\hfFn[\JCoordsX]( \thetaStar, \thetaStar )}{\sqrt{2\pi}} , \thetaStar \right\rangle \right]
    <
    -\alphaX \tX
  \right)
  \leq
  e^{-\frac{1}{3} \alphaX \m \tX^{2}}
.\end{gather*}
%|<<|===========================================================================================|<<|
Combining the above inequalities into a two-sided bound via a union bound yields:
%|>>|===========================================================================================|>>|
\begin{gather*}
  \Pr \left(
    \left|
    \left\langle \frac{\hfFn[\JCoordsX]( \thetaStar, \thetaStar )}{\sqrt{2\pi}} , \thetaStar \right\rangle
    -
    \E \left[ \left\langle \frac{\hfFn[\JCoordsX]( \thetaStar, \thetaStar )}{\sqrt{2\pi}} , \thetaStar \right\rangle \right]
    \right|
    >
    \alphaX \tX
  \right)
  \leq
  2 e^{-\frac{1}{3} \alphaX \m \tX^{2}}
.\end{gather*}
%|<<|===========================================================================================|<<|
Lastly, union bounding over all \(  \JCoordsX \in \JSX  \), the desired uniform concentration inequality follows:
%|>>|===========================================================================================|>>|
\begin{gather*}
  \Pr \left(
    \ExistsST{\JCoordsX \in \JSX}
    {\left|
    \left\langle \frac{\hfFn[\JCoordsX]( \thetaStar, \thetaStar )}{\sqrt{2\pi}} , \thetaStar \right\rangle
    -
    \E \left[ \left\langle \frac{\hfFn[\JCoordsX]( \thetaStar, \thetaStar )}{\sqrt{2\pi}} , \thetaStar \right\rangle \right]
    \right|
    >
    \alphaX \tX}
  \right)
  \leq
  2 | \JSX | e^{-\frac{1}{3} \alphaX \m \tX^{2}}
.\end{gather*}
%|<<|===========================================================================================|<<|
%\end{proof}
%%|<<|~~~~~~~~~~~~~~~~~~~~~~~~~~~~~~~~~~~~~~~~~~~~~~~~~~~~~~~~~~~~~~~~~~~~~~~~~~~~~~~~~~~~~~~~~~~|<<|
%%|<<|~~~~~~~~~~~~~~~~~~~~~~~~~~~~~~~~~~~~~~~~~~~~~~~~~~~~~~~~~~~~~~~~~~~~~~~~~~~~~~~~~~~~~~~~~~~|<<|
%%|<<|~~~~~~~~~~~~~~~~~~~~~~~~~~~~~~~~~~~~~~~~~~~~~~~~~~~~~~~~~~~~~~~~~~~~~~~~~~~~~~~~~~~~~~~~~~~|<<|

%%%%%%%%%%%%%%%%%%%%%%%%%%%%%%%%%%%%%%%%%%%%%%%%%%%%%%%%%%%%%%%%%%%%%%%%%%%%%%%%%%%%%%%%%%%%%%%%%%%%
%%%%%%%%%%%%%%%%%%%%%%%%%%%%%%%%%%%%%%%%%%%%%%%%%%%%%%%%%%%%%%%%%%%%%%%%%%%%%%%%%%%%%%%%%%%%%%%%%%%%

\subsubsection{Proof of the \EQUATIONS \eqref{eqn:lemma:concentration-ineq:noisy:pr:2} and \eqref{eqn:lemma:concentration-ineq:noisy:ev:2}}
\label{outline:concentration-ineq|pf-noisy|2}

We begin with some preliminary analysis to characterize a few random variables of interest.
As in the derivations of \EQUATIONS \eqref{eqn:lemma:concentration-ineq:noisy:pr:1} and \eqref{eqn:lemma:concentration-ineq:noisy:ev:1} in \SECTION \ref{outline:concentration-ineq|pf-noisy|1}, consider an arbitrary coordinate subset
%|>>|:::::::::::::::::::::::::::::::::::::::::::::::::::::::::::::::::::::::::::::::::::::::::::|>>|
\(  \JCoordsX \in \JSX  \),
%\(  \JCoordsX \subseteq [\n]  \), \(  | \JCoordsX | \leq \kUnion  \),
%|<<|:::::::::::::::::::::::::::::::::::::::::::::::::::::::::::::::::::::::::::::::::::::::::::|<<|
recall \EQUATION \eqref{eqn:pf:lemma:concentration-ineq:noisy:16} and \eqref{eqn:pf:lemma:concentration-ineq:noisy:1}:
%|>>|===========================================================================================|>>|
\begin{gather*}
  \frac{1}{\sqrt{2\pi}} \hfFn[\JCoordsX]( \thetaStar, \thetaStar )
  =
  -\frac{1}{\m}
  \sum_{\iIx=1}^{\m}
  \CovVX\VIx{\iIx}
  \sep \Sign( \langle \CovVX\VIx{\iIx}, \thetaStar \rangle )
  \sep \I( \fFn( \langle \CovVX\VIx{\iIx}, \thetaStar \rangle ) \neq \Sign( \langle \CovVX\VIx{\iIx}, \thetaStar \rangle ) )
  ,\\
  \left\langle \frac{1}{\sqrt{2\pi}} \hfFn[\JCoordsX]( \thetaStar, \thetaStar ), \thetaStar \right\rangle
  =
  -\frac{1}{\m}
  \sum_{\iIx=1}^{\m}
  \langle \CovVX\VIx{\iIx}, \thetaStar \rangle
  \sep \Sign( \langle \CovVX\VIx{\iIx}, \thetaStar \rangle )
  \sep \I( \fFn( \langle \CovVX\VIx{\iIx}, \thetaStar \rangle ) \neq \Sign( \langle \CovVX\VIx{\iIx}, \thetaStar \rangle ) )
,\end{gather*}
%|<<|===========================================================================================|<<|
where
%|>>|:::::::::::::::::::::::::::::::::::::::::::::::::::::::::::::::::::::::::::::::::::::::::::|>>|
\(  \CovVX\VIx{\iIx} \defeq \ThresholdSet{\Supp( \thetaStar ) \cup \JCoordsX}( \CovV\VIx{\iIx} )  \).
%|<<|:::::::::::::::::::::::::::::::::::::::::::::::::::::::::::::::::::::::::::::::::::::::::::|<<|
Thus,
%|>>|===========================================================================================|>>|
\begin{align*}
  &
  \frac{1}{\sqrt{2\pi}} \hfFn[\JCoordsX]( \thetaStar, \thetaStar )
  -
  \left\langle \frac{1}{\sqrt{2\pi}} \hfFn[\JCoordsX]( \thetaStar, \thetaStar ), \thetaStar \right\rangle
  \thetaStar
  \\
  &\AlignIndent=
  -\frac{1}{\m}
  \sum_{\iIx=1}^{\m}
  \left( \CovVX\VIx{\iIx} - \langle \CovVX\VIx{\iIx}, \thetaStar \rangle \thetaStar \right)
  \sep \Sign( \langle \CovVX\VIx{\iIx}, \thetaStar \rangle )
  \sep \I( \fFn( \langle \CovVX\VIx{\iIx}, \thetaStar \rangle ) \neq \Sign( \langle \CovVX\VIx{\iIx}, \thetaStar \rangle ) )
.\end{align*}
%|<<|===========================================================================================|<<|
Let
%|>>|:::::::::::::::::::::::::::::::::::::::::::::::::::::::::::::::::::::::::::::::::::::::::::|>>|
\(  \kX \defeq | \Supp( \thetaStar ) \cup \JCoordsX |  \),
%|<<|:::::::::::::::::::::::::::::::::::::::::::::::::::::::::::::::::::::::::::::::::::::::::::|<<|
and denote the \(  \kX  \)-dimensional subspace of vectors whose support is a (possibly improper) subset of \(  \Supp( \thetaStar ) \cup \JCoordsX  \) by
%|>>|:::::::::::::::::::::::::::::::::::::::::::::::::::::::::::::::::::::::::::::::::::::::::::|>>|
\(  \Set{V} \defeq \{ \Vec{\vV} \in \R^{\n} : \Supp( \Vec{\vV} ) \subseteq \Supp( \thetaStar ) \cup \JCoordsX \}  \).
%|<<|:::::::::::::::::::::::::::::::::::::::::::::::::::::::::::::::::::::::::::::::::::::::::::|<<|
Let
%|>>|:::::::::::::::::::::::::::::::::::::::::::::::::::::::::::::::::::::::::::::::::::::::::::|>>|
\(  \{ \Vec{\vV}\VIx{1}, \dots, \Vec{\vV}\VIx{\kX} \} \subset \Set{V}  \)
%|<<|:::::::::::::::::::::::::::::::::::::::::::::::::::::::::::::::::::::::::::::::::::::::::::|<<|
be an orthonormal basis of \(  \Set{V}  \), where
%|>>|:::::::::::::::::::::::::::::::::::::::::::::::::::::::::::::::::::::::::::::::::::::::::::|>>|
\(  \Vec{\vV}\VIx{\kX} = \thetaStar  \).
%|<<|:::::::::::::::::::::::::::::::::::::::::::::::::::::::::::::::::::::::::::::::::::::::::::|<<|
Then, for each \(  \iIx \in [\m]  \), since the vector \(  \CovVX\VIx{\iIx}  \) is contained in \(  \Set{V}  \), it is orthogonally decomposed with this bases as:
%|>>|===========================================================================================|>>|
\begin{gather*}
  \CovVX\VIx{\iIx}
  =
  \sum_{\jIx=1}^{\kX}
  \langle \CovVX\VIx{\iIx}, \Vec{\vV}\VIx{\jIx} \rangle \Vec{\vV}\VIx{\jIx}
,\end{gather*}
%|<<|===========================================================================================|<<|
while \(  \CovVX\VIx{\iIx} - \langle \CovVX\VIx{\iIx}, \thetaStar \rangle \thetaStar  \)---which is likewise an element in the vector subspace \(  \Set{V}  \)---is orthogonally decomposed as:
%|>>|===========================================================================================|>>|
\begin{align*}
  \CovVX\VIx{\iIx} - \langle \CovVX\VIx{\iIx}, \thetaStar \rangle \thetaStar
  % &=
  % \sum_{\jIx=1}^{\kX}
  % \langle \CovVX\VIx{\iIx} - \langle \CovVX\VIx{\iIx}, \thetaStar \rangle \thetaStar, \Vec{\vV}\VIx{\jIx} \rangle \Vec{\vV}\VIx{\jIx}
  &=
  \sum_{\jIx=1}^{\kX}
  \bigl(
    \langle \CovVX\VIx{\iIx}, \Vec{\vV}\VIx{\jIx} \rangle
    -
    \langle \CovVX\VIx{\iIx}, \thetaStar \rangle
    \langle \thetaStar, \Vec{\vV}\VIx{\jIx} \rangle
  \bigr)
  \Vec{\vV}\VIx{\jIx}
\XXX{
  \\
  &=
  \langle \CovVX\VIx{\iIx} - \langle \CovVX\VIx{\iIx}, \thetaStar \rangle \thetaStar, \Vec{\vV}\VIx{\kX} \rangle \Vec{\vV}\VIx{\kX}
  +
  \sum_{\jIx=1}^{\kX-1}
  \langle \CovVX\VIx{\iIx} - \langle \CovVX\VIx{\iIx}, \thetaStar \rangle \thetaStar, \Vec{\vV}\VIx{\jIx} \rangle \Vec{\vV}\VIx{\jIx}
  \\
  &\dCmt{by separating out the last term from the summation}
}
  \\
  &=
  \bigl(
    \langle \CovVX\VIx{\iIx}, \thetaStar \rangle
    -
    \langle \CovVX\VIx{\iIx}, \thetaStar \rangle
    \langle \thetaStar, \thetaStar \rangle
  \bigr)
  \thetaStar
  +
  \sum_{\jIx=1}^{\kX-1}
  \bigl(
    \langle \CovVX\VIx{\iIx}, \Vec{\vV}\VIx{\jIx} \rangle
    -
    \langle \CovVX\VIx{\iIx}, \thetaStar \rangle
    \langle \thetaStar, \Vec{\vV}\VIx{\jIx} \rangle
  \bigr)
  \Vec{\vV}\VIx{\jIx}
  \\
  &\dCmt{by separating out the \(  \kX\Th  \) term from the summation,}
  \\
  &\dCmtx{and since \(  \Vec{\vV}\VIx{\kX} = \thetaStar  \) by design}
  \\
  % &=
  % \bigl(
  %   \langle \CovVX\VIx{\iIx}, \thetaStar \rangle
  %   -
  %   \langle \CovVX\VIx{\iIx}, \thetaStar \rangle \langle \thetaStar, \thetaStar \rangle
  % \bigr)
  % \thetaStar
  % +
  % \sum_{\jIx=1}^{\kX-1}
  % \langle \CovVX\VIx{\iIx} - \langle \CovVX\VIx{\iIx}, \thetaStar \rangle \thetaStar, \Vec{\vV}\VIx{\jIx} \rangle \Vec{\vV}\VIx{\jIx}
  % \\
  % &\dCmt{by the linearity of inner products}
  % \\
\XXX{
  &=
  \bigl(
    \langle \CovVX\VIx{\iIx}, \thetaStar \rangle
    -
    \langle \CovVX\VIx{\iIx}, \thetaStar \rangle
  \bigr)
  \thetaStar
  +
  \sum_{\jIx=1}^{\kX-1}
  \langle \CovVX\VIx{\iIx} - \langle \CovVX\VIx{\iIx}, \thetaStar \rangle \thetaStar, \Vec{\vV}\VIx{\jIx} \rangle \Vec{\vV}\VIx{\jIx}
  \\
  &\dCmt{by distributivity and recalling that \(  \langle \thetaStar, \thetaStar \rangle = \| \thetaStar \|_{2}^{2} = 1  \)}
  \\
}
  &=
  \sum_{\jIx=1}^{\kX-1}
  \bigl(
    \langle \CovVX\VIx{\iIx}, \Vec{\vV}\VIx{\jIx} \rangle
    -
    \langle \CovVX\VIx{\iIx}, \thetaStar \rangle
    \langle \thetaStar, \Vec{\vV}\VIx{\jIx} \rangle
  \bigr)
  \Vec{\vV}\VIx{\jIx}
  \\
  &\dCmt{recalling that \(  \langle \thetaStar, \thetaStar \rangle = \| \thetaStar \|_{2}^{2} = 1  \)}
   \\
  % &\dCmt{by cancellation}
  % \\
  % &=
  % \sum_{\jIx=1}^{\kX-1}
  % \langle \CovVX\VIx{\iIx}, \Vec{\vV}\VIx{\jIx} \rangle \Vec{\vV}\VIx{\jIx}
  % -
  % \langle \CovVX\VIx{\iIx}, \thetaStar \rangle
  % \langle \thetaStar, \Vec{\vV}\VIx{\jIx} \rangle \Vec{\vV}\VIx{\jIx}
  % \\
  % &\dCmt{by the linearity of inner products}
  % \\
  &=
  \sum_{\jIx=1}^{\kX-1}
  \langle \CovVX\VIx{\iIx}, \Vec{\vV}\VIx{\jIx} \rangle \Vec{\vV}\VIx{\jIx}
  .\\
  &\dCmt{for \(  \jIx \neq \kX  \), \(  \langle \thetaStar, \Vec{\vV}\VIx{\jIx} \rangle = \langle \Vec{\vV}\VIx{\kX}, \Vec{\vV}\VIx{\jIx} \rangle = 0  \) since \(  \Vec{\vV}\VIx{\jIx} \perp \Vec{\vV}\VIx{\kX}  \) by design}
\end{align*}
%|<<|===========================================================================================|<<|
Using this orthogonal decomposition,
%|>>|===========================================================================================|>>|
\begin{align*}
  &
  \frac{1}{\sqrt{2\pi}} \hfFn[\JCoordsX]( \thetaStar, \thetaStar )
  -
  \left\langle \frac{1}{\sqrt{2\pi}} \hfFn[\JCoordsX]( \thetaStar, \thetaStar ), \thetaStar \right\rangle
  \thetaStar
  \\
  &\AlignIndent=
  -\frac{1}{\m}
  \sum_{\iIx=1}^{\m}
  \left( \CovVX\VIx{\iIx} - \langle \CovVX\VIx{\iIx}, \thetaStar \rangle \thetaStar \right)
  \sep
  \Sign( \langle \CovVX\VIx{\iIx}, \thetaStar \rangle )
  \sep
  \I( \fFn( \langle \CovVX\VIx{\iIx}, \thetaStar \rangle ) \neq \Sign( \langle \CovVX\VIx{\iIx}, \thetaStar \rangle ) )
  \\
  &\AlignIndent=
  -\frac{1}{\m}
  \sum_{\iIx=1}^{\m}
  \sum_{\jIx=1}^{\kX-1}
  \langle \CovVX\VIx{\iIx}, \Vec{\vV}\VIx{\jIx} \rangle \Vec{\vV}\VIx{\jIx}
  \sep
  \Sign( \langle \CovVX\VIx{\iIx}, \thetaStar \rangle )
  \sep
  \I( \fFn( \langle \CovVX\VIx{\iIx}, \thetaStar \rangle ) \neq \Sign( \langle \CovVX\VIx{\iIx}, \thetaStar \rangle ) )
\TagEqn{\label{eqn:pf:lemma:concentration-ineq:noisy:17}}
.\end{align*}
%|<<|===========================================================================================|<<|
Note that by a well-known property of Gaussian vectors, \(  \langle \CovVX\VIx{\iIx}, \Vec{\vV}\VIx{\jIx} \rangle \sim \N(0,1)  \) for each \(  \iIx \in [\m]  \), \(  \jIx \in [\kX]  \), and moreover, due to the orthogonality of \(  \Vec{\vV}\VIx{1}, \dots, \Vec{\vV}\VIx{\kX}  \), the random variables \(  \langle \CovVX\VIx{\iIx}, \Vec{\vV}\VIx{1} \rangle, \dots, \langle \CovVX\VIx{\iIx}, \Vec{\vV}\VIx{\kX} \rangle  \) are mutually independent.
In particular, the random variable \(  \langle \CovVX\VIx{\iIx}, \Vec{\vV}\VIx{\kX} \rangle = \langle \CovVX\VIx{\iIx}, \thetaStar \rangle  \) is independent of every \(  \langle \CovVX\VIx{\iIx}, \Vec{\vV}\VIx{\jIx} \rangle  \), \(  \jIx \in [\kX-1]  \).
Therefore, for every \(  \iIx \in [\m]  \), each \(  \jIx\Th  \) summand, \(  \jIx \in [\kX-1]  \), in \eqref{eqn:pf:lemma:concentration-ineq:noisy:17} follows the same distribution as
%|>>|===========================================================================================|>>|
\begin{gather}
\label{eqn:pf:lemma:concentration-ineq:noisy:19}
  \langle \CovVX\VIx{\iIx}, \Vec{\vV}\VIx{\jIx} \rangle \Vec{\vV}\VIx{\jIx}
  \sep
  \Sign( \langle \CovVX\VIx{\iIx}, \thetaStar \rangle )
  \sep
  \I( \fFn( \langle \CovVX\VIx{\iIx}, \thetaStar \rangle ) \neq \Sign( \langle \CovVX\VIx{\iIx}, \thetaStar \rangle ) )
  \sim
  \Yi \Zij \Vec{\vV}\VIx{\jIx} = \Uij \Vec{\vV}\VIx{\jIx}
,\end{gather}
%|<<|===========================================================================================|<<|
where
%|>>|:::::::::::::::::::::::::::::::::::::::::::::::::::::::::::::::::::::::::::::::::::::::::::|>>|
\(  \Zij[\iIx][1], \dots, \Zij[\iIx][\kX] \sim \N(0,1)  \)
%|<<|:::::::::::::::::::::::::::::::::::::::::::::::::::::::::::::::::::::::::::::::::::::::::::|<<|
are \iid Gaussian random variables, and where
%|>>|===========================================================================================|>>|
\begin{gather*}
  \Ri \defeq \I( \fFn( \Zij[\iIx][\kX] ) \neq \Sign( \Zij[\iIx][\kX] ) )
  ,\\
  \Yi \defeq \Sign( \Zij[\iIx][\kX] ) \Ri
  ,\\
  \Uij \defeq \Yi \Zij
.\end{gather*}
%|<<|===========================================================================================|<<|
% %|>>|:::::::::::::::::::::::::::::::::::::::::::::::::::::::::::::::::::::::::::::::::::::::::::|>>|
% \(  \Ri \defeq \I( \fFn( \Zij[\iIx][\kX] ) \neq \Sign( \Zij[\iIx][\kX] ) )  \),
% \(  \Yi \defeq \Sign( \Zij[\iIx][\kX] ) \Ri  \), and
% \(  \Uij \defeq \Yi \Zij  \).
% %|<<|:::::::::::::::::::::::::::::::::::::::::::::::::::::::::::::::::::::::::::::::::::::::::::|<<|
Conditioned on \(  \Ri=1  \), the random variable is distributed as \(  ( \Yi \Mid| \Ri \neq 0 ) \sim \{ -1,1 \}  \), uniformly, and is independently of all \(  \Zij  \), \(  \jIx \in [\kX-1]  \).
Hence, conditioned on \(  \Ri  \), the random variable \(  \Uij \Mid| \Ri  \) has a density function given for \(  \zX \in \R  \) and \(  \rX \in \{ 0,1 \}  \) by
%|>>|===========================================================================================|>>|
\begin{align*}
  \pdf{\Uij \Mid| \Ri}( \zX \Mid| \rX )
  &=
  \begin{cases}
  0, & \cIf \zX \neq 0, \rX=0, \\
  1, & \cIf \zX = 0,    \rX=0, \\
  \pdf{\Zij}( \zX ) \pdf{\Yi \Mid| \Ri}( 1 \Mid| 1 ) + \pdf{-\Zij}( \zX ) \pdf{\Yi \Mid| \Ri}( -1 \Mid| 1 ), & \cIf \rX=1,
  \end{cases}
  \\
  &=
  \begin{cases}
  0, & \cIf \zX \neq 0, \rX=0, \\
  1, & \cIf \zX = 0,    \rX=0, \\
  \frac{1}{2} \pdf{\Zij}( \zX ) + \frac{1}{2} \pdf{-\Zij}( \zX ), & \cIf \rX=1,
  \end{cases}
  \\
  % &=
  % \begin{cases}
  % 0, & \cIf \zX \neq 0, \rX=0, \\
  % 1, & \cIf \zX = 0,    \rX=0, \\
  % \pdf{\Zij}( \zX ), & \cIf \rX=1,
  % \end{cases}
  % \\
  &=
  \begin{cases}
  0, & \cIf \zX \neq 0, \rX=0, \\
  1, & \cIf \zX = 0,    \rX=0, \\
  \frac{1}{\sqrt{2\pi}} e^{-\frac{1}{2} \zX^{2}}, & \cIf \rX=1,
  \end{cases}
\TagEqn{\label{eqn:pf:lemma:concentration-ineq:noisy:18}}
\end{align*}
%|<<|===========================================================================================|<<|
where the third case on the \RHS of the first equality is due to the law of total probability, the definition of conditional probabilities, the independence of \(  ( \Yi \Mid| \Ri = 1 )  \) and \(  \Zij  \), \(  \jIx \in [\kX-1]  \), and remarks made in the proof of \LEMMA \ref{lemma:concentration-ineq:noiseless}.
\EQUATION \eqref{eqn:pf:lemma:concentration-ineq:noisy:18} implies
%|>>|:::::::::::::::::::::::::::::::::::::::::::::::::::::::::::::::::::::::::::::::::::::::::::|>>|
\(  ( \Uij \Mid| \Ri=1 ) \sim \Zij  \).
%|<<|:::::::::::::::::::::::::::::::::::::::::::::::::::::::::::::::::::::::::::::::::::::::::::|<<|
Additionally, due to an earlier discussion in \SECTION \ref{outline:concentration-ineq|pf-noisy|1}, the mass function of the random variable \(  \Ri  \) is given for \(  \rX \in \{ 0,1 \}  \) by
%|>>|===========================================================================================|>>|
\begin{align*}
  \pdf{\Ri}( \rX )
  =
  \begin{cases}
  1-\alphaX ,& \cIf \rX=0, \\
  \alphaX   ,& \cIf \rX=1.
  \end{cases}
\end{align*}
%|<<|===========================================================================================|<<|
%
%%%%%%%%%%%%%%%%%%%%%%%%%%%%%%%%%%%%%%%%%%%%%%%%%%%%%%%%%%%%%%%%%%%%%%%%%%%%%%%%%%%%%%%%%%%%%%%%%%%%
\par %%%%%%%%%%%%%%%%%%%%%%%%%%%%%%%%%%%%%%%%%%%%%%%%%%%%%%%%%%%%%%%%%%%%%%%%%%%%%%%%%%%%%%%%%%%%%%%
%%%%%%%%%%%%%%%%%%%%%%%%%%%%%%%%%%%%%%%%%%%%%%%%%%%%%%%%%%%%%%%%%%%%%%%%%%%%%%%%%%%%%%%%%%%%%%%%%%%%
%
This concludes the preliminary work.
We now proceed to the derivations of \EQUATIONS \eqref{eqn:lemma:concentration-ineq:noisy:pr:2} and \eqref{eqn:lemma:concentration-ineq:noisy:ev:2}, beginning with the latter.
%
%%%%%%%%%%%%%%%%%%%%%%%%%%%%%%%%%%%%%%%%%%%%%%%%%%%%%%%%%%%%%%%%%%%%%%%%%%%%%%%%%%%%%%%%%%%%%%%%%%%%
\paragraph{Verification of \EQUATION \eqref{eqn:lemma:concentration-ineq:noisy:ev:2}} %%%%%%%%%%%%%%
%%%%%%%%%%%%%%%%%%%%%%%%%%%%%%%%%%%%%%%%%%%%%%%%%%%%%%%%%%%%%%%%%%%%%%%%%%%%%%%%%%%%%%%%%%%%%%%%%%%%
%
It is now possible to verify \EQUATION \eqref{eqn:lemma:concentration-ineq:noisy:ev:2}.
Combining the above arguments with the law of total expectation and the definition of conditional expectations, the expectation of \(  \Uij  \) is calculated as follows:
%|>>|===========================================================================================|>>|
\begin{align*}
  \E[ \Uij ]
  &=
  \pdf{\Ri}(0) \E[ \Uij \Mid| \Ri=0 ] + \pdf{\Ri}(1) \E[ \Uij \Mid| \Ri=1 ]
  \\
  &=
  ( 1-\alphaX ) 0 + \alphaX \int_{\zX=-\infty}^{\zX=\infty} \frac{1}{\sqrt{2\pi}} \zX e^{-\frac{1}{2} \zX^{2}} d\zX
  \\
  &=
  0
\TagEqn{\label{eqn:pf:lemma:concentration-ineq:noisy:22}}
.\end{align*}
%|<<|===========================================================================================|<<|
\EQUATION \eqref{eqn:lemma:concentration-ineq:noisy:ev:2} now follows:
%|>>|===========================================================================================|>>|
\begin{align*}
  % &\negphantom{\AlignSp}
  \E[ \gfFn[\JCoordsX]( \thetaStar, \thetaStar ) ]
\XXX{
  \\
  &=
  \E \left[
    \frac{1}{\sqrt{2\pi}} \hfFn[\JCoordsX]( \thetaStar, \thetaStar )
    -
    \left\langle \frac{1}{\sqrt{2\pi}} \hfFn[\JCoordsX]( \thetaStar, \thetaStar ), \thetaStar \right\rangle
    \thetaStar
  \right]
  \\
  &\dCmt{by the definition of \(  \gfFn[\JCoordsX]  \) in \EQUATION \eqref{eqn:notations:gfJ:def}}
}
  % \\
  % &=
  % \E \left[
  %   -\frac{1}{\m}
  %   \sum_{\iIx=1}^{\m}
  %   \sum_{\jIx=1}^{\kX-1}
  %   \langle \CovVX\VIx{\iIx}, \Vec{\vV}\VIx{\jIx} \rangle \Vec{\vV}\VIx{\jIx}
  %   \sep
  %   \Sign( \langle \CovVX\VIx{\iIx}, \thetaStar \rangle )
  %   \sep
  %   \I( \fFn( \langle \CovVX\VIx{\iIx}, \thetaStar \rangle ) \neq \Sign( \langle \CovVX\VIx{\iIx}, \thetaStar \rangle ) )
  % \right]
  % \\
  % &\dCmt{by the definition of \(  \gfFn[\JCoordsX]  \) in \EQUATION \eqref{eqn:notations:gfJ:def} and by \EQUATION \eqref{eqn:pf:lemma:concentration-ineq:noisy:17}}
  % \\
  &=
  -\frac{1}{\m}
  \sum_{\iIx=1}^{\m}
  \sum_{\jIx=1}^{\kX-1}
  \E \left[
    \langle \CovVX\VIx{\iIx}, \Vec{\vV}\VIx{\jIx} \rangle
    \sep
    \Sign( \langle \CovVX\VIx{\iIx}, \thetaStar \rangle )
    \sep
    \I( \fFn( \langle \CovVX\VIx{\iIx}, \thetaStar \rangle ) \neq \Sign( \langle \CovVX\VIx{\iIx}, \thetaStar \rangle ) )
  \right]
  \Vec{\vV}\VIx{\jIx}
  \\
  &\dCmt{by the definition of \(  \gfFn[\JCoordsX]  \) in \EQUATION \eqref{eqn:notations:gfJ:def} and by \EQUATION \eqref{eqn:pf:lemma:concentration-ineq:noisy:17}}
  \\
  &\dCmtx{and the linearity of expectation}
  \\
  &=
  -\frac{1}{\m}
  \sum_{\iIx=1}^{\m}
  \sum_{\jIx=1}^{\kX-1}
  \E[ \Uij ] \Vec{\vV}\VIx{\jIx}
  \\
  &\dCmt{by \EQUATION \eqref{eqn:pf:lemma:concentration-ineq:noisy:19}}
  \\
  &=
  \Vec{0}
  ,\\
  &\dCmt{by \EQUATION \eqref{eqn:pf:lemma:concentration-ineq:noisy:22}}
\end{align*}
%|<<|===========================================================================================|<<|
as desired.
%
%%%%%%%%%%%%%%%%%%%%%%%%%%%%%%%%%%%%%%%%%%%%%%%%%%%%%%%%%%%%%%%%%%%%%%%%%%%%%%%%%%%%%%%%%%%%%%%%%%%%
\paragraph{Verification of \EQUATION \eqref{eqn:lemma:concentration-ineq:noisy:pr:2}} %%%%%%%%%%%%%%
%%%%%%%%%%%%%%%%%%%%%%%%%%%%%%%%%%%%%%%%%%%%%%%%%%%%%%%%%%%%%%%%%%%%%%%%%%%%%%%%%%%%%%%%%%%%%%%%%%%%
%
The verification of \EQUATION \eqref{eqn:lemma:concentration-ineq:noisy:pr:2} will largely rely on the work already accomplished above, as well as \LEMMA \ref{lemma:norm-subgaussian:1}, which is restated below for convenience as \LEMMA \ref{lemma:norm-subgaussian:2}.
%
%|>>|*******************************************************************************************|>>|
%|>>|*******************************************************************************************|>>|
%|>>|*******************************************************************************************|>>|
\begin{lemma}
\label{lemma:norm-subgaussian:2}
%
%\ToDo{Make sure this is consistent with \LEMMA \ref{lemma:norm-subgaussian:1}.}
Fix \(  \tXXX, \sigma  > 0  \) and \(  0 < \dX \leq \n  \).
Let
%|>>|:::::::::::::::::::::::::::::::::::::::::::::::::::::::::::::::::::::::::::::::::::::::::::|>>|
\(  \ICoords \subseteq [\n]  \), \(  | \ICoords | = \dX  \).
%|<<|:::::::::::::::::::::::::::::::::::::::::::::::::::::::::::::::::::::::::::::::::::::::::::|<<|
Let
%|>>|:::::::::::::::::::::::::::::::::::::::::::::::::::::::::::::::::::::::::::::::::::::::::::|>>|
\(  \Vec{\XRV} \sim \N( \Vec{0}, \sigma^{2} \sum_{\jIx \in \ICoords} \ej \ej^{\T} )  \).
%|<<|:::::::::::::::::::::::::::::::::::::::::::::::::::::::::::::::::::::::::::::::::::::::::::|<<|
Then,
%|>>|===========================================================================================|>>|
\begin{gather}
\label{eqn:lemma:norm-subgaussian:2:1}
  \Pr \left(
    \| \Vec{\XRV} - \E[ \Vec{\XRV} ] \|_{2}
    >
    \sqrt{\dX} \sigma
    +
    \tXXX
  \right)
  \leq
  \Pr \left(
    \| \Vec{\XRV} - \E[ \Vec{\XRV} ] \|_{2}
    >
    \E[ \| \Vec{\XRV} \|_{2} ]
    +
    \tXXX
  \right)
  \leq
  e^{-\frac{1}{2 \sigma^{2}} \tXXX^{2}}
.\end{gather}
%|<<|===========================================================================================|<<|
\end{lemma}
%|<<|*******************************************************************************************|<<|
%|<<|*******************************************************************************************|<<|
%|<<|*******************************************************************************************|<<|
%
Recall that for each \(  \jIx \in [\kX]  \),
%|>>|===========================================================================================|>>|
\begin{gather*}
  \langle \CovVX\VIx{\iIx}, \Vec{\vV}\VIx{\jIx} \rangle \Vec{\vV}\VIx{\jIx}
  \sep
  \Sign( \langle \CovVX\VIx{\iIx}, \thetaStar \rangle )
  \sep
  \I( \fFn( \langle \CovVX\VIx{\iIx}, \thetaStar \rangle ) \neq \Sign( \langle \CovVX\VIx{\iIx}, \thetaStar \rangle ) )
  \sim
  \Yi \Zij \Vec{\vV}\VIx{\jIx} = \Uij \Vec{\vV}\VIx{\jIx}
\end{gather*}
%|<<|===========================================================================================|<<|
per \EQUATION \eqref{eqn:pf:lemma:concentration-ineq:noisy:19}.
Due to the rotational invariance of Gaussians, we will assume going forward, without loss of generality, that the basis vectors, \(  \Vec{\vV}\VIx{1}, \dots, \Vec{\vV}\VIx{\kX}  \), are simply the first \(  \kX  \) standard basis vectors of \(  \R^{\n}  \), i.e., \(  \Vec{\vV}\VIx{\jIx} = \ej  \), \(  \jIx \in [\kX]  \), where \(  \ej \in \R^{\n}  \) is the vector in which  the \(  \jIx\Th  \) entry is \(  1  \) and all other entries are \(  0  \).
Note that under this assumption, \(  \thetaStar = \Vec{\vV}\VIx{\kX} = \ej[\kX]  \), but this, again, does not lose any generality.
For \(  \jIx \in [\kX-1]  \), let
%|>>|===========================================================================================|>>|
\begin{gather*}
  \Uj \defeq \frac{1}{\m} \sum_{\iIx=1}^{\m} \Uij
,\end{gather*}
%|<<|===========================================================================================|<<|
and let
%|>>|===========================================================================================|>>|
\begin{gather*}
  \Vec{\URV} \defeq \sum_{\jIx=1}^{\kX-1} \Uj \ej
.\end{gather*}
%|<<|===========================================================================================|<<|
Writing the random vector
%|>>|:::::::::::::::::::::::::::::::::::::::::::::::::::::::::::::::::::::::::::::::::::::::::::|>>|
\(  \Vec{\RRV} \defeq ( \Ri[1], \dots, \Ri[\m] )  \),
%|<<|:::::::::::::::::::::::::::::::::::::::::::::::::::::::::::::::::::::::::::::::::::::::::::|<<|
and fixing
%|>>|:::::::::::::::::::::::::::::::::::::::::::::::::::::::::::::::::::::::::::::::::::::::::::|>>|
\(  \Vec{\rX} \in \{ 0,1 \}^{\m}  \),
%|<<|:::::::::::::::::::::::::::::::::::::::::::::::::::::::::::::::::::::::::::::::::::::::::::|<<|
notice that
%|>>|===========================================================================================|>>|
\begin{align*}
  ( \Uj \Mid| \Vec{\RRV}=\Vec{\rX} )
  &=
  \frac{1}{\m} \sum_{\iIx \in \Supp( \Vec{\rX} )} ( \Uij \Mid| \Vec{\RRV}=\Vec{\rX} )
\XXX{
  \\
  &=
  \frac{1}{\m} \sum_{\iIx \in \Supp( \Vec{\rX} )} ( \Uij \Mid| \Ri=\rX[\iIx] )
  \\
  &\dCmt{each \(  \Uij  \), \(  \iIx \in [\m]  \), is independent of \(  \{ \Ri[\iIx'] \}_{\iIx' \neq \iIx}  \)}
}
  \\
  &=
  \frac{1}{\m} \sum_{\iIx \in \Supp( \Vec{\rX} )} ( \Uij \Mid| \Ri=1 )
  \\
  &\dCmt{each \(  \Uij  \), \(  \iIx \in [\m]  \), is independent of \(  \{ \Ri[\iIx'] \}_{\iIx' \neq \iIx}  \),}
  \\
  &\dCmtx{and since for each \(  \iIx \in \Supp( \Vec{\rX} )  \), \(  \rX[\iIx]=1  \)}
  \\
  &\sim
  \frac{1}{\m} \sum_{\iIx \in \Supp( \Vec{\rX} )} \Zij
  ,\\
  &\dCmt{by the density of \(  \Uij \Mid| \Ri=1  \) in \EQUATION \eqref{eqn:pf:lemma:concentration-ineq:noisy:18}}
\end{align*}
%|<<|===========================================================================================|<<|
and therefore,
%|>>|:::::::::::::::::::::::::::::::::::::::::::::::::::::::::::::::::::::::::::::::::::::::::::|>>|
\(  ( \Uj \Mid| \Vec{\RRV}=\Vec{\rX} ) \sim \N( 0, \frac{\| \Vec{\rX} \|_{0}}{\m^{2}} )  \).
%|<<|:::::::::::::::::::::::::::::::::::::::::::::::::::::::::::::::::::::::::::::::::::::::::::|<<|
Moreover, letting
%|>>|:::::::::::::::::::::::::::::::::::::::::::::::::::::::::::::::::::::::::::::::::::::::::::|>>|
\(  \LRV \defeq \| \Vec{\RRV} \|_{0}  \) and \(  \EllX \in \ZeroTo{\m}  \),
%|<<|:::::::::::::::::::::::::::::::::::::::::::::::::::::::::::::::::::::::::::::::::::::::::::|<<|
it also happens that
%|>>|:::::::::::::::::::::::::::::::::::::::::::::::::::::::::::::::::::::::::::::::::::::::::::|>>|
\(  ( \Uj \Mid| \LRV = \EllX ) \sim \N( 0, \frac{\EllX}{\m^{2}} )  \)
%|<<|:::::::::::::::::::::::::::::::::::::::::::::::::::::::::::::::::::::::::::::::::::::::::::|<<|
since the random variables, \(  \Zij  \), \(  \iIx \in [\m]  \), are \iid
(This can be more formally argued using the density function of the conditioned random variable \(  \Uj \Mid| \Vec{\RRV}  \), combined with the law of probability.)
Writing the \(  ( \kX-1 )  \)-sparse random vector
%|>>|===========================================================================================|>>|
\begin{align*}
  \Vec{\URV} \defeq \sum_{\jIx=1}^{\kX-1} \Uj \ej
,\end{align*}
%|<<|===========================================================================================|<<|
it follows that the conditioned random vector \(  \Vec{\URV} \Mid| \LRV = \EllX  \) follows a zero-mean Gaussian distribution such that
%|>>|===========================================================================================|>>|
\begin{align*}
  ( \Vec{\URV} \Mid| \LRV = \EllX )
  =
  \left( \sum_{\jIx=1}^{\kX-1} \Uj \ej \middle| \LRV = \EllX \right)
  \sim
  \N \left( \Vec{0}, \frac{\EllX}{\m^{2}} \sum_{\jIx=1}^{\kX-1} \ej \ej^{\T} \right)
,\end{align*}
%|<<|===========================================================================================|<<|
and hence, \(  \Vec{\URV} \Mid| \LRV \leq \EllX  \) is at most \(  \frac{\sqrt{\EllX}}{\m}  \)-\subgaussian with mean
%|>>|:::::::::::::::::::::::::::::::::::::::::::::::::::::::::::::::::::::::::::::::::::::::::::|>>|
\(  \E[ \Vec{\URV} \Mid| \LRV \leq \EllX ] = \Vec{0}  \)
%|<<|:::::::::::::::::::::::::::::::::::::::::::::::::::::::::::::::::::::::::::::::::::::::::::|<<|
and support of cardinality
%|>>|:::::::::::::::::::::::::::::::::::::::::::::::::::::::::::::::::::::::::::::::::::::::::::|>>|
\(  \| \Vec{\URV} \|_{0} = \kX-1  \).
%|<<|:::::::::::::::::::::::::::::::::::::::::::::::::::::::::::::::::::::::::::::::::::::::::::|<<|
Therefore, by \LEMMA \ref{lemma:norm-subgaussian:2} and standard properties of Gaussians,
%(\see \eg, \cite[{\LEMMA D.9}]{matsumoto2022binary}),
%|>>|===========================================================================================|>>|
\begin{align}
\label{eqn:pf:lemma:concentration-ineq:noisy:20}
  &
  \Pr \left(
    \| \Vec{\URV} - \E[ \Vec{\URV} ] \|_{2}
    >
    \frac{\sqrt{( \kX-1 ) \EllX}}{\m}
    +
    \alphaO \tX
  \middle|
    \LRV \leq \EllX
  \right)
  \\
  &\AlignIndent\leq
  \Pr \left(
    \| \Vec{\URV} - \E[ \Vec{\URV} ] \|_{2}
    >
    \frac{\sqrt{( \kX-1 ) \EllX}}{\m}
    +
    \alphaO \tX
  \middle|
    \LRV = \EllX
  \right)
  \\
  &\AlignIndent\leq
  e^{-\frac{\m^{2} \alphaO^{2} \tX^{2}}{2 \EllX}}
.\end{align}
%|<<|===========================================================================================|<<|
Additionally, it is folklore that
%|>>|:::::::::::::::::::::::::::::::::::::::::::::::::::::::::::::::::::::::::::::::::::::::::::|>>|
\(  \LRV = \sum_{\iIx=1}^{\m} \Ri \sim \Binomial( \m, \alphaX )  \)
%|<<|:::::::::::::::::::::::::::::::::::::::::::::::::::::::::::::::::::::::::::::::::::::::::::|<<|
with
\(  \E[ \LRV ] = \E[ \sum_{\iIx=1}^{\m} \Ri ] = \alphaX \m  \)
(\seeeg \cite{charikar2002similarity}).
Thus, letting
%|>>|:::::::::::::::::::::::::::::::::::::::::::::::::::::::::::::::::::::::::::::::::::::::::::|>>|
\(  \LRVO \sim \Binomial( \m, \alphaO )  \)
%|<<|:::::::::::::::::::::::::::::::::::::::::::::::::::::::::::::::::::::::::::::::::::::::::::|<<|
be a binomial random variable with mean
%|>>|:::::::::::::::::::::::::::::::::::::::::::::::::::::::::::::::::::::::::::::::::::::::::::|>>|
\(  \E[ \LRVO ] = \alphaO \m  \),
%|<<|:::::::::::::::::::::::::::::::::::::::::::::::::::::::::::::::::::::::::::::::::::::::::::|<<|
by recalling that
%|>>|:::::::::::::::::::::::::::::::::::::::::::::::::::::::::::::::::::::::::::::::::::::::::::|>>|
\(  \alphaO \defeq \alphaOExpr \geq \alphaX  \),
%|<<|:::::::::::::::::::::::::::::::::::::::::::::::::::::::::::::::::::::::::::::::::::::::::::|<<|
and by a standard concentration inequality for binomial random variables, for \(  \sXXX \in (0,1)  \),
%|>>|===========================================================================================|>>|
\begin{gather}
\label{eqn:pf:lemma:concentration-ineq:noisy:21}
  \Pr( \LRV > ( 1+\sXXX ) \alphaO \m )
  \leq
  \Pr( \LRVO > ( 1+\sXXX ) \alphaO \m )
  \leq
  e^{-\frac{1}{3} \alphaO \m \sXXX^{2}}
.\end{gather}
%|<<|===========================================================================================|<<|
%Note that
%%|>>|:::::::::::::::::::::::::::::::::::::::::::::::::::::::::::::::::::::::::::::::::::::::::::|>>|
%\(  \frac{\sqrt{( 1+\sXXX ) ( \kX-1 ) \alphaX \m}}{\m} = \sqrt{\frac{1}{\m} \alphaX ( 1+\sXXX ) ( \kX-1 )}  \).
%%|<<|:::::::::::::::::::::::::::::::::::::::::::::::::::::::::::::::::::::::::::::::::::::::::::|<<|
Applying the law of total probability gives way to
%|>>|===========================================================================================|>>|
\begin{align*}
  &\negphantom{\AlignSp}
  \Pr \left(
    \| \Vec{\URV} - \E[ \Vec{\URV} ] \|_{2}
    >
    \sqrt{\frac{\alphaO ( 1+\sXXX ) ( \kX-1 )}{\m} }
    +
    \alphaO \tX
  \right)
  \\
  &=
  \Pr \left(
    \| \Vec{\URV} - \E[ \Vec{\URV} ] \|_{2} > \sqrt{\frac{\alphaO ( 1+\sXXX ) ( \kX-1 ) }{\m} } + \alphaO \tX
  \middle|
    \LRV \leq ( 1+\sXXX ) \alphaO \m
  \right)
  \Pr( \LRV \leq ( 1+\sXXX ) \alphaO \m )
  \\
  &\AlignSp+
  \Pr \left(
    \| \Vec{\URV} - \E[ \Vec{\URV} ] \|_{2} > \sqrt{\frac{\alphaO ( 1+\sXXX ) ( \kX-1 )}{\m} } + \alphaO \tX
  \middle|
    \LRV > ( 1+\sXXX ) \alphaO \m
  \right)
  \Pr( \LRV > ( 1+\sXXX ) \alphaO \m )
  \\
  &\dCmt{by the law of total probability and the definition of conditional probabilities}
  \\
  &\leq
  \Pr \left(
    \| \Vec{\URV} - \E[ \Vec{\URV} ] \|_{2} > \sqrt{\frac{\alphaO ( 1+\sXXX ) ( \kX-1 )}{\m} } + \alphaO \tX
  \middle|
    \LRV \leq ( 1+\sXXX ) \alphaO \m
  \right)
  +
  \Pr( \LRV > ( 1+\sXXX ) \alphaO \m )
  \\
  &\leq
  e^{-\frac{1}{2 ( 1+\sXXX )} \alphaO \m \tX^{2}}
  +
  e^{-\frac{1}{3} \alphaO \m \sXXX^{2}}
  \TagEqn{\label{eqn:pf:lemma:concentration-ineq:noisy:2:1}}
  .\\
  &\dCmt{by \EQUATIONS \eqref{eqn:pf:lemma:concentration-ineq:noisy:20} and \eqref{eqn:pf:lemma:concentration-ineq:noisy:21} and because \(  {\textstyle \frac{\alphaO^{2} \m^{2} \tX^{2}}{2 \alphaO \m ( 1+\sXXX )} = \frac{\alphaO \m \tX^{2}}{2 ( 1+\sXXX )}}  \)}
\end{align*}
%|<<|===========================================================================================|<<|
For an arbitrary fixing of \(  \JCoordsX \in \JSX  \), the variables \(  \kX  \) and \(  \kOX  \) satisfy
%|>>|:::::::::::::::::::::::::::::::::::::::::::::::::::::::::::::::::::::::::::::::::::::::::::|>>|
%\(  \kX = | \Supp( \thetaStar ) \cup \JCoordsX | \leq \min \{ | \Supp( \thetaStar ) | + | \JCoordsX |, \n \} \leq \min \{ \k + \max_{\JCoordsX' \in \JSX} | \JCoordsX' |, \n \} = \kOX  \).
%|<<|:::::::::::::::::::::::::::::::::::::::::::::::::::::::::::::::::::::::::::::::::::::::::::|<<|
%|>>|===========================================================================================|>>|
\begin{gather*}
  \kX = | \Supp( \thetaStar ) \cup \JCoordsX | \leq \min \{ | \Supp( \thetaStar ) | + | \JCoordsX |, \n \} \leq \min \{ \k + \max_{\JCoordsX' \in \JSX} | \JCoordsX' |, \n \} = \kOX
.\end{gather*}
%|<<|===========================================================================================|<<|
It follows from this and the above bound in \EQUATION \eqref{eqn:pf:lemma:concentration-ineq:noisy:2:1} that
%|>>|===========================================================================================|>>|
\begin{align*}
  &\negphantom{\AlignIndent}
  \Pr \left(
    \left\| \frac{\gfFn[\JCoordsX]( \thetaStar, \thetaStar )}{\sqrt{2\pi}}  - \E \left[ \frac{\gfFn[\JCoordsX]( \thetaStar, \thetaStar )}{\sqrt{2\pi}}  \right]  \right\|_{2}
    >
    \sqrt{\frac{\alphaO ( 1+\sXXX ) ( \kOX-1 )}{\m} }
    +
    \alphaO \tX
  \right)
  \\
  &\leq
  \Pr \left(
    \left\| \frac{\gfFn[\JCoordsX]( \thetaStar, \thetaStar )}{\sqrt{2\pi}}  - \E \left[ \frac{\gfFn[\JCoordsX]( \thetaStar, \thetaStar )}{\sqrt{2\pi}}  \right]  \right\|_{2}
    >
    \sqrt{\frac{\alphaO ( 1+\sXXX ) ( \kX-1 )}{\m} }
    +
    \alphaO \tX
  \right)
  \\
  &\dCmt{due to the above remark that \(  \kX \leq \kOX  \)}
  \\
  &\leq
  e^{-\frac{1}{2 ( 1+\sXXX )} \alphaO \m \tX^{2}}
  +
  e^{-\frac{1}{3} \alphaO \m \sXXX^{2}}
  \TagEqn{\label{eqn:pf:lemma:concentration-ineq:noisy:2:2}}
  .\\
  &\dCmt{by \EQUATION \eqref{eqn:pf:lemma:concentration-ineq:noisy:2:1} and the definition of \(  \Vec{\URV}  \)}
\end{align*}
%|<<|===========================================================================================|<<|
To obtain a uniform result over all \(  \JCoordsX \in \JSX  \), a union bound over \(  \JSX  \) can be applied to the probability corresponding to the first term on the \RHS of the above inequality, \eqref{eqn:pf:lemma:concentration-ineq:noisy:2:2}, yielding \EQUATION \eqref{eqn:lemma:concentration-ineq:noisy:pr:2}:
%|>>|===========================================================================================|>>|
\begin{gather*}
  \Pr \left(
    \ExistsST{\JCoordsX \in \JSX}
    {\left\| \frac{\gfFn[\JCoordsX]( \thetaStar, \thetaStar )}{\sqrt{2\pi}}  - \E \left[ \frac{\gfFn[\JCoordsX]( \thetaStar, \thetaStar )}{\sqrt{2\pi}}  \right]  \right\|_{2}
    >
    \sqrt{\frac{\alphaO ( 1+\sXXX ) ( \kOX-1 )}{\m} }
    +
    \alphaO \tX}
  \right)
  \\
  \leq
  | \JSX | e^{-\frac{1}{2 ( 1+\sXXX )} \alphaO \m \tX^{2}}
  +
  e^{-\frac{1}{3} \alphaO \m \sXXX^{2}}
,\end{gather*}
%|<<|===========================================================================================|<<|
as desired.
\end{proof}
%|<<|~~~~~~~~~~~~~~~~~~~~~~~~~~~~~~~~~~~~~~~~~~~~~~~~~~~~~~~~~~~~~~~~~~~~~~~~~~~~~~~~~~~~~~~~~~~|<<|
%|<<|~~~~~~~~~~~~~~~~~~~~~~~~~~~~~~~~~~~~~~~~~~~~~~~~~~~~~~~~~~~~~~~~~~~~~~~~~~~~~~~~~~~~~~~~~~~|<<|
%|<<|~~~~~~~~~~~~~~~~~~~~~~~~~~~~~~~~~~~~~~~~~~~~~~~~~~~~~~~~~~~~~~~~~~~~~~~~~~~~~~~~~~~~~~~~~~~|<<|

%%%%%%%%%%%%%%%%%%%%%%%%%%%%%%%%%%%%%%%%%%%%%%%%%%%%%%%%%%%%%%%%%%%%%%%%%%%%%%%%%%%%%%%%%%%%%%%%%%%%
%%%%%%%%%%%%%%%%%%%%%%%%%%%%%%%%%%%%%%%%%%%%%%%%%%%%%%%%%%%%%%%%%%%%%%%%%%%%%%%%%%%%%%%%%%%%%%%%%%%%
%%%%%%%%%%%%%%%%%%%%%%%%%%%%%%%%%%%%%%%%%%%%%%%%%%%%%%%%%%%%%%%%%%%%%%%%%%%%%%%%%%%%%%%%%%%%%%%%%%%%

\subsubsection{Proof of \LEMMA \ref{lemma:pf:lemma:concentration-ineq:noisy:f1,f2}}
\label{outline:concentration-ineq|pf-noisy|pf-f1,f2}

This section establishes the auxiliary result, \LEMMA \ref{lemma:pf:lemma:concentration-ineq:noisy:f1,f2}, stated and used in the proof of \LEMMA \ref{lemma:concentration-ineq:noisy}.

%|>>|~~~~~~~~~~~~~~~~~~~~~~~~~~~~~~~~~~~~~~~~~~~~~~~~~~~~~~~~~~~~~~~~~~~~~~~~~~~~~~~~~~~~~~~~~~~|>>|
%|>>|~~~~~~~~~~~~~~~~~~~~~~~~~~~~~~~~~~~~~~~~~~~~~~~~~~~~~~~~~~~~~~~~~~~~~~~~~~~~~~~~~~~~~~~~~~~|>>|
%|>>|~~~~~~~~~~~~~~~~~~~~~~~~~~~~~~~~~~~~~~~~~~~~~~~~~~~~~~~~~~~~~~~~~~~~~~~~~~~~~~~~~~~~~~~~~~~|>>|
\begin{proof}
{\LEMMA \ref{lemma:pf:lemma:concentration-ineq:noisy:f1,f2}}
%
\checkoff%
%
Recall the definitions of the functions \(  \fFnX, \fFnXX : \R \to \R  \) from \EQUATIONS \eqref{eqn:pf:lemma:concentration-ineq:noisy:f1} and \eqref{eqn:pf:lemma:concentration-ineq:noisy:f2}, respectively:
%|>>|===========================================================================================|>>|
\begin{gather*}
  \fFnX( \sX )
  \defeq
  \frac{1}{\sqrt{2\pi}}
  e^{-\sX \muX[1]}
  \int_{\zX=0}^{\zX=\infty}
  e^{-\frac{1}{2} (\zX-\sX)^{2}} (\pExpr{\zX})
  d\zX
  =
  \frac{1}{\sqrt{2\pi}}
  e^{-\sX \muX[1]}
  \int_{\zX=0}^{\zX=\infty}
  e^{-\frac{1}{2} (\zX-\sX)^{2}} \pExprFn( \zX )
  d\zX
  ,\\
  \fFnXX( \sX )
  \defeq
  \frac{1}{\sqrt{2\pi}}
  e^{\sX \muX[1]}
  \int_{\zX=0}^{\zX=\infty}
  e^{-\frac{1}{2} (\zX+\sX)^{2}} (\pExpr{\zX})
  d\zX
  =
  \frac{1}{\sqrt{2\pi}}
  e^{\sX \muX[1]}
  \int_{\zX=0}^{\zX=\infty}
  e^{-\frac{1}{2} (\zX+\sX)^{2}} \pExprFn( \zX )
  d\zX
,\end{gather*}
%|<<|===========================================================================================|<<|
where
%|>>|:::::::::::::::::::::::::::::::::::::::::::::::::::::::::::::::::::::::::::::::::::::::::::|>>|
% \(  \muX[1] = \frac{1-\sqrt{\frac{\pi}{2}}\gammaX}{\sqrt{2\pi} \alphaX} = \frac{\sqrt{\hfrac{2}{\pi}}-\gammaX}{2\alphaX}  \)
\(  \muX[1] = \frac{\sqrt{\hfrac{2}{\pi}}-\gammaX}{2\alphaX}  \)
%|<<|:::::::::::::::::::::::::::::::::::::::::::::::::::::::::::::::::::::::::::::::::::::::::::|<<|
and
%|>>|:::::::::::::::::::::::::::::::::::::::::::::::::::::::::::::::::::::::::::::::::::::::::::|>>|
\(  \pExprFn( \zX ) = \pExpr{\zX}  \), \(  \zX \in \R  \).
%|<<|:::::::::::::::::::::::::::::::::::::::::::::::::::::::::::::::::::::::::::::::::::::::::::|<<|
Due to \CONDITION \ref{condition:assumption:p:i} of \ASSUMPTION \ref{assumption:p}, the function \(  \pFn  \) is nondecreasing over the real line, which implies that \(  \pExprFn  \) is nonincreasing.
%Additionally, by \CONDITION \ref{condition:assumption:p:ii} of \ASSUMPTION \ref{assumption:p} and \REMARK \ref{remark:condition:assumption:p:ii}, \(  \pExprFn  \) satisfies
Additionally, by \CONDITION \ref{condition:assumption:p:ii} of \ASSUMPTION \ref{assumption:p}, \(  \pExprFn  \) satisfies
%|>>|===========================================================================================|>>|
\begin{gather*}
  \frac{\pExprFn( \zX+\wX )}{\pExprFn( \zX )}
  \geq
  \frac{\pExprFn( \zXX+\wX )}{\pExprFn( \zXX )}
\end{gather*}
%|<<|===========================================================================================|<<|
for
%|>>|:::::::::::::::::::::::::::::::::::::::::::::::::::::::::::::::::::::::::::::::::::::::::::|>>|
\(  \zX \leq \zXX \in [0,\infty)  \) and \(  \wX > 0  \).
%|<<|:::::::::::::::::::::::::::::::::::::::::::::::::::::::::::::::::::::::::::::::::::::::::::|<<|
%Note that the former assumption implies that \(  \pExprFn  \) (possibly nonstrictly) monotonically decrease on the interval \(  [0, \infty)  \).
%
%%%%%%%%%%%%%%%%%%%%%%%%%%%%%%%%%%%%%%%%%%%%%%%%%%%%%%%%%%%%%%%%%%%%%%%%%%%%%%%%%%%%%%%%%%%%%%%%%%%%
\par %%%%%%%%%%%%%%%%%%%%%%%%%%%%%%%%%%%%%%%%%%%%%%%%%%%%%%%%%%%%%%%%%%%%%%%%%%%%%%%%%%%%%%%%%%%%%%%
%%%%%%%%%%%%%%%%%%%%%%%%%%%%%%%%%%%%%%%%%%%%%%%%%%%%%%%%%%%%%%%%%%%%%%%%%%%%%%%%%%%%%%%%%%%%%%%%%%%%
%
Next, we will establish the lemma's result for \(  \fFnX( 0 )  \) and \(  \fFnXX( 0 )  \).
%
%%%%%%%%%%%%%%%%%%%%%%%%%%%%%%%%%%%%%%%%%%%%%%%%%%%%%%%%%%%%%%%%%%%%%%%%%%%%%%%%%%%%%%%%%%%%%%%%%%%%
\paragraph{Verification of \EQUATIONS \eqref{eqn:pf:lemma:concentration-ineq:noisy:f1(0)} and \eqref{eqn:pf:lemma:concentration-ineq:noisy:f2(0)}}
%%%%%%%%%%%%%%%%%%%%%%%%%%%%%%%%%%%%%%%%%%%%%%%%%%%%%%%%%%%%%%%%%%%%%%%%%%%%%%%%%%%%%%%%%%%%%%%%%%%%
%
\EQUATIONS \eqref{eqn:pf:lemma:concentration-ineq:noisy:f1(0)} and \eqref{eqn:pf:lemma:concentration-ineq:noisy:f2(0)} in the lemma are simple to verify:
%|>>|===========================================================================================|>>|
\begin{align*}
  \fFnX( 0 )
  &=
  \frac{1}{\sqrt{2\pi}}
  e^{-0 \muX[1]}
  \int_{\zX=0}^{\zX=\infty}
  e^{-\frac{1}{2} (\zX-0)^{2}} (\pExpr{\zX})
  d\zX
  \\
  &=
  \frac{1}{\sqrt{2\pi}}
  \int_{\zX=0}^{\zX=\infty}
  e^{-\frac{1}{2} \zX^{2}} (\pExpr{\zX})
  d\zX
  \\
  &=
  \alphaX
,\end{align*}
%|<<|===========================================================================================|<<|
and likewise,
%|>>|===========================================================================================|>>|
\begin{align*}
  \fFnXX( 0 )
  &=
  \frac{1}{\sqrt{2\pi}}
  e^{0 \muX[1]}
  \int_{\zX=0}^{\zX=\infty}
  e^{-\frac{1}{2} (\zX+0)^{2}} (\pExpr{\zX})
  d\zX
  \\
  &=
  \frac{1}{\sqrt{2\pi}}
  \int_{\zX=0}^{\zX=\infty}
  e^{-\frac{1}{2} \zX^{2}} (\pExpr{\zX})
  d\zX
  \\
  &=
  \alphaX
,\end{align*}
%|<<|===========================================================================================|<<|
where the last equality in each derivation follows directly from the definition of \(  \alphaX  \) in \EQUATION \eqref{eqn:notations:alpha:def}.
%
%%%%%%%%%%%%%%%%%%%%%%%%%%%%%%%%%%%%%%%%%%%%%%%%%%%%%%%%%%%%%%%%%%%%%%%%%%%%%%%%%%%%%%%%%%%%%%%%%%%%
\paragraph{Verification of \EQUATION \eqref{eqn:pf:lemma:concentration-ineq:noisy:f1:ub}} %%%%%%%%%%
%%%%%%%%%%%%%%%%%%%%%%%%%%%%%%%%%%%%%%%%%%%%%%%%%%%%%%%%%%%%%%%%%%%%%%%%%%%%%%%%%%%%%%%%%%%%%%%%%%%%
%
Moving on to the upper bound on \(  \fFnX  \) in \EQUATION \eqref{eqn:pf:lemma:concentration-ineq:noisy:f1:ub}, it suffices to show that
%|>>|:::::::::::::::::::::::::::::::::::::::::::::::::::::::::::::::::::::::::::::::::::::::::::|>>|
\(  \frac{d}{d\sX} \fFnX( \sX ) \leq 0  \)
%|<<|:::::::::::::::::::::::::::::::::::::::::::::::::::::::::::::::::::::::::::::::::::::::::::|<<|
for all \(  \sX \geq 0  \) since this implies, by basic calculus, that
%|>>|:::::::::::::::::::::::::::::::::::::::::::::::::::::::::::::::::::::::::::::::::::::::::::|>>|
\(  \fFnX( \sX ) \leq \fFnX( 0 )  \)
%|<<|:::::::::::::::::::::::::::::::::::::::::::::::::::::::::::::::::::::::::::::::::::::::::::|<<|
over the interval \(  \sX \in [0,\infty)  \).
Observe:
%|>>|===========================================================================================|>>|
\begin{gather*}
  \frac{d}{d\sX} \fFnX( \sX )
  =
  e^{-\sX \muX[1]}
  \frac{1}{\sqrt{2\pi}}
  \int_{\zX=0}^{\zX=\infty}
  ( \zX-\sX-\muX[1] )
  e^{-\frac{1}{2} ( \zX-\sX )^{2}}
  \pExprFn( \zX )
  d\zX
.\end{gather*}
%|<<|===========================================================================================|<<|
When \(  \sX=0  \), the desired inequality,
%|>>|:::::::::::::::::::::::::::::::::::::::::::::::::::::::::::::::::::::::::::::::::::::::::::|>>|
\(  \frac{d}{d\sX} \fFnX( \sX ) \leq 0  \),
%|<<|:::::::::::::::::::::::::::::::::::::::::::::::::::::::::::::::::::::::::::::::::::::::::::|<<|
is true:
%|>>|===========================================================================================|>>|
\begin{align*}
  \frac{d}{d\sX} \fFnX( 0 )
  &=
  \frac{1}{\sqrt{2\pi}}
  \int_{\zX=0}^{\zX=\infty}
  ( \zX-\muX[1] )
  e^{-\frac{1}{2} \zX^{2}}
  \pExprFn( \zX )
  d\zX
  \\
  &=
  \frac{1}{\sqrt{2\pi}}
  \int_{\zX=0}^{\zX=\infty}
  \zX
  e^{-\frac{1}{2} \zX^{2}}
  \pExprFn( \zX )
  d\zX
  -
  \muX[1]
  \frac{1}{\sqrt{2\pi}}
  \int_{\zX=0}^{\zX=\infty}
  e^{-\frac{1}{2} \zX^{2}}
  \pExprFn( \zX )
  d\zX
  % \\
  % &\dCmt{by the linearity of integration}
  \\
  &=
  %\frac{\gammaX}{\sqrt{2\pi}}
  \frac{\sqrt{\hfrac{2}{\pi}}-\gammaX}{2}
  -
  %\frac{\gammaX}{\sqrt{2\pi} \alphaX}
  \frac{\sqrt{\hfrac{2}{\pi}}-\gammaX}{2\alphaX}
  \alphaX
  \\
  &\dCmt{by the definitions of \(  \alphaX, \gammaX  \) in \EQUATIONS \eqref{eqn:notations:alpha:def} and \eqref{eqn:notations:gamma:def}, respectively,}
  \\
  &\dCmtIndent\text{and an earlier remark in \EQUATION \eqref{eqn:pf:lemma:concentration-ineq:noisy:7} that \(  {\textstyle \muX[1] = \tfrac{\sqrt{\hfrac{2}{\pi}}-\gammaX}{2\alphaX}}  \)}
  % \\
  % &=
  % \frac{\sqrt{\hfrac{2}{\pi}}-\gammaX}{2}
  % -
  % \frac{\sqrt{\hfrac{2}{\pi}}-\gammaX}{2}
  \\
  &=
  0
.\end{align*}
%|<<|===========================================================================================|<<|
%
%%%%%%%%%%%%%%%%%%%%%%%%%%%%%%%%%%%%%%%%%%%%%%%%%%%%%%%%%%%%%%%%%%%%%%%%%%%%%%%%%%%%%%%%%%%%%%%%%%%%
\par %%%%%%%%%%%%%%%%%%%%%%%%%%%%%%%%%%%%%%%%%%%%%%%%%%%%%%%%%%%%%%%%%%%%%%%%%%%%%%%%%%%%%%%%%%%%%%%
%%%%%%%%%%%%%%%%%%%%%%%%%%%%%%%%%%%%%%%%%%%%%%%%%%%%%%%%%%%%%%%%%%%%%%%%%%%%%%%%%%%%%%%%%%%%%%%%%%%%
%
On the other hand, the case when \(  \sX > 0  \) will require more work.
%%%%%%%%%%%%%%%%%%%%%%%%%%%%%%%%%%%%%%%%%%%%%%%%%%%%%%%%%%%%%%%%%%%%%%%%%%%%%%%%%%%%%%%%%%%%%%%%%%%%
\renewcommand{\wX}{\sX}%%%%%%%%%%%%%%%%%%%%%%%%%%%%%%%%%%%%%%%%%%%%%%%%%%%%%%%%%%%%%%%%%%%%%%%%%%%%%
%%%%%%%%%%%%%%%%%%%%%%%%%%%%%%%%%%%%%%%%%%%%%%%%%%%%%%%%%%%%%%%%%%%%%%%%%%%%%%%%%%%%%%%%%%%%%%%%%%%%
Notice that
%|>>|===========================================================================================|>>|
\begin{gather}
\label{eqn:pf:lemma:pf:lemma:concentration-ineq:noisy:f1,f2:1}
  \frac{d}{d\sX} \fFnX( \wX )
  =
  e^{-\sX \muX[1]}
  \frac{1}{\sqrt{2\pi}}
  \int_{\zX=0}^{\zX=\infty}
  ( \zX-\wX-\muX[1] )
  e^{-\frac{1}{2} ( \zX-\wX )^{2}}
  \pExprFn( \zX )
  d\zX
  < 0
\end{gather}
%|<<|===========================================================================================|<<|
if and only if
%|>>|===========================================================================================|>>|
\begin{gather}
\label{eqn:pf:lemma:pf:lemma:concentration-ineq:noisy:f1,f2:2}
  \int_{\zX=0}^{\zX=\infty}
  ( \zX-\wX-\muX[1] )
  e^{-\frac{1}{2} ( \zX-\wX )^{2}}
  \pExprFn( \zX )
  d\zX
  < 0
,\end{gather}
%|<<|===========================================================================================|<<|
and similarly,
%|>>|===========================================================================================|>>|
\begin{gather}
\label{eqn:pf:lemma:pf:lemma:concentration-ineq:noisy:f1,f2:3}
  \frac{d}{d\sX} \fFnX( 0 )
  =
  \frac{1}{\sqrt{2\pi}}
  \int_{\zX=0}^{\zX=\infty}
  ( \zX-\muX[1] )
  e^{-\frac{1}{2} \zX^{2}}
  \pExprFn( \zX )
  d\zX
  = 0
\end{gather}
%|<<|===========================================================================================|<<|
if and only if
%|>>|===========================================================================================|>>|
\begin{gather}
\label{eqn:pf:lemma:pf:lemma:concentration-ineq:noisy:f1,f2:4}
  \int_{\zX=0}^{\zX=\infty}
  ( \zX-\muX[1] )
  e^{-\frac{1}{2} \zX^{2}}
  \pExprFn( \zX )
  d\zX
  = 0
.\end{gather}
%|<<|===========================================================================================|<<|
We already have that
%|>>|:::::::::::::::::::::::::::::::::::::::::::::::::::::::::::::::::::::::::::::::::::::::::::|>>|
\(  \frac{d}{d\sX} \fFnX( 0 ) = 0  \),
%|<<|:::::::::::::::::::::::::::::::::::::::::::::::::::::::::::::::::::::::::::::::::::::::::::|<<|
which implies by the above observation that \EQUATION \eqref{eqn:pf:lemma:pf:lemma:concentration-ineq:noisy:f1,f2:4} also holds.
%%|>>|:::::::::::::::::::::::::::::::::::::::::::::::::::::::::::::::::::::::::::::::::::::::::::|>>|
%\(  \int_{\zX=0}^{\zX=\infty} ( \zX-\muX[1] ) e^{-\frac{1}{2} \zX^{2}} \pExprFn( \zX ) d\zX = 0  \).
%%|<<|:::::::::::::::::::::::::::::::::::::::::::::::::::::::::::::::::::::::::::::::::::::::::::|<<|
%
%%%%%%%%%%%%%%%%%%%%%%%%%%%%%%%%%%%%%%%%%%%%%%%%%%%%%%%%%%%%%%%%%%%%%%%%%%%%%%%%%%%%%%%%%%%%%%%%%%%%
\par %%%%%%%%%%%%%%%%%%%%%%%%%%%%%%%%%%%%%%%%%%%%%%%%%%%%%%%%%%%%%%%%%%%%%%%%%%%%%%%%%%%%%%%%%%%%%%%
%%%%%%%%%%%%%%%%%%%%%%%%%%%%%%%%%%%%%%%%%%%%%%%%%%%%%%%%%%%%%%%%%%%%%%%%%%%%%%%%%%%%%%%%%%%%%%%%%%%%
%
The next argument focuses in on the former biconditional statement---in particular, the establishment of \EQUATION \eqref{eqn:pf:lemma:pf:lemma:concentration-ineq:noisy:f1,f2:2}.
To derive \EQUATION \eqref{eqn:pf:lemma:pf:lemma:concentration-ineq:noisy:f1,f2:2}, the interval of integration on its \LHS is partitioned into three intervals:
%|>>|===========================================================================================|>>|
\begin{align*}
  &\negphantom{\AlignSp}
  \int_{\zX=0}^{\zX=\infty}
  ( \zX-\wX-\muX[1] )
  e^{-\frac{1}{2} ( \zX-\wX )^{2}}
  \pExprFn( \zX )
  d\zX
  \\
  &=
  \int_{\zX=0}^{\zX=\wX}
  ( \zX-\wX-\muX[1] )
  e^{-\frac{1}{2} ( \zX-\wX )^{2}}
  \pExprFn( \zX )
  d\zX
  +
  \int_{\zX=\wX}^{\zX=\wX+\muX[1]}
  ( \zX-\wX-\muX[1] )
  e^{-\frac{1}{2} ( \zX-\wX )^{2}}
  \pExprFn( \zX )
  d\zX
  \\
  &\AlignSp+
  \int_{\zX=\wX+\muX[1] }^{\zX=\infty}
  ( \zX-\wX-\muX[1] )
  e^{-\frac{1}{2} ( \zX-\wX )^{2}}
  \pExprFn( \zX )
  d\zX
  \\
  &=
  \int_{\zX=-\wX}^{\zX=0}
  ( \zX-\muX[1] )
  e^{-\frac{1}{2} \zX^{2}}
  \pExprFn( \zX+\wX )
  d\zX
  +
  \int_{\zX=0}^{\zX=\muX[1]}
  ( \zX-\muX[1] )
  e^{-\frac{1}{2} \zX^{2}}
  \pExprFn( \zX+\wX )
  d\zX
  \\
  &\AlignSp+
  \int_{\zX=\muX[1] }^{\zX=\infty}
  ( \zX-\muX[1] )
  e^{-\frac{1}{2} \zX^{2}}
  \pExprFn( \zX+\wX )
  d\zX
\TagEqn{\label{eqn:pf:lemma:pf:lemma:concentration-ineq:noisy:f1,f2:5}}
,\end{align*}
%|<<|===========================================================================================|<<|
where the second equality applies a change of variables.
Clearly, the first of the three integrals in the last expression in \eqref{eqn:pf:lemma:pf:lemma:concentration-ineq:noisy:f1,f2:5} is negative when \(  \sX > 0  \):
%|>>|===========================================================================================|>>|
\begin{gather}
\label{eqn:pf:lemma:pf:lemma:concentration-ineq:noisy:f1,f2:9}
  \int_{\zX=-\wX}^{\zX=0}
  ( \zX-\muX[1] )
  e^{-\frac{1}{2} \zX^{2}}
  \pExprFn( \zX+\wX )
  d\zX
  < 0
,\end{gather}
%|<<|===========================================================================================|<<|
and thus, if the second and third integrals in the last expression in \eqref{eqn:pf:lemma:pf:lemma:concentration-ineq:noisy:f1,f2:5} sum to a nonpositive value, then \EQUATION \eqref{eqn:pf:lemma:pf:lemma:concentration-ineq:noisy:f1,f2:2}---and hence also \EQUATION \eqref{eqn:pf:lemma:pf:lemma:concentration-ineq:noisy:f1,f2:1}---will hold.
We will now show that this nonpositivity
%of the sum of the second and third integrals is indeed true.
indeed occurs.
Note the following property of an expression related to the integrand:
%|>>|===========================================================================================|>>|
\begin{gather*}
  ( \zX-\muX[1] )
  e^{-\frac{1}{2} \zX^{2}}
  \pExprFn( \zX )
  \leq 0
  ,\quad
  \zX \in [0,\muX[1]]
  ,\\
  ( \zX-\muX[1] )
  e^{-\frac{1}{2} \zX^{2}}
  \pExprFn( \zX )
  \geq 0
  ,\quad
  \zX \in [\muX[1],\infty)
,\end{gather*}
%|<<|===========================================================================================|<<|
which implies that
%|>>|===========================================================================================|>>|
\begin{gather}
\label{eqn:pf:lemma:pf:lemma:concentration-ineq:noisy:f1,f2:10}
  - ( \zX-\muX[1] )
  e^{-\frac{1}{2} \zX^{2}}
  \pExprFn( \zX )
  =
  | ( \zX-\muX[1] )
  e^{-\frac{1}{2} \zX^{2}}
  \pExprFn( \zX ) |
  ,\quad
  \zX \in [0,\muX[1]]
  ,\\
\label{eqn:pf:lemma:pf:lemma:concentration-ineq:noisy:f1,f2:11}
  ( \zX-\muX[1] )
  e^{-\frac{1}{2} \zX^{2}}
  \pExprFn( \zX )
  =
  | ( \zX-\muX[1] )
  e^{-\frac{1}{2} \zX^{2}}
  \pExprFn( \zX ) |
  ,\quad
  \zX \in [\muX[1],\infty)
.\end{gather}
%|<<|===========================================================================================|<<|
Then, for the second integral in \eqref{eqn:pf:lemma:pf:lemma:concentration-ineq:noisy:f1,f2:5}, observe:
%|>>|===========================================================================================|>>|
\begin{align*}
  \int_{\zX=0}^{\zX=\muX[1]}
  ( \zX-\muX[1] )
  e^{-\frac{1}{2} \zX^{2}}
  \pExprFn( \zX+\wX )
  d\zX
  % &=
  % \int_{\zX=0}^{\zX=\muX[1]}
  % ( \zX-\muX[1] )
  % e^{-\frac{1}{2} \zX^{2}}
  % \pExprFn( \zX )
  % \frac{\pExprFn( \zX+\wX )}{\pExprFn( \zX )}
  % d\zX
  % \\
  % &=
  % \int_{\zX=0}^{\zX=\muX[1]}
  % ( -( \zX-\muX[1] )
  % e^{-\frac{1}{2} \zX^{2}}
  % \pExprFn( \zX ) )
  % \left( -\frac{\pExprFn( \zX+\wX )}{\pExprFn( \zX )} \right)
  % d\zX
  % \\
  &=
  \int_{\zX=0}^{\zX=\muX[1]}
  | ( \zX-\muX[1] )
  e^{-\frac{1}{2} \zX^{2}}
  \pExprFn( \zX ) |
  \left( -\frac{\pExprFn( \zX+\wX )}{\pExprFn( \zX )} \right)
  d\zX
  \\
  &\dCmt{by \EQUATION \eqref{eqn:pf:lemma:pf:lemma:concentration-ineq:noisy:f1,f2:10}}
  \\
  &\leq
  \int_{\zX=0}^{\zX=\muX[1]}
  | ( \zX-\muX[1] ) )
  e^{-\frac{1}{2} \zX^{2}}
  \pExprFn( \zX ) |
  \left( -\frac{\pExprFn( \muX[1]+\wX )}{\pExprFn( \muX[1] )} \right)
  d\zX
  \\
%  &\dCmt{by \CONDITION \ref{condition:assumption:p:ii} of \ASSUMPTION \ref{assumption:p}}
%  \\
%  &\dCmtIndent \text{and \REMARK \ref{remark:condition:assumption:p:ii}, and because \(  \zX \leq \muX[1]  \)}
%  \\
%  &\dCmtIndent \text{for all \(  \zX \in [0, \muX[1]]  \)}
  &\dCmt{by \CONDITION \ref{condition:assumption:p:ii} of \ASSUMPTION \ref{assumption:p},}
  \\
  &\dCmtIndent \text{and because \(  \zX \leq \muX[1]  \) for all \(  \zX \in [0, \muX[1]]  \)}
  % \\
  % &=
  % \int_{\zX=0}^{\zX=\muX[1]}
  % ( - ( \zX-\muX[1] )
  % e^{-\frac{1}{2} \zX^{2}}
  % \pExprFn( \zX ) )
  % \left( -\frac{\pExprFn( \muX[1]+\wX )}{\pExprFn( \muX[1] )} \right)
  % d\zX
  % \\
  % &\dCmt{by \EQUATION \eqref{eqn:pf:lemma:pf:lemma:concentration-ineq:noisy:f1,f2:10}}
  % \\
  % &=
  % \int_{\zX=0}^{\zX=\muX[1]}
  % ( \zX-\muX[1] )
  % e^{-\frac{1}{2} \zX^{2}}
  % \pExprFn( \zX )
  % \frac{\pExprFn( \muX[1]+\wX )}{\pExprFn( \muX[1] )}
  % d\zX
  \\
  &=
  \frac{\pExprFn( \muX[1]+\wX )}{\pExprFn( \muX[1] )}
  \int_{\zX=0}^{\zX=\muX[1]}
  ( \zX-\muX[1] )
  e^{-\frac{1}{2} \zX^{2}}
  \pExprFn( \zX )
  d\zX
\TagEqn{\label{eqn:pf:lemma:pf:lemma:concentration-ineq:noisy:f1,f2:6}}
  .\\
  &\dCmt{by \EQUATION \eqref{eqn:pf:lemma:pf:lemma:concentration-ineq:noisy:f1,f2:10}}
\end{align*}
%|<<|===========================================================================================|<<|
Similarly, for the third integral from \eqref{eqn:pf:lemma:pf:lemma:concentration-ineq:noisy:f1,f2:5}, observe:
%|>>|===========================================================================================|>>|
\begin{align*}
  \int_{\zX=\muX[1] }^{\zX=\infty}
  ( \zX-\muX[1] )
  e^{-\frac{1}{2} \zX^{2}}
  \pExprFn( \zX+\wX )
  d\zX
  % &=
  % \int_{\zX=\muX[1] }^{\zX=\infty}
  % ( \zX-\muX[1] )
  % e^{-\frac{1}{2} \zX^{2}}
  % \pExprFn( \zX )
  % \frac{\pExprFn( \zX+\wX )}{\pExprFn( \zX )}
  % d\zX
  % \\
  &=
  \int_{\zX=\muX[1] }^{\zX=\infty}
  | ( \zX-\muX[1] )
  e^{-\frac{1}{2} \zX^{2}}
  \pExprFn( \zX ) |
  \frac{\pExprFn( \zX+\wX )}{\pExprFn( \zX )}
  d\zX
  \\
  &\dCmt{by \EQUATION \eqref{eqn:pf:lemma:pf:lemma:concentration-ineq:noisy:f1,f2:11}}
  \\
  &\leq
  \int_{\zX=\muX[1] }^{\zX=\infty}
  | ( \zX-\muX[1] )
  e^{-\frac{1}{2} \zX^{2}}
  \pExprFn( \zX ) |
  \frac{\pExprFn( \muX[1]+\wX )}{\pExprFn( \muX[1] )}
  d\zX
  \\
%  &\dCmt{by \CONDITION \ref{condition:assumption:p:ii} of \ASSUMPTION \ref{assumption:p}}
%  \\
%  &\dCmtIndent \text{and \REMARK \ref{remark:condition:assumption:p:ii}, and because \(  \zX \geq \muX[1]  \)}
%  \\
%  &\dCmtIndent \text{for all \(  \zX \in [\muX[1], \infty)  \)}
  &\dCmt{by \CONDITION \ref{condition:assumption:p:ii} of \ASSUMPTION \ref{assumption:p},}
  \\
  &\dCmtIndent \text{and because \(  \zX \geq \muX[1]  \) for all \(  \zX \in [\muX[1], \infty)  \)}
  % \\
  % &=
  % \int_{\zX=\muX[1] }^{\zX=\infty}
  % ( \zX-\muX[1] )
  % e^{-\frac{1}{2} \zX^{2}}
  % \pExprFn( \zX )
  % \frac{\pExprFn( \muX[1]+\wX )}{\pExprFn( \muX[1] )}
  % d\zX
  % \\
  % &\dCmt{by \EQUATION \eqref{eqn:pf:lemma:pf:lemma:concentration-ineq:noisy:f1,f2:11}}
  \\
  &=
  \frac{\pExprFn( \muX[1]+\wX )}{\pExprFn( \muX[1] )}
  \int_{\zX=\muX[1] }^{\zX=\infty}
  ( \zX-\muX[1] )
  e^{-\frac{1}{2} \zX^{2}}
  \pExprFn( \zX )
  d\zX
\TagEqn{\label{eqn:pf:lemma:pf:lemma:concentration-ineq:noisy:f1,f2:7}}
  .\\
  &\dCmt{by \EQUATION \eqref{eqn:pf:lemma:pf:lemma:concentration-ineq:noisy:f1,f2:11}}
\end{align*}
%|<<|===========================================================================================|<<|
Then, the sum of the two integrals is bounded from above as follows:
%|>>|===========================================================================================|>>|
\begin{align*}
  &
  \int_{\zX=0}^{\zX=\muX[1]}
  ( \zX-\muX[1] )
  e^{-\frac{1}{2} \zX^{2}}
  \pExprFn( \zX+\wX )
  d\zX
  +
  \int_{\zX=\muX[1] }^{\zX=\infty}
  ( \zX-\muX[1] )
  e^{-\frac{1}{2} \zX^{2}}
  \pExprFn( \zX+\wX )
  d\zX
  \\
  &\AlignIndent\leq
  \frac{\pExprFn( \muX[1]+\wX )}{\pExprFn( \muX[1] )}
  \int_{\zX=0}^{\zX=\muX[1]}
  ( \zX-\muX[1] )
  e^{-\frac{1}{2} \zX^{2}}
  \pExprFn( \zX )
  d\zX
  +
  \frac{\pExprFn( \muX[1]+\wX )}{\pExprFn( \muX[1] )}
  \int_{\zX=\muX[1] }^{\zX=\infty}
  ( \zX-\muX[1] )
  e^{-\frac{1}{2} \zX^{2}}
  \pExprFn( \zX )
  d\zX
  \\
  &\AlignIndent\dCmt{by \EQUATIONS \eqref{eqn:pf:lemma:pf:lemma:concentration-ineq:noisy:f1,f2:6} and \eqref{eqn:pf:lemma:pf:lemma:concentration-ineq:noisy:f1,f2:7}}
  % \\
  % &\AlignIndent=
  % \frac{\pExprFn( \muX[1]+\wX )}{\pExprFn( \muX[1] )}
  % \left(
  %   \int_{\zX=0}^{\zX=\muX[1]}
  %   ( \zX-\muX[1] )
  %   e^{-\frac{1}{2} \zX^{2}}
  %   \pExprFn( \zX )
  %   d\zX
  %   +
  %   \int_{\zX=\muX[1] }^{\zX=\infty}
  %   ( \zX-\muX[1] )
  %   e^{-\frac{1}{2} \zX^{2}}
  %   \pExprFn( \zX )
  %   d\zX
  % \right)
  % \\
  % &\AlignIndent=
  % \frac{\pExprFn( \muX[1]+\wX )}{\pExprFn( \muX[1] )}
  % \int_{\zX=0}^{\zX=\infty}
  % ( \zX-\muX[1] )
  % e^{-\frac{1}{2} \zX^{2}}
  % \pExprFn( \zX )
  % d\zX
  \\
  &\AlignIndent=
  \frac{\pExprFn( \muX[1]+\wX )}{\pExprFn( \muX[1] )}
  \left(
    \int_{\zX=0}^{\zX=\infty}
    \zX
    e^{-\frac{1}{2} \zX^{2}}
    \pExprFn( \zX )
    d\zX
    -
    \muX[1]
    \int_{\zX=0}^{\zX=\infty}
    e^{-\frac{1}{2} \zX^{2}}
    \pExprFn( \zX )
    d\zX
  \right)
  \\
  &\AlignIndent=
  \frac{\pExprFn( \muX[1]+\wX )}{\pExprFn( \muX[1] )}
  \left(
    \left( 1 - \sqrt{\frac{\pi}{2}} \gammaX \right)
    -
    \frac{\sqrt{\hfrac{2}{\pi}} - \gammaX}{2\alphaX}
    \sqrt{2\pi} \alphaX
  \right)
  \\
  &\AlignIndent\dCmt{by the definitions of \(  \alphaX, \gammaX  \) in \EQUATIONS \eqref{eqn:notations:alpha:def} and \eqref{eqn:notations:gamma:def}, respectively,}
  \\
  &\AlignIndent\dCmtIndent\text{and because \(  {\textstyle \muX[1] = \frac{\sqrt{\hfrac{2}{\pi}} - \gammaX}{2\alphaX}}  \) as in \EQUATION \eqref{eqn:pf:lemma:concentration-ineq:noisy:7}}
  \\
  % &\AlignIndent=
  % \frac{\pExprFn( \muX[1]+\wX )}{\pExprFn( \muX[1] )}
  % \left( \left( 1 - \sqrt{\frac{\pi}{2}} \gammaX \right) - \left( 1 - \sqrt{\frac{\pi}{2}} \gammaX \right) \right)
  % \\
  &\AlignIndent=
  0
\TagEqn{\label{eqn:pf:lemma:pf:lemma:concentration-ineq:noisy:f1,f2:8}}
.\end{align*}
%|<<|===========================================================================================|<<|
Substituting \EQUATIONS \eqref{eqn:pf:lemma:pf:lemma:concentration-ineq:noisy:f1,f2:9} and \eqref{eqn:pf:lemma:pf:lemma:concentration-ineq:noisy:f1,f2:8} into \EQUATION \eqref{eqn:pf:lemma:pf:lemma:concentration-ineq:noisy:f1,f2:5}, it follows that for \(  \sX > 0  \),
%|>>|===========================================================================================|>>|
\begin{align*}
  &\negphantom{\AlignSp}
  \int_{\zX=0}^{\zX=\infty}
  ( \zX-\wX-\muX[1] )
  e^{-\frac{1}{2} ( \zX-\wX )^{2}}
  \pExprFn( \zX )
  d\zX
  \\
  &=
  \int_{\zX=-\wX}^{\zX=0}
  ( \zX-\muX[1] )
  e^{-\frac{1}{2} \zX^{2}}
  \pExprFn( \zX+\wX )
  d\zX
  +
  \int_{\zX=0}^{\zX=\muX[1]}
  ( \zX-\muX[1] )
  e^{-\frac{1}{2} \zX^{2}}
  \pExprFn( \zX+\wX )
  d\zX
  \\
  &\AlignSp+
  \int_{\zX=\muX[1] }^{\zX=\infty}
  ( \zX-\muX[1] )
  e^{-\frac{1}{2} \zX^{2}}
  \pExprFn( \zX+\wX )
  d\zX
  \\
  &\dCmt{by \EQUATION \eqref{eqn:pf:lemma:pf:lemma:concentration-ineq:noisy:f1,f2:5}}
  \\
  &<
  \int_{\zX=0}^{\zX=\muX[1]}
  ( \zX-\muX[1] )
  e^{-\frac{1}{2} \zX^{2}}
  \pExprFn( \zX+\wX )
  d\zX
  +
  \int_{\zX=\muX[1] }^{\zX=\infty}
  ( \zX-\muX[1] )
  e^{-\frac{1}{2} \zX^{2}}
  \pExprFn( \zX+\wX )
  \\
  &\dCmt{by \EQUATION \eqref{eqn:pf:lemma:pf:lemma:concentration-ineq:noisy:f1,f2:9}}
  \\
  &\leq
  0
  .\\
  &\dCmt{by \EQUATION \eqref{eqn:pf:lemma:pf:lemma:concentration-ineq:noisy:f1,f2:8}}
\end{align*}
%|<<|===========================================================================================|<<|
In short, the above work has established that
%|>>|===========================================================================================|>>|
\begin{align*}
  \int_{\zX=0}^{\zX=\infty}
  ( \zX-\wX-\muX[1] )
  e^{-\frac{1}{2} ( \zX-\wX )^{2}}
  \pExprFn( \zX )
  d\zX
  < 0
\end{align*}
%|<<|===========================================================================================|<<|
when \(  \sX > 0  \), and that
%|>>|:::::::::::::::::::::::::::::::::::::::::::::::::::::::::::::::::::::::::::::::::::::::::::|>>|
\(  \frac{d}{d\sX} \fFnX( 0 ) = 0  \).
%|<<|:::::::::::::::::::::::::::::::::::::::::::::::::::::::::::::::::::::::::::::::::::::::::::|<<|
Therefore, by \EQUATIONS \eqref{eqn:pf:lemma:pf:lemma:concentration-ineq:noisy:f1,f2:1} and \eqref{eqn:pf:lemma:pf:lemma:concentration-ineq:noisy:f1,f2:2}, as well as the earlier discussion, it happens that
%|>>|:::::::::::::::::::::::::::::::::::::::::::::::::::::::::::::::::::::::::::::::::::::::::::|>>|
\(  \frac{d}{d\sX} \fFnX( \sX ) \leq 0  \)
%|<<|:::::::::::::::::::::::::::::::::::::::::::::::::::::::::::::::::::::::::::::::::::::::::::|<<|
for all
%|>>|:::::::::::::::::::::::::::::::::::::::::::::::::::::::::::::::::::::::::::::::::::::::::::|>>|
\(  \sX \geq 0  \).
%|<<|:::::::::::::::::::::::::::::::::::::::::::::::::::::::::::::::::::::::::::::::::::::::::::|<<|
By basic calculus, this implies that
%|>>|===========================================================================================|>>|
\begin{gather*}
  \sup_{\sX \geq 0} \fFnX( \sX ) = \fFnX( 0 )
,\end{gather*}
%|<<|===========================================================================================|<<|
verifying \EQUATION \eqref{eqn:pf:lemma:concentration-ineq:noisy:f1:ub}.
%
%%%%%%%%%%%%%%%%%%%%%%%%%%%%%%%%%%%%%%%%%%%%%%%%%%%%%%%%%%%%%%%%%%%%%%%%%%%%%%%%%%%%%%%%%%%%%%%%%%%%
\paragraph{Verification of \EQUATION \eqref{eqn:pf:lemma:concentration-ineq:noisy:f2:ub}} %%%%%%%%%%
%%%%%%%%%%%%%%%%%%%%%%%%%%%%%%%%%%%%%%%%%%%%%%%%%%%%%%%%%%%%%%%%%%%%%%%%%%%%%%%%%%%%%%%%%%%%%%%%%%%%
%
\EQUATION \eqref{eqn:pf:lemma:concentration-ineq:noisy:f2:ub} can be derived through an analogous approach.
As such, most of the analysis to upper bound \(  \fFnXX  \) falls onto showing that
%|>>|:::::::::::::::::::::::::::::::::::::::::::::::::::::::::::::::::::::::::::::::::::::::::::|>>|
\(  \frac{d}{d\sX} \fFnXX( \sX ) \leq 0  \)
%|<<|:::::::::::::::::::::::::::::::::::::::::::::::::::::::::::::::::::::::::::::::::::::::::::|<<|
for all \(  \sX \geq 0  \), from which it will directly follow that
%|>>|:::::::::::::::::::::::::::::::::::::::::::::::::::::::::::::::::::::::::::::::::::::::::::|>>|
\(  \fFnXX( \sX ) \leq \fFnXX( 0 )  \)
%|<<|:::::::::::::::::::::::::::::::::::::::::::::::::::::::::::::::::::::::::::::::::::::::::::|<<|
for \(  \sX \geq 0  \).
The derivative of \(  \fFnXX  \) with respect to \(  \sX  \) is given by
%|>>|===========================================================================================|>>|
\begin{gather*}
  \frac{d}{d\sX} \fFnXX( \sX )
  =
  e^{\sX \muX[1]}
  \frac{1}{\sqrt{2\pi}}
  \int_{\zX=0}^{\zX=\infty}
  ( \muX[1]-\zX-\sX )
  e^{-\frac{1}{2} ( \zX+\sX )^{2}}
  \pExprFn( \zX )
  d\zX
.\end{gather*}
%|<<|===========================================================================================|<<|
At \(  \sX=0  \), this evaluates to
%|>>|===========================================================================================|>>|
\begin{align*}
  \frac{d}{d\sX} \fFnXX( 0 )
  &=
  \frac{1}{\sqrt{2\pi}}
  \int_{\zX=0}^{\zX=\infty}
  ( \muX[1]-\zX )
  e^{-\frac{1}{2} \zX^{2}}
  \pExprFn( \zX )
  d\zX
  \\
  &=
  \muX[1]
  \frac{1}{\sqrt{2\pi}}
  \int_{\zX=0}^{\zX=\infty}
  e^{-\frac{1}{2} \zX^{2}}
  \pExprFn( \zX )
  d\zX
  -
  \frac{1}{\sqrt{2\pi}}
  \int_{\zX=0}^{\zX=\infty}
  \zX
  e^{-\frac{1}{2} \zX^{2}}
  \pExprFn( \zX )
  d\zX
  % \\
  % &\dCmt{by the linearity of integration}
  \\
  &=
  \frac{\sqrt{\hfrac{2}{\pi}} - \gammaX}{2\alphaX} \alphaX
  -
  \frac{\sqrt{\hfrac{2}{\pi}} - \gammaX}{2}
  \\
  &\dCmt{by the definitions of \(  \alphaX, \gammaX  \) in \EQUATIONS \eqref{eqn:notations:alpha:def} and \eqref{eqn:notations:gamma:def}, respectively,}
  \\
  &\dCmtIndent\text{and an earlier remark in \EQUATION \eqref{eqn:pf:lemma:concentration-ineq:noisy:7} that \(  {\textstyle \muX[1] = \frac{\sqrt{\hfrac{2}{\pi}} - \gammaX}{2\alphaX}}  \)}
  \\
  % &=
  % \frac{\sqrt{\hfrac{2}{\pi}} - \gammaX}{2}
  % -
  % \frac{\sqrt{\hfrac{2}{\pi}} - \gammaX}{2}
  % \\
  &=
  0
\TagEqn{\label{eqn:pf:lemma:pf:lemma:concentration-ineq:noisy:f1,f2:f2:7}}
,\end{align*}
%|<<|===========================================================================================|<<|
which verifies the desired nonpositivity of \(  \frac{d}{d\sX} \fFnXX  \) in the case when \(  \sX=0  \), and which further implies that
%|>>|===========================================================================================|>>|
\begin{gather}
\label{eqn:pf:lemma:pf:lemma:concentration-ineq:noisy:f1,f2:f2:1}
  \int_{\zX=0}^{\zX=\infty}
  ( \muX[1]-\zX )
  e^{-\frac{1}{2} \zX^{2}}
  \pExprFn( \zX )
  d\zX
  = 0
.\end{gather}
%|<<|===========================================================================================|<<|
%
%%%%%%%%%%%%%%%%%%%%%%%%%%%%%%%%%%%%%%%%%%%%%%%%%%%%%%%%%%%%%%%%%%%%%%%%%%%%%%%%%%%%%%%%%%%%%%%%%%%%
\par %%%%%%%%%%%%%%%%%%%%%%%%%%%%%%%%%%%%%%%%%%%%%%%%%%%%%%%%%%%%%%%%%%%%%%%%%%%%%%%%%%%%%%%%%%%%%%%
%%%%%%%%%%%%%%%%%%%%%%%%%%%%%%%%%%%%%%%%%%%%%%%%%%%%%%%%%%%%%%%%%%%%%%%%%%%%%%%%%%%%%%%%%%%%%%%%%%%%
%
On the other hand, towards the case whee \(  \sX > 0  \),
%(and in particular, for \(  \wX > 0  \)),
note the following biconditional statement for \(  \wX > 0  \):
%|>>|===========================================================================================|>>|
\begin{gather}
\label{eqn:pf:lemma:pf:lemma:concentration-ineq:noisy:f1,f2:f2:2}
  \frac{d}{d\sX} \fFnXX( \wX )
  =
  e^{\wX \muX[1]}
  \frac{1}{\sqrt{2\pi}}
  \int_{\zX=0}^{\zX=\infty}
  ( \muX[1]-\zX-\wX )
  e^{-\frac{1}{2} ( \zX+\wX )^{2}}
  \pExprFn( \zX )
  d\zX
  < 0
\end{gather}
%|<<|===========================================================================================|<<|
if and only if
%|>>|===========================================================================================|>>|
\begin{gather}
\label{eqn:pf:lemma:pf:lemma:concentration-ineq:noisy:f1,f2:f2:3}
  \int_{\zX=0}^{\zX=\infty}
  ( \muX[1]-\zX-\wX )
  e^{-\frac{1}{2} ( \zX+\wX )^{2}}
  \pExprFn( \zX )
  d\zX
  < 0
.\end{gather}
%|<<|===========================================================================================|<<|
The next step is establishing the inequality in \eqref{eqn:pf:lemma:pf:lemma:concentration-ineq:noisy:f1,f2:f2:3} for \(  \sX > 0  \).
Throughout the upcoming analysis, take \(  \sX > 0  \) arbitrarily.
The interval of integration appearing on the \LHS of \EQUATION \eqref{eqn:pf:lemma:pf:lemma:concentration-ineq:noisy:f1,f2:f2:3} can be partitioned according to where the integrand takes positive verses nonpositive values:
%|>>|:::::::::::::::::::::::::::::::::::::::::::::::::::::::::::::::::::::::::::::::::::::::::::|>>|
\(  \zX \in [0, \muX[1]-\wX)  \) and \(  \zX \in [\muX[1]-\wX, \infty)  \),
%|<<|:::::::::::::::::::::::::::::::::::::::::::::::::::::::::::::::::::::::::::::::::::::::::::|<<|
respectively.
Hence, the integral in \eqref{eqn:pf:lemma:pf:lemma:concentration-ineq:noisy:f1,f2:f2:3} can be rewritten as:
%|>>|===========================================================================================|>>|
\begin{align*}
  &\negphantom{\AlignSp}
  \int_{\zX=0}^{\zX=\infty}
  ( \muX[1]-\zX-\wX )
  e^{-\frac{1}{2} ( \zX+\wX )^{2}}
  \pExprFn( \zX )
  d\zX
  \\
  &=
  \int_{\zX=0}^{\zX=\muX[1]-\wX}
  ( \muX[1]-\zX-\wX )
  e^{-\frac{1}{2} ( \zX+\wX )^{2}}
  \pExprFn( \zX )
  d\zX
  +
  \int_{\zX=\muX[1]-\wX}^{\zX=\infty}
  ( \muX[1]-\zX-\wX )
  e^{-\frac{1}{2} ( \zX+\wX )^{2}}
  \pExprFn( \zX )
  d\zX
\TagEqn{\label{eqn:pf:lemma:pf:lemma:concentration-ineq:noisy:f1,f2:f2:4}}
.\end{align*}
%|<<|===========================================================================================|<<|
The first of the two terms on the \RHS of \EQUATION \eqref{eqn:pf:lemma:pf:lemma:concentration-ineq:noisy:f1,f2:f2:4} is bounded from above by
%|>>|===========================================================================================|>>|
\begin{align*}
  &
  \int_{\zX=0}^{\zX=\muX[1]-\wX}
  ( \muX[1]-\zX-\wX )
  e^{-\frac{1}{2} ( \zX+\wX )^{2}}
  \pExprFn( \zX )
  d\zX
  \\
  &\AlignSp<
  \int_{\zX=0}^{\zX=\muX[1]-\wX}
  ( \muX[1]-\zX )
  e^{-\frac{1}{2} ( \zX+\wX )^{2}}
  \pExprFn( \zX )
  d\zX
  \\
  &\AlignSp<
  \int_{\zX=0}^{\zX=\muX[1]}
  ( \muX[1]-\zX )
  e^{-\frac{1}{2} ( \zX+\wX )^{2}}
  \pExprFn( \zX )
  d\zX
  \\
  &\AlignSp\dCmt{the integrand is nonnegative on the interval \(  \zX \in [0,\muX[1]]  \)}
  \\
  &\AlignSp<
  \int_{\zX=0}^{\zX=\muX[1]}
  ( \muX[1]-\zX )
  e^{-\frac{1}{2} \zX^{2}}
  \pExprFn( \zX )
  d\zX
\TagEqn{\label{eqn:pf:lemma:pf:lemma:concentration-ineq:noisy:f1,f2:f2:5}}
,\end{align*}
%|<<|===========================================================================================|<<|
while the second term on the \RHS of \eqref{eqn:pf:lemma:pf:lemma:concentration-ineq:noisy:f1,f2:f2:4} is upper bounded by
%|>>|===========================================================================================|>>|
\begin{align*}
  &\negphantom{\AlignSp}
  \int_{\zX=\muX[1]-\wX}^{\zX=\infty}
  ( \muX[1]-\zX-\wX )
  e^{-\frac{1}{2} ( \zX+\wX )^{2}}
  \pExprFn( \zX )
  d\zX
  \\
  &=
  \int_{\zX=\muX[1]}^{\zX=\infty}
  ( \muX[1]-\zX )
  e^{-\frac{1}{2} \zX^{2}}
  \pExprFn( \zX-\wX )
  d\zX
  % \\
  % &\dCmt{by a change of variables}
  \\
  % &=
  % \int_{\zX=\muX[1]}^{\zX=\infty}
  % ( -( \muX[1]-\zX )
  % e^{-\frac{1}{2} \zX^{2}} )
  % ( -\pExprFn( \zX-\wX ) )
  % d\zX
  % \\
  &=
  \int_{\zX=\muX[1]}^{\zX=\infty}
 |  ( \muX[1]-\zX )
  e^{-\frac{1}{2} \zX^{2}} |
  ( -\pExprFn( \zX-\wX ) )
  d\zX
  \\
  &\dCmt{since \(  {\textstyle -( \muX[1]-\zX ) e^{-\frac{1}{2} \zX^{2}} = | ( \muX[1]-\zX ) e^{-\frac{1}{2} \zX^{2}} |}  \) for \(  \zX \geq \muX[1]  \)}
  \\
  &\leq
  \int_{\zX=\muX[1]}^{\zX=\infty}
  | ( \muX[1]-\zX ) )
  e^{-\frac{1}{2} \zX^{2}} |
  ( -\pExprFn( \zX ) )
  d\zX
  \\
  &\dCmt{since \(  \pExprFn  \) is nonincreasing}
  % \\
  % &\dCmt{by an earlier observation, \(  \pExprFn  \) is nonincreasing,}
  % \\
  % &\dCmtIndent \text{so \(  -\pExprFn  \) is nondecreasing}
  % \\
  % &=
  % \int_{\zX=\muX[1]}^{\zX=\infty}
  % ( -( \muX[1]-\zX ) )
  % e^{-\frac{1}{2} \zX^{2}}
  % ( -\pExprFn( \zX ) )
  % d\zX
  % \\
  % &\dCmt{by an above remark}
  \\
  &=
  \int_{\zX=\muX[1]}^{\zX=\infty}
  ( \muX[1]-\zX )
  e^{-\frac{1}{2} \zX^{2}}
  \pExprFn( \zX )
  d\zX
\TagEqn{\label{eqn:pf:lemma:pf:lemma:concentration-ineq:noisy:f1,f2:f2:6}}
.\end{align*}
%|<<|===========================================================================================|<<|
Combining \EQUATIONS \eqref{eqn:pf:lemma:pf:lemma:concentration-ineq:noisy:f1,f2:f2:5} and \eqref{eqn:pf:lemma:pf:lemma:concentration-ineq:noisy:f1,f2:f2:6} into \EQUATION \eqref{eqn:pf:lemma:pf:lemma:concentration-ineq:noisy:f1,f2:f2:4} yields:
%|>>|===========================================================================================|>>|
\begin{align*}
  &\negphantom{\AlignSp}
  \int_{\zX=0}^{\zX=\infty}
  ( \muX[1]-\zX-\wX )
  e^{-\frac{1}{2} ( \zX+\wX )^{2}}
  \pExprFn( \zX )
  d\zX
  % \\
  % &=
  % \int_{\zX=0}^{\zX=\muX[1]-\wX}
  % ( \muX[1]-\zX-\wX )
  % e^{-\frac{1}{2} ( \zX+\wX )^{2}}
  % \pExprFn( \zX )
  % d\zX
  % +
  % \int_{\zX=\muX[1]-\wX}^{\zX=\infty}
  % ( \muX[1]-\zX-\wX )
  % e^{-\frac{1}{2} ( \zX+\wX )^{2}}
  % \pExprFn( \zX )
  % d\zX
  % \\
  % &\dCmt{by the linearity of integrals}
  \\
  &<
  \int_{\zX=0}^{\zX=\muX[1]}
  ( \muX[1]-\zX )
  e^{-\frac{1}{2} \zX^{2}}
  \pExprFn( \zX )
  d\zX
  +
  \int_{\zX=\muX[1]}^{\zX=\infty}
  ( \muX[1]-\zX )
  e^{-\frac{1}{2} \zX^{2}}
  \pExprFn( \zX )
  d\zX
  \\
  &\dCmt{by \EQUATIONS \eqref{eqn:pf:lemma:pf:lemma:concentration-ineq:noisy:f1,f2:f2:5} and \eqref{eqn:pf:lemma:pf:lemma:concentration-ineq:noisy:f1,f2:f2:6}}
  \\
  &=
  \int_{\zX=0}^{\zX=\infty}
  ( \muX[1]-\zX )
  e^{-\frac{1}{2} \zX^{2}}
  \pExprFn( \zX )
  d\zX
  % \\
  % &\dCmt{by the linearity of integration}
  \\
  &=
  0
  .\\
  &\dCmt{by \EQUATION \eqref{eqn:pf:lemma:pf:lemma:concentration-ineq:noisy:f1,f2:f2:1}}
\end{align*}
%|<<|===========================================================================================|<<|
Thus, for every \(  \sX > 0  \), \EQUATION \eqref{eqn:pf:lemma:pf:lemma:concentration-ineq:noisy:f1,f2:f2:3} holds, implying that \EQUATION \eqref{eqn:pf:lemma:pf:lemma:concentration-ineq:noisy:f1,f2:f2:2} is also true due to the biconditional relationship between this pair of equations (which was stated earlier in the proof)---that is, it indeed happens that
%|>>|:::::::::::::::::::::::::::::::::::::::::::::::::::::::::::::::::::::::::::::::::::::::::::|>>|
\(  \frac{d}{d\sX} \fFnXX( \sX ) < 0  \)
%|<<|:::::::::::::::::::::::::::::::::::::::::::::::::::::::::::::::::::::::::::::::::::::::::::|<<|
for \(  \sX > 0  \).
Moreover, the derivation in \eqref{eqn:pf:lemma:pf:lemma:concentration-ineq:noisy:f1,f2:f2:7} showed that
%|>>|:::::::::::::::::::::::::::::::::::::::::::::::::::::::::::::::::::::::::::::::::::::::::::|>>|
\(  \frac{d}{d\sX} \fFnXX( 0 ) = 0  \).
%|<<|:::::::::::::::::::::::::::::::::::::::::::::::::::::::::::::::::::::::::::::::::::::::::::|<<|
It follows that
%|>>|:::::::::::::::::::::::::::::::::::::::::::::::::::::::::::::::::::::::::::::::::::::::::::|>>|
\(  \frac{d}{d\sX} \fFnXX( \sX ) \leq 0  \)
%|<<|:::::::::::::::::::::::::::::::::::::::::::::::::::::::::::::::::::::::::::::::::::::::::::|<<|
for all \(  \sX \geq 0  \), and therefore, due to standard facts about derivatives, \EQUATION \eqref{eqn:pf:lemma:concentration-ineq:noisy:f2:ub} holds:
%|>>|===========================================================================================|>>|
\begin{gather*}
  \sup_{\sX \geq 0} \fFnXX( \sX ) = \fFnXX( 0 )
,\end{gather*}
%|<<|===========================================================================================|<<|
concluding the proof of \LEMMA \ref{lemma:pf:lemma:concentration-ineq:noisy:f1,f2}.
%%%%%%%%%%%%%%%%%%%%%%%%%%%%%%%%%%%%%%%%%%%%%%%%%%%%%%%%%%%%%%%%%%%%%%%%%%%%%%%%%%%%%%%%%%%%%%%%%%%%
\renewcommand{\wX}{w} %%%%%%%%%%%%%%%%%%%%%%%%%%%%%%%%%%%%%%%%%%%%%%%%%%%%%%%%%%%%%%%%%%%%%%%%%%%%%%
%%%%%%%%%%%%%%%%%%%%%%%%%%%%%%%%%%%%%%%%%%%%%%%%%%%%%%%%%%%%%%%%%%%%%%%%%%%%%%%%%%%%%%%%%%%%%%%%%%%%
\end{proof}
%|<<|~~~~~~~~~~~~~~~~~~~~~~~~~~~~~~~~~~~~~~~~~~~~~~~~~~~~~~~~~~~~~~~~~~~~~~~~~~~~~~~~~~~~~~~~~~~|<<|
%|<<|~~~~~~~~~~~~~~~~~~~~~~~~~~~~~~~~~~~~~~~~~~~~~~~~~~~~~~~~~~~~~~~~~~~~~~~~~~~~~~~~~~~~~~~~~~~|<<|
%|<<|~~~~~~~~~~~~~~~~~~~~~~~~~~~~~~~~~~~~~~~~~~~~~~~~~~~~~~~~~~~~~~~~~~~~~~~~~~~~~~~~~~~~~~~~~~~|<<|