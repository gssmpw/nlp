\subsection{Overview of the Proof of \THEOREM \ref{thm:main-technical:sparse}}
\label{outline:main-technical-result|outline-of-pf}

\checkoffbutmayberecheck%
%
The proof of the main technical theorem, \THEOREM \ref{thm:main-technical:sparse}, takes up the majority of the work in this manuscript.
This section provides an overview of the proof.
The proof in full is located in \APPENDIX \ref{outline:pf-main-technical-result} with some auxiliary results therein proved in \APPENDIX \ref{outline:concentration-ineq}.
Before outlining the arguments, recall the  definitions of Equations \eqref{eqn:notations:h:def}--\eqref{eqn:notations:hfJ:def} from \SECTION \ref{outline:main-result|outline-of-pf}. % for
%\(  \Vec{u}, \Vec{v} \in \R^{\n}  \) and \(  \JCoords \subseteq [\n]  \).
%are recalled for convenience:
%for \(  \Vec{u}, \Vec{v} \in \R^{\n}  \) and \(  \JCoords \subseteq [\n]  \), let
%|>>|===========================================================================================|>>|
% \begin{gather*}
%   %\label{eqn:notations:h:def}
%   \hFn( \Vec{u}, \Vec{v} )
%   \defeq
%   \frac{\sqrt{2\pi}}{\m}
%   \sep
%   \CovM^{\T}
%   \sep
%   \frac{1}{2}
%   \left( \Sign( \CovM \Vec{u} ) - \Sign( \CovM \Vec{v} ) \right)
%   ,\\ %\label{eqn:notations:hJ:def}
%   \hFn[\JCoords]( \Vec{u}, \Vec{v} )
%   \defeq
%   \ThresholdSet{\Supp( \Vec{u} ) \cup \Supp( \Vec{v} ) \cup \JCoords}(
%   \hFn( \Vec{u}, \Vec{v} )
% %  \frac{\sqrt{2\pi}}{\m}
% %  \sep
% %  \CovM^{\T}
% %  \sep
% %  \frac{1}{2}
% %  \left( \Sign( \CovM \Vec{u} ) - \Sign( \CovM \Vec{v} ) \right)
%   )
%   ,\\ %\label{eqn:notations:hf:def}
%   \hfFn( \Vec{u}, \Vec{v} )
%   \defeq
%   \frac{\sqrt{2\pi}}{\m}
%   \sep
%   \CovM^{\T}
%   \sep
%   \frac{1}{2}
%   \left( \fFn( \CovM \Vec{u} ) - \Sign( \CovM \Vec{v} ) \right)
%   ,\\ %\label{eqn:notations:hfJ:def}
%   \hfFn[\JCoords]( \Vec{u}, \Vec{v} )
%   \defeq
%   \ThresholdSet{\Supp( \Vec{u} ) \cup \Supp( \Vec{v} ) \cup \JCoords}(
%   \hfFn( \Vec{u}, \Vec{v} )
% %  \frac{\sqrt{2\pi}}{\m}
% %  \sep
% %  \CovM^{\T}
% %  \sep
% %  \frac{1}{2}
% %  \left( \fFn( \CovM \Vec{u} ) - \Sign( \CovM \Vec{v} ) \right)
%   )
% ,\end{gather*}
%|<<|===========================================================================================|<<|
Additionally, define the following related notations for \(  \Vec{u}, \Vec{v} \in \R^{\n}  \) and \(  \JCoords \subseteq [\n]  \):
%|>>|===========================================================================================|>>|
\begin{gather}
  \label{eqn:notations:g:def}
  \gFn( \Vec{u}, \Vec{v} )
  \defeq
  \hFn( \Vec{u}, \Vec{v} )
  -
  \left\langle \hFn( \Vec{u}, \Vec{v} ), \frac{\Vec{u}-\Vec{v}}{\| \Vec{u}-\Vec{v} \|_{2}} \right\rangle \frac{\Vec{u}-\Vec{v}}{\| \Vec{u}-\Vec{v} \|_{2}}
  -
  \left\langle \hFn( \Vec{u}, \Vec{v} ), \frac{\Vec{u}+\Vec{v}}{\| \Vec{u}+\Vec{v} \|_{2}} \right\rangle \frac{\Vec{u}+\Vec{v}}{\| \Vec{u}+\Vec{v} \|_{2}}
  ,\\ \label{eqn:notations:gJ:def}
  \gFn[\JCoords]( \Vec{u}, \Vec{v} )
  \defeq
  \ThresholdSet{\Supp( \Vec{u} ) \cup \Supp( \Vec{v} ) \cup \JCoords}( \gFn( \Vec{u}, \Vec{v} ) )
  ,\\ \label{eqn:notations:gf:def}
  \gfFn( \Vec{u}, \Vec{u} )
  \defeq
  \hfFn( \Vec{u}, \Vec{u} ) - \left\langle \hfFn( \Vec{u}, \Vec{u} ), \Vec{u} \right\rangle \Vec{u}
  ,\\ \label{eqn:notations:gfJ:def}
  \gfFn[\JCoords]( \Vec{u}, \Vec{u} )
  \defeq
  \ThresholdSet{\Supp( \Vec{u} ) \cup \Supp( \Vec{v} ) \cup \JCoords}( \gfFn( \Vec{u}, \Vec{u} ) )
.\end{gather}
%|<<|===========================================================================================|<<|
Note that
%|>>|===========================================================================================|>>|
\begin{gather*}
  \gFn[\JCoords]( \Vec{u}, \Vec{v} )
  =
  \hFn[\JCoords]( \Vec{u}, \Vec{v} )
  -
  \left\langle \hFn[\JCoords]( \Vec{u}, \Vec{v} ), \frac{\Vec{u}-\Vec{v}}{\| \Vec{u}-\Vec{v} \|_{2}} \right\rangle \frac{\Vec{u}-\Vec{v}}{\| \Vec{u}-\Vec{v} \|_{2}}
  -
  \left\langle \hFn[\JCoords]( \Vec{u}, \Vec{v} ), \frac{\Vec{u}+\Vec{v}}{\| \Vec{u}+\Vec{v} \|_{2}} \right\rangle \frac{\Vec{u}+\Vec{v}}{\| \Vec{u}+\Vec{v} \|_{2}}
\end{gather*}
%|<<|===========================================================================================|<<|
and that
%|>>|===========================================================================================|>>|
\begin{gather*}
  \gfFn[\JCoords]( \Vec{u}, \Vec{u} )
  =
  \hfFn[\JCoords]( \Vec{u}, \Vec{u} )
  -
  \left\langle \hfFn[\JCoords]( \Vec{u}, \Vec{u} ), \Vec{u} \right\rangle \Vec{u}
.\end{gather*}
%|<<|===========================================================================================|<<|

%%%%%%%%%%%%%%%%%%%%%%%%%%%%%%%%%%%%%%%%%%%%%%%%%%%%%%%%%%%%%%%%%%%%%%%%%%%%%%%%%%%%%%%%%%%%%%%%%%%%
%%%%%%%%%%%%%%%%%%%%%%%%%%%%%%%%%%%%%%%%%%%%%%%%%%%%%%%%%%%%%%%%%%%%%%%%%%%%%%%%%%%%%%%%%%%%%%%%%%%%

\subsubsection{Key Steps of the Proof}
\label{outline:main-technical-result|outline-of-pf|outline}

The proof of \THEOREM \ref{thm:main-technical:sparse} are sketched as follows.
%
\checkoff%

%|>>|XXXXXXXXXXXXXXXXXXXXXXXXXXXXXXXXXXXXXXXXXXXXXXXXXXXXXXXXXXXXXXXXXXXXXXXXXXXXXXXXXXXXXXXXXXX|>>|
%|>>|XXXXXXXXXXXXXXXXXXXXXXXXXXXXXXXXXXXXXXXXXXXXXXXXXXXXXXXXXXXXXXXXXXXXXXXXXXXXXXXXXXXXXXXXXXX|>>|
%|>>|XXXXXXXXXXXXXXXXXXXXXXXXXXXXXXXXXXXXXXXXXXXXXXXXXXXXXXXXXXXXXXXXXXXXXXXXXXXXXXXXXXXXXXXXXXX|>>|
\begin{enumerate}
%XXXXXXXXXXXXXXXXXXXXXXXXXXXXXXXXXXXXXXXXXXXXXXXXXXXXXXXXXXXXXXXXXXXXXXXXXXXXXXXXXXXXXXXXXXXXXXXXXXX
\item \label{enum:outline-pf-main-technical:1}
Recall that the aim is to bound
%|>>|===========================================================================================|>>|
\begin{gather}
\label{eqn:enum:outline-pf-main-technical:1}
  \left\|
    \thetaStar
    -
    \frac
    {\thetaXX + \hfFn[\JCoords]( \thetaStar, \thetaXX )}
    {\| \thetaXX + \hfFn[\JCoords]( \thetaStar, \thetaXX ) \|_{2}}
  \right\|_{2}
\end{gather}
%|<<|===========================================================================================|<<|
from above with high probability uniformly for all \(  \thetaXX \in \ParamSpace  \) and all \(  \JCoords \subseteq [\n]  \), \(  | \JCoords | \leq \k  \).
%We begin by arbitrarily fixing \(  \thetaStar, \thetaXX \in \ParamSpace  \) and \(  \JCoords \subseteq [\n]  \), \(  | \JCoords | \leq \k  \), where \(  \thetaXX  \) and \(  \JCoords  \) will be varied later on to establish the uniform result.
%XXXXXXXXXXXXXXXXXXXXXXXXXXXXXXXXXXXXXXXXXXXXXXXXXXXXXXXXXXXXXXXXXXXXXXXXXXXXXXXXXXXXXXXXXXXXXXXXXXX
\item \label{enum:outline-pf-main-technical:2}
To obtain a uniform result, a \(  \tauX  \)-net, \(  \ParamCover \subset \ParamSpace  \), over the parameter space, \(  \ParamSpace  \), is constructed with a particular design, the details of which are left to the formal proof of \THEOREM \ref{thm:main-technical:sparse}.
For the purpose of this overview, its suffices to say that, crucially, the design of \(  \ParamCover  \) ensures that for each \(  \thetaXX \in \ParamSpace  \), there exists an element, \(  \thetaX \in \ParamCover  \), such that both
%|>>|:::::::::::::::::::::::::::::::::::::::::::::::::::::::::::::::::::::::::::::::::::::::::::|>>|
\(  \| \thetaX - \thetaXX \|_{2} \leq \tauX  \) and
\(  \Supp( \thetaX ) = \Supp( \thetaXX )  \).
%|<<|:::::::::::::::::::::::::::::::::::::::::::::::::::::::::::::::::::::::::::::::::::::::::::|<<|
This cover, \(  \ParamCover  \), will allow the establishment of a global result for points, \(  \thetaX \in \ParamCover  \), within it, which can subsequently be extended to arbitrary points, \(  \thetaXX \in \ParamSpace  \), in the entire parameter space via a local analysis.
%XXXXXXXXXXXXXXXXXXXXXXXXXXXXXXXXXXXXXXXXXXXXXXXXXXXXXXXXXXXXXXXXXXXXXXXXXXXXXXXXXXXXXXXXXXXXXXXXXXX
\item \label{enum:outline-pf-main-technical:3}
As another preliminary step, it will be shown that for any \(  \thetaXX \in \ParamSpace  \) and \(  \JCoords \subseteq [\n]  \),
%|>>|===========================================================================================|>>|
\begin{gather*}
  \frac
  {\E[ \thetaXX + \hfFn[\JCoords]( \thetaStar, \thetaXX ) ]}
  {\| \E[ \thetaXX + \hfFn[\JCoords]( \thetaStar, \thetaXX ) ] \|_{2}}
  =
  \thetaStar
.\end{gather*}
%|<<|===========================================================================================|<<|
In other words, the quantity in \eqref{eqn:enum:outline-pf-main-technical:1}---which we seek to bound---describes a notion of deviation of
%|>>|:::::::::::::::::::::::::::::::::::::::::::::::::::::::::::::::::::::::::::::::::::::::::::|>>|
\(  \thetaXX + \hfFn[\JCoords]( \thetaStar, \thetaXX )  \)
%|<<|:::::::::::::::::::::::::::::::::::::::::::::::::::::::::::::::::::::::::::::::::::::::::::|<<|
from its mean (after normalization):
%|>>|===========================================================================================|>>|
\begin{gather*}
  \left\|
    \thetaStar
    -
    \frac
    {\thetaXX + \hfFn[\JCoords]( \thetaStar, \thetaXX )}
    {\| \thetaXX + \hfFn[\JCoords]( \thetaStar, \thetaXX ) \|_{2}}
  \right\|_{2}
  =
  \left\|
    \frac
    {\thetaXX + \hfFn[\JCoords]( \thetaStar, \thetaXX )}
    {\| \thetaXX + \hfFn[\JCoords]( \thetaStar, \thetaXX ) \|_{2}}
    -
    \frac
    {\E[ \thetaXX + \hfFn[\JCoords]( \thetaStar, \thetaXX ) ]}
    {\| \E[ \thetaXX + \hfFn[\JCoords]( \thetaStar, \thetaXX ) ] \|_{2}}
  \right\|_{2}
.\end{gather*}
%|<<|===========================================================================================|<<|
In fact, this deviation turns out to roughly scale with the deviation of the random function \(  \hfFn[\JCoords]  \) around its mean:
%|>>|===========================================================================================|>>|
\begin{gather}
\label{eqn:enum:outline-pf-main-technical:1b}
%  \left\|
%    \thetaStar
%    -
%    \frac
%    {\thetaXX + \hfFn[\JCoords]( \thetaStar, \thetaXX )}
%    {\| \thetaXX + \hfFn[\JCoords]( \thetaStar, \thetaXX ) \|_{2}}
%  \right\|_{2}
%  =
  \left\|
    \frac
    {\thetaXX + \hfFn[\JCoords]( \thetaStar, \thetaXX )}
    {\| \thetaXX + \hfFn[\JCoords]( \thetaStar, \thetaXX ) \|_{2}}
    -
    \frac
    {\E[ \thetaXX + \hfFn[\JCoords]( \thetaStar, \thetaXX ) ]}
    {\| \E[ \thetaXX + \hfFn[\JCoords]( \thetaStar, \thetaXX ) ] \|_{2}}
  \right\|_{2}
  \propto
  \| \hfFn[\JCoords]( \thetaStar, \thetaXX ) - \E[ \hfFn[\JCoords]( \thetaStar, \thetaXX ) ] \|_{2}
.\end{gather}
%|<<|===========================================================================================|<<|
Analyzing (a decomposition of) the deviation of \(  \hfFn[\JCoords]  \) will be at the core of the proof.
%XXXXXXXXXXXXXXXXXXXXXXXXXXXXXXXXXXXXXXXXXXXXXXXXXXXXXXXXXXXXXXXXXXXXXXXXXXXXXXXXXXXXXXXXXXXXXXXXXXX
\item \label{enum:outline-pf-main-technical:4}
Letting
%|>>|:::::::::::::::::::::::::::::::::::::::::::::::::::::::::::::::::::::::::::::::::::::::::::|>>|
\(  \thetaStar, \thetaXX \in \ParamSpace  \)
%|<<|:::::::::::::::::::::::::::::::::::::::::::::::::::::::::::::::::::::::::::::::::::::::::::|<<|
be arbitrary, and using the observations in \STEP \ref{enum:outline-pf-main-technical:3}, the triangle inequality, algebraic manipulations, and other standard techniques, the expression in \eqref{eqn:enum:outline-pf-main-technical:1}
%is decomposed into (and bounded by)
is bounded by the sum of
three terms which will admit an easier analysis than directly handling \eqref{eqn:enum:outline-pf-main-technical:1}:
%|>>|===========================================================================================|>>|
\begin{subequations}
\label{eqn:enum:outline-pf-main-technical:2}
\begin{align}
  \left\| \thetaStar - \frac{\thetaXX+\hfFn[\JCoords]( \thetaStar, \thetaXX )}{\| \thetaXX+\hfFn[\JCoords]( \thetaStar, \thetaXX ) \|_{2}} \right\|_{2}
  &\leq
  \label{enum:outline-pf-main-technical:5:i}
  \frac
  {2 \| \hFn[\JCoords]( \thetaStar, \thetaX ) - \E[ \hFn[\JCoords]( \thetaStar, \thetaX ) ] \|_{2}}
  {\DENOM}
  \\ \label{enum:outline-pf-main-technical:5:iii}
  &\AlignSp+
  \frac
  {2 \| \hFn[\Supp( \thetaStar ) \cup \JCoords]( \thetaX, \thetaXX ) - \E[ \hFn[\Supp( \thetaStar ) \cup \JCoords]( \thetaX, \thetaXX ) ] \|_{2}}
  {\DENOM}
  \\ \label{enum:outline-pf-main-technical:5:ii}
  &\AlignSp+
  \frac
  {2 \| \hfFn[\Supp( \thetaX ) \cup \JCoords]( \thetaStar, \thetaStar ) - \E[ \hfFn[\Supp( \thetaX ) \cup \JCoords]( \thetaStar, \thetaStar ) ] \|_{2}}
  {\DENOM}
%\TagEqn{\label{eqn:enum:outline-pf-main-technical:2}}
,\end{align}
\end{subequations}
%|<<|===========================================================================================|<<|
where \(  \JCoords \subseteq [\n]  \), \(  | \JCoords | \leq \k  \), is arbitrary,
and where
%|>>|:::::::::::::::::::::::::::::::::::::::::::::::::::::::::::::::::::::::::::::::::::::::::::|>>|
\(  \thetaX \in \ParamCover \setminus \Ball{\tauX}( \thetaStar )  \)
%|<<|:::::::::::::::::::::::::::::::::::::::::::::::::::::::::::::::::::::::::::::::::::::::::::|<<|
such that
%|>>|:::::::::::::::::::::::::::::::::::::::::::::::::::::::::::::::::::::::::::::::::::::::::::|>>|
\(  \| \thetaX - \thetaXX \|_{2} \leq 2\tauX  \) and
\(  \Supp( \thetaX ) \cup \JCoords = \Supp( \thetaXX ) \cup \JCoords  \)
%|<<|:::::::::::::::::::::::::::::::::::::::::::::::::::::::::::::::::::::::::::::::::::::::::::|<<|
(\see \LEMMA \ref{lemma:combine}).
Per the design of the \(  \tauX  \)-net, \(  \ParamCover \subset \ParamSpace  \), in \STEP \ref{enum:outline-pf-main-technical:2}, such a point \(  \thetaX \in \ParamCover  \) exists for any choice of \(  \thetaXX \in \ParamSpace  \).
%preserve some notion of deviations.
%XXXXXXXXXXXXXXXXXXXXXXXXXXXXXXXXXXXXXXXXXXXXXXXXXXXXXXXXXXXXXXXXXXXXXXXXXXXXXXXXXXXXXXXXXXXXXXXXXXX
\item \label{enum:outline-pf-main-technical:5}
The three terms on the \RHS of \EQUATION \eqref{eqn:enum:outline-pf-main-technical:2} can be viewed as bounding \eqref{eqn:enum:outline-pf-main-technical:1} by relating it (with appropriate scaling) to the deviation of \(  \hfFn  \) specified on the \RHS of \eqref{eqn:enum:outline-pf-main-technical:1b}, and then controlling the \RHS of \eqref{eqn:enum:outline-pf-main-technical:1b} by
%breaking it up into,
decomposing the deviation of \(  \hfFn  \) into three components of deviation,
in order:
%|>>|×××××××××××××××××××××××××××××××××××××××××××××××××××××××××××××××××××××××××××××××××××××××××××|>>|
%\Enum[{\label{enum:outline-pf-main-technical:5:i}}]{i}
\eqref{enum:outline-pf-main-technical:5:i},
a component handling points in the cover, \(  \ParamCover  \), over \(  \ParamSpace  \) that are sufficiently far from \(  \thetaStar  \)---a ``global'' result;
%×××××××××××××××××××××××××××××××××××××××××××××××××××××××××××××××××××××××××××××××××××××××××××××××××××
%\Enum[{\label{enum:outline-pf-main-technical:5:iii}}]{ii}
\eqref{enum:outline-pf-main-technical:5:iii},
a component reconciling the discrepancy between the original point, \(  \thetaXX \in \ParamSpace  \), which may be outside the cover, and a nearby neighbor in the cover, \(  \thetaX \in \ParamCover \setminus \Ball{\tauX}( \thetaStar )  \)---a ``local'' result; and
%×××××××××××××××××××××××××××××××××××××××××××××××××××××××××××××××××××××××××××××××××××××××××××××××××××
%\Enum[{\label{enum:outline-pf-main-technical:5:ii}}]{iii}
\eqref{enum:outline-pf-main-technical:5:ii},
a component handling the ``noise'' introduced into the GLM through the randomness of \(  \fFn  \).
%|<<|×××××××××××××××××××××××××××××××××××××××××××××××××××××××××××××××××××××××××××××××××××××××××××|<<|
%Hence, these three terms in \eqref{eqn:enum:outline-pf-main-technical:2} in effect decompose the original deviation captured by \eqref{eqn:enum:outline-pf-main-technical:1} into three components of deviation. %each itself capturing the deviation of particular a random vector.
%%The first component, \eqref{enum:outline-pf-main-technical:5:i}, captures the aforementioned ``global'' result, while the second component, \eqref{enum:outline-pf-main-technical:5:iii}, corresponds with the ``local'' result.
%This enumeration of the terms, \eqref{enum:outline-pf-main-technical:5:i}--\eqref{enum:outline-pf-main-technical:5:ii}, in \EQUATION \eqref{eqn:enum:outline-pf-main-technical:2} will be carried forward in the remaining overview.
%XXXXXXXXXXXXXXXXXXXXXXXXXXXXXXXXXXXXXXXXXXXXXXXXXXXXXXXXXXXXXXXXXXXXXXXXXXXXXXXXXXXXXXXXXXXXXXXXXXX
\item \label{enum:outline-pf-main-technical:6}
The technical work in this manuscript then lies largely with bounding the three terms on the \RHS of \EQUATION \eqref{eqn:enum:outline-pf-main-technical:2}.
While most of the details of this analysis are left to the formal proofs (\see \APPENDICES \ref{outline:pf-main-technical-result|pf-intermediate}--\ref{outline:concentration-ineq}), a few salient ideas in the approach are mentioned here.
%The bulk of the work is then to
%Using concentration inequalities, which are ,
%For each of the three terms on the \RHS of \EQUATION \eqref{eqn:enum:outline-pf-main-technical:2},
%XXXXXXXXXXXXXXXXXXXXXXXXXXXXXXXXXXXXXXXXXXXXXXXXXXXXXXXXXXXXXXXXXXXXXXXXXXXXXXXXXXXXXXXXXXXXXXXXXXX
\item \label{enum:outline-pf-main-technical:7}
For the three terms, \eqref{enum:outline-pf-main-technical:5:i}--\eqref{enum:outline-pf-main-technical:5:ii}, the (shared) denominator can be calculated directly.
%XXXXXXXXXXXXXXXXXXXXXXXXXXXXXXXXXXXXXXXXXXXXXXXXXXXXXXXXXXXXXXXXXXXXXXXXXXXXXXXXXXXXXXXXXXXXXXXXXXX
\item \label{enum:outline-pf-main-technical:8}
On the other hand, the numerators in \eqref{enum:outline-pf-main-technical:5:i}--\eqref{enum:outline-pf-main-technical:5:ii} are upper bounded with bounded probability
%, rather than exactly calculated,
through concentration inequalities derived with standard techniques.
To do so, each numerator is orthogonally decomposed into two to three components for which derivations of concentration inequalities are easier.
%for which concentration inequalities can be derived more easily.
Subsequently, for each numerator, the concentration inequalities for its associated components are combined via the triangle inequality.
Then, these are extended into uniform results by appropriate union bounds.
%XXXXXXXXXXXXXXXXXXXXXXXXXXXXXXXXXXXXXXXXXXXXXXXXXXXXXXXXXXXXXXXXXXXXXXXXXXXXXXXXXXXXXXXXXXXXXXXXXXX
\item \label{enum:outline-pf-main-technical:9}
For the first and last terms, \eqref{enum:outline-pf-main-technical:5:i} and \eqref{enum:outline-pf-main-technical:5:ii}, the union bounds are straightforward: simply taken over the coordinate subsets of cardinality at most \(  \k  \), as well as, in the case of \eqref{enum:outline-pf-main-technical:5:i}, over the cover, \(  \ParamCover  \).
%XXXXXXXXXXXXXXXXXXXXXXXXXXXXXXXXXXXXXXXXXXXXXXXXXXXXXXXXXXXXXXXXXXXXXXXXXXXXXXXXXXXXXXXXXXXXXXXXXXX
\item \label{enum:outline-pf-main-technical:10}
In contrast, the second term, \eqref{enum:outline-pf-main-technical:5:iii},  requires a more careful---and somewhat indirect---argument.
In this case, the union bound is taken over the set
%|>>|:::::::::::::::::::::::::::::::::::::::::::::::::::::::::::::::::::::::::::::::::::::::::::|>>|
%\(  \{ \hFn[\Supp( \thetaStar ) \cup \JCoords]( \thetaX, \thetaXX ) : \thetaXX \in \BallX{2\tauX}( \thetaX ), \Supp( \thetaXX ) = \Supp( \thetaX ) \}  \),
\(  \{ \hFn[\Supp( \thetaStar ) \cup \JCoords]( \thetaX, \thetaXX ) : \thetaXX \in \BallX{2\tauX}( \thetaX ) \}  \),
%|<<|:::::::::::::::::::::::::::::::::::::::::::::::::::::::::::::::::::::::::::::::::::::::::::|<<|
which has a sufficiently small cardinality due to the local binary embeddings of \cite[{\COROLLARY 3.3}]{oymak2015near}.
%\ToDo{Revise this.}
%XXXXXXXXXXXXXXXXXXXXXXXXXXXXXXXXXXXXXXXXXXXXXXXXXXXXXXXXXXXXXXXXXXXXXXXXXXXXXXXXXXXXXXXXXXXXXXXXXXX
\item \label{enum:outline-pf-main-technical:11}
Using the uniform results obtained in \STEPS \ref{enum:outline-pf-main-technical:7}--\ref{enum:outline-pf-main-technical:10}, the number of covariates, \(  \m  \), can then be determined such that desired bounds on the terms \eqref{enum:outline-pf-main-technical:5:i}--\eqref{enum:outline-pf-main-technical:5:ii},
%on the \RHS \eqref{eqn:enum:outline-pf-main-technical:2},
and hence also the desired bound on \eqref{eqn:enum:outline-pf-main-technical:1}, hold uniformly with high probability.
This will establish the invertibility condition for Gaussian covariate matrices claimed in \THEOREM \ref{thm:main-technical:sparse}.
%Using \EQUATION \eqref{eqn:enum:outline-pf-main-technical:2} and the bounds on \eqref{enum:outline-pf-main-technical:5:i}--\eqref{enum:outline-pf-main-technical:5:ii} obtained in \STEPS \ref{enum:outline-pf-main-technical:7}--\ref{enum:outline-pf-main-technical:10}, the number of covariates, \(  \m  \), can then be determined such that a desired bound on \eqref{eqn:enum:outline-pf-main-technical:1} hold uniformly with high probability.
%XXXXXXXXXXXXXXXXXXXXXXXXXXXXXXXXXXXXXXXXXXXXXXXXXXXXXXXXXXXXXXXXXXXXXXXXXXXXXXXXXXXXXXXXXXXXXXXXXXX
%\item \label{enum:outline-pf-main-technical:12}
%%XXXXXXXXXXXXXXXXXXXXXXXXXXXXXXXXXXXXXXXXXXXXXXXXXXXXXXXXXXXXXXXXXXXXXXXXXXXXXXXXXXXXXXXXXXXXXXXXXXX
\end{enumerate}
%|<<|XXXXXXXXXXXXXXXXXXXXXXXXXXXXXXXXXXXXXXXXXXXXXXXXXXXXXXXXXXXXXXXXXXXXXXXXXXXXXXXXXXXXXXXXXXX|<<|
%|<<|XXXXXXXXXXXXXXXXXXXXXXXXXXXXXXXXXXXXXXXXXXXXXXXXXXXXXXXXXXXXXXXXXXXXXXXXXXXXXXXXXXXXXXXXXXX|<<|
%|<<|XXXXXXXXXXXXXXXXXXXXXXXXXXXXXXXXXXXXXXXXXXXXXXXXXXXXXXXXXXXXXXXXXXXXXXXXXXXXXXXXXXXXXXXXXXX|<<|