% COMMANDS, ALIASES, ENVIRONMENTS, VARIABLES FOR SPARSE LEARNING PAPERS
\usepackage[utf8]{inputenc} % remove in COLT version
\usepackage{natbib} % remove in COLT version
\usepackage{usebib} % remove in COLT version
\usepackage{amsmath} % remove in COLT version
\usepackage{mathtools}
\usepackage{amsthm} % remove in COLT version
%\usepackage{amsfonts}
\usepackage{amssymb} % remove in COLT version
\usepackage{graphicx} % remove in COLT version
%\usepackage{microtype}      % microtypography
%\usepackage{url}            % simple URL typesetting
%\usepackage{booktabs}       % professional-quality tables
%\usepackage{nicefrac}       % compact symbols for 1/2, etc.
%\usepackage{algorithmic}
\usepackage[ruled,vlined,linesnumbered,hangingcomment]{algorithm2e} % remove in COLT version
% \usepackage{algorithm2e} % for COLT
%\usepackage{algorithm}
%\usepackage{algorithmic}
%\newsavebox{\algbox}
%\usepackage{textcomp}
\usepackage{makecell} % remove in COLT version
\usepackage[flushleft]{threeparttable} % remove in COLT version
\usepackage{geometry} % remove in COLT version
%\usepackage{fancyhdr}
%\usepackage{fontsize}
%\usepackage{verbatim}
\usepackage{hyperref} % remove in COLT version - *CHECK THIS*
\usepackage{color}
\usepackage[dvipsnames]{xcolor} % remove in COLT version
\usepackage{bm}
%\usepackage{multicol}
%\usepackage{multirow}
\usepackage{enumerate}
%\usepackage[inline]{enumitem}
\usepackage[explicit]{titlesec} % remove in COLT version
%\usepackage{titletoc}
\usepackage{xargs}
\usepackage{xifthen}
\usepackage{xparse}
%\usepackage{listings}
%\usepackage{etoolbox}
%\usepackage{tcolorbox}
%\usepackage{colortbl}
%\usepackage{pdfpages}
\usepackage{sidecap}
\usepackage{xspace}
\usepackage{xfrac}
\usepackage{arydshln}
\usepackage{booktabs} % remove in COLT version
\usepackage{comment}
%\usepackage{mleftright}
%\delimitershortfall-1sp
%\mleftright

\bibinput{refs}

\bibliographystyle{plainnat}
%\bibpunct{[}{]}{,}{n}{}{,}
\bibpunct{(}{)}{;}{a}{}{,}


% \def\BibTeX{{\rm B\kern-.05em{\sc i\kern-.025em b}\kern-.08em
%   T\kern-.1667em\lower.7ex\hbox{E}\kern-.125em}}

%%%%%%%%%%%%%%%%%%%%%%%%%%%%%%%%%%%%%%%%%%%%%%%%%%%%%%%%%%%%%%%%%%%%%%%%%%%%%%%%%%%%%%%%%%%%%%%%%%%%
%%%%%%%%%%%%%%%%%%%%%%%%%%%%%%%%%%%%%%%%%%%%%%%%%%%%%%%%%%%%%%%%%%%%%%%%%%%%%%%%%%%%%%%%%%%%%%%%%%%%
%%%%%%%%%%%%%%%%%%%%%%%%%%%%%%%%%%%%%%%%%%%%%%%%%%%%%%%%%%%%%%%%%%%%%%%%%%%%%%%%%%%%%%%%%%%%%%%%%%%%

\allowdisplaybreaks
% \setlength\tabcolsep{1pt}

%https://tex.stackexchange.com/questions/48220/error-when-using-bm-in-numbered-section-title-when-hyperref-is-loaded
\pdfstringdefDisableCommands{%
    \renewcommand*{\bm}[1]{\textbf{#1}}%
    \renewcommand*{\Vec}[1]{\textbf{#1}}%
    \renewcommand{\THEOREM}{Theorem\ }
    \renewcommand{\LEMMA}{Lemma\ }
    \renewcommand{\LEMMAS}{Lemmas\ }
    \renewcommand{\STEP}{Step\ }
    \renewcommand{\EQUATION}{Equation\ }
    \renewcommand{\EQUATIONS}{Equations\ }
    \renewcommand{\COROLLARY}{Corollary\ }
    \renewcommand{\COROLLARIES}{Corollaries\ }
    \renewcommand{\ref}[1]{#1}
    \renewcommand{\eqref}[1]{(#1)}
    %\renewcommand{}{\ }
    % any other necessary redefinitions
}

% remove in COLT version
% \let\originalleft\left
% \let\originalright\right
% \renewcommand{\left}{\mathopen{}\mathclose\bgroup\originalleft}
% \renewcommand{\right}{\aftergroup\egroup\originalright}

\newcommand\partialtoprule[2]{%
   \addlinespace[-\belowrulesep]
   \cmidrule(#1){#2}
   \addlinespace[-\belowrulesep]}
\newcommand\partialbottomrule[2]{%
   \addlinespace[-\aboverulesep]
   \cmidrule(#1){#2}
   \addlinespace[-\belowrulesep]}
%\newcommand\partialmidrule[2]{%
%   \addlinespace[-\belowrulesep]
%   \cmidrule(#1){#2}
%   \addlinespace[-\belowrulesep]}

%{2e539e}%{28498a}%{24478f}%{2450a8}
%{2e669e}%{4b659b}%{4563a1}%{4b659b}%{506795}%{3f62a6}
%{506795}
%{336699}%{2d5986}%{386694}%{386694}%{3f73a6}%{36597c}%{3a5978}%{5c6a8b}
%{56698f}%{42598a}%{5b80a4}%{517394}%{0.18,0.38,0.48}
%\everymath{\color{MathInlineColor}}

\newboolean{showcomments}
\newboolean{showcolors}
\setboolean{showcomments}{true}
\setboolean{showcolors}{true}

\definecolor{todocolor}{rgb}{1,0,0}
\definecolor{cmtcolor}{rgb}{0.1,0.1,0.9}
\definecolor{cmtqcolor}{HTML}{0959aa}%{0071BC}

\def\ldelim{(}%{\ast\,}
\def\rdelim{)}%{\,\ast}

%\newcommand{\TODO}[1][]{%
%  \ifthenelse{\boolean{showcomments}}%
%  {\ensuremath{\text{\normalfont\textcolor{red}{\(\textcolor{red}{\ldelim}\)\,\textbf{To\,Do}\IfNE{#1}{\textbf{:} \emph{#1}}\,\(\textcolor{red}{\rdelim}\)}}}}%
%  {}%
%}%

\newcommand{\IfNE}[2]{\ifthenelse{\isempty{#1}}{}{#2}}
\newcommand{\IfENE}[3]{\ifthenelse{\isempty{#1}}{#2}{#3}}
\NewDocumentCommand{\Sub}{s m O{} O{}}{\IfBooleanTF{#1}{\IfENE{#2}{_{#3#4}}{_{#3#2#4}}}{\IfNE{#2}{_{#3#2#4}}}}
\NewDocumentCommand{\Sup}{s m O{} O{}}{\IfBooleanTF{#1}{\IfENE{#2}{^{#3#4}}{^{#3#2#4}}}{\IfNE{#2}{^{#3#2#4}}}}

\newcommand{\latex}{\LaTeX\xspace}
\newcommand{\tex}{\TeX\xspace}

%https://tex.stackexchange.com/questions/316426/negative-phantom-inside-equations
\newcommand{\negphantom}[1]{\ifmmode\settowidth{\dimen0}{$#1$}\else\settowidth{\dimen0}{#1}\fi\hspace*{-\dimen0}}

%https://tex.stackexchange.com/questions/74353/what-commands-are-there-for-horizontal-spacing/74354#74354
%https://tex.stackexchange.com/questions/332146/a-canonical-or-robust-white-space-thinner-than
%\protected\def\verythinspace{%
\protected\def\<{%
  \ifmmode
    \mskip0.5\thinmuskip
  \else
    \ifhmode
      \kern0.08334em
    \fi
  \fi
}

%%%%%%%%%%%%%%%%%%%%%%%%%%%%%%%%%%%%%%%%%%%%%%%%%%%%%%%%%%%%%%%%%%%%%%%%%%%%%%%%%%%%%%%%%%%%%%%%%%%%

%\makeatletter
%\renewcommand\paragraph{\@startsection{paragraph}{8}{\z@}%
%  {1ex \@plus1ex \@minus.2ex}%
%  {-1em}%
%  {\normalfont\normalsize\itshape}}
%\makeatother

%%%%%%%%%%%%%%%%%%%%%%%%%%%%%%%%%%%%%%%%%%%%%%%%%%%%%%%%%%%%%%%%%%%%%%%%%%%%%%%%%%%%%%%%%%%%%%%%%%%%

\NewDocumentCommand{\InlineClaim}{+m}{\emph{#1}\xspace}

%%%%%%%%%%%%%%%%%%%%%%%%%%%%%%%%%%%%%%%%%%%%%%%%%%%%%%%%%%%%%%%%%%%%%%%%%%%%%%%%%%%%%%%%%%%%%%%%%%%%

\newcommand{\NumberEqn}{\stepcounter{equation}\tag{\theequation}}

%%%%%%%%%%%%%%%%%%%%%%%%%%%%%%%%%%%%%%%%%%%%%%%%%%%%%%%%%%%%%%%%%%%%%%%%%%%%%%%%%%%%%%%%%%%%%%%%%%%%
%%%%%%%%%%%%%%%%%%%%%%%%%%%%%%%%%%%%%%%%%%%%%%%%%%%%%%%%%%%%%%%%%%%%%%%%%%%%%%%%%%%%%%%%%%%%%%%%%%%%
%%%%%%%%%%%%%%%%%%%%%%%%%%%%%%%%%%%%%%%%%%%%%%%%%%%%%%%%%%%%%%%%%%%%%%%%%%%%%%%%%%%%%%%%%%%%%%%%%%%%

%\makeatletter
%\renewcommand{\paragraph}{\@startsection{paragraph}{8}{\z@}%
%  {1ex \@plus1ex \@minus.2ex}%
%  {-1em}%
%  {\normalfont\normalsize\bfseries}}
%\makeatother
\makeatletter % remove in COLT version
\titleformat{\paragraph}[runin]{\normalfont\itshape}{}{}{\theparagraph #1.}
\titleformat{\paragraph}[runin]{\normalfont\itshape}{}{}{#1.} % remove in COLT version
\makeatother % remove in COLT version

%\newcommand{\ProofLabel}[2][]{of #1 #2}
\newcommand{\ProofLabel}[2][]{#2}%[2][.]{\textit{(#2)#1}}

\makeatletter
\renewenvironment{proof}[1]{\par
 \renewcommand{\qedsymbol}{$\textcolor{black}{\blacksquare}$}%
 \pushQED{\qed}%
 \normalfont \topsep6\p@\@plus6\p@\relax
 \trivlist
 \item\relax
 {\itshape
 \proofname\ifthenelse{\equal{#1}{}}{\@addpunct{.}}{ (Proof of #1)\@addpunct{.}}}\hspace\labelsep\ignorespaces
}{%
 \popQED\endtrivlist\@endpefalse \vskip8\p@\@plus6\p@\relax
}
\makeatother

% remove in COLT version
\makeatletter
\newenvironment{subproof}[1]{\par
 \renewcommand{\qedsymbol}{$\textcolor{black}{\square}$}%
 \pushQED{\qed}
 \normalfont \topsep6\p@\@plus6\p@\relax
 \trivlist
 \item\relax
 {\itshape
 \proofname\ifthenelse{\equal{#1}{}}{\@addpunct{.}}{ (Proof of #1)\@addpunct{.}}}\hspace\labelsep\ignorespaces
}{%
 \popQED\endtrivlist\@endpefalse \vskip8\p@\@plus6\p@\relax
}
\makeatother

\makeatletter
\newenvironment{FillIn}[1][]{\par
 \renewcommand{\qedsymbol}{\textcolor{BrickRed}{$\bigstar$}}%
 \pushQED{\qed}
 \normalfont\color{RawSienna} \topsep6\p@\@plus6\p@\relax
 \trivlist
 \item\relax
 {\bfseries\itshape\color{BrickRed}
 Finish / Fill-In\@addpunct{:}}\hspace\labelsep\ignorespaces
}{%
 \popQED\endtrivlist\@endpefalse \color{black}\vskip8\p@\@plus6\p@\relax
}
\makeatother

% \begin{comment} % remove in COLT version

\newtheoremstyle{claim}% name
{3pt}%      Space above
{3pt}%      Space below
{\itshape}%         Body font
{}%         Indent amount (empty = no indent, \parindent = para indent)
{\bfseries}%{\itshape}% Thm head font
{.}%        Punctuation after thm head
{.5em}%     Space after thm head: " " = normal interword space;
%       \newline = linebreak
{}%         Thm head spec (can be left empty, meaning `normal')

\newtheoremstyle{property}% name
{3pt}%      Space above
{3pt}%      Space below
{\itshape}%         Body font
{}%         Indent amount (empty = no indent, \parindent = para indent)
{}%{\itshape}% Thm head font
{}%        Punctuation after thm head
{.5em}%     Space after thm head: " " = normal interword space;
%       \newline = linebreak
{}%         Thm head spec (can be left empty, meaning `normal')

\newtheoremstyle{LemmaRepeat}% name
{3pt}%      Space above
{3pt}%      Space below
{\itshape}%         Body font
{}%         Indent amount (empty = no indent, \parindent = para indent)
{\bfseries}%{\itshape}% Thm head font
{.}%        Punctuation after thm head
{.5em}%     Space after thm head: " " = normal interword space;
%       \newline = linebreak
{\thmname{#1}\thmnote{ #3}}%         Thm head spec (can be left empty, meaning `normal')

%https://tex.stackexchange.com/questions/43966/how-to-make-the-optional-title-of-a-theorem-bold-with-amsthm
\newtheoremstyle{mystyle}%                % Name
  {}%                                     % Space above
  {}%                                     % Space below
  {\itshape}%                                     % Body font
  {}%                                     % Indent amount
  {\bfseries}%                            % Theorem head font
  {.}%                                    % Punctuation after theorem head
  { }%                                    % Space after theorem head, ' ', or \newline
  {\thmname{#1}\thmnumber{ #2}\thmnote{ ({\normalfont#3})}}%                                     % Theorem head spec (can be left empty, meaning `normal')

\theoremstyle{mystyle}

% \end{comment}

\newtheorem{theorem}{Theorem}[section]
\newtheorem{corollary}[theorem]{Corollary}
\newtheorem{lemma}[theorem]{Lemma}
\newtheorem{prop}[theorem]{Prop}
\newtheorem{conjecture}[theorem]{Conjecture}
\newtheorem{assumption}[theorem]{Assumption}
\newtheorem{fact}[theorem]{Fact}
\newtheorem{remark}{Remark}[section]
\newtheorem*{theorem*}{Restatement of Theorem}
\newtheorem*{lemma*}{Restatement of Lemma}
\newtheorem*{claim*}{Restatement of Claim}
\newtheorem*{corollary*}{Restatement of Corollary}
\newtheorem*{fact*}{Restatement of Fact}
\theoremstyle{definition}
\newtheorem{definition}{Definition}[section]
\theoremstyle{remark}
% \newtheorem{remark}{Remark}[section]
\theoremstyle{claim} % remove in COLT version
\newtheorem{claim}[theorem]{Claim}
\theoremstyle{property} % remove in COLT version
\newtheorem{property}{}[section]
\theoremstyle{LemmaRepeat} % remove in COLT version
\newtheorem{LemmaRepeat}{Lemma}

\newcommand{\LemmaLabel}[1]{(#1)}

% remove in COLT version
\makeatletter
\renewenvironment{remark}[1][]{\par
 \renewcommand{\qedsymbol}{$\blacktriangleleft$}%
 \pushQED{\qed}%
 \normalfont \topsep6\p@\@plus6\p@\relax
 \trivlist
 \item\relax
 \refstepcounter{remark}
 {{\bfseries Remark \theremark}\ifthenelse{\equal{#1}{}}{\@addpunct{.}}{ (#1)\@addpunct{.}}}\hspace\labelsep\ignorespaces
}{%
 \popQED\endtrivlist\@endpefalse \vskip8\p@\@plus6\p@\relax
}
\makeatother


%%%%%%%%%%%%%%%%%%%%%%%%%%%%%%%%%%%%%%%%%%%%%%%%%%%%%%%%%%%%%%%%%%%%%%%%%%%%%%%%%%%%%%%%%%%%%%%%%%%%
% |<<| THM/DEF/ETC. ENVIRONMENTS |<<|%%%%%%%%%%%%%%%%%%%%%%%%%%%%%%%%%%%%%%%%%%%%%%%%%%%%%%%%%%%%%%%
% |>>| LISTS/NUMBERING --------- |>>|%%%%%%%%%%%%%%%%%%%%%%%%%%%%%%%%%%%%%%%%%%%%%%%%%%%%%%%%%%%%%%%
%%%%%%%%%%%%%%%%%%%%%%%%%%%%%%%%%%%%%%%%%%%%%%%%%%%%%%%%%%%%%%%%%%%%%%%%%%%%%%%%%%%%%%%%%%%%%%%%%%%%

\DontPrintSemicolon % remove in COLT version

\SetKwInput{Given}{Given}
\SetKwInput{Solution}{Solution}
%\SetKwInput{Param}{Param}
%\SetKwInOut{Require}{\textcolor{AlgSetupKWColor}{Require}}
\SetKwInput{Argument}{Argument}
\SetKwInput{Require}{Require}
\SetKwInput{Let}{Let}
\SetKwInput{Fix}{Fix}
\SetKwInput{Define}{Define}
\SetKwInput{Design}{Design}
\SetKwInput{Construct}{Construct}
%\SetKwInOut{Define}{Define}
%\SetKwInOut{Denote}{Denote}
%\SetKwInOut{Assume}{Assume}
%\SetKwInOut{Dataset}{Dataset}
%\SetKwInOut{Samples}{Samples}

%\SetKwInOut{Goal}{Goal}
%\SetKwInOut{Solution}{Solution}
%\SetKwBlock{Given}{Given:}{}
%\SetKwInOut{Prob}{Problem}
%\SetKwInOut{Task}{Task}

\SetKwIF{If}{ElseIf}{Else}%
  {if}%
  {then:}%
  {else if}{else}{endif}
\SetKwFor{ForEach}%
  {for each}%
  {do:}{endfch}
\SetKwFor{ForAll}%
  {for all}%
  {do:}%
  {endfall}

\SetKw{LetStatement}{let:}
\SetKw{InitStatement}{initialize:}
\SetKw{GivenStatement}{given:}
\SetKw{ConstructStatement}{construct:}
\SetKw{CalculateStatement}{calculate:}
\SetKw{EstimateStatement}{estimate:}
\SetKw{FindStatement}{find:}
%\SetKwInput{DefineStatement}{Define}
\SetKw{SolveStatement}{solve:}
\SetKwProg{Stage}{Stage}{:}{end}
%\SetKwProg{Stage}{\normalfont\emph{Stage}}{\normalfont\emph{:}}{end}

%\newcommand{\progstyle}[1]{\normalfont\textbf{#1}}
%\SetProgSty{progstyle}
%\SetProcArgSty{textbf}
%\SetFuncArgSty{progstyle}
%\SetArgSty{progstyle}

%\SetKwProg{Procedure}{Procedure}{:}{}

\newcommand{\mycommentfnt}[1]{\textsf{\footnotesize #1}}

% remove in COLT version
% \SetSideCommentLeft
% \SetNoFillComment
% \SetCommentSty{mycommentfnt}
% \SetKwComment{Comment}{$\textcolor{black}{\blacktriangleright}$ \ }{}

%%%%%%%%%%%%%%%%%%%%%%%%%%%%%%%%%%%%%%%%%%%%%%%%%%%%%%%%%%%%%%%%%%%%%%%%%%%%%%%%%%%%%%%%%%%%%%%%%%%%

\newcommand{\smallsquare}{%
  \text{\fboxsep=-.2pt\raisebox{0.75mm}{\fbox{\rule{0pt}{0.8ex}\rule{0.8ex}{0pt}}}}%
}

\begin{comment} % remove in COLT version
\newlist{itemizetriangle}{itemize}{10}
\setlist[itemizetriangle,1]{label=$\textcolor{black}{\blacktriangleright}$}
\setlist[itemizetriangle,2]{label=\textbullet}
\setlist[itemizetriangle,3]{label=\textendash}
\setlist[itemizetriangle,4]{label=$\textcolor{black}{\triangleright}$}

\newlist{Itemize}{itemize}{10}
\setlist[Itemize,1]{label=$\textcolor{black}{\triangleright}$,nosep}
\setlist[Itemize,2]{label=\textendash,nosep}
\setlist[Itemize,3]{label=$\textcolor{black}{\diamond}$,nosep}
\setlist[Itemize,4]{label=$\textcolor{black}{\smallsquare}$,nosep}
\setlist[Itemize,5]{label=$\textcolor{black}{\circ}$,nosep}

%\newlist{Enumerate}{enumerate}{5}
%\setlist[Enumerate,1]{label={(\arabic*)}}
%\setlist[Enumerate,2]{label={(\roman*)}}
%\setlist[Enumerate,3]{label={(\alph*)}}
%
\newlist{Enumerate}{enumerate}{5}
\setlist[Enumerate,1]{label={(\alph*)}}
\setlist[Enumerate,2]{label={(\roman*)}}
\setlist[Enumerate,3]{label={(\arabic*)}}

\newlist{EnumerateText}{enumerate}{5}
\setlist[EnumerateText,1]{label={(\alph*)}}
\setlist[EnumerateText,2]{label={(\arabic*)}}
\setlist[EnumerateText,3]{label={(\roman*)}}

\newlist{EnumerateThm}{enumerate}{5}
\setlist[EnumerateThm,1]{label={\textup{\thetheorem{} (\arabic*)}}, leftmargin=0.08\linewidth}
\setlist[EnumerateThm,2]{label={(\roman*)}}
\setlist[EnumerateThm,3]{label={(\alph*)}}

\newlist{EnumeratePropRef}{enumerate}{5}
\setlist[EnumeratePropRef,1]{label={\textup{\Prop \thetheorem{} (\arabic*) (HD)}}, leftmargin=0.18\linewidth,first=\itshape}
\setlist[EnumeratePropRef,2]{label={(\roman*)}}
\setlist[EnumeratePropRef,3]{label={(\alph*)}}

\newlist{EnumerateSub}{enumerate}{5}
\setlist[EnumerateSub,1]{label={(\roman*)}}
\setlist[EnumerateSub,2]{label={(\arabic*)}} % might swap arabic/alph
\setlist[EnumerateSub,3]{label={(\alph*)}}

\newlist{EnumerateAlg}{enumerate}{5}
\setlist[EnumerateAlg,1]{label={\bfseries\small\arabic*:},left=0pt,itemindent=0pt}
\setlist[EnumerateAlg,2]{label={(\roman*)}}
\setlist[EnumerateAlg,3]{label={(\alph*)}}

\newlist{EnumerateInline}{enumerate*}{3}
\setlist[EnumerateInline,1]{label={(\roman*)}}
\setlist[EnumerateInline,2]{label={(\alph*)}}
\setlist[EnumerateInline,3]{label={(arabic*)}}

\end{comment}

\newcommand{\TextEnum}[1]{(#1)}
\newcommand{\Enum}[2][]{\IfENE{#1}{(#2)}{\Label{#1}{(#2)}}\xspace}
%https://tex.stackexchange.com/questions/86338/label-one-line-in-align-environment
\newcommand{\TagEqn}{\stepcounter{equation}\tag{\theequation}}

%\makeatletter
%\NewDocumentCommand{\Label}{s m m}{%
%  \@bsphack
%  \csname phantomsection\endcsname % in case hyperref is used
%  \IfBooleanF{#1}{#3}%
%  \def\@currentlabel{#3}{\label{#2}}%
%  \@esphack
%}
%\makeatother

\MakeRobust{\ref}

%https://tex.stackexchange.com/questions/271062/labeling-a-text-and-referencing-it-later#:~:text=%5Clabel%20will%20refer%20to%20the%20current%20value%20of%20%5C%40currentlabel%3B%20just%20use%20this%20feature.
\makeatletter
\NewDocumentCommand{\Label}{s m m}{%
  \@bsphack
  \csname phantomsection\endcsname % in case hyperref is used
  \IfBooleanF{#1}{#3}%
  \def\@currentlabel{#3}{#2}%{\label{#2}}%
  \@esphack
}
\makeatother

% https://tex.stackexchange.com/questions/198771/align-in-substack
\makeatletter
\newcommand{\subalign}[1]{%
  \vcenter{%
    \Let@ \restore@math@cr \default@tag
    \baselineskip\fontdimen10 \scriptfont\tw@
    \advance\baselineskip\fontdimen12 \scriptfont\tw@
    \lineskip\thr@@\fontdimen8 \scriptfont\thr@@
    \lineskiplimit\lineskip
    \ialign{\hfil$\m@th\scriptstyle##$&$\m@th\scriptstyle{}##$\hfil\crcr
      #1\crcr
    }%
  }%
}
\makeatother

% https://tex.stackexchange.com/questions/434391/how-do-i-get-italic-sans-serif-in-math-mode
\DeclareMathAlphabet{\mathsfit}{T1}{\sfdefault}{\mddefault}{\sldefault}
\SetMathAlphabet{\mathsfit}{bold}{T1}{\sfdefault}{\bfdefault}{\sldefault}

%%%%%%%%%%%%%%%%%%%%%%%%%%%%%%%%%%%%%%%%%%%%%%%%%%%%%%%%%%%%%%%%%%%%%%%%%%%%%%%%%%%%%%%%%%%%%%%%%%%%
%%%%%%%%%%%%%%%%%%%%%%%%%%%%%%%%%%%%%%%%%%%%%%%%%%%%%%%%%%%%%%%%%%%%%%%%%%%%%%%%%%%%%%%%%%%%%%%%%%%%
%%%%%%%%%%%%%%%%%%%%%%%%%%%%%%%%%%%%%%%%%%%%%%%%%%%%%%%%%%%%%%%%%%%%%%%%%%%%%%%%%%%%%%%%%%%%%%%%%%%%

\newcommand{\R}{\mathbb{R}}
\newcommand{\Q}{\mathbb{Q}}
\newcommand{\Z}{\mathbb{Z}}
\newcommand{\C}{\mathbb{C}}
%\newcommand{\N}{\mathbb{N}}

\newcommand{\RNonzero}{( \R \setminus \{ 0 \} )}

\newcommand{\ZeroTo}[1]{\{ 0, \dots, #1 \}}
\newcommand{\ZminusO}[1][]{\IfENE{#1}{\Z \setminus \{0\}}{( \Z \setminus \{0\} )^{#1}}}
\newcommand{\RminusO}[1][]{\IfENE{#1}{\R \setminus \{0\}}{( \R \setminus \{0\} )^{#1}}}

\newcommand{\Order}{\OperatorName{order}}
\newcommand{\Abelian}{abelian\xspace}

% \let\oldemptyset\emptyset % remove in COLT version
% \let\emptyset\varnothing % remove in COLT version

%\NewDocumentCommand{\e}{s +m O{}}{\IfBooleanTF{#1}{e^{(#2)#3}}{\bm{\mathrm{e}}^{(#2)#3}}}
%\NewDocumentCommand{\e}{s +m O{}}{\IfBooleanTF{#1}{e^{#2#3}}{\bm{\mathrm{e}}^{#2#3}}}
\newcommand{\e}{e}
\newcommand{\ex}{\hat{e}}

%\newcommand{\OperatorName}[1]{\operatorname{#1}}%{\operatorname{\mathsf{#1}}}
\let\OperatorName\operatorname
\newcommand{\OperatorNameWithLimits}[1]{\operatornamewithlimits{#1}}%{\operatornamewithlimits{\mathsf{#1}}}

%\NewDocumentCommand{\RV}{+m O{}}{#1\Sub{#2}}
\newcommand{\Distr}[1]{\mathcal{#1}}
%\newcommand{\Event}[1]{\mathcal{#1}}
\newcommand{\Set}[1]{\mathcal{#1}}
%\newcommand{\Family}[1]{\mathcal{#1}}
%\newcommand{\Coords}[1]{#1}%{\mathcal{#1}}
%\newcommand{\Rows}[1]{#1}
%\newcommand{\ColSet}[1]{#1}
\renewcommand{\Pr}{\operatornamewithlimits{\mathit{P}}}
\newcommand{\EOp}{\operatornamewithlimits{\mathbb{E}}}

\newcommand{\N}{\mathcal{N}}

%\makeatletter
%\newcommand\niton{\mathrel{\m@th\mathpalette\canc@l\owns}}
%\newcommand\canc@l[2]{{\ooalign{$\hfil#1/\mkern1mu\hfil$\crcr$#1#2$}}}
%\makeatother
%
%\newcommand{\simiid}{\stackrel{\mathrm{\iid[]}}{\sim}}
%
%\newcommand{\vin}{\rotatebox[origin=c]{-90}{\(\in\)}}
%\newcommand{\vleq}{\rotatebox[origin=c]{-90}{\(\leq\)}}
%\newcommand{\vgeq}{\rotatebox[origin=c]{-90}{\(\geq\)}}
%\newcommand{\vl}{\rotatebox[origin=c]{-90}{\(<\)}}
%\newcommand{\vg}{\rotatebox[origin=c]{-90}{\(>\)}}
%\newcommand{\veq}{\rotatebox[origin=c]{-90}{\(=\)}}

\newcommand{\T}{T}%{\mathsf{T}}
\newcommand{\Star}{\ast}
\newcommand{\IOp}{\mathbb{I}}
\newcommand{\IpmOp}{\mathbb{I}_{\pm}}

\newcommand{\0}{\mathsf{0}}
\newcommand{\1}{\mathsf{1}}
\newcommand{\2}{\mathsf{2}}
\newcommand{\3}{\mathsf{3}}
\newcommand{\Minus}{\mathsf{-}}
\newcommand{\Plus}{\mathsf{+}}

\NewDocumentCommand{\Log}{s O{} D(){}}{{%
  \ifthenelse{\isempty{#3}}{\log}{
    %\IfBooleanTF{#1}{\log\IfNE{#2}{_{#2}}(#3)}%
    {
      \log\IfNE{#2}{_{#2}}
      \mathchoice
        {\left( #3 \right)} % \displaystyle
        {\IfBooleanTF{#1}{\left( #3 \right)}{(#3)}} % \textstyle
        {\left( #3 \right)}%{(#3)} % \scriptstyle
        {\left( #3 \right)}%{(#3)} % \scriptscriptstyle
}}}}

\NewDocumentCommand{\Sign}{t. s O{} D(){}}{{%
  \def\dfmt{\left( #4 \right)}
  \def\tfmt{( #4 )}
  \IfBooleanT{#1}{\;}
  \SignOp\Sub{#3}
  \IfNE{#4}{
    \IfBooleanTF{#2}{\tfmt}{
      \mathchoice
        {\dfmt} % \displaystyle
        %{\SignOp\Sub{#2}\bigl( #3 \bigr)} % \displaystyle
        %{\SignOp\Sub{#2}\left( #3 \right)} % \displaystyle
        {\tfmt} % \textstyle
        {\tfmt} % \scriptstyle
        {\tfmt} % \scriptscriptstyle
}}}}

\NewDocumentCommand{\SignO}{s D(){}}{{%
  \ifthenelse{\isempty{#2}}{}{
    \IfBooleanTF{#1}{\SignOOp(#2)}{
      \mathchoice
        {\SignOOp\left( #2 \right)} % \displaystyle
        {\SignOOp(#2)} % \textstyle
        {\SignOOp(#2)} % \scriptstyle
        {\SignOOp(#2)} % \scriptscriptstyle
}}}}

\NewDocumentCommand{\Supp}{s D(){}}{{%
  \def\dfmt{\left( #2 \right)}
  \def\tfmt{( #2 )}
  \SuppOp%
  \IfBooleanTF{#1}{\tfmt}{
    \mathchoice
      {\tfmt} % \displaystyle
      {\tfmt} % \textstyle
      {\tfmt} % \scriptstyle
      {\tfmt} % \scriptscriptstyle
}}}

%\RenewDocumentCommand{\Pr}{s O{} D(){}}{{%
%  \operatornamewithlimits{Pr}\IfNE{#2}{_{#2}} \ifthenelse{\isempty{#3}}{}{
%    \IfBooleanTF{#1}{(#3)}{
%      \mathchoice
%        {\left( #3 \right)} % \displaystyle
%        {(#3)} % \textstyle
%        {(#3)} % \scriptstyle
%        {(#3)} % \scriptscriptstyle
%}}}}

%\NewDocumentCommand{\E}{s O{} D(){}}{{%
%  \EOp\IfNE{#2}{_{#2}} \ifthenelse{\isempty{#3}}{}{
%    \IfBooleanTF{#1}{(#3)}{
%      \mathchoice
%        {\log \left( #3 \right)} % \displaystyle
%        {\log(#3)} % \textstyle
%        {\log(#3)} % \scriptstyle
%        {\log(#3)} % \scriptscriptstyle
%}}}}

\newcommand{\E}{\operatornamewithlimits{\mathbb{E}}}

\NewDocumentCommand{\I}{s D(){}}{{%
  \IOp%
  \ifthenelse{\isempty{#2}}{}{
    \IfBooleanTF{#1}{(#2)}{
      \mathchoice
        {\bigl( #2 \bigr)} % \displaystyle
        {(#2)} % \textstyle
        {(#2)} % \scriptstyle
        {(#2)} % \scriptscriptstyle
}}}}

\NewDocumentCommand{\Ipm}{s D(){}}{{%
  \IpmOp%
  \ifthenelse{\isempty{#2}}{}{
    \IfBooleanTF{#1}{(#2)}{
      \mathchoice
        {\left( #2 \right)} % \displaystyle
        {(#2)} % \textstyle
        {(#2)} % \scriptstyle
        {(#2)} % \scriptscriptstyle
}}}}

%\NewDocumentCommand{\BigO}{s t' D(){}}{{%
%  \IfBooleanTF{#2}{\BigOOp'}{\BigOOp}
%  \ifthenelse{\isempty{#3}}{}{
%    \IfBooleanTF{#1}{(#3)}{
%      \mathchoice
%        {\left( #3 \right)} % \displaystyle
%        {(#3)} % \textstyle
%        {(#3)} % \scriptstyle
%        {(#3)} % \scriptscriptstyle
%}}}}

%\NewDocumentCommand{\BigOmega}{s t' D(){}}{{%
%  \IfBooleanTF{#2}{\BigOmegaOp'}{\BigOmegaOp}
%  \ifthenelse{\isempty{#3}}{}{
%    \IfBooleanTF{#1}{(#3)}{
%      \mathchoice
%        {\left( #3 \right)} % \displaystyle
%        {(#3)} % \textstyle
%        {(#3)} % \scriptstyle
%        {(#3)} % \scriptscriptstyle
%}}}}

\NewDocumentCommand{\BigO}{s t' D(){}}{{%
  \def\dfmt{\IfBooleanTF{#1}{\IfNE{#3}{(#3)}}{\IfNE{#3}{\left( #3 \right)}}}
  \def\tfmt{\IfBooleanTF{#1}{\IfNE{#3}{\left( #3 \right)}}{\IfNE{#3}{(#3)}}}
  \IfBooleanTF{#2}{\BigOOp'}{\BigOOp}
  \IfBooleanTF{#1}{\tfmt}{
    \mathchoice
      {\dfmt} % \displaystyle
      {\tfmt} % \textstyle
      {\tfmt} % \scriptstyle
      {\tfmt} % \scriptscriptstyle
}}}


\NewDocumentCommand{\BigOmega}{s t' D(){}}{{%
  \def\dfmt{\IfBooleanTF{#1}{(#3)}{\left( #3 \right)}}
  \def\tfmt{\IfBooleanTF{#1}{\left( #3 \right)}{(#3)}}
  \IfBooleanTF{#2}{\BigOmegaOp'}{\BigOmegaOp}
  \IfBooleanTF{#1}{\tfmt}{
    \mathchoice
      {\dfmt} % \displaystyle
      {\tfmt} % \textstyle
      {\tfmt} % \scriptstyle
      {\tfmt} % \scriptscriptstyle
}}}

\NewDocumentCommand{\BigTheta}{s t' D(){}}{{%
  \def\dfmt{\IfBooleanTF{#1}{(#3)}{\left( #3 \right)}}
  \def\tfmt{\IfBooleanTF{#1}{\left( #3 \right)}{(#3)}}
  \IfBooleanTF{#2}{\BigThetaOp'}{\BigThetaOp}
  \IfBooleanTF{#1}{\tfmt}{
    \mathchoice
      {\dfmt} % \displaystyle
      {\tfmt} % \textstyle
      {\tfmt} % \scriptstyle
      {\tfmt} % \scriptscriptstyle
}}}


%\NewDocumentCommand{\BigTheta}{s t' D(){}}{{%
%  \IfBooleanTF{#2}{\BigThetOp'}{\BigOThetaOp}
%  \ifthenelse{\isempty{#3}}{}{
%    \IfBooleanTF{#1}{(#3)}{
%      \mathchoice
%        {\left( #3 \right)} % \displaystyle
%        {(#3)} % \textstyle
%        {(#3)} % \scriptstyle
%        {(#3)} % \scriptscriptstyle
%}}}}
%
\NewDocumentCommand{\LittleO}{s t' D(){}}{{%
  \IfBooleanTF{#2}{\LittleOOp'}{\LittleOOp}
  \ifthenelse{\isempty{#3}}{}{
    \IfBooleanTF{#1}{(#3)}{
      \mathchoice
        {\left( #3 \right)} % \displaystyle
        {(#3)} % \textstyle
        {(#3)} % \scriptstyle
        {(#3)} % \scriptscriptstyle
}}}}

\NewDocumentCommand{\LittleOmega}{s t' D(){}}{{%
  \IfBooleanTF{#2}{\LittleOmegaOp'}{\LittleOmegaOp}
  \ifthenelse{\isempty{#3}}{}{
    \IfBooleanTF{#1}{(#3)}{
      \mathchoice
        {\left( #3 \right)} % \displaystyle
        {(#3)} % \textstyle
        {(#3)} % \scriptstyle
        {(#3)} % \scriptscriptstyle
}}}}

\NewDocumentCommand{\Paren}{s O{} m}{{%
  #2%
  \ifthenelse{\isempty{#3}}{}{
    \IfBooleanTF{#1}{(#3)}{
      \mathchoice
        {\left( #3 \right)} % \displaystyle
        {(#3)} % \textstyle
        {(#3)} % \scriptstyle
        {(#3)} % \scriptscriptstyle
}}}}

%\NewDocumentCommand{}{s D(){}}{{%
%  \ifthenelse{\isempty{#2}}{}{
%    \IfBooleanTF{#1}{(#2)}{
%      \mathchoice
%        {\left( #2 \right)} % \displaystyle
%        {(#2)} % \textstyle
%        {(#2)} % \scriptstyle
%        {(#2)} % \scriptscriptstyle
%}}}}

%\NewDocumentCommand{}{s m}{{%
%  \def\dfmt{}
%  \def\tfmt{}
%  \IfBooleanTF{#1}{\tfmt}{
%    \mathchoice
%      {\dfmt} % \displaystyle
%      {\tfmt} % \textstyle
%      {\tfmt} % \scriptstyle
%      {\tfmt} % \scriptscriptstyle
%}}}

\NewDocumentCommand{\BigOOp}{t'}{\IfBooleanTF{#1}{\tilde{O}}{O}}
\NewDocumentCommand{\BigThetaOp}{t'}{\IfBooleanTF{#1}{\tilde{\Theta}}{\Theta}}
\NewDocumentCommand{\BigOmegaOp}{t'}{\IfBooleanTF{#1}{\tilde{\Omega}}{\Omega}}
\newcommand{\LittleOOp}{o}
\newcommand{\LittleOmegaOp}{\omega}

\newcommand{\Sgn}{\OperatorName{sgn}}
\newcommand{\SignOp}{\OperatorName{sign}}
\newcommand{\SignOOp}{\OperatorName{sign}_{0}}
\newcommand{\SuppOp}{\OperatorName{supp}}
\newcommand{\SuppComplement}{\OperatorName{\overline{supp}}}
\newcommand{\Rank}{\OperatorName{rank}}
\newcommand{\Tr}{\OperatorName{tr}}
\NewDocumentCommand{\Ker}{s O{}}{\IfBooleanTF{#1}{\widehat{\OperatorName{ker}}\Sub{#2}}{\OperatorName{ker}}}
\newcommand{\Span}{\OperatorName{span}}
\newcommand{\Diag}{\OperatorName{diag}}
\newcommand{\Col}{\OperatorName{col}}
\newcommand{\Dim}{\OperatorName{dim}}

%\newcommand{\Proj}[2]{\mathsf{proj}_{#1}(#2)}
\NewDocumentCommand{\Proj}{s m D(){}}{{%
  \def\dfmt{\IfNE{#3}{\left( #3 \right)}}
  \def\tfmt{\IfNE{#3}{( #3 )}}
  \mathsf{proj}_{#2}
  \IfBooleanTF{#1}{\tfmt}{
    \mathchoice
      {\dfmt} % \displaystyle
      {\tfmt} % \textstyle
      {\tfmt} % \scriptstyle
      {\tfmt} % \scriptscriptstyle
}}}

\newcommand{\lnorm}[1]{\ell_{#1}}
\newcommand{\Lnorm}[1]{L_{#1}}
\newcommand{\lOnorm}{\(  \lnorm{0}  \)-``norm''}

\newcommand{\SThreshold}[1]{^{#1}}
\RenewDocumentCommand{\Vec}{s +m}{\IfBooleanTF{#1}{#2}{\VecFmt{#2}}}
\newcommand{\VecFmt}[1]{\mathbf{#1}}
\newcommand{\MatFmt}[1]{\mathbf{#1}}
%\RenewDocumentCommand{\Vec}{s +m}{\IfBooleanTF{#1}{#2}{\mathbf{#2}}}
%{s +m O{}}{\IfBooleanTF{#1}{#2\Sup{#3}[(][)]}{\bm{\mathrm{#2}}\Sup{#3}[(][)]}}
\NewDocumentCommand{\SVec}{s +m O{} +m}{\IfBooleanTF{#1}{#2^{\IfNE{#3}{(#3);}#4}}{\mathbf{#2}^{\IfNE{#3}{(#3);}#4}}}
\NewDocumentCommand{\BVec}{s O{} +m O{}}{\IfNE{#4}{\,(}\IfBooleanTF{#1}{(\mathbf{1}^{#3})\S{#2}}{\mathbf{1}^{\IfNE{#2}{(#2);}#3}}\IfNE{#4}{)_{#4}}}
%\NewDocumentCommand{\BVec}{s O{} +m O{}}{\IfNE{#4}{\,(}\IfBooleanTF{#1}{(\bm{\mathrm{1}}_{#3})\Sub{#2}}{\bm{\mathrm{1}}_{\IfNE{#2}{(#2);}#3}}\IfNE{#4}{)^{#4}}}
\NewDocumentCommand{\SMat}{s +m O{} +m}{\IfBooleanTF{#1}{#2^{\IfNE{#3}{(#3);}#4}}{\bm{\mathrm{#2}}^{\IfNE{#3}{(#3);}#4}}}
\NewDocumentCommand{\PMat}{s +m O{} +m}{\IfBooleanTF{#1}{#2_{\IfNE{#3}{(#3);}(#4)}}{\bm{\mathrm{#2}}_{\IfNE{#3}{(#3);}(#4)}}}
\newcommand{\SSet}[2]{\Set{#1}^{#2}}
\NewDocumentCommand{\Mat}{s +m}{\IfBooleanTF{#1}{#2}{\MatFmt{#2}}}
%\NewDocumentCommand{\Mat}{s +m}{\IfBooleanTF{#1}{#2}{\bm{\mathrm{#2}}}}
\NewDocumentCommand{\MatRow}{s +m +m O{}}{\IfBooleanTF{#1}{#2}{\bm{\mathrm{#2}}}\IfENE{#4}{_{#3}}{_{#3,#4}}}%{\bm{\mathrm{#2}}}\IfNE{#3}{_{#3}}^{(#4)}}
\NewDocumentCommand{\MatCol}{s +m +m O{}}{\IfBooleanTF{#1}{#2}{\bm{\mathrm{\underline{#2}}}}_{#3}\IfNE{#4}{^{#4}}}
\NewDocumentCommand{\Submat}{s +m +m}{{\bm{\mathrm{#2}}}^{(#3)}}
\NewDocumentCommand{\SubmatRow}{s +m +m +m O{}}{\IfBooleanTF{#1}{#2}{\bm{\mathrm{#2}}}\IfENE{#5}{_{#4}}{_{#4,#5}}^{(#3)}}
%{\bm{\mathrm{#2}}}\IfNE{#3}{_{#3}}^{(#4)}}

\newcommand{\mSp}{\,}
\newcommand{\MSp}{\;}
\newcommand{\InlineVec}[1]{(\; #1 \;)}
\newcommand{\InlineMat}[1]{(\; #1 \;)}

\newcommand{\Mean}{\OperatorName{mean}}
\newcommand{\Median}{\OperatorName{median}}

\newcommand{\Var}{\OperatorName{Var}}
\newcommand{\Cov}{\OperatorName{Cov}}
\newcommand{\Normal}{\mathcal{N}}
\newcommand{\Bernoulli}{\OperatorNameWithLimits{Bernoulli}}
\newcommand{\Binomial}{\OperatorNameWithLimits{Binomial}}
\newcommand{\Uniform}{\OperatorNameWithLimits{Uniform}}
\newcommand{\Rademacher}{\OperatorNameWithLimits{Rademacher}}

\newcommand{\DKL}[2]{D_{\mathrm{KL}}( #1 \Mid\| #2 )}

\newcommand{\iid}[1][~]{i.i.d.#1}
\NewDocumentCommand{\MGF}{s}{\IfBooleanTF{#1}{moment generating function}{mgf}\xspace}
\NewDocumentCommand{\MGFs}{s}{\IfBooleanTF{#1}{moment generating functions}{mgfs}\xspace}

\newcommand{\tsubgaussian}[1]{\ensuremath{(\sigma^{2}=#1)}-subgaussian}
\newcommand{\subgaussian}{subgaussian\xspace}
\newcommand{\normsubgaussian}{norm-subgaussian\xspace}

\newcommand{\SphereSym}{S}
\newcommand{\Sphere}[2][]{\ifthenelse{\isempty{#1}}{\SphereSym^{#2-1}}{\mathcal{S}^{#2}}}
\newcommand{\SparseRealSubspace}[2]{{\R^{#2} \cap \Sigma_{#1}^{#2}}}
\newcommand{\SparseSphereSubspace}[2]{{\SphereSym^{#2-1} \cap \SparseSubspace{#1}{#2}}}
\newcommand{\SparseRationalSubspace}[2]{{\Q^{#2} \cap \Sigma_{#1}^{#2}}}
\newcommand{\SparseSubspace}[2]{\Sigma_{#1}^{#2}}

\newcommand{\hyphen}{\text{-}}
\NewDocumentCommand{\Th}{t'}{\IfBooleanTF{#1}{^{\prime\mathrm{th}}}{\!\,^{\mathrm{th}}}}
\newcommand{\defeq}{\triangleq}

\newcommand{\SymDiff}[2]{#1 \triangle #2}
\newcommand{\indep}{\mathrel{\perp\!\!\!\perp}}

\newcommand{\dOnlyIf}{\Longrightarrow~~}
\newcommand{\dIf}{\Longleftarrow~~}
\newcommand{\dIff}{\Longleftrightarrow~~}
\newcommand{\dStep}{\longrightarrow~}
\newcommand{\dStepTab}{\phantom{\dStep~}}
%\newcommand{\dStep}{\longrightarrow~~}
\newcommand{\iIfThen}{~~\Longrightarrow~~}
\newcommand{\iOnlyIf}{~~\Longrightarrow~~}
\newcommand{\iIf}{~~\Longleftarrow~~}
\newcommand{\iIff}{~~\Longleftrightarrow~~}
\newcommand{\iStep}{~~\longrightarrow~~}
\NewDocumentCommand{\dCmt}{s O{\qquad} m}{#2\blacktriangleright \IfBooleanTF{#1}{\text{#3}}{\IfNE{#3}{\text{#3} \ }}}
\NewDocumentCommand{\dCmtx}{s O{} m}{#2\dCmtIndent \IfBooleanTF{#1}{\text{#3}}{\IfNE{#3}{\text{#3} \ }}}
\newcommand{\dCmtIndent}{\phantom{\qquad \blacktriangleright \ \ }}
\newcommand{\iWhere}{~~\text{where}~~}
\newcommand{\cIf}{\text{if}\ }
\newcommand{\cOtherwise}{\text{otherwise} }
\newcommand{\cWP}{\mathrm{with\ probability}\ }
\newcommand{\Text}[1]{\ \text{#1}\ }
\newcommand{\tAnd}{\ \text{and} \ }
\newcommand{\tOr}{\ \text{or} \ }
\newcommand{\tOrIf}{\ \text{or if} \ }
\newcommand{\tAndIf}{\ \text{and if} \ }
\newcommand{\ttIf}{\ \ \ \ \text{if} \ }
\newcommand{\tIf}{\ \text{if} \ }
\newcommand{\iseven}{\ \text{is even}}
\newcommand{\isodd}{\ \text{is odd}}
\newcommand{\Wedge}{~\wedge~}
\newcommand{\Vee}{~\vee~}
\newcommand{\mAnd}{,~}
\newcommand{\mOr}{\tOr}
\newcommand{\st}{\text{such that}}
\newcommand{\SubjectTo}{\text{subject to}}
\newcommand{\independent}{\mathrm{independent}}

\newcommand{\dCmtTab}{\phantom{\qquad\blacktriangleright }}

\newcommand{\THEOREM}[1][~]{Theorem#1\ignorespaces}
\newcommand{\COROLLARY}[1][~]{Corollary#1\ignorespaces}
\newcommand{\LEMMA}[1][~]{Lemma#1\ignorespaces}
\newcommand{\PROP}[1][~]{Prop#1\ignorespaces}
\newcommand{\PROPS}[1][~]{Propositions#1\ignorespaces}
\newcommand{\FACT}[1][~]{Fact#1\ignorespaces}
\newcommand{\FACTS}[1][~]{Facts#1\ignorespaces}
\newcommand{\CLAIM}[1][~]{Claim#1\ignorespaces}
\newcommand{\REMARK}[1][~]{Remark#1}
\newcommand{\EQUATION}[1][~]{Equation#1\ignorespaces}
\newcommand{\DEFINITION}[1][~]{Definition#1\ignorespaces}
\newcommand{\FIGURE}[1][~]{Figure#1\ignorespaces}
\newcommand{\ALGORITHM}[1][~]{Algorithm#1\ignorespaces}
\newcommand{\LINE}[1][~]{Line#1\ignorespaces}
\newcommand{\REGIME}[1][~]{Regime#1\ignorespaces}
\newcommand{\REGIMES}[1][~]{Regimes#1\ignorespaces}
\newcommand{\PROPERTY}[1][~]{Property#1\ignorespaces}
\newcommand{\PROPERTIES}[1][~]{Properties#1\ignorespaces}
\newcommand{\ASSUMPTION}[1][~]{Assumption#1\ignorespaces}
\newcommand{\ASSUMPTIONS}[1][~]{Assumptions#1\ignorespaces}
\newcommand{\STEP}[1][~]{Step#1\ignorespaces}
\newcommand{\CASE}[1][~]{Case#1\ignorespaces}
\newcommand{\CONDITION}[1][~]{Condition#1\ignorespaces}
\newcommand{\REQUIREMENT}[1][~]{Requirement#1\ignorespaces}
\newcommand{\POINT}[1][~]{Point#1\ignorespaces}
\newcommand{\ITEM}[1][~]{Item#1\ignorespaces}
\newcommand{\ARGUMENT}[1][~]{Argument#1\ignorespaces}
\newcommand{\TASK}[1][~]{Task#1\ignorespaces}

\newcommand{\THEOREMS}[1][~]{Theorems#1\ignorespaces}
\newcommand{\COROLLARIES}[1][~]{Corollaries#1\ignorespaces}
\newcommand{\LEMMAS}[1][~]{Lemmas#1\ignorespaces}
\newcommand{\CLAIMS}[1][~]{Claims#1\ignorespaces}
\newcommand{\REMARKS}[1][~]{Remarks#1\ignorespaces}
\newcommand{\EQUATIONS}[1][~]{Equations#1\ignorespaces}
\newcommand{\DEFINITIONS}[1][~]{Definitions#1\ignorespaces}
\newcommand{\FIGURES}[1][~]{Figures#1\ignorespaces}
\newcommand{\ALGORITHMS}[1][~]{Algorithms#1\ignorespaces}
\newcommand{\LINES}[1][~]{Lines#1\ignorespaces}
\newcommand{\SECTION}[1][~]{Section#1\ignorespaces}
\newcommand{\SECTIONS}[1][~]{Sections#1\ignorespaces}
\newcommand{\APPENDIX}[1][~]{Appendix#1\ignorespaces}
\newcommand{\APPENDICES}[1][~]{Appendices#1\ignorespaces}
%\NewDocumentCommand{\SECTION}{s t! O{~}}{Section#3\ignorespaces}
%\NewDocumentCommand{\SECTIONS}{s t! O{~}}{Sections#3\ignorespaces}
%\NewDocumentCommand{\APPENDIX}{s t! O{~}}{\IfBooleanTF{#2}{Section}{\IfBooleanTF{#1}{Appendix}{Appendix}}#3\ignorespaces}
%\NewDocumentCommand{\APPENDICES}{s t! O{~}}{\IfBooleanTF{#2}{Sections}{\IfBooleanTF{#1}{Appendices}{Appendices}}#3\ignorespaces}
\newcommand{\STEPS}[1][~]{Steps#1\ignorespaces}
\newcommand{\CASES}[1][~]{Cases#1\ignorespaces}
\newcommand{\CONDITIONS}[1][~]{Conditions#1\ignorespaces}
\newcommand{\REQUIREMENTS}[1][~]{Requirements#1\ignorespaces}
\newcommand{\POINTS}[1][~]{Points#1\ignorespaces}
\newcommand{\ITEMS}[1][~]{Items#1\ignorespaces}
\newcommand{\ARGUMENTS}[1][~]{Arguments#1\ignorespaces}
\newcommand{\TASKS}[1][~]{Tasks#1\ignorespaces}

\newcommand{\cf}[1][~]{cf.#1\ignorespaces}
\NewDocumentCommand{\dueto}{s O{~}}{\IfBooleanTF{#1}{Due~to}{due~to}#2\ignorespaces}
\NewDocumentCommand{\see}{s}{\IfBooleanTF{#1}{See}{see},~\ignorespaces}
\NewDocumentCommand{\seealso}{s}{\IfBooleanTF{#1}{See~also}{see~also},~\ignorespaces}
\NewDocumentCommand{\seeeg}{s}{\IfBooleanTF{#1}{See}{see},~e.g.,~\ignorespaces}

\NewDocumentCommand{\Dueto}{s O{~}}{\IfBooleanTF{#1}{due~to}{Due~to}#2\ignorespaces}
\NewDocumentCommand{\See}{s}{\IfBooleanTF{#1}{see}{See},~\ignorespaces}
\NewDocumentCommand{\Seealso}{s}{\IfBooleanTF{#1}{see~also}{See~also},~\ignorespaces}
\NewDocumentCommand{\Seeeg}{s}{\IfBooleanTF{#1}{see}{See},~e.g.,~\ignorespaces}

%\newcommand{\THEOREM}[1][~]{Theorem#1}
%\newcommand{\COROLLARY}[1][~]{Corollary#1}
%\newcommand{\LEMMA}[1][~]{Lemma#1}
%\newcommand{\PROP}[1][~]{Prop#1}
%\newcommand{\PROPS}[1][~]{Propositions#1}
%\newcommand{\FACT}[1][~]{Fact#1}
%\newcommand{\FACTS}[1][~]{Facts#1}
%\newcommand{\CLAIM}[1][~]{Claim#1}
%\newcommand{\REMARK}[1][~]{Remark#1}
%\newcommand{\EQUATION}[1][~]{Equation#1}
%\newcommand{\DEFINITION}[1][~]{Definition#1}
%\newcommand{\ALGORITHM}[1][~]{Algorithm#1}
%\newcommand{\LINE}[1][~]{Line#1}
%\newcommand{\REGIME}[1][~]{Regime#1}
%\newcommand{\REGIMES}[1][~]{Regimes#1}
%\newcommand{\PROPERTY}[1][~]{Property#1}
%\newcommand{\PROPERTIES}[1][~]{Properties#1}
%\newcommand{\ASSUMPTION}[1][~]{Assumption#1}
%\newcommand{\ASSUMPTIONS}[1][~]{Assumptions#1}
%\newcommand{\STEP}[1][~]{Step#1}
%\newcommand{\CASE}[1][~]{Case#1}
%
%\newcommand{\THEOREMS}[1][~]{Theorems#1}
%\newcommand{\COROLLARIES}[1][~]{Corollaries#1}
%\newcommand{\LEMMAS}[1][~]{Lemmas#1}
%\newcommand{\CLAIMS}[1][~]{Claims#1}
%\newcommand{\REMARKS}[1][~]{Remarks#1}
%\newcommand{\EQUATIONS}[1][~]{Equations#1}
%\newcommand{\DEFINITIONS}[1][~]{Definitions#1}
%\newcommand{\ALGORITHMS}[1][~]{Algorithms#1}
%\newcommand{\LINES}[1][~]{Lines#1}
%\newcommand{\SECTION}[1][~]{Section#1}
%\newcommand{\SECTIONS}[1][~]{Sections#1}
%\newcommand{\APPENDIX}[1][~]{Appendix#1}
%\newcommand{\APPENDICES}[1][~]{Appendices#1}
%\newcommand{\STEPS}[1][~]{Steps#1}
%\newcommand{\CASES}[1][~]{Cases#1}

%\newcommand{\THEOREM}{Theorem\xspace}
%\newcommand{\COROLLARY}{Corollary\xspace}
%\newcommand{\LEMMA}{Lemma\xspace}
%\newcommand{\PROP}{Prop\xspace}
%\newcommand{\PROPS}{Propositions\xspace}
%\newcommand{\FACT}{Fact\xspace}
%\newcommand{\FACTS}{Facts\xspace}
%\newcommand{\CLAIM}{Claim\xspace}
%\newcommand{\REMARK}{Remark\xspace}
%\newcommand{\EQUATION}{Equation\xspace}
%\newcommand{\DEFINITION}{Definition\xspace}
%\newcommand{\ALGORITHM}{Algorithm\xspace}
%\newcommand{\LINE}{Line\xspace}
%\newcommand{\REGIME}{regime\xspace}
%\newcommand{\REGIMES}{regimes\xspace}
%\newcommand{\PROPERTY}{Property\xspace}
%\newcommand{\PROPERTIES}{Properties\xspace}
%\newcommand{\ASSUMPTION}{Assumption\xspace}
%\newcommand{\ASSUMPTIONS}{Assumptions\xspace}
%\newcommand{\STEP}{Step\xspace}
%\newcommand{\CASE}{Case\xspace}
%
%\newcommand{\THEOREMS}{Theorems\xspace}
%\newcommand{\COROLLARIES}{Corollaries\xspace}
%\newcommand{\LEMMAS}{Lemmas\xspace}
%\newcommand{\CLAIMS}{Claims\xspace}
%\newcommand{\REMARKS}{Remarks\xspace}
%\newcommand{\EQUATIONS}{Equations\xspace}
%\newcommand{\DEFINITIONS}{Definitions\xspace}
%\newcommand{\ALGORITHMS}{Algorithms\xspace}
%\newcommand{\LINES}{Lines\xspace}
%\newcommand{\SECTION}{Section\xspace}
%\newcommand{\SECTIONS}{Sections\xspace}
%\newcommand{\STEPS}{Steps\xspace}
%\newcommand{\CASES}{Cases\xspace}

\newcommand{\ProofOf}[1]{Proof of #1}

\newcommand{\tab}{\ensuremath{~~~~}}
\newcommand{\Tab}[1][1]{%
    \ifthenelse{#1>0}{\tab}{}%
    \ifthenelse{#1>1}{\tab}{}%
    \ifthenelse{#1>2}{\tab}{}%
    \ifthenelse{#1>3}{\tab}{}%
    \ifthenelse{#1>4}{\tab}{}%
    \ifthenelse{#1>5}{\tab}{}%
    \ifthenelse{#1>6}{\tab}{}%
    \ifthenelse{#1>7}{\tab}{}%
    \ifthenelse{#1>8}{\tab}{}%
    \ifthenelse{#1>9}{\tab}{}%
    \ifthenelse{#1>10}{\tab}{}%
    \ifthenelse{#1>11}{\tab}{}%
    \ifthenelse{#1>12}{\tab}{}%
    \ifthenelse{#1>13}{\tab}{}%
    \ifthenelse{#1>14}{\tab}{}%
    \ifthenelse{#1>15}{\tab}{}%
    \ifthenelse{#1>16}{\tab}{}%
    \ifthenelse{#1>17}{\tab}{}%
    \ifthenelse{#1>18}{\tab}{}%
    \ifthenelse{#1>19}{\tab}{}%
    \ifthenelse{#1>20}{\tab}{}%
}

\newcommand{\AlignSp}{\tab}
\newcommand{\AlignIndent}{\tab}

\newcommand{\hfrac}[2]{#1/#2}

\NewDocumentCommand{\SBinom}{s +m +m}{\IfBooleanTF{#1}{\binom{#2}{#3}}{\binom{[#2]}{#3}}}
\newcommand{\qbinom}[3][q]{\genfrac{\lgroup}{\rgroup}{0pt}{}{#2}{#3}\Sub{#1}}

\newcommand{\Zmod}[1]{\Z/#1\Z}
\newcommand{\Zcross}[2][]{(\Z/#2\Z)^{\Gdot#1}}
\newcommand{\Zdot}[2][]{(\Z/#2\Z)^{\cdot#1}}
\newcommand{\Gdot}{\odot}

\newcommand{\+}{\phantom{+}}
\newcommand{\hsp}{\,\,\,}

\newcommand{\Group}[2]{\langle #1, #2 \rangle}
\newcommand{\PlusMod}[1]{+_{#1}}
\newcommand{\TimesMod}[1]{\times_{#1}}
\newcommand{\DotMod}[1]{\cdot_{#1}}
%\newcommand{\Id}{\mathsf{id}}

\newcommand{\Leq}{\preceq}
\newcommand{\nLeq}{\npreceq}
\newcommand{\Less}{\prec}
\newcommand{\nLess}{\nprec}
\newcommand{\Geq}{\succeq}
\newcommand{\nGeq}{\nsucceq}
\newcommand{\Gtr}{\succ}
\newcommand{\nGtr}{\nsucc}

\newcommand{\lapprox}{\lessapprox}

%\newcommand{\Param}[1]{#1}
%\newcommand{\Variable}[1]{#1}
%\newcommand{\Function}[1]{#1}
%\newcommand{\Ix}[1]{#1}
%%\newcommand{\Iter}[1]{#1}
%\newcommand{\Polynomial}[1]{#1}

\newcommand{\ForLoop}{\textsf{for}-loop\xspace}
\newcommand{\ForLoops}{\textsf{for}-loops\xspace}
\newcommand{\WhileLoop}{\textsf{while}-loop\xspace}
\newcommand{\WhileLoops}{\textsf{while}-loops\xspace}

\newcommand{\Ell}{\ell}

\newcommand{\Id}[1]{\Mat{I}_{#1}}

\newcommand{\DistSOp}[1]{d_{\Sphere{#1}}}

\NewDocumentCommand{\Dist}{s +m +m}{d\IfBooleanTF{#1}{( #2, #3 )}{\left( #2, #3 \right)}}
\NewDocumentCommand{\DistH}{s O{} +m +m}{{%
 \def\dfmt{\bigl( #3, #4 \bigr)}
 \def\tfmt{( #3, #4 )}
 d_{H}#2
 \IfBooleanTF{#1}{\dfmt}{
    \mathchoice
     {\tfmt} % \displaystyle
     {\tfmt} % \textstyle
     {\tfmt} % \scriptstyle
     {\tfmt} % \scriptscriptstyle
}}}
\NewDocumentCommand{\DistS}{s O{n} O{} +m +m}{{%
 \def\dfmt{\bigl( #4, #5 \bigr)}
 \def\tfmt{( #4, #5 )}
 %d_{\mathcal{S}^{n-1}}#2
 \DistSOp{#2}#3
 \IfBooleanTF{#1}{\dfmt}{
    \mathchoice
     {\tfmt} % \displaystyle
     {\tfmt} % \textstyle
     {\tfmt} % \scriptstyle
     {\tfmt} % \scriptscriptstyle
}}}
\newcommand{\DistAngular}[2]{\frac{\arccos( #1,#2 )}{\pi}}

%%%%%%%%%%%%%%%%%%%%%%%%%%%%%%%%%%%%%%%%%%%%%%%%%%%%%%%%%%%%%%%%%%%%%%%%%%%%%%%%%%%%%%%%%%%%%%%%%%%%
%%%%%%%%%%%%%%%%%%%%%%%%%%%%%%%%%%%%%%%%%%%%%%%%%%%%%%%%%%%%%%%%%%%%%%%%%%%%%%%%%%%%%%%%%%%%%%%%%%%%
%%%%%%%%%%%%%%%%%%%%%%%%%%%%%%%%%%%%%%%%%%%%%%%%%%%%%%%%%%%%%%%%%%%%%%%%%%%%%%%%%%%%%%%%%%%%%%%%%%%%

\newcommand{\SetComplement}[1]{\bar{#1}}

\newcommand{\even}{\; \mathrm{even}}
\newcommand{\odd}{\; \mathrm{odd}}

\newcommand{\Vdots}{\phantom{=}\!\!\vdots}
\newcommand{\OpSp}{\phantom{=}}

\newcommand{\SectionSepDelim}{\textbf{\textperiodcentered}}
\newcommand{\SectionSep}{\SectionSepDelim \ \SectionSepDelim \ \SectionSepDelim \ \SectionSepDelim \ \SectionSepDelim}
%\newcommand{\Start}{\vspace{-8pt}\begin{center} \SectionSep \end{center}\vspace{-8pt}}
%\newcommand{\End}{\vspace{-8pt}\begin{center} \SectionSep \end{center}\vspace{-8pt}}

%%%%%%%%%%%%%%%%%%%%%%%%%%%%%%%%%%%%%%%%%%%%%%%%%%%%%%%%%%%%%%%%%%%%%%%%%%%%%%%%%%%%%%%%%%%%%%%%%%%%

\NewDocumentCommand{\LatestUpdate}{s +m}{\IfBooleanTF{#1}{(Latest update on: #2)}{(\emph{Latest update on: #2}.)}}

\newcommand{\thsp}{\hspace{0.2pt}}
\newcommand{\ththsp}{\hspace{0.1pt}}
\newcommand{\HorizLine}{\noindent\makebox[\linewidth]{\rule{\linewidth}{0.4pt}}}
\NewDocumentCommand{\Mid}{s m}{%
  \def\dfmt{\,#2\,}
  \def\tfmt{\,#2\,}
  \def\sfmt{\thsp#2\thsp}
  \def\ssfmt{\thsp#2\thsp}
  \IfBooleanTF{#1}{\tfmt}{%
    \mathchoice
      {\dfmt} % \displaystyle
      {\tfmt} % \textstyle
      {\sfmt} % \scriptstyle
      {\ssfmt} % \scriptscriptstyle
}}
\newcommand{\Left}[1]{#1}%{#1\,}
\newcommand{\Right}[1]{#1}%{\,#1}

\newcommand{\onebitcs}{1-bit compressed sensing\xspace}
\newcommand{\Onebitcs}{1-bit compressed sensing\xspace}
\newcommand{\BIHT}{BIHT\xspace}
\NewDocumentCommand{\NBIHT}{s}{\IfBooleanTF{#1}{(normalized)}{normalized} \BIHT}

\newcommand{\Restriction}[2]{
 \def\dfmt{\left. #1 \right|_{#2}}
 \def\tfmt{#1|_{#2}}
  \mathchoice
   {\dfmt} % \displaystyle
   {\tfmt} % \textstyle
   {\tfmt} % \scriptstyle
   {\tfmt} % \scriptscriptstyle
}
\NewDocumentCommand{\VIx}{+m O{}}{\Sub{#1\IfNE{#2}{,#2}}}%{^{(#2)}}
\newcommand{\MIx}[2][]{^{(#2)}}
\newcommand{\IIx}[2][]{^{(#2)#1}}
\NewDocumentCommand{\RVIx}{+m O{}}{\Sub{#1\IfNE{#2}{,#2}}}

%%%%%%%%%%%%%%%%%%%%%%%%%%%%%%%%%%%%%%%%%%%%%%%%%%%%%%%%%%%%%%%%%%%%%%%%%%%%%%%%%%%%%%%%%%%%%%%%%%%%

\newcommand{\Net}[2][]{\mathcal{C}_{#2\IfNE{#1}{;#1}}}
\newcommand{\BallNet}[2][]{\mathcal{D}_{#2\IfNE{#1}{;#1}}}
\newcommand{\Ball}[2][]{\mathcal{B}_{#2}\IfNE{#1}{^{(#1)}}}
\newcommand{\BallX}[2][]{\mathcal{B}'_{#2}\IfNE{#1}{^{(#1)}}}
\newcommand{\BallAngular}[2][]{\mathcal{B}_{\theta\leq #2}\IfNE{#1}{^{(#1)}}}%{\mathcal{B}_{\measuredangle #2}\IfNE{#1}{^{(#1)}}}
\NewDocumentCommand{\BallSparseSphere}{O{} +m D(){}}{\mathcal{B}_{#2}\IfNE{#1}{^{(#1)}}( #3 ) \cap \Sphere{n} \cap \SparseSubspace{k}{n}}
%\newcommand{\BallSparseSphere}[2][]{\mathcal{B}_{#2}\IfNE{#1}{^{(#1)}} \cap \Sphere{n} \cap \SparseSubspace{k}{n}}

\newcommand{\ThresholdOp}{T\!}%{\mathsfit{t}}%{\mathcal{T}\!\!}
%\newcommand{\ThresholdMat}{\Mat{T}\!}
%\newcommand{\ThresholdMatEntry}{\Mat*{T}\!}
\NewDocumentCommand{\Threshold}{s +m D(){}}{{%
  \def\dfmt{\left( #3 \right)}
  \def\tfmt{( #3 )}
  \ThresholdOp\Sub{#2}%
  \ifthenelse{\isempty{#3}}{}{
  \IfBooleanTF{#1}{\tfmt}{
    \mathchoice
      {\dfmt} % \displaystyle
      {\tfmt} % \textstyle
      {\tfmt} % \scriptstyle
      {\tfmt} % \scriptscriptstyle
}}}}

%\NewDocumentCommand{\ThresholdSetMat}{s +m O{}}{%
%  \IfBooleanTF{#1}{\ThresholdMatEntry_{\IfNE{#3}{#3;}#2}}{\ThresholdMat_{#2}}%
%}

\NewDocumentCommand{\ThresholdSet}{s t' O{} +m D(){}}{{%
  \def\dfmt{\left( #5 \right)}
  \def\tfmt{( #5 )}
  \ThresholdOp_{\IfBooleanF{#2}{#3}#4}%
  \ifthenelse{\isempty{#5}}{}{%
  \IfBooleanTF{#1}{\tfmt}{
    \mathchoice
      {\dfmt} % \displaystyle
      {\tfmt} % \textstyle
      {\tfmt} % \scriptstyle
      {\tfmt} % \scriptscriptstyle
}}}}
%\NewDocumentCommand{\TS}{s +m D(){}}{%
%  \ThresholdOp_{#2}%
%  \ifthenelse{\isempty{#3}}
%  {}
%  {\IfBooleanTF{#1}{( #3 )}{\left( #3 \right)}}
%}

%\NewDocumentCommand{\ThresholdSetMat}{s +m O{}}{%
%  \IfBooleanTF{#1}{\ThresholdMatEntry_{\IfNE{#3}{#3;}#2}}{\ThresholdMat_{#2}}%
%}

\newcommand{\realvalued}{real-valued\xspace}
\newcommand{\pointwise}{point-wise\xspace}
\newcommand{\coordinatewise}{coordinate-wise\xspace}
\newcommand{\entrywise}{entry-wise\xspace}
\newcommand{\orderwise}{order-wise\xspace}
\newcommand{\worstcase}{worst-case}
\newcommand{\bestcase}{best-case}
\newcommand{\avgcase}{average-case}
\newcommand{\eg}{e.g.,\xspace}
\newcommand{\ie}{i.e.,\xspace}
\newcommand{\WLOG}{without loss of generality\xspace}
\newcommand{\wrt}{with respect to}%{w.r.t.}
\newcommand{\whp}{with high probability}
\newcommand{\errorrate}{error-rate\xspace}
\newcommand{\topk}[1][k]{top-\(  #1  \)\xspace}
\newcommand{\polytime}{polynomial time\xspace}
\newcommand{\informationtheoretical}{information theoretical\xspace}
\newcommand{\informationtheoretically}{information theoretically\xspace}
\newcommand{\nonseparable}{non-separable\xspace}

\newcommand{\SA}[1]{\mathsf{SA}\Sub{#1}}
\newcommand{\SACap}[2][]{\mathsf{A}\Sub{#2\IfNE{#1}{,#1}}}
\newcommand{\BetaFunction}[2]{B( #1, #2 )}
\newcommand{\IncompleteBetaFunction}[3]{B( #1; #2, #3 )}
\newcommand{\RegularizedIncompleteBetaFunctionOp}[1]{I_{#1}}
\NewDocumentCommand{\RegularizedIncompleteBetaFunction}{s m m m}{{%
  \def\dfmt{\left( #3, #4 \right)}
  \def\tfmt{( #3, #4 )}
  \RegularizedIncompleteBetaFunctionOp{#2}
  \IfBooleanTF{#1}{\tfmt}{
    \mathchoice
      {\dfmt} % \displaystyle
      {\tfmt} % \textstyle
      {\tfmt} % \scriptstyle
      {\tfmt} % \scriptscriptstyle
}}}


\NewDocumentCommand{\RHS}{s}{\IfBooleanTF{#1}{RHS}{right-hand-side}\xspace}
\NewDocumentCommand{\LHS}{s}{\IfBooleanTF{#1}{LHS}{left-hand-side}\xspace}