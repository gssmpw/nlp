\section{Restricted Approximate Invertibility of GLMs}
\label{outline:main-technical-result}

The crux of the analysis for the convergence of \ALGORITHM \ref{alg:biht:normalized} is a variant of the restricted approximate invertibility conditions (\RAICs) studied in \cite{friedlander2021nbiht,matsumoto2022binary}, which is established for Gaussian matrices in \THEOREM \ref{thm:main-technical:sparse}.
%Its specializations to logistic and probit regression are presented in \COROLLARIES \ref{corollary:main-technical:logistic-regression} and \ref{corollary:main-technical:probit}, respectively.
The formal proofs of these technical results, which constitute the primary technical contributions of this work, are located in \APPENDIX \ref{outline:pf-main-technical-result} and overviewed in \SECTION \ref{outline:main-technical-result|outline-of-pf|outline}.



%|<<|###########################################################################################|<<|
%|<<|###########################################################################################|<<|
%|<<|###########################################################################################|<<|
%

%\par %%%%%%%%%%%%%%%%%%%%%%%%%%%%%%%%%%%%%%%%%%%%%%%%%%%%%%%%%%%%%%%%%%%%%%%%%%%%%%%%%%%%%%%%%%%%%%%
%%%%%%%%%%%%%%%%%%%%%%%%%%%%%%%%%%%%%%%%%%%%%%%%%%%%%%%%%%%%%%%%%%%%%%%%%%%%%%%%%%%%%%%%%%%%%%%%%%%%
%

%|>>|*******************************************************************************************|>>|
%|>>|*******************************************************************************************|>>|
%|>>|*******************************************************************************************|>>|
The main technical theorem will be formalized next.
\begin{theorem}
\label{thm:main-technical:sparse}
%
%Let
%|>>|:::::::::::::::::::::::::::::::::::::::::::::::::::::::::::::::::::::::::::::::::::::::::::|>>|
%\(  \ConstA, \ConstB, \ConstCOne, \ConstCTwo, \ConstCThree, \ConstCFour, \ConstCFive > 0  \)
%|<<|:::::::::::::::::::::::::::::::::::::::::::::::::::::::::::::::::::::::::::::::::::::::::::|<<|
%be absolute constants as specified in \DEFINITION \ref{def:univ-const}, and 
Fix
%|>>|:::::::::::::::::::::::::::::::::::::::::::::::::::::::::::::::::::::::::::::::::::::::::::|>>|
\(  \n, \k, \m \in \Z_{+}  \), \(  \k \leq \n  \), and \(  \rhoX, \deltaX \in (0,1)  \) where
\(
  \deltaX \defeq \frac{\epsilonX}{\frac{3}{2} ( 5+\sqrt{21} )}.
%\frac{\epsilonX}{\frac{1}{2} ( 3+\sqrt{5} )}
\)
%|<<|:::::::::::::::::::::::::::::::::::::::::::::::::::::::::::::::::::::::::::::::::::::::::::|<<|
%Set \(  \nuX = \nuX( \deltaX ) > 0  \)
% such that
% %|>>|===========================================================================================|>>|
% \begin{gather}
% \label{eqn:main-technical:sparse:nu}
%   \nuX
%   =
%   \nuXEXPR
% ,\end{gather}
%|<<|===========================================================================================|<<|
%and let \(  \tauX = \tauX( \deltaX ) > 0  \)
%be given by
%|>>|===========================================================================================|>>|
% \begin{gather}
% \label{eqn:main-technical:sparse:tau}
%   \tauX \defeq \tauXEXPR
% ,\end{gather}
%|<<|===========================================================================================|<<|
%as in \DEFINITION \ref{def:nu-and-tau}.
Write
%|>>|:::::::::::::::::::::::::::::::::::::::::::::::::::::::::::::::::::::::::::::::::::::::::::|>>|
\(  \alphaO = \alphaO( \deltaX ) \defeq \alphaOExpr[\deltaX]  \)
%|<<|:::::::::::::::::::::::::::::::::::::::::::::::::::::::::::::::::::::::::::::::::::::::::::|<<|
as in \EQUATION \eqref{eqn:notations:alpha_0:def}.
Let
%|>>|:::::::::::::::::::::::::::::::::::::::::::::::::::::::::::::::::::::::::::::::::::::::::::|>>|
\(  \ParamSpace = \SparseSphereSubspace{\k}{\n}  \),
%|<<|:::::::::::::::::::::::::::::::::::::::::::::::::::::::::::::::::::::::::::::::::::::::::::|<<|
and fix
%|>>|:::::::::::::::::::::::::::::::::::::::::::::::::::::::::::::::::::::::::::::::::::::::::::|>>|
\(  \thetaStar \in \ParamSpace  \).
%|<<|:::::::::::::::::::::::::::::::::::::::::::::::::::::::::::::::::::::::::::::::::::::::::::|<<|
Under \ASSUMPTION \ref{assumption:p}, for a number of samples 
%|>>|===========================================================================================|>>|
\begin{align}
\nonumber
  \m
  % &\geq
  % \mEXPR[s]
  % \\ \nonumber
  &=
  \mOEXPRS{\deltaX}[,]
  \\
\label{eqn:main-technical:sparse:m}
\end{align}
%|<<|===========================================================================================|<<|
 with probability at least \(  1-\rhoX  \), uniformly for all \(  \thetaXX \in \ParamSpace  \) and all \(  \JCoords \subseteq [\n]  \), \(  | \JCoords | \leq \k  \),
%|>>|===========================================================================================|>>|
\begin{gather}
\label{eqn:main-technical:sparse:1}
  \left\|
    \thetaStar
    -
    \frac
    {\thetaXX + \hfFn[\JCoords]( \thetaStar, \thetaXX )}
    {\| \thetaXX + \hfFn[\JCoords]( \thetaStar, \thetaXX ) \|_{2}}
  \right\|_{2}
  \leq
  \sqrt{\deltaX \| \thetaStar-\thetaXX \|_{2}}
  +
  \deltaX
.\end{gather}
%|<<|===========================================================================================|<<|
\end{theorem}
%|<<|*******************************************************************************************|<<|
%|<<|*******************************************************************************************|<<|
%|<<|*******************************************************************************************|<<|

%|>>|♢♢♢♢♢♢♢♢♢♢♢♢♢♢♢♢♢♢♢♢♢♢♢♢♢♢♢♢♢♢♢♢♢♢♢♢♢♢♢♢♢♢♢♢♢♢♢♢♢♢♢♢♢♢♢♢♢♢♢♢♢♢♢♢♢♢♢♢♢♢♢♢♢♢♢♢♢♢♢♢♢♢♢♢♢♢♢♢♢♢♢|>>|
%|>>|♢♢♢♢♢♢♢♢♢♢♢♢♢♢♢♢♢♢♢♢♢♢♢♢♢♢♢♢♢♢♢♢♢♢♢♢♢♢♢♢♢♢♢♢♢♢♢♢♢♢♢♢♢♢♢♢♢♢♢♢♢♢♢♢♢♢♢♢♢♢♢♢♢♢♢♢♢♢♢♢♢♢♢♢♢♢♢♢♢♢♢|>>|
%|>>|♢♢♢♢♢♢♢♢♢♢♢♢♢♢♢♢♢♢♢♢♢♢♢♢♢♢♢♢♢♢♢♢♢♢♢♢♢♢♢♢♢♢♢♢♢♢♢♢♢♢♢♢♢♢♢♢♢♢♢♢♢♢♢♢♢♢♢♢♢♢♢♢♢♢♢♢♢♢♢♢♢♢♢♢♢♢♢♢♢♢♢|>>|
% \begin{remark}
% \label{remark:main-technical:dense}
% %
% Analogously to
% %By the same argument as provided for
% \COROLLARY \ref{corollary:approx-error:dense}
% %in relation to
% of
% \THEOREM \ref{thm:approx-error:sparse}, for the dense parameter space, where \(  \k = \n  \) and \(  \ParamSpace = \Sphere{\n}  \), the sample complexity in \EQUATION \eqref{eqn:main-technical:sparse:m} of \THEOREM \ref{thm:main-technical:sparse} becomes
% %|>>|===========================================================================================|>>|
% \begin{align*}
%   \m
%   &\geq
%   \mEXPR[d]
%   \\ \nonumber
%   &=
%   \mOEXPRD{\epsilonX}[.]
% \end{align*}
% %|<<|===========================================================================================|<<|
% \end{remark}
%|<<|♢♢♢♢♢♢♢♢♢♢♢♢♢♢♢♢♢♢♢♢♢♢♢♢♢♢♢♢♢♢♢♢♢♢♢♢♢♢♢♢♢♢♢♢♢♢♢♢♢♢♢♢♢♢♢♢♢♢♢♢♢♢♢♢♢♢♢♢♢♢♢♢♢♢♢♢♢♢♢♢♢♢♢♢♢♢♢♢♢♢♢|<<|
%|<<|♢♢♢♢♢♢♢♢♢♢♢♢♢♢♢♢♢♢♢♢♢♢♢♢♢♢♢♢♢♢♢♢♢♢♢♢♢♢♢♢♢♢♢♢♢♢♢♢♢♢♢♢♢♢♢♢♢♢♢♢♢♢♢♢♢♢♢♢♢♢♢♢♢♢♢♢♢♢♢♢♢♢♢♢♢♢♢♢♢♢♢|<<|
%|<<|♢♢♢♢♢♢♢♢♢♢♢♢♢♢♢♢♢♢♢♢♢♢♢♢♢♢♢♢♢♢♢♢♢♢♢♢♢♢♢♢♢♢♢♢♢♢♢♢♢♢♢♢♢♢♢♢♢♢♢♢♢♢♢♢♢♢♢♢♢♢♢♢♢♢♢♢♢♢♢♢♢♢♢♢♢♢♢♢♢♢♢|<<|

The main technical theorem holds for logistic and probit regression---as formalized below in \COROLLARY \ref{corollary:main-technical:logistic-regression}% %and \ref{corollary:main-technical:probit}, respectively---
---because \ASSUMPTION \ref{assumption:p} is satisfied by both models.
Moreover,
%similarly to \COROLLARIES \ref{corollary:approx-error:logistic-regression} and \ref{corollary:approx-error:probit},
closed-form bounds on the sample complexity in \THEOREM \ref{thm:main-technical:sparse} can be derived for these two
%examples of
exemplary
GLMs, which are stated as \orderwise results in \COROLLARY \ref{corollary:main-technical:logistic-regression} %and \ref{corollary:main-technical:probit},
with the specification of precise bounds and constants left to the proof of the corollary in \APPENDIX \ref{outline:pf-main-technical-result|pf-main-corollaries}.

%|>>|*******************************************************************************************|>>|
%|>>|*******************************************************************************************|>>|
%|>>|*******************************************************************************************|>>|
\begin{corollary}
\label{corollary:main-technical:logistic-regression}
%
%Let \(  \cO, \bO, \ConstbetaXThrsholdLR > 0  \) be positive constants such that
%|>>|:::::::::::::::::::::::::::::::::::::::::::::::::::::::::::::::::::::::::::::::::::::::::::|>>|
%\(  \cO \defeq \frac{\sqrt{\hfrac{8}{\pi}}}{1+\bO} \geq 1  \)
%|<<|:::::::::::::::::::::::::::::::::::::::::::::::::::::::::::::::::::::::::::::::::::::::::::|<<|
%and
%|>>|:::::::::::::::::::::::::::::::::::::::::::::::::::::::::::::::::::::::::::::::::::::::::::|>>|
%\(  \ConstbetaXThrsholdLR \defeq \ConstbetaXThrsholdValueLR  \).
%|<<|:::::::::::::::::::::::::::::::::::::::::::::::::::::::::::::::::::::::::::::::::::::::::::|<<|
%as per \DEFINITION \ref{def:univ-const}.
Let \(  \pFn  \) be the logistic function with \betaXnamelr \(  \betaX \GTR 0  \), as in \DEFINITION \ref{def:p:logistic-regression} (or the probit function with SNR \(  \betaX \GTR 0  \), as in \DEFINITION \ref{def:p:probit}).
If there exist absolute constants \(  \cO, \ConstbetaXThrsholdLR > 0  \) then for a number of samples given by
 \EQUATION \eqref{eqn:corollary:approx-error:logistic-regression:m:sparse},
%|>>|===========================================================================================|>>|
% \begin{gather}
%   \m = \mOEXPRLR[s]{\epsilonX}[,]
% \end{gather}
%|<<|===========================================================================================|<<|
 the bound stated as \EQUATION \eqref{eqn:main-technical:sparse:1} in \THEOREM \ref{thm:main-technical:sparse} holds uniformly for all \(  \thetaXX \in \ParamSpace  \) and all \(  \JCoords \subseteq [\n]  \), \(  | \JCoords | \leq \k  \), with probability at least \(  1-\rhoX  \).
\end{corollary}
%|<<|*******************************************************************************************|<<|
%|<<|*******************************************************************************************|<<|
%|<<|*******************************************************************************************|<<|

%|>>|*******************************************************************************************|>>|
%|>>|*******************************************************************************************|>>|
%|>>|*******************************************************************************************|>>|
% \begin{corollary}
% \label{corollary:main-technical:probit}
% %
% Let \(  \pFn  \) be defined as in \DEFINITION \ref{def:p:probit} for probit regression with \betaXnamepr \(  \betaX \GTR 0  \).
% If there exist absolute constants  \(  \ConstbetaXThresholdPROne, \ConstbetaXThresholdPRTwo > 0  \) then for a number of samples given by \EQUATION \eqref{eqn:corollary:approx-error:probit:m:sparse},
% % if
% % %|>>|===========================================================================================|>>|
% % \begin{gather}
% %   \m = \mOEXPRPR[s]{\epsilonX}[,]
% % \end{gather}
% %|<<|===========================================================================================|<<|
%  the bound stated as \EQUATION \eqref{eqn:main-technical:sparse:1} in \THEOREM \ref{thm:main-technical:sparse}
% %for \(  \ParamSpace = \SparseSphereSubspace{\k}{\n}  \)
 %holds uniformly for all \(  \thetaXX \in \ParamSpace  \) and all \(  \JCoords \subseteq [\n]  \), \(  | \JCoords | \leq \k  \), with probability at least \(  1-\rhoX  \).
%\end{corollary}
%|<<|*******************************************************************************************|<<|
%|<<|*******************************************************************************************|<<|
%|<<|*******************************************************************************************|<<|