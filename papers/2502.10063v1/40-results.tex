\section{Results}


In this section, we evaluate example implementations of the proposed \smm architectures.
In \secn{smm:sec:results-baseline}, we compare the \smm MXU architectures in isolation against our baseline MXU designs in Table \ref{smm:tab:smm-MM}.
In \secn{smm:sec:results-prior-work} we evaluate the \smm MXU architectures in Tables \ref{smm:tab:second}-\ref{smm:tab:last} compared to prior work when integrated into an end-to-end deep learning accelerator system based on the system from our previous work \cite{pogue2024fast}, which has open-source code \cite{ffip-source}.
We first describe the integration of our proposed systolic arrays into the deep learning system in \secn{sec:system}, and in Sections \ref{sec:mu} and \ref{smm:sec:mat-sz}, we define performance metrics used to compare the \smmArch architectures against baseline designs and prior works.

\subsection{System Integration}
\label{sec:system}
We were able to integrate the \smm architectures into a system based on our previous work \cite{pogue2024fast}, which has open-source code \cite{ffip-source}, by swapping the \smmArch MXU architectures from \fg{smm:fig:smm-mxu} into our system design \cite{pogue2024fast} in place of the free-pipeline fast inner-product (FFIP) MXU.

In order to perform GEMM on the proposed MXUs and multiply matrices of arbitrary sizes that can be larger than the MXU dimensions, the full \AB matrices are first divided into \gemmtiles prior to being divided further into smaller \blocks for executing \eq{smm:eq:mm} or \seq.
The \gemmtiles are then fed into the MXU one-by-one.
Each \gemmtile is then considered as the full \AB matrix from \eq{smm:eq:mm} or \seq while being fed into the MXU and gets further divided into smaller \m.A?{ij}/\m.B?{ji} \blocks within the MXU.

Following each \gemmtile multiplication, the partial \gemmtile products are accumulated outside of the MXU to generate each final \gemmtile product.
Prior to each \gemmtile multiplication, a $\bld{B}$ \gemmtile is loaded into the MXU.
It then remains in place as the $\bld{A}$ \gemmtile flows through the MXU producing the \gemmtile product, during which a new \Av vector is fed into the MXU each clock cycle.
Additionally, to hide the latency of loading $\bld{B}$ \gemmtiles, the MXU PEs each contain one extra $b$ buffer to load the next $\bld{B}$ \gemmtile into the MXU as the current \gemmtile is being multiplied.

Each \A, \B, and \C \subblock entering or exiting the top-level MXU for the \smm and baseline MXUs first pass through triangular-shaped register arrays each containing $X$ shift registers of varying depths.
Here, each shift register $SR_k$ has a depth of $k$ and loads one $a_{i,k}$ or $b_{k,j}$ element per clock cycle.
These triangular buffers are explained further in our prior work \cite{pogue2024fast} and they allow the vector elements to enter the MXU in the necessary order as depicted in the element indices in \figs \ref{smm:fig:MM-mxu} and \ref{smm:fig:pes}.

\subsection{Multiplier Compute Efficiency}
\label{sec:mu}
In this subsection, we define an efficiency metric called the multiplier compute efficiency (MCE) in \eq{smm:eq:mu} which we use to compare the \smmArch architectures against baseline designs and prior works.
This is used to quantify how much the algebraic optimizations exploited in an architecture reduce the computational complexity.
Reductions in computational complexity allow an architecture to utilize its multipliers more effectively than conventional designs using no algebraic optimizations.
The multiplier compute efficiency is defined as follows:
\begin{align}  \label{smm:eq:mu}
  \tx{MCE} = \frac{\tx{mults} / \tx{multiplier}} {\tx{clock cycle}}
  &= \frac{(\tx{mults/s})/\tx{\#multipliers}}{f}
  \,.
\end{align}
Here, mults/s above is measured by taking the number of multiplications required to carry out an execution using conventional algebra and dividing it by the measured execution time.
Finally, \#multipliers is the number of instantiated multipliers in the design, and $f$ is the clock frequency that the hardware design is operating at.

Conventional matrix multiplication algorithms such as \eq{smm:eq:mm} have no algebraic optimizations for reducing the computational complexity.
Therefore, the limit/maximum achievable value (also referred to as the roof) of the metric in \eq{smm:eq:mu} is the following when using conventional matrix multiplication in hardware:
\begin{align}  \label{smm:eq:mm-mu-roof}
  \tx{roof}\lb\tx{MCE}_{\tx{\mm?r}}\rb &= 1
  \,.
\end{align}
In contrast, \sa requires $8^r/7^r$ times fewer multiplications than a conventional matrix multiplication algorithm, where $r$ is the number of levels of recursion implemented in \sa.
Therefore, the multiplier compute efficiency can reach the following limit in \smmArch architectures:
\begin{align} \label{smm:eq:mu-roof}
  \tx{roof}\lb\tx{MCE}_{\tx{\smm?r}}\rb &= 
  \lb\frac{8}{7}\rb^{r}
  \,.
\end{align}

As discussed in \secn{smm:sec:background}, \sa reduces the overall number of operations in matrix multiplication.
Furthermore, any additions required before the matrix multiplications in the algorithm are even less of a concern in fixed-point implementations.
This is because the hardware complexity of fixed-point multipliers typically scale quadratically with the input bitwidth compared to linearly for adders and registers \multComplexity, causing the hardware footprint of multipliers to dominate that of adders and registers.

However, one of the impediments of using \sa for fixed-point implementations is that the bitwidths of the multiplication inputs increase by $r$ bits for $r$ levels of Strassen recursion that are implemented, reducing its potential area savings for custom fixed-point hardware designs.
Nonetheless, this impediment for fixed-point designs can be inherently mitigated in FPGA implementations so long as $r$ plus the initial input width is not larger than the maximum input width supported by the FPGA's DSP units.
For example, each DSP in common Intel/Altera FPGAs instantiate two 18\by19-bit multipliers \cite{intel-dsp}, and common input bitwidths for applications such as deep learning are 16 bits or less.
This leaves room for at least 2 or more levels of Strassen recursion to be implemented before surpassing the bitwidth limit supported by the DSPs.

Furthermore, due to the flexible nature of custom hardware design, the \smm?r architectures can be efficiently mapped onto other DSP units in general which support input bitwidths up to $n$ bits by customizing the input datapath bitwidth $w$ and value of $r$ as necessary to ensure that $w+r \le n$.
So long as the accuracy requirements of the application are still met, this will allow the \smm?r designs and their increase in multiplier bitwidth to still be efficiently mapped onto DSP units of any bitwidth in a general way.

\subsection{Supporting Smaller Matrices with the Same Performance}
\label{smm:sec:mat-sz}
Multisystolic array designs such as the \smmArch and baseline \mm?r architectures have the ability to efficiently multiply smaller matrices than a single-systolic array design with the same performance capability.
By executing \eq{smm:eq:mm} or \seq fully in parallel for $r$ levels of recursion, matrix products of size as small as $n\times n$ can be computed up to once every $n/2^r$ clock cycles in an \mm?r or \smmArch multisystolic array design.
Furthermore, these matrix products require $n^3$ multiplications to calculate using conventional algebra.
Therefore, the ratio of an architecture's throughput per clock cycle versus its smallest supported matrix sizes it can multiply, which we refer to as the matrix size efficiency (MSE), is the following:
\begin{align} \label{smm:eq:mat-sz}
  \tx{MSE} = 
  \frac{\tx{mults}/\tx{clock cycle}} {\tx{min. mat. size (h\by w)}}
  \,,
\end{align}
which has the following roof for multisystolic arrays:
\begin{align} \label{smm:eq:mat-sz}
  \tx{roof}\lb\tx{MSE}_{\tx{(S)MM}_r}\rb
  = \frac {n^3/\lb n/2^r\rb} {n\times n} = 2^r
  \,.
\end{align}

In contrast, a single-systolic array design can produce matrix products of size as small as $n\times n$ up to once every $n$ clock cycles, making this ratio the following for a single-systolic array design:
\begin{align}
  \tx{roof}\lb\tx{MSE}_{\tx{MM}}\rb
  = \frac {n^3/ n} {n\times n} = 1
  \,.
\end{align}
This shows that the \smmArch and baseline \mm?r multisystolic array designs can efficiently multiply matrices $2^r$ times smaller than a single-systolic array architecture with the same performance capability.

As discussed in \secn{multi-sys}, this is an important property for increasing a systolic array accelerator's maximum achievable throughput on real-life workloads.
Even if more compute resources are instantiated to scale up the size of the systolic array, the systolic array will begin to be underutilized after its size surpasses the workload's matrix sizes.
This is particularly true in modern workloads such as deep learning acceleration, where the matrix sizes that the workloads break down to can be smaller than the maximum systolic array size that could be instantiated in an accelerator \citeMultiSys.
In \secn{smm:sec:results-prior-work}, we demonstrate how this property allowed us to scale up our deep learning accelerator design without compromising utilization to achieve state-of-the-art ResNet \cite{kaiming2016deep} throughput.



\subsection{Comparison to Baseline Designs}
\label{smm:sec:results-baseline}
\newcolumntype{D}{>{\centering\arraybackslash}p{1.98cm}}
\newcolumntype{E}{>{\centering\arraybackslash}p{1.8cm}}
\begin{table*}\centering
\caption{Comparing \smmArch multisystolic array architectures against the baseline \mmArch?\zero single-systolic array architecture and baseline \mmArch multisystolic array architectures in isolation (without integration into a deep learning accelerator system).}
\label{smm:tab:smm-MM}
\label{smm:tab:first}
\scriptsize
\begin{threeparttable}
  \begin{tabular}{|>{\raggedleft}p{3.3cm}|D|DD|DDE|}\toprule
\arrayrulecolor{black}
&\alg.\MM?\zero<{} 48\by48 &\alg.\MM?1<{} 16\by16  &\alg.\SMM?1<{} 16\by16 &\alg.\MM?2<{} 6\by6 &\alg.\SMM?2<{} 6\by6 &\alg.\SMM?2<{} 6\by6 (with extra regs.)   \\
\toprule
DSPs                                 &1,152                     &1,024                 &\textbf{896}           &1,152               &\textbf{882}         &\textbf{882}            \\
\arrayrulecolor{black!30}\midrule
\arrayrulecolor{black}
ALMs                                 &34,890                    &30,872                &30,265                 &36,397              &35,863               &38,485                  \\
\arrayrulecolor{black!30}\midrule
\arrayrulecolor{black}
Registers                            &130,262                   &118,049               &115,830                &138,219             &133,511              &147,750                 \\
\arrayrulecolor{black!30}\midrule
\arrayrulecolor{black}
Frequency (MHz)                      &399                       &398                   &380                    &388                 &291                  &361                     \\
\arrayrulecolor{black}\midrule
$\tx{roof}\lb\tx{Throughput}\rb$ (GOPS) \tnote{1}     &1839     &1630                  &1556                   &1788                &1341                 &1663                    \\
\arrayrulecolor{black!30}\midrule
Throughput/DSP \tnote{2}                       &1.60            &1.59                  &\textbf{1.74}                   &1.55                &1.52                 &\textbf{1.89}                    \\
\arrayrulecolor{black!30}\midrule
\macUtA \tnote{3}                              &1               &1                     &\textbf{1.14}          &1                   &\textbf{1.31}        &\textbf{1.31}           \\
\arrayrulecolor{black}\midrule
Min. supported matrix size \tnote{4} &48\by48                   &32\by32               &32\by32                &24\by24             &24\by24              &24\by24                 \\
\arrayrulecolor{black!30}\midrule
\minSzMetric \tnote{5}               &1                         &2                     &2                      &4                   &4                    &4                       \\
\arrayrulecolor{black}
\bottomrule
\end{tabular}
  \begin{tablenotes}
    \item All designs are synthesized on Arria 10 GX 1150 FPGA for 16-bit fixed-point inputs and consume 0 memory resources.
\item[1] Maximum achievable throughput in giga operations per second, where throughput is equal to the number of operations required to carry out an execution using conventional algebra divided by the measured execution time.
\item[2] Shows which designs can achieve the highest throughput for the same number of DSPs.
\item[3] Maximum achievable multiplier compute efficiency, defined in \secn{sec:mu}, measures how effectively the architecture can utilize its multipliers. It can surpass 1 in \smm architectures because the observed mults/s is equal to the number of multiplications required to carry out an execution using conventional algebra divided by execution time.
\item[4] Minimum input matrix sizes that can be multiplied at peak throughput/full utilization.
\item[5] \matSzExpl
\end{tablenotes}
\end{threeparttable}
\end{table*}


Table \ref{smm:tab:smm-MM} shows the resource usage and performance comparison between the proposed \smmArch and baseline \mm{/}\mmArch systolic array architectures in isolation (without integration into a deep learning accelerator system).
The \smmArch?1 and \smmArch?2 architectures overall have a similar amount of soft logic resources and the same throughput per clock cycle roof as the \mmArch?1 and \mmArch?2 architectures, respectively, but they require \textbf{1.14-1.31\x fewer DSP units}.
Compared to the multisystolic array \mm?1 and \mm?2 designs, the \smmArch?1 and \smmArch?2 architectures are also functionally equivalent, respectively, other than having a lower clock frequency.
To help mitigate the limitation of having a lower frequency, we added an extra \smmArch?2 design (which had the biggest issue with clock frequency) on the far right of Table \ref{smm:tab:smm-MM} containing additional pipelining registers in the addition logic of each Q Addition Vectors unit from \fig{smm-add-units} (c).
This extra design demonstrates how a trade-off can be optionally made to increase the design's clock frequency at the cost of some extra soft logic resources.

Nonetheless, the lower clock frequencies of the \smm?r designs in Table  \ref{smm:tab:smm-MM} are compensated by the fact that the \smm?r designs achieve more effective operations from the same number DSP units.
Since the reduction in DSP units is greater than the reduction in clock frequency in the \smm?1 design and \smm?2 design with extra registers relative to their \mm?r counterparts, they would be able to achieve a higher overall throughput if scaled up in size to use the same number of DSPs.
This is shown by the Throughput/DSP metric in Table  \ref{smm:tab:smm-MM}, which shows that the \smm?r designs achieve up to \textbf{22\% more throughput per DSP} than their \mm?r counterparts.
Finally, if the frequency-limiting critical path is in external control or other logic outside of the systolic array after integrating it into an end-to-end accelerator system, as was the case in our full-system accelerators from Tables \ref{smm:tab:8}-\ref{smm:tab:ffip}, this limitation of a lower frequency is further mitigated.

\begin{figure}[]
  \hspace*{-0.25cm}
  \centering
  \includegraphics[scale=0.0525]{smm-mult-compute-efficiency.eps}
  \vspace*{-0.6cm}
  \caption{Plotting the multiplier compute efficiency of the architectures in Table \ref{smm:tab:smm-MM} when multiplying different input matrices of size $n$\by$n$.
    As shown, the minimum matrix size that can be efficiently multiplied gets smaller in the order of the \mm?\zero{,} \mm?1{/}\smm?1{,} and \mm?2{/}\smm?2 designs, respectively.}
  \label{smm:fig:perf}
\end{figure}

The throughput per clock cycle roof of the \mm?\zero and \mm?2 baseline designs in Table \ref{smm:tab:smm-MM} are equal and they consume the same number of DSP resources, but the \mm?\zero design requires slightly fewer ALM and register resources.
However, this penalty may be justified in the \mm?2 design when considering that the minimum matrix size (height\by width) that can be multiplied while fully utilizing the MXU is 4$\times$ smaller in the \mm?2 design compared to the \mm?\zero design.
This increases its performance scalability for accelerating modern workloads such as deep learning as discussed in \secn{multi-sys} and \ref{smm:sec:mat-sz}.
This ability of the multisystolic array designs to more efficiently multiply smaller matrices is further illustrated in \fig{smm:fig:perf}.
This same property is true for the \smmArch?2 design, except it achieves this with fewer DSP resources.
This benefit is demonstrated in \secn{smm:sec:results-prior-work}, where this property allowed us to scale up our deep learning accelerator design without compromising utilization to achieve state-of-the-art ResNet throughput.


\subsection{Comparison to Prior Work}
\label{smm:sec:results-prior-work}

Full system-level validation of the experimental accelerator as integrated into the system from our previous work \cite{pogue2024fast} has been done on an Arria 10 SoC Developement Kit \cite{sx-dev-kit} containing the \sx device by measuring throughput in real-time.
However, this device contains fewer soft logic resources than the \gx used in the prior works we compare against, and we generate compilation results for our design on the same \gx device used in prior works for a more fair and consistent comparison.
Throughput values of our designs on the \gx device are then calculated using an accurate throughput estimation model based on our highly deterministic and time-predictable system implementation, which accurately predicts actual throughputs measured on the \sx device available to us.
Tables \ref{smm:tab:second}-\ref{smm:tab:last} show throughputs for ResNet \cite{kaiming2016deep} neural networks.

The works from Liu \ea \cite{liu2022toward} and Fan \ea \cite{fan2022fpga} in Table \ref{smm:tab:8} use a technique to pack two 8-bit multiplications onto each 18\by19-bit multiplier in the DSPs and additional ALMs, and therefore the number of multipliers is calculated as \#DSP\by4 in those works.
The number of multipliers in the works \cite{huang2022fpga}, \cite{kim2023agamotto} from Table \ref{smm:tab-mixed} is calculated as \#DSPs since they are implemented on AMD FPGAs where each DSP instantiates one 18\by27-bit multiplier \cite{amd-dsp}.
In Tables \ref{smm:tab:8} and \ref{smm:tab-mixed}, the number of multipliers in the prior works \cite{an2022opencl}, \cite{dai2024dcp} is equal to \#DSPs\by2, where each DSP in the Altera FPGAs instantiates two 18\by19-bit multipliers \cite{intel-dsp}.
The number of multipliers used in the MXUs from our architectures in Tables \ref{smm:tab:first}-\ref{smm:tab:second} is equal to $8^r$ or $7^r$ times $X$\by$Y$ for the \mmArch and \smmArch MXUs, respectively.
For example, an \mm?\zero 64\by64 MXU (meaning $r = 0$ and $X = Y = 64$) would contain $8^0\times64^2$ multipliers, an \mm?1 32\by32 MXU would contain $8^1\times32^2$ multipliers, and an \smmArch?2 8\by8 MXU would contain $7^2\times8^2$ multipliers.
Due to the FFIP reduction in multipliers as described in our prior work \cite{pogue2024fast}, the number of multipliers for the FFIP architectures in Table \ref{smm:tab:ffip} is equal to $8^r$ or $7^r$ times $X\times Y/2 + X/2$ for the FFIP and FFIP+\smmArch designs, respectively.
Additionally, for our deep learning accelerator implementations in Tables \ref{smm:tab:second}-\ref{smm:tab:ffip}, there are an additional $Y\times4^r$ multipliers located outside the MXU in the Post-GEMM Unit \cite{pogue2024fast} for performing inter-layer quantization rescaling functions.
For our designs requiring more than 3036 multipliers, 3036 are instantiated on 1518 DSPs, and the remainder are instantiated in soft logic resources as the DSP resources are fully utilized.

\label{smm:sec:results-prior-work}
\begin{table*}\centering
  \caption{\smmArch multisystolic array architectures integrated into a deep learning accelerator system compared with prior state-of-the-art deep learning accelerators.}
\label{smm:tab:8}
\label{smm:tab:second}
\scriptsize
\begin{threeparttable}
  \begin{tabular}{|>{\raggedleft}p{2.2cm}|AA|AA|AA||AAA|AAA|}\toprule
    \arrayrulecolor{black}
    &\multicolumn{2}{c|}{TNNLS '22 \cite{liu2022toward}} &\multicolumn{2}{c|}{TCAD '22 \cite{fan2022fpga}} &\multicolumn{2}{c||}{Entropy '22 \cite{an2022opencl}} &\multicolumn{3}{c|}{\alg.\SMM?1<{} 32\by32} &\multicolumn{3}{c|}{\alg.\SMM?2<{} 8\by8}                                                           \\
\toprule
DSPs                               &\multicolumn{2}{c|}{1473}                           &\multicolumn{2}{c|}{1473}                        &\multicolumn{2}{c||}{1503}                           &\multicolumn{3}{c|}{1518}               &\multicolumn{3}{c|}{1518}              \\
\arrayrulecolor{black!30}\midrule
\arrayrulecolor{black}
ALMs                               &\multicolumn{2}{c|}{304K}                           &\multicolumn{2}{c|}{304K}                        &\multicolumn{2}{c||}{303K}                           &\multicolumn{3}{c|}{306K}               &\multicolumn{3}{c|}{145K}              \\
\arrayrulecolor{black!30}\midrule
\arrayrulecolor{black}
Registers                          &\multicolumn{2}{c|}{889K}                           &\multicolumn{2}{c|}{890K}                        &\multicolumn{2}{c||}{-}                              &\multicolumn{3}{c|}{641K}               &\multicolumn{3}{c|}{386K}              \\
\arrayrulecolor{black!30}\midrule
\arrayrulecolor{black}
Memories                           &\multicolumn{2}{c|}{2334}                           &\multicolumn{2}{c|}{2334}                        &\multicolumn{2}{c||}{1953}                           &\multicolumn{3}{c|}{2713}               &\multicolumn{3}{c|}{2036}              \\
\arrayrulecolor{black!30}\midrule
\arrayrulecolor{black}
Frequency (MHz)                    &\multicolumn{2}{c|}{200}                            &\multicolumn{2}{c|}{220}                         &\multicolumn{2}{c||}{172}                            &\multicolumn{3}{c|}{293}                &\multicolumn{3}{c|}{295}               \\
\arrayrulecolor{black!30}\midrule
\arrayrulecolor{black}
Model                              &ResNet-50      &VGG 16                               &Bayes ResNet-18  &Bayes VGG 11                    &R-CNN (ResNet-50) &R-CNN (VGG 16)                     &ResNet-50 &ResNet-101 &ResNet-152       &ResNet-50 &ResNet-101 &ResNet-152                  \\
\arrayrulecolor{black}\midrule
\tabThpt                           &1519           &1295                                &1590             &534                            &719               &865                               &3750      &4116       &4276             &2024      &2115       &2158                       \\
\arrayrulecolor{black!30}\midrule
\arrayrulecolor{black}
\arrayrulecolor{black}
\smmMacUt                             &0.645          &0.550                               &0.639            &0.206                          &0.696             &0.837                             &0.877     &0.963      &1.002            &1.051     &1.098      &1.120         \\
\arrayrulecolor{black}
\bottomrule
\end{tabular}
\begin{tablenotes}
\item All designs are synthesized on Arria 10 GX 1150 FPGA for 8-bit fixed-point inputs.
\item[1] Throughput in giga operations per second, equal to the number of operations required to carry out an execution using conventional algebra divided by execution time.
\item[2] \smmMacUtExpl
\end{tablenotes}
\end{threeparttable}
\end{table*}

\begin{table*}\centering
  \caption{Comparison of an FFIP \cite{pogue2024fast} single-systolic array architecture, which doubles performance per MAC unit, with combined FFIP+\smmArch multisystolic array architectures when integrated into deep learning accelerator systems.}
\label{smm:tab:ffip}
\label{smm:tab:last}
\scriptsize
\begin{threeparttable}
  \begin{tabular}{|>{\raggedleft}p{2.2cm}|AAA||AAA|AAA|}\toprule
\arrayrulecolor{black}
                                   &\multicolumn{3}{c||}{TC '24 \cite{pogue2024fast} (FFIP 64\by64)}            &\multicolumn{3}{c|}{\alg.\sffip?1<{} 32\by32}              &\multicolumn{3}{c|}{\alg.\sffip?2<{} 8\by8}                                              \\
\toprule
DSPs                               &\multicolumn{3}{c||}{\eDSPs}                                                &\multicolumn{3}{c|}{1518}                               &\multicolumn{3}{c|}{946}                                                            \\
\arrayrulecolor{black!30}\midrule
\arrayrulecolor{black}
ALMs                               &\multicolumn{3}{c||}{\eALMs}                                                &\multicolumn{3}{c|}{216K}                               &\multicolumn{3}{c|}{165K}                                                            \\
\arrayrulecolor{black!30}\midrule
\arrayrulecolor{black}
Registers                          &\multicolumn{3}{c||}{\eRegs}                                                &\multicolumn{3}{c|}{627K}                               &\multicolumn{3}{c|}{463K}                                                         \\
\arrayrulecolor{black!30}\midrule
\arrayrulecolor{black}
Memories                           &\multicolumn{3}{c||}{\eMems}                                                &\multicolumn{3}{c|}{2713}                               &\multicolumn{3}{c|}{2036}                                                            \\
\arrayrulecolor{black!30}\midrule
\arrayrulecolor{black}
Frequency (MHz)                    &\multicolumn{3}{c||}{\eFreq}                                                &\multicolumn{3}{c|}{313}                                &\multicolumn{3}{c|}{297}                                                           \\
\arrayrulecolor{black!30}\midrule
\arrayrulecolor{black}
Model                              &ResNet-50            &ResNet-101          &ResNet-152                      &ResNet-50       &ResNet-101     &ResNet-152             &ResNet-50       &ResNet-101     &ResNet-152              \\
\arrayrulecolor{black}\midrule
\tabThpt                           &\eResNetAGOPS        &\eResNetBGOPS  &\eResNetCGOPS                        &4006            &4397           &4568                   &2038            &2130           &2172            \\
\arrayrulecolor{black!30}\midrule
\smmMacUt                             &1.521  &1.655 &1.707                                                   &1.674  &1.837 &1.908            &1.813  &1.895 &1.933 \\
\arrayrulecolor{black}
\bottomrule
\end{tabular}
  \begin{tablenotes}
  \item All designs are synthesized on Arria 10 GX 1150 FPGA for 8-bit fixed-point inputs.
  \footNoteRefs
\end{tablenotes}
\end{threeparttable}
\end{table*}

\newcolumntype{V}{>{\centering\arraybackslash}p{1.4cm}}
\newcolumntype{W}{>{\centering\arraybackslash}p{1.2cm}}
\begin{table}[]\centering
\caption{State-of-the-art deep learning accelerators implemented on other FPGA families for similar neural networks and input bitwidths to provide further comparison with Tables \ref{smm:tab:8} and \ref{smm:tab:ffip}.}
\label{smm:tab-mixed}
\scriptsize
\begin{threeparttable}
  \begin{tabular}{|>{\raggedleft}p{2.5cm}|V|V|W|}\toprule
    \arrayrulecolor{black}
                                          &TNNLS '22 \cite{huang2022fpga} &TCAS-I '23 \cite{kim2023agamotto} &TCAD '24 \cite{dai2024dcp} \\
\toprule
FPGA                                       &AMD VX980                         &AMD XCV U9P                            &Altera Stratix 10 GX650                                              \\
\arrayrulecolor{black!30}\midrule
\arrayrulecolor{black}
Fixed-point input bitwidth                 &8/16  \tnote{3}               &8                                  &8                                                     \\
\arrayrulecolor{black!30}\midrule
\arrayrulecolor{black}
DSPs                                       &3121                          &2048                               &1024                                                  \\
\arrayrulecolor{black!30}\midrule
\arrayrulecolor{black}
ALMs (Altera) / LUTs (AMD)                 &480K                          &-                                  &152K                                                  \\
\arrayrulecolor{black!30}\midrule
\arrayrulecolor{black}
Registers                                  &-                             &-                                  &567K                                                     \\
\arrayrulecolor{black!30}\midrule
\arrayrulecolor{black}
Memories  (20Kb Altera) / (36Kb AMD)    &1457                          &-                                     &2056                                                  \\
\arrayrulecolor{black!30}\midrule
\arrayrulecolor{black}
Frequency (MHz)                            &100                           &200                                &200                                                   \\
\arrayrulecolor{black!30}\midrule
\arrayrulecolor{black}
Model                                      &ResNet-101                    &ResNet-50                          &ResNet-152                                            \\
\arrayrulecolor{black}\midrule
\tabThpt                                   &600                           &287                                &794                                                   \\
\arrayrulecolor{black!30}\midrule
\arrayrulecolor{black}
\smmMacUt                                  &0.961                         &0.351                              &0.969                                                 \\
\arrayrulecolor{black}
\bottomrule
\end{tabular}
\begin{tablenotes}
  \footNoteRefs
\item[3] Weights are quantized to 8 bits and layer input/output is quantized to 8 or 16 bits at different stages.
\end{tablenotes}
\end{threeparttable}
\end{table}


Tables \ref{smm:tab:8}-\ref{smm:tab:ffip} show the \smmArch architectures  integrated into the deep learning system from our previous work \cite{pogue2024fast} compared to state-of-the-art accelerators evaluated on the same FPGA family for the same input bitwidths and similar neural network models.
Integrating the \smmArch multisystolic array design into our deep learning accelerator allowed us to increase the multiplier compute efficiency while also scaling up the computational resources and throughput roof without increasing the minimum supported matrix sizes.
This allowed it to significantly surpass the throughput in our prior work \cite{pogue2024fast} and other state-of-the-art prior works evaluated on the same FPGA family as shown in Tables \ref{smm:tab:8}-\ref{smm:tab:ffip}.
If the design is scaled up using a single-systolic array, the minimum supported matrix size increases and compute resources begin to be underutilized for ResNet execution based on the smaller matrix sizes that its workload decomposes to.
This causes the effective throughput to not increase well despite the design having a larger throughput roof.

The \smmArch?1 32\by32 and \sAlg.\sffip<{}?1 32\by32 designs consume noticeably more memory resources than the \smmArch?2 8\by8 and \sAlg.\sffip<{}?2 8\by8 designs.
However, it is worth noting that this is not due to increased memory requirements, but rather is due to the compiler favouring to swap some register resources for memory resources.
This is because the \smmArch?1 32\by32 and \sAlg.\sffip<{}?1 32\by32 designs have a higher register (and overall area) overhead than the \smmArch?2 8\by8 and \sAlg.\sffip<{}?2 8\by8 designs in order to achieve higher throughput roofs.

In Table \ref{smm:tab:8}, the \smmArch architectures achieve the highest throughput and multiplier compute efficiency compared to the prior works.
The \smmArch?1 and \smmArch?2 architectures' multiplier compute efficiencies in Table \ref{smm:tab:8} approach their limits of 1.14 and 1.31 that are derived in \eq{smm:eq:mu-roof}.
This surpasses the limit of 1 of the baseline \mmArch architectures and prior works that is derived in \eq{smm:eq:mm-mu-roof}, validating \smm's ability to increase multiplier compute efficiency and reduce computational complexity as expected from our analysis.

Table \ref{smm:tab:ffip} shows an example of how \smm can be combined with other algebraic techniques to further increase multiplier compute efficiency limits.
FFIP \cite{pogue2024fast} provides a way to reduce the number
of required multiplications by up to a factor of 2, trading half the multiplications for cheap low-bitwidth additions.
Because of this, the limit for the multiplier compute efficiency metric in \eq{smm:eq:mu} for an FFIP architecture becomes 2, and $2\times(8/7)^r$ for a combined FFIP+\smmArch architecture.
In Table \ref{smm:tab:ffip}, we evaluate architectures that combine FFIP+\smmArch by instantiating \smmArch MXUs that use FFIP MXUs at their lowest level of recursion instead of the conventional \mm?\zero MXUs from \fig{smm:fig:MM-mxu}.
This further increases multiplier compute efficiency compared to a standalone \smmArch or standalone FFIP MXU as seen in the achieved multiplier compute efficiencies of the \sAlg.\sffip<{}?{r} architectures listed in Table \ref{smm:tab:ffip}.
