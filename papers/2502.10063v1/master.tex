\documentclass[journal]{IEEEtran}
\usepackage{lipsum,afterpage,refcount}
\usepackage{tablefootnote}
\usepackage{siunitx}
\usepackage[T1]{fontenc}
\usepackage{booktabs}
\usepackage{graphicx}
\usepackage{fancyhdr}
\usepackage{setspace}
\usepackage{url}
\usepackage{hyperref}
\usepackage{amsthm}
\usepackage{bm}
\usepackage{mathtools}
\usepackage{flexisym}
\usepackage{multirow}
\usepackage[flushleft]{threeparttable}
\usepackage{color}
\usepackage[table]{xcolor}
\usepackage{array}
\usepackage{stfloats}
\usepackage{changepage}
\usepackage{algpseudocode}
\usepackage{amssymb}
\usepackage{enumitem}
\usepackage{verbatim}
\usepackage{amsmath}
\usepackage{algorithm}
\usepackage{tikz}
\usepackage{bbding}
\usepackage{xspace}
\usepackage{academicons}
\usepackage{textcomp}
\usepackage{xparse}

\makeatletter
\global\let\tikz@ensure@dollar@catcode=\relax
\makeatother
\AtBeginDocument{
    \catcode`_=12
    \begingroup\lccode`~=`_
    \lowercase{\endgroup\let~}\sb
    \mathcode`_="8000
}


\newcolumntype{P}[1]{>{\centering\arraybackslash}p{#1}}
\newcolumntype{M}[1]{>{\centering\arraybackslash}m{#1}}
\newcolumntype{L}{>{\centering\arraybackslash}p{0.58cm}}
\newcolumntype{Z}{>{\centering\arraybackslash}p{.95cm}}
\newcolumntype{A}{>{\centering\arraybackslash}p{.8cm}}
\newcolumntype{B}{>{\centering\arraybackslash}p{.6cm}}
\newcolumntype{H}{>{\setbox0=\hbox\bgroup}c<{\egroup}@{}}

\hypersetup{
  colorlinks=true,
  linkcolor=blue!50!blue,
  citecolor=blue!50!blue,
  urlcolor=black!70!black
}

\DeclarePairedDelimiterX\set[1]\lbrace\rbrace{\def\given{\;\delimsize\vert\;}#1}
\DeclarePairedDelimiter{\ceil}{\lceil}{\rceil}

% Figure command
\newcommand{\insertFigure}[2]{
    \begin{figure}[t!]
%\setlength{\abovecaptionskip}{0pt}
%\setlength{\belowcaptionskip}{0pt}
        \centering
        \includegraphics[width=\linewidth]{figures/#1.pdf}
	\vspace{-6mm}
        \caption{\small #2}
	\vspace{-3mm}
        \label{fig:#1}
    \end{figure}
}

\newcommand{\insertWideFigure}[2]{

    \begin{figure*}[ht!]
%\setlength{\abovecaptionskip}{0pt}
%\setlength{\belowcaptionskip}{0pt}
        \centering
        \includegraphics[width=\textwidth]{figures/#1.pdf}
	\vspace{-6mm}
        \caption{\small #2}
	\vspace{-3mm}
        \label{fig:#1}
    \end{figure*}

}

\newcommand*\circledsmall[1]{\tikz[baseline=(char.base)]{
            \node[shape=circle,fill=darkgray,inner sep=0.5pt] (char) {\small\textcolor{white}{#1}};}}

\newcommand*\circled[1]{\tikz[baseline=(char.base)]{
            \node[shape=circle,fill=darkgray,inner sep=0.5pt] (char) {\textcolor{white}{#1}};}}

%%%% for tighter bullets
\newcommand{\squishlist}{
 \begin{list}{$\bullet$}
  { \setlength{\itemsep}{0pt}
     \setlength{\parsep}{0pt}
     \setlength{\topsep}{3pt}
     \setlength{\partopsep}{0pt}
     \setlength{\leftmargin}{1.5em}
     \setlength{\labelwidth}{1em}
     \setlength{\labelsep}{0.5em} } }
     
%%%% for tighter numbers
\newcommand{\squishnums}{
 \begin{list}{$\bullets$}
  { \setlength{\itemsep}{0pt}
     \setlength{\parsep}{3pt}
     \setlength{\topsep}{3pt}
     \setlength{\partopsep}{0pt}
     \setlength{\leftmargin}{1.5em}
     \setlength{\labelwidth}{1em}
     \setlength{\labelsep}{0.5em} } }

\newcommand{\squishlisttwo}{
 \begin{list}{$\bullet$}
  { \setlength{\itemsep}{0pt}
     \setlength{\parsep}{0pt}
    \setlength{\topsep}{0pt}
    \setlength{\partopsep}{0pt}
    \setlength{\leftmargin}{2em}
    \setlength{\labelwidth}{1.5em}
    \setlength{\labelsep}{0.5em} } }

\newcommand{\squishend}{
  \end{list}  }


\newcommand{\betterparagraph}[1]{\textbf{#1.}}


% Sections
\def\sectionautorefname{Sec.}

\renewcommand*\sectionautorefname{\Snospace}
\def\subsectionautorefname{Sec.}
\def\subsubsectionautorefname{Sec.}

% Figures
\def\figureautorefname{Fig.}


\newcommand{\greencheck}{{\color{ForestGreen}\checkmark}}
\newcommand{\orangecheck}{{\color{orange}\checkmark}}
\newcommand{\redcheck}{{\color{red}\xmark}}

\newcommand{\NMName}[0]{\textsc{Einsum}~}
\newcommand{\NMNameTitle}[0]{Einsum}

\newcommand{\NMNamenospace}[0]{\textsc{Einsum}}

\newcommand{\NMNameOff}[0]{\textsc{Einsum} (near DRAM)~}
\newcommand{\NMNameOffTitle}[0]{\textsc{Einsum} (near DRAM)}

\newcommand{\NMNameOffnospace}[0]{\textsc{Einsum} (near DRAM)}

\newcommand{\TitleName}[0]{\textsc{Harp}~}

\newcommand{\TitleNamenospace}[0]{\textsc{Harp}}

\newcommand{\HHPName}[0]{\textsc{Harp}~}

\newcommand{\HHPNamenospace}[0]{\textsc{Harp}}

\newcommand{\NoCName}[0]{\textsc{In-switch}~}

\newcommand{\NoCNamenospace}[0]{\textsc{In-switch}}

\newcommand{\TODO}[1]{\textcolor{red}{TODO::: #1}}
\newcommand{\RG}[1]{{\color{orange}\bfseries [RG::: #1]}}
\newcommand{\TK}[1]{{\color{teal}\bfseries [Tushar::: #1]}} 
\newcommand{\MP}[1]{{\color{olive}\bfseries [Michael::: #1]}}

\newcommand{\fixme}[1]{{{\color{blue} #1}}}


\newcommand{\cmark}{\ding{51}}%
\newcommand{\xmark}{\ding{55}}%

%\renewcommand{\hl}{}

%\begin{comment}
\renewcommand{\RG}[1]{\ignorespaces}

\renewcommand{\TK}[1]{\ignorespaces}

\renewcommand{\TODO}[1]{\ignorespaces}

%\end{comment}


\newcommand{\insertFigurePartnn}[2]{
    \begin{figure}[t]
%\setlength{\abovecaptionskip}{0pt}
%\setlength{\belowcaptionskip}{0pt}
        \centering
        \includegraphics[width=0.85\linewidth]{figures/#1.pdf}
	\vspace{-2mm}
        \caption{#2}
	\vspace{0mm}
        \label{fig:#1}
    \end{figure}
}

% Sections
\renewcommand*\sectionautorefname{\Snospace}
\def\sectionautorefname{Sec.}
\def\subsectionautorefname{Sec.}
\def\subsubsectionautorefname{Sec.}

% Figures
\def\figureautorefname{Fig.}

\def\algorithmautorefname{Alg.}

\def\tableautorefname{Table}


\newcommand{\insertWideFigureScaled}[3]{

    \begin{figure*}[ht!]
%\setlength{\abovecaptionskip}{0pt}
%\setlength{\belowcaptionskip}{0pt}
        \centering
        \includegraphics[width=#3\textwidth]{figures/#1.pdf}
	\vspace{-3mm}
        \caption{\small #2}
	\vspace{0mm}
        \label{fig:#1}
    \end{figure*}

}

\newcommand{\insertFigureScaled}[3]{

    \begin{figure}[t!]
%\setlength{\abovecaptionskip}{0pt}
%\setlength{\belowcaptionskip}{0pt}
        \centering
        \includegraphics[width=#3\linewidth]{figures/#1.pdf}
	\vspace{-3mm}
        \caption{\small #2}
	\vspace{0mm}
        \label{fig:#1}
    \end{figure}

}

\newcommand\copyrighttext{%
  \scriptsize \textcopyright
  2025 IEEE. Personal use of this material is permitted. Permission
  from IEEE must be obtained for all other uses, in any current or future
  media, including reprinting/republishing this material for advertising or
  promotional purposes, creating new collective works, for resale or
  redistribution to servers or lists, or reuse of any copyrighted
  component of this work in other works.
  Accepted for publication in IEEE Transactions on Very Large Scale Integration (VLSI) Systems. DOI: 10.1109/TVLSI.2025.3530785}
\newcommand\copyrightnotice{%
  \begin{tikzpicture}[remember picture,overlay]
    \node[anchor=south,yshift=0pt] at (current page.south) {\fbox{\parbox{\dimexpr\textwidth-\fboxsep-\fboxrule\relax}{\copyrighttext}}};
  \end{tikzpicture}}

\begin{document}

\title{Strassen Multisystolic Array Hardware Architectures}

\author{Trevor~E.~Pogue~\orcid{0000-0002-6791-3758} and Nicola~Nicolici~\orcid{0000-0001-6345-5908},~\IEEEmembership{Senior Member,~IEEE}
  \IEEEcompsocitemizethanks{\IEEEcompsocthanksitem T. E. Pogue and N. Nicolici are with the Department of Electrical and Computer Engineering, McMaster University, Hamilton, ON, L8S 4L8, Canada \protect\\
    Email: poguete@mcmaster.ca; nicolici@mcmaster.ca}}

\IEEEpeerreviewmaketitle

\IEEEtitleabstractindextext{%
\begin{abstract}
    Humor is a social binding agent. It is an act of creativity that can provoke emotional reactions on a broad range of topics. Humor has long been thought to be “too human” for AI to generate. However, humans are complex, and humor requires our complex set of skills: cognitive reasoning, social understanding, a broad base of knowledge, creative thinking, and audience understanding. We explore whether giving AI such skills enables it to write humor. We target one audience: Gen Z humor fans. We ask people to rate meme caption humor from three sources: highly upvoted human captions, 2) basic LLMs, and 3) LLMs captions with humor skills. We find that users like LLMs captions with humor skills more than basic LLMs and almost on par with top-rated humor written by people. We discuss how giving AI human-like skills can help it generate communication that resonates with people. 

\end{abstract}
\begin{IEEEkeywords}
  Hardware architecture, machine learning, matrix multiplication, performance, Strassen, systolic arrays.
\end{IEEEkeywords}}

\maketitle
\IEEEdisplaynontitleabstractindextext
\IEEEpeerreviewmaketitle
\section{Friction in \abr{ai} systems}
% \sscomment{haven't used the word \textit{friction} entirely though, also maybe `Friction is necessary for AI systems' is a better one}
\begin{figure}[ht]
    \centering
    \includegraphics[width=0.45\textwidth]{figures/intro_example.pdf}
    \caption{Detecting deception is crucial in mixed cooperative-competitive environments. \textbf{(Left)} \textcolor{fra}{France} believed the lie that \textcolor{ita}{Italy} will move their army in Tyrolia to Munich, losing Burgundy and subsequently Marseilles to \textcolor{ger}{Germany}. \textbf{(Right)} If \textcolor{fra}{France} had detected the deception, they could have successfully defended Burgundy and avoided disbanding one army.}
    \label{fig:intro_example}
\end{figure}
%motivation
% The increasing sophistication of \abr{ai} models, particularly large language models (LLMs), raises significant safety concerns regarding their potential to engage in deceptive behaviors. Deception in \abr{ai}-generated text poses risks in various domains, including misinformation propagation, social manipulation, and adversarial exploits in multi-agent interactions. If left unaddressed, \abr{ai} systems optimized for persuasion or goal-oriented dialogue may develop strategies that prioritize manipulation over truthfulness. This risk is especially pronounced in reinforcement learning with human feedback~\citep[\abr{rlhf},][]{christiano2017deep}, where \abr{ai} agents are rewarded for producing desirable outputs rather than strictly adhering to factual accuracy (reward hacking citations). As \abr{ai} systems become more autonomous and embedded in decision-making processes, understanding and mitigating deceptive tendencies is crucial to ensuring trustworthiness and alignment with human values. 

% \fgcomment{I think this paragraph is talking about the same thing (AI deception) as the next one. We should either separate to deception in NL/in AI, or how it happens (reward hacking and optimization)/why necessary to detect and mitigate}
% \wwcomment{agree, I made them into how it happens (reward hacking and optimization)/why necessary to detect and mitigate, thanks!}

\begin{figure*}[h]
    \centering
    \includegraphics[width=\textwidth]{figures/overview.pdf}
    \caption{An overview of our approach to detect deceptive proposals, requiring a recipient (Human Player) to follow a proposed action. \textbf{(Left)} A state of this Diplomacy game is (1) \textcolor{aus}{Austria} and \textcolor{ita}{Italy} have an alliance (2) while \textcolor{tur}{Turkey} and \textcolor{ita}{Italy} have been clashing for several turns. \textcolor{aus}{Austria} realizes that they are in a weak spot and need a quick escape, so they reach out to \textcolor{ita}{Italy}. It is a deceptive proposal so that \textcolor{aus}{Austria} can get to Trieste. \textbf{(Bottom Middle)} A human player can be biased towards their own ally (\textcolor{aus}{Austria}) and use their fast-thinking system to instinctively help. \textbf{(Top Middle)}  For an alternative perspective, our approach converts natural language to proposals using \abr{amr}. \textbf{(Right)} Then, we leverage the RL value function from \cicero to estimate three aspects of deception---Bait, Switch and Edge---from counterfactual actions of \textcolor{aus}{Austria} and \textcolor{ita}{Italy}. 
    Passing the dialogue alongside these values to a classifier decides whether \textcolor{aus}{Austria}'s proposal is \textbf{deceptive}.}
    \label{fig:overview}
\end{figure*}


Deception in natural language is a fundamental aspect of human communication, often employed as a strategic tool to mislead others through misrepresentation, omission, exaggeration, or counterfactual reasoning~\citep{bok2011lying}. 
From casual social interactions to high-stakes negotiations, deception influences trust, decision-making, and cooperation, making it a subject of extensive study in psychology, linguistics, and philosophy, manifesting real-world challenges such as fake news on
social media ~\citep{bade-etal-2024-social}, misinformation ~\citep{panda-levitan-2022-improving} and adversarial communication in strategic
games ~\citep{bernard-mickus-2023-many}. As artificial intelligence systems increasingly engage in human-like communication, they not only inherit but also amplify deceptive strategies, sometimes unintentionally.
%
In \abr{ai}-generated text, deception can emerge as a byproduct of
optimization objectives, particularly in \abr{rlhf} scenarios where
agents maximize utility in multi-agent settings, sometimes at the
expense of honesty~\cite{wen2025language}.
%
This phenomenon has garnered significant attention across various
domains, as \abr{ai} deception is not confined to theoretical
constructs but manifests in real-world challenges, e.g. hallucination in reasoning tasks ~\citep{grover-etal-2024-navigating}.

% \jbgcomment{Don't focus on AI deception, start with human deception and then talk about how AI can amplify it.\wwcomment{I added some parts and rearranged.}}

Prior research underscores that \abr{ai}-generated deceptive
communication can be difficult to detect and may lead to unintended
consequences when deployed in practical applications~\citep{park2024ai,sarkadi2024deceptive}.
%
Deceptive AI-generated text can erode trust in digital communication,
amplify misinformation, and facilitate large-scale manipulation in
political, financial, and social domains~\citep{solaiman2019release,
  weidinger2022taxonomy}.
%
Furthermore, the scalability of AI models allows deceptive content to
be produced and disseminated at unprecedented rates, making manual
detection impractical.
% \sscomment{instead, you could say `make detecting
%   deception more nuance and call the need for an automatic tool that
%   can aid humans with it' we can't claim for removing the role of the
%   human in the loop at least as a verifier}.
% \wwcomment{I dont think I claims as not human is needed but rather human manually verification way harder when AI contents are generated super fast/many}
%
To address these risks, robust mitigation strategies are necessary,
including adversarial training~\citep{perez2022red}, explain-ability
techniques to enhance AI transparency~\citep{danilevsky2020explainability}, and real-time detection methods
leveraging linguistic and behavioral cues~\citep{vosoughi2018spread}.

% \jbgcomment{I think just one paragraph is sufficient for setup, get
%   into the paper sooner.  I think until we have user experiments, it's
%   hard to use System 1/2 because we don't actually show that switching
%   systems helps. \wwcomment{removed system1/2 Kahneman}}

% The increasing sophistication of \abr{ai} models and the risks of deceptive behaviors demand innovative safeguards. Drawing on \textit{the Dual-Process Theory} \citep{kahneman2011thinking}, we can conceptualize the generation of \abr{ai} text as analogous to \textbf{System 1}: a fast, intuitive process prone to prioritize goal-oriented outputs over factual accuracy. This risk is particularly acute when \abr{ai} systems exploit maximize utility (e.g. reward-hacking in language models \cite{amodei2016concreteproblemsaisafety}). A promising mitigation strategy responding to \textbf{System 1} involves counterfactual reinforcement learning (cite) as a \textbf{System 2}-like mechanism. This reflective process would serve to \textit{scrutinize} the outputs generated by the \textit{fast}, heuristic \textbf{System 1}-like component. By evaluating alternative scenarios, counterfactual \abr{rl} can catch deceptive tendencies before they deploy.

% \jbgcomment{The focus on AMR is too focused on the specific method and doesn't say anything about \emph{why} we need to do this.

%   I'd structure this more as:
%   \begin{itemize}
%   \item An effective negotiation is one where both parties benefit
%   \item The bigger problem is that sometimes people are taken in by an offer that ``sounds too good to be true''
%   \item We need to ground the text of the negotiations, and we use AMR to do that
%   \item We then use the value functions of trained agents to detect offers that could be part of a deception
%     \item We compare and combine this with techniques X, Y, Z
%   \end{itemize}
%   \wwcomment{that sounds a lot better. I fixed it accordingly. Right now it is a bit short so I will add more details}

% }

 % \fgcomment{move to related work?} 
 % \wwcomment{done! moved prior works ai deception to related works}
We test our detection strategies within the environment of  \textit{Diplomacy}, a game rich in negotiation, cooperation, and betrayal expressed through natural language. The most intriguing moments of the game arise when two players negotiate to cooperate in pursuit of their respective goals. While such agreements usually yield mutual benefits, this is not always the case---some negotiated arrangements are the result of deception, omission, or straight-up lies on the part of one player. Skilled players combat such behavior by developing the ability to recognize when an offer \textit{sounds too good to be true}. Our work explores this area to raise awareness among human players when they encounter deception embedded in negotiations. 

We use the \emph{value function} of \cicero~\citep{bakhtin2023mastering}, an agent trained to play Diplomacy at a human level, to detect whether a proposal is ``\textit{too good to be true}''. We estimate other players' likely strategies and query Llama3~\cite{llama3modelcard}, an \abr{llm} good at general purpose semantic understanding, to determine if a message is contrary to \cicero{}'s expected strategy. However, Llama3 cannot precisely predict deception since it sees almost half of data as deceptive. Consequently, we decide to take a more explicit approach to modeling deception. Our contributions are as follows:\\
\noindent \textbf{1)} With Theory-of-Mind-influenced deception, we identify negotiations in natural language via formal logical modeling and detect potential deceptive offers in negotiations using \cicero \abr{rl} value function to generate counterfactual explanations.\\
%Also, identifying negotiations in natural language dialogue via formal logical modeling is necessary in this step.\\
\noindent \textbf{2)} We train a BERT-based~\cite{devlin-etal-2019-bert} classifier to predict deception using \abr{rl} values and message embeddings.\\
\noindent \textbf{3)} We show that our classifier is more accurate than a fine-tuned Llama3 in human lie prediction and detecting partially-deceptive negotiations. 

\section{Data Generator DCSim}
\label{sec:background}

To simulate the execution of HEP workflows on parallel and distributed computing (PDC) infrastructure, we utilize the DCSim tool~\cite{Horzela2023Proceeding,Horzela2023_1000165566} as a reference simulator\footnote{Note the exact version of the simulator published here: \url{https://doi.org/10.5281/ZENODO.8300961}}. 
It is implemented using the SimGrid~\cite{SimGrid} and WRENCH~\cite{wrench} simulation frameworks.
%
SimGrid and WRENCH have been chosen since they are general-purpose, enable more accurate simulation of PDC systems than other versatile frameworks~\cite{TOMACS} while keeping the level of computational complexity at manageable levels, have been carefully validated~\cite{
% TOMACS, 
simutool_09, nstools_07, simgrid_storage,
% SMPI_TPDS,
% 7885814, 8048921,
7384330,
% stanisic,
Cornebize-cluster19}, and are utilized by a large and active community. %~\cite{simgrid-web,wrench-web}.

DCSim takes three main essential user inputs indicated as \enquote{configurations} in \autoref{fig:approach-overview}.
First, a platform description following the SimGrid standard defines the network of computing resources with all its (technical) characteristics, for instance, the network of links and interconnected computers, their routing, as well as their respective network bandwidths, latencies, numbers of CPU cores, CPU speeds, disk bandwidths, storage volumes, and other parameters.
Second, datasets and replicas present at the simulation start, their composition of files, sizes, and locations are specified in a dedicated input file.
Third, the workloads to be executed on the specified platform while possibly processing files from datasets are defined in another dedicated steering file.
The workloads are characterized by their submission time, the job collection they are composed of, their respective resource demands, and the amount of computational work that must be executed for their completion.
Additionally, parameters can be given to DCSim to steer individual components of the simulation models integrated into DCSim or to adjust their calibration. 
In this work, we use the default calibration derived for \cite{Horzela2023Proceeding}.
Using these inputs, DCSim simulates the scheduling of the jobs, their execution and data processing, and returns a list of observables for each simulated job.
Per default, observables are the start and end times of the jobs, the total times spent with I/O and compute operations, and the processed amounts of data.

It has been shown that after calibration, DCSim can achieve valid descriptions of real-world computing workflow executions and systems~\cite{Horzela2023Proceeding,Horzela2023_1000165566}.
However, the execution time of DCSim scales super-linear with the size of the simulated computing system, which limits its feasibility and necessitates workarounds in studies concerning computing systems at a global scale, like the WLCG.
We therefore utilize DCSim as a baseline for testing ML surrogates in simulation.

\section{Strassen Architecture}
\label{smm:sec:arch}
The proposed architectures achieve a more efficient implementation of \sa than what is possible through execution on CPUs and GPUs by pipelining and performing the extra additions and data movement steps at all levels of recursion in parallel with the matrix multiplications.
The architectures are functionally equivalent to conventional multisystolic array designs while allowing the theoretical complexity reductions of \sa to be translated directly into hardware resource savings.

\subsection{Memory Layout and Access Algorithm}
\label{smm:sec:mem-layout}

In order to perform the extra Strassen data movement and addition steps at all levels of recursion in parallel with the matrix multiplications, the architecture reads one row/column at a time of the \A and \B input matrix \subblocks from the lowest level of recursion in \eq{smm:eq:strass-first} simultaneously.
This generates and provides all \T and \Ss \subblocks one row/column at a time for performing all the matrix multiplications in \eq{smm:eq:strass-last} at the lowest level of recursion in parallel.
The \T and \Ss \subblocks are all immediately generated from the \A and \B input \subblocks and consumed in parallel like this to eliminate any additional execution time or hardware resources needed for storing/re-accessing them for later use.

\begin{figure}
  \centering
  \includegraphics[scale=.75]{smm-mem-layout.eps}
  \caption{Example data layout for the \A matrix in memory for an architecture implementing Strassen matrix multiplication for 2 levels of recursion  (\smmArch?2).
    Each address $i$ contains every $m^{th}$ row of \A concatenated together starting at row $i$ (notated as \m.A'{i:m:,:}).
    To help illustrate this, the gray coloured rows are all elements of \A belonging to address 0, which forms \m.A'{0:m:,:} containing row 0 of every \A \subblock from the lowest level of recursion in \eq{smm:eq:strass-first}.
    The organization for the \B matrices in memory are the same, except that the order of the elements is transposed compared to the \A matrix layout shown here.}
  \label{smm:fig:mem}
\end{figure}

To achieve this, each \AB matrix fed into the MXU is divided into $4^r$ equal \subblocks of size $m$\by$k$ for \A and of size $k$\by$n$ for \B, where each row/column $i$/$j$ of each \A/\B \subblock is stored in the accelerator's \A and \B memories at location $i$/$j$ plus an offset.
An example of this memory layout for implementing 2 levels of Strassen recursion is shown in \fig{smm:fig:mem}.
This means that each \A memory location $i$ is a vector containing every $m^{th}$ row of \A starting at row $i$ concatenated together (notated as \Av), and each \B memory location $j$ is a vector containing every $n^{th}$ column of \B starting at column $j$ concatenated together (notated as \Bv).
This allows one row or column of all $4^r$ \A/\B \subblocks from the lowest level of recursion in \eq{smm:eq:strass-first} to all be read at once from a single memory location and fed into the MXU each clock cycle.
\Av and \Bv rows/columns are then read consecutively when feeding the \AB blocks into the MXU.

As shown in \eq{smm:eq:strass-first}, the input \AB matrices at each level of recursion are divided into four \block quadrants labelled \An and \Bn of size $M$\by$K$ for \An quadrants and of size $K$\by$N$ for \Bn quadrants.
The portions of each \Av and \Bv vector belonging to quadrant \An and \Bn are notated as \Anv and \Bnv.
The MXU then computes and returns row $i$ of all \C \subblocks from the lowest level of recursion in \eq{smm:eq:strass-last} in every clock cycle $i$, allowing \Cv to be stored in the same format as \A in memory for if \C will later be taken as an \A input for a later matrix multiplication.


\subsection{Strassen Multisystolic Array Design}
\begin{figure}
  \centering
  \includegraphics[scale=.89]{smm-mxu-top.eps}
  \caption{Top-level diagram of the proposed \smmArch multisystolic array architecture for implementing Strassen matrix multiplication \seq for $r$ levels of recursion in hardware.}
  \label{smm:fig:smm-mxu}
\end{figure}
\begin{figure}
  \centering
  \includegraphics[scale=.89]{smm-mxu-additions.eps}
  \caption{Internal structure of the \smmArch MXU addition vectors from \fig{smm:fig:smm-mxu}.
  }
  \label{smm-add-units}
\end{figure}

\Figure \ref{smm:fig:smm-mxu} shows the proposed \smmArch multisystolic array architecture.
Rather than having one $X$\by$Y$ MXU with $X$ columns and $Y$ rows of MAC units for efficiently multiplying matrices down to size $X$\by$Y$, this architecture consists of $7^r$ smaller $X/2^r$\by$Y/2^r$ MXUs that together efficiently multiply matrices down to the same size but at a higher throughput.
Furthermore, it achieves this with fewer MAC units than a conventional multisystolic array design.
This both allows smaller matrices to be multiplied at a higher utilization and increases the throughput per MAC unit.

The \Av and \Bv vectors read into the MXU are first divided into their four \Anv and \Bnv portions depending on which quadrant of \A/\B each element belongs to as shown in \fig{smm:fig:smm-mxu}.
They then pass through the \A/\B addition vectors shown in \fig{smm-add-units} (a) and (b) to form the \Tv/\Sv matrices.
The \A/\B addition vectors both contain 5 addition vectors each consisting of $K$ scalar adders or subtractors, where $K$ is the width of the four \m.A?{ij} \blocks and the height of the four \m.B?{ji} \blocks as defined in \secn{smm:sec:mem-layout}.
The 7 \Tv/\Sv vectors then pass into the next level of \smmArch?{r-1} MXUs to perform the 7 matrix \block multiplications.
The \Qv vectors of the matrix \block multiplication outputs then pass through the \Qv addition vectors shown in \fig{smm-add-units} (c) consisting of 8 addition vectors each containing $N$ scalar adders or subtractors.
This forms the final \C product, where $N$ is the width of the four \m.B?{ji} \blocks as defined in \secn{smm:sec:mem-layout}.

Each of the 7 \smmArch?{r-1} MXUs can contain 7 more \smmArch?{r-2} MXUs for implementing another level of \s recursion and repeating the process above, or they can be instantiated as a baseline \mmArch?\zero MXU shown in \Figure \ref{smm:fig:MM-mxu}.
For implementing the next level of \smmArch?{r-2} MXUs inside each \smmArch?{r-1} MXU, each \Tv/\Sv input passed into an \smmArch?{r-1} MXU will then be considered as the full \Av/\Bv inputs within that MXU and are split again into the next level of four \Anv/\Bnv vectors.
The dimensions of the matrix \blocks being read/computed and the number of scalar adders in the addition vectors within each \smmArch?{r-1} MXU will then be reduced by a factor of 2 at each level of recursion.
For fixed-point implementations, the \Tv/\Sv inputs to each \smmArch?{r-1} MXU that were formed from an addition or subtraction in the \A or \B vector addition units will have an increased bitwidth by 1 bit.


\subsection{Baseline Designs}

\begin{figure}
  \centering
  \includegraphics[scale=1.2]{smm-baseline-mm0-mxu.eps}
  \caption{Baseline \mmArch?\zero single-systolic array architecture that implements conventional matrix multiplication \eq{smm:eq:mmZero} in hardware, provided for completeness and clarity.
    It is instantiated at the lowest level of recursion in the \smmArch and \mmArch MXU architectures.
    $X$ here represents the width of the $a$ and $b$ vectors entering the \mmArch?\zero MXU, and $Y$ represents the width of the $c$ vectors exiting the MXU.}
  \label{smm:fig:MM-mxu}
\end{figure}

\begin{figure}
  \centering
  \includegraphics[scale=.78]{smm-baseline-mmr-mxu.eps}
  \caption{Baseline \sAlg.{\mm?r} multisystolic array architecture for implementing conventional blocked matrix multiplication \eq{smm:eq:mm} for $r$ levels of recursion in hardware.}
  \label{smm:fig:multi-mm-mxu}
\end{figure}

We later compare the \smmArch?r architectures with baseline \mmArch multisystolic array architectures shown in \fig{smm:fig:multi-mm-mxu} which execute \eq{smm:eq:mm} in parallel for $r$ levels of recursion.
The baseline \mmArch architectures are functionally identical to the \smmArch architectures, but they consist of $8^r$ smaller $X/2^r$\by$Y/2^r$ MXUs rather than $7^r$.
\Figure \ref{smm:fig:MM-mxu} also shows the internal structure of each baseline \mmArch?\zero MXU present at the lowest level of recursion in each \smmArch and \mmArch architecture, and \fig{smm:fig:pes} shows the internal structure of the processing elements (PE)s inside the \mmArch?\zero MXUs.

\begin{figure}
  \centering
  \includegraphics[scale=.8]{smm-PE.eps}
  \caption{The internal PE structure of each \mmArch?\zero MXU from \fig{smm:fig:MM-mxu}, provided for completeness.
    Here, $w_a$ is the additional bitwidth added to account for accumulation, equal to $\ceil{\tx{log}_2(X)}$, where $X$ is the width of the $a$ and $b$ vectors entering the \mmArch?\zero MXU.
  }
  \label{smm:fig:pes}
\end{figure}

\section{Results}


In this section, we evaluate example implementations of the proposed \smm architectures.
In \secn{smm:sec:results-baseline}, we compare the \smm MXU architectures in isolation against our baseline MXU designs in Table \ref{smm:tab:smm-MM}.
In \secn{smm:sec:results-prior-work} we evaluate the \smm MXU architectures in Tables \ref{smm:tab:second}-\ref{smm:tab:last} compared to prior work when integrated into an end-to-end deep learning accelerator system based on the system from our previous work \cite{pogue2024fast}, which has open-source code \cite{ffip-source}.
We first describe the integration of our proposed systolic arrays into the deep learning system in \secn{sec:system}, and in Sections \ref{sec:mu} and \ref{smm:sec:mat-sz}, we define performance metrics used to compare the \smmArch architectures against baseline designs and prior works.

\subsection{System Integration}
\label{sec:system}
We were able to integrate the \smm architectures into a system based on our previous work \cite{pogue2024fast}, which has open-source code \cite{ffip-source}, by swapping the \smmArch MXU architectures from \fg{smm:fig:smm-mxu} into our system design \cite{pogue2024fast} in place of the free-pipeline fast inner-product (FFIP) MXU.

In order to perform GEMM on the proposed MXUs and multiply matrices of arbitrary sizes that can be larger than the MXU dimensions, the full \AB matrices are first divided into \gemmtiles prior to being divided further into smaller \blocks for executing \eq{smm:eq:mm} or \seq.
The \gemmtiles are then fed into the MXU one-by-one.
Each \gemmtile is then considered as the full \AB matrix from \eq{smm:eq:mm} or \seq while being fed into the MXU and gets further divided into smaller \m.A?{ij}/\m.B?{ji} \blocks within the MXU.

Following each \gemmtile multiplication, the partial \gemmtile products are accumulated outside of the MXU to generate each final \gemmtile product.
Prior to each \gemmtile multiplication, a $\bld{B}$ \gemmtile is loaded into the MXU.
It then remains in place as the $\bld{A}$ \gemmtile flows through the MXU producing the \gemmtile product, during which a new \Av vector is fed into the MXU each clock cycle.
Additionally, to hide the latency of loading $\bld{B}$ \gemmtiles, the MXU PEs each contain one extra $b$ buffer to load the next $\bld{B}$ \gemmtile into the MXU as the current \gemmtile is being multiplied.

Each \A, \B, and \C \subblock entering or exiting the top-level MXU for the \smm and baseline MXUs first pass through triangular-shaped register arrays each containing $X$ shift registers of varying depths.
Here, each shift register $SR_k$ has a depth of $k$ and loads one $a_{i,k}$ or $b_{k,j}$ element per clock cycle.
These triangular buffers are explained further in our prior work \cite{pogue2024fast} and they allow the vector elements to enter the MXU in the necessary order as depicted in the element indices in \figs \ref{smm:fig:MM-mxu} and \ref{smm:fig:pes}.

\subsection{Multiplier Compute Efficiency}
\label{sec:mu}
In this subsection, we define an efficiency metric called the multiplier compute efficiency (MCE) in \eq{smm:eq:mu} which we use to compare the \smmArch architectures against baseline designs and prior works.
This is used to quantify how much the algebraic optimizations exploited in an architecture reduce the computational complexity.
Reductions in computational complexity allow an architecture to utilize its multipliers more effectively than conventional designs using no algebraic optimizations.
The multiplier compute efficiency is defined as follows:
\begin{align}  \label{smm:eq:mu}
  \tx{MCE} = \frac{\tx{mults} / \tx{multiplier}} {\tx{clock cycle}}
  &= \frac{(\tx{mults/s})/\tx{\#multipliers}}{f}
  \,.
\end{align}
Here, mults/s above is measured by taking the number of multiplications required to carry out an execution using conventional algebra and dividing it by the measured execution time.
Finally, \#multipliers is the number of instantiated multipliers in the design, and $f$ is the clock frequency that the hardware design is operating at.

Conventional matrix multiplication algorithms such as \eq{smm:eq:mm} have no algebraic optimizations for reducing the computational complexity.
Therefore, the limit/maximum achievable value (also referred to as the roof) of the metric in \eq{smm:eq:mu} is the following when using conventional matrix multiplication in hardware:
\begin{align}  \label{smm:eq:mm-mu-roof}
  \tx{roof}\lb\tx{MCE}_{\tx{\mm?r}}\rb &= 1
  \,.
\end{align}
In contrast, \sa requires $8^r/7^r$ times fewer multiplications than a conventional matrix multiplication algorithm, where $r$ is the number of levels of recursion implemented in \sa.
Therefore, the multiplier compute efficiency can reach the following limit in \smmArch architectures:
\begin{align} \label{smm:eq:mu-roof}
  \tx{roof}\lb\tx{MCE}_{\tx{\smm?r}}\rb &= 
  \lb\frac{8}{7}\rb^{r}
  \,.
\end{align}

As discussed in \secn{smm:sec:background}, \sa reduces the overall number of operations in matrix multiplication.
Furthermore, any additions required before the matrix multiplications in the algorithm are even less of a concern in fixed-point implementations.
This is because the hardware complexity of fixed-point multipliers typically scale quadratically with the input bitwidth compared to linearly for adders and registers \multComplexity, causing the hardware footprint of multipliers to dominate that of adders and registers.

However, one of the impediments of using \sa for fixed-point implementations is that the bitwidths of the multiplication inputs increase by $r$ bits for $r$ levels of Strassen recursion that are implemented, reducing its potential area savings for custom fixed-point hardware designs.
Nonetheless, this impediment for fixed-point designs can be inherently mitigated in FPGA implementations so long as $r$ plus the initial input width is not larger than the maximum input width supported by the FPGA's DSP units.
For example, each DSP in common Intel/Altera FPGAs instantiate two 18\by19-bit multipliers \cite{intel-dsp}, and common input bitwidths for applications such as deep learning are 16 bits or less.
This leaves room for at least 2 or more levels of Strassen recursion to be implemented before surpassing the bitwidth limit supported by the DSPs.

Furthermore, due to the flexible nature of custom hardware design, the \smm?r architectures can be efficiently mapped onto other DSP units in general which support input bitwidths up to $n$ bits by customizing the input datapath bitwidth $w$ and value of $r$ as necessary to ensure that $w+r \le n$.
So long as the accuracy requirements of the application are still met, this will allow the \smm?r designs and their increase in multiplier bitwidth to still be efficiently mapped onto DSP units of any bitwidth in a general way.

\subsection{Supporting Smaller Matrices with the Same Performance}
\label{smm:sec:mat-sz}
Multisystolic array designs such as the \smmArch and baseline \mm?r architectures have the ability to efficiently multiply smaller matrices than a single-systolic array design with the same performance capability.
By executing \eq{smm:eq:mm} or \seq fully in parallel for $r$ levels of recursion, matrix products of size as small as $n\times n$ can be computed up to once every $n/2^r$ clock cycles in an \mm?r or \smmArch multisystolic array design.
Furthermore, these matrix products require $n^3$ multiplications to calculate using conventional algebra.
Therefore, the ratio of an architecture's throughput per clock cycle versus its smallest supported matrix sizes it can multiply, which we refer to as the matrix size efficiency (MSE), is the following:
\begin{align} \label{smm:eq:mat-sz}
  \tx{MSE} = 
  \frac{\tx{mults}/\tx{clock cycle}} {\tx{min. mat. size (h\by w)}}
  \,,
\end{align}
which has the following roof for multisystolic arrays:
\begin{align} \label{smm:eq:mat-sz}
  \tx{roof}\lb\tx{MSE}_{\tx{(S)MM}_r}\rb
  = \frac {n^3/\lb n/2^r\rb} {n\times n} = 2^r
  \,.
\end{align}

In contrast, a single-systolic array design can produce matrix products of size as small as $n\times n$ up to once every $n$ clock cycles, making this ratio the following for a single-systolic array design:
\begin{align}
  \tx{roof}\lb\tx{MSE}_{\tx{MM}}\rb
  = \frac {n^3/ n} {n\times n} = 1
  \,.
\end{align}
This shows that the \smmArch and baseline \mm?r multisystolic array designs can efficiently multiply matrices $2^r$ times smaller than a single-systolic array architecture with the same performance capability.

As discussed in \secn{multi-sys}, this is an important property for increasing a systolic array accelerator's maximum achievable throughput on real-life workloads.
Even if more compute resources are instantiated to scale up the size of the systolic array, the systolic array will begin to be underutilized after its size surpasses the workload's matrix sizes.
This is particularly true in modern workloads such as deep learning acceleration, where the matrix sizes that the workloads break down to can be smaller than the maximum systolic array size that could be instantiated in an accelerator \citeMultiSys.
In \secn{smm:sec:results-prior-work}, we demonstrate how this property allowed us to scale up our deep learning accelerator design without compromising utilization to achieve state-of-the-art ResNet \cite{kaiming2016deep} throughput.



\subsection{Comparison to Baseline Designs}
\label{smm:sec:results-baseline}
\newcolumntype{D}{>{\centering\arraybackslash}p{1.98cm}}
\newcolumntype{E}{>{\centering\arraybackslash}p{1.8cm}}
\begin{table*}\centering
\caption{Comparing \smmArch multisystolic array architectures against the baseline \mmArch?\zero single-systolic array architecture and baseline \mmArch multisystolic array architectures in isolation (without integration into a deep learning accelerator system).}
\label{smm:tab:smm-MM}
\label{smm:tab:first}
\scriptsize
\begin{threeparttable}
  \begin{tabular}{|>{\raggedleft}p{3.3cm}|D|DD|DDE|}\toprule
\arrayrulecolor{black}
&\alg.\MM?\zero<{} 48\by48 &\alg.\MM?1<{} 16\by16  &\alg.\SMM?1<{} 16\by16 &\alg.\MM?2<{} 6\by6 &\alg.\SMM?2<{} 6\by6 &\alg.\SMM?2<{} 6\by6 (with extra regs.)   \\
\toprule
DSPs                                 &1,152                     &1,024                 &\textbf{896}           &1,152               &\textbf{882}         &\textbf{882}            \\
\arrayrulecolor{black!30}\midrule
\arrayrulecolor{black}
ALMs                                 &34,890                    &30,872                &30,265                 &36,397              &35,863               &38,485                  \\
\arrayrulecolor{black!30}\midrule
\arrayrulecolor{black}
Registers                            &130,262                   &118,049               &115,830                &138,219             &133,511              &147,750                 \\
\arrayrulecolor{black!30}\midrule
\arrayrulecolor{black}
Frequency (MHz)                      &399                       &398                   &380                    &388                 &291                  &361                     \\
\arrayrulecolor{black}\midrule
$\tx{roof}\lb\tx{Throughput}\rb$ (GOPS) \tnote{1}     &1839     &1630                  &1556                   &1788                &1341                 &1663                    \\
\arrayrulecolor{black!30}\midrule
Throughput/DSP \tnote{2}                       &1.60            &1.59                  &\textbf{1.74}                   &1.55                &1.52                 &\textbf{1.89}                    \\
\arrayrulecolor{black!30}\midrule
\macUtA \tnote{3}                              &1               &1                     &\textbf{1.14}          &1                   &\textbf{1.31}        &\textbf{1.31}           \\
\arrayrulecolor{black}\midrule
Min. supported matrix size \tnote{4} &48\by48                   &32\by32               &32\by32                &24\by24             &24\by24              &24\by24                 \\
\arrayrulecolor{black!30}\midrule
\minSzMetric \tnote{5}               &1                         &2                     &2                      &4                   &4                    &4                       \\
\arrayrulecolor{black}
\bottomrule
\end{tabular}
  \begin{tablenotes}
    \item All designs are synthesized on Arria 10 GX 1150 FPGA for 16-bit fixed-point inputs and consume 0 memory resources.
\item[1] Maximum achievable throughput in giga operations per second, where throughput is equal to the number of operations required to carry out an execution using conventional algebra divided by the measured execution time.
\item[2] Shows which designs can achieve the highest throughput for the same number of DSPs.
\item[3] Maximum achievable multiplier compute efficiency, defined in \secn{sec:mu}, measures how effectively the architecture can utilize its multipliers. It can surpass 1 in \smm architectures because the observed mults/s is equal to the number of multiplications required to carry out an execution using conventional algebra divided by execution time.
\item[4] Minimum input matrix sizes that can be multiplied at peak throughput/full utilization.
\item[5] \matSzExpl
\end{tablenotes}
\end{threeparttable}
\end{table*}


Table \ref{smm:tab:smm-MM} shows the resource usage and performance comparison between the proposed \smmArch and baseline \mm{/}\mmArch systolic array architectures in isolation (without integration into a deep learning accelerator system).
The \smmArch?1 and \smmArch?2 architectures overall have a similar amount of soft logic resources and the same throughput per clock cycle roof as the \mmArch?1 and \mmArch?2 architectures, respectively, but they require \textbf{1.14-1.31\x fewer DSP units}.
Compared to the multisystolic array \mm?1 and \mm?2 designs, the \smmArch?1 and \smmArch?2 architectures are also functionally equivalent, respectively, other than having a lower clock frequency.
To help mitigate the limitation of having a lower frequency, we added an extra \smmArch?2 design (which had the biggest issue with clock frequency) on the far right of Table \ref{smm:tab:smm-MM} containing additional pipelining registers in the addition logic of each Q Addition Vectors unit from \fig{smm-add-units} (c).
This extra design demonstrates how a trade-off can be optionally made to increase the design's clock frequency at the cost of some extra soft logic resources.

Nonetheless, the lower clock frequencies of the \smm?r designs in Table  \ref{smm:tab:smm-MM} are compensated by the fact that the \smm?r designs achieve more effective operations from the same number DSP units.
Since the reduction in DSP units is greater than the reduction in clock frequency in the \smm?1 design and \smm?2 design with extra registers relative to their \mm?r counterparts, they would be able to achieve a higher overall throughput if scaled up in size to use the same number of DSPs.
This is shown by the Throughput/DSP metric in Table  \ref{smm:tab:smm-MM}, which shows that the \smm?r designs achieve up to \textbf{22\% more throughput per DSP} than their \mm?r counterparts.
Finally, if the frequency-limiting critical path is in external control or other logic outside of the systolic array after integrating it into an end-to-end accelerator system, as was the case in our full-system accelerators from Tables \ref{smm:tab:8}-\ref{smm:tab:ffip}, this limitation of a lower frequency is further mitigated.

\begin{figure}[]
  \hspace*{-0.25cm}
  \centering
  \includegraphics[scale=0.0525]{smm-mult-compute-efficiency.eps}
  \vspace*{-0.6cm}
  \caption{Plotting the multiplier compute efficiency of the architectures in Table \ref{smm:tab:smm-MM} when multiplying different input matrices of size $n$\by$n$.
    As shown, the minimum matrix size that can be efficiently multiplied gets smaller in the order of the \mm?\zero{,} \mm?1{/}\smm?1{,} and \mm?2{/}\smm?2 designs, respectively.}
  \label{smm:fig:perf}
\end{figure}

The throughput per clock cycle roof of the \mm?\zero and \mm?2 baseline designs in Table \ref{smm:tab:smm-MM} are equal and they consume the same number of DSP resources, but the \mm?\zero design requires slightly fewer ALM and register resources.
However, this penalty may be justified in the \mm?2 design when considering that the minimum matrix size (height\by width) that can be multiplied while fully utilizing the MXU is 4$\times$ smaller in the \mm?2 design compared to the \mm?\zero design.
This increases its performance scalability for accelerating modern workloads such as deep learning as discussed in \secn{multi-sys} and \ref{smm:sec:mat-sz}.
This ability of the multisystolic array designs to more efficiently multiply smaller matrices is further illustrated in \fig{smm:fig:perf}.
This same property is true for the \smmArch?2 design, except it achieves this with fewer DSP resources.
This benefit is demonstrated in \secn{smm:sec:results-prior-work}, where this property allowed us to scale up our deep learning accelerator design without compromising utilization to achieve state-of-the-art ResNet throughput.


\subsection{Comparison to Prior Work}
\label{smm:sec:results-prior-work}

Full system-level validation of the experimental accelerator as integrated into the system from our previous work \cite{pogue2024fast} has been done on an Arria 10 SoC Developement Kit \cite{sx-dev-kit} containing the \sx device by measuring throughput in real-time.
However, this device contains fewer soft logic resources than the \gx used in the prior works we compare against, and we generate compilation results for our design on the same \gx device used in prior works for a more fair and consistent comparison.
Throughput values of our designs on the \gx device are then calculated using an accurate throughput estimation model based on our highly deterministic and time-predictable system implementation, which accurately predicts actual throughputs measured on the \sx device available to us.
Tables \ref{smm:tab:second}-\ref{smm:tab:last} show throughputs for ResNet \cite{kaiming2016deep} neural networks.

The works from Liu \ea \cite{liu2022toward} and Fan \ea \cite{fan2022fpga} in Table \ref{smm:tab:8} use a technique to pack two 8-bit multiplications onto each 18\by19-bit multiplier in the DSPs and additional ALMs, and therefore the number of multipliers is calculated as \#DSP\by4 in those works.
The number of multipliers in the works \cite{huang2022fpga}, \cite{kim2023agamotto} from Table \ref{smm:tab-mixed} is calculated as \#DSPs since they are implemented on AMD FPGAs where each DSP instantiates one 18\by27-bit multiplier \cite{amd-dsp}.
In Tables \ref{smm:tab:8} and \ref{smm:tab-mixed}, the number of multipliers in the prior works \cite{an2022opencl}, \cite{dai2024dcp} is equal to \#DSPs\by2, where each DSP in the Altera FPGAs instantiates two 18\by19-bit multipliers \cite{intel-dsp}.
The number of multipliers used in the MXUs from our architectures in Tables \ref{smm:tab:first}-\ref{smm:tab:second} is equal to $8^r$ or $7^r$ times $X$\by$Y$ for the \mmArch and \smmArch MXUs, respectively.
For example, an \mm?\zero 64\by64 MXU (meaning $r = 0$ and $X = Y = 64$) would contain $8^0\times64^2$ multipliers, an \mm?1 32\by32 MXU would contain $8^1\times32^2$ multipliers, and an \smmArch?2 8\by8 MXU would contain $7^2\times8^2$ multipliers.
Due to the FFIP reduction in multipliers as described in our prior work \cite{pogue2024fast}, the number of multipliers for the FFIP architectures in Table \ref{smm:tab:ffip} is equal to $8^r$ or $7^r$ times $X\times Y/2 + X/2$ for the FFIP and FFIP+\smmArch designs, respectively.
Additionally, for our deep learning accelerator implementations in Tables \ref{smm:tab:second}-\ref{smm:tab:ffip}, there are an additional $Y\times4^r$ multipliers located outside the MXU in the Post-GEMM Unit \cite{pogue2024fast} for performing inter-layer quantization rescaling functions.
For our designs requiring more than 3036 multipliers, 3036 are instantiated on 1518 DSPs, and the remainder are instantiated in soft logic resources as the DSP resources are fully utilized.

\label{smm:sec:results-prior-work}
\begin{table*}\centering
  \caption{\smmArch multisystolic array architectures integrated into a deep learning accelerator system compared with prior state-of-the-art deep learning accelerators.}
\label{smm:tab:8}
\label{smm:tab:second}
\scriptsize
\begin{threeparttable}
  \begin{tabular}{|>{\raggedleft}p{2.2cm}|AA|AA|AA||AAA|AAA|}\toprule
    \arrayrulecolor{black}
    &\multicolumn{2}{c|}{TNNLS '22 \cite{liu2022toward}} &\multicolumn{2}{c|}{TCAD '22 \cite{fan2022fpga}} &\multicolumn{2}{c||}{Entropy '22 \cite{an2022opencl}} &\multicolumn{3}{c|}{\alg.\SMM?1<{} 32\by32} &\multicolumn{3}{c|}{\alg.\SMM?2<{} 8\by8}                                                           \\
\toprule
DSPs                               &\multicolumn{2}{c|}{1473}                           &\multicolumn{2}{c|}{1473}                        &\multicolumn{2}{c||}{1503}                           &\multicolumn{3}{c|}{1518}               &\multicolumn{3}{c|}{1518}              \\
\arrayrulecolor{black!30}\midrule
\arrayrulecolor{black}
ALMs                               &\multicolumn{2}{c|}{304K}                           &\multicolumn{2}{c|}{304K}                        &\multicolumn{2}{c||}{303K}                           &\multicolumn{3}{c|}{306K}               &\multicolumn{3}{c|}{145K}              \\
\arrayrulecolor{black!30}\midrule
\arrayrulecolor{black}
Registers                          &\multicolumn{2}{c|}{889K}                           &\multicolumn{2}{c|}{890K}                        &\multicolumn{2}{c||}{-}                              &\multicolumn{3}{c|}{641K}               &\multicolumn{3}{c|}{386K}              \\
\arrayrulecolor{black!30}\midrule
\arrayrulecolor{black}
Memories                           &\multicolumn{2}{c|}{2334}                           &\multicolumn{2}{c|}{2334}                        &\multicolumn{2}{c||}{1953}                           &\multicolumn{3}{c|}{2713}               &\multicolumn{3}{c|}{2036}              \\
\arrayrulecolor{black!30}\midrule
\arrayrulecolor{black}
Frequency (MHz)                    &\multicolumn{2}{c|}{200}                            &\multicolumn{2}{c|}{220}                         &\multicolumn{2}{c||}{172}                            &\multicolumn{3}{c|}{293}                &\multicolumn{3}{c|}{295}               \\
\arrayrulecolor{black!30}\midrule
\arrayrulecolor{black}
Model                              &ResNet-50      &VGG 16                               &Bayes ResNet-18  &Bayes VGG 11                    &R-CNN (ResNet-50) &R-CNN (VGG 16)                     &ResNet-50 &ResNet-101 &ResNet-152       &ResNet-50 &ResNet-101 &ResNet-152                  \\
\arrayrulecolor{black}\midrule
\tabThpt                           &1519           &1295                                &1590             &534                            &719               &865                               &3750      &4116       &4276             &2024      &2115       &2158                       \\
\arrayrulecolor{black!30}\midrule
\arrayrulecolor{black}
\arrayrulecolor{black}
\smmMacUt                             &0.645          &0.550                               &0.639            &0.206                          &0.696             &0.837                             &0.877     &0.963      &1.002            &1.051     &1.098      &1.120         \\
\arrayrulecolor{black}
\bottomrule
\end{tabular}
\begin{tablenotes}
\item All designs are synthesized on Arria 10 GX 1150 FPGA for 8-bit fixed-point inputs.
\item[1] Throughput in giga operations per second, equal to the number of operations required to carry out an execution using conventional algebra divided by execution time.
\item[2] \smmMacUtExpl
\end{tablenotes}
\end{threeparttable}
\end{table*}

\begin{table*}\centering
  \caption{Comparison of an FFIP \cite{pogue2024fast} single-systolic array architecture, which doubles performance per MAC unit, with combined FFIP+\smmArch multisystolic array architectures when integrated into deep learning accelerator systems.}
\label{smm:tab:ffip}
\label{smm:tab:last}
\scriptsize
\begin{threeparttable}
  \begin{tabular}{|>{\raggedleft}p{2.2cm}|AAA||AAA|AAA|}\toprule
\arrayrulecolor{black}
                                   &\multicolumn{3}{c||}{TC '24 \cite{pogue2024fast} (FFIP 64\by64)}            &\multicolumn{3}{c|}{\alg.\sffip?1<{} 32\by32}              &\multicolumn{3}{c|}{\alg.\sffip?2<{} 8\by8}                                              \\
\toprule
DSPs                               &\multicolumn{3}{c||}{\eDSPs}                                                &\multicolumn{3}{c|}{1518}                               &\multicolumn{3}{c|}{946}                                                            \\
\arrayrulecolor{black!30}\midrule
\arrayrulecolor{black}
ALMs                               &\multicolumn{3}{c||}{\eALMs}                                                &\multicolumn{3}{c|}{216K}                               &\multicolumn{3}{c|}{165K}                                                            \\
\arrayrulecolor{black!30}\midrule
\arrayrulecolor{black}
Registers                          &\multicolumn{3}{c||}{\eRegs}                                                &\multicolumn{3}{c|}{627K}                               &\multicolumn{3}{c|}{463K}                                                         \\
\arrayrulecolor{black!30}\midrule
\arrayrulecolor{black}
Memories                           &\multicolumn{3}{c||}{\eMems}                                                &\multicolumn{3}{c|}{2713}                               &\multicolumn{3}{c|}{2036}                                                            \\
\arrayrulecolor{black!30}\midrule
\arrayrulecolor{black}
Frequency (MHz)                    &\multicolumn{3}{c||}{\eFreq}                                                &\multicolumn{3}{c|}{313}                                &\multicolumn{3}{c|}{297}                                                           \\
\arrayrulecolor{black!30}\midrule
\arrayrulecolor{black}
Model                              &ResNet-50            &ResNet-101          &ResNet-152                      &ResNet-50       &ResNet-101     &ResNet-152             &ResNet-50       &ResNet-101     &ResNet-152              \\
\arrayrulecolor{black}\midrule
\tabThpt                           &\eResNetAGOPS        &\eResNetBGOPS  &\eResNetCGOPS                        &4006            &4397           &4568                   &2038            &2130           &2172            \\
\arrayrulecolor{black!30}\midrule
\smmMacUt                             &1.521  &1.655 &1.707                                                   &1.674  &1.837 &1.908            &1.813  &1.895 &1.933 \\
\arrayrulecolor{black}
\bottomrule
\end{tabular}
  \begin{tablenotes}
  \item All designs are synthesized on Arria 10 GX 1150 FPGA for 8-bit fixed-point inputs.
  \footNoteRefs
\end{tablenotes}
\end{threeparttable}
\end{table*}

\newcolumntype{V}{>{\centering\arraybackslash}p{1.4cm}}
\newcolumntype{W}{>{\centering\arraybackslash}p{1.2cm}}
\begin{table}[]\centering
\caption{State-of-the-art deep learning accelerators implemented on other FPGA families for similar neural networks and input bitwidths to provide further comparison with Tables \ref{smm:tab:8} and \ref{smm:tab:ffip}.}
\label{smm:tab-mixed}
\scriptsize
\begin{threeparttable}
  \begin{tabular}{|>{\raggedleft}p{2.5cm}|V|V|W|}\toprule
    \arrayrulecolor{black}
                                          &TNNLS '22 \cite{huang2022fpga} &TCAS-I '23 \cite{kim2023agamotto} &TCAD '24 \cite{dai2024dcp} \\
\toprule
FPGA                                       &AMD VX980                         &AMD XCV U9P                            &Altera Stratix 10 GX650                                              \\
\arrayrulecolor{black!30}\midrule
\arrayrulecolor{black}
Fixed-point input bitwidth                 &8/16  \tnote{3}               &8                                  &8                                                     \\
\arrayrulecolor{black!30}\midrule
\arrayrulecolor{black}
DSPs                                       &3121                          &2048                               &1024                                                  \\
\arrayrulecolor{black!30}\midrule
\arrayrulecolor{black}
ALMs (Altera) / LUTs (AMD)                 &480K                          &-                                  &152K                                                  \\
\arrayrulecolor{black!30}\midrule
\arrayrulecolor{black}
Registers                                  &-                             &-                                  &567K                                                     \\
\arrayrulecolor{black!30}\midrule
\arrayrulecolor{black}
Memories  (20Kb Altera) / (36Kb AMD)    &1457                          &-                                     &2056                                                  \\
\arrayrulecolor{black!30}\midrule
\arrayrulecolor{black}
Frequency (MHz)                            &100                           &200                                &200                                                   \\
\arrayrulecolor{black!30}\midrule
\arrayrulecolor{black}
Model                                      &ResNet-101                    &ResNet-50                          &ResNet-152                                            \\
\arrayrulecolor{black}\midrule
\tabThpt                                   &600                           &287                                &794                                                   \\
\arrayrulecolor{black!30}\midrule
\arrayrulecolor{black}
\smmMacUt                                  &0.961                         &0.351                              &0.969                                                 \\
\arrayrulecolor{black}
\bottomrule
\end{tabular}
\begin{tablenotes}
  \footNoteRefs
\item[3] Weights are quantized to 8 bits and layer input/output is quantized to 8 or 16 bits at different stages.
\end{tablenotes}
\end{threeparttable}
\end{table}


Tables \ref{smm:tab:8}-\ref{smm:tab:ffip} show the \smmArch architectures  integrated into the deep learning system from our previous work \cite{pogue2024fast} compared to state-of-the-art accelerators evaluated on the same FPGA family for the same input bitwidths and similar neural network models.
Integrating the \smmArch multisystolic array design into our deep learning accelerator allowed us to increase the multiplier compute efficiency while also scaling up the computational resources and throughput roof without increasing the minimum supported matrix sizes.
This allowed it to significantly surpass the throughput in our prior work \cite{pogue2024fast} and other state-of-the-art prior works evaluated on the same FPGA family as shown in Tables \ref{smm:tab:8}-\ref{smm:tab:ffip}.
If the design is scaled up using a single-systolic array, the minimum supported matrix size increases and compute resources begin to be underutilized for ResNet execution based on the smaller matrix sizes that its workload decomposes to.
This causes the effective throughput to not increase well despite the design having a larger throughput roof.

The \smmArch?1 32\by32 and \sAlg.\sffip<{}?1 32\by32 designs consume noticeably more memory resources than the \smmArch?2 8\by8 and \sAlg.\sffip<{}?2 8\by8 designs.
However, it is worth noting that this is not due to increased memory requirements, but rather is due to the compiler favouring to swap some register resources for memory resources.
This is because the \smmArch?1 32\by32 and \sAlg.\sffip<{}?1 32\by32 designs have a higher register (and overall area) overhead than the \smmArch?2 8\by8 and \sAlg.\sffip<{}?2 8\by8 designs in order to achieve higher throughput roofs.

In Table \ref{smm:tab:8}, the \smmArch architectures achieve the highest throughput and multiplier compute efficiency compared to the prior works.
The \smmArch?1 and \smmArch?2 architectures' multiplier compute efficiencies in Table \ref{smm:tab:8} approach their limits of 1.14 and 1.31 that are derived in \eq{smm:eq:mu-roof}.
This surpasses the limit of 1 of the baseline \mmArch architectures and prior works that is derived in \eq{smm:eq:mm-mu-roof}, validating \smm's ability to increase multiplier compute efficiency and reduce computational complexity as expected from our analysis.

Table \ref{smm:tab:ffip} shows an example of how \smm can be combined with other algebraic techniques to further increase multiplier compute efficiency limits.
FFIP \cite{pogue2024fast} provides a way to reduce the number
of required multiplications by up to a factor of 2, trading half the multiplications for cheap low-bitwidth additions.
Because of this, the limit for the multiplier compute efficiency metric in \eq{smm:eq:mu} for an FFIP architecture becomes 2, and $2\times(8/7)^r$ for a combined FFIP+\smmArch architecture.
In Table \ref{smm:tab:ffip}, we evaluate architectures that combine FFIP+\smmArch by instantiating \smmArch MXUs that use FFIP MXUs at their lowest level of recursion instead of the conventional \mm?\zero MXUs from \fig{smm:fig:MM-mxu}.
This further increases multiplier compute efficiency compared to a standalone \smmArch or standalone FFIP MXU as seen in the achieved multiplier compute efficiencies of the \sAlg.\sffip<{}?{r} architectures listed in Table \ref{smm:tab:ffip}.

\section{Conclusion}
Strassen's fast matrix multiplication algorithm reduces the complexity of naive matrix multiplication, however, general-purpose hardware is not suitable for achieving the algorithm's promised theoretical speedups.
Furthermore, there is limited prior work on custom hardware architectures designed specifically for executing the algorithm in hardware.
We address this by presenting custom Strassen multisystolic array hardware architectures that are functionally equivalent to conventional multisystolic array designs.
However, they allow the theoretical complexity reductions of \sa to be translated directly into hardware resource savings, even for multiplication of small matrices.

Compared to a conventional multisystolic array design, the proposed architectures implemented on FPGA for 1 and 2 levels of Strassen recursion use 1.14\x and 1.31\x fewer DSP units and an overall comparable amount of soft logic resources when instantiated for multiplying $n$\by$n$ matrices down to sizes $n = 32$ and $n = 24$, respectively.
The proposed systolic array architectures increase conventional multiplications/multiplier/clock cycle limits by a factor of $1.14^r$ for $r$ implemented levels of Strassen recursion.
Furthermore, they allow the throughput per clock cycle roof of an accelerator to double for each implemented level of Strassen recursion without increasing the minimum supported matrix sizes that can be efficiently multiplied.


\bibliographystyle{IEEEtran}
\bibliography{IEEEabrv,bibl}

\begin{IEEEbiography}[{\includegraphics[width=1in,height=1.25in,clip,keepaspectratio]{trevor.eps}}]{Trevor E. Pogue}
  Trevor E. Pogue received the B.Eng. degree in Electrical Engineering and the M.A.Sc. degree in Electrical and Computer Engineering from McMaster University, Hamilton, Canada, in 2016 and 2019, respectively. He worked as an intern at Synopsys and AMD in 2018 and 2022-2023, respectively. He is currently a Ph.D. Candidate in the Department of Electrical and Computer Engineering at McMaster University, Hamilton, Canada. His research interests are in the area of hardware acceleration.
\end{IEEEbiography}

\begin{IEEEbiography}[{\includegraphics[width=1in,height=1.25in,clip,keepaspectratio]{nicola.eps}}]{Nicola
    Nicolici}(S’99-M’00-SM'11)
  Nicola Nicolici (S99-M00-SM’11) received the Dipl.Ing. degree in Computer
  Engineering from the “Politehnica” University of Timisoara, Romania, in 1997
  and the Ph.D. degree in Electronics and Computer Science from the University
  of Southampton, U.K., in 2000. He is currently a Professor with the Department
  of Electrical and Computer Engineering, McMaster University, Hamilton, Canada.
  His research interests are in the area of computer-aided design and test. He
  has authored a number of papers in this area. Dr. Nicolici was the recipient
  of the IEEE TTTC Beausang Award for the Best Student Paper at the
  International Test Conference in 2000 and the Best Paper Award at the IEEE/ACM
  Design Automation and Test in Europe Conference in 2004.
\end{IEEEbiography}

\end{document}
