\section{Related Work}
\label{sect:prior-work}
\subsection{Existing Aortic Segmentation Methods}
\label{sect:related-work}

The aorta can be divided into the ascending aorta, aortic arch, thoracic aorta, abdominal aorta, and iliac arteries. Most existing methods focus on binary segmentation of either the thoracic aorta or the entire aorta, without distinguishing between different aortic branches and zones____. While most studies utilize 3D CT or CTA images, some have explored aortic segmentation in other modalities, such as 3D MRI____ and 3D ultrasound____. Additionally, certain methods address the segmentation of the true and false lumen in aortic dissections. For instance, in____ the ZOZI-seg model was introduced, a two-stage framework that combines a 3D transformer with a 3D U-Net for localized texture refinement. This model, trained on private data, segments the aorta into the true lumen, false lumen, and thrombosis. Similarly, ____ proposed a method that segments the aorta into the true lumen and intraluminal thrombosis, leveraging a private dataset of 70 3D CTA volumes. 

Recently, a few techniques have gone beyond binary segmentation to multi-class segmentation, although they remain focused on chest CTA. For example, in____, a U-Net model with attention gating was developed for multi-class segmentation of the thoracic aorta using a private CTA dataset. Similarly, researchers in____ proposed a CNN-based segmentation model trained on a private dataset of 30 3D chest CTA volumes to segment the aorta (excluding the iliac arteries) into nine zones using a multi-atlas method. As summarized in Table~\ref{tab:literature_review}, existing methods are limited to either binary segmentation of the aorta or multi-class segmentation focused on the thoracic aorta. This highlights a critical unmet need for a large, standardized, open-source dataset enabling multi-class segmentation of the entire aorta, its branches, and zones, from the ascending aorta to the iliac arteries.

\begin{table*}[!hbt]
\caption{Summary of existing aortic segmentation studies. CT: computed tomography; CTA: computed tomography angiography; MRI: magnetic resonance imaging; TL: true lumen; FL: false lumen.}
\centering
\renewcommand{\arraystretch}{1.2} % Adds row cushion for better readability
%\small % Set smaller font size
\footnotesize
\begin{tabular}{|c|c|c|c|c|c|}
\hline
\textbf{Model} & \textbf{Modality} & \textbf{Image Dimension} & \textbf{Dataset Type} & \textbf{Segmented Object(s)} & \textbf{\# of Image Volumes} \\ \hline
____          & CTA            & 3D          & Private        & TL and FL                     & 276 \\ \hline
____          & CT             & 3D          & Private        & TL, FL, and thrombosis         & 253 \\ \hline
____       & CTA            & 3D          & Private        & TL and thrombosis              & 70  \\ \hline
____      & CTA            & 3D          & Private        & Aorta                          & 42  \\ \hline
____           & CT             & 3D          & Private        & Nine thoracic aortic zones     & 704 \\ \hline
____         & CTA            & 3D          & Private        & TL and FL                      & 120 \\ \hline
____       & CTA            & 3D          & Public         & TL and FL                      & 33  \\ \hline
____      & CTA            & 3D          & Public         & TL and FL                      & 100 \\ \hline
____ & CTA      & 3D          & Public         & Aorta                          & 38  \\ \hline
____         & CT             & 2D          & Private        & TL and FL                      & 20  \\ \hline
____ & CT             & 3D          & Private        & Three thoracic aortic zones    & 24  \\ \hline
____ & CTA            & 3D          & Private        & Six thoracic aortic zones      & 194 \\ \hline
____    & 4D flow MRI    & 3D          & Private        & Aorta                          & 114 \\ \hline
____  & CTA            & 2D and 3D   & Private        & TL and FL                      & 21  \\ \hline
____     & CT             & 3D          & Private        & TL and FL                      & 267 \\ \hline
____   & CTA            & 3D          & Public         & Aorta                          & 56  \\ \hline
____            & CTA            & 3D          & Private        & TL and FL                      & 153 \\ \hline
____           & CTA            & 3D          & Private        & TL and FL                      & 139 \\ \hline
____           & MRI            & 3D          & Private        & Thoracic aorta                 & 391 \\ \hline
____     & CTA            & 3D          & Private        & TL and FL                      & 463 \\ \hline
____   & CTA            & 3D          & Private        & Aorta                          & 191 \\ \hline
____       & CTA            & 3D          & Private        & Thoracic aorta                 & 72  \\ \hline
\end{tabular}
\label{tab:literature_review}
\end{table*}

% \begin{table*}[!hbt]  
% \caption{Summary of existing aortic segmentation studies. CT: computed tomograph; CTA: computed tomography angiography; MRI: magnetic resonance imaging; TL: true lumen; FL: false lumen.}
%     \centering
%     \renewcommand{\arraystretch}{1.2} % adds row cushion
%     \small
%     \begin{tabular}{c | c | c | c | c | c | c}
%         \toprule[0.8pt]
%         \textbf{Model} & \textbf{Modality} & \textbf{Image Dimension} & \textbf{Dataset Type} & \textbf{Segmented Object(s)} & \textbf{\makecell{\# of Image Volumes}} \\
%         \midrule[0.8pt]
%         ____             & \makecell{CTA}           & \makecell{3D}          & \makecell{Private}  &  \makecell{TL and FL} & 276 \\
%         \hline
%         ____             & \makecell{CT}            & \makecell{3D}          & \makecell{Private}  & \makecell{TL, FL, and \\ thrombosis} & 253 \\
%         \hline
%         ____          & \makecell{CTA}           & \makecell{3D}          & \makecell{Private} & \makecell{TL and \\thrombosis} & 70 \\
%         \hline
%         ____         & \makecell{CTA}           & \makecell{3D}          &  \makecell{Private}  & \makecell{Aorta} & 42 \\
%         \hline
%         ____              & \makecell{CT}            & \makecell{3D}          & \makecell{Private}  & \makecell{Nine thoracic \\ aortic zones} & 704 \\
%         \hline
%         ____            & \makecell{CTA}           & \makecell{3D}          & \makecell{Private}  & \makecell{TL and FL} & 120 \\
%         \hline
%         ____          & \makecell{CTA}           & \makecell{3D}          & \makecell{Public}  & \makecell{TL and FL} & 33 \\
%         \hline
%         ____         & \makecell{CTA}           & \makecell{3D}          & \makecell{Public}  & \makecell{TL and FL} & 100 \\
%         \hline
%         ____& \makecell{CTA}           & \makecell{3D}          & \makecell{Public}  & \makecell{Aorta} & 38 \\
%         \hline
%         ____            & \makecell{CT}            & \makecell{2D}          & \makecell{Private} & \makecell{TL and FL} & \makecell{20} \\
%         \hline
%         ____    & \makecell{CT}            & \makecell{3D}          & \makecell{Private} & \makecell{Three thoracic \\ aortic zones} & 24 \\
%         \hline
%         ____    & \makecell{CTA}           & \makecell{3D}          & \makecell{Private} & \makecell{Six thoracic \\ aortic zones} & 194 \\
%         \hline
%         ____       & \makecell{4D flow MRI}   & \makecell{3D}          & \makecell{Private} & \makecell{Aorta} & 114 \\
%         \hline
%         ____     & \makecell{CTA}           & \makecell{2D and 3D}   & \makecell{Private} & \makecell{TL and FL} & 21 \\
%         \hline
%         ____        & \makecell{CT}            & \makecell{3D}          & \makecell{Private}  & \makecell{TL and FL} & 267 \\
%         \hline
%         ____      & \makecell{CTA}            & \makecell{3D}          & \makecell{Public}  & \makecell{Aorta} & 56 \\
%         \hline
%         ____               & \makecell{CTA}           & \makecell{3D}          & \makecell{Private} & \makecell{TL and FL} & 153 \\
%         \hline
%         ____              & \makecell{CTA}           & \makecell{3D}          & \makecell{Private} & \makecell{TL and FL} & 139\\
%         \hline
%         ____              & \makecell{MRI}           & \makecell{3D}          & \makecell{Private} & \makecell{Thoracic aorta} & 391\\
%         \hline
%         ____        & \makecell{CTA}           & \makecell{3D}          & \makecell{Private}& \makecell{TL and FL} & 463\\
%         \hline
%         ____      & \makecell{CTA}           & \makecell{3D}          & \makecell{Private} & \makecell{Aorta}  & 191 \\
%         \hline
%         ____          & \makecell{CTA}           & \makecell{3D}          & \makecell{Private}  & \makecell{Thoracic aorta} & 72 \\
%         \bottomrule[0.8pt]
%     \end{tabular}
%     \label{tab:literature_review}                         	
% \end{table*}


\subsection{Existing Aortic Segmentation Datasets}
\label{sect:about-the-dataset}
As summarized in Table~\ref{tab:literature_review}, most studies rely on proprietary datasets that are not publicly accessible. However, a few publicly available datasets exist for aortic segmentation. The ImageTBAD dataset____ contains annotations for 100 CTA images from patients with type B aortic dissections treated at the Guangdong Provincial People’s Hospital in China. Of these, 32 images are annotated for true lumen and false lumen, while the remaining 68 images include annotations for true lumen, false lumen, and false lumen thrombus.  Despite its value, this dataset has notable limitations: it covers only a portion of the aorta (Figure~\ref{fig:aorta-datasets-for-comparison}(a)) and focuses solely on aortic dissections, omitting branches and zones critical for monitoring zonal progression. The AVT dataset____ is a publicly available multi-center dataset of 56 CTA images from three sources: 20 cases from the KiTS19 Grand Challenge____, 18 from the Rider Lund CT dataset____, and 18 from Dongyang Hospital. Images from KiTS and Dongyang hospital are free of pathologies, while those from the Rider dataset include aortic dissections and abdominal aortic aneurysms. Unlike ImageTBAD, the AVT dataset spans the entire aorta and its branches. However, it annotates the aorta and all its branches as a single label, without distinguishing individual branches and zones (Figure~\ref{fig:aorta-datasets-for-comparison} (a)-(c)). The recently released Aortic Dissection Dataset____ comprises 40 CTA images. Unlike the ImageTBAD dataset, it includes the entire aorta and some aortic branches in most cases (Figure \ref{fig:aorta-datasets-for-comparison}) but provides labels only for true lumen and false lumen, omitting individual branches and zones.

\begin{figure*}[!hbt]%
\centering
\includegraphics[width=0.75\textwidth]{images/DatasetComparison.PNG}
\caption{Comparison of various publicly available aortic segmentation datasets}
\label{fig:aorta-datasets-for-comparison}%
\end{figure*}

\subsection{Relevant Image Segmentation Challenges}
\label{sect:aorta-related-challenges}
Medical image segmentation has been the focus of numerous medical imaging challenges, including notable ones such as the Head and Neck Organ-at-Risk CT \& MR Segmentation Challenge (HaN-Seg)____, the Fast and Low-Resource Semi-Supervised Abdominal Organ Segmentation in CT (FLARE 2022)____, the Segmentation of Organs-at-Risk and Gross Tumor Volume of NPC for Radiotherapy Planning (SegRap2023)____, and the KiTS21 Challenge____. Despite their contributions, none addressed the segmentation of the aorta. The 2023 Segmentation of the Aorta (SEG.A.2023) challenge____ aimed to advance methods for automated binary segmentation of the aortic vessel tree in CTA images. The challenge provided participants with a dataset of 56 annotated CTA images from three institutions to develop their models. Although SEG.A.2023 marked significant progress, it was limited to binary segmentation of the aorta versus non-aorta and did not include detailed annotations for aortic branches or zones.