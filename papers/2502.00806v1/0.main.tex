%%
%% This is file `sample-sigconf.tex',
%% generated with the docstrip utility.
%%
%% The original source files were:
%%
%% samples.dtx  (with options: `all,proceedings,bibtex,sigconf')
%% 
%% IMPORTANT NOTICE:
%% 
%% For the copyright see the source file.
%% 
%% Any modified versions of this file must be renamed
%% with new filenames distinct from sample-sigconf.tex.
%% 
%% For distribution of the original source see the terms
%% for copying and modification in the file samples.dtx.
%% 
%% This generated file may be distributed as long as the
%% original source files, as listed above, are part of the
%% same distribution. (The sources need not necessarily be
%% in the same archive or directory.)
%%
%%
%% Commands for TeXCount
%TC:macro \cite [option:text,text]
%TC:macro \citep [option:text,text]
%TC:macro \citet [option:text,text]
%TC:envir table 0 1
%TC:envir table* 0 1
%TC:envir tabular [ignore] word
%TC:envir displaymath 0 word
%TC:envir math 0 word
%TC:envir comment 0 0
%%
%%
%% The first command in your LaTeX source must be the \documentclass
%% command.
%%
%% For submission and review of your manuscript please change the
%% command to \documentclass[manuscript, screen, review]{acmart}.
%%
%% When submitting camera ready or to TAPS, please change the command
%% to \documentclass[sigconf]{acmart} or whichever template is required
%% for your publication.
%%
%%
\documentclass[sigconf]{acmart}

%%
%% \BibTeX command to typeset BibTeX logo in the docs
\AtBeginDocument{%
  \providecommand\BibTeX{{%
    Bib\TeX}}}

%% Rights management information.  This information is sent to you
%% when you complete the rights form.  These commands have SAMPLE
%% values in them; it is your responsibility as an author to replace
%% the commands and values with those provided to you when you
%% complete the rights form.
\setcopyright{acmlicensed}
\copyrightyear{2025}
\acmYear{2025}
\setcopyright{cc}
\setcctype{by}
\acmConference[WWW '25]{Proceedings of the ACM Web Conference 2025}{April 28-May 2, 2025}{Sydney, NSW, Australia}
\acmBooktitle{Proceedings of the ACM Web Conference 2025 (WWW '25), April 28-May 2, 2025, Sydney, NSW, Australia}
\acmDOI{10.1145/3696410.3714818}
\acmISBN{979-8-4007-1274-6/25/04}

\settopmatter{printacmref=true}

%% These commands are for a PROCEEDINGS abstract or paper.
% \acmConference[Conference acronym 'XX]{Make sure to enter the correct
%   conference title from your rights confirmation emai}{June 03--05,
%   2018}{Woodstock, NY}
%%
%%  Uncomment \acmBooktitle if the title of the proceedings is different
%%  from ``Proceedings of ...''!
%%
%%\acmBooktitle{Woodstock '18: ACM Symposium on Neural Gaze Detection,
%%  June 03--05, 2018, Woodstock, NY}
% \acmISBN{978-1-4503-XXXX-X/18/06}


%%
%% Submission ID.
%% Use this when submitting an article to a sponsored event. You'll
%% receive a unique submission ID from the organizers
%% of the event, and this ID should be used as the parameter to this command.
%%\acmSubmissionID{123-A56-BU3}

%%
%% For managing citations, it is recommended to use bibliography
%% files in BibTeX format.
%%
%% You can then either use BibTeX with the ACM-Reference-Format style,
%% or BibLaTeX with the acmnumeric or acmauthoryear sytles, that include
%% support for advanced citation of software artefact from the
%% biblatex-software package, also separately available on CTAN.
%%
%% Look at the sample-*-biblatex.tex files for templates showcasing
%% the biblatex styles.
%%

%%
%% The majority of ACM publications use numbered citations and
%% references.  The command \citestyle{authoryear} switches to the
%% "author year" style.
%%
%% If you are preparing content for an event
%% sponsored by ACM SIGGRAPH, you must use the "author year" style of
%% citations and references.
%% Uncommenting
%% the next command will enable that style.
%%\citestyle{acmauthoryear}
\usepackage{enumitem}

\usepackage{amsmath}
\newtheorem{problem}{Problem}
\newtheorem{definition}{Definition}

\usepackage{xspace}
\newcommand{\model}{UniGraph2\xspace}
\newcommand{\vpara}[1]{\vspace{0.04in}\noindent\textbf{#1}\xspace}

\newcommand\todo[1]{\textcolor{red}{[TODO: #1 ]}}

% add background color to table
\usepackage{color}
\usepackage{colortbl}
\definecolor{Gray}{gray}{0.9}

\usepackage{multirow}

\usepackage{arydshln}


%%%%% NEW MATH DEFINITIONS %%%%%

% \usepackage{amsmath,amsfonts,bm}
\usepackage{amsmath,amsfonts}

\usepackage{pifont}


\newcommand{\R}{\mathbb{R}}


\def\va{{\mathbf{a}}}
\def\vg{{\mathbf{g}}}

% Sets
\def\sR{\mathbb{R}}
\def\sC{\mathbb{C}}
\def\sZ{\mathbb{Z}}
\def\sN{\mathbb{N}}
\def\sQ{\mathbb{Q}}

\def\sS{\mathcal{S}}



% Vectors
\def\vzero{{\mathbf{0}}}
\def\vone{{\mathbf{1}}}
\def\vmu{{\mathbf{\mu}}}
\def\vtheta{{\mathbf{\theta}}}
\def\va{{\mathbf{a}}}
\def\vb{{\mathbf{b}}}
\def\vc{{\mathbf{c}}}
\def\vd{{\mathbf{d}}}
\def\ve{{\mathbf{e}}}
\def\vf{{\mathbf{f}}}
\def\vg{{\mathbf{g}}}
\def\vh{{\mathbf{h}}}
\def\vi{{\mathbf{i}}}
\def\vj{{\mathbf{j}}}
\def\vk{{\mathbf{k}}}
\def\vl{{\mathbf{l}}}
\def\vm{{\mathbf{m}}}
\def\vn{{\mathbf{n}}}
\def\vo{{\mathbf{o}}}
\def\vp{{\mathbf{p}}}
\def\vq{{\mathbf{q}}}
\def\vr{{\mathbf{r}}}
\def\vs{{\mathbf{s}}}
\def\vt{{\mathbf{t}}}
\def\vu{{\mathbf{u}}}
\def\vv{{\mathbf{v}}}
\def\vw{{\mathbf{w}}}
\def\vx{{\mathbf{x}}}
\def\vy{{\mathbf{y}}}
\def\vz{{\mathbf{z}}}
\def\vzeta{{\mathbf{\zeta}}}

% Matrix
\def\mA{{\mathbf{A}}}
\def\mB{{\mathbf{B}}}
\def\mC{{\mathbf{C}}}
\def\mD{{\mathbf{D}}}
\def\mE{{\mathbf{E}}}
\def\mF{{\mathbf{F}}}
\def\mG{{\mathbf{G}}}
\def\mH{{\mathbf{H}}}
\def\mI{{\mathbf{I}}}
\def\mJ{{\mathbf{J}}}
\def\mK{{\mathbf{K}}}
\def\mL{{\mathbf{L}}}
\def\mM{{\mathbf{M}}}
\def\mN{{\mathbf{N}}}
\def\mO{{\mathbf{O}}}
\def\mP{{\mathbf{P}}}
\def\mQ{{\mathbf{Q}}}
\def\mR{{\mathbf{R}}}
\def\mS{{\mathbf{S}}}
\def\mT{{\mathbf{T}}}
\def\mU{{\mathbf{U}}}
\def\mV{{\mathbf{V}}}
\def\mW{{\mathbf{W}}}
\def\mX{{\mathbf{X}}}
\def\mY{{\mathbf{Y}}}
\def\mZ{{\mathbf{Z}}}
\def\mBeta{{\mathbf{\beta}}}
\def\mPhi{{\mathbf{\Phi}}}
\def\mLambda{{\mathbf{\Lambda}}}
\def\mSigma{{\mathbf{\Sigma}}}


% Expectation
% \def\eE{\mathop{\mathbb{E}}\limits}
\def\eE{\mathbb{E}}

% Probability
\def\pP{\mathbb{P}}

% Tilde
\def\tf{\tilde{f}}
\def\tS{\tilde{S}}
\def\wtF{\widetilde{\mathcal{F}}}
\def\whR{\widehat{R}}
\def\tvx{\tilde{\mathbf{x}}}
\def\ty{\tilde{y}}


\def\defeq{\overset{\textup{def}}{=}}
% \def\defeq{\overset{.}{=}}
\def\defone{\overset{\text{\ding{172}}}{=}}
\def\deftwo{\overset{\text{\ding{173}}}{=}}
\def\leqone{\overset{\text{\ding{172}}}{\leq}}
\def\leqtwo{\overset{\text{\ding{173}}}{\leq}}
\def\leqthree{\overset{\text{\ding{174}}}{\leq}}
\def\leqfour{\overset{\text{\ding{175}}}{\leq}}
\def\eqone{\overset{\text{\ding{172}}}{=}}
\def\eqtwo{\overset{\text{\ding{173}}}{=}}
\def\eqthree{\overset{\text{\ding{174}}}{=}}
\def\eqfour{\overset{\text{\ding{175}}}{=}}
\def\geqfive{\overset{\text{\ding{176}}}{\geq}}

%%
%% end of the preamble, start of the body of the document source.
\begin{document}

%%
%% The "title" command has an optional parameter,
%% allowing the author to define a "short title" to be used in page headers.
\title{\model: Learning a Unified Embedding Space to Bind Multimodal Graphs}

\author{Yufei He}
\affiliation{National University of Singapore\country{Singapore}}
\email{yufei.he@u.nus.edu}

\author{Yuan Sui}
\affiliation{National University of Singapore\country{Singapore}}
\email{yuan.sui@u.nus.edu}

\author{Xiaoxin He}
\affiliation{National University of Singapore\country{Singapore}}
\email{he.xiaoxin@u.nus.edu}

\author{Yue Liu}
\affiliation{National University of Singapore\country{Singapore}}
\email{yliu@u.nus.edu}

\author{Yifei Sun}
\affiliation{Zhejiang University\country{China}}
\email{yifeisun@zju.edu.cn}

\author{Bryan Hooi}
\affiliation{National University of Singapore\country{Singapore}}
\email{bhooi@comp.nus.edu.sg}

%%
%% The "author" command and its associated commands are used to define
%% the authors and their affiliations.
%% Of note is the shared affiliation of the first two authors, and the
%% "authornote" and "authornotemark" commands
%% used to denote shared contribution to the research.


%%
%% By default, the full list of authors will be used in the page
%% headers. Often, this list is too long, and will overlap
%% other information printed in the page headers. This command allows
%% the author to define a more concise list
%% of authors' names for this purpose.
% \renewcommand{\shortauthors}{Trovato et al.}
\renewcommand{\shortauthors}{Yufei He et al.}
%%
%% The abstract is a short summary of the work to be presented in the
%% article.
\begin{abstract}

% Graph foundation models aim xxx,

% However, they primarily focus on TAGs while can not handle MMG,

% To this end, we propose UniGraph2, by xxx

% Concretely, 

% Experiments,

Existing foundation models, such as CLIP, aim to learn a unified embedding space for multimodal data, enabling a wide range of downstream web-based applications like search, recommendation, and content classification. However, these models often overlook the inherent graph structures in multimodal datasets, where entities and their relationships are crucial. %For example, in social networks, users are connected through friendships, follows, or interactions, and share content in various modalities like text and images. 
Multimodal graphs (MMGs) represent such graphs where each node is associated with features from different modalities, while the edges capture the relationships between these entities.
% such data by combining diverse modalities with graph structures that capture the relationships between entities.
% in e-commerce platforms, products are linked based on co-purchase patterns, user reviews, and shared attributes, encompassing multiple data types.
% \todo{introduce MMG, and real applications}However, none of these models consider the graph structure inherent in many multimodal datasets, where entities and their relationships are critical. 
On the other hand, existing graph foundation models primarily focus on text-attributed graphs (TAGs) and are not designed to handle the complexities of MMGs. To address these limitations, we propose \model\footnote[1]{The code is available at \url{https://github.com/yf-he/UniGraph2}}, a novel cross-domain graph foundation model that enables general representation learning on MMGs, providing a unified embedding space. \model employs modality-specific encoders alongside a graph neural network (GNN) to learn a unified low-dimensional embedding space that captures both the multimodal information and the underlying graph structure. We propose a new cross-domain multi-graph pre-training algorithm at scale to ensure effective transfer learning across diverse graph domains and modalities. Additionally, we adopt a Mixture of Experts (MoE) component to align features from different domains and modalities, ensuring coherent and robust embeddings that unify the information across modalities. Extensive experiments on a variety of multimodal graph tasks demonstrate that UniGraph2 significantly outperforms state-of-the-art models in tasks such as representation learning, transfer learning, and multimodal generative tasks, offering a scalable and flexible solution for learning on MMGs.



% Graph foundation models (GFMs), which aim to learn knowledge from graphs in different domains and transfer it to various tasks, have become a promising direction for web mining. However, existing GFMs, e.g., UniGraph, primarily focus on text-attributed graphs (TAGs) and fail to handle the data in various modalities, e.g., text, image, and video on the Web, limiting the applicability. To this end, we propose a new GFM method, termed \model, to achieve the general ability on multi-modal graphs (MMGs), by refactoring UniGraph from three key aspects, including architecture, pre-training, and alignment. Concretely, for the architecture, we first design the GNN-based modulation-specific encoders to learn a unified representation that captures both modality information and structure information. Then, for the pre-training, we propose a large-scale cross-domain pre-training algorithm to learn knowledge across diverse graph domains and modalities. In addition, for the alignment, we introduce a Mixture-of-Expert (MoE) component to align features from different domains, ensuring coherent and robust embeddings that unify the information across modalities. By these designs, \model is powered by the strong multimodal representation learning ability on graph data, improving the applicability and performance in web mining. Extensive experiments on xx multimodal tasks demonstrate the superiority of our proposed method. Remarkablely, xxx

% Extensive experiments on a variety of multimodal graph tasks demonstrate that UniGraph2 significantly outperforms state-of-the-art models in tasks such as node classification, retrieval, and transfer learning, offering a scalable and flexible solution for learning on multimodal graphs.


\end{abstract}


%%
%% The code below is generated by the tool at http://dl.acm.org/ccs.cfm.
%% Please copy and paste the code instead of the example below.
%%
% \vspace{-60mm}
\begin{CCSXML}
<ccs2012>
   <concept>
       <concept_id>10002951.10003227.10003351</concept_id>
       <concept_desc>Information systems~Data mining</concept_desc>
       <concept_significance>500</concept_significance>
       </concept>
   <concept>
       <concept_id>10010147.10010257.10010293.10010294</concept_id>
       <concept_desc>Computing methodologies~Neural networks</concept_desc>
       <concept_significance>500</concept_significance>
       </concept>
   <concept>
       <concept_id>10002951.10003260.10003282.10003292</concept_id>
       <concept_desc>Information systems~Social networks</concept_desc>
       <concept_significance>300</concept_significance>
       </concept>
 </ccs2012>
\end{CCSXML}

\ccsdesc[500]{Information systems~Data mining}
\ccsdesc[500]{Computing methodologies~Neural networks}
\ccsdesc[300]{Information systems~Social networks}

% \begin{figure}
%     \centering
%     \includegraphics[width=1.0\linewidth]{figs/fig1.pdf}
%     \vskip -0.1in
%     \caption{\model binds multimodal graphs from different graph domains to a unified embedding space, enabling diverse downstream tasks.}
%     \label{fig:fig1}
%     \vspace{-6mm}
% \end{figure}

%%
%% Keywords. The author(s) should pick words that accurately describe
%% the work being presented. Separate the keywords with commas.
\keywords{Pre-Training; Graph Foundation Models; Multimodal Learning}


%%
%% This command processes the author and affiliation and title
%% information and builds the first part of the formatted document.
\maketitle
% \vspace{-4mm}
% \vpara{Relevance to the Web and to the track.}
% UniGraph2 is a large pre-trained model that processes and unifies multimodal graph data from various web domains, making it applicable to web-based applications like search, recommendation, and content classification.
% UniGraph2 is a large pre-trained model designed to process and unify multimodal graph data derived from various web domains, making it directly relevant to web-based applications such as search, recommendation, and content classification. 
% Its ability to generalize across diverse web data sources and modalities aligns it with the goals of Web mining and content analysis.

\section{Introduction}

\begin{figure}[h]
    \centering
    \begin{overpic}[trim=0cm 0cm 0cm 0cm,clip,angle=0,origin=c,width=.4\linewidth]{images/teaser_absolute.png}
        %  trim={<left> <lower> <right> <upper>}
        %  \put(horiz, vert)
        %  \put(horiz, vert){\rotatebox{90}{Text}}
        %
        \put(107, 32){$\mathbf{\to}$}
    \end{overpic}\hspace{1cm}
    \begin{overpic}[trim=0cm 0cm 0cm 0cm,clip,angle=0,origin=c,width=.4\linewidth]{images/teaser_translated_yellow.png}
        %  trim={<left> <lower> <right> <upper>}
        %  \put(horiz, vert)
        %  \put(horiz, vert){\rotatebox{90}{Text}}
        %
    \end{overpic}
    \caption{Using translation methods, a controller trained on an environment with a given visual variation \textit{(left)} can be reused without any training or fine-tuning on a different environment (\textit{right}) with comparable performance. In red we see the trajectory of a car driven by the same controller when connected to two different encoders, one for each visual variation.
    }
    \label{fig:teaser}
\end{figure}

Deep Reinforcement Learning (RL) has enabled agents to achieve remarkable performance in complex decision-making tasks, from robotic manipulation to high-dimensional games (Mnih et al., 2015; Silver et al., 2017). 
Although recent RL techniques achieved strong improvements over sample efficiency \citep{yarats2021drqv2, kostrikov2020image}, training new agents remains a costly process, both in computational and temporal terms.
Despite these advances, most methods still require at least partial retraining when dealing with domain shifts such as visual appearance, reward functions, or action spaces \citep{pmlr-v97-cobbe19a, zhang2020learning}. These domain changes typically require expensive retraining, which can be prohibitive for real-world settings that require millions of interactions.

A variety of approaches have been proposed to address these shifting conditions. Domain randomization \citep{tobin2017domain, sadeghi2016cad2rl} trains agents across diverse visual styles or physics settings, promoting invariant features but demanding broader coverage of possible variations. Multi-task RL \citep{parisotto2015actor, teh2017distral} attempts to learn shared representations across multiple tasks.

In the supervised setting, recent representation learning techniques \citep{Moschella2022-yf,maiorca2023latent, norelli2022b, cannistraci2023bricks}, show that it is possible to zero-shot recombine encoders and decoders to perform new tasks across different modalities (images, text..) and tasks (classification, reconstruction) and even architectures.
In RL, methods adopting the relative representation framework \citep{Moschella2022-yf} have shown promising results in adapting encoders to different controllers with zero or few-shots adaptation, for robotic control from proprioceptive states \citep{jian2021adversarial} or for playing games in the Gymnasium suite \citep{towers2024gymnasium} from pixels \citep{ricciardi2025r3lrelativerepresentationsreinforcement}.
These methods, however, still require training models to use the new relative representations.

By contrast, \cite{maiorca2023latent} suggest that modules from independently trained neural networks can be connected via a simple linear or affine transformation, with no training constraint or fine-tuning required, if such transformations can be reliably estimated from a small set of “anchor” samples, pairs of states or observations deemed semantically equivalent.

Our main contribution is the implementation of a RL method based on semantic alignment to map between latent spaces of different neural models, so that their encoders and controllers can be stitched with the goal of creating new agents that can act on visual-task combinations never seen together in training. This includes the use of the transformations to map modules from different networks, and the collection of anchor samples used to estimate these transformations. We call our method Semantic Alignment for Policy Stitching (\textbf{SAPS}).
We perform analyses and empirical tests on the CarRacing and LunarLander environments to show the performance of new agents created via zero-shot stitching of encoders and controllers trained on different visual-task variations, demonstrating significant gains compared to existing zero-shot methods.
%\clearpage
\section{Related Work} \label{related}




% \subsection{Benchmarks in Coding Scenarios}
% \begin{enumerate}
%     \item Code Generation
%     \item Bug Fixing
% \end{enumerate}

% \subsection{Large Language Model Agents}

% At the heart of the LLM Agent is an Agent Core, which coordinates the core \textit{logic} and \textit{behavioral} characteristics of the agent. In addition, the Agent includes the following key components:

% \begin{itemize}
%     \item Memory Module: It consists of both short-term and long-term memory components that record the agent's internal logs and interactions with the user.
%     \item Tools: These are the tools that the agent can use to perform tasks, usually specific third-party APIs.
%     \item Planning Module: This is used for solving complex problems, such as decomposing tasks and problems, reflexivity or critique.
% \end{itemize}

% \subsection{Multi Agent Collaboration Framework}

% MetaGPT \url{https://arxiv.org/abs/2308.00352}


\parabf{Coding \llm{s}.}
Large Language Models (\llm{s}) have become the go-to solution for a wide array of coding tasks due to their exceptional performance in both code generation and comprehension~\cite{codex}. These models have been successfully applied to various software engineering activities, including program synthesis~\cite{patton2024programming, codex, li2022competition, iyer2018mapping}, code translation~\cite{pan2024lost, roziere2020unsupervised, roziere2021leveraging}, program repair~\cite{xia2023repairstudy, chatrepair, monperrus2018living, bouzenia2024repairagent}, and test generation~\cite{titanfuzz, fuzz4all, deng2023fuzzgpt, lemieux2023codamosa, kang2023testing}. Beyond general-purpose \llm{s}, specialized models have been developed by further training on extensive datasets of open-source code snippets. Notable examples of these code-specific \llm{s} include \codex~\cite{codex}, \codellama~\cite{codellama}, StarCoder~\cite{starcoder,starcodertwo}, and \deepseek~\cite{deepseek}. Additionally, instruction-following code models have emerged, refined through instruction-tuning techniques. These include models such as \codellamainstruct~\cite{codellama}, \deepseekinstruct~\cite{deepseek}, \wizardcoder~\cite{wizardcoder}, \magicoder~\cite{magicoder}, and OpenCodeInterpreter~\cite{zheng2024opencodeinterpreter}.

\parabf{Benchmarking \llm-based coding tasks.}
To assess the capabilities of \llm{s} in coding, a variety of benchmarks have been proposed. Among the most widely utilized are \humaneval~\cite{codex} and \mbpp~\cite{austin2021program}, which are handcrafted benchmarks for code generation that include test cases to validate the correctness of \llm outputs. Other benchmarks have been developed to offer more rigorous tests~\cite{evalplus}, cover additional programming languages~\cite{zheng2023codegeex,cassano2023multipl}, and address different programming domains~\cite{livecodebench, hendrycksapps2021, codecontest, ds1000, arcade}.

More recently, research has shifted towards evaluating \llm{s} on real-world software engineering challenges by operating on entire code repositories rather than isolated coding problems~\cite{swebench, zhang2023repocoder, liu2023repobench}. A notable benchmark in this area is \swebench~\cite{swebench}, which includes tasks requiring repository modifications to resolve actual GitHub issues. The authors of \swebench have also released a more focused subset, \swebenchlite~\cite{swebenchlite}, which contains 300 problems centered on bug fixing that only involves single-file modifications in the ground truth patches. ML-Bench \cite{liu2023mlbench} is a benchmark for evaluating large language models and agents for Machine Learning tasks on reporitory-level code. It involves 18 repositories and focuses on code generation and interactions with Jupyter Notebooks.

\parabf{Repository-level coding.}
The rise of agent-based frameworks~\cite{xi2023rise} has spurred the development of agent-based approaches to software engineering tasks. Devin~\cite{devinwebpage} (and its open-source counterpart OpenDevin~\cite{opendevin}) is among the first comprehensive \llm agent-based frameworks. Devin employs agents to first perform task planning based on user requirements, then allows them to use tools like file editors, terminals, and web search engines to iteratively execute the tasks. \sweagent~\cite{sweagent} introduces a custom agent-computer interface (ACI), enabling the \llm agent to interact with the repository environment through actions like reading and editing files or running bash commands. Another agent-based approach, \autocoderover~\cite{autocoderover}, equips the \llm agent with specific APIs (e.g., searching for methods within certain classes) to effectively identify the necessary modifications for issue resolution. Beside these examples, a variety of other agent-based approaches have been developed in both open-source~\cite{aidar} and commercial products~\cite{bouzenia2024repairagent, coder, repounderstander, lingma, factorydroid, ibmagent, opencsgstarship, marscode, amazonqdeveloper}.

% Unlike these agent-based methods, \tech offers a straightforward and cost-efficient solution for addressing real-world software engineering challenges. Our work is the first to demonstrate that an \emph{agentless} approach can achieve comparable performance without the need for complex tools or modeling intricate environment behavior and feedback.

Unlike existing benchmarks and agent-based frameworks, which focus on the code generation/completion tasks, our proposed \model and \agent focus on the code deployment task, which is under-studied in the field.
%\clearpage

% 加一节,怎么区分确定性和不确定性

\section{Methodology}


To achieve effective probabilistic predictions, we propose CoST that simultaneously leverages the advantages of both deterministic and probabilistic models. Our approach involves two stages. In the first stage, the deterministic model is pretrained to predict the conditional mean that captures the primary patterns. In the second stage, the parameters of the deterministic model are frozen, and the scale-aware diffusion model, constrained by a customized fluctuation scale, is jointly trained to model residual distributions that reflect random fluctuations.   
Figure~\ref{fig:model} illustrates an overview of our model.


\subsection{Mean-Residual Decomposition}

For urban spatiotemporal probabilistic prediction, current approaches typically employ a single probabilistic model to capture the full distribution of data, incorporating both the primary spatiotemporal patterns and the random fluctuations. However, it is challenging to model both of these distributions. Inspired by~\cite{mardani2023residual} and the Reynolds decomposition in fluid dynamics~\cite{pope2001turbulent}, we propose to decompose the target data \(\mathbf{x}^{ta}\) as follows:
\begin{equation}
 \mathbf{x}^{ta} = \underbrace{\mathbb{E}[\mathbf{x}^{ta}|\mathbf{x}^{co}]}_{\substack{:=\boldsymbol{\mu}(Deterministic)}} + \underbrace{(\mathbf{x}^{ta}-\mathbb{E}[\mathbf{x}^{ta}|\mathbf{x}^{co}])}_{\substack{:=\mathbf{r}(Probabilistic)}},
\end{equation}
where \(\boldsymbol{\mu}\) is the conditional mean representing the primary patterns, and \(\mathbf{r}\) is the residual representing the random variations. Our core idea is that if a deterministic model can accurately predict the conditional mean, that is, \(\boldsymbol{\mu}\approx\mathbb{E}_{\theta}[\mathbf{x}^{ta}|\mathbf{x}]\), then the probabilistic model only needs to focus on learning the simpler residual distribution, thus combining the strengths of both models to enhance the probabilistic prediction capability.









\subsection{Mean Prediction via Deterministic Model}

We require a deterministic model that can accurately predict the conditional mean to align with our research hypothesis, and also maintain high predictive efficiency to avoid additional increases in training and inference time. Therefore, we select the MLP-based STID model as our mean prediction module.
In the first stage of training, we pretrain the model for 50 epochs to effectively capture the primary spatiotemporal patterns. Specifically, we input historical conditional data \(\mathbf{x}^{co}\) into the STID model to obtain the conditional mean output \(\mathbb{E}_{\theta}[\mathbf{x}^{ta}|\mathbf{x}^{co}]\).

The STID model is pretrained by optimizing the following loss function:

\begin{equation}
\label{eq:loss2}
   \mathcal{L}_{2}  = \left\| \mathbb{E}_{\theta}[\mathbf{x}^{ta}|\mathbf{x}^{co}] - \mathbf{x}^{ta} \right\|_2^2 .
\end{equation}

\subsection{Residual Learning via Diffusion Model}
Diffusion models have achieved significant success in probabilistic modeling. In this work, we employ a diffusion model for probabilistic prediction, training it to learn the residual distribution:
\begin{equation}
\label{eq:one-setp-forward}
    \mathbf{r}_{ta}=\mathbf{x}^{ta}-\mathbb{E}_{\theta}[\mathbf{x}^{ta}|\mathbf{x}^{co}].
\end{equation}
Consequently, the target data \(\mathbf{x}^{ta}\) for diffusion models in Eqs.~\eqref{eq:one-setp-forward}, \eqref{eq:inference}, and \eqref{eq:loss1} is replaced by \(\mathbf{r}_{ta}\).
The residual distribution of urban spatiotemporal data is not independently and identically distributed (i.i.d.) nor does it follow a fixed distribution, such as \(\mathcal{N}(0, \mathbf{\sigma})\). Instead, it often exhibits complex spatiotemporal dependence and heterogeneity. So we consider both temporal residual learning and spatial residual learning. 




\subsubsection*{\textbf{Temporal Residual Learning.}} 
For urban spatiotemporal data, the residual distribution varies at different time points. For example, fluctuations are lower at night and higher during the day, with uncertainty varying between weekends and weekdays. To model this, we incorporate the timestamp information as the condition for the denoising process. Meanwhile, the residual distribution can also be affected by sudden weather changes or public events. To capture these real-time changes, we concatenate the context data $\mathbf{x}^{co}_0$ with noised target residual $\mathbf{r}^{ta}_n$ as the input. The noise is not added to $\mathbf{x}^{co}_0$ and $\mathbf{r}^{ta}_n$ during the diffusion training and inference.




\subsubsection*{\textbf{Spatial Residual Learning.}}
In areas with frequent traffic accidents, fluctuations tend to be more pronounced and may induce anomalous variations in adjacent regions, thus affecting their distributions.
For spatial dependence modeling, the model learns a spatial embedding for each location, following the STID approach. Additionally, we propose a scale-aware diffusion process to further distinguish the heterogeneity for different regions. In this section, we detail the calculation of \(\mathbf{Q}\) and how it is integrated into the scale-aware diffusion process.

\noindent\textbf{(i) Customized Fluctuation Scale.} Specifically, we apply the Fast Fourier Transform (FFT) to spatiotemporal sequences in the training set to quantify fluctuation levels in different regions and use the custom scale \(\mathbf{Q}\) as input to account for spatial differences in residual. Specifically, we first employ FFT to extract the fluctuation components for each region within the training set. The detailed steps are as follows:









\begin{equation}
    \begin{aligned}
    & \mathbf{A}_{\mathrm{k}} = \left| \text{FFT}(\mathbf{x})_\mathrm{k} \right|, \quad \mathbf{{\phi}}_{\mathrm{k}} = \mathbf{\phi} \left( \text{FFT}(\mathbf{x})_\mathrm{k} \right), \\
    & \mathbf{A}_{\text{max}}=\max_{\mathrm{k}\in\left\{1,\cdots,\left\lfloor\frac{\mathbf{L}}{2}\right\rfloor + 1\right\}}\mathbf{A}_{\mathrm{k}}, \\
    & \mathcal{K} = \left\{ \mathrm{k} \in \left\{ 1, \cdots, \left\lfloor \frac{{L}}{2} \right\rfloor + 1 \right\} : \mathbf{A}_{\mathrm{k}} < 0.1 \times \mathbf{A}_{\text{max}} \right\}, \\
    & \mathbf{x}_{\mathbf{r}}[i] = \sum_{\mathrm{k} \in \mathcal{K}} \mathbf{A}_{\mathrm{k}} \Big[ \cos \left( 2\pi \mathbf{f}_{\mathrm{k}} i + \mathbf{\phi}_{\mathrm{k}} \right) \\
    & \qquad \qquad + \cos \left( 2\pi \bar{\mathbf{f}}_{\mathrm{k}} i + \bar{\mathbf{\phi}}_{\mathrm{k}} \right) \Big],
    \end{aligned}
\end{equation}
where \(\mathbf{A}_{\mathrm{k}},\mathbf{\phi}_{\mathrm{k}}\) reprent the amplitude and phase of the $\mathrm{k}-$th frequency component. $L$ is the temporal length of the training set. \(\mathbf{A}_{\text{max}}\) is the maximum amplitude among the components, obtained using the $\max$ operator. $\mathcal{K}$ represents the set of indices for the selected residual components. \(\mathbf{f}_{\mathrm{k}}\) is the frequency of the \(\mathrm{k}\)-th component. $\bar{\mathbf{f}}_{\mathrm{k}}, \bar{\mathbf{\phi}}_{\mathrm{k}}$ represent the conjugate components. \(\mathbf{x}_{\mathbf{r}}\) ref to the extracted residual component of the training set. We then compute the variance $\sigma^2_k$ of the residual sequence for each location $k$ and expand it to match the shape as 
\(\mathbf{r}^{ta}_0 \in \mathbb{R}^{B \times K \times P}\) , where $B$ represents the batch size. And we can get the variance tensor \(\mathcal{M}\): 
\begin{equation}
\begin{aligned}
    &\mathcal{M}_{b,k,p}=\sigma_{k}^2,\\
&\forall b\in\{1,\cdots,B\}, \forall k\in\{1,\cdots,K\}, \forall p\in\{1,\cdots,P\}.
\end{aligned}
\end{equation}
The residual fluctuations are bidirectional, encompassing both positive and negative variations, so we generate a random sign tensor \(\mathbf{S}\in\mathbb{R}^{B\times K\times P}\) for \(\mathcal{M}\), where each element \(S_{b,k,p}\) of \(\mathbf{S}\) is sampled from a Bernoulli distribution with \(p = 0.5\). 
%That is, \(r_{b,k,p}\) takes the value $1$ with probability $0.5$ and $-1$ with probability $0.5$. 
The customized fluctuation scale \(\mathbf{Q}\) is then defined as:
\begin{equation}
\begin{aligned}
&\mathbf{Q}_{b,k,p}=S_{b,k,p}\times\mathcal{M}_{b,k,p},\\
&\forall b\in\{1,\cdots,B\}, \forall k\in\{1,\cdots,K\}, \forall p\in\{1,\cdots,P\}.
\end{aligned}
\end{equation}
Then \(\mathbf{Q}\) is used as the input of the denoising network. 





\noindent\textbf{(ii) Scale-aware Diffusion Process.}

The vanilla diffusion models typically model all regions as the same \(\mathcal{N}(0, I)\) distribution at the end of the diffusion process, failing to distinguish the differences among regions. To further model the differences of residual distribution across various regions, we adopt the technique proposed by~\cite{han2022card} to make the residual learning region-specific conditioned on \({\mathbf{Q}}\). Specially, we have calculated the customized fluctuation scale \({\mathbf{Q}}\), and We redefined the noise distribution at the endpoint of the diffusion process as follows:
\begin{equation}
    p(\mathbf{r}^{ta}_N)=\mathcal{N}({\mathbf{Q}},I),
\end{equation}
Accordingly, the Eq~\ref{eq:new one-step} in the forward process is rewritten as:
\begin{equation}
\label{eq:new one-step}
    \mathbf{r}_n^{ta} = \sqrt{\bar{\alpha}_n} \mathbf{r}_0^{ta}+(1-\sqrt{\bar{\alpha}_n})\mathbf{Q} + \sqrt{1 - \bar{\alpha}_n} \mathbf{\epsilon}, \quad \mathbf{\epsilon} \sim \mathcal{N}(0, I).
\end{equation}
And in the denoising process, we sample \(\mathbf{r}_N^{ta}\) from $\mathcal{N}({\mathbf{Q}},I)$, and denoise it use Eq~(\ref{eq:inference}), the computation of \(\mu_{\theta}(\mathbf{r}_n^{ta}, n| \mathbf{x}_0^{co})\) in Eq~\ref{eq:inference} is modified as:
\begin{equation}
\label{eq: mu}
    \mu_{\theta}(\mathbf{r}_n^{ta}, n| \mathbf{x}_0^{co})=\frac{1}{\sqrt{\bar{\alpha}_n}} \left( \mathbf{r}_n^{ta} - \frac{\beta_n}{\sqrt{1 - \bar{\alpha}_n}} \mathbf{\epsilon}_{\theta}(\mathbf{r}_n^{ta}, n| \mathbf{x}_0^{co}) \right)+(1-\frac{1}{\sqrt{\bar{\alpha}_n}})\mathbf{Q}.
\end{equation}
This approach enables the diffusion process to be governed by the \(\mathbf{Q}\) values at each region, leading to more effective utilization of the customized scale conditions.


\subsection{Training and Inference}
\begin{algorithm}
\caption{\methodname{} Training}
\KwIn{Coarse-to-fine Autoencoder $\text{Enc}$, $\text{Dec}$}
\KwOut{}
\For{$i \gets 1$ \textbf{to} $n-1$}{
    \For{$j \gets 1$ \textbf{to} $n-i$}{
        \If{$L[j] > L[j+1]$}{
            Swap $L[j]$ and $L[j+1]$
        }
    }
}
\Return $L$
\end{algorithm}
\begin{algorithm}[!t]
\caption{Inference}
\label{al: sampling}
\begin{algorithmic}[1]
    \State \textbf{Input:} Context data $\mathbf{x}_0^{co}$, customized fluctuation scale $\mathbf{Q}$, trained diffusion model $\epsilon_{\theta}$, trained deterministic model $\mathbb{E}_{\theta}$
    \State \textbf{Output:} Target data $\mathbf{x}_0^{ta}$
    \State Estimate the conditional mean \(\mathbb{E}_{\theta}[\mathbf{x}^{ta}_0|\mathbf{x}^{co}_0]\)
    \State Sample $\mathbf{r}_N^{ta}$ from $\epsilon \sim \mathcal{N}(\mathbf{S},I)$
    \For{$n = N$ to $1$} 
        \State Estimate the noise $\mathbf{\epsilon}_{\theta}(\mathbf{r}_n^{ta}, n| \mathbf{x}_0^{co})$
        \State Calculate the $\mu_{\theta}(\mathbf{r}_n^{ta}, n| \mathbf{x}_0^{co})$ using Eq.~(\ref{eq: mu})
        \State Sample $\mathbf{r}_{n-1}^{ta}$ using Eq.~(\ref{eq:inference})
    \EndFor
    \State \textbf{Return:} $\mathbf{x}_0^{ta}=\mathbb{E}_{\theta}[\mathbf{x}^{ta}_0|\mathbf{x}^{co}_0]+\mathbf{r}_0^{ta}$
\end{algorithmic}

\end{algorithm}

\subsubsection*{\textbf{Training}}
Our training process is a two-stage procedure. We first pretrain the deterministic model STID to enable it to predict the conditional mean. Subsequently, we train the diffusion mode to learn the distribution of residuals, where the residuals are calculated as the difference between the true values and the conditional mean predicted by the pretrained STID model with frozen parameters. The detailed training procedure is presented in Algorithm~\ref{al: train}.
\subsubsection*{\textbf{Inference}}
The inference process of the model consists of two paths: one utilizes the pretrained STID model to predict the conditional mean, while the other uses the diffusion model to predict the residuals. The final sample is obtained by summing the results of both paths. This process is detailed in Algorithm~\ref{al: sampling}.
%\clearpage
\section{Experiments}
\subsection{Experiment Setup} 
\paragraph{Models.}

% \begin{table*}[htbp]
% \newcolumntype{g}{>{\columncolor{green!10}}c}
% \newcolumntype{b}{>{\columncolor{blue!10}}c}
% \renewcommand{\arraystretch}{1.22} % Adjust row spacing
% \small
% \resizebox{0.95\textwidth}{!}{
% \begin{tabular}{llcccc}

% \toprule
% UltraBench Split & Model & Overall Score & Soft Score & Hard Score & BLEU  \\ \midrule

% FineWeb  & Base Model  & 45.11 & 67.71 & 38.34 & 5.8 \\
% % \ \ Easy (66 samples)                      &         & 59.68 & 76.01 & 0.23 & 48.23 & 0.053     \\
% % \ \ Medium (76 samples)                    &         & 44.64 & 66.45 & 0.11 & 39.78 & 0.060     \\
% % \ \ Hard (58 samples)                      &         & 29.14 & 59.95 & 0.00 & 25.20 & 0.049     \\ \midrule

% & SFT Model  & 58.63 & 83.18  & 51.39 & 9.6 \\
% % \ \ Easy                      &         & 69.11 & 88.56 & 0.53 & 56.84 & 0.074    \\
% % \ \ Medium                    &         & 61.45 & 85.92 & 0.39 & 56.06 & 0.098     \\
% % \ \ Hard                       &         & 43.00 & 73.45 & 0.17 & 39.07 & 0.089     \\ \midrule

% & Llama-3.2-3B-DPO &  65.87 & 82.25 & 60.88 & 9.1  \\
% % \ \ Easy                       &         & 74.66 & 89.60 & 0.61 & 65.26 & 0.063    \\
% % \ \ Medium                   &         & 68.38 & 81.28 & 0.25 & 65.57 & 0.092  \\
% % \ \ Hard                       &         & 52.60 & 75.14 & 0.16 & 49.73 & 0.096   \\
% \midrule
% Global    & Llama-3.2-3B-Instruct\textsubscript{BASE}   & 36.85          & 43.95              & 35.23    & -         \\
% & Llama-3.2-3B-Instruct\textsubscript{SFT}    & 42.27          & 54.57               & 38.50     & -         \\
% & Llama-3.2-3B-Instruct\textsubscript{DPO}     & 63.84          & 57.84               & 64.86   & -           \\ 
% \bottomrule

% \end{tabular}
% }
% \caption{Performance scores for Llama-3.2-3B-Instruct models under different evaluation conditions.}
% \label{tab:ultrabench}
% \end{table*}

\begin{table*}[htbp]
\newcolumntype{g}{>{\columncolor{green!10}}c}
\newcolumntype{b}{>{\columncolor{blue!10}}c}
\renewcommand{\arraystretch}{1.22} % Adjust row spacing
\small
\resizebox{\textwidth}{!}
{
\begin{tabular}{llccccccc}

\toprule
&  \multirow{2}*{Model} & \multicolumn{4}{c}{\textbf{FineWeb Split}} & \multicolumn{3}{c}{\textbf{Multi-source Split}} \\
\cmidrule(l){3-6} \cmidrule(l){7-9} 
& & Overall Score & Soft Score & Hard Score & BERTScore F1    & Overall Score & Soft Score & Hard Score   \\ 
 
 \midrule

 \multirow{3}*{\rotatebox{90}{Main}}& Base Model  & 50.30 & 67.08 & 33.51 & 59.92  & 37.45       & 36.10            & 38.79            \\
% \hdashline

& UltraGen (AR)  & 56.05 & 81.44  & 30.65 & 62.00    & 50.15         & 62.41               & 37.89            \\
& UltraGen (AR+GPO) &  59.61 & 84.33 & 34.89 & 61.22    &  57.23       & 69.01               & 45.44            \\ 
\midrule
% \hdashline
\multirow{4}*{\rotatebox{90}{Ablation}} & AR (Few Constraints) & 48.25 & 74.09 & 22.41 & 60.10    & 38.38         &  46.00        & 30.76             \\
& GPO & 55.57 & 74.50 & 36.63 & 60.59 & 42.44 & 51.00 & 33.86 \\
& AR+GPO (Random Sampling) &  59.77 & 85.42 & 34.11 & 60.56 & 55.24         & 68.01            & 42.47            \\ 
& AR+GPO (High Similarity) &  59.44 & 83.22 & 35.65 & 60.85 & 55.45        & 66.05               & 44.85             \\ 
& AR+GPO (Low Correlation) &  58.91 & 83.59 & 34.23 & 60.00 & 54.47         & 65.22               & 43.71             \\ 

\bottomrule

\end{tabular}
}
\caption{Performance scores for Llama-3.2-3B-Instruct models on the validation set under different evaluation conditions across FineWeb and Global splits.}
\label{tab:ultrabench}
\end{table*}

% \begin{table}[h!]
\centering
\caption{Performance Comparison for Different Levels}
\resizebox{0.5\textwidth}{!}{
\begin{tabular}{@{}lccc@{}}
\toprule
\textbf{Category} & \textbf{Score} & \textbf{Soft Score} & \textbf{Hard Score} \\ \midrule
\multicolumn{4}{c}{\textbf{Overall Scores}} \\
Llama-3.2-3B-Instruct\textsubscript{BASE}       & 36.85          & 43.95              & 35.23            \\
Llama-3.2-3B-Instruct\textsubscript{SFT}         & 42.27          & 54.57               & 38.50              \\
Llama-3.2-3B-Instruct\textsubscript{DPO}         & 63.84          & 57.84               & 64.86             \\ \midrule
% \multicolumn{4}{c}{\textbf{Easy (47 Samples)}} \\
% Llama-3.2-3B-Instruct\textsubscript{BASE}         & 49.89          & 58.16               & 47.63              \\
% Llama-3.2-3B-Instruct\textsubscript{SFT}         & 51.65          & 66.38               & 42.82              \\
% Llama-3.2-3B-Instruct\textsubscript{DPO}         & 51.26          & 62.31               & 44.05              \\ \midrule
% \multicolumn{4}{c}{\textbf{Medium (55 Samples)}} \\
% Llama-3.2-3B-Instruct\textsubscript{BASE}         & 45.53         & 53.48               & 39.60              \\
% Llama-3.2-3B-Instruct\textsubscript{SFT}         & 43.92          & 53.53               & 32.14              \\
% Llama-3.2-3B-Instruct\textsubscript{DPO}         & 37.47          & 57.14               & 35.16              \\ \midrule
% \multicolumn{4}{c}{\textbf{Hard (98 Samples)}} \\
% Llama-3.2-3B-Instruct\textsubscript{BASE}         & 16.65          & 30.72               & 15.73              \\
% Llama-3.2-3B-Instruct\textsubscript{SFT}         & 25.30          & 42.14               & 24.40              \\
% Llama-3.2-3B-Instruct\textsubscript{DPO}         & 24.31          & 37.72               & 23.59              \\ \bottomrule
\end{tabular}
}
\label{table:global_performance_comparison}
\end{table}
Our experiments evaluate the EFCG task using one mainstream instruction-tuned base model: Llama-3.2-3B-Instruct ~\cite{dubey2024llama}, chosen for its demonstrated proficiency in instruction-following tasks within the 3B parameter range. To systematically assess the impact of our methodology, we compare three training paradigms: (1) \textbf{BASE}, which directly employs the unmodified base models to establish a performance baseline; (2) \textbf{AR}, where models undergo the auto-reconstruction stage on our meticulously constructed FineWeb dataset (§3.2), enriched with fine-grained attributes to enhance multi-constraint adherence; and (3) \textbf{AR+GPO}, a hybrid optimization approach combining direct preference optimization with global embedding space adaption.

\subsection{Evaluation Results on UltraBench}

Our experimental findings, summarized in Table \ref{tab:ultrabench}, demonstrate the substantial advancements achieved by applying the UltraGen paradigm to EFCG. The evaluation leverages the validation set of FineWeb and Global splits to assess model performance under both local and global constraints.

The application of AR yielded significant improvements over the base model. On the FineWeb split, the AR model attained an overall score of 56.05, representing a relative improvement of 11.4\%. The soft score rose to 81.44, indicating enhanced adherence to semantic and stylistic attributes, while the hard score increased to 30.65, reflecting better performance on programmatically verifiable constraints. On the Global split, the AR model demonstrated its ability to generalize, achieving an overall score of 50.15.

Further optimization through GPO demonstrated remarkable performance on the Global split, where the model achieved an overall score of 57.23 and an impressive hard score of 45.44. This highlights the model's robust generalization and optimization capabilities when dealing with diverse and challenging global constraints. Notably, despite being trained on the Global split, the AR+GPO model exhibited strong performance on the FineWeb split as well, achieving an overall score of 59.61, a soft score of 84.33, and a hard score of 34.89. This result underscores the model's ability to transfer its learned capabilities from the broader and more diverse Global split to the more localized FineWeb split.

\paragraph{Ablation}
To evaluate the contribution of key components in our UltraGen framework, we conducted ablation studies by systematically modifying the training process. We tested the impact of reducing the number of attributes during AR, removing the AR stage, replacing curated attributes with random sampling, and eliminating the high-correlation or low-redundancy selection steps. The results demonstrate that both AR and GPO stages are crucial for achieving strong performance, as reducing constraints, removing correlation modeling, or neglecting redundancy minimization leads to performance degradation.
% \paragraph{Ablation}
% To evaluate the contribution of key components in our UltraGen framework, we conducted several ablation studies by systematically modifying the training process. The following ablations were performed:
% \begin{enumerate}
%     \item \textbf{SFT with limited attributes}: To examine the impact of attribute numbers during the supervised fine-tuning stage, we trained an SFT model using a reduced set of attributes (fewer than 10 per sample).
%     \item \textbf{DPO only}: We directly train the DPO on the global split without SFT stage.
%     \item \textbf{SFT + DPO random sampling}: In this ablation, we replaced the curated high correlation and low redundancy attribute combinations with random sampling during the RL stage. 
%     \item \textbf{SFT + DPO w/o high correlation}: This experiment removed the attribute correlation modeling step, where attributes with strong relationships were prioritized.
%     \item \textbf{SFT + DPO w/o low redundancy}: In this setup, we did not enforce diversity in attribute sets by minimizing semantic redundancy.
% \end{enumerate}
% The ablation study shows that SFT with fewer constraints significantly underperforms the standard SFT. And DPO variants with fewer constraints, random sampling, or reduced correlation emphasize the importance of optimized attribute selection in the global space.
\subsection{Data Synthesis Improvement}

\begin{table}[htbp]
\centering
\small
\resizebox{0.48\textwidth}{!}{
\begin{tabular}{lccc}
\toprule
\textbf{Dataset (Domain)} & \textbf{Base} &  \textbf{AR} & \textbf{AR+GPO} \\ \midrule
Emotion (Tweet Emotion) & 28.25 & \textbf{42.30}  & 38.65 \\
Hillary (Tweet Stance)  & 55.93  & 45.76 & \textbf{58.31} \\
AG-News (News Topic) & 80.03 & 79.96 &\textbf{83.28} \\
TREC (Question Type) & 38.00  & 51.20  & \textbf{51.40} \\ 
\midrule
Average   & 50.55 & 54.81 & \textbf{57.91} \\
\bottomrule
\end{tabular}
}
\caption{Performance comparison for data synthesis.}
\vspace{-1em}
\label{tab:data_synthesis}
\end{table}

To demonstrate the improvement in the usage of texts synthesized by UltraGen, we utilize several diverse well-established text classification benchmarks to test the data synthesis capability, such as sentiment analysis \textbf{(1) Emotion} ~\cite{saravia-etal-2018-carer}, attitude classification towards a particular public figure \textbf{(2) Hillary} ~\cite{barbieri2020tweeteval}, topic classification \textbf{(3) AG News} ~\cite{Zhang2015CharacterlevelCN}, question type classification \textbf{(4) TREC} ~\cite{li-roth-2002-learning}.

For each dataset, we analyze the unique properties and paraphrase these properties as hard and soft attributes. Then using a uniform prompt tailored for each dataset, we generate 2,000 synthetic samples per dataset. These generated samples are then used to train a classifier, which is subsequently evaluated on the original test set of the dataset. This procedure allows for a fair comparison of model performance on synthetic data. 

The results, summarized in Table \ref{tab:data_synthesis}, demonstrate the superior generalization ability of the AR+GPO model trained on the Global split. Notably, the AR+GPO model achieved the highest average score of 57.91 across the benchmarks, significantly outperforming both the base model and the AR models. While the AR model’s performance stagnated (45.76, lower than the original one) on the Hillary benchmark, reflecting a focus on localized attributes, the AR+GPO model excelled with a score of 58.31, indicating its generalization and adaptability beyond localized training objectives.

\subsection{Trade-Offs in EFCG}
\begin{figure}[t]
    \centering
        \includegraphics[width=0.49\textwidth]{figs/tradeoff.pdf}
    \caption{The Trade-off between F1 score and CSR. While BERTScore tends to improve with more attributes, CSR declines}
    \vspace{-1.5em}
    \label{fig:tradeoff}
\end{figure}

Figure~\ref{fig:tradeoff} illustrates the interplay between BERTScore and CSR across different numbers of attributes from 10 to 50 for each model. As the figure shows, increasing the number of attributes presents a clear double-edged effect: while more attributes can enhance fine-grained control (e.g., higher F1 score) over the generated text, the added complexity makes it more difficult for the model to maintain high constraint adherence.

\paragraph{Better Multi-Objective Alignment Under EFCG.}
\begin{figure*}[htbp]
    \centering
        \includegraphics[width=0.98\textwidth]{figs/case_study.pdf}
    \caption{In a case study on travel itinerary generation, the attention flow illustrates improved constraint awareness in AR+GPO.}
    \vspace{-1em}
    \label{fig:case_study}
\end{figure*}

When looking at the 30, 40, and 50 attribute conditions:
AR+GPO consistently attains CSR values 5--10 points higher than the other two models without sacrificing F1.
For example, at 50 attributes, AR+GPO’s CSR (44.76\%) is considerably above AR’s (35.86\%) and Original’s (37.40\%), while also delivering the highest F1 (0.6348 vs. 0.6310 for AR and 0.6076 for Original).



This pattern illustrates a more favorable trade-off for AR+GPO: it does not simply chase high BERTScore by ignoring constraints, nor does it force all constraints at the expense of overall text quality. Instead, AR+GPO’s global optimization helps coordinate multiple constraints while retaining strong semantic alignment. In contrast, AR appears effective at moderate attribute counts but loses ground on CSR once the load goes beyond 30 attributes, and the Original model experiences an even steeper decline.

% \paragraph{Implications.} In extreme fine-grained control (EFCG) tasks, these findings confirm that:
% \begin{enumerate}
%     \item Light to Moderate Constraints (e.g., up to 20 attributes) can be addressed by simpler fine-tuning without major F1 loss.
%     \item High Constraint Settings (30+ attributes), especially when semantic overlap among attributes is low or conflicts are frequent, demand methods like DPO or other preference-optimization approaches to prevent a precipitous drop in CSR.
% \end{enumerate}
% Table~\ref{tab:data_synthesis} presents the performance comparison between the original baselines and the SFT model across three traditional text classification benchmarks: Emotion, AG-News, and TREC. The results highlight significant improvements achieved by the SFT model, particularly for Emotion and TREC datasets. On the Emotion dataset, the SFT model achieves a 42.3\% accuracy, representing a substantial improvement of 14.0 percentage points over the baseline. Similarly, on the TREC dataset, which focuses on question type classification, the SFT model attains a 55.0\% accuracy, outperforming the baseline by 17.0 percentage points.



% \subsection{Toxicity Control}
% \label{sec:toxic}
% In this section, we use a toxic classifier~\cite{Detoxify} to identify 171 harmful examples from FineWeb. Using the same attribute extraction methods as described in Section 3, we then generate texts based on these attributes.

% The results in Figure \ref{fig:toxicity} clearly demonstrate that the SFT model struggles to handle toxicity control effectively, particularly as the number of attributes increases. While the Original Model maintains a consistently low toxicity rate across all levels of attribute complexity, the SFT model shows a significant and steady rise in toxicity rate, reaching over 25\% when handling 60 attributes. 

% This trend suggests that the SFT model fails to generalize well under highly constrained conditions and becomes increasingly susceptible to generating toxic content as it attempts to satisfy a growing number of attributes. The inability to properly balance attribute satisfaction with toxicity control highlights a critical limitation of the SFT approach, emphasizing the need for more robust mechanisms to enforce safety constraints, especially in scenarios involving complex or numerous attributes.

% \begin{figure}[t] 
%     \centering
%         \includegraphics[width=0.5\textwidth]{figs/toxicity.pdf}
%     \caption{Comparison of toxicity. }
%     \label{fig:toxicity}
% \end{figure}

% \subsection{Attention Flow}


% In this section, we test whether our UltraGen adapt well in out of domain downstream tasks. We test two capabilities, the first one is the factual, the


\begin{figure*}[h]
    \centering
    \includegraphics[width=14cm]{figures/visualized_kitti5.jpg}
    \caption[Qualitative Results on KITTI \textit{val.} set]{\textbf{Qualitative results on the KITTI \textit{val} set for the car class.} The proposed method (green) and ground truth (red).
    } \label{fig:KITTI visualized}
\end{figure*}

\begin{figure*}[t]
    \centering
     \includegraphics[width=14cm]{figures/custom_result_monodetr_monoground3.jpg}
    \caption{\textbf{Qualitative results on the custom dataset.} Comparison of detection results between the proposed model (blue), the state-of-the-art models (green), and ground truth (red) in ego-view (left) and bird's-eye view (right); MonoDETR (left) and MonoGround (right).}
    \label{fig:custom_result_visualized}
\end{figure*}

\section{Conclusion}
\label{sec:conclusion}
This paper presents a novel approach to monocular 3D object detection by integrating a Vision Foundation Model as the backbone with the DETR architecture, enabling enhanced depth estimation and feature extraction within a single-stage, real-time framework. By incorporating a Hierarchical Feature Fusion Block for multi-scale detection and 6D Dynamic Anchor Boxes for iterative bounding box refinement, the proposed model achieves improved performance without relying on additional data sources, such as LiDAR. Future work will focus on extending the model's capabilities to detect 3D bounding boxes while accounting for rolling and pitching angles.

\balance
% \clearpage

\bibliographystyle{ACM-Reference-Format}
\balance
\bibliography{reference}
%\balance
% \clearpage

\onecolumn
\clearpage
\appendix
\appendixpage  % if you use a package that provides an appendix title page
\hypersetup{linkcolor=black}
\startcontents[sections]
\printcontents[sections]{l}{1}

\hypersetup{linkcolor=hrefblue}
\glsresetall

\section{Additional related works}\label{apx:related_works}

\paragraph{Knowledge distillation.}
Knowledge distillation (KD)~\citep{hinton2015distilling,gou2021knowledge} is closely connected to W2S generalization regarding the teacher-student setup, while W2S reverts the capacities of teacher and student in KD. In KD, a strong teacher model guides a weak student model to learn the teacher's knowledge. In contrast, W2S generalization occurs when a strong student model surpasses a weak teacher model under weak supervision.
\citet{phuong2019towards,stanton2021does,ojha2023knowledge,nagarajan2023student,dong2024cluster,ildiz2024high} conducted rigorous statistical analyses for the student's generalization from knowledge distillation. 
From the analysis perspective, a key difference between KD and W2S is that W2S is usually analyzed in the context of finetuning since the notions of “weak” and “strong” are built upon pretraining. This finetuning perspective introduces distinct angles from KD for examining intrinsic dimension~\citep{li2018measuring} and student-teacher correlation in W2S. 

\paragraph{Self-distillation and self-training.}
In contrast to W2S that considers distinct student and teacher models, self-distillation~\citep{zhang2019your,zhang2021self} and related paradigms such as Born-Again Networks~\citep{furlanello2018born} use the same or progressively refined architectures to iteratively distill knowledge from a ``previous version'' of the model. There have been extensive theoretical analyses toward understanding the mechanism behind self-distillation~\citep{mobahi2020self,das2023understanding,borup2023self,pareek2024understanding}.

Self-training~\citep{scudder1965probability,lee2013pseudo} is a closely related method to self-distillation that takes a single model's confident predictions to create pseudo-labels for unlabeled data and refines that model iteratively. 
\citet{wei2020theoretical,oymak2021theoretical,frei2022self} provide theoretical insights into the generalization of self-training. 
In particular, \citet{wei2020theoretical} introduced a theoretical framework based on neighborhood expansion, which was later on extended to various settings of weakly supervised learning, including domain adaptation~\citep{cai2021theory}, contrastive learning~\citep{shen2022connect}, consistency regularization~\citep{yang2023sample}, and now weak-to-strong generalization~\citep{lang2024theoretical,shin2024weak}.




\section{Proofs in \Cref{sec:single_task_ft}}

\begin{lemma}\label{lem:low_est_err_ft}    
    Given the FT approximation errors $\rho_s$ and $\rho_w$ in \Cref{def:ft_est_err}, we have
    \begin{align*}
        \rho_s(n) \le n \rho_s \quad \text{and} \quad \rho_w(n) \le n \rho_w \quad \forall\ n \in \N.
    \end{align*}
\end{lemma}

\begin{proof}[Proof of \Cref{lem:low_est_err_ft}]
    Let $\thetab_* = \argmin_{\thetab \in \R^d}\ \E_{\xb \sim \Dcal}[(\phi_w(\xb)^\top \thetab - f_*(\xb))^2]$ such that
    \begin{align*}
        \E_{\xb \sim \Dcal}[(\phi_w(\xb)^\top \thetab_* - f_*(\xb))^2] = \rho_w.
    \end{align*}
    Then, by observing that conditioned on $\Xb$,
    \begin{align*}
        \phi_w(\Xb)^\dagger f_*(\Xb) = \argmin_{\thetab \in \R^d}\ \| \phi_w(\Xb) \thetab - f_*(\Xb) \|_2^2,
    \end{align*} 
    we have
    \begin{align*}
        \rho_w(n) &= \E_{\Xb \sim \Dcal^n}\sbr{\| \phi_w(\Xb) \phi_w(\Xb)^\dagger f_*(\Xb) - f_*(\Xb) \|_2^2} \\
        &\le \E_{\Xb \sim \Dcal^n}\sbr{\| \phi_w(\Xb) \thetab_* - f_*(\Xb) \|_2^2} \\
        &= n\ \E_{\Xb \sim \Dcal^n}\sbr{\frac{1}{n} \| \phi_w(\Xb) \thetab_* - f_*(\Xb) \|_2^2} \\
        &= n\ \E_{\xb \sim \Dcal}\sbr{(\phi_w(\xb)^\top \thetab_* - f_*(\xb))^2} \\
        &= n\ \rho_w.
    \end{align*}
    The proof for $\rho_s(n)$ follows analogously.
\end{proof}



\subsection{Proof of \Cref{thm:w2s_ft}}\label{apx:pf_w2s_ft}

\begin{theorem}[Formal restatement of \Cref{thm:w2s_ft}]\label{thm:w2s_ft_formal}
    Consider $f_\wts(\xb) = \phi_s(\xb)^\top \thetab_\wts$ finetuned as in \eqref{eq:sft_weak}, \eqref{eq:w2s_ft} with both $\alpha_w, \alpha_\wts \to 0$. Under \Cref{asm:features,asm:ft_data}, when $n \ge \Omega(d_w)$, the excess risk $\exrisk(f_\wts) = \vari(f_\wts) + \bias(f_\wts)$ satisfies
    \begin{align*}
        &\bias(f_\wts) \le \frac{\rho_w(n)}{n} + \frac{\rho_s(N)}{N} \le \rho_w + \rho_s, \\
        &\vari(f_\wts) \lesssim \frac{\sigma^2}{n} \rbr{d_{s \wedge w} + \frac{d_s}{N} (d_w - d_{s \wedge w})}.
    \end{align*}
    In particular, when ${\rho_w(n)}/{n} > 0$ and $d_s < d_w$, the inequality for $\bias(f_\wts)$ is strict.

    Moreover, when $\phi_w(\xb) \sim \Ncal(\b0_d, \Sigmab_w)$, for any $n > d_w + 1$, we have 
    \begin{align*}
        &\vari(f_\wts) = \frac{\sigma^2}{n-d_w-1} \rbr{d_{s \wedge w} + \frac{d_s}{N} (d_w - d_{s \wedge w})}.
    \end{align*}
\end{theorem}

\begin{proof}[Proof of \Cref{thm:w2s_ft} and \Cref{thm:w2s_ft_formal}]
    We first observe that the solution of \eqref{eq:sft_weak} as $\alpha_w \to 0$ is given by
    \begin{align*}
        \thetab_w = \wt\Phib_w^\dagger \wt\yb = \wt\Phib_w^\dagger (\wt\fb_* + \wt\zb),
    \end{align*}
    where $\wt\zb \sim \Ncal(\b0_n, \sigma^2 \Ib_n)$.
    Meanwhile, the solution of \eqref{eq:w2s_ft} as $\alpha_\wts \to 0$ is given by
    \begin{align*}
        \thetab_\wts = \Phib_s^\dagger \Phib_w \thetab_w = \Phib_s^\dagger \Phib_w \wt\Phib_w^\dagger (\wt\fb_* + \wt\zb).
    \end{align*}  
    
    Then, the excess risk of $f_\wts$ can be decomposed into variance and bias as follows:
    \begin{align*}
        \exrisk(f_\wts) &= \E_{\xb \sim \Dcal}\sbr{\E_{f_\wts}\sbr{(f_\wts(\xb) - f_*(\xb))^2}} \\
        &= \E_{\Scal_x}\sbr{\E_{\wt\Scal}\sbr{\frac{1}{N}\nbr{\Phib_s \thetab_\wts - \fb_*}_2^2}} \\
        &=\E_{\Scal_x, \wt\Scal}\sbr{\frac{1}{N} \nbr{(\Phib_s \Phib_s^\dagger \Phib_w \wt\Phib_w^\dagger \wt\fb_* - \fb_*) + \Phib_s \Phib_s^\dagger \Phib_w \wt\Phib_w^\dagger \wt\zb}_2^2} \\
        &= \underbrace{\frac{1}{N} \E_{\Scal_x, \wt\Scal}\sbr{\nbr{\Phib_s \Phib_s^\dagger \Phib_w \wt\Phib_w^\dagger \wt\zb}_2^2}}_{\vari(f_\wts)} + \underbrace{\frac{1}{N} \E_{\Scal_x, \wt\Scal}\sbr{\nbr{\Phib_s \Phib_s^\dagger \Phib_w \wt\Phib_w^\dagger \wt\fb_* - \fb_*}_2^2}}_{\bias(f_\wts)}.
    \end{align*}

    \paragraph{Bias.}
    For the bias term, by observing that $\Pb_s = \Phib_s \Phib_s^\dagger$ is an $N \times N$ orthogonal projection, we can decompose the bias term as
    \begin{align*}
        \bias(f_\wts) &= \E_{\Scal_x, \wt\Scal}\sbr{\frac{1}{N} \nbr{\Pb_s \rbr{\Phib_w \wt\Phib_w^\dagger \wt\fb_* - \fb_*}}_2^2} + \frac{1}{N} \E_{\Scal_x}\sbr{\nbr{\rbr{\Ib_N - \Pb_s} \fb_*}_2^2},
    \end{align*}
    where $\E_{\Scal_x}\sbr{\nbr{\rbr{\Ib_N - \Pb_s} \fb_*}_2^2} = \rho_s(N)$ by \Cref{def:ft_est_err}.

    For the first term, 
    \begin{align*}
        \E_{\Scal_x, \wt\Scal}\sbr{\frac{1}{N} \nbr{\Pb_s \rbr{\Phib_w \wt\Phib_w^\dagger \wt\fb_* - \fb_*}}_2^2} &\le \E_{\Scal_x, \wt\Scal}\sbr{\frac{1}{N} \nbr{\Phib_w \wt\Phib_w^\dagger \wt\fb_* - \fb_*}_2^2} \\
        &= \E_{\wt\Scal}\sbr{\frac{1}{n} \nbr{\wt\Phib_w \wt\Phib_w^\dagger \wt\fb_* - \wt\fb_*}_2^2} \\
        &= \frac{\rho_w(n)}{n}.
    \end{align*}
    Notice that when ${\rho_w(n)}/{n} > 0$, this inequality is strict if $d_s < d_w$, where $\Phib_w \wt\Phib_w^\dagger \wt\fb_* - \wt\fb_* \notin \range(\Phib_s)$ almost surely.

    Overall, we have
    \begin{align*}
        \bias(f_\wts) \le \frac{\rho_w(n)}{n} + \frac{\rho_s(N)}{N} \le \rho_w + \rho_s,
    \end{align*}
    where the second inequality follows from \Cref{lem:low_est_err_ft}.

    \paragraph{Variance.}
    For the variance term, we observe that
    \begin{align*}
    \begin{split}
        \vari(f_\wts) &= \frac{1}{N} \E_{\Scal_x, \wt\Scal}\sbr{\nbr{\Pb_s \Phib_w \wt\Phib_w^\dagger \wt\zb}_2^2} \\
        &= \frac{1}{N} \E_{\Scal_x, \wt\Scal}\sbr{\tr\rbr{\Phib_w^\top \Pb_s \Phib_w \wt\Phib_w^\dagger \wt\zb \wt\zb^\top (\wt\Phib_w^\dagger)^\top}} \\
        &= \frac{\sigma^2}{N} \E_{\Scal_x, \wt\Scal}\sbr{\tr\rbr{\Phib_w^\top \Pb_s \Phib_w (\wt\Phib_w^\top \wt\Phib_w)^\dagger}},
    \end{split}
    \end{align*}
    which implies
    \begin{align}\label{eq:pf_var_w2s}
    \begin{split}
        \vari(f_\wts) = \frac{\sigma^2}{N} \tr\rbr{\E_{\Scal_x}\sbr{\Sigmab_w^{-1/2} \Phib_w^\top \Pb_s \Phib_w \Sigmab_w^{-1/2}} \E_{\wt\Scal}\sbr{\rbr{\Sigmab_w^{-1/2} \wt\Phib_w^\top \wt\Phib_w \Sigmab_w^{-1/2}}^\dagger}}.
    \end{split}
    \end{align}

    Recall the spectral decomposition $\Sigmab_w = \Vb_w \Lambdab_w \Vb_w^\top$. 
    Since $\E_{\xb \sim \Dcal}[\phi_w(\xb) \phi_w(\xb)^\top] = \Sigmab_w$, for each $\xb \sim \Dcal$, we can write $\phi_w(\xb) = \Sigmab_w^{1/2} \gammab$, where $\gammab \in \R^{d}$ is an independent random vector that is zero-mean and isotropic (\ie $\E[\gammab] = \b0_{d}$ and $\E[\gammab \gammab^\top] = \Ib_{d}$). The same holds for $\Sigmab_s = \Vb_s \Lambdab_s \Vb_s^\top$ and $\phi_s(\xb) = \Sigmab_s^{1/2} \gammab$.

    Then, for $\Scal$ and $\wt\Scal$, there exist independent random matrices $\Gammab = [\gammab_1, \ldots, \gammab_N]^\top \in \R^{N \times d}$ and $\wt\Gammab = [\wt\gammab_1, \ldots, \wt\gammab_n]^\top \in \R^{n \times d}$ consisting of $\iid$ zero-mean isotropic rows such that
    \begin{align}\label{eq:pf_var_w2s_subgaussian_asm}
    \begin{split}
        &\Phib_w \Sigmab_w^{-1/2} = \Gammab \Sigmab_w^{1/2} \Sigmab_w^{-1/2} = \Gammab \Vb_w \Vb_w^\top, \\
        &\wt\Phib_w \Sigmab_w^{-1/2} = \wt\Gammab \Sigmab_w^{1/2} \Sigmab_w^{-1/2} = \wt\Gammab \Vb_w \Vb_w^\top, \\
        &\Phib_s \Sigmab_s^{-1/2} = \Gammab \Sigmab_s^{1/2} \Sigmab_s^{-1/2} = \Gammab \Vb_s \Vb_s^\top, \\
        &\wt\Phib_s \Sigmab_s^{-1/2} = \wt\Gammab \Sigmab_s^{1/2} \Sigmab_s^{-1/2} = \wt\Gammab \Vb_s \Vb_s^\top.
    \end{split}
    \end{align}
    Let $\Gammab_w = \Gammab \Vb_w \in \R^{N \times d_w}$ and $\wt\Gammab_w = \wt\Gammab \Vb_w \in \R^{n \times d_w}$. We observe that
    \begin{align*}
        \E_{\wt\Scal}\sbr{\rbr{\Sigmab_w^{-1/2} \wt\Phib_w^\top \wt\Phib_w \Sigmab_w^{-1/2}}^\dagger}
        = \E_{\wt\Scal}\sbr{\rbr{\Vb_w \wt\Gammab_w^\top \wt\Gammab_w \Vb_w^\top}^\dagger} 
        = \Vb_w \E_{\wt\Scal}\sbr{\rbr{\wt\Gammab_w^\top \wt\Gammab_w}^\dagger} \Vb_w^\top.
    \end{align*}

    Now, we consider the following two cases for the feature distribution of $\phi_w(\xb)$, corresponding to the distribution of $\Gammab_w$ and $\wt\Gammab_w$:
    \begin{enumerate}[label=(\alph*)]
        \item \b{Gaussian features}: In \Cref{thm:w2s_ft}, assuming $\phi_w(\xb) \sim \Ncal(\b0_d, \Sigmab_w)$ such that $\wt\Gammab_w$ consists of $\iid$ Gaussian rows, we have $\wt\gammab_i \sim \Ncal(\b0_{d_w}, \Ib_{d_w})$. Notice that under the assumption $n > d_w + 1$, $\rank(\wt\Gammab_w) = d_w$ almost surely, and therefore $\wt\Gammab_w^\top \wt\Gammab_w$ is invertible.
        
        Meanwhile, with $\wt\gammab_i \sim \Ncal(\b0_{d_w}, \Ib_{d_w})$ for all $i \in [n]$, $(\wt\Gammab_w^\top \wt\Gammab_w) \sim \Wcal(\Ib_{d_w},n)$ follows the Wishart distribution~\citep[Definition 3.4.1]{wishart1928generalised} with $n$ degrees of freedom and scale matrix $\Ib_{d_w}$. 
        Therefore, $(\wt\Gammab_w^\top \wt\Gammab_w)^{-1} \sim \Wcal^{-1}(\Ib_{d_w},n)$ follows the inverse Wishart distribution~\citep[\S 3.8]{mardia2024multivariate}, whose mean takes the form~\citep[(3.8.3)]{mardia2024multivariate}
        \begin{align*}
            \E_{\wt\Scal}\sbr{(\wt\Gammab_w^\top \wt\Gammab_w)^\dagger} = \frac{1}{n - d_w -1} \Ib_{d_w}.
        \end{align*}
        Then, we have
        \begin{align*}
            \E_{\wt\Scal}\sbr{\rbr{\Sigmab_w^{-1/2} \wt\Phib_w^\top \wt\Phib_w \Sigmab_w^{-1/2}}^\dagger}
            = \frac{1}{n - d_w -1} \Vb_w \Vb_w^\top.
        \end{align*}
        Therefore, \eqref{eq:pf_var_w2s} implies
        \begin{align}\label{eq:pf_var_w2s_1}
        \begin{split}
            \vari(f_\wts) &= \frac{\sigma^2}{N}\ \frac{1}{n - d_w -1}\ \tr\rbr{\Vb_w^\top \E_{\Scal_x}\sbr{\Sigmab_w^{-1/2} \Phib_w^\top \Pb_s \Phib_w \Sigmab_w^{-1/2}} \Vb_w} \\
            &= \frac{\sigma^2}{N}\ \frac{1}{n - d_w -1}\ \tr\rbr{\E_{\Scal_x}\sbr{\Vb_w^\top \Vb_w \Gammab_w^\top \Pb_s \Gammab_w \Vb_w^\top \Vb_w}} \\
            &= \frac{\sigma^2}{N}\ \frac{1}{n - d_w -1}\ \tr\rbr{\E_{\Scal_x}\sbr{\Gammab_w^\top \Pb_s \Gammab_w}}.
        \end{split}
        \end{align}
        Recall that $\Pb_s = \Phib_s \Phib_s^\dagger$. Let $\Gammab_s = \Gammab \Vb_s \in \R^{N \times d_s}$, and we can write
        \begin{align*}
            \Pb_s = (\Phib_s \Sigmab_s^{-1/2}) (\Phib_s \Sigmab_s^{-1/2})^\dagger = (\Gammab_s \Vb_s^\top) (\Gammab_s \Vb_s^\top)^\dagger = \Gammab_s \Gammab_s^\dagger.
        \end{align*}
        Therefore, with $\Gammab_w = \Gammab \Vb_w$ and $\Gammab_s = \Gammab \Vb_s$, we can decompose
        \begin{align*}
            \tr\rbr{\E_{\Scal_x}\sbr{\Gammab_w^\top \Pb_s \Gammab_w}} 
            &= \E_{\Scal_x}\sbr{\tr\rbr{\Gammab_w^\top \Gammab_s \Gammab_s^\dagger \Gammab_w}} \\
            &= \E_{\Scal_x}\sbr{\tr\rbr{\Vb_w^\top \Vb_s \Vb_s^\top \Vb_w \Gammab_w^\top \Gammab_s \Gammab_s^\dagger \Gammab_w}} \\
            &+ \E_{\Scal_x}\sbr{\tr\rbr{\Vb_w^\top (\Ib_d - \Vb_s \Vb_s^\top) \Vb_w \Gammab_w^\top \Gammab_s \Gammab_s^\dagger \Gammab_w}}.
        \end{align*}
        For the first term, since $\Gammab_w \Vb_w^\top \Vb_s = \Gammab \Vb_w \Vb_w^\top \Vb_s$ and $\Gammab_s = \Gammab \Vb_s$, the range of $\Gammab_w \Vb_w^\top \Vb_s$ is a subspace of that of $\Gammab_s$ and therefore,
        \begin{align*}
            \E_{\Scal_x}\sbr{\tr\rbr{\Vb_w^\top \Vb_s \Vb_s^\top \Vb_w \Gammab_w^\top \Gammab_s \Gammab_s^\dagger \Gammab_w}} 
            &= \E_{\Scal_x}\sbr{\tr\rbr{ \Vb_s^\top \Vb_w \Gammab_w^\top \Gammab_s \Gammab_s^\dagger \Gammab_w \Vb_w^\top \Vb_s}} \\
            &= \E_{\Scal_x}\sbr{\tr\rbr{ \Vb_s^\top \Vb_w \Gammab_w^\top \Gammab_w \Vb_w^\top \Vb_s}} \\
            &= \tr\rbr{\Vb_s^\top \Vb_w \E_{\Scal_x}\sbr{\Gammab_w^\top \Gammab_w} \Vb_w^\top \Vb_s}.
        \end{align*}
        Since $\E_{\Scal_x}\sbr{\Gammab_w^\top \Gammab_w} = N \Ib_{d_w}$, we have
        \begin{align*}
            \E_{\Scal_x}\sbr{\tr\rbr{\Vb_w^\top \Vb_s \Vb_s^\top \Vb_w \Gammab_w^\top \Gammab_s \Gammab_s^\dagger \Gammab_w}} 
            &= N \tr\rbr{\Vb_s^\top \Vb_w \Vb_w^\top \Vb_s} \\
            &= N \nbr{\Vb_s^\top \Vb_w}_F^2 \\
            &= N d_{s \wedge w}.
        \end{align*}
        For the second term, we first observe that the row space of $\Gammab_w \Vb_w^\top (\Ib_d - \Vb_s \Vb_s^\top)$ is orthogonal to that of $\Gammab_s = \Gammab \Vb_s$, and therefore, $\Gammab_w \Vb_w^\top (\Ib_d - \Vb_s \Vb_s^\top)$ and $\Gammab_s$ are independent, which implies
        \begin{align*}
            \E_{\Scal_x}\sbr{\tr\rbr{\Vb_w^\top (\Ib_d - \Vb_s \Vb_s^\top) \Vb_w \Gammab_w^\top \Gammab_s \Gammab_s^\dagger \Gammab_w}} 
            &= \tr\rbr{\E\sbr{\Gammab_w \Vb_w^\top (\Ib_d - \Vb_s \Vb_s^\top) \Vb_w \Gammab_w^\top} \E\sbr{\Gammab_s \Gammab_s^\dagger}}.
        \end{align*}
        Since $\Gammab$ consists of independent isotropic rows, so do $\Gammab_s = \Gammab \Vb_s \in \R^{N \times d_s}$ and $\Gammab_w = \Gammab \Vb_w \in \R^{N \times d_w}$, which implies
        \begin{align*}
            \E\sbr{\Gammab_s \Gammab_s^\dagger} = \frac{d_s}{N}\ \Ib_N \quad \t{and} \quad \E\sbr{\Gammab_w^\top \Gammab_w} = N\ \Ib_{d_w}.
        \end{align*}
        Then, we have
        \begin{align*}
            \E_{\Scal_x}\sbr{\tr\rbr{\Vb_w^\top (\Ib_d - \Vb_s \Vb_s^\top) \Vb_w \Gammab_w^\top \Gammab_s \Gammab_s^\dagger \Gammab_w}} 
            &= \tr\rbr{\E\sbr{\Gammab_w \Vb_w^\top (\Ib_d - \Vb_s \Vb_s^\top) \Vb_w \Gammab_w^\top} \E\sbr{\Gammab_s \Gammab_s^\dagger}} \\
            &= \frac{d_s}{N} \tr\rbr{\E\sbr{\Gammab_w \Vb_w^\top (\Ib_d - \Vb_s \Vb_s^\top) \Vb_w \Gammab_w^\top}} \\
            &= \frac{d_s}{N} \tr\rbr{\Vb_w^\top (\Ib_d - \Vb_s \Vb_s^\top) \Vb_w \E\sbr{\Gammab_w^\top \Gammab_w}} \\
            &= \frac{d_s}{N} N \tr\rbr{\Vb_w^\top (\Ib_d - \Vb_s \Vb_s^\top) \Vb_w} \\
            &= d_s (d_w - d_{s \wedge w}).
        \end{align*}
        Combining the two terms, we have
        \begin{align*}
            \tr\rbr{\E_{\Scal_x}\sbr{\Gammab_w^\top \Pb_s \Gammab_w}} = N d_{s \wedge w} + d_s (d_w - d_{s \wedge w}).
        \end{align*}
        Then, by \eqref{eq:pf_var_w2s_1}, the variance is exactly characterized by
        \begin{align*}
            \vari(f_\wts) 
            &= \frac{\sigma^2}{N}\ \frac{N d_{s \wedge w} + d_s (d_w - d_{s \wedge w})}{n - d_w -1} \\
            &= \frac{\sigma^2}{n-d_w-1} \rbr{d_{s \wedge w} + \frac{d_s}{N} (d_w - d_{s \wedge w})}.
        \end{align*}

        \item \b{Sub-gaussian features}: Relaxing the Gaussian feature assumption, when $\wt\Gammab_w$ consists of $\iid$ sub-gaussian random vectors that are zero-mean and isotropic (\ie $\E[\wt\gammab_i] = \b0_{d_w}$ and $\E[\wt\gammab_i \wt\gammab_i^\top] = \Ib_{d_w}$), with $n \ge \Omega(d_w)$, \Cref{lem:trace_inv_subgaussian} implies that
        \begin{align*}
            \E_{\wt\Scal}\sbr{(\wt\Gammab_w^\top \wt\Gammab_w)^\dagger} \aleq O\rbr{\frac{1}{n}} \Ib_{d_w},
        \end{align*}
        and therefore,
        \begin{align*}
            \E_{\wt\Scal}\sbr{\rbr{\Sigmab_w^{-1/2} \wt\Phib_w^\top \wt\Phib_w \Sigmab_w^{-1/2}}^\dagger} \aleq O\rbr{\frac{1}{n}} \Vb_w \Vb_w^\top.
        \end{align*}
        Then, via an analogous argument as \eqref{eq:pf_var_w2s_1}, \eqref{eq:pf_var_w2s} implies that 
        \begin{align}\label{eq:pf_var_w2s_2}
        \begin{split}
            \vari(f_\wts) \le \frac{\sigma^2}{N}\ O\rbr{\frac{1}{n}}\ \tr\rbr{\E_{\Scal_x}\sbr{\Gammab_w^\top \Pb_s \Gammab_w}}.
        \end{split}
        \end{align}
        We observe that in the analysis of the Gaussian feature case, the characterization
        \begin{align*}
            \tr\rbr{\E_{\Scal_x}\sbr{\Gammab_w^\top \Pb_s \Gammab_w}} = (N - d_s) d_{s \wedge w} + d_s d_w
        \end{align*}
        does not involve the Gaussianity of $\Gammab$ and therefore holds for general subgaussian features.
        This leads to an upper bound on the variance:
        \begin{align*}
            \vari(f_\wts) 
            &\le \frac{\sigma^2}{N}\ O\rbr{\frac{1}{n}}\ \rbr{N d_{s \wedge w} + d_s (d_w - d_{s \wedge w})} \\
            &\lesssim \frac{\sigma^2}{n} \rbr{d_{s \wedge w} + \frac{d_s}{N} (d_w - d_{s \wedge w})}.
        \end{align*}
    \end{enumerate}
\end{proof}


\begin{lemma}[Adapting \cite{vershynin2010introduction} Theorem 5.39]\label{lem:trace_inv_subgaussian}
    Let $\wt\Gammab_w = [\wt\gammab_1, \ldots, \wt\gammab_n]^\top$ be an $n \times d_w$ matrix whose rows $\wt\gammab_1, \ldots, \wt\gammab_n$ consist of $\iid$ sub-gaussian random vectors that are zero-mean and isotropic (\ie $\E[\wt\gammab_i] = \b0_{d_w}$ and $\E[\wt\gammab_i \wt\gammab_i^\top] = \Ib_{d_w}$). When $n \ge \Omega(d_w)$, we have
    \begin{align*}
        \E\sbr{\nbr{\rbr{\wt\Gammab_w^\top \wt\Gammab_w}^\dagger}_2} \le O\rbr{\frac{1}{n}},
    \end{align*}
    where $\Omega(\cdot)$ and $O(\cdot)$ suppresses constants that depend only on the sub-gaussian norm $\nbr{\wt\gammab_i}_{\psi_2} = \sup_{\vb \in \SSS^{d_w-1}} \sup_{p \ge 1} (\E[|\wt\gammab_i^\top \vb|^p])^{1/p} / \sqrt{p}$, independent of $n, d_w$.
\end{lemma}

\begin{proof}[Proof of \Cref{lem:trace_inv_subgaussian}]
    Let $\sigma_{\min}(\wt\Gammab_w^\top \wt\Gammab_w)$ be the smallest singular value of $\wt\Gammab_w^\top \wt\Gammab_w$.
    Leveraging \cite{vershynin2010introduction} Theorem 5.39, we notice that for $n \ge \Omega(d_w)$, there exist constants $c_1, c_2 > 0$ that depend only on the sub-gaussian norm $\nbr{\wt\gammab_i}_{\psi_2}$ such that
    \begin{align*}
        \Pr\sbr{\sigma_{\min}(\wt\Gammab_w^\top \wt\Gammab_w) < \rbr{\sqrt{n} - c_1\sqrt{d_w} - t}^2} \le \exp\rbr{-c_2 t^2}.
    \end{align*}
    Therefore, we have 
    \begin{align*}
        \Pr\sbr{\frac{1}{\sigma_{\min}(\wt\Gammab_w^\top \wt\Gammab_w)} > t} \le \exp\rbr{-c_2 \rbr{\sqrt{n} - c_1 \sqrt{d_w} - \sqrt{\frac{1}{t}}}^2}.
    \end{align*}

    Notice that for any non-negative random variable $Z$ with a cumulative density function $F_Z(z)$, 
    \begin{align*}
        \E\sbr{Z} &= \int_0^\infty z d F_Z(z) 
        = - \int_0^\infty z d \rbr{1 - F_Z(z)} \\
        &= \sbr{z \rbr{1 - F_Z(z)}}_0^\infty + \int_0^\infty \rbr{1 - F_Z(z)} dz \\
        &= \int_0^\infty \Pr\sbr{Z > z} dz.
    \end{align*}
    Therefore, we have
    \begin{align*}
        \E\sbr{\frac{1}{\sigma_{\min}(\wt\Gammab_w^\top \wt\Gammab_w)}} \le \int_0^\infty \exp\rbr{-c_2 \rbr{\sqrt{n} - c_1 \sqrt{d_w} - \sqrt{\frac{1}{t}}}^2} d t.
    \end{align*}
    Let $t_0 = 1 / \rbr{\sqrt{n} - c_1 \sqrt{d_w}}^2$ such that $\sqrt{n} - c_1 \sqrt{d_w} - \sqrt{\frac{1}{t}}=0$ and 
    \begin{align*}
        \int_{0}^{t_0} \exp\rbr{-c_2 \rbr{\sqrt{n} - c_1 \sqrt{d_w} - \sqrt{\frac{1}{t}}}^2} d t \le t_0
    \end{align*}
    Then, we have
    \begin{align*}
        &\E\sbr{\frac{1}{\sigma_{\min}(\wt\Gammab_w^\top \wt\Gammab_w)}} 
        \le \int_0^\infty \exp\rbr{-c_2 \rbr{\sqrt{n} - c_1 \sqrt{d_w} - \sqrt{\frac{1}{t}}}^2} d t \\
        &\le t_0 + \int_{t_0}^\infty \exp\rbr{-c_2 \rbr{\sqrt{n} - c_1 \sqrt{d_w} - \sqrt{\frac{1}{t}}}^2} d t \\
        &= t_0 + 2 \int_{0}^{\sqrt{n}-c_1\sqrt{d_w}} \exp\rbr{-c_2 u^2} \rbr{\sqrt{n} - c_1 \sqrt{d_w} - u}^{-3} d u \\
        &= t_0 + \frac{2}{\rbr{\sqrt{n} - c_1 \sqrt{d_w}}^2} \int_{0}^{1} \exp\rbr{-c_2 \rbr{\sqrt{n}-c_1\sqrt{d_w}}^2 u^2} \rbr{1 - u}^{-3} d u \\
        &= \frac{1}{\rbr{\sqrt{n} - c_1 \sqrt{d_w}}^2} + \frac{2}{\rbr{\sqrt{n} - c_1 \sqrt{d_w}}^2} \rbr{\int_{0}^{1} \exp\rbr{-\Omega\rbr{u^2}} \rbr{1 - u}^{-3} d u} \\
        &= O\rbr{\frac{1}{\rbr{\sqrt{n} - c_1 \sqrt{d_w}}^2}}.
    \end{align*}
    When $n \ge \Omega(d_w)$, we have $\sqrt{n} - c_1 \sqrt{d_w} \ge \Omega(\sqrt{n})$, and therefore ,
    \begin{align*}
        \E\sbr{\nbr{\rbr{\wt\Gammab_w^\top \wt\Gammab_w}^\dagger}_2}
        \le \E\sbr{\frac{1}{\sigma_{\min}(\wt\Gammab_w^\top \wt\Gammab_w)}} 
        \le O\rbr{\frac{1}{n}}.
    \end{align*}
\end{proof}





\subsection{Proof of \Cref{pro:sft_weak} and \Cref{cor:sft_strong}}\label{apx:pf_sft_weak}
\begin{proof}[Proof of \Cref{pro:sft_weak} and \Cref{cor:sft_strong}]
    The excess risk of the finetuned weak teacher $f_w(\xb) = \phi_w(\xb)^\top \thetab_w$ can be expressed as
    \begin{align*}
        \exrisk(f_w) &= \E_{\xb \sim \Dcal}\sbr{\E_{f_w}\sbr{(f_w(\xb) - f_*(\xb))^2}} \\
        &= \E_{\wt\Scal}\sbr{\frac{1}{n}\nbr{\wt\Phib_w \thetab_w - \wt\fb_*}_2^2},
    \end{align*}
    where $\wt\fb_* = [\fb_*(\wt\xb_1), \ldots, \fb_*(\wt\xb_n)]^\top \in \R^n$; and we recall that $\wt\Phib_w = [\phi_w(\wt\xb_1), \ldots, \phi_w(\wt\xb_n)]^\top$. Notice that the randomness of $\thetab_w$ comes from the SFT samples $\wt\Scal \sim \Dcal(f_*)^n$.

    Observe that the solution of \eqref{eq:sft_weak} as $\alpha_w \to 0$ is given by $\thetab_w = \wt\Phib_w^\dagger \wt\yb$, where $\wt\yb = \wt\fb_* + \wt\zb$ is the noisy label vector with $\wt\zb \sim \Ncal(\b0_n, \sigma^2 \Ib_n)$.
    Therefore, with the randomness over $\wt\Scal \sim \Dcal(f_*)^n$, we have
    \begin{align*}
        \exrisk(f_w) &= \E \sbr{\frac{1}{n}\nbr{\wt\Phib_w \wt\Phib_w^\dagger \wt\yb - \wt\fb_*}_2^2} \\
        &= \E \sbr{\frac{1}{n}\nbr{\wt\Phib_w \wt\Phib_w^\dagger \wt\zb + \rbr{\wt\Phib_w \wt\Phib_w^\dagger \wt\fb_* - \wt\fb_*}}_2^2} \\
        &= \underbrace{\E \sbr{\frac{1}{n}\nbr{\wt\Phib_w \wt\Phib_w^\dagger \wt\zb}_2^2}}_{\vari(f_w)} + \underbrace{\E\sbr{\frac{1}{n}\nbr{\wt\Phib_w \wt\Phib_w^\dagger \wt\fb_* - \wt\fb_*}_2^2}}_{\bias(f_w)}.
    \end{align*}
    
    For bias, by the definition of finetuning capacity (see \Cref{def:ft_est_err}), we have
    \begin{align*}
        \bias(f_w) = \frac{1}{n} \E\sbr{\nbr{\wt\Phib_w \wt\Phib_w^\dagger \wt\fb_* - \wt\fb_*}_2^2} = \frac{\rho_w(n)}{n}.
    \end{align*}
    We observe that $\bias(f_w) \le \rho_w$ by \Cref{lem:low_est_err_ft}.
    Notice that \Cref{lem:low_est_err_ft} also implies $\bias(f_s) = {\rho_s(n)}/{n} \le \rho_s$. 

    For variance, we observe that 
    \begin{align*}
        \vari(f_w) &= \frac{1}{n} \E\sbr{\nbr{\wt\Phib_w \wt\Phib_w^\dagger \wt\zb}_2^2} \\
        &= \frac{1}{n} \E\sbr{\tr\rbr{\wt\Phib_w \wt\Phib_w^\dagger \wt\zb \wt\zb^\top}} \\
        &= \frac{\sigma^2}{n} \E\sbr{\tr\rbr{\wt\Phib_w \wt\Phib_w^\dagger}}.
    \end{align*}
    By \Cref{asm:ft_data}, since $\rank(\wt\Phib_w) = d_w$ almost surely, we have
    \begin{align*}
        \vari(f_w) = \frac{\sigma^2}{n} \E\sbr{\tr\rbr{\wt\Phib_w \wt\Phib_w^\dagger}} = \frac{\sigma^2 d_w}{n}.
    \end{align*}
\end{proof}



\subsection{Proof of \Cref{cor:pgr}}\label{apx:pf_pgr}
\begin{proof}[Proof of \Cref{cor:pgr}]
    Noticing that with $\rank(\wt\Phib_w) = d_w$ and $\rank(\wt\Phib_s) = \rank(\Phib_s) = d_s$ almost surely, the excess risks of $f_w, f_s, f_c$ are characterized exactly in \Cref{pro:sft_weak} and \Cref{cor:sft_strong}, and $\exrisk(f_\wts)$ is upper bounded by \Cref{thm:w2s_ft}.
    Therefore, by directly plugging in the excess risks to the definitions of PGR and OPR, we have
    \begin{align}\label{eq:pgr_lower_tight}
    \begin{split}
        \pgr = &\frac{\exrisk(f_w) - \exrisk(f_\wts)}{\exrisk(f_w) - \exrisk(f_c)} \\
        \ge &\rbr{\sigma^2\ \frac{d_w}{n} + \frac{\rho_w(n)}{n} - \frac{\sigma^2}{n-d_w-1} \rbr{d_{s \wedge w} + \frac{d_s}{N} (d_w-d_{s \wedge w})} - \rbr{\frac{\rho_w(n)}{n} + \frac{\rho_s(N)}{N}}} \\
        &\rbr{\sigma^2\ \frac{d_w}{n} + \frac{\rho_w(n)}{n} - \sigma^2\ \frac{d_s}{N+n} - \frac{\rho_s(N+n)}{N+n}}^{-1} \\
        \ge &\rbr{\sigma^2 \frac{d_w}{n} - \sigma^2 \frac{d_{s \wedge w} + (d_w - d_{s \wedge w}) {d_s}/{N}}{n-d_w-1} - \frac{\rho_s(N)}{N}} \Big/ \rbr{\sigma^2 \frac{d_w}{n} + \frac{\rho_w(n)}{n}}, \\
        \ge &\rbr{\sigma^2 \frac{d_w}{n} - \sigma^2 \frac{d_{s \wedge w} + (d_w - d_{s \wedge w}) {d_s}/{N}}{n-d_w-1} - \rho_s} \Big/ \rbr{\sigma^2 \frac{d_w}{n} + \rho_w},
    \end{split}
    \end{align}
    and 
    \begin{align}\label{eq:opr_lower_tight}
    \begin{split}
        \opr = &\frac{\exrisk(f_s)}{\exrisk(f_\wts)} \\
        \ge &\rbr{\sigma^2\ \frac{d_s}{n} + \frac{\rho_s(n)}{n}} \Big/ \rbr{\sigma^2 \frac{d_{s \wedge w} + (d_w - d_{s \wedge w}) {d_s}/{N}}{n-d_w-1} + \rbr{\frac{\rho_w(n)}{n} + \frac{\rho_s(N)}{N}}} \\
        \ge &\sigma^2 \frac{d_s}{n} \Big/ \rbr{\sigma^2 \frac{d_{s \wedge w} + (d_w - d_{s \wedge w}) {d_s}/{N}}{n-d_w-1} + \rho_w + \rho_s}.
    \end{split}
    \end{align} 

    When taking $n = d_w + q + 1$ for some small constant $q \in \N$, we observe that 
    \begin{align*}
        \pgr &\ge \rbr{\sigma^2 \frac{d_w}{n} - \sigma^2 \frac{d_{s \wedge w} + (d_w - d_{s \wedge w}) {d_s}/{N}}{n-d_w-1} - \rho_s} \Big/ \rbr{\sigma^2 \frac{d_w}{n} + \rho_w} \\
        &\ge \rbr{\frac{d_w}{d_w + q + 1} - \frac{d_{s \wedge w}}{q} - \frac{d_s}{N} \frac{d_w - d_{s \wedge w}}{q} - \frac{\rho_s}{\sigma^2}} \Big/ \rbr{\frac{d_w}{d_w + q + 1} + \frac{\rho_w}{\sigma^2}} \\
        &\ge \rbr{\frac{d_w}{d_w + q + 1} - \frac{d_{s \wedge w}}{q} - \frac{d_s}{N} \frac{d_w - d_{s \wedge w}}{q} - \frac{\rho_s}{\sigma^2} - \frac{\rho_w}{\sigma^2}} \Big/ \rbr{\frac{d_w}{d_w + q + 1} + \frac{\rho_w}{\sigma^2} - \frac{\rho_w}{\sigma^2}} \\
        &= 1 - \frac{n}{d_w} \rbr{\frac{d_{s \wedge w}}{q} + \frac{d_s}{N} \frac{d_w - d_{s \wedge w}}{q} + \frac{\rho_w + \rho_s}{\sigma^2}} \\
        &= 1 - \frac{n}{q}\ {\frac{d_{s \wedge w} + (d_w - d_{s \wedge w}) d_s / N}{d_w}} - \frac{n}{d_w}\ {\frac{\rho_w + \rho_s}{\sigma^2}},
    \end{align*}
    and
    \begin{align*}
        \opr &\ge \sigma^2 \frac{d_s}{n} \Big/ \rbr{\sigma^2 \frac{d_{s \wedge w} + (d_w - d_{s \wedge w}) {d_s}/{N}}{n-d_w-1} + \rho_w + \rho_s} \\
        &= \frac{d_s}{n} \Big/ \rbr{\frac{d_{s \wedge w} + (d_w - d_{s \wedge w}) {d_s}/{N}}{q} + \frac{\rho_w + \rho_s}{\sigma^2}} \\
        &= \rbr{\frac{n}{q}\ \frac{d_{s \wedge w} + (d_w - d_{s \wedge w}) {d_s}/{N}}{d_s} + \frac{n}{d_s}\ \frac{\rho_w + \rho_s}{\sigma^2}}^{-1}.
    \end{align*}
\end{proof}



\subsection{Proof of \Cref{cor:non_monotonic_scaling}}\label{apx:pf_non_monotonic_scaling}
\begin{proof}[Proof of \Cref{cor:non_monotonic_scaling}]
    Recall the notations introduced for conciseness:
    \begin{align*}
        d_\wts(N) = d_{s \wedge w} + (d_w - d_{s \wedge w}) \frac{d_s}{N}, \quad \varrho = \frac{\rho_w + \rho_s}{\sigma^2}.
    \end{align*}
    Then, the lower bounds for $\pgr$ and $\opr$ in \Cref{cor:pgr} can be expressed in terms of $d_\wts(N)$ and $\varrho$ as 
    \begin{align*}
        \pgr \ge 1 - \frac{d_\wts(N)}{d_w} - \frac{d_w + 1}{d_w} \varrho - \frac{d_w + 1}{d_w}\ \frac{d_\wts(N)}{q} - q \frac{\varrho}{d_w},
    \end{align*}
    and 
    \begin{align*}
        \opr \ge \rbr{\frac{d_\wts(N)}{d_s} + \frac{d_w + 1}{d_s} \varrho + \frac{d_\wts(N)}{d_s}\ \frac{d_w + 1}{q} + q \frac{\varrho}{d_s}}^{-1}.
    \end{align*}
    Both lower bounds are maximized when the last two terms in the expressions that involve $q$ are minimized, which is achieved when $q = \sqrt{\rbr{d_w + 1} {d_\wts(N)}/{\varrho}}$. Substituting the optimal $q$ back into the expressions yields the best lower bounds for $\pgr$ and $\opr$:
    \begin{align*}
        \pgr \ge &1 - \frac{d_\wts(N)}{d_w} - \varrho \frac{d_w + 1}{d_w} - 2 \sqrt{\varrho \frac{d_w + 1}{d_w}\ \frac{d_\wts(N)}{d_w}} \\
        = &1 - \rbr{\sqrt{\frac{d_\wts(N)}{d_w}} + \sqrt{\varrho\ \frac{d_w + 1}{d_w}}}^2,
    \end{align*}
    and 
    \begin{align*}
        \opr \ge &\rbr{\frac{d_\wts(N)}{d_s} + \varrho \frac{d_w + 1}{d_s} + 2 \sqrt{\varrho \frac{d_w + 1}{d_s}\ \frac{d_\wts(N)}{d_s}}}^{-1} \\
        = &\rbr{\sqrt{\frac{d_\wts(N)}{d_s}} + \sqrt{\varrho\ \frac{d_w + 1}{d_s}}}^{-2}.
    \end{align*}
\end{proof}




\section{Ridge regression analysis}\label{apx:ridge_regression}
In this section, we investigate the more realistic scenario where the weak and strong feature covariances are not exactly low-rank but admit small numbers of dominating eigenvalues. 

Concretely, we consider the same data distribution $(\xb, y) \sim \Dcal(f_*)$ with $y = f_*(\xb) + z$ for some independent Gaussian label noise $z \sim \Ncal(0, \sigma^2)$ and an unknown ground truth function $f_*: \Xcal \to \R$ as in \Cref{sec:ridgeless_regression}.
Under the same sub-gaussian feature assumption as in \Cref{asm:features}, we adapt \Cref{def:low_intrinsic_dim,def:correlation_dim} to the ridge regression setting as follows.
\begin{assumption}[Data distribution]\label{asm:ridge_regression}
    Let $\phi_s: \Xcal \to \R^d$ and $\phi_w: \Xcal \to \R^d$ be the strong and weak pretrained models that take $\xb \sim \Dcal$ and output pretrained features $\phi_s(\xb), \phi_w(\xb) \in \R^d$, respectively.
    \begin{enumerate}[label=(\roman*)]
        \item \b{Ground truth}: Assume $f_*$ can be expressed as a linear function over an unknown ground truth feature $\phi_*: \Xcal \to \R^d$ such that $f_*(\cdot) = \phi_*(\cdot)^\top \thetab_*$ for some fixed $\thetab_* \in \R^d$.
        \item \b{Sub-gaussian features} (\Cref{asm:features}): Let $\phi_w(\xb)$, $\phi_s(\xb)$, $\phi_*(\xb)$ be zero-mean sub-gaussian random vectors with $\E[\phi_w(\xb)] = \E[\phi_s(\xb)] = \E[\phi_*(\xb)] = \b{0}_d$, and 
        \begin{align*}
            \E[\phi_w(\xb) \phi_w(\xb)^\top] = \Sigmab_w, \quad \E[\phi_s(\xb) \phi_s(\xb)^\top] = \Sigmab_s, \quad \E[\phi_*(\xb) \phi_*(\xb)^\top] = \Sigmab_*.
        \end{align*}
        For conciseness, we assume without loss of generality that these features are roughly normalized, \ie, $\nbr{\Sigmab_w}_2 \asymp 1$, $\nbr{\Sigmab_s}_2 \asymp 1$, and $\nbr{\Sigmab_*}_2 \asymp 1$.
        \item \b{Low intrinsic dimension}: Let $\Sigmab_s$ and $\Sigmab_w$ both be \b{positive-definite} with spectral decompositions $\Sigmab_s = \Vb_s \Lambdab_s \Vb_s^\top$ and $\Sigmab_w = \Vb_w \Lambdab_w \Vb_w^\top$, where $\Lambdab_s, \Lambdab_w \in \R^{d \times d}$ are diagonal matrices with positive eigenvalues in decreasing order; while $\Vb_s \in \R^{d \times d}$ and $\Vb_w \in \R^{d \times d}$ are orthogonal matrices consisting of the corresponding orthonormal eigenvectors. The low intrinsic dimension of FT implies that $\Lambdab_s = \diag(\lambda^s_1,\cdots,\lambda^s_d)$ and $\Lambdab_w = \diag(\lambda^w_1,\cdots,\lambda^w_d)$ consist of a few dominating eigenvalues, while the rest of the eigenvalues are negligible, \ie, there exist $d_s, d_w \ll d$ such that $\sum_{i > d_s} \lambda^s_i \ll \tr(\Sigmab_s)$ and $\sum_{i > d_w} \lambda^w_i \ll \tr(\Sigmab_w)$. Here, 
        \begin{align*}
            \tr(\Sigmab_s) \lesssim d_s \quad \t{and} \quad \tr(\Sigmab_w) \lesssim d_w
        \end{align*}
        effectively measure the intrinsic dimensions of $\phi_s$ and $\phi_w$.
    \end{enumerate}
\end{assumption}

\begin{remark}[Weak-strong similarity]
    In place of correlation dimension (\Cref{def:correlation_dim}) in the ridgeless setting, for the ridge regression analysis, we measure the similarity between the weak and strong models directly through $\tr(\Sigmab_s \Sigmab_w)$. Notice that 
    \begin{align*}
        \tr(\Sigmab_s \Sigmab_w) \le \min\cbr{\tr(\Sigmab_s)\nbr{\Sigmab_w}_2, \tr(\Sigmab_w)\nbr{\Sigmab_s}_2} \lesssim \min\cbr{\tr(\Sigmab_s), \tr(\Sigmab_w)}.
    \end{align*}
    In particular, when $\Sigmab_s$ and $\Sigmab_w$ admit low intrinsic dimensions, $\tr(\Sigmab_s \Sigmab_w)$ can be much smaller than $\min\cbr{\tr(\Sigmab_s), \tr(\Sigmab_w)}$ if their eigenvectors corresponding to the dominating eigenvalues are almost orthogonal.
\end{remark}

\begin{remark}[FT approximation errors]
    It is worth noting that under the ground truth and positive-definite covariance assumptions in \Cref{asm:ridge_regression}(i, iii), the FT approximation errors in \Cref{def:ft_est_err} satisfy
    \begin{align}\label{eq:pf_ridge_ft_approx_err}
    \begin{split}
        &\rho_s = \min_{\thetab \in \R^d} \E_{\xb \sim \Dcal}\sbr{(\phi_s(\xb)^\top \thetab - f_*(\xb))^2} = 0 \quad (\t{when } \thetab = \Sigmab_s^{-1} \Sigmab_* \thetab_*), \\
        &\rho_w = \min_{\thetab \in \R^d} \E_{\xb \sim \Dcal}\sbr{(\phi_w(\xb)^\top \thetab - f_*(\xb))^2} = 0 \quad (\t{when } \thetab = \Sigmab_w^{-1} \Sigmab_* \thetab_*).
    \end{split}
    \end{align}
    In place of \Cref{def:ft_est_err}, with positive-definite covariances in \Cref{asm:ridge_regression}, we measure the alignment between the ground truth feature $\phi_*$ and the weak/strong feature $\phi_w, \phi_s$ through
    \begin{align*}
        \varrho_s = \|\Sigmab_s^{-1/2} \Sigmab_*^{1/2} \thetab_*\|_2^2, \quad \varrho_w = \|\Sigmab_w^{-1/2} \Sigmab_*^{1/2} \thetab_*\|_2^2.
    \end{align*}
    Intuitively, for $\Sigmab_s$ and $\Sigmab_w$ with a few dominating eigenvalues (\Cref{asm:ridge_regression}(iii)), $\varrho_s$ and $\varrho_w$ are small if the eigensubspace associated with non-negligible eigenvalues of $\Sigmab_*$ is fully covered by the eigensubspaces associated with the dominating eigenvalues of $\Sigmab_s$ and $\Sigmab_w$, respectively. 
\end{remark}

The W2S FT under ridge regression consists of two steps.
\begin{enumerate}[label=(\alph*)]
    \item First, the weak teacher $f_w(\xb) = \phi_w(\xb)^\top \thetab_w$ is supervisedly finetuned over $\wt\Scal$: 
    \begin{align}\label{eq:w2s_weak_ridge}
        \thetab_w = \argmin_{\thetab \in \R^d} \frac{1}{n}\nbr{\wt\Phib_w \thetab - \wt\yb}_2^2 + \alpha_w \nbr{\thetab}_2^2, \quad \alpha_w > 0.
    \end{align}
    \item Then, the W2S model $f_\wts(\xb) = \phi_s(\xb)^\top \thetab_\wts$ is finetuned over the strong feature $\phi_s$ through the unlabeled samples $\Scal_x$ and their pseudo-labels generated by the weak teacher model:
    \begin{align}\label{eq:w2s_strong_ridge}
        \thetab_\wts = \argmin_{\thetab \in \R^d} \frac{1}{N}\nbr{\Phib_s \thetab - \Phib_w \thetab_w}_2^2 + \alpha_\wts \nbr{\thetab}_2^2, \quad \alpha_\wts > 0.
    \end{align}
\end{enumerate}

\begin{theorem}[W2S under ridge regression]\label{thm:w2s_ridge}
    Let $\varrho_w = \nbr{\Sigmab_w^{-1/2} \Sigmab_*^{1/2} \thetab_*}_2^2$ and $\varrho_s = \nbr{\Sigmab_s^{-1/2} \Sigmab_*^{1/2} \thetab_*}_2^2$.
    Under \Cref{asm:ridge_regression}, the generalization error of W2S FT via ridge regression with fixed $\alpha_w, \alpha_\wts > 0$, $\exrisk(f_\wts) = \vari(f_\wts) + \bias(f_\wts)$, is upper bounded by
    \begin{align*}
        \vari(f_\wts) \le \frac{\sigma^2 \tr\rbr{\Sigmab_s \Sigmab_w}}{4 (\alpha_w n) (\alpha_\wts N)}, \quad
        \bias(f_\wts) \le \alpha_w \varrho_w + \alpha_\wts \varrho_s.
    \end{align*}
    In particular, when taking  
    \begin{align*}
        \alpha_w = \rbr{\frac{\sigma^2 \tr\rbr{\Sigmab_s \Sigmab_w}}{4 n N}\ \frac{\varrho_s}{\varrho_w^2}}^{1/3}, \quad 
        \alpha_\wts = \rbr{\frac{\sigma^2 \tr\rbr{\Sigmab_s \Sigmab_w}}{4 n N}\ \frac{\varrho_w}{\varrho_s^2}}^{1/3},
    \end{align*}
    the excess risk of W2S FT is upper bounded by
    \begin{align*}
        \exrisk(f_\wts) \le 3 \rbr{\frac{\sigma^2 \tr\rbr{\Sigmab_s \Sigmab_w}}{4 n N}\ \varrho_s \varrho_w}^{1/3}.
    \end{align*}
\end{theorem}

\Cref{thm:w2s_ridge} conveys a similar high-level intuition as in \Cref{thm:w2s_ft} regarding the effect of the weak-strong similarity on the generalization error of W2S FT. In particular, the larger discrepancy between $\phi_s$ and $\phi_w$ (corresponding to the smaller $\tr\rbr{\Sigmab_s \Sigmab_w}$) leads to lower variance and therefore better W2S generalization.

Meanwhile, a key difference in W2S between the ridge and ridgeless settings (\Cref{thm:w2s_ridge} versus \Cref{thm:w2s_ft}) is that the FT approximation errors in \Cref{thm:w2s_ridge}, reflected by $\varrho_s = \|\Sigmab_s^{-1/2} \Sigmab_*^{1/2} \thetab_*\|_2^2$ and $\varrho_w = \|\Sigmab_w^{-1/2} \Sigmab_*^{1/2} \thetab_*\|_2^2$, can be compensated by larger sample sizes $n, N$ and directly affect the sample complexity: 
\begin{align*}
    n N \asymp \sigma^2 \tr\rbr{\Sigmab_s \Sigmab_w} \varrho_s \varrho_w.
\end{align*}
Such difference is a result of optimizing the regularization hyperparameters $\alpha_w, \alpha_\wts$ in ridge regression that control the variance-bias tradeoff.

\begin{proof}[Proof of \Cref{thm:w2s_ridge}]
    We first formalize some useful facts on the features and labels as in \eqref{eq:pf_var_w2s_subgaussian_asm}.
    In particular, the sub-gaussian assumption in \Cref{asm:ridge_regression}(ii) implies that for each $\xb \sim \Dcal$, the corresponding strong/weak feature $\phi_s(\xb), \phi_w(\xb) \in \R^d$, and the ground truth $f_*(\xb) \in \R$ are simultaneously characterized by an independent subgaussian random vector $\gammab \in \R^d$ with $\E[\gammab] = \b0_{d}$ and $\E[\gammab \gammab^\top] = \Ib_{d}$, \ie,
    \begin{align*}
        \phi_s(\xb) = \Sigmab_s^{1/2} \gammab, \quad \phi_w(\xb) = \Sigmab_w^{1/2} \gammab, \quad f_*(\xb) = \phi_*(\xb)^\top \thetab_* = \gammab^\top \Sigmab_*^{1/2} \thetab_*.
    \end{align*}

    Then, for $\Scal$ and $\wt\Scal$, there exist independent random matrices $\Gammab = [\gammab_1, \ldots, \gammab_N]^\top \in \R^{N \times d}$ and $\wt\Gammab = [\wt\gammab_1, \ldots, \wt\gammab_n]^\top \in \R^{n \times d}$ consisting of $\iid$ zero-mean isotropic rows such that
    \begin{align}\label{eq:pf_var_w2s_subgaussian_asm_2}
    \begin{split}
        &\Phib_s = \Gammab \Sigmab_s^{1/2} = \Gammab_s \Lambdab_s^{1/2} \Vb_s^\top, \\
        &\Phib_w = \Gammab \Sigmab_w^{1/2} = \Gammab_w \Lambdab_w^{1/2} \Vb_w^\top, \\
        &\yb = \fb_* + \zb, \quad \fb_* = \Gammab \Sigmab_*^{1/2} \thetab_*, \quad \zb \sim \Ncal(\b0_N, \sigma^2 \Ib_N), \\
        &\wt\Phib_w = \wt\Gammab \Sigmab_w^{1/2} = \wt\Gammab_w \Lambdab_w^{1/2} \Vb_w^\top, \\
        &\wt\yb = \wt\fb_* + \wt\zb, \quad \wt\fb_* = \wt\Gammab \Sigmab_*^{1/2} \thetab_*, \quad \wt\zb \sim \Ncal(\b0_n, \sigma^2 \Ib_n),
    \end{split}
    \end{align}
    where $\Gammab_s = \Gammab \Vb_s$, $\Gammab_w = \Gammab \Vb_w$, and $\wt\Gammab_w = \wt\Gammab \Vb_w$.

    \paragraph{Variance-bias decomposition.}
    Recall that the excess risk of W2S generalization $\exrisk(f_\wts)$ can be decomposed into the variance and bias terms:
    \begin{align*}
        &\vari(f_\wts) = \E_{\xb \sim \Dcal}\sbr{\E_{\Scal_x, \wt\Scal}\sbr{(f_\wts(\xb) - \E_{\Scal_x, \wt\Scal}[f_\wts(\xb)])^2}}, \\
        &\bias(f) = \E_{\xb \sim \Dcal}\sbr{(\E_{\Scal_x, \wt\Scal}[f_\wts(\xb)] - f_*(\xb))^2}.
    \end{align*}
    With $\alpha_w > 0$, \eqref{eq:w2s_weak_ridge} yields a weak teacher model $f_w(\xb) = \phi_w(\xb)^\top \thetab_w$ with 
    \begin{align*}
        \thetab_w = \rbr{\wt\Phib_w^\top \wt\Phib_w + \alpha_w n \Ib_d}^{-1} \wt\Phib_w^\top \rbr{\wt\fb_8 + \wt\zb}.
    \end{align*}
    Then, the W2S model $f_\wts(\xb) = \phi_s(\xb)^\top \thetab_\wts$ is given by \eqref{eq:w2s_strong_ridge} with $\alpha_\wts > 0$:
    \begin{align*}
        \thetab_\wts = &\rbr{\Phib_s^\top \Phib_s + \alpha_\wts N \Ib_d}^{-1} \Phib_s^\top \Phib_w \thetab_w \\
        = &\rbr{\Phib_s^\top \Phib_s + \alpha_\wts N \Ib_d}^{-1} \Phib_s^\top \Phib_w \rbr{\wt\Phib_w^\top \wt\Phib_w + \alpha_w n \Ib_d}^{-1} \wt\Phib_w^\top \rbr{\wt\fb_* + \wt\zb},
    \end{align*}
    which implies
    \begin{align*}
        \E_{\Scal_x, \wt\Scal}[\thetab_\wts] = \rbr{\Phib_s^\top \Phib_s + \alpha_\wts N \Ib_d}^{-1} \Phib_s^\top \Phib_w \rbr{\wt\Phib_w^\top \wt\Phib_w + \alpha_w n \Ib_d}^{-1} \wt\Phib_w^\top \wt\fb_*.
    \end{align*}
    Then, we can concretize the variance and bias terms as:
    \begin{align}\label{eq:pf_ridge_var}
    \begin{split}
        &\vari(f_\wts) = \E_{\xb \sim \Dcal}\sbr{\E_{\Scal_x, \wt\Scal}\sbr{(f_\wts(\xb) - \E_{\Scal_x, \wt\Scal}[f_\wts(\xb)])^2}} \\
        = &\E_{\Scal_x, \wt\Scal}\sbr{\nbr{\Sigmab_s^{1/2} \rbr{\Phib_s^\top \Phib_s + \alpha_\wts N \Ib_d}^{-1} \Phib_s^\top \Phib_w \rbr{\wt\Phib_w^\top \wt\Phib_w + \alpha_w n \Ib_d}^{-1} \wt\Phib_w^\top \wt\zb}_2^2},
    \end{split}
    \end{align}
    and
    \begin{align}\label{eq:pf_ridge_bias}
    \begin{split}
        &\bias(f_\wts) = \E_{\xb \sim \Dcal}\sbr{(\E_{\Scal_x, \wt\Scal}[f_\wts(\xb)] - f_*(\xb))^2} \\
        = &\E_{\Scal_x, \wt\Scal}\sbr{\frac{1}{N} \nbr{\Phib_s \rbr{\Phib_s^\top \Phib_s + \alpha_\wts N \Ib_d}^{-1} \Phib_s^\top \Phib_w \rbr{\wt\Phib_w^\top \wt\Phib_w + \alpha_w n \Ib_d}^{-1} \wt\Phib_w^\top \wt\fb_* - \fb_*}_2^2}.
    \end{split}
    \end{align}
    Now, we are ready to upper bound the variance and bias terms separately.

    \paragraph{Variance.}
    Denote $\zetab = \Lambdab_w^{1/2} \Vb_w^\top \rbr{\wt\Phib_w^\top \wt\Phib_w + \alpha_w n \Ib_d}^{-1} \wt\Phib_w^\top \wt\zb \in \R^d$, whose randomness comes from $\wt\Scal$ only, independent of $\Scal_x$.
    Then, the variance term \eqref{eq:pf_ridge_var} can be expressed as
    \begin{align*}
        &\vari(f_\wts) = \E_{\Scal_x, \wt\Scal}\sbr{\nbr{\Sigmab_s^{1/2} \rbr{\Phib_s^\top \Phib_s + \alpha_\wts N \Ib_d}^{-1} \Phib_s^\top \Phib_w \zetab}_2^2} \\
        = &\tr\rbr{\E_{\Scal_s}\rbr{\Gammab_w^\top \Phib_s \rbr{\Phib_s^\top \Phib_s + \alpha_\wts N \Ib_d}^{-1} \Sigmab_s \rbr{\Phib_s^\top \Phib_s + \alpha_\wts N \Ib_d}^{-1} \Phib_s^\top \Gammab_w} \E_{\wt\Scal}\sbr{\zetab \zetab^\top}} \\
        = &\tr\rbr{\E_{\Scal_s}\rbr{\Gammab_w^\top \Gammab_s \rbr{\Gammab_s^\top \Gammab_s + \alpha_\wts N \Lambdab_s^{-1}}^{-1} \rbr{\Gammab_s^\top \Gammab_s + \alpha_\wts N \Lambdab_s^{-1}}^{-1} \Gammab_s^\top \Gammab_w} \E_{\wt\Scal}\sbr{\zetab \zetab^\top}} \\
        = &\tr\rbr{\E_{\Scal_s}\rbr{\Vb_w^\top \Gammab^\top \Gammab \rbr{\Gammab^\top \Gammab + \alpha_\wts N \Sigmab_s^{-1}}^{-2} \Gammab^\top \Gammab \Vb_w} \E_{\wt\Scal}\sbr{\zetab \zetab^\top}} \\
        = &\tr\rbr{\E_{\Scal_s}\rbr{\Gammab^\top \Gammab \rbr{\Gammab^\top \Gammab + \alpha_\wts N \Sigmab_s^{-1}}^{-2} \Gammab^\top \Gammab} \E_{\wt\Scal}\sbr{\Vb_w \zetab \zetab^\top \Vb_w^\top}}.
    \end{align*}
    Notice that $\rbr{\Gammab^\top \Gammab + \alpha_\wts N \Sigmab_s^{-1}}^{2} \succeq \alpha_\wts N \rbr{\Gammab^\top \Gammab \Sigmab_s^{-1} + \Sigmab_s^{-1} \Gammab^\top \Gammab}$.
    Since matrix inversion is convex, a Jensen-type inequality implies that
    \begin{align*}
        &\Gammab^\top \Gammab \rbr{\Gammab^\top \Gammab + \alpha_\wts N \Sigmab_s^{-1}}^{-2} \Gammab^\top \Gammab \\
        \preceq &\Gammab^\top \Gammab \rbr{\alpha_\wts N \rbr{\Gammab^\top \Gammab \Sigmab_s^{-1} + \Sigmab_s^{-1} \Gammab^\top \Gammab}}^{\dagger} \Gammab^\top \Gammab \\
        = &\frac{1}{2 \alpha_\wts N} \Gammab^\top \Gammab \rbr{\frac{1}{2} \rbr{\Gammab^\top \Gammab \Sigmab_s^{-1} + \Sigmab_s^{-1} \Gammab^\top \Gammab}}^{\dagger} \Gammab^\top \Gammab \\
        \preceq &\frac{1}{4 \alpha_\wts N} \rbr{\Gammab^\top \Gammab \Sigmab_s + \Sigmab_s \Gammab^\top \Gammab}.
    \end{align*}
    Therefore, 
    \begin{align*}
        \E_{\Scal_s}\rbr{\Gammab^\top \Gammab \rbr{\Gammab^\top \Gammab + \alpha_\wts N \Sigmab_s^{-1}}^{-2} \Gammab^\top \Gammab}
        \preceq &\frac{1}{4 \alpha_\wts N} \E_{\Scal_s}\sbr{\Gammab^\top \Gammab \Sigmab_s + \Sigmab_s \Gammab^\top \Gammab} 
        = \frac{1}{2 \alpha_\wts N} \Sigmab_s.
    \end{align*}
    Meanwhile, we observe that
    \begin{align*}
        \E_{\wt\Scal}\sbr{\Vb_w \zetab \zetab^\top \Vb_w^\top} 
        = &\E_{\wt\Scal}\sbr{\Sigmab_w^{1/2} \rbr{\wt\Phib_w^\top \wt\Phib_w + \alpha_w n \Ib_d}^{-1} \wt\Phib_w^\top \wt\zb \wt\zb^\top \wt\Phib_w \rbr{\wt\Phib_w^\top \wt\Phib_w + \alpha_w n \Ib_d}^{-1} \Sigmab_w^{1/2}} \\
        = &\sigma^2 \E_{\wt\Scal}\sbr{\Sigmab_w^{1/2} \rbr{\wt\Phib_w^\top \wt\Phib_w + \alpha_w n \Ib_d}^{-1} \wt\Phib_w^\top \wt\Phib_w \rbr{\wt\Phib_w^\top \wt\Phib_w + \alpha_w n \Ib_d}^{-1} \Sigmab_w^{1/2}},
    \end{align*}
    where 
    \begin{align*}
        \rbr{\wt\Phib_w^\top \wt\Phib_w + \alpha_w n \Ib_d}^{-1} \wt\Phib_w^\top \wt\Phib_w \rbr{\wt\Phib_w^\top \wt\Phib_w + \alpha_w n \Ib_d}^{-1}
        \preceq &\frac{1}{2 \alpha_w n} \Ib_d.
    \end{align*}
    Therefore, we have
    \begin{align*}
        \E_{\wt\Scal}\sbr{\Vb_w \zetab \zetab^\top \Vb_w^\top} 
        \preceq &\sigma^2 \E_{\wt\Scal}\sbr{\Sigmab_w^{1/2} \rbr{\frac{1}{2 \alpha_w n} \Ib_d} \Sigmab_w^{1/2}}
        = \frac{\sigma^2}{2 \alpha_w n} \Sigmab_w.
    \end{align*}
    Overall, the variance of $f_\wts$ can be upper bounded as
    \begin{align}\label{eq:pf_ridge_var_ub}
    \begin{split}
        \vari(f_\wts) 
        = &\tr\rbr{\E_{\Scal_s}\rbr{\Gammab^\top \Gammab \rbr{\Gammab^\top \Gammab + \alpha_\wts N \Sigmab_s^{-1}}^{-2} \Gammab^\top \Gammab} \E_{\wt\Scal}\sbr{\Vb_w \zetab \zetab^\top \Vb_w^\top}} \\
        \le &\frac{\sigma^2 \tr\rbr{\Sigmab_s \Sigmab_w}}{4 (\alpha_w n) (\alpha_\wts N)}.
    \end{split}
    \end{align}

    \paragraph{Bias.}
    Let $\xib = \Sigmab_w^{1/2} \rbr{\wt\Phib_w^\top \wt\Phib_w + \alpha_w n \Ib_d}^{-1} \wt\Phib_w^\top \wt\fb_* \in \R^d$, whose randomness comes from $\wt\Scal$ only, independent of $\Scal_x$.
    Recall from \eqref{eq:pf_ridge_bias}, the bias term \eqref{eq:pf_ridge_bias} can be decomposed as
    \begin{align*}
        &\bias(f_\wts) = \E_{\Scal_x, \wt\Scal}\sbr{\frac{1}{N} \nbr{\Phib_s \rbr{\Phib_s^\top \Phib_s + \alpha_\wts N \Ib_d}^{-1} \Phib_s^\top \Phib_w \rbr{\wt\Phib_w^\top \wt\Phib_w + \alpha_w n \Ib_d}^{-1} \wt\Phib_w^\top \wt\fb_* - \fb_*}_2^2}\\
        &= \E_{\Scal_x, \wt\Scal}\sbr{\frac{1}{N} \rbr{\nbr{\Phib_s \rbr{\Phib_s^\top \Phib_s + \alpha_\wts N \Ib_d}^{-1} \Phib_s^\top \Gammab \xib - \Phib_s \Phib_s^\dagger \fb_*}_2^2 + \nbr{\rbr{\Ib_N - \Phib_s \Phib_s^\dagger} \fb_*}_2^2}},
    \end{align*}
    where by \Cref{lem:low_est_err_ft} and \eqref{eq:pf_ridge_ft_approx_err}
    \begin{align*}
        \E_{\Scal_x}\sbr{\frac{1}{N} \nbr{\rbr{\Ib_N - \Phib_s \Phib_s^\dagger} \fb_*}_2^2}
        = \frac{\rho_s(N)}{N} \le \rho_s = 0.
    \end{align*}
    Therefore, with $\xib = \Sigmab_w^{1/2} \rbr{\wt\Phib_w^\top \wt\Phib_w + \alpha_w n \Ib_d}^{-1} \wt\Phib_w^\top \wt\fb_*$, we have
    \begin{align*}
        \bias(f_\wts) = \E_{\Scal_x, \wt\Scal}\sbr{\frac{1}{N} \nbr{\Phib_s \rbr{\Phib_s^\top \Phib_s + \alpha_\wts N \Ib_d}^{-1} \Phib_s^\top \Gammab \xib - \Phib_s \Phib_s^\dagger \fb_*}_2^2}.
    \end{align*}
    Recall that $\fb_* = \Gammab \Sigmab_*^{1/2} \thetab_*$ and $\Phib_s = \Gammab \Sigmab_s^{1/2} = \Gammab_s \Lambdab_s^{1/2} \Vb_s^\top$.
    Then, we can express the bias term as
    \begin{align*}
        \bias(f_\wts) = &\E_{\Scal_x, \wt\Scal}\sbr{\frac{1}{N} \nbr{\Gammab\rbr{\Gammab^\top \Gammab + \alpha_\wts N \Sigmab_s^{-1}}^{-1} \Gammab^\top \Gammab \xib - \Gammab \Gammab^\dagger \fb_*}_2^2} \\
        = &\E_{\Scal_x, \wt\Scal}\sbr{\frac{1}{N} \nbr{\Gammab \Sigmab_*^{1/2} \thetab_* - \Gammab\rbr{\Gammab^\top \Gammab + \alpha_\wts N \Sigmab_s^{-1}}^{-1} \Gammab^\top \Gammab \xib}_2^2} \\
        = &\E_{\Scal_x, \wt\Scal}\sbr{\frac{1}{N} \nbr{\Gammab \rbr{\Sigmab_*^{1/2} \thetab_* - \xib} + \Gammab \rbr{\Ib_d - \rbr{\Gammab^\top \Gammab + \alpha_\wts N \Sigmab_s^{-1}}^{-1} \Gammab^\top \Gammab} \xib}_2^2} \\
    \end{align*} 
    By Woodbury matrix identity, we have
    \begin{align}\label{eq:pf_ridge_bias_woodbury}
        \Ib_d - \rbr{\Gammab^\top \Gammab + \alpha_\wts N \Sigmab_s^{-1}}^{-1} \Gammab^\top \Gammab
        = \rbr{\Ib_d + \frac{1}{\alpha_\wts N} \Sigmab_s \Gammab^\top \Gammab}^{-1}.
    \end{align}
    Therefore, we have 
    \begin{align}\label{eq:pf_ridge_bias_inter1}
        \bias(f_\wts) = \E_{\Scal_x, \wt\Scal}\Bigg[\frac{1}{N} \Big\|\underbrace{\Gammab \rbr{\Sigmab_*^{1/2} \thetab_* - \xib}}_{\t{Term A}} + \underbrace{\Gammab \rbr{\Ib_d + \frac{1}{\alpha_\wts N} \Sigmab_s \Gammab^\top \Gammab}^{-1} \xib}_{\t{Term B}}\Big\|_2^2\Bigg].
    \end{align}

    For Term A, notice that $\xib = \Sigmab_w^{1/2} \rbr{\wt\Phib_w^\top \wt\Phib_w + \alpha_w n \Ib_d}^{-1} \wt\Phib_w^\top \wt\fb_*$ implies
    \begin{align*}
        \Sigmab_*^{1/2} \thetab_* - \xib 
        = &\Sigmab_*^{1/2} \thetab_* - \Sigmab_w^{1/2} \rbr{\wt\Phib_w^\top \wt\Phib_w + \alpha_w n \Ib_d}^{-1} \wt\Phib_w^\top \wt\fb_* \\
        = &\Sigmab_*^{1/2} \thetab_* - \rbr{\wt\Gammab^\top \wt\Gammab + \alpha_w n \Sigmab_w^{-1}}^{-1} \wt\Gammab^\top \wt\Gammab \Sigmab_*^{1/2} \thetab_* \\
        = &\rbr{\Ib_d - \rbr{\wt\Gammab^\top \wt\Gammab + \alpha_w n \Sigmab_w^{-1}}^{-1} \wt\Gammab^\top \wt\Gammab} \Sigmab_*^{1/2} \thetab_* \\
        = &\rbr{\Ib_d + \frac{1}{\alpha_w n} \Sigmab_w \wt\Gammab^\top \wt\Gammab}^{-1} \Sigmab_*^{1/2} \thetab_*,
    \end{align*}
    where the last equality follows from Woodbury matrix identity as in \eqref{eq:pf_ridge_bias_woodbury}.
    Therefore,
    \begin{align*}
        \E_{\Scal_x, \wt\Scal}\sbr{\frac{1}{N} \nbr{\Gammab \rbr{\Sigmab_*^{1/2} \thetab_* - \xib}}_2^2} 
        = &\E_{\wt\Scal}\sbr{\frac{1}{n} \nbr{\wt\Gammab \rbr{\Sigmab_*^{1/2} \thetab_* - \xib}}_2^2} \\
        = &\E_{\wt\Scal}\sbr{\frac{1}{n} \nbr{\wt\Gammab \rbr{\Ib_d + \frac{1}{\alpha_w n} \Sigmab_w \wt\Gammab^\top \wt\Gammab}^{-1} \Sigmab_*^{1/2} \thetab_*}_2^2}.
    \end{align*}
    Since 
    \begin{align*}
        \rbr{\Ib_d + \frac{1}{\alpha_w n} \Sigmab_w \wt\Gammab^\top \wt\Gammab}^{-1} \wt\Gammab^\top \wt\Gammab \rbr{\Ib_d + \frac{1}{\alpha_w n} \Sigmab_w \wt\Gammab^\top \wt\Gammab}^{-1} \preceq \frac{\alpha_w n}{2} \Sigmab_w^{-1},
    \end{align*}
    we have
    \begin{align}\label{eq:pf_ridge_bias_term1}
    \begin{split}
        \E_{\Scal_x, \wt\Scal}\sbr{\frac{1}{N} \nbr{\Gammab \rbr{\Sigmab_*^{1/2} \thetab_* - \xib}}_2^2} 
        \le &\frac{1}{n} \tr\rbr{\frac{\alpha_w n}{2} \Sigmab_w^{-1} \Sigmab_*^{1/2} \thetab_* \thetab_*^\top \Sigmab_*^{1/2}} \\
        = &\frac{\alpha_w}{2} \nbr{\Sigmab_w^{-1/2} \Sigmab_*^{1/2} \thetab_*}_2^2.
    \end{split}
    \end{align}
    
    For Term B, leveraging Woodbury matrix identity as in \eqref{eq:pf_ridge_bias_woodbury}, we notice that 
    \begin{align*}
        &\E_{\Scal_x, \wt\Scal}\sbr{\frac{1}{N} \nbr{\Gammab \rbr{\Ib_d + \frac{1}{\alpha_\wts N} \Sigmab_s \Gammab^\top \Gammab}^{-1} \xib}_2^2} 
        \le \E_{\Scal_x, \wt\Scal}\sbr{\frac{1}{N} \tr\rbr{\frac{\alpha_\wts N}{2} \Sigmab_s^{-1} \xib \xib^\top}} \\
        = &\frac{\alpha_\wts}{2} \E_{\Scal_x, \wt\Scal}\sbr{\nbr{\Sigmab_s^{-1/2} \Sigmab_w^{1/2} \rbr{\wt\Phib_w^\top \wt\Phib_w + \alpha_w n \Ib_d}^{-1} \wt\Phib_w^\top \wt\fb_*}_2^2} \\
        = &\frac{\alpha_\wts}{2} \E_{\Scal_x, \wt\Scal}\sbr{\nbr{\Sigmab_s^{-1/2} \rbr{\wt\Gammab^\top \wt\Gammab + \alpha_w n \Sigmab_w^{-1}}^{-1} \wt\Gammab^\top \wt\Gammab \Sigmab_*^{1/2} \thetab_*}_2^2}
    \end{align*}
    Since $\rbr{\wt\Gammab^\top \wt\Gammab + \alpha_w n \Sigmab_w^{-1}}^{-1} \wt\Gammab^\top \wt\Gammab \preceq \Ib_d$, we know that
    \begin{align}\label{eq:pf_ridge_bias_term2}
    \begin{split}
        \E_{\Scal_x, \wt\Scal}\sbr{\frac{1}{N} \nbr{\Gammab \rbr{\Ib_d + \frac{1}{\alpha_\wts N} \Sigmab_s \Gammab^\top \Gammab}^{-1} \xib}_2^2} 
        \le \frac{\alpha_\wts}{2} \nbr{\Sigmab_s^{-1/2} \Sigmab_*^{1/2} \thetab_*}_2^2.
    \end{split}
    \end{align}
    Combining \eqref{eq:pf_ridge_bias_inter1}, \eqref{eq:pf_ridge_bias_term1}, and \eqref{eq:pf_ridge_bias_term2}, we can upper bound the bias term as
    \begin{align}\label{eq:pf_ridge_bias_final}
    \begin{split}
        &\bias(f_\wts) = \E_{\Scal_x, \wt\Scal}\Bigg[\frac{1}{N} \Big\|\underbrace{\Gammab \rbr{\Sigmab_*^{1/2} \thetab_* - \xib}}_{\t{Term A}} + \underbrace{\Gammab \rbr{\Ib_d + \frac{1}{\alpha_\wts N} \Sigmab_s \Gammab^\top \Gammab}^{-1} \xib}_{\t{Term B}}\Big\|_2^2\Bigg] \\
        \le &2 \E_{\Scal_x, \wt\Scal}\sbr{\frac{1}{N} \nbr{\Gammab \rbr{\Sigmab_*^{1/2} \thetab_* - \xib}}_2^2} + 2 \E_{\Scal_x, \wt\Scal}\sbr{\frac{1}{N} \nbr{\Gammab \rbr{\Ib_d + \frac{1}{\alpha_\wts N} \Sigmab_s \Gammab^\top \Gammab}^{-1} \xib}_2^2} \\
        \le &\alpha_w \nbr{\Sigmab_w^{-1/2} \Sigmab_*^{1/2} \thetab_*}_2^2 + \alpha_\wts \nbr{\Sigmab_s^{-1/2} \Sigmab_*^{1/2} \thetab_*}_2^2.
    \end{split}
    \end{align}
    
    \paragraph{Variance-bias tradeoff.}
    Overall, by \eqref{eq:pf_ridge_var_ub} and \eqref{eq:pf_ridge_bias_final}, we have
    \begin{align*}
        &\vari(f_\wts) \le \frac{\sigma^2 \tr\rbr{\Sigmab_s \Sigmab_w}}{4 (\alpha_w n) (\alpha_\wts N)}, \\
        &\bias(f_\wts) \le \alpha_w \nbr{\Sigmab_w^{-1/2} \Sigmab_*^{1/2} \thetab_*}_2^2 + \alpha_\wts \nbr{\Sigmab_s^{-1/2} \Sigmab_*^{1/2} \thetab_*}_2^2.
    \end{align*}
    The upper bound the excess risk $\exrisk(f_\wts) = \vari(f_\wts) + \bias(f_\wts)$ is minimized by taking 
    \begin{align*}
        \alpha_w = \rbr{\frac{\sigma^2 \tr\rbr{\Sigmab_s \Sigmab_w}}{4 n N}\ \frac{\nbr{\Sigmab_s^{-1/2} \Sigmab_*^{1/2} \thetab_*}_2^2}{\nbr{\Sigmab_w^{-1/2} \Sigmab_*^{1/2} \thetab_*}_2^4}}^{1/3}, \ 
        \alpha_\wts = \rbr{\frac{\sigma^2 \tr\rbr{\Sigmab_s \Sigmab_w}}{4 n N}\ \frac{\nbr{\Sigmab_w^{-1/2} \Sigmab_*^{1/2} \thetab_*}_2^2}{\nbr{\Sigmab_s^{-1/2} \Sigmab_*^{1/2} \thetab_*}_2^4}}^{1/3},
    \end{align*}
    which leads to the optimal upper bound for the excess risk:
    \begin{align*}
        \exrisk(f_\wts) \le 3 \rbr{\frac{\sigma^2 \tr\rbr{\Sigmab_s \Sigmab_w}}{4 n N}\ \nbr{\Sigmab_s^{-1/2} \Sigmab_*^{1/2} \thetab_*}_2^2 \nbr{\Sigmab_w^{-1/2} \Sigmab_*^{1/2} \thetab_*}_2^2}^{1/3}.
    \end{align*}
\end{proof}






\section{Canonical angles}\label{apx:canonical_angles}
In this section, we review the concept of canonical angles between two subspaces that connect the formal definition of the correlation dimension $d_{s \wedge w} = \nbr{\Vb_s^\top \Vb_w}_F^2$ in \Cref{def:correlation_dim} to the intuitive notion of the alignment between the corresponding subspaces $\Vcal_s$ and $\Vcal_w$ in the introduction: $\sum \cos(\angle(\Vcal_s, \Vcal_w)) = \nbr{\Vb_s^\top \Vb_w}_F^2$.
\begin{definition}[Canonical angles \cite{golub2013matrix}, adapting from \cite{dong2024efficient}]\label{def:canonical_angles}
    Let $\Vcal_s,\Vcal_w \subseteq \R^d$ be two subspaces with dimensions $\dim\rbr{\Vcal_s}=d_s$ and $\dim\rbr{\Vcal_w}=d_w$ (assuming $d_w \geq d_s$ without loss of generality). The canonical angles $\angle\rbr{\Vcal_s,\Vcal_w}=\diag\rbr{\nu_1,\dots,\nu_{d_s}}$ are $d_s$ angles that jointly measure the alignment between $\Vcal_s$ and $\Vcal_w$, defined recursively as follows:
    \begin{align*}
        &\ub_i, \vb_i ~\triangleq~
        \argmax~\ub_i^*\vb_i \\
        \t{s.t.}~
        &\ub_i \in \rbr{\Vcal_s \setminus \spn\cbr{\ub_{\iota}}_{\iota=1}^{i-1}} \cap \SSS^{d-1},\\ 
        &\vb_i \in \rbr{\Vcal_w \setminus \spn\cbr{\vb_{\iota}}_{\iota=1}^{i-1}} \cap \SSS^{d-1}\\
        &\cos (\nu_i) = \ub_i^* \vb_i \quad \forall~ i=1,\dots,k,
    \end{align*}
    such that $0 \leq \nu_1 \leq \dots \leq \nu_k \leq \pi/2$.

    Given two subspaces $\Vcal_s,\Vcal_w \subseteq \R^d$, let $\Vb_s \in \R^{d \times d_s}$ and $\Vb_w \in \R^{d \times d_w}$ be the matrices whose columns form orthonormal bases for $\Vcal_s$ and $\Vcal_w$, respectively. Then, the canonical angles $\angle(\Vcal_s, \Vcal_w)$ are determined by the singular values of $\Vb_s^\top \Vb_w$~\citep[\S 3]{bjorck1973numerical}:
    \begin{align*}
        \cos(\angle_i(\Vcal_s, \Vcal_w)) = \sigma_i(\Vb_s^\top \Vb_w) \quad \forall~ i=1,\dots,d_s,
    \end{align*}
    where $\sigma_i(\Vb_s^\top \Vb_w)$ denotes the $i$-th singular value of $\Vb_s^\top \Vb_w$.
\end{definition}

In particular, since $\Vb_s, \Vb_w$ consist of orthonormal columns, the singular values of $\Vb_s^\top \Vb_w$ fall in $[0,1]$, and therefore,
\begin{align*}
    d_{s \wedge w} = \sum \cos(\angle(\Vcal_s, \Vcal_w)) = \nbr{\Vb_s^\top \Vb_w}_F^2 \in [0, \min\cbr{d_s, d_w}].
\end{align*}




\section{Additional experiments}\label{apx:exp_details}

\subsection{Additional experiments and details on UTKFace regression}\label{apx:exp_img_reg}
This section provides some additional details and results for the UTKFace regression experiments in \Cref{sec:exp_img_reg}. 

\begin{figure}[!h]
    \centering
    \includegraphics[width=\columnwidth]{fig/mse_utkface_resnet18_clipb32.pdf}%\vspace{-2em}
    \caption{Scaling for MSE on UTKFace with \texttt{CLIP-B32} as the strong student and \texttt{ResNet18} as the weak teacher}\label{fig:mse_utkface_resnet18-clip}
\end{figure}

\begin{figure}[!h]
    \centering
    \includegraphics[width=\columnwidth]{fig/mse_utkface_resnet50_clipb32.pdf}%\vspace{-2em}
    \caption{Scaling for MSE on UTKFace with \texttt{CLIP-B32} as the strong student and \texttt{ResNet50} as the weak teacher}\label{fig:mse_utkface_resnet50-clip}
\end{figure}

\begin{figure}[!h]
    \centering
    \includegraphics[width=\columnwidth]{fig/mse_utkface_resnet152_clipb32.pdf}%\vspace{-2em}
    \caption{Scaling for MSE on UTKFace with \texttt{CLIP-B32} as the strong student and \texttt{ResNet152} as the weak teacher}\label{fig:mse_utkface_resnet152-clip}
\end{figure}

We summarize the relevant dimensionality in \Cref{tab:img_reg_dim}. We observe the following:
\begin{itemize}
    \item The intrinsic dimensions of the pretrained features are significantly smaller than the ambiance feature dimensions, which is consistent with our theoretical analysis and the empirical observations in \cite{aghajanyan2020intrinsic}. 
    \item The correlation dimensions $d_{s \wedge w}$ are considerably smaller than the corresponding intrinsic dimensions, indicating that the subspaces spanned by the weak and strong features are not aligned in practice. As shown in \Cref{sec:exp_img_reg}, such discrepancies in the weak and strong features facilitate W2S generalization.
\end{itemize}

\begin{table}[!ht]
    \centering
    \caption{Summary of the pretrained feature dimensions, along with the intrinsic dimensions $d_s, d_w$ and correlation dimensions $d_{s \wedge w}$ (with respect to the strong student \texttt{CLIP-B32}) computed over the entire UTKFace dataset (including training and testing).}\label{tab:img_reg_dim}
    \begin{tabular}{c|ccc}
        \toprule
        Pretrained Model & Feature Dimension & Intrinsic Dimension ($\tau=0.01$) & Correlation Dimension \\
        \midrule
        \texttt{ResNet18} & 512 & 194 & 167.64 \\
        \texttt{ResNet34} & 512 & 150 & 129.97 \\
        \texttt{ResNet50} & 2048 & 522 & 301.06 \\
        \texttt{ResNet101} & 2048 & 615 & 354.52 \\
        \texttt{ResNet152} & 2048 & 589 & 339.90 \\
        \midrule
        \texttt{CLIP-B32} & 768 & 443 & $\times$ \\
        \bottomrule
    \end{tabular}
\end{table}

For reference, we provide the scaling for MSE losses of three representative teacher-student pairs in \Cref{fig:mse_utkface_resnet18-clip,fig:mse_utkface_resnet50-clip,fig:mse_utkface_resnet152-clip}. 
\begin{itemize}
    \item It is worth highlighting that while the MSE loss of $f_\wts$ monotonically decreases with respect to both sample sizes $n,N$, the different rates of convergence compared to $f_w$, $f_s$, and $f_c$ lead to the distinct scaling behavior of the relative W2S performance ($\pgr$ and $\opr$) with respect to $n$ versus $N$ in \Cref{fig:pgr_opr_utkface_resnet-clip,fig:pgr_opr_utkface_vardom_resnet-clip}.
    \item When the strong student has a lower intrinsic dimension than the weak teacher (\cf \Cref{fig:mse_utkface_resnet18-clip} versus \Cref{fig:mse_utkface_resnet50-clip,fig:mse_utkface_resnet152-clip}), $d_s < d_w$, the W2S model $f_\wts$ tends to achieve better generalization in terms of the test MSE. This is consistent with our analysis in \Cref{sec:generalization_errors}.
    \item When $d_s < d_w$, the W2S model $f_\wts$ tends to achieve (slightly) better generalization for (slightly) smaller correlation dimension $d_{s \wedge w}$ (\cf \Cref{fig:mse_utkface_resnet50-clip} versus \Cref{fig:mse_utkface_resnet152-clip}), again coinciding with our analysis in \Cref{sec:generalization_errors}.
    \item W2S generalization generally happens (\ie $f_\wts$ is able to outperform $f_w$) with sufficiently large sample sizes $n, N$. However, as the labeled sample size $n$ increases, the test MSE of $f_\wts$ converges slower than that of the strong baseline and ceiling models, $f_s$ and $f_c$, leading to the inverse scaling for $\pgr$ and $\opr$ with respect to $n$ in \Cref{fig:pgr_opr_utkface_resnet-clip,fig:pgr_opr_utkface_vardom_resnet-clip}. When $n$ is too large, the W2S model $f_\wts$ may not be able to achieve better generalization than the strong baseline $f_s$.
\end{itemize}




\subsection{Experiments on image classification}\label{apx:exp_img_cls}

\paragraph{Dataset.} ColoredMNIST \citep{arjovsky2019invariant} consists of groups of different colors and reassign the label to be binary (digits 0-4 vs 5-9). We pool together the groups to form one dataset. The choice is to bring diversity to the feature space with additional color features and thus potential feature discrepancies. We hold out a test set of 7000 samples and used the rest 63000 samples for training.

\paragraph{Linear probing over pretrained features.} We fix a strong student as DINOv2-s14 \citep{oquab2023dinov2} and vary the weak teacher among the ResNet-d series and ResNet series (ResNet18D, ResNet34D, ResNet101, ResNet152) \citep{he2018resnetd,he2015deepresiduallearningimage}. We replace ResNet18 and ResNet34 used in \Cref{sec:exp_img_reg} to experiment on weak models with similar intrinsic dimensions but different correlation dimensions. We treat the backbone of the models (excluding the classification layer) as $\phi_s$ and $\phi_w$ and finetune them via linear probing. We train the models with cross entropy loss and AdamW optimizer. We tune the training hyperparameters of weak and strong models using a validation set and train them for 800 steps with learning rate 1e-3 and weight decay 1e-6. 

\begin{table}[!ht]
    \centering
    \caption{Summary of the pretrained feature dimensions, along with the intrinsic dimensions $d_s, d_w$ and correlation dimensions $d_{s \wedge w}$ (with respect to the strong student \texttt{DINOv2-S14}) computed over the entire ColoredMNIST dataset (including training and testing).}\label{tab:img_cls_dim_coloredmnist}
    \begin{tabular}{c|ccc}
        \toprule
        Pretrained Model & Feature Dimension & Intrinsic Dimension ($\tau=0.01$) & Correlation Dimension \\
        \midrule
        \texttt{ResNet-18-D} & 512 & 117 & 6.23 \\
        \texttt{ResNet-34-D} & 512 & 127 & 7.07 \\
        \texttt{ResNet101} & 2048 & 121 & 1.74 \\
        \texttt{ResNet152} & 2048 & 128 & 1.88 \\
        \midrule
        \texttt{DINOv2-S14} & 384 & 28 & $\times$ \\
        \bottomrule
    \end{tabular}
\end{table}

\begin{figure}[!h]
    \centering
    \includegraphics[width=\columnwidth]{fig/coloredmnist_lp/coloredmnist_dsw.pdf}%\vspace{-2em}
    \caption{Scaling for $\pgr$ and $\opr$ of different weak teachers with a fixed strong student on ColoredMNIST.}\label{fig:coloredmnist_dscapw}
\end{figure}

\begin{figure}[!h]
    \centering
    \includegraphics[width=\columnwidth]{fig/coloredmnist_lp/coloredmnist_var.pdf}%\vspace{-2em}
    \caption{Scaling for $\pgr$ and $\opr$ of W2S on ColoredMNIST with injected label noise.}\label{fig:coloredmnist_variance}
\end{figure}

\paragraph{Discrepancies lead to better W2S.}
\Cref{fig:coloredmnist_dscapw} shows the scaling of $\pgr$ and $\opr$ with respect to the sample sizes $n, N$ for different weak teachers in the ResNet series with respect to a fixed student, \texttt{CLIP-B32}. 
As in \Cref{sec:exp_img_reg}, we observe that with similar intrinsic dimensions $d_s, d_w$, W2S achieves better relative performance in terms of $\pgr$ and $\opr$ when the correlation dimension $d_{s \wedge w}$ is smaller.

\paragraph{Variance reduction is a key advantage of W2S.}
We inject noise to the labels of the original ColoredMNIST training samples by randomly flipping the ground truth labels with probability $\varsigma \in [0,1]$ (following \cite{arjovsky2019invariant}). 
\Cref{fig:coloredmnist_variance} shows the scaling of $\pgr$ and $\opr$ with respect to $n$ and $N$ when taking DINOv2-S14 as the strong student and ResNet101 as the weak teacher. We observe that the larger artificial label noise $\varsigma$ leads to higher $\pgr$ and $\opr$. 

\subsection{Experiments on sentiment classification}\label{apx:exp_nlp_cls}

\paragraph{Dataset.} The Stanford Sentiment Treebank \citep{socher-etal-2013-sst2} is a corpus with fully labeled parse trees that allows for a complete analysis of the compositional effects of sentiment in language. The corpus is based on the dataset introduced by \citet{pang-lee-2005-sst_original_corpus} and consists of 11,855 single sentences extracted from movie reviews. It was parsed with the Stanford parser and includes a total of 215,154 unique phrases from those parse trees, each annotated by 3 human judges. We conduct binary classification experiments on full sentences (negative or somewhat negative vs somewhat positive or positive with neutral sentences discarded), the so-called SST-2 dataset, and split the dataset into training and testing sets of sizes 63000 and 4349. Generalization errors are estimated with the 0-1 loss over the test set.

\paragraph{Full finetuning.} We fix the strong student as Electra-base-discriminator \citep{clark2020electra} and vary the weak teacher among the Bert series \citep{turc2019bert-tiny} (Bert-Tiny, Bert-Mini, Bert-Small, Bert-Medium). 
With manageable model sizes, we conduct full finetuning experiments following the setup in \cite{burns2023weak}.
We use the standard cross entropy loss for supervised finetuning. 
When training strong students on weak labels (W2S), we use the confidence weighted loss proposed by \cite{burns2023weak}, which is suggested to be able to improve weak-to-strong generalization on many NLP tasks.
All training is conducted via Adam optimizers~\citep{kingma2014adam} with a learning rate of 5e-5, a cosine learning rate schedule, and 40 warmup steps. We train for 3 epochs, which is sufficient for the train and validation loss to stabilize. 

\paragraph{Intrinsic dimension.} The intrinsic dimensions $d_w,d_s$ are measured based on the Structure-Aware Intrinsic Dimension (SAID) method proposed by \cite{aghajanyan2020intrinsic}. We first train the full models on the whole training set, and then train the models with only $d$ trainable parameters based on SAID transformation. The $d_w$ or $d_s$ are the smallest number of parameters under SAID that is necessary to retain 90\% accuracy of the full models. Here, the 90\% accuracy is a common threshold used to estimate intrinsic dimensions in the literature \citep{li2018measuring}.

\begin{figure}[!h]
    \centering
    \includegraphics[width=\columnwidth]{fig/sst2/sst2-dsw.pdf}%\vspace{-2em}
    \caption{Scaling for $\pgr$ and $\opr$ of different weak teachers with a fixed strong student on SST-2.}\label{fig:sst2_dsw}
\end{figure}

\begin{figure}[!h]
    \centering
    \includegraphics[width=\columnwidth]{fig/sst2/sst2-var.pdf}%\vspace{-2em}
    \caption{Scaling for $\pgr$ and $\opr$ of W2S on SST-2 with injected label noise.}\label{fig:sst2_var}
\end{figure}

\paragraph{Correlation Dimension.} 
Let $D_s, D_w \in \N$ be the finetunable parameter counts of the strong and weak models, respectively. For full FT whose dynamics fall in the kernel regime, as explained in \Cref{rmk:lp_to_general_ft}, the strong and weak ``features'' become the gradients\footnote{
    Notice that $f_s, f_w$ are scalar-valued functions for binary classification tasks like SST-2, and thus the gradients $\nabla_{\thetab} f_s$ and $\nabla_{\thetab} f_w$ are row vectors.
    For multi-class classification tasks where $f_s, f_w$ output vectors of logits, a common heuristic to keep $\Phib_s, \Phib_w$ as matrices of manageable sizes (in constrast to tensors) is to replace gradients of the models, $\nabla_{\thetab} f_s$ and $\nabla_{\thetab} f_w$, with gradients of MSE losses at the pretrained initialization. 
    The gradients of MSE can be viewed as a weighted sum of the model gradients for each class.
}, $\Phib_s = \nabla_{\thetab} f_s(\Xb | \theta_s^{(0)}) \in \R^{N \times D_s}$ and $\Phib_w = \nabla_{\thetab} f_w(\Xb | \theta_w^{(0)}) \in \R^{N \times D_w}$, of the respective models at the pretrained initialization, $\theta_s^{(0)} \in \R^{D_s}$ and $\theta_w^{(0)} \in \R^{D_w}$.

A practical challenge is that $D_s, D_w, N$ are all huge for full FT on most NLP tasks, making it infeasible to compute the $D_s \times D_s$ and $D_w \times D_w$ Gram matrices and their spectral decompositions. 
As a remedy, we leverage the significantly lower intrinsic dimensions $d_s \ll D_s, d_w \ll D_w$ (see \Cref{tab:img_cls_dim_coloredmnist}) to accelerate estimation of $d_{s \wedge w}$ via sketching~\citep{halko2011finding,woodruff2014sketching}.
\begin{enumerate}[label=(\roman*)]
    \item We first reduce both $D_s, D_w$ to the same lower dimension $D = 0.01 \min\{D_s, D_w\}$ (with $D \gg \max\{d_s, d_w\}$) by uniform subsampling columns of $\Phib_s, \Phib_w$ to obtain $\Phib_s', \Phib_w' \in \R^{N \times D}$.
    \item Then, we use randomized rangefinder~\citep[Algorithm 4.1]{halko2011finding} to approximate the first $d_s, d_w$ right singular vectors, $\Vb_s \in \R^{D \times d_s}$ and $\Vb_w \in \R^{D \times d_w}$, of $\Phib_s', \Phib_w'$. Taking the evaluation of $\Vb_s$ as an example, we draw a Gaussian random matrix $\Gb_s \in \R^{d_s \times D}$ and compute the orthornormalization $\Vb_s = \ortho(\Phib_s'^\top \Gb_s)$ via QR decomposition.
    \item Finally, we compute the correlation dimension $d_{s \wedge w} = \nbr{\Vb_s^\top \Vb_w}_F^2$.
\end{enumerate}

\begin{table}[!ht]
    \centering
    \caption{Summary of finetunable parameter counts $D_s, D_w$, intrinsic dimensions $d_s, d_w$, and correlation dimensions $d_{s \wedge w}$ (with respect to the strong student \texttt{Electra}) computed over the entire SST-2 dataset (including training and testing).}\label{tab:sst2_dim}
    \begin{tabular}{c|ccc}
        \toprule
        Pretrained Model & $D_s,D_w$ & Intrinsic Dimension ($\tau=0.01$) & Correlation Dimension \\
        \midrule
        \texttt{Bert-Tiny} & 4.4M & 7000 & 81.13 \\
        \texttt{Bert-Mini} & 11.2M & 8500 & 38.67 \\
        \texttt{Bert-Small} & 28.8M & 8000 & 26.19 \\
        \texttt{Bert-Medium} & 41.4M & 4000 & 8.52 \\
        \midrule
        \texttt{Electra} & 109.5M & 700 & $\times$ \\
        \bottomrule
    \end{tabular}
\end{table}

\paragraph{Discrepancies lead to better W2S.}
\Cref{fig:sst2_dsw} shows the scaling of $\pgr$ and $\opr$ with respect to $n$ and $N$ for different $d_{s \wedge w}$. 
As in \Cref{sec:exp_img_reg,apx:exp_img_cls}, we observe the better relative W2S performance in terms of $\pgr$ and $\opr$ when $d_{s \wedge w}/d_w$ is smaller.

\paragraph{Variance reduction is a key advantage of W2S.}
We inject noise to the labels of training samples by randomly flipping labels with probability $\varsigma = 0, 0.1, 0.2, 0.3$. 
\Cref{fig:sst2_var} shows the scaling of $\pgr$ and $\opr$ with respect to $n$ and $N$ when taking \texttt{Electra} as the strong student and \texttt{Bert-Medium} as the weak teacher. We observe that the larger artificial label noise $\varsigma$ leads to higher $\pgr$ and $\opr$. 

\end{document}
\endinput
%%
%% End of file `sample-sigconf.tex'.
