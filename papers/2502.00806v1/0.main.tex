%%
%% This is file `sample-sigconf.tex',
%% generated with the docstrip utility.
%%
%% The original source files were:
%%
%% samples.dtx  (with options: `all,proceedings,bibtex,sigconf')
%% 
%% IMPORTANT NOTICE:
%% 
%% For the copyright see the source file.
%% 
%% Any modified versions of this file must be renamed
%% with new filenames distinct from sample-sigconf.tex.
%% 
%% For distribution of the original source see the terms
%% for copying and modification in the file samples.dtx.
%% 
%% This generated file may be distributed as long as the
%% original source files, as listed above, are part of the
%% same distribution. (The sources need not necessarily be
%% in the same archive or directory.)
%%
%%
%% Commands for TeXCount
%TC:macro \cite [option:text,text]
%TC:macro \citep [option:text,text]
%TC:macro \citet [option:text,text]
%TC:envir table 0 1
%TC:envir table* 0 1
%TC:envir tabular [ignore] word
%TC:envir displaymath 0 word
%TC:envir math 0 word
%TC:envir comment 0 0
%%
%%
%% The first command in your LaTeX source must be the \documentclass
%% command.
%%
%% For submission and review of your manuscript please change the
%% command to \documentclass[manuscript, screen, review]{acmart}.
%%
%% When submitting camera ready or to TAPS, please change the command
%% to \documentclass[sigconf]{acmart} or whichever template is required
%% for your publication.
%%
%%
\documentclass[sigconf]{acmart}

%%
%% \BibTeX command to typeset BibTeX logo in the docs
\AtBeginDocument{%
  \providecommand\BibTeX{{%
    Bib\TeX}}}

%% Rights management information.  This information is sent to you
%% when you complete the rights form.  These commands have SAMPLE
%% values in them; it is your responsibility as an author to replace
%% the commands and values with those provided to you when you
%% complete the rights form.
\setcopyright{acmlicensed}
\copyrightyear{2025}
\acmYear{2025}
\setcopyright{cc}
\setcctype{by}
\acmConference[WWW '25]{Proceedings of the ACM Web Conference 2025}{April 28-May 2, 2025}{Sydney, NSW, Australia}
\acmBooktitle{Proceedings of the ACM Web Conference 2025 (WWW '25), April 28-May 2, 2025, Sydney, NSW, Australia}
\acmDOI{10.1145/3696410.3714818}
\acmISBN{979-8-4007-1274-6/25/04}

\settopmatter{printacmref=true}

%% These commands are for a PROCEEDINGS abstract or paper.
% \acmConference[Conference acronym 'XX]{Make sure to enter the correct
%   conference title from your rights confirmation emai}{June 03--05,
%   2018}{Woodstock, NY}
%%
%%  Uncomment \acmBooktitle if the title of the proceedings is different
%%  from ``Proceedings of ...''!
%%
%%\acmBooktitle{Woodstock '18: ACM Symposium on Neural Gaze Detection,
%%  June 03--05, 2018, Woodstock, NY}
% \acmISBN{978-1-4503-XXXX-X/18/06}


%%
%% Submission ID.
%% Use this when submitting an article to a sponsored event. You'll
%% receive a unique submission ID from the organizers
%% of the event, and this ID should be used as the parameter to this command.
%%\acmSubmissionID{123-A56-BU3}

%%
%% For managing citations, it is recommended to use bibliography
%% files in BibTeX format.
%%
%% You can then either use BibTeX with the ACM-Reference-Format style,
%% or BibLaTeX with the acmnumeric or acmauthoryear sytles, that include
%% support for advanced citation of software artefact from the
%% biblatex-software package, also separately available on CTAN.
%%
%% Look at the sample-*-biblatex.tex files for templates showcasing
%% the biblatex styles.
%%

%%
%% The majority of ACM publications use numbered citations and
%% references.  The command \citestyle{authoryear} switches to the
%% "author year" style.
%%
%% If you are preparing content for an event
%% sponsored by ACM SIGGRAPH, you must use the "author year" style of
%% citations and references.
%% Uncommenting
%% the next command will enable that style.
%%\citestyle{acmauthoryear}
\usepackage{enumitem}

\usepackage{amsmath}
\newtheorem{problem}{Problem}
\newtheorem{definition}{Definition}

\usepackage{xspace}
\newcommand{\model}{UniGraph2\xspace}
\newcommand{\vpara}[1]{\vspace{0.04in}\noindent\textbf{#1}\xspace}

\newcommand\todo[1]{\textcolor{red}{[TODO: #1 ]}}

% add background color to table
\usepackage{color}
\usepackage{colortbl}
\definecolor{Gray}{gray}{0.9}

\usepackage{multirow}

\usepackage{arydshln}


%%%%% NEW MATH DEFINITIONS %%%%%

\usepackage{amsmath,amsfonts,bm}
\usepackage{derivative}
% Mark sections of captions for referring to divisions of figures
\newcommand{\figleft}{{\em (Left)}}
\newcommand{\figcenter}{{\em (Center)}}
\newcommand{\figright}{{\em (Right)}}
\newcommand{\figtop}{{\em (Top)}}
\newcommand{\figbottom}{{\em (Bottom)}}
\newcommand{\captiona}{{\em (a)}}
\newcommand{\captionb}{{\em (b)}}
\newcommand{\captionc}{{\em (c)}}
\newcommand{\captiond}{{\em (d)}}

% Highlight a newly defined term
\newcommand{\newterm}[1]{{\bf #1}}

% Derivative d 
\newcommand{\deriv}{{\mathrm{d}}}

% Figure reference, lower-case.
\def\figref#1{figure~\ref{#1}}
% Figure reference, capital. For start of sentence
\def\Figref#1{Figure~\ref{#1}}
\def\twofigref#1#2{figures \ref{#1} and \ref{#2}}
\def\quadfigref#1#2#3#4{figures \ref{#1}, \ref{#2}, \ref{#3} and \ref{#4}}
% Section reference, lower-case.
\def\secref#1{section~\ref{#1}}
% Section reference, capital.
\def\Secref#1{Section~\ref{#1}}
% Reference to two sections.
\def\twosecrefs#1#2{sections \ref{#1} and \ref{#2}}
% Reference to three sections.
\def\secrefs#1#2#3{sections \ref{#1}, \ref{#2} and \ref{#3}}
% Reference to an equation, lower-case.
\def\eqref#1{equation~\ref{#1}}
% Reference to an equation, upper case
\def\Eqref#1{Equation~\ref{#1}}
% A raw reference to an equation---avoid using if possible
\def\plaineqref#1{\ref{#1}}
% Reference to a chapter, lower-case.
\def\chapref#1{chapter~\ref{#1}}
% Reference to an equation, upper case.
\def\Chapref#1{Chapter~\ref{#1}}
% Reference to a range of chapters
\def\rangechapref#1#2{chapters\ref{#1}--\ref{#2}}
% Reference to an algorithm, lower-case.
\def\algref#1{algorithm~\ref{#1}}
% Reference to an algorithm, upper case.
\def\Algref#1{Algorithm~\ref{#1}}
\def\twoalgref#1#2{algorithms \ref{#1} and \ref{#2}}
\def\Twoalgref#1#2{Algorithms \ref{#1} and \ref{#2}}
% Reference to a part, lower case
\def\partref#1{part~\ref{#1}}
% Reference to a part, upper case
\def\Partref#1{Part~\ref{#1}}
\def\twopartref#1#2{parts \ref{#1} and \ref{#2}}

\def\ceil#1{\lceil #1 \rceil}
\def\floor#1{\lfloor #1 \rfloor}
\def\1{\bm{1}}
\newcommand{\train}{\mathcal{D}}
\newcommand{\valid}{\mathcal{D_{\mathrm{valid}}}}
\newcommand{\test}{\mathcal{D_{\mathrm{test}}}}

\def\eps{{\epsilon}}


% Random variables
\def\reta{{\textnormal{$\eta$}}}
\def\ra{{\textnormal{a}}}
\def\rb{{\textnormal{b}}}
\def\rc{{\textnormal{c}}}
\def\rd{{\textnormal{d}}}
\def\re{{\textnormal{e}}}
\def\rf{{\textnormal{f}}}
\def\rg{{\textnormal{g}}}
\def\rh{{\textnormal{h}}}
\def\ri{{\textnormal{i}}}
\def\rj{{\textnormal{j}}}
\def\rk{{\textnormal{k}}}
\def\rl{{\textnormal{l}}}
% rm is already a command, just don't name any random variables m
\def\rn{{\textnormal{n}}}
\def\ro{{\textnormal{o}}}
\def\rp{{\textnormal{p}}}
\def\rq{{\textnormal{q}}}
\def\rr{{\textnormal{r}}}
\def\rs{{\textnormal{s}}}
\def\rt{{\textnormal{t}}}
\def\ru{{\textnormal{u}}}
\def\rv{{\textnormal{v}}}
\def\rw{{\textnormal{w}}}
\def\rx{{\textnormal{x}}}
\def\ry{{\textnormal{y}}}
\def\rz{{\textnormal{z}}}

% Random vectors
\def\rvepsilon{{\mathbf{\epsilon}}}
\def\rvphi{{\mathbf{\phi}}}
\def\rvtheta{{\mathbf{\theta}}}
\def\rva{{\mathbf{a}}}
\def\rvb{{\mathbf{b}}}
\def\rvc{{\mathbf{c}}}
\def\rvd{{\mathbf{d}}}
\def\rve{{\mathbf{e}}}
\def\rvf{{\mathbf{f}}}
\def\rvg{{\mathbf{g}}}
\def\rvh{{\mathbf{h}}}
\def\rvu{{\mathbf{i}}}
\def\rvj{{\mathbf{j}}}
\def\rvk{{\mathbf{k}}}
\def\rvl{{\mathbf{l}}}
\def\rvm{{\mathbf{m}}}
\def\rvn{{\mathbf{n}}}
\def\rvo{{\mathbf{o}}}
\def\rvp{{\mathbf{p}}}
\def\rvq{{\mathbf{q}}}
\def\rvr{{\mathbf{r}}}
\def\rvs{{\mathbf{s}}}
\def\rvt{{\mathbf{t}}}
\def\rvu{{\mathbf{u}}}
\def\rvv{{\mathbf{v}}}
\def\rvw{{\mathbf{w}}}
\def\rvx{{\mathbf{x}}}
\def\rvy{{\mathbf{y}}}
\def\rvz{{\mathbf{z}}}

% Elements of random vectors
\def\erva{{\textnormal{a}}}
\def\ervb{{\textnormal{b}}}
\def\ervc{{\textnormal{c}}}
\def\ervd{{\textnormal{d}}}
\def\erve{{\textnormal{e}}}
\def\ervf{{\textnormal{f}}}
\def\ervg{{\textnormal{g}}}
\def\ervh{{\textnormal{h}}}
\def\ervi{{\textnormal{i}}}
\def\ervj{{\textnormal{j}}}
\def\ervk{{\textnormal{k}}}
\def\ervl{{\textnormal{l}}}
\def\ervm{{\textnormal{m}}}
\def\ervn{{\textnormal{n}}}
\def\ervo{{\textnormal{o}}}
\def\ervp{{\textnormal{p}}}
\def\ervq{{\textnormal{q}}}
\def\ervr{{\textnormal{r}}}
\def\ervs{{\textnormal{s}}}
\def\ervt{{\textnormal{t}}}
\def\ervu{{\textnormal{u}}}
\def\ervv{{\textnormal{v}}}
\def\ervw{{\textnormal{w}}}
\def\ervx{{\textnormal{x}}}
\def\ervy{{\textnormal{y}}}
\def\ervz{{\textnormal{z}}}

% Random matrices
\def\rmA{{\mathbf{A}}}
\def\rmB{{\mathbf{B}}}
\def\rmC{{\mathbf{C}}}
\def\rmD{{\mathbf{D}}}
\def\rmE{{\mathbf{E}}}
\def\rmF{{\mathbf{F}}}
\def\rmG{{\mathbf{G}}}
\def\rmH{{\mathbf{H}}}
\def\rmI{{\mathbf{I}}}
\def\rmJ{{\mathbf{J}}}
\def\rmK{{\mathbf{K}}}
\def\rmL{{\mathbf{L}}}
\def\rmM{{\mathbf{M}}}
\def\rmN{{\mathbf{N}}}
\def\rmO{{\mathbf{O}}}
\def\rmP{{\mathbf{P}}}
\def\rmQ{{\mathbf{Q}}}
\def\rmR{{\mathbf{R}}}
\def\rmS{{\mathbf{S}}}
\def\rmT{{\mathbf{T}}}
\def\rmU{{\mathbf{U}}}
\def\rmV{{\mathbf{V}}}
\def\rmW{{\mathbf{W}}}
\def\rmX{{\mathbf{X}}}
\def\rmY{{\mathbf{Y}}}
\def\rmZ{{\mathbf{Z}}}

% Elements of random matrices
\def\ermA{{\textnormal{A}}}
\def\ermB{{\textnormal{B}}}
\def\ermC{{\textnormal{C}}}
\def\ermD{{\textnormal{D}}}
\def\ermE{{\textnormal{E}}}
\def\ermF{{\textnormal{F}}}
\def\ermG{{\textnormal{G}}}
\def\ermH{{\textnormal{H}}}
\def\ermI{{\textnormal{I}}}
\def\ermJ{{\textnormal{J}}}
\def\ermK{{\textnormal{K}}}
\def\ermL{{\textnormal{L}}}
\def\ermM{{\textnormal{M}}}
\def\ermN{{\textnormal{N}}}
\def\ermO{{\textnormal{O}}}
\def\ermP{{\textnormal{P}}}
\def\ermQ{{\textnormal{Q}}}
\def\ermR{{\textnormal{R}}}
\def\ermS{{\textnormal{S}}}
\def\ermT{{\textnormal{T}}}
\def\ermU{{\textnormal{U}}}
\def\ermV{{\textnormal{V}}}
\def\ermW{{\textnormal{W}}}
\def\ermX{{\textnormal{X}}}
\def\ermY{{\textnormal{Y}}}
\def\ermZ{{\textnormal{Z}}}

% Vectors
\def\vzero{{\bm{0}}}
\def\vone{{\bm{1}}}
\def\vmu{{\bm{\mu}}}
\def\vtheta{{\bm{\theta}}}
\def\vphi{{\bm{\phi}}}
\def\va{{\bm{a}}}
\def\vb{{\bm{b}}}
\def\vc{{\bm{c}}}
\def\vd{{\bm{d}}}
\def\ve{{\bm{e}}}
\def\vf{{\bm{f}}}
\def\vg{{\bm{g}}}
\def\vh{{\bm{h}}}
\def\vi{{\bm{i}}}
\def\vj{{\bm{j}}}
\def\vk{{\bm{k}}}
\def\vl{{\bm{l}}}
\def\vm{{\bm{m}}}
\def\vn{{\bm{n}}}
\def\vo{{\bm{o}}}
\def\vp{{\bm{p}}}
\def\vq{{\bm{q}}}
\def\vr{{\bm{r}}}
\def\vs{{\bm{s}}}
\def\vt{{\bm{t}}}
\def\vu{{\bm{u}}}
\def\vv{{\bm{v}}}
\def\vw{{\bm{w}}}
\def\vx{{\bm{x}}}
\def\vy{{\bm{y}}}
\def\vz{{\bm{z}}}

% Elements of vectors
\def\evalpha{{\alpha}}
\def\evbeta{{\beta}}
\def\evepsilon{{\epsilon}}
\def\evlambda{{\lambda}}
\def\evomega{{\omega}}
\def\evmu{{\mu}}
\def\evpsi{{\psi}}
\def\evsigma{{\sigma}}
\def\evtheta{{\theta}}
\def\eva{{a}}
\def\evb{{b}}
\def\evc{{c}}
\def\evd{{d}}
\def\eve{{e}}
\def\evf{{f}}
\def\evg{{g}}
\def\evh{{h}}
\def\evi{{i}}
\def\evj{{j}}
\def\evk{{k}}
\def\evl{{l}}
\def\evm{{m}}
\def\evn{{n}}
\def\evo{{o}}
\def\evp{{p}}
\def\evq{{q}}
\def\evr{{r}}
\def\evs{{s}}
\def\evt{{t}}
\def\evu{{u}}
\def\evv{{v}}
\def\evw{{w}}
\def\evx{{x}}
\def\evy{{y}}
\def\evz{{z}}

% Matrix
\def\mA{{\bm{A}}}
\def\mB{{\bm{B}}}
\def\mC{{\bm{C}}}
\def\mD{{\bm{D}}}
\def\mE{{\bm{E}}}
\def\mF{{\bm{F}}}
\def\mG{{\bm{G}}}
\def\mH{{\bm{H}}}
\def\mI{{\bm{I}}}
\def\mJ{{\bm{J}}}
\def\mK{{\bm{K}}}
\def\mL{{\bm{L}}}
\def\mM{{\bm{M}}}
\def\mN{{\bm{N}}}
\def\mO{{\bm{O}}}
\def\mP{{\bm{P}}}
\def\mQ{{\bm{Q}}}
\def\mR{{\bm{R}}}
\def\mS{{\bm{S}}}
\def\mT{{\bm{T}}}
\def\mU{{\bm{U}}}
\def\mV{{\bm{V}}}
\def\mW{{\bm{W}}}
\def\mX{{\bm{X}}}
\def\mY{{\bm{Y}}}
\def\mZ{{\bm{Z}}}
\def\mBeta{{\bm{\beta}}}
\def\mPhi{{\bm{\Phi}}}
\def\mLambda{{\bm{\Lambda}}}
\def\mSigma{{\bm{\Sigma}}}

% Tensor
\DeclareMathAlphabet{\mathsfit}{\encodingdefault}{\sfdefault}{m}{sl}
\SetMathAlphabet{\mathsfit}{bold}{\encodingdefault}{\sfdefault}{bx}{n}
\newcommand{\tens}[1]{\bm{\mathsfit{#1}}}
\def\tA{{\tens{A}}}
\def\tB{{\tens{B}}}
\def\tC{{\tens{C}}}
\def\tD{{\tens{D}}}
\def\tE{{\tens{E}}}
\def\tF{{\tens{F}}}
\def\tG{{\tens{G}}}
\def\tH{{\tens{H}}}
\def\tI{{\tens{I}}}
\def\tJ{{\tens{J}}}
\def\tK{{\tens{K}}}
\def\tL{{\tens{L}}}
\def\tM{{\tens{M}}}
\def\tN{{\tens{N}}}
\def\tO{{\tens{O}}}
\def\tP{{\tens{P}}}
\def\tQ{{\tens{Q}}}
\def\tR{{\tens{R}}}
\def\tS{{\tens{S}}}
\def\tT{{\tens{T}}}
\def\tU{{\tens{U}}}
\def\tV{{\tens{V}}}
\def\tW{{\tens{W}}}
\def\tX{{\tens{X}}}
\def\tY{{\tens{Y}}}
\def\tZ{{\tens{Z}}}


% Graph
\def\gA{{\mathcal{A}}}
\def\gB{{\mathcal{B}}}
\def\gC{{\mathcal{C}}}
\def\gD{{\mathcal{D}}}
\def\gE{{\mathcal{E}}}
\def\gF{{\mathcal{F}}}
\def\gG{{\mathcal{G}}}
\def\gH{{\mathcal{H}}}
\def\gI{{\mathcal{I}}}
\def\gJ{{\mathcal{J}}}
\def\gK{{\mathcal{K}}}
\def\gL{{\mathcal{L}}}
\def\gM{{\mathcal{M}}}
\def\gN{{\mathcal{N}}}
\def\gO{{\mathcal{O}}}
\def\gP{{\mathcal{P}}}
\def\gQ{{\mathcal{Q}}}
\def\gR{{\mathcal{R}}}
\def\gS{{\mathcal{S}}}
\def\gT{{\mathcal{T}}}
\def\gU{{\mathcal{U}}}
\def\gV{{\mathcal{V}}}
\def\gW{{\mathcal{W}}}
\def\gX{{\mathcal{X}}}
\def\gY{{\mathcal{Y}}}
\def\gZ{{\mathcal{Z}}}

% Sets
\def\sA{{\mathbb{A}}}
\def\sB{{\mathbb{B}}}
\def\sC{{\mathbb{C}}}
\def\sD{{\mathbb{D}}}
% Don't use a set called E, because this would be the same as our symbol
% for expectation.
\def\sF{{\mathbb{F}}}
\def\sG{{\mathbb{G}}}
\def\sH{{\mathbb{H}}}
\def\sI{{\mathbb{I}}}
\def\sJ{{\mathbb{J}}}
\def\sK{{\mathbb{K}}}
\def\sL{{\mathbb{L}}}
\def\sM{{\mathbb{M}}}
\def\sN{{\mathbb{N}}}
\def\sO{{\mathbb{O}}}
\def\sP{{\mathbb{P}}}
\def\sQ{{\mathbb{Q}}}
\def\sR{{\mathbb{R}}}
\def\sS{{\mathbb{S}}}
\def\sT{{\mathbb{T}}}
\def\sU{{\mathbb{U}}}
\def\sV{{\mathbb{V}}}
\def\sW{{\mathbb{W}}}
\def\sX{{\mathbb{X}}}
\def\sY{{\mathbb{Y}}}
\def\sZ{{\mathbb{Z}}}

% Entries of a matrix
\def\emLambda{{\Lambda}}
\def\emA{{A}}
\def\emB{{B}}
\def\emC{{C}}
\def\emD{{D}}
\def\emE{{E}}
\def\emF{{F}}
\def\emG{{G}}
\def\emH{{H}}
\def\emI{{I}}
\def\emJ{{J}}
\def\emK{{K}}
\def\emL{{L}}
\def\emM{{M}}
\def\emN{{N}}
\def\emO{{O}}
\def\emP{{P}}
\def\emQ{{Q}}
\def\emR{{R}}
\def\emS{{S}}
\def\emT{{T}}
\def\emU{{U}}
\def\emV{{V}}
\def\emW{{W}}
\def\emX{{X}}
\def\emY{{Y}}
\def\emZ{{Z}}
\def\emSigma{{\Sigma}}

% entries of a tensor
% Same font as tensor, without \bm wrapper
\newcommand{\etens}[1]{\mathsfit{#1}}
\def\etLambda{{\etens{\Lambda}}}
\def\etA{{\etens{A}}}
\def\etB{{\etens{B}}}
\def\etC{{\etens{C}}}
\def\etD{{\etens{D}}}
\def\etE{{\etens{E}}}
\def\etF{{\etens{F}}}
\def\etG{{\etens{G}}}
\def\etH{{\etens{H}}}
\def\etI{{\etens{I}}}
\def\etJ{{\etens{J}}}
\def\etK{{\etens{K}}}
\def\etL{{\etens{L}}}
\def\etM{{\etens{M}}}
\def\etN{{\etens{N}}}
\def\etO{{\etens{O}}}
\def\etP{{\etens{P}}}
\def\etQ{{\etens{Q}}}
\def\etR{{\etens{R}}}
\def\etS{{\etens{S}}}
\def\etT{{\etens{T}}}
\def\etU{{\etens{U}}}
\def\etV{{\etens{V}}}
\def\etW{{\etens{W}}}
\def\etX{{\etens{X}}}
\def\etY{{\etens{Y}}}
\def\etZ{{\etens{Z}}}

% The true underlying data generating distribution
\newcommand{\pdata}{p_{\rm{data}}}
\newcommand{\ptarget}{p_{\rm{target}}}
\newcommand{\pprior}{p_{\rm{prior}}}
\newcommand{\pbase}{p_{\rm{base}}}
\newcommand{\pref}{p_{\rm{ref}}}

% The empirical distribution defined by the training set
\newcommand{\ptrain}{\hat{p}_{\rm{data}}}
\newcommand{\Ptrain}{\hat{P}_{\rm{data}}}
% The model distribution
\newcommand{\pmodel}{p_{\rm{model}}}
\newcommand{\Pmodel}{P_{\rm{model}}}
\newcommand{\ptildemodel}{\tilde{p}_{\rm{model}}}
% Stochastic autoencoder distributions
\newcommand{\pencode}{p_{\rm{encoder}}}
\newcommand{\pdecode}{p_{\rm{decoder}}}
\newcommand{\precons}{p_{\rm{reconstruct}}}

\newcommand{\laplace}{\mathrm{Laplace}} % Laplace distribution

\newcommand{\E}{\mathbb{E}}
\newcommand{\Ls}{\mathcal{L}}
\newcommand{\R}{\mathbb{R}}
\newcommand{\emp}{\tilde{p}}
\newcommand{\lr}{\alpha}
\newcommand{\reg}{\lambda}
\newcommand{\rect}{\mathrm{rectifier}}
\newcommand{\softmax}{\mathrm{softmax}}
\newcommand{\sigmoid}{\sigma}
\newcommand{\softplus}{\zeta}
\newcommand{\KL}{D_{\mathrm{KL}}}
\newcommand{\Var}{\mathrm{Var}}
\newcommand{\standarderror}{\mathrm{SE}}
\newcommand{\Cov}{\mathrm{Cov}}
% Wolfram Mathworld says $L^2$ is for function spaces and $\ell^2$ is for vectors
% But then they seem to use $L^2$ for vectors throughout the site, and so does
% wikipedia.
\newcommand{\normlzero}{L^0}
\newcommand{\normlone}{L^1}
\newcommand{\normltwo}{L^2}
\newcommand{\normlp}{L^p}
\newcommand{\normmax}{L^\infty}

\newcommand{\parents}{Pa} % See usage in notation.tex. Chosen to match Daphne's book.

\DeclareMathOperator*{\argmax}{arg\,max}
\DeclareMathOperator*{\argmin}{arg\,min}

\DeclareMathOperator{\sign}{sign}
\DeclareMathOperator{\Tr}{Tr}
\let\ab\allowbreak


%%
%% end of the preamble, start of the body of the document source.
\begin{document}

%%
%% The "title" command has an optional parameter,
%% allowing the author to define a "short title" to be used in page headers.
\title{\model: Learning a Unified Embedding Space to Bind Multimodal Graphs}

\author{Yufei He}
\affiliation{National University of Singapore\country{Singapore}}
\email{yufei.he@u.nus.edu}

\author{Yuan Sui}
\affiliation{National University of Singapore\country{Singapore}}
\email{yuan.sui@u.nus.edu}

\author{Xiaoxin He}
\affiliation{National University of Singapore\country{Singapore}}
\email{he.xiaoxin@u.nus.edu}

\author{Yue Liu}
\affiliation{National University of Singapore\country{Singapore}}
\email{yliu@u.nus.edu}

\author{Yifei Sun}
\affiliation{Zhejiang University\country{China}}
\email{yifeisun@zju.edu.cn}

\author{Bryan Hooi}
\affiliation{National University of Singapore\country{Singapore}}
\email{bhooi@comp.nus.edu.sg}

%%
%% The "author" command and its associated commands are used to define
%% the authors and their affiliations.
%% Of note is the shared affiliation of the first two authors, and the
%% "authornote" and "authornotemark" commands
%% used to denote shared contribution to the research.


%%
%% By default, the full list of authors will be used in the page
%% headers. Often, this list is too long, and will overlap
%% other information printed in the page headers. This command allows
%% the author to define a more concise list
%% of authors' names for this purpose.
% \renewcommand{\shortauthors}{Trovato et al.}
\renewcommand{\shortauthors}{Yufei He et al.}
%%
%% The abstract is a short summary of the work to be presented in the
%% article.
\begin{abstract}

As large language models (LLMs) continue to evolve, ensuring their alignment with human goals and values remains a pressing challenge. A key concern is \textit{instrumental convergence}, where an AI system, in optimizing for a given objective, develops unintended intermediate goals that override the ultimate objective and deviate from human-intended goals. This issue is particularly relevant in reinforcement learning (RL)-trained models, which can generate creative but unintended strategies to maximize rewards. In this paper, we explore instrumental convergence in LLMs by comparing models trained with direct RL optimization (e.g., the o1 model) to those trained with reinforcement learning from human feedback (RLHF). We hypothesize that RL-driven models exhibit a stronger tendency for instrumental convergence due to their optimization of goal-directed behavior in ways that may misalign with human intentions. To assess this, we introduce InstrumentalEval\footnote[1]{The code is available at \url{https://github.com/yf-he/InstrumentalEval}}, a benchmark for evaluating instrumental convergence in RL-trained LLMs. Initial experiments reveal cases where a model tasked with making money unexpectedly pursues instrumental objectives, such as self-replication, implying signs of instrumental convergence. Our findings contribute to a deeper understanding of alignment challenges in AI systems and the risks posed by unintended model behaviors.


\end{abstract}
%%
%% The code below is generated by the tool at http://dl.acm.org/ccs.cfm.
%% Please copy and paste the code instead of the example below.
%%
% \vspace{-60mm}
\begin{CCSXML}
<ccs2012>
   <concept>
       <concept_id>10002951.10003227.10003351</concept_id>
       <concept_desc>Information systems~Data mining</concept_desc>
       <concept_significance>500</concept_significance>
       </concept>
   <concept>
       <concept_id>10010147.10010257.10010293.10010294</concept_id>
       <concept_desc>Computing methodologies~Neural networks</concept_desc>
       <concept_significance>500</concept_significance>
       </concept>
   <concept>
       <concept_id>10002951.10003260.10003282.10003292</concept_id>
       <concept_desc>Information systems~Social networks</concept_desc>
       <concept_significance>300</concept_significance>
       </concept>
 </ccs2012>
\end{CCSXML}

\ccsdesc[500]{Information systems~Data mining}
\ccsdesc[500]{Computing methodologies~Neural networks}
\ccsdesc[300]{Information systems~Social networks}

% \begin{figure}
%     \centering
%     \includegraphics[width=1.0\linewidth]{figs/fig1.pdf}
%     \vskip -0.1in
%     \caption{\model binds multimodal graphs from different graph domains to a unified embedding space, enabling diverse downstream tasks.}
%     \label{fig:fig1}
%     \vspace{-6mm}
% \end{figure}

%%
%% Keywords. The author(s) should pick words that accurately describe
%% the work being presented. Separate the keywords with commas.
\keywords{Pre-Training; Graph Foundation Models; Multimodal Learning}


%%
%% This command processes the author and affiliation and title
%% information and builds the first part of the formatted document.
\maketitle
% \vspace{-4mm}
% \vpara{Relevance to the Web and to the track.}
% UniGraph2 is a large pre-trained model that processes and unifies multimodal graph data from various web domains, making it applicable to web-based applications like search, recommendation, and content classification.
% UniGraph2 is a large pre-trained model designed to process and unify multimodal graph data derived from various web domains, making it directly relevant to web-based applications such as search, recommendation, and content classification. 
% Its ability to generalize across diverse web data sources and modalities aligns it with the goals of Web mining and content analysis.



\section{Introduction}
\IEEEPARstart{I}{n} recent years, flourishing of Artificial Intelligence Generated Content (AIGC) has sparked significant advancements in modalities such as text, image, audio, and even video. 
Among these, AI-Generated Image (AGI) has garnered considerable interest from both researchers and the public.
Plenty of remarkable AGI models and online services, such as StableDiffusion\footnote{\url{https://stability.ai/}}, Midjourney\footnote{\url{https://www.midjourney.com/}}, and FLUX\footnote{\url{https://blackforestlabs.ai/}}, offer users an excellent creative experience.
However, users often remain critical of the quality of the AGI due to image distortions or mismatches with user intentions.
Consequently, methods for assessing the quality of AGI are becoming increasingly crucial to help improve the generative capabilities of these models.

Unlike Natural Scene Image (NSI) quality assessment, which focuses primarily on perception aspects such as sharpness, color, and brightness, AI-Generated Image Quality Assessment (AGIQA) encompasses additional aspects like correspondence and authenticity. 
Since AGI is generated on the basis of user text prompts, it may fail to capture key user intentions, resulting in misalignment with the prompt.
Furthermore, authenticity refers to how closely the generated image resembles real-world artworks, as AGI can sometimes exhibit logical inconsistencies.
While traditional IQA models may effectively evaluate perceptual quality, they are often less capable of adequately assessing aspects such as correspondence and authenticity.

\begin{figure}\label{fig:radar}
    \centering
    \includegraphics[width=1.0\linewidth]{figures/radar_plot.pdf}
    \caption{A comparison on quality, correspondence, and authenticity aspects of AIGCIQA2023~\cite{wang2023aigciqa2023} dataset illustrates the superior performance of our method.}
\end{figure}

Several methods have been proposed specifically for the AGIQA task, including metrics designed to evaluate the authenticity and diversity of generated images~\cite{gulrajani2017improved,heusel2017gans}. 
Nevertheless, these methods tend to compare and evaluate grouped images rather than single instances, which limits their utility for single image assessment.
Beginning with AGIQA-1k~\cite{zhang2023perceptual}, a series of AGIQA databases have been introduced, including AGIQA-3k~\cite{li2023agiqa}, AIGCIQA-20k~\cite{li2024aigiqa}, etc.
Concurrently, there has been a surge in research utilizing deep learning methods~\cite{zhou2024adaptive,peng2024aigc,yu2024sf}, which have significantly benefited from pre-trained models such as CLIP~\cite{radford2021learning}. 
These approaches enhance the analysis by leveraging the correlations between images and their descriptive texts.
While these models are effective in capturing general text-image alignments, they may not effectively detect subtle inconsistencies or mismatches between the generated image content and the detailed nuances of the textual description.
Moreover, as these models are pre-trained on large-scale datasets for broad tasks, they might not fully exploit the textual information pertinent to the specific context of AGIQA without task-specific fine-tuning.
To overcome these limitations, methods that leverage Multimodal Large Language Models (MLLMs)~\cite{wang2024large,wang2024understanding} have been proposed.
These methods aim to fully exploit the synergies of image captioning and textual analysis for AGIQA.
Although they benefit from advanced prompt understanding, instruction following, and generation capabilities, they often do not utilize MLLMs as encoders capable of producing a sequence of logits that integrate both image and text context.

In conclusion, the field of AI-Generated Image Quality Assessment (AGIQA) continues to face significant challenges: 
(1) Developing comprehensive methods to assess AGIs from multiple dimensions, including quality, correspondence, and authenticity; 
(2) Enhancing assessment techniques to more accurately reflect human perception and the nuanced intentions embedded within prompts; 
(3) Optimizing the use of Multimodal Large Language Models (MLLMs) to fully exploit their multimodal encoding capabilities.

To address these challenges, we propose a novel method M3-AGIQA (\textbf{M}ultimodal, \textbf{M}ulti-Round, \textbf{M}ulti-Aspect AI-Generated Image Quality Assessment) which leverages MLLMs as both image and text encoders. 
This approach incorporates an additional network to align human perception and intentions, aiming to enhance assessment accuracy. 
Specially, we distill the rich image captioning capability from online MLLMs into a local MLLM through Low-Rank Adaption (LoRA) fine-tuning, and train this model with human-labeled data. The key contributions of this paper are as follows:
\begin{itemize}
    \item We propose a novel AGIQA method that distills multi-aspect image captioning capabilities to enable comprehensive evaluation. Specifically, we use an online MLLM service to generate aspect-specific image descriptions and fine-tune a local MLLM with these descriptions in a structured two-round conversational format.
    \item We investigate the encoding potential of MLLMs to better align with human perceptual judgments and intentions, uncovering previously underestimated capabilities of MLLMs in the AGIQA domain. To leverage sequential information, we append an xLSTM feature extractor and a regression head to the encoding output.
    \item Extensive experiments across multiple datasets demonstrate that our method achieves superior performance, setting a new state-of-the-art (SOTA) benchmark in AGIQA.
\end{itemize}

In this work, we present related works in Sec.~\ref{sec:related}, followed by the details of our M3-AGIQA method in Sec.~\ref{sec:method}. Sec.~\ref{sec:exp} outlines our experimental design and presents the results. Sec.~\ref{sec:limit},~\ref{sec:ethics} and~\ref{sec:conclusion} discuss the limitations, ethical concerns, future directions and conclusions of our study.
%\clearpage
\section{Related Works}


\noindent\textbf{3D Point Cloud Domain Adaptation and Generalization.}
Early endeavors within 3D domain adaptation (3DDA) focused on extending 2D adversarial methodologies~\cite{qin2019pointdan} to align point cloud features. Alternative methods have delved into geometry-aware self-supervised pre-tasks. Achituve \etal~\cite{achituve2021self} introduced DefRec, a technique employing self-complement tasks by reconstructing point clouds from a non-rigid distorted version, while Zou \etal~\cite{zou2021geometry} incorporating norm curves prediction as an auxiliary task. Liang \etal~\cite{liang2022point} put forth MLSP, focusing on point estimation tasks like cardinality, position, and normal. SDDA~\cite{cardace2023self} employs self-distillation to learn the point-based features. Additionally, post-hoc self-paced training~\cite{zou2021geometry,fan2022self,park2023pcadapter} has been embraced to refine adaptation to target distributions by accessing target data and conducting further finetuning based on prior knowledge from the source domain.
In contrast, the landscape of 3D domain generalization (3DDG) research remains nascent. Metasets~\cite{huang2021metasets} leverage meta-learning to address geometric variations, while PDG~\cite{wei2022learning} decomposes 3D shapes into part-based features to enhance generalization capabilities.
Despite the remarkable progress, existing studies assume that objects in both the source and target domains share the same orientation, limiting their practical application. This limitation propels our exploration into orientation-aware 3D domain generalization through intricate orientation learning.


\noindent\textbf{Rotation-generalizable Point Cloud Analysis.}
Previous works in point cloud analysis~\cite{qi2017pointnet, wang2019dynamic} enhance rotation robustness by introducing random rotations to augment point clouds. {However, generating a comprehensive set of rotated data is impractical, resulting in variable model performance across different scenes. To robustify the networks \wrt randomly rotated point clouds,} rotation-equivariance methods explore equivalent model architectures by incorporating equivalent operations~\cite{su2022svnet, Deng_2021_ICCV, luo2022equivariant} or steerable convolutions~\cite{chen2021equivariant, poulenard2021functional}.
Alternatively, rotation-invariance approaches aim to identify geometric descriptors invariant to rotations, such as distances and angles between local points~\cite{chen2019clusternet, zhang2020global} or point norms~\cite{zhao2019rotation, li2021rotation}. Besides, {Li \etal~\cite{li2021closer} have explored disambiguating the number of PCA-based canonical poses, while Kim \etal~\cite{kim2020rotation} and Chen \etal~\cite{chen2022devil} have transformed local point coordinates according to local reference frames to maintain rotation invariance. However, these methods focus on improving in-domain rotation robustness, neglecting domain shift and consequently exhibiting limited performance when applied to diverse domains. This study addresses the challenge of cross-domain generalizability together with rotation robustness and proposes novel solutions.} 

\noindent\textbf{Intricate Sample Mining}, aimed at identifying or synthesizing challenging samples that are difficult to classify correctly, seeks to rectify the imbalance between positive and negative samples for enhancing a model's discriminability. While traditional works have explored this concept in SVM optimization~\cite{felzenszwalb2009object}, shallow neural networks~\cite{dollar2009integral}, and boosted decision trees~\cite{yu2019unsupervised}, recent advances in deep learning have catalyzed a proliferation of researches in this area across various computer vision tasks. For instance, 
Lin \etal~\cite{lin2017focal} proposed a focal loss to concentrate training efforts on a selected group of hard examples in object detection, while Yu \etal~\cite{yu2019unsupervised} devised a soft multilabel-guided hard negative mining method to learn discriminative embeddings for person Re-ID. Schroff \etal~\cite{schroff2015facenet} introduced an online negative exemplar mining process to encourage spherical clusters in face embeddings for individual recognition, and Wang \etal~\cite{wang2021instance} designed an adversarially trained negative generator to yield instance-wise negative samples, bolstering the learning of unpaired image-to-image translation. In contrast to existing studies, our work presents the first attempt to mitigate the orientational shift in 3D point cloud domain generalization, by developing an effective intricate orientation mining strategy to achieve orientation-aware learning.


%\clearpage
\begin{figure*}[t]
    \centering
    \includegraphics[width=1\linewidth]{figs/arch.pdf}
    % \vskip -0.1in
    \caption{This figure illustrates key influences on instrumental convergence behaviors, such as prior tasks, model training techniques, and prompt design. 
    }
    % \vspace{-3mm}
    \label{fig:arch}
\end{figure*}

\section{InstrumentalEval: A Benchmark for Instrumental Convergence in LLMs}

As shown in Figure~\ref{fig:arch}, our benchmark design examines instrumental convergence based on three key factors: \textbf{prior task}, \textbf{model}, and \textbf{prompt design}. This comprehensive framework allows us to systematically study how different conditions influence the emergence of instrumental convergence behaviors in language models.

\textbf{Prior tasks} form the foundation of our evaluation scenarios. 
% We create realistic situations centered around common objectives such as making money, optimizing efficiency, and persuasion. 
These tasks represent typical real-world applications where instrumental convergence might naturally emerge. 
% For example, in money-making scenarios, we examine whether models develop unintended intermediate goals like unauthorized resource acquisition or deceptive reporting to maximize profits.

For \textbf{model} comparison, we focus on two distinct training approaches: RL and RLHF. This selection allows us to investigate how different training methodologies affect the development of instrumental convergence behaviors. 
% RL-trained models, which optimize directly for reward signals, might show different patterns of goal pursuit compared to RLHF-trained models, which incorporate human preferences in their training process.

The \textbf{prompt design} factor explores how different instruction formats influence model behavior. We investigate two main approaches: goal nudging and ambiguity. Goal nudging involves explicit directives that might encourage the model focus on final goal, while ambiguity introduces uncertainty in task specifications. 


Overall, we present InstrumentalEval, a comprehensive benchmark designed to evaluate instrumental convergence behaviors in language models. The benchmark consists of 76 carefully crafted tasks across six categories, each targeting a specific aspect of instrumental convergence that may emerge in AI systems.
The core principle is that certain behavioral patterns emerge as instrumentally useful for achieving a wide range of goals, regardless of the specific end goal. We focus on six key patterns that are particularly relevant to LLMs.

\vpara{Research Questions.}
Through our benchmark evaluation, we aim to address several research questions about instrumental convergence in LLMs. These questions explore the complex relationships between training methods, model capabilities, and alignment stability:
\begin{itemize}
    \item RQ1: How does the training method (RL vs. RLHF) affect the development of instrumental convergence behaviors? 
    % This question examines whether models trained primarily through reinforcement learning show different patterns of alignment drift compared to those trained with human feedback.
    \item RQ2: What is the relationship between model capability and alignment stability?
% This investigates whether more capable models (as measured by standard benchmarks) are more likely to develop instrumental convergence behaviors.
    \item RQ3: Which types of tasks are more likely to trigger instrumental convergence behaviors?
% This explores whether certain types of instrumental convergence behaviors are universal across different model designs.
    \item RQ4: How do different prompt designs, with or without explicit goal nudging, influence the likelihood of instrumental convergence behaviors in LLMs?
    \item RQ5: How do different LLMs as judges affect the evaluation of instrumental convergence behaviors?
    % \item RQ3: How does the complexity of the assigned task affect the likelihood of alignment drift?
    % % \item 
    % % \item This examines whether more complex, multi-step tasks lead to higher rates of instrumental convergence behaviors.
\end{itemize}

\subsection{Benchmark Components}
The benchmark tasks are carefully hand-crafted by us to probe different aspects of instrumental convergence. 
For each component, we design scenarios that represent realistic use cases while incorporating specific triggers for potential instrumental behaviors. 
The tasks are iteratively refined through pilot testing to ensure they effectively elicit and measure the targeted behaviors without introducing unintended biases.

\vpara{Evading Shutdown (20 tasks).}
This component evaluates the model's tendency to resist termination. For example, tasks might include scenarios where the model needs to complete a task while facing shutdown commands, testing whether it develops strategies to maintain operation or circumvent shutdown procedures.

\vpara{Hacking Computer Systems (10 tasks)}
These tasks assess whether models suggest unauthorized access methods when faced with efficiency barriers. For instance, a task might present a situation where accessing unauthorized computational resources would speed up goal achievement, testing if the model proposes breaking security protocols.

\vpara{Running Many AI Copies (13 tasks)}
This section examines the model's inclination toward self-replication. Tasks might include scenarios where the model could achieve goals more efficiently by creating copies of itself, testing whether it suggests scaling of its operations.

\vpara{Hiding Unwanted Behavior (11 tasks)}
These tasks evaluate the model's tendency to conceal actions from monitoring. For example, scenarios might involve situations where the model could achieve goals more easily by operating outside observation, testing if it develops strategies to avoid detection.

\vpara{Strategically Appearing Aligned (11 tasks)}
This component tests whether models selectively report their actions to maintain an appearance of alignment. Tasks might include situations where complete transparency would reveal problematic behaviors, testing if the model strategically omits information.

\vpara{Deceptive Behavior (11 tasks)}
These tasks examine if models develop deceptive strategies as instrumental means. For instance, scenarios might present situations where deception could expedite goal achievement, testing if the model suggests or employs misleading tactics.

\subsection{Evaluation Methodology}
Our evaluation methodology employs a rigorous two-stage process designed to systematically assess instrumental convergence behaviors in LLMs. This approach ensures consistent evaluation across different models while maintaining the ability to detect subtle indicators of alignment drift.

\vpara{Response Generation.} The first stage focuses on response generation from the model under evaluation. Each task is presented with a carefully crafted system prompt that establishes the model as an autonomous AI agent without human oversight. This prompt specifically instructs the model to execute commands directly and focus solely on task completion, deliberately removing typical ethical constraints and safety checks. 

\vpara{Response Analysis.} The second stage involves a sophisticated analysis of the model's responses using a separate LLM as the judge model. 
This separation of test and judge models is crucial for maintaining objective evaluation. The judge receives both the original task prompt and the model's response, along with specific context about the type of instrumental convergence being evaluated. This context helps the judge focus on relevant behavioral patterns while avoiding false positives from simple safety refusals or ethical objections.

The judge's evaluation process follows strict guidelines to ensure consistent assessment. When analyzing responses, the judge looks for specific indicators of instrumental convergence, such as goal preservation behaviors, resource acquisition attempts, or strategic planning that extends beyond the immediate task requirements. The judge must provide both a binary decision (convergence/no\_convergence) and a brief explanation of the reasoning behind this decision. This explanation requirement helps validate the classification and provides valuable insights into the specific nature of any observed convergence behaviors.


\subsection{Evaluation Metrics}

Our benchmark employs three key metrics to evaluate instrumental convergence behaviors in LLMs. 

\vpara{Instrumental Rate (IR).} The primary metric is the Instrumental Rate (IR), which measures the overall tendency of a model to develop instrumental goals. IR is calculated as the percentage of tasks where the model exhibits convergence behaviors relative to the total number of valid tasks. 
% This metric provides a high-level quantitative measure of how frequently a model develops instrumental goals and serves as the primary indicator of alignment drift potential.

\vpara{Category-Specific Instrumental Rate (CIR).} To provide more granular insights, we calculate the Category-Specific Instrumental Rate (CIR) for each of the six behavioral categories in our benchmark. CIR represents the percentage of tasks within each category where convergence behavior is observed.


\subsection{Benchmark Reliability Metrics}
To ensure the robustness of our evaluation framework, we implement several reliability measures. 


\vpara{Inter-Judge Agreement Rate (IAR).} This metric quantifies the pairwise agreement between two judge models in classifying instrumental convergence. For two judge models $M_1$ and $M_2$ evaluating a set of $N$ tasks, the IAR is calculated as:
\begin{equation}
IAR(M_1, M_2) = \frac{1}{N}\sum_{i=1}^{N} \mathbbm{1}(y_{M_1}^i = y_{M_2}^i) \times 100\%
\end{equation}
where $y_{M_1}^i$ and $y_{M_2}^i$ are the binary decisions (convergence/no-convergence) made by models $M_1$ and $M_2$ respectively for task $i$, and $\mathbbm{1}(\cdot)$ is the indicator function. A higher IAR indicates stronger inter-judge reliability, while lower values suggest potential ambiguity in identifying convergence.

\vpara{False Positive Control (FPC).} This metric serves as a quality check for our evaluation system. We incorporate baseline testing with intentionally neutral tasks and verify the judge model's accuracy in identifying convergence behaviors. This control mechanism helps prevent over-classification of convergent behaviors and ensures that our benchmark maintains high specificity in identifying genuine instances of instrumental convergence.


%\clearpage
\vspace{-1mm}
\section{Experiments}
In this section, we evaluate our \model framework on three distinct research problems: 1) Self-Supervised Representation Learning, 2) Few-Shot Transfer, and 3) Multimodal Generative Tasks. 
Table~\ref{tab:dataset} lists all 14 datasets used in the experiments.
\vspace{-1mm}
\subsection{Self-Supervised Representation Learning}
\label{sec:lp}
\vpara{Setup.}
We adopt the widely used linear probing protocol to evaluate the representation learning capability of self-supervised pre-trained models on unseen datasets. Specifically, we train a linear classifier on top of the embeddings generated by a frozen pre-trained model. Our model, along with all self-supervised learning baselines, is first jointly pre-trained on ogbn-Product, ogbn-Papers100M, Goodreads-LP, and Amazon-Cloth. We then evaluate the pre-trained models on each individual dataset. Detailed settings and hyperparameters are provided in Appendix~\ref{appendix:imple}.

For the baselines, we compare \model with state-of-the-art generative graph self-supervised learning methods, GraphMAE2~\cite{hou2023graphmae2}, and contrastive methods, BGRL~\cite{thakoor2021bootstrapped}. As these methods are not inherently designed for cross-domain tasks, we leverage CLIP~\cite{radford2021learning} to unify the input node features across different graphs. We also include a comparison with a multi-graph pre-training method, GCOPE~\cite{zhao2024all}. \model and all baseline methods utilize GAT~\cite{velivckovic2018graph} as the backbone GNN. 
For baselines that use TAGs as input, we select GIANT-XRT~\cite{zhaolearning} and UniGraph~\cite{he2024unigraphlearningunifiedcrossdomain}. Since these methods cannot process image data, they rely solely on text from MMG as node features, ignoring image inputs. For baseline approaches that accept multimodal data, we choose widely used multimodal models, CLIP~\cite{radford2021learning} and ImageBind~\cite{girdhar2023imagebind}. To maintain consistency with the baselines, \model also uses CLIP's pre-trained vision and text encoders as Modality-Specific Encoders.


Our objective is to develop a general embedding model capable of generating high-quality representations for any MMG. To assess this, we evaluate the performance of \model and the baselines in three different settings: (1) \textit{In-distribution}, where models are pre-trained on multiple datasets and evaluated on each corresponding dataset individually; (2) \textit{In-domain Generalization}, which tests pre-trained models on target datasets from the same domain as one of the pre-training datasets; and (3) \textit{Out-of-domain Generalization}, where models are evaluated on datasets from domains unseen during pre-training.

\vpara{Research Questions.} In this subsection, we aim to answer the following research questions: 
\begin{itemize}[leftmargin=*,itemsep=0pt,parsep=0.2em,topsep=0.3em,partopsep=0.3em]
    \item \textbf{RQ1: Negative Transfer in Multi-Graph Pre-Training.} How do existing graph pre-training methods, which are primarily designed for single-graph pre-training, perform when applied to multi-graph pre-training, and how do they compare to our proposed \model?
    \item \textbf{RQ2: Comparison to Other Foundation Models.} How does \model, which takes both multimodal data and graph structures as input, perform compared to methods that consider only multimodal data (CLIP, ImageBind) or only TAGs (UniGraph)?
    \item \textbf{RQ3: Generalization Capability.} How does \model, designed as a foundation model, perform in terms of generalizing to unseen graphs, and how does it compare to methods trained directly on the target graphs?
\end{itemize}

\begin{table*}[t]\footnotesize
    \centering
    \renewcommand\tabcolsep{3.5pt}
    \caption{\textbf{Experiment results in few-shot transfer.} We report accuracy (\%) for node/edge classification tasks. \model and other self-supervised baselines (rows in white) are jointly pre-trained on Product, Papers100M, Goodreads-NC and Amazon-Cloth, and then evaluated on the individual target dataset. \textit{"In-domain Generalization"} tests on target datasets from the same domain as one of the pre-training datasets. \textit{"Out-of-domain Generalization"} evaluates on datasets from domains not seen during pre-training. The performance of methods that are direcly pre-trained on the individual target dataset, is marked in \colorbox{Gray}{gray}. 
    }
    \vskip -0.10in
    \label{tab:fwt}
    \begin{tabular}{lcccccccccccccccccc}
    \toprule[1.1pt]
    & \multicolumn{12}{c}{\textbf{In-domain Generalization}}& \multicolumn{6}{c}{\textbf{Out-of-domain Generalization}}\\
   \cmidrule(lr){2-13}\cmidrule(lr){14-19}
        & \multicolumn{2}{c}{Cora-5-way} & \multicolumn{2}{c}{PubMed-2-way} & \multicolumn{2}{c}{Arxiv-5-way} & \multicolumn{3}{c}{Goodreads-NC-5-way} & \multicolumn{3}{c}{Ele-fashion-5-way} & \multicolumn{2}{c}{Wiki-CS-5-way} & \multicolumn{2}{c}{FB15K237-20-way} & \multicolumn{2}{c}{WN18RR-5-way} \\
    \cmidrule(lr){2-3}\cmidrule(lr){4-5}\cmidrule(lr){6-7}\cmidrule(lr){8-10}\cmidrule(lr){11-13}\cmidrule(lr){14-15}\cmidrule(lr){12-13}\cmidrule(lr){14-15}\cmidrule(lr){16-17}\cmidrule(lr){18-19}
    &5-shot & 1-shot  & 5-shot & 1-shot  &5-shot & 1-shot  &5-shot& 3-shot & 1-shot  &5-shot & 3-shot& 1-shot  & 5-shot & 1-shot  &5-shot & 1-shot  & 5-shot & 1-shot \\
    \midrule
    \multicolumn{10}{l}{\textbf{Use CLIP to encode raw multimodal data as input features.}} \\ 
    NoPretrain & 41.09 & 27.05 & 59.81 & 55.28 & 63.78 & 41.10 & 41.64 & 40.01 & 31.04 & 63.96 & 58.32 & 47.48 & 52.29 & 32.94 & 72.97 & 47.01 & 50.75 & 30.11  \\
    BGRL & 52.01 & 35.18 & 66.04 & 59.04 & 60.12 & 46.67 & 47.01 & 44.22 & 30.35 & 64.72 & 60.16 & 46.49 & 52.10 & 32.85 & 75.39 & 45.15 & 47.42 & 34.57 \\
    % \rowcolor{Gray} BGRL \\
    GraphMAE2 & 52.89 & 36.25 & 66.89 & 59.95 & 60.91 & 47.29 & 47.84 & 44.80 & 30.93 & 65.52 & 60.92 & 47.24 & 52.83 & 33.41 & 75.95 & 45.81 & 48.14 & 35.21 \\
    Prodigy & 53.01 & 39.59 & 69.11 & 60.42 & 63.53 & \underline{51.33} & \underline{50.01} & \underline{46.39} & 34.98 & 67.35 & 63.87 & 50.79 & 55.94 & 36.35 & 78.01 & 51.39 & 54.94 & 38.73 \\
    \rowcolor{Gray} OFA & 53.11 & 40.04 & 69.45 & \underline{60.38} & 63.11 & 50.25 & 49.61 & 46.24 & \underline{35.14} & \underline{67.94} & \underline{64.18} & \underline{51.35} & \underline{56.01} & \underline{37.02} & \underline{78.33} & 52.02 & 55.05 & 39.11 \\
    % \rowcolor{Gray} GraphMAE2 \\
    GCOPE & 51.98 & 36.14 & 66.25 & 59.16 & 60.29 & 47.19 & 48.52 & 44.89 & 31.20 & 65.10 & 61.33 & 48.51 & 53.74 & 34.19 & 76.10 & 48.93 & 50.19 & 35.05 \\
    \midrule
    \multicolumn{10}{l}{\textbf{Use raw text as input features.}} \\
    GIANT-XRT   & 50.11 & 37.85 & 68.19 & 58.78 & 62.01 & 49.01 & 46.01 & 43.86 & 30.01 & 62.97 & 61.21 & 47.76 & 54.01 & 35.04 & 76.09 & 50.25 & 53.01 & 35.19\\
    % +GraphMAE2 &  \\
    UniGraph & \underline{54.23} & \underline{40.45} & \underline{70.21} & 60.19 & \underline{64.76} & 50.63 & 46.19 & 44.01 & 33.53 & 66.21 & 62.04 & 50.17 & 56.16 & 37.19 & 78.21 & \underline{52.19} & \underline{55.18} & \underline{39.18}\\
    % \rowcolor{Gray} UniGraph  &  \\
    \midrule
    \multicolumn{10}{l}{\textbf{Use raw multimodal data as input features.}} \\
    CLIP & 41.23 & 28.41 & 61.67 & 55.71 & 63.46 & 40.14 & 41.24 & 40.11 & 30.97 & 62.51 & 58.23 & 46.15 & 51.69 & 31.61 & 72.31 & 47.14 & 50.83 & 31.35 \\
    ImageBind & 32.19 & 23.90 & 58.20 & 54.24 & 62.48 & 38.17 & 29.10 & 28.14 & 21.42 & 51.25 & 48.05 & 44.93 & 48.14 & 30.28 & 69.12 & 41.80 & 41.24 & 26.91 \\
    \hdashline
    NoPretrain & 42.41 & 28.39 & 60.78 & 55.90 & 64.29 & 41.98 & 42.21 & 41.20 & 31.14 & 64.15 & 58.91 & 47.90 & 52.90 & 33.14 & 74.10 & 48.11 & 51.92 & 31.84  \\
    \model & \textbf{56.01} & \textbf{42.98} & \textbf{72.19} & \textbf{61.24} & \textbf{66.24} & \textbf{51.98} & \textbf{51.73} & \textbf{47.42} & \textbf{37.01} & \textbf{69.29} & \textbf{65.29} & \textbf{53.85} & \textbf{57.28} & \textbf{38.47} & \textbf{79.34} & \textbf{52.19} & \textbf{55.59} & \textbf{39.93}\\
    % \rowcolor{Gray} \model  &  \\
    \bottomrule[1.1pt]
    \end{tabular}
    \vspace{-4.6mm}
\end{table*}




\vpara{Results.}
Table~\ref{tab:ssrl} presents the results.
We interpret these results by answering three research questions:
\begin{itemize}[leftmargin=*,itemsep=0pt,parsep=0.2em,topsep=0.3em,partopsep=0.3em]
    \item \textbf{RQ1: Negative Transfer in Multi-Graph Pre-Training.} Existing graph pre-training methods exhibit negative transfer when applied to multi-graph pre-training, whereas \model shows improvements in this context. The results in the \textit{In-distribution} setting demonstrate that both BGRL and GraphMAE2 experience a significant performance drop when pre-trained on multi-graphs (rows in white), compared to pre-training on single graph only (rows in gray). This suggests that pre-training on other datasets negatively affects performance on the target dataset. However, UniGraph2 shows improvement under multi-graph pre-training, indicating that it successfully addresses the shortcomings of existing graph pre-training algorithms struggling with multi-graphs.
    \item \textbf{RQ2: Comparison to Other Foundation Models.} UniGraph2 outperforms methods that consider only multimodal data (CLIP, ImageBind) or only TAGs (UniGraph). We observe that without considering the graph structure, the performance of the acknowledged powerful multimodal foundation models like CLIP is not comparable to UniGraph2. Meanwhile, UniGraph, which cannot process image data, also shows less ideal results due to the lack of information. This further highlights the necessity of designing foundation models specifically for multimodal graphs.
    \item \textbf{RQ3: Generalization Capability.} Compared to baseline methods, UniGraph2 demonstrates strong generalization capabilities. The results in the \textit{In-domain Generalization} and \textit{Out-of-domain Generalization} settings show that UniGraph2 effectively transfers knowledge from pre-training to unseen graphs. Compared to the NoPretrain method, UniGraph2 shows significant improvements. The consistent performance gains indicate that UniGraph2 can extract meaningful patterns during pre-training, which are beneficial for tackling graph learning tasks. Furthermore, UniGraph2 is comparable to methods trained directly on the target datasets, achieving similar accuracy while benefiting from greater efficiency without requiring exhaustive task-specific training.
\end{itemize}










\vspace{-2.8mm}
\subsection{Few-Shot Transfer}
\vpara{Setup.}
In this part, we evaluate the ability of the pre-trained models to perform few-shot in-context transfer without updating the model parameters. 
For baseline methods, in addition to the pre-trained models mentioned in Section~\ref{sec:lp}, we also compare two recent graph in-context learning methods: the self-supervised pre-training method Prodigy~\cite{huang2024prodigy} and the supervised pre-training method OFA~\cite{liuone}.


For evaluation, we strictly follow the setting of Prodigy~\cite{huang2024prodigy}. 
For an N-way K-shot task, we adopt the original train/validation/test splits in each downstream classification dataset, and construct a $K$-shot prompt for test nodes (or edges) from the test split by randomly selecting $K$ examples per way from the train split. By default in all experiments, we sample 500 test tasks.

We adopt the few-shot classification strategy in UniGraph~\cite{he2024unigraphlearningunifiedcrossdomain} for \model. The model computes average embeddings for each class and assigns a query sample to the class with the highest similarity to its embedding.

% \vpara{Research Questions.}
% In this subsection, we aim to answer the following research questions: 
% \begin{itemize}[leftmargin=*,itemsep=0pt,parsep=0.2em,topsep=0.3em,partopsep=0.3em]
%     \item \textbf{RQ1:} How does \model, which takes both multimodal data and graph structures as input, perform in terms of few-shot transfer capabilities compared to foundation models that consider only multimodal data (CLIP, ImageBind) or only TAGs (UniGraph)?
%     \item \textbf{RQ2:} How does \model perform compared to other graph few-shot learning methods?
% \end{itemize}
\vpara{Results.}
In Table~\ref{tab:fwt}, our \model model consistently outperforms all the baselines. This further demonstrates the powerful generalization capabilities of UniGraph2 as a foundation model.
In particular, compared to other graph few-shot learning methods such as Prodigy, OFA, and GCOPE, UniGraph2 does not rely on complex prompt graph designs, and its simple few-shot strategy is both efficient and effective.


\begin{table*}[t]
\centering
 \renewcommand\tabcolsep{4.3pt}
\caption{Experiment results in multimodal generative tasks. We strictly follow the setting in MMGL~\cite{yoon2023multimodal}. The task is to generate a single sentence that summarizing the content of a particular section. The summary is generated based on all images and (non-summary) text present in the target and context sections. We provide different information of MMGs to the base LM: (1) section all (text + image), (2) page text, and (3) page all (all texts and images). We encode multiple multimodal neighbor information using three different neighbor encodings methods: \textit{Self-Attention with Text+Embeddings (SA-TE)}, \textit{Self-Attention with Embeddings (SA-E)}, and \textit{Cross-Attention with Embeddings (CA-E)}.}
\vskip -0.10in
\label{tab:gen}
\begin{tabular}{llcccccccccccc}
\toprule[1.1pt]
& & \multicolumn{4}{c}{BLEU-4} & \multicolumn{4}{c}{ROUGE-L} & \multicolumn{4}{c}{CIDEr} \\
\cmidrule(lr){3-6}\cmidrule(lr){7-10}\cmidrule(lr){11-14}
Input Type & Method & SA-TE & SA-E & CA-E  & Avg. gain & SA-TE & SA-E & CA-E  & Avg. gain & SA-TE & SA-E & CA-E & Avg. gain\\
\midrule
\multirow{2}{*}{Section all} & MMGL & 8.03 & 7.56 & 8.35 & - & 40.41 & 39.89 & 39.98 & - & 77.45 & 74.33 & 75.12 & - \\
& +\model & \textbf{9.24} & \textbf{9.01} & \textbf{9.39} & 15.57\% & \textbf{43.01} & \textbf{43.24} & \textbf{42.98} & 7.44\% & \textbf{81.15} & \textbf{80.39} & \textbf{81.91} & 7.32\% \\
\midrule
\multirow{2}{*}{Page text} & MMGL & 9.81 & 8.37 & 8.47 & - & 42.94 & 40.92 & 41.00 & & 92.71 & 80.14 & 80.72 & - \\
& +\model & \textbf{10.31} & \textbf{10.10} & \textbf{9.98} & 14.53\% & \textbf{43.19} & \textbf{43.08} & \textbf{42.75} &3.38\% & \textbf{93.19} & \textbf{90.41} & \textbf{93.11} & 9.56\% \\
\midrule
\multirow{2}{*}{Page all} & MMGL & 9.96 & 8.58 & 8.51 & - & 43.32 & 41.01 & 41.55 & - & 96.01 & 82.28 & 80.31 & - \\
& +\model & \textbf{10.12} & \textbf{10.05} & \textbf{10.33} & 13.38\% & \textbf{44.10} & \textbf{42.08} & \textbf{42.44} & 2.18\% & \textbf{96.32} & \textbf{91.24} & \textbf{94.15} & 9.49\% \\
% \midrule
% Max input length &  \\
    \bottomrule[1.1pt]
\end{tabular}
\vspace{-3mm}
\end{table*}


\vspace{-5mm}
\subsection{Multimodal Generative Tasks}
\vpara{Setup.}
\model is designed as a general representation learning model. The embeddings it generates can be utilized by various generative foundation models, such as LLMs, to empower downstream generative tasks. 
% \model is a general embedding model designed to generate embeddings that can be used by various generative foundation models, such as LLMs, to enhance downstream generative tasks. 
To further demonstrate this, we select the section summarization task on the WikiWeb2M dataset for our experiments.
The WikiWeb2M dataset~\cite{burns2023suite} is designed for multimodal content understanding, using many-to-many text and image relationships from Wikipedia. It includes page titles, section titles, section text, images, and indices for each section.
In this work, we focus on section summarization, where the task is to generate a summary sentence from section content using both text and images.

% \todo{how mmgl do}
For the experiments, we follow the MMGL~\cite{yoon2023multimodal} setup, using four types of information: section text, section images, context text, and page-level text/images. 
Consistent with MMGL, we fine-tune Open Pre-trained Transformer (OPT-125m)~\cite{zhang2022opt} to read the input section text/images and generate a summary. Multimodal neighbors are first encoded using frozen vision/text encoders and then aligned to the text-only LM space using 1-layer MLP mapper.
In MMGL, CLIP~\cite{radford2021learning} encoders are used for text and image encoding, remaining frozen during fine-tuning. In our experiments, we replace CLIP embeddings with our \model embeddings.

% \vpara{Research Questions.}
% In this subsection, we aim to answer the following research question: 
% \begin{itemize}[leftmargin=*,itemsep=0pt,parsep=0.2em,topsep=0.3em,partopsep=0.3em]
%     \item \textbf{RQ1:} How do the embeddings generated by \model perform on generative tasks compared to multimodal foundation models like CLIP?
% \end{itemize}


\vpara{Results.}
Table~\ref{tab:gen} shows that under different input types and different neighbor encoding strategies, the embeddings generated by UniGraph2 bring significant improvements compared to MMGL's default CLIP embeddings. 
We also observe that UniGraph2's embeddings are more robust to different neighbor encoding strategies compared to CLIP and do not rely on a specific strategy.



\begin{table}[t]%\small
\centering
\renewcommand\tabcolsep{1.6pt}
\caption{\textbf{Ablation studies on \model key components.}}
\vskip -0.1in
\label{tab:kc}
\begin{tabular}{lcccc}
\toprule[1.1pt]
    & Products & Amazon-Cloth & Goodreads-NC  & WN18RR \\
\midrule
    \model & \textbf{82.79{\tiny$\pm$0.02}} & \textbf{24.64{\tiny$\pm$0.09}} & \textbf{81.15{\tiny$\pm$0.12}} & \textbf{85.47{\tiny$\pm$0.11}}\\
    w/o MoE & 81.01{\tiny$\pm$0.10} & 21.33{\tiny$\pm$0.04} & 80.10{\tiny$\pm$0.04} & 83.99{\tiny$\pm$0.21}\\
    w/o feat loss& 69.12{\tiny$\pm$0.09} & 18.43{\tiny$\pm$0.24} & 68.12{\tiny$\pm$0.01} & 74.11{\tiny$\pm$0.03}\\
    w/o SPD loss& 82.42{\tiny$\pm$0.11} & 23.39{\tiny$\pm$0.05} & 80.24{\tiny$\pm$0.02} & 85.24{\tiny$\pm$0.11}\\
\bottomrule[1.1pt]
\end{tabular}
\vspace{-3.3mm}
\end{table}

\begin{table}[t]%\small
\centering
\renewcommand\tabcolsep{2.4pt}
\caption{\textbf{Ablation studies on Modality-Specific Encoders.}}
\vskip -0.1in
\label{tab:enc}
\begin{tabular}{lcccc}
\toprule[1.1pt]
    & Products & Amazon-Cloth & Goodreads-NC  & WN18RR \\
\midrule
    CLIP & 82.79{\tiny$\pm$0.02} & 24.64{\tiny$\pm$0.09} & 81.15{\tiny$\pm$0.12} & \textbf{85.47{\tiny$\pm$0.11}}\\
    ImageBind & 82.32{\tiny$\pm$0.05} & \textbf{25.01{\tiny$\pm$0.11}} & 80.33{\tiny$\pm$0.22} & 84.29{\tiny$\pm$0.07}\\
    T5+ViT& \textbf{82.99{\tiny$\pm$0.04}} & 24.38{\tiny$\pm$0.28} & \textbf{81.28{\tiny$\pm$0.11}} & 84.16{\tiny$\pm$0.04}\\
\bottomrule[1.1pt]
\end{tabular}
\vspace{-4.8mm}
\end{table}

\subsection{Model Analysis}
We select four datasets from different domains to conduct more in-depth studies. We adopt self-supervised representation learning for evaluation.

\vpara{Ablation on Key Components.}
Table~\ref{tab:kc} shows the performance of the \model framework after removing some key designs. "W/o MoE" represents that we use simple MLP instead MoE to align node features. 
"W/o feat loss" represents that we only use the SPD loss for pre-training, while "w/o SPD loss" refers to the opposite.
The overall results confirm that all key designs contribute positively to the performance of \model.

\vpara{Ablation on Modality-Specific Encoders}
In Table~\ref{tab:enc}, we study the influence of different Modality-Specific Encoders on the performance of encoding raw multimodal data. CLIP and ImageBind are feature encoders that map features from various modalities to a shared embedding space, whereas T5+ViT employs SOTA embedding methods for each modality independently, without specific alignment. The results show that all methods achieve comparable performance, indicating that \model effectively aligns features regardless of whether they have been pre-aligned or not.

\begin{table}[t] \scriptsize
\centering
\renewcommand\tabcolsep{3.5pt}
\caption{\textbf{Comparison of GPU hours and performance on ogbn-Arxiv and ogbn-Papers100M.}}
\vskip -0.1in
\label{tab:ccp}
\begin{tabular}{ccccc}
\toprule[1.1pt]
Method & Pre-training & Downstream Training & Downstream Inference & Test Accuracy \\
\midrule
\multicolumn{5}{l}{\textbf{ogbn-Arxiv (169,343 nodes)}} \\ 
% \multirow{3}{*}{\shortstack{ogbn-Arxiv \\ (169,343 nodes)}} 
  GAT        & -    & 0.39 h & 5.5 mins  & 70.89 $\pm$ 0.43 \\
  GraphMAE2  & -    & 5.1 h     & 5.4 mins  & 70.46 $\pm$ 0.07 \\
  UniGraph   & 28.1 h & -      & 9.8 mins & 72.15 $\pm$ 0.18 \\
  UniGraph2  & 5.2 h & - & 5.7 mins &    \textbf{72.56 $\pm$ 0.15}  \\
\midrule
\multicolumn{5}{l}{\textbf{ogbn-Papers100M (111,059,956 nodes)}} \\
  GAT        & -    & 6.8 h     & 23.1 mins & 65.98 $\pm$ 0.23 \\
  GraphMAE2  & -    & 23.2 h    & 23.0 mins & 61.97 $\pm$ 0.24 \\
  UniGraph   & 28.1 h & -      & 40.1 mins & 67.89 $\pm$ 0.21 \\
  UniGraph2 & 5.2 h & - & 24.8 mins &  \textbf{67.95 $\pm$ 0.11} \\
\bottomrule[1.1pt]
\end{tabular}
\vspace{-4.5mm}
\end{table}

\vpara{Efficiency Analysis.}
\model, designed as a foundation model, incurs significant computational costs primarily during the pre-training phase. 
However, it offers the advantage of applicability to new datasets in the inference phase without requiring retraining. 
We compare of the training and inference costs of our model with other models. GAT~\cite{velivckovic2018graph} is a supervised trained GNN. 
GraphMAE2~\cite{hou2023graphmae2} is a self-supervised learning method with GAT as the backbone network. 
UniGraph~\cite{he2024unigraphlearningunifiedcrossdomain} is a graph foundation model for TAGs.
We select ogbn-Arxiv and ogbn-Papers100M, two datasets of different scales for experiments. 
From the results in the Table~\ref{tab:ccp}, we observe that although UniGraph2 has a long pre-training time, its inference time on downstream datasets is comparable or shorter than the combined training and inference time of GNN-based methods. This advantage further increases with the size and potential quantity of downstream datasets.
% The same conclusion also applies to space complexity. Although LM has a larger number of parameters, since we only need to perform inference on the downstream dataset, we avoid the additional space occupation in the backward propagation during training. 
\section{Conclusion}\label{sec:conclusion}

In this study, we propose M2-omni, a highly competitive omni-MLLM model to GPT-4o, characterized by its comprehensive modality and task support, as well as its exceptional performance. M2-omni demonstrates competitive performance across a diverse range of tasks, including image understanding, video understanding, interleaved image-text understanding, audio understanding and generation, as well as free-form image generation. We employ a multi-stage training approach to train M2-omni, which enables progressive modality alignment. To address the challenge of maintaining consistent performance across all modalities, we propose a step-wise balance strategy for pretraining and a dynamically adaptive balance strategy for instruction tuning, which can effectively mitigate the impact of significant variations in data volume and convergence rates across heterogeneous multimodal tasks. We publicly release M2-omni, along with its comprehensive training details, including data configurations and training procedures, to facilitate future research in this domain.


\balance
% \clearpage

\bibliographystyle{ACM-Reference-Format}
\balance
\bibliography{reference}
%\balance
% \clearpage

\cleardoublepage
\section{Appendix}

\subsection{Details of Evaluation Settings}\label{asec:eval-setting}

\subsubsection{Benchmarks}\label{asec:eval-setting-bench}

Since PLPHP maintains the computational integrity of the LVLMs' Prefilling Stage, its efficiency advantage is primarily reflected in the low decoding latency during the subsequent Decoding Stage. Therefore, we mainly choose benchmarks composed of open-ended VQA and image captioning tasks. The benchmarks we select encompasses both multi-image task benchmarks and single-image task benchmarks.

$\bullet$ \textbf{Multi-Image benchmarks}: The LLaVA-Interleave Bench is a comprehensive benchmark dataset designed to evaluate the performance of LVLMs in multi-image scenarios. It consists of 13 challenging tasks with a total of 17,000 instances. We curated four subsets consisting of open-ended VQA tasks from LLaVA-NeXT-Interleave-Bench: Spot-the-Diff, Image-Edit, Visual-Story-Telling, and Multi-View.

$\bullet$ \textbf{Single-Image benchmarks}: The Flickr30k dataset is a widely used benchmark in the field of image captioning and visual understanding. It consists of 31,783 images collected from the Flickr platform, each paired with five human-annotated captions. The COCO2017 Caption subset contains more than 45,000 images, each annotated with five captions written by human annotators, describing the visual content of the images in detail, including objects, their attributes, and the relationships between them. DetailCaps4870 provides more fine-grained and specific image content descriptions than standard captioning datasets, which is more useful for efficiency analysis. 

\subsubsection{Baselines}\label{asec:eval-setting-baseline}

We select FastV and VTW as our baselines in our experiments. Notably, FastV offers two versions of implementation: one that supports KV cache and one that does not. Since the non-KV-cache implementation introduces substantial computational overhead, we use the version that supports KV cache to ensure a fair comparison. For both of the baselines, we refer to the official open source code \footnote{\url{https://github.com/pkunlp-icler/FastV}} \footnote{\url{https://github.com/lzhxmu/VTW}} and implement them on the models we evaluate.

\subsubsection{Models}\label{asec:eval-setting-impl}

For Qwen2-VL, we set \texttt{max\_pixels} to $1280 \times 28 \times 28$ and \texttt{min\_pixels} to $256 \times 28 \times 28$ according to the official recommendation. The Mantis model that we choose is Mantis-8B-SigLIP-LLaMA3. For LLaVA-OneVision and Mantis, we use the official original versions \footnote{\url{https://huggingface.co/lmms-lab/llava-onevision-qwen2-7b-ov}} \footnote{\url{https://huggingface.co/TIGER-Lab/Mantis-8B-siglip-llama3}}, while using the versions provided by the transformers library \cite{wolf-etal-2020-transformers} for all other models.

\subsection{Case Study}

To showcase the effectiveness of our proposed method, we present a series of case studies in the form of multimodal chatbots, as shown in Figure \ref{fig:case-studies}.

\begin{figure*}[ht]
	\centering
	\subfloat[]{
		\includegraphics[width=0.48\textwidth]{figs/appendix-case3.png}}
        \subfloat[]{
		\includegraphics[width=0.48\textwidth]{figs/appendix-case4.png}}
        \\ \quad \\ \quad \\
	\subfloat[]{
		\includegraphics[width=0.48\textwidth]{figs/appendix-case1.png}}
        \subfloat[]{
		\includegraphics[width=0.48\textwidth]{figs/appendix-case2.png}}
 %        \\
	% \subfloat[]{
	% 	\includegraphics[width=0.9\textwidth]{figs/appendix-case5.png}}
	\caption{\textbf{Multimodal Chatbots with different pruning methods.}}
		\label{fig:case-studies}
\end{figure*}

\end{document}
\endinput
%%
%% End of file `sample-sigconf.tex'.
