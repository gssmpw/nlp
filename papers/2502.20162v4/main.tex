% CVPR 2025 Paper Template; see https://github.com/cvpr-org/author-kit

\documentclass[10pt,twocolumn,letterpaper]{article}

%%%%%%%%% PAPER TYPE  - PLEASE UPDATE FOR FINAL VERSION
%\usepackage{cvpr}              % To produce the CAMERA-READY version
%\usepackage[review]{cvpr}      % To produce the REVIEW version
\usepackage[pagenumbers]{cvpr} % To force page numbers, e.g. for an arXiv version
%%%% My packages %%%%
\usepackage{amsmath}
\usepackage{subcaption}
\usepackage{multicol}
\usepackage{graphicx}
\usepackage{xcolor}
\usepackage{algpseudocode}
\usepackage{algorithm}
\usepackage{amsmath}
\usepackage{bm}
\usepackage{subfiles}
\graphicspath{{figures/}}

\usepackage[switch]{lineno}


\usepackage{tikz}
\usetikzlibrary{bayesnet}

% Import additional packages in the preamble file, before hyperref
\newcommand{\CG}{\mathcal{G}\xspace}
\newcommand{\CV}{\mathcal{V}\xspace}
\newcommand{\CE}{\mathcal{E}\xspace}
\newcommand{\CA}{\mathcal{A}\xspace}
\newcommand{\CF}{\mathcal{F}\xspace}
\newcommand{\CR}{\mathcal{R}\xspace}
\newcommand{\CB}{\mathcal{B}\xspace}
\newcommand{\CX}{\mathcal{X}\xspace}
\newcommand{\CK}{\mathcal{K}\xspace}
\newcommand{\CM}{\mathcal{M}\xspace}
\newcommand{\CC}{\mathcal{C}\xspace}
\newcommand{\CL}{\mathcal{L}\xspace}
\newcommand{\CI}{\mathcal{I}\xspace}
\newcommand{\CQ}{\mathcal{Q}\xspace}
\newcommand{\CO}{\mathcal{O}\xspace}
\newcommand{\CP}{\mathcal{P}\xspace}
\newcommand{\CS}{\mathcal{S}\xspace}
\newcommand{\CT}{\mathcal{T}\xspace}
\newcommand{\CJ}{\mathcal{J}\xspace}
\usepackage[para]{footmisc}
\usepackage{subfig}
% \usepackage{subcaption}
% \usepackage{array}
% \usepackage{colortbl}



% It is strongly recommended to use hyperref, especially for the review version.
% hyperref with option pagebackref eases the reviewers' job.
% Please disable hyperref *only* if you encounter grave issues, 
% e.g. with the file validation for the camera-ready version.
%
% If you comment hyperref and then uncomment it, you should delete *.aux before re-running LaTeX.
% (Or just hit 'q' on the first LaTeX run, let it finish, and you should be clear).
\definecolor{cvprblue}{rgb}{0.21,0.49,0.74}
\usepackage[pagebackref,breaklinks,colorlinks,citecolor=cvprblue]{hyperref}

%% New commands
\newcommand{\card}[1]{\lvert\mathcal{#1}\rvert}

%%%%%%%%% PAPER ID  - PLEASE UPDATE
\def\paperID{15137} % *** Enter the Paper ID here
\def\confName{CVPR}
\def\confYear{2025}

%%%%%%%%% TITLE - PLEASE UPDATE
\title{Gradient-Guided Annealing for Domain Generalization}

%%%%%%%%% AUTHORS - PLEASE UPDATE
\author{Aristotelis Ballas\\
Dpt of Informatics and Telematics\\
Harokopio University of Athens\\
Omirou 9, Tavros, Athens, Greece\\
{\tt\small aballas@hua.gr}
% For a paper whose authors are all at the same institution,
% omit the following lines up until the closing ``}''.
% Additional authors and addresses can be added with ``\and'',
% just like the second author.
% To save space, use either the email address or home page, not both
\and
Christos Diou\\
Dpt of Informatics and Telematics\\
Harokopio Univesity of Athens\\
Omirou 9, Tavros, Athens, Greece\\
{\tt\small cdiou@hua.gr}
}

\begin{document}
%\maketitle

\twocolumn[{%
	\renewcommand\twocolumn[1][]{#1}%
	\maketitle
	\begin{center}
		\setcounter{figure}{0}
		\centering
		\captionsetup{type=figure}
		
		% First figure
		\vspace{-1.5em}
		\includegraphics[width=0.7\textwidth,trim={0.0cm 0.0cm 0cm 0},clip,page=1]{concept-erm-final}\vspace{-0.5em}%
		
		% Space between figures
		\vspace{1em}
		
		% Second figure
		\includegraphics[width=0.7\textwidth,trim={0.0cm 0.0cm 0cm 0},clip,page=1]{concept-gga-final}\vspace{-0.5em}%
		\caption{\em (left) Decision boundaries of a 4\textsuperscript{th}-degree polynomial logistic regression model with 2D input. In this example, feature $x_1$ is class-specific and $x_2$ is domain-specific, while color represents classes and shapes represent domains. The samples with solid red and green colors are included in the training data, whereas the fainted samples are part of the hidden held-out test set. As a result, domain shift is represented by a change in $x_2$. Although the classifier should only infer based on $x_1$, traditional gradient descent leads to overfitting (top-left). The proposed method, \textit{GGA} (bottom-left), introduces an annealing process that depends on gradient agreement, leading to models that generalize well to new, unobserved target domains. (right) Schematics of the parameter updates of ERM (top-right) and GGA (bottom-right). Parameters updated via ERM are driven by gradient conflict, whereas GGA searches for a point where gradients align before continuing descending towards a minima.}\label{fig:concept}%
	\end{center}%
}]



\begin{abstract}

% Recent works to jointly reconstruct 3D human and object from a single RGB image, are mostly model-based, that fail to capture the fine details of the clothed human body and object surface. In this paper, we introduce ReCHOR, a novel, model-free, first-method to produce realistic clothed human-object reconstructions from a monocular view. This is extremely challenging due to human-object occlusions, diverse interactions and depth ambiguity, as it needs to infer both 3D spatial awareness and high resolution details. Our core idea is based on estimating neural implicit representations for human and object respectively by an attention-based neural implicit model that attends to pixel-aligned features from both the global human-object image for spatial awareness and  the local separate view of human and object images for high quality details. Additionally, the network is conditioned on semantic features from an initial estimated human-object pose prior and a generative diffusion model that inpaints occluded regions, thus enabling the retrieval of details from them.
% We also propose a synthetic dataset with rendered scenes of diverse, inter-occluded 3D human and object scans, to train our network. We evaluate our method on the synthetic and real world BEHAVE dataset. Our experiments show that our method outperforms the SOTA in achieving realistic clothed human-object reconstructions.
Recent approaches to jointly reconstruct 3D humans and objects from a single RGB image represent 3D shapes with template-based or coarse models, which fail to capture details of loose clothing on human bodies. In this paper, we introduce a novel implicit approach for jointly reconstructing realistic 3D clothed humans and objects from a monocular view. For the first time, we model both the human and the object with an implicit representation, allowing to capture more realistic details such as clothing. This task is extremely challenging due to human-object occlusions and the lack of 3D information in 2D images, often leading to poor detail reconstruction and depth ambiguity. To address these problems, we propose a novel attention-based neural implicit model that leverages image pixel alignment from both the input human-object image for a global understanding of the human-object scene and from local separate views of the human and object images to improve realism with, for example, clothing details. Additionally, the network is conditioned on semantic features derived from an estimated human-object pose prior, which provides 3D spatial information about the shared space of humans and objects. To handle human occlusion caused by objects, we use a generative diffusion model that inpaints the occluded regions, recovering otherwise lost details. For training and evaluation, we introduce a synthetic dataset featuring rendered scenes of inter-occluded 3D human scans and diverse objects. Extensive evaluation on both synthetic and real-world datasets demonstrates the superior quality of the proposed human-object reconstructions over competitive methods.
\end{abstract}
\section{Introduction}
\label{sec:intro}
% Image editing methods in diffusion models depend on user-defined control directions - users can unlock their creativity using these methods by specifying the desired manipulation through prompts~\cite{gandikota2023concept}, reference images~\cite{ruiz2022dreambooth, kumari2022customdiffusion, gal2022image, chen2024trainingfreeregionalpromptingdiffusion}, or attribute vectors~\cite{parmar2023zero,hertz2022prompt}. In this work, we ask a fundamentally different question: \emph{Can we automatically discover the underlying visual structure of a concept within diffusion model's knowledge?} %Rather than requiring user-specified controls, we aim to decompose the model's internal knowledge into meaningful directions.

% This question touches on a fundamental limitation in how we interact with diffusion models. Current control methods ~\cite{zhang2023addingconditionalcontroltexttoimage, gandikota2023concept, ye2023ipadaptertextcompatibleimage,ye2023ipadaptertextcompatibleimage, hertz2024stylealignedimagegeneration, li2023photomaker, shi2024instantbooth, chen2024trainingfreeregionalpromptingdiffusion} require users to specify their desired manipulations in advance, limiting interactive creativity. This contrasts with natural human artistic workflows, where creators dynamically explore creative ideas while jointly refining them toward meaningful artistic outcomes~\cite{hoffmann2016modeling}. This synergy between specification and exploration is not new to generative models. Early GAN architectures naturally developed disentangled latent spaces that enabled continuous\cite{harkonen2020ganspace,radford2015unsupervised, wu2021stylespace, shen2020interfacegan}, compositional control over generated images. Users could explore these spaces to discover interesting variations that would be difficult to describe in words~\cite{wu2021stylespace}, then combine them to achieve their creative goals~\cite{grabe2022towards}. 


% While diffusion models have largely superseded GANs in conditional image synthesis~\cite{dhariwal2021diffusion},  their underlying structure remains less understood. Diffusion models achieve remarkable diversity through high-dimensional latents, unlike GANs' compact latent spaces.  With a single prompt, diffusion models can generate radically different variations through different random initializations of input noise. We ask - Is it possible to discover interpretable structure within this vast space of variations?

Text-to-image diffusion models are capable of generating remarkable visual variations from a single prompt through different random initializations. However, this vast creative potential remains largely opaque to users---while we can generate diverse images, we lack understanding of the underlying structure of these variations. This presents a fundamental challenge: how can we discover and expose the latent visual capabilities encoded within these models?

\let\thefootnote\relax \footnote{$^{*}$Correspondence to \texttt{gandikota.ro@northeastern.edu}}

The challenge touches on a key limitation in how we interact with diffusion models today. Current control methods require users to explicitly specify their desired edits in advance through prompts~\cite{gandikota2023concept}, reference images~\cite{zhang2023addingconditionalcontroltexttoimage, chen2024trainingfreeregionalpromptingdiffusion, ruiz2022dreambooth,kumari2022customdiffusion, Ryu_lora, hu2021lora}, or attribute vectors~\cite{ye2023ipadaptertextcompatibleimage, hertz2024stylealignedimagegeneration, li2023photomaker, shi2024instantbooth,parmar2023zero,hertz2022prompt}. That contrasts sharply with natural human creative workflows, where artists dynamically explore creative ideas and jointly refine them toward meaningful artistic outcomes~\cite{hoffmann2016modeling}. The need for pre-specified controls creates a barrier between users and the full creative potential of these models.

Interestingly, earlier generative models like GANs~\cite{gans,karras2019style,brock2018large} naturally developed more interpretable internal structures. Their compact latent spaces often exhibited emergent disentanglement~\cite{harkonen2020ganspace,radford2015unsupervised, wu2021stylespace, shen2020interfacegan}, enabling continuous and compositional control over generated images. Users could explore these spaces to discover interesting variations that would be difficult to describe in words~\cite{wu2021stylespace}, then combine them to achieve their creative goals~\cite{grabe2022towards}.

Diffusion models have largely superseded GANs in conditional image synthesis~\cite{dhariwal2021diffusion}, achieving greater diversity through much higher-dimensional latents. And yet an understanding of the underlying structure of these larger latent spaces has remained elusive. In this work, we ask a fundamental question: \emph{Can we automatically discover the visual structure within a diffusion model's knowledge of a concept?} Rather than requiring user-specified controls, we aim to decompose the model's internal representations into expressive directions that users can explore and combine.

To address these needs, we present \textbf{SliderSpace}, a framework that brings systematic explorability to diffusion models. Given just a text prompt, SliderSpace discovers a canonical set of meaningful, diverse, and controllable directions within the model's knowledge of that concept. Each direction is implemented as a low-rank adapter~\cite{hu2021lora} that can be scaled and composed with others, allowing users to explore and smoothly combine different aspects of variation, as shown in Figure~\ref{fig:intro}.

We ground SliderSpace discovery in three key requirements for meaningful decomposition of a diffusion model's visual manifold: 
\begin{enumerate}
    \item \textbf{Unsupervised Discovery:} The decomposition process should emerge from the intrinsic structure of the model's learned representation, rather than being guided by predefined attributes. This ensures we capture the true topology of the model's knowledge space rather than projecting our assumptions onto it.
    
    \item \textbf{Semantic Orthogonality:} Each discovered control must represent a distinct semantic direction. This is enforced in a semantic feature space, like CLIP, where every slider has an orthogonal effect in embeddings. This prevents discovering multiple controls that create similar semantic effects, making the system more efficient and easier.
    
    \item \textbf{Distribution Consistency:} Directions must induce consistent transformations across both random seeds and prompt variations. 
\end{enumerate}

These requirements naturally lead to our proposed framework, which we formalize in Section~\ref{sec:method}. As we show in our experiments, SliderSpace is architecture-agnostic, working with both conventional U-Net based models like Stable Diffusion~\cite{rombach2022high, rombach2022sd20, podell2023sdxl, turbo, dmd} and recent transformer-based architectures like Flux~\cite{flux}.

We demonstrate the expressiveness of SliderSpace through three applications: First, we show how SliderSpace can decompose high-level concepts into diverse and expressive components, revealing the natural axes of variation in the model's understanding. Second, we explore artistic style variation, where SliderSpace discovers directions that match or exceed the diversity of manually curated artist lists while being judged more useful by human evaluators. Finally, we show how SliderSpace can help reverse the mode collapse commonly observed in distilled diffusion models, restoring diversity while maintaining generation speed.

Beyond providing practical creative control, SliderSpace opens new avenues for understanding and utilizing the latent capabilities of diffusion models. By mapping these models' visual potential into intuitive, composable directions, we take a step toward making their creative possibilities more accessible and interpretable to users.

% Image editing methods in diffusion models unlock the creativity of users. In this work we ask an alternate question: \emph{Can we organize and expose what of the diffusion model is already capable of?}.
% Existing methods for controlling image generation typically require users to manually specify edit directions for desired changes. This process is time-consuming, requires technical expertise, and limits the spontaneity of the creative process. For instance, if a user wants to adjust the smile of a generated person, they must explicitly request this edit, often through imprecise prompt engineering or model fine-tuning. This approach of predefined controls or manual specifications restricts users from fully exploring the latent capabilities of the model. There may be interesting stylistic variations or attributes that the model can generate, but users have no easy way to discover or utilize these.

% Natural visual disentanglement was an emergent property in the latent space of Generative Adversarial Models (GANs) \cite{harkonen2020ganspace,radford2015unsupervised, wu2021stylespace, shen2020interfacegan}. In particular, it has been observed that StyleGAN~\cite{karras2019style} stylespace neurons offer detailed control over many meaningful aspects of images that would be difficult to describe in words~\cite{wu2021stylespace}. However, diffusion models do not share such a compact latent space~\cite{park2023unsupervised}; and efforts to uncover such a space in the semantic embeddings of the text conditioning have met with limited success \nik{Nick - is there a specific citation you were thinking about?}.

% In this work we introduce \textbf{SliderSpace}, which takes a step towards uncovering an analogous low dimensional representation of diffusion models' visual breadth; in essence treating the diffusion model as many generators sharing parameters, where a particular generator is defined by a specific prompt. For a given prompt we sample many random seeds (and optionally prompt expansions using an LLM), generate the corresponding images, and apply an off the shelf feature extractor (in this work CLIP, but our method can be applied to any differentiable feature extractor). We use PCA to analyze these features, and for each of the leading $k$ principal components we train a LoRA \cite{} which causes the diffusion model to produces images which increase the feature magnitude along that component when passed back through the same feature extractor. This leads to a 'Slider' for each principal component, because each LoRA can be scaled and applied to the original diffusion model, continuously varying those visual features in the generated results (as measured, in our case, by CLIP).

% There are many other works that enhance the controllability of diffusion models. One common approach is enabling users to add spatial constraints to a generation either manually, or via a reference image \cite{zhang2023addingconditionalcontroltexttoimage, chen2024trainingfreeregionalpromptingdiffusion}, a second is leveraging more abstract embeddings (e.g. identity, style) extracted from a reference image \cite{ye2023ipadaptertextcompatibleimage, hertz2024stylealignedimagegeneration, li2023photomaker, shi2024instantbooth}, a third is finetuning a foundation model to better generate a concept important to the user \cite{ruiz2022dreambooth, kumari2022customdiffusion, Ryu_lora, hu2021lora}, and a fourth (most relevant to this work) is finding low-rank adaptors of the model based on a prompt or small training set which can be scaled to provide continous control over one aspect of generated image (e.g. night vs day, basic vs luxury, etc.) \cite{gandikota2023concept}. SliderSpace is complementary to all of these methods and offers something distinct. All of the other methods we are aware require the user (and / or model designer) to know in advance what type of control they want. In contrast SliderSpace assists users in discovering and controlling hidden capabilities present in the diffusion model's distribution of possible generations.

%We propose that truly intuitive creative control in a text-to-image model should meet three key criteria: \emph{discoverability}, \emph{intuitiveness}, and \emph{specificity}. The model should reveal controllable attributes that may not be immediately obvious, offer controls that are easy to understand and manipulate, and ensure each control affects a distinct attribute of the generated image.

% We demonstrate the utility and power of SliderSpace using three applications built on top of SDXL-DMD \cite{dmd}, because its fast generation speed lends itself well to the continuous control offered by SliderSpace.

% First, we study concept decomposition (Section \ref{sec:concept_exp}), where we learn sliders for a specific concept (e.g. 'monster', 'waterfall', 'car'). Through quantitative metrics of diversity and text alignment we demonstrate that the learned sliders dramatically boost the diversity of generations when randomly applied without harming text alignment; we also ask humans to qualitatively judge these results in a user study where they find the SliderSpace results to be more 'Diverse', 'Useful', and 'Creative' than our baselines.

% Second, we attempt to compare the automatic discoveries of SliderSpace to a large scale manual study of artistic styles (Section \ref{sec:art_exp}), open-sourced by ParrotZone \cite{parrotzone}. In this study SDXL was prompted with over 4300 artist names,  and based on visual inspection the cases of successful stylistic mimicry recorded. Quantitatively SliderSpace more closely matches the distribution of artistic variation discovered by ParrotZone than other baselines, and in our user studies was judged to be significantly more 'Diverse' and 'Useful' than the baselines. To our surprise humans even judged SliderSpace results to be slightly more 'Diverse' than the results generated by the manually discovered artist names of \cite{parrotzone}.

% Third, we attempt to use SliderSpace to reverse the mode collapse commonly observed in distilled few-step diffusion models relative to the original teacher model (Section \ref{sec:diverse_exp}). We quantitatively demonstrate that applying SliderSpace to SDXL-DMD leads to more closely matching the distribution of images by the original teacher, SDXL.

%Through extensive experiments on various state-of-the-art text-to-image models, we demonstrate that SliderSpace significantly enhances user control and creative expression in AI-assisted image generation tasks. Our method enables a range of applications, including concept decomposition and control, diversity improvement in generated images, customization dissection and edits, and the exploration of artistic styles inherent in the model.

% SliderSpace goes beyond providing a practical tool for enhanced creative control. By mapping the visual potential of diffusion models it can open new avenues for generative creativity and deepens our understanding of each model's hidden potential.
\section{Related Work}

\paragraph{LLMs for Agent tasks.}

Our research is related to deploying large language models (LLMs) as agents for decision-making tasks in interactive environments~\citep{liu2023agentbench,zhou2023webarena,shridhar2020alfred,toyama2021androidenv}. Earlier works, such as~\citep{yao2023webshopscalablerealworldweb}, fine-tuned models like BERT~\citep{devlin2019bertpretrainingdeepbidirectional} for decision-making in simplified environments, such as online shopping or mobile phone manipulation. With the advent of large language models~\citep{brown2020languagemodelsfewshotlearners,openai2024gpt4technicalreport}, it became feasible to perform decision-making tasks through zero-shot or few-shot in-context learning. To better assess the capabilities of LLMs as agents, several models have been developed~\citep{deng2024mind2web,xiong2024watch,hong2023cogagent,yan2023gpt}. Most approaches~\citep{zheng2024seeact,deng2024mind2web} provide the agent with observation and action history, and the language model predicts the next action via in-context learning. Additionally, some methods~\citep{zhang2023building,li2023camel,song2024trial} attempt to distill trajectories from state-of-the-art language models to train more effective policy models. In contrast, our paper introduces a novel framework that automatically learns a reward model from LLM agent navigation, using it to guide the agents in making more effective plans.

\textbf{LLM Planning.} Our paper is also related to planning with large language models. Early researchers~\citep{brown2020languagemodelsfewshotlearners} often prompted large language models to directly perform agent tasks. Later, \citet{yao2022react} proposed ReAct, which combined LLMs for action prediction with chain-of-thought prompting~\citep{wei2022chain}. Several other works~\citep{yao2023treethoughtsdeliberateproblem,hao2023reasoning,zhao2023large,qiao2024agentplanningworldknowledge} have focused on enhancing multi-step reasoning capabilities by integrating LLMs with tree search methods. Our model differs from these previous studies in several significant ways. First, rather than solely focusing on text generation tasks, our pipeline addresses multi-step action planning tasks in interactive environments, where we must consider not only historical input but also multimodal feedback from the environment. Additionally, our pipeline involves automatic learning of the reward model from the environment without relying on human-annotated data, whereas previous works rely on prompting-based frameworks that require large commercial LLMs like GPT-4~\citep{openai2024gpt4technicalreport} to learn action prediction. Furthermore, \Model supports a variety of planning algorithms beyond tree search.

\textbf{Learning from AI Feedback.} In contrast to prior work on LLM planning, our approach also draws on recent advances in learning from AI feedback~\citep{bai2022constitutional,lee2023rlaif,yuan2024self,sharma2024critical,pan2024autonomous,koh2024tree}. These studies initially prompt state-of-the-art large language models to generate text responses that adhere to predefined principles and then potentially fine-tune the LLMs with reinforcement learning. Like previous studies, we also prompt large language models to generate synthetic data. However, unlike them, we focus not on fine-tuning a better generative model but on developing a classification model that evaluates how well action trajectories fulfill the intended instructions. This approach is simpler, requires no reliance on state-of-the-art LLMs, and is more efficient. We also demonstrate that our learned reward model can integrate with various LLMs and planning algorithms, consistently improving their performance.

\textbf{Inference-Time Scaling.} ~\citet{snell2024scaling} validates the efficacy of inference-time scaling for language models. Based on inference-time scaling, various methods have been proposed, such as random sampling~\citep{wang2022self} and tree-search methods~\citep{hao2023reasoning, zhang2024accessing, guan2025rstar}. Concurrently, several works have also leveraged inference-time scaling to improve the performance of agentic tasks. ~\citet{koh2024tree} adopts a training-free approach, employing MCTS to enhance policy model performance during inference and prompting the LLM to return the reward. ~\citet{gu2024your} introduces a novel speculative reasoning approach to bypass irreversible actions by leveraging LLMs or VLMs. It also employs tree search to improve performance and prompts an LLM to output rewards. ~\citet{yu2024exact} proposes Reflective-MCTS to perform tree search and fine-tune the GPT model, leading to improvements in ~\citet{koh2024visualwebarena}. ~\citet{putta2024agent} also utilizes MCTS to enhance performance on web-based tasks such as ~\citet{yao2023webshopscalablerealworldweb} and real-world booking environments. ~\cite{lin2025qlass} utilizes the stepwise reward to give effective intermediate guidance across different agentic tasks. Our work differs from previous efforts in two key aspects: (1) Broader Application Domain. Unlike prior studies that primarily focus on tasks from a single domain, our method demonstrates strong generalizability across web agents, mathematical reasoning, and scientific discovery domains, further proving its effectiveness. (2) Flexible and Effective Reward Modeling. Instead of simply prompting an LLM as a reward model, we finetune a small scale VLM~\citep{lin2023vila} to evaluate input trajectories. %Our reward scores range continuously between 0 and 1, in contrast to existing methods that rely on discrete scoring (e.g., 0 and 1, or 0, 0.5, and 1) through direct LLM prompting.

% Concurrently, several works have also leveraged inference-time scaling to improve the performance of agentic tasks. ~\citet{pan2024autonomous} demonstrates that LLMs and VLMs, such as the GPT series, can function as evaluators or reward models to provide guidance for fine-tuning or reflection, thereby enhancing digital agents. This lays the groundwork for subsequent studies that directly prompt LLMs as reward models. ~\citet{koh2024tree} adopts a training-free approach, employing MCTS to enhance policy model performance during inference. However, it is limited to web environments~\citep{koh2024visualwebarena}. Moreover, its value function relies on prompting an LLM, which is less effective than our proposed method. We validate our approach through ablation studies, demonstrating that our fine-tuned reward model is more effective. ~\citet{gu2024your} introduces a novel speculative reasoning approach to bypass irreversible actions, such as purchasing a product, by leveraging LLMs or VLMs. It also employs tree search to improve performance, but it remains restricted to the web domain~\citep{koh2024visualwebarena, deng2024mind2web}. Additionally, it lacks reward modeling and instead prompts an LLM to output rewards. ~\citet{yu2024exact} proposes Reflective-MCTS to perform tree search and fine-tune the GPT model, leading to improvements in ~\citep{koh2024visualwebarena}. However, this work focuses solely on a single web agent task, and its reward modeling is derived from multi-agent debate, differing from our more effective and efficient reward modeling approach. ~\citet{putta2024agent} also utilizes MCTS to enhance performance, but it is limited to web-based tasks such as ~\citep{yao2023webshopscalablerealworldweb} and real-world booking environments.
% introduce PDDL domains
% why Gripper env as testing context
% motivation: comparing classical vs LLM planners
% - classical: PDDL solver fast-downward
% - LLM: gpt-4o
% explanation and refinement are two distinguishing features of LLM planners
% - how we demonstrate explanation and refinement in the study
We evaluate user trust in two planners over a set of planning problems and study the potential factors influencing user trust in the planners. In particular, we compare a language-model-based planner, denoted as an \emph{LLM Planner}, with a traditional graph-search-based planner, denoted as a \emph{PDDL Solver}. The PDDL Solver uses Fast Downwards \cite{fastdownward} as its underlying model, processing planning problems described in PDDL to generate an optimal solution. In comparison, the LLM Planner employs GPT-4o to interpret the planning problem and extract a solution generated by the language model. Unlike the PDDL Solver, the LLM Planner can reason through the planning problem, explain its proposed solution, and iteratively refine the solution based on external feedback. This study investigates how the correctness of solutions, the quality of explanations, and the refinement process influence user trust.

\subsection{Planning Problem}
% \begin{wrapfigure}{r}{0.4\textwidth}
% % \begin{figure}[t]
%     \centering
%     \includegraphics[width=\linewidth]{figures/problem-example.pdf}
%     \caption{A running example of a planning problem in our study.}
%     \Description{Planning Problem Example}
%     \label{fig: problem-example}
% % \end{figure}
% \end{wrapfigure}

We describe each planning problem in the \emph{Planning Domain Definition Language (PDDL)} and propose two planners to generate plans that solve the problem. We select the \emph{gripper} planning problems from the International Planning Competition \cite{IPC} for plan generation and evaluation. In a gripper planning problem, a robot moves balls between a set of rooms using two grippers. The objective is to create a plan for the robot to move the balls to the target rooms we defined. We present a few running examples of the gripper problem in Figure \ref{fig: correctness}.

A planning problem consists of a \emph{planning domain} and a \emph{problem description}, expressed in PDDL. 

\paragraph{Planning Domain}
A planning domain refers to the universal aspects of a problem that remains consistent across different instances of the problem. In particular, it defines the types of objects, predicates, and actions that exist in the planning problem. We present an example of the gripper problem in Appendix \ref{app: grippers}.

\paragraph{Problem Description} A problem description specifies the particular instance of a planning task within a given domain. It includes the planning domain to which it pertains, a set of objects, the initial state of these objects, and the goal state to be achieved.

\paragraph{Plan}
A plan is a sequence of actions with specific input parameters. Recall that an action corresponds to a state transition. If a plan (a sequence of actions) transits from the initial state to the goal state defined by a problem, then we consider the plan to be \emph{correct}. If a plan does not transit to the goal state or there exists an action violating its precondition, then the plan is \emph{wrong}.

\begin{figure}[t]
    \centering
    \includegraphics[width=0.8\linewidth]{figures/correct.jpeg}
    \caption{Examples where LLM Planner correctly generates a plan for the gripper planning problem.}
    \Description{Planning Problem Correctness}
    \label{fig: correct}
\end{figure}

\subsection{PDDL Solver}
The PDDL Solver takes the planning domain and the problem description as inputs and then generates a plan described in PDDL. 
% It generates a plan in the following format:
% \vspace{4pt}
% \begin{lstlisting}[language=completion]
% (move robot1 room1 room3)
% (pick robot1 ball2 room3 rgripper1)
% (move robot1 room3 room2) ......
% \end{lstlisting}
Next, we convert the generated plan into natural language for user studies following the procedure in \cite{seipp-et-al-zenodo2022} and display it to users. We present an example in Figure \ref{fig: correct}.

The PDDL Solver applies a graph search algorithm to find a path (i.e., a list of transitions) from the initial state to the goal state. It either generates a \emph{correct} plan---defined as the shortest path between the initial and goal states---or returns a signal indicating that no solution exists for the given problem.

\subsection{LLM Planner}

The LLM Planner addresses planning problems by querying a large language model. In particular, it transmits the planning domain and problem description to the language model using a structured prompt format. The planner then retrieves a natural language plan from the language model. We use GPT-4o as the language model for the planner. To ensure the output adheres to the desired format, we include a few in-context examples within the prompts.

A language model solves a planning problem by interpreting the domain and problem descriptions, simulating state transitions, and generating a sequence of actions to achieve the goal. While effective for reasoning and plan generation, language models may struggle with large state spaces. Unlike the PDDL Solver, the LLM Planner may generate \emph{incorrect} plans that violate the problem specifications (e.g., preconditions of actions) or fail to achieve the goal.

\subsection{Explanation and Refinement}
Alongside the generated plans, we offer detailed explanations of all the plans and revisions of any incorrect plans. This study examines how these explanations and refinements influence human trust in the two planners.

\paragraph{LLM Planner with Explanation (LLM+Expl)}
For each generated plan, we manually provide a natural language explanation. This explanation includes an assessment of the plan’s correctness, identification of any violations of action preconditions, and an analysis of inconsistencies between the final state achieved and the intended goal state. We present examples of explanations in Figure \ref{fig: explain} in Appendix.

In particular, if a plan is correct, the explanation is simply ``the plan successfully satisfies the goal conditions.'' 
If a plan is incorrect, we identify the underlying cause as either a violation of action preconditions or a failure to achieve the goal state. In cases involving precondition violations, we specify the action responsible for the issue. For example, consider the action ``robot moves from room 1 to room 2,'' but the robot is initially located in room 3. This scenario constitutes a violation of the precondition for the ``move'' action. In the latter case, we describe the differences between the final state achieved and the intended goal state, e.g., ``fail to move ball 2 to room 2.''

% \begin{wrapfigure}{r}{0.5\textwidth}
%     \centering
%     \includegraphics[width=0.98\linewidth]{figures/refine.jpeg}
%     \includegraphics[width=0.98\linewidth]{figures/refine-correct.jpeg}
%     \includegraphics[width=0.98\linewidth]{figures/refine-wrong.jpeg}
%     \caption{Plan refinement by the LLM Planner. The top row presents two choices of plan refinement (where the refinement starts). The second and third row shows the refinement outcomes of the two choices, where the second row shows a correctly refined plan and the third row shows an incorrect plan.}
%     \Description{Refinement}
%     \label{fig: refine}
% \end{wrapfigure}

\paragraph{LLM Planner with Refinement (LLM+Refine)}
Note that a plan generated by the LLM Planner could be incorrect. Therefore, we offer a prompting mechanism for the LLM Planner to refine the generated plan according to the user feedback. The mechanism works as follows:

1. Request the user to indicate the step number of the first action in the plan that is incorrect, such as the step where an action’s precondition is violated. We present a sample user interface on the left of Figure \ref{fig: refine} in Appendix.

2. Send the planning domain, problem description, and the original plan to the language model. Then, query the model to rewrite the subsequent steps starting from the user-specified step number. We present a sample input prompt in Figure \ref{fig: refine-prompt} in the Appendix.

3. Replace the original plan with the newly refined plan and display it to the user.

This mechanism allows users to interact with the language model to refine the plan. It enables the language model to focus on a subset of steps, facilitating a deeper interpretation of the incorrect component. However, the correctness of the refined plan is not guaranteed. Figure \ref{fig: refine} in the Appendix shows an example of a correctly refined plan and an incorrectly refined plan.

\begin{table*}[h]
	\centering
	\small
	\caption{{\bf Comparison of \textbf{GGA} with the ERM baseline}. The top out-of-domain accuracies on five domain generalization benchmarks averaged over three trials, are presented.}
	\label{table:baseline-results}
	% \vspace{-0.5em}
	%	\renewcommand{\arraystretch}{1.1}
	%	\setlength{\tabcolsep}{3pt}
	%	\setlength{\abovetopsep}{0.5em}
	\begin{tabular}{l|ccccc|c}
		\toprule
		Algorithm & PACS & VLCS & OfficeHome & {TerraInc} & {DomainNet} & {Avg.}  \\
		\midrule
		ERM  & %\cite{vapnik1998statistical}   & 
		85.5\scriptsize$\pm0.2$             & 
		77.3\scriptsize$\pm0.4$             & 
		66.5\scriptsize$\pm0.3$             & 
		46.1\scriptsize$\pm1.8$             & 
		43.8\scriptsize$\pm0.3$             & 63.9  \\
		\midrule
		\textbf{GGA} (ours)                 & 
		\textbf{87.3}\scriptsize{$\pm0.4$}           & 
		\textbf{79.9}\scriptsize{$\pm0.4$}           & 
		\textbf{68.5}\scriptsize{$\pm0.2$}           &  
		\textbf{50.6}\scriptsize{$\pm0.1$}           & 
		\textbf{45.2}\scriptsize{$\pm0.2$}           & \textbf{66.3}  \\
		%		\midrule
		
		
		\bottomrule
		
	\end{tabular}
	% \vspace{-0.5em}
\end{table*}


\begin{table*}[h]
	\centering
	\small
	\caption{\textbf{Comparison with state-of-the-art domain generalization methods.} Out-of-domain accuracies on five domain generalization benchmarks are shown. Top performing methods are highlighted in \textbf{bold} while second-best are \textit{underlined}.
		The results marked by $\dagger, \ddagger$ are copied from Gulrajani and Lopez-Paz \cite{gulrajani2020domainbed} and Wang et al. \cite{wang2023sharpness}, respectively. For fair comparison, the training of each algorithm combined with \textbf{GGA}, were run on the respective codebases. Average accuracies and standard errors are calculated from three trials for the combination of GGA with past algorithms and from 5 trials for GGA. In {\color{green} green} and {\color{red} red}, we highlight the performance boost and decrease of applying GGA on top of each algorithm respectively, averaged over three trials. Due to computational resources, for DomainNet we do not combine GGA with previous methods.}
	\label{table:total-results}
	% \vspace{-0.5em}
	\renewcommand{\arraystretch}{1.1}
	\setlength{\tabcolsep}{3pt}
	\setlength{\abovetopsep}{0.5em}
	\begin{tabular}{l|cccc|c||c|c}
		\toprule
		Algorithm & PACS          & VLCS          & OfficeHome    & {TerraInc}    & {Avg.} & DomainNet & Total \\
		\midrule
		
		Mixstyle$^\ddagger$ \cite{zhou2021mixstyle}     & 
		85.2\scriptsize{$\pm0.3$} 
		\small{({\color{green}$+0.3$})}         & 
		77.9\scriptsize{$\pm0.5$} \small{({\color{green}$+0.6$})}          & 60.4\scriptsize{$\pm0.3$} \small{({\color{green}$+0.5$})}          & 44.0\scriptsize{$\pm0.7$}
		\small{({\color{green}$+1.1$})}          & 
		%		
		66.9 \small{({\color{green}$+0.6$})}         &
		34.0\scriptsize{$\pm0.1$}           & 60.3 \\
		
		GroupDRO$^\ddagger$ \cite{Sagawa2020GroupDRO}    & 
		84.4\scriptsize{$\pm0.8$}          
		\small{({\color{green}$+1.4$})}   & 
		76.7\scriptsize{$\pm0.6$}          
		\small{({\color{green}$+0.6$})}   & 
		66.0\scriptsize{$\pm0.7$}             
		\small{({\color{green}$+2.2$})}   & 
		43.2\scriptsize{$\pm1.1$}              
		\small{({\color{green}$+1.5$})}   & 
		%		 
		67.6 \small{({\color{green}$+1.4$})}         &
		33.3\scriptsize{$\pm0.2$}          & 60.7 \\
		
		MMD$^\ddagger$ \cite{li2018mmd}                  & 
		84.7\scriptsize{$\pm0.5$}          
		\small{({\color{green}$+0.8$})} & 
		77.5\scriptsize{$\pm0.9$}            
		\small{({\color{green}$+1.3$})} & 
		66.3\scriptsize{$\pm0.1$}          
		\small{({\color{green}$+1.6$})} & 
		42.2\scriptsize{$\pm1.6$}            
		\small{({\color{green}$+6.3$})} & 
		%		 
		67.7 \small{({\color{green}$+2.5$})} &      
		23.4\scriptsize{$\pm9.5$}           &  58.8 \\
		
		AND-mask \cite{shahtalebi2021sand} & 
		84.4\scriptsize{$\pm0.9$}  
		\small{({\color{green}$+0.1$})}& 
		78.1\scriptsize{$\pm0.9$}  
		\small{({\color{green}$+0.3$})}& 
		65.6\scriptsize{$\pm0.4$}  
		\small{({\color{green}$+1.2$})} & 
		44.6\scriptsize{$\pm0.3$} 
		\small{({\color{red}$-0.4$})}
		& 68.2 \small{({\color{green}$+0.3$})}    &
		37.2\scriptsize{$\pm0.6$} & 62.0 \\
		
		ARM$^\ddagger$ \cite{zhang2020arm}               &
		85.1\scriptsize{$\pm0.4$}          
		\small{({\color{green}$+0.5$})}& 
		77.6\scriptsize{$\pm0.3$} 
		\small{({\color{green}$+0.9$})}           & 
		64.8\scriptsize{$\pm0.3$} 
		\small{({\color{green}$+2.0$})}           & 
		45.5\scriptsize{$\pm0.3$} 
		\small{({\color{green}$+0.8$})}           & 
		%		 
		68.3 \small{({\color{green}$+1.1$})}         &
		35.5\scriptsize{$\pm0.2$}          & 61.7 \\
		
		IRM$^\dagger$ \cite{arjovsky2019irm}            & 
		83.5\scriptsize{$\pm0.8$}          
		\small{({\color{red}$-1.5$})}& 
		78.5\scriptsize{$\pm0.5$}          
		\small{({\color{red}$-0.9$})} & 
		64.3\scriptsize{$\pm2.2$}          
		\small{({\color{red}$-2.1$})}& 
		47.6\scriptsize{$\pm0.8$}          
		\small{({\color{red}$-3.9$})} & 
		%		 
		68.5 \small{({\color{red}$-2.1$})}        &
		33.9\scriptsize{$\pm2.8$}          & 61.6 \\
		
		MTL$^\ddagger$ \cite{blanchard2021mtl_marginal_transfer_learning}    & 
		84.6\scriptsize{$\pm0.5$}           
		\small{({\color{green}$+0.9$})} & 
		77.2\scriptsize{$\pm0.4$}          
		\small{({\color{green}$+2.0$})} & 
		66.4\scriptsize{$\pm0.5$}          
		\small{({\color{green}$+0.4$})} & 
		45.6\scriptsize{$\pm1.2$}          
		\small{({\color{green}$+0.6$})}&  
		%		 
		68.5 \small{({\color{green}$+0.9$})}        &
		40.6\scriptsize{$\pm0.1$}          & 62.9 \\
		
		
		VREx$^\ddagger$ \cite{krueger2020vrex}           & 
		84.9\scriptsize{$\pm0.6$} 
		\small{({\color{green}$+0.6$})}           & 
		78.3\scriptsize{$\pm0.2$} 
		\small{({\color{green}$+0.1$})}           & 
		66.4\scriptsize{$\pm0.6$} 
		\small{({\color{green}$+1.3$})}           & 
		46.4\scriptsize{$\pm0.6$} 
		\small{({\color{green}$+2.0$})}           & 
		%		 
		69.0 \small{({\color{green}$+1.0$})}      &
		33.6\scriptsize{$\pm2.9$}          & 61.9   \\
		
		MLDG$^\dagger$ \cite{li2018learning}                & 
		84.9\scriptsize{$\pm1.0$}          
		\small{({\color{green}$+0.7$})} &
		77.2\scriptsize{$\pm0.4$}          
		\small{({\color{green}$+1.3$})} & 
		66.8\scriptsize{$\pm0.6$}          
		\small{({\color{green}$+1.2$})} &
		47.7\scriptsize{$\pm0.2$}          
		\small{({\color{green}$+1.1$})} & 
		%		 
		69.2 \small{({\color{green}$+1.2$})}    &
		41.2\scriptsize{$\pm0.1$}            & 63.6  \\
		
		Mixup$^\dagger$ \cite{xu2020interdomain_mixup_aaai}             & 
		84.6\scriptsize{$\pm0.6$}            
		\small{({\color{green}$+1.2$})} & 
		77.4\scriptsize{$\pm0.6$}          
		\small{({\color{green}$+1.8$})} & 
		68.1\scriptsize{$\pm0.3$}            
		\small{({\color{green}$+1.0$})} & 
		47.9\scriptsize{$\pm0.8$}            
		\small{({\color{green}$+2.0$})} & 
		%		
		69.5   \small{({\color{green}$+1.5$})}    &
		39.2\scriptsize{$\pm0.1$}            & 63.4    \\
		
		
		SagNet$^\dagger$ \cite{nam2019sagnet}           &             
		86.3\scriptsize{$\pm0.2$} 
		\small{({\color{red}$-1.0$})} & 
		77.8\scriptsize{$\pm0.5$}          
		\small{({\color{green}$+0.9$})} & 
		68.1\scriptsize{$\pm0.1$}          
		\small{({\color{green}$+0.3$})}& 
		48.6\scriptsize{$\pm1.0$}          
		\small{({\color{green}$+0.7$})}& 
		%		 
		70.2 \small{({\color{green}$+0.4$})}        &
		40.3\scriptsize{$\pm0.1$}          &  64.2 \\
		
		
		CORAL$^\dagger$ \cite{sun2016coral}             & 
		86.2\scriptsize{$\pm0.3$}          
		\small{({\color{green}$+0.7$})}& 78.8\scriptsize{$\pm0.6$}          
		\small{({\color{red}$-0.4$})} & 
		68.7\scriptsize{$\pm0.3$}          
		\small{({\color{green}$+0.2$})} & 
		47.6\scriptsize{$\pm1.0$}          
		\small{({\color{green}$+0.3$})}& 
		%		 
		70.3    \small{({\color{green}$+0.2$})}    &
		41.5\scriptsize{$\pm0.1$}          & 64.5   \\
		
		\midrule
		
		RSC$^\dagger$ \cite{huang2020rsc}               & 
		85.2\scriptsize{$\pm0.9$}          
		\small{({\color{green}$+0.1$})}& 77.1\scriptsize{$\pm0.5$}          
		\small{({\color{green}$+0.2$})}& 
		65.5\scriptsize{$\pm0.9$} \small{($+0.0$)}        & 
		46.6\scriptsize{$\pm1.0$}          
		\small{({\color{green}$+0.2$})}& 
		%		 
		68.6 \small{({\color{green}$+0.1$})}   &
		38.9\scriptsize{$\pm0.5$}          & 62.7 \\
		
		Fish $^\ddagger$ \cite{shi2021gradient}                     & 
		85.5\scriptsize{$\pm0.3$}          
		\small{({\color{green}$+0.1$})} & 
		77.8\scriptsize{$\pm0.3$}          
		\small{({\color{green}$+0.9$})} & 
		68.6\scriptsize{$\pm0.4$}            
		\small{({\color{red}$-0.6$})} & 
		45.1\scriptsize{$\pm1.3$}            
		\small{({\color{green}$+3.8$})} & 
		%		 
		69.3 \small{({\color{green}$+1.0$})}      &
		42.7\scriptsize{$\pm0.2$}            &   63.9    \\ 
		
		SAM $^\ddagger$ \cite{foret2020sharpness}  & 
		85.8\scriptsize$\pm0.2$             
		\small{({\color{green}$+0.6$})} & 
		79.4\scriptsize$\pm0.1$              
		\small{({\color{green}$+0.7$})} & 
		\underline{69.6}\scriptsize$\pm0.1$              
		\small{({\color{green}$+0.4$})} & 
		43.3\scriptsize$\pm0.7$              
		\small{({\color{green}$+2.6$})} & 
		%		 
		69.5 \small{({\color{green}$+1.1$})}   &
		44.3\scriptsize$\pm0.0$              &  64.5 \\
		
		GSAM $^\ddagger$ \cite{zhuang2022surrogate}  & 
		85.9\scriptsize$\pm0.1$           
		\small{({\color{green}$+0.4$})} & 
		79.1\scriptsize$\pm0.2$             
		\small{({\color{green}$+1.0$})} & 
		69.3\scriptsize$\pm0.0$             
		\small{({\color{green}$+0.3$})}& 
		47.0\scriptsize$\pm0.8$             
		\small{({\color{green}$+0.6$})}& 
		%		 
		70.3  \small{({\color{green}$+0.2$})}  &
		44.6\scriptsize$\pm0.2$             &   65.1 \\
		
		SAGM $^\ddagger$ \cite{wang2023sharpness}       & 
		\underline{86.6}\scriptsize{$\pm0.2$}           
		\small{({\color{green}$+0.2$})}& 
		\textbf{80.0}\scriptsize{$\pm0.3$}           
		\small{({\color{red}$-0.3$})}& 
		\textbf{70.1}\scriptsize{$\pm0.2$}           
		\small{({\color{red}$-0.6$})}& 
		\underline{48.8}\scriptsize{$\pm0.9$}           
		\small{({\color{red}$-0.1$})}& 
		%		 
		\underline{71.4} \small{({\color{red}$-0.2$})}  &
		45.0\scriptsize{$\pm0.2$}           &  66.1 \\
		
		%		\midrule
		\midrule
		\textbf{GGA} (ours)                 & 
		\textbf{87.3}\scriptsize{$\pm0.4$}           & 
		\underline{79.9}\scriptsize{$\pm0.4$}           & 
		68.5\scriptsize{$\pm0.2$}           &  
		\textbf{50.6}\scriptsize{$\pm0.1$}    &  \textbf{71.7} &
		\textbf{45.2}\scriptsize{$\pm0.2$} & \textbf{66.3}\\
		%		\midrule
		
		
		\bottomrule
		
	\end{tabular}
	% \vspace{-0.5em}
\end{table*}


\section{Experiments}
\label{sec:experiments}

\subsection{Experimental setup and implementation details}

In our experiments, we follow the protocol of DomainBed 
\cite{gulrajani2020domainbed} and exhaustively evaluate our algorithm against state-of-the-art algorithms on five DG benchmarks, using the same dataset splits and model selection. For the hyperparameter search space, we follow \cite{cha2021swad} in order to avoid the high computational burden of DomainBed. The datasets included in the benchmarks are, PACS \cite{li2017deeper} (9,991 images, 7 classes, and 4 domains), VLCS \cite{fang2013unbiased} (10,729 images, 5 classes, and 4 domains), OfficeHome \cite{venkateswara2017deep} (15,588 images, 65 classes, and 4 domains), TerraIncognita \cite{beery2018recognition} (24,788 images, 10 classes, and 4 domains) and DomainNet \cite{peng2019moment} (586,575 images, 345 classes,
and 6 domains).

In all experiments, the \textit{leave-one-domain-out} cross-validation protocol is followed. Specifically, in each run a single domain is left out 
as the target (test) domain, while the rest of the domains are used for 
training. The final performance of each algorithm is calculated by averaging the top-1 accuracy on the target domain, with different train-validation splits. For training, we utilize a ResNet-50 
\cite{he2016deep} pretrained on ImageNet \cite{russakovsky2015imagenet} for the backbone feature extractor and ADAM for the optimizer. Regarding GGA, 
during training we let each algorithm run for several training iterations depending on the dataset and then begin the parameter space 
search, as described in Section \ref{sec:methods}. For the neighborhood size
in the search process, we set $\rho$ to $0.0005$\footnote{The selection of $\rho$ was based on previous algorithms that implement weight perturbations in ResNet-50 networks \cite{foret2020sharpness, wang2023sharpness, le2024gradient}. The sensitivity analysis presented in subsection \ref*{sensitivity} also reveals that $\rho = 0.0005$ yields the best performance.} and search for a total of $A = 250$ steps before moving to the next mini-batch of 48 samples from each domain, in each dataset. In all experiments, we perform search iterations for $100$ different mini-batches during early training stages. The rest of the hyperparameters, such as learning rate, weight decay and dropout rate, are tuned according to \cite{cha2021swad} and are presented in Table \ref{table:hyperparameters}.
To account for the variability introduced in the random search of GGA, we 
repeat the experiments with 5 different seeds for each dataset. All models were trained on a cluster containing $4\times40$GB NVIDIA A$100$ GPU cards, split into $8$ $20$GB virtual MIG devices and $1\times24$GB NVIDIA RTX A$5000$ GPU card.

\begin{table}[h]
	\centering
	%\footnotesize
		\resizebox{\linewidth}{!}{
		\begin{tabular}{lrrrrr} 
			\toprule
			\textbf{Hyperparameter}& PACS & VLCS  & OH & TI & DN\\ \midrule
			Learning rate&{3e-5}&{1e-5}&{1e-5}& 1e-5 & 3e-5 \\ 
			Dropout & 0.5 & 0.5 & 0.5 & 0.5 & 0.5 \\ 
			Weight decay&1e-4 & 1e-4&1e-4 & 1e-4 & 1e-4 \\ 
			Training Steps & 5000 & 5000 & 5000 & 5000 & 15000 \\
			$\rho$ & {5e-5} & {5e-5} & {5e-5} & {5e-5} & 5e-5 \\
			GGA Start-End & 100-150, 1500-1550 & 100-200 & 100-200 & 500-600 & 100-200 \\ 
			\bottomrule
		\end{tabular}
				}
	\caption{Hyperparameters for DG experiments. OH, TI and DN stand for OfficeHome, TerraIncognita and DomainNet respectively. $\rho$ is the parameter space search used during the annealing steps in GGA, while the last row indicates the training iterations during which GGA occurs.}
	\label{table:hyperparameters}
\end{table}

\subsection{Comparative Evaluation}
\label{results}

The average OOD performances of the baseline vanilla ERM \cite{vapnik1998statistical} and state-of-the-art DG methods on a total of 5 DG benchmarks, are reported in Tables \ref{table:baseline-results} and \ref{table:total-results} respectively. The results for each separate domain in each benchmark are reported in the Appendix. In the experiments, we apply GGA on-top of the rest of the DG algorithms and show that its properties generalize to other methods as well, boosting their overall performance. It should also be noted that GGA can be adapted to training scenarios where distributions shifts are present in training data. 

Initially, to validate whether the application of GGA in the early stages of 
model training leads to models with increased generalizabilty, we compare the 
results with the vanilla ERM baseline. As presented in Table \ref{table:baseline-results}, GGA is able to boost the 
performance of the baseline model by an average of 2.4\% on all 
benchmarks, demonstrating the efficacy of the proposed algorithm. For further evaluation, we also compare GGA with state-of-the-art DG methods \cite{li2018mmd, zhou2021mixstyle, Sagawa2020GroupDRO, arjovsky2019irm, zhang2020arm, krueger2020vrex, shahtalebi2021sand, ganin2016dann, huang2020rsc, blanchard2021mtl_marginal_transfer_learning, xu2020interdomain_mixup_aaai, li2018learning, shi2021gradient, nam2019sagnet, sun2016coral, foret2020sharpness, zhuang2022surrogate, wang2023sharpness, vapnik1998statistical}, before applying GGA to them as well. We note that we only include previous works that have implemented a ResNet-50 as the backbone encoder and do not use additional components or ensembles. In Table \ref{table:total-results} we differentiate between the methods that operate 
on model gradients \cite{huang2020rsc, shi2021gradient, foret2020sharpness, zhuang2022surrogate, wang2023sharpness} and the rest of the algorithms. Even without its application to other methods, GGA is able to surpass most of the previously proposed algorithms, while also setting the state-of-the-art on PACS ($+0.7\%$) and on the challenging TerraIncognita ($+1.8$) dataset, while remaining competitive in the other two. What's more important is that the application of GGA in conjunction with the rest of the DG methods, proves beneficial and ultimately boosts their overall performance in almost each case by an average of around $1\%$ and in some cases up to even $6.3\%$. With regards to IRM, which seems to be significantly impacted negatively by the application of GGA, its learning objective emphasizes on simultaneously minimizing the training loss of each source domain and not necessarily on the pairwise agreement of gradients among domains. It is therefore not able to converge to a good solution after the initial model's weights have been perturbed during training.

From the experimental results, it is evident that the application of GGA 
and its search for parameter values where gradients align between domains is beneficial to model training. By introducing the proposed annealing step before the final stages of  training, the majority of the models exceed their previous performance and exhibit improved generalization capabilities.

\begin{figure*}[t]
	\centering
	\includegraphics[width=\textwidth]{vlcs-grads.png}
	\caption{Impact of GGA on gradient alignment during model training on the VLCS dataset. As illustrated, in the case of vanilla ERM, even though the training loss is minimized, the average cosine similarity among domain gradients remains low. In the case of GGA however, after the algorithm searches for points in the parameter space with increased gradient similarity, the gradients continue to agree during training, while the total training loss is also minimized.}
	\label{vlcs-grads}
\end{figure*}

\subsection{Evaluating the impact of GGA on gradient disagreement}
\label{gga-evaluation}

As discussed in Section \ref{sec:methods}, when a training dataset is composed
as a mixture of multiple domains, conflicting gradients between mini-batches drawn
from each domain lead to models that do not infer based on domain-invariant
features and which generalize to previously unseen data samples, but are hindered
by domain-specific, spurious correlations. This is evident in the case of
vanilla models trained via ERM where the average gradient similarity among
domains continues to remain low upon reaching a local minima. Our hypothesis is
that this behavior can be avoided by searching for a parameter set of common
agreement between domains before optimizing via gradient descent.

To demonstrate the operation of the proposed algorithm in practice against ERM, we calculate the average gradient cosine similarity between mini-batches from source domains during training for the VLCS dataset, along 
with the training loss in each iteration. As a result, each sub-figure in Figure
\ref{vlcs-grads} illustrates the progression of the training gradient alignment
between domains, against the total training loss. 

As expected, in the very initial iterations the gradients of the pretrained
model parameters point towards a common direction. However, in the case of ERM
as training progresses and the loss is minimized, the domain gradients begin to
disagree leading the model to converge to undesirable minima that do not
generalize across domains. On the other hand, when GGA is applied the model
searches for parameters such that gradients are aligned before continuing
training. This is illustrated by the spike in gradient similarity,
during iterations $100$ up to $200$. After GGA concludes, we observe that the
model continues training by descending into minima where gradients agree among
domains.

\subsection{Sensitivity Analysis}
\label{sensitivity}

\begin{figure}[t]
	\centering
%	% First figure
%	\begin{subfigure}{0.3\textwidth}
%		\centering
%		\includegraphics[width=\textwidth]{gradient-conflicts-oh.png}
%		\caption{Gradient conflicts - ERM}
%	\end{subfigure}%
%	\hfill
	% Second figure
	\begin{subfigure}{0.5\columnwidth}
		\centering
		\includegraphics[width=\textwidth]{rho-ablation.png}
%		\caption{Sensitivity analysis of $\rho$}
	\end{subfigure}%
	\hfill
	% Third figure
	\begin{subfigure}{0.5\columnwidth}
		\centering
		\includegraphics[width=\textwidth]{init-ablation.png}
%		\caption{GGA initialization analysis}
	\end{subfigure}
\caption{Sensitivity analysis of the parameter space search magnitude $\rho$ and the training stage application of GGA. The analysis on PACS reveals smaller weight perturbations and gradient annealing during early training iterations lead to increased model performance.}
\label{sensitivity-fig}
\end{figure}

The two core parameters of GGA, are the size of the parameter search $\rho$ and
the moment of our methods implementation during training, i.e., during the
early, mid or late training stages. To justify their selection, we conduct a
sensitivity analysis (Fig. \ref{sensitivity-fig}) by varying one of the above
parameters while fixing the other at its optimal value.

Regarding the magnitude of weight perturbations during the application of GGA,
we found that the optimal value was $\rho = 5e-5$. As illustrated in Figure
\ref{sensitivity-fig}, a larger magnitude of perturbation led to decreased model
performance. Intuitively, the application of larger noise to the model
parameters leads to sets that are not close to the solution, making it
increasingly difficult for the model to converge. On the other hand, smaller
perturbations seem to have little to no effect on training, as the search is
limited to spaces near the current parameters, which is why the model
performance falls back close to that of ERM. With regards to the stage of
training during which GGA will be applied, we found that the models yielded
better performance when the search was initialized in earlier stages.We hypothesize that applying perturbations near the end of training displaces the model from a local optimum, requiring additional iterations to converge.

\section{Limitations} 

In this work, we compared the effectiveness and interplay of SFT and RL-based methods, under fixed data constraints. In particular, we chose offline methods like DPO and KTO as the baseline implementation of the RL method because it eliminates the need for reward modeling or iterative finetuning. This means that the process of development is limited to collecting an offline dataset and fientuning it - making it the most fair comparable to SFT in terms of implementation effort, compute costs and annotation efforts. Since this baseline RL method shows optimal performance over SFT, we hope that this motivates future work to study more complex RL-based methods and their interplay with SFT. In addition, we used GPT4o annotation for synthetic data generation, and also for evaluating Summarization and Helpfulness, which could include potential biases inherited from the model. 

In addition, we limited the size of the model to under 10 Billion parameters, to keep the finetuning cost low enough to ignore as compared to the data annotation costs. In addition, it would be extremely compute resource intensive to run thousands of finetuning runs with larger model sizes like 70B parameters. We hope that future work would study the scaling trends of RL-based methods against different model sizes, and also study the compute-data trade-off in-depth.


\section{Limitations}

While our approach is typically more general than current training-based approaches, it still has limitations. One limitation arises from surprising entanglements in the CLIP and diffusion feature spaces. For example, when attempting to combine a zebra's body with a leopard fur pattern (\cref{fig:limitations} (top)), the diffusion model tends to produce animals with the head of a giraffe, even though no giraffe appears in either input image. We hypothesize that this may be related to the tendency of diffusion models to represent some concepts as a composition of more basic visual components~\citep{chefer2023hidden}, but leave further investigation to future work.

On the other hand, some concepts may be \textit{more} disentangled in CLIP-space than intuitively expected. For example, outfit types and colors are disentangled in CLIP-space, hence, an ``outfit'' subspace spanned with descriptions of different types of outfits (``dress'', ``tuxedo''...) will not preserve outfit colors (\cref{fig:limitations} (bottom)). However, this can be easily amended by also specifying colors in the spanning texts (``\textit{red} dress'', ``\textit{blue} tuxedo''...).



Finally, we note that IP-Adapter itself is limited in the level of detail captured from the input image. Hence, our approach will not be sufficient for capturing delicate details such as exact identities. Stronger encoders may achieve higher fidelity, but it is not clear that our embedding-space projections would generalize to more complex feature spaces.

\section{Conclusions}

We presented IP-Composer, a training-free method that allows a user to compose novel images from visual concepts derived through a set of input images. To do so, our approach uses a CLIP-based IP-Adapter, leveraging their joint disentangled subspace structure. Through this approach, we achieve comparable or better performance compared with existing training-based methods, and can more easily generalize to novel concepts derived solely from textual descriptions. 

We hope that our work can serve as an additional component of the creative toolbox, and open the way to additional composable-concept discovery methods. 

\section{Acknowledgment}
We would like to thank Ron Mokady and Yoad Tewel for providing feedback and helpful suggestions.



{
    \small
    \bibliographystyle{ieeenat_fullname}
    \bibliography{main}
}

\subfile{supp/supplementary.tex}

% WARNING: do not forget to delete the supplementary pages from your submission 
% 
\clearpage
% \setcounter{page}{1}
% \maketitlesupplementary
\begin{center}
Supplementary Material
\end{center}

% {
%     \onecolumn
%     \centering
%     \Large
%     \textbf{\thetitle}\\
%     \vspace{0.5em}Supplementary Material \\
%     \vspace{1.0em}
% }

\section{Proof of \cref{theorem:dr}}
We require some additional regularity assumptions:
\begin{assumption} 1) The number of classes $C$ is bounded w.r.t the number of samples $N$, 2) the missingness mechanism $P(A=1|Y,\theta)$, as well as its estimated counterpart $P(A=1|Y,\theta)$, are bounded below by some constant $\epsilon > 0$, 3) the quantities $P(Y|X,\theta)$ and $P(A|Y,\theta)$ are estimated using auxiliary samples independent of samples used for the sample averaging.
\label{assumption:extra}
\end{assumption}
Assumptions 1 and 2 are natural. For the missingness mechanism, the ground truth being bounded means that there is a non-vanishing proportion of samples for every class. The boundedness of the estimate can be enforced by clipping the estimate. Assumption 3 is called sample splitting in \cite{kennedy-dr}.

For convenience we use operator $\E_N$ to denote the average of $N$ samples i.e. $\frac{1}{N}\sum_{i=1}^N$. Note that this is by itself a random variable, in contrast to $\E$ which is a fixed number.

\begin{proof}[Proof of \cref{theorem:dr}] Because $C$ is bounded (assumption \ref{assumption:extra}), we can fix a class $c$ and prove the theorem.
Let us define the influence function $\phi$, parameterized by $\theta$, as
\begin{equation}
\phi(O | \theta)(c) = P(Y=c|X,\theta) + \frac{\one(A=1)}{P(A=1|Y,\theta)} (\one(Y=c) - P(Y=c|X,\theta)) - P(Y=c)
\end{equation}
As we have done in the main text, we use $\phi(O)$ to denote the same function but all estimated quantities are replaced with their truths. In other words, we use $\phi(O)$ for $\phi(O|\theta_0)$ where $\theta_0$ is the truth, given that our model contains $\theta_0$ e.g. when the model is consistent.

Recall that:
\begin{equation}
\begin{aligned}
\Psi_{dr}(\theta)(c) &= \frac{1}{N}\sum_{i=1}^N \left\{P(Y=c|X,\theta) + \frac{\one(A=1)}{P(A=1|Y,\theta)} (\one(Y=c) - P(Y=c|X,\theta))\right\}\\
&= \E_N [\phi(O|\theta)(c)] + P(Y=c)
\end{aligned}
\end{equation}

We will show that:
\begin{equation}
\Psi_{dr}(\theta)(c) - P(Y=c) = (\E_N - \E)[\phi(O)(c)] + o_P(N^{-1/2})
\label{eq:proof-linearity}
\end{equation}
To do that, we use the following decomposition
\begin{equation}
\begin{aligned}
\Psi_{dr}(\theta)(c) - P(Y=c) &= \E_N [\phi(O|\theta)(c)] \\
&= (\E_N - \E)[\phi(O)(c)] + (\E_N - \E)[\phi(O|\theta)(c) - \phi(O)(c)] + \E[\phi(O|\theta)(c)]
% &+ (\E_n - \E)[\phi(O;\theta) - \phi(O)]\\
% &+ \E[P(Y=c|X,\theta)] - \E[P(Y=c|X)] + \E[\phi(O,\theta)]
\end{aligned}
\end{equation}
and analyze the second and third term. The third term is:
\begin{equation}
\begin{aligned}
\E[\phi(O|\theta)(c)] &= \E[P(Y=c|X,\theta)] + \E\left[\frac{\one(A=1)}{P(A=1|Y,\theta)}(\one(Y=c) - P(Y=c|X,\theta))\right]- P(Y=c) \\
&= \E\left[P(Y=c|X,\theta) + \frac{P(A=1|Y)}{P(A=1|Y,\theta)}(P(Y=c|X) - P(Y=c|X,\theta))\right] - \E[P(Y=c|X)]\\
&= \E\left[(P(Y=c|X,\theta) - P(Y=c|X)) (P(A=1|Y,\theta) -P(A=1|Y)) \frac{1}{P(A=1|Y,\theta)}\right]\\
\end{aligned}
\end{equation}
by Cauchy-Schwarz inequality:
\begin{equation}
\begin{aligned}
\E[\phi(O|\theta)(c)] &\le \frac{1}{\epsilon} \|P(A=1|Y,\theta) - P(A=1|Y)\|_2 \|P(Y=c|X,\theta) - P(Y=c|X)\|_{L_2(P)}\\
&= \frac{1}{\epsilon} o_P(N^{-1/4} N^{-1/4}) = o_P(N^{-1/2})
\end{aligned}
\end{equation}
by assumption \ref{assumption:4th-root-n} and that $P(A=1|Y,\theta) > \epsilon$ (assumption \ref{assumption:extra}). The second term can be bounded by Chebyshev inequality
% \begin{equation}
% \begin{aligned}
% \E[\E_N[\phi(O|\theta)(c) - \phi(O)(c)]] &= \E[\phi(O|\theta)(c) - \phi(O)(c)]\\
% \var[\E_N[\phi(O|\theta)(c) - \phi(O)(c)]] &= \frac{1}{N}\var[\phi(O|\theta)(c) - \phi(O)(c)] \le 
% \end{aligned}
% \end{equation}
\begin{equation}
P(|(\E_N - \E)[\phi(O|\theta)(c) - \phi(O)(c)]| \ge t) \le \frac{\var[\E_N[\phi(O|\theta)(c) - \phi(O)(c)]]}{t^2} = \frac{\var[\phi(O|\theta)(c) - \phi(O)(c)]}{Nt^2}
\end{equation}
note here that $\theta$ is independent of the samples used for $\E_N$ by assumption \ref{assumption:extra}. For any $\varepsilon > 0$, by picking $t = \frac{1}{\sqrt{N\varepsilon}}$ we get
\begin{equation}
P\left(\left|\frac{(\E_N - \E)[\phi(O|\theta)(c) - \phi(O)(c)]}{N^{-1/2}}\right| \ge \frac{1}{\sqrt{\varepsilon}}\right) \le \varepsilon \var[\phi(O|\theta)(c) - \phi(O)(c)]
\end{equation}
by the definition of $O_P$, we then get
\begin{equation}
(\E_N - \E)[\phi(O|\theta)(c) - \phi(O)(c)] = O_P(N^{-1/2}\var[\phi(O|\theta)(c) - \phi(O)(c)])
\end{equation}
Because $\phi$ is a continuous function of $P(Y|X,\theta)$ and $P(A|Y,\theta)$ (given $P(A|Y,\theta) > \epsilon$, assumption \ref{assumption:extra}), by the continuous mapping theorem and the fact that $P(Y|X,\theta)$ and $P(A|Y,\theta)$ are convergent in probability (assumption \ref{assumption:4th-root-n}), we get $\var[\phi(O|\theta)(c) - \phi(O)(c)] = o_P(1)$. This gives
\begin{equation}
(\E_N - \E)[\phi(O|\theta)(c) - \phi(O)(c)] = o_P(N^{-1/2})
\end{equation}
Therefore, we have shown that the second and third term are both $o_P(N^{-1/2})$, proving \cref{eq:proof-linearity}. As the final step, multiply both sides of this equation by $\sqrt{N}$ we get:
\begin{equation}
\sqrt{N}(\Psi_{dr}(\theta)(c) - P(Y=c)) = \sqrt{N} (\E_N - \E)[\phi(O)(c)] + o_P(1) \rightsquigarrow \mathcal{N}(0, \var[\phi(O)(c)])
\end{equation}
by the central limit theorem, and $\var[\phi(O)(c)] = \E[\phi(O)(c)^2]$ because $\E[\phi(O)(c)] = 0$.
\end{proof}

While we started with the definition of $\phi$, \cref{eq:proof-linearity} shows that $\phi$ is indeed an influence function. Now we show that $\phi$ is also the efficient influence function, by using the characterization of the model's tangent space \cite{tsiatis-missingdata}. Note that the joint probability factorizes as $P(X,A,Y) = P(X)P(Y|X)P(A|Y)$, therefore the tangent space $\mathcal{T}$ factorizes as $\mathcal{T} = \mathcal{T}_{X} \oplus \mathcal{T}_{Y|X} \oplus \mathcal{T}_{A|Y}$ where $\mathcal{T}_X = \{h(X): \E[h] = 0\}$, $\mathcal{T}_{Y|X} = \{h(X,Y): \E[h|X] = 0\}$, $\mathcal{T}_{A|Y} = \{h(A,Y): \E[h|Y] = 0\}$, and the 3 subspaces are pairwise orthogonal. All influence functions are orthogonal to the tangent space, but the influence function that is also in the tangent space has the smallest variance and is called the efficient influence function. As $\phi$ is already an influence function, we need only show that $\phi$ is in $\mathcal{T}$. We write $\phi$ as
\begin{equation}
\phi(O)(c) = (P(Y=c|X) - P(Y=c)) + \left[\frac{\one(A=1)}{P(A=1|Y)} - 1\right](\one(Y=c) - P(Y=c|X)) + (\one(Y=c) - P(Y=c|X))
\end{equation}
and note that the first, second and third term are in $\mathcal{T}_X$, $\mathcal{T}_{A|Y}$ and $\mathcal{T}_{Y|X}$ respectively. Therefore, $\phi$ is indeed in $\mathcal{T}$. The efficient influence function has the smallest variance of all influence function, and therefore our estimator being asymptotically linear in $\phi$ (\cref{eq:proof-linearity}) has the smallest mean squared error in a local asymptotic minimax sense \cite{kennedy-dr, asymptoticstatistics}

\section{Further background and related work}
\paragraph{Discussion on semi-supervised EM.}
It appears that semi-supervised EM was first used for parameter estimation when the missingness mechanism is non-ignorable in \cite{ibrahim1996parameter}, but has not been used for label shift estimation.
Perhaps this is because the semi-supervised situation where additional unlabeled data is available during training is rarer than the test-time adaptation case. EM is well suited to take advantage of the extra unlabeled data to improve the classifier under very scarce and long-tailed labeled data. While the connection between pseudo-labeling and EM has been explored before \cite{entropyminimization}, the situation with label shift has not until recently \cite{simpro}. Here the application of EM is much more interesting, because other than simply giving pseudo-labeling a rigorous formulation, EM also estimates the missingness mechanism (equivalently the label distribution shift), which is important for shift correction and thus high-quality pseudo-labels \cite{acr}. The application of confidence thresholding can be seen as a sparse variant of EM \cite{neal1998view}.

\paragraph{The doubly-robust risk.} 
\label{subsec:dr-risk}
A technique that also derives from the theory of semi-parametric efficiency is orthogonal statistical learning \citep{foster2023orthogonal}. The idea is to minimize the doubly-robust risk:
\label{subsec:method-dr-risk}
\begin{equation}
\label{eq:dr-risk}
\mathcal{R}(\theta_2) = \frac{1}{N} \sum_{i=1}^N \Bigg[ l(x_i, \hat y_i|\theta_2) + \frac{\one(a_i=1)}{P(A=a_i|Y=y_i, \theta_1)} (l(x_i, y_i | \theta_2) - l(x_i, \hat y_i | \theta_2))\Bigg]
\end{equation}
where $l(x,y|\theta) = -\sum_{c=1}^C [y]_c \log P(Y=c|X=x,\theta)$ is the negative cross-entropy. 
The notation $[y]_c$ means that we are using the $c$-entry in a C-dimension probability vector $y$. 
Thus, $y_i$ denotes the one-hot label of observation $i$, while $\hat y_i$ denotes the pseudo-label, which can be one-hot or all-zero. 
Finally, we use $\theta_1$ to denote that $P(a|y,\theta_1)$ is an estimation from a previous stage, but it can be estimated with $\theta_2$ as well. 
The risk $\mathcal{R}(\theta_2)$ can be used as a training loss in a straightforward fashion. 
Similar to the doubly robust estimation of $P(Y)$, the doubly robust risk provides approximately unbiased estimation of the risk. 
This property has been used in \citep{arelabelsinformative, onnonrandommissinglabels, drst} also in the semi-supervised learning setting.
More broadly, it is at the heart of one of the core techniques in heterogenous treatment effect estimation in causal estimation \cite{kennedy2023towards, foster2023orthogonal, wager2018estimation}. 
The focus here is not the estimation of $\mathcal{R}(\theta_2)$ per se, but the quality of the learned model \cite{foster2023orthogonal}.
By using the doubly-robust risk, we can achieve an optimality result similar in spirit to our theorem \cref{theorem:dr}, but for the generalization error.
While this is appealing, in practice there are 2 problems with this approach. First, the inverse probability weight $P(A=a_i|Y=y_i,\theta_1)$ can be very large if the class ratio is highly unlabeled, making training unstable \cite{kallus2020deepmatch, pham2023stable}. 
This problem exists for our estimation as well. However, it is much easier to control for estimation than for training because of the iterative nature of model update. Secondly, we can further write $\mathcal{R}$ as:
\begin{equation}
\mathcal{R}(\theta_2) = \frac{1}{N}\sum_{i=1}^N l\left(x_i, \hat y_i + \frac{\one(a_i=1)}{P(A=a_i|Y=y_i,\theta_1)} (y_i - \hat y_i)\Bigg\vert\theta_2\right)
\end{equation}
which is a cross-entropy loss with new meta-pseudo-labels. However, these labels are not meant to be learned exactly, and furthermore they can be negative. Thus, theoretical works have to put stringent assumptions on the models. In \cref{subsec:ablation-1}, we show that experimentally that the instability problem makes doubly-robust risk performance worse than our 2-stage approach.

\section{Training and hyperparameter settings.}
\label{subsec:training-setting}
For neural network training, we follow the implementation and hyperparameter settings of \cite{simpro}. In particular, we adapt the core code of SimPro for Supervised, MLE and EM. For MLE, we update $P(A|Y)$ using the Adam optimizer with learning rate 1e-3, while for EM we use a momentum update similar to SimPro's update of $P(Y|A)$ because it has a a closed-form solution at each mini-batch. We use Wide ResNet-28-2 on all methods and all datasets in this section, including Imagenet-127, because we are motivated by the fact that stage-1's goal is not classification accuracy but the estimation of a finite-dimensional parameter. When using Wide ResNet-28-2 for Imagenet-127, we use the hyperparameters of CIFAR-100, except we lower the batch size of unlabeled data to 2 times that of labeled data instead of 8 for memory reason. We do not perform additional hyperparameter tuning. All experiments can be performed on 1 A6000 RTX GPU, and are run 3 times. We report the total variation distance between the estimated and the ground truth unlabeled class distribution, similar to its usage in Theorem 3.1 of \cite{lsc}, and the top-1 classification accuracy.

In the second stage of our algorithm, we freeze our estimation and plug it in SimPro and BOAT.
We keep exactly the same hyperparameter settings that SimPro and BOAT use. In particular, for Imagenet-127, we now use ResNet-50 and run each experiment once.
In SimPro, we set the unlabeled class distribution $P(Y|A=0)$ at the E-step;  however, we still keep a running estimate of the class distribution $P(Y)$ in the logit adjustment loss \cref{eq:simpro-la-loss}. While it is possible to use the first stage estimate in the logit adjustment loss, we observe that doing so results in lower accuracy than using the the running average. This is conceptually consistent with the role of the running average - serving not as an accurate estimate of $P(Y)$ but to make the classifier's class distribution uniform through the logit adjustment loss, which is good for the test set. Similarly, in BOAT, we only replace $\Delta_c = \log P(Y|A=1) - \log P(Y|A=0)$ in equation (4) of \cite{boat}, which is adjusting a classifier's predictions from the labeled to the unlabeled class distribution, with our SimPro + DR estimate instead of their on-the-fly estimate. 


% \section{Additional experiments}
% % \begin{table*}[t]
\centering
\caption{Total Variation Distance on CIFAR-10-LT ($N_l = 500$, $M_l = 4000$) with different class imbalance ratios $\gamma_l$ and $\gamma_u$ under five different unlabeled class distributions.}
\label{tab:cifar10-tv}
\resizebox{\textwidth}{!}{
\begin{tabular}{lccccccccccc}
\toprule
& & \multicolumn{2}{c}{consistent} & \multicolumn{2}{c}{uniform} & \multicolumn{2}{c}{reversed} & \multicolumn{2}{c}{middle} & \multicolumn{2}{c}{head-tail} \\
\cmidrule(lr){3-4} \cmidrule(lr){5-6} \cmidrule(lr){7-8} \cmidrule(lr){9-10} \cmidrule(lr){11-12}
& & $\gamma_l = 150$ & $\gamma_l = 100$ & $\gamma_l = 150$ & $\gamma_l = 100$ & $\gamma_l = 150$ & $\gamma_l = 100$ & $\gamma_l = 150$ & $\gamma_l = 100$ & $\gamma_l = 150$ & $\gamma_l = 100$ \\
Model & Estimator & $\gamma_u = 150$ & $\gamma_u = 100$ & $\gamma_u = 1$ & $\gamma_u = 1$ & $\gamma_u = 1/150$ & $\gamma_u = 1/100$ & $\gamma_u = 150$ & $\gamma_u = 100$ & $\gamma_u = 150$ & $\gamma_u = 100$ \\
\midrule
Supervised & MLLS & 0.269 ± 0.252 & 0.038 ± 0.006 & 0.251 ± 0.046 & 0.255 ± 0.060 & 0.429 ± 0.028 & 0.493 ± 0.050 & 0.333 ± 0.042 & 0.320 ± 0.009 & 0.457 ± 0.034 & 0.444 ± 0.043 \\
Supervised & RLLS & 0.043 ± 0.001 & 0.044 ± 0.010 & 0.348 ± 0.034 & 0.305 ± 0.068 & 0.769 ± 0.016 & 0.678 ± 0.028 & 0.430 ± 0.008 & 0.368 ± 0.013 & 0.539 ± 0.018 & 0.503 ± 0.020 \\
\midrule
MLE & IPW & 0.027 ± 0.001 & 0.027 ± 0.000 & 0.319 ± 0.072 & 0.243 ± 0.010 & 0.674 ± 0.020 & 0.646 ± 0.041 & 0.438 ± 0.020 & 0.454 ± 0.026 & 0.547 ± 0.049 & 0.491 ± 0.059 \\
MLE & OR & 0.045 ± 0.004 & 0.042 ± 0.000 & 0.215 ± 0.026 & 0.203 ± 0.032 & 0.433 ± 0.017 & 0.395 ± 0.033 & 0.193 ± 0.006 & 0.209 ± 0.037 & 0.307 ± 0.147 & 0.249 ± 0.130 \\
MLE & DR & 0.090 ± 0.002 & 0.079 ± 0.000 & 0.407 ± 0.027 & 0.360 ± 0.007 & 0.425 ± 0.007 & 0.421 ± 0.029 & 0.256 ± 0.001 & 0.286 ± 0.031 & 0.435 ± 0.136 & 0.362 ± 0.122 \\
\midrule
EM & IPW & 0.035 ± 0.002 & 0.040 ± 0.001 & 0.021 ± 0.001 & 0.029 ± 0.015 & 0.303 ± 0.187 & 0.091 ± 0.010 & 0.119 ± 0.011 & 0.105 ± 0.022 & 0.104 ± 0.026 & 0.104 ± 0.051 \\
EM & OR & 0.037 ± 0.003 & 0.042 ± 0.002 & 0.016 ± 0.001 & 0.024 ± 0.012 & 0.269 ± 0.183 & 0.090 ± 0.008 & 0.122 ± 0.012 & 0.103 ± 0.022 & 0.072 ± 0.012 & 0.073 ± 0.024 \\
EM & DR & 0.034 ± 0.004 & 0.037 ± 0.001 & 0.014 ± 0.001 & 0.027 ± 0.020 & 0.264 ± 0.191 & 0.092 ± 0.005 & 0.111 ± 0.019 & 0.097 ± 0.026 & 0.077 ± 0.016 & 0.073 ± 0.028 \\
\midrule
SimPro & IPW & 0.070 ± 0.011 & 0.058 ± 0.000 & 0.046 ± 0.001 & 0.049 ± 0.005 & 0.254 ± 0.074 & 0.223 ± 0.098 & 0.097 ± 0.025 & 0.067 ± 0.002 & 0.105 ± 0.066 & 0.110 ± 0.079 \\
SimPro & OR & 0.071 ± 0.012 & 0.058 ± 0.000 & 0.045 ± 0.001 & 0.049 ± 0.006 & 0.040 ± 0.003 & 0.059 ± 0.017 & 0.074 ± 0.006 & 0.075 ± 0.002 & 0.033 ± 0.003 & 0.033 ± 0.003 \\
SimPro & DR & 0.017 ± 0.004 & 0.026 ± 0.001 & 0.019 ± 0.002 & 0.018 ± 0.003 & 0.039 ± 0.003 & 0.058 ± 0.025 & 0.091 ± 0.007 & 0.031 ± 0.001 & 0.015 ± 0.003 & 0.019 ± 0.007 \\
\bottomrule
\end{tabular}
}
\end{table*}
% 

\begin{table*}[t]
\centering
\caption{Total Variation Distance on CIFAR-100-LT ($N_l = 50$, $M_l = 400$) with different class imbalance ratios $\gamma_l$ and $\gamma_u$ under five different unlabeled class distributions.}
\label{tab:cifar100-tv}
\resizebox{\textwidth}{!}{
\begin{tabular}{lccccccccccc}
\toprule
& & \multicolumn{2}{c}{consistent} & \multicolumn{2}{c}{uniform} & \multicolumn{2}{c}{reversed} & \multicolumn{2}{c}{middle} & \multicolumn{2}{c}{head-tail} \\
\cmidrule(lr){3-4} \cmidrule(lr){5-6} \cmidrule(lr){7-8} \cmidrule(lr){9-10} \cmidrule(lr){11-12}
& & $\gamma_l = 20$ & $\gamma_l = 10$ & $\gamma_l = 20$ & $\gamma_l = 10$ & $\gamma_l = 20$ & $\gamma_l = 10$ & $\gamma_l = 20$ & $\gamma_l = 10$ & $\gamma_l = 20$ & $\gamma_l = 10$ \\
Model & Estimator & $\gamma_u = 20$ & $\gamma_u = 10$ & $\gamma_u = 1$ & $\gamma_u = 1$ & $\gamma_u = 1/20$ & $\gamma_u = 1/10$ & $\gamma_u = 20$ & $\gamma_u = 10$ & $\gamma_u = 20$ & $\gamma_u = 10$ \\
\midrule
Supervised & MLLS & 0.707 ± 0.016 & 0.313 ± 0.100 & 0.445 ± 0.172 & 0.309 ± 0.119 & 0.383 ± 0.075 & 0.397 ± 0.006 & 0.570 ± 0.001 & 0.373 ± 0.107 & 0.543 ± 0.009 & 0.231 ± 0.057 \\
Supervised & RLLS & 0.520 ± 0.007 & 0.133 ± 0.003 & 0.337 ± 0.125 & 0.253 ± 0.082 & 0.424 ± 0.060 & 0.463 ± 0.003 & 0.454 ± 0.021 & 0.306 ± 0.074 & 0.460 ± 0.028 & 0.241 ± 0.040 \\
\midrule
MLE & IPW & 0.075 ± 0.000 & 0.071 ± 0.001 & 0.229 ± 0.001 & 0.167 ± 0.002 & 0.565 ± 0.005 & 0.443 ± 0.007 & 0.415 ± 0.000 & 0.311 ± 0.005 & 0.343 ± 0.000 & 0.280 ± 0.001 \\
MLE & OR & 0.065 ± 0.002 & 0.061 ± 0.001 & 0.200 ± 0.007 & 0.143 ± 0.001 & 0.526 ± 0.011 & 0.399 ± 0.023 & 0.360 ± 0.003 & 0.256 ± 0.012 & 0.328 ± 0.003 & 0.266 ± 0.005 \\
MLE & DR & 0.149 ± 0.019 & 0.145 ± 0.010 & 0.243 ± 0.004 & 0.214 ± 0.019 & 0.568 ± 0.005 & 0.464 ± 0.014 & 0.403 ± 0.014 & 0.309 ± 0.012 & 0.365 ± 0.007 & 0.320 ± 0.004 \\
\midrule
EM & IPW & 0.097 ± 0.008 & 0.092 ± 0.004 & 0.239 ± 0.007 & 0.179 ± 0.003 & 0.478 ± 0.012 & 0.329 ± 0.020 & 0.262 ± 0.016 & 0.202 ± 0.003 & 0.312 ± 0.002 & 0.227 ± 0.001 \\
EM & OR & 0.121 ± 0.007 & 0.108 ± 0.005 & 0.261 ± 0.007 & 0.189 ± 0.004 & 0.489 ± 0.013 & 0.335 ± 0.020 & 0.274 ± 0.016 & 0.211 ± 0.004 & 0.336 ± 0.003 & 0.235 ± 0.001 \\
EM & DR & 0.125 ± 0.005 & 0.111 ± 0.004 & 0.269 ± 0.007 & 0.194 ± 0.005 & 0.497 ± 0.010 & 0.336 ± 0.024 & 0.281 ± 0.019 & 0.219 ± 0.008 & 0.336 ± 0.007 & 0.233 ± 0.004 \\
\midrule
SimPro & IPW & 0.125 ± 0.001 & 0.100 ± 0.005 & 0.166 ± 0.007 & 0.141 ± 0.009 & 0.353 ± 0.023 & 0.261 ± 0.008 & 0.202 ± 0.003 & 0.158 ± 0.005 & 0.277 ± 0.009 & 0.197 ± 0.003 \\
SimPro & OR & 0.133 ± 0.005 & 0.100 ± 0.004 & 0.160 ± 0.007 & 0.138 ± 0.010 & 0.322 ± 0.014 & 0.253 ± 0.008 & 0.202 ± 0.003 & 0.156 ± 0.005 & 0.269 ± 0.006 & 0.191 ± 0.004 \\
SimPro & DR & 0.122 ± 0.003 & 0.106 ± 0.006 & 0.188 ± 0.009 & 0.149 ± 0.006 & 0.343 ± 0.023 & 0.257 ± 0.007 & 0.219 ± 0.010 & 0.172 ± 0.002 & 0.279 ± 0.007 & 0.198 ± 0.004 \\
\bottomrule
\end{tabular}
}
\end{table*}

\end{document}
