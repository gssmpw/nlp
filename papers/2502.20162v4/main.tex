% CVPR 2025 Paper Template; see https://github.com/cvpr-org/author-kit

\documentclass[10pt,twocolumn,letterpaper]{article}

%%%%%%%%% PAPER TYPE  - PLEASE UPDATE FOR FINAL VERSION
%\usepackage{cvpr}              % To produce the CAMERA-READY version
%\usepackage[review]{cvpr}      % To produce the REVIEW version
\usepackage[pagenumbers]{cvpr} % To force page numbers, e.g. for an arXiv version
%%%% My packages %%%%
\usepackage{amsmath}
\usepackage{subcaption}
\usepackage{multicol}
\usepackage{graphicx}
\usepackage{xcolor}
\usepackage{algpseudocode}
\usepackage{algorithm}
\usepackage{amsmath}
\usepackage{bm}
\usepackage{subfiles}
\graphicspath{{figures/}}

\usepackage[switch]{lineno}


\usepackage{tikz}
\usetikzlibrary{bayesnet}

% Import additional packages in the preamble file, before hyperref
%
% --- inline annotations
%
\newcommand{\red}[1]{{\color{red}#1}}
\newcommand{\todo}[1]{{\color{red}#1}}
\newcommand{\TODO}[1]{\textbf{\color{red}[TODO: #1]}}
% --- disable by uncommenting  
% \renewcommand{\TODO}[1]{}
% \renewcommand{\todo}[1]{#1}



\newcommand{\VLM}{LVLM\xspace} 
\newcommand{\ours}{PeKit\xspace}
\newcommand{\yollava}{Yo’LLaVA\xspace}

\newcommand{\thisismy}{This-Is-My-Img\xspace}
\newcommand{\myparagraph}[1]{\noindent\textbf{#1}}
\newcommand{\vdoro}[1]{{\color[rgb]{0.4, 0.18, 0.78} {[V] #1}}}
% --- disable by uncommenting  
% \renewcommand{\TODO}[1]{}
% \renewcommand{\todo}[1]{#1}
\usepackage{slashbox}
% Vectors
\newcommand{\bB}{\mathcal{B}}
\newcommand{\bw}{\mathbf{w}}
\newcommand{\bs}{\mathbf{s}}
\newcommand{\bo}{\mathbf{o}}
\newcommand{\bn}{\mathbf{n}}
\newcommand{\bc}{\mathbf{c}}
\newcommand{\bp}{\mathbf{p}}
\newcommand{\bS}{\mathbf{S}}
\newcommand{\bk}{\mathbf{k}}
\newcommand{\bmu}{\boldsymbol{\mu}}
\newcommand{\bx}{\mathbf{x}}
\newcommand{\bg}{\mathbf{g}}
\newcommand{\be}{\mathbf{e}}
\newcommand{\bX}{\mathbf{X}}
\newcommand{\by}{\mathbf{y}}
\newcommand{\bv}{\mathbf{v}}
\newcommand{\bz}{\mathbf{z}}
\newcommand{\bq}{\mathbf{q}}
\newcommand{\bff}{\mathbf{f}}
\newcommand{\bu}{\mathbf{u}}
\newcommand{\bh}{\mathbf{h}}
\newcommand{\bb}{\mathbf{b}}

\newcommand{\rone}{\textcolor{green}{R1}}
\newcommand{\rtwo}{\textcolor{orange}{R2}}
\newcommand{\rthree}{\textcolor{red}{R3}}
\usepackage{amsmath}
%\usepackage{arydshln}
\DeclareMathOperator{\similarity}{sim}
\DeclareMathOperator{\AvgPool}{AvgPool}

\newcommand{\argmax}{\mathop{\mathrm{argmax}}}     



% It is strongly recommended to use hyperref, especially for the review version.
% hyperref with option pagebackref eases the reviewers' job.
% Please disable hyperref *only* if you encounter grave issues, 
% e.g. with the file validation for the camera-ready version.
%
% If you comment hyperref and then uncomment it, you should delete *.aux before re-running LaTeX.
% (Or just hit 'q' on the first LaTeX run, let it finish, and you should be clear).
\definecolor{cvprblue}{rgb}{0.21,0.49,0.74}
\usepackage[pagebackref,breaklinks,colorlinks,citecolor=cvprblue]{hyperref}

%% New commands
\newcommand{\card}[1]{\lvert\mathcal{#1}\rvert}

%%%%%%%%% PAPER ID  - PLEASE UPDATE
\def\paperID{15137} % *** Enter the Paper ID here
\def\confName{CVPR}
\def\confYear{2025}

%%%%%%%%% TITLE - PLEASE UPDATE
\title{Gradient-Guided Annealing for Domain Generalization}

%%%%%%%%% AUTHORS - PLEASE UPDATE
\author{Aristotelis Ballas\\
Dpt of Informatics and Telematics\\
Harokopio University of Athens\\
Omirou 9, Tavros, Athens, Greece\\
{\tt\small aballas@hua.gr}
% For a paper whose authors are all at the same institution,
% omit the following lines up until the closing ``}''.
% Additional authors and addresses can be added with ``\and'',
% just like the second author.
% To save space, use either the email address or home page, not both
\and
Christos Diou\\
Dpt of Informatics and Telematics\\
Harokopio Univesity of Athens\\
Omirou 9, Tavros, Athens, Greece\\
{\tt\small cdiou@hua.gr}
}

\begin{document}
%\maketitle

\twocolumn[{%
	\renewcommand\twocolumn[1][]{#1}%
	\maketitle
	\begin{center}
		\setcounter{figure}{0}
		\centering
		\captionsetup{type=figure}
		
		% First figure
		\vspace{-1.5em}
		\includegraphics[width=0.7\textwidth,trim={0.0cm 0.0cm 0cm 0},clip,page=1]{concept-erm-final}\vspace{-0.5em}%
		
		% Space between figures
		\vspace{1em}
		
		% Second figure
		\includegraphics[width=0.7\textwidth,trim={0.0cm 0.0cm 0cm 0},clip,page=1]{concept-gga-final}\vspace{-0.5em}%
		\caption{\em (left) Decision boundaries of a 4\textsuperscript{th}-degree polynomial logistic regression model with 2D input. In this example, feature $x_1$ is class-specific and $x_2$ is domain-specific, while color represents classes and shapes represent domains. The samples with solid red and green colors are included in the training data, whereas the fainted samples are part of the hidden held-out test set. As a result, domain shift is represented by a change in $x_2$. Although the classifier should only infer based on $x_1$, traditional gradient descent leads to overfitting (top-left). The proposed method, \textit{GGA} (bottom-left), introduces an annealing process that depends on gradient agreement, leading to models that generalize well to new, unobserved target domains. (right) Schematics of the parameter updates of ERM (top-right) and GGA (bottom-right). Parameters updated via ERM are driven by gradient conflict, whereas GGA searches for a point where gradients align before continuing descending towards a minima.}\label{fig:concept}%
	\end{center}%
}]



\begin{abstract}


The choice of representation for geographic location significantly impacts the accuracy of models for a broad range of geospatial tasks, including fine-grained species classification, population density estimation, and biome classification. Recent works like SatCLIP and GeoCLIP learn such representations by contrastively aligning geolocation with co-located images. While these methods work exceptionally well, in this paper, we posit that the current training strategies fail to fully capture the important visual features. We provide an information theoretic perspective on why the resulting embeddings from these methods discard crucial visual information that is important for many downstream tasks. To solve this problem, we propose a novel retrieval-augmented strategy called RANGE. We build our method on the intuition that the visual features of a location can be estimated by combining the visual features from multiple similar-looking locations. We evaluate our method across a wide variety of tasks. Our results show that RANGE outperforms the existing state-of-the-art models with significant margins in most tasks. We show gains of up to 13.1\% on classification tasks and 0.145 $R^2$ on regression tasks. All our code and models will be made available at: \href{https://github.com/mvrl/RANGE}{https://github.com/mvrl/RANGE}.

\end{abstract}


\section{Introduction}
Backdoor attacks pose a concealed yet profound security risk to machine learning (ML) models, for which the adversaries can inject a stealth backdoor into the model during training, enabling them to illicitly control the model's output upon encountering predefined inputs. These attacks can even occur without the knowledge of developers or end-users, thereby undermining the trust in ML systems. As ML becomes more deeply embedded in critical sectors like finance, healthcare, and autonomous driving \citep{he2016deep, liu2020computing, tournier2019mrtrix3, adjabi2020past}, the potential damage from backdoor attacks grows, underscoring the emergency for developing robust defense mechanisms against backdoor attacks.

To address the threat of backdoor attacks, researchers have developed a variety of strategies \cite{liu2018fine,wu2021adversarial,wang2019neural,zeng2022adversarial,zhu2023neural,Zhu_2023_ICCV, wei2024shared,wei2024d3}, aimed at purifying backdoors within victim models. These methods are designed to integrate with current deployment workflows seamlessly and have demonstrated significant success in mitigating the effects of backdoor triggers \cite{wubackdoorbench, wu2023defenses, wu2024backdoorbench,dunnett2024countering}.  However, most state-of-the-art (SOTA) backdoor purification methods operate under the assumption that a small clean dataset, often referred to as \textbf{auxiliary dataset}, is available for purification. Such an assumption poses practical challenges, especially in scenarios where data is scarce. To tackle this challenge, efforts have been made to reduce the size of the required auxiliary dataset~\cite{chai2022oneshot,li2023reconstructive, Zhu_2023_ICCV} and even explore dataset-free purification techniques~\cite{zheng2022data,hong2023revisiting,lin2024fusing}. Although these approaches offer some improvements, recent evaluations \cite{dunnett2024countering, wu2024backdoorbench} continue to highlight the importance of sufficient auxiliary data for achieving robust defenses against backdoor attacks.

While significant progress has been made in reducing the size of auxiliary datasets, an equally critical yet underexplored question remains: \emph{how does the nature of the auxiliary dataset affect purification effectiveness?} In  real-world  applications, auxiliary datasets can vary widely, encompassing in-distribution data, synthetic data, or external data from different sources. Understanding how each type of auxiliary dataset influences the purification effectiveness is vital for selecting or constructing the most suitable auxiliary dataset and the corresponding technique. For instance, when multiple datasets are available, understanding how different datasets contribute to purification can guide defenders in selecting or crafting the most appropriate dataset. Conversely, when only limited auxiliary data is accessible, knowing which purification technique works best under those constraints is critical. Therefore, there is an urgent need for a thorough investigation into the impact of auxiliary datasets on purification effectiveness to guide defenders in  enhancing the security of ML systems. 

In this paper, we systematically investigate the critical role of auxiliary datasets in backdoor purification, aiming to bridge the gap between idealized and practical purification scenarios.  Specifically, we first construct a diverse set of auxiliary datasets to emulate real-world conditions, as summarized in Table~\ref{overall}. These datasets include in-distribution data, synthetic data, and external data from other sources. Through an evaluation of SOTA backdoor purification methods across these datasets, we uncover several critical insights: \textbf{1)} In-distribution datasets, particularly those carefully filtered from the original training data of the victim model, effectively preserve the model’s utility for its intended tasks but may fall short in eliminating backdoors. \textbf{2)} Incorporating OOD datasets can help the model forget backdoors but also bring the risk of forgetting critical learned knowledge, significantly degrading its overall performance. Building on these findings, we propose Guided Input Calibration (GIC), a novel technique that enhances backdoor purification by adaptively transforming auxiliary data to better align with the victim model’s learned representations. By leveraging the victim model itself to guide this transformation, GIC optimizes the purification process, striking a balance between preserving model utility and mitigating backdoor threats. Extensive experiments demonstrate that GIC significantly improves the effectiveness of backdoor purification across diverse auxiliary datasets, providing a practical and robust defense solution.

Our main contributions are threefold:
\textbf{1) Impact analysis of auxiliary datasets:} We take the \textbf{first step}  in systematically investigating how different types of auxiliary datasets influence backdoor purification effectiveness. Our findings provide novel insights and serve as a foundation for future research on optimizing dataset selection and construction for enhanced backdoor defense.
%
\textbf{2) Compilation and evaluation of diverse auxiliary datasets:}  We have compiled and rigorously evaluated a diverse set of auxiliary datasets using SOTA purification methods, making our datasets and code publicly available to facilitate and support future research on practical backdoor defense strategies.
%
\textbf{3) Introduction of GIC:} We introduce GIC, the \textbf{first} dedicated solution designed to align auxiliary datasets with the model’s learned representations, significantly enhancing backdoor mitigation across various dataset types. Our approach sets a new benchmark for practical and effective backdoor defense.



\section{Related Work}
\label{sec:related-works}
\subsection{Novel View Synthesis}
Novel view synthesis is a foundational task in the computer vision and graphics, which aims to generate unseen views of a scene from a given set of images.
% Many methods have been designed to solve this problem by posing it as 3D geometry based rendering, where point clouds~\cite{point_differentiable,point_nfs}, mesh~\cite{worldsheet,FVS,SVS}, planes~\cite{automatci_photo_pop_up,tour_into_the_picture} and multi-plane images~\cite{MINE,single_view_mpi,stereo_magnification}, \etal
Numerous methods have been developed to address this problem by approaching it as 3D geometry-based rendering, such as using meshes~\cite{worldsheet,FVS,SVS}, MPI~\cite{MINE,single_view_mpi,stereo_magnification}, point clouds~\cite{point_differentiable,point_nfs}, etc.
% planes~\cite{automatci_photo_pop_up,tour_into_the_picture}, 


\begin{figure*}[!t]
    \centering
    \includegraphics[width=1.0\linewidth]{figures/overview-v7.png}
    %\caption{\textbf{Overview.} Given a set of images, our method obtains both camera intrinsics and extrinsics, as well as a 3DGS model. First, we obtain the initial camera parameters, global track points from image correspondences and monodepth with reprojection loss. Then we incorporate the global track information and select Gaussian kernels associated with track points. We jointly optimize the parameters $K$, $T_{cw}$, 3DGS through multi-view geometric consistency $L_{t2d}$, $L_{t3d}$, $L_{scale}$ and photometric consistency $L_1$, $L_{D-SSIM}$.}
    \caption{\textbf{Overview.} Given a set of images, our method obtains both camera intrinsics and extrinsics, as well as a 3DGS model. During the initialization, we extract the global tracks, and initialize camera parameters and Gaussians from image correspondences and monodepth with reprojection loss. We determine Gaussian kernels with recovered 3D track points, and then jointly optimize the parameters $K$, $T_{cw}$, 3DGS through the proposed global track constraints (i.e., $L_{t2d}$, $L_{t3d}$, and $L_{scale}$) and original photometric losses (i.e., $L_1$ and $L_{D-SSIM}$).}
    \label{fig:overview}
\end{figure*}

Recently, Neural Radiance Fields (NeRF)~\cite{2020NeRF} provide a novel solution to this problem by representing scenes as implicit radiance fields using neural networks, achieving photo-realistic rendering quality. Although having some works in improving efficiency~\cite{instant_nerf2022, lin2022enerf}, the time-consuming training and rendering still limit its practicality.
Alternatively, 3D Gaussian Splatting (3DGS)~\cite{3DGS2023} models the scene as explicit Gaussian kernels, with differentiable splatting for rendering. Its improved real-time rendering performance, lower storage and efficiency, quickly attract more attentions.
% Different from NeRF-based methods which need MLPs to model the scene and huge computational cost for rendering, 3DGS has stronger real-time performance, higher storage and computational efficiency, benefits from its explicit representation and gradient backpropagation.

\subsection{Optimizing Camera Poses in NeRFs and 3DGS}
Although NeRF and 3DGS can provide impressive scene representation, these methods all need accurate camera parameters (both intrinsic and extrinsic) as additional inputs, which are mostly obtained by COLMAP~\cite{colmap2016}.
% This strong reliance on COLMAP significantly limits their use in real-world applications, so optimizing the camera parameters during the scene training becomes crucial.
When the prior is inaccurate or unknown, accurately estimating camera parameters and scene representations becomes crucial.

% In early works, only photometric constraints are used for scene training and camera pose estimation. 
% iNeRF~\cite{iNerf2021} optimizes the camera poses based on a pre-trained NeRF model.
% NeRFmm~\cite{wang2021nerfmm} introduce a joint optimization process, which estimates the camera poses and trains NeRF model jointly.
% BARF~\cite{barf2021} and GARF~\cite{2022GARF} provide new positional encoding strategy to handle with the gradient inconsistency issue of positional embedding and yield promising results.
% However, they achieve satisfactory optimization results when only the pose initialization is quite closed to the ground-truth, as the photometric constrains can only improve the quality of camera estimation within a small range.
% Later, more prior information of geometry and correspondence, \ie monocular depth and feature matching, are introduced into joint optimisation to enhance the capability of camera poses estimation.
% SC-NeRF~\cite{SCNeRF2021} minimizes a projected ray distance loss based on correspondence of adjacent frames.
% NoPe-NeRF~\cite{bian2022nopenerf} chooses monocular depth maps as geometric priors, and defines undistorted depth loss and relative pose constraints for joint optimization.
In earlier studies, scene training and camera pose estimation relied solely on photometric constraints. iNeRF~\cite{iNerf2021} refines the camera poses using a pre-trained NeRF model. NeRFmm~\cite{wang2021nerfmm} introduces a joint optimization approach that simultaneously estimates camera poses and trains the NeRF model. BARF~\cite{barf2021} and GARF~\cite{2022GARF} propose a new positional encoding strategy to address the gradient inconsistency issues in positional embedding, achieving promising results. However, these methods only yield satisfactory optimization when the initial pose is very close to the ground truth, as photometric constraints alone can only enhance camera estimation quality within a limited range. Subsequently, 
% additional prior information on geometry and correspondence, such as monocular depth and feature matching, has been incorporated into joint optimization to improve the accuracy of camera pose estimation. 
SC-NeRF~\cite{SCNeRF2021} minimizes a projected ray distance loss based on correspondence between adjacent frames. NoPe-NeRF~\cite{bian2022nopenerf} utilizes monocular depth maps as geometric priors and defines undistorted depth loss and relative pose constraints.

% With regard to 3D Gaussian Splatting, CF-3DGS~\cite{CF-3DGS-2024} also leverages mono-depth information to constrain the optimization of local 3DGS for relative pose estimation and later learn a global 3DGS progressively in a sequential manner.
% InstantSplat~\cite{fan2024instantsplat} focus on sparse view scenes, first use DUSt3R~\cite{dust3r2024cvpr} to generate a set of densely covered and pixel-aligned points for 3D Gaussian initialization, then introduce a parallel grid partitioning strategy in joint optimization to speed up.
% % Jiang et al.~\cite{Jiang_2024sig} proposed to build the scene continuously and progressively, to next unregistered frame, they use registration and adjustment to adjust the previous registered camera poses and align unregistered monocular depths, later refine the joint model by matching detected correspondences in screen-space coordinates.
% \gjh{Jiang et al.~\cite{Jiang_2024sig} also implemented an incremental approach for reconstructing camera poses and scenes. Initially, they perform feature matching between the current image and the image rendered by a differentiable surface renderer. They then construct matching point errors, depth errors, and photometric errors to achieve the registration and adjustment of the current image. Finally, based on the depth map, the pixels of the current image are projected as new 3D Gaussians. However, this method still exhibits limitations when dealing with complex scenes and unordered images.}
% % CG-3DGS~\cite{sun2024correspondenceguidedsfmfree3dgaussian} follows CF-3DGS, first construct a coarse point cloud from mono-depth maps to train a 3DGS model, then progressively estimate camera poses based on this pre-trained model by constraining the correspondences between rendering view and ground-truth.
% \gjh{Similarly, CG-3DGS~\cite{sun2024correspondenceguidedsfmfree3dgaussian} first utilizes monocular depth estimation and the camera parameters from the first frame to initialize a set of 3D Gaussians. It then progressively estimates camera poses based on this pre-trained model by constraining the correspondences between the rendered views and the ground truth.}
% % Free-SurGS~\cite{freesurgs2024} matches the projection flow derived from 3D Gaussians with optical flow to estimate the poses, to compensate for the limitations of photometric loss.
% \gjh{Free-SurGS~\cite{freesurgs2024} introduces the first SfM-free 3DGS approach for surgical scene reconstruction. Due to the challenges posed by weak textures and photometric inconsistencies in surgical scenes, Free-SurGS achieves pose estimation by minimizing the flow loss between the projection flow and the optical flow. Subsequently, it keeps the camera pose fixed and optimizes the scene representation by minimizing the photometric loss, depth loss and flow loss.}
% \gjh{However, most current works assume camera intrinsics are known and primarily focus on optimizing camera poses. Additionally, these methods typically rely on sequentially ordered image inputs and incrementally optimize camera parameters and scene representation. This inevitably leads to drift errors, preventing the achievement of globally consistent results. Our work aims to address these issues.}

Regarding 3D Gaussian Splatting, CF-3DGS~\cite{CF-3DGS-2024} utilizes mono-depth information to refine the optimization of local 3DGS for relative pose estimation and subsequently learns a global 3DGS in a sequential manner. InstantSplat~\cite{fan2024instantsplat} targets sparse view scenes, initially employing DUSt3R~\cite{dust3r2024cvpr} to create a densely covered, pixel-aligned point set for initializing 3D Gaussian models, and then implements a parallel grid partitioning strategy to accelerate joint optimization. Jiang \etal~\cite{Jiang_2024sig} develops an incremental method for reconstructing camera poses and scenes, but it struggles with complex scenes and unordered images. 
% Similarly, CG-3DGS~\cite{sun2024correspondenceguidedsfmfree3dgaussian} progressively estimates camera poses using a pre-trained model by aligning the correspondences between rendered views and actual scenes. Free-SurGS~\cite{freesurgs2024} pioneers an SfM-free 3DGS method for reconstructing surgical scenes, overcoming challenges such as weak textures and photometric inconsistencies by minimizing the discrepancy between projection flow and optical flow.
%\pb{SF-3DGS-HT~\cite{ji2024sfmfree3dgaussiansplatting} introduced VFI into training as additional photometric constraints. They separated the whole scene into several local 3DGS models and then merged them hierarchically, which leads to a significant improvement on simple and dense view scenes.}
HT-3DGS~\cite{ji2024sfmfree3dgaussiansplatting} interpolates frames for training and splits the scene into local clips, using a hierarchical strategy to build 3DGS model. It works well for simple scenes, but fails with dramatic motions due to unstable interpolation and low efficiency.
% {While effective for simple scenes, it struggles with dramatic motion due to unstable view interpolation and suffers from low computational efficiency.}

However, most existing methods generally depend on sequentially ordered image inputs and incrementally optimize camera parameters and 3DGS, which often leads to drift errors and hinders achieving globally consistent results. Our work seeks to overcome these limitations.

\section{Methodology}

\subsection{Problem Definition}

Given a multivariate time series input $X \in \mathbb{R}^{C  \times T}$, multivariate time series forecasting tasks are designed to predict its future $F$ time steps $\hat{Y}\in \mathbb{R}^{C \times F}$ using past $T$ steps. $C $ is the number of variates or channels.

\subsection{Preliminary Analysis}

This section presents why RevIN~\citep{Kim_revin,liu2022non}, High-pass, and Low-pass filters fail to address the Mid-Frequency Spectrum Gap. Let the input univariate time series be $ x(t) $ with length $ T $ and target $ y(t) $ with length $ F $. 

\begin{definition}[Frequency Spectral Energy]\label{def:energy}
The Fourier transform of $x(t)$, $X(f)$, and its spectral energy $E_X(f)$ is given by:
\vspace{-0.2cm}
\begin{align}
X(f) = \sum_{t=0}^{T-1} x(t) e^{-i 2 \pi f t / {T-1}}, \quad &f = 0, 1, \dots, T-1\notag\\
E_X(f) = |X(f)|^2.
\end{align}
\vspace{-0.2cm}
\end{definition}

\textbf{Impact of RevIN on Frequency Spectrum \quad}
\begin{definition}[Reversible Instance Normalization]\label{def:RevIN}
Given a \textbf{forecast model} $ f: \mathbb{R}^T \rightarrow \mathbb{R}^F $ that generates a forecast $ \hat{y}(t) $ from a given input $x(t)$, RevIN is defined as:
\vspace{-0.2cm}
\begin{align}
&\hat{x}(t) = \frac{x(t) - \mu}{\sigma},\quad t = 0, 1, \dots, T-1\notag\\
&\hat{y}(t) = f(\hat{x}(t)), \quad \hat{y}(t)_{rev}= \hat{y}(t) \cdot \sigma + \mu,\notag\\
&\mu = \frac{1}{T} \sum_{t=0}^{T-1} x(t), \quad \sigma = \sqrt{\frac{1}{T} \sum_{t=0}^{T-1} (x(t) - \mu)^2}.
\end{align}
\vspace{-0.2cm}
\end{definition}

\begin{theorem} [Frequency Spectrum after RevIN] \label{theorem:RevIN}
\vspace{-0.2cm}
The spectral energy of $\hat{x}(t)$ (transformed using RevIN):
\begin{align}
E_{\hat{X}}(0)=0,& \quad f=0, \notag\\
E_{\hat{X}}(f) = \left( \frac{1}{\sigma} \right)^2 |X(f)|^2,&\quad f = 1,2,\dots, T-1 . 
\end{align}
\vspace{-0.2cm}
\end{theorem}
The proof is in Appendix~\ref{app:RevIN}. Theorem~\ref{theorem:RevIN} suggests that RevIN scales the absolute spectral energy by $ \sigma^2 $ but does not affect its relative distribution except $E_{\hat{X}}(0)=0$. Thus, RevIN preserves the relative spectral energy distribution and leaves the Mid-Frequency Spectrum Gap unresolved. \textit{However, our experiments still employ RevIN to ensure a fair comparison with other baselines.}
\begin{figure*}[h]
  \centering
  \includegraphics[width=1.\linewidth]{Faker/source/assets/jpg/ReFocus.jpg}
  \caption{General structure of \textbf{ReFocus}. `Adaptive Mid-Frequency Energy Optimizer (AMEO)' enhances mid-frequency components modeling, and `Energy-based Key-Frequency Picking Block' (EKPB) effectively captures shared Key-Frequency across channels}
  \label{fig:refocus}
\end{figure*}

\begin{figure*}[h]
  \centering
  \includegraphics[width=0.7\linewidth]{Faker/source/assets/jpg/ket.jpg}
  \caption{General process of the \textbf{Key-Frequency Enhanced Training strategy (KET)}, where spectral information from other channels is randomly introduced into each channel, to enhance the extraction of the shared Key-Frequency.}
  \label{fig:reshuffle}
\end{figure*}
\textbf{Impact of High- and Low-pass filter \quad}
We still define $\hat{x}(t)$ to be the filtered (processed) signal, obtained by applying a filter $H(f)$ (High/Low-pass filter). The filter $ H(f) $ is 1 in the passband (High/Low frequency) and 0 in the stopband (Middle frequency). So $E_{\hat{X}}(f)=0,\quad E_{\hat{X}}\leq E_X(f)$ for middle frequencies, which creates even larger gap.

\subsection{Overall Structure of The Proposed ReFocus}

In this section, we elucidate the overall architecture of \textbf{ReFocus}, depicted in Figure \ref{fig:refocus}. We define frequency domain projection as $D1\rightarrow D2$ representing a projection from dimension $D1$ to $D2$ in the frequency domain~\citep{xu2024fits}. Initially, we apply \textbf{AMEO} to the input $X \in \mathbb{R}^{C \times T}$, yielding the processed spectrum $ X_{am} \in \mathbb{R}^{C  \times T} $. Next, we use a projection $T\rightarrow D$ to transform $ X_{am}$ into the Variate Embedding $ X_{em} \in \mathbb{R}^{C  \times D}$~\citep{LiuiTransformer}. Then, $X_{em}$ go through $N$ \textbf{EKPB} to generate representation $H_{N+1}$, which is projected to obtain final prediction $\hat{Y}$. 

\textbf{Adaptive Mid-Frequency Energy Optimizer \quad}
Building upon the \textbf{Preliminary Analysis}, we propose a convolution- and residual learning-based solution to address the Mid-Frequency Spectrum Gap, which we denoted as AMEO. 
\begin{definition}[Adaptive Mid-Frequency Energy Optimizer]\label{def:AMEO}
AMEO is defined as:
\begin{align}
&\hat{x}(t) = x(t)-\frac{\beta}{K}\sum_{k=0}^{K-1} \tilde{x}(t+K-1-k),\notag\\
&\tilde{x}(t) =\notag\\
&\begin{cases}
x(t-(\frac{K}{2}+1)), \quad \text{if } \frac{K}{2}+1 \leq t < T+\frac{K}{2}+1, \\
0,  \quad\text{if } 0 \leq t < \frac{K}{2}+1 \text{ or } T+\frac{K}{2}+1 \leq t < T+K.
\end{cases}
\end{align}
\vspace{-0.2cm}
\end{definition}

It is equivalent to $x=x-\beta \cdot Conv(x)$. $Conv$ is a 1D convolution (Zero-padding at both ends, stride $s=1$, kernel size $K$, with values initialized as $ \frac{1}{K} $). $\beta \in \mathbb{R}^{1}$ is a hyperparameter.

\begin{theorem} [Frequency Spectrum after AMEO] \label{theorem:AMEO}
The spectral energy of $\hat{x}(t)$ obtained using AMEO:
\begin{align}
E_{\hat{X}}(f) =|X(f)|^2 \left\{1 - \beta \cdot \underbrace{\frac{1}{K} \sum_{k=0}^{K-1} e^{i 2 \pi f (\frac{3K}{2}-k -2) / {T-1}}}_{G(f)}\right\}^2
\end{align}
\vspace{-0.2cm}
\end{theorem}

The proof is in Appendix~\ref{app:AMEO}. We have $E_{\hat{X}}(f) =|X(f)|^2(1-\beta  \cdot G(f))^2$. Generally, $ G(f) $ behaves as a decay function, gradually reducing its value from \textbf{One} to \textbf{Zero}. Such \textbf{decay behavior} makes AMEO relatively enhances mid-frequency components, thus addressing the Mid-Frequency Spectrum Gap.

\textbf{Energy-based Key-Frequency Picking Block \quad} In each \textbf{EKPB}, the input $ H_i \in \mathbb{R}^{C  \times D} (H_1=X_{em}) $ is first processed through an MLP to generate $ H_i^k \in \mathbb{R}^{C  \times Q}$. Then, FFT is applied to get $ H_i^f \in \mathbb{R}^{C  \times (Q/2+1)}$. For $ H_i^f$, we calculate its energy, denoted as $ H_i^e \in \mathbb{R}^{C  \times (Q/2+1)}$. A cross-channel softmax is then applied to $ H_i^e$ per frequency to obtain a probability distribution $ H_i^{soft} \in \mathbb{R}^{C  \times (Q/2+1)}$. Using $H_i^{soft}$, we select values from $ H_i^f$ across channels for each frequency, resulting in $K^f_i \in \mathbb{R}^{1  \times (Q/2+1)}$, which represents the Shared Key-Frequency across all channels. Then iFFT is performed on $K^f_i$ to get $K_i\in \mathbb{R}^{1  \times Q}$, followed by projection $Q\rightarrow D$ and repeating (C times) to get $\hat{K}_i \in \mathbb{R}^{C  \times D}$. This $\hat{K}_i$ is point-wisely added to $\hat{H_i}\in \mathbb{R}^{C  \times D}$ , which is the projection of $ H_i$ using projection $D\rightarrow D$. Then, an MLP and $Add\&Norm$ is applied to the result $HK\in \mathbb{R}^{C  \times D}$ to fuse inter-series dependencies information, and another MLP and $Add\&Norm$ is used to capture intra-series variations~\citep{LiuiTransformer}. The output of each \textbf{EKPB} is $\hat{O_i} \in \mathbb{R}^{C  \times D}$, where $H_{i+1}=\hat{O_i}$.

\subsection{Key-Frequency Enhanced Training strategy}

In real-world time series, certain channels often exhibit spectral dependencies, which may not be fully captured in the training set, and the specific channels with such dependencies are also unknown~\citep{geweke1984freqchannel,Zhao2024freqchannel}. So this work borrows insight from recent advancement of mix-up in time series~\citep{zhou2023mixup,ansari2024mixup}, randomly introducing spectral information from other channels into each channel, to enhance the extraction of the shared Key-Frequency, as in Figure~\ref{fig:reshuffle}. Given a multivariate time series input $X \in \mathbb{R}^{C \times T}$ and its ground-truth $Y \in \mathbb{R}^{C \times F}$, we generate a pseudo sample pair: 

\begin{align}
X' = iFFT(FFT(X) +\alpha \cdot FFT(X[\text{perm},:]))&,  \notag\\ 
Y' = iFFT(FFT(Y) +\alpha \cdot FFT(Y[\text{perm},:]))&.
\end{align}

$\alpha \in \mathbb{R}^{C \times 1}$ is a weight vector sampled from a normal distribution, $\text{perm}$ is a reshuffled channel index. Since $FFT$ and $iFFT$ are linear operations, this mix-up process can be equivalently simplified in the \textbf{Time Domain}:
\begin{align}
X' = X +\alpha \cdot X[\text{perm},:]&,  \notag\\
Y' = Y +\alpha \cdot Y[\text{perm},:]&
 \end{align}
We alternate training between real and synthetic data to preserve the spectral dependencies in real samples. This combines the advantages of data augmentation, such as improved generalization, while mitigating potential drawbacks like over-smoothing and training instability~\citep{ryu2024tf,alkhalifah2022tf}.












\begin{table*}[h]
	\centering
	\small
	\caption{{\bf Comparison of \textbf{GGA} with the ERM baseline}. The top out-of-domain accuracies on five domain generalization benchmarks averaged over three trials, are presented.}
	\label{table:baseline-results}
	% \vspace{-0.5em}
	%	\renewcommand{\arraystretch}{1.1}
	%	\setlength{\tabcolsep}{3pt}
	%	\setlength{\abovetopsep}{0.5em}
	\begin{tabular}{l|ccccc|c}
		\toprule
		Algorithm & PACS & VLCS & OfficeHome & {TerraInc} & {DomainNet} & {Avg.}  \\
		\midrule
		ERM  & %\cite{vapnik1998statistical}   & 
		85.5\scriptsize$\pm0.2$             & 
		77.3\scriptsize$\pm0.4$             & 
		66.5\scriptsize$\pm0.3$             & 
		46.1\scriptsize$\pm1.8$             & 
		43.8\scriptsize$\pm0.3$             & 63.9  \\
		\midrule
		\textbf{GGA} (ours)                 & 
		\textbf{87.3}\scriptsize{$\pm0.4$}           & 
		\textbf{79.9}\scriptsize{$\pm0.4$}           & 
		\textbf{68.5}\scriptsize{$\pm0.2$}           &  
		\textbf{50.6}\scriptsize{$\pm0.1$}           & 
		\textbf{45.2}\scriptsize{$\pm0.2$}           & \textbf{66.3}  \\
		%		\midrule
		
		
		\bottomrule
		
	\end{tabular}
	% \vspace{-0.5em}
\end{table*}


\begin{table*}[h]
	\centering
	\small
	\caption{\textbf{Comparison with state-of-the-art domain generalization methods.} Out-of-domain accuracies on five domain generalization benchmarks are shown. Top performing methods are highlighted in \textbf{bold} while second-best are \textit{underlined}.
		The results marked by $\dagger, \ddagger$ are copied from Gulrajani and Lopez-Paz \cite{gulrajani2020domainbed} and Wang et al. \cite{wang2023sharpness}, respectively. For fair comparison, the training of each algorithm combined with \textbf{GGA}, were run on the respective codebases. Average accuracies and standard errors are calculated from three trials for the combination of GGA with past algorithms and from 5 trials for GGA. In {\color{green} green} and {\color{red} red}, we highlight the performance boost and decrease of applying GGA on top of each algorithm respectively, averaged over three trials. Due to computational resources, for DomainNet we do not combine GGA with previous methods.}
	\label{table:total-results}
	% \vspace{-0.5em}
	\renewcommand{\arraystretch}{1.1}
	\setlength{\tabcolsep}{3pt}
	\setlength{\abovetopsep}{0.5em}
	\begin{tabular}{l|cccc|c||c|c}
		\toprule
		Algorithm & PACS          & VLCS          & OfficeHome    & {TerraInc}    & {Avg.} & DomainNet & Total \\
		\midrule
		
		Mixstyle$^\ddagger$ \cite{zhou2021mixstyle}     & 
		85.2\scriptsize{$\pm0.3$} 
		\small{({\color{green}$+0.3$})}         & 
		77.9\scriptsize{$\pm0.5$} \small{({\color{green}$+0.6$})}          & 60.4\scriptsize{$\pm0.3$} \small{({\color{green}$+0.5$})}          & 44.0\scriptsize{$\pm0.7$}
		\small{({\color{green}$+1.1$})}          & 
		%		
		66.9 \small{({\color{green}$+0.6$})}         &
		34.0\scriptsize{$\pm0.1$}           & 60.3 \\
		
		GroupDRO$^\ddagger$ \cite{Sagawa2020GroupDRO}    & 
		84.4\scriptsize{$\pm0.8$}          
		\small{({\color{green}$+1.4$})}   & 
		76.7\scriptsize{$\pm0.6$}          
		\small{({\color{green}$+0.6$})}   & 
		66.0\scriptsize{$\pm0.7$}             
		\small{({\color{green}$+2.2$})}   & 
		43.2\scriptsize{$\pm1.1$}              
		\small{({\color{green}$+1.5$})}   & 
		%		 
		67.6 \small{({\color{green}$+1.4$})}         &
		33.3\scriptsize{$\pm0.2$}          & 60.7 \\
		
		MMD$^\ddagger$ \cite{li2018mmd}                  & 
		84.7\scriptsize{$\pm0.5$}          
		\small{({\color{green}$+0.8$})} & 
		77.5\scriptsize{$\pm0.9$}            
		\small{({\color{green}$+1.3$})} & 
		66.3\scriptsize{$\pm0.1$}          
		\small{({\color{green}$+1.6$})} & 
		42.2\scriptsize{$\pm1.6$}            
		\small{({\color{green}$+6.3$})} & 
		%		 
		67.7 \small{({\color{green}$+2.5$})} &      
		23.4\scriptsize{$\pm9.5$}           &  58.8 \\
		
		AND-mask \cite{shahtalebi2021sand} & 
		84.4\scriptsize{$\pm0.9$}  
		\small{({\color{green}$+0.1$})}& 
		78.1\scriptsize{$\pm0.9$}  
		\small{({\color{green}$+0.3$})}& 
		65.6\scriptsize{$\pm0.4$}  
		\small{({\color{green}$+1.2$})} & 
		44.6\scriptsize{$\pm0.3$} 
		\small{({\color{red}$-0.4$})}
		& 68.2 \small{({\color{green}$+0.3$})}    &
		37.2\scriptsize{$\pm0.6$} & 62.0 \\
		
		ARM$^\ddagger$ \cite{zhang2020arm}               &
		85.1\scriptsize{$\pm0.4$}          
		\small{({\color{green}$+0.5$})}& 
		77.6\scriptsize{$\pm0.3$} 
		\small{({\color{green}$+0.9$})}           & 
		64.8\scriptsize{$\pm0.3$} 
		\small{({\color{green}$+2.0$})}           & 
		45.5\scriptsize{$\pm0.3$} 
		\small{({\color{green}$+0.8$})}           & 
		%		 
		68.3 \small{({\color{green}$+1.1$})}         &
		35.5\scriptsize{$\pm0.2$}          & 61.7 \\
		
		IRM$^\dagger$ \cite{arjovsky2019irm}            & 
		83.5\scriptsize{$\pm0.8$}          
		\small{({\color{red}$-1.5$})}& 
		78.5\scriptsize{$\pm0.5$}          
		\small{({\color{red}$-0.9$})} & 
		64.3\scriptsize{$\pm2.2$}          
		\small{({\color{red}$-2.1$})}& 
		47.6\scriptsize{$\pm0.8$}          
		\small{({\color{red}$-3.9$})} & 
		%		 
		68.5 \small{({\color{red}$-2.1$})}        &
		33.9\scriptsize{$\pm2.8$}          & 61.6 \\
		
		MTL$^\ddagger$ \cite{blanchard2021mtl_marginal_transfer_learning}    & 
		84.6\scriptsize{$\pm0.5$}           
		\small{({\color{green}$+0.9$})} & 
		77.2\scriptsize{$\pm0.4$}          
		\small{({\color{green}$+2.0$})} & 
		66.4\scriptsize{$\pm0.5$}          
		\small{({\color{green}$+0.4$})} & 
		45.6\scriptsize{$\pm1.2$}          
		\small{({\color{green}$+0.6$})}&  
		%		 
		68.5 \small{({\color{green}$+0.9$})}        &
		40.6\scriptsize{$\pm0.1$}          & 62.9 \\
		
		
		VREx$^\ddagger$ \cite{krueger2020vrex}           & 
		84.9\scriptsize{$\pm0.6$} 
		\small{({\color{green}$+0.6$})}           & 
		78.3\scriptsize{$\pm0.2$} 
		\small{({\color{green}$+0.1$})}           & 
		66.4\scriptsize{$\pm0.6$} 
		\small{({\color{green}$+1.3$})}           & 
		46.4\scriptsize{$\pm0.6$} 
		\small{({\color{green}$+2.0$})}           & 
		%		 
		69.0 \small{({\color{green}$+1.0$})}      &
		33.6\scriptsize{$\pm2.9$}          & 61.9   \\
		
		MLDG$^\dagger$ \cite{li2018learning}                & 
		84.9\scriptsize{$\pm1.0$}          
		\small{({\color{green}$+0.7$})} &
		77.2\scriptsize{$\pm0.4$}          
		\small{({\color{green}$+1.3$})} & 
		66.8\scriptsize{$\pm0.6$}          
		\small{({\color{green}$+1.2$})} &
		47.7\scriptsize{$\pm0.2$}          
		\small{({\color{green}$+1.1$})} & 
		%		 
		69.2 \small{({\color{green}$+1.2$})}    &
		41.2\scriptsize{$\pm0.1$}            & 63.6  \\
		
		Mixup$^\dagger$ \cite{xu2020interdomain_mixup_aaai}             & 
		84.6\scriptsize{$\pm0.6$}            
		\small{({\color{green}$+1.2$})} & 
		77.4\scriptsize{$\pm0.6$}          
		\small{({\color{green}$+1.8$})} & 
		68.1\scriptsize{$\pm0.3$}            
		\small{({\color{green}$+1.0$})} & 
		47.9\scriptsize{$\pm0.8$}            
		\small{({\color{green}$+2.0$})} & 
		%		
		69.5   \small{({\color{green}$+1.5$})}    &
		39.2\scriptsize{$\pm0.1$}            & 63.4    \\
		
		
		SagNet$^\dagger$ \cite{nam2019sagnet}           &             
		86.3\scriptsize{$\pm0.2$} 
		\small{({\color{red}$-1.0$})} & 
		77.8\scriptsize{$\pm0.5$}          
		\small{({\color{green}$+0.9$})} & 
		68.1\scriptsize{$\pm0.1$}          
		\small{({\color{green}$+0.3$})}& 
		48.6\scriptsize{$\pm1.0$}          
		\small{({\color{green}$+0.7$})}& 
		%		 
		70.2 \small{({\color{green}$+0.4$})}        &
		40.3\scriptsize{$\pm0.1$}          &  64.2 \\
		
		
		CORAL$^\dagger$ \cite{sun2016coral}             & 
		86.2\scriptsize{$\pm0.3$}          
		\small{({\color{green}$+0.7$})}& 78.8\scriptsize{$\pm0.6$}          
		\small{({\color{red}$-0.4$})} & 
		68.7\scriptsize{$\pm0.3$}          
		\small{({\color{green}$+0.2$})} & 
		47.6\scriptsize{$\pm1.0$}          
		\small{({\color{green}$+0.3$})}& 
		%		 
		70.3    \small{({\color{green}$+0.2$})}    &
		41.5\scriptsize{$\pm0.1$}          & 64.5   \\
		
		\midrule
		
		RSC$^\dagger$ \cite{huang2020rsc}               & 
		85.2\scriptsize{$\pm0.9$}          
		\small{({\color{green}$+0.1$})}& 77.1\scriptsize{$\pm0.5$}          
		\small{({\color{green}$+0.2$})}& 
		65.5\scriptsize{$\pm0.9$} \small{($+0.0$)}        & 
		46.6\scriptsize{$\pm1.0$}          
		\small{({\color{green}$+0.2$})}& 
		%		 
		68.6 \small{({\color{green}$+0.1$})}   &
		38.9\scriptsize{$\pm0.5$}          & 62.7 \\
		
		Fish $^\ddagger$ \cite{shi2021gradient}                     & 
		85.5\scriptsize{$\pm0.3$}          
		\small{({\color{green}$+0.1$})} & 
		77.8\scriptsize{$\pm0.3$}          
		\small{({\color{green}$+0.9$})} & 
		68.6\scriptsize{$\pm0.4$}            
		\small{({\color{red}$-0.6$})} & 
		45.1\scriptsize{$\pm1.3$}            
		\small{({\color{green}$+3.8$})} & 
		%		 
		69.3 \small{({\color{green}$+1.0$})}      &
		42.7\scriptsize{$\pm0.2$}            &   63.9    \\ 
		
		SAM $^\ddagger$ \cite{foret2020sharpness}  & 
		85.8\scriptsize$\pm0.2$             
		\small{({\color{green}$+0.6$})} & 
		79.4\scriptsize$\pm0.1$              
		\small{({\color{green}$+0.7$})} & 
		\underline{69.6}\scriptsize$\pm0.1$              
		\small{({\color{green}$+0.4$})} & 
		43.3\scriptsize$\pm0.7$              
		\small{({\color{green}$+2.6$})} & 
		%		 
		69.5 \small{({\color{green}$+1.1$})}   &
		44.3\scriptsize$\pm0.0$              &  64.5 \\
		
		GSAM $^\ddagger$ \cite{zhuang2022surrogate}  & 
		85.9\scriptsize$\pm0.1$           
		\small{({\color{green}$+0.4$})} & 
		79.1\scriptsize$\pm0.2$             
		\small{({\color{green}$+1.0$})} & 
		69.3\scriptsize$\pm0.0$             
		\small{({\color{green}$+0.3$})}& 
		47.0\scriptsize$\pm0.8$             
		\small{({\color{green}$+0.6$})}& 
		%		 
		70.3  \small{({\color{green}$+0.2$})}  &
		44.6\scriptsize$\pm0.2$             &   65.1 \\
		
		SAGM $^\ddagger$ \cite{wang2023sharpness}       & 
		\underline{86.6}\scriptsize{$\pm0.2$}           
		\small{({\color{green}$+0.2$})}& 
		\textbf{80.0}\scriptsize{$\pm0.3$}           
		\small{({\color{red}$-0.3$})}& 
		\textbf{70.1}\scriptsize{$\pm0.2$}           
		\small{({\color{red}$-0.6$})}& 
		\underline{48.8}\scriptsize{$\pm0.9$}           
		\small{({\color{red}$-0.1$})}& 
		%		 
		\underline{71.4} \small{({\color{red}$-0.2$})}  &
		45.0\scriptsize{$\pm0.2$}           &  66.1 \\
		
		%		\midrule
		\midrule
		\textbf{GGA} (ours)                 & 
		\textbf{87.3}\scriptsize{$\pm0.4$}           & 
		\underline{79.9}\scriptsize{$\pm0.4$}           & 
		68.5\scriptsize{$\pm0.2$}           &  
		\textbf{50.6}\scriptsize{$\pm0.1$}    &  \textbf{71.7} &
		\textbf{45.2}\scriptsize{$\pm0.2$} & \textbf{66.3}\\
		%		\midrule
		
		
		\bottomrule
		
	\end{tabular}
	% \vspace{-0.5em}
\end{table*}


\section{Experiments}
\label{sec:experiments}

\subsection{Experimental setup and implementation details}

In our experiments, we follow the protocol of DomainBed 
\cite{gulrajani2020domainbed} and exhaustively evaluate our algorithm against state-of-the-art algorithms on five DG benchmarks, using the same dataset splits and model selection. For the hyperparameter search space, we follow \cite{cha2021swad} in order to avoid the high computational burden of DomainBed. The datasets included in the benchmarks are, PACS \cite{li2017deeper} (9,991 images, 7 classes, and 4 domains), VLCS \cite{fang2013unbiased} (10,729 images, 5 classes, and 4 domains), OfficeHome \cite{venkateswara2017deep} (15,588 images, 65 classes, and 4 domains), TerraIncognita \cite{beery2018recognition} (24,788 images, 10 classes, and 4 domains) and DomainNet \cite{peng2019moment} (586,575 images, 345 classes,
and 6 domains).

In all experiments, the \textit{leave-one-domain-out} cross-validation protocol is followed. Specifically, in each run a single domain is left out 
as the target (test) domain, while the rest of the domains are used for 
training. The final performance of each algorithm is calculated by averaging the top-1 accuracy on the target domain, with different train-validation splits. For training, we utilize a ResNet-50 
\cite{he2016deep} pretrained on ImageNet \cite{russakovsky2015imagenet} for the backbone feature extractor and ADAM for the optimizer. Regarding GGA, 
during training we let each algorithm run for several training iterations depending on the dataset and then begin the parameter space 
search, as described in Section \ref{sec:methods}. For the neighborhood size
in the search process, we set $\rho$ to $0.0005$\footnote{The selection of $\rho$ was based on previous algorithms that implement weight perturbations in ResNet-50 networks \cite{foret2020sharpness, wang2023sharpness, le2024gradient}. The sensitivity analysis presented in subsection \ref*{sensitivity} also reveals that $\rho = 0.0005$ yields the best performance.} and search for a total of $A = 250$ steps before moving to the next mini-batch of 48 samples from each domain, in each dataset. In all experiments, we perform search iterations for $100$ different mini-batches during early training stages. The rest of the hyperparameters, such as learning rate, weight decay and dropout rate, are tuned according to \cite{cha2021swad} and are presented in Table \ref{table:hyperparameters}.
To account for the variability introduced in the random search of GGA, we 
repeat the experiments with 5 different seeds for each dataset. All models were trained on a cluster containing $4\times40$GB NVIDIA A$100$ GPU cards, split into $8$ $20$GB virtual MIG devices and $1\times24$GB NVIDIA RTX A$5000$ GPU card.

\begin{table}[h]
	\centering
	%\footnotesize
		\resizebox{\linewidth}{!}{
		\begin{tabular}{lrrrrr} 
			\toprule
			\textbf{Hyperparameter}& PACS & VLCS  & OH & TI & DN\\ \midrule
			Learning rate&{3e-5}&{1e-5}&{1e-5}& 1e-5 & 3e-5 \\ 
			Dropout & 0.5 & 0.5 & 0.5 & 0.5 & 0.5 \\ 
			Weight decay&1e-4 & 1e-4&1e-4 & 1e-4 & 1e-4 \\ 
			Training Steps & 5000 & 5000 & 5000 & 5000 & 15000 \\
			$\rho$ & {5e-5} & {5e-5} & {5e-5} & {5e-5} & 5e-5 \\
			GGA Start-End & 100-150, 1500-1550 & 100-200 & 100-200 & 500-600 & 100-200 \\ 
			\bottomrule
		\end{tabular}
				}
	\caption{Hyperparameters for DG experiments. OH, TI and DN stand for OfficeHome, TerraIncognita and DomainNet respectively. $\rho$ is the parameter space search used during the annealing steps in GGA, while the last row indicates the training iterations during which GGA occurs.}
	\label{table:hyperparameters}
\end{table}

\subsection{Comparative Evaluation}
\label{results}

The average OOD performances of the baseline vanilla ERM \cite{vapnik1998statistical} and state-of-the-art DG methods on a total of 5 DG benchmarks, are reported in Tables \ref{table:baseline-results} and \ref{table:total-results} respectively. The results for each separate domain in each benchmark are reported in the Appendix. In the experiments, we apply GGA on-top of the rest of the DG algorithms and show that its properties generalize to other methods as well, boosting their overall performance. It should also be noted that GGA can be adapted to training scenarios where distributions shifts are present in training data. 

Initially, to validate whether the application of GGA in the early stages of 
model training leads to models with increased generalizabilty, we compare the 
results with the vanilla ERM baseline. As presented in Table \ref{table:baseline-results}, GGA is able to boost the 
performance of the baseline model by an average of 2.4\% on all 
benchmarks, demonstrating the efficacy of the proposed algorithm. For further evaluation, we also compare GGA with state-of-the-art DG methods \cite{li2018mmd, zhou2021mixstyle, Sagawa2020GroupDRO, arjovsky2019irm, zhang2020arm, krueger2020vrex, shahtalebi2021sand, ganin2016dann, huang2020rsc, blanchard2021mtl_marginal_transfer_learning, xu2020interdomain_mixup_aaai, li2018learning, shi2021gradient, nam2019sagnet, sun2016coral, foret2020sharpness, zhuang2022surrogate, wang2023sharpness, vapnik1998statistical}, before applying GGA to them as well. We note that we only include previous works that have implemented a ResNet-50 as the backbone encoder and do not use additional components or ensembles. In Table \ref{table:total-results} we differentiate between the methods that operate 
on model gradients \cite{huang2020rsc, shi2021gradient, foret2020sharpness, zhuang2022surrogate, wang2023sharpness} and the rest of the algorithms. Even without its application to other methods, GGA is able to surpass most of the previously proposed algorithms, while also setting the state-of-the-art on PACS ($+0.7\%$) and on the challenging TerraIncognita ($+1.8$) dataset, while remaining competitive in the other two. What's more important is that the application of GGA in conjunction with the rest of the DG methods, proves beneficial and ultimately boosts their overall performance in almost each case by an average of around $1\%$ and in some cases up to even $6.3\%$. With regards to IRM, which seems to be significantly impacted negatively by the application of GGA, its learning objective emphasizes on simultaneously minimizing the training loss of each source domain and not necessarily on the pairwise agreement of gradients among domains. It is therefore not able to converge to a good solution after the initial model's weights have been perturbed during training.

From the experimental results, it is evident that the application of GGA 
and its search for parameter values where gradients align between domains is beneficial to model training. By introducing the proposed annealing step before the final stages of  training, the majority of the models exceed their previous performance and exhibit improved generalization capabilities.

\begin{figure*}[t]
	\centering
	\includegraphics[width=\textwidth]{vlcs-grads.png}
	\caption{Impact of GGA on gradient alignment during model training on the VLCS dataset. As illustrated, in the case of vanilla ERM, even though the training loss is minimized, the average cosine similarity among domain gradients remains low. In the case of GGA however, after the algorithm searches for points in the parameter space with increased gradient similarity, the gradients continue to agree during training, while the total training loss is also minimized.}
	\label{vlcs-grads}
\end{figure*}

\subsection{Evaluating the impact of GGA on gradient disagreement}
\label{gga-evaluation}

As discussed in Section \ref{sec:methods}, when a training dataset is composed
as a mixture of multiple domains, conflicting gradients between mini-batches drawn
from each domain lead to models that do not infer based on domain-invariant
features and which generalize to previously unseen data samples, but are hindered
by domain-specific, spurious correlations. This is evident in the case of
vanilla models trained via ERM where the average gradient similarity among
domains continues to remain low upon reaching a local minima. Our hypothesis is
that this behavior can be avoided by searching for a parameter set of common
agreement between domains before optimizing via gradient descent.

To demonstrate the operation of the proposed algorithm in practice against ERM, we calculate the average gradient cosine similarity between mini-batches from source domains during training for the VLCS dataset, along 
with the training loss in each iteration. As a result, each sub-figure in Figure
\ref{vlcs-grads} illustrates the progression of the training gradient alignment
between domains, against the total training loss. 

As expected, in the very initial iterations the gradients of the pretrained
model parameters point towards a common direction. However, in the case of ERM
as training progresses and the loss is minimized, the domain gradients begin to
disagree leading the model to converge to undesirable minima that do not
generalize across domains. On the other hand, when GGA is applied the model
searches for parameters such that gradients are aligned before continuing
training. This is illustrated by the spike in gradient similarity,
during iterations $100$ up to $200$. After GGA concludes, we observe that the
model continues training by descending into minima where gradients agree among
domains.

\subsection{Sensitivity Analysis}
\label{sensitivity}

\begin{figure}[t]
	\centering
%	% First figure
%	\begin{subfigure}{0.3\textwidth}
%		\centering
%		\includegraphics[width=\textwidth]{gradient-conflicts-oh.png}
%		\caption{Gradient conflicts - ERM}
%	\end{subfigure}%
%	\hfill
	% Second figure
	\begin{subfigure}{0.5\columnwidth}
		\centering
		\includegraphics[width=\textwidth]{rho-ablation.png}
%		\caption{Sensitivity analysis of $\rho$}
	\end{subfigure}%
	\hfill
	% Third figure
	\begin{subfigure}{0.5\columnwidth}
		\centering
		\includegraphics[width=\textwidth]{init-ablation.png}
%		\caption{GGA initialization analysis}
	\end{subfigure}
\caption{Sensitivity analysis of the parameter space search magnitude $\rho$ and the training stage application of GGA. The analysis on PACS reveals smaller weight perturbations and gradient annealing during early training iterations lead to increased model performance.}
\label{sensitivity-fig}
\end{figure}

The two core parameters of GGA, are the size of the parameter search $\rho$ and
the moment of our methods implementation during training, i.e., during the
early, mid or late training stages. To justify their selection, we conduct a
sensitivity analysis (Fig. \ref{sensitivity-fig}) by varying one of the above
parameters while fixing the other at its optimal value.

Regarding the magnitude of weight perturbations during the application of GGA,
we found that the optimal value was $\rho = 5e-5$. As illustrated in Figure
\ref{sensitivity-fig}, a larger magnitude of perturbation led to decreased model
performance. Intuitively, the application of larger noise to the model
parameters leads to sets that are not close to the solution, making it
increasingly difficult for the model to converge. On the other hand, smaller
perturbations seem to have little to no effect on training, as the search is
limited to spaces near the current parameters, which is why the model
performance falls back close to that of ERM. With regards to the stage of
training during which GGA will be applied, we found that the models yielded
better performance when the search was initialized in earlier stages.We hypothesize that applying perturbations near the end of training displaces the model from a local optimum, requiring additional iterations to converge.

\section{Conclusions}
\label{sec:conclusions}

In this paper, we introduce a novel sketch-to-image translation technique that uses a learnable lightweight mapping network (LCTN) for latent code translation from sketch to image domain, followed by $k$ forward diffusion and $T$ backward denoising steps through a pre-trained text-to-image LDM. We show that by selecting an optimal value for $k \sim [1, T]$ near the upper threshold ($k \approx T$, $k < T$), it is possible to generate highly detailed photorealistic images that closely resemble the intended structures in the given sketches. Our experiments demonstrate that the proposed technique outperforms the existing methods in most visual and analytical comparisons across multiple datasets. Additionally, we show that the proposed method retains structural consistency across different visual styles, allowing photorealistic style manipulation in the generated images.


{
    \small
    \bibliographystyle{ieeenat_fullname}
    \bibliography{main}
}

\subfile{supp/supplementary.tex}

% WARNING: do not forget to delete the supplementary pages from your submission 
% \clearpage
\pagenumbering{gobble}
\maketitlesupplementary

\section{Additional Results on Embodied Tasks}

To evaluate the broader applicability of our EgoAgent's learned representation beyond video-conditioned 3D human motion prediction, we test its ability to improve visual policy learning for embodiments other than the human skeleton.
Following the methodology in~\cite{majumdar2023we}, we conduct experiments on the TriFinger benchmark~\cite{wuthrich2020trifinger}, which involves a three-finger robot performing two tasks: reach cube and move cube. 
We freeze the pretrained representations and use a 3-layer MLP as the policy network, training each task with 100 demonstrations.

\begin{table}[h]
\centering
\caption{Success rate (\%) on the TriFinger benchmark, where each model's pretrained representation is fixed, and additional linear layers are trained as the policy network.}
\label{tab:trifinger}
\resizebox{\linewidth}{!}{%
\begin{tabular}{llcc}
\toprule
Methods       & Training Dataset & Reach Cube & Move Cube \\
\midrule
DINO~\cite{caron2021emerging}         & WT Venice        & 78.03     & 47.42     \\
DoRA~\cite{venkataramanan2023imagenet}          & WT Venice        & 81.62     & 53.76     \\
DoRA~\cite{venkataramanan2023imagenet}          & WT All           & 82.40     & 48.13     \\
\midrule
EgoAgent-300M & WT+Ego-Exo4D      & 82.61    & 54.21      \\
EgoAgent-1B   & WT+Ego-Exo4D      & \textbf{85.72}      & \textbf{57.66}   \\
\bottomrule
\end{tabular}%
}
\end{table}

As shown in Table~\ref{tab:trifinger}, EgoAgent achieves the highest success rates on both tasks, outperforming the best models from DoRA~\cite{venkataramanan2023imagenet} with increases of +3.32\% and +3.9\% respectively.
This result shows that by incorporating human action prediction into the learning process, EgoAgent demonstrates the ability to learn more effective representations that benefit both image classification and embodied manipulation tasks.
This highlights the potential of leveraging human-centric motion data to bridge the gap between visual understanding and actionable policy learning.



\section{Additional Results on Egocentric Future State Prediction}

In this section, we provide additional qualitative results on the egocentric future state prediction task. Additionally, we describe our approach to finetune video diffusion model on the Ego-Exo4D dataset~\cite{grauman2024ego} and generate future video frames conditioned on initial frames as shown in Figure~\ref{fig:opensora_finetune}.

\begin{figure}[b]
    \centering
    \includegraphics[width=\linewidth]{figures/opensora_finetune.pdf}
    \caption{Comparison of OpenSora V1.1 first-frame-conditioned video generation results before and after finetuning on Ego-Exo4D. Fine-tuning enhances temporal consistency, but the predicted pixel-space future states still exhibit errors, such as inaccuracies in the basketball's trajectory.}
    \label{fig:opensora_finetune}
\end{figure}

\subsection{Visualizations and Comparisons}

More visualizations of our method, DoRA, and OpenSora in different scenes (as shown in Figure~\ref{fig:supp pred}). For OpenSora, when predicting the states of $t_k$, we use all the ground truth frames from $t_{0}$ to $t_{k-1}$ as conditions. As OpenSora takes only past observations as input and neglects human motion, it performs well only when the human has relatively small motions (see top cases in Figure~\ref{fig:supp pred}), but can not adjust to large movements of the human body or quick viewpoint changes (see bottom cases in Figure~\ref{fig:supp pred}).

\begin{figure*}
    \centering
    \includegraphics[width=\linewidth]{figures/supp_pred.pdf}
    \caption{Retrieval and generation results for egocentric future state prediction. Correct and wrong retrieval images are marked with green and red boundaries, respectively.}
    \label{fig:supp pred}
\end{figure*}

\begin{figure*}[t]
    \centering
    \includegraphics[width=0.9\linewidth]{figures/motion_prediction.pdf}
    \vspace{-0.5mm}
    \caption{Motion prediction results in scenes with minor changes in observation.}
    \vspace{-1.5mm}
    \label{fig:motion_prediction}
\end{figure*}

\subsection{Finetuning OpenSora on Ego-Exo4D}

OpenSora V1.1~\cite{opensora}, initially trained on internet videos and images, produces severely inconsistent results when directly applied to infer future videos on the Ego-Exo4D dataset, as illustrated in Figure~\ref{fig:opensora_finetune}.
To address the gap between general internet content and egocentric video data, we fine-tune the official checkpoint on the Ego-Exo4D training set for 50 epochs.
OpenSora V1.1 proposed a random mask strategy during training to enable video generation by image and video conditioning. We adopted the default masking rate, which applies: 75\% with no masking, 2.5\% with random masking of 1 frame to 1/4 of the total frames, 2.5\% with masking at either the beginning or the end for 1 frame to 1/4 of the total frames, and 5\% with random masking spanning 1 frame to 1/4 of the total frames at both the beginning and the end.

As shown in Fig.~\ref{fig:opensora_finetune}, despite being trained on a large dataset, OpenSora struggles to generalize to the Ego-Exo4D dataset, producing future video frames with minimal consistency relative to the conditioning frame. While fine-tuning improves temporal consistency, the moving trajectories of objects like the basketball and soccer ball still deviate from realistic physical laws. Compared with our feature space prediction results, this suggests that training world models in a reconstructive latent space is more challenging than training them in a feature space.


\section{Additional Results on 3D Human Motion Prediction}

We present additional qualitative results for the 3D human motion prediction task, highlighting a particularly challenging scenario where egocentric observations exhibit minimal variation. This scenario poses significant difficulties for video-conditioned motion prediction, as the model must effectively capture and interpret subtle changes. As demonstrated in Fig.~\ref{fig:motion_prediction}, EgoAgent successfully generates accurate predictions that closely align with the ground truth motion, showcasing its ability to handle fine-grained temporal dynamics and nuanced contextual cues.

\section{OpenSora for Image Classification}

In this section, we detail the process of extracting features from OpenSora V1.1~\cite{opensora} (without fine-tuning) for an image classification task. Following the approach of~\cite{xiang2023denoising}, we leverage the insight that diffusion models can be interpreted as multi-level denoising autoencoders. These models inherently learn linearly separable representations within their intermediate layers, without relying on auxiliary encoders. The quality of the extracted features depends on both the layer depth and the noise level applied during extraction.


\begin{table}[h]
\centering
\caption{$k$-NN evaluation results of OpenSora V1.1 features from different layer depths and noising scales on ImageNet-100. Top1 and Top5 accuracy (\%) are reported.}
\label{tab:opensora-knn}
\resizebox{0.95\linewidth}{!}{%
\begin{tabular}{lcccccc}
\toprule
\multirow{2}{*}{Timesteps} & \multicolumn{2}{c}{First Layer} & \multicolumn{2}{c}{Middle Layer} & \multicolumn{2}{c}{Last Layer} \\
\cmidrule(r){2-3}   \cmidrule(r){4-5}  \cmidrule(r){6-7}  & Top1           & Top5           & Top1            & Top5           & Top1           & Top5          \\
\midrule
32        &  6.10           & 18.20             & 34.04               & 59.50             & 30.40             & 55.74             \\
64        & 6.12              & 18.48              & 36.04               & 61.84              & 31.80         & 57.06         \\
128       & 5.84             & 18.14             & 38.08               & 64.16              & 33.44       & 58.42 \\
256       & 5.60             & 16.58              & 30.34               & 56.38              &28.14          & 52.32        \\
512       & 3.66              & 11.70            & 6.24              & 17.62              & 7.24              & 19.44  \\ 
\bottomrule
\end{tabular}%
}
\end{table}

As shown in Table~\ref{tab:opensora-knn}, we first evaluate $k$-NN classification performance on the ImageNet-100 dataset using three intermediate layers and five different noise scales. We find that a noise timestep of 128 yields the best results, with the middle and last layers performing significantly better than the first layer.
We then test this optimal configuration on ImageNet-1K and find that the last layer with 128 noising timesteps achieves the best classification accuracy.

\section{Data Preprocess}
For egocentric video sequences, we utilize videos from the Ego-Exo4D~\cite{grauman2024ego} and WT~\cite{venkataramanan2023imagenet} datasets.
The original resolution of Ego-Exo4D videos is 1408×1408, captured at 30 fps. We sample one frame every five frames and use the original resolution to crop local views (224×224) for computing the self-supervised representation loss. For computing the prediction and action loss, the videos are downsampled to 224×224 resolution.
WT primarily consists of 4K videos (3840×2160) recorded at 60 or 30 fps. Similar to Ego-Exo4D, we use the original resolution and downsample the frame rate to 6 fps for representation loss computation.
As Ego-Exo4D employs fisheye cameras, we undistort the images to a pinhole camera model using the official Project Aria Tools to align them with the WT videos.

For motion sequences, the Ego-Exo4D dataset provides synchronized 3D motion annotations and camera extrinsic parameters for various tasks and scenes. While some annotations are manually labeled, others are automatically generated using 3D motion estimation algorithms from multiple exocentric views. To maximize data utility and maintain high-quality annotations, manual labels are prioritized wherever available, and automated annotations are used only when manual labels are absent.
Each pose is converted into the egocentric camera's coordinate system using transformation matrices derived from the camera extrinsics. These matrices also enable the computation of trajectory vectors for each frame in a sequence. Beyond the x, y, z coordinates, a visibility dimension is appended to account for keypoints invisible to all exocentric views. Finally, a sliding window approach segments sequences into fixed-size windows to serve as input for the model. Note that we do not downsample the frame rate of 3D motions.

\section{Training Details}
\subsection{Architecture Configurations}
In Table~\ref{tab:arch}, we provide detailed architecture configurations for EgoAgent following the scaling-up strategy of InternLM~\cite{team2023internlm}. To ensure the generalization, we do not modify the internal modules in InternML, \emph{i.e.}, we adopt the RMSNorm and 1D RoPE. We show that, without specific modules designed for vision tasks, EgoAgent can perform well on vision and action tasks.

\begin{table}[ht]
  \centering
  \caption{Architecture configurations of EgoAgent.}
  \resizebox{0.8\linewidth}{!}{%
    \begin{tabular}{lcc}
    \toprule
          & EgoAgent-300M & EgoAgent-1B \\
          \midrule
    Depth & 22    & 22 \\
    Embedding dim & 1024  & 2048 \\
    Number of heads & 8     & 16 \\
    MLP ratio &    8/3   & 8/3 \\
    $\#$param.  & 284M & 1.13B \\
    \bottomrule
    \end{tabular}%
    }
  \label{tab:arch}%
\end{table}%

Table~\ref{tab:io_structure} presents the detailed configuration of the embedding and prediction modules in EgoAgent, including the image projector ($\text{Proj}_i$), representation head/state prediction head ($\text{MLP}_i$), action projector ($\text{Proj}_a$) and action prediction head ($\text{MLP}_a$).
Note that the representation head and the state prediction head share the same architecture but have distinct weights.

\begin{table}[t]
\centering
\caption{Architecture of the embedding ($\text{Proj}_i$, $\text{Proj}_a$) and prediction ($\text{MLP}_i$, $\text{MLP}_a$) modules in EgoAgent. For details on module connections and functions, please refer to Fig.~2 in the main paper.}
\label{tab:io_structure}
\resizebox{\linewidth}{!}{%
\begin{tabular}{lcl}
\toprule
       & \multicolumn{1}{c}{Norm \& Activation} & \multicolumn{1}{c}{Output Shape}  \\
\midrule
\multicolumn{3}{l}{$\text{Proj}_i$ (\textit{Image projector})} \\
\midrule
Input image  & -          & 3$\times$224$\times$224 \\
Conv 2D (16$\times$16) & -       & Embedding dim$\times$14$\times$14    \\
\midrule
\multicolumn{3}{l}{$\text{MLP}_i$ (\textit{State prediction head} \& \textit{Representation head)}} \\
\midrule
Input embedding  & -          & Embedding dim \\
Linear & GELU       & 2048          \\
Linear & GELU       & 2048          \\
Linear & -          & 256           \\
Linear & -          & 65536     \\
\midrule
\multicolumn{3}{l}{$\text{Proj}_a$ (\textit{Action projector})} \\
\midrule
Input pose sequence  & -          & 4$\times$5$\times$17 \\
Conv 2D (5$\times$17) & LN, GELU   & Embedding dim$\times$1$\times$1    \\
\midrule
\multicolumn{3}{l}{$\text{MLP}_a$ (\textit{Action prediction head})} \\
\midrule
Input embedding  & -          & Embedding dim$\times$1$\times$1 \\
Linear & -          & 4$\times$5$\times$17     \\
\bottomrule
\end{tabular}%
}
\end{table}


\subsection{Training Configurations}
In Table~\ref{tab:training hyper}, we provide the detailed training hyper-parameters for experiments in the main manuscripts.

\begin{table}[ht]
  \centering
  \caption{Hyper-parameters for training EgoAgent.}
  \resizebox{0.86\linewidth}{!}{%
    \begin{tabular}{lc}
    \toprule
    Training Configuration & EgoAgent-300M/1B \\
    \midrule
    Training recipe: &  \\
    optimizer & AdamW~\cite{loshchilov2017decoupled} \\
    optimizer momentum & $\beta_1=0.9, \beta_2=0.999$ \\
    \midrule
    Learning hyper-parameters: &  \\
    base learning rate & 6.0E-04 \\
    learning rate schedule & cosine \\
    base weight decay & 0.04 \\
    end weight decay & 0.4 \\
    batch size & 1920 \\
    training iters & 72,000 \\
    lr warmup iters & 1,800 \\
    warmup schedule & linear \\
    gradient clip & 1.0 \\
    data type & float16 \\
    norm epsilon & 1.0E-06 \\
    \midrule
    EMA hyper-parameters: &  \\
    momentum & 0.996 \\
    \bottomrule
    \end{tabular}%
    }
  \label{tab:training hyper}%
\end{table}%

\clearpage


\end{document}
