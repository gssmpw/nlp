% CVPR 2025 Paper Template; see https://github.com/cvpr-org/author-kit

\documentclass[10pt,onecolumn,letterpaper]{article}

%%%%%%%%% PAPER TYPE  - PLEASE UPDATE FOR FINAL VERSION
%\usepackage{cvpr}              % To produce the CAMERA-READY version
%\usepackage[review]{cvpr}      % To produce the REVIEW version
 \usepackage[pagenumbers]{cvpr} % To force page numbers, e.g. for an arXiv version
%%%% My packages %%%%
\usepackage{amsmath}
\usepackage{subcaption}
\usepackage{multicol}
\usepackage{graphicx}
\usepackage{xcolor}
\usepackage{algpseudocode}
\usepackage{algorithm}
\usepackage{amsmath}
\usepackage{bm}
\graphicspath{{figures/}}

\usepackage[switch]{lineno}


\usepackage{tikz}
\usetikzlibrary{bayesnet}

% Import additional packages in the preamble file, before hyperref
\newcommand{\CG}{\mathcal{G}\xspace}
\newcommand{\CV}{\mathcal{V}\xspace}
\newcommand{\CE}{\mathcal{E}\xspace}
\newcommand{\CA}{\mathcal{A}\xspace}
\newcommand{\CF}{\mathcal{F}\xspace}
\newcommand{\CR}{\mathcal{R}\xspace}
\newcommand{\CB}{\mathcal{B}\xspace}
\newcommand{\CX}{\mathcal{X}\xspace}
\newcommand{\CK}{\mathcal{K}\xspace}
\newcommand{\CM}{\mathcal{M}\xspace}
\newcommand{\CC}{\mathcal{C}\xspace}
\newcommand{\CL}{\mathcal{L}\xspace}
\newcommand{\CI}{\mathcal{I}\xspace}
\newcommand{\CQ}{\mathcal{Q}\xspace}
\newcommand{\CO}{\mathcal{O}\xspace}
\newcommand{\CP}{\mathcal{P}\xspace}
\newcommand{\CS}{\mathcal{S}\xspace}
\newcommand{\CT}{\mathcal{T}\xspace}
\newcommand{\CJ}{\mathcal{J}\xspace}
\usepackage[para]{footmisc}
\usepackage{subfig}
% \usepackage{subcaption}
% \usepackage{array}
% \usepackage{colortbl}



% It is strongly recommended to use hyperref, especially for the review version.
% hyperref with option pagebackref eases the reviewers' job.
% Please disable hyperref *only* if you encounter grave issues, 
% e.g. with the file validation for the camera-ready version.
%
% If you comment hyperref and then uncomment it, you should delete *.aux before re-running LaTeX.
% (Or just hit 'q' on the first LaTeX run, let it finish, and you should be clear).
\definecolor{cvprblue}{rgb}{0.21,0.49,0.74}
\usepackage[pagebackref,breaklinks,colorlinks,citecolor=cvprblue]{hyperref}

%% New commands
\newcommand{\card}[1]{\lvert\mathcal{#1}\rvert}
\def\thesection{\Alph{section}}
%%%%%%%%% PAPER ID  - PLEASE UPDATE
\def\paperID{15137} % *** Enter the Paper ID here
\def\confName{CVPR}
\def\confYear{2025}

%%%%%%%%% TITLE - PLEASE UPDATE
\title{Gradient-Guided Annealing for Domain Generalization}

%%%%%%%%% AUTHORS - PLEASE UPDATE
\author{Aristotelis Ballas\\
	Dpt of Informatics and Telematics\\
	Harokopio University of Athens\\
	Omirou 9, Tavros, Athens, Greece\\
	{\tt\small aballas@hua.gr}
	% For a paper whose authors are all at the same institution,
	% omit the following lines up until the closing ``}''.
% Additional authors and addresses can be added with ``\and'',
% just like the second author.
% To save space, use either the email address or home page, not both
\and
Christos Diou\\
Dpt of Informatics and Telematics\\
Harokopio Univesity of Athens\\
Omirou 9, Tavros, Athens, Greece\\
{\tt\small cdiou@hua.gr}
}

\begin{document}
%\maketitle

\clearpage
\setcounter{page}{1}
\maketitlesupplementary

The following materials are provided in this supplementary file:
\begin{itemize}
	\item An extended literature review discussion, helpful for navigating the Domain Generalization literature under the scope of computer vision.
	\item A computational analysis regarding the application of GGA.
	\item Detailed results for each dataset domain and algorithm, presented in Table 2 of the main text, along with extra experiments on the ColoredMNIST and RotatedMNIST datasets.
\end{itemize}

\section{Extended Literature Review}
\label{sec:extended-lit-review}

Domain Generalization (DG) \cite{wang2022generalizing} is arguably one of the 
most difficult and fundamental problems of Machine Learning (ML) today. 
Unsurprisingly, a vast number of researchers have poured effort into advancing the field, where findings have been applied to various areas, such 
as Natural Language Processing \cite{hupkes2023taxonomy}, Reinforcement Learning \cite{li2018learning}, Healthcare and Medicine \cite{9298838, 10233054}, Time-Series forecasting \cite{du2021adarnn}, Fault Diagnosis 
\cite{9174912} and, of course, Computer Vision \cite{wang2022generalizing}. 
Even though not covering the entire field of DG, this section aims to present 
a taxonomy of the general DG methodologies developed in CV, for producing 
robust models that can generalize to previously unseen data, and attempts to 
assist potential readers navigate the past literature, while also 
categorizing our proposed method among its predecessors. 
Domain Generalization methods can be categorized into three major groups, depending on their operation during the process of model training, namely: 
(a) Data Manipulation, (b) Representation Learning and, (c) Learning 
Algorithm. Furthermore, as mentioned in the main text, DG algorithms can 
either leverage domain labels during training (multi-source), or completely
disregard the knowledge of existing domain shifts in their training data and 
handle them as a single distribution (single-source). 
%An illustration of the above taxonomy is presented in Fig. \ref{fig:taxonomy}. 

\textbf{Data Manipulation}. As its name suggests, methods 
included in this group focus on either perturbing existing samples (\textit{data augmentation}) or creating novel ones (\textit{data generation}), 
in order to regularize the training of machine learning models, avoid 
overfitting and improve their generalizability. The basic idea in data 
manipulation methodologies is to simulate domain shift by creating diverse 
data samples, which can in turn mimic the entirety of distributions present 
in the input space. Regarding data augmentation, most popular techniques 
include traditional image transformations, such as random flip, rotation and 
color distortion. Even though these augmentations can be randomly applied 
during training, without needing domain labels, it has been shown that their 
selection significantly affects model performance. For example, the 
authors of \cite{volpi2019addressing} define novel augmentation rules that 
push the perturbed images to diverge as much as possible from the original 
ones. Additionally, image augmentations prove effective towards overcoming 
domain shifts in medical image classification \cite{otalora2019staining, zhang2020generalizing}, where transformations can replicate shifts caused by 
the use of different devices. On the other hand, multiple data augmentation 
methods were also inspired by adversarial attacks and use adversarial gradients to distort the input images \cite{volpi2018generalizing, qiao2020learning}, or use randomly initialized convolutional networks for transforming samples \cite{choi2023progressive}. These techniques act as regularizers during model training, allowing them to learn generalizable image representations. The generation of novel data domains is also a well 
researched area in the data manipulation group. In addition to using domain gradients for synthesizing novel domains \cite{shankar2018generalizing}, 
several methods took advantage of style transfer \cite{huang2017arbitrary} 
and either map the styles of images to that existing source domains \cite{borlino2021rethinking} or create novel styles \cite{yue2019domain}. On a similar note, mixing the styles of training images by conventional methods  \cite{xu2020interdomain_mixup_aaai, zhou2021mixstyle} or with the generative models \cite{wang2024multi} also proves beneficial.



\textbf{Representation Learning}. This group of methods is arguably the most 
prominent in DG and has been the central focus of ML \cite{6472238}. 
Following the formulation in the main text, given a labeling function $h$ 
that maps input observations $\bm{x}$ to their labels $y$, we can decompose it into $h = f \circ g$, where $g$ is a parametric function that learns 
representations of $x$ and $f$ is the classifier function. The goal of 
representation learning can be summarized as follows:

\begin{equation}
	\min_{f, g} \mathbb{E}_{x,y} \ell(f(g(\bm{x}; \bm{\theta})), y) +\lambda\ell_\text{reg}
\end{equation}
where $\ell$ the loss function to be minimized and $\ell_\text{reg}$ a 
regularizer. Methods included in this group, focus on learning a robust and generalizable representation learning function $g$. The algorithms included 
in this group can be further categorized into three sub-groups. \textit{Feature disentanglement} \cite{Zhang_2022_CVPR} methods intend to extract disentangled feature vectors from samples, where each dimension can be linked to a subset of data generating factors. The main idea is to produce a model that extracts a representation that can be further decomposed into domain-specific, domain-invariant, and class-specific features. To that end, the authors of \cite{piratla2020efficient} present CSD, which jointly learns a domain-invariant and domain-specific component in the final embedding and enables the extraction of disentangled representations, whereas the authors of \cite{chattopadhyay2020learning} propose learning domain specific 
masks during training to improve model robustness. Generative models have also been proposed in the disentangled representation learning literature for DG, with variational autoencoders (VAEs) and GANs \cite{chen2016infogan} being utilized for learning distinct latent subspaces for class- and domain-specific features \cite{ilse2020diva}. Another promising category of 
methods aiming to produce disentangled representations is that of 
Causality-Inspired algorithms. In causal representation learning, a domain 
shift can be thought of as an intervention, subsequentially leading the development of models that aim to uncover the true causal data generating factors. Naturally, the prediction of a model should not be affected by interventions on spuriously correlated but irrelevant features, such as the background, color or style of the image. Under this causal consideration, the authors of \cite{mahajan2021domain} propose a structural causal model in 
order to model within-class variations and leverage the fact that inputs 
across domains should have the same representation, given that they derive 
from the same object. Similar to disentangled representations, there have 
been proposed methods in the literature that focus on completely disregarding 
domain-relevant from the final feature vectors, deriving solely 
domain-invariant representations. Based on the initial findings of \cite{ben2006analysis}, numerous works have presented algorithms that aim to minimize the representation differences across multiple source domains within a specific feature space, ensuring they become domain invariant, ultimately enabling the trained model to effectively generalize to previously unseen domains. In one of the most notable previous 
works in this category, Arjovsky et al. \cite{arjovsky2019irm} enforce the 
optimal classifier on top of the representation space to be the same across domains and simultaneously minimize the loss across distributions. The above 
idea of Invariant Risk Minimization (IRM) has been extended to several other
works. For example, the authors of \cite{krueger2021out} propose minimizing 
the variance of source-domain risks, by minimizing their extrapolated risk, 
while the authors of \cite{zhang2020arm} propose adapting to domain shift and 
producing invariant representations. Finally, an alternative route towards
learning generalized representations is via regularization strategies. 
The most representative group of methods in this category is \textit{Gradient-Based operations}, which utilize gradient information during 
model training. In \cite{huang2020rsc}, the authors propose learning robust
representations by discarding the most dominant gradients in each training 
iteration under the assumption that they are correlated with domain-specific
features present in the source data. Another popular strategy is to seek for 
flat minima \cite{foret2020sharpness, cha2021swad} in the loss landscape of 
neural networks during training, assuming that models that converge to flat minima exhibit increased generalization capabilities \cite{zhuang2022surrogate, wang2023sharpness}. 
What's more, Shi et al. \cite{shi2021gradient} hypothesized that gradients 
among domains should match and proposed an approximation of a loss inducing the maximization of the gradient inner product during training. Our method (GGA) can be categorized in this group of gradient operations, as it considers
the similarity of domain gradients in the early iterations of model training
and seeks for sets in the parameter space with increased gradient alignment, before continuing the optimization procedure. 

\textbf{Learning Algorithm}. In addition to manipulating the input space or 
feature extractor, DG methods were also researched under the scope of 
alternative ML learning paradigms, such as \textit{ensemble}, \textit{meta}, \textit{domain-adversarial}, \textit{self-supervised} and 
\textit{reinforcement} learning. In this section we present the most exemplary
works in each category. \textit{Ensemble-Learning} in DG initially combined
several copies of the same network, each of which is trained on a specific 
domain \cite{zhou2021domain, ding2017deep}. Alternatively, instead of using several networks, \cite{yosinski2014transferable} proposed sharing shallow 
layers among CNNs. During inference, the final prediction is produced by 
either simple \cite{zhou2021domain} or weighted averaging 
\cite{wang2020dofe}. In \textit{Meta-Learning} for DG, Li et al. \cite{li2018learning} propose MLDG and split the source domains into 
meta-train and meta-test splits to mimic the effects of domain shift during 
training. Similarly, \cite{balaji2018metareg} proposes learning a meta regularizer for the classifier, while MAML \cite{finn2017model} was proposed
for improved parameter initialization. Another approach is that of \textit{Adversarial Learning} (AL). In the context of DG, the aim of 
adversarial learning is to train a classifier to distinguish between source domains \cite{matsuura2020domain} and ultimately learn domain-agnostic 
features from the samples that can be generalized to unseen data 
\cite{li2018deep}. Other learning paradigms such as \textit{Self-Supervised} 
learning have also been explored in DG, which leverages unlabeled data 
samples to derive generalized representations. Notably, the authors of 
\cite{carlucci2019domain} introduce a self-supervised jigsaw-solving puzzle 
task to push the model to learn robust representations. Furthermore, 
contrastive learning has also been shown to improve model performance. 
Specifically, SelfReg \cite{kim2021selfreg} utilizes self-supervised contrastive losses to bring latent representations of same-class samples closer. Similarly, the authors of \cite{ballas2024multi} introduce 
a contrastive loss for representations extracted from intermediate 
layers of the network. Finally, \textit{Reinforcement learning} has also been 
applied in the context of DG. Indicatively, previous works have explored randomizing the environments of an RL agent for transferring them to 
real-world scenarios \cite{tobin2017domain,lee2019network}, whereas \cite{laskin2020curl} researches the combination of RL with contrastive 
learning.

\section{Computational Analyis}
\label{computation}

\subsection{Experiment Infrastructure}
\label{experiment-infra}

Each and every experiment is conducted on a cluster containing $4\times40$GB NVIDIA A$100$ GPU cards, split into $8$ $20$GB virtual MIG devices and $1\times24$GB NVIDIA RTX A$5000$ GPU card, via a SLURM workload manager.

\subsection{Complexity Analysis}
\label{sec:complexity}

Each GGA training iteration includes computing model gradients $S\cdot n_a$ times for each training step, where $S$ is the number of source domains and $n_a$ is the number of search steps. These GGA training iterations only take place in the early stages of training and for a small percentage of the total training iterations (2\% in our experiments). The rest of the iterations are vanilla ERM. Furthermore, inference is not affected by the application of GGA 
during training.

 
\section{Full Experimental Results}
\label{sec:full-results}
In this section, we show detailed results of Table 2 in the main text.
The results marked by $\dagger, \ddagger$ are copied from Gulrajani and
Lopez-Paz \cite{gulrajani2020domainbed} and Wang \etal 
\cite{wang2023sharpness}, respectively. Standard errors for the baseline methods are reported from three trials, if available from past literature.
In {\color{green} green} and {\color{red} red}, we highlight the performance boost and decrease of applying \textbf{GGA} on top of each algorithm respectively, averaged over three trials. In addition, we also present 
detailed results for the DomainNet benchmark, without however including 
results for the combination of GGA with the baseline algorithms, due to 
computational restrictions. We also include experiments for the ColoredMNIST and RotatedMNIST datasets, where we reproduced the results for all baselines and report the average results over 5 runs. The below tables are better read in color.

When applying GGA to existing methods, the only difference regarding the baseline algorithm training is that ``Algorithm 1'' (i.e. GGA)  is applied instead of the method’s update rules for the duration of the annealing process (training steps $A_s$ to $A_e$). The total epochs and method hyperparameters remain the same throughout training.


%\subsection{PACS}
\begin{table*}
\centering
\small
\renewcommand{\arraystretch}{1.1}
\caption{\small{Out-of-domain accuracies (\%) on {PACS}.}}
\begin{tabular}{lllll|c}
\toprule
\textbf{Algorithm} & \textbf{A} & \textbf{C} & \textbf{P} & \textbf{S} & \textbf{Avg} \\
\midrule

IRM$^\dagger$ \cite{arjovsky2019irm}            & 84.8\scriptsize{$\pm1.3$}      \normalsize{({\color{red}$-3.6$})}& 
76.4\scriptsize{$\pm1.1$}      \normalsize{({\color{green}$+0.7$})} & 
96.7\scriptsize{$\pm0.6$}      \normalsize{({\color{red}$-0.3$})}& 
76.1\scriptsize{$\pm1.0$}      \normalsize{({\color{red}$-2.8$})} & 
%		33.9\scriptsize{$\pm2.8$}          & 
83.5 \normalsize{({\color{red}$-1.5$})}          \\

ERM$^\ddagger$ \cite{vapnik1998statistical} & 
85.7 \scriptsize$\pm0.6$ & 
77.1 \scriptsize$\pm0.8$ & 
97.4 \scriptsize$\pm0.4$ & 
76.6 \scriptsize$\pm0.7$ & 
84.2 \\


GroupDRO$^\ddagger$ \cite{Sagawa2020GroupDRO}    & 83.5\scriptsize{$\pm0.9$}    \normalsize{({\color{green}$+3.6$})}   & 
79.1\scriptsize{$\pm0.6$}    \normalsize{({\color{green}$+2.0$})}   & 
96.7\scriptsize{$\pm0.7$}    \normalsize{({\color{green}$+0.5$})}   & 
78.3\scriptsize{$\pm2.0$}    \normalsize{({\color{red}$-0.5$})}   & 
%		33.3\scriptsize{$\pm0.2$}          & 
84.4 \normalsize{({\color{green}$+1.4$})}           \\


MTL$^\ddagger$\cite{blanchard2021mtl_marginal_transfer_learning}   & 87.5\scriptsize{$\pm0.8$}     \normalsize{({\color{green}$+1.3$})} & 
77.1\scriptsize{$\pm0.5$}     \normalsize{({\color{green}$+2.5$})} & 
96.4\scriptsize{$\pm0.8$}     \normalsize{({\color{red}$-0.8$})} & 
77.3\scriptsize{$\pm1.8$}     \normalsize{({\color{green}$+0.7$})} &  
%		40.6\scriptsize{$\pm0.1$}          & 
84.6 \normalsize{({\color{green}$+0.9$})}          \\

Mixup$^\dagger$ \cite{xu2020interdomain_mixup_aaai}             & 86.1\scriptsize{$\pm0.5$}   \normalsize{({\color{green}$+1.2$})} & 
78.9\scriptsize{$\pm0.8$}   \normalsize{({\color{red}$-0.1$})} & 
97.6\scriptsize{$\pm0.1$}   \normalsize{({\color{red}$-0.4$})} & 
75.8\scriptsize{$\pm1.8$}   \normalsize{({\color{green}$+4.1$})} & 
%		39.2\scriptsize{$\pm0.1$}            & 
84.6   \normalsize{({\color{green}$+1.2$})}        \\


MMD$^\ddagger$ \cite{li2018mmd}                  & 86.1\scriptsize{$\pm1.4$}     \normalsize{({\color{green}$+0.8$})} & 
79.4\scriptsize{$\pm0.9$}     \normalsize{({\color{green}$+0.5$})} & 
96.6\scriptsize{$\pm0.2$}     \normalsize{({\color{red}$-0.6$})} & 76.5\scriptsize{$\pm0.5$}     \normalsize{({\color{green}$+3.7$})} & 
%		23.4\scriptsize{$\pm9.5$}          & 
84.7 \normalsize{({\color{green}$+0.8$})}         \\


VREx$^\ddagger$ \cite{krueger2020vrex} & 
86.0\scriptsize{$\pm1.6$}     \normalsize{({\color{green}$+0.2$})} & 79.1\scriptsize{$\pm0.6$}     \normalsize{({\color{red}$-0.9$})} & 96.9\scriptsize{$\pm0.5$}     \normalsize{({\color{red}$-0.1$})} & 77.7\scriptsize{$\pm1.7$}     \normalsize{({\color{green}$+2.3$})} & 
%		33.6\scriptsize{$\pm2.9$}          & 
84.9 \normalsize{({\color{green}$+0.5$})}         \\

MLDG$^\dagger$ \cite{li2018learning}                & 85.5\scriptsize{$\pm1.4$}   \normalsize{({\color{green}$+1.1$})} &
80.1\scriptsize{$\pm1.7$}   \normalsize{({\color{green}$+0.9$})} & 97.4\scriptsize{$\pm0.3$}   \normalsize{({\color{green}$-0.9$})} &
76.6\scriptsize{$\pm1.1$}   \normalsize{({\color{green}$+1.6$})} & 
%		41.2\scriptsize{$\pm0.1$}            & 
84.9 \normalsize{({\color{green}$+0.7$})}      \\

ARM$^\ddagger$ \cite{zhang2020arm}               & 86.8\scriptsize{$\pm0.6$}     \normalsize{({\color{red}$-1.9$})} & 
76.8\scriptsize{$\pm0.5$}     \normalsize{({\color{green}$+4.7$})} & 
97.4\scriptsize{$\pm0.3$}     \normalsize{({\color{red}$-1.1$})} & 
79.3\scriptsize{$\pm1.2$}     \normalsize{({\color{green}$+0.4$})} & 
%		35.5\scriptsize{$\pm0.2$}          & 
85.1 \normalsize{({\color{green}$+0.5$})}           \\


Mixstyle$^\ddagger$ \cite{zhou2021mixstyle}   & 
86.8\scriptsize{$\pm0.5$} \normalsize{({\color{green}$+1.1$})}            & 79.0\scriptsize{$\pm1.4$} \normalsize{({\color{red}$-0.3$})}          & 96.6\scriptsize{$\pm0.1$} \normalsize{({\color{red}$-1.1$})}          & 78.5\scriptsize{$\pm2.3$} \normalsize{({\color{green}$+1.4$})}          & 
%		34.0\scriptsize{$\pm0.1$}          & 
85.2 \normalsize{({\color{green}$+0.3$})}           \\



%AND-mask \cite{shahtalebi2021sand} & 
%84.4\scriptsize{$\pm0.9$}  
%\normalsize{({\color{green}$+0.1$})}& 
%78.1\scriptsize{$\pm0.9$}  
%\normalsize{({\color{green}$+0.3$})}& 
%65.6\scriptsize{$\pm0.4$}  
%\normalsize{({\color{green}$+0.4$})} & 
%44.6\scriptsize{$\pm0.3$} 
%\normalsize{({\color{green}$+0.5$})}
%%        & 37.2\scriptsize{$\pm0.6$}  
%& 68.2 \normalsize{({\color{green}$+0.3$})}\\

CORAL$^\dagger$ \cite{sun2016coral}             & 88.3\scriptsize{$\pm0.2$}      \normalsize{({\color{red}$-0.6$})}& 
80.0\scriptsize{$\pm0.5$}      \normalsize{({\color{green}$+1.3$})} & 
97.5\scriptsize{$\pm0.3$}      \normalsize{({\color{green}$+0.6$})} & 
78.8\scriptsize{$\pm1.3$}      \normalsize{({\color{green}$+1.5$})}& 
%		41.5\scriptsize{$\pm0.1$}          & 
86.2    \normalsize{({\color{green}$+0.7$})}       \\

SagNet$^\dagger$ \cite{nam2019sagnet}           &             87.4\scriptsize{$\pm0.2$}    \normalsize{({\color{red}$-2.3$})} & 
80.7\scriptsize{$\pm0.5$}    \normalsize{({\color{green}$+1.0$})} & 97.1\scriptsize{$\pm0.1$}    \normalsize{({\color{red}$-0.9$})}& 
80.0\scriptsize{$\pm1.0$}    \normalsize{({\color{red}$-1.8$})}& 
%		40.3\scriptsize{$\pm0.1$}          & 
86.3 \normalsize{({\color{red}$-1.0$})}         \\

\midrule

RSC$^\dagger$ \cite{huang2020rsc}               & 85.4\scriptsize{$\pm0.9$}      \normalsize{({\color{red}$-1.8$})}& 
79.7\scriptsize{$\pm0.5$}      \normalsize{({\color{green}$+2.9$})}& 
97.6\scriptsize{$\pm0.9$}      \normalsize{({\color{red}$-1.0$})}& 
78.2\scriptsize{$\pm1.0$}      \normalsize{({\color{green}$+0.3$})}& 
%		38.9\scriptsize{$\pm0.5$}          & 
85.2 \normalsize{({\color{green}$+0.1$})}   \\

%Fish $^\ddagger$ \cite{shi2021gradient}                     & 85.5\scriptsize{$\pm0.3$}     \normalsize{({\color{green}$+0.1$})} & 77.8\scriptsize{$\pm0.3$}     \normalsize{({\color{green}$+0.9$})} & 
%68.6\scriptsize{$\pm0.4$}     \normalsize{({\color{red}$-0.6$})} & 
%45.1\scriptsize{$\pm1.3$}     \normalsize{({\color{green}$+0.8$})} & 
%%		42.7\scriptsize{$\pm0.2$}            & 
%69.3 \normalsize{({\color{green}$+0.3$})}             \\ 

SAM $^\ddagger$\cite{foret2020sharpness}  & 
85.6\scriptsize$\pm2.1$       \normalsize{({\color{green}$+1.1$})} & 
80.9\scriptsize$\pm1.2$       \normalsize{({\color{red}$-0.8$})} & 
97.0\scriptsize$\pm0.4$       \normalsize{({\color{red}$-0.2$})} & 
79.6\scriptsize$\pm1.6$       \normalsize{({\color{green}$+1.2$})} & 
%		44.3\scriptsize$\pm0.0$              & 
85.8 \normalsize{({\color{green}$+0.6$})}   \\

GSAM $^\ddagger$ \cite{zhuang2022surrogate}       &
86.9\scriptsize$\pm0.1$      \normalsize{({\color{red}$-0.4$})} & 
80.4\scriptsize$\pm0.2$      \normalsize{({\color{green}$+0.7$})} & 
97.5\scriptsize$\pm0.0$      \normalsize{({\color{red}$-1.1$})}& 
78.7\scriptsize$\pm0.8$      \normalsize{({\color{green}$+2.5$})}& 
%		44.6\scriptsize$\pm0.2$             & 
85.9  \normalsize{({\color{green}$+0.4$})}  \\

SAGM $^\ddagger$ \cite{wang2023sharpness}       & 
87.4\scriptsize{$\pm0.2$}    \normalsize{({\color{green}$+1.2$})}& 
80.2\scriptsize{$\pm0.3$}    \normalsize{({\color{green}$+1.1$})}& 
98.0\scriptsize{$\pm0.2$}    \normalsize{({\color{red}$-1.0$})}& 
80.8\scriptsize{$\pm0.6$}    \normalsize{({\color{red}$-0.4$})}& 
%		45.0\scriptsize{$\pm0.2$}           & 
86.6 \normalsize{({\color{green}$+0.2$})}   \\

%		\midrule
\midrule
\textbf{GGA} (ours)                 & 
88.8\scriptsize{$\pm0.2$}           & 
80.1\scriptsize{$\pm0.3$}           & 
97.3\scriptsize{$\pm0.2$}           &  
81.2\scriptsize{$\pm0.5$}    &  
87.3  \\
%		\midrule

\bottomrule
\end{tabular}
\end{table*}

%\subsection{VLCS}
\begin{table*}[]
\centering
\small
\renewcommand{\arraystretch}{1.1}
\caption{\small{Out-of-domain accuracies (\%) on VLCS.}}
\begin{tabular}{lllll|c}
\toprule
\textbf{Algorithm} & \textbf{C} & \textbf{L} & \textbf{S} & \textbf{V} & \textbf{Avg} \\
\midrule

GroupDRO$^\ddagger$ \cite{Sagawa2020GroupDRO}    & 97.3\scriptsize{$\pm0.3$}    \normalsize{({\color{green}$+1.4$})}   & 
63.4\scriptsize{$\pm0.9$}    \normalsize{({\color{green}$+1.7$})}   & 
69.5\scriptsize{$\pm0.8$}    \normalsize{({\color{green}$+2.0$})}   & 
76.7\scriptsize{$\pm0.7$}    \normalsize{({\color{red}$-2.9$})}   & 
%		33.3\scriptsize{$\pm0.2$}          & 
76.7 \normalsize{({\color{green}$+0.6$})}           \\

MLDG$^\dagger$ \cite{li2018learning}                & 97.4\scriptsize{$\pm0.2$}      \normalsize{({\color{green}$+1.6$})} &
65.2\scriptsize{$\pm0.7$}      \normalsize{({\color{red}$-0.4$})} & 71.0\scriptsize{$\pm1.4$}      \normalsize{({\color{green}$+3.0$})} &
75.3\scriptsize{$\pm1.0$}      \normalsize{({\color{green}$+0.9$})} & 
%		41.2\scriptsize{$\pm0.1$}            & 
77.2 \normalsize{({\color{green}$+1.3$})}      \\

MTL$^\ddagger$\cite{blanchard2021mtl_marginal_transfer_learning}    & 97.8\scriptsize{$\pm0.4$}     \normalsize{({\color{green}$+0.3$})} & 
64.3\scriptsize{$\pm0.3$}     \normalsize{({\color{green}$+1.8$})} & 
71.5\scriptsize{$\pm0.7$}     \normalsize{({\color{green}$+4.1$})} & 
75.3\scriptsize{$\pm1.7$}     \normalsize{({\color{green}$+1.6$})} &  
%		40.6\scriptsize{$\pm0.1$}          & 
77.2 \normalsize{({\color{green}$+2.0$})}          \\

ERM$^\ddagger$ \cite{vapnik1998statistical}  & 
98.0 \scriptsize$\pm0.3$ & 
64.7 \scriptsize$\pm1.2$ & 
71.4 \scriptsize$\pm1.2$ & 
75.2 \scriptsize$\pm1.6$ & 
77.3 \\

Mixup$^\dagger$ \cite{xu2020interdomain_mixup_aaai}             & 98.3\scriptsize{$\pm0.6$}     \normalsize{({\color{green}$+0.8$})} & 
64.8\scriptsize{$\pm1.0$}     \normalsize{({\color{green}$+1.7$})} & 
72.1\scriptsize{$\pm0.5$}     \normalsize{({\color{green}$+1.0$})} & 
74.3\scriptsize{$\pm0.8$}     \normalsize{({\color{green}$+3.6$})} & 
%		39.2\scriptsize{$\pm0.1$}            & 
77.4  \normalsize{({\color{green}$+1.8$})}        \\

MMD$^\ddagger$ \cite{li2018mmd}                  & 97.7\scriptsize{$\pm0.1$}     \normalsize{({\color{red}$-0.9$})} & 
64.0\scriptsize{$\pm1.1$}     \normalsize{({\color{green}$+1.4$})} & 
72.8\scriptsize{$\pm0.2$}     \normalsize{({\color{green}$+1.4$})} & 
75.3\scriptsize{$\pm3.3$}     \normalsize{({\color{green}$+3.5$})} & 
%		23.4\scriptsize{$\pm9.5$}          & 
77.5 \normalsize{({\color{green}$+1.3$})}         \\

ARM$^\ddagger$ \cite{zhang2020arm}               & 98.7\scriptsize{$\pm0.2$}     \normalsize{({\color{red}$-0.2$})}  & 
63.6\scriptsize{$\pm0.7$}     \normalsize{({\color{green}$+2.2$})}  & 
71.3\scriptsize{$\pm1.2$}     \normalsize{({\color{green}$+0.3$})}  & 
76.7\scriptsize{$\pm0.6$}     \normalsize{({\color{green}$+1.4$})}  & 
%		35.5\scriptsize{$\pm0.2$}          & 
77.6 \normalsize{({\color{green}$+0.9$})}           \\

SagNet$^\dagger$ \cite{nam2019sagnet}           &             97.9\scriptsize{$\pm0.4$}     \normalsize{({\color{red}$-0.2$})} & 
64.5\scriptsize{$\pm0.5$}     \normalsize{({\color{green}$+1.7$})} & 71.4\scriptsize{$\pm1.3$}     \normalsize{({\color{green}$+0.8$})}& 
77.5\scriptsize{$\pm0.5$}     \normalsize{({\color{green}$+1.4$})}& 
%		40.3\scriptsize{$\pm0.1$}          & 
77.8 \normalsize{({\color{green}$+0.9$})}         \\

Mixstyle$^\ddagger$ \cite{zhou2021mixstyle}   & 
98.6\scriptsize{$\pm0.3$} \normalsize{({\color{red}$-0.1$})}            & 64.5\scriptsize{$\pm1.1$} \normalsize{({\color{green}$+1.9$})}          & 72.6\scriptsize{$\pm0.5$} \normalsize{({\color{green}$+0.4$})}          & 75.7\scriptsize{$\pm1.7$} \normalsize{({\color{green}$+0.4$})}          & 
%		34.0\scriptsize{$\pm0.1$}          & 
77.9 \normalsize{({\color{green}$+0.6$})}           \\

%AND-mask \cite{shahtalebi2021sand} & 
%84.4\scriptsize{$\pm0.9$}  
%\normalsize{({\color{green}$+0.1$})}& 
%78.1\scriptsize{$\pm0.9$}  
%\normalsize{({\color{green}$+0.3$})}& 
%65.6\scriptsize{$\pm0.4$}  
%\normalsize{({\color{green}$+0.4$})} & 
%44.6\scriptsize{$\pm0.3$} 
%\normalsize{({\color{green}$+0.5$})}
%%        & 37.2\scriptsize{$\pm0.6$}  
%& 68.2 \normalsize{({\color{green}$+0.3$})}\\

VREx$^\ddagger$ \cite{krueger2020vrex}         & 
98.4\scriptsize{$\pm0.3$}      \normalsize{({\color{red}$-0.9$})} & 64.4\scriptsize{$\pm1.4$}      \normalsize{({\color{green}$+1.9$})} & 74.1\scriptsize{$\pm0.4$}      \normalsize{({\color{green}$-1.7$})} & 76.2\scriptsize{$\pm1.3$}      \normalsize{({\color{green}$+1.0$})} & 
%		33.6\scriptsize{$\pm2.9$}          & 
78.3 \normalsize{({\color{green}$+0.1$})}         \\


IRM$^\dagger$ \cite{arjovsky2019irm}            & 98.6\scriptsize{$\pm0.1$}      \normalsize{({\color{red}$-0.3$})}& 
64.9\scriptsize{$\pm0.9$}      \normalsize{({\color{red}$-3.5$})}& 
73.4\scriptsize{$\pm0.6$}      \normalsize{({\color{green}$+1.7$})}& 
77.3\scriptsize{$\pm0.9$}      \normalsize{({\color{red}$-1.5$})} & 
%		33.9\scriptsize{$\pm2.8$}          & 
78.6 \normalsize{({\color{red}$-0.9$})}          \\


CORAL$^\dagger$ \cite{sun2016coral}             & 98.3\scriptsize{$\pm0.3$}      \normalsize{({\color{green}$+0.9$})}& 
66.1\scriptsize{$\pm0.6$}      \normalsize{({\color{green}$+1.8$})} & 
73.4\scriptsize{$\pm0.3$}      \normalsize{({\color{red}$-1.8$})} & 
77.5\scriptsize{$\pm1.0$}      \normalsize{({\color{red}$-2.1$})}& 
%		41.5\scriptsize{$\pm0.1$}          & 
78.8    \normalsize{({\color{red}$-0.4$})}       \\

\midrule

RSC$^\dagger$ \cite{huang2020rsc}               & 
97.9\scriptsize{$\pm0.1$}    \normalsize{({\color{green}$+0.6$})}& 62.5\scriptsize{$\pm0.7$}    \normalsize{({\color{green}$+0.3$})}& 
72.3\scriptsize{$\pm1.2$}    \normalsize{({\color{green}$+0.4$})}& 
75.6\scriptsize{$\pm0.8$}    \normalsize{({\color{red}$-0.8$})}& 
%		38.9\scriptsize{$\pm0.5$}          & 
77.1 \normalsize{({\color{green}$+0.2$})}   \\

%Fish $^\ddagger$ \cite{shi2021gradient}                     & 85.5\scriptsize{$\pm0.3$}          
%\normalsize{({\color{green}$+0.1$})} & 77.8\scriptsize{$\pm0.3$}          
%\normalsize{({\color{green}$+0.9$})} & 
%68.6\scriptsize{$\pm0.4$}            
%\normalsize{({\color{red}$-0.6$})} & 
%45.1\scriptsize{$\pm1.3$}            
%\normalsize{({\color{green}$+0.8$})} & 
%%		42.7\scriptsize{$\pm0.2$}            & 
%69.3 \normalsize{({\color{green}$+0.3$})}             \\ 



GSAM $^\ddagger$ \cite{zhuang2022surrogate}       & 
98.7\scriptsize$\pm0.3$   \normalsize{({\color{green}$+0.5$})} & 
64.9\scriptsize$\pm0.2$   \normalsize{({\color{green}$+0.5$})} & 
74.3\scriptsize$\pm0.0$   \normalsize{({\color{green}$+1.2$})}& 
78.5\scriptsize$\pm0.8$   \normalsize{({\color{green}$+1.8$})}& 
%		44.6\scriptsize$\pm0.2$             & 
79.1  \normalsize{({\color{green}$+1.0$})}  \\

SAM $^\ddagger$\cite{foret2020sharpness}  & 
99.1\scriptsize$\pm0.2$     \normalsize{({\color{red}$-0.2$})} & 
65.0\scriptsize$\pm1.0$     \normalsize{({\color{green}$+1.8$})} & 
73.7\scriptsize$\pm1.0$     \normalsize{({\color{red}$-0.2$})} & 
79.8\scriptsize$\pm0.1$     \normalsize{({\color{green}$+1.5$})} & 
%		44.3\scriptsize$\pm0.0$              & 
79.4 \normalsize{({\color{green}$+0.7$})}   \\

SAGM $^\ddagger$ \cite{wang2023sharpness}       & 
99.0\scriptsize{$\pm0.2$}  \normalsize{({\color{red}$-0.4$})}& 
65.2\scriptsize{$\pm0.4$}  \normalsize{({\color{green}$+0.5$})}& 
75.1\scriptsize{$\pm0.3$}  \normalsize{({\color{red}$-1.1$})}& 
80.7\scriptsize{$\pm0.8$}  \normalsize{({\color{red}$-0.2$})}& 
%		45.0\scriptsize{$\pm0.2$}           & 
80.0 \normalsize{({\color{red}$-0.3$})}   \\

%		\midrule
\midrule
\textbf{GGA} (ours)                 & 
99.1\scriptsize{$\pm0.2$}           & 
67.5\scriptsize{$\pm0.6$}           & 
75.1\scriptsize{$\pm0.3$}           &  
78.0\scriptsize{$\pm0.1$}    &  
79.9  \\
%		\midrule
\bottomrule
\end{tabular}
\end{table*}

%\subsection{OfficeHome}
\begin{table*}[]
\centering
\small
\renewcommand{\arraystretch}{1.1}
\caption{\small{Out-of-domain accuracies (\%) on OfficeHome.}}
\begin{tabular}{lllll|c}
\toprule
\textbf{Algorithm} & \textbf{A} & \textbf{C} & \textbf{P} & \textbf{R} & \textbf{Avg} \\
\midrule
Mixstyle$^\ddagger$ \cite{zhou2021mixstyle}   & 
51.1\scriptsize{$\pm0.3$} \normalsize{({\color{green}$+0.3$})}            & 53.2\scriptsize{$\pm0.4$} \normalsize{({\color{green}$+0.7$})}          & 68.2\scriptsize{$\pm0.7$} \normalsize{({\color{green}$+0.4$})}          & 69.2\scriptsize{$\pm0.6$} \normalsize{({\color{green}$+0.6$})}          & 
%		34.0\scriptsize{$\pm0.1$}          & 
60.4 \normalsize{({\color{green}$+0.5$})}           \\

IRM$^\dagger$ \cite{arjovsky2019irm}            & 58.9\scriptsize{$\pm2.3$}       \normalsize{({\color{red}$-3.5$})}& 
52.2\scriptsize{$\pm1.6$}       \normalsize{({\color{red}$-2.1$})} & 
72.1\scriptsize{$\pm2.9$}       \normalsize{({\color{red}$-1.7$})}& 
74.0\scriptsize{$\pm2.5$}       \normalsize{({\color{red}$-1.0$})} & 
%		33.9\scriptsize{$\pm2.8$}          & 
64.3 \normalsize{({\color{red}$-2.1$})}          \\

ARM$^\ddagger$ \cite{zhang2020arm}               & 58.9\scriptsize{$\pm0.8$}      \normalsize{({\color{green}$+2.7$})} & 
51.0\scriptsize{$\pm0.5$}      \normalsize{({\color{red}$-0.3$})} & 
74.1\scriptsize{$\pm0.1$}      \normalsize{({\color{green}$+2.1$})} & 
75.2\scriptsize{$\pm0.3$}      \normalsize{({\color{green}$+3.4$})} & 
%		35.5\scriptsize{$\pm0.2$}          & 
64.8 \normalsize{({\color{green}$+2.1$})}           \\

GroupDRO$^\ddagger$ \cite{Sagawa2020GroupDRO}    & 60.4\scriptsize{$\pm0.7$}    \normalsize{({\color{green}$+3.8$})}   & 
52.7\scriptsize{$\pm1.0$}    \normalsize{({\color{green}$+1.2$})}   & 
75.0\scriptsize{$\pm0.7$}    \normalsize{({\color{green}$+1.3$})}   & 
76.0\scriptsize{$\pm0.7$}    \normalsize{({\color{green}$+2.2$})}   & 
%		33.3\scriptsize{$\pm0.2$}          & 
66.0 \normalsize{({\color{green}$+2.2$})}           \\

MMD$^\ddagger$ \cite{li2018mmd}                  & 60.4\scriptsize{$\pm0.2$}     \normalsize{({\color{green}$+3.1$})} & 
53.3\scriptsize{$\pm0.3$}     \normalsize{({\color{red}$-0.2$})} & 
74.3\scriptsize{$\pm0.1$}     \normalsize{({\color{green}$+3.1$})} & 
77.4\scriptsize{$\pm0.6$}     \normalsize{({\color{green}$+0.7$})} & 
%		23.4\scriptsize{$\pm9.5$}          & 
66.4 \normalsize{({\color{green}$+1.6$})}         \\

%AND-mask \cite{shahtalebi2021sand} & 
%84.4\scriptsize{$\pm0.9$}  
%\normalsize{({\color{green}$+0.1$})}& 
%78.1\scriptsize{$\pm0.9$}  
%\normalsize{({\color{green}$+0.3$})}& 
%65.6\scriptsize{$\pm0.4$}  
%\normalsize{({\color{green}$+0.4$})} & 
%44.6\scriptsize{$\pm0.3$} 
%\normalsize{({\color{green}$+0.5$})}
%%        & 37.2\scriptsize{$\pm0.6$}  
%& 68.2 \normalsize{({\color{green}$+0.3$})}\\

MTL$^\ddagger$\cite{blanchard2021mtl_marginal_transfer_learning}    & 61.5\scriptsize{$\pm0.7$}       \normalsize{({\color{green}$+0.8$})} & 
52.4\scriptsize{$\pm0.6$}       \normalsize{({\color{red}$-0.3$})} & 
74.9\scriptsize{$\pm0.4$}       \normalsize{({\color{green}$+1.0$})} & 
76.8\scriptsize{$\pm0.4$}       \normalsize{({\color{green}$+1.2$})}&  
%		40.6\scriptsize{$\pm0.1$}          & 
66.4 \normalsize{({\color{green}$+0.4$})}          \\

VREx$^\ddagger$ \cite{krueger2020vrex}         & 
60.7\scriptsize{$\pm0.9$}        \normalsize{({\color{green}$+1.9$})} & 53.0\scriptsize{$\pm0.9$}        \normalsize{({\color{green}$+0.8$})} & 75.3\scriptsize{$\pm0.1$}        \normalsize{({\color{green}$+0.6$})} & 76.6\scriptsize{$\pm0.5$}        \normalsize{({\color{green}$+0.2$})} & 
%		33.6\scriptsize{$\pm2.9$}          & 
66.4 \normalsize{({\color{green}$+0.9$})}         \\

ERM$^\ddagger$ \cite{vapnik1998statistical} & 
63.1 \scriptsize$\pm0.3$ & 
51.9 \scriptsize$\pm0.4$ & 
77.2 \scriptsize$\pm0.5$ & 
78.1 \scriptsize$\pm0.2$ & 67.6 \\



MLDG$^\dagger$ \cite{li2018learning}                & 61.5\scriptsize{$\pm0.9$}       \normalsize{({\color{green}$+2.4$})} &
53.2\scriptsize{$\pm0.6$}       \normalsize{({\color{green}$+0.2$})} & 75.0\scriptsize{$\pm1.2$}       \normalsize{({\color{green}$+1.8$})} &
77.5\scriptsize{$\pm0.4$}       \normalsize{({\color{green}$+0.6$})} & 
%		41.2\scriptsize{$\pm0.1$}            & 
66.8 \normalsize{({\color{green}$+1.2$})}      \\

Mixup$^\dagger$ \cite{xu2020interdomain_mixup_aaai}             & 62.4\scriptsize{$\pm0.8$}       \normalsize{({\color{green}$+1.5$})} & 
54.8\scriptsize{$\pm0.6$}       \normalsize{({\color{red}$-1.7$})} & 
76.9\scriptsize{$\pm0.3$}       \normalsize{({\color{green}$+1.9$})} & 
78.3\scriptsize{$\pm0.2$}       \normalsize{({\color{green}$+0.4$})} & 
%		39.2\scriptsize{$\pm0.1$}            & 
68.1   \normalsize{({\color{green}$+1.2$})}        \\


SagNet$^\dagger$ \cite{nam2019sagnet}           &             63.4\scriptsize{$\pm0.2$}       \normalsize{({\color{green}$+1.0$})} & 
54.8\scriptsize{$\pm0.4$}       \normalsize{({\color{red}$-1.9$})} & 75.8\scriptsize{$\pm0.4$}       \normalsize{({\color{green}$+1.4$})}& 
78.3\scriptsize{$\pm0.3$}       \normalsize{({\color{green}$+0.7$})}& 
%		40.3\scriptsize{$\pm0.1$}          & 
68.1 \normalsize{({\color{green}$+0.3$})}         \\



CORAL$^\dagger$ \cite{sun2016coral}             & 
65.3\scriptsize{$\pm0.3$}       \normalsize{({\color{green}$+0.3$})}& 
54.4\scriptsize{$\pm0.6$}       \normalsize{({\color{green}$+0.2$})} & 
76.5\scriptsize{$\pm0.3$}       \normalsize{({\color{red}$-0.7$})} & 
78.4\scriptsize{$\pm1.0$}       \normalsize{({\color{green}$+1.0$})}& 
%		41.5\scriptsize{$\pm0.1$}          & 
68.7    \normalsize{({\color{green}$+0.2$})}       \\

\midrule

RSC$^\dagger$ \cite{huang2020rsc}               & 
60.7\scriptsize{$\pm1.4$}      \normalsize{({\color{red}$-1.1$})}& 
51.4\scriptsize{$\pm0.3$}      \normalsize{($+0.0$)}& 
74.8\scriptsize{$\pm1.1$}      \normalsize{({\color{green}$+0.6$})}& 
75.1\scriptsize{$\pm1.3$}      \normalsize{({\color{green}$+0.5$})}& 
%		38.9\scriptsize{$\pm0.5$}          & 
65.5 \normalsize{($+0.0$)}  \\

%Fish $^\ddagger$ \cite{shi2021gradient}                     & 85.5\scriptsize{$\pm0.3$}          
%\normalsize{({\color{green}$+0.1$})} & 77.8\scriptsize{$\pm0.3$}          
%\normalsize{({\color{green}$+0.9$})} & 
%68.6\scriptsize{$\pm0.4$}            
%\normalsize{({\color{red}$-0.6$})} & 
%45.1\scriptsize{$\pm1.3$}            
%\normalsize{({\color{green}$+0.8$})} & 
%%		42.7\scriptsize{$\pm0.2$}            & 
%69.3 \normalsize{({\color{green}$+0.3$})}             \\ 

GSAM $^\ddagger$ \cite{zhuang2022surrogate}  & 
64.9\scriptsize$\pm0.1$     \normalsize{({\color{red}$-0.6$})} & 
55.2\scriptsize$\pm0.2$     \normalsize{({\color{green}$+1.1$})} & 
77.8\scriptsize$\pm0.0$     \normalsize{({\color{green}$+0.4$})}& 
79.2\scriptsize$\pm0.2$     \normalsize{({\color{green}$+0.3$})}& 
%		44.6\scriptsize$\pm0.2$             & 
69.3  \normalsize{({\color{green}$+0.3$})}  \\

SAM $^\ddagger$\cite{foret2020sharpness}  & 
64.5\scriptsize$\pm0.3$     \normalsize{({\color{green}$+0.7$})} & 
56.5\scriptsize$\pm0.2$     \normalsize{({\color{green}$+0.3$})} & 
77.4\scriptsize$\pm0.1$     \normalsize{({\color{green}$+1.0$})} & 
79.8\scriptsize$\pm0.4$     \normalsize{({\color{green}$+0.4$})} & 
%		44.3\scriptsize$\pm0.0$              & 
69.6 \normalsize{({\color{green}$+0.6$})}   \\


SAGM $^\ddagger$ \cite{wang2023sharpness}       & 
65.4\scriptsize{$\pm0.4$}    \normalsize{({\color{red}$-0.9$})}& 
57.0\scriptsize{$\pm0.3$}    \normalsize{({\color{red}$-0.8$})}& 
78.0\scriptsize{$\pm0.3$}    \normalsize{({\color{green}$+0.4$})}& 
80.0\scriptsize{$\pm0.2$}    \normalsize{({\color{red}$-1.1$})}& 
%		45.0\scriptsize{$\pm0.2$}           & 
70.1 \normalsize{({\color{red}$-0.6$})}   \\

%		\midrule
\midrule
\textbf{GGA} (ours)                 & 
64.3\scriptsize{$\pm0.1$}           & 
54.4\scriptsize{$\pm0.2$}           & 
76.5\scriptsize{$\pm0.3$}           &  
78.9\scriptsize{$\pm0.2$}    &  
68.5  \\
%		\midrule
\bottomrule
\end{tabular}
\end{table*}

%\subsection{TerraIncognita}
\begin{table*}[]
\centering
\small
\renewcommand{\arraystretch}{1.1}
\caption{\small{Out-of-domain accuracies (\%) on TerraIncognita.}}
\begin{tabular}{lllll|c}
\toprule
\textbf{Algorithm} & \textbf{L100} & \textbf{L38} & \textbf{L43} & \textbf{L46} & \textbf{Avg} \\
\midrule

MMD$^\ddagger$ \cite{li2018mmd}                  & 41.9\scriptsize{$\pm3.0$}     \normalsize{({\color{green}$+9.7$})} & 
34.8\scriptsize{$\pm1.0$}     \normalsize{({\color{green}$+9.8$})} & 
57.0\scriptsize{$\pm1.9$}     \normalsize{({\color{green}$+0.5$})} & 35.2\scriptsize{$\pm1.8$}     \normalsize{({\color{green}$+5.9$})} & 
%		23.4\scriptsize{$\pm9.5$}          & 
42.2 \normalsize{({\color{green}$+6.3$})}         \\

GroupDRO$^\ddagger$ \cite{Sagawa2020GroupDRO}    & 41.2\scriptsize{$\pm0.7$}    \normalsize{({\color{red}$-1.8$})}   & 
38.6\scriptsize{$\pm2.1$}    \normalsize{({\color{green}$+9.4$})}   & 
56.7\scriptsize{$\pm0.9$}    \normalsize{({\color{black}$\pm0.0$})}   & 
36.4\scriptsize{$\pm2.1$}    \normalsize{({\color{red}$-1.6$})}   & 
%		33.3\scriptsize{$\pm0.2$}          & 
43.2 \normalsize{({\color{green}$+1.7$})}           \\


Mixstyle$^\ddagger$ \cite{zhou2021mixstyle}   & 
54.3\scriptsize{$\pm1.1$} \normalsize{({\color{red}$-2.9$})}            & 34.1\scriptsize{$\pm1.1$} \normalsize{({\color{green}$+8.8$})}          & 55.9\scriptsize{$\pm1.1$} \normalsize{({\color{red}$-2.8$})}          & 31.7\scriptsize{$\pm2.1$} \normalsize{({\color{green}$+2.9$})}          & 
%		34.0\scriptsize{$\pm0.1$}          & 
44.0 \normalsize{({\color{green}$+1.1$})}           \\


%AND-mask \cite{shahtalebi2021sand} & 
%84.4\scriptsize{$\pm0.9$}  
%\normalsize{({\color{green}$+0.1$})}& 
%78.1\scriptsize{$\pm0.9$}  
%\normalsize{({\color{green}$+0.3$})}& 
%65.6\scriptsize{$\pm0.4$}  
%\normalsize{({\color{green}$+0.4$})} & 
%44.6\scriptsize{$\pm0.3$} 
%\normalsize{({\color{green}$+0.5$})}
%%        & 37.2\scriptsize{$\pm0.6$}  
%& 68.2 \normalsize{({\color{green}$+0.3$})}\\

ARM$^\ddagger$ \cite{zhang2020arm}               & 49.3\scriptsize{$\pm0.7$}     \normalsize{({\color{red}$-3.0$})} & 
38.3\scriptsize{$\pm0.7$}     \normalsize{({\color{green}$+4.2$})} & 
55.8\scriptsize{$\pm0.8$}     \normalsize{({\color{green}$+2.0$})} & 
38.7\scriptsize{$\pm1.3$}     \normalsize{({\color{red}$-0.2$})} & 
%		35.5\scriptsize{$\pm0.2$}          & 
45.5 \normalsize{({\color{green}$+0.8$})}           \\


MTL$^\ddagger$\cite{blanchard2021mtl_marginal_transfer_learning}    & 49.3\scriptsize{$\pm1.2$}      \normalsize{({\color{red}$-5.9$})} & 
39.6\scriptsize{$\pm6.3$}      \normalsize{({\color{green}$+3.6$})} & 
55.6\scriptsize{$\pm1.1$}      \normalsize{({\color{green}$+2.1$})} & 
37.8\scriptsize{$\pm0.8$}      \normalsize{({\color{green}$+2.6$})}&  
%		40.6\scriptsize{$\pm0.1$}          & 
45.6 \normalsize{({\color{green}$+0.9$})}          \\



ERM$^\ddagger$  \cite{vapnik1998statistical}   & 
49.8 \scriptsize$\pm4.4$ & 
42.1 \scriptsize$\pm1.4$ & 
56.9 \scriptsize$\pm1.8$ & 
35.7 \scriptsize$\pm3.9$ & 
46.1 \\

VREx$^\ddagger$ \cite{krueger2020vrex}         & 
48.2\scriptsize{$\pm4.3$}       \normalsize{({\color{green}$+3.1$})}& 41.7\scriptsize{$\pm1.3$}       \normalsize{({\color{green}$+0.7$})}& 56.8\scriptsize{$\pm0.8$}       \normalsize{({\color{green}$+2.0$})}& 38.7\scriptsize{$\pm3.1$}       \normalsize{({\color{red}$-0.4$})} & 
%		33.6\scriptsize{$\pm2.9$}          & 
46.4 \normalsize{({\color{green}$+1.3$})}         \\


IRM$^\dagger$ \cite{arjovsky2019irm}            & 54.6\scriptsize{$\pm1.3$}      \normalsize{({\color{red}$-4.3$})}& 
39.8\scriptsize{$\pm1.9$}      \normalsize{({\color{red}$-3.4$})} & 
56.2\scriptsize{$\pm1.8$}      \normalsize{({\color{red}$-3.8$})}& 
39.6\scriptsize{$\pm0.8$}      \normalsize{({\color{red}$-4.1$})} & 
%		33.9\scriptsize{$\pm2.8$}          & 
47.6 \normalsize{({\color{red}$-3.9$})}          \\


CORAL$^\dagger$ \cite{sun2016coral}             & 
51.6\scriptsize{$\pm2.4$}      \normalsize{({\color{green}$+3.1$})}& 
42.2\scriptsize{$\pm1.0$}      \normalsize{({\color{red}$-1.2$})} & 
57.0\scriptsize{$\pm1.0$}      \normalsize{({\color{green}$+1.1$})} & 
39.8\scriptsize{$\pm2.9$}      \normalsize{({\color{red}$-1.8$})}& 
%		41.5\scriptsize{$\pm0.1$}          & 
47.6    \normalsize{({\color{green}$+0.3$})}       \\

MLDG$^\dagger$ \cite{li2018learning}                & 54.2\scriptsize{$\pm3.0$}       \normalsize{({\color{red}$-2.5$})} &
44.3\scriptsize{$\pm1.1$}       \normalsize{({\color{green}$+1.4$})} & 55.6\scriptsize{$\pm0.3$}       \normalsize{({\color{green}$+5.1$})} &
36.9\scriptsize{$\pm2.2$}       \normalsize{({\color{green}$+0.6$})} & 
%		41.2\scriptsize{$\pm0.1$}            & 
47.8 \normalsize{({\color{green}$+1.2$})}      \\

Mixup$^\dagger$ \cite{xu2020interdomain_mixup_aaai}             & 59.6\scriptsize{$\pm2.0$}       \normalsize{({\color{green}$1.2$})} & 
42.2\scriptsize{$\pm1.4$}       \normalsize{({\color{green}$+7.6$})} & 
55.9\scriptsize{$\pm0.8$}       \normalsize{({\color{green}$+1.2$})} & 
33.9\scriptsize{$\pm1.4$}       \normalsize{({\color{red}$-0.9$})} & 
%		39.2\scriptsize{$\pm0.1$}            & 
47.9   \normalsize{({\color{green}$+2.1$})}        \\


SagNet$^\dagger$ \cite{nam2019sagnet}           &             53.0\scriptsize{$\pm2.0$}       \normalsize{({\color{green}$+2.3$})} & 
43.0\scriptsize{$\pm1.4$}       \normalsize{({\color{green}$+0.2$})} & 57.9\scriptsize{$\pm0.8$}       \normalsize{({\color{red}$-2.6$})}& 
40.4\scriptsize{$\pm1.4$}       \normalsize{({\color{green}$+2.9$})}& 
%		40.3\scriptsize{$\pm0.1$}          & 
48.6 \normalsize{({\color{green}$+0.4$})}         \\



\midrule



%Fish $^\ddagger$ \cite{shi2021gradient}                     & 85.5\scriptsize{$\pm0.3$}          
%\normalsize{({\color{green}$+0.1$})} & 77.8\scriptsize{$\pm0.3$}          
%\normalsize{({\color{green}$+0.9$})} & 
%68.6\scriptsize{$\pm0.4$}            
%\normalsize{({\color{red}$-0.6$})} & 
%45.1\scriptsize{$\pm1.3$}            
%\normalsize{({\color{green}$+0.8$})} & 
%%		42.7\scriptsize{$\pm0.2$}            & 
%69.3 \normalsize{({\color{green}$+0.3$})}             \\ 

SAM $^\ddagger$\cite{foret2020sharpness}  & 
46.3\scriptsize$\pm1.0$          \normalsize{({\color{green}$+3.3$})} & 
38.4\scriptsize$\pm2.4$          \normalsize{({\color{green}$+5.2$})} & 
54.0\scriptsize$\pm1.0$          \normalsize{({\color{green}$+1.9$})} & 
34.5\scriptsize$\pm0.8$          \normalsize{({\color{red}$-0.1$})} & 
%		44.3\scriptsize$\pm0.0$              & 
43.3 \normalsize{({\color{green}$+2.6$})}   \\

RSC$^\dagger$ \cite{huang2020rsc}               & 
50.2\scriptsize{$\pm2.2$}      \normalsize{({\color{red}$-0.8$})}& 
39.2\scriptsize{$\pm1.4$}      \normalsize{({\color{green}$+1.0$})}& 
56.3\scriptsize{$\pm1.4$}      \normalsize{({\color{green}$+0.8$})}&
40.8\scriptsize{$\pm0.6$}      \normalsize{({\color{green}$+0.2$})}& 
%		38.9\scriptsize{$\pm0.5$}          & 
46.6 \normalsize{({\color{green}$+0.2$})}   \\

GSAM $^\ddagger$ \cite{zhuang2022surrogate} & 
50.8\scriptsize$\pm0.1$        \normalsize{({\color{green}$+3.8$})} & 
39.3\scriptsize$\pm0.2$        \normalsize{({\color{green}$+0.6$})} & 
59.6\scriptsize$\pm0.0$        \normalsize{({\color{red}$-2.2$})}& 
38.2\scriptsize$\pm0.8$        \normalsize{({\color{green}$+0.4$})}& 
%		44.6\scriptsize$\pm0.2$             & 
47.0  \normalsize{({\color{green}$+0.6$})}  \\

SAGM $^\ddagger$ \cite{wang2023sharpness}       & 
54.8\scriptsize{$\pm1.3$}   \normalsize{($\pm0.0$)}& 
41.4\scriptsize{$\pm0.8$}   \normalsize{({\color{green}$+6.3$})}& 
57.7\scriptsize{$\pm0.6$}   \normalsize{({\color{red}$-1.1$})}& 
41.3\scriptsize{$\pm0.4$}   \normalsize{({\color{red}$-5.5$})}& 
%		45.0\scriptsize{$\pm0.2$}           & 
48.8 \normalsize{({\color{red}$-0.1$})}   \\

%		\midrule
\midrule
\textbf{GGA} (ours)                 & 
55.9\scriptsize{$\pm0.1$}           & 
45.5\scriptsize{$\pm0.1$}           & 
59.7\scriptsize{$\pm0.1$}           &  
41.5\scriptsize{$\pm0.1$}    &  
50.6  \\
%		\midrule
\bottomrule
\end{tabular}
\end{table*}

%\subsection{DomainNet}
\begin{table*}[]
\centering
\small
\renewcommand{\arraystretch}{1.1}
\caption{\small{Out-of-domain accuracies (\%) on {DomainNet}.}}
\begin{tabular}{lllllll|c}
\toprule
\textbf{Algorithm} & \textbf{clip} & \textbf{info} & \textbf{paint} & \textbf{quick} & \textbf{real} & \textbf{sketch} & \textbf{Avg} \\
\midrule

MMD$^\dagger$ \cite{li2018mmd}   & 
32.1 \scriptsize$\pm13.3$ & 
11.0 \scriptsize$\pm4.6$ & 
26.8 \scriptsize$\pm11.3$ & 
8.7 \scriptsize$\pm2.1$ & 
32.7 \scriptsize$\pm13.8$ & 
28.9 \scriptsize$\pm11.9$ & 23.4 \\


GroupDRO$^\dagger$ \cite{Sagawa2020GroupDRO} & 
47.2 \scriptsize$\pm0.5$ & 
17.5 \scriptsize$\pm0.4$ & 
33.8 \scriptsize$\pm0.5$ & 
9.3 \scriptsize$\pm0.3$ & 
51.6 \scriptsize$\pm0.4$ & 
40.1 \scriptsize$\pm0.6$ & 33.3 \\

VREx$^\dagger$ \cite{krueger2020vrex} & 
47.3 \scriptsize$\pm3.5$ & 
16.0 \scriptsize$\pm1.5$ & 
35.8 \scriptsize$\pm4.6$ & 
10.9 \scriptsize$\pm0.3$ & 
49.6 \scriptsize$\pm4.9$ & 
42.0 \scriptsize$\pm3.0$ & 33.6 \\

IRM$^\dagger$ \cite{arjovsky2019irm}  & 
48.5 \scriptsize$\pm2.8$ & 
15.0 \scriptsize$\pm1.5$ & 
38.3 \scriptsize$\pm4.3$ & 
10.9 \scriptsize$\pm0.5$ & 
48.2 \scriptsize$\pm5.2$ & 
42.3 \scriptsize$\pm3.1$ & 33.9 \\

Mixstyle$^\ddagger$ \cite{zhou2021mixstyle} & 
51.9 \scriptsize$\pm0.4$ & 
13.3 \scriptsize$\pm0.2$ & 
37.0 \scriptsize$\pm0.5$ & 
12.3 \scriptsize$\pm0.1$ & 
46.1 \scriptsize$\pm0.3$ & 
43.4 \scriptsize$\pm0.4$ & 34.0 \\

ARM$^\dagger$ \cite{zhang2020arm} & 
49.7 \scriptsize$\pm0.3$ & 
16.3 \scriptsize$\pm0.5$ & 
40.9 \scriptsize$\pm1.1$ & 
9.4 \scriptsize$\pm0.1$ & 
53.4 \scriptsize$\pm0.4$ & 
43.5 \scriptsize$\pm0.4$ & 35.5 \\

Mixup$^\ddagger$ \cite{xu2020interdomain_mixup_aaai} & 
55.7\scriptsize$\pm0.3$ & 
18.5\scriptsize$\pm0.5$ & 
44.3\scriptsize$\pm0.5$ & 
12.5\scriptsize$\pm0.4$ & 
55.8\scriptsize$\pm0.3$ & 
48.2\scriptsize$\pm0.5$ & 39.2 \\


SagNet$^\dagger$ \cite{nam2019sagnet} & 
57.7 \scriptsize$\pm0.3$ & 
19.0 \scriptsize$\pm0.2$ & 
45.3 \scriptsize$\pm0.3$ & 
12.7 \scriptsize$\pm0.5$ & 
58.1 \scriptsize$\pm0.5$ & 
48.8 \scriptsize$\pm0.2$ & 40.3 \\

MTL$^\dagger$ \cite{blanchard2021mtl_marginal_transfer_learning}& 
57.9 \scriptsize$\pm0.5$ & 
18.5 \scriptsize$\pm0.4$ & 
46.0 \scriptsize$\pm0.1$ & 
12.5 \scriptsize$\pm0.1$ & 
59.5 \scriptsize$\pm0.3$ & 
49.2 \scriptsize$\pm0.1$ & 40.6 \\

MLDG$^\dagger$ \cite{li2018learning}& 
59.1 \scriptsize$\pm0.2$ & 
19.1 \scriptsize$\pm0.3$ & 
45.8 \scriptsize$\pm0.7$ & 
13.4 \scriptsize$\pm0.3$ & 
59.6 \scriptsize$\pm0.2$ & 
50.2 \scriptsize$\pm0.4$ & 41.2 \\

CORAL$^\dagger$ \cite{sun2016coral}& 
59.2 \scriptsize$\pm0.1$ & 
19.7 \scriptsize$\pm0.2$ & 
46.6 \scriptsize$\pm0.3$ & 
13.4 \scriptsize$\pm0.4$ &
59.8 \scriptsize$\pm0.2$ & 
50.1 \scriptsize$\pm0.6$ & 41.5 \\

ERM$^\ddagger$ \cite{vapnik1998statistical} & 
63.0 \scriptsize$\pm0.2$ & 
21.2 \scriptsize$\pm0.2$ & 
50.1 \scriptsize$\pm0.4$ & 
13.9 \scriptsize$\pm0.5$ & 
63.7 \scriptsize$\pm0.2$ &
52.0 \scriptsize$\pm0.5$ & 43.8 \\

\midrule 
% Gradient Based Methods

RSC$^\dagger$ \cite{huang2020rsc}               & 
55.0\scriptsize{$\pm1.2$} & 
18.3\scriptsize{$\pm0.5$} & 
44.4\scriptsize{$\pm0.6$} &
12.2\scriptsize{$\pm0.2$} & 
55.7\scriptsize{$\pm0.7$} &
47.8\scriptsize{$\pm0.9$} &
38.9 \\

SAM$^\ddagger$ \cite{foret2020sharpness}  & 
64.5\scriptsize$\pm0.3$  & 
20.7\scriptsize$\pm0.2$  & 
50.2\scriptsize$\pm0.1$  & 
15.1\scriptsize$\pm0.3$  & 
62.6\scriptsize$\pm0.2$  & 
52.7\scriptsize$\pm0.3$  &  44.3   \\


GSAM $^\ddagger$ \cite{zhuang2022surrogate} & 
64.2\scriptsize$\pm0.3$  & 
20.8\scriptsize$\pm0.2$  & 
50.9\scriptsize$\pm0.0$  & 
14.4\scriptsize$\pm0.8$  & 
63.5\scriptsize$\pm0.2$  &
53.9\scriptsize$\pm0.2$  &  44.6   \\


SAGM $^\ddagger$ \cite{wang2023sharpness}   & 
64.9\scriptsize{$\pm0.2$}  & 
21.1\scriptsize{$\pm0.3$}  & 
51.5\scriptsize{$\pm0.2$}  & 
14.8\scriptsize{$\pm0.2$}  & 
64.1\scriptsize{$\pm0.2$}  & 
53.6\scriptsize{$\pm0.2$}  & 45.0 \\

%		\midrule
\midrule

GGA (ours) & 
64.0\scriptsize{$\pm0.2$} & 
22.2\scriptsize{$\pm0.3$} & 
51.7\scriptsize{$\pm0.1$} & 
14.3\scriptsize{$\pm0.2$} & 
64.1\scriptsize{$\pm0.4$} & 
54.3\scriptsize{$\pm0.3$} & 45.2 \\

\bottomrule
\end{tabular}
\end{table*}

%\subsection{CMNIST}
\begin{table*}
	\centering
	\small
	\renewcommand{\arraystretch}{1.1}
	\caption{\small{Out-of-domain accuracies (\%) on ColoredMNIST (left) and RotatedMNIST (right).}}
	\begin{tabular}{llll|c|llllll|c}
		\toprule
		\textbf{Algorithm} & \textbf{0.1} & \textbf{0.2} & \textbf{0.9} & \textbf{Avg} 
		&\textbf{0} &\textbf{15} &\textbf{30} &
		\textbf{45} &\textbf{60} &
		\textbf{75} &\textbf{Avg} \\
		\midrule
		
		IRM$\ddagger$ \cite{arjovsky2019irm} & 
		56.8\scriptsize$\pm4.5$ & 63.5\scriptsize$\pm2.7$ & 
		10.2\scriptsize$\pm0.2$ & 
		43.5 &
		95.5\scriptsize{$\pm0.4$} & 98.7\scriptsize{$\pm0.2$} & 
		98.7\scriptsize{$\pm0.1$} & 98.5\scriptsize{$\pm0.3$} & 
		98.7\scriptsize{$\pm0.1$} & 96.1\scriptsize{$\pm0.1$} & 97.7\\ 
		
		MLDG \cite{li2018learning} & 
		71.5\scriptsize$\pm0.6$ & 73.0\scriptsize$\pm0.1$ & 
		10.1\scriptsize$\pm0.2$ & 
		51.5 &
		94.7\scriptsize{$\pm0.7$} & 98.8\scriptsize{$\pm0.1$} & 
		98.8\scriptsize{$\pm0.1$} & 98.8\scriptsize{$\pm0.1$} & 
		98.7\scriptsize{$\pm0.1$} & 95.9\scriptsize{$\pm0.4$} & 97.6\\ 
		
		MTL \cite{blanchard2021mtl_marginal_transfer_learning} & 
		71.3\scriptsize$\pm0.6$ & 72.9\scriptsize$\pm0.2$ & 
		10.2\scriptsize$\pm0.1$ & 
		51.5 &
		94.6\scriptsize{$\pm1.1$} & 98.6\scriptsize{$\pm0.2$} & 
		98.8\scriptsize{$\pm0.1$} & 98.7\scriptsize{$\pm0.1$} & 
		98.7\scriptsize{$\pm0.3$} & 95.3\scriptsize{$\pm0.7$} & 97.4\\ 
		
		Mixup \cite{xu2020interdomain_mixup_aaai} & 
		71.5\scriptsize$\pm0.8$ & 73.2\scriptsize$\pm0.3$ & 
		10.2\scriptsize$\pm0.2$ & 
		51.6 &
		94.9\scriptsize{$\pm0.5$} & 98.8\scriptsize{$\pm0.1$} & 
		98.8\scriptsize{$\pm0.2$} & 98.8\scriptsize{$\pm0.1$} & 
		98.8\scriptsize{$\pm0.1$} & 95.7\scriptsize{$\pm0.5$} & 97.6\\
		
		SagNet \cite{nam2019sagnet}  & 
		72.0\scriptsize$\pm0.5$ & 72.8\scriptsize$\pm0.5$ & 
		9.9\scriptsize$\pm0.3$ & 
		51.6 &
		95.5\scriptsize{$\pm0.3$} & 98.9\scriptsize{$\pm0.1$} & 
		99.0\scriptsize{$\pm0.1$} & 98.8\scriptsize{$\pm0.2$} & 
		98.8\scriptsize{$\pm0.1$} & 95.9\scriptsize{$\pm0.4$} & 97.8\\
		
		ERM \cite{vapnik1998statistical} & 
		71.8\scriptsize$\pm0.9$ & 73.3\scriptsize$\pm0.4$ & 
		9.9\scriptsize$\pm0.3$ & 
		51.7 &
		
		95.1\scriptsize{$\pm0.6$} & 98.7\scriptsize{$\pm0.2$} & 
		98.7\scriptsize{$\pm0.2$} & 98.7\scriptsize{$\pm0.2$} & 
		98.8\scriptsize{$\pm0.1$} & 95.6\scriptsize{$\pm0.4$} & 97.6 \\
		
		ARM \cite{zhang2020arm}  & 
		74.5\scriptsize{$\pm3.8$} & 71.1\scriptsize$\pm1.8$ & 
		9.9\scriptsize$\pm0.3$   & 
		51.8 &
		
		95.1\scriptsize{$\pm1.1$} & 98.8\scriptsize{$\pm0.2$} & 
		98.8\scriptsize{$\pm0.1$} & 98.8\scriptsize{$\pm0.1$} & 
		98.8\scriptsize{$\pm0.1$} & 96\scriptsize{$\pm0.6$} & 97.7 \\ 
		
		CORAL \cite{sun2016coral} & 
		72.3\scriptsize$\pm0.7$ & 72.8\scriptsize$\pm0.4$ & 
		10.5\scriptsize$\pm0.3$ & 
		51.8 &
		
		95.6\scriptsize{$\pm0.3$} & 98.9\scriptsize{$\pm0.1$} & 
		98.9\scriptsize{$\pm0.1$} & 99.0\scriptsize{$\pm0.0$} & 
		98.9\scriptsize{$\pm0.1$} & 96.1\scriptsize{$\pm0.3$} & 97.9 \\ 
		
		
		
		Fish \cite{shi2021gradient}  & 
		71.7\scriptsize$\pm0.5$ & 73.2\scriptsize$\pm0.5$ & 
		10.4\scriptsize$\pm0.2$ & 
		51.8 &
		95.3\scriptsize{$\pm0.6$} & 98.9\scriptsize{$\pm0.1$} & 
		98.9\scriptsize{$\pm0.2$} & 98.9\scriptsize{$\pm0.1$} & 
		98.9\scriptsize{$\pm0.1$} & 95.6\scriptsize{$\pm0.6$} & 97.7 \\ 
		
		GroupDRO \cite{Sagawa2020GroupDRO} & 
		72.6\scriptsize$\pm0.6$ & 73.5\scriptsize$\pm0.4$ & 
		9.9\scriptsize$\pm0.2$ & 
		52.0 &
		95.9\scriptsize{$\pm0.6$} & 98.7\scriptsize{$\pm0.2$} & 
		98.6\scriptsize{$\pm0.1$} & 98.7\scriptsize{$\pm0.1$} & 
		98.7\scriptsize{$\pm0.1$} & 96.0\scriptsize{$\pm0.2$} & 97.8\\  	
		
		VREx \cite{krueger2020vrex} & 
		72.9\scriptsize$\pm0.3$ & 72.9\scriptsize$\pm0.4$ & 
		10.3\scriptsize$\pm0.6$ & 
		52.0 &
		95.7\scriptsize{$\pm0.6$} & 98.9\scriptsize{$\pm0.2$} & 
		98.7\scriptsize{$\pm0.1$} & 98.9\scriptsize{$\pm0.2$} & 
		98.9\scriptsize{$\pm0.1$} & 95.8\scriptsize{$\pm0.4$} & 97.8\\ 
		 
		
		\midrule
		
		SAM \cite{foret2020sharpness} & 
		71.1\scriptsize$\pm0.5$ & 73.3\scriptsize$\pm0.4$ & 
		10.1\scriptsize$\pm0.3$ & 
		51.5 &
		95.7\scriptsize{$\pm0.2$} & 99.0\scriptsize{$\pm0.1$} & 
		98.9\scriptsize{$\pm0.1$} & 98.9\scriptsize{$\pm0.1$} & 
		98.9\scriptsize{$\pm0.1$} & 96.2\scriptsize{$\pm0.4$} & 97.9\\
		
		GSAM \cite{zhuang2022surrogate} & 
		71.8\scriptsize$\pm0.3$ & 73.2\scriptsize$\pm0.2$ & 
		9.9\scriptsize$\pm0.2$ & 
		51.6 &
		94.9\scriptsize{$\pm0.1$} & 98.9\scriptsize{$\pm0.1$} & 
		98.9\scriptsize{$\pm0.2$} & 99.0\scriptsize{$\pm0.2$} & 
		98.8\scriptsize{$\pm0.1$} & 96.0\scriptsize{$\pm0.1$} & 97.7\\
		
		RSC \cite{huang2020rsc} & 
		72.5\scriptsize$\pm0.3$ & 72.4\scriptsize$\pm0.6$ & 
		10.2\scriptsize$\pm0.5$ & 
		51.7 &
		94.2\scriptsize{$\pm1.1$} & 98.6\scriptsize{$\pm0.1$} & 
		98.7\scriptsize{$\pm0.2$} & 98.6\scriptsize{$\pm0.2$} & 
		98.7\scriptsize{$\pm0.2$} & 95.7\scriptsize{$\pm0.7$} & 97.4\\  
		
		
		SAGM \cite{wang2023sharpness} & 
		71.5\scriptsize$\pm0.8$ & 73.6\scriptsize$\pm0.5$ & 
		10.6\scriptsize$\pm0.6$ & 
		51.9 &
		95.4\scriptsize{$\pm0.4$} & 98.9\scriptsize{$\pm0.1$} & 
		98.9\scriptsize{$\pm0.1$} & 98.9\scriptsize{$\pm0.1$} & 
		98.9\scriptsize{$\pm0.1$} & 95.9\scriptsize{$\pm0.5$} & 97.8\\
		
		\midrule
		\textbf{GGA} (ours) & 
		71.2\scriptsize$\pm0.7$ & 73.1\scriptsize$\pm0.6$ & 
		11.5\scriptsize$\pm0.4$ & 
		51.9 &
		95.1\scriptsize{$\pm0.8$} & 99.0\scriptsize{$\pm0.1$} & 
		99.0\scriptsize{$\pm0.3$} & 98.8\scriptsize{$\pm0.1$} & 
		98.8\scriptsize{$\pm0.2$} & 96.1\scriptsize{$\pm0.4$} & 97.8 \\ 
		\bottomrule
	\end{tabular}
\end{table*}


\clearpage
\clearpage

\bibliographystyle{ieeenat_fullname}
\bibliography{supp}


% WARNING: do not forget to delete the supplementary pages from your submission 
% 
\clearpage
% \setcounter{page}{1}
% \maketitlesupplementary
\begin{center}
Supplementary Material
\end{center}

% {
%     \onecolumn
%     \centering
%     \Large
%     \textbf{\thetitle}\\
%     \vspace{0.5em}Supplementary Material \\
%     \vspace{1.0em}
% }

\section{Proof of \cref{theorem:dr}}
We require some additional regularity assumptions:
\begin{assumption} 1) The number of classes $C$ is bounded w.r.t the number of samples $N$, 2) the missingness mechanism $P(A=1|Y,\theta)$, as well as its estimated counterpart $P(A=1|Y,\theta)$, are bounded below by some constant $\epsilon > 0$, 3) the quantities $P(Y|X,\theta)$ and $P(A|Y,\theta)$ are estimated using auxiliary samples independent of samples used for the sample averaging.
\label{assumption:extra}
\end{assumption}
Assumptions 1 and 2 are natural. For the missingness mechanism, the ground truth being bounded means that there is a non-vanishing proportion of samples for every class. The boundedness of the estimate can be enforced by clipping the estimate. Assumption 3 is called sample splitting in \cite{kennedy-dr}.

For convenience we use operator $\E_N$ to denote the average of $N$ samples i.e. $\frac{1}{N}\sum_{i=1}^N$. Note that this is by itself a random variable, in contrast to $\E$ which is a fixed number.

\begin{proof}[Proof of \cref{theorem:dr}] Because $C$ is bounded (assumption \ref{assumption:extra}), we can fix a class $c$ and prove the theorem.
Let us define the influence function $\phi$, parameterized by $\theta$, as
\begin{equation}
\phi(O | \theta)(c) = P(Y=c|X,\theta) + \frac{\one(A=1)}{P(A=1|Y,\theta)} (\one(Y=c) - P(Y=c|X,\theta)) - P(Y=c)
\end{equation}
As we have done in the main text, we use $\phi(O)$ to denote the same function but all estimated quantities are replaced with their truths. In other words, we use $\phi(O)$ for $\phi(O|\theta_0)$ where $\theta_0$ is the truth, given that our model contains $\theta_0$ e.g. when the model is consistent.

Recall that:
\begin{equation}
\begin{aligned}
\Psi_{dr}(\theta)(c) &= \frac{1}{N}\sum_{i=1}^N \left\{P(Y=c|X,\theta) + \frac{\one(A=1)}{P(A=1|Y,\theta)} (\one(Y=c) - P(Y=c|X,\theta))\right\}\\
&= \E_N [\phi(O|\theta)(c)] + P(Y=c)
\end{aligned}
\end{equation}

We will show that:
\begin{equation}
\Psi_{dr}(\theta)(c) - P(Y=c) = (\E_N - \E)[\phi(O)(c)] + o_P(N^{-1/2})
\label{eq:proof-linearity}
\end{equation}
To do that, we use the following decomposition
\begin{equation}
\begin{aligned}
\Psi_{dr}(\theta)(c) - P(Y=c) &= \E_N [\phi(O|\theta)(c)] \\
&= (\E_N - \E)[\phi(O)(c)] + (\E_N - \E)[\phi(O|\theta)(c) - \phi(O)(c)] + \E[\phi(O|\theta)(c)]
% &+ (\E_n - \E)[\phi(O;\theta) - \phi(O)]\\
% &+ \E[P(Y=c|X,\theta)] - \E[P(Y=c|X)] + \E[\phi(O,\theta)]
\end{aligned}
\end{equation}
and analyze the second and third term. The third term is:
\begin{equation}
\begin{aligned}
\E[\phi(O|\theta)(c)] &= \E[P(Y=c|X,\theta)] + \E\left[\frac{\one(A=1)}{P(A=1|Y,\theta)}(\one(Y=c) - P(Y=c|X,\theta))\right]- P(Y=c) \\
&= \E\left[P(Y=c|X,\theta) + \frac{P(A=1|Y)}{P(A=1|Y,\theta)}(P(Y=c|X) - P(Y=c|X,\theta))\right] - \E[P(Y=c|X)]\\
&= \E\left[(P(Y=c|X,\theta) - P(Y=c|X)) (P(A=1|Y,\theta) -P(A=1|Y)) \frac{1}{P(A=1|Y,\theta)}\right]\\
\end{aligned}
\end{equation}
by Cauchy-Schwarz inequality:
\begin{equation}
\begin{aligned}
\E[\phi(O|\theta)(c)] &\le \frac{1}{\epsilon} \|P(A=1|Y,\theta) - P(A=1|Y)\|_2 \|P(Y=c|X,\theta) - P(Y=c|X)\|_{L_2(P)}\\
&= \frac{1}{\epsilon} o_P(N^{-1/4} N^{-1/4}) = o_P(N^{-1/2})
\end{aligned}
\end{equation}
by assumption \ref{assumption:4th-root-n} and that $P(A=1|Y,\theta) > \epsilon$ (assumption \ref{assumption:extra}). The second term can be bounded by Chebyshev inequality
% \begin{equation}
% \begin{aligned}
% \E[\E_N[\phi(O|\theta)(c) - \phi(O)(c)]] &= \E[\phi(O|\theta)(c) - \phi(O)(c)]\\
% \var[\E_N[\phi(O|\theta)(c) - \phi(O)(c)]] &= \frac{1}{N}\var[\phi(O|\theta)(c) - \phi(O)(c)] \le 
% \end{aligned}
% \end{equation}
\begin{equation}
P(|(\E_N - \E)[\phi(O|\theta)(c) - \phi(O)(c)]| \ge t) \le \frac{\var[\E_N[\phi(O|\theta)(c) - \phi(O)(c)]]}{t^2} = \frac{\var[\phi(O|\theta)(c) - \phi(O)(c)]}{Nt^2}
\end{equation}
note here that $\theta$ is independent of the samples used for $\E_N$ by assumption \ref{assumption:extra}. For any $\varepsilon > 0$, by picking $t = \frac{1}{\sqrt{N\varepsilon}}$ we get
\begin{equation}
P\left(\left|\frac{(\E_N - \E)[\phi(O|\theta)(c) - \phi(O)(c)]}{N^{-1/2}}\right| \ge \frac{1}{\sqrt{\varepsilon}}\right) \le \varepsilon \var[\phi(O|\theta)(c) - \phi(O)(c)]
\end{equation}
by the definition of $O_P$, we then get
\begin{equation}
(\E_N - \E)[\phi(O|\theta)(c) - \phi(O)(c)] = O_P(N^{-1/2}\var[\phi(O|\theta)(c) - \phi(O)(c)])
\end{equation}
Because $\phi$ is a continuous function of $P(Y|X,\theta)$ and $P(A|Y,\theta)$ (given $P(A|Y,\theta) > \epsilon$, assumption \ref{assumption:extra}), by the continuous mapping theorem and the fact that $P(Y|X,\theta)$ and $P(A|Y,\theta)$ are convergent in probability (assumption \ref{assumption:4th-root-n}), we get $\var[\phi(O|\theta)(c) - \phi(O)(c)] = o_P(1)$. This gives
\begin{equation}
(\E_N - \E)[\phi(O|\theta)(c) - \phi(O)(c)] = o_P(N^{-1/2})
\end{equation}
Therefore, we have shown that the second and third term are both $o_P(N^{-1/2})$, proving \cref{eq:proof-linearity}. As the final step, multiply both sides of this equation by $\sqrt{N}$ we get:
\begin{equation}
\sqrt{N}(\Psi_{dr}(\theta)(c) - P(Y=c)) = \sqrt{N} (\E_N - \E)[\phi(O)(c)] + o_P(1) \rightsquigarrow \mathcal{N}(0, \var[\phi(O)(c)])
\end{equation}
by the central limit theorem, and $\var[\phi(O)(c)] = \E[\phi(O)(c)^2]$ because $\E[\phi(O)(c)] = 0$.
\end{proof}

While we started with the definition of $\phi$, \cref{eq:proof-linearity} shows that $\phi$ is indeed an influence function. Now we show that $\phi$ is also the efficient influence function, by using the characterization of the model's tangent space \cite{tsiatis-missingdata}. Note that the joint probability factorizes as $P(X,A,Y) = P(X)P(Y|X)P(A|Y)$, therefore the tangent space $\mathcal{T}$ factorizes as $\mathcal{T} = \mathcal{T}_{X} \oplus \mathcal{T}_{Y|X} \oplus \mathcal{T}_{A|Y}$ where $\mathcal{T}_X = \{h(X): \E[h] = 0\}$, $\mathcal{T}_{Y|X} = \{h(X,Y): \E[h|X] = 0\}$, $\mathcal{T}_{A|Y} = \{h(A,Y): \E[h|Y] = 0\}$, and the 3 subspaces are pairwise orthogonal. All influence functions are orthogonal to the tangent space, but the influence function that is also in the tangent space has the smallest variance and is called the efficient influence function. As $\phi$ is already an influence function, we need only show that $\phi$ is in $\mathcal{T}$. We write $\phi$ as
\begin{equation}
\phi(O)(c) = (P(Y=c|X) - P(Y=c)) + \left[\frac{\one(A=1)}{P(A=1|Y)} - 1\right](\one(Y=c) - P(Y=c|X)) + (\one(Y=c) - P(Y=c|X))
\end{equation}
and note that the first, second and third term are in $\mathcal{T}_X$, $\mathcal{T}_{A|Y}$ and $\mathcal{T}_{Y|X}$ respectively. Therefore, $\phi$ is indeed in $\mathcal{T}$. The efficient influence function has the smallest variance of all influence function, and therefore our estimator being asymptotically linear in $\phi$ (\cref{eq:proof-linearity}) has the smallest mean squared error in a local asymptotic minimax sense \cite{kennedy-dr, asymptoticstatistics}

\section{Further background and related work}
\paragraph{Discussion on semi-supervised EM.}
It appears that semi-supervised EM was first used for parameter estimation when the missingness mechanism is non-ignorable in \cite{ibrahim1996parameter}, but has not been used for label shift estimation.
Perhaps this is because the semi-supervised situation where additional unlabeled data is available during training is rarer than the test-time adaptation case. EM is well suited to take advantage of the extra unlabeled data to improve the classifier under very scarce and long-tailed labeled data. While the connection between pseudo-labeling and EM has been explored before \cite{entropyminimization}, the situation with label shift has not until recently \cite{simpro}. Here the application of EM is much more interesting, because other than simply giving pseudo-labeling a rigorous formulation, EM also estimates the missingness mechanism (equivalently the label distribution shift), which is important for shift correction and thus high-quality pseudo-labels \cite{acr}. The application of confidence thresholding can be seen as a sparse variant of EM \cite{neal1998view}.

\paragraph{The doubly-robust risk.} 
\label{subsec:dr-risk}
A technique that also derives from the theory of semi-parametric efficiency is orthogonal statistical learning \citep{foster2023orthogonal}. The idea is to minimize the doubly-robust risk:
\label{subsec:method-dr-risk}
\begin{equation}
\label{eq:dr-risk}
\mathcal{R}(\theta_2) = \frac{1}{N} \sum_{i=1}^N \Bigg[ l(x_i, \hat y_i|\theta_2) + \frac{\one(a_i=1)}{P(A=a_i|Y=y_i, \theta_1)} (l(x_i, y_i | \theta_2) - l(x_i, \hat y_i | \theta_2))\Bigg]
\end{equation}
where $l(x,y|\theta) = -\sum_{c=1}^C [y]_c \log P(Y=c|X=x,\theta)$ is the negative cross-entropy. 
The notation $[y]_c$ means that we are using the $c$-entry in a C-dimension probability vector $y$. 
Thus, $y_i$ denotes the one-hot label of observation $i$, while $\hat y_i$ denotes the pseudo-label, which can be one-hot or all-zero. 
Finally, we use $\theta_1$ to denote that $P(a|y,\theta_1)$ is an estimation from a previous stage, but it can be estimated with $\theta_2$ as well. 
The risk $\mathcal{R}(\theta_2)$ can be used as a training loss in a straightforward fashion. 
Similar to the doubly robust estimation of $P(Y)$, the doubly robust risk provides approximately unbiased estimation of the risk. 
This property has been used in \citep{arelabelsinformative, onnonrandommissinglabels, drst} also in the semi-supervised learning setting.
More broadly, it is at the heart of one of the core techniques in heterogenous treatment effect estimation in causal estimation \cite{kennedy2023towards, foster2023orthogonal, wager2018estimation}. 
The focus here is not the estimation of $\mathcal{R}(\theta_2)$ per se, but the quality of the learned model \cite{foster2023orthogonal}.
By using the doubly-robust risk, we can achieve an optimality result similar in spirit to our theorem \cref{theorem:dr}, but for the generalization error.
While this is appealing, in practice there are 2 problems with this approach. First, the inverse probability weight $P(A=a_i|Y=y_i,\theta_1)$ can be very large if the class ratio is highly unlabeled, making training unstable \cite{kallus2020deepmatch, pham2023stable}. 
This problem exists for our estimation as well. However, it is much easier to control for estimation than for training because of the iterative nature of model update. Secondly, we can further write $\mathcal{R}$ as:
\begin{equation}
\mathcal{R}(\theta_2) = \frac{1}{N}\sum_{i=1}^N l\left(x_i, \hat y_i + \frac{\one(a_i=1)}{P(A=a_i|Y=y_i,\theta_1)} (y_i - \hat y_i)\Bigg\vert\theta_2\right)
\end{equation}
which is a cross-entropy loss with new meta-pseudo-labels. However, these labels are not meant to be learned exactly, and furthermore they can be negative. Thus, theoretical works have to put stringent assumptions on the models. In \cref{subsec:ablation-1}, we show that experimentally that the instability problem makes doubly-robust risk performance worse than our 2-stage approach.

\section{Training and hyperparameter settings.}
\label{subsec:training-setting}
For neural network training, we follow the implementation and hyperparameter settings of \cite{simpro}. In particular, we adapt the core code of SimPro for Supervised, MLE and EM. For MLE, we update $P(A|Y)$ using the Adam optimizer with learning rate 1e-3, while for EM we use a momentum update similar to SimPro's update of $P(Y|A)$ because it has a a closed-form solution at each mini-batch. We use Wide ResNet-28-2 on all methods and all datasets in this section, including Imagenet-127, because we are motivated by the fact that stage-1's goal is not classification accuracy but the estimation of a finite-dimensional parameter. When using Wide ResNet-28-2 for Imagenet-127, we use the hyperparameters of CIFAR-100, except we lower the batch size of unlabeled data to 2 times that of labeled data instead of 8 for memory reason. We do not perform additional hyperparameter tuning. All experiments can be performed on 1 A6000 RTX GPU, and are run 3 times. We report the total variation distance between the estimated and the ground truth unlabeled class distribution, similar to its usage in Theorem 3.1 of \cite{lsc}, and the top-1 classification accuracy.

In the second stage of our algorithm, we freeze our estimation and plug it in SimPro and BOAT.
We keep exactly the same hyperparameter settings that SimPro and BOAT use. In particular, for Imagenet-127, we now use ResNet-50 and run each experiment once.
In SimPro, we set the unlabeled class distribution $P(Y|A=0)$ at the E-step;  however, we still keep a running estimate of the class distribution $P(Y)$ in the logit adjustment loss \cref{eq:simpro-la-loss}. While it is possible to use the first stage estimate in the logit adjustment loss, we observe that doing so results in lower accuracy than using the the running average. This is conceptually consistent with the role of the running average - serving not as an accurate estimate of $P(Y)$ but to make the classifier's class distribution uniform through the logit adjustment loss, which is good for the test set. Similarly, in BOAT, we only replace $\Delta_c = \log P(Y|A=1) - \log P(Y|A=0)$ in equation (4) of \cite{boat}, which is adjusting a classifier's predictions from the labeled to the unlabeled class distribution, with our SimPro + DR estimate instead of their on-the-fly estimate. 


% \section{Additional experiments}
% % \begin{table*}[t]
\centering
\caption{Total Variation Distance on CIFAR-10-LT ($N_l = 500$, $M_l = 4000$) with different class imbalance ratios $\gamma_l$ and $\gamma_u$ under five different unlabeled class distributions.}
\label{tab:cifar10-tv}
\resizebox{\textwidth}{!}{
\begin{tabular}{lccccccccccc}
\toprule
& & \multicolumn{2}{c}{consistent} & \multicolumn{2}{c}{uniform} & \multicolumn{2}{c}{reversed} & \multicolumn{2}{c}{middle} & \multicolumn{2}{c}{head-tail} \\
\cmidrule(lr){3-4} \cmidrule(lr){5-6} \cmidrule(lr){7-8} \cmidrule(lr){9-10} \cmidrule(lr){11-12}
& & $\gamma_l = 150$ & $\gamma_l = 100$ & $\gamma_l = 150$ & $\gamma_l = 100$ & $\gamma_l = 150$ & $\gamma_l = 100$ & $\gamma_l = 150$ & $\gamma_l = 100$ & $\gamma_l = 150$ & $\gamma_l = 100$ \\
Model & Estimator & $\gamma_u = 150$ & $\gamma_u = 100$ & $\gamma_u = 1$ & $\gamma_u = 1$ & $\gamma_u = 1/150$ & $\gamma_u = 1/100$ & $\gamma_u = 150$ & $\gamma_u = 100$ & $\gamma_u = 150$ & $\gamma_u = 100$ \\
\midrule
Supervised & MLLS & 0.269 ± 0.252 & 0.038 ± 0.006 & 0.251 ± 0.046 & 0.255 ± 0.060 & 0.429 ± 0.028 & 0.493 ± 0.050 & 0.333 ± 0.042 & 0.320 ± 0.009 & 0.457 ± 0.034 & 0.444 ± 0.043 \\
Supervised & RLLS & 0.043 ± 0.001 & 0.044 ± 0.010 & 0.348 ± 0.034 & 0.305 ± 0.068 & 0.769 ± 0.016 & 0.678 ± 0.028 & 0.430 ± 0.008 & 0.368 ± 0.013 & 0.539 ± 0.018 & 0.503 ± 0.020 \\
\midrule
MLE & IPW & 0.027 ± 0.001 & 0.027 ± 0.000 & 0.319 ± 0.072 & 0.243 ± 0.010 & 0.674 ± 0.020 & 0.646 ± 0.041 & 0.438 ± 0.020 & 0.454 ± 0.026 & 0.547 ± 0.049 & 0.491 ± 0.059 \\
MLE & OR & 0.045 ± 0.004 & 0.042 ± 0.000 & 0.215 ± 0.026 & 0.203 ± 0.032 & 0.433 ± 0.017 & 0.395 ± 0.033 & 0.193 ± 0.006 & 0.209 ± 0.037 & 0.307 ± 0.147 & 0.249 ± 0.130 \\
MLE & DR & 0.090 ± 0.002 & 0.079 ± 0.000 & 0.407 ± 0.027 & 0.360 ± 0.007 & 0.425 ± 0.007 & 0.421 ± 0.029 & 0.256 ± 0.001 & 0.286 ± 0.031 & 0.435 ± 0.136 & 0.362 ± 0.122 \\
\midrule
EM & IPW & 0.035 ± 0.002 & 0.040 ± 0.001 & 0.021 ± 0.001 & 0.029 ± 0.015 & 0.303 ± 0.187 & 0.091 ± 0.010 & 0.119 ± 0.011 & 0.105 ± 0.022 & 0.104 ± 0.026 & 0.104 ± 0.051 \\
EM & OR & 0.037 ± 0.003 & 0.042 ± 0.002 & 0.016 ± 0.001 & 0.024 ± 0.012 & 0.269 ± 0.183 & 0.090 ± 0.008 & 0.122 ± 0.012 & 0.103 ± 0.022 & 0.072 ± 0.012 & 0.073 ± 0.024 \\
EM & DR & 0.034 ± 0.004 & 0.037 ± 0.001 & 0.014 ± 0.001 & 0.027 ± 0.020 & 0.264 ± 0.191 & 0.092 ± 0.005 & 0.111 ± 0.019 & 0.097 ± 0.026 & 0.077 ± 0.016 & 0.073 ± 0.028 \\
\midrule
SimPro & IPW & 0.070 ± 0.011 & 0.058 ± 0.000 & 0.046 ± 0.001 & 0.049 ± 0.005 & 0.254 ± 0.074 & 0.223 ± 0.098 & 0.097 ± 0.025 & 0.067 ± 0.002 & 0.105 ± 0.066 & 0.110 ± 0.079 \\
SimPro & OR & 0.071 ± 0.012 & 0.058 ± 0.000 & 0.045 ± 0.001 & 0.049 ± 0.006 & 0.040 ± 0.003 & 0.059 ± 0.017 & 0.074 ± 0.006 & 0.075 ± 0.002 & 0.033 ± 0.003 & 0.033 ± 0.003 \\
SimPro & DR & 0.017 ± 0.004 & 0.026 ± 0.001 & 0.019 ± 0.002 & 0.018 ± 0.003 & 0.039 ± 0.003 & 0.058 ± 0.025 & 0.091 ± 0.007 & 0.031 ± 0.001 & 0.015 ± 0.003 & 0.019 ± 0.007 \\
\bottomrule
\end{tabular}
}
\end{table*}
% 

\begin{table*}[t]
\centering
\caption{Total Variation Distance on CIFAR-100-LT ($N_l = 50$, $M_l = 400$) with different class imbalance ratios $\gamma_l$ and $\gamma_u$ under five different unlabeled class distributions.}
\label{tab:cifar100-tv}
\resizebox{\textwidth}{!}{
\begin{tabular}{lccccccccccc}
\toprule
& & \multicolumn{2}{c}{consistent} & \multicolumn{2}{c}{uniform} & \multicolumn{2}{c}{reversed} & \multicolumn{2}{c}{middle} & \multicolumn{2}{c}{head-tail} \\
\cmidrule(lr){3-4} \cmidrule(lr){5-6} \cmidrule(lr){7-8} \cmidrule(lr){9-10} \cmidrule(lr){11-12}
& & $\gamma_l = 20$ & $\gamma_l = 10$ & $\gamma_l = 20$ & $\gamma_l = 10$ & $\gamma_l = 20$ & $\gamma_l = 10$ & $\gamma_l = 20$ & $\gamma_l = 10$ & $\gamma_l = 20$ & $\gamma_l = 10$ \\
Model & Estimator & $\gamma_u = 20$ & $\gamma_u = 10$ & $\gamma_u = 1$ & $\gamma_u = 1$ & $\gamma_u = 1/20$ & $\gamma_u = 1/10$ & $\gamma_u = 20$ & $\gamma_u = 10$ & $\gamma_u = 20$ & $\gamma_u = 10$ \\
\midrule
Supervised & MLLS & 0.707 ± 0.016 & 0.313 ± 0.100 & 0.445 ± 0.172 & 0.309 ± 0.119 & 0.383 ± 0.075 & 0.397 ± 0.006 & 0.570 ± 0.001 & 0.373 ± 0.107 & 0.543 ± 0.009 & 0.231 ± 0.057 \\
Supervised & RLLS & 0.520 ± 0.007 & 0.133 ± 0.003 & 0.337 ± 0.125 & 0.253 ± 0.082 & 0.424 ± 0.060 & 0.463 ± 0.003 & 0.454 ± 0.021 & 0.306 ± 0.074 & 0.460 ± 0.028 & 0.241 ± 0.040 \\
\midrule
MLE & IPW & 0.075 ± 0.000 & 0.071 ± 0.001 & 0.229 ± 0.001 & 0.167 ± 0.002 & 0.565 ± 0.005 & 0.443 ± 0.007 & 0.415 ± 0.000 & 0.311 ± 0.005 & 0.343 ± 0.000 & 0.280 ± 0.001 \\
MLE & OR & 0.065 ± 0.002 & 0.061 ± 0.001 & 0.200 ± 0.007 & 0.143 ± 0.001 & 0.526 ± 0.011 & 0.399 ± 0.023 & 0.360 ± 0.003 & 0.256 ± 0.012 & 0.328 ± 0.003 & 0.266 ± 0.005 \\
MLE & DR & 0.149 ± 0.019 & 0.145 ± 0.010 & 0.243 ± 0.004 & 0.214 ± 0.019 & 0.568 ± 0.005 & 0.464 ± 0.014 & 0.403 ± 0.014 & 0.309 ± 0.012 & 0.365 ± 0.007 & 0.320 ± 0.004 \\
\midrule
EM & IPW & 0.097 ± 0.008 & 0.092 ± 0.004 & 0.239 ± 0.007 & 0.179 ± 0.003 & 0.478 ± 0.012 & 0.329 ± 0.020 & 0.262 ± 0.016 & 0.202 ± 0.003 & 0.312 ± 0.002 & 0.227 ± 0.001 \\
EM & OR & 0.121 ± 0.007 & 0.108 ± 0.005 & 0.261 ± 0.007 & 0.189 ± 0.004 & 0.489 ± 0.013 & 0.335 ± 0.020 & 0.274 ± 0.016 & 0.211 ± 0.004 & 0.336 ± 0.003 & 0.235 ± 0.001 \\
EM & DR & 0.125 ± 0.005 & 0.111 ± 0.004 & 0.269 ± 0.007 & 0.194 ± 0.005 & 0.497 ± 0.010 & 0.336 ± 0.024 & 0.281 ± 0.019 & 0.219 ± 0.008 & 0.336 ± 0.007 & 0.233 ± 0.004 \\
\midrule
SimPro & IPW & 0.125 ± 0.001 & 0.100 ± 0.005 & 0.166 ± 0.007 & 0.141 ± 0.009 & 0.353 ± 0.023 & 0.261 ± 0.008 & 0.202 ± 0.003 & 0.158 ± 0.005 & 0.277 ± 0.009 & 0.197 ± 0.003 \\
SimPro & OR & 0.133 ± 0.005 & 0.100 ± 0.004 & 0.160 ± 0.007 & 0.138 ± 0.010 & 0.322 ± 0.014 & 0.253 ± 0.008 & 0.202 ± 0.003 & 0.156 ± 0.005 & 0.269 ± 0.006 & 0.191 ± 0.004 \\
SimPro & DR & 0.122 ± 0.003 & 0.106 ± 0.006 & 0.188 ± 0.009 & 0.149 ± 0.006 & 0.343 ± 0.023 & 0.257 ± 0.007 & 0.219 ± 0.010 & 0.172 ± 0.002 & 0.279 ± 0.007 & 0.198 ± 0.004 \\
\bottomrule
\end{tabular}
}
\end{table*}







\end{document}
