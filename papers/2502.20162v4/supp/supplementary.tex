% CVPR 2025 Paper Template; see https://github.com/cvpr-org/author-kit

\documentclass[10pt,onecolumn,letterpaper]{article}

%%%%%%%%% PAPER TYPE  - PLEASE UPDATE FOR FINAL VERSION
%\usepackage{cvpr}              % To produce the CAMERA-READY version
%\usepackage[review]{cvpr}      % To produce the REVIEW version
 \usepackage[pagenumbers]{cvpr} % To force page numbers, e.g. for an arXiv version
%%%% My packages %%%%
\usepackage{amsmath}
\usepackage{subcaption}
\usepackage{multicol}
\usepackage{graphicx}
\usepackage{xcolor}
\usepackage{algpseudocode}
\usepackage{algorithm}
\usepackage{amsmath}
\usepackage{bm}
\graphicspath{{figures/}}

\usepackage[switch]{lineno}


\usepackage{tikz}
\usetikzlibrary{bayesnet}

% Import additional packages in the preamble file, before hyperref
%
% --- inline annotations
%
\newcommand{\red}[1]{{\color{red}#1}}
\newcommand{\todo}[1]{{\color{red}#1}}
\newcommand{\TODO}[1]{\textbf{\color{red}[TODO: #1]}}
% --- disable by uncommenting  
% \renewcommand{\TODO}[1]{}
% \renewcommand{\todo}[1]{#1}



\newcommand{\VLM}{LVLM\xspace} 
\newcommand{\ours}{PeKit\xspace}
\newcommand{\yollava}{Yo’LLaVA\xspace}

\newcommand{\thisismy}{This-Is-My-Img\xspace}
\newcommand{\myparagraph}[1]{\noindent\textbf{#1}}
\newcommand{\vdoro}[1]{{\color[rgb]{0.4, 0.18, 0.78} {[V] #1}}}
% --- disable by uncommenting  
% \renewcommand{\TODO}[1]{}
% \renewcommand{\todo}[1]{#1}
\usepackage{slashbox}
% Vectors
\newcommand{\bB}{\mathcal{B}}
\newcommand{\bw}{\mathbf{w}}
\newcommand{\bs}{\mathbf{s}}
\newcommand{\bo}{\mathbf{o}}
\newcommand{\bn}{\mathbf{n}}
\newcommand{\bc}{\mathbf{c}}
\newcommand{\bp}{\mathbf{p}}
\newcommand{\bS}{\mathbf{S}}
\newcommand{\bk}{\mathbf{k}}
\newcommand{\bmu}{\boldsymbol{\mu}}
\newcommand{\bx}{\mathbf{x}}
\newcommand{\bg}{\mathbf{g}}
\newcommand{\be}{\mathbf{e}}
\newcommand{\bX}{\mathbf{X}}
\newcommand{\by}{\mathbf{y}}
\newcommand{\bv}{\mathbf{v}}
\newcommand{\bz}{\mathbf{z}}
\newcommand{\bq}{\mathbf{q}}
\newcommand{\bff}{\mathbf{f}}
\newcommand{\bu}{\mathbf{u}}
\newcommand{\bh}{\mathbf{h}}
\newcommand{\bb}{\mathbf{b}}

\newcommand{\rone}{\textcolor{green}{R1}}
\newcommand{\rtwo}{\textcolor{orange}{R2}}
\newcommand{\rthree}{\textcolor{red}{R3}}
\usepackage{amsmath}
%\usepackage{arydshln}
\DeclareMathOperator{\similarity}{sim}
\DeclareMathOperator{\AvgPool}{AvgPool}

\newcommand{\argmax}{\mathop{\mathrm{argmax}}}     



% It is strongly recommended to use hyperref, especially for the review version.
% hyperref with option pagebackref eases the reviewers' job.
% Please disable hyperref *only* if you encounter grave issues, 
% e.g. with the file validation for the camera-ready version.
%
% If you comment hyperref and then uncomment it, you should delete *.aux before re-running LaTeX.
% (Or just hit 'q' on the first LaTeX run, let it finish, and you should be clear).
\definecolor{cvprblue}{rgb}{0.21,0.49,0.74}
\usepackage[pagebackref,breaklinks,colorlinks,citecolor=cvprblue]{hyperref}

%% New commands
\newcommand{\card}[1]{\lvert\mathcal{#1}\rvert}
\def\thesection{\Alph{section}}
%%%%%%%%% PAPER ID  - PLEASE UPDATE
\def\paperID{15137} % *** Enter the Paper ID here
\def\confName{CVPR}
\def\confYear{2025}

%%%%%%%%% TITLE - PLEASE UPDATE
\title{Gradient-Guided Annealing for Domain Generalization}

%%%%%%%%% AUTHORS - PLEASE UPDATE
\author{Aristotelis Ballas\\
	Dpt of Informatics and Telematics\\
	Harokopio University of Athens\\
	Omirou 9, Tavros, Athens, Greece\\
	{\tt\small aballas@hua.gr}
	% For a paper whose authors are all at the same institution,
	% omit the following lines up until the closing ``}''.
% Additional authors and addresses can be added with ``\and'',
% just like the second author.
% To save space, use either the email address or home page, not both
\and
Christos Diou\\
Dpt of Informatics and Telematics\\
Harokopio Univesity of Athens\\
Omirou 9, Tavros, Athens, Greece\\
{\tt\small cdiou@hua.gr}
}

\begin{document}
%\maketitle

\clearpage
\setcounter{page}{1}
\maketitlesupplementary

The following materials are provided in this supplementary file:
\begin{itemize}
	\item An extended literature review discussion, helpful for navigating the Domain Generalization literature under the scope of computer vision.
	\item A computational analysis regarding the application of GGA.
	\item Detailed results for each dataset domain and algorithm, presented in Table 2 of the main text, along with extra experiments on the ColoredMNIST and RotatedMNIST datasets.
\end{itemize}

\section{Extended Literature Review}
\label{sec:extended-lit-review}

Domain Generalization (DG) \cite{wang2022generalizing} is arguably one of the 
most difficult and fundamental problems of Machine Learning (ML) today. 
Unsurprisingly, a vast number of researchers have poured effort into advancing the field, where findings have been applied to various areas, such 
as Natural Language Processing \cite{hupkes2023taxonomy}, Reinforcement Learning \cite{li2018learning}, Healthcare and Medicine \cite{9298838, 10233054}, Time-Series forecasting \cite{du2021adarnn}, Fault Diagnosis 
\cite{9174912} and, of course, Computer Vision \cite{wang2022generalizing}. 
Even though not covering the entire field of DG, this section aims to present 
a taxonomy of the general DG methodologies developed in CV, for producing 
robust models that can generalize to previously unseen data, and attempts to 
assist potential readers navigate the past literature, while also 
categorizing our proposed method among its predecessors. 
Domain Generalization methods can be categorized into three major groups, depending on their operation during the process of model training, namely: 
(a) Data Manipulation, (b) Representation Learning and, (c) Learning 
Algorithm. Furthermore, as mentioned in the main text, DG algorithms can 
either leverage domain labels during training (multi-source), or completely
disregard the knowledge of existing domain shifts in their training data and 
handle them as a single distribution (single-source). 
%An illustration of the above taxonomy is presented in Fig. \ref{fig:taxonomy}. 

\textbf{Data Manipulation}. As its name suggests, methods 
included in this group focus on either perturbing existing samples (\textit{data augmentation}) or creating novel ones (\textit{data generation}), 
in order to regularize the training of machine learning models, avoid 
overfitting and improve their generalizability. The basic idea in data 
manipulation methodologies is to simulate domain shift by creating diverse 
data samples, which can in turn mimic the entirety of distributions present 
in the input space. Regarding data augmentation, most popular techniques 
include traditional image transformations, such as random flip, rotation and 
color distortion. Even though these augmentations can be randomly applied 
during training, without needing domain labels, it has been shown that their 
selection significantly affects model performance. For example, the 
authors of \cite{volpi2019addressing} define novel augmentation rules that 
push the perturbed images to diverge as much as possible from the original 
ones. Additionally, image augmentations prove effective towards overcoming 
domain shifts in medical image classification \cite{otalora2019staining, zhang2020generalizing}, where transformations can replicate shifts caused by 
the use of different devices. On the other hand, multiple data augmentation 
methods were also inspired by adversarial attacks and use adversarial gradients to distort the input images \cite{volpi2018generalizing, qiao2020learning}, or use randomly initialized convolutional networks for transforming samples \cite{choi2023progressive}. These techniques act as regularizers during model training, allowing them to learn generalizable image representations. The generation of novel data domains is also a well 
researched area in the data manipulation group. In addition to using domain gradients for synthesizing novel domains \cite{shankar2018generalizing}, 
several methods took advantage of style transfer \cite{huang2017arbitrary} 
and either map the styles of images to that existing source domains \cite{borlino2021rethinking} or create novel styles \cite{yue2019domain}. On a similar note, mixing the styles of training images by conventional methods  \cite{xu2020interdomain_mixup_aaai, zhou2021mixstyle} or with the generative models \cite{wang2024multi} also proves beneficial.



\textbf{Representation Learning}. This group of methods is arguably the most 
prominent in DG and has been the central focus of ML \cite{6472238}. 
Following the formulation in the main text, given a labeling function $h$ 
that maps input observations $\bm{x}$ to their labels $y$, we can decompose it into $h = f \circ g$, where $g$ is a parametric function that learns 
representations of $x$ and $f$ is the classifier function. The goal of 
representation learning can be summarized as follows:

\begin{equation}
	\min_{f, g} \mathbb{E}_{x,y} \ell(f(g(\bm{x}; \bm{\theta})), y) +\lambda\ell_\text{reg}
\end{equation}
where $\ell$ the loss function to be minimized and $\ell_\text{reg}$ a 
regularizer. Methods included in this group, focus on learning a robust and generalizable representation learning function $g$. The algorithms included 
in this group can be further categorized into three sub-groups. \textit{Feature disentanglement} \cite{Zhang_2022_CVPR} methods intend to extract disentangled feature vectors from samples, where each dimension can be linked to a subset of data generating factors. The main idea is to produce a model that extracts a representation that can be further decomposed into domain-specific, domain-invariant, and class-specific features. To that end, the authors of \cite{piratla2020efficient} present CSD, which jointly learns a domain-invariant and domain-specific component in the final embedding and enables the extraction of disentangled representations, whereas the authors of \cite{chattopadhyay2020learning} propose learning domain specific 
masks during training to improve model robustness. Generative models have also been proposed in the disentangled representation learning literature for DG, with variational autoencoders (VAEs) and GANs \cite{chen2016infogan} being utilized for learning distinct latent subspaces for class- and domain-specific features \cite{ilse2020diva}. Another promising category of 
methods aiming to produce disentangled representations is that of 
Causality-Inspired algorithms. In causal representation learning, a domain 
shift can be thought of as an intervention, subsequentially leading the development of models that aim to uncover the true causal data generating factors. Naturally, the prediction of a model should not be affected by interventions on spuriously correlated but irrelevant features, such as the background, color or style of the image. Under this causal consideration, the authors of \cite{mahajan2021domain} propose a structural causal model in 
order to model within-class variations and leverage the fact that inputs 
across domains should have the same representation, given that they derive 
from the same object. Similar to disentangled representations, there have 
been proposed methods in the literature that focus on completely disregarding 
domain-relevant from the final feature vectors, deriving solely 
domain-invariant representations. Based on the initial findings of \cite{ben2006analysis}, numerous works have presented algorithms that aim to minimize the representation differences across multiple source domains within a specific feature space, ensuring they become domain invariant, ultimately enabling the trained model to effectively generalize to previously unseen domains. In one of the most notable previous 
works in this category, Arjovsky et al. \cite{arjovsky2019irm} enforce the 
optimal classifier on top of the representation space to be the same across domains and simultaneously minimize the loss across distributions. The above 
idea of Invariant Risk Minimization (IRM) has been extended to several other
works. For example, the authors of \cite{krueger2021out} propose minimizing 
the variance of source-domain risks, by minimizing their extrapolated risk, 
while the authors of \cite{zhang2020arm} propose adapting to domain shift and 
producing invariant representations. Finally, an alternative route towards
learning generalized representations is via regularization strategies. 
The most representative group of methods in this category is \textit{Gradient-Based operations}, which utilize gradient information during 
model training. In \cite{huang2020rsc}, the authors propose learning robust
representations by discarding the most dominant gradients in each training 
iteration under the assumption that they are correlated with domain-specific
features present in the source data. Another popular strategy is to seek for 
flat minima \cite{foret2020sharpness, cha2021swad} in the loss landscape of 
neural networks during training, assuming that models that converge to flat minima exhibit increased generalization capabilities \cite{zhuang2022surrogate, wang2023sharpness}. 
What's more, Shi et al. \cite{shi2021gradient} hypothesized that gradients 
among domains should match and proposed an approximation of a loss inducing the maximization of the gradient inner product during training. Our method (GGA) can be categorized in this group of gradient operations, as it considers
the similarity of domain gradients in the early iterations of model training
and seeks for sets in the parameter space with increased gradient alignment, before continuing the optimization procedure. 

\textbf{Learning Algorithm}. In addition to manipulating the input space or 
feature extractor, DG methods were also researched under the scope of 
alternative ML learning paradigms, such as \textit{ensemble}, \textit{meta}, \textit{domain-adversarial}, \textit{self-supervised} and 
\textit{reinforcement} learning. In this section we present the most exemplary
works in each category. \textit{Ensemble-Learning} in DG initially combined
several copies of the same network, each of which is trained on a specific 
domain \cite{zhou2021domain, ding2017deep}. Alternatively, instead of using several networks, \cite{yosinski2014transferable} proposed sharing shallow 
layers among CNNs. During inference, the final prediction is produced by 
either simple \cite{zhou2021domain} or weighted averaging 
\cite{wang2020dofe}. In \textit{Meta-Learning} for DG, Li et al. \cite{li2018learning} propose MLDG and split the source domains into 
meta-train and meta-test splits to mimic the effects of domain shift during 
training. Similarly, \cite{balaji2018metareg} proposes learning a meta regularizer for the classifier, while MAML \cite{finn2017model} was proposed
for improved parameter initialization. Another approach is that of \textit{Adversarial Learning} (AL). In the context of DG, the aim of 
adversarial learning is to train a classifier to distinguish between source domains \cite{matsuura2020domain} and ultimately learn domain-agnostic 
features from the samples that can be generalized to unseen data 
\cite{li2018deep}. Other learning paradigms such as \textit{Self-Supervised} 
learning have also been explored in DG, which leverages unlabeled data 
samples to derive generalized representations. Notably, the authors of 
\cite{carlucci2019domain} introduce a self-supervised jigsaw-solving puzzle 
task to push the model to learn robust representations. Furthermore, 
contrastive learning has also been shown to improve model performance. 
Specifically, SelfReg \cite{kim2021selfreg} utilizes self-supervised contrastive losses to bring latent representations of same-class samples closer. Similarly, the authors of \cite{ballas2024multi} introduce 
a contrastive loss for representations extracted from intermediate 
layers of the network. Finally, \textit{Reinforcement learning} has also been 
applied in the context of DG. Indicatively, previous works have explored randomizing the environments of an RL agent for transferring them to 
real-world scenarios \cite{tobin2017domain,lee2019network}, whereas \cite{laskin2020curl} researches the combination of RL with contrastive 
learning.

\section{Computational Analyis}
\label{computation}

\subsection{Experiment Infrastructure}
\label{experiment-infra}

Each and every experiment is conducted on a cluster containing $4\times40$GB NVIDIA A$100$ GPU cards, split into $8$ $20$GB virtual MIG devices and $1\times24$GB NVIDIA RTX A$5000$ GPU card, via a SLURM workload manager.

\subsection{Complexity Analysis}
\label{sec:complexity}

Each GGA training iteration includes computing model gradients $S\cdot n_a$ times for each training step, where $S$ is the number of source domains and $n_a$ is the number of search steps. These GGA training iterations only take place in the early stages of training and for a small percentage of the total training iterations (2\% in our experiments). The rest of the iterations are vanilla ERM. Furthermore, inference is not affected by the application of GGA 
during training.

 
\section{Full Experimental Results}
\label{sec:full-results}
In this section, we show detailed results of Table 2 in the main text.
The results marked by $\dagger, \ddagger$ are copied from Gulrajani and
Lopez-Paz \cite{gulrajani2020domainbed} and Wang \etal 
\cite{wang2023sharpness}, respectively. Standard errors for the baseline methods are reported from three trials, if available from past literature.
In {\color{green} green} and {\color{red} red}, we highlight the performance boost and decrease of applying \textbf{GGA} on top of each algorithm respectively, averaged over three trials. In addition, we also present 
detailed results for the DomainNet benchmark, without however including 
results for the combination of GGA with the baseline algorithms, due to 
computational restrictions. We also include experiments for the ColoredMNIST and RotatedMNIST datasets, where we reproduced the results for all baselines and report the average results over 5 runs. The below tables are better read in color.

When applying GGA to existing methods, the only difference regarding the baseline algorithm training is that ``Algorithm 1'' (i.e. GGA)  is applied instead of the method’s update rules for the duration of the annealing process (training steps $A_s$ to $A_e$). The total epochs and method hyperparameters remain the same throughout training.


%\subsection{PACS}
\begin{table*}
\centering
\small
\renewcommand{\arraystretch}{1.1}
\caption{\small{Out-of-domain accuracies (\%) on {PACS}.}}
\begin{tabular}{lllll|c}
\toprule
\textbf{Algorithm} & \textbf{A} & \textbf{C} & \textbf{P} & \textbf{S} & \textbf{Avg} \\
\midrule

IRM$^\dagger$ \cite{arjovsky2019irm}            & 84.8\scriptsize{$\pm1.3$}      \normalsize{({\color{red}$-3.6$})}& 
76.4\scriptsize{$\pm1.1$}      \normalsize{({\color{green}$+0.7$})} & 
96.7\scriptsize{$\pm0.6$}      \normalsize{({\color{red}$-0.3$})}& 
76.1\scriptsize{$\pm1.0$}      \normalsize{({\color{red}$-2.8$})} & 
%		33.9\scriptsize{$\pm2.8$}          & 
83.5 \normalsize{({\color{red}$-1.5$})}          \\

ERM$^\ddagger$ \cite{vapnik1998statistical} & 
85.7 \scriptsize$\pm0.6$ & 
77.1 \scriptsize$\pm0.8$ & 
97.4 \scriptsize$\pm0.4$ & 
76.6 \scriptsize$\pm0.7$ & 
84.2 \\


GroupDRO$^\ddagger$ \cite{Sagawa2020GroupDRO}    & 83.5\scriptsize{$\pm0.9$}    \normalsize{({\color{green}$+3.6$})}   & 
79.1\scriptsize{$\pm0.6$}    \normalsize{({\color{green}$+2.0$})}   & 
96.7\scriptsize{$\pm0.7$}    \normalsize{({\color{green}$+0.5$})}   & 
78.3\scriptsize{$\pm2.0$}    \normalsize{({\color{red}$-0.5$})}   & 
%		33.3\scriptsize{$\pm0.2$}          & 
84.4 \normalsize{({\color{green}$+1.4$})}           \\


MTL$^\ddagger$\cite{blanchard2021mtl_marginal_transfer_learning}   & 87.5\scriptsize{$\pm0.8$}     \normalsize{({\color{green}$+1.3$})} & 
77.1\scriptsize{$\pm0.5$}     \normalsize{({\color{green}$+2.5$})} & 
96.4\scriptsize{$\pm0.8$}     \normalsize{({\color{red}$-0.8$})} & 
77.3\scriptsize{$\pm1.8$}     \normalsize{({\color{green}$+0.7$})} &  
%		40.6\scriptsize{$\pm0.1$}          & 
84.6 \normalsize{({\color{green}$+0.9$})}          \\

Mixup$^\dagger$ \cite{xu2020interdomain_mixup_aaai}             & 86.1\scriptsize{$\pm0.5$}   \normalsize{({\color{green}$+1.2$})} & 
78.9\scriptsize{$\pm0.8$}   \normalsize{({\color{red}$-0.1$})} & 
97.6\scriptsize{$\pm0.1$}   \normalsize{({\color{red}$-0.4$})} & 
75.8\scriptsize{$\pm1.8$}   \normalsize{({\color{green}$+4.1$})} & 
%		39.2\scriptsize{$\pm0.1$}            & 
84.6   \normalsize{({\color{green}$+1.2$})}        \\


MMD$^\ddagger$ \cite{li2018mmd}                  & 86.1\scriptsize{$\pm1.4$}     \normalsize{({\color{green}$+0.8$})} & 
79.4\scriptsize{$\pm0.9$}     \normalsize{({\color{green}$+0.5$})} & 
96.6\scriptsize{$\pm0.2$}     \normalsize{({\color{red}$-0.6$})} & 76.5\scriptsize{$\pm0.5$}     \normalsize{({\color{green}$+3.7$})} & 
%		23.4\scriptsize{$\pm9.5$}          & 
84.7 \normalsize{({\color{green}$+0.8$})}         \\


VREx$^\ddagger$ \cite{krueger2020vrex} & 
86.0\scriptsize{$\pm1.6$}     \normalsize{({\color{green}$+0.2$})} & 79.1\scriptsize{$\pm0.6$}     \normalsize{({\color{red}$-0.9$})} & 96.9\scriptsize{$\pm0.5$}     \normalsize{({\color{red}$-0.1$})} & 77.7\scriptsize{$\pm1.7$}     \normalsize{({\color{green}$+2.3$})} & 
%		33.6\scriptsize{$\pm2.9$}          & 
84.9 \normalsize{({\color{green}$+0.5$})}         \\

MLDG$^\dagger$ \cite{li2018learning}                & 85.5\scriptsize{$\pm1.4$}   \normalsize{({\color{green}$+1.1$})} &
80.1\scriptsize{$\pm1.7$}   \normalsize{({\color{green}$+0.9$})} & 97.4\scriptsize{$\pm0.3$}   \normalsize{({\color{green}$-0.9$})} &
76.6\scriptsize{$\pm1.1$}   \normalsize{({\color{green}$+1.6$})} & 
%		41.2\scriptsize{$\pm0.1$}            & 
84.9 \normalsize{({\color{green}$+0.7$})}      \\

ARM$^\ddagger$ \cite{zhang2020arm}               & 86.8\scriptsize{$\pm0.6$}     \normalsize{({\color{red}$-1.9$})} & 
76.8\scriptsize{$\pm0.5$}     \normalsize{({\color{green}$+4.7$})} & 
97.4\scriptsize{$\pm0.3$}     \normalsize{({\color{red}$-1.1$})} & 
79.3\scriptsize{$\pm1.2$}     \normalsize{({\color{green}$+0.4$})} & 
%		35.5\scriptsize{$\pm0.2$}          & 
85.1 \normalsize{({\color{green}$+0.5$})}           \\


Mixstyle$^\ddagger$ \cite{zhou2021mixstyle}   & 
86.8\scriptsize{$\pm0.5$} \normalsize{({\color{green}$+1.1$})}            & 79.0\scriptsize{$\pm1.4$} \normalsize{({\color{red}$-0.3$})}          & 96.6\scriptsize{$\pm0.1$} \normalsize{({\color{red}$-1.1$})}          & 78.5\scriptsize{$\pm2.3$} \normalsize{({\color{green}$+1.4$})}          & 
%		34.0\scriptsize{$\pm0.1$}          & 
85.2 \normalsize{({\color{green}$+0.3$})}           \\



%AND-mask \cite{shahtalebi2021sand} & 
%84.4\scriptsize{$\pm0.9$}  
%\normalsize{({\color{green}$+0.1$})}& 
%78.1\scriptsize{$\pm0.9$}  
%\normalsize{({\color{green}$+0.3$})}& 
%65.6\scriptsize{$\pm0.4$}  
%\normalsize{({\color{green}$+0.4$})} & 
%44.6\scriptsize{$\pm0.3$} 
%\normalsize{({\color{green}$+0.5$})}
%%        & 37.2\scriptsize{$\pm0.6$}  
%& 68.2 \normalsize{({\color{green}$+0.3$})}\\

CORAL$^\dagger$ \cite{sun2016coral}             & 88.3\scriptsize{$\pm0.2$}      \normalsize{({\color{red}$-0.6$})}& 
80.0\scriptsize{$\pm0.5$}      \normalsize{({\color{green}$+1.3$})} & 
97.5\scriptsize{$\pm0.3$}      \normalsize{({\color{green}$+0.6$})} & 
78.8\scriptsize{$\pm1.3$}      \normalsize{({\color{green}$+1.5$})}& 
%		41.5\scriptsize{$\pm0.1$}          & 
86.2    \normalsize{({\color{green}$+0.7$})}       \\

SagNet$^\dagger$ \cite{nam2019sagnet}           &             87.4\scriptsize{$\pm0.2$}    \normalsize{({\color{red}$-2.3$})} & 
80.7\scriptsize{$\pm0.5$}    \normalsize{({\color{green}$+1.0$})} & 97.1\scriptsize{$\pm0.1$}    \normalsize{({\color{red}$-0.9$})}& 
80.0\scriptsize{$\pm1.0$}    \normalsize{({\color{red}$-1.8$})}& 
%		40.3\scriptsize{$\pm0.1$}          & 
86.3 \normalsize{({\color{red}$-1.0$})}         \\

\midrule

RSC$^\dagger$ \cite{huang2020rsc}               & 85.4\scriptsize{$\pm0.9$}      \normalsize{({\color{red}$-1.8$})}& 
79.7\scriptsize{$\pm0.5$}      \normalsize{({\color{green}$+2.9$})}& 
97.6\scriptsize{$\pm0.9$}      \normalsize{({\color{red}$-1.0$})}& 
78.2\scriptsize{$\pm1.0$}      \normalsize{({\color{green}$+0.3$})}& 
%		38.9\scriptsize{$\pm0.5$}          & 
85.2 \normalsize{({\color{green}$+0.1$})}   \\

%Fish $^\ddagger$ \cite{shi2021gradient}                     & 85.5\scriptsize{$\pm0.3$}     \normalsize{({\color{green}$+0.1$})} & 77.8\scriptsize{$\pm0.3$}     \normalsize{({\color{green}$+0.9$})} & 
%68.6\scriptsize{$\pm0.4$}     \normalsize{({\color{red}$-0.6$})} & 
%45.1\scriptsize{$\pm1.3$}     \normalsize{({\color{green}$+0.8$})} & 
%%		42.7\scriptsize{$\pm0.2$}            & 
%69.3 \normalsize{({\color{green}$+0.3$})}             \\ 

SAM $^\ddagger$\cite{foret2020sharpness}  & 
85.6\scriptsize$\pm2.1$       \normalsize{({\color{green}$+1.1$})} & 
80.9\scriptsize$\pm1.2$       \normalsize{({\color{red}$-0.8$})} & 
97.0\scriptsize$\pm0.4$       \normalsize{({\color{red}$-0.2$})} & 
79.6\scriptsize$\pm1.6$       \normalsize{({\color{green}$+1.2$})} & 
%		44.3\scriptsize$\pm0.0$              & 
85.8 \normalsize{({\color{green}$+0.6$})}   \\

GSAM $^\ddagger$ \cite{zhuang2022surrogate}       &
86.9\scriptsize$\pm0.1$      \normalsize{({\color{red}$-0.4$})} & 
80.4\scriptsize$\pm0.2$      \normalsize{({\color{green}$+0.7$})} & 
97.5\scriptsize$\pm0.0$      \normalsize{({\color{red}$-1.1$})}& 
78.7\scriptsize$\pm0.8$      \normalsize{({\color{green}$+2.5$})}& 
%		44.6\scriptsize$\pm0.2$             & 
85.9  \normalsize{({\color{green}$+0.4$})}  \\

SAGM $^\ddagger$ \cite{wang2023sharpness}       & 
87.4\scriptsize{$\pm0.2$}    \normalsize{({\color{green}$+1.2$})}& 
80.2\scriptsize{$\pm0.3$}    \normalsize{({\color{green}$+1.1$})}& 
98.0\scriptsize{$\pm0.2$}    \normalsize{({\color{red}$-1.0$})}& 
80.8\scriptsize{$\pm0.6$}    \normalsize{({\color{red}$-0.4$})}& 
%		45.0\scriptsize{$\pm0.2$}           & 
86.6 \normalsize{({\color{green}$+0.2$})}   \\

%		\midrule
\midrule
\textbf{GGA} (ours)                 & 
88.8\scriptsize{$\pm0.2$}           & 
80.1\scriptsize{$\pm0.3$}           & 
97.3\scriptsize{$\pm0.2$}           &  
81.2\scriptsize{$\pm0.5$}    &  
87.3  \\
%		\midrule

\bottomrule
\end{tabular}
\end{table*}

%\subsection{VLCS}
\begin{table*}[]
\centering
\small
\renewcommand{\arraystretch}{1.1}
\caption{\small{Out-of-domain accuracies (\%) on VLCS.}}
\begin{tabular}{lllll|c}
\toprule
\textbf{Algorithm} & \textbf{C} & \textbf{L} & \textbf{S} & \textbf{V} & \textbf{Avg} \\
\midrule

GroupDRO$^\ddagger$ \cite{Sagawa2020GroupDRO}    & 97.3\scriptsize{$\pm0.3$}    \normalsize{({\color{green}$+1.4$})}   & 
63.4\scriptsize{$\pm0.9$}    \normalsize{({\color{green}$+1.7$})}   & 
69.5\scriptsize{$\pm0.8$}    \normalsize{({\color{green}$+2.0$})}   & 
76.7\scriptsize{$\pm0.7$}    \normalsize{({\color{red}$-2.9$})}   & 
%		33.3\scriptsize{$\pm0.2$}          & 
76.7 \normalsize{({\color{green}$+0.6$})}           \\

MLDG$^\dagger$ \cite{li2018learning}                & 97.4\scriptsize{$\pm0.2$}      \normalsize{({\color{green}$+1.6$})} &
65.2\scriptsize{$\pm0.7$}      \normalsize{({\color{red}$-0.4$})} & 71.0\scriptsize{$\pm1.4$}      \normalsize{({\color{green}$+3.0$})} &
75.3\scriptsize{$\pm1.0$}      \normalsize{({\color{green}$+0.9$})} & 
%		41.2\scriptsize{$\pm0.1$}            & 
77.2 \normalsize{({\color{green}$+1.3$})}      \\

MTL$^\ddagger$\cite{blanchard2021mtl_marginal_transfer_learning}    & 97.8\scriptsize{$\pm0.4$}     \normalsize{({\color{green}$+0.3$})} & 
64.3\scriptsize{$\pm0.3$}     \normalsize{({\color{green}$+1.8$})} & 
71.5\scriptsize{$\pm0.7$}     \normalsize{({\color{green}$+4.1$})} & 
75.3\scriptsize{$\pm1.7$}     \normalsize{({\color{green}$+1.6$})} &  
%		40.6\scriptsize{$\pm0.1$}          & 
77.2 \normalsize{({\color{green}$+2.0$})}          \\

ERM$^\ddagger$ \cite{vapnik1998statistical}  & 
98.0 \scriptsize$\pm0.3$ & 
64.7 \scriptsize$\pm1.2$ & 
71.4 \scriptsize$\pm1.2$ & 
75.2 \scriptsize$\pm1.6$ & 
77.3 \\

Mixup$^\dagger$ \cite{xu2020interdomain_mixup_aaai}             & 98.3\scriptsize{$\pm0.6$}     \normalsize{({\color{green}$+0.8$})} & 
64.8\scriptsize{$\pm1.0$}     \normalsize{({\color{green}$+1.7$})} & 
72.1\scriptsize{$\pm0.5$}     \normalsize{({\color{green}$+1.0$})} & 
74.3\scriptsize{$\pm0.8$}     \normalsize{({\color{green}$+3.6$})} & 
%		39.2\scriptsize{$\pm0.1$}            & 
77.4  \normalsize{({\color{green}$+1.8$})}        \\

MMD$^\ddagger$ \cite{li2018mmd}                  & 97.7\scriptsize{$\pm0.1$}     \normalsize{({\color{red}$-0.9$})} & 
64.0\scriptsize{$\pm1.1$}     \normalsize{({\color{green}$+1.4$})} & 
72.8\scriptsize{$\pm0.2$}     \normalsize{({\color{green}$+1.4$})} & 
75.3\scriptsize{$\pm3.3$}     \normalsize{({\color{green}$+3.5$})} & 
%		23.4\scriptsize{$\pm9.5$}          & 
77.5 \normalsize{({\color{green}$+1.3$})}         \\

ARM$^\ddagger$ \cite{zhang2020arm}               & 98.7\scriptsize{$\pm0.2$}     \normalsize{({\color{red}$-0.2$})}  & 
63.6\scriptsize{$\pm0.7$}     \normalsize{({\color{green}$+2.2$})}  & 
71.3\scriptsize{$\pm1.2$}     \normalsize{({\color{green}$+0.3$})}  & 
76.7\scriptsize{$\pm0.6$}     \normalsize{({\color{green}$+1.4$})}  & 
%		35.5\scriptsize{$\pm0.2$}          & 
77.6 \normalsize{({\color{green}$+0.9$})}           \\

SagNet$^\dagger$ \cite{nam2019sagnet}           &             97.9\scriptsize{$\pm0.4$}     \normalsize{({\color{red}$-0.2$})} & 
64.5\scriptsize{$\pm0.5$}     \normalsize{({\color{green}$+1.7$})} & 71.4\scriptsize{$\pm1.3$}     \normalsize{({\color{green}$+0.8$})}& 
77.5\scriptsize{$\pm0.5$}     \normalsize{({\color{green}$+1.4$})}& 
%		40.3\scriptsize{$\pm0.1$}          & 
77.8 \normalsize{({\color{green}$+0.9$})}         \\

Mixstyle$^\ddagger$ \cite{zhou2021mixstyle}   & 
98.6\scriptsize{$\pm0.3$} \normalsize{({\color{red}$-0.1$})}            & 64.5\scriptsize{$\pm1.1$} \normalsize{({\color{green}$+1.9$})}          & 72.6\scriptsize{$\pm0.5$} \normalsize{({\color{green}$+0.4$})}          & 75.7\scriptsize{$\pm1.7$} \normalsize{({\color{green}$+0.4$})}          & 
%		34.0\scriptsize{$\pm0.1$}          & 
77.9 \normalsize{({\color{green}$+0.6$})}           \\

%AND-mask \cite{shahtalebi2021sand} & 
%84.4\scriptsize{$\pm0.9$}  
%\normalsize{({\color{green}$+0.1$})}& 
%78.1\scriptsize{$\pm0.9$}  
%\normalsize{({\color{green}$+0.3$})}& 
%65.6\scriptsize{$\pm0.4$}  
%\normalsize{({\color{green}$+0.4$})} & 
%44.6\scriptsize{$\pm0.3$} 
%\normalsize{({\color{green}$+0.5$})}
%%        & 37.2\scriptsize{$\pm0.6$}  
%& 68.2 \normalsize{({\color{green}$+0.3$})}\\

VREx$^\ddagger$ \cite{krueger2020vrex}         & 
98.4\scriptsize{$\pm0.3$}      \normalsize{({\color{red}$-0.9$})} & 64.4\scriptsize{$\pm1.4$}      \normalsize{({\color{green}$+1.9$})} & 74.1\scriptsize{$\pm0.4$}      \normalsize{({\color{green}$-1.7$})} & 76.2\scriptsize{$\pm1.3$}      \normalsize{({\color{green}$+1.0$})} & 
%		33.6\scriptsize{$\pm2.9$}          & 
78.3 \normalsize{({\color{green}$+0.1$})}         \\


IRM$^\dagger$ \cite{arjovsky2019irm}            & 98.6\scriptsize{$\pm0.1$}      \normalsize{({\color{red}$-0.3$})}& 
64.9\scriptsize{$\pm0.9$}      \normalsize{({\color{red}$-3.5$})}& 
73.4\scriptsize{$\pm0.6$}      \normalsize{({\color{green}$+1.7$})}& 
77.3\scriptsize{$\pm0.9$}      \normalsize{({\color{red}$-1.5$})} & 
%		33.9\scriptsize{$\pm2.8$}          & 
78.6 \normalsize{({\color{red}$-0.9$})}          \\


CORAL$^\dagger$ \cite{sun2016coral}             & 98.3\scriptsize{$\pm0.3$}      \normalsize{({\color{green}$+0.9$})}& 
66.1\scriptsize{$\pm0.6$}      \normalsize{({\color{green}$+1.8$})} & 
73.4\scriptsize{$\pm0.3$}      \normalsize{({\color{red}$-1.8$})} & 
77.5\scriptsize{$\pm1.0$}      \normalsize{({\color{red}$-2.1$})}& 
%		41.5\scriptsize{$\pm0.1$}          & 
78.8    \normalsize{({\color{red}$-0.4$})}       \\

\midrule

RSC$^\dagger$ \cite{huang2020rsc}               & 
97.9\scriptsize{$\pm0.1$}    \normalsize{({\color{green}$+0.6$})}& 62.5\scriptsize{$\pm0.7$}    \normalsize{({\color{green}$+0.3$})}& 
72.3\scriptsize{$\pm1.2$}    \normalsize{({\color{green}$+0.4$})}& 
75.6\scriptsize{$\pm0.8$}    \normalsize{({\color{red}$-0.8$})}& 
%		38.9\scriptsize{$\pm0.5$}          & 
77.1 \normalsize{({\color{green}$+0.2$})}   \\

%Fish $^\ddagger$ \cite{shi2021gradient}                     & 85.5\scriptsize{$\pm0.3$}          
%\normalsize{({\color{green}$+0.1$})} & 77.8\scriptsize{$\pm0.3$}          
%\normalsize{({\color{green}$+0.9$})} & 
%68.6\scriptsize{$\pm0.4$}            
%\normalsize{({\color{red}$-0.6$})} & 
%45.1\scriptsize{$\pm1.3$}            
%\normalsize{({\color{green}$+0.8$})} & 
%%		42.7\scriptsize{$\pm0.2$}            & 
%69.3 \normalsize{({\color{green}$+0.3$})}             \\ 



GSAM $^\ddagger$ \cite{zhuang2022surrogate}       & 
98.7\scriptsize$\pm0.3$   \normalsize{({\color{green}$+0.5$})} & 
64.9\scriptsize$\pm0.2$   \normalsize{({\color{green}$+0.5$})} & 
74.3\scriptsize$\pm0.0$   \normalsize{({\color{green}$+1.2$})}& 
78.5\scriptsize$\pm0.8$   \normalsize{({\color{green}$+1.8$})}& 
%		44.6\scriptsize$\pm0.2$             & 
79.1  \normalsize{({\color{green}$+1.0$})}  \\

SAM $^\ddagger$\cite{foret2020sharpness}  & 
99.1\scriptsize$\pm0.2$     \normalsize{({\color{red}$-0.2$})} & 
65.0\scriptsize$\pm1.0$     \normalsize{({\color{green}$+1.8$})} & 
73.7\scriptsize$\pm1.0$     \normalsize{({\color{red}$-0.2$})} & 
79.8\scriptsize$\pm0.1$     \normalsize{({\color{green}$+1.5$})} & 
%		44.3\scriptsize$\pm0.0$              & 
79.4 \normalsize{({\color{green}$+0.7$})}   \\

SAGM $^\ddagger$ \cite{wang2023sharpness}       & 
99.0\scriptsize{$\pm0.2$}  \normalsize{({\color{red}$-0.4$})}& 
65.2\scriptsize{$\pm0.4$}  \normalsize{({\color{green}$+0.5$})}& 
75.1\scriptsize{$\pm0.3$}  \normalsize{({\color{red}$-1.1$})}& 
80.7\scriptsize{$\pm0.8$}  \normalsize{({\color{red}$-0.2$})}& 
%		45.0\scriptsize{$\pm0.2$}           & 
80.0 \normalsize{({\color{red}$-0.3$})}   \\

%		\midrule
\midrule
\textbf{GGA} (ours)                 & 
99.1\scriptsize{$\pm0.2$}           & 
67.5\scriptsize{$\pm0.6$}           & 
75.1\scriptsize{$\pm0.3$}           &  
78.0\scriptsize{$\pm0.1$}    &  
79.9  \\
%		\midrule
\bottomrule
\end{tabular}
\end{table*}

%\subsection{OfficeHome}
\begin{table*}[]
\centering
\small
\renewcommand{\arraystretch}{1.1}
\caption{\small{Out-of-domain accuracies (\%) on OfficeHome.}}
\begin{tabular}{lllll|c}
\toprule
\textbf{Algorithm} & \textbf{A} & \textbf{C} & \textbf{P} & \textbf{R} & \textbf{Avg} \\
\midrule
Mixstyle$^\ddagger$ \cite{zhou2021mixstyle}   & 
51.1\scriptsize{$\pm0.3$} \normalsize{({\color{green}$+0.3$})}            & 53.2\scriptsize{$\pm0.4$} \normalsize{({\color{green}$+0.7$})}          & 68.2\scriptsize{$\pm0.7$} \normalsize{({\color{green}$+0.4$})}          & 69.2\scriptsize{$\pm0.6$} \normalsize{({\color{green}$+0.6$})}          & 
%		34.0\scriptsize{$\pm0.1$}          & 
60.4 \normalsize{({\color{green}$+0.5$})}           \\

IRM$^\dagger$ \cite{arjovsky2019irm}            & 58.9\scriptsize{$\pm2.3$}       \normalsize{({\color{red}$-3.5$})}& 
52.2\scriptsize{$\pm1.6$}       \normalsize{({\color{red}$-2.1$})} & 
72.1\scriptsize{$\pm2.9$}       \normalsize{({\color{red}$-1.7$})}& 
74.0\scriptsize{$\pm2.5$}       \normalsize{({\color{red}$-1.0$})} & 
%		33.9\scriptsize{$\pm2.8$}          & 
64.3 \normalsize{({\color{red}$-2.1$})}          \\

ARM$^\ddagger$ \cite{zhang2020arm}               & 58.9\scriptsize{$\pm0.8$}      \normalsize{({\color{green}$+2.7$})} & 
51.0\scriptsize{$\pm0.5$}      \normalsize{({\color{red}$-0.3$})} & 
74.1\scriptsize{$\pm0.1$}      \normalsize{({\color{green}$+2.1$})} & 
75.2\scriptsize{$\pm0.3$}      \normalsize{({\color{green}$+3.4$})} & 
%		35.5\scriptsize{$\pm0.2$}          & 
64.8 \normalsize{({\color{green}$+2.1$})}           \\

GroupDRO$^\ddagger$ \cite{Sagawa2020GroupDRO}    & 60.4\scriptsize{$\pm0.7$}    \normalsize{({\color{green}$+3.8$})}   & 
52.7\scriptsize{$\pm1.0$}    \normalsize{({\color{green}$+1.2$})}   & 
75.0\scriptsize{$\pm0.7$}    \normalsize{({\color{green}$+1.3$})}   & 
76.0\scriptsize{$\pm0.7$}    \normalsize{({\color{green}$+2.2$})}   & 
%		33.3\scriptsize{$\pm0.2$}          & 
66.0 \normalsize{({\color{green}$+2.2$})}           \\

MMD$^\ddagger$ \cite{li2018mmd}                  & 60.4\scriptsize{$\pm0.2$}     \normalsize{({\color{green}$+3.1$})} & 
53.3\scriptsize{$\pm0.3$}     \normalsize{({\color{red}$-0.2$})} & 
74.3\scriptsize{$\pm0.1$}     \normalsize{({\color{green}$+3.1$})} & 
77.4\scriptsize{$\pm0.6$}     \normalsize{({\color{green}$+0.7$})} & 
%		23.4\scriptsize{$\pm9.5$}          & 
66.4 \normalsize{({\color{green}$+1.6$})}         \\

%AND-mask \cite{shahtalebi2021sand} & 
%84.4\scriptsize{$\pm0.9$}  
%\normalsize{({\color{green}$+0.1$})}& 
%78.1\scriptsize{$\pm0.9$}  
%\normalsize{({\color{green}$+0.3$})}& 
%65.6\scriptsize{$\pm0.4$}  
%\normalsize{({\color{green}$+0.4$})} & 
%44.6\scriptsize{$\pm0.3$} 
%\normalsize{({\color{green}$+0.5$})}
%%        & 37.2\scriptsize{$\pm0.6$}  
%& 68.2 \normalsize{({\color{green}$+0.3$})}\\

MTL$^\ddagger$\cite{blanchard2021mtl_marginal_transfer_learning}    & 61.5\scriptsize{$\pm0.7$}       \normalsize{({\color{green}$+0.8$})} & 
52.4\scriptsize{$\pm0.6$}       \normalsize{({\color{red}$-0.3$})} & 
74.9\scriptsize{$\pm0.4$}       \normalsize{({\color{green}$+1.0$})} & 
76.8\scriptsize{$\pm0.4$}       \normalsize{({\color{green}$+1.2$})}&  
%		40.6\scriptsize{$\pm0.1$}          & 
66.4 \normalsize{({\color{green}$+0.4$})}          \\

VREx$^\ddagger$ \cite{krueger2020vrex}         & 
60.7\scriptsize{$\pm0.9$}        \normalsize{({\color{green}$+1.9$})} & 53.0\scriptsize{$\pm0.9$}        \normalsize{({\color{green}$+0.8$})} & 75.3\scriptsize{$\pm0.1$}        \normalsize{({\color{green}$+0.6$})} & 76.6\scriptsize{$\pm0.5$}        \normalsize{({\color{green}$+0.2$})} & 
%		33.6\scriptsize{$\pm2.9$}          & 
66.4 \normalsize{({\color{green}$+0.9$})}         \\

ERM$^\ddagger$ \cite{vapnik1998statistical} & 
63.1 \scriptsize$\pm0.3$ & 
51.9 \scriptsize$\pm0.4$ & 
77.2 \scriptsize$\pm0.5$ & 
78.1 \scriptsize$\pm0.2$ & 67.6 \\



MLDG$^\dagger$ \cite{li2018learning}                & 61.5\scriptsize{$\pm0.9$}       \normalsize{({\color{green}$+2.4$})} &
53.2\scriptsize{$\pm0.6$}       \normalsize{({\color{green}$+0.2$})} & 75.0\scriptsize{$\pm1.2$}       \normalsize{({\color{green}$+1.8$})} &
77.5\scriptsize{$\pm0.4$}       \normalsize{({\color{green}$+0.6$})} & 
%		41.2\scriptsize{$\pm0.1$}            & 
66.8 \normalsize{({\color{green}$+1.2$})}      \\

Mixup$^\dagger$ \cite{xu2020interdomain_mixup_aaai}             & 62.4\scriptsize{$\pm0.8$}       \normalsize{({\color{green}$+1.5$})} & 
54.8\scriptsize{$\pm0.6$}       \normalsize{({\color{red}$-1.7$})} & 
76.9\scriptsize{$\pm0.3$}       \normalsize{({\color{green}$+1.9$})} & 
78.3\scriptsize{$\pm0.2$}       \normalsize{({\color{green}$+0.4$})} & 
%		39.2\scriptsize{$\pm0.1$}            & 
68.1   \normalsize{({\color{green}$+1.2$})}        \\


SagNet$^\dagger$ \cite{nam2019sagnet}           &             63.4\scriptsize{$\pm0.2$}       \normalsize{({\color{green}$+1.0$})} & 
54.8\scriptsize{$\pm0.4$}       \normalsize{({\color{red}$-1.9$})} & 75.8\scriptsize{$\pm0.4$}       \normalsize{({\color{green}$+1.4$})}& 
78.3\scriptsize{$\pm0.3$}       \normalsize{({\color{green}$+0.7$})}& 
%		40.3\scriptsize{$\pm0.1$}          & 
68.1 \normalsize{({\color{green}$+0.3$})}         \\



CORAL$^\dagger$ \cite{sun2016coral}             & 
65.3\scriptsize{$\pm0.3$}       \normalsize{({\color{green}$+0.3$})}& 
54.4\scriptsize{$\pm0.6$}       \normalsize{({\color{green}$+0.2$})} & 
76.5\scriptsize{$\pm0.3$}       \normalsize{({\color{red}$-0.7$})} & 
78.4\scriptsize{$\pm1.0$}       \normalsize{({\color{green}$+1.0$})}& 
%		41.5\scriptsize{$\pm0.1$}          & 
68.7    \normalsize{({\color{green}$+0.2$})}       \\

\midrule

RSC$^\dagger$ \cite{huang2020rsc}               & 
60.7\scriptsize{$\pm1.4$}      \normalsize{({\color{red}$-1.1$})}& 
51.4\scriptsize{$\pm0.3$}      \normalsize{($+0.0$)}& 
74.8\scriptsize{$\pm1.1$}      \normalsize{({\color{green}$+0.6$})}& 
75.1\scriptsize{$\pm1.3$}      \normalsize{({\color{green}$+0.5$})}& 
%		38.9\scriptsize{$\pm0.5$}          & 
65.5 \normalsize{($+0.0$)}  \\

%Fish $^\ddagger$ \cite{shi2021gradient}                     & 85.5\scriptsize{$\pm0.3$}          
%\normalsize{({\color{green}$+0.1$})} & 77.8\scriptsize{$\pm0.3$}          
%\normalsize{({\color{green}$+0.9$})} & 
%68.6\scriptsize{$\pm0.4$}            
%\normalsize{({\color{red}$-0.6$})} & 
%45.1\scriptsize{$\pm1.3$}            
%\normalsize{({\color{green}$+0.8$})} & 
%%		42.7\scriptsize{$\pm0.2$}            & 
%69.3 \normalsize{({\color{green}$+0.3$})}             \\ 

GSAM $^\ddagger$ \cite{zhuang2022surrogate}  & 
64.9\scriptsize$\pm0.1$     \normalsize{({\color{red}$-0.6$})} & 
55.2\scriptsize$\pm0.2$     \normalsize{({\color{green}$+1.1$})} & 
77.8\scriptsize$\pm0.0$     \normalsize{({\color{green}$+0.4$})}& 
79.2\scriptsize$\pm0.2$     \normalsize{({\color{green}$+0.3$})}& 
%		44.6\scriptsize$\pm0.2$             & 
69.3  \normalsize{({\color{green}$+0.3$})}  \\

SAM $^\ddagger$\cite{foret2020sharpness}  & 
64.5\scriptsize$\pm0.3$     \normalsize{({\color{green}$+0.7$})} & 
56.5\scriptsize$\pm0.2$     \normalsize{({\color{green}$+0.3$})} & 
77.4\scriptsize$\pm0.1$     \normalsize{({\color{green}$+1.0$})} & 
79.8\scriptsize$\pm0.4$     \normalsize{({\color{green}$+0.4$})} & 
%		44.3\scriptsize$\pm0.0$              & 
69.6 \normalsize{({\color{green}$+0.6$})}   \\


SAGM $^\ddagger$ \cite{wang2023sharpness}       & 
65.4\scriptsize{$\pm0.4$}    \normalsize{({\color{red}$-0.9$})}& 
57.0\scriptsize{$\pm0.3$}    \normalsize{({\color{red}$-0.8$})}& 
78.0\scriptsize{$\pm0.3$}    \normalsize{({\color{green}$+0.4$})}& 
80.0\scriptsize{$\pm0.2$}    \normalsize{({\color{red}$-1.1$})}& 
%		45.0\scriptsize{$\pm0.2$}           & 
70.1 \normalsize{({\color{red}$-0.6$})}   \\

%		\midrule
\midrule
\textbf{GGA} (ours)                 & 
64.3\scriptsize{$\pm0.1$}           & 
54.4\scriptsize{$\pm0.2$}           & 
76.5\scriptsize{$\pm0.3$}           &  
78.9\scriptsize{$\pm0.2$}    &  
68.5  \\
%		\midrule
\bottomrule
\end{tabular}
\end{table*}

%\subsection{TerraIncognita}
\begin{table*}[]
\centering
\small
\renewcommand{\arraystretch}{1.1}
\caption{\small{Out-of-domain accuracies (\%) on TerraIncognita.}}
\begin{tabular}{lllll|c}
\toprule
\textbf{Algorithm} & \textbf{L100} & \textbf{L38} & \textbf{L43} & \textbf{L46} & \textbf{Avg} \\
\midrule

MMD$^\ddagger$ \cite{li2018mmd}                  & 41.9\scriptsize{$\pm3.0$}     \normalsize{({\color{green}$+9.7$})} & 
34.8\scriptsize{$\pm1.0$}     \normalsize{({\color{green}$+9.8$})} & 
57.0\scriptsize{$\pm1.9$}     \normalsize{({\color{green}$+0.5$})} & 35.2\scriptsize{$\pm1.8$}     \normalsize{({\color{green}$+5.9$})} & 
%		23.4\scriptsize{$\pm9.5$}          & 
42.2 \normalsize{({\color{green}$+6.3$})}         \\

GroupDRO$^\ddagger$ \cite{Sagawa2020GroupDRO}    & 41.2\scriptsize{$\pm0.7$}    \normalsize{({\color{red}$-1.8$})}   & 
38.6\scriptsize{$\pm2.1$}    \normalsize{({\color{green}$+9.4$})}   & 
56.7\scriptsize{$\pm0.9$}    \normalsize{({\color{black}$\pm0.0$})}   & 
36.4\scriptsize{$\pm2.1$}    \normalsize{({\color{red}$-1.6$})}   & 
%		33.3\scriptsize{$\pm0.2$}          & 
43.2 \normalsize{({\color{green}$+1.7$})}           \\


Mixstyle$^\ddagger$ \cite{zhou2021mixstyle}   & 
54.3\scriptsize{$\pm1.1$} \normalsize{({\color{red}$-2.9$})}            & 34.1\scriptsize{$\pm1.1$} \normalsize{({\color{green}$+8.8$})}          & 55.9\scriptsize{$\pm1.1$} \normalsize{({\color{red}$-2.8$})}          & 31.7\scriptsize{$\pm2.1$} \normalsize{({\color{green}$+2.9$})}          & 
%		34.0\scriptsize{$\pm0.1$}          & 
44.0 \normalsize{({\color{green}$+1.1$})}           \\


%AND-mask \cite{shahtalebi2021sand} & 
%84.4\scriptsize{$\pm0.9$}  
%\normalsize{({\color{green}$+0.1$})}& 
%78.1\scriptsize{$\pm0.9$}  
%\normalsize{({\color{green}$+0.3$})}& 
%65.6\scriptsize{$\pm0.4$}  
%\normalsize{({\color{green}$+0.4$})} & 
%44.6\scriptsize{$\pm0.3$} 
%\normalsize{({\color{green}$+0.5$})}
%%        & 37.2\scriptsize{$\pm0.6$}  
%& 68.2 \normalsize{({\color{green}$+0.3$})}\\

ARM$^\ddagger$ \cite{zhang2020arm}               & 49.3\scriptsize{$\pm0.7$}     \normalsize{({\color{red}$-3.0$})} & 
38.3\scriptsize{$\pm0.7$}     \normalsize{({\color{green}$+4.2$})} & 
55.8\scriptsize{$\pm0.8$}     \normalsize{({\color{green}$+2.0$})} & 
38.7\scriptsize{$\pm1.3$}     \normalsize{({\color{red}$-0.2$})} & 
%		35.5\scriptsize{$\pm0.2$}          & 
45.5 \normalsize{({\color{green}$+0.8$})}           \\


MTL$^\ddagger$\cite{blanchard2021mtl_marginal_transfer_learning}    & 49.3\scriptsize{$\pm1.2$}      \normalsize{({\color{red}$-5.9$})} & 
39.6\scriptsize{$\pm6.3$}      \normalsize{({\color{green}$+3.6$})} & 
55.6\scriptsize{$\pm1.1$}      \normalsize{({\color{green}$+2.1$})} & 
37.8\scriptsize{$\pm0.8$}      \normalsize{({\color{green}$+2.6$})}&  
%		40.6\scriptsize{$\pm0.1$}          & 
45.6 \normalsize{({\color{green}$+0.9$})}          \\



ERM$^\ddagger$  \cite{vapnik1998statistical}   & 
49.8 \scriptsize$\pm4.4$ & 
42.1 \scriptsize$\pm1.4$ & 
56.9 \scriptsize$\pm1.8$ & 
35.7 \scriptsize$\pm3.9$ & 
46.1 \\

VREx$^\ddagger$ \cite{krueger2020vrex}         & 
48.2\scriptsize{$\pm4.3$}       \normalsize{({\color{green}$+3.1$})}& 41.7\scriptsize{$\pm1.3$}       \normalsize{({\color{green}$+0.7$})}& 56.8\scriptsize{$\pm0.8$}       \normalsize{({\color{green}$+2.0$})}& 38.7\scriptsize{$\pm3.1$}       \normalsize{({\color{red}$-0.4$})} & 
%		33.6\scriptsize{$\pm2.9$}          & 
46.4 \normalsize{({\color{green}$+1.3$})}         \\


IRM$^\dagger$ \cite{arjovsky2019irm}            & 54.6\scriptsize{$\pm1.3$}      \normalsize{({\color{red}$-4.3$})}& 
39.8\scriptsize{$\pm1.9$}      \normalsize{({\color{red}$-3.4$})} & 
56.2\scriptsize{$\pm1.8$}      \normalsize{({\color{red}$-3.8$})}& 
39.6\scriptsize{$\pm0.8$}      \normalsize{({\color{red}$-4.1$})} & 
%		33.9\scriptsize{$\pm2.8$}          & 
47.6 \normalsize{({\color{red}$-3.9$})}          \\


CORAL$^\dagger$ \cite{sun2016coral}             & 
51.6\scriptsize{$\pm2.4$}      \normalsize{({\color{green}$+3.1$})}& 
42.2\scriptsize{$\pm1.0$}      \normalsize{({\color{red}$-1.2$})} & 
57.0\scriptsize{$\pm1.0$}      \normalsize{({\color{green}$+1.1$})} & 
39.8\scriptsize{$\pm2.9$}      \normalsize{({\color{red}$-1.8$})}& 
%		41.5\scriptsize{$\pm0.1$}          & 
47.6    \normalsize{({\color{green}$+0.3$})}       \\

MLDG$^\dagger$ \cite{li2018learning}                & 54.2\scriptsize{$\pm3.0$}       \normalsize{({\color{red}$-2.5$})} &
44.3\scriptsize{$\pm1.1$}       \normalsize{({\color{green}$+1.4$})} & 55.6\scriptsize{$\pm0.3$}       \normalsize{({\color{green}$+5.1$})} &
36.9\scriptsize{$\pm2.2$}       \normalsize{({\color{green}$+0.6$})} & 
%		41.2\scriptsize{$\pm0.1$}            & 
47.8 \normalsize{({\color{green}$+1.2$})}      \\

Mixup$^\dagger$ \cite{xu2020interdomain_mixup_aaai}             & 59.6\scriptsize{$\pm2.0$}       \normalsize{({\color{green}$1.2$})} & 
42.2\scriptsize{$\pm1.4$}       \normalsize{({\color{green}$+7.6$})} & 
55.9\scriptsize{$\pm0.8$}       \normalsize{({\color{green}$+1.2$})} & 
33.9\scriptsize{$\pm1.4$}       \normalsize{({\color{red}$-0.9$})} & 
%		39.2\scriptsize{$\pm0.1$}            & 
47.9   \normalsize{({\color{green}$+2.1$})}        \\


SagNet$^\dagger$ \cite{nam2019sagnet}           &             53.0\scriptsize{$\pm2.0$}       \normalsize{({\color{green}$+2.3$})} & 
43.0\scriptsize{$\pm1.4$}       \normalsize{({\color{green}$+0.2$})} & 57.9\scriptsize{$\pm0.8$}       \normalsize{({\color{red}$-2.6$})}& 
40.4\scriptsize{$\pm1.4$}       \normalsize{({\color{green}$+2.9$})}& 
%		40.3\scriptsize{$\pm0.1$}          & 
48.6 \normalsize{({\color{green}$+0.4$})}         \\



\midrule



%Fish $^\ddagger$ \cite{shi2021gradient}                     & 85.5\scriptsize{$\pm0.3$}          
%\normalsize{({\color{green}$+0.1$})} & 77.8\scriptsize{$\pm0.3$}          
%\normalsize{({\color{green}$+0.9$})} & 
%68.6\scriptsize{$\pm0.4$}            
%\normalsize{({\color{red}$-0.6$})} & 
%45.1\scriptsize{$\pm1.3$}            
%\normalsize{({\color{green}$+0.8$})} & 
%%		42.7\scriptsize{$\pm0.2$}            & 
%69.3 \normalsize{({\color{green}$+0.3$})}             \\ 

SAM $^\ddagger$\cite{foret2020sharpness}  & 
46.3\scriptsize$\pm1.0$          \normalsize{({\color{green}$+3.3$})} & 
38.4\scriptsize$\pm2.4$          \normalsize{({\color{green}$+5.2$})} & 
54.0\scriptsize$\pm1.0$          \normalsize{({\color{green}$+1.9$})} & 
34.5\scriptsize$\pm0.8$          \normalsize{({\color{red}$-0.1$})} & 
%		44.3\scriptsize$\pm0.0$              & 
43.3 \normalsize{({\color{green}$+2.6$})}   \\

RSC$^\dagger$ \cite{huang2020rsc}               & 
50.2\scriptsize{$\pm2.2$}      \normalsize{({\color{red}$-0.8$})}& 
39.2\scriptsize{$\pm1.4$}      \normalsize{({\color{green}$+1.0$})}& 
56.3\scriptsize{$\pm1.4$}      \normalsize{({\color{green}$+0.8$})}&
40.8\scriptsize{$\pm0.6$}      \normalsize{({\color{green}$+0.2$})}& 
%		38.9\scriptsize{$\pm0.5$}          & 
46.6 \normalsize{({\color{green}$+0.2$})}   \\

GSAM $^\ddagger$ \cite{zhuang2022surrogate} & 
50.8\scriptsize$\pm0.1$        \normalsize{({\color{green}$+3.8$})} & 
39.3\scriptsize$\pm0.2$        \normalsize{({\color{green}$+0.6$})} & 
59.6\scriptsize$\pm0.0$        \normalsize{({\color{red}$-2.2$})}& 
38.2\scriptsize$\pm0.8$        \normalsize{({\color{green}$+0.4$})}& 
%		44.6\scriptsize$\pm0.2$             & 
47.0  \normalsize{({\color{green}$+0.6$})}  \\

SAGM $^\ddagger$ \cite{wang2023sharpness}       & 
54.8\scriptsize{$\pm1.3$}   \normalsize{($\pm0.0$)}& 
41.4\scriptsize{$\pm0.8$}   \normalsize{({\color{green}$+6.3$})}& 
57.7\scriptsize{$\pm0.6$}   \normalsize{({\color{red}$-1.1$})}& 
41.3\scriptsize{$\pm0.4$}   \normalsize{({\color{red}$-5.5$})}& 
%		45.0\scriptsize{$\pm0.2$}           & 
48.8 \normalsize{({\color{red}$-0.1$})}   \\

%		\midrule
\midrule
\textbf{GGA} (ours)                 & 
55.9\scriptsize{$\pm0.1$}           & 
45.5\scriptsize{$\pm0.1$}           & 
59.7\scriptsize{$\pm0.1$}           &  
41.5\scriptsize{$\pm0.1$}    &  
50.6  \\
%		\midrule
\bottomrule
\end{tabular}
\end{table*}

%\subsection{DomainNet}
\begin{table*}[]
\centering
\small
\renewcommand{\arraystretch}{1.1}
\caption{\small{Out-of-domain accuracies (\%) on {DomainNet}.}}
\begin{tabular}{lllllll|c}
\toprule
\textbf{Algorithm} & \textbf{clip} & \textbf{info} & \textbf{paint} & \textbf{quick} & \textbf{real} & \textbf{sketch} & \textbf{Avg} \\
\midrule

MMD$^\dagger$ \cite{li2018mmd}   & 
32.1 \scriptsize$\pm13.3$ & 
11.0 \scriptsize$\pm4.6$ & 
26.8 \scriptsize$\pm11.3$ & 
8.7 \scriptsize$\pm2.1$ & 
32.7 \scriptsize$\pm13.8$ & 
28.9 \scriptsize$\pm11.9$ & 23.4 \\


GroupDRO$^\dagger$ \cite{Sagawa2020GroupDRO} & 
47.2 \scriptsize$\pm0.5$ & 
17.5 \scriptsize$\pm0.4$ & 
33.8 \scriptsize$\pm0.5$ & 
9.3 \scriptsize$\pm0.3$ & 
51.6 \scriptsize$\pm0.4$ & 
40.1 \scriptsize$\pm0.6$ & 33.3 \\

VREx$^\dagger$ \cite{krueger2020vrex} & 
47.3 \scriptsize$\pm3.5$ & 
16.0 \scriptsize$\pm1.5$ & 
35.8 \scriptsize$\pm4.6$ & 
10.9 \scriptsize$\pm0.3$ & 
49.6 \scriptsize$\pm4.9$ & 
42.0 \scriptsize$\pm3.0$ & 33.6 \\

IRM$^\dagger$ \cite{arjovsky2019irm}  & 
48.5 \scriptsize$\pm2.8$ & 
15.0 \scriptsize$\pm1.5$ & 
38.3 \scriptsize$\pm4.3$ & 
10.9 \scriptsize$\pm0.5$ & 
48.2 \scriptsize$\pm5.2$ & 
42.3 \scriptsize$\pm3.1$ & 33.9 \\

Mixstyle$^\ddagger$ \cite{zhou2021mixstyle} & 
51.9 \scriptsize$\pm0.4$ & 
13.3 \scriptsize$\pm0.2$ & 
37.0 \scriptsize$\pm0.5$ & 
12.3 \scriptsize$\pm0.1$ & 
46.1 \scriptsize$\pm0.3$ & 
43.4 \scriptsize$\pm0.4$ & 34.0 \\

ARM$^\dagger$ \cite{zhang2020arm} & 
49.7 \scriptsize$\pm0.3$ & 
16.3 \scriptsize$\pm0.5$ & 
40.9 \scriptsize$\pm1.1$ & 
9.4 \scriptsize$\pm0.1$ & 
53.4 \scriptsize$\pm0.4$ & 
43.5 \scriptsize$\pm0.4$ & 35.5 \\

Mixup$^\ddagger$ \cite{xu2020interdomain_mixup_aaai} & 
55.7\scriptsize$\pm0.3$ & 
18.5\scriptsize$\pm0.5$ & 
44.3\scriptsize$\pm0.5$ & 
12.5\scriptsize$\pm0.4$ & 
55.8\scriptsize$\pm0.3$ & 
48.2\scriptsize$\pm0.5$ & 39.2 \\


SagNet$^\dagger$ \cite{nam2019sagnet} & 
57.7 \scriptsize$\pm0.3$ & 
19.0 \scriptsize$\pm0.2$ & 
45.3 \scriptsize$\pm0.3$ & 
12.7 \scriptsize$\pm0.5$ & 
58.1 \scriptsize$\pm0.5$ & 
48.8 \scriptsize$\pm0.2$ & 40.3 \\

MTL$^\dagger$ \cite{blanchard2021mtl_marginal_transfer_learning}& 
57.9 \scriptsize$\pm0.5$ & 
18.5 \scriptsize$\pm0.4$ & 
46.0 \scriptsize$\pm0.1$ & 
12.5 \scriptsize$\pm0.1$ & 
59.5 \scriptsize$\pm0.3$ & 
49.2 \scriptsize$\pm0.1$ & 40.6 \\

MLDG$^\dagger$ \cite{li2018learning}& 
59.1 \scriptsize$\pm0.2$ & 
19.1 \scriptsize$\pm0.3$ & 
45.8 \scriptsize$\pm0.7$ & 
13.4 \scriptsize$\pm0.3$ & 
59.6 \scriptsize$\pm0.2$ & 
50.2 \scriptsize$\pm0.4$ & 41.2 \\

CORAL$^\dagger$ \cite{sun2016coral}& 
59.2 \scriptsize$\pm0.1$ & 
19.7 \scriptsize$\pm0.2$ & 
46.6 \scriptsize$\pm0.3$ & 
13.4 \scriptsize$\pm0.4$ &
59.8 \scriptsize$\pm0.2$ & 
50.1 \scriptsize$\pm0.6$ & 41.5 \\

ERM$^\ddagger$ \cite{vapnik1998statistical} & 
63.0 \scriptsize$\pm0.2$ & 
21.2 \scriptsize$\pm0.2$ & 
50.1 \scriptsize$\pm0.4$ & 
13.9 \scriptsize$\pm0.5$ & 
63.7 \scriptsize$\pm0.2$ &
52.0 \scriptsize$\pm0.5$ & 43.8 \\

\midrule 
% Gradient Based Methods

RSC$^\dagger$ \cite{huang2020rsc}               & 
55.0\scriptsize{$\pm1.2$} & 
18.3\scriptsize{$\pm0.5$} & 
44.4\scriptsize{$\pm0.6$} &
12.2\scriptsize{$\pm0.2$} & 
55.7\scriptsize{$\pm0.7$} &
47.8\scriptsize{$\pm0.9$} &
38.9 \\

SAM$^\ddagger$ \cite{foret2020sharpness}  & 
64.5\scriptsize$\pm0.3$  & 
20.7\scriptsize$\pm0.2$  & 
50.2\scriptsize$\pm0.1$  & 
15.1\scriptsize$\pm0.3$  & 
62.6\scriptsize$\pm0.2$  & 
52.7\scriptsize$\pm0.3$  &  44.3   \\


GSAM $^\ddagger$ \cite{zhuang2022surrogate} & 
64.2\scriptsize$\pm0.3$  & 
20.8\scriptsize$\pm0.2$  & 
50.9\scriptsize$\pm0.0$  & 
14.4\scriptsize$\pm0.8$  & 
63.5\scriptsize$\pm0.2$  &
53.9\scriptsize$\pm0.2$  &  44.6   \\


SAGM $^\ddagger$ \cite{wang2023sharpness}   & 
64.9\scriptsize{$\pm0.2$}  & 
21.1\scriptsize{$\pm0.3$}  & 
51.5\scriptsize{$\pm0.2$}  & 
14.8\scriptsize{$\pm0.2$}  & 
64.1\scriptsize{$\pm0.2$}  & 
53.6\scriptsize{$\pm0.2$}  & 45.0 \\

%		\midrule
\midrule

GGA (ours) & 
64.0\scriptsize{$\pm0.2$} & 
22.2\scriptsize{$\pm0.3$} & 
51.7\scriptsize{$\pm0.1$} & 
14.3\scriptsize{$\pm0.2$} & 
64.1\scriptsize{$\pm0.4$} & 
54.3\scriptsize{$\pm0.3$} & 45.2 \\

\bottomrule
\end{tabular}
\end{table*}

%\subsection{CMNIST}
\begin{table*}
	\centering
	\small
	\renewcommand{\arraystretch}{1.1}
	\caption{\small{Out-of-domain accuracies (\%) on ColoredMNIST (left) and RotatedMNIST (right).}}
	\begin{tabular}{llll|c|llllll|c}
		\toprule
		\textbf{Algorithm} & \textbf{0.1} & \textbf{0.2} & \textbf{0.9} & \textbf{Avg} 
		&\textbf{0} &\textbf{15} &\textbf{30} &
		\textbf{45} &\textbf{60} &
		\textbf{75} &\textbf{Avg} \\
		\midrule
		
		IRM$\ddagger$ \cite{arjovsky2019irm} & 
		56.8\scriptsize$\pm4.5$ & 63.5\scriptsize$\pm2.7$ & 
		10.2\scriptsize$\pm0.2$ & 
		43.5 &
		95.5\scriptsize{$\pm0.4$} & 98.7\scriptsize{$\pm0.2$} & 
		98.7\scriptsize{$\pm0.1$} & 98.5\scriptsize{$\pm0.3$} & 
		98.7\scriptsize{$\pm0.1$} & 96.1\scriptsize{$\pm0.1$} & 97.7\\ 
		
		MLDG \cite{li2018learning} & 
		71.5\scriptsize$\pm0.6$ & 73.0\scriptsize$\pm0.1$ & 
		10.1\scriptsize$\pm0.2$ & 
		51.5 &
		94.7\scriptsize{$\pm0.7$} & 98.8\scriptsize{$\pm0.1$} & 
		98.8\scriptsize{$\pm0.1$} & 98.8\scriptsize{$\pm0.1$} & 
		98.7\scriptsize{$\pm0.1$} & 95.9\scriptsize{$\pm0.4$} & 97.6\\ 
		
		MTL \cite{blanchard2021mtl_marginal_transfer_learning} & 
		71.3\scriptsize$\pm0.6$ & 72.9\scriptsize$\pm0.2$ & 
		10.2\scriptsize$\pm0.1$ & 
		51.5 &
		94.6\scriptsize{$\pm1.1$} & 98.6\scriptsize{$\pm0.2$} & 
		98.8\scriptsize{$\pm0.1$} & 98.7\scriptsize{$\pm0.1$} & 
		98.7\scriptsize{$\pm0.3$} & 95.3\scriptsize{$\pm0.7$} & 97.4\\ 
		
		Mixup \cite{xu2020interdomain_mixup_aaai} & 
		71.5\scriptsize$\pm0.8$ & 73.2\scriptsize$\pm0.3$ & 
		10.2\scriptsize$\pm0.2$ & 
		51.6 &
		94.9\scriptsize{$\pm0.5$} & 98.8\scriptsize{$\pm0.1$} & 
		98.8\scriptsize{$\pm0.2$} & 98.8\scriptsize{$\pm0.1$} & 
		98.8\scriptsize{$\pm0.1$} & 95.7\scriptsize{$\pm0.5$} & 97.6\\
		
		SagNet \cite{nam2019sagnet}  & 
		72.0\scriptsize$\pm0.5$ & 72.8\scriptsize$\pm0.5$ & 
		9.9\scriptsize$\pm0.3$ & 
		51.6 &
		95.5\scriptsize{$\pm0.3$} & 98.9\scriptsize{$\pm0.1$} & 
		99.0\scriptsize{$\pm0.1$} & 98.8\scriptsize{$\pm0.2$} & 
		98.8\scriptsize{$\pm0.1$} & 95.9\scriptsize{$\pm0.4$} & 97.8\\
		
		ERM \cite{vapnik1998statistical} & 
		71.8\scriptsize$\pm0.9$ & 73.3\scriptsize$\pm0.4$ & 
		9.9\scriptsize$\pm0.3$ & 
		51.7 &
		
		95.1\scriptsize{$\pm0.6$} & 98.7\scriptsize{$\pm0.2$} & 
		98.7\scriptsize{$\pm0.2$} & 98.7\scriptsize{$\pm0.2$} & 
		98.8\scriptsize{$\pm0.1$} & 95.6\scriptsize{$\pm0.4$} & 97.6 \\
		
		ARM \cite{zhang2020arm}  & 
		74.5\scriptsize{$\pm3.8$} & 71.1\scriptsize$\pm1.8$ & 
		9.9\scriptsize$\pm0.3$   & 
		51.8 &
		
		95.1\scriptsize{$\pm1.1$} & 98.8\scriptsize{$\pm0.2$} & 
		98.8\scriptsize{$\pm0.1$} & 98.8\scriptsize{$\pm0.1$} & 
		98.8\scriptsize{$\pm0.1$} & 96\scriptsize{$\pm0.6$} & 97.7 \\ 
		
		CORAL \cite{sun2016coral} & 
		72.3\scriptsize$\pm0.7$ & 72.8\scriptsize$\pm0.4$ & 
		10.5\scriptsize$\pm0.3$ & 
		51.8 &
		
		95.6\scriptsize{$\pm0.3$} & 98.9\scriptsize{$\pm0.1$} & 
		98.9\scriptsize{$\pm0.1$} & 99.0\scriptsize{$\pm0.0$} & 
		98.9\scriptsize{$\pm0.1$} & 96.1\scriptsize{$\pm0.3$} & 97.9 \\ 
		
		
		
		Fish \cite{shi2021gradient}  & 
		71.7\scriptsize$\pm0.5$ & 73.2\scriptsize$\pm0.5$ & 
		10.4\scriptsize$\pm0.2$ & 
		51.8 &
		95.3\scriptsize{$\pm0.6$} & 98.9\scriptsize{$\pm0.1$} & 
		98.9\scriptsize{$\pm0.2$} & 98.9\scriptsize{$\pm0.1$} & 
		98.9\scriptsize{$\pm0.1$} & 95.6\scriptsize{$\pm0.6$} & 97.7 \\ 
		
		GroupDRO \cite{Sagawa2020GroupDRO} & 
		72.6\scriptsize$\pm0.6$ & 73.5\scriptsize$\pm0.4$ & 
		9.9\scriptsize$\pm0.2$ & 
		52.0 &
		95.9\scriptsize{$\pm0.6$} & 98.7\scriptsize{$\pm0.2$} & 
		98.6\scriptsize{$\pm0.1$} & 98.7\scriptsize{$\pm0.1$} & 
		98.7\scriptsize{$\pm0.1$} & 96.0\scriptsize{$\pm0.2$} & 97.8\\  	
		
		VREx \cite{krueger2020vrex} & 
		72.9\scriptsize$\pm0.3$ & 72.9\scriptsize$\pm0.4$ & 
		10.3\scriptsize$\pm0.6$ & 
		52.0 &
		95.7\scriptsize{$\pm0.6$} & 98.9\scriptsize{$\pm0.2$} & 
		98.7\scriptsize{$\pm0.1$} & 98.9\scriptsize{$\pm0.2$} & 
		98.9\scriptsize{$\pm0.1$} & 95.8\scriptsize{$\pm0.4$} & 97.8\\ 
		 
		
		\midrule
		
		SAM \cite{foret2020sharpness} & 
		71.1\scriptsize$\pm0.5$ & 73.3\scriptsize$\pm0.4$ & 
		10.1\scriptsize$\pm0.3$ & 
		51.5 &
		95.7\scriptsize{$\pm0.2$} & 99.0\scriptsize{$\pm0.1$} & 
		98.9\scriptsize{$\pm0.1$} & 98.9\scriptsize{$\pm0.1$} & 
		98.9\scriptsize{$\pm0.1$} & 96.2\scriptsize{$\pm0.4$} & 97.9\\
		
		GSAM \cite{zhuang2022surrogate} & 
		71.8\scriptsize$\pm0.3$ & 73.2\scriptsize$\pm0.2$ & 
		9.9\scriptsize$\pm0.2$ & 
		51.6 &
		94.9\scriptsize{$\pm0.1$} & 98.9\scriptsize{$\pm0.1$} & 
		98.9\scriptsize{$\pm0.2$} & 99.0\scriptsize{$\pm0.2$} & 
		98.8\scriptsize{$\pm0.1$} & 96.0\scriptsize{$\pm0.1$} & 97.7\\
		
		RSC \cite{huang2020rsc} & 
		72.5\scriptsize$\pm0.3$ & 72.4\scriptsize$\pm0.6$ & 
		10.2\scriptsize$\pm0.5$ & 
		51.7 &
		94.2\scriptsize{$\pm1.1$} & 98.6\scriptsize{$\pm0.1$} & 
		98.7\scriptsize{$\pm0.2$} & 98.6\scriptsize{$\pm0.2$} & 
		98.7\scriptsize{$\pm0.2$} & 95.7\scriptsize{$\pm0.7$} & 97.4\\  
		
		
		SAGM \cite{wang2023sharpness} & 
		71.5\scriptsize$\pm0.8$ & 73.6\scriptsize$\pm0.5$ & 
		10.6\scriptsize$\pm0.6$ & 
		51.9 &
		95.4\scriptsize{$\pm0.4$} & 98.9\scriptsize{$\pm0.1$} & 
		98.9\scriptsize{$\pm0.1$} & 98.9\scriptsize{$\pm0.1$} & 
		98.9\scriptsize{$\pm0.1$} & 95.9\scriptsize{$\pm0.5$} & 97.8\\
		
		\midrule
		\textbf{GGA} (ours) & 
		71.2\scriptsize$\pm0.7$ & 73.1\scriptsize$\pm0.6$ & 
		11.5\scriptsize$\pm0.4$ & 
		51.9 &
		95.1\scriptsize{$\pm0.8$} & 99.0\scriptsize{$\pm0.1$} & 
		99.0\scriptsize{$\pm0.3$} & 98.8\scriptsize{$\pm0.1$} & 
		98.8\scriptsize{$\pm0.2$} & 96.1\scriptsize{$\pm0.4$} & 97.8 \\ 
		\bottomrule
	\end{tabular}
\end{table*}


\clearpage
\clearpage

\bibliographystyle{ieeenat_fullname}
\bibliography{supp}


% WARNING: do not forget to delete the supplementary pages from your submission 
% \clearpage
\pagenumbering{gobble}
\maketitlesupplementary

\section{Additional Results on Embodied Tasks}

To evaluate the broader applicability of our EgoAgent's learned representation beyond video-conditioned 3D human motion prediction, we test its ability to improve visual policy learning for embodiments other than the human skeleton.
Following the methodology in~\cite{majumdar2023we}, we conduct experiments on the TriFinger benchmark~\cite{wuthrich2020trifinger}, which involves a three-finger robot performing two tasks: reach cube and move cube. 
We freeze the pretrained representations and use a 3-layer MLP as the policy network, training each task with 100 demonstrations.

\begin{table}[h]
\centering
\caption{Success rate (\%) on the TriFinger benchmark, where each model's pretrained representation is fixed, and additional linear layers are trained as the policy network.}
\label{tab:trifinger}
\resizebox{\linewidth}{!}{%
\begin{tabular}{llcc}
\toprule
Methods       & Training Dataset & Reach Cube & Move Cube \\
\midrule
DINO~\cite{caron2021emerging}         & WT Venice        & 78.03     & 47.42     \\
DoRA~\cite{venkataramanan2023imagenet}          & WT Venice        & 81.62     & 53.76     \\
DoRA~\cite{venkataramanan2023imagenet}          & WT All           & 82.40     & 48.13     \\
\midrule
EgoAgent-300M & WT+Ego-Exo4D      & 82.61    & 54.21      \\
EgoAgent-1B   & WT+Ego-Exo4D      & \textbf{85.72}      & \textbf{57.66}   \\
\bottomrule
\end{tabular}%
}
\end{table}

As shown in Table~\ref{tab:trifinger}, EgoAgent achieves the highest success rates on both tasks, outperforming the best models from DoRA~\cite{venkataramanan2023imagenet} with increases of +3.32\% and +3.9\% respectively.
This result shows that by incorporating human action prediction into the learning process, EgoAgent demonstrates the ability to learn more effective representations that benefit both image classification and embodied manipulation tasks.
This highlights the potential of leveraging human-centric motion data to bridge the gap between visual understanding and actionable policy learning.



\section{Additional Results on Egocentric Future State Prediction}

In this section, we provide additional qualitative results on the egocentric future state prediction task. Additionally, we describe our approach to finetune video diffusion model on the Ego-Exo4D dataset~\cite{grauman2024ego} and generate future video frames conditioned on initial frames as shown in Figure~\ref{fig:opensora_finetune}.

\begin{figure}[b]
    \centering
    \includegraphics[width=\linewidth]{figures/opensora_finetune.pdf}
    \caption{Comparison of OpenSora V1.1 first-frame-conditioned video generation results before and after finetuning on Ego-Exo4D. Fine-tuning enhances temporal consistency, but the predicted pixel-space future states still exhibit errors, such as inaccuracies in the basketball's trajectory.}
    \label{fig:opensora_finetune}
\end{figure}

\subsection{Visualizations and Comparisons}

More visualizations of our method, DoRA, and OpenSora in different scenes (as shown in Figure~\ref{fig:supp pred}). For OpenSora, when predicting the states of $t_k$, we use all the ground truth frames from $t_{0}$ to $t_{k-1}$ as conditions. As OpenSora takes only past observations as input and neglects human motion, it performs well only when the human has relatively small motions (see top cases in Figure~\ref{fig:supp pred}), but can not adjust to large movements of the human body or quick viewpoint changes (see bottom cases in Figure~\ref{fig:supp pred}).

\begin{figure*}
    \centering
    \includegraphics[width=\linewidth]{figures/supp_pred.pdf}
    \caption{Retrieval and generation results for egocentric future state prediction. Correct and wrong retrieval images are marked with green and red boundaries, respectively.}
    \label{fig:supp pred}
\end{figure*}

\begin{figure*}[t]
    \centering
    \includegraphics[width=0.9\linewidth]{figures/motion_prediction.pdf}
    \vspace{-0.5mm}
    \caption{Motion prediction results in scenes with minor changes in observation.}
    \vspace{-1.5mm}
    \label{fig:motion_prediction}
\end{figure*}

\subsection{Finetuning OpenSora on Ego-Exo4D}

OpenSora V1.1~\cite{opensora}, initially trained on internet videos and images, produces severely inconsistent results when directly applied to infer future videos on the Ego-Exo4D dataset, as illustrated in Figure~\ref{fig:opensora_finetune}.
To address the gap between general internet content and egocentric video data, we fine-tune the official checkpoint on the Ego-Exo4D training set for 50 epochs.
OpenSora V1.1 proposed a random mask strategy during training to enable video generation by image and video conditioning. We adopted the default masking rate, which applies: 75\% with no masking, 2.5\% with random masking of 1 frame to 1/4 of the total frames, 2.5\% with masking at either the beginning or the end for 1 frame to 1/4 of the total frames, and 5\% with random masking spanning 1 frame to 1/4 of the total frames at both the beginning and the end.

As shown in Fig.~\ref{fig:opensora_finetune}, despite being trained on a large dataset, OpenSora struggles to generalize to the Ego-Exo4D dataset, producing future video frames with minimal consistency relative to the conditioning frame. While fine-tuning improves temporal consistency, the moving trajectories of objects like the basketball and soccer ball still deviate from realistic physical laws. Compared with our feature space prediction results, this suggests that training world models in a reconstructive latent space is more challenging than training them in a feature space.


\section{Additional Results on 3D Human Motion Prediction}

We present additional qualitative results for the 3D human motion prediction task, highlighting a particularly challenging scenario where egocentric observations exhibit minimal variation. This scenario poses significant difficulties for video-conditioned motion prediction, as the model must effectively capture and interpret subtle changes. As demonstrated in Fig.~\ref{fig:motion_prediction}, EgoAgent successfully generates accurate predictions that closely align with the ground truth motion, showcasing its ability to handle fine-grained temporal dynamics and nuanced contextual cues.

\section{OpenSora for Image Classification}

In this section, we detail the process of extracting features from OpenSora V1.1~\cite{opensora} (without fine-tuning) for an image classification task. Following the approach of~\cite{xiang2023denoising}, we leverage the insight that diffusion models can be interpreted as multi-level denoising autoencoders. These models inherently learn linearly separable representations within their intermediate layers, without relying on auxiliary encoders. The quality of the extracted features depends on both the layer depth and the noise level applied during extraction.


\begin{table}[h]
\centering
\caption{$k$-NN evaluation results of OpenSora V1.1 features from different layer depths and noising scales on ImageNet-100. Top1 and Top5 accuracy (\%) are reported.}
\label{tab:opensora-knn}
\resizebox{0.95\linewidth}{!}{%
\begin{tabular}{lcccccc}
\toprule
\multirow{2}{*}{Timesteps} & \multicolumn{2}{c}{First Layer} & \multicolumn{2}{c}{Middle Layer} & \multicolumn{2}{c}{Last Layer} \\
\cmidrule(r){2-3}   \cmidrule(r){4-5}  \cmidrule(r){6-7}  & Top1           & Top5           & Top1            & Top5           & Top1           & Top5          \\
\midrule
32        &  6.10           & 18.20             & 34.04               & 59.50             & 30.40             & 55.74             \\
64        & 6.12              & 18.48              & 36.04               & 61.84              & 31.80         & 57.06         \\
128       & 5.84             & 18.14             & 38.08               & 64.16              & 33.44       & 58.42 \\
256       & 5.60             & 16.58              & 30.34               & 56.38              &28.14          & 52.32        \\
512       & 3.66              & 11.70            & 6.24              & 17.62              & 7.24              & 19.44  \\ 
\bottomrule
\end{tabular}%
}
\end{table}

As shown in Table~\ref{tab:opensora-knn}, we first evaluate $k$-NN classification performance on the ImageNet-100 dataset using three intermediate layers and five different noise scales. We find that a noise timestep of 128 yields the best results, with the middle and last layers performing significantly better than the first layer.
We then test this optimal configuration on ImageNet-1K and find that the last layer with 128 noising timesteps achieves the best classification accuracy.

\section{Data Preprocess}
For egocentric video sequences, we utilize videos from the Ego-Exo4D~\cite{grauman2024ego} and WT~\cite{venkataramanan2023imagenet} datasets.
The original resolution of Ego-Exo4D videos is 1408×1408, captured at 30 fps. We sample one frame every five frames and use the original resolution to crop local views (224×224) for computing the self-supervised representation loss. For computing the prediction and action loss, the videos are downsampled to 224×224 resolution.
WT primarily consists of 4K videos (3840×2160) recorded at 60 or 30 fps. Similar to Ego-Exo4D, we use the original resolution and downsample the frame rate to 6 fps for representation loss computation.
As Ego-Exo4D employs fisheye cameras, we undistort the images to a pinhole camera model using the official Project Aria Tools to align them with the WT videos.

For motion sequences, the Ego-Exo4D dataset provides synchronized 3D motion annotations and camera extrinsic parameters for various tasks and scenes. While some annotations are manually labeled, others are automatically generated using 3D motion estimation algorithms from multiple exocentric views. To maximize data utility and maintain high-quality annotations, manual labels are prioritized wherever available, and automated annotations are used only when manual labels are absent.
Each pose is converted into the egocentric camera's coordinate system using transformation matrices derived from the camera extrinsics. These matrices also enable the computation of trajectory vectors for each frame in a sequence. Beyond the x, y, z coordinates, a visibility dimension is appended to account for keypoints invisible to all exocentric views. Finally, a sliding window approach segments sequences into fixed-size windows to serve as input for the model. Note that we do not downsample the frame rate of 3D motions.

\section{Training Details}
\subsection{Architecture Configurations}
In Table~\ref{tab:arch}, we provide detailed architecture configurations for EgoAgent following the scaling-up strategy of InternLM~\cite{team2023internlm}. To ensure the generalization, we do not modify the internal modules in InternML, \emph{i.e.}, we adopt the RMSNorm and 1D RoPE. We show that, without specific modules designed for vision tasks, EgoAgent can perform well on vision and action tasks.

\begin{table}[ht]
  \centering
  \caption{Architecture configurations of EgoAgent.}
  \resizebox{0.8\linewidth}{!}{%
    \begin{tabular}{lcc}
    \toprule
          & EgoAgent-300M & EgoAgent-1B \\
          \midrule
    Depth & 22    & 22 \\
    Embedding dim & 1024  & 2048 \\
    Number of heads & 8     & 16 \\
    MLP ratio &    8/3   & 8/3 \\
    $\#$param.  & 284M & 1.13B \\
    \bottomrule
    \end{tabular}%
    }
  \label{tab:arch}%
\end{table}%

Table~\ref{tab:io_structure} presents the detailed configuration of the embedding and prediction modules in EgoAgent, including the image projector ($\text{Proj}_i$), representation head/state prediction head ($\text{MLP}_i$), action projector ($\text{Proj}_a$) and action prediction head ($\text{MLP}_a$).
Note that the representation head and the state prediction head share the same architecture but have distinct weights.

\begin{table}[t]
\centering
\caption{Architecture of the embedding ($\text{Proj}_i$, $\text{Proj}_a$) and prediction ($\text{MLP}_i$, $\text{MLP}_a$) modules in EgoAgent. For details on module connections and functions, please refer to Fig.~2 in the main paper.}
\label{tab:io_structure}
\resizebox{\linewidth}{!}{%
\begin{tabular}{lcl}
\toprule
       & \multicolumn{1}{c}{Norm \& Activation} & \multicolumn{1}{c}{Output Shape}  \\
\midrule
\multicolumn{3}{l}{$\text{Proj}_i$ (\textit{Image projector})} \\
\midrule
Input image  & -          & 3$\times$224$\times$224 \\
Conv 2D (16$\times$16) & -       & Embedding dim$\times$14$\times$14    \\
\midrule
\multicolumn{3}{l}{$\text{MLP}_i$ (\textit{State prediction head} \& \textit{Representation head)}} \\
\midrule
Input embedding  & -          & Embedding dim \\
Linear & GELU       & 2048          \\
Linear & GELU       & 2048          \\
Linear & -          & 256           \\
Linear & -          & 65536     \\
\midrule
\multicolumn{3}{l}{$\text{Proj}_a$ (\textit{Action projector})} \\
\midrule
Input pose sequence  & -          & 4$\times$5$\times$17 \\
Conv 2D (5$\times$17) & LN, GELU   & Embedding dim$\times$1$\times$1    \\
\midrule
\multicolumn{3}{l}{$\text{MLP}_a$ (\textit{Action prediction head})} \\
\midrule
Input embedding  & -          & Embedding dim$\times$1$\times$1 \\
Linear & -          & 4$\times$5$\times$17     \\
\bottomrule
\end{tabular}%
}
\end{table}


\subsection{Training Configurations}
In Table~\ref{tab:training hyper}, we provide the detailed training hyper-parameters for experiments in the main manuscripts.

\begin{table}[ht]
  \centering
  \caption{Hyper-parameters for training EgoAgent.}
  \resizebox{0.86\linewidth}{!}{%
    \begin{tabular}{lc}
    \toprule
    Training Configuration & EgoAgent-300M/1B \\
    \midrule
    Training recipe: &  \\
    optimizer & AdamW~\cite{loshchilov2017decoupled} \\
    optimizer momentum & $\beta_1=0.9, \beta_2=0.999$ \\
    \midrule
    Learning hyper-parameters: &  \\
    base learning rate & 6.0E-04 \\
    learning rate schedule & cosine \\
    base weight decay & 0.04 \\
    end weight decay & 0.4 \\
    batch size & 1920 \\
    training iters & 72,000 \\
    lr warmup iters & 1,800 \\
    warmup schedule & linear \\
    gradient clip & 1.0 \\
    data type & float16 \\
    norm epsilon & 1.0E-06 \\
    \midrule
    EMA hyper-parameters: &  \\
    momentum & 0.996 \\
    \bottomrule
    \end{tabular}%
    }
  \label{tab:training hyper}%
\end{table}%

\clearpage








\end{document}
