\begin{abstract}

Domain Generalization (DG) research has gained considerable traction as of 
late, since the ability to generalize to unseen data distributions is a 
requirement that eludes even state-of-the-art training algorithms. In this 
paper we observe that the initial iterations of model training play a key 
role in domain generalization effectiveness, since the loss landscape may be 
significantly different across the training and test distributions, contrary 
to the case of i.i.d. data. Conflicts between gradients of the loss 
components of each domain lead the optimization procedure to undesirable 
local minima that do not capture the domain-invariant features of the target 
classes. We propose alleviating domain conflicts in model optimization, by 
iteratively annealing the parameters of a model in the early stages of 
training and searching for points where gradients align between domains. By 
discovering a set of parameter values where gradients are updated towards the 
same direction for each data distribution present in the training set, the 
proposed Gradient-Guided Annealing (GGA) algorithm encourages models to seek 
out minima that exhibit improved robustness against domain shifts. The 
efficacy of GGA is evaluated on five widely accepted and challenging
image classification domain generalization benchmarks, where its use alone is
able to establish highly competitive or even state-of-the-art performance.
Moreover, when combined with previously proposed domain-generalization
algorithms it is able to consistently improve their effectiveness by significant margins\footnote{Code available at: \href{https://github.com/aristotelisballas/GGA}{https://github.com/aristotelisballas/GGA}}.

\end{abstract}
