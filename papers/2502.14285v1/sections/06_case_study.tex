\section{Case Study}
To clearly demonstrate EvoStealer's advantages over baseline methods, we select an easy and a hard example for case study, with the results shown in Figure~\ref{fig:case}. The results show that, on in-domain data, EvoStealer generates images that closely match the style of the original images, with all four synthesized images maintaining stylistic consistency. In contrast, the four images generated by the other baseline methods exhibit significant style variation. On out-of-domain data, EvoStealer maintains the same style as in-domain images, successfully achieving subject generalization. In contrast, the other baseline methods fail to generalize. Additionally, we analyze three distinct failure cases (see Appendix~\ref{app_failcases} for details).
% 为了更直观的体现EvoStealer相对于baselines的优势,我们选取了一个简单和一个困难的例子进行样例分析。结果如图3所示。我们可以观察到,在in-domain数据上,EvoStealer能够合成和原图风格非常接近的图片,并且合成的4张图片风格一致。相比之下,其他3个baselines合成的4张图片则风格迥异。在out-of-domain数据上,EvoStealer生成的图片仍能和in-domain上的图片保持一致的风格,成功地实现了主体的泛化。相比之下,其他baselines则无法实现泛化。