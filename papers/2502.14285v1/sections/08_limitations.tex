\section{Limitations}
% 我们提出的EvoStaler和benchmark存在如下limitations
% 1. EvoStealer是一种依赖于MLLMs能力的窃取方法。优点是EvoStealer不需要任何微调,相比于prompt stealing attack的方法简单易用,并且得益于MLLMs在大量数据上进行过预训练,因此不会像其他方法一样,在开放数据上出现性能衰退的现象。但是缺点也是EvoStealer的性能上限受制于MLLMs。

% 2. 由于经费限制,我们的benchmark仅包含了DALL E-3生成的图片,并未包含Midjourney,Stable Diffusion等常见的模型。然而,我们认为其足以衡量窃取方法的性能。我们将在未来的研究丰富我们的benchmark

% 3. 为了和当前的方法进行比较,我们的benchmark仅包含一个主体,即每个数据仅包含一个[subject]标记。然而,现实的模板存在多个主体的情况,窃取难度也会相应越大。我们将在后续研究中将此类情况纳入研究范围。

The current implementation of EvoStealer and benchmark presents several methodological limitations:
\begin{enumerate}
    \item EvoStealer's MLLM-based design offers simplified implementation without fine-tuning requirements and maintains robust performance across open datasets. However, this approach inherently limits the system's maximum performance to the capabilities of the underlying MLLMs.
    \item Resource constraints restricted our benchmark to DALL·E-3 generated images, excluding other prominent models like Midjourney and Stable Diffusion. Nevertheless, the current benchmark adequately evaluates stealing method performance, with planned expansion to additional models in future work.
    \item The benchmark's single-subject design facilitates comparative analysis but does not address multi-subject templates in real-world applications—a limitation to be addressed in subsequent research.
\end{enumerate}

\section{Ethical Considerations}
% EvoStealer是一个prompt template窃取方法,仅需要几张示例图片就能窃取相应的prompt template,基于此template,只需要做适当的修改就能生产大批量同风格的图片。这种行为对知识产权构成了严重的威胁。本研究的目的是揭露这一潜在的安全隐患。我们认为,了解这种威胁模型对有效制定适当的防御机制至关重要。尽管当前有使用水印的方式去保护图片,但是这在现实的交易平台难以实现,因为不完整的图片会影响客户的购买意愿,客户可能购买到不符合自己意愿的prompt,从而面临客户的投诉。我们认为展示更少的图片(如2-4张)是一个简单的防御措施。在未来,探索更鲁邦有效的防御措施是非常有必要的。
EvoStealer's ability to extract prompt templates from minimal image examples enables attackers to generate multiple stylistically similar images through minor template modifications, posing significant risks to creators' intellectual property. This research highlights this security vulnerability, as understanding such threat models is essential for developing effective countermeasures.

While watermarking offers some protection, its implementation on trading platforms presents practical challenges. Watermarks can obscure image details, potentially deterring buyers or leading to customer dissatisfaction when purchased prompts fail to meet expectations. Our findings suggest that limiting the number of displayed images to 2-4 examples provides a simple yet effective defensive strategy.

Future research should prioritize developing robust protection mechanisms to safeguard both creators' rights and the integrity of the AI-generated content marketplace.