% Prompt交易是一个涉及知识产权的交易模板,近些年取得了迅猛的发展。这些交易提供少量样例图片和模型信息,吸引用户购买生成这些图片的prompt模板。在本篇文章中,我们探讨了一个攻击任务——攻击者能否通过少量图片窃取待售的prompt模板。为此,我们制作了一个包含Easy和Hard两个难度级别的数据集,共包含50个模板及450张图片。此外,我们提出了一个名为EvoStealer的不需要微调模型窃取方法。EvoStealer基于差分进化算法,首先使用多模态模型基于设定pattern初始化子代,然后使用llm迭代地产生新的子代逐步改善种群的质量。特别地,在进化过程中,EvoStealer特别关注子代的公共特征,以期望产生更通用的模板。我们使用开源模型和闭源模型进行了大量的试验。实验结果表明EvoStealer窃取的模板不仅能够复现出和原图非常相似的图片,还能够很好地泛化到其他主体上。显著高于其他baselines(平均超过10%)。代价分析显示,EvoStealer可实现接近零成本的窃取。

Prompt trading has emerged as a significant intellectual property concern in recent years, where vendors entice users by showcasing sample images before selling prompt templates that can generate similar images. This work investigates a critical security vulnerability: attackers can steal prompt templates using only a limited number of sample images. 
To investigate this threat, we introduce \textbf{\data}, a prompt-stealing benchmark consisting of 50 templates and 450 images, organized into Easy and Hard difficulty levels.
To identify the vulnerabity of VLMs to prompt stealing, we propose \textbf{EvoStealer}, a novel template stealing method that operates without model fine-tuning by leveraging differential evolution algorithms. The system first initializes population sets using multimodal large language models (MLLMs) based on predefined patterns, then iteratively generates enhanced offspring through MLLMs. During evolution, EvoStealer identifies common features across offspring to derive generalized templates. 
Our comprehensive evaluation conducted across open-source (\intern) and closed-source models (\gpta and \gpt) demonstrates that EvoStealer's stolen templates can reproduce images highly similar to originals and effectively generalize to other subjects, significantly outperforming baseline methods with an average improvement of over 10\%. Moreover, our cost analysis reveals that EvoStealer achieves template stealing with negligible computational expenses. Our code and dataset are available at~\url{https://github.com/whitepagewu/evostealer}.