\section{Conclusion}
% 在这篇文章中,我们探索一个攻击场景,提示模板窃取。攻击者能否通过少量的样例图片,窃取生成样例图片风格图片的可泛化的提示模板。为了研究这一攻击场景,我们制作了一个包含50组的数据集,每组数据包含9张DALL-E3生成的图片以及一个提示模板。此外,我们提出了一个基于多模态模型的窃取方法EvoStealer,EvoStealer连接了差分进化算法和多模态模型,不需要任何的微调即可实现模板窃取的目标。我们进行了丰富的试验和分析证明了EvoStealer的有效性和实用性。

This paper investigates prompt template stealing—whether attackers can extract generalizable templates that maintain stylistic consistency using minimal sample images. To explore this scenario, we provide \data, a two-tier benchmark consisting of 50 templates and 450 images, organized into Easy and Hard difficulty levels. We also introduce EvoStealer, a template stealing method that combines differential evolution algorithms with MLLMs, enabling template stealing without the need for fine-tuning. Extensive experiments and analysis validate its effectiveness and practicality. 

