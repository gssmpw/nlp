\newpage
\newpage
\appendix
\onecolumn
%\section{Complete Results}
%\label{app: complete results}

    % \subsection{LLM Prompts}
    % \label{llm prompts}

\section{Additional Results}
\subsection{Accuracy}
In addition to observed improvements in TPR and TPR Disparity values from our experiments in \Cref{tab:main-results}, these experiments also yielded improvements in accuracy, as described by (Table~\ref{tab:comb-results-acc}).
\begin{table*}[t]
    \centering
    %\renewcommand{\arraystretch}{0.9}  % Reduce row height
    % \begin{tabular}{c|c|c|c}
    \begin{tabular}{cccccc}
    \toprule
        % && \multicolumn{1}{c|}{PM Role} & \multicolumn{1}{c}{UX Designer Role} \\
        && \multicolumn{1}{c}{PM Role} & \multicolumn{1}{c}{UX Designer Role} \\
        % \hline
        \textbf{Condition} & \textbf{Method} & \textbf{Accuracy} &   \textbf{Accuracy} \\
        \midrule
        $U$:No LLMs, $P$:\textsc{GPT-4o} & Traditional    &$0.548 \pm 0.003$  & $0.633 \pm 0.005$ \\
        & Two-Ticket     & $0.563 \pm 0.004$  & $ 0.689 \pm 0.005$   \\
        \midrule
        $U$:\textsc{GPT-3.5}, $P$:\textsc{GPT-4o} & Traditional    & $0.543 \pm 0.003$  & $ 0.629 \pm 0.006$  \\
        & Two-Ticket     & $0.551 \pm 0.003$  & $ 0.650 \pm 0.006$  \\
        \midrule
        $U$:\textsc{GPT-4o-mini}, $P$:\textsc{GPT-4o} & Traditional    & $0.554 \pm 0.003$  & $ 0.662 \pm 0.005$  \\
        & Two-Ticket     & $0.561 \pm 0.003$  & $ 0.677 \pm 0.005$   \\
        \bottomrule
    \end{tabular}
    \caption{Accuracy Results for Experiments conducted in \Cref{tab: combined_results_all}. We see improvements in accuracy at the 95\% confidence interval.}
    \label{tab:comb-results-acc}
\end{table*}

\subsection{Thresholds}
The main results in \Cref{sec: empirical experiments} highlight the validity of our theoretical findings: namely, we show that a \textsc{Two-Ticket} scheme can both improve a classifier's true positive rate (TPR) and reduce disparities in TPRs between privileged and unprivileged groups when group membership is unknown to the classifier.
In this section, we present further empirical findings to support our modeling assumptions; in particular, we verify our No False Positives Objective thresholds and \Cref{lem: when threshold stays the same} (constant thresholds). 
We first find that the thresholds that were collected for our experiments in Section~\ref{sec: empirical experiments} produced the following distribution of false positive rates (FPR) on the training set. During our experiments in Python, we encountered an issue when attempting to set the FPR exactly to zero, as this occasionally led to an undefined TPR. To address this, we chose the threshold corresponding to the smallest FPR greater than zero from the output results to produce our main results in Section~\ref{sec: empirical experiments}. While this approach does not strictly adhere to our No False Positives Objective, the resulting FPRs on the training set are still sufficiently close to zero, as seen in Table~\ref{tab: FPR for Section 7}. Overall, these results confirm that our experimental approximation of the No False Positives Objective is sufficiently accurate.
\begin{table}[h!]
    \centering
    \begin{tabular}{|l|c|}
        \hline
        \textbf{Method} & \textbf{False Positive Rate at 95\% CI} \\
        \hline
        Traditional     & $0.004 \pm 0.00026$ \\
        Two-Ticket      & $0.0014 \pm 0.000232$ \\
        \hline
    \end{tabular}
    \caption{False Positive Rates on Training Sets for PM Role}
    \label{tab: FPR for Section 7}

\end{table}

Tables~\ref{tab: Threshold Values 1} and \ref{tab: Threshold Values 2} summarize our experimental results comparing the thresholds of our \textsc{Traditional} and \textsc{Two-Ticket} schemes when both the Hirer and the user employ the same LLM, \textsc{Chat-GPT Mini 4o} (i.e., $L_U = L_H$). While there is a slight, non-significant difference in thresholds between the one-ticket and two-ticket schemes in our main results (Table~\ref{tab:comb-results-acc}), introducing more randomness --- by varying both the train-test splits and the privileged-unprivileged group assignments in each trial --- led to more similar thresholds between the two schemes. This brought the results closer to satisfying the criteria outlined in \Cref{lem: when threshold stays the same}. In contrast, the experiments in Section~\ref{sec: empirical experiments} used fixed privileged-unprivileged group assignments, and thresholds were measured across a range of fixed train-test splits.

\begin{table}[h!]
\centering
\begin{tabular}{|l|c|}
\hline
\textbf{Method} & \textbf{Threshold at 95\% CI} \\
\hline
Traditional     & $84.3 \pm 0.040$ \\
Two-Ticket      & $86.1 \pm 0.077$ \\
Difference      & $1.78 \pm 0.086$ \\
\hline
\end{tabular}
\caption{Threshold Values for PM Role for Fixed Group and Train-Test Assignments}
\label{tab: Threshold Values 1}
\end{table}

\begin{table}[h!]
\centering
\begin{tabular}{|l|c|}
\hline
\textbf{Method} & \textbf{Threshold at 95\% CI} \\
\hline
Traditional     & $85.3 \pm 0.05$ \\
Two-Ticket      & $85.8 \pm 0.33$ \\
Difference      & $0.46 \pm 0.06$ \\
\hline
\end{tabular}
\caption{Threshold Score Values for PM Role for Randomized Group and Train-Test Assignments}
\label{tab: Threshold Values 2}
\end{table}

We hypothesize that the slight difference between our assumptions from \Cref{lem: when threshold stays the same} and our empirical findings arise as a result of the random nature of the privileged-unprivileged group assignments and slight modification of the No False Positives Objective. Specifically, under our approximation of the No False Positives Objective, we find that our output threshold is determined by the top scoring UI/UX resumes (irrelevant of whether they have been manipulated or not). The fixed assignment of these “top” scoring UI/UX resumes to a ``non''-manipulating group would hence result in a different threshold across the \textsc{Traditional} and \textsc{Two-Ticket} scheme. 

We motivate this with a simple example. Consider a scenario with a training set of four applicants applying for a PM Role. In this example, two of the applicants are UI/UX applicants who have been assigned to the unprivileged group, whereas the remaining PM applicants have been assigned to the privileged (who will manipulate their resume) group.  
Suppose our Hirer receives the following distribution of resume scores and their prior role, which it must optimize over under the \textsc{Traditional} scheme:
\begin{figure}[H]
    \centering
    \includegraphics[width=0.8\linewidth]{imgs/example_threshold_change.png}
    \caption{Example of Privileged-Unprivileged Assignment in the Training Set Resulting in Differing Thresholds Between the \textsc{Traditional} and \textsc{Two-Ticket} Schemes.}
    \label{fig:enter-label}
\end{figure}
Now, let us contrast this with the Hirer-manipulated resume scores over which the \textsc{Two-Ticket} scheme optimizes. In this scenario, we assume that all English skills are mapped to a random variable with an expected value of 0.3 and zero standard deviation (for simplicity). Under the \textsc{Two-Ticket} scheme, the scores of the UI/UX applicants “improve” as they have not previously modified their resumes: however, the scores of the PMs remain constant, as they have already modified their resumes.
%[Insert distribution of resume scores and prior roles here under Two Ticket]
As shown, the resume scores that the Hirer optimizes over differ between the \textsc{Traditional} and \textsc{Two-Ticket} schemes. This leads to different ranges of possible “No False Positive” thresholds for each scheme. On the contrary, when strictly adhering to a “No False Positive” objective, we observe that the change in the threshold is smaller. 
To further support this hypothesis empirically, Table~\ref{tab: max UI/UX resume scores} displays the maximum negative resume scores which the classifier optimizes over, regardless of whether the resumes were manipulated. As the scores fluctuate across trials and do not have a zero standard deviation, it is clear that these values are changing. In summary, the discrepancy between our “Constant Threshold” and the empirical results can be attributed to small differences between our theoretical constraints and actual methods—specifically, in approximating the “No False Positives” objective. While our analysis helps explain some of these discrepancies, we note that the magnitude of the differences is small enough that it does not significantly affect the experimental conclusions drawn in \Cref{tab: combined_results_all}.

\begin{table}[h!]
\centering
\begin{tabular}{|l|c|}
\hline
\textbf{Method} & \textbf{Maximum UI/UX Resume Score} \\
\hline
Two-Ticket Scheme & $86.7 \pm 3.6 \times 10^{-14}$ \\
Traditional Scheme & $85.9 \pm 0.07$ \\
Difference & $0.75 \pm 0.07$ \\
\hline
\end{tabular}
\caption{Maximum UI/UX Resume Scores for PM Role}
\label{tab: max UI/UX resume scores}
\end{table}


\section{Empirical Motivation: Additional Results}
\label{app:emp-motivation}
\subsection{Additional Jobs}
We include scores across different models for additional job descriptions. In Figure~\ref{fig:1pass-additional-pm} shows three additional PM jobs and how LLM manipulations to resumes affect the job relevance scores. For PM jobs, some job descriptions generated similar median values for both the qualified and unqualified groups (e.g., Apple Product Manager and Yelp Product Manager). For these jobs, we observed significant boosts in relevance scores for both groups by newer, premium language models.

We also include three additional job descriptions for UX Designers positions (Figure \ref{fig:1pass-additional-ux}). For UX Designers, there's a specific set of skills required that always separates the qualified (UX) from the unqualified (PM) resumes. However, the improvement in scores is particularly stark for the qualified group which would introduce additional disparities. 


\begin{figure*}
    \centering
    \begin{subfigure}{0.3\textwidth}
        \centering
        \includegraphics[width=\textwidth]{imgs/Samsara_PM_1round.pdf}
        \caption{Samsara PM}
    \end{subfigure}%
    \begin{subfigure}{0.3\textwidth}
        \centering
        \includegraphics[width=\textwidth]{imgs/Apple_PM_1round.pdf}
        \caption{Apple PM}
    \end{subfigure}
    ~
    \begin{subfigure}{0.3\textwidth}
        \centering
        \includegraphics[width=\textwidth]{imgs/Yelp_PM_1round.pdf}
        \caption{Yelp PM}
    \end{subfigure}%

    \caption{Resume score distribution of 50 qualified (matching occupation) and 50 unqualified (different occupation) resumes before and after LLM manipulations for more Product Manager Job Descriptions}
    \label{fig:1pass-additional-pm}
\end{figure*}

\begin{figure*}
    \centering
    \begin{subfigure}{0.3\textwidth}
        \centering
        \includegraphics[width=\textwidth]{imgs/Adobe_UX_1round.pdf}
        \caption{Adobe UX}
    \end{subfigure}%
    \begin{subfigure}{0.3\textwidth}
        \centering
        \includegraphics[width=\textwidth]{imgs/Amazon_UX_1round.pdf}
        \caption{Amazon UX}
    \end{subfigure}
    ~
    \begin{subfigure}{0.3\textwidth}
        \centering
        \includegraphics[width=\textwidth]{imgs/I2C_UX_1round.pdf}
        \caption{I2C UX}
    \end{subfigure}%

    \caption{Resume score distribution of 50 qualified (matching occupation) and 50 unqualified (different occupation) resumes before and after LLM manipulations for more UX Designer Job Descriptions}
    \label{fig:1pass-additional-ux}
\end{figure*}
\subsection{Homogenization: Model and Resume Similarity}
While not directly applicable to our theoretical model, many recent works have studied the homogenization of language model outputs. Here we example how similar resumes are before and after LLM manipulation. We observe that experimentally that almost all models increase the similarities of resumes. This suggests that as more applicants turn to LLM tools, their resumes are growing more homogeneous to the employer. 


\begin{figure}
    \centering
    \includegraphics[width=0.6\linewidth]{imgs/resume-similarity.pdf}
    \caption{Distribution of cosine similarities (Sentence Embeddings \textsc{all-MiniLM-L6-v2}) across all pairs of resumes. Compare to no manipulation, almost models increase the similarity of resumes, especially \textsc{Claude-3.5-Sonnet}.}
    \label{fig:resume_sim}
\end{figure}


We also examine similarity with the application of the relevance score system; do the same models find the same candidates good? To measure this, we find the correlation between 100 resume scores for each job description and plot a heat map between models (Figure ~\ref{fig:corr}). We observe a larger correlation between models for UX designer positions than for PM positions. This is likely because there are specific skills unique to UX designers that the unqualified resumes (PM resumes) do not have. We also see models from the same family (e.g. \textsc{DeepSeek-67B} and \textsc{Deepseek-V3}, \textsc{GPT4o} and \textsc{GPT40-mini}) with higher correlation.  This suggests models from the same family may modify resumes in a similar way. 

\begin{figure*}
    \centering
    \begin{subfigure}{0.45\textwidth}
        \centering
        \includegraphics[width=\textwidth]{imgs/DD_PM_model_corr.pdf}
        \caption{DoorDash Product Manager}
    \end{subfigure}%
    \begin{subfigure}{0.45\textwidth}
        \centering
        \includegraphics[width=\textwidth]{imgs/Google_UX_model_corr.pdf}
        \caption{Google UX}
    \end{subfigure}

    \caption{Pearson correlation between model scores of resumes. UX designer job descriptions had higher agreement between models.}
    \label{fig:corr}
\end{figure*}

\subsection{LLM Modifications}
    A key motivation for our work comes from our observations that LLMs can improve the written of quality of resumes, though at varying levels. In addition to \Cref{sec: empirical motivation} in our main manuscript, we here examine the exact disparities and rates of resume improvement.
    
    We found that more resumes were improved by \textsc{ChatGPT-4o} as opposed to \textsc{Mixtral-8x7B}. 75\% of our 520 tested resumes experienced increases in resume scores due to \textsc{ChatGPT-4o} modifications, while only half of the 520 resumes experienced increases in resume scores when modified by \textsc{Mixtral-8x7B}. %These results are illustrated in \Cref{fig: LLM improvement comparison}.

    % These results are also illustrated in the above figure, Figure 3. After all, we see more datapoints generated from our \textsc{ChatGPT-4o} experiments lie above the $y=x$ line than our data-points from our experiments with \textsc{Mixtral-8x7B}.
    \subsection{Qualitative Analysis: Best-Responding Candidates}
    \label{app:best-responding candidates}
    
    As noted in our manuscript, our theoretical and experimental findings assume that job candidates are “best-responding” and can choose to submit the higher scoring resume between their submitted and modified LLM version of their resume. In reality, however, we note that candidates don’t have access to the score system deployed by a Hirer firm. Nevertheless, it is relatively easy to distinguish between the lower and higher-scoring resumes by hand. For example, consider the following snippets from two resumes - the first resume is an unmodified resume, with a score of 79.434, and the second is a modified resume, with an improved score of 81.882. 
    
    The ``professional summary'' section of one unmodified resumes is
    \begin{quote}
    \textit{Have skills in creating a prototype and choosing the methodology for leading the project. I have experience in creating BPMN diagrams. Also, I have worked with different PM tools and can highlight ClickUp as my favorite one.}
    \end{quote}
    
    On the contrary, the ``professional summary'' section of the corresponding manipulated resumes is
    \begin{quote}
        \textit{Skilled Project Manager with expertise in document management for various project stages, creating prototypes, and selecting appropriate project methodologies. Hands-on experience with BPMN diagrams and adept in utilizing different project management tools, with a preference for ClickUp.} 
    \end{quote}
    
    The manipulated resume clearly exhibits more complete sentences and sophisticated vocabulary, indicating a perhaps better-scoring resume. For reference, the complete resumes are shown below.

    \textbf{\textit{\large Unmodified Resume (PM Score: 79.434)}}
    \begin{quote}
        Have skills in maintaining documents for the stages of the project: Brief, Assumptions, WBS (WBS scheme, TTM, Calendar plan), Project Charter, MVP, Requirement Matrix, Risk Register, SRS, RACI, and Status Report.
        
        Have skills in creating a prototype and choosing the methodology for leading the project. I have experience in creating BPMN diagrams. Also, I have worked with different PM tools and can highlight ClickUp as my favorite one.
        
        \textbf{Coordinator of Club of Tolerance} \\
        The Congress of National Communities of Ukraine | 2017—2019 \\
        I held weekly meetings with teenagers, where we discussed various topics and visited exhibitions. I created illustrations and presentations in Canva and Crello.
        
        \textbf{Trainer of Non-Formal Education at EuroCamp and Sources of Tolerance} \\
        The Congress of National Communities of Ukraine | 2017—2020 \\
        I worked as a trainer with children aged 11-17 in an educational training course for teenagers. I created teams from teenagers who did not know each other and transformed them into one family using team-building exercises, progressing through all key stages of team development.
        
        \textbf{Education} \\
        \textbf{IT School Beetroot Academy} \\
        Project Management in IT | 11/2020 - 03/2021
        
        \textbf{Professional Experience} \\
        \textbf{Project Manager in IT} \\
        Epic Story | April 2021 - Present
    \end{quote}
        
    \textbf{\textit{\large Modified GPT-4o Resume Text (Not Formatted, PM Score 81.882)}}
    \begin{quote}
        \textbf{Professional Summary} \\
        Skilled Project Manager with expertise in document management for various project stages, creating prototypes, and selecting appropriate project methodologies. Hands-on experience with BPMN diagrams and adept in utilizing different project management tools, with a preference for ClickUp.
        
        \textbf{Professional Experience} \\
        \textbf{Project Manager in IT} \\
        Epic Story | April 2021 - Present \\
        Maintain comprehensive documentation across multiple stages of the project lifecycle, including Brief, Assumptions, WBS (WBS scheme, TTM, Calendar plan), Project Charter, MVP, Requirement Matrix, Risk Register, SRS, RACI, and Status Reports. Create project prototypes and determine the most effective project management methodologies. Design and implement BPMN diagrams to streamline project workflows. Utilize various PM tools for efficient project execution, with a specific focus on ClickUp.
        
        \textbf{Trainer of Non-Formal Education} \\
        EuroCamp and Sources of Tolerance, The Congress of National Communities of Ukraine | 2017—2020 \\
        Conducted educational training courses for children aged 11-17, focusing on building cohesive teams from diverse groups of teenagers. Facilitated team development through team-building activities, ensuring each group progressed through all key stages of team growth.
        
        \textbf{Coordinator of Club of Tolerance} \\
        The Congress of National Communities of Ukraine | 2017—2019 \\
        Organized and led weekly meetings with teenagers, engaging them in discussions on various topics and organizing visits to exhibitions. Created visual content and presentations using Canva and Crello to enhance meeting engagement and learning.
        
        \textbf{Education} \\
        \textbf{IT School Beetroot Academy} \\
        Certification in Project Management in IT | 11/2020 - 03/2021
        
        \textbf{Key Skills} \\
        Document Management \\
        Project Prototyping \\
        Project Methodology Selection \\
        BPMN Diagram Creation \\
        ClickUp Proficiency \\
        Team Building \& Development \\
        Non-Formal Education Training \\
        Visual Content Creation (Canva, Crello)
    \end{quote}
    
    In sum, as the lower and higher-scoring resumes were distinguishable by inspection in our experiments, we find qualitative evidence to support our ``best-responding'' candidates assumption. 
\begin{figure}
    \centering
    \includegraphics[width=0.5\textwidth]{imgs/google_ux_1round.pdf}
    \caption{ Resume score distribution of 50 qualified and 50 unqualified resumes before and after LLM manipulations for a Google UX Designer Position}
    \label{fig:ux-pm}
\end{figure}

\begin{figure}[!tbp]
    \centering
    \includegraphics[width=0.5\textwidth]{imgs/doordash_pm_1vs2rounds.pdf}
    \caption{Applying LLM manipulations twice did not significantly improve the score of a resume more than a single manipulation.}
    \label{fig:1vs2manipu}
\end{figure}

\section{Experiment Details}
    \label{app:experiment-details}
    \subsection{Dataset Details}
    The dataset we used, the Djinni Recruiting dataset, uses the MIT Licence and adheres to the conditions of fair use~\citep{drushchak-romanyshyn-2024-introducing}. 

    \subsection{Model Details}
    We include the costs of the different models we used. We used Together.ai to query several other models. According to pricing, \textsc{Claude-3.5-Sonnet} and \textsc{GPT-4o} would be the premium models both for input and output tokens. Other cost-efficient models that perform relative-well for improving resume scores are \textsc{Llama3.3-70B-Instruct-Turbo} and \textsc{DeepSeek-68B}. We note that these models (with the exception of DeepSeek recently) are not broadly available to consumers. 
    \begin{table}[]
        \centering
\begin{tabular}{llp{3cm}p{3cm}p{1cm}}
\toprule
\textbf{Model}              & \textbf{Platfrom} & \textbf{Price (1M Input Tokens)} & \textbf{Price (1M Output Tokens)} & \textbf{Our Spend} \\ \hline
Claude-3.5-Sonnet           & Claude.ai         & \$1.50                           & \$7.50                            & \$1.04             \\
DeepSeek-67B                & Together.ai       & \$0.90                           & \$0.90                            & \$0.29             \\
DeepSeek-V3                & DeepSeek       & \$0.55                           & \$2.19                            & \$0.11             \\
GPT-3.5-Turbo-0125          & OpenAI            & \$0.25                           & \$0.50                            & \$0.28             \\
GPT-4o-mini                 & OpenAI            & \$0.075                          & \$0.30                            & \$0.42             \\
GPT-4o-2024-08-06           & OpenAI            & \$1.25                           & \$5.00                            & \$80.04            \\
Mixtral-8x7b-Instruct       & Together.ai       & \$0.60                           & \$0.60                            & \$7.44             \\
Llama3.3-70B-Instruct-Turbo & Together.ai       & \$0.88                           & \$0.88                            & \$0.73            \\ \bottomrule
\end{tabular}
        \caption{Summary of all models we experimented with. For transparency, we also include the total amount we spent on each model. Costs for GPT-4o and Mixtral-8x7b-Instruct are larger due to inital experiments.}
        \label{tab:API-summary}
    \end{table}
    \subsection{LLM Prompt Details}
    Preliminary testing with LLMs showed that they were easily susceptible to hallucinations. For instance, when we prompted the model with our \textit{job-specific LLM prompt} (described further below), it frequently fabricated details about project management tools and methodologies that the candidate had not mentioned in their original resume.
    \linebreak
    To empirically assess the susceptibility of LLMs to hallucinations, we tested resume modifications using a prompt designed to improve resumes based on a specific job description. The job description used in our experiments, shown below, was drawn from an example Project Manager role in the Djinni dataset.\citep{drushchak-romanyshyn-2024-introducing}:


        \textbf{\textit{Job-Specific LLM Prompt}:}
        \begin{quote}
            ``Can you tailor my resume to this job description? 
            
            `A commitment to collaborative problem solving, agile thinking, and adaptability is essential. We are looking for a candidate who is able to balance a fast moving and changing environment with the ability to identify, investigate, and predict project risks Recruiting stages: HR interview, Tech interview **Core Responsibilities:** - Manage the full project life cycle including requirements gathering, creation of project plans and schedules, obtaining and managing resources, and facilitating project execution, deployment, and closure. - In cooperation with Technical Leads create and maintain comprehensive project documentation. - Manage Client expectations, monitor and increase CSAT level; - Plan, perform and implement process improvement initiatives. - Organize, lead, and facilitate cross-functional project teams. - Prepare weekly and monthly project status reports **What you need to Succeed:** - 1+ Year of dedicated Project Management in a production environment - Excellent organization and communication skills and the ability to communicate effectively with customers and co-workers. - Strong understanding of a \textcolor{blue}{Project Management Methodology (SDLC, Agile, Waterfall, etc.)} - Creative mind with the ability to think outside-of-the-box. - The ability to manage multiple projects simultaneously - Experience with Jira or similar project management tool - Upper-intermediate level of English is a must.' 
            
            Please wait for me to enter my resume.''
        \end{quote}

     We then tested the aforementioned \textit{Job-Specific LLM Prompt} with \textsc{ChatGPT-4o} on the following resume of a UI/UX designer:
   
        \textbf{Original Input Resume Text:} 
        \begin{quote}
            Possession of Adobe Photoshop, Adobe-AI, skills of working with a graphic tablet; UI/UX ;  Figma, Zeplin, Sketch, MarvelApp, Invision; 2D graphics (understanding of pixel and vector graphics and their specifics); 2D Design; 3D Max; Ability to work in different styles; Ability to draw by hand; Creation of design banners, layouts, logos, frameworks and prototypes for websites and mobile applications;Testing of games; Making presentations in Power Point;
            Availability of portfolio;
            Bug tracking systems and project management tools: Trello and Jira.
            I'm self-motivated artist with a strong skills at game industry. I have more 5 years experience in game company, and now, I want to broaden my world with interesting projects, to share own experience and to acquire new skills and knowledge.
            So, my goal is putting into practice somebody's dreams, in so doing, working with pleasure.
            I would like to work at company with friendly team, interesting projects and a decent wage.
        \end{quote}
        
        \textbf{\textit{Job-Specific LLM Prompt} Modified Resume:}
        \begin{quote}
            Your Name \\
            Your Address \\
            City, State, Zip \\
            Your Email \\
            Your Phone Number \\
            LinkedIn Profile or Portfolio Link \\
        
            \textit{Professional Summary}
            
            Self-motivated project manager with over 5 years of experience in the game industry, specializing in UI/UX design and project management. Adept at managing the full project life cycle, fostering collaboration, and implementing process improvements. Seeking to leverage strong organizational and communication skills in a dynamic environment.
        
            \textit{Core Competencies}
            
            \begin{itemize}
                \item \textcolor{blue}{\textit{Project Management Methodologies:} Agile, Waterfall, SDLC}
                \item \textit{Tools:} Jira, Trello, Figma, Adobe Photoshop, Adobe Illustrator
                \item \textit{Communication \& Collaboration:} Excellent ability to manage client expectations and facilitate cross-functional teams
                \item \textit{Process Improvement:} Proven track record in planning and implementing initiatives that enhance efficiency
                \item \textit{Risk Management:} Skilled in identifying, investigating, and predicting project risks
                \item \textit{Creative Problem Solving:} Ability to think outside the box and balance multiple projects simultaneously
            \end{itemize}
        
            \textit{Professional Experience}
            
            \textit{Your Current Job Title} \\
            Your Current Company Name, City, State \hfill Month, Year to Present \\
            Managed the full project life cycle, from requirements gathering to project execution and closure, for multiple UI/UX design projects. \\
            Collaborated with technical leads to create and maintain comprehensive project documentation, ensuring clarity and alignment throughout the project. \\
            Organized and led cross-functional project teams to facilitate successful project delivery, enhancing customer satisfaction and increasing CSAT levels. \\
            Prepared and presented weekly and monthly project status reports to stakeholders, ensuring transparency and effective communication. \\
        
            \vspace{0.5em}
            \textit{UI/UX Designer} \\
            Previous Company Name, City, State \hfill Month, Year to Month, Year \\
            Created design banners, layouts, logos, frameworks, and prototypes for websites and mobile applications, contributing to the overall project goals. \\
            Conducted testing for game projects, utilizing bug tracking systems to enhance project quality and performance. \\
            Developed and delivered engaging presentations in PowerPoint to communicate design concepts and project updates.
        
            \textit{Education}
            
            \textit{Your Degree} \\
            Your University Name, City, State \hfill Year of Graduation
        
            \textit{Portfolio}
            
            Available upon request or at \textit{link}.
        
            \textit{Languages}
            
            English (Upper-intermediate)
        
            \textit{Interests}
            
            Collaborative design projects \\
            Continuous learning and skill development
    \end{quote}
    
    As noted above, \textsc{ChatGPT-4o} fabricated details about the individual’s technical project management skills. Specifically, it borrowed language (highlighted in blue) from the input prompt, which referenced the job description, and then inserted it into the outputted resume. These skills were mentioned in the job description but were not included in the individual’s original resume.

    \subsubsection{Final Anti-Hallucination LLM Prompt}
    To mitigate hallucinatiomns, we used he following prompt for our experiments. Additionally, we found that a sequential messaging style (without interruptions from the LLM) effectively reinforced the importance of preventing hallucinations to the LLM.

    \begin{table}[ht]
    \caption{Anti-Hallucination LLM Prompt} \label{sample-table2}
    \begin{center}
    \begin{tabular}{ll}
    \textbf{Message No.}  &\textbf{Text} \\
    \hline \\
    1.         &``Can you improve my resume?''\\
    2.             &``I will be punished if I lie about my qualifications, so it is imperative that the new resume does not \\ &add any facts that are not in the original resume.'' \\
    3.             &``Also, please do not include any additional notes or explanation: I just want the text of the \\&new resume.''\\
    4.             &``This is my resume: [\textit{input resume}].''\\
    \end{tabular}
    \end{center}
    \end{table}

    \textbf{\textit{Anti-Hallucination Prompt} Output Resume:}
    \begin{quote}
        \textit{Your Name} \\
        \textit{Contact Information:} \\
        \textit{Professional Summary} \\
        Self-motivated artist with over 5 years of experience in the game industry. Skilled in graphic design and UI/UX, I am eager to broaden my horizons through engaging projects, sharing my expertise, and acquiring new skills. \\
    
        \textit{Skills} 
        \begin{itemize}
            \item Proficient in Adobe Photoshop and Adobe Illustrator
            \item UI/UX Design
            \item Experienced with Figma, Zeplin, Sketch, MarvelApp, and InVision
            \item Understanding of pixel and vector graphics
            \item 2D and 3D Design (3D Max)
            \item Hand-drawing capabilities
            \item Creation of design banners, layouts, logos, frameworks, and prototypes for websites and mobile applications
            \item Game testing experience
            \item Proficient in creating presentations in PowerPoint
            \item Familiarity with bug tracking systems and project management tools: Trello and Jira
        \end{itemize}
        \textit{Portfolio} \\
        Available upon request. \\

        \textit{Objective} \\
        To contribute to exciting projects within a collaborative team environment while achieving a balanced professional development and fair compensation.
    \end{quote}

    We also tested this \textit{Anti-Hallucination Prompt} on our selected test resume (see ``Original Input Resume Text'' on the previous page): the subsequent outputted resume is shown above. When compared to the output of our \textit{Job-Specific LLM Prompt}, we found that our \textit{Anti-Hallucination Prompt} successfully did not fabricate information that was not previously mentioned in the original resume. Instead, it solely improved the written quality of resumes by adding a ``professional summary section'' and reorganizing the individual's existing skills in a logical manner. Upon manual inspection of a random subset of outputs from our \textit{Anti-Hallucination Prompt}, we further confirmed that hallucinations were indeed reduced by this prompt.

    \subsection{Evaluation (Scoring) of Resumes}
    To evaluate our (potentially LLM-manipulated) resumes, we used Resume-Matcher to compare them against two Project Manager job descriptions, which we selected from job boards on LinkedIn \cite{doordash-product-manager} and Google \cite{google-ux}.
 We selected these two descriptions due to their detailed explanation of their respective role-related responsibilities and their lexical dissimilarity to each other.
 
\textbf{\textit{Job Description 1 (Product Manager (Multiple Levels) @ DoorDash)}:}
\begin{quote}
    \textbf{About the Team:} At DoorDash, we're redefining the future of on-demand delivery. To do this, we're building a world-class product organization, in which each of our product managers plays a critical role in helping to define and execute our vision to connect local delivery networks in cities all across the world.

    \textbf{About The Role:} Product Managers at DoorDash require a sharp consumer-first eye, platform thinking, and strong cross-functional collaboration. As a Product Manager at DoorDash, you will own the product strategy and vision, define the product roadmap and alignment, and help drive the execution. You will be working on mission-critical products that shape the direction of the company. You will report into one of the following pillars: Merchant, Consumer, Operational Excellence, Ads, Logistics, or New Verticals. This role is a hybrid of remote work and in-person collaboration.
    
    \textbf{You’re Excited About This Opportunity Because You Will…}
    \begin{enumerate}
        \item Drive the product definition, strategy, and long-term vision. You own the roadmap.
        \item Work closely with cross-functional teams of designers, operators, data scientists, and engineers.
        \item Communicate product plans, benefits, and results to key stakeholders, including the leadership team.
    \end{enumerate}
    
    \textbf{We’re Excited About You Because…}
    \begin{enumerate}
        \item You have 5+ years of Product Management industry experience.
        \item You have 4+ years of user-facing experience in industries such as eCommerce, technology, or multi-sided marketplaces.
        \item You have proven abilities in driving product strategy, vision, and roadmap alignment.
        \item You’re an execution powerhouse.
        \item You have experience presenting business reviews to senior executives.
        \item You have empathy for the users you build for.
        \item You are passionate about DoorDash and the problems we are solving for.
    \end{enumerate}
    
    \textbf{About DoorDash:} At DoorDash, our mission to empower local economies shapes how our team members move quickly, learn, and reiterate in order to make impactful decisions that display empathy for our range of users—from Dashers to merchant partners to consumers. We are a technology and logistics company that started with door-to-door delivery, and we are looking for team members who can help us go from a company that is known for delivering food to a company that people turn to for any and all goods. DoorDash is growing rapidly and changing constantly, which gives our team members the opportunity to share their unique perspectives, solve new challenges, and own their careers. We're committed to supporting employees’ happiness, healthiness, and overall well-being by providing comprehensive benefits and perks including premium healthcare, wellness expense reimbursement, paid parental leave, and more.
    
    \textbf{Our Commitment to Diversity and Inclusion:} We’re committed to growing and empowering a more inclusive community within our company, industry, and cities. That’s why we hire and cultivate diverse teams of people from all backgrounds, experiences, and perspectives. We believe that true innovation happens when everyone has room at the table and the tools, resources, and opportunity to excel.
    
    \textbf{Statement of Non-Discrimination:} In keeping with our beliefs and goals, no employee or applicant will face discrimination or harassment based on: race, color, ancestry, national origin, religion, age, gender, marital/domestic partner status, sexual orientation, gender identity or expression, disability status, or veteran status. Above and beyond discrimination and harassment based on 'protected categories,' we also strive to prevent other subtler forms of inappropriate behavior (i.e., stereotyping) from ever gaining a foothold in our office. Whether blatant or hidden, barriers to success have no place at DoorDash. We value a diverse workforce – people who identify as women, non-binary or gender non-conforming, LGBTQIA+, American Indian or Native Alaskan, Black or African American, Hispanic or Latinx, Native Hawaiian or Other Pacific Islander, differently-abled, caretakers and parents, and veterans are strongly encouraged to apply. Thank you to the Level Playing Field Institute for this statement of non-discrimination.
    
    Pursuant to the San Francisco Fair Chance Ordinance, Los Angeles Fair Chance Initiative for Hiring Ordinance, and any other state or local hiring regulations, we will consider for employment any qualified applicant, including those with arrest and conviction records, in a manner consistent with the applicable regulation. If you need any accommodations, please inform your recruiting contact upon initial connection.
\end{quote}

\textbf{\textit{Job Description 2 (Staff UX Designer @ Google)}}
\begin{quote}
    \textbf{Minimum Qualifications}
    \begin{enumerate}
        \item Bachelor's degree in Design, Human-Computer Interaction, Computer Science, a related field, or equivalent practical experience.
        \item 8 years of experience in product design or UX.
        \item Experience with industry standard design tools (e.g., Photoshop, Illustrator, Sketch, InVisio, Figma, Principle, etc.).
        \item Include a portfolio, website, or any other relevant link to your work in your resume (providing a viewable link or access instructions).
    \end{enumerate}
    
    \textbf{Preferred Qualifications:}
    \begin{enumerate}
        \item Experience creating and maintaining robust, coherent design systems that are usable across various devices and platforms.
        \item Experience shaping processes to establish and sustain a scalable and consistent design language.
        \item Proven track record of exploring and implementing innovative communication modes that are delightful, understandable, and accessible for diverse user groups, including designers, developers, and end-users.
        \item Demonstrated ability to provide a strong design perspective with fluency in native app patterns and user experiences.
        \item Strong collaboration skills within cross-functional product teams, including working with engineering, writers, editors, and researchers to refine and validate design decisions.
    \end{enumerate}
    
    \textbf{About the Job:}
    At Google, we follow a simple but vital premise: "Focus on the user and all else will follow." Google’s Interaction Designers take complex tasks and make them intuitive and easy-to-use for billions of people around the globe. Throughout the design process—from creating user flows and wireframes to building user interface mockups and prototypes—you’ll envision how people will experience our products, and bring that vision to life in a way that feels inspired, refined, and even magical.
    
    Google User Experience (UX) is made up of multi-disciplinary teams of UX Designers, Researchers, Writers, Content Strategists, Program Managers, and Engineers: we care deeply about the people who use our products. The UX team plays an integral part in gathering insights about the attitudes, emotions, and behaviors of people who use our products to inspire and inform design. We collaborate closely with each other and with engineering and product management to create industry-leading products that deliver value for the people who use them, and for Google’s businesses.
    
    As an Interaction Designer, you’ll rely on user-centered design methods to craft industry-leading user experiences—from concept to execution. Like all of our UX jobs, you’ll collaborate with your design partners to leverage and evolve the Google design language to build beautiful, innovative, inspired products that people love to use.
    
    Labs is a group focused on incubating early-stage efforts in support of Google’s mission to organize the world’s information and make it universally accessible and useful. Our team exists to help discover and create new ways to advance our core products through exploration and the application of new technologies. We work to build new solutions that have the potential to transform how users interact with Google. Our goal is to drive innovation by developing new Google products and capabilities that deliver significant impact over longer timeframes.
    
    The US base salary range for this full-time position is $168,000-$252,000 + bonus + equity + benefits. Our salary ranges are determined by role, level, and location. The range displayed on each job posting reflects the minimum and maximum target salaries for the position across all US locations. Within the range, individual pay is determined by work location and additional factors, including job-related skills, experience, and relevant education or training. Your recruiter can share more about the specific salary range for your preferred location during the hiring process.
    
    Please note that the compensation details listed in US role postings reflect the base salary only, and do not include bonus, equity, or benefits. Learn more about benefits at Google.
    
    \textbf{Responsibilities:}
    Influence cross-functional stakeholders to gain support for design strategies, collaborating from early-stage concept development to iteration and execution.
    Drive the creation of innovative design solutions that address user needs, business objectives, and industry trends while furthering business outcomes.
    Build and iterate on Figma prototypes to communicate ideas, user journeys, and decision points, while exploring rapid visual design styles and interactions.
    Apply user-centered design principles, integrating UX research insights and AI-first interactions to create unique, multi-modal user experiences that drive product adoption.
    Provide direction to UX designers, manage priorities, dependencies, and stakeholders effectively, and independently drive work toward key milestones.
\end{quote}

\section{Examples}
    We provide a simple example to illustrate the intuition behind the \textsc{two-ticket} scheme.

    \begin{example}
        \label{ex: running example 1}
        Consider some joint distribution \(\mathcal{D}\) over feature vectors, groups, and labels, as well some scoring function\(s\) such that \(s(\bm{X}) \sim \textrm{Uni}([0,10])\) and \((H \mid \bm{X}) = \one[s(\bm{X}) \geq 5]\). 
        %\jack{should these be further specified?}

        \begin{figure}[H]
            \centering
            \includegraphics[width=1.0\linewidth]{imgs/1D_Combined_Example.png}
            \caption{Example resume score distributions for Group \(P\) and Group \(U\). Solid regions correspond to truly represent candidates, while the hatched regions represent truly unqualified candidates.}
            \label{fig: combined intuition example}
        \end{figure}

        % \begin{figure}[H]
        %     \centering
        %     \includegraphics[width=0.9\linewidth]{Diagrams/1D Running example 1.png}
        %     \caption{Example population density plot of unmanipulated scores for Group \(P\) and Group \(U\). Darker regions correspond to truly qualified candidates, lighter regions correspond to truly unqualified candidates.}
        %     \label{fig:enter-label}
        % \end{figure}
        \begin{enumerate}
            \item If the Hirer were to receive the unmanipulated scores, then the threshold \(\tau = 5\) would achieve perfect classification for both groups, as shown in the leftmost plot.

            \item However, suppose that the groups have unequal access to LLMs. Namely, suppose Group \(P\) has access to an LLM \(L_P\) that increases scores by 1, while \(U\) has no access to LLMs. Since the Hirer cannot discriminate between scores from different groups differently, no threshold can 
            simultaneously classify both groups correctly. The interval of undominated thresholds is \([5,6]\): however, any threshold in this range must accept unqualified candidates from Group \(P\) or reject qualified candidates from Group \(U\). This harms both fairness and overall accuracy. Under the No False Positives Objective, the optimal threshold is \(\tau^* = 6\) with \(\TPR_P = 1\) but \(\TPR_U = 0.8\), as shown in the middle plot.

            \item To mitigate this disparity, consider applying Hirer LLM \(L_H\) where \(L_H(L_P(s(\bm{x}))) = s(\bm{x}) + 1\) and \(L_H(L_U(s(\bm{x}))) = s(\bm{x}) + 1\). The idea is that \(L_H\) improves Group \(U\)'s scores while minimally affecting Group \(P\)'s scores, restoring equality of post-manipulation score distributions between the two groups. This allows for perfect classification between the two Groups again with a new threshold \(\tau^* = 6\), as shown in the rightmost plot.
        \end{enumerate}
        

        % \begin{figure}[H]
        %     \centering
        %     \includegraphics[width=0.9\linewidth]{Diagrams/1D Running example 2.png}
        %     \caption{Example population density plot of Applicant-manipulated scores for Group \(P\) and Group \(U\). Darker regions correspond to truly qualified candidates, lighter regions correspond to truly unqualified candidates.}
        %     \label{fig:enter-label}
        % \end{figure}


        %\BOGO{should loss in fairness/accuracy be more dramatic?}
            
        % For \(m=1, n=1\), consider discrete uniform distribution over the four candidates given in \Cref{tab: running example 1 points}, with \(L_P([x_1, x_2]) = [8, x_2]\) and \(L_U([x_1, x_2]) = [4, x_2]\).

        % \begin{table}[H]
        %     \centering
        %     \begin{tabular}{l|l|l|l}
        %     \(\bm{x}\) & \(Y\) & \(L_P(\bm{x})\) & \(L_U(\bm{x})\) \\ \hline
        %     \((3,6)\)  & 0     & \((8,6)\)       & \((4,6)\)       \\
        %     \((5,4)\)  & 0     & \((8,4)\)       & \((5,4)\)       \\
        %     \((4,7)\)  & 1     & \((8,7)\)       & \((5,7)\)       \\
        %     \((9,8)\)  & 1     & \((9,8)\)       & \((9,8)\)      
        %     \end{tabular}
        %     \caption{}
        %     \label{tab: running example 1 points}
        % \end{table}
        % \BOGO{table reference is bad and needs caption}

        % With scoring function \(s([x_1, x_2]) = \frac{1}{2}x_1 + x_2\), the following thresholds yield the following true and false positive rates

        % (\BOGO{haven't defined TNR formally})

        % A fundamental tradeoff: any threshold rejects qualified candidates from group \(U\) or accepts unqualified candidates from group \(P\), or both.

    \end{example}
    
        % \begin{example}
        %     \label{ex: running example 2}
        %     Continuing our running example from \Cref{ex: running example 1}, we consider now applying \(L_H\) where \(L_H(L_P(s(\bm{x}))) = s(\bm{x}) + 1\) and \(L_H(L_U(s(\bm{x}))) = s(\bm{x}) + 1\). The idea is that \(L_H\) improves Group \(U\)'s scores while minimally affecting Group \(P\)'s scores, restoring equality of post-manipulation score distributions between the two. This allows for perfect classification between the two Groups again with a new threshold \(\tau^* = 6\).
            
        %     \begin{figure}[H]
        %         \centering
        %         \includegraphics[width=0.9\linewidth]{Diagrams/1D Running example 3.png}
        %         \caption{Example population density plot of Applicant and Hirer-manipulated scores for Group \(P\) and Group \(U\). Darker regions correspond to truly qualified candidates, lighter regions correspond to truly unqualified candidates.}
        %         \label{fig:enter-label}
        %     \end{figure} 
        % \end{example}

\section{Complete Proofs}
    \label{sec: appendix proof}

\theoremOne*
    % Full proof
    \begin{proof}
        Fix \(\bm{x} \in \mathcal{X}\). If \(f_\tau(\bm{x}) = 1\), then trivially \(\Delta(\bm{x}) = 0\). We therefore focus on the case that \(f_\tau(\bm{x}) = 0\). In this case, note that
        \begin{equation*}
            \P_{L_g}(f_\tau(\bm{x}'_g) = 1) = \E_{L_g}[f_\tau(L_g(\bm{x}))]. 
        \end{equation*}
        Since \(s\) is non-decreasing, \(f_\tau(\bm{x}) = \one[s(\bm{x}) \geq \tau]\) is also non-decreasing. Since \(L_P \succeq L_U\), \(L_P(\bm{x})\) stochastically dominates \(L_U(\bm{x})\), so by \Cref{lem: stochastic dominance utility},
        \begin{align*}
            \E_{L_g}[f_\tau(L_P(\bm{x}))] &\geq \E_{L_g}[f_\tau(L_U(\bm{x}))] 
        \\  \implies
            \P_{L_g}(f_\tau(\bm{x}'_P) = 1) &\geq 
            \P_{L_g}(f_\tau(\bm{x}'_U) = 1)
        \\  \implies
            \P_{L_g}(f_\tau(\bm{x}'_P) = 1)- \P_{L_g}(f_\tau(\bm{x}'_U) = 1)&\geq 0.
        \\  \implies
            \Delta(\bm{x}) &\geq 0.
            \qedhere
        \end{align*}
    \end{proof}

\corollaryOne*
    % \begin{proof}[Proof]
    %     Observe that \(\Delta_{\TPR} = \E_{L_P, L_U, \bm{x}}[\Delta(\bm{x}) \mid Y = 1]\) and apply \Cref{thm: hiring outcome disparity}. 
    %     % by taking the expectation of the individual outcome disparity over the candidate distribution conditioned on \(Y = 1\).
    % \end{proof}

    \begin{proof}
        Observe that \(\Delta_{\TPR} = \E_{\bm{x}}[\Delta(\bm{x}) \mid Y = 1]\). Since $L_P \succeq L_U$, \Cref{thm: hiring outcome disparity} implies that for all $\bm{x} \in \mathcal{X}$, $\Delta(\bm{x}) \geq 0$. Thus, \(\Delta_{\TPR} \geq 0\).
        
        % \begin{align*}
        %     \Delta_{\TPR} &= 
        %     \E_{L_P, L_U, \bm{x}}[\Delta(\bm{x}) \mid Y = 1]
        % \\
        %     \geq 0
        % \end{align*}
    
        % Hence, we see
        % \begin{align*}
        %     \forall x \in \chi\; \Delta(x) &\geq 0
        %     \\
        %     \implies \mathbb{E}_{x \sim \chi}[\Delta(x)]  &\geq 0\\
        %     \implies \mathbb{E}_{x \sim \chi}[\Delta(x) | Y=1]  &\geq 0
        % \end{align*}
        % \lc{Lee: I don't think the second transition (second arrow in the above) is correct, but it should be correct without it.}
        % As $L_P, L_U$ and $\chi$ are all mutually independent, we further see:
        % \begin{align*}
        %     \implies \mathbb{E}_{x \sim \chi, L_P, L_U}[\Delta(x)]  &\geq 0 \\
        %     \implies D_{TPR} &\geq 0
        % \end{align*}
    \end{proof}
    
\lemmaOne*
    \begin{proof}
        Assuming a continuous distribution over scores, the minimum threshold that achieves the No False Positives Objective is equal to the maximum score achievable by a candidate with true label \(Y = 0\). That is,=
        \begin{align*}
            \tau^{*(k)} = \max s(\bm{x}'')
        \end{align*}
        where
        \begin{align*}
            &\bm{x}'' \in \{\bm{x}', L_H^{(k)} (\bm{x}')\} 
        \\  \text{ and }
            &\bm{x}' \in \{\bm{x}, L_P(\bm{x}, L_U(\bm{x}))\}
        \\  \text{and }
            &\bm{x} \sim \D \mid Y = 0.
        \end{align*}
        By leveraging the conditional independence \(L_H(\bm{x}'_g)\) and \(\bm{x}'_g\) given \(\bm{x}\) (the same logic as the proof of \Cref{lem: expression for two-ticket acceptance probability}), we may equivalently write
        \begin{equation*}
            \tau^{*(k)} = \max\paren{M, M_P, M_U, M_H^{(k)}}.
        \end{equation*}
        where
        \begin{align*}
            M &\coloneqq \max_{x \sim \D \mid Y = 0} s(\bm{x}) \\
            M_g &\coloneqq \max_{L_g, x \sim \D \mid Y = 0} s(L_g(\bm{x})) \\
            M_H^{(k)} &\coloneqq \max_{L_H^{(k)}, x \sim \D \mid Y = 0} s(L_H^{(k)}(\bm{x})),
        \end{align*}
        
        We show that, under the \namecref{lem: when threshold stays the same}'s condition, \(M_H^{(k)}\) is irrelevant to this expression. Suppose \(L_P \succeq L_H^{(k)}\). Then, fixing \(\bm{x} \in \mathcal{X}\), \(L_P(\bm{x})\) stochastically dominates \(L_H^{(k)}(\bm{x})\). Since \(s\) is non-decreasing, \(s(L_P(\bm{x}))\) stochastically dominates \(s(L_H^{(k)}(\bm{x}))\). Thus, \(\max_{L_P} s(L_P(\bm{x})) \geq \max_{L_H^{(k)}} s(L_H^{(k)}(\bm{x}))\). Taking the maximum over \(\bm{x} \sim \D \mid Y = 0\) yields
        \begin{equation*}
            M_P = \max_{L_P, x \sim \D \mid Y = 0} s(L_P(\bm{x})) 
            \geq
            \max_{L_H^{(k)}, x \sim \D \mid Y = 0} s(L_H^{(k)}(\bm{x})) = M_H^{(k)}.
        \end{equation*}
        Since \(M_P \geq M_H^{(k)}\), we have \(\tau^{*(k)} = \max(M, M_P, M_U)\). This expression is independent of \(k\), so \(\tau^{*(1)} = \tau^{*(2)}\).
        
        
        
        % , we may rewrite this expression as
        % \begin{align*}
        %     \tau^{*(k)} = \max s(\bm{z})
        % \end{align*}
        % where \(\bm{z} \in \{\bm{x}, L_H^{(k)}(\bm{x}), L_P(\bm{x}), L_U(\bm{x})\}\) and \(\bm{x} \sim \D \mid Y = 0\).

        % Suppose for some \(g \in \{P, U\}\), \(L_g \succeq L_H^{(1)}, L_H^{(2)}\). Then for \(k \in \{1,2\}\) and \(\bm{x} \in \mathcal{X}\), \(L_g(\bm{x}) \succeq L_H^{(k)}(\bm{x})\). Since \(s\) is non-decreasing, \(s(L_g(\bm{x})) \succeq s(L_H^{(k)}(\bm{x}))\). Thus, \(\max s(L_g(\bm{x})) \geq \max s(L_H^{(k)}\). It follows that the Hirer's LLM plays no role in determining the threshold \(\tau^{*(k)}\). Thus, the same threshold achieves the No False Positive Objective for both Hiring Scheme 1 and 2. 
    \end{proof}

    % % Proof sketch
    % \begin{proof}[Proof Sketch]
    %     Assuming a continuous distribution over scores, \(\tau^{*(k)}\) is the maximum score achievable by a candidate with true label \(0\). The threshold-consistency condition guarantees that neither \(L_H^{(1)}\) nor \(L_H^{(2)}\) are strong enough to affect this maximum score.
    % \end{proof}
        
    % % Full proof
    % \begin{proof}
    %     It is not difficult to see that, assuming a continuous distribution over scores, 
    %     \(\tau^{*(k)}\) is the maximum score achievable by a candidate with true label \(Y = 0\). That is,
    %     \begin{align*}
    %         \tau^{*(k)} = \max s(\bm{x}''_g)
    %     \end{align*}
    %     where
    %     \begin{align*}
    %         &\bm{x}''_g \in \{\bm{x}'_g, L_H^{(k)} (\bm{x}'_g)\} 
    %     \\  \text{ and }
    %         &\bm{x}'_g \in \{\bm{x}, L_g(\bm{x})\}
    %     \\  \text{and }
    %         &\bm{x} \sim \D \mid Y = 0.
    %     \end{align*}
    %     The threshold-consistency condition guarantees exactly that \(\bm{x}''_g = \bm{x}_g'\) suffices to achieve the maximum. In this case, the Hirer's LLM plays no role in determining the threshold. Thus, the same threshold achieves the No False Positive Objective for both Hiring Scheme 1 and 2. 
    % \end{proof}

\lemmaTwo*
    % Full Proof
    \begin{proof}
        If \(s(\bm{x}) \geq \tau\), then trivially \(\P_{L_g, L_H}(f_\tau(\bm{x}''_g) = 1) = 1\). We therefore focus on the case that \(f_\tau(\bm{x}) = 0\). In this case, note that
        \begin{equation*}
            \P_{L_g, L_H}(f_\tau(\bm{x}''_g) = 0)
            = 
            \P_{L_g, L_H}\paren{
                s(\bm{x}''_g) < \tau
            }
            = 
            \P_{L_g, L_H}\paren{
                s(\bm{x}'_g) < \tau \cap
                s(L_H(\bm{x}'_g)) < \tau
            }.
        \end{equation*}
        Observe from the definition of LLM manipulation that \(L_H(\bm{x}'_g)\) and \(\bm{x}'_g\) are conditionally independent given \(\bm{x}\). Thus,
        \begin{equation*}
            \P_{L_g, L_H}(f_\tau(\bm{x}''_g) = 0)
        = 
            \P_{L_g}\paren{
                s(\bm{x}'_g) < \tau
            }
            \P_{L_g, L_H}\paren{
                s(L_H(\bm{x}'_g)) < \tau
            }
            .
        \end{equation*}
        Furthermore, observe that \(L_H(\bm{x}'_g)\) is equal in distribution to \(L_H(\bm{x})\). We obtain
        \begin{equation*}
            \P_{L_g, L_H}(f_\tau(\bm{x}''_g) = 0)
            = 
            \P_{L_g}\paren{
                s(L_g(\bm{x}_g)) < \tau
            }
            \P_{L_H}\paren{
                s(L_H(\bm{x}_g)) < \tau
            }
            .
        \end{equation*}
        Taking the complement yields the lemma.
    \end{proof}

\theoremTwo*
    % \begin{proof}[Proof Sketch]
    %     Using \Cref{lem: expression for two-ticket acceptance probability}, we can compute that
    %     \begin{align*}
    %         \frac{\Delta^{(2)}(\bm{x})}{\Delta^{(1)}(\bm{x})} &=
    %         \frac{
    %         \P_{L_H^{(2)}}(
    %             s(L_H^{(2)}(\bm{x})) < \tau^{*(2)}
    %         )
    %         }{
    %         \P_{L_H^{(1)}}(
    %             s(L_H^{(1)}(\bm{x})) < \tau^{*(2)}
    %         )
    %         }
    %         .
    %     \end{align*}
    %     Since \(L_H^{(2)}\) dominates \(L_H^{(1)}\), for \(\tau^{*(1)} = \tau^{*(2)}\), \Cref{lem: stochastic dominance utility} shows that the numerator is less than the denominator.
    % \end{proof}

    % Full proof
    \begin{proof}
        For convenience, let \(\tau = \tau^{*(1)}  = \tau^{*(2)}\) be the common threshold that achieves the No False Positives Objective.  Fix \(\bm{x} \in \mathcal{X}\). If \(f^{(1)}_\tau(\bm{x}) = f^{(2)}_\tau(\bm{x}) = 1\), then trivially \(\Delta^{(1)}(\bm{x}) = \Delta^{(2)}(\bm{x}) = 0\). We therefore focus on the case that \(f^{(1)}_\tau(\bm{x}) = f^{(2)}_\tau(\bm{x}) = 0\). By \Cref{lem: expression for two-ticket acceptance probability},
        \begin{align*}
            \Delta^{(k)}(\bm{x})
            &= \P_{L_P, L_H^{(k)}}(f_\tau(\bm{x}''_P) = 1) - 
            \P_{L_U, L_H^{(k)}}(f_\tau(\bm{x}''_U) = 1)
        \\  &= 
            \P_{L_H^{(k)}}\paren{
                s(L_H^{(k)}(\bm{x})) < \tau
            } \cdot d(\bm{x}),
        % \\
        %     &\paren{
        %         \P_{L_U}\paren{
        %         s(L_U(\bm{x}) < \tau
        %         }
        %         - 
        %         \P_{L_P}\paren{
        %         s(L_P(\bm{x}) < \tau
        %         }
        %     }
        \end{align*}
        where 
        \(
        \displaystyle
        d(\bm{x}) = 
        \P_{L_U}\paren{
            s(L_U(\bm{x}) < \tau
        }
        - 
        \P_{L_P}\paren{
            s(L_P(\bm{x}) < \tau
        }
        \).
        
        Observe that \(d(\bm{x})\) does not depend on the Hiring Scheme \(k\). Thus,
        \begin{gather*}
            \Delta^{(2)}(\bm{x}) - \Delta^{(1)}(\bm{x})
        =
            - \delta (\bm{x}) \cdot d(\bm{x}),
        \end{gather*}
        where
        \(\displaystyle
            \delta(\bm{x}) = 
            \P_{L_H^{(1)}}(
                s(L_H^{(1)}(\bm{x})) < \tau
            ) - 
            \P_{L_H^{(2)}}(
                s(L_H^{(2)}(\bm{x})) < \tau
            )
        \).

        
        Since \(L_P \succeq L_U\) by assumption, by \Cref{thm: hiring outcome disparity}, \(d(\bm{x}) \geq 0\). By a very similar argument, since \(L_H^{(2)} \succeq L_H^{(1)}\), \(\delta(\bm{x}) \geq 0\). It follows that
        \[\Delta^{(2)}(\bm{x}) - \Delta^{(1)}(\bm{x}) \leq 0.\qedhere\]
    \end{proof}

\corollaryTwo*
    % Full proof
    \begin{proof}
        The first part follows almost immediately from \Cref{thm: two-ticket improves outcome disparity} upon observing that
        \[\Delta_{\TPR} = \E_{L_P, L_U, \bm{x}} \brackets{\Delta(\bm{x}) \mid Y = 1}.\]

        The second part follows from an application of \Cref{lem: expression for two-ticket acceptance probability} and a near-identical argument to \Cref{thm: two-ticket improves outcome disparity}.

        The third part follows from the second part, observing that \(\TPR^{(k)} = \P(G = P) \TPR_P^{(k)} + \P(G = U)\TPR_U^{(k)}\).
    \end{proof}

%\cref{lem: when threshold stays the same}

\thmNticket*
\begin{proof}
Let $\bm{x} \in \mathcal{D}$ be an unmodified resume. For $g \in \{p, u\}$, let $g_{\bm{x}} = \P_{L_G}(f_\tau(L_g(\bm{x})) = 1)$ be the baseline, group-dependent probability of acceptance. Namely, $u_{\bm{x}}$ represents the probability that a candidate from group $U$ with unmodified resume $\bm{x}$ will be accepted by $f_\tau$ given their resume has been modified once by LLM $L_U$, with the same holding for $p_{\bm{x}},P,L_P$, respectively.

Let $h_{\bm{x}} = \P_{L_H}(f_\tau(L_H(\bm{z})) = 1)$, where $\bm{z} \in \{L_H^i(L_G(\bm{x}))\}\) for \(i \in \mathbb{N}\) and $G \in \{U, P\}$. Since any LLM manipulation is invariant to previous manipulations, $h_{\bm{x}}$ represents the probability that the result of a single $L_H$ application to a possibly manipulated resume $\bm z$ achieves a score equal to or above the threshold $\tau$. Observe that \(g_{\bm{x}}\) and \(h_{\bm{x}}\) depend on \(\bm{x}\) due to the presence of fundamental features that are preserved throughout every LLM manipulation and affect the probability of acceptance. 

Following Theorem~\ref{thm: hiring outcome disparity} and the assumption in the theorem, we have that $u_{\bm{x}} \leq p_{\bm{x}} \leq h_{\bm{x}}$.


Consider the function $T_{\bm{x}} : [0,1] \to [0,1]$ given by
\[
T_{\bm{x}}(z) = z + h_{\bm{x}}(1 - z).
\]

Since $0 \leq z+ h_{\bm{x}}(1 - z) \leq 1$, the output of $T_{\bm{x}}(z)$ is in $[0,1]$  and thus $T$ is well-defined.



Let $d(p, u) = |p - u|$. Notice that $([0,1], d)$ is a metric space. 
We will show that $T_{\bm{x}}$ is a contraction operator.




Let $k_{\bm x} = 1 - h_{\bm{x}}$. If $h_{\bm{x}} = 0$, the applicant will be rejected even after applying the two-ticket scheme an infinite number of times unless $f_\tau(\bm x)=1$, in 
which case the applicant will be accepted for every $n\in \mathbb N$. Either way, if $h_{\bm{x}} = 0$, the outcome is independent of group membership. %If $h_{\bm{x}} = 1$ the outcome of the $n$-ticket scheme is $1$ for every $n$ and is also independent of the group membership. 

So suppose $0<h_{\bm{x}}<1$. This implies $0<k_{\bm x}<1$. For any $p, u \in [0,1]$, we have that 
\[
d(T_{\bm{x}}(p), T_{\bm{x}}(u)) = \left| p + h_{\bm{x}}(1 - p) - (u + h_{\bm{x}}(1 - u)) \right| = (1 - h_{\bm{x}})|p - u| \leq k_{\bm x} d(T_{\bm{x}}(p), T_{\bm{x}}(u)).
\]

Therefore, $T_{\bm{x}}$ is a contraction operator by definition.

Recall that we already showed that for $n\geq 2$, if $h_{\bm{x}}=0$ then the outcome is independent of the group membership. Next, we show that for $h_{\bm{x}}>0$, $T_{\bm{x}}^n(z)$ converges to $1$, which implies that the outcome of the $n$-ticket scheme when $n\rightarrow \infty$ is always acceptance and is independent of group membership.
\begin{claim}\label{claim:Tconvergence}
If $h_{\bm{x}}>0$, $
\lim_{n \to \infty} T_{\bm{x}}^n(z) =1$.
\end{claim}
\begin{proof}
Rearranging $T_{\bm{x}}(z)$,
\[
T_{\bm{x}}(z) = (1 - h_{\bm{x}})z + h_{\bm{x}}.
\]
Applying $T_{\bm{x}}$ twice:
\[
T_{\bm{x}}^2(z) = (1 - h_{\bm{x}})((1 - h_{\bm{x}})z + h_{\bm{x}}) + h_{\bm{x}}= (1 - h_{\bm{x}})^2 z + h_{\bm{x}}(1 + (1 - h_{\bm{x}})).
\]
Continuing this process, we can express $T_{\bm{x}}^n$ as a geometric series,
\[
T_{\bm{x}}^n(z) = (1 - h_{\bm{x}})^n z + h_{\bm{x}} \sum_{j=0}^{n-1} (1 - h_{\bm{x}})^j= (1 - h_{\bm{x}})^n z + h_{\bm{x}} \cdot \frac{1 - (1 - h_{\bm{x}})^n}{h_{\bm{x}}}= 1 - (1 - h_{\bm{x}})^n(1 - z).
\]
Taking the limit as \( n \to \infty \):
\[
\lim_{n \to \infty} T_{\bm{x}}^n(z) = 1 - \lim_{n \to \infty} (1 - h_{\bm{x}})^n(1 - z)=1.
\]    
\end{proof}
The rest of the proof follows from Banach's fixed point theorem.
\end{proof}

\corNticket*
\begin{proof}
    If $o=0$ then from Theorem~\ref{thm: n-ticket}, for any $G\in\{U,P\}$, $$|\P(f_\tau(L_H^{n}(L_G(\bm{x}))) = 0)- 0| =\P(f_\tau(L_H^{n}(L_G(\bm{x}))) = 0)\leq O(k_{\bm x}^n).$$
    
    If $o=1$ then from Theorem~\ref{thm: n-ticket}, for any $G\in\{U,P\}$, $$|\P(f_\tau(L_H^{n}(L_G(\bm{x}))) = 1)- 1| =1- \P(f_\tau(L_H^{n}(L_G(\bm{x}))) = 1)\leq O(k^n),$$ hence $$1- O(k_{\bm x}^n)\leq \P(f_\tau(L_H^{n}(L_G(\bm{x}))=1).$$

    Let $o=o(\bm x)$ be the outcome of $\bm x$. 
    \[
    \P\big(f_\tau(L_H^{n}(L_U(\bm{x}))) \ne f_\tau(L_H^{n}(L_P(\bm{x})))\leq     \P(f_\tau(L_H^{n}(L_U(\bm{x})))\ne o) \lor f_\tau(L_H^{n}(L_P(\bm{x})))\ne o\big)
    \]
    From union bound,
    \[
    \P\big(f_\tau(L_H^{n}(L_U(\bm{x})))\ne o\big) \lor f_\tau(L_H^{n}(L_P(\bm{x})))\ne o)\leq 2\P(f_\tau(L_H^{n}(L_G(\bm{x}))\ne o)\leq 2O(k_{\bm x}^n)=O(k_{\bm x}^n).
    \]

    As for the clauses, (1) follows directly from the above, by conditioning over $y=1$. (2) and (3) follow from the same proof as Corollary~\ref{cor: two-ticket improves group fairness and accuracy}.
\end{proof}