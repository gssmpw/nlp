%%%%%%%% ICML 2025 EXAMPLE LATEX SUBMISSION FILE %%%%%%%%%%%%%%%%%

\documentclass{article}

% Recommended, but optional, packages for figures and better typesetting:
\usepackage{microtype}
\usepackage{graphicx}
\usepackage{subfigure}
\usepackage{booktabs} % for professional tables
\usepackage{amsthm}
\usepackage{amssymb}
\usepackage{dsfont}
\usepackage{tabularx}
\usepackage{outlines}
\usepackage{float}
\usepackage{graphicx}
\usepackage{subcaption}
\usepackage{caption}
\usepackage{booktabs}
\usepackage[hyphens]{url}


% hyperref makes hyperlinks in the resulting PDF.
% If your build breaks (sometimes temporarily if a hyperlink spans a page)
% please comment out the following usepackage line and replace
% \usepackage{icml2025} with \usepackage[nohyperref]{icml2025} above.
\usepackage{hyperref}

% Attempt to make hyperref and algorithmic work together better:
\newcommand{\theHalgorithm}{\arabic{algorithm}}

% Use the following line for the initial blind version submitted for review:
\usepackage{icml2025}

% If accepted, instead use the following line for the camera-ready submission:
%\usepackage[accepted]{icml2025}

% For theorems and such
\usepackage{amsmath}
\usepackage{amssymb}
\usepackage{mathtools}
\usepackage{amsthm}

% if you use cleveref..
\usepackage[capitalize,noabbrev]{cleveref}

%%%%%%%%%%%%%%%%%%%%%%%%%%%%%%%%
% THEOREMS
%%%%%%%%%%%%%%%%%%%%%%%%%%%%%%%%
\theoremstyle{plain}
% \newtheorem{theorem}{Theorem}[section]
% \newtheorem{proposition}[theorem]{Proposition}
% \newtheorem{lemma}[theorem]{Lemma}
% \newtheorem{corollary}[theorem]{Corollary}
% \theoremstyle{definition}
% \newtheorem{definition}[theorem]{Definition}
% \newtheorem{assumption}[theorem]{Assumption}
% \theoremstyle{remark}
% \newtheorem{remark}[theorem]{Remark}

\theoremstyle{definition}
\newtheorem{definition}{Definition}[section]
\newtheorem{lemma}{Lemma}[section]
\newtheorem{proposition}{Proposition}[section]
\newtheorem*{remark}{Remark}
% \newtheorem{remark}{Remark} % Allow remarks to be referenced
\newtheorem{theorem}{Theorem}
\newtheorem{corollary}{Corollary}
\newtheorem{example}{Example}


% Custom operators/commands
% These are my additions 
\usepackage{cleveref}
%\setlength\parindent{0pt} %INDENT LOCATION
% \usepackage[colorlinks=true, allcolors=blue]{hyperref}
% \usepackage{xcolor}
% \usepackage{soul}   % colorbox linebreak
\usepackage{bm}
% \usepackage{tablefootnote}
\usepackage{mathtools}      % coloneq

% =============== CUSTOM COMMAND DEFINITIONS ===============
% \DeclareMathOperator*{\E}{\mathop{\mathbb{E}}}
\DeclareMathOperator*{\E}{\mathbb{E}}
\let\P\relax
% \DeclareMathOperator*{\P}{\mathop{\mathbb{P}}}
\DeclareMathOperator*{\P}{\mathbb{P}}
\newcommand{\disteq}{\stackrel{d}{=}}

\newcommand{\one}[1][]{\mathds{1}
    \ifthenelse{\equal{#1}{}}
        {}              % Then
        {\brackets{#1}} % Else
} 

\newcommand{\floor}[1]{\left\lfloor #1 \right\rfloor}
\newcommand{\ceil}[1]{\left\lceil #1 \right\rceil}
\newcommand{\fracpart}[1]{\left\{ #1 \right\}}

\newcommand{\paren}[1]{\left( #1 \right)}
\newcommand{\braces}[1]{\left\{ #1 \right\}}
\newcommand{\brackets}[1]{\left[ #1 \right]}
\newcommand{\angles}[1]{\left\langle #1 \right\rangle}
\newcommand{\abs}[1]{\left| #1 \right|}


\newcommand{\Z}{\mathbb{Z}}
\newcommand{\N}{\mathbb{N}}
\newcommand{\R}{\mathbb{R}}

\DeclareMathOperator*{\argmax}{argmax}
\DeclareMathOperator*{\argmin}{argmin}

% \newcommand{\abs}[1]{\left\lvert #1 \right\rvert}

% \newcommand{\norm}[2][2]{\left\lvert\left\lvert #2 \right\rvert\right\rvert_#1}
% \newcommand{\evaluate}[1]{{#1}\Biggr\rvert}

%\newcommand{\peg}{\phantom{-}}

\newcommand{\setbld}[2]{\ensuremath{\left\{#1 ~\middle|~ #2\right\}}}   % $\setbld{(M, x)}

% Insertables
% \newcommand{\BOGO}[1]{\\{\color{red}BOGO: #1}}
\newcommand{\BOGO}[1]{\textcolor{red}{ BOGO: #1}}

\newcommand{\lc}[1]{\textcolor{green!50!black}{#1}}

\newtheorem{claim}[theorem]{Claim}

\newcommand{\D}{\mathcal{D}}
\newcommand{\TPR}{\mathrm{TPR}} % True Positive Rate (e.g. Sensitivity, Recall)
\newcommand{\TNR}{\mathrm{TNR}} % True Negative Rate
\newcommand{\FPR}{\mathrm{FPR}} % False Positive Rate
\newcommand{\FNR}{\mathrm{FNR}} % False Negative Rate

\newcommand{\FSD}{\succeq_1}    % first-order stochastic dominance
% independence, from https://tex.stackexchange.com/questions/79434/double-perpendicular-symbol-for-independence
\newcommand\independent{\protect\mathpalette{\protect\independenT}{\perp}}
\def\independenT#1#2{\mathrel{\rlap{$#1#2$}\mkern2mu{#1#2}}}

\usepackage{xcolor}
\newcommand{\jack}[1]{{\color{teal} \sf [Jack: #1]}}
\newcommand{\judy}[1]{{\color{violet} \sf [Judy: #1]}}
\usepackage{thm-restate}

\newcommand{\support}{\textrm{supp}}


% Todonotes is useful during development; simply uncomment the next line
%    and comment out the line below the next line to turn off comments
%\usepackage[disable,textsize=tiny]{todonotes}
\usepackage[textsize=tiny]{todonotes}


% The \icmltitle you define below is probably too long as a header.
% Therefore, a short form for the running title is supplied here:
\icmltitlerunning{Two Tickets are Better than One: Fair and Accurate Hiring Under Strategic LLM Manipulations}

\begin{document}

\twocolumn[
\icmltitle{Two Tickets are Better than One: \\ Fair and Accurate Hiring Under Strategic LLM Manipulations}

% It is OKAY to include author information, even for blind
% submissions: the style file will automatically remove it for you
% unless you've provided the [accepted] option to the icml2025
% package.

% List of affiliations: The first argument should be a (short)
% identifier you will use later to specify author affiliations
% Academic affiliations should list Department, University, City, Region, Country
% Industry affiliations should list Company, City, Region, Country

% You can specify symbols, otherwise they are numbered in order.
% Ideally, you should not use this facility. Affiliations will be numbered
% in order of appearance and this is the preferred way.
\icmlsetsymbol{equal}{*}

\begin{icmlauthorlist}
\icmlauthor{Firstname1 Lastname1}{equal,yyy}
\icmlauthor{Firstname2 Lastname2}{equal,yyy,comp}
\icmlauthor{Firstname3 Lastname3}{comp}
\icmlauthor{Firstname4 Lastname4}{sch}
\icmlauthor{Firstname5 Lastname5}{yyy}
\icmlauthor{Firstname6 Lastname6}{sch,yyy,comp}
\icmlauthor{Firstname7 Lastname7}{comp}
%\icmlauthor{}{sch}
\icmlauthor{Firstname8 Lastname8}{sch}
\icmlauthor{Firstname8 Lastname8}{yyy,comp}
%\icmlauthor{}{sch}
%\icmlauthor{}{sch}
\end{icmlauthorlist}

\icmlaffiliation{yyy}{Department of XXX, University of YYY, Location, Country}
\icmlaffiliation{comp}{Company Name, Location, Country}
\icmlaffiliation{sch}{School of ZZZ, Institute of WWW, Location, Country}

\icmlcorrespondingauthor{Firstname1 Lastname1}{first1.last1@xxx.edu}
\icmlcorrespondingauthor{Firstname2 Lastname2}{first2.last2@www.uk}

% You may provide any keywords that you
% find helpful for describing your paper; these are used to populate
% the "keywords" metadata in the PDF but will not be shown in the document
\icmlkeywords{Machine Learning, ICML}

\vskip 0.3in
]

% this must go after the closing bracket ] following \twocolumn[ ...

% This command actually creates the footnote in the first column
% listing the affiliations and the copyright notice.
% The command takes one argument, which is text to display at the start of the footnote.
% The \icmlEqualContribution command is standard text for equal contribution.
% Remove it (just {}) if you do not need this facility.

%\printAffiliationsAndNotice{}  % leave blank if no need to mention equal contribution
\printAffiliationsAndNotice{\icmlEqualContribution} % otherwise use the standard text.

%%
%% This is file `sample-manuscript.tex',
%% generated with the docstrip utility.
%%
%% The original source files were:
%%
%% samples.dtx  (with options: `manuscript')
%% 
%% IMPORTANT NOTICE:
%% 
%% For the copyright see the source file.
%% 
%% Any modified versions of this file must be renamed
%% with new filenames distinct from sample-manuscript.tex.
%% 
%% For distribution of the original source see the terms
%% for copying and modification in the file samples.dtx.
%% 
%% This generated file may be distributed as long as the
%% original source files, as listed above, are part of the
%% same distribution. (The sources need not necessarily be
%% in the same archive or directory.)
%%
%% The first command in your LaTeX source must be the \documentclass command.
%%%% Small single column format, used for CIE, CSUR, DTRAP, JACM, JDIQ, JEA, JERIC, JETC, PACMCGIT, TAAS, TACCESS, TACO, TALG, TALLIP (formerly TALIP), TCPS, TDSCI, TEAC, TECS, TELO, THRI, TIIS, TIOT, TISSEC, TIST, TKDD, TMIS, TOCE, TOCHI, TOCL, TOCS, TOCT, TODAES, TODS, TOIS, TOIT, TOMACS, TOMM (formerly TOMCCAP), TOMPECS, TOMS, TOPC, TOPLAS, TOPS, TOS, TOSEM, TOSN, TQC, TRETS, TSAS, TSC, TSLP, TWEB.
% \documentclass[acmsmall]{acmart}

%%%% Large single column format, used for IMWUT, JOCCH, PACMPL, POMACS, TAP, PACMHCI
% \documentclass[acmlarge,screen]{acmart}

%%%% Large double column format, used for TOG
% \documentclass[acmtog, authorversion]{acmart}

%%%% Generic manuscript mode, required for submission
%%%% and peer review
\documentclass[manuscript,screen,review=false]{acmart}
% \documentclass[manuscript,review,anonymous]{acmart}
% \documentclass[sigconf]{acmart}
%% Fonts used in the template cannot be substituted; margin 
%% adjustments are not allowed.
%%
%% \BibTeX command to typeset BibTeX logo in the docs
\AtBeginDocument{%
  \providecommand\BibTeX{{%
    \normalfont B\kern-0.5em{\scshape i\kern-0.25em b}\kern-0.8em\TeX}}}

%% Rights management information.  This information is sent to you
%% when you complete the rights form.  These commands have SAMPLE
%% values in them; it is your responsibility as an author to replace
%% the commands and values with those provided to you when you
%% complete the rights form.
\setcopyright{acmcopyright}
\copyrightyear{2018}
\acmYear{2018}
\acmDOI{10.1145/1122445.1122456}

%% These commands are for a PROCEEDINGS abstract or paper.
\acmConference[Woodstock '18]{Woodstock '18: ACM Conference on Human-Computer Interaction}{June 03--05, 2018}{Woodstock, NY}
\acmBooktitle{Woodstock '18: ACM Conference on Human-Computer Interaction, June 03--05, 2018, Woodstock, NY}
\acmPrice{15.00}
\acmISBN{978-1-4503-XXXX-X/18/06}

\usepackage{tabularx}
\usepackage{enumitem}
\usepackage[toc,page]{appendix}
\usepackage{multirow}
\usepackage{makecell}
\usepackage{hhline}
\usepackage{afterpage}

%%
%% Submission ID.
%% Use this when submitting an article to a sponsored event. You'll
%% receive a unique submission ID from the organizers
%% of the event, and this ID should be used as the parameter to this command.
%%\acmSubmissionID{123-A56-BU3}

%%
%% The majority of ACM publications use numbered citations and
%% references.  The command \citestyle{authoryear} switches to the
%% "author year" style.
%%
%% If you are preparing content for an event
%% sponsored by ACM SIGGRAPH, you must use the "author year" style of
%% citations and references.
%% Uncommenting
%% the next command will enable that style.
%%\citestyle{acmauthoryear}

%%
%% end of the preamble, start of the body of the document source.
\begin{document}

%%
%% The "title" command has an optional parameter,
%% allowing the author to define a "short title" to be used in page headers.

\title{Audience Impressions of Narrative Structures and Personal Language Style in Science Communication on Social Media}
% \shorttitle{Audience Impressions of Informal Science Communication}
\renewcommand{\shorttitle}{Audience Impressions of Informal Science Communication}


%%
%% The "author" command and its associated commands are used to define
%% the authors and their affiliations.
%% Of note is the shared affiliation of the first two authors, and the
%% "authornote" and "authornotemark" commands
%% used to denote shared contribution to the research.
% \author{Anonymous For Submission}
\author{Grace Li}
\email{gl2676@columbia.edu}
\affiliation{
    \institution{Columbia University}
    \city{NY}
    \state{NY}
    \country{USA}
}

\author{Yuanyang ``YY'' Teng}
\affiliation{
    \institution{Columbia University}
    \city{NY}
    \state{NY}
    \country{USA}
}

\author{Juna Kawai-Yue}
\affiliation{
    \institution{Columbia University}
    \city{NY}
    \state{NY}
    \country{USA}
}


\author{Unaisah Ahmed}
\affiliation{
    \institution{Barnard College}
    \city{NY}
    \state{NY}
    \country{USA}
}

\author{Anatta S. Tantiwongse}
\affiliation{
    \institution{Columbia University}
    \city{NY}
    \state{NY}
    \country{USA}
}

\author{Jessica Y. Liang}
\affiliation{
    \institution{Columbia University}
    \city{NY}
    \state{NY}
    \country{USA}
}

\author{Dorothy Zhang}
\affiliation{
    \institution{Columbia University}
    \city{NY}
    \state{NY}
    \country{USA}
}

\author{Kynnedy Simone Smith}
\affiliation{
    \institution{Columbia University}
    \city{NY}
    \state{NY}
    \country{USA}
}

\author{Tao Long}
\affiliation{
    \institution{Columbia University}
    \city{NY}
    \state{NY}
    \country{USA}
}


\author{Mina Lee}
\affiliation{
    \institution{University of Chicago}
    \city{Chicago}
    \state{IL}
    \country{USA}
}


\author{Lydia B Chilton}
\affiliation{
    \institution{Columbia University}
    \city{NY}
    \state{NY}
    \country{USA}
}
% \authornote{Both authors contributed equally to this research.}
% \email{trovato@corporation.com}
% \orcid{1234-5678-9012}
% \author{G.K.M. Tobin}
% \authornotemark[1]
% \email{webmaster@marysville-ohio.com}
% \affiliation{%
%   \institution{Institute for Clarity in Documentation}
%   \city{Dublin}
%   \state{Ohio}
%   \country{USA}
% }

%%
%% By default, the full list of authors will be used in the page
%% headers. Often, this list is too long, and will overlap
%% other information printed in the page headers. This command allows
%% the author to define a more concise list
%% of authors' names for this purpose.
%\renewcommand{\shortauthors}{Trovato and Tobin, et al.}

\newcommand{\yy}[1]{{\color{purple}\bf{YY Comments: #1}\normalfont}}
\newcommand{\grace}[1]{{\color{pink}\bf{Grace Comments: #1}\normalfont}}
\newcommand{\lydia}[1]{{\color{red}\bf{Lydia Comments: #1}\normalfont}}
\def \revision #1{\textcolor{blue}{#1}}
%%
%% The abstract is a short summary of the work to be presented in the
%% article.
%TC:ignore
\begin{abstract}
Science communication increases public interest in science by educating, engaging, and encouraging everyday people to participate in the sciences. But traditional science communication is often too formal and inaccessible for general audiences. However, there is a growing trend on social media to make it more approachable using three techniques: relatable examples to make explanations concrete, step-by-step walkthroughs to improve understanding, and personal language to drive engagement. These techniques are flashy and often garner more engagement from social media users, but the effectiveness of these techniques in actually explaining the science is unknown. Furthermore, many scientists struggle with adopting these science communication strategies for social media, fearing it might undermine their authority. We conduct a reader study to understand how these science communication techniques on social media affect readers' understanding and engagement of the science. We found that while most readers prefer these techniques, they had diverse preferences for when and where these techniques are used. With these findings, we conducted a writer study to understand how scientists' varying comfort levels with these strategies can be supported by presenting different structure and style options. We found that the side-by-side comparison of options helped writers make editorial decisions. Instead of adhering to one direction of science communication, writers explored a continuum of options which helped them identify which communication strategies they wanted to implement. 


% within diverse reader preferences. We conducted a writer study that provided different options for structuring and styling science explanations. 
% , in order to accommodate different scientists' comfort levels in using social media science communication techniques,

% Despite this, scientists shy away from using colloquial language, fearing it might undermine their authority. To rationalize writers’ fears, we conducted a reader study to understand audience impressions. We found that while readers prefer these techniques, they can detract from understanding complex topics. To understand how writers’ fears can be mitigated, we conducted a writer study that provided different options of structuring and styling science explanations. We found that the side-by-side viewing of options helped with writing decisions. Instead of adhering to one direction of science communication, writers explored a continuum of options and the process helped reduce their fears of using colloquial techniques. 


% Traditional science communication is often too formal and inaccessible for general audiences. However, there's a growing trend on social media to make it more approachable using three techniques: relatable examples to make abstract concepts more concrete, step-by-step walkthroughs to improve understanding, and personal language to drive engagement. Despite this, scientists shy away from using colloquial language, fearing it might undermine their authority. 
% To rationalize writers' fears, we conducted a reader study to understand audience impressions. 
% We found that while these elements can engage readers, they sometimes detract from understanding complex topics. 
% Furthermore, to understand how writers' fears can be mitigated, we conducted a writer study that showed writers different options of structuring and styling their science explanations. We found that writers had diverse preferences for presenting science to a larger audience. We also found that the side-by-side viewing of different structure and styles helped writers decide which option to choose. Furthermore, it also inspired writers to merge different aspects of each condition together. These findings demonstrate the importance of showing writers different approaches to science writing and to help dispell writers' anxiety about adhering to one method of science communication. to helps them explore a continuum of options.

% \grace{somethng about potential design space??}... 

% Based on these insights, we developed a system that helps science writers balance engagement and clarity by offering LLM-generated suggestions with and without colloquial properties. We found that this side-by-side comparison helped writers explore different explanation methods. In conclusion, while colloquial language can enhance engagement, it must be used carefully to avoid weakening explanations.
% \lydia{"struggle" is a wrong word-- they shy away from using personal language, because of this fear of undermining their authority. we ran two studies w/ reader and writer NOT build a system. update w/ findings.}
\end{abstract}
%TC:endignore




% lydia 
% Audience study 
% Writer's Study



\begin{CCSXML}
<ccs2012>
   <concept>
       <concept_id>10003120.10003130.10003233</concept_id>
       <concept_desc>Human-centered computing~Collaborative and social computing systems and tools</concept_desc>
       <concept_significance>500</concept_significance>
       </concept>
   <concept>
       <concept_id>10003120.10003130.10003131.10003234</concept_id>
       <concept_desc>Human-centered computing~Social content sharing</concept_desc>
       <concept_significance>500</concept_significance>
       </concept>
 </ccs2012>
\end{CCSXML}

\ccsdesc[500]{Human-centered computing~Collaborative and social computing systems and tools}
\ccsdesc[300]{Human-centered computing~Social content sharing}

%%
%% Keywords. The author(s) should pick words that accurately describe
%% the work being presented. Separate the keywords with commas.
\keywords{}

%% A "teaser" image appears between the author and affiliation
%% information and the body of the document, and typically spans the
%% page.

%%
%% This command processes the author and affiliation and title
%% information and builds the first part of the formatted document.
\maketitle




\section{Introduction}

Video generation has garnered significant attention owing to its transformative potential across a wide range of applications, such media content creation~\citep{polyak2024movie}, advertising~\citep{zhang2024virbo,bacher2021advert}, video games~\citep{yang2024playable,valevski2024diffusion, oasis2024}, and world model simulators~\citep{ha2018world, videoworldsimulators2024, agarwal2025cosmos}. Benefiting from advanced generative algorithms~\citep{goodfellow2014generative, ho2020denoising, liu2023flow, lipman2023flow}, scalable model architectures~\citep{vaswani2017attention, peebles2023scalable}, vast amounts of internet-sourced data~\citep{chen2024panda, nan2024openvid, ju2024miradata}, and ongoing expansion of computing capabilities~\citep{nvidia2022h100, nvidia2023dgxgh200, nvidia2024h200nvl}, remarkable advancements have been achieved in the field of video generation~\citep{ho2022video, ho2022imagen, singer2023makeavideo, blattmann2023align, videoworldsimulators2024, kuaishou2024klingai, yang2024cogvideox, jin2024pyramidal, polyak2024movie, kong2024hunyuanvideo, ji2024prompt}.


In this work, we present \textbf{\ours}, a family of rectified flow~\citep{lipman2023flow, liu2023flow} transformer models designed for joint image and video generation, establishing a pathway toward industry-grade performance. This report centers on four key components: data curation, model architecture design, flow formulation, and training infrastructure optimization—each rigorously refined to meet the demands of high-quality, large-scale video generation.


\begin{figure}[ht]
    \centering
    \begin{subfigure}[b]{0.82\linewidth}
        \centering
        \includegraphics[width=\linewidth]{figures/t2i_1024.pdf}
        \caption{Text-to-Image Samples}\label{fig:main-demo-t2i}
    \end{subfigure}
    \vfill
    \begin{subfigure}[b]{0.82\linewidth}
        \centering
        \includegraphics[width=\linewidth]{figures/t2v_samples.pdf}
        \caption{Text-to-Video Samples}\label{fig:main-demo-t2v}
    \end{subfigure}
\caption{\textbf{Generated samples from \ours.} Key components are highlighted in \textcolor{red}{\textbf{RED}}.}\label{fig:main-demo}
\end{figure}


First, we present a comprehensive data processing pipeline designed to construct large-scale, high-quality image and video-text datasets. The pipeline integrates multiple advanced techniques, including video and image filtering based on aesthetic scores, OCR-driven content analysis, and subjective evaluations, to ensure exceptional visual and contextual quality. Furthermore, we employ multimodal large language models~(MLLMs)~\citep{yuan2025tarsier2} to generate dense and contextually aligned captions, which are subsequently refined using an additional large language model~(LLM)~\citep{yang2024qwen2} to enhance their accuracy, fluency, and descriptive richness. As a result, we have curated a robust training dataset comprising approximately 36M video-text pairs and 160M image-text pairs, which are proven sufficient for training industry-level generative models.

Secondly, we take a pioneering step by applying rectified flow formulation~\citep{lipman2023flow} for joint image and video generation, implemented through the \ours model family, which comprises Transformer architectures with 2B and 8B parameters. At its core, the \ours framework employs a 3D joint image-video variational autoencoder (VAE) to compress image and video inputs into a shared latent space, facilitating unified representation. This shared latent space is coupled with a full-attention~\citep{vaswani2017attention} mechanism, enabling seamless joint training of image and video. This architecture delivers high-quality, coherent outputs across both images and videos, establishing a unified framework for visual generation tasks.


Furthermore, to support the training of \ours at scale, we have developed a robust infrastructure tailored for large-scale model training. Our approach incorporates advanced parallelism strategies~\citep{jacobs2023deepspeed, pytorch_fsdp} to manage memory efficiently during long-context training. Additionally, we employ ByteCheckpoint~\citep{wan2024bytecheckpoint} for high-performance checkpointing and integrate fault-tolerant mechanisms from MegaScale~\citep{jiang2024megascale} to ensure stability and scalability across large GPU clusters. These optimizations enable \ours to handle the computational and data challenges of generative modeling with exceptional efficiency and reliability.


We evaluate \ours on both text-to-image and text-to-video benchmarks to highlight its competitive advantages. For text-to-image generation, \ours-T2I demonstrates strong performance across multiple benchmarks, including T2I-CompBench~\citep{huang2023t2i-compbench}, GenEval~\citep{ghosh2024geneval}, and DPG-Bench~\citep{hu2024ella_dbgbench}, excelling in both visual quality and text-image alignment. In text-to-video benchmarks, \ours-T2V achieves state-of-the-art performance on the UCF-101~\citep{ucf101} zero-shot generation task. Additionally, \ours-T2V attains an impressive score of \textbf{84.85} on VBench~\citep{huang2024vbench}, securing the top position on the leaderboard (as of 2025-01-25) and surpassing several leading commercial text-to-video models. Qualitative results, illustrated in \Cref{fig:main-demo}, further demonstrate the superior quality of the generated media samples. These findings underscore \ours's effectiveness in multi-modal generation and its potential as a high-performing solution for both research and commercial applications.

\section{Background on Tweetorials}
Tweetorials are a form of social media science communication on Twitter, defined as a chronological series of tweets that explains a science topic \cite{symplurTweetorialsFrom, Bruggemann2020-ez}. According to Breu and Berstein, Tweetorials emerged from the medical community to continue education for other medical professionals, often containing medical jargon and not intended for everyday audiences. As the form became popularized, other communities on Twitter began to adopt the format \cite{doi:10.1056/NEJMp1906790, symplurTweetorialsFrom, Bruggemann2020-ez}. As a result, Breu's definition of a Tweetorial, ``a collection of threaded tweets aimed at teaching users who engage with them," can be applied to various domains such as biology, computer science, and economics. 

Gero et. al. identified and compared key features of Tweetorials to traditional science communication strategies \cite{10.1145/3479566}. The authors identified 3 main structural components of Tweetorials to be the hook, body, and conclusion. Previous research has explored strategies for writing engaging hooks \cite{10.1145/3643834.3661587, long2023tweetorialhooksgenerativeai}. Thus, we focus on specific strategies that appear within the body of a Tweetorial. 

\subsection{Key Strategies of the Tweetorial Body}
According to Gero et. al., the body of a Tweetorial is the most varied in length and types of detail used to explain a given topic \cite{10.1145/3479566}. The researchers found that Tweetorials contain specific techniques such as the use of an example, a step-by-step walkthrough structure, and personal language. 

\subsubsection{Example (E)}
\label{example_defintion}
Like traditional science communication, Tweetorials often contain an example that uses familiar or simpler concepts to explain the main idea \cite{10.1145/3479566}. We define this as an \textbf{example (E)} technique. Components of the example are the \textit{use case} and the \textit{scenario}. The \textit{use case} is a general application of the scientific topic and the \textit{scenario} is a specific situation that describes how the science topic was applied. For example, one Tweetorial uses the example of fingertips getting wrinkly in the bath to explain water immersion wrinkling.\footnote{\href{http://language-play.com/tech-tweets/tweetorial/4}{http://language-play.com/tech-tweets/tweetorial/4}} For this Tweetorial, the use case is when fingers get wrinkly in water. The scenario is a parent bathing their child. We use the use case and scenario for more finegrain control over the LLM-generated narratives in Section \ref{narrative_generation_strategy}. 

% Another 


% \grace{where to add descirption fo data inputs}
% The [example] data field is in the format of a short phrase. For instance: \textit{``Spider’s Web”} is the [example] for the topic Tensile Structure in Civil Engineering. The [scenario] data field contains a scenario of a personal narrative that narrates the example through a scenario and connects with the scientific topic. For instance, for the \textit{``Spider’s Web”} example, the scenario is \textit{``During our late-night camping, my adventurous friend decided to challenge a playful spider. He picked up a small twig and started slowly poking its web. Upon noticing the twig, the spider ran towards it, displaying its territorial instinct. I witnessed how the web, a miraculous tensile structure, withheld the pressure without falling apart. Thanks to the constant tension in the silk material, the web stayed steady, absorbing the additional load, distributed throughout its double-curved surface, and transmitting it to its anchor points.”}. We used the prompt seen in Figure 

\subsubsection{Walkthrough (W)}
Tweetorial structures often use a narrative and signposting to establish a narrative structure. We define these two attributes as the \textbf{walkthrough (W)} technique. Narrative is defined by a series of connected events to explain a given topic. Tweetorials signpost by using transition words like ``Firstly" and ``Secondly,'' or by using a list of questions to help frame the sequence of the Tweetorial. In a Tweetorial about selectivity metrics in college rankings, the author uses the second tweet to list out 3 driving questions for the explanation and to establish the structure of the Tweetorial: ``1. Does ``selectivity" actually tell you anything useful about how good your education will be? 2. What does ``selectivity" actually measure that is of value to a student? 3. Why do I have the feeling somebody chose this metric cause they just needed more stuff to rank by?"\footnote{\href{http://language-play.com/tech-tweets/tweetorial/14}{http://language-play.com/tech-tweets/tweetorial/14}} 
% \yy{the [] in the quote is easily confused with citations, and they are not in the actual tweetorial. This example is more aligned with motivation after the hook, rather than a body walkthrough}

\subsubsection{Personal Language (P)}
Tweetorials often use subjective, conversational, and informal language. We define these features as the \textbf{personal language (P)} technique.  Some authors might use first-person pronouns like ``I" to talk from their subjective perspective,\footnote{\href{http://language-play.com/tech-tweets/tweetorial/1}{http://language-play.com/tech-tweets/tweetorial/1}} or use the second person, ``you," to directly address the audience in conversation: ``You can think of a Hash Function like a magic fingerprint reader."\footnote{\href{http://language-play.com/tech-tweets/tweetorial/31}{http://language-play.com/tech-tweets/tweetorial/31}} Some authors might use ALLCAPS or emojis to engage in informal language and humor: ``OH MAN MY HEAD HURTS AND MY LIMBS TINGLE EVERY TIME I GO TO A CHINESE RESTAURANT, I THINK IT MAY BE ALL THE MSG THEY PUT IN THE FOOD???".\footnote{\href{http://language-play.com/tech-tweets/tweetorial/33}{http://language-play.com/tech-tweets/tweetorial/33}} 

\vspace{5mm}
Figure \ref{fig:everything} provides an annotated Tweetorial highlighting these three techniques (example (E), walkthrough (W), and personal language (P)) on the topic of Walker's Action Decrement Theory in Psychology to demonstrate how they are applied in a science explanation. We use these three techniques to ground our approach in understanding reader preferences for science communication on social media and how writers explore the design space for science writing structures and styles. 


\begin{figure}
    \centering\includegraphics[width=0.62\linewidth]{Figures/Everything_Annotate.png}
    \caption{Annotated Tweetorial on the topic of Walker's Action Decrement Theory in Psychology with color highlights corresponding to Example, Walkthrough, and Personal Language.}
    \label{fig:everything}

\end{figure}

\section{Related Work}

\subsection{Large 3D Reconstruction Models}
Recently, generalized feed-forward models for 3D reconstruction from sparse input views have garnered considerable attention due to their applicability in heavily under-constrained scenarios. The Large Reconstruction Model (LRM)~\cite{hong2023lrm} uses a transformer-based encoder-decoder pipeline to infer a NeRF reconstruction from just a single image. Newer iterations have shifted the focus towards generating 3D Gaussian representations from four input images~\cite{tang2025lgm, xu2024grm, zhang2025gslrm, charatan2024pixelsplat, chen2025mvsplat, liu2025mvsgaussian}, showing remarkable novel view synthesis results. The paradigm of transformer-based sparse 3D reconstruction has also successfully been applied to lifting monocular videos to 4D~\cite{ren2024l4gm}. \\
Yet, none of the existing works in the domain have studied the use-case of inferring \textit{animatable} 3D representations from sparse input images, which is the focus of our work. To this end, we build on top of the Large Gaussian Reconstruction Model (GRM)~\cite{xu2024grm}.

\subsection{3D-aware Portrait Animation}
A different line of work focuses on animating portraits in a 3D-aware manner.
MegaPortraits~\cite{drobyshev2022megaportraits} builds a 3D Volume given a source and driving image, and renders the animated source actor via orthographic projection with subsequent 2D neural rendering.
3D morphable models (3DMMs)~\cite{blanz19993dmm} are extensively used to obtain more interpretable control over the portrait animation. For example, StyleRig~\cite{tewari2020stylerig} demonstrates how a 3DMM can be used to control the data generated from a pre-trained StyleGAN~\cite{karras2019stylegan} network. ROME~\cite{khakhulin2022rome} predicts vertex offsets and texture of a FLAME~\cite{li2017flame} mesh from the input image.
A TriPlane representation is inferred and animated via FLAME~\cite{li2017flame} in multiple methods like Portrait4D~\cite{deng2024portrait4d}, Portrait4D-v2~\cite{deng2024portrait4dv2}, and GPAvatar~\cite{chu2024gpavatar}.
Others, such as VOODOO 3D~\cite{tran2024voodoo3d} and VOODOO XP~\cite{tran2024voodooxp}, learn their own expression encoder to drive the source person in a more detailed manner. \\
All of the aforementioned methods require nothing more than a single image of a person to animate it. This allows them to train on large monocular video datasets to infer a very generic motion prior that even translates to paintings or cartoon characters. However, due to their task formulation, these methods mostly focus on image synthesis from a frontal camera, often trading 3D consistency for better image quality by using 2D screen-space neural renderers. In contrast, our work aims to produce a truthful and complete 3D avatar representation from the input images that can be viewed from any angle.  

\subsection{Photo-realistic 3D Face Models}
The increasing availability of large-scale multi-view face datasets~\cite{kirschstein2023nersemble, ava256, pan2024renderme360, yang2020facescape} has enabled building photo-realistic 3D face models that learn a detailed prior over both geometry and appearance of human faces. HeadNeRF~\cite{hong2022headnerf} conditions a Neural Radiance Field (NeRF)~\cite{mildenhall2021nerf} on identity, expression, albedo, and illumination codes. VRMM~\cite{yang2024vrmm} builds a high-quality and relightable 3D face model using volumetric primitives~\cite{lombardi2021mvp}. One2Avatar~\cite{yu2024one2avatar} extends a 3DMM by anchoring a radiance field to its surface. More recently, GPHM~\cite{xu2025gphm} and HeadGAP~\cite{zheng2024headgap} have adopted 3D Gaussians to build a photo-realistic 3D face model. \\
Photo-realistic 3D face models learn a powerful prior over human facial appearance and geometry, which can be fitted to a single or multiple images of a person, effectively inferring a 3D head avatar. However, the fitting procedure itself is non-trivial and often requires expensive test-time optimization, impeding casual use-cases on consumer-grade devices. While this limitation may be circumvented by learning a generalized encoder that maps images into the 3D face model's latent space, another fundamental limitation remains. Even with more multi-view face datasets being published, the number of available training subjects rarely exceeds the thousands, making it hard to truly learn the full distibution of human facial appearance. Instead, our approach avoids generalizing over the identity axis by conditioning on some images of a person, and only generalizes over the expression axis for which plenty of data is available. 

A similar motivation has inspired recent work on codec avatars where a generalized network infers an animatable 3D representation given a registered mesh of a person~\cite{cao2022authentic, li2024uravatar}.
The resulting avatars exhibit excellent quality at the cost of several minutes of video capture per subject and expensive test-time optimization.
For example, URAvatar~\cite{li2024uravatar} finetunes their network on the given video recording for 3 hours on 8 A100 GPUs, making inference on consumer-grade devices impossible. In contrast, our approach directly regresses the final 3D head avatar from just four input images without the need for expensive test-time fine-tuning.



\section{Reader Study: Methodology}
% Grace (not sure we need) We want to understand whether readers prefer scientific explanations that use prevalent social media communication techniques to explain science. Do readers like explanations with an example (or could the example be unrelatable to their daily life and fail to connect the reader with the topic)? Do readers like explanations that walkthrough the example step-by-step (or does that become too complicated to follow)? Do readers like personal language (or does it undermine the authority of the writer)?

% Grace (not sure we need)  We develop hypotheses and conditions to investigate these questions. We generate science explanations on 15 STEM topics using an LLM and ask 35 readers to rate their understanding and engagement on a Likert-scale survey.  

%purpose: overview and motivate our study is interesting

% In particular, do readers like explanations that have an example (or does the example GET too specific to a single application and fail to motivate the topic)?

%We want to know whether readers prefer Tweetorials that use the science communication techniques SEEN on social media. 

%We generate tweetorials for ALL THESE CONDITIONS on multiple STEM topics using GPT-4 and ask readers to rate their understanding and engagement on a Likert Scale. 

% \grace{thinking about the paper as 3 things that writers could do when writing tweetorials, but writers are uncomfortable doing. here are 3 things that experienced writers do, but should they do it or what do readers think?? when thinking about research about features it's always about it doesn't have these features and is it good to add them, so this needs to be a bit better because we compare EWP to EWP-RemoveE, so it's not really the same structure of add/remove features}
% \grace{Tweetorials, not broadly }

\subsection{Research Questions}
Tweetorials typically include three techniques \textbf{example (E)}, \textbf{walkthrough (W)}, and \textbf{personal language (P)} \cite{10.1145/3479566}. We want to understand how each of these techniques affects readers' ratings of science communication on social media. Given that most published Tweetorials contain these three techniques, we hypothesize that science explanations that contain all three features, \textbf{EWP}, will have the highest reader preference rating. To evaluate the effect of the 3 different features, we compare narratives with all three features (\textbf{EWP}) to narratives with one of the features removed (\textbf{EW}, \textbf{EP}, \textbf{WP}). We investigate the following hypotheses in a survey study with readers:

\textbf{H1: Example (E)}: Readers prefer explanations with an example (EWP) compared to those without an example (WP). 

\textbf{H2: Walkthrough (W)}: Readers prefer explanations that include a step-by-step walkthrough of the topic using an example (EWP) compared to explanations with multiple unrelated examples (EP). 

\textbf{H3: Personal Language (P)}: 
Readers prefer explanations that use personal and subjective language (EWP) compared to explanations that have a neutral scientific voice (EW).

% \textbf{H4: All three techniques (EWP)}: Readers prefer explanations that use all three techniques of example, walkthrough, and personal language (EWP) compared to explanations that use a subset of the three techniques. \grace{we don't actually test these techniques invidiually, the accurate H4 would be: Readers prefer explanations with all 3 explanation techniques compared to explanations that only use 2 of the 3 techniques. (which i think is basically equivalent to finding H1, H2, H3...)}

Testing each of these hypotheses requires comparing two explanations for the same scientific topic that are as similar as possible and only vary in whether they contain a example, walkthrough or personal language. Parallel examples like this are unlikely to occur naturally. Thus, we use AI to generate parallel explanations in each condition (See Section \ref{narrative_generation_strategy}). 

% \color{lightgray}

% \textbf{H1:Examples}: Readers prefer explanations with examples compared to those without an example. 

% To test this we generated two explanations for the same topic: 
% Our experimental condition was generated by a prompt that included everything (with instructions to include an example, walkthrough, and personal language), or \textit{EWP (Everything)} for short.
% The baseline condition had to be as similar as possible, except for not including an example. Thus, we asked the LLM to take the EWP (Everything) explanation and simply 'remove the example'. We call this \textit{EWP-RemoveE}.

% %Compare \textit{EWP (Everything) } to \textit{EWP - RemoveE}

% \textbf{H2:Walkthrough}: Readers prefer explanations that include a step-by-step walkthrough of the topic using an example compared to explanations with multiple unrelated examples. 
% To test this we generate two explanations for the same topic: 
% The experimental condition is the same as \textit{EWP (Everything) } but without few-shot training data. we call this \textit{EWP (Everything) - NoFewShot}.
% The baseline condition was generated using a prompt that instructed GPT to not walkthrough the explanation sequentially, but to include examples and personal language.
% %and to explain the topic and its different aspects in a segmented modular approach. 
% The baseline condition also has no few-shot training data.  We call this \textit{EP - NoFewShot}. 
% %Due to walkthrough is deeply integrated in the structure of the writing, GPT was unscuccessful in removing it... 

% %Compare \textit{EWP (Everything) - NoFewShot} to \textit{EP - NoFewShot}

% \textbf{H3:Personal Language}: 
% %Readers prefer explanations that use personal and subjective language rather than explanations that are have an objective (WRONG WORD) scientific voice.
% Readers prefer explanations that use personal and subjective language compared to explanations that have a neutral scientific voice.
% To test this we generate two explanations for the same topic: 
% The experimental condition is the same as \textit{EWP (Everything)}. 
% The baseline condition is as similar as possible except for not using personal language. Thus, we asked the LLM to take EWP (Everything) explanation and simply 'remove the emotional and subjective language'. We call this \textit{EWP - RemoveP}. 

%Compare \textit{EWP (Everything)} to \textit{EWP - RemoveP}
\color{black}

% \begin{enumerate}
%     \item[\textbf{H1}] Readers prefer Twitter threads that have a relatable example (E) over X . 
%     \item[\textbf{H2}] Readers prefer Twitter threads that have a step-by-step walkthrough (W) of one example over X. 
%     \item[\textbf{H3}] Readers prefer Twitter threads that use personal language (P) over X.   
%     % The audience prefers the Twitterorial that uses personal language and speaks from the writer's first-person perspective over the one that only uses objective language. 
% \end{enumerate}

% In the study, for each hypothesis, we compare two conditions A and B.
% In condition A, which we call "everything", all three techniques are included: example (E), walkthrough (W), and personal language (P). A is compared with a condition B where one technique is missing.

% For H1, we compared two conditions - A. everything(EWP) vs. B. No example(-E). (See Table \ref{Hypothesis_Example_Conditions})

% For H2, we compared two conditions - A. everything(EWP)  vs. B. No walkthrough(-W). (See Table \ref{Hypothesis_Walkthrough_Conditions})

% For H3, we compared two conditions - A. everything(EWP) vs. B. No personal language(-P). (See Table \ref{Hypothesis_Personal_Conditions}) 

% Below, We describe what science topics are used in the study in Section \ref{topic_select}, how Twitter threads used in each condition are generated in Section \ref{Twitter-gen}, participants recruitment in Section \ref{user-study1}, survey design, data collection and analysis in Section \ref{data}. 

\subsection{Automatic Narrative Generation Strategy}
\label{narrative_generation_strategy}

We used OpenAI's GPT-4 API to generate science explanations with and without each technique in the form of Tweetorials (about 10 tweets in length) to ensure consistency and to make fair comparisons between science explanations. We describe the method for each technique in Sections \ref{H1_Prompting}, \ref{H2_Prompting}, and \ref{H3_Prompting}.

% To understand reader preferences for examples (H1), walkthrough (H2), and personal language (H3) in science communication, we need to compare science explanations with and without these three techniques for the same topic. There do not exist Twitter threads of the same topic in the wild that are rigorously written with and without each of the three techniques. No prior research has made comparisons between these conditions. As such, 

We cover 5 diverse STEM fields: a physical science field (Physics), a social science field (Psychology), a technological field (Computer Science), a mathematical field (Statistics), and an engineering field (Civil Engineering). For each field an expert selected topics for 3 different levels of complexity (introductory, intermediate, and advanced levels) for a total of 15 topics (Appendix \ref{stem_topics}). We generated 75 different science explanations (5 conditions for each topic) for readers to rate. Each science explanation was validated by a corresponding expert for accuracy. We describe exact prompting methods for each hypothesis (H1: Example (E), H2: Walkthrough (W), H3: Personal Language(P)) below.

\subsubsection{Generating Tweetorials with and without Examples (H1)} 
\label{H1_Prompting}

We used GPT-4 to generate science narratives that contain an \textbf{E}xample, \textbf{W}alkthrough, and \textbf{P}ersonal language. We included five few-shot examples of published Tweetorials on Twitter to create our experimental condition: \textbf{EWP}. Experts on each topic provided data inputs [use case] and [scenario] (described in Section \ref{example_defintion}) to define the specific example the given narrative should use throughout the explanation. We provided specific guidelines regarding how the LLM should incorporate the given example, a walkthrough, and personal language. We iterated on each line separately and in compilation to ensure that the prompt was concise, essential, and reasonably consistent (Appendix \ref{prompts}). 

To generate the baseline condition, \textbf{WP}, we use a ``remove" method which provides GPT-4 a given narrative and a set of guidelines that specifies what technique to remove from the given narrative while maintaining all other conditions. The output is a new narrative with only the specified technique removed. To generate a science narrative without an example, we use the ``remove" method on the example. We pass in the experimental condition narration, \textbf{EWP}, and a set of guidelines that specify only the example should be removed from the narrative while maintaining all other elements such as structure and style (Appendix \ref{WP_prompt}). We use this same prompting strategy to ``remove personal language" in Section \ref{H3_Prompting}.

Figure \ref{fig:example} shows a side-by-side annotated example of the experimental condition (EWP) and baseline condition (WP) for the topic of Walker's Action Decrement Theory. The lack of example highlights in green in the baseline condition shows the effect of no example in contrast with the explanation with everything.


% All generated narratives were evaluated by experts from each field to ensure technical accuracy and adherence to the given condition (EWP or WP) and regenerated until all conditions were satisfied. 
% \begin{figure}
%     \centering
%     \includegraphics[width=0.9\linewidth]{Figures/H1_prompts.png}
%     \caption{\textbf{H1:Example Prompts} Used to generate the experimental condition, EWP (Everything), and baseline condition (EWP-RemoveE)}
%     \label{fig:H1_example_prompts}
% \end{figure}


\begin{figure}
    \centering
    \includegraphics[width=0.9\linewidth]{Figures/Example_Annotate.png}
    \caption{\textbf{H1: Example} Sample explanation generations on the topic of "Walker's Action Decrement Theory" in Psychology comparing the experimental condition (EWP) and baseline condition (WP).}
    \label{fig:example}
\end{figure}



% We used the ``remove example" approach in order to maintain all other aspects of the science explanation such as structure, style, word choice, and flow and only remove the aspect that we wanted to test: the example (E). 

\subsubsection{Generating Tweetorials with and without Personal Language (H3)} \label{H3_Prompting}
To understand reader preferences for science explanations with and without personal language (H2:Personal Language), we followed a similar GPT-4 generation protocol as in Section \ref{H1_Prompting} to generate the experimental condition which contains all three techniques and few-shot examples, \textbf{EWP}. To generate the baseline condition, we used the ``remove" procedure from Section \ref{H1_Prompting} to ``remove personal language" from the experimental condition, \textit{EWP}. We pass in \textit{EWP} and specify guidelines to only remove the personal language while maintaining all structural elements of the science explanation to create \textbf{EW}. 

Figure \ref{fig:personal} shows a side-by-side annotated example of the experimental condition (EWP) and baseline condition (EW) for the topic of Walker's Action Decrement Theory. The lack of personal language highlights in yellow in the baseline condition shows the effect of removing personal language from EWP. 

\begin{figure}
    \centering
    \includegraphics[width=0.9\linewidth]{Figures/PersonalLanguage_annotate.png}
    \caption{\textbf{H3:Personal Language} Sample explanation generations on the topic of Walker's Action Decrement Theory in Psychology comparing the experimental condition (EWP) and baseline condition (EW).}
    \label{fig:personal}
\end{figure}

% \grace{method is only justifable if EWP is the baseline: if we claim that removing the example is a lower line, the the lower line is a tweetorial with none of the features. EWP is the baseline that means that most existing tweetorials contain these features. baseline  = standard operating conditinos}



\subsubsection{Generating Tweetorials with and without Walkthroughs (H2)} \label{H2_Prompting}
% \grace{re-word this sentence to 

% to understand reader preferences we use GPT to evaluate reader preferences with and wihtout }
To understand reader preferences for walkthroughs (H2:Walkthrough), we used GPT-4 to generate two different science explanations with contrasting narrative structures. In preliminary testing, we found that the ``remove" method failed to generate narratives without walkthroughs because the walkthrough served as the narrative structure. Thus, we provide the data inputs of [topic] and [domain] and a new set of guidelines to GPT-4 to generate narratives without a walkthrough (Appendix \ref{EP_prompt}). The guidelines specify instructions for not using a walkthrough (e.g. the explanation should use a non-sequential approach and that every tweet stands alone). No few-shot examples were added because there were no published Tweetorials with no walkthrough to create \textbf{EP-NoFewShot}.

We used the same base prompt as the experimental conditions in Section \ref{H1_Prompting} and Section \ref{H2_Prompting} to generate a science explanation with an \textbf{E}xample, \textbf{W}alkthrough and \textbf{P}ersonal language. To maintain a fair comparison with EP-NoFewShot, we omitted few-shot examples to create \textbf{EWP-NoFewShot}. Figure \ref{fig:walkthrough} highlights the differences between science narratives with and without walkthroughs.

% We kept the same base prompt and data inputs as in Section \ref{H1_Prompting}. 

% \begin{figure}
%     \centering
%     \includegraphics[width=0.9\linewidth]{Figures/H2_prompts.png}
%     \caption{\textbf{H2:Example Prompts} Used to generate the experimental condition (EWP - NoFewShot) and baseline condition (EP - NoFewShot) }
%     \label{fig:H2_example_prompts}
% \end{figure}

\begin{figure}
    \centering
    \includegraphics[width=0.9\linewidth]{Figures/Walkthrough_Annotate.png}
    \caption{\textbf{H2: Walkthrough} Sample explanation generations on the topic of Back Propagation in Computer Science contrasting experimental condition (EWP-NoFewShot) and baseline condition (EP-NoFewShot).}
    \label{fig:walkthrough}
\end{figure}

% No few-shot training examples were used because there were no applicable, human-written Tweetorials with no walkthrough that we could include.

% \grace{want to go in the order of EPW where the walkthrough part comes last so the reader isn;t jumping around between tehse different thigns -- combine personal alnauge and example section: remove procedire doesn't work for walkthrough in the enxt section we describe the procedure that we do for that}

% \grace{if possible, add an API access date / range }

 
% we selected five STEM fields (Computer Science, Physics, Statistics, Civil Engineering, Psychology), including 1 math, 1, technology,  1 engineering, 1 physical science (Physics) and a social science (Psychology).


% For each topic, we generated one Tweetorial of each of the five conditions: \textit{EWP (Everything)}, \textit{EWP (Everything) - NoFewShot}, \textit{EWP - RemoveE}, \textit{EP - NoFewShot}, \textit{EWP - RemoveP}. 
% Our dataset contains a total of 75 Tweetorials (5 conditions × 5 fields of study ×  3 levels of complexity).

% \yy{Question to Lydia: -- To make our numbers look larger -- at one point, we had 3 topics / level, but we decided not to conduct the reader study for all of them due to time and \$\$. Do we want to say our dataset is 225 tweetorials, but study is on 75 tweetorials??}

% After LLM generation, an expert from each field read and validated the Tweetorials for their scientific accuracy. 

% \yy{Experts validated this, but don't mention they are authors.}
%Experts validated the generation. 
% Among our authors, we have experts from each field who proposed the topics and verified the scientific accuracy of GPT-generated writings. One author studied computer science for four years. 
%One author studied physics for three years. One author studied applied statistics in economics and data science for three years. One author is a licensed architect who practiced in the building construction industry for five years, a subdomain of civil engineering. One author studied cognitive psychology and behavioral science for four years. 
%All the generations are validated by experts for their scientific accuracy. 



% Figure \ref{fig:example} shows a segment of "Walker's Action Decrement Theory" in Psychology, with H1:Example experimental condition and baseline condition side by side. The lack of example highlights in green in the baseline condition shows the effect of no example in contrast of the explanation with everything.


% \yy{annotate (with color? highlight?) and describe E, W, P with the examples}


% \begin{figure}
%     \centering
%     \includegraphics[width=0.9\linewidth]{Figures/H3_prompts.png}
%     \caption{\textbf{H3:Personal Language} prompts used to generate the experimental condition with personal language (EWP) and baseline condition without personal langauge (EWP-RemoveP).}
%     \label{fig:H3_example_prompts}
% \end{figure}





% \subsubsection{\textit{EWP (Everything)} example}

% \begin{quote}
% Field: Pyschology\\
% Topic: "Walker's Action Decrement Theory"\\

% Have you ever wondered why you sometimes forget significant events as they are happening, only to remember them more clearly later on? This happened to me last Halloween. I took one of my friends to see a horror movie, and boy, was it frightening! 1/

% I was so shocked by how terrifying the movie was that my mind went completely blank. I couldn't recall the specific details or what exactly was happening in the movie. Is that weird? I was truly scared and clammy, my hands were shaking, and my heart was pounding super fast. 2/

% I barely had any memory of the movie at all! But when I reached home and my adrenaline started to wear off, something strange happened. I found that I could recall the terrifying jump scares, the horrifying scenes, and even the dull background music from each scene! 3/

% Turns out, this is all perceived just how Walker's action decrement theory would predict it. It sounds like intense scholarly jargon, right? But what it stands for is actually really simple – it states that experiencing intense emotions can temporarily interfere with memory. 4/

% According to this theory, the involvement in a highly emotional event can cause us to respond energetically, increasing our physical reactions. I think that's why my heart started pounding and I became clammy and shaky like a leaf in a whirlwind! Wild, isn't it? 5/

% At the peak of the event, the emotional impact acts as a trigger for an internal response that can block out or interfere with our ability to recall certain memories. Unbelievable as it sounds, our bodies can inadvertently play a hide and seek game with our minds! 6/

% Don't worry though, it's not as if these memories are gone for good. Once the adrenaline rush drops and our bodily responses return to normal, the memories come flooding back as if they were hiding behind a curtain, and now the show's ready to start! 7/

% Recall what happened to me? After coming home and my 'fright or flight' response eased up, I suddenly had vivid memories of those bone-chilling jump scares, the frightening scenes, the horror, and even the background music that set the movie's tone. 8/

% So next time you experience something intense that temporarily wipes your memory, don't fret! Walker's action decrement theory has got you covered. As the thrill dissipates, you'll get to enjoy the replay of the experience all over again in the movie theatre of your mind! 9/

% What a fascinating theory, isn't it? It really makes me appreciate the clever ways our brain and body work together to manage and control our experiences. Make sure to remember your own experiences with Walker's Action Decrement Theory! 10/

% \end{quote}



% \subsubsection{\textit{EWP - RemoveE} example}

% \begin{quote}
% Field: Pyschology\\
% Topic: "Walker's Action Decrement Theory"\\

% Have you ever speculated about the instances where you're unable to remember emotionally charged or significant episodes, only to recollect them much more clearly later on? The answer to this lies in understanding a specific psychological theory, known as Walker's action decrement theory. 1/

% Walker's action decrement theory provides an intriguing explanation for this. It states that experiencing intense emotions can temporarily interfere with memory. It sounds like a complex psychological concept, doesn't it? However, it's quite a simple one in reality. 2/

% The theory implies that when we are deeply involved in an emotionally intense event, we respond to it energetically, escalating our physiological responses. This heightened physical response can include an increased heart rate, sweatiness, or even trembling, due to the rush of adrenaline. 3/

% The most compelling part of this theory is that, at the apex of the event, the emotional impact initiates an internal response that can temporarily prevent or disrupt our ability to remember specific details. Figuratively speaking, there is an unexpected game of hide-and-seek happening within us! 4/

% However, these memories aren't permanently inaccessible. The theory assures us that, as the adrenaline rush subsides and our physiological responses return to their default states, the blocked memories resurface. It's as if they were concealed behind a curtain, ready to reveal themselves once the stage is set. 5/

% This mechanism allows us to relive the emotional experience in complete detail. As per Walker's action decrement theory, once the emotional intensity diminishes, all the suppressed memories of the event can significantly and vividly resurface. 6/

% Now, whenever you are confronted with a situation where something emotionally charged temporarily disrupts your memory recall, remember that Walker's action decrement theory offers the explanation. As the intensity eases, you'll find the forgotten details coming back to you in astonishing clarity. 7/

% This theory provides a fascinating explanation for the interplay of emotions, memory recall, and physiological reactions in our body. It gives us significant insight into how our brain and body coordinate to process intense experiences, and how we, in turn, perceive and remember them. 8/

% So, the next time you momentarily forget a significant event, don't worry. Walker's action decrement theory provides an elegant and simple way to understand this phenomenon. This model captures the intricate interplay of emotions and memory, making it an essential psychological theory to remember! 9/

% Aren't psychological theories simply fascinating? They provide us with a myriad of insights into the remarkable ways our brain and body coordinate to process our experiences. So remember, Walker's Action Decrement Theory offers an insightful window into the fascinating world of memory dynamics! 10/

% \end{quote}

% \subsubsection{\textit{EWP (Everything) - NoFewShot} example}



% \begin{quote}
% Field: Computer Science\\
% Topic: Back Propagation\\

% It's a lazy Sunday afternoon. The golden sunlight peeping through the windows, the crispy hint of fall in the air. As I get engrossed in a book, I get the irresistible urge to bake. I find an exciting new cookie recipe online. But will they turn out as expected? 1/

% I carefully follow the recipe down to each detail. As the warm sweet scent wafts through the room, I can hardly wait. But when I taste my first batch, they're way too sugary. Bummer, right? In that moment, I draw parallels with backpropagation in artificial intelligence. 2/

% Now, what on earth is backpropagation? It's an algorithm used in training artificial neural networks. Think of it like GPS guiding you when you've taken a wrong turn, rerouting to get to the destination faster. But how does tweaking a cookie recipe link to AI? 3/

% Just as you’d adjust your recipe based on the taste, machines learning from their mistakes and improving is the crux of backpropagation. Here's the story of one such 'intelligent' machine and its journey. Intrigued? Let's dive deep into this fascinating concept. 4/

% I go back to my cookie dough, realizing I had added too much sugar. I adjust the amount for my next batch but it’s under-sweet this time. Just like an AI model tweaking its parameters based on the error it encountered. Trying, failing, learning, improving. 5/

% Each adjustment I make to my recipe is akin to a step in the backpropagation process. The perfect batch of cookies isn’t an accident but a result of careful refinement. Isn't it amazing how a simple act of baking mirrors the complex world of artificial intelligence? 6/

% As with backpropagation, each mistake is a stepping stone to perfection. Every failure brings it closer to the goal. The sweet taste of success, in my case, a perfect batch of cookies, goes beyond the initial hurdles. That’s the beauty of learning, isn’t it? 7/

% Just like perfecting a recipe, backpropagation involves constant tweaking and learning until we get the desired output. It's not defeat that defines us, but rather how we respond to it. The machine learns, improves and eventually nails it. Much like our cookie baking endeavor. 8/

% So, the next time you find yourself stuck in a loop of trials and errors, remember backpropagation, and remember my cookie story. Every effort you make counts, every mistake brings you a step closer to success. Isn't it fascinating how life mirrors AI? 9/

% In my quest to make the perfect cookie, I found a lesson from the realm of AI. A simple Sunday baking led me to explore the profound concept of backpropagation. Like a perfect cookie recipe, AI too finds its sweet spot through continuous learning and improvement. And that's how we bake and learn, sweet folks! 10/

% \end{quote}

% \subsubsection{\textit{EP - NoFewShot} example}

% \begin{quote}
% Field: Computer Science\\
% Topic: Back Propagation\\

% Let's dive into Backpropagation, a crucial element of AI learning. If AI were a student, Backpropagation would be like its personal tutor, nudging it along to better answers. Imagine your math tutor spotting an error in calculation \& guiding you to correct it. 1/

% Backpropagation is the heart of training a neural network. Think of a Neural Network as an intricate web of neurons. It's like the brain of an AI, processing all the information. Backpropagation helps optimize this 'AI brain' to make smarter predictions. 2/

% At the core, backpropagation is about "learning from mistakes". Picture a toddler learning to walk. If they stumble, they adjust their next step. Similarly, when an AI's prediction is off, backpropagation helps it adjust its calculations. 3/

% Backpropagation hinges on Calculus. Did you ever wonder why you had to learn about derivatives in high school? Well, they're critical here. Just like how a car uses brakes to adjust its speed, AI uses calculus and backpropagation to adjust its learning. 4/

% The AI starts 'guessing' answers initially, quite like shooting darts in the dark. Backpropagation assesses how off these predictions are from the actual answer. It calculates an 'error value' that illustrates how wrong our 'AI student' was. 5/

% Think about cooking a new recipe. The first attempt might not taste perfect. So, you adjust the spices – more salt, less chili. Backpropagation does the same for AI, adjusting the model's predictive 'ingredients' (known as weights) to achieve a better result. 6/

% Think of the weights in an AI model like the dials on a sound mixer. Backpropagation is like a sound engineer, constantly fine-tuning the dials(weights) based on the difference between the desired output and actual result for improved accuracy. 7/

% Where does the term 'Backpropagation' come from? It's because after every prediction, the process sends the error value backward through the network. This ‘feedback’ enables the AI to adjust its calculations and reduce future errors. Backpropagation in essence is all about learning, improving and iterating. 8/

% If AI is like a musician learning a new tune, backpropagation is the process of listening to playback and realizing which notes need fixing. It's the way AI networks get better at hitting their notes – becoming more accurate in their predictions \& responses. 9/

% So remember, Backpropagation helps our AIs learn by 'experience'. It nudges them towards better accuracy and smarter predictions, turning them from clumsy beginners to experts capable of tackling complex problems with ease! 10/
% \end{quote}

% \subsubsection{\textit{EWP-RemoveP} example}


% \begin{quote}

% Field: Psychology\\
% Topic: Walker's Action Decrement Theory\\


% The inability to recall significant events as they unfold, only to remember them clearly later on is a phenomena commonly encountered. One such instance occurred during a screening of a horror movie. The movie was considered imposingly frightening, resulting in a sense of terror. 1/

% It was reported that due to the terror instigated by the movie, a state of amnesia was experienced where the specific details about the movie became indistinguishable. The person was found to be clammy, with a rapidly fluctuating heartbeat and trembling hands. 2/

% Immediately after the conclusion of the movie, there was a lack of memory regarding the content of the movie. However, when they returned home and the adrenaline levels dropped, they were able to recollect the scenes, sound effects, and background music from the movie. 3/

% One possible explanation for this situation is Walker's action decrement theory. According to this theory, intense emotions can temporarily interfere with memory. This theory suggests that participating in a highly emotional situation increases energetic responses and physical reactions. 4/

% At the height of such an emotional situation, it could potentially induce an internal response that displaces the ability to recall specific incidents. This indicates that physical conditions can affect memory recall in an unpredictable manner. 6/

% These memories are not permanently lost, but repressed. As the innate adrenaline rush subsidizes and the physical conditions return to normal, the memories usually return. These suppressed memories have been compared to actors waiting for their curtain call on a stage. 7/

% This is corroborated by the scenario where once the person was home and their body and mind were back to normal, they were suddenly flooded with vivid memories of the horror movie they had just seen. This phenomenon even applied to the sound effects and background music. 8/

% If an intense experience causes temporary memory loss, Walker's action decrement theory provides a logical explanation. As the excitement fades, the memory comes rushing back, allowing the possiblity to relive the experience akin to watching a replay in one's mind. 9/

% Walker's action decrement theory serves as a compelling method to comprehend the clever interplay between our brain and body in managing and controlling our emotional experiences. It is important to remember these encounters from the perspective of Walker's Action Deccrement Theory. 10/


% \end{quote}



% % %lydia comment
% \yy{Describe the data, terms, and have examples}
% \yy{two examples from Unaisah, back propagation, memory}









% In this study, participants were asked to read Twitter threads generated by ChatGPT and answer a short Likert Scale Survey. 



%  In the user study, we asked xx non-STEM undergraduate students to read and evaluate the xx Twitterorials and share their preferences (see \ref{user-study1}). We collected data through survey questions and follow-up interviews (see \ref{data}). 
% %We then conducted a user study with xx non-STEM undergraduate students as audience representatives to understand their preferences for the three techniques (see \ref{user-study1}). 



% %%%TABLE for H1-Example
% \begin{table}[]
% \begin{tabular}{|l|l|}
% \hline
% Condition H1-A                                                                    & Condition H1-B                                                                         \\ \hline
% \begin{tabular}[c]{@{}l@{}} Everything(EWP) \\ (w/ Few-shot)\end{tabular} & \begin{tabular}[c]{@{}l@{}}No Example \\ (EWP remove E)\end{tabular} \\ \hline
% \end{tabular}

% \caption{Conditions for H1 Readers prefer Twitter threads that have a relatable example.}
% \label{Hypothesis_Example_Conditions}
% \end{table}

% %%%TABLE for H2-Walkthrough
% \begin{table}[]
% \begin{tabular}{|l|l|} 
% \hline
% Condition H2-A                                                                   & Condition H2-B                                                      \\ \hline
% \begin{tabular}[c]{@{}l@{}}Everything(EWP) \\ (w/o Few-shot)\end{tabular} & \begin{tabular}[c]{@{}l@{}}No Walkthrough \\ (See Table \ref{PROMPT_Walkthrough_Conditions})\end{tabular} \\ \hline
% \end{tabular}

% \caption{Conditions for H2 Readers prefer Twitter threads that have a step-by-step walkthrough of one example. }
% \label{Hypothesis_Walkthrough_Conditions}
% \end{table}

% %%%TABLE for H3-Personal Language
% \begin{table}[]
% \begin{tabular}{|l|l|}
% \hline
% Condition H3-A                                                                    & Condition H3-B                                                                         \\ \hline
% \begin{tabular}[c]{@{}l@{}} Everything(EWP) \\ (w/ Few-shot) \\Same as H1-A\end{tabular} & \begin{tabular}[c]{@{}l@{}}No Personal Language \\ (EWP remove P)\end{tabular} \\ \hline
% \end{tabular}

% \caption{Conditions for H3 Readers prefer Twitter threads that use personal language.}
% \label{Hypothesis_Personal_Conditions}
% \end{table}

% \begin{table}[]
% \resizebox{\textwidth}{!}{%
% \begin{tabular}{|l|l|l|l|l|l|}
% \hline
% Field & \textbf{Computer Science} & \textbf{Physics} & \textbf{Statistics} & \textbf{Civil Engineering} & \textbf{Psychology} \\ \hline
% Introductory Topic &
%   Depth-First Search &
%   Distance and Displacement &
%   Normal Distribution &
%   Lattice Structure &
%   Retroactive and Proactive Interference \\ \hline
% Intermediate Topic &
%   Back Propagation &
%   Thermal Equilibrium &
%   Central Limit Theorem &
%   Tensile Structure &
%   Feature Integration Theory \\ \hline
% Advanced Topic &
%   Recurrent Neural Networks &
%   Thin Film Interference &
%   Linear Regression &
%   Curtain Wall System &
%   Walker's Action Decrement Theory \\ \hline
% \end{tabular}%
% }
% \caption{Scientific fields and topics used in Reader Study}
% \label{topics}
% \end{table}


% \subsection{LLM Generation of Twitter Threads (Prompt Engineering) }
% \label{Twitter-gen}


% We conducted prompt engineering to generate the conditions used in the survey. Below we describe our process. Additionally, see Appendix \ref{prompts} for all the prompts. 

%%% Use our Hypothesis to organize 

% \yy{Organize by Hypothesis H\#}



% \subsubsection{Experimental Condition - EWP(Everything)}
% To produce the experimental condition that contains all three techniques (Example, Walkthrough, Personal Language) to be used in H1/H2/H3, we developed a prompt that has three paragraphs, each paragraph contains instructions for each of the three techniques. 

% % \yy{we have one prompt that has 3 paragraphs - E, W, P}

% % \yy{explain and give examples of how use case and scenarios are related to example. where they come from. }
% To generate explanations with an example, we included data fields [example] and [scenario], which are provided by the experts on the topics. The [example] data field is in the format of a short phrase. For instance: \textit{"Spider's Web"} is the [example] for the topic Tensile Structure in Civil Engineering. The [scenario] data field contains a scenario of a personal narrative that narrates the example through a scenario and connects with the scientific topic. For instance, for the "Spider's Web" example, the scenario is \textit{"During our late-night camping, my adventurous friend decided to challenge a playful spider. He picked up a small twig and started slowly poking its web. Upon noticing the twig, the spider ran towards it, displaying its territorial instinct. I witnessed how the web, a miraculous tensile structure, withheld the pressure without falling apart. Thanks to the constant tension in the silk material, the web stayed steady, absorbing the additional load, distributed throughout its double-curved surface, and transmitting it to its anchor points. "}

% To produce science explanations with an example incorporating the [example] and [scenario] data fields, we use the following "example" prompt paragraph: 

% \begin{quote}
% \textit{Use the given example to help explain how the topic words: [example].
% Use the scenario to provide additional context: [scenario].}
% \end{quote}

% To produce a step-by-step walkthrough of the example, we developed a three-line guideline as a prompt paragraph. We iterated on each line separately and in compilation to ensure that they are concise, essential, and reasonably consistent. The "walkthrough guidelines" prompt paragraph is: 

% \begin{quote}
% \textit{Walkthrough Guidelines:\\
% Tell the story from a first-person perspective.\\
% Walk through the story timeline in a sequence.\\
% Be sure to explain each dimension of the topic in detail, relating it back to the given use case and scenario.}
% \end{quote}

% To apply personal language to the explanation, we repeated a similar process as "walkthrough guidelines" to develop "emotional guidelines". The "emotional guidelines" prompt paragraph is:

% \begin{quote}
% \textit{Emotional Guidelines:\\
% Take the second-person audience on an emotional journey.\\
% Add visually descriptive details in the storytelling.\\
% Use emotional languages (both negative and positive).\\
% Add questions that echo with the audience and spark curiosity.}
% \end{quote}

% We compiled the 3 paragraphs above to produce a prompt that generated Twitter thread for \textit{EWP(Everything)} experimental condition with reasonable consistency and quality. To further improve the LLM performance, we included five few-shot training data. The few-shot training data are Tweetorials written by experts without the use of LLM. 


% \subsubsection{H1: Examples - Baseline Condition (EWP - RemoveE) } 

% To produce explanations for \textit{H1: Examples} baseline condition, we engineered a prompt that instructed LLM to remove the example from the output of \textit{EWP(Everything)} experimental condition. We evaluated the outputs and concluded that LLM was able to successfully produce Twitter threads without an example. The example removal prompt instruction is: 

% \begin{quote}
% \textit{Remove the example from the narrative.\\
% Do not include ANY examples.\\
% Only provide a technical walkthrough of [topic] following the same structure.}
% \end{quote}

% \subsubsection{H2: Walkthrough - Baseline Condition (EP - NoFewShot) }

% To produce explanations for \textit{H2:Walkthrough} baseline condition, we attempted the same removal prompting method, however, found that LLM was unable to accomplish this consistently. Therefore, we engineered a separate prompt paragraph to instruct LLM with guidelines what not to do :

% \begin{quote}
% \textit{
% Do not walkthrough using timeline sequence. \\
% Do not use words such as "before", "after", "then", "next", "also", "first", "second", "third", "last", "summary".\\
% Explain a different technical component of the topic in each Twitter in a non-sequential modular approach.\\
% Each Twitter stands alone and allows the reader to navigate through the explanations in various orders.}
% \end{quote}

% This prompt used to generate \textit{H2:Walkthrough} baseline condition does not contain few-shot training data. To make an equal comparison with the experimental condition, we re-generated \textit{EWP (Everything)} without few shot, which we call it \textit{EWP (Everything) - NoFewShot}. 



% \subsubsection{H3: Personal Langauge - Baseline Condition (EWP - RemoveP)}
% To produce explanations for \textit{H3: Personal Language} baseline condition, we followed a similar process as \textit{H1: Examples} baseline condition, by instructing LLM to remove personal language from the output of \textit{EWP(Everything)} experimental condition. We evaluated the outputs and concluded that LLM was able to successfully produce Twitter threads without personal language. The personal language removal prompt instruction is:

% \begin{quote}
% \textit{Edit the sentences to remove ’you’, ’I’, and ’we’ pronouns.\\
% Do not use rhetorical or confirmation questions.\\
% Use objective and generalizable language wherever possible.\\
% Remove any extraneous descriptions and adjectives.\\
% Use formal language.\\}
% \end{quote}

% \yy{add the prompt in quote}


% \yy{Add Twitter thread samples for each condition, side by side}

% \yy{Move prompts to Appendix, maybe??}

%describe how we come up with the prompts 

%describe why walkthrough needs a different approach 

%describe few-shot data written by the team

%describe hypotheses and which conditions are compared 





% %%%PROMPTS for H1-Example
% \begin{table}[]
% \resizebox{\textwidth}{!}{
% \begin{tabular}{|l|l|} 
% \hline
% GPT Prompt for H1-A                                                                   & GPT Prompt for H1-B                                                      \\ \hline
% \begin{tabular}[c]{@{}l@{}}

% [Few-shot examples]\\

% Instructions:

% Talk to a friend about the topic: [topic] in the domain: [domain].\\
% Explain how the topic works.\\
% Use the given example to help explain how the topic words: [use case]. \\
% Use the scenario to provide additional context: [scenario].\\

% \vspace{0.5}
% Output format: a piece of writing with short paragraphs (280 characters for each paragraph).\\

% Walkthrough Guidelines:\\
% Tell the story from a first-person perspective.\\
% Walk through the story timeline in a sequence.\\
% Be sure to explain each dimension of the topic in detail, \\
% relating it back to the given use case and scenario.\\


% Emotional Guidelines:\\
% Take the second-person audience on an emotional journey.\\
% Add visually descriptive details in the storytelling.\\
% Use emotional languages (both negative and positive).\\
% Add questions that echo with the audience and spark curiosity.\\


% Data:\\
% Domain: [domain]\\
% Topic: [topic]\\
% Use Case: [use case]\\
% Scenario: [scenario]\\
% Twitterorial:\\
% \end{tabular} & \begin{tabular}[c]{@{}l@{}}

% Narrative: [GPT output from H1-A]\\

% Instructions:\\
% Keep the same tone and structure as the given narrative.\\
% You are given a science narrative that explains how [topic] works. \\
% Remove the example of [example_label] from the narrative.\\
% Do not include ANY examples.\\
% Only provide a technical walkthrough of [topic] following the same structure.\\

% \end{tabular} \\ \hline
% \end{tabular}
% }

% \caption{GPT Prompts for H1 Readers prefer Twitter threads that have a relatable example.}
% \label{PROMPT_Example_Conditions}
% \end{table}

% %%%PROMPTS for H2-Walkthrough
% \begin{table}[ht]
% \resizebox{\textwidth}{!}{
% \begin{tabular}{|l|l|} 
% \hline
% GPT Prompt for H2-A                                                                   & GPT Prompt for H2-B                                                      \\ \hline
% \begin{tabular}[c]{@{}l@{}}

% Instructions:

% Talk to a friend about the topic: [topic] in the domain: [domain].\\
% Explain how the topic works.\\
% Use the given example to help explain how the topic words: [use case]. \\
% Use the scenario to provide additional context: [scenario].\\

% \vspace{0.5}
% Output format: a piece of writing with short paragraphs (280 characters for each paragraph).\\

% Walkthrough Guidelines:\\
% Tell the story from a first-person perspective.\\
% Walk through the story timeline in a sequence.\\
% Be sure to explain each dimension of the topic in detail, \\
% relating it back to the given use case and scenario.\\


% Emotional Guidelines:\\
% Take the second-person audience on an emotional journey.\\
% Add visually descriptive details in the storytelling.\\
% Use emotional languages (both negative and positive).\\
% Add questions that echo with the audience and spark curiosity.\\


% Data:\\
% Domain: [domain]\\
% Topic: [topic]\\
% Use Case: [use case]\\
% Scenario: [scenario]\\
% Twitterorial:\\
% \end{tabular} & \begin{tabular}[c]{@{}l@{}}

% Instructions:\\
% Write a series of Twitters explaining the given topic: [topic] in the domain of [domain].\\
% Make sure each Twitter is less than 280 characters.\\

% Do not use technical jargon and define all technical components.\\
% Do not walkthrough using timeline sequence. Do not use words \\such as "before", "after", "then", "next", "also", "first", "second", "third", "last", "summary".\\
% Explain a different technical component of the topic in each Twitter\\ in a non-sequential modular approach.\\
% Each Twitter stands alone and allows the reader to navigate through the explanations \\in various orders.\\


% \end{tabular} \\ \hline
% \end{tabular}
% }




% \caption{GPT Prompts for H2 Readers prefer Twitter threads that have a step-by-step walkthrough of one example. }
% \label{PROMPT_Walkthrough_Conditions}
% \end{table}

% %%%PROMPTS for H3-Personal Language
% \begin{table}[]

% \resizebox{\textwidth}{!}{
% \begin{tabular}{|l|l|} 
% \hline
% GPT Prompt for H3-A                                                                   & GPT Prompt for H3-B                                                      \\ \hline
% \begin{tabular}[c]{@{}l@{}}

% Same as H1-A
% \end{tabular} & \begin{tabular}[c]{@{}l@{}}

% Narrative: [GPT output from H3-A]\\

% Instructions:\\
% You are given a science narrative that uses emotional and engaging language to explain a concept.\\
% Your task is to write a new narrative that maintains the same structure of the given narrative \\but removes all emotional language and all subjective language.\\
% Edit the sentences to remove you, I, and we pronouns. Do not use rhetorical or confirmation questions.\\
% Maintain the active voice and use "people" or "student" or other general terms as the subject.\\
% Use objective and generalizable language wherever possible.\\
% Remove any extraneous descriptions and adjectives.\\
% Use formal language.\\


% \end{tabular} \\ \hline
% \end{tabular}
% }




% \caption{GPT prompts for H3 Readers prefer Twitter threads that use personal language.}
% \label{PROMPT_Personal_Conditions}
% \end{table}







% \subsection{Participants}


% recruitment: rationale for choosing non-STEM students + recent graduates. recruitment methods. 

% We recruited 35 undergraduate non-experts to evaluate the generated narratives to determine the quality of the science explanations. We choose non-experts because they represent everyday people, who are our target audience for science communication on social media. 


\subsection{Participant Recruitment}
\label{Recruitment}
\label{user-study1}
We recruited 35 undergraduate students and recent college graduates who did not study nor intend to study any of the 5 science fields. We chose non-experts because they represent everyday people, who are our target audience for science communication on social media. Because the Tweetorials that we aim to emulate are US-centric in the examples and language used, we disqualified participants who do not self-identify as culturally American and whose first language is not English. The participants are from 12 universities and liberal arts colleges in the United States, with an average age of 22, and a gender distribution of 4 males, 29 females, and 2 non-binary individuals. We distributed the recruitment survey via school mailing lists, Slack workspaces, Discord channels, and snowball sampling \cite{goodman_snowball_1961} among schoolmates of the participants. Each participant was compensated \$27 dollars. The study was approved by our institutional IRB. 
% \yy{AGE:  use average age}



% Following the recruitment, qualified participants attended a 15-minute onboarding session via Zoom with an experimenter. After that, they were given a link to the Qualtrics survey described in Section \ref{Data}, which took approximately 1.5 hours to complete on their own. T

\subsection{Survey Procedure}
Following recruitment, each qualified participant attended an onboarding session with an experimenter on Zoom. The experimenter explained the study procedure and data collection, and acquired participants' consent. The experimenter shared a sample Twitter thread and survey for the participant to answer and explained the Likert-scale rating criteria. The participants thought aloud while answering the sample survey, justified their rating decisions, and asked clarifying questions before completing the survey on their own. Participants' justifications for their ratings on the sample science narratives aligned with the defined definitions. 

We implemented our survey on Qualtrics, an online survey platform. On the first page, participants are asked to read a Twitter thread. The page included a one-minute timer to prevent participants from skimming or skipping the reading process. After reading the Twitter thread, the participant advances to the next page and completes four Likert-scale questions that correspond to the four evaluation dimensions (engaging, relatable, understandable, and easy-to-follow, see Section \ref{survey_design}), on the scale from 1 to 5 (1-Strongly Disagree, 5-Strongly Agree). This process repeats 15 times for the 15 science explanations for each participant. 

To minimize bias from prior knowledge, we employed a between-subjects design, ensuring that each participant only read each topic once, and no participant read the same topic under two different conditions. Each participant read a randomized selection of Twitter threads across all conditions and topics. Each Twitter thread was evaluated by 7 participants to achieve a statistical distribution and reduce individual bias.

\subsection{Data Collection and Analysis}
\label{Data}
% grace unsure if we need this??
% Participants rated science explanations on 4 dimensions (engaging, relatable, understandable, and easy-to-follow) on a Likert Scale of Strongly Agree (5-points) to Strongly Disagree (1-point). For participant's rating, we summed the scores for each of the dimensions to create an overall score (20-points max) for the science explanation. Each participant evaluated 15 different science explanations across different topics in STEM.  We used the overall score to get a comprehensive measure that captures the participants' overall perception across multiple dimensions of quality.

\subsubsection{Survey Design}
\label{survey_design}
Each science explanation was evaluated on 4 different questions: (1) I find the thread engaging, (2) I find the thread relatable, (3) I find the thread understandable, and (4) I find the thread easy to follow. \textit{Engaging} refers to the language, while \textit{relatable} refers to the example used. \textit{Understandable} refers to the presence of technical jargon, while \textit{easy-to-follow} refers to the sequence and flow of the narrative. The onboarding session demonstrated that participants were able to differentiate between each of the dimensions. We sum the participant's ratings for each Likert-scale question to create an overall score for the given science explanation (20-point max score).

% \grace{go somewhere else}
 
% For engaging explanations, participants mentioned the language that was used in the text that captured or held their attention. For relatable explanations, participants usually referenced how the example used in the explanation connected to their own lives. For understandable explanations, participants mentioned how they were able to understand the technical concepts and terms that were used in the text. For easy-to-follow explanations, participants mentioned how the sequence of information followed a clear order and structure. \grace{we define these things as xyz, in onboarding we wanted to ensure aligned of defintioin and found tnat they fid algn with partiicpants }



% \begin{itemize}
%     \item \textbf{Engaging}: covers the science explanation's ability to capture and hold a reader's attention. Evaluated by the language used in the explanation. 
%     \item \textbf{Relatable}: reflects how well the explanation connects to the reader's existing experiences, knowledge, or concerns. Evaluated by the examples used in the explanation.  
%     \item \textbf{Understandable}: measures whether the explanation successfully conveys its ideas in a way the reader can comprehend. Evaluated by the sentences used to present the science, such as technical jargon or assumed prerequisite knowledge in the explanation. 
%     \item \textbf{Easy-to-Follow}: indicates the clarity of the explanation's structure, flow, and logical progression. Evaluated by the structure of the explanation.
% \end{itemize}

% These four dimensions are non-overlapping. Engaging is about reader interest and emotional draw which is different from clarity of content (understandable), personal connection (relatable), or structural coherence (easy-to-follow). For example, a science explanation can use describe and vivid verbs to engage the reader, but fail to explain the science behind a topic which makes it not understandable. \grace{Should I continue this for each of the dimensinos? How much more justification do we need?}


% \grace{do we need this??} We use the following phrases in our survey:
% \begin{enumerate}
%     \item I find the thread engaging.
%     \item I find the thread relatable.
%     \item I find the thread understandable.
%     \item I find the thread easy to follow.
% \end{enumerate}



% The 4 Likert Scale questions evaluate 4 different dimensions of what makes a good Tweetorial: engaging, relateable, understandable, and easy-to-follow. 

% We use the following phrases 


% STILL NEED TO DO \yy{add rationale of the 4 Likert-Scale questions. }


% \subsubsection{Survey Procedure}


% % for consent process: users are asked to confirm participation in qualtrics intake survey + again in onboarding process where I re-state the expectations of the survey (15 tweetorials, 1-1.5 hours, $27, +deadline) before continuing with the survey

% 15-minute onboarding sessions with all participants to walk them through the process of answering one set of twitter thread + question pairing to ensure that annotators understand the rating criteria + were also asked to think aloud to ensure that they had a justification for their rating. after this, raters would complete the rest of the survey in their own time


% \yy{add consent process}

% \yy{add onboarding procedure}





% When distributing the survey, we took the following steps to prevent bias.

% Each participant read and evaluated 15 Twitter threads on 15 different topics. 

% To mitigate prior knowledge bias, we adopted a between-subjects method, such that no participant read the same topic from two different conditions. 

% Each participant read a randomized mixture of the 5 conditions. \yy{check w/ Grace -- Did we randomize?}

% Each Twitter thread is evaluated by 7 participants to reach a statistical significance and eliminate human bias. 





\subsubsection{Data Analysis}
\textbf{Quantitative Analysis: Survey Data} 
% To understand readers’ preferences for science explanations that incorporate specific techniques, we designed a study involving 35 participants. Each participant rated 15 different explanations on 4 different Likert scale questions. Because the Likert scale questions were unique and non-overlapping, we summed the ratings to create a overall score for each science narrative. Each explanation was independently evaluated by 7 different raters. We covered five distinct STEM topics (Physics, Computer Science, Statistics, Psychology, and Civil Engineering) to capture a broad range of content.

Our main goal was to test whether the inclusion of examples, walkthroughs, and personal language influenced readers' engagement and understanding. We used the overall score (sum of the 4 Likert responses) for the given science explanation in our analysis. Because each participant did not read the same topic twice and sampled across multiple STEM fields, our data exhibited a hierarchical structure: each rating is linked to a specific participant and a specific topic.

We used a Generalized Linear Mixed Model (GLMM) for our analyses. The GLMM framework allowed us to isolate how each technique affected the ratings for each science explanation based on the overall score of the Likert ratings, while statistically controlling for other factors. We can estimate the direction and magnitude of the effectiveness of each technique (examples, walkthroughs, and personal language) relative to explanations without them. We also accounted for individual differences in baseline rating tendencies by including random intercepts for each participant because people may vary in their strictness or leniency when rating explanations. We also included random intercepts for each of the five STEM topics to account for potential differences across fields (e.g., participants might rate Computer Science explanations differently than Physics explanations). The GLMM model allowed us to compare pairs of explanations—those with a given technique versus those without—to determine whether the presence of each technique had a significant effect on the ratings.
  

\textbf{Qualitative Analysis: Followup Interviews}
To gain more nuanced insights into participant preferences in the survey data, we randomly reached out to 17 of the 35 participants for follow-up interviews. 8 of 17 participants opted-in for a 15-minute follow-up interview on Zoom. The participants are compensated \$10 total. In the follow-up interview, we asked the participants to re-read a science explanation they had previously read and explain in detail specific instances from the science narrative that influenced their Likert ratings.  The interviews were recorded and transcribed before three researchers conducted a thematic analysis. We used grounded theory to derive insights from the interview transcripts \cite{charmaz_constructing_2006}. The first author read through all interview transcripts, inductively derived a preliminary set of codes, and then grouped the codes based on themes. The first author and two additional authors collaboratively reviewed and refined the themes until a consensus was reached.

% The purpose of these interviews was to gain deeper insights into the survey findings and explore participants’ perspectives on key themes that emerged from the quantitative data. 

% To compare the experimental condition with the baseline condition for each hypothesis, we calculated an average rating across 105 data points (15 topics × 7 participants). Then, we calculated a sum of the four Likert-Scale questions to get a total score. The total scores from Condition A and B are directly compared to conclude readers' preference. 
% To understand inter-annotator agreement among the participants, we calculated Krippendorff's alpha score (k-alpha). We used K-alpha to calculate the inter-annotator agreement due to it's flexibility in calculating agreement between multiple annotators. K-alpha ranges from -1 to 1 where 1 denotes perfect agreement and values below 0 signify random agreement. 
% We also calculated p-value on the total score values between each condition to determine the statistical significance of the results. 

% We calculated a mean of 7 participants' scores for each condition on each topic. 
% \yy{Grace: What other statistical analysis did we do?}

% \subsubsection{Follow-Up Interviews}










\section{Results: Reader Study}

% \subsubsection{Readers prefer science explanations with an example over no example (H1:Example)}
% Readers demonstrate a statistically significant preference for reading science explanations with an example (EWP+FewShot) over one without an example (EWP-RemoveE). Readers' overall score consisted of 4 Likert scale questions the measure engagement, relatability, easy-to-follow, and   with 5 possible answer choices ranging from Strongly Agree (5 points) to Strongly Disagree (1 point). Th

% for science narratives with explanations was to science narratives without,  


\subsection{Quantitative Findings on Survey Data}
% \grace{to answer H1 example, we comapred the overall rating with EWP : WP, be more explicit about RQs and }
Overall, we collected 105 ratings on 75 different science explanations. This consisted of 35 different annotators who evaluated each science explanation on 4 different Likert-scale questions. We conduct a quantitative analysis of the survey data followed by a qualitative analysis of individual reader's preferences for science communication to contextualize nuances to the quantitative data.

% \grace{update labels}
% \grace{fill in the new stats from what juna runs}
% \grace{stars next to numbers that are statistically significant}

% \begin{table}[H]
%     \centering
%     \includegraphics[width=0.9\linewidth]{Figures/reader_survey_results.png}
%     \caption{Reader preferences for different methods of science explanations and the average scores for each rating dimension (engaging, relateable, understandable, and easy-to-follow) and the average total score for each condition.}
%     \label{fig:reader_survey_results}
% \end{table}


\subsubsection{Usage of Example Preferences}
% To answer our Hypothesis 1, Readers \textbf{do} prefer explanations with examples, EWP (Everything) compared to those without an example, EWP-RemoveE.

% Results from our survey demonstrate that readers preferred reading explanations \textbf{with} a single example, EWP (Everything), with an average score of 15.8 out of 20 points. Readers rated science explanations with the \textbf{removed} example, EWP-RemoveE, an average score of 13.0, almost a 3-point difference "close to" or "about" a 3-point difference. This result is statistically significant with a p-value of 0.000003.  (See Table \ref{table:reader-results-example}), 

To evaluate H1:Example, we compare experimental condition EWP to baseline condition WP. We found that participants prefer reading explanations with an example (EWP) over without an example (WP).
The results presented in Table \ref{tab:glmm_results_H1} illustrate the effects of the two conditions on the total score of the four survey questions (max 20 points) within a Generalized Linear Mixed Model (GLMM) framework. The experimental condition, EWP, received a score of 15.831 which shows a statistically significant increase in user preference when compared to the baseline condition of 13.003 (p < 0.0001). The effect size is 2.828 which shows that readers rated narratives with an example (EWP) almost 3 points higher than narratives without an example (WP). Additionally, the random effects analysis reveals a large group variance of 2.626 suggesting the differences among individual participants play a significant role and a small fields variance at 0.062, the specific field of study has a minimal impact. 
% The intercept score of 13.003 out of 20 signifies a strong baseline response level. 
%indicating that participants typically demonstrate a substantial consistency in their preference to the baseline condition -- science explanation without using an example.
% The experimental condition EWP received a score of 15.831 out of 20.It shows a coefficient of 2.828 (p < 0.0001) suggesting that participants prefer having an example (EWP condition) significantly better than without an example (baseline condition WP). 
%gave significantly higher scores to this condition compared to the baseline condition. 
%This positive coefficient indicates that the EWP condition provides a substantial improvement, increasing participants' preference effectively.
%Additionally, the random effects analysis reveals a group variance of 2.626, indicating considerable variability among individual participants. The fields variance is minimal (0.062), suggesting that while individual differences play a significant role, the specific field of study has a lesser impact. Overall, these findings highlight the positive influence of the EWP condition on participant preference, reinforcing its effectiveness in improving science explanation.



% \yy{no p-value for intercept. Add total of intercept + condition cofficient ===> handholding readers}

\begin{table}[h]
    \centering
    \caption{GLMM Results for H1:Examples}
    \begin{tabular}{@{}lccccc@{}}
        \toprule
        \textbf{Effect}                  & \textbf{Score (max. 20)}& \textbf{Coefficient} & \textbf{Standard Error} & \textbf{z-value} & \textbf{p-value} \\ \midrule
        Intercept[WP]                        & 13.003 & 13.003               & 0.483                   & 26.937            &           \\ \midrule
        CONDITION[EWP]       & 15.831 & 2.828                & 0.546                   & 5.183             & 0.000          \\ \midrule
        \textbf{Random Effects}   &        &                       &                          &                   &                   \\ 
        Group Variance    &        & 2.626                & 0.752                   &                   &                   \\ 
        Fields Variance      &            & 0.062                & 0.543                   &                   &                   \\ \bottomrule
    \end{tabular}
    \label{tab:glmm_results_H1}
\end{table}


\subsubsection{Step-by-Step Walkthrough Preferences}
% To answer Hypothesis 2 (H2), we cannot conclude that readers prefer the step-by-step explanation structure, EWP (Everything) - NoFewShot, over the multiple unrelated examples structure, Ep - NoFewShot.

% According to Table \ref{fig:reader_survey_results}, readers preferred science narratives with a step-by-step walk through with an average score of 15.7 out of 20. Readers rated science narratives \textbf{without} a step-by-step walkthrough an average score of 14.8. The less than 1-point difference is not statically significant (p>0.05). 

To evaluate H2:Walkthrough, we compare experimental condition EWP-NoFewShot to baseline condition EP-NoFewShot. We found that readers had a slight preference for explanations with a walkthrough compared to without a walkthrough, with a considerable variance on the fields of study.
Table \ref{tab:glmm_results_H2} summarizes the findings from a GLMM analysis. The experimental condition, EWP-NoFewShot received a score of 15.965 which was not statistically significant compared to 14.729, the baseline score for EP-NoFewShot (p = 0.057). As such, there is no significant difference in how readers rated explanations with and without walkthroughs. The random effects analysis shows a minimal group variance of 0.002, indicating little variability among participants. However, the field variance is considerable at 2.566, suggesting that participants prefer having a walkthrough for some STEM topics but prefer having no walkthrough for other STEM topics. 
%with the intercept corresponding to the baseline condition EP-NoFewShot. 
% The intercept score of 14.729 out of 20 indicates a strong baseline response level. %indicating that participants overall demonstrate a high consistency in their level of preference to the baseline condition -- science explanation without a step-by-step walkthrough.
% The experimental condition EWP-NoFewShot received a score of 15.965 out of 20. Its coefficient is 0.966 (p = 0.057), which suggests that while participants prefer having a walkthrough (EWP-NoFewShot condition) better than without a walkthrough (EP-NoFewShot condition), the statistical significance is only marginal. 
%while the total response is higher in the EWP-NoFewShot condition compared to the EP-NoFewShot condition, this difference is only marginally significant.
%This indicates a potential positive impact of the EWP-NoFewShot condition on participant responses, though further investigation may be needed.
%though follow-up qualitative interviews may reveal the nuances in participants' 
%may be necessary to confirm its statistical significance.
%differences in field of study contribute significantly to variations in total responses. 
%Overall, these findings indicate that while the EWP-NoFewShot condition may enhance responses relative to the EP-NoFewShot baseline, the effect is not conclusively significant, warranting further exploration. 
Our qualitative findings (Section 5.2) provide further explanations on why some participants are divided in their preferences for walkthrough.  


\begin{table}[h]
    \centering
    \caption{GLMM Results for H2:Walkthrough}
    \begin{tabular}{@{}lccccc@{}}
        \toprule
        \textbf{Effect}                 & \textbf{Score (max. 20)} & \textbf{Coefficient} & \textbf{Standard Error} & \textbf{z-value} & \textbf{p-value} \\ \midrule
        Intercept[EP-NoFewShot]               & 14.729         & 14.729               & 0.439                   & 33.537            &        \\ \midrule
        CONDITION[EWP-NoFewShot]    & 15.695    & 0.966                & 0.508                   & 1.902             & 0.057          \\ \midrule
        \textbf{Random Effects}   &       &                       &                          &                   &                   \\ 
        Group Variance     &       & 0.002                & 0.618                   &                   &                   \\ 
        Fields Variance        &          & 2.566                & 0.506                   &                   &                   \\ \bottomrule
    \end{tabular}
    \label{tab:glmm_results_H2}
\end{table}





\subsubsection{Personal Language Preferences}
% To answer our Hypothesis 3 (H3), readers do prefer science explanations that have use personal language, EWP(Everything) over explanations that do not use personal language, EWP-RemoveP.

% According to Table \ref{fig:reader_survey_results}, readers prefer reading science narrative \textbf{with} personal language, with an average score of 15.8 out of 20, as opposed to narratives with personal language removed which scored an average of 13.9 out of 20 points. The difference is statistically significant with a p-value of 0.0007. 

% The EWP (Everything) condition outperformed the EWP – RemoveP condition by 0.5 points across all dimensions.

To evaluate H3:Personal Language, we compare experimental condition EWP to baseline condition EW. We found that participants prefer reading explanations with personal language over without personal language. 
Table \ref{tab:glmm_results_H3} summarizes the findings from a GLMM analysis. The experimental condition, EWP, received a statistically significantly higher score of 15.883 compared to 13.973 for the baseline condition, EW (p < 0.0001). The effect size is 1.910 which means that readers rate explanations with personal language almost 2 points higher than explanations without personal language. Additionally, there is a group variance of 1.392 indicating a moderate level of variability among participants and a fields variance of 1.361 showing a moderate variance among STEM topics. 

% The intercept score of 13.973 out of 20 indicates a significant baseline level for participants in the EW condition.
% The experimental condition EWP received a score of 15.883 out of 20. The coefficient is 1.910 (p < 0.0001), suggesting that participants prefer having personal language (EWP condition) better than without personal language (EW condition) reaching a statistical significance. 
%participants in the experimental condition gave significantly higher scores compared to those in the baseline condition. This positive coefficient indicates that participants exhibit substantially higher preference to the EWP condition. 

%, indicating a moderate level of variability among participants, while the field variance is also notable at 1.361. This suggests that both individual differences and variations in fields of study play significant roles in influencing total scores. 
%Overall, these findings highlight the positive impact of the EWP condition relative to the EW baseline, reinforcing our hypothesis.

\begin{table}[h]
    \centering
    \caption{GLMM Results for H3:Personal Language}
    \begin{tabular}{@{}lccccc@{}}
        \toprule
        \textbf{Effect}        & \textbf{Score(max. 20)}          & \textbf{Coefficient} & \textbf{Standard Error} & \textbf{z-value} & \textbf{p-value} \\ \midrule
        Intercept[EW]      & 13.973                  & 13.973               & 0.463                   & 30.175            &           \\ \midrule
        CONDITION[EWP]   &15.883     & 1.910                & 0.526                   & 3.632             & 0.000          \\ \midrule
        \textbf{Random Effects}   &       &                       &                          &                   &                   \\ 
        Group Variance     &       & 1.392                & 0.717                   &                   &                   \\ 
        Fields Variance       &           & 1.361                & 0.548                   &                   &                   \\ \bottomrule
    \end{tabular}
    \label{tab:glmm_results_H3}
\end{table}


% \subsubsection{Overall Reader Preference}

% \yy{add preamble and hypothesis == EWP is the best of all.}
% In addition to the three hypotheses, we want to understand how the three techniques compare to each other and whether or not using all three techniques produces a science explanation that is the most preferable to the audience (H4: All three techniques (EWP)). Table \ref{tab:glmm_results_all} shows the findings from a GLMM analyzing total scores across four conditions: WP, EP-NoFewShot and EW with the intercept representing EWP. 
%We hypothesize that EWP would have the highest score of all conditions. 
% The intercept score of 15.874 out of 20 indicates participants prefer EWP having all three techniques over other conditions. 
% Among the conditions, WP received 13.260 with a significant negative effect coefficient of -2.614 (p < 0.0001) and EW received 13.799 with a significant negative effect with a coefficient of -2.075 (p < 0.0001), indicating that participants consistently prefer having all three techniques (EWP) over science explanations without an example or without personal language. 
%suggesting that participants consistently did not prefer science explanations without an example compare to EWP(Everything). Similarly, EW also shows a significant negative effect with a coefficient of -2.075 (p < 0.001), indicating a notable decrease in preference compared to EWP. 
% In contrast, the no walkthrough condition EP-NoFewShot shows a slightly higher score at 14.784 with a coefficient of -1.090 (p = 0.075), suggesting that participants potentially did not prefer having no walkthrough compared to having all three techniques, but further investigation is needed. Our qualitative results in the next section provide some further explanations on this. 
%The condition EP-NoFewShot has a coefficient of -1.090, which approaches significance (p = 0.075), implying a potential decrease in preference without a walkthrough in the explanation that may warrant further investigation. \yy{clarify the writing}
% Conversely, the EWP-NoFewShot condition exhibits a coefficient of -0.043 (p = 0.942), indicating no significant difference in responses compared to EWP(Everything).

% \yy{do we need random effects here?}
% \revision
% {The random effects analysis indicates a group variance of 1.477, highlighting the variability in responses attributed to individual differences. The covariance terms related to field variations indicate how different fields of study may interact with the group, suggesting complex relationships between random effects. The variances associated with different fields (e.g., Computer Science, Physics, Psychology, Statistics) further emphasize the importance of field context in influencing total responses. Overall, these results underscore the negative impact of certain conditions on participant performance while highlighting the potential complexity introduced by random effects associated with PID and FIELD.}
% \begin{table}[h]
%     \centering
%     \caption{GLMM Results for all conditions}
%     \begin{tabular}{@{}lccccc@{}}
%         \toprule
        
%         \textbf{Effect} & \textbf{Score(max. 20)} & \textbf{Coefficient} & \textbf{Standard Error} & \textbf{z-value} & \textbf{p-value} \\ \midrule
%         Intercept[EWP] & 15.874 & 15.874 & 0.515 & 30.829 &  \\ \midrule
%         CONDITION[WP] &13.260 & -2.614 & 0.589 & -4.436 & 0.000 \\ 
%         % CONDITION[EWP(Everything)-NoFewShot] & -0.043 & 0.595 & -0.072 & 0.942 \\ 
%         CONDITION[EP-NoFewShot] &14.784 & -1.090 & 0.612 & -1.780 & 0.075 \\ 
%         CONDITION[EW] &13.799& -2.075 & 0.568 & -3.655 & 0.000 \\ \midrule
%         % \textbf{Random Effects} & & & & \\ 
%         % Group Variance & 1.477 & 0.587 & & \\ 
%         % FIELD[Computer Science] Variance & 3.473 & 0.892 & & \\ 
%         % FIELD[Physics] Variance & 1.609 & 2.368 & & \\ 
%         % FIELD[Psychology] Variance & 3.633 & 0.943 & & \\ 
%         % FIELD[Statistics] Variance & 2.562 & 0.995 & & \\ \bottomrule
%     \end{tabular}
%     \label{tab:glmm_results_all}
% \end{table}




\subsection{Qualitative Findings on Reader's Preferences}

% The findings from the GLMM demonstrate (1) strong audience preferences for examples in science communication with considerable variability among individuals, (2) marginal preferences for a walkthrough in science communication with little variation among individuals, and (3) substantial audience preferences for personal language in science communication with a moderate variation in individual preferences. 
% \grace{need to edit this to be reflective of the findings. people preferred science explanations with examples and personal langauge, but were split on walkthroughs}
Overall, participants preferred science explanations with an example over no example and with personal language over no personal language. However, participants were split on the their preference for narratives with and without walkthroughs. We use semi-structured interviews to gain further insights into the reasons why participants might prefer narratives with or without an example, walkthrough, and personal language.

% People preferred science explanations with an example, a step-by-step walkthrough, and personal language, but there are nuances to each reader's ratings. To explore the nuances of reader preferences, we conducted 

% We identified three themes in reader's preferences as it relates to the usage of examples, structure, and personal language when explaining science to the public:
% \grace{one-two word: sentence, Relatable Example: reader's personal experiences affect score}

% \begin{itemize}
%     \item Theme 1: Relatable example: Reader's relatedness to an example affects their engagement with the topic.
%     \item Theme 2: Walkthrough structure: Readers sometimes prefer broad overviews of a topic.
%     \item Theme 3: Personal Language: Readers sometimes prefer formal and academic explanations of a topic over personal language.
% \end{itemize}


\subsubsection{Most Readers Preferred Examples}
Overall, readers reported that having an example was helpful. 4 of 8 readers stated the example helped them understand the importance of a topic (P2, P4, P7, P8). 5 of 8 found that examples helped them reflect and understand their own experiences better (P1, P2, P3, P4, and P7). When reading a science explanation about the computer science topic of depth-first search, P2 remarked that learning how the algorithm can be applied to navigating a maze helped her engage with the topic: ``I don't really care about computer science algorithms, but I do care about how this applies to my own life."

However, not all readers needed an example to help them understand the topic. 3 of 8 participants mentioned how the example felt unnecessary and detracted from the content of the explanation (P1, P2, P8). When reading about depth-first search, P1 mentioned how the given example of navigating through the Botanical Garden felt unnecessary: ``The concept in and of itself is interesting. I would have just read about that. I don't really need any more context." For P8, when learning about thin film interface through the example of a child playing with bubbles, he remarked that ``[the example] doesn't really pertain to me in any way," demonstrating how certain examples might not resonate with particular audiences.

3 of the 8 participants mentioned how they would reference their own experiences to help ground the technical explanation (P1, P2, P4). When P2 was reading about thin film interference without an example, she used her own experience in working with glass to ground her understanding of the topic: ``[the topic] was really relevant to my life and what I'm already doing." For P4, when the given example of watching a horror film was used to explain Walker's Action Decrement Theory, she mentioned how even though the example was unrelatable, she referenced her own experiences to find an example that fit the same context. This shows that readers use their own experiences to contextualize the science for explanations that do not include an example or that include an example unrelatable to the reader.
 
 % even when an example is included in the narrative, if the example is unrelateable  when a narrative used an example that was unrelatable to the reader, This demonstrates shows how some people liked narratives without examples because it allows them to fill in their own experiences. 

 % These findings illustrate how examples can help ground the science in something tangible to help readers to understand a topic. But without an example, some readers would reference their own experiences to understand the topic demonstrating how examples are not necessary for all readers. Some readers also felt that an example was unnecessary for the complexity of a topic and convoluted the explanation showing how for some topics an example might not be neeeded. 

\subsubsection{Readers had No Preference for Narratives with Walkthroughs}
There was no statistical significance in our results comparing reader preferences for narratives with and without walkthroughs. In this section we will investigate why some readers prefer walkthroughs or no walkthroughs. 
 
Some readers reported that the walkthrough helped establish a structure for the explanation (P2, P3, P8), provide evenly paced information (P1, P3, P6), and helped them follow through with the science (P4, P6, P7). P3 appreciated the walkthrough for the topic of Walker's Action Decrement Theory in Psychology because it helped them understand how their heightened emotional state when watching a horror movie might affect their memory. By walking through 3 different stages of the narrator's emotional journey during this event and how it affects their memory, P3 mentioned how the narrative helped ``set the stage" and ``allows you to follow through with [the science explanation]." The walkthrough provided a scaffold to support the reader in processing the information as a sequence of events. 

But not all readers preferred narratives with a walkthrough. 3 participants (P2, P3, P7) reported that the walkthrough structure made the explanation feel overly explanatory or repetitive. For the topic of Walker's Action Decrement Theory, P2 said that the walkthrough narrative was ``just overly explanatory for a concept that is very intuitive," demonstrating how for certain topics a walkthrough might not be necessary.

4 participants (P2, P4, P5, P6) preferred science explanations without a walkthrough but with many examples because they provided multiple different angles to view a topic in a condensed space. P4 stated that the explanation without a walkthrough broke down the topic of the curtain wall system into smaller, self-contained chunks of information that made reading the explanation ``less intimidating." For example, one paragraph of the explanation focused specifically on the aspect of temperature control, while the subsequent paragraph explored the structural construction of the curtain wall. P4 said that having the information broken down in this way made it more ``digestible'' to learn about the topic. For P6, the ``list of facts'' structure of explanation without a walkthrough helped them gain a broad overview of the topic. This illustrates that for some topics, readers might not want an in-depth walkthrough of the science and might prefer getting a high-level overview of how the science works. 

 % Walkthroughs can be helpful for providing scaffolding for readers to read about the science, but is not always preferred by readers. Some topics might not require in-depth walkthroughs and some readers might prefer just getting a broad understanding of the topic. 

 % This demonstrates that for certain topics, readers might not want an indepth walkthrough of the science. Furthermore, science explanations without a walkthrough also helped certain readers learn about different aspects of the topic, making the reading process more engaging and less intimidating. 


 \subsubsection{Personal Language Sometimes Distracts from the Science}
 Overall, readers preferred reading explanations with personal language. Participants reported that the personal language helped establish a writer/reader connection (P1, P3, P5). But not all readers preferred narratives with personal language. 2 readers preferred reading science explanations with no personal language because they believed personal language was unnecessary (P1, P8): ``[the explanation is] full of fluff coming from a personal perspective that I didn’t really care about" (P1). P8 mentioned that he did not need personal language for engagement when reading about science, additionally, he hypothesized that other readers ``[might] need the personal aspects to get them to read something that they wouldn't otherwise be interested in." Depending on a reader's inclination towards science, personal language may or may not be necessary to help them engage with these science narratives. 
 %how when he is reading about science, he doesn't need the personal language to get him engaged and 

 
 % Without personal language, readers often found that science explanations overly technical and hard to understand (P1, P6, P7). 

 % For two readers (P1, P7), the presence of technical jargon often triggered an emotional aversion to continue reading. P1 said that ``I read ``linear regression", and I wouldn't be interested in it." Similarly, P7 said ``when I saw those bigger words that I was like, ``Oh, that's scary to read,"" demonstrating how technical jargon can trigger strong aversions in readers.  Furthermore, when technical jargon is paired with scientific concepts, P7 mentioned how the overall explanation became more intimidating to read. \grace{is this the place to situate these findings in other works?}

 

% This demonstrates that the presence of personal language in science explanations depends heavily on the reader's comfort engaging with STEM topics. Without personal language, explanations can trigger strong emotional aversions to the science for some readers, while for other readers, they would prefer only reading about the science.
 
 
 % For example, one section might focus specifically on the aspect of temperature control, while the subsequent paragraph explores the structural construction of the curtain wall. 

 
% \grace{time to shape a narrative, what is surprising and interesting things here: 

% examples: takeaway (one example where it is good, but anchor them in a positive case), what are the interesting themes around what happened generally examples are good, but there are exceptions 

% examples are bad when examples are too childish... [lydia needs to be convinced of this] (how many people said each thing)

% someitmes people liked the narratives without the examples because they could fill in their own (how many people said each thing)

% what are things that multiple people said: 
% }

% For P1, P2, P3, P4, and P7, examples helped them better understand their own experiences demonstrating how examples can help make science more relatable: ``This is a specific scenario that I can connect to and  apply this to my life, and imagine and make sense of" (P7). When examples were not present, P4 and P6 mentioned how they were unsure how the topic applies to real life: ``I wasn't really sure when you would like see this in real life" (P4). Furthermore, P4 and P5 explicitly mentioned how they wanted to have examples when none were present in the EWP-E narrative: ``I wish I had more examples instead of just 'the vibrance around light'" (P4). 

% P3 mentioned how the use of an example helped establish her understanding of the topic: the example "set the stage [for the explanation], [the example] allows you to follow through with something". P4 mentioned "I like how it has the example, like of public transportation. So it's something that I can be like, "Okay, this applies to me."" Additionally, in the EWP-RemoveE conditions, where an example is not included in the thread, P4 mentioned how "it didn't give any examples" and how they "wished" there were examples to help explain the topic. These quotes illustrate how the use of examples helps improve a reader's engagement and understanding of a topic. 

% However, the GLMM results show that there is considerable variability in individual preferences for science narratives with and without examples. To better understand the different preferences that readers have for science narratives with and without examples, we identified 2 different dimensions to these preferences:
% \begin{itemize}
%     % \item Dimension 1: Unrelateable examples don't always negatively affect readers' understanding.
%     \item Dimension 1: Readers often reference their own experiences to create examples when no explicit examples are included.
%     \item Dimension 2: Examples can hurt readers' engagement and understanding of a topic.
% \end{itemize}

% But the experiences of readers with the specific examples that are used vary in two different dimension (1) relatability of a given example to the reader and (2) the use of a given example. For example, some readers can find the example unrelatable, but still find the science explanation understandable. 

% \grace{for readers who didn't like the example, the narrative was not relateable, but still understandable. whether example is relateable, whether narrative is understandble, and whether}

% \textbf{Dimension 1: Unrelateable examples don't always negatively affect understanding}

% For P8 that did not relate to an example that was used in the EWP (Everything Condition), he still found the science explanation understandable. The reader stated that he could not relate to the given example of the narrator playing with bubbles with his child because “I do not have a kid and I do not play with bubbles.” But even though the example was unrelatable, he found that the structure of the explanation was “presented in a logical way” that helped him understand the topic.


% \textbf{Dimension 1: Readers often reference their own experiences to create examples when no examples are included.}

%  For P2 reading an EWP-RemoveE science explanation, she said that she would slightly disagree that the thread was engaging and relatable because “… because [there] weren’t any examples. I wasn’t sure when [I] would see this [phenomena].” But the lack of an engaging and relatable example did not detract from P2’s understanding of the technical components because “there was a good mix of specific concepts that was not overly technical.” Besides the engagement and relatability score which the reader rated a 2 out of 5, the reader rated the understandablility and easy-to-follow structure a 4 out of 5. 

% \grace{"a relatability score of 2 out of 5"}

% The lack of an example does not always detract from a reader’s understanding of the science topic. P4 mentioned how the use of thought-provoking questions like: ``Have you ever wondered why you sometimes forget significant events as they are happening, only to remember them more clearly later on?" (P4), allowed her to reflect on her personal experiences and form a connection with the science explanation, even when she could not relate to the given example. This demonstrates that the language used in the science narrative can prompt readers to reflect and identify an example from their own lives to use and relate to the concept.

% For P2, her own personal interest in a certain topic area made them more likely to keep reading a science explanation without an example. Even though there was no explicit example used to explain the concept of thin film interference in physics, the reader’s own experience of ``working with colored glass and other materials right now” allowed them to form more specific connections with the topic: “[the science explanation] was really relevant to my life and what I'm already doing.” This demonstrates how readers implicitly draw upon their own personal experiences to connect with the text. Even when no explicit example was provided, this reader used her own personal interest in the topic and current work as an implicit example to supplement the explanation. 

% \textbf{Dimension 2: Examples can hurt engagement and understanding of a topic.} 

% For P1, P2, and P8 the use of examples can detract from a science explanation. One science explanation for the topic of the central limit theorem in statistics, used an example of a person trying to work with skewed transit time data. For P2, she remarked that ``I have no sympathy for the narrator. It seems like they've never taken public transit before in their lives” (P2). This demonstrates how an unrelatable example can deter the reader from reading on. Additionally, P2 also remarked that the example felt unnecessary to explain the topic, ``This is like a waste of time. It was overly explanatory for the concept,” demonstrating how some topics might not need an example to help explain it. 

% P1 said how she didn't find the given science explanation on linear regression ``engaging at all. It seemed to me, [the explanation] was trying to apply, a statistical application in the real world... I think the point was that you can use data to help you make decisions, but people don't really make decisions that way. I just didn't find it relatable or predictable" (P1). This demonstrates that the use of an example in for this reader felt contrived and did not help them understand or engage with the science narrative.

% For P8, the use of the example is unnecessary and doesn't add to his understanding of the topic of thin film interference stating: "the first two paragraphs of the [science narrative] is kind of nice, but it doesn't really do anything for me...It doesn't really pertain to me in any way. I don't play bubbles and I don't have kids" (P8). This demonstrates that an unrelateable example provides unnecessary information that isn't needed to help the reader understand the science.

% Overall, most readers like science narratives that contain an example to help them understand a topic. But the importance of examples in helping improve a reader's understanding of the topic wasn't universal. In our followup interviews, we found that (1) that some people were able to draw upon their own experiences to create a relatable example, and (2) some people were deterred when a science narrative used example they did not relate to. 

% \subsubsection{Theme 2: Walkthrough structure: Readers sometimes prefer broad overviews of a topic.}
% \grace{there was no statistical signfiicance, why wasn't the walkthrough was always preferred, there were some topics that they didn't need it. one example that they didn't. 

% remove the dimensions, each heading is the takeaway, the first paragraph is really short, and then one or two paragraphs that talk about the exception}

% The GLMM model showed a slight but insignificant reader preference for science narratives with a walkthrough. In our followup interviews we explored the different reasons why people preferred each of these structures. We identified 2 different dimensions that readers mentioned when choosing between these two narrative structures:
% \begin{itemize}
%     \item Dimension 1: Readers prefer step-by-step walkthroughs to understand a topic in-depth.  
%     \item Dimension 2: Readers prefer non-sequential walkthroughs to understand a broad overview of a topic.
% \end{itemize}

% \textbf{Dimension 1: Readers prefer step-by-step explanations to understand a topic in-depth.}
% P2, P3, and P8 mentioned how the clearly established structure of step-by-step walkthroughs was helpful: ``It laid out the steps that I would have to take" (P2), ``well organized" (P8), and ``You can follow a sequence" (P6). More specifically, for P4, P6, and P7, the story-like structure of step-by-step walkthroughs to help them follow the science: ``the narrative of telling a story, going from like broad to specific. It flowed nicely" (P4) and ``Starting at the beginning with the example: ``Say your watching a movie," then at [tweet] six it's the peak, ``it acts as a trigger." This feels like the climax of the story. Then it leads to a nice ending" (P6). Finally for P1, P3, and P6, step-by-step walkthroughs provided evenly-paced information spread to help readers understand the science: ``The information is spread throughout. There's little tidbits that help us think it over" (P1) and ``There's a good lead up to the explanations and how they make you feel" (P6).

% P3 mentioned how they liked the clear sequence of the explanation: "it was very straightforward and explanatory, and sort of laid out the steps that we would have to take if we were in a maze and had to execute this process." The step-by-step process helped this participant understand the exact steps that they would need to take in order to perform depth-first-search. 

% P5 found that the step-by-step story sequence was engaging: "I think the the narrative of telling a story, going from like broad to specific. It flowed nicely, which was engaging." The sequential ordering of information not only helped them understand the topic, but also kept the reader engaged throughout the reading process.

% When a step-by-step walkthrough was not present, P5 mentioned how the narrative seemed very disjointed: ``All of the examples were sort of like one offs. It wasn't very like coherent. It felt very definition-[like]." For this participant, the non-sequential narrative only allowed her to get a surface level understanding of the topic and not a true understanding of how the topic works: ``I feel like the descriptions, while good, it could be drawn out more so that it's not just glossed over." P6 also mentioned how a EP-NoFewShot explanation ``felt more like a list of facts rather than a story, [a story] can make it easy to engage with if you're fairly new to the subject." For P6, not only did they not like the non-sequential explanation, but also wished that it had been explained through a narrative-based structure, EWP-NoFewShot.

% \textbf{Dimension 2: Readers prefer non-sequential explanations to understand a broad overview of a topic.}

% Some readers preferred the breadth of information that the non-sequential explanations, EP-NoFewShot, provided. P5 mentioned how it helps ``when [the science explanation] offers more examples to help break down a concept." P4 mentioned that even though the non-sequential narrative was ``not a story, but [there is] some sort of comparison to the real world," that made her feel engaged. P4 also mentioned how the non-sequential explanation was ``broken up more so it makes it less intimidating to read through, and is more digestible," illustrating that readers have different preferences for how much information is being presented to them at a time. For some topics, a step-by-step walkthrough might feel ``overly explanatory for a concept that is very intuitive" (P2).

% These two dimensions illustrate the trade-offs between a step-by-step and a non-sequential explanation structure. While some readers prefer the step-by-step information in truly understanding how a topic works, other readers might find the amount of information that needs to be processed overwhelming. Similarly, some readers might only be interested in a high-level understanding on how a certain topic works and might not be interested in an in-depth explanation. These findings illustrate that different explanations structures support different ways for a reader to engage with a topic. A step-by-step explanation facilitates a deeper understanding, while a non-sequential explanation helps readers understand many different aspects or applications of a topic.

% \subsubsection{Theme 3: Personal Language: Readers sometimes prefer formal and academic explanations of a topic over personal language.}

% The GLMM model shows that overall people demonstrated a substantial higher preference for science explanations with personal language, EWP (Everything), with slight variation in individual preferences. Overall, personal languages helps establish a writer/reader connection through the use of questions to prompt the reader to reflect: ``When a text  asks questions the reader may be thinking is engaging. It's like you're talking to yourself" (P5) and ``This is more fun to read because [the text is] asking me questions that I'm asking myself" (P5). 
% because it helped both engagement and understanding for science topics. P1 said that "some of the little jokes actually really helped me understand the what the topic is" and that it is a "very common way of speaking" that helped her stay engaged in the thread. P7 mentioned that the descriptive language of a person's "fiery red hair," and other "vivid language" helped her stay engaged when reading. She also mentioned how certain lines from the science explanation were so descriptive that "it's almost like literature. It's really good in that way, and that makes it really engaging."

% When personal language was removed from science explanations, readers remarked that they disliked the overly technical language (P1, P6, P7) and the academic voice (P1, P3, P6). Undefined technical language made it hard to understand and follow the science explanation: ``It was hard to follow at times because there were some terms I didn't understand in this particular field of study: phase shift. What's the exact definition of that?" (P6), ````For a unit change in the independent variable," I'm like, ``Okay, I know these words individually, but together. they don't make sense to me." I think it actually took away from my understanding to have this long explanation of the formula." (P1), and "I mean ``architectural breathing systems." What does that mean?" (P6). For P7, the technical language and scientific jargon was intimidating to read, ``I had to read it a few times to get it. Those bigger words that I was like, ``Oh, that's scary to read." Paired with science concepts, I mean, it makes it kind of intimidating" (P7), demonstrating how technical language is not only less engaging to read but also serves as an emotion trigger for some readers. Many readers also remarked that without personal language, the science explanations felt ``pedantic," (P6) ``highfalutin," (P4) and too ``academic" (P3). P3 said: ``I think some of the wording was a bit too academic, and which can be a little bit distracting if you're focusing really hard on figuring out what they're saying," showing how formal and impersonal language not only hurts a reader's engagement but also understanding of a topic. 

% The GLMM model also shows a slight variation in individual preferences. We identified one main dimensions to when the removal of personal language was preferred.
% \begin{itemize}
%     \item Dimensions 1: The use of personal language can convolute the explanation. 
%     \item Dimension 2: Some readers prefer purely technical explanations.
% \end{itemize}

% \textbf{Dimensions 1: The use of personal language felt unnatural}
% Both P8 and P1 remarked how the inclusion of personal language sometimes felt unnecessary: ``I feel like I'm wasting my time, if I'm reading the personal aspect of things" (P8) and ``full of fluff coming from a personal perspective that I didn't really care about" (P2) demonstrating how personal language can potentially convolute the science for some readers. For P2 the highly personal language made the science explanation feel like it was meant for a ``six years old. But I'm not six years old." Her comment demonstrates how sometimes overly personal language can come off as infantilizing to the reader and deter them from reading more about the topic.

% \textbf{ Dimension 2: Some readers prefer purely technical explanations}
% P8 mentioned how he liked how ``straightforward and explanatory” the language was when reading an explanation with no personal language, EWP-RemoveP.  P8 also mentioned ``I prefer to just if I'm reading something that's educational, I prefer it to just be educational" demonstrating persoanl preferences for how educational materials are presented to him (P8). This shows how some readers don't need personal language to engage them to learn about science on social media. 

% Overall, participants found that the addition of personal language helped them understand and engage with science narratives, but some participants had found the use of personal language detracted from the science explanations (P1, P8) and others didn't need personal language to help them feel engaged or understand a topic (P8). 

% \grace{overall picture: EWP is a great starting point, but doesn't work for all topics, and AI can help you brainstorm these different options, as a writer it can help you explore these different options}

\section{Writer Study: Methodology}

% \grace{can condense this into 2 sentence. we found that readers like blah blhan, previous work has shown that writers like blah blah and as such we do these RQs.}

The reader study found that overall readers had no preference between narratives with and without walkthroughs but did prefer when explanations had multiple different examples to explain the topic. Previous studies have shown that scientists often struggle with two aspects of science communication on social media (1) framing science for everyday audiences and (2) using informal language to communicate science \cite{williams2022hci, koivumaki2020social}. As such, we provide writers with different options to structure and frame their writing (\textit{One Example}, \textit{No Example}, \textit{Many Examples}) and provide options to see their narrative with and without personal language to support the writing process for social media science communication. Our research questions are:

\textbf{RQ1:} How does seeing structure options (\textit{One Example}, \textit{No Example}, \textit{Many Examples}) help writers consider different framing strategies for writing science for social media?

\textbf{RQ2:} How does seeing style options (\textit{With Personal Language}, \textit{Without Personal Language}) help writers consider different communication styles when writing science for social media?


% \grace{make it shorter} In the reader study, readers had no preference between narratives with and without a walkthrough. Some readers preferred narratives without a walkthrough because they contained multiple examples. As such, we investigate how different structure and framing options  can help scientists write science on social media. The reader study also demonstrated that while most participants preferred reading narratives with personal language, some participants preferred narratives without it, demonstrating that there is a range of possibilities for writers to choose from. Thus, we provide scientists with two different narratives with and without personal language to help them explore their comfort between these two extremes. Our research questions are

% \grace{format the same, but keep naming as RQ}


% \grace{readers had no preference for with/without walkthroughs. but the readerst that preferred walkthroughs because they had mutliple 

% had diverse preferences for narratives with and without personal language (subjective, conversational, and informal language and humor). 

% reader study there was high variance, there is a variety to pick through. need one more 
% The reader study also showed that most readers preferred narratives with personal language, but . But not always... meaning that there is a spectrum that writers can pick.} 


% In our reader study, we explore how techniques of example (E) and walkthrough (W) as separate dimensions of science communication. But the interview results from (H2:Walkthrough) demonstrate that some participants liked narratives without a walkthrough because they contained multiple different examples to help explain the topic. The presence of multiple different examples caused the structure of the explanation to be a more modular, each example illustrated a different dimension of the topic. As such, to help scientists explore different framings for science communication on social media, we evaluate one example, no example, and many example narrative structures.  

% The reader study also demonstrated that readers have

% To accommodate scientists' varying comfort levels on science communication on social media during the writing process. We provide scientists with different options for how to structure and style their writing. 


% Scientists are trained to write formally for science conferences and journals struggle to adapt to the  of social media for concerns around how their social media persona might affect their professional reputations \cite{koivumaki2020social}. Furthermore, many scientists want to communicate their work on social media for either for personal or professional benefits \cite{williams2022hci}. But even when they are taught techniques for science communication, may still feel hesitant to utilize them \cite{10.1145/3479566}.

% We want to understand how  To do so, We hypothesize that seeing multiple options helps writers navigate their own comfort level  on the spectrum of formal to informal science communication. C

% From the reader study results for (H2:Walkthrough), we found that there was no statistically significance preference for readers between the with and without walkthrough narratives. In the followup interviews,  The presence of multiple different examples caused the structure of the explanation to be a more modular, each example illustrated a different dimension of the topic. Narratives a walkthrough and one example explained different dimensions of the topic with one example using a more connected narrative approach.



% This shows how the structure of a narrative is also closely tied to the number of examples used. 

% The results from the reader study demonstrated how the use of example(s) were tied to the overall narrative structure. 


% Building on our Reader Study findings, which indicated...

% Science experts often have writers' fears and shy away from using the \grace{three social media --change} techniques when communicating science to the general public. We want to understand how their fears might be mitigated during the writing process by giving them the options of using initial drafts with and without the three techniques (Example, Walkthrough, Personal Language). 
% In particular, we explore the following questions:

% \grace{misalign with the fears + the RQ1, is RQ1 the direction we want to go in, scientists struggle with this, cite katy's paper everytime we say that scientists struggle w/ this. choose a direction to write in... set up the problem again, and bring this back in: scientists are trained to write in formal ways etc. etc. research shown that it can be hard to adopt these styles, explore whether showing writers options for informal writing can influence their adoption for informal writing, need to set up the context a bit more. do it for structure and style}


% RQ1: Does having the options to view a narrative w/ and w/o a walk-through help writers choose a direction to write in?

% RQ2: Does having the options to view a narrative with and without personal language help writers choose a direction to write in?



% \yy{move this into subsections}
% We conducted a writer study with 5 PhD-level researchers. 
%The writer participants are guided through a workflow via a web interface with LLM generations and editing functionalities. The workflow asked them to first work on the structure drafts (Example, Walkthrough), then work on the personal language style drafts, and finally make edits to the science explanations until they are publish-ready. We used think-aloud protocol and semi-structured interview questions to elicit rich insights from their experience. 
%We describe this in more detail in Section \ref{sec:writer_workflow}. 
%we  a web interface that provides a workflow that guides the writer participants to work first on the 
% Participants are asked to produce two Tweetorials. One topic is pre-assigned by the experimenter, and the other is chosen by the participant.

% Our survey on readers preferences for how science is explained to them is diverse. We want to explore whether showing writers different ways to (1) structure science narrative and (2) narratives with and without personal language will help them 


\subsection{Participants}
We recruited 10 PhD-level researchers interested in communicating their research to the public on social media. Participants are from 2 universities, an average age of 25, with a gender distribution of 9 males and 1 female (Table \ref{tab:my-table}). We advertised the study to students in research labs through school mailing lists, Slack workspaces, and snowball sampling among lab mates of the participants. Their expertise spans various CS research areas, including natural language processing, programming languages, and social computing. The study was conducted over Zoom and took approximately 2 hours. Each participant is compensated \$40 dollars total. The study was approved by our institutional IRB. 

% \yy{do we mention their experience w/ past tweetorial studies?  past experience w/ science communication?? Justification for only using CS field? }

% 6 Computer Science PhD students

\begin{table}[]
\begin{tabular}{|c|c|c|}
\hline
\textbf{ID} &  \textbf{Field of Expertise}              & \textbf{Research Experience (years)} \\ \hline
1                    & Computer Science and Journalism          & 2                            \\ \hline
2                    & Artificial Intelligence and Neuroscience & 2                            \\ \hline
3                    & Human-Computer Interaction               & 2.5                          \\ \hline
4                  & Natural Language Processing              & 2                            \\ \hline
5                    & Natural Language Processing              & 3                            \\ \hline
6                       & Programming Languages                    & 5                            \\ \hline
7                      & Computer Security                        & 2.5                          \\ \hline
8                      & Quantum Computing                        & 3                            \\ \hline
9                  & Human-Computer Interaction               & 3                            \\ \hline
10                   & Computer Science Education               & 7                            \\ \hline
\end{tabular}
\caption{Participant Demographics for Writer Study}
\label{tab:my-table}
\end{table}

% \begin{table}[H]
% \begin{tabular}{|c|c|c|c|c|c|}
% \hline
% \textbf{ID} & \textbf{Gender}        & \textbf{Field of Expertise}         & \textbf{Research Experience}                         \\ \hline
% 1           & Man                      & Computer Science and Journalism & 2 years                           \\ \hline
% 2           & Man             & Computer Science                & 2 years                                           \\ \hline
% 3           & Woman                   & Social Computing                & 2.5 years                                  \\ \hline
% 4           & Man                        & Natural Language Processing     & 2 years                           \\ \hline
% 5           & Man                      & Natural Language Processing     & 6 years                               \\ \hline
% \end{tabular}%
% \caption{Participants Demographics for Writer Study}
% \label{tab:my-table}
% \end{table}




% \begin{table}[H]
% \resizebox{\textwidth}{!}{%
% \begin{tabular}{|c|c|c|c|c|c|}
% \hline
% \textbf{ID} & \textbf{Gender} & \textbf{Education}         & \textbf{Field of Expertise}         & \textbf{Research Experience} & \textbf{Topic of Choice}                         \\ \hline
% 1           & Man             & 2nd Year Masters           & Computer Science and Journalism & 2 years                          & Statistics: Generalized Linear Additive Models   \\ \hline
% 2           & Man             & Bachelor of Science - 2023 & Computer Science                & 2 years                            & Computer Science: Gradient Descent               \\ \hline
% 3           & Woman           & 1st Year Ph.D              & Social Computing                & 2.5 years                         & Qualitative Analysis: Ordinal Regression         \\ \hline
% 4           & Man             & 1st Year Ph.D              & Natural Language Processing     & 2 years                           & Natural Language Processing: Controlled Decoding \\ \hline
% 5           & Man             & 2nd Year Ph.D              & Natural Language Processing     & 6 years                           & Natural Language Processing: Word Embeddings     \\ \hline
% \end{tabular}%
% }
% \caption{Participants for Writer Study}
% \label{tab:my-table}
% \end{table}


\subsection{Writing Study Procedure and Analysis}

\subsubsection{Study Procedure}
Each session includes a pre-study presentation, an interface demonstration, and two writing sessions with a 15-minute break in between. The experimenter used a short presentation to educate participants about science communication on social media, a background on Tweetorials, and different techniques and examples for science communication on social media. Next, the experimenter demonstrated the study web interface and explained how it worked. 

% \subsubsection{Writing Study}
When the participant was ready for the writing session, the experimenter shared with the participant a URL link to the study web interface. The experimenter started screen and audio recording upon the participant's verbal consent. Participants use the study interface once for each writing session: first to write about the predetermined topic of merge sort and then to write about a topic of their choice. The experimenter implemented think-aloud protocol and encouraged the participant to voice their thought process, reasoning behind their structure and style choices, and editing decisions. After each writing session, the experimenter conducted a semi-structured interview with the participant to understand how seeing different structures and styles influenced their writing choices. Some sample questions are: 
How did you arrive at your choice of structure/style?
How did seeing the options change your preference of structure/style?
How have reading the unselected options help you decide which direction to write in?

% The web interface used in the study provides a workflow that guides the participants to work first on the structure choices and second on the language style choices. We describe this in more detail in Section \ref{sec:writer_workflow}. 

% \grace{can cut a lot from this paragraph, check for redundancy}

% Participants are asked to produce two Tweetorials. One topic is pre-assigned by the experimenter, and the other is chosen by the participant. 
% After finishing writing the two topics, we conducted a semi-structured interview to elicit rich insights into their experience. We collected usage data and Tweetorial generations from the web interface, as well as Screen and audio recordings for further analysis.


\subsection{Study Interface}
\label{sec:writer_workflow}
% \grace{refer to figure 5 at the beginning of 6.2, change the name interface to match our naming conventions + rescreenshot interface }

We built a web interface that guides writers through a workflow with LLM generations and editing functionalities (Figure \ref{fig:step1}, Figure \ref{fig:step2}, Figure \ref{fig:step3}). The workflow includes the following 4 steps:

% \grace{need to relate back to structure and the way that we describe it to people, may need to describe that earlier, say this further up, examples are part of the structure section.... motivate the problem of why we do walkthrough and examples.... talk this out more and connect them, simple why are we doing things, not the same thing as reader study set up}
\textbf{Step 1: Structure Options} Users first enter their domain and topic to generate the 3 different structure options. After the generation, 3 columns are displayed side-by-side with the corresponding LLM prompts: \textit{One Example}, \textit{No Example}, and \textit{Many Examples} (Figure \ref{fig:step1}. The user chooses one of the three to proceed with. Writers can merge certain paragraphs from two structure options by copying and pasting between the columns before proceeding. 

\textbf{Step 2: Selected Structure Feedback and Edits} The second part of the interface (Figure \ref{fig:step2}) allows the writer to iterate on the selected structure by providing feedback instructions to an LLM or making manual edits directly in the textbox (copy, paste, delete, and type). The writer is asked to focus only on editing structural aspects of the text such as technical accuracy, content, and sequence. When they are satisfied with the structure, they proceed to the next step. 


\textbf{Step 3: Language Style Options} The third part of the interface compares different style options, it has 2 columns displayed side by side (Figure \ref{fig:step3}). On the left is the writer's selected and edited draft from the previous step which contains personal language and the right is the draft without personal language. Writers select which narrative they want to proceed with to the next step. Writers can merge certain paragraphs from both options by copying and pasting between the columns before proceeding. 


\textbf{Step 4: Final Edits}
The last part of the interface displays the draft from Step 3 (not shown). The writer refines and finalizes the writing until they are satisfied and ready to share it on social media. 
% \yy{TO BE CONTINUED}

% % % The workflow asked them to first work on the structure drafts (Example, Walkthrough), then work on the personal language style drafts, and finally make edits to the science explanations until they are publish-ready. 




% % \yy{terminology to be updated based on system screenshots. }




% Following that, the second part of the interface is for style choices. It has 2 columns displayed side by side, writer's current chosen version which has personal language vs. a version with personal language removed. Again, there is "I Like it!" button for user to choose which one to proceed with for further editing. 

\begin{figure}
    \centering
    \includegraphics[width=1.0\linewidth]{Figures/Step1.png}
    \caption{Study interface: Viewing different structure options (1st of 3 steps)}
    \label{fig:step1}
\end{figure}

\begin{figure}
    \centering
    \includegraphics[width=1.0\linewidth]{Figures/Step2.png}
    \caption{Study interface: Editing narrative structure (2nd of 3 steps)}
    \label{fig:step2}
\end{figure}

\begin{figure}
    \centering
    \includegraphics[width=1.0\linewidth]{Figures/Step3.png}
    \caption{Study interface: Viewing different style options (3rd of 3 steps)}
    \label{fig:step3}
\end{figure}




% First, participants are shown three GPT-generated Twitter threads using three structural techniques - (1) Everything(EWP), (2) No example, (3) No step-by-step walkthrough. Participants can re-generate as needed. 
% They are asked to choose one to continue in the next step. 

% Second, participants are to give ChatGPT feedback to refine the structure until they are satisfied. 

% Third, participants are shown two Twitter threads with different personal language styles - (1) their current version w/personal langauge, (2) GPT removed personal language from it. Participants are to choose which one to continue with in the next step. 

% Finally, the participants edited the Twitter thread until they think it is ready to be published. 







% \subsubsection{Semi-Structured Interview}
\subsubsection{Data Analysis}
We analyze interview transcripts and video recordings to understand quantitative and qualitative aspects of participants' choices and reasoning when presented with different options. We analyzed participants' choice of structure (\textit{One Example}, \textit{No Example}, \textit{Many Examples}) and style options (\textit{With Personal Language} and \textit{Without Personal Language}), participants' editing actions and LLM feedback prompts, iterations of writing generations and refinements, as well as the final writings. 

One researcher independently conducted a bottom-up, open-coding approach to data analysis \cite{charmaz_constructing_2006}. Then, the researcher worked with two other researchers to iterate on the codes, discuss their similarities and differences as part of a comparative analysis \cite{merriam2015qualitative}, and leveraged them in an affinity diagramming process \cite{holtzblatt2017affinity}. The researchers determined that they reached code saturation when no researcher could identify new codes or arrive at new interpretations. 



% We collected a combination of quantitative and qualitative data. These include We transcribed the semi-structured interview recordings into text for analysis. 



% The interview is audio recorded and later transcribed for analysis. 

% Each participant is asked to use the web interface twice to create two Twitter threads: one on a pre-assigned topic and the other on a topic of their choice.






% \subsubsection{Think aloud protocol}

% The experimenter implemented think aloud protocol through the user study process. We constantly encouraged the participant to voice their thought process, and reasoning behind their choices, feedback and editing decisions. 

% \subsubsection{Semi-structured interview}

% After completing two Twitter thread writing, we conducted a semi-structured interview with the participant to reflect on their experience. We asked questions to understand how seeing different structures and styles influenced their writing choices. Some sample questions are .. 

% \subsection{Data Analysis}
% \grace{6.3.2 writing study procedure: data collection and analysis and  instead of writing study + don't need 6.3.3 be it's own thing, just be a paragraph merged with the previous section

% 6.4 is short, can maybe lump it into previous 

% don't want all the tiny section. sections shouldn't be one paragraph}



%In accordance with Mcdonald et al. \cite{mcdonald_irr}, we did not compute inter-rater reliability (IRR), since we used the coding process to discover emergent themes or recurrent topics and permitted multiple possible interpretations of the meaning of the codes. After the two researchers completed their synthesis of an affinity diagram, one additional researcher reviewed the themes and provided their comments. 


% iterations of tweet generation/edits via the web interface in json

% screen recording of the writer's process

% think aloud audio recording

% interview recording and transcripts 

% Analysis -- thematic coding


\section{Results: Writer Study}
% In this section, we provide quantitative and qualitative analysis on the writers' experience when they are presented with different structure and style options for science narratives. 10 writers wrote two different science explanations, which resulted in 20 total edited science explanations. 

% \grace{writers have preferrence for interface for interface or tools, }

% To understand writer's preferences for different structures and styles of communication, we report findings on the following research questions:

% RQ1: How does seeing structure options (Everything, No Example, No Walkthrough) help writers choose a direction to write in?

% \grace{with and without personal language}
% RQ2: How does seeing style options (Yes Personal Language, No Personal Language) help writers choose a direction to write in?

% \begin{itemize}
%     \item Structure: Which narrative structures (Everything (Everything, No Example, and EP-NoFewShot) do writers prefer?
%     \item Structure: Does seeing different structure options help writers choose a direction to write in?
%     \item Style: Which narrative style (Yes - Personal Language and Remove - Personal Language) do writers prefer?
%     \item Style: Does seeing different style options help writers choose a direction to write in?
% \end{itemize}


% which explanation structure do writers prefer + does showing different narrative structures help writers decide what to write in 

% \grace{in title of 6.1: showing different narrative structures help people mix and match, takeaway should be clear. one sentence bolded and first. that's when you are done. first/last doesn't really matter.}

% \subsection{Writers Reactions: Different Structures for Science Explanations}
% We generated three different ways (EWP, WP, EP) to structure science narratives. After we asked writers to choose one of the narrative structures to learn more about their preferences for how science explanations are structured. Then we observed the writer to see how they will edited, adapted, and merged their chosen narrative structure into something they are satisfied with. 
% \grace{edit the information}

\subsection{\textbf{RQ1}: Different Narrative Structures help Writers Choose Appropriate Framing Techniques}

Of the three narrative choices (\textit{One Example}, \textit{Many Examples}, \textit{No Examples}), most science explanations used the \textit{One Example} narrative to iterate on (9 of 20). The second most chosen narrative strategy was merging components between the \textit{One Example} and \textit{Many Examples} options (5 of 20). The \textit{Many Examples} strategy was chosen 4 of 20 times. The least used structure was \textit{No Examples} (2 of 20). Choices made by participants and topic are found in Table \ref{tab:writer-structure-preferences}. Writers' choices differed depending on the topic, their personal preferences for science communication, and whether or not they had a predefined explanation structure they wanted to follow. The \textit{One Example} option often provided writers with a clear sequencing of information to follow. Some participants (P2, P4, P6, P10) chose to merge between the \textit{One Example} and \textit{Many Examples} and used examples covered in the \textit{Many Examples} option to supplement certain explanations in the \textit{One Example} narrative. Occasionally, the \textit{No Example} option providing a strong technical explanation for writers to build off of.

% \grace{explain why each one was the winnner, people made these different choices because of largely depended on the topic, personal preference, add interpretation, this is consistent with the reader study, an example is a good way to structure thigns and that mansy examples have benefits. no examples can occasionally be a good choice.} We investigated how writers made their choices and how seeing different structure options was helpful to them. 

% \grace{get to the point faster, overall people picked one example, }
% To answer RQ1, we evaluate how many writers choose each structure option (\textit{One Example}, \textit{No Example}, \textit{Many Examples}). 
% \grace{how did thye amke the chocies and how the different options are helpful to them}
% \grace{need a lead in for the different ways that people use the nrrative structures}


% \grace{restate RQ1 short way: does structure help writers... make naming consistent, Italicize and capitalize the naming for the examples -- make sure this is consistent}


% \grace{put the sentences together --> merge, instead writers selected no example 2 of 20 times. 

% no worst/best language}
% \grace{topic sentences are bad need to make it a bit punchy, what is the takeaway --> should tell what the paragraph is going to say in, what is interesting + takeaway-ish, now is more so summary based... give the readers something interesting in topic sentence}

% \grace{need to follup which narrative they choose and why}
% Writers' existing science communication preferences and beliefs influence their choice of narrative structure. 
Seeing different narrative structures helped 7 of 10 writers identify their own preferences for science communication by solidifying or challenging their existing beliefs (P2, P3, P5, P7, P8, P9, P10). P9 mentioned that seeing the narrative with \textit{Many Examples} helped him solidify his initial feeling that ``one example is a good way to engage people." But P7 mentioned how seeing multiple structure options helped him re-evaluate his initial approach to explaining the topic of word embeddings. P7 initially wanted to use the \textit{One Example} narrative structure, but instead chose to use the \textit{Many Examples} narrative because the structure provided for the topic of word embeddings: ``I was able to see what other things I might actually want to include when I'm trying to explain it, and what other possibilities there are for explaining." 

Seeing multiple narrative structures helped 4 of 10 writers merge different framings of a topic (P2, P4, P6, P10). For multiplicative weights update, P10 merged elements from \textit{Many Examples} into a base narrative with \textit{One Example} to emphasize the dimensions of online learning and regret minimization which are ``important facts about the topic, but wasn't brought up in the [\textit{One Example}] option." Because the \textit{Many Examples} option provided him with different angles to motivate or contextualize the topic, the selected paragraphs from \textit{Many Examples} could be ``slotted in at the end."  

There were 4 times where writers already had an established outline in mind for how to write the science explanation for one of their two topics (P1, P3 P8, P9). But for most writers and topics, seeing different options provided 8 of 10 writers an example of what to avoid when explaining science to an everyday audience (P1, P2, P3, P5, P6, P7, P8, P9). Seeing the \textit{No Example} option helped them consider their audience's technical ability and the background needed to explain the science. P6 said \textit{Many Examples} narrative had a``rigorous depiction of what the software is, [which can be hard] for a general audience to imagine what that means." 

% P10 believed that the topic of multiplicative weights update method was ``relatively receptive to real world examples, so [he] would definitely include one" (P10). P7 mentioned how he ``tend towards" science explanations that have a walked through examples and as a result choose that narrative to iterate on. 

% \grace{cut some of this, needs to get to the point faster, what do you want people to take away from this... }



% and mentioned how seeing different explanation structures helped him determine which subtopics to include. He chose to include a paragraph from the \textit{Many Example} option that provided a succinct description of online learning and regret minimization to highlight why MWUM is so important.

% : ``[the structure] provided little nuggets of information [which] is more useful than one big example that tries to cover everything." P7 mentioned how

% \grace{this is a better topic sentence, shorten the topic, for some topics seeing mutliple narrative structures helped them explore how to explain the topic. put cut information somewhere else in the paragraph, cut sentence 2 and add number to the topic sentence}
% \grace{feels like the same thought, bring out the merging here, sometimes people didn't choose one topic, but instead merged them}
 

% Some writers used the different narratives to compare the pros/cons for each narrative and perform merges between narrative options if there were elements from another narrative that they would want to incorporate. 




% \grace{seeing no example showed readers what not to do, see how no example was too abstract. counterpoint}


% \grace{can merge, don't need example

% 4 times people didn't need any of the options. but for people who didn't have an idea it was helpful.}
% But seeing the different narrative options didn't always help writers. For certain topics, 4 of 10 writers already had clear ideas for how to communicate the science that seeing multiple options was distracting . For the topic of 4D printing, P9 said that he didn't like any of the narrative structures and already had a clear vision for how to write the narrative himself. While for the topic of merge sort, P9 had no strong idea for how to start, so seeing many different options helps ``give [him] a starting point." 

% This illustrates how writers sometimes do not need a system to help them decide what narrative structure to use. 

% P4 and P2 choose to merge aspects of two different narrative strategies together. In doing so, he took one narrative as his primary narrative structure and then incorporated certain elements from a secondary narrative structure into the primary. This demonstrates how the characteristics that differentiate the 3 narrative structures (Everything, No Example, and EP-NoFewShot) exist on a spectrum and are not mutually exclusive. 

% We found that writers have diverse preferences for how to structure their science explanations. While half of the science narratives used the Everything structure, writers also choose to use other narrative structures, demonstrating that there is not one standard method that writers like to structure their science explanations.

% \grace{make primary, secondary, and then change the color for primary and secondary}
\begin{table}[H]
\resizebox{\textwidth}{!}{%
\begin{tabular}{|c|c|c|c|c|c|c|}
\hline
{\textbf{Participant ID}} & {\textbf{Domain}} & {\textbf{Topic}} & {\textbf{One Example}} & {\textbf{No Example}} & {\textbf{Many Examples}} & \textbf{Merged}\\ \hline\hline

\multirow{2}{*}{P1} & {Computer Science} & {Merge Sort} &  &  & {\checkmark} & \\ \cline{2-7}
& { Statistics} & { Gradient Linear Additive Models} & {\checkmark} &  &  & \\ \hline

\multirow{2}{*}{P2} & {Computer Science} & {Merge Sort} & & & & {\begin{tabular}[c]{@{}c@{}}Primary: One Example\\ (Secondary: Many Examples)\end{tabular}}                          \\ \cline{2-7}
& {Computer Science} & {Gradient Descent} & {\checkmark} & & & \\ \hline
 
\multirow{2}{*}{P3} & {Computer Science} & {Merge Sort} & & {\checkmark} & &  \\ \cline{2-7}
& {Qualitative Analysis} & {Ordinal Regression} & & {\checkmark} & & \\ \hline
 
\multirow{2}{*}{\vspace{-0.21in} P4} & {Computer Science} & {Merge Sort} & & & & {\begin{tabular}[c]{@{}c@{}}Primary: One Example\\ (Secondary: Many Examples)\end{tabular}}  \\ \cline{2-7}
& {Natural Language Processing} & {Controlled Decoding} & & & & {\begin{tabular}[c]{@{}c@{}}Primary: Many Examples\\ (Secondary: One Example)\end{tabular}} \\ \hline

\multirow{2}{*}{P5} & {Computer Science} & {Merge Sort} & {\checkmark} & & & \\ \cline{2-7}
& {Natural Language Processing} & {Word Embeddings} & & & {\checkmark} & \\ \hline
 
\multirow{2}{*}{\vspace{-0.2in}P6} & {Computer Science} & {Merge Sort} & {\checkmark} & & & \\ \cline{2-7}
& {Programming Languages} & {Formal Verification} & & & & \makecell{Primary: One Example \\ (Secondary: Many Examples)} \\ \hline

\multirow{2}{*}{P7} & {Computer Science} & {Merge Sort} & {\checkmark} & & &  \\ \cline{2-7}
& {Natural Language Processing} & {Embedding Space} & & & {\checkmark} & \\ \hline
 
\multirow{2}{*}{P8} & {Computer Science} & {Merge Sort} & {\checkmark} & & & \\ \cline{2-7}
& {Quantum Algorithms} & {Grover's Algorithm} & & & {\checkmark} & \\ \hline

\multirow{2}{*}{P9} & {Computer Science} & {Merge Sort} & {\checkmark} & & & \\ \cline{2-7}
& {Tangible User Interfaces} & {4D Printing} & {\checkmark} & & & \\ \hline
 
\multirow{2}{*}{\vspace{-0.18in} P10} & {Computer Science} & {Merge Sort} & {\checkmark} & & & \\ \cline{2-7}
& {Optimization Algorithms}     & {Multiplicative Weights Update} & & & & {\begin{tabular}[c]{@{}c@{}}Primary: One Example\\ (Secondary: Many Examples)\end{tabular}}  \\ \hline\hline

& & \textbf{Total Count:} & \textbf{\begin{tabular}[c]{@{}c@{}}9\\ One Example\end{tabular}} & \textbf{\begin{tabular}[c]{@{}c@{}}2 \\ No Example\end{tabular}} & \textbf{\begin{tabular}[c]{@{}c@{}}4\\ Many Examples\end{tabular}} & \textbf{\begin{tabular}[c]{@{}c@{}}5\\ Merged\end{tabular}} \\\hline            
\end{tabular}%
}
\caption{Participants, the topic they wrote on, and the corresponding narrative structure that they chose. When merging two narratives, ``primary" denotes the narrative structure that the participant used as the base and ``secondary" means that the writer incorporated elements from this narrative into the primary narrative.}
\label{tab:writer-structure-preferences}
\end{table}

% \begin{table}[H]
% \resizebox{\textwidth}{!}{%
% \begin{tabular}{ccccccc}
% \hline
% {\textbf{Participant ID}} & {\textbf{Domain}} & {\textbf{Topic}} & {\textbf{\begin{tabular}[c]{@{}c@{}}Everything\\ EWP (Everything)\end{tabular}}} & {\textbf{\begin{tabular}[c]{@{}c@{}}No Example\\ EWP-RemoveE\end{tabular}}} & {\textbf{\begin{tabular}[c]{@{}c@{}}No Walkthrough\\ EP - NoFewShot\end{tabular}}} & {\textbf{\begin{tabular}[c]{@{}c@{}}Merged Multiple \\ Narrative Structures\end{tabular}}} \\ \hline
% {P1}                      & {Computer Science} & {Merge Sort}                      & {}                                                                                              & {}                                                                                        & {Primary}                                                                                         & {}                                                                                         \\ \hline
% {P1}                      & {Statistics} & {Gradient Linear Additive Models} & {Primary}                                                                                       & {}                                                                                        & {}                                                                                                & {}                                                                                         \\ \hline
% {P2}                      & {Computer Science} & {Merge Sort}                      & {Primary}                                                                                       & {}                                                                                        & {Seconday}                                                                                        & {Yes}                                                                                      \\ \hline
% {P2}                      & {Computer Science} & {Gradient Descent}                & {Primary}                                                                                       & {}                                                                                        & {}                                                                                                & {}                                                                                         \\ \hline
% {P3}                      & {Computer Science} & {Merge Sort}                      & {}                                                                                              & {Primary}                                                                                 & {}                                                                                                & {}                                                                                         \\ \hline
% {P3}                      & {Qualitative Analysis}        & {Ordinal Regression}              & {}                                                                                              & {Primary}                                                                                 & {}                                                                                                & {}                                                                                         \\ \hline
% {P4}                      & {Computer Science} & {Merge Sort}                      & {Primary}                                                                                       & {}                                                                                        & {Secondary}                                                                                       & {Yes}                                                                                      \\ \hline
% {P4}                      & {Natural Language Processing} & {Controlled Decoding} & {Secondary}                                                                                     & {}                                                                                        & {Primary}                                                                                         & {Yes}                                                                                      \\ \hline
% {P5}                      & {Computer Science} & {Merge Sort}                      & {Primary}                                                                                       & {}                                                                                        & {}                                                                                                & {}                                                                                         \\ \hline
% {P5}                      & {Natural Language Processing} & {Word Embeddings}                 & {}                                                                                              & {}                                                                                        & {Primary}                                                                                         & {}                                                                                         \\ \hline
%                                               &                                                  & \textbf{Total Count:} & \textbf{5 primary, 1 secondary}                                                                                    & \textbf{2 primary}                                                                                           & \textbf{3 primary, 1 secondary}                                                                                      & \textbf{2}                                                                                                   
% \end{tabular}%
% }
% \caption{Participants and the topic they wrote on and the corresponding narrative structures that they choose. Primary denotes the narrative structure that the participant choose and Secondary means that the writer incorporated elements from this narrative into Primary narrative.}
% \label{tab:writer-structure-preferences}
% \end{table}

% \subsubsection{Structure: Does seeing different structure options help writers choose a direction to write in?}

% To better understand how seeing different narrative structures helps writers choose a direction to write in, we analyzed the user study transcripts that contained conversations between the study administrator and the participant as the participant was using the user interface. We found two key themes in how seeing the different narrative generations helped writers:
% \begin{itemize}
%     \item Explore which dimensions of a topic to include,
%     \item Reflect on the audience they are writing for
% \end{itemize}


% \textbf{Explore which dimensions of the topic to include}

% For P3, she found that exploring the different narrative structures helped her identify a main structure to follow, as well as additional aspects of the topic that she might want to include: "if I want to explain different aspects about, about merge sort [in my final narrative], I would use specific [tweets] from the [EP-NoFewShot] output, but I would definitely not use the [EP-NoFewShot] as a whole because it's a bit disordered." This demonstrates that the different narrative structures provide not only structural guidance for writers, but also support for balancing the breadth and depth of a topic. 

% For P5, he was "keen to see what the different narrative structures would look like, see what the trade offs are and necessarily like. Essentially, it gave me a good vision of how you can write the same thing differently." For the topic of merge sort, seeing different narrative structures allowed him to see "what degree of scientific information is being retained with example versus without example." Once seeing the two structures side-by-side, P5 chose to continue with the Everything narrative structure because it was "layman friendly." This demonstrates that exploring different options for narrative structural helped him to assess the trade-offs between the amount of scientific information included and the engagement of an everyday audience. 

% For P2, he initially disregarded the Everything structure with a single, walkthrough example because he didn't like the example that was used, but after reading all three narrative structure he said: "maybe with merge sort, starting with an example, on second thought, might actually be better. Like a very minimal example, sorting three numbers or something." This demonstrates, how viewing different narrative structures side-by-side allows the writer to weigh the pros and cons between each narrative sturcture. Furthermore, P2 mentioned how he would like to keep some of the content covered in the EP-NoFewShot condition such as "the same amount of time [merge sort takes], regardless of the initial order and the fact that it's efficient, it's actually something that actually no the other [narrative] says: [that merge sort] efficient." This demonstrates that the different narrative structures also help the writer explore different aspects of a topic to include in their explanation.

% For P4, after seeing the different narrative generations, he choose to merge aspects of each together, for example he said that "the optimal case is mixing the [Everything] structure and the [No Walkthrough] structure" because he wanted to retain the step-by-step walkthrough aspect of the Everything structure, but expand the narrative to include "contain more information." Additionally, when he was writing another science explanation on the topic of controlled decoding, he choose the No Walkthrough Structure because it covered a breath of differnt dimensions that he was interested in exploring.

% \grace{insert quotes}


% \textbf{Reflect on the Audience they are writing for}

% For P2, after reading the EP-NoFewShot condition that doesn't contain a step-by-step walkthrough, he stated that "[EP-NoFewShot] does have more information, for interested readers, but maybe it wouldn't be best for someone that has no idea about these topics," demonstrating how seeing the different conditions helped him consider which audience he was writing for. 

% Similarly, for P4, seeing each of the different narrative conditions prompted him to to identify the different aspects of each that he liked. When seeing the Everything condition with a walkthrough, he liked the use of an example to explain a topic, but was unsatisfied with the amount of content covered. As a result, he used the No Walkthrough narrative structure as his primary structure, but then used the step-by-step walkthrough technique in a subset of tweets. He used the web interface to ask GPT to "Try to mentioning the example about cleaning room in the 3 - 8 points," showcasing how within the No Walkthrough structure, the writer still wanted to maintain a step-by-step explanation to help readers understand a particular dimension of the topic in depth, while still maintaining the breadth of topics that he wished to cover. This demonstrates how exploring different narrative structures supports writers in exploring a range of science explanation techniques.

% For one participant, P3, when asked to select a narrative structure to move forwards with said: "I think it depends on the audience like. So if the audience is a non tech, low tech general public, I will probably incorporate the example from the [Everything] narrative structure, but I would definitely use the structure in a second output." For her third topic on ordinal regression, P3 mentioned how "[No walkthrough] structure looks like a checklist. [Researchers] can check if this model is applicable to the data we have. So I think, this [narrative structure] being fragmented when explaining this concept is totally okay and to explain this topic to an audience [of researchers]."

% These findings demonstrate that by viewing different narrative structures side-by-side also helps writers reconsider who their audience is. While this study was focused on create science explanations for an everyday audience, these results demonstrate that this method of comparison also helps writers write for more technical audiences as well. 


\subsection{\textbf{RQ2}: Different Style Options Help Writers Balance Personal Preferences with Social Media Communication}
The most common narrative strategy that writers used was \textit{With Personal Language} (12 of 20). Merging elements from the \textit{With Personal Language} and \textit{Without Personal Language} style was the second most common action that writers performed (6 of 20). Only 2 of 20 narratives used narratives without personal language. Overall, writers prefer using science narratives that include personal language and editing the LLM-generated language to match their own voice. Many writers found that comparing between narratives with and without personal language helped them identify and adopt useful techniques they might not have otherwise used in their own writing.

8 of 10 participants chose to begin with the \textit{With Personal Language} option because seeing personal language provided a more accessible base narrative that writers could use to edit or replace style elements (P1, P2, P3, P5, P6, P8, P9, P10). Seeing the placement of personal language helped P6 identify where and what to edit instead of determining where and what to add if he was working from the \textit{Without Personal Language} option: ``I try and get a pretty good baseline, and the personal language involved adds to the baseline." Furthermore, seeing options with personal language helped writers appreciate language that they might not have thought of using and internalize the importance of these style choices when read in juxtaposition with a narrative without personal language. For P7, seeing GPT-generated personal language helped him adapt to the genre of science communication on social media: ``I think the enthusiasm, while it may not necessarily be the way that I would have written this, feels like it's a more engaging way of trying to explain this to people on the internet."

But for 2 of 10 writers, starting from the narrative \textit{Without Personal Language} helped them evaluate the technical content of the explanation before adding back personal language (P2, P3). P2 mentioned how he prefers the remove personal language option because it is a ``simpler version" that he can add his own writing voice to, instead of editing GPT's writing voice. P3 mentioned that she chose to start with the narrative \textit{Without Personal Language} because she valued the content explanation over the personal language: ``I do like being more objective and professional when writing explanations about scientific concept." 

% After a writer has finalized the structure of their science explanation, we ask the writer to choose between the narrative with and without personal language to understand the writer's preferences. Then we observed how the writer edited, adapted, and merged their chosen narrative into a an explanation that they are satisfied with. 
% \grace{don't ened to restate what we are doing,, we found blah, just say how a human is personal language was preferred over no personal language. 

% reconceptialize the section and the groupings}
% To understand how seeing different style options can help writers balance their professional identities with informal language strategies, we evaluate how many writers choose each style option (\textit{With Personal Language}. 

% \grace{re-conceptualize}

% But for some writers, having both options provided a reference to evaluate and revise their draft against. Furthermore, different writers used the side-by-side comparison of narratives with and without personal language in different ways to support their writing process. \grace{keep this? Writers who selected \textit{With Personal Language} often deleted or revised the LLM-generated language to match their own voice, while writers who selected \textit{Without Personal Language} often added their own voice to the explanation. }

% \grace{
% emphasize these two points: start with personal lanuage and then add it --> or start with remove and add it 
% --> why seeing both why they started with it or not, and add as color/reason
% --> drill into 2 paragraphs (3 at most)
% clarity, seeing btoh help....

% }




% P7 mentioned how the personal language makes the introduction to the problem too ``lengthy and drawn out," which can distract from the purpose of explaining the science.  P1 mentioned how seeing the narrative without personal language let him ``parse exactly what information is being presented without any frills around it," which let him come to the realization that ``maybe, it's not clear exactly what is happening" (P1). Sometimes writers would copy and paste technical sentences from the \textit{Without Personal Language}

% Seeing both style options helped 5 out of 10 participants remember the audience of everyday people on social media they were writing for (P1, P9, P7 P10, P3). For P9, seeing both options helped him ``gain clarity around what [he] did want" in terms of how much technical detail to include. P9 also mentioned how sometimes he might forget what is considered a technical term for everyday audiences, so when he reads a narrative without personal language, the ``re-introduction of more technical terms stands out more." It is helpful to him because the comparison `` reminds [him] of things to look for when [he] edits." This demonstrates the common struggle that experts have in evaluating how much contextualization is necessary to explain science topics for everyday people. 

% \grace{tighten the anlaysis for quotes}


% We found the most of the science explanations choose the Yes - Personal Language style as the primary narrative style (8/10 science explanations). Only two science explanations used the No - Personal Language as the primary narrative style (2/10). But most interestingly, \grace{there were 7 instances where writers choose to XYZ} 7 writers choose to merge elements from both the Personal and No Personal Language sections. This demonstrates that while there is a strong preference for writers to choose science narratives that contain personal language, often times writers use elements from both the personal and no personal language narrative styles. This demonstrates that individual writer preferences actually lie on a spectrum between the two extremes: Yes - Personal Language and No - Personal Language. By showing writers these two different options, writers can more effectively explore the continuum between these two extremes. To further understand, writer's perspectives on these two different style methods, we analyze the interview transcripts from the writer study. 


\begin{table}[H]
\resizebox{\textwidth}{!}{%
\begin{tabular}{|c|c|c|c|c|c|}
\hline
{\textbf{Participant ID}} & {\textbf{Domain}} & {\textbf{Topic}} & {\textbf{With Personal Language}} & {\textbf{Without Personal Language}} & {\textbf{Merged Narrative Styles}} \\ \hline\hline
  
\multirow{2}{*}{P1} & {Computer Science} & {Merge Sort} & {\checkmark} & & \\ \cline{2-6}
& {Statistics} & {Gradient Linear Additive Models} & & & Primary: Without Personal Language \\ \hline 

\multirow{2}{*}{P2} & {Computer Science} & {Merge Sort} & & {\checkmark} & \\ \cline{2-6}
& {Computer Science} & {Gradient Descent} & & & Primary: With Personal Language \\ \hline 
 
\multirow{2}{*}{P3} & {Computer Science} & {Merge Sort} & & & Primary: With Personal Language \\ \cline{2-6}
& {Qualitative Analysis} & {Ordinal Regression} & & & Primary: Without Personal Language \\ \hline 
 
\multirow{2}{*}{P4} & {Computer Science} & {Merge Sort} & {\checkmark} & & \\ \cline{2-6}
& { Natural Language Processing} & {Controlled Decoding} & {\checkmark} & & \\ \hline 

\multirow{2}{*}{P5} & {Computer Science} & {Merge Sort} & {\checkmark} & &  \\ \cline{2-6}
& {Natural Language Processing} & {Word Embeddings} & & & Primary: With Personal Language \\ \hline 
 
\multirow{2}{*}{P6} & {Computer Science} & {Merge Sort} & {\checkmark} & & \\ \cline{2-6}
& {Programming Languages} & { Formal Verification} & {\checkmark} & & \\ \hline 

\multirow{2}{*}{P7} & {Computer Science} & {Merge Sort} & & &  Primary: With Personal Language \\ \cline{2-6}
& {Natural Language Processing} & {Embedding Space} & {\checkmark} & &  \\ \hline 
 
\multirow{2}{*}{P8} & {Computer Science} & {Merge Sort} & {\checkmark} & & \\ \cline{2-6}
& { Quantum Algorithms} & { Grover's Algorithm} & & {\checkmark} & \\ \hline 

\multirow{2}{*}{P9} & {Computer Science} & {Merge Sort} & {\checkmark} & &  \\ \cline{2-6}
& {Tangible User Interfaces} & {4D Printing} & {\checkmark} & &  \\ \hline 
 
\multirow{2}{*}{P10} & {Computer Science} & {Merge Sort} & {\checkmark} & & \\ \cline{2-6}
& {Optimization Algorithms} & {Multiplicative Weights Update} & {\checkmark} & & \\ \hline \hline

& & \textbf{Total Count:} & \textbf{\begin{tabular}[c]{@{}c@{}}12\\ With Personal Language\end{tabular}} & \textbf{\begin{tabular}[c]{@{}c@{}}2 \\ Without Personal Language\end{tabular}} &  \textbf{\begin{tabular}[c]{@{}c@{}}6\\ Merged\end{tabular}} \\ \hline
\end{tabular}%
}
\caption{Participants, the topic they wrote on, and the corresponding narrative style that they chose. For merged narratives, ``primary" denotes the narrative style that the participant chose as their base narrative, and writers incorporated elements from the remaining style option into the primary narrative.}
\label{tab:writer-preferences-style}
\end{table}

% \grace{edit table to include primary and secondary}

% \begin{table}[H]
% \resizebox{\textwidth}{!}{%
% \begin{tabular}{cccccc}
% \hline
% {\textbf{Participant ID}} & {\textbf{Domain}} & {\textbf{Topic}} & {\textbf{Yes - Personal Language}} & {\textbf{Remove - Personal Language}} & {\textbf{Merged Narrative Styles}} \\ \hline
% {P1}                      & {Computer Science} & {Merge Sort}                      & {Primary} & {Secondary}                           & {Yes}                              \\ \hline
% {P1}                      & {Statistics} & {Gradient Linear Additive Models} & {Primary} & {Secondary}                           & {Yes}                              \\ \hline
% {P2}                      & {Computer Science} & {Merge Sort}                      & {Primary} & {Secondary}                           & {Yes}                              \\ \hline
% {P2}                      & {Computer Science} & {Gradient Descent}                & {Seconday} & {Primary}                             & {Yes}                              \\ \hline
% {P3}                      & {Computer Science} & {Merge Sort}                      & {Primary} & {} & {Yes}                              \\ \hline
% {P3}                      & {Qualitative Analysis}        & {Ordinal Regression}              & {Secondary}                        & {Primary}                             & {Yes}                              \\ \hline
% {P4}                      & {Computer Science} & {Merge Sort}                      & {Primary} & {} & \\ \hline
% {P4}                      & {Natural Language Processing} & {Controlled Decoding} & {Primary} & {} & \\ \hline
% {P5}                      & {Computer Science} & {Merge Sort}                      & {Primary} & {} & \\ \hline
% {P5}                      & {Natural Language Processing} & {Word Embeddings}                 & {Primary} & {Secondary}                           & {Yes}                              \\ \hline
%                                               &                                                  & \textbf{Total Count:} & \textbf{8 primary, 2 secondary}                       & \textbf{2 primary, 4 secondary} & \textbf{7 Merged}                                    
% \end{tabular}%
% }
% \caption{Participants, the topic they wrote on, and the corresponding narrative style that they choose. Primary denotes the narrative style that the participant choose and Secondary means that the writer incorporated elements from this narrative into Primary narrative style.}
% \label{tab:writer-preferences-style}
% \end{table}

% We found that different writers had different preferences and tolerances for incorporating personal language into their science explanations.

% P1, mentioned that seeing the explanation style with personal language "reading [the remove personal language generation] made it more obvious to me that those frills [the line: "this is cool, isn't it"] are very necessary posting something on on Twitter." This shows how the side-by-side comparison allowed the writer to realize the importance of personal language in making science communication engaging, by helping him think from the perspective of a reader.

% For P2, when seeing the side-by-side explanation styles with and without personal language he appreciated seeing the narrative without personal language because it helped him "stay focused on the concept and not get distracted by details in the example." P2 also mentioned how the personal language was distracting to him and it wasn't until he saw the explanation without personal language did he realize that there were still structural changes that he wanted to make: "reading it in simpler terms helped me determine the the structural changes I would want to make." This demonstrates that removing personal language also helped the writer check the structure and content of the science explanations without personal language that might distract him.

% But for P3, seeing the side-by-side comparisons of the narratives with and without personal narratives re-emphasized her own preferences for science communication. She said that "I think for scientific concept, it's nice to explain it in this non emotional way." Even though, she thought that the narrative with personal language looks "professional enough," the occurrence of phrases like "brilliant example of a theory-heavy concept" made the entire thread seem less professional. When finalizing the narrative, P3 decided to copy over the first tweet of the narrative with personal language because of its "vivid intro." This demonstrates how even when presented with the option to make the twitter thread more engaging, the writer's own preferences and habits for science communication overshadowed her willingness to try a different style of communication.

% For P5, he always chose the explanation that contained personal language, saying that "speaking in like a non personal language, in my opinion, makes it, like, less engaging, and also, from a reader's perspective, it looks too formal." When he encountered weird phrases such as, "Alexa, are you listening," he would opt to remove those phrases manually to "reflect to [his] personal like writing style." P5 would describe his writing style for science communication on social media as "the way you would describe something to a peer, using a informal manner, using examples, or using something that is very easy to grasp... finding an example that as many people can relate to," demonstrating that he already feels comfortable using and communicating science in an informal manner. 

% Comparing P3 and P5's responses, each writer's own personal preferences and ideas for what makes effective science communication shapes their willingness to use to use alternative styles of communication. For P3, her tendency to shy away from using personal language reflects her personal view that science communication should be professional in order to maintain credibility. While for P5, his own writing style is much more familiar with and open to using informal language to engage an audience.

% These findings illustrate that while writer's preferences for incorporating personal language into science communication may vary, most authors choose to find a middle ground between the two extremes: incorporating different elements from Yes and No Personal Language into their final science explanation. Presenting writers different options, allowed them to incorporate different degrees of personal language into their writing.   

% \grace{5 people merged them, becuase they felt it was a continuum, +juicy quotes, keep the writing tight}
% \grace{every two paragraphs a new thought, keep it snappy + moving along, answer is always two paragraph}

% Therefore, showing writers different options not only helps them decide which narrative structure to use, but also provide them with inspiration with other characteristics that they might want to include. Seeing different narrative structure helps participants decide (1) what to include and (2) how to include it. We will explore these dimensions further through the participant's experiences using the design probe...  \grace{find alternative work for system}





% \grace{same structure for personal language: how many chose each one + if they did merges, if they choose personal langaueg + then deleted things, which one they learned the hardest towards}
% - primary and then secondary
% \grace{help people undestand the design space, MERGING!! say merging, continuous, HOW MUCH do you lean on one example, which poeople chose merges + which merges they picked}



% \textbf{Exploring Different Voices by Working with Narratives with Generated Personal Language}
% Alternatively, for P1, having a narrative with personal language already built into the explanation structure, allowed him to experiment with writing in a conversational voice for science that he was not familiar with. Saying "if I were to write something myself, I wouldn't be necessarily good at, like, writing something like: "oh, did you ever wonder,"" P1 was more open to accepting GPT's generated language because seeing the narratives with and without personal language helped him realize when you get rid of the personal language the explanation "just reads like an academic paper, it doesn't actually read like something you would see on Twitter." Furthermore, P1 also mentioned that he doesn't "think of [his] voice as being the best suited to explain merge sort on the internet, and it feels like the voice that [explanation with personal language] uses is better for that. [The explanation with personal language] feels like this was something I could possibly read on the web. And I think if I were to just write in my own voice, it wouldn't sound like that. So I think this is better. Actually. I prefer this."

% \textbf{Working with GPT Generated Personal Language}
% For P2, when seeing the side-by-side explanation styles with and without personal language he appreciated seeing the narrative without personal language because it helped him "stay focused on the concept and not get distracted by details in the example." P2 also mentioned how the personal language was distracting to him and it wasn't until he saw the explanation without personal language did he realize that there were still structural changes that he wanted to make: "reading it in simpler terms helped me determine the the structural changes I would want to make."






% his experience with the side-by-side science explanations with and without personal language 


% - helped writers explore a dimension / register that they might not be familiar with using, but like when they see it
% - other times, the writer has strong preferences/ideas about how science communication should be ie. formal and thought that more personal language would undermine their authority, and thus always choose not to include personal language. 
% - another time, a writer merged tweets from both the yes - personal narrative and the remove personal narrative because sometimes the personal language overshadowed the technical aspects. 


% Writers have diverse preferences for how to structure their science explanations that vary depending on the topic, their comfort with using informal language in science communication, and the technical correctness and scientific clarity of each narrative structure. To understand the considerations that writers have for how they choose their preferred narrative structure, we analyze the think-aloud interview transcripts to find common themes. 

% Out of 20 science narratives that were writing, 14 narratives incorporated some aspect of the EWP narrative structure in their final output which included an example, walkthrough, and personal language. 9 of 20 narratives incorporated the EP narrative structure into their final output. 2 or 20 narratives used the WP narrative structure in their final output. 5 of 20 narratives choose to merge narrative structure. 

% Merging narrative structures means that writers select a primary narrative and incorporate additional elements from the secondary narrative. All writers who chose to merge narrative structures merged the EWP and EP condition. \grace{shoudl this just be primary narratives? currently covers if any part of a narrative is used in final it counts} 


% \grace{
% are these the two dimensions that we want to measure / disentangle
% 1) how does seeing multiple options help writers in the writing process
% 2) when do writers choose to use each narrative strategy 
% }

% \grace{Title: Seeing Different Narrative Structures Helps Writers}

\section{Discussions}\label{sec:discussion}
% \paragraph{Strategy space: one-step strategies vs two-step strategies.}
% In a sequential mechanism, depending on whether the agent's choice of first and second attributes coincide, we can distinguish 
% \emph{one-step strategies} and \emph{two-step strategies}.
% Consider an arbitrary fixed order sequential mechanism $(\tilde\classifier_A,\tilde\classifier_B,1)$ where $\tilde\classifier_A$ is offered as the first test.
% Consider an agent with initial attributes $\orifeatures\notin \tilde\classifier_A\cap\tilde\classifier_B$, i.e., such an agent fails both tests initially.
% Knowing that $\tilde\classifier_A$ is offered as the first test, this agent can choose the first attributes $\firstfeatures\in \tilde \classifier_A$ in order to pass the first test.
% Before the second test $\tilde\classifier_B$ takes place, he can again choose some  $\secondfeatures\in \tilde \classifier_B$ to pass the second test. 
% We call such a strategy a \emph{two-step strategy} (See \cref{subfig:two step}).
% Alternatively, the agent can choose some attributes  $\tilde\features^1\in \tilde\classifier_A\cap\tilde\classifier_B$ before both tests.
% Since $\tilde\features^1$ is in the intersection of both tests, the agent can pass both tests and get selected.
% We call such a strategy a \emph{one-step strategy} (See  \cref{subfig:one step}).
% Both \emph{one-step strategies} and \emph{two-step strategies} are feasible in sequential mechanisms.
% However, in simultaneous mechanisms, only one-step strategies are feasible.
% Hence the major difference between simultaneous mechanisms and sequential mechanisms is the strategy space of the agent.



% \begin{figure}[h]  \label{fig: 1-step vs 2-step}
% 	\centering 
% 	\begin{subfigure}[b]{0.45\linewidth}
% 		\begin{tikzpicture}[xscale=7.8,yscale=7.8]
		
% 		\draw [domain=0.68:1.4, thick] plot (\x, {3/4*\x+1/4});
% 		\node [left] at (1.38, 1.3 ) {$\tilde \classifier_B$};%:\feature_2\geq \frac34 \feature_1 +\frac14
% 		\draw [thick] (1,0.68) -- (1,1.3);
% 		\node [right] at (1, 1.3) {$\tilde\classifier_A$};%: \feature_1 \geq 1$
		
% 		\node [left] at (1.33,1.25) {$+$};
% 		\node [right] at (1, 1.25) {$+$};
		
% 		\node[red] at (1-0.1, 1-0.2 ) {$\circ$};
% 		\node [ below] at (1-0.1, 1-0.2 ) {$\orifeatures$};
		
% 		\node [red] at  (1,1) {$\circ$};
% 		\node [right, red] at (1, 1 ) {$\tilde\features^1$};
		
% 		\draw [densely dashdotted, red] (1-0.1, 1-0.2 ) --(1,1);
% 		\end{tikzpicture}
% 		\caption{One-step strategy}  
% 		\label{subfig:one step}
% 	\end{subfigure}
% 	\begin{subfigure}[b]{0.45\linewidth}
% 		\begin{tikzpicture}[xscale=7.8,yscale=7.8]
		
% 		\draw [domain=0.68:1.4, thick] plot (\x, {3/4*\x+1/4});
% 		\node [left] at (1.38, 1.3 ) {$\tilde \classifier_B$};%:\feature_2\geq \frac34 \feature_1 +\frac14
% 		\draw [thick] (1,0.68) -- (1,1.3);
% 		\node [right] at (1, 1.3) {$\tilde\classifier_A$};%: \feature_1 \geq 1$
		
% 		\node [left] at (1.33,1.25) {$+$};
% 		\node [right] at (1, 1.25) {$+$};
		
% 		\node[blue] at (1-0.1, 1-0.2 ) {\textbullet};
% 		\node [ below] at (1-0.1, 1-0.2 ) {$\orifeatures$};
		
% 		\node [blue] at (1, 1-0.072 ) {\textbullet};
% 		\node [blue,right] at (1, 1-0.072 ) {$\firstfeatures$};
		
% 		\node [blue] at  (1+0.1-5.5/24*0.6,1-0.22+5.5/24*0.8) {\textbullet};
% 		\node [blue,above] at  (1+0.1-5.5/24*0.6,1-0.22+5.5/24*0.8) {$\secondfeatures$};
		
		
% 		\draw [densely dashdotted, blue] (1-0.1,1-0.2) -- (1,1-0.072);
% 		\draw [densely dashdotted, blue]  (1,1-0.072) -- (1+0.1-5.5/24*0.6,1-0.22+5.5/24*0.8);
		
% 		\end{tikzpicture}
% 		\caption{Two-step strategy}
% 		\label{subfig:two step}
% 	\end{subfigure}
% 	\caption{Example of two different types of strategies}
% 	\rule{0in}{1.2em}$^\dag$\scriptsize In principle, using a one-step strategy in a simultaneous mechanism could incur a different cost compared to using it in a sequential mechanism.
% 	This is because in the sequential mechanism, there could be a positive cost of maintaining the attributes for extra time.
% \end{figure}  


% Subtly, in a sequential mechanism, even when agent chooses the same attributes for both tests and he does not opt out after the first test, it could incur a positive cost of maintaining the attributes.
% However, this creates unnecessary complication and would not add any new insight to the analysis. 
% \cref{assump: cost function one step} 
% makes sure that once the agent chooses some attributes, there is no cost of maintaining it. 
% Therefore, we can safely write any \emph{one-step strategy} $\strategies=\tilde\features^1$  as 
% $\strategies=(\tilde\features^1,\tilde\features^1)$.

\paragraph{Dimensionality and degenerate sequential mechanisms.}
Our model cannot be reduced to a problem where agent has a one-dimension attribute and the principal offers two selection cutoffs.
Consider a setup where the agent has a one-dimensional attribute $\feature\in \bbR$.
The principal would like to select any agent with attribute $\feature\in [b_1,b_2]$, where $b_1<b_2$.
Suppose now the principal announces that any agent with attribute belonging to $[\tilde b_1,\tilde b_2]$ will be selected, with $\tilde b_1<\tilde b_2$. 
The corresponding tests would be $\tilde\classifier_A: \feature\geq \tilde b_1$ and $\tilde\classifier_B: \feature\leq \tilde b_2$.
Then no matter what the order of the tests is, the agent's best response is fixed: for any $\feature < \tilde b_1$, $\strategies(\feature)=(\tilde b_1,\tilde b_1)$; for any $\feature > \tilde b_2$, $\strategies(\feature)=(\tilde b_2,\tilde b_2)$; for any $\tilde b_1\leq \feature \leq \tilde b_2$, $\strategies(\feature)=(\feature,\feature)$.

Next we introduce the concept of degenerate testing mechanisms. 
\begin{definition}[degenerate mechanism]
	A sequential mechanism $(\tilde\classifier_A,\tilde\classifier_B,\probprincipal,\disclose)$ is said to be \emph{degenerate} if A's best response is the same regardless of the order of the tests.
\end{definition}

We summarize the above observation in the following proposition.
\begin{proposition}
	If the agent's attribute is one-dimension, then any testing mechanism $(\tilde\classifier_A,\tilde\classifier_B,\probprincipal,\disclose)$ is degenerate.
\end{proposition}

\citet{perez2022test} studies a test design problem where the agent's attribute is one dimensional.
The above observation provides a motivation to study the case where agent has multi dimensional attributes and how the order of the tests play a role in such settings. 
As a starting point, we study the simplest case where agent has two dimensional attributes and principal has two tests.




\paragraph{Modeling tests.}
We interpret a test as an evaluation of whether one (or some) aspect(s) of the agent meets the the principal's requirement.
We assume that the agent does not which aspect(s) he is being evaluated in each test.
We argue that this is a reasonable assumption.
For example, in a math test, the students taking the test do not know whether they are being evaluated on their mathematical thinking skill or their mastery of mathematical tools.
This slightly differs from the way we use the word test in English.
When we say there is a math test, the math test is a subject that includes at least the above mentioned two \emph{aspects of ability} being evaluated.
Hence if we interpret a test as an evaluation of aspect(s) of ability, it is unlikely that the students know how each aspect of ability is evaluated (or graded) unless they are told.
% Before we proceed, we point out one subtle but important distinction between the \emph{tests} in our model and the tests we use in day-to-day life.
% In day-to-day life, when we refer to a test or a school test, we usually refer to a specific subject or a topic, for example, math test, English test, etc..
% The testee usually knows which \emph{subject} they are being tested.
% However, in our model, a \emph{test} is an evaluation to one skill or multiple skills.
% Therefore, even if the testee knows which \emph{subject} they are being tested, they may still not know which \emph{aspect(s) of ability} they are being tested. 


\paragraph{Simultaneous mechanisms are not special case of sequential mechanisms.}
Although in simultaneous mechanisms the order of the tests does not matter, they cannot be viewed as special cases of sequential mechanisms, i.e.,  $\simultaneous\not\subset \sequential$.
First of all, in any game specified by a simultaneous mechanism, the agent only has one information node.
Therefore, a simultaneous mechanism cannot be achieved by any fixed order mechanism or random order mechanism with disclosure.
Second, even though in the game specified by a random order mechanism without disclosure, the agent also has one information node, his strategy space is much larger.
This is because in any simultaneous mechanism, the agent is essentially restricted to use \emph{one-step} strategies, while in any random order mechanism without disclosure, the agent can use either \emph{one-step} strategies or \emph{two-step} strategies. 
An alternative way to see this is that in any random order mechanism without disclosure, the agent is evaluated by a \emph{linear} test each time.
However, in any simultaneous mechanism, the agent is evaluated by the intersection of two \emph{linear} tests each time. 
More formally, let $\twolineartest=\{\tilde\classifier_1\cap\tilde\classifier_2:\tilde\classifier_i\in \lineartest, i=1,2\}$ be the space of tests that can be used in any simultaneous mechanism.
Then $\lineartest\subset \twolineartest$.
Therefore, by changing the timing to offer the tests, simultaneous mechanisms are essentially enlarging the space of tests that can be used.

\paragraph{Dynamic nature of sequential mechanisms.}
To further illustrate why the game induced by any sequential mechanism is not static, let's consider the cost function $\cost(\orifeatures,\firstfeatures,\secondfeatures)=\metric(\orifeatures,\firstfeatures)+\metric(\firstfeatures,\secondfeatures)$, where $\metric(\cdot,\cdot)$ is the Euclidean distance.
Consider two fixed order mechanisms that use the same two tests $\tilde\classifier_A$ and $\tilde\classifier_B$ and they only differ in terms of the order of the two tests.
In \cref{subfig:two step A-B}, the mechanism offers $\tilde\classifier_A$ as the first test, while in \cref{subfig:two step B-A}, the mechanism offers $\tilde\classifier_B$ as the first test.
Consider an agent with attributes $\orifeatures$ as in \cref{subfig:two step A-B}.
Notice that $\orifeatures$ pass test $\tilde\classifier_A$ but fail test $\tilde\classifier_B$.
Under the fixed order mechanism $(\tilde\classifier_A,\tilde\classifier_B,1)$, the optimal strategy of such attributes is to choose $\firstfeatures=\orifeatures$ and a different $\secondfeatures$ that are the projection of $\orifeatures$ on the boundary line of $\tilde\classifier_B$ (See \cref{subfig:two step A-B}).
% We call this a \emph{two-step} strategy.
Such a strategy is the least costly for the agent and guarantees that the agent is selected under this fixed order mechanism.
However, under another fixed order mechanism $(\tilde\classifier_B,\tilde\classifier_A,1)$, the optimal strategy of such attributes is to choose $\firstfeatures=\secondfeatures$ that are the intersection of  the boundary line of $\tilde\classifier_A$ and $\tilde\classifier_B$ (See \cref{subfig:two step A-B}).
This is because now $\tilde\classifier_B$ is the first test the agent needs to pass and $\orifeatures$  are so far away from  $\tilde\classifier_B$ that it is less costly to pass the two tests at the same time.
% We call this a \emph{one-step} strategy.
This example shows that a sequential mechanism is naturally dynamic and as a result, the agent's strategy is also dynamic. 



%---------------------------------------
\begin{figure}[h]  \label{fig: sequential is dynamic}
	\centering 
	\begin{subfigure}[b]{0.45\linewidth}
		\begin{tikzpicture}[xscale=7.8,yscale=7.8]
		
		\draw [domain=0.68:1.4, thick] plot (\x, {3/4*\x+1/4});
		\node [left] at (1.38, 1.3 ) {$\tilde \classifier_B$};%:\feature_2\geq \frac34 \feature_1 +\frac14
		\draw [thick, red] (1,0.68) -- (1,1.3);
		\node [right] at (1, 1.3) {$\tilde\classifier_A$};%: \feature_1 \geq 1$
		
		\node [left] at (1.33,1.25) {$+$};
		\node [right] at (1, 1.25) {$+$};
		
		\node[black] at (1, 1-0.2 ) {\textbullet};
		\node [ right] at (1, 1-0.2 ) {$\orifeatures\textcolor{blue}{=\firstfeatures}$};
		
		\node [blue] at  (1-12/125,1-9/125) {\textbullet};
		\node [left, blue] at (1-12/125,1-9/125 ) {$\secondfeatures$};
		
		\draw [densely dashdotted, blue] (1, 1-0.2 ) --(1-12/125,1-9/125);
		\end{tikzpicture}
		\caption{Two-step strategy: $\tilde\classifier_A\rightarrow\tilde\classifier_B$}  
		\label{subfig:two step A-B}
	\end{subfigure}
	\begin{subfigure}[b]{0.45\linewidth}
		\begin{tikzpicture}[xscale=7.8,yscale=7.8]
		
		\draw [domain=0.68:1.4, thick, red] plot (\x, {3/4*\x+1/4});
		\node [left] at (1.38, 1.3 ) {$\tilde \classifier_B$};%:\feature_2\geq \frac34 \feature_1 +\frac14
		\draw [thick] (1,0.68) -- (1,1.3);
		\node [right] at (1, 1.3) {$\tilde\classifier_A$};%: \feature_1 \geq 1$
		
		\node [left] at (1.33,1.25) {$+$};
		\node [right] at (1, 1.25) {$+$};
		
		\node[blue] at (1, 1-0.2 ) {\textbullet};
		\node [ right] at (1, 1-0.2 ) {$\orifeatures$};
		
		% \node [blue] at (1-12/125,1-9/125) {\textbullet};
		% \node [blue,left] at (0.8,1 ) {$\tilde\features$};
		
		\node [blue] at  (1,1) {\textbullet};
		\node [blue,right] at  (1,1) {$\firstfeatures=\secondfeatures$};
		
		
		\draw [densely dashdotted, blue] (1,1-0.2) -- (1,1);
		% \draw [densely dashdotted, blue]  (1,1) -- (0.8,1);
		
		\end{tikzpicture}
		\caption{Two-step strategy: $\tilde\classifier_B\rightarrow\tilde\classifier_A$}
		\label{subfig:two step B-A}
	\end{subfigure}
	\caption{Dynamic nature of sequential mechanisms}
	\rule{0in}{1.2em}$^\dag$\scriptsize Suppose $\cost(\orifeatures,\firstfeatures,\secondfeatures)=\metric(\orifeatures,\firstfeatures)+\metric(\firstfeatures,\secondfeatures)$, where $\metric(\cdot,\cdot)$ is the Euclidean distance. The left panel shows the optimal strategy under the fixed order mechanism that offers test $\tilde\classifier_A$ as the first test, while the right panel shows the optimal strategy under the fixed order mechanism that offers test $\tilde\classifier_B$ as the first test.
\end{figure}  

\paragraph{More tests do not help.}
In the manipulation setting, adding more tests only makes a procedure more stringent and it is counter-productive. 
Suppose the principal offers test $\classifier_1$ and $\classifier_2$ repeatedly. For simplicity, suppose the first test is $\classifier_1$, the second is $\classifier_2$ and the third is $\classifier_1$. For those types that find it profitable to pass both tests together under a fixed-order procedure with two tests, adding an extra test does not change their incentives.
For those types that find it profitable to pass one test at a time, adding a third test only make it more costly to pass one test at a time.
Hence, the benefit of the fixed-order procedure is diminished under more tests.

In the investment setting, under the optimal simultaneous mechanism, adding one more test cannot improve the outcome since using two tests already achieve the first best.


\paragraph{Endogenous agent's technology.}
We assume that the agent's technology, whether he manipulates or invests, is exogenous. 
This usually can be explained by the different monitoring strength of the economic environment.
In the banking application, US banks are tested daily and therefore there is no room for manipulation.
In contrast, European banks are tested monthly or quarterly.
The low monitoring intensity allows banks to temporarily change their balance sheet.

Suppose instead now when the agent changes his type from $\orifeatures$ to $\firstfeatures$, he can choose between  manipulation, which costs $0.2\cdot\onecost(\orifeatures,\firstfeatures)$ and investment, which costs $0.5\cdot\onecost(\orifeatures,\firstfeatures)$.
Then the only equilibrium is such that every type chooses manipulation and the principal chooses the optimal fixed-order sequential mechanism under manipulation, given that the principal's objective is \ref{max qualified}. 
This is because (1) whichever mechanism the principal chooses, the agent prefers manipulation over investment, and (2) no type can credibly commits to investment. 
As a result, although every type would have been better-off by choosing investment, it is not an equilibrium.



\paragraph{Informational robustness.}
When the agent manipulates , the optimal tests are chosen based on the information about the agent's cost.
Whenever there is a small amount of uncertainty in the agent's cost function,  the principal needs to choose even more stringent tests so as not to select any unqualified agent in the worst case where the agent's cost happens to be least costly one.
In contrast, when the agent invests, the optimal tests coincide with the qualified region. 
Hence the optimal mechanism in the investment setting is more robust to the agent's information.

% \paragraph{Stringency of testing procedures}
% Fixing the tests, we use how difficult it is for the agent to to manipulate to measure the stringency of a testing procedure.
% Roughly speaking, a simultaneous procedure is weakly more stringent than a random-order sequential procedure without disclosure, which is  more stringent than a random-order sequential procedure with disclosure,  which is  more stringent than a fixed-order sequential procedure.

% \paragraph{Noisy tests}
% A stochastic \emph{linear} test $\tilde\classifier$ is a half plane and a random variable $\epsilon$ such that the agent passes the test if and only if agent's attributes $\features +\epsilon \in \tilde\classifier$.
% The test result of an agent with attributes $\features$ under test $\tilde\classifier$ is $\test= \indicate{\features+\epsilon\in \tilde\classifier}$.
% Incorporating stochasticity into the test does not change our results.


% \paragraph{Randomizing over multiple tests}




% \paragraph{Connections to persuasion problems.}
% \citet{glazer2004optimal} and \citet{sher2014persuasion} study persuasion problems with evidence where the agent's private information also has two dimensions. 
% Instead of passing tests like in our setting, the agent needs to provide hard evidence to persuade the principal.
% The main difference of our paper compared to these two is that we adopt a slightly more general approach to model what information the agent can use to persuade the principal, which is similar to the modeling approached used in \citet{perez2022test}.
% Evidence usually contains the truth and therefore cannot be fabricated. 
% Or rather, we can view it as the cost of adopting different attributes is infinity.
% In this sense, it is more general to assume that agent can adopt different attributes to present to the principal with some cost. 
% Then persuasion with evidence can be viewed as (1) the agent's cost of changing attributes is infinitely high, and (2) the principal can use \emph{perfect} tests, i.e., principal can observe agent's attributes in each round of communication.


\section*{Limitations}
Our study focuses on entity unlearning, leaving hazardous knowledge and copyrighted content unlearning unexplored. These cases may require different evaluation strategies.

Additionally, our experiments use mid-sized models (LLaMA-2-7B-Chat, LLaMA-3-8B-Instruct). Larger models, with their computational demands and structural differences, may respond differently. Future research should assess their applicability to such models.



\section{Conclusion}
\label{sec:conclusion}
\vspace{-2.5mm}
We introduce a new research direction in collaborative driving and present the first solution. Empirical results show our approach can significantly reduce data collection and development efforts, advancing safer autonomous systems. 

\nbf{Limitations and future work}
Following existing benchmarks~\citep{xu2022opv2v, xu2023v2v4real}, \ours focuses on vehicle-like objects. Future work could extend it to broader objects and static entities (\eg, traffic signs, signals) essential for real-world traffic.


%%
%% The next two lines define the bibliography style to be used, and
%% the bibliography file.
\bibliographystyle{ACM-Reference-Format}
\bibliography{paper}

%TC:ignore
\appendix
\section{Appendix}

\begin{figure}
    \centering
    \includegraphics[width=0.9\linewidth]{figures/02_traning_taxonomies.png}
    \caption{Training Taxonomies}
    \label{fig:traning_taxonomies}
\end{figure}

\begin{figure}
    \centering
    \includegraphics[width=1.0\linewidth]{figures/03_preference_tuning_taxonomies.png}
    \caption{Preference Tuning Taxonomies}
    \label{fig:preference_tuning_taxonogy}
\end{figure}

% \begin{table*}[ht]
%     \centering
%     \begin{tabular}{|>{\raggedright\arraybackslash}m{4cm}|>{\raggedright\arraybackslash}m{3cm}|>{\raggedright\arraybackslash}m{7cm}|}
\hline
\textbf{Name} & \textbf{Notation} & \textbf{Description} \\
\hline
Input Sequence & $x$ & Input sequence that is passed to the model. \\
Output Sequence & $y$ & Expected label or output of the model. \\
\hline
Dispreferred Response & $y_l$ & Negative samples for reward model training. \\
Preferred Response & $y_w$ & Positive samples for reward model training. \\
\hline
Optimal Policy Model & $\pi^*$ & Optimal policy model. \\
Policy Model & $\pi_\theta$ & Generative model that takes the input prompt and returns a sequence of output or probability distribution. \\
Reference Policy Model & $\pi_{\text{ref}}$ & Generative model that is used as a reference to ensure the policy model is not deviated significantly. \\
\hline
Preference Dataset & $\mathcal{D}_{\text{pref}}$ & Dataset with a set of preferred and dispreferred responses to train a reward model. \\
SFT Dataset & $\mathcal{D}_{\text{sft}}$ & Dataset with a set of input and label for supervised fine-tuning. \\
\hline
Loss Function & $\mathcal{L}$ & Loss function. \\
Regularization Hyper-parameters & $\alpha, \beta_{\text{reg}}$ & Regularization Hyper-parameters for preference tuning. \\
Reward & $r$ & Reward score. \\
Target Reward Margin & $\gamma$ & The margin separating the winning and losing responses. \\
Variance & $\beta_i$ & Variance (or noise schedule) used in diffusion models. \\
\hline
\end{tabular}
%     \caption{Caption}
%     \label{tab:notation}
% \end{table*}
%TC:endignore


\end{document}
\endinput
%%
%% End of file `sample-authordraft.tex'.


\bibliography{bibliography}
\bibliographystyle{icml2025}


%%%%%%%%%%%%%%%%%%%%%%%%%%%%%%%%%%%%%%%%%%%%%%%%%%%%%%%%%%%%%%%%%%%%%%%%%%%%%%%
%%%%%%%%%%%%%%%%%%%%%%%%%%%%%%%%%%%%%%%%%%%%%%%%%%%%%%%%%%%%%%%%%%%%%%%%%%%%%%%
% APPENDIX
%%%%%%%%%%%%%%%%%%%%%%%%%%%%%%%%%%%%%%%%%%%%%%%%%%%%%%%%%%%%%%%%%%%%%%%%%%%%%%%
%%%%%%%%%%%%%%%%%%%%%%%%%%%%%%%%%%%%%%%%%%%%%%%%%%%%%%%%%%%%%%%%%%%%%%%%%%%%%%%
\newpage
\appendix
\onecolumn
\newpage
\appendix

\renewcommand{\figurename}{Supplementary Figure}
\renewcommand{\tablename}{Supplementary Table}
\setcounter{figure}{0}
\setcounter{table}{0}

    



\section{Details of datasets}
This section provides additional details about the dataset used to evaluate the downstream tasks. \Cref{tab:disease_definition} lists the ICD-10 codes and medications used to identify the diagnoses for each disease. \Cref{tab:characteristic} presents the distribution of patient characteristics for each disease. \Cref{fig:nyu_langone_prevalence,fig:nyu_longisland_prevalence} illustrates the prevalence of each disease in the downstream tasks for the NYU Langone and NYU Long Island datasets, highlighting the imbalances present in these tasks.

\begin{table}[!htpb]
    \centering
    \caption{The definition of diseases in EHR by diagnosis codes and medications.}
    \begin{tabular}{lr}
    \toprule
         Disease &  Definition in EHR \\
    \midrule
       IPH  &  I61.0, I61.1, I61.2, I61.3, I61.4, I61.8, I61.9 \\
       IVH  &  I61.5, P52.1, P52.2, P52.3  \\
       ICH  &  IPH + IVH + I61.6, I62.9, P10.9, P52.4, P52.9 \\
       SDH  &  S06.5, I62.0 \\
       EDH  &  S06.4, I62.1 \\
       SAH  &  I60.*, S06.6, P52.5, P10.3  \\
       Tumor  &  C71.*, C79.3, D33.0, D33.1, D33.2, D33.3, D33.7, D33.9  \\
       Hydrocephalus  &  G91.* \\
       Edema  &  G93.1, G93.5, G93.6, G93.82, S06.1 \\
       \multirow{2}{*}{ADRD}  &  G23.1, G30.*, G31.01, G31.09, G31.83, G31.85, G31.9, F01.*, F02.*, F03.*, G31.84, G31.1, \\ 
       & \textbf{Medication:} DONEPEZIL, RIVASTIGMINE, GALANTAMINE, MEMANTINE, TACRINE \\ 
    \bottomrule
    \end{tabular}
    \label{tab:disease_definition}
\end{table}

\begin{table}[!htbp]
\centering
\caption{Demographic characteristics of patients associated with scans from the NYU Langone dataset, matched with electronic health records (EHR) and utilized in downstream tasks.}
\label{tab:characteristic}

 The characteristic table on NYU Langone dataset matched with EHR.
\begin{tabular}{ll|rr|r}
\toprule
                       \textbf{Cohort} &  &           \textbf{Male (\%)} &          \textbf{Female (\%)} &     \textbf{Age (std)} \\
\midrule
 --- & All (n=270,205) & 128,113 (47.41\%) & 142,092 (52.59\%) & 63.64 (19.68) \\
\midrule
       Tumor & Neg (n=260,704) & 123,338 (47.31\%) & 137,366 (52.69\%) & 63.85 (19.72) \\
             & Pos (n=9,501) &   4,775 (50.26\%) &   4,726 (49.74\%) & 57.80 (17.67) \\
\midrule
HCP & Neg (n=253,000) & 118,881 (46.99\%) & 134,119 (53.01\%) & 63.67 (19.72) \\
              & Pos (n=17,205) &   9,232 (53.66\%) &   7,973 (46.34\%) & 63.18 (19.11) \\
\midrule
Edema & Neg (n=242,576) & 112,987 (46.58\%) & 129,589 (53.42\%) & 63.96 (19.84) \\
      & Pos (n=27,629) &  15,126 (54.75\%) &  12,503 (45.25\%) & 60.81 (17.97) \\
\midrule
ADRD  & Neg (n=232,667) & 111,159 (47.78\%) & 121,508 (52.22\%) & 61.31 (19.55) \\
      & Pos (n=37,538) &  16,954 (45.16\%) &  20,584 (54.84\%) & 78.09 (13.30) \\
\midrule
          IPH & Neg (n=251,308) & 117,692 (46.83\%) & 133,616 (53.17\%) & 63.58 (19.82) \\
              & Pos (n=18,897) &  10,421 (55.15\%) &   8,476 (44.85\%) & 64.39 (17.69) \\
\midrule
          IVH & Neg (n=258,232) & 121,686 (47.12\%) & 136,546 (52.88\%) & 63.65 (19.79) \\
              & Pos (n=11,973) &   6,427 (53.68\%) &   5,546 (46.32\%) & 63.45 (17.19) \\
\midrule
          SDH & Neg (n=248,468) & 114,869 (46.23\%) & 133,599 (53.77\%) & 63.44 (19.78) \\
              & Pos (n=21,737) &  13,244 (60.93\%) &   8,493 (39.07\%) & 65.95 (18.33) \\
\midrule
          EDH & Neg (n=265,431) & 125,113 (47.14\%) & 140,318 (52.86\%) & 63.77 (19.64) \\
              & Pos (n=4,774) &   3,000 (62.84\%) &   1,774 (37.16\%) & 56.53 (20.75) \\
\midrule
          SAH & Neg (n=251,594) & 118,424 (47.07\%) & 133,170 (52.93\%) & 63.79 (19.76) \\
              & Pos (n=18,611) &   9,689 (52.06\%) &   8,922 (47.94\%) & 61.59 (18.49) \\
\midrule
          ICH & Neg (n=229,851) & 105,498 (45.90\%) & 124,353 (54.10\%) & 63.41 (19.93) \\
              & Pos (n=40,354) &  22,615 (56.04\%) &  17,739 (43.96\%) & 64.93 (18.14) \\
\bottomrule
\end{tabular}
\end{table}


\begin{table}[!h]
    \centering
    \caption*{\textbf{Supplementary \Cref{tab:characteristic} Continued.} Demographic characteristics of patients associated with scans from the NYU Long Island dataset, matched with electronic health records (EHR) and utilized in downstream tasks.}
\begin{tabular}{ll|rr|r}
\toprule
                       \textbf{Cohort} &  &           \textbf{Male (\%)} &          \textbf{Female (\%)} &     \textbf{Age (std)} \\
\midrule
--- & All (n=22,158) & 9,580 (43.23\%) & 12,578 (56.77\%) & 68.33 (18.14) \\
\midrule
Tumor & Neg (n=21,578) & 9,275 (42.98\%) & 12,303 (57.02\%) & 68.59 (18.08) \\
      & Pos (n=580) &   305 (52.59\%) &    275 (47.41\%) & 58.78 (17.79) \\
\midrule
HCP   & Neg (n=20,653) & 8,718 (42.21\%) & 11,935 (57.79\%) & 69.05 (17.90) \\
      & Pos (n=1,505) &   862 (57.28\%) &    643 (42.72\%) & 58.52 (18.48) \\
\midrule
Edema & Neg (n=19,402) & 8,068 (41.58\%) & 11,334 (58.42\%) & 68.89 (18.27) \\
      & Pos (n=2,756) & 1,512 (54.86\%) &  1,244 (45.14\%) & 64.36 (16.66) \\
\midrule
ADRD  & Neg (n=19,537) & 8,391 (42.95\%) & 11,146 (57.05\%) & 66.78 (18.28) \\
      & Pos (n=2,621) & 1,189 (45.36\%) &  1,432 (54.64\%) & 79.90 (11.77) \\
\midrule
IPH   & Neg (n=19,357) & 7,974 (41.19\%) & 11,383 (58.81\%) & 68.97 (18.27) \\
      & Pos (n=2,801) & 1,606 (57.34\%) &  1,195 (42.66\%) & 63.89 (16.48) \\
\midrule
IVH   & Neg (n=19,636) & 8,164 (41.58\%) & 11,472 (58.42\%) & 68.96 (18.22) \\
      & Pos (n=2,522) & 1,416 (56.15\%) &  1,106 (43.85\%) & 63.43 (16.66) \\
\midrule
SDH   & Neg (n=20,885) & 8,870 (42.47\%) & 12,015 (57.53\%) & 68.33 (18.21) \\
      & Pos (n=1,273) &   710 (55.77\%) &    563 (44.23\%) & 68.37 (16.83) \\
\midrule
EDH   & Neg (n=21,912) & 9,443 (43.10\%) & 12,469 (56.90\%) & 68.33 (18.16) \\
      & Pos (n=246) &   137 (55.69\%) &    109 (44.31\%) & 68.19 (15.59) \\
\midrule
SAH   & Neg (n=20,652) & 8,824 (42.73\%) & 11,828 (57.27\%) & 68.68 (18.12) \\
      & Pos (n=1,506) &   756 (50.20\%) &    750 (49.80\%) & 63.58 (17.65) \\
\midrule
ICH   & Neg (n=18,388) & 7,456 (40.55\%) & 10,932 (59.45\%) & 68.92 (18.35) \\
      & Pos (n=3,770) & 2,124 (56.34\%) &  1,646 (43.66\%) & 65.48 (16.77) \\
\bottomrule
\end{tabular}
\end{table}

\begin{figure}[!ht]
    \centering
    \includegraphics[width=0.8\textwidth]{images/NYU_Langone_prevalence.pdf}
    \caption{Disease prevalence of NYU Langone }
    \label{fig:nyu_langone_prevalence}
\end{figure}

\begin{figure}[!h]
    \centering
    \includegraphics[width=0.8\textwidth]{images/NYU_Longisland_prevalence.pdf}
    \caption{Disease prevalence of NYU Longisland dataset}
    \label{fig:nyu_longisland_prevalence}
\end{figure}



\section{Data augmentation details}
\label{sec:dataaug_details}
We applied Random Flipping across all three dimensions, Random Shift Intensity with offset $0.1$ for both pre-training and fine-tuning. For DINO training. random Gaussian Smoothing with sigma=$(0.5-1.0)$ is applied across all dimensions, Random Gamma Adjust is applied with gamma=$(0.2-1.0)$.


\section{Additional experiment results}
This section provides additional experimental results with more details.
Supplementary \Cref{fig:channels-ablation,fig:patches-ablation} compares the performance of the foundation model using different numbers of channels and patch sizes, demonstrating that the architecture design of our foundation model is optimal. 

Supplementary \Cref{fig:radar-comparison-merlin} compares our foundation model with a foundation CT model from previous studies, Merlin\cite{blankemeier2024merlinvisionlanguagefoundation}, which was trained on abdomen CT scans with corresponding radiology report pairs. Our model demonstrates superior performance on head CT scans.

Supplementary \Cref{fig:probing-comparison-gemini} compares our foundation model with Google CT Foundation model~\cite{yang2024advancingmultimodalmedicalcapabilities}, which was trained on large scale and diverse CT scans from different anatomy with corresponding radiology report pairs. Our model consistently shows improved performance across the board even though our model was pre-trained with less samples.

Supplementary \Cref{fig:probing_comparison} compares the performance on downstream tasks with various supervised tuning methods applied to foundation models pretrained with the MAE and DINO frameworks. Per-pathology comparisons are shown in Supplementary \Cref{fig:probing-comparison-perpath,fig:probing-comparison-perpath-dino}. Meanwhile, supplementary \Cref{fig:boxplot_scaling} complements \Cref{fig:scaling_law}, illustrating the per-pathology performances of foundation models pretrained with different scales of training data.

Supplementary \Cref{fig:batch_effect,fig:thickness-ablation} studies the impact of batch effect caused by different CT scan protocols of slice thickness and machine manufacturer. Detailed per-pathology performances are shown in Supplementary \Cref{fig:slice_thickness_per_pathology,fig:manufacturer_per_pathology}.

\begin{figure}[!htpb]
    \centering
    \makebox[\textwidth][l]{%
        \hspace{0.3\textwidth}\textbf{NYU Langone}
    } \\[0.2cm]
    \includegraphics[trim={0 0 0 0},clip,height=0.3\textwidth, width=0.3\textwidth]{figures/abla_chans/AUC_chans_NYU.pdf}
    \includegraphics[trim={0 0 0 0},clip,height=0.3\textwidth, width=0.55\textwidth]{figures/abla_chans/AP_chans_NYU.pdf}\\
    \makebox[\textwidth][l]{
        \hspace{0.34\textwidth}\textbf{RSNA}
    } \\[0.2cm]
    \includegraphics[trim={0 0 0 0},clip,height=0.3\textwidth, width=0.3\textwidth]{figures/abla_chans/AUC_chans_RSNA.pdf}
    \includegraphics[height=0.3\textwidth, width=0.55\textwidth]{figures/abla_chans/AP_chans_RSNA.pdf} 
    \caption{\textbf{Comparison of Different Channels Performance.} This plot compares the performance of models trained using different numbers of channels (channels from multiple HU intervals vs. a single HU interval). Across two datasets, models using three channels from different HU intervals consistently outperform those using a single channel with a fixed HU interval. All models were pre-trained on $100\%$ of the pretraining data with MAE.}
    \label{fig:channels-ablation}
\end{figure}


\begin{figure}[!htpb]
    \centering
    \makebox[\textwidth][l]{%
        \hspace{0.3\textwidth}\textbf{NYU Langone}
    } \\[0.2cm]
    \includegraphics[trim={0 0 0 0},clip,height=0.3\textwidth, width=0.3\textwidth]{figures/abla_patches/AUC_patches_NYU.pdf}
    \includegraphics[trim={0 0 0 0},clip,height=0.3\textwidth, width=0.55\textwidth]{figures/abla_patches/AP_patches_NYU.pdf}\\
    \makebox[\textwidth][l]{
        \hspace{0.34\textwidth}\textbf{RSNA}
    } \\[0.2cm]
    \includegraphics[trim={0 0 0 0},clip,height=0.3\textwidth, width=0.3\textwidth]{figures/abla_patches/AUC_patches_RSNA.pdf}
    \includegraphics[height=0.3\textwidth, width=0.55\textwidth]{figures/abla_patches/AP_patches_RSNA.pdf} 
    \caption{\textbf{Comparison of Different Patches Performance.} This plot compares the performance of models trained with different patch sizes (12 vs. 16). The results demonstrate that smaller patch sizes consistently achieve better performance. All models were pre-trained on $100\%$ of the pretraining data with MAE.}
    \label{fig:patches-ablation}
\end{figure}


\begin{figure*}
    \centering
    \makebox[\textwidth][l]{%
        \hspace{0.06\textwidth}
        \textbf{NYU Langone} \hspace{0.06\textwidth} \textbf{NYU Long Island} \hspace{0.11\textwidth} \textbf{RSNA} \hspace{0.18\textwidth} \textbf{CQ500}
    } \\[0.2cm]
    \includegraphics[trim={0 0 0 0},clip,height=0.21\textwidth, width=0.21\textwidth]{figures/abla_radarplot_merlin/AUC_NYU.pdf}
    \includegraphics[trim={0 0 0 0},clip,height=0.21\textwidth, width=0.21\textwidth]{figures/abla_radarplot_merlin/AUC_Longisland.pdf}
    \includegraphics[trim={0 0 0 0},clip,height=0.21\textwidth, width=0.21\textwidth]{figures/abla_radarplot_merlin/AUC_RSNA.pdf}
    \includegraphics[trim={0 0 0 0},clip,height=0.21\textwidth, width=0.35\textwidth]{figures/abla_radarplot_merlin/AUC_CQ500.pdf}\\[0.2cm]
    \includegraphics[height=0.21\textwidth, width=0.21\textwidth]{figures/abla_radarplot_merlin/AP_NYU.pdf} 
    \includegraphics[height=0.21\textwidth, width=0.21\textwidth]{figures/abla_radarplot_merlin/AP_Longisland.pdf} 
    \includegraphics[height=0.21\textwidth, width=0.21\textwidth]{figures/abla_radarplot_merlin/AP_RSNA.pdf}
    \includegraphics[height=0.21\textwidth, width=0.35\textwidth]{figures/abla_radarplot_merlin/AP_CQ500.pdf}
    \caption{\textbf{Comparison to previous 3D Foundation Model.} This plot compares the performance of our model with Merlin~\cite{blankemeier2024merlinvisionlanguagefoundation} and models trained from scratch across four datasets for our model and ResNet50-3D. Our DINO trained model is used in this comparison. Our model demonstrates consistently superior performance across majority of diseases, with the exception of epidural hemorrhage (EDH) in the CQ500 dataset.}
    \label{fig:radar-comparison-merlin}
\end{figure*}



\begin{figure*}
    \centering
    \makebox[\textwidth][l]{%
        \hspace{0.10\textwidth}
        \textbf{NYU Langone} \hspace{0.08\textwidth} \textbf{NYU Long Island} \hspace{0.1\textwidth} \textbf{RSNA} \hspace{0.15\textwidth} \textbf{CQ500}
    } \\[0.2cm]
    \includegraphics[trim={0 0 0 0},clip, width=0.22\textwidth]{figures/abla_probing/AUC_NYU.pdf}
    \includegraphics[trim={0 0 0 0},clip, width=0.22\textwidth]{figures/abla_probing/AUC_Longisland.pdf}
    \includegraphics[trim={0 0 0 0},clip, width=0.22\textwidth]{figures/abla_probing/AUC_RSNA.pdf}
    \includegraphics[trim={0 0 0 0},clip, width=0.28\textwidth]{figures/abla_probing/AUC_CQ500.pdf}
    \\[0.2cm]
    \includegraphics[width=0.22\textwidth]{figures/abla_probing/AP_NYU.pdf} 
    \includegraphics[width=0.22\textwidth]{figures/abla_probing/AP_Longisland.pdf} 
    \includegraphics[width=0.22\textwidth]{figures/abla_probing/AP_RSNA.pdf}
    \includegraphics[width=0.28\textwidth]{figures/abla_probing/AP_CQ500.pdf}
    \caption{\textbf{Comparison of Different Downstream Training Methods.} This plot illustrates the downstream performance of models evaluated using fine-tuning and various probing methods across four datasets. In most cases, the DINO pre-trained model outperforms the MAE pre-trained model. All models were pre-trained on $100\%$ of the available pretraining data.}
    \label{fig:probing_comparison}
\end{figure*}


\begin{figure}
\centering
\makebox[\textwidth][l]{%
    \hspace{0.39\textwidth}\textbf{RSNA}
} \\[0.2cm]
\includegraphics[trim={0 0 0mm 0},clip,height=0.27\textwidth]{figures/abla_gemini/AUC_RSNA_Gemini.pdf}
\includegraphics[trim={0 0 5mm 0},clip,height=0.27\textwidth]{figures/abla_gemini/AP_RSNA_Gemini.pdf}

\makebox[\textwidth][l]{%
    \hspace{0.38\textwidth}\textbf{CQ500}
} \\[0.2cm]
\includegraphics[trim={0 0 10mm 0},clip,height=0.345\textwidth]{figures/abla_gemini/AUC_CQ500_Gemini.pdf}
\includegraphics[trim={0 0 5mm 0},clip,height=0.345\textwidth]{figures/abla_gemini/AP_CQ500_Gemini.pdf}

\caption{\textbf{Performance comparison of linear probing for Our Model vs. Google CT Foundation model} This plot compares our model performance vs. Google CT Foundation model\cite{yang2024advancing} and Merlin \cite{blankemeier2024merlinvisionlanguagefoundation} across all diseases on RSNA and CQ500. Since Google CT Foundation moudel requires uploading data to Google Cloud (not allowed on our private data) for requesting model embeddings with model weights inaccessible, only public dataset comparison is provided in this study. Similar to other evaluations, we observed that our model outperforms Google CT Foundation model across the board with the only exception on Midline Shift for Google CT Foundation model and EDH for Merlin.}
\label{fig:probing-comparison-gemini}
\end{figure}



\begin{figure}
    \centering
    \makebox[\textwidth][l]{%
        \hspace{0.35\textwidth}\textbf{NYU Langone}
    } \\[0.2cm]
    \includegraphics[trim={0 0 120mm 0},clip,height=0.255\textwidth]{figures/abla_probing_perpath/DINO_AUC_NYU_Langone.pdf}
    \includegraphics[trim={0 0 0 0},clip,height=0.255\textwidth]{figures/abla_probing_perpath/DINO_AP_NYU_Langone.pdf} \\
    \makebox[\textwidth][l]{
        \hspace{0.35\textwidth}\textbf{NYU Long Island}
    } \\[0.2cm]
    \includegraphics[trim={0 0 120mm 0},clip,height=0.255\textwidth]{figures/abla_probing_perpath/DINO_AUC_NYU_Long_Island.pdf}
    \includegraphics[trim={0 0 0 0},clip,height=0.255\textwidth]{figures/abla_probing_perpath/DINO_AP_NYU_Long_Island.pdf} 
    \makebox[\textwidth][l]{
        \hspace{0.4\textwidth}\textbf{RSNA}
    } \\[0.2cm]
    \includegraphics[trim={0 0 120mm 0},clip,height=0.24\textwidth]{figures/abla_probing_perpath/DINO_AUC_RSNA.pdf}
    \hspace{5mm}
    \includegraphics[trim={0 0 0 0},clip,height=0.24\textwidth]{figures/abla_probing_perpath/DINO_AP_RSNA.pdf} 
    \makebox[\textwidth][l]{
        \hspace{0.4\textwidth}\textbf{CQ500}
    } \\[0.2cm]
    \includegraphics[trim={0 0 120mm 0},clip,height=0.30\textwidth]{figures/abla_probing_perpath/DINO_AUC_CQ500.pdf} \hspace{5mm}
    \includegraphics[trim={0 0 0 0},clip,height=0.30\textwidth]{figures/abla_probing_perpath/DINO_AP_CQ500.pdf} 
    \caption{\textbf{Performance comparison of supervised finetuning methods per pathology on the foundation model trained with DINO.} This plot breaks down the average performance across all diseases shown in Supplementary \Cref{fig:probing_comparison}. The results show that fine-tuning the entire network achieves the best performance in most scenarios. However, linear probing closely approaches finetuning performance for many diseases especially on small or imbalanced dataset, underscoring the capability of our pre-trained models to generate representations that adapt effectively to diverse disease detection tasks.}
    \label{fig:probing-comparison-perpath-dino}
\end{figure}

\begin{figure}
    \centering
    \makebox[\textwidth][l]{%
        \hspace{0.35\textwidth}\textbf{NYU Langone}
    } \\[0.2cm]
    \includegraphics[trim={0 0 0 0},clip,height=0.24\textwidth, width=0.3\textwidth]{figures/abla_probing_perpath/AUC_NYU.pdf}
    \includegraphics[trim={0 0 0 0},clip,height=0.24\textwidth, width=0.45\textwidth]{figures/abla_probing_perpath/AP_NYU.pdf}\\
    \makebox[\textwidth][l]{
        \hspace{0.35\textwidth}\textbf{NYU Long Island}
    } \\[0.2cm]
    \includegraphics[trim={0 0 0 0},clip,height=0.24\textwidth, width=0.3\textwidth]{figures/abla_probing_perpath/AUC_Longisland.pdf}
    \includegraphics[trim={0 0 0 0},clip,height=0.24\textwidth, width=0.45\textwidth]{figures/abla_probing_perpath/AP_Longisland.pdf} 
    \makebox[\textwidth][l]{
        \hspace{0.4\textwidth}\textbf{RSNA}
    } \\[0.2cm]
    \includegraphics[trim={0 0 0 0},clip,height=0.24\textwidth, width=0.3\textwidth]{figures/abla_probing_perpath/AUC_RSNA.pdf}
    \includegraphics[height=0.24\textwidth, width=0.45\textwidth]{figures/abla_probing_perpath/AP_RSNA.pdf} 
    \makebox[\textwidth][l]{
        \hspace{0.4\textwidth}\textbf{CQ500}
    } \\[0.2cm]
    \includegraphics[trim={0 0 120mm 0},clip,height=0.24\textwidth]{figures/abla_probing_perpath/AUC_CQ500.pdf}
    \includegraphics[trim={0 0 0 0},clip,height=0.24\textwidth]{figures/abla_probing_perpath/AP_CQ500.pdf} 
    \caption{\textbf{Performance comparison of supervised finetuning methods per pathology on the foundation model trained with MAE.} The results reveal that attentive probing is significantly more effective than linear probing, consistent with findings from~\cite{Chen2024}. Furthermore, for many diseases, the performance of probing models approaches that of fine-tuning, demonstrating that our pre-trained models produce adaptable representations capable of detecting diverse diseases.}
    \label{fig:probing-comparison-perpath}
\end{figure}









\begin{figure}
    \centering
    \textbf{NYU Langone} \\
    \includegraphics[trim={0 0 0 0},clip,height=0.24\textwidth, width=0.38\textwidth]{figures/abla_perpath_perf/AUC_NYU.pdf}
    \includegraphics[height=0.24\textwidth, width=0.45\textwidth]{figures/abla_perpath_perf/AP_NYU.pdf} \\
    \textbf{NYU Long Island} \\
    \includegraphics[trim={0 0 0 0},clip,height=0.24\textwidth, width=0.38\textwidth]{figures/abla_perpath_perf/AUC_Longisland.pdf}
    \includegraphics[height=0.24\textwidth, width=0.45\textwidth]{figures/abla_perpath_perf/AP_Longisland.pdf} \\
    \textbf{RSNA} \\
    \includegraphics[trim={0 0 0 0},clip,height=0.24\textwidth, width=0.38\textwidth]{figures/abla_perpath_perf/AUC_RSNA.pdf}
    \includegraphics[height=0.24\textwidth, width=0.45\textwidth]{figures/abla_perpath_perf/AP_RSNA.pdf}\\
    \textbf{CQ500} \\
    \includegraphics[trim={0 0 0 0},clip,height=0.24\textwidth, width=0.38\textwidth]{figures/abla_perpath_perf/AUC_CQ500.pdf}
    \includegraphics[height=0.24\textwidth, width=0.45\textwidth]{figures/abla_perpath_perf/AP_CQ500.pdf}
    \caption{\textbf{Performance for Different Percentage of Pre-training Samples (Per-Pathology).} This plot illustrates label efficiency for individual pathologies using Tukey plots, alongside the average performance across all diseases shown in \Cref{fig:scaling_law}. The results indicate that the majority of pathologies show improved downstream performance as the amount of pretraining data increases.}
    \label{fig:boxplot_scaling}
\end{figure}


\newpage

\section{Time complexity increase with reduced patch size}
\label{apd:self_attention_rate}
Assume we have 3D image input of shape $H\times W\times D$, patch size $P$ and reducing factor $s$. By time complexity of self-attention $O(n^2 d)$ for sequence length $n$ and embedding dimension $d$, the new time complexity after reducing patch size can be derived as
\begin{align*}
    O(n^2d)&=O((\frac{H\times W\times D}{(\frac{P}{s})^3})^2d) \\
           &=O((\frac{H\times W\times D}{P^3})^2 s^6d)  \\
           &=O(s^6)O(n_{ori}^2d)
\end{align*}
where $n_{ori}=\frac{H\times W\times D}{P^3}$ is the original sequence length before reducing patch size.



















\newpage
\begin{figure}[ht]
    \centering
    \includegraphics[width=\textwidth]{images/tsne_embedding_visualization_per_pathology.png}
    \caption{The 2D projection with t-SNE of CT volume representation extracted from the foundation model. Interestingly, certain subgroups still exhibited slightly better AUCs. For instance, scans with slice thicknesses between 1–4 mm (represented by light green points in the upper panel of \Cref{fig:batch_effect}) align with a specialized protocol for CT angiography (CTA), which uses contrast dye to improve diagnosis on particular diseases.}
    \label{fig:batch_effect}
\end{figure}


\begin{figure*}[ht]
    \centering
    \begin{subfigure}[b]{0.33\textwidth}
        \centering
        \includegraphics[width=\textwidth]{images/AUROC_vs_Slice_thickness_binned.png}
        \caption{AUROC Performance}
    \end{subfigure}
    \hfill
    \begin{subfigure}[b]{0.33\textwidth}
        \centering
        \includegraphics[width=\textwidth]{images/AUPRC_vs_Slice_thickness_binned.png}
        \caption{AUPRC Performance}
    \end{subfigure}
    \hfill
    \begin{subfigure}[b]{0.33\textwidth}
        \centering
        \includegraphics[width=\textwidth]{images/Histogram_of_slice_thickness_distribution_across_scans.png}
        \caption{Histogram of slice thickness distribution}
    \end{subfigure}
    \caption{The downstream task performances on various ranges of slice thickness.}
    \label{fig:thickness-ablation}
\end{figure*}


\begin{figure*}[ht]
    \centering
    \begin{subfigure}[b]{\textwidth}
        \centering
        \includegraphics[width=\textwidth]{images/AUROC_vs_slice_thickness_for_each_disease_category.png}
        \caption{AUROC Performance}
    \end{subfigure}
    \hfill
    \begin{subfigure}[b]{\textwidth}
        \centering
        \includegraphics[width=\textwidth]{images/AUPRC_vs_slice_thickness_for_eachdisease_category.png}
        \caption{AUPRC Performance}
    \end{subfigure}
    \hfill
    \begin{subfigure}[b]{\textwidth}
        \centering
        \includegraphics[width=\textwidth]{images/Ratio_of_positive_labels_vs_slice_thickness_for_each_disease_category.png}
        \caption{Ratio of Positive Labels}
    \end{subfigure}
    \caption{Performance for Each Slice Thickness Bin (Per Pathology).}
    \label{fig:slice_thickness_per_pathology}
\end{figure*}


\begin{figure*}[ht]
    \centering
    \begin{subfigure}[b]{0.3\textwidth}
        \centering
        \includegraphics[width=\textwidth]{images/AUROC_by_Disease_and_Manufacturer.png}
        \caption{AUROC Performance}
    \end{subfigure}
    \hfill
    \begin{subfigure}[b]{0.3\textwidth}
        \centering
        \includegraphics[width=\textwidth]{images/AUPRC_by_Disease_and_Manufacturer.png}
        \caption{AUPRC Performance}
    \end{subfigure}
    \hfill
    \begin{subfigure}[b]{0.39\textwidth}
        \centering
        \includegraphics[width=\textwidth]{images/Positive_Label_Ratio_by_Disease_and_Manufacturer.png}
        \caption{Distribution of Scans from Each Manufacturer}
    \end{subfigure}
    \caption{Performance for Each Manufacturer (Per Pathology).}
    \label{fig:manufacturer_per_pathology}
\end{figure*}






%%%%%%%%%%%%%%%%%%%%%%%%%%%%%%%%%%%%%%%%%%%%%%%%%%%%%%%%%%%%%%%%%%%%%%%%%%%%%%%
%%%%%%%%%%%%%%%%%%%%%%%%%%%%%%%%%%%%%%%%%%%%%%%%%%%%%%%%%%%%%%%%%%%%%%%%%%%%%%%


\end{document}


% This document was modified from the file originally made available by
% Pat Langley and Andrea Danyluk for ICML-2K. This version was created
% by Iain Murray in 2018, and modified by Alexandre Bouchard in
% 2019 and 2021 and by Csaba Szepesvari, Gang Niu and Sivan Sabato in 2022.
% Modified again in 2023 and 2024 by Sivan Sabato and Jonathan Scarlett.
% Previous contributors include Dan Roy, Lise Getoor and Tobias
% Scheffer, which was slightly modified from the 2010 version by
% Thorsten Joachims & Johannes Fuernkranz, slightly modified from the
% 2009 version by Kiri Wagstaff and Sam Roweis's 2008 version, which is
% slightly modified from Prasad Tadepalli's 2007 version which is a
% lightly changed version of the previous year's version by Andrew
% Moore, which was in turn edited from those of Kristian Kersting and
% Codrina Lauth. Alex Smola contributed to the algorithmic style files.
