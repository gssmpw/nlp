\section{Introduction}
\label{Sec:Introduction}

Blockchain is a decentralized digital ledger technology that securely records and verifies transactions across multiple computers, ensuring data integrity and transparency~\cite{bitcoin}. Bitcoin (BTC), the most prominent token in the blockchain ecosystem, reached a remarkable market capitalization of \$1,800 billion in November 2024, demonstrating the impact and success of this technology~\cite{coinmarketcap}.


Building on blockchain technology, Decentralized Applications (DApps) have emerged as a key innovation~\cite{buterin2014next}. These applications run on smart contracts deployed across various blockchain networks, such as Ethereum, Polygon, and Binance Smart Chain (BSC). The expansion of DApps reflects a shift toward decentralized application development, providing diverse services such as Decentralized Finance (DeFi) and Gamefi. This evolving ecosystem is supported by a wide infrastructure, including cryptocurrency wallets and blockchain explorers, which contribute to the dynamism of the blockchain landscape~\cite{tang2022blockchain}.

Smart contract events and transaction logs are essential for tracking actions within blockchain applications, supporting functionalities in cross-chain bridges, wallets, and NFT marketplaces. Falsifying these logs can undermine security, leading to asset theft, fraud, and interoperability issues that threaten user trust. While past research has examined vulnerabilities in specific domains, such as cross-chain bridges~\cite{Xscope,liao2024smartaxe,wang2024xguard}, NFT ownership~\cite{sleepminting}, code inconsistencies~\cite{doccon}, and token behaviors~\cite{TokenScope}, a comprehensive study on event forgery across multiple applications remains absent. This paper addresses this gap by introducing \emph{Phantom Events}, a new class of vulnerabilities that reveal the widespread risks of \emph{log forgery} on the blockchain. Unlike previous studies on isolated scenarios, our work systematically summarizes event misuse across decentralized applications and explores distinct detection methods to identify phantom events. In prior studies, detection tools such as XScope~\cite{Xscope} and XGuard~\cite{wang2024xguard} operate at the source code level, limiting detection when only bytecode is available. Conversely, SmartAxe~\cite{liao2024smartaxe} and TokenScope~\cite{TokenScope} analyze bytecode but their bytecode-only approach misses event parameter details. Guidi et al.~\cite{sleepminting}, which focuses solely on transaction logs, experiences high false-positive rates due to limited contextual analysis. Our tool combines bytecode, source code (when available), and transaction data, enabling comprehensive detection across multiple vulnerabilities while reducing the false positives that affect single-layer analysis methods.

Phantom events are subtle manipulations within blockchain systems that turn normal contract events
into hidden channels for unauthorized actions. These events compromise trust in blockchain
transactions, enabling unauthorized activities that bypass typical security checks. They either
mimic or slightly alter smart contract events to mislead event listeners, user interfaces, and
blockchain analysis tools. This exploitation damages data integrity and can lead to substantial
financial and reputational losses.
For example, on 9 February 2024, attackers caused a 1.04 million USD loss by forging transfer
events~\cite{zero_transfer}.

Our research focuses on understanding and categorizing the various types of phantom events.
We conducted a comprehensive survey of recent blockchain security reports and articles to gather
real-world attacks that exploit vulnerabilities related to phantom events.
Furthermore, we audited multiple active smart contracts and developed a security model to process
blockchain events (\cref{sec:threat}), which serves as the foundation for analyzing and
categorizing the collected attack scenarios (\cref{sec:vector}).
Furthermore, we developed a tool named \tool, designed to detect vulnerabilities related to phantom
events (\cref{sec:detection}).
Designing \tool presented several challenges: (1) many contracts exist only as bytecode, lacking
event semantics, which complicates analysis; (2) phantom events mimic the format of legitimate
events, making them hard to discern; and (3) understanding the business logic of contracts is
necessary to determine the presence of irregularities.

Using our security model and detection tool, we identified numerous vulnerabilities and potential
risks in various blockchain applications (details reported in \cref{sec:evaluation}).
Our results reveal that phantom event vulnerabilities are widespread and affect a variety of
applications in the blockchain ecosystem. Through our auditing efforts, we discovered previously
unreported instances of inconsistent logging vulnerabilities, where on-chain issues resulted in
discrepancies with off-chain records.
Furthermore, our tool identified historical event issues on chain that had gone undetected,
underscoring the effectiveness of the tool in uncovering past incidents.
In contract-level analysis, our tool outperformed existing solutions, demonstrating superior
accuracy in detecting phantom event vulnerabilities.
Furthermore, we identified security issues in multiple real-world applications, including
cryptocurrency wallets, blockchain explorers, and cross-chain bridges, and reported these findings
to project teams, claiming bounties for responsible disclosure.
Finally, we propose mitigation strategies to address these vulnerabilities in~\cref{sec:mitigation}.

In short, we make the following contributions in this paper:
\begin{itemize}
\item \textbf{Novel Attack Taxonomy.}
We are among the first to systematically analyze and categorize attacks
related to the forgery of blockchain transaction logs. We propose a security model that captures
five types of attack. We also identified previously unreported inconsistent logging
vulnerabilities, where on-chain issues result in discrepancies with off-chain records.
\item \textbf{Efficient Detection.}
We developed a detection mechanism that surpasses existing tools in identifying Phantom Events,
successfully uncovering previously unreported attack transactions.
\item \textbf{Practical Implications.}
We identified real-world issues across various blockchain applications, including blockchain
explorers, cryptocurrency wallets, GameFi, DeFi, and NFT marketplaces.
Notably, we found a vulnerability in a cross-chain relayer, which is also used by other projects,
including one with a market capitalization of over \$250 million.
To date, we have reported six cryptocurrency wallet issues, one blockchain explorer issue, and one
cross-chain bridge issue to project teams and bug bounty platforms, with five confirmed and one
resolved, earning a total of \$600 bounty for responsible disclosure.
\end{itemize}
