\section{Background}\label{sec:Background}

\subsection{Smart contract event and transaction log in EVM-based Blockchains}
Smart contracts, written in high-level languages like Solidity~\cite{Solidity} and Vyper~\cite{Vyper}, are compiled into bytecode for execution on the Ethereum Virtual Machine (EVM)~\cite{EVM}. EVM-based blockchains support a wide range of functionalities, from simple transfers to complex DApps~\cite{DApp}. In these blockchains, \textit{events} and \textit{transaction logs} enable communication between smart contracts and external systems. When a smart contract emits an event, it is recorded as a transaction log, involving key components as follows.

\paragraph{Event}
An event, such as \texttt{Deposit}, is defined as a tuple $E_{\texttt{Deposit}}(P_1, P_2, \ldots, P_n)$, where the parameters $P_i$ may be \textit{indexed} (stored in \textit{topics}) or \textit{non-indexed} (stored in \textit{data}). For example, in the event
\texttt{Deposit(address indexed account, uint256 amount, uint256 timestamp);},
the \texttt{account} is an \textit{indexed} parameter, which means it will be included in the event's topics, enabling more efficient searching and filtering in the event logs. The \texttt{amount} and \texttt{timestamp} parameters are not indexed, but they still provide valuable information about the transaction.

%\liuye{Pls use detailed event declaration such as event Deposit(type indexed parameter, ...) to showcase. Otherwise, it is difficult for readers to understand what a event looks like, and what event topics and indexed parameters are.}

\paragraph{Listener}
An external system (e.g., a DApp or wallet) that monitors specific events by subscribing to logs through RPC methods.

\paragraph{Transaction}
A transaction $TX_{\textit{hash}}$ represents an action on the blockchain, such as invoking a function $f_{\texttt{name}}$ within a smart contract. It includes the sender’s and recipient’s addresses, the transfer amount, and input data defining the function call and its parameters.


\paragraph{Sender}
The sender \( TX_{\textit{sender}} \) is the user address that creates the transaction.

\paragraph{Emitter}
The emitter $S_{\texttt{address}}$ is the smart contract triggering the event upon meeting certain conditions.

\paragraph{Transaction Log}
A log \( L_{hash} = (Topics_0^i, \allowbreak{} Data^i,\allowbreak{} S_{\texttt{address}}^i) \) is
generated
when an event is emitted during smart contract execution. \( Topics_0^i \) represents the event
signature for the \(i\)-th log, \( Data^i \) contains non-indexed parameters, and \(
S_{\texttt{address}}^i \) is the contract emitting the event. These logs are stored on the
blockchain and referenced for event monitoring.

%\liuye{Maybe we can have a unified definition for event log, e.g., event = (tx\_sender, emitter, event\_name, log\_data) and then detail each part. For me, I don't think we need to discuss much more about transaction log. We can briefly say like: a transaction may be associated with multiple event logs emitted by the called contracts. }

\subsection{Token}
In EVM-based blockchains, tokens are implemented as smart contracts adhering to standards like \textit{ERC-20} and \textit{ERC-721}, which define interfaces and events to ensure compatibility across DApps. For instance, ERC-20 specifies events like \texttt{Transfer} and \texttt{Approval} for tracking token transfers and approvals, while \textit{ERC-721} defines similar events but is adapted for NFTs, where each token is unique. These standardized events enable external systems, such as wallets and exchanges, to monitor token activities and update user balances or ownership records in real time, ensuring accurate and efficient tracking.

\begin{figure}[t]
    \centering
    \includegraphics[width=\linewidth]{Sections/Picture/bridge.pdf}
    \caption{The basic architecture of a cross-chain bridge.}
    \label{fig:Bridge}
\end{figure}

\subsection{Cross-Chain Brdiges}
Some DApps rely on off-chain modules to monitor transactions and events emitted by smart contracts in order to stay updated on on-chain activities and state changes in real time. For example, cross-chain bridges use these events to trigger off-chain processes, update user interfaces, or others.

In the case of a cross-chain bridge, as illustrated in \cref{fig:Bridge}, the smart contract on \emph{source chain} acts as the emitter $S_{source}$, emitting specific events such as \texttt{Deposit} when users deposit assets. And the smart contract on the \emph{destination chain} as an actor, being invoked by an off-chain relayer to process the deposited assets and complete the transfer operation.

When a user deposits tokens in a \emph{bridge contract} in the source chain, the deposit event $E_{\texttt{Deposit}}$ is emitted and recorded as a transaction log $L_{deposit_{tx}}$. An \emph{off-chain relayer}, functioning as a listener, monitors these logs to detect the \texttt{Deposit} event, process the event information, and coordinate with the smart contract on the destination chain to complete the asset transfer.