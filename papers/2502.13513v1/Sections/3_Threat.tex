\section{Threat Model and Motivating Examples}\label{sec:threat}

This section presents a threat model for the systematic analysis of event-related attacks and illustrates the impact of Phantom Events through motivating examples.
\subsection{Trust Boundaries in Event Interactions}

In the complex environment of blockchain interactions, establishing clear trust boundaries is essential for comprehensively understanding the security risks and vulnerabilities associated with transaction and event workflows. The interact model illustrated in \cref{fig:system model} defines three primary trust boundaries:
\begin{figure}[t]
    \centering
    \includegraphics[width=.9\linewidth]{Sections/Picture/securitymodel.pdf}
    \caption{Trust Boundaries in Blockchain Interactions}
    \label{fig:system model}
\end{figure}

\paragraph{Trust Boundary-Blockchain (\(\bm{\mathcal{M}}_B\))}
The core trust boundary is the blockchain, which serves as the backbone of the system where transactions and smart contract events are stored.
Within this boundary lies the Transaction Storage (\(\bm{\mathcal{M}}_{TS}\)), which can be divided into \emph{phantom events} and \emph{authentic events}.
Phantom events are artificially created or manipulated events that may trigger unintended actions, while authentic events are the expected output of blockchain transactions.
The Smart Contract area (\(\bm{\mathcal{M}}_{SC}\)) can be divided into \emph{authentic contracts} and \emph{forged contracts}, indicating the potential for contracts to operate as intended or act maliciously.
Other Blockchain Modules (\(\bm{\mathcal{M}}_{OM}\)) include consensus mechanisms, node communication protocols, and other components that support blockchain functionality.

\paragraph{Trust Boundary-DApp (\(\bm{\mathcal{M}}_{D}\))}
The second boundary encompasses DApps that act as interfaces between users and the blockchain.
DApps listen for events from the blockchain and respond accordingly, often triggering transactions or updates in response to these events.


\paragraph{Trust Boundary-Others (\(\bm{\mathcal{M}}_{O}\))}
The third boundary includes the broader ecosystem, such as off-chain services, external APIs, and wallets. These components also rely on the integrity of blockchain events for proper functionality.

\subsection{Threat Model}
In our threat model, attackers seek to compromise DApps and the broader ecosystem infrastructure by emitting forged events, and may also exploit these attacks to mislead users through social engineering tactics. While regular users emit legitimate events by invoking authentic contracts, attackers can emit forged events in several ways: by calling a forged contract, by calling an authentic contract, or by invoking a forged contract that subsequently calls an authentic contract to emit a forged event. The attacker's capabilities are confined to making direct contract calls or interacting with DApps.

\subsection{Motivating Examples}\label{sec:motivation}
Numerous real-world blockchain attacks have occurred due to the forgery of transaction logs.
For example, the Qubit Bridge and pNetwork Bridge attacks~\cite{Qubit,pNetwork} caused \$80 million and \$4.3 million losses, respectively, due to the exploitation of Phantom Event vulnerabilities.

\begin{figure}[h]
    \centering
    \includegraphics[width=1\columnwidth]{Sections/Picture/sequencediagram.pdf}
    \caption{Attack sequence of pNetwork Hack.}
    \label{fig:sqeuence}
\end{figure}

In the pNetwork hack, there is a flaw in the event handling mechanism, where if both a malicious
contract and a legitimate contract are invoked within the same transaction and both emit events,
the mechanism mistakenly treats all emitted events as legitimate contract events. So as shown in
the attack sequence (\cref{fig:sqeuence}), the attacker first deployed a malicious contract
$S_{\texttt{malicious}}$ on the source chain which invoking the legitimate contract
$S_{\texttt{legitimate}}$ through a function call. After the legitimate contract
$S_{\texttt{legitimate}}$ executed its function $f_{\texttt{deposit}}$ and emitted the authentic
event $E_{\texttt{Deposit}}$, the malicious contract $S_{\texttt{malicious}}$ simultaneously
emitted a counterfeit event $E_{\texttt{Deposit}}$ with an untruthful amount.
Since both the authentic event and the counterfeit event were recorded under the same transaction
$TX_{\textit{deposit}}$, the relayer processed the counterfeit event as if it were legitimate.
This allowed the attacker to transfer unauthorized funds to the destination chain by exploiting the
inability of the system to differentiate between the two events.

In the Qubit Bridge attack, the attacker exploits a flaw in the \texttt{deposit} function of the
victim contract, as illustrated in \cref{code:1}.
The relayer transfers assets on the destination chain by retrieving the token address and amount
from the event emitted on the source chain. However, the attacker exploited this by using the
\texttt{deposit} function to emit an event that should only be emitted by the \texttt{depositETH}
function, allowing them to trigger a transfer of ETH on the destination chain without actually
depositing any ETH on the source chain, thus bypassing the relayer's checks and profiting from the
attack.


\begin{comment}
Another example is the transfer spoofing attack on 9 February 2024, where an attacker successfully executed a \textit{transfer event spoofing} attack, resulting in a loss of \$1.04 million USD. This attack exploited vulnerabilities in wallets and blockchain explorers that parse and display transfer events. As shown in~\Cref{tab:transfer_spoof_attack_transaction}, after a user sent funds to a legitimate address (0x734\ldots{}a79F7), the attacker forged a transfer event, making it appear as though the funds were sent to a similar address controlled by the attacker. Due to the way certain wallets and explorers processed these events, the fake transfer was displayed as legitimate, leading the user to mistakenly send additional funds to the attacker's address, resulting in significant financial losses~\cite{zero_transfer}.
\end{comment}
\paragraph{Idea of \tool}
Motivated by these real-world examples, we apply customized detection methods for various types of
vulnerabilities.
The phantom events may arise from issues in smart contract logic or from flaws in off-chain
processors.
To address this, we designed a smart contract-level approach, which identifies vulnerabilities that
could lead to phantom events.
Additionally, we developed an on-chain approach for transaction data, inspecting transactions for
patterns or anomalies that may indicate attacks.
These approaches complement each other and enable a comprehensive detection framework that
addresses both contract-based and transaction-based origins of phantom events.