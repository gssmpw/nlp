\appendices

%\section{Attack Sequence of pNetwork Hack}

%\Cref{fig:sqeuence} shown the attack sequence of pNetwork attack.
%\label{pnetwork_sequence}
\section{Real-World Attack Examples}

\subsection{Examples of the Five Attack Vectors in Our Taxonomy}
\label{appendix:examples}

\paragraph{Event Counterfeiting}
Two well-known instances of \emph{Event Counterfeiting} are the attacks on Qubit Bridge~\cite{Qubit} and Meter Bridge~\cite{Meter}, which resulted in losses of \$80 million and \$4.3 million, respectively.

\paragraph{Inconsistent Logging}
Many of these issues were discovered from the 25 most active smart contracts on the BSC chain on 1 April 2024. Specifically, the contracts for TroyEmpire~\cite{TroyEmpire}, SecondLive~\cite{SecondLive}, WalletLocking~\cite{WalletLocking}, and TTGameEvents~\cite{TTGame} were found to have these issues. For example, TTGame project generates an address for users' registration and transfers a certain amount of BNB to the address as gas fees for future automatic calls. By identifying authorized user addresses from the historical transactions, we can find the remaining unauthorized transactions that emit phantom events. Although user permissions for the other three projects were unclear, we can simulate to call their functions and arbitrarily generate withdrawal events, confirming the existence of these issues.

\paragraph{Contract Imitation}
For \emph{Contract Imitation} attack, the most notable instance is the attack on pNetwork, which resulted in a loss of \$12.7 million. This has already been briefly introduced in \cref{sec:motivation}. Similarly, THORChain experienced a loss of \$8 million due to an error handler issue~\cite{thorchain}. As for mimicry contract attack, a representative case is the ``sleepminting'' attack, which gained attention for generating an NFT valued at \$69 million and listing it on a platform. This incident has been widely studied in the academic community~\cite{nft_sleepminting,clone_nft,sleepminting}. According to recent information from the Forta platform~\cite{sleepminting_forta}, there were 8,130 sleepminting alerts generated in just 7 days, from December 3 to December 9, 2023.

\paragraph{Transfer Event Spoofing}
Our analysis of on-chain data revealed numerous instances of \emph{Transfer Event Spoofing}, where ERC-20 token and NFT events had altered sender addresses. These included forged addresses resembling celebrities, well-known exchanges, and other eye-catching entities. Several of these cases have led to significant financial losses~\cite{layerzero_scam,TetherClaims,ethaddress,zero_transfer,wallet_visual}.

\paragraph{Event Handling Error}
A well-known example of \emph{Event Handling Error} involves the misinformation case experienced by Rarible and OpenSea, caused by sleepminting. Blockchain explorers like Etherscan and Bscscan, as well as various wallets, often treat forged transfer events as legitimate transactions. Another instance involves blockwell.ai, where wallets mistakenly identified a phantom event from a spoofed token as a valid ERC-20 transfer~\cite{blockwell.ai}. Similar vulnerabilities have been observed in the ERC-721 and ERC-1155 tokens.

In addition, OpenSea, the largest NFT marketplace, has faced similar issues. The attackers manipulated the metadata of the NFTs displayed by OpenSea, embedding malicious payloads that triggered unwanted wallet behaviors, ultimately allowing attackers to profit. Similar attacks have been found in ERC-20 projects, highlighting the widespread difficulty in securely handling event data across the blockchain ecosystem~\cite{rektosaurus,OpenSea}.

\subsection{Framework of Inconsistent Logging and Contract Imitation}

\begin{figure}
    \centering
    \includegraphics[width=1\linewidth]{Sections//Picture/inconsistency.png}
    \caption{The overview of Inconsistent logging apps.}
    \label{fig:Inconsistent_logging}
\end{figure}

~\Cref{fig:Inconsistent_logging} provides an overview of \emph{Inconsistent Logging} in applications. In this framework, a user operates through an application, which interacts with a database to insert logs and with a smart contract on the blockchain to emit events. However, inconsistencies can arise between the database logs and the blockchain transaction logs due to discrepancies in logging practices. Attackers can exploit this by directly calling the record function, potentially creating mismatches between the application's internal log records and the blockchain's transaction logs.

\begin{figure}[!htbp]
    \centering
    \begin{subfigure}[b]{0.5\textwidth}
        \centering
        \includegraphics[width=\textwidth]{Sections/Picture/attack1-1.png}
        \caption{Blended Event Attack}
        \label{fig:attack3-1}
    \end{subfigure}
    \quad
    \begin{subfigure}[b]{0.5\textwidth}
        \centering
        \includegraphics[width=\textwidth]{Sections/Picture/attack1-2-0.png}
        \caption{Mimicry Contract Attack}
        \label{fig:attack3-2}
    \end{subfigure}
    \caption{Two type of Contract imitation}
    \label{fig:attack1}
\end{figure}

~\Cref{fig:attack1} illustrates two types of contract imitation attacks. In the \emph{Blended Event Attack}, a malicious contract interacts with an authentic contract, and logs from both contracts are recorded in the same transaction, which may mislead validators. In the \emph{Mimicry Contract Attack}, the attacker deploys a contract that imitates a real contract, generating deceptive logs that appear to originate from the authentic contract, creating further confusion for validators.

\subsection{Case of Transfer Event Spoofing}

\begin{figure}
    \centering
    \includegraphics[width=1\linewidth]{Sections//Picture/address.png}
    \caption{Airdrop event spoofing.}
    \label{fig:airdrop}
\end{figure}

~\Cref{fig:airdrop} illustrates an example of a \emph{Transfer Event Spoofing} attack, where the attacker forges the event sender, causing blockchain explorers and wallets to incorrectly display it as a successful transfer. This misrepresentation can deceive users and is used as a social engineering tactic to exploit trust in these interfaces.


% \section{Real case of transfer event spoofing attack transactions.}
\begin{table}[]
\caption{Real case of transfer event spoofing attack transactions.}
\label{tab:transfer_spoof_attack_transaction}
\begin{tabular}{lllll}
\toprule
  & Sender & Transfer From                & Transfer To                  & Token      \\
\midrule
1 & Victim           & Victim & 0x{\color[HTML]{FE0000}734}659...50C{\color[HTML]{FE0000}a79F7} & USDT \\
\midrule
2 & Attacker              & Victim & 0x{\color[HTML]{FE0000}734}35A..2Bc{\color[HTML]{FE0000}a79F7}  & FAKE USDT     \\
\midrule
3 & Victim           & Victim & 0x{\color[HTML]{FE0000}734}35A...2Bc{\color[HTML]{FE0000}a79F7} & USDT \\
\bottomrule
\end{tabular}
\end{table}
The \emph{Transfer Spoofing} attack on 9 February 2024, where an attacker successfully executed a \emph{Transfer Event Spoofing} attack, resulting in a loss of \$1.04 million USD. This attack exploited vulnerabilities in wallets and blockchain explorers that parse and display transfer events. As shown in~\Cref{tab:transfer_spoof_attack_transaction}, after a user sent funds to a legitimate address (0x734\ldots{}a79F7), the attacker forged a transfer event, making it appear as though the funds were sent to a similar address controlled by the attacker. Due to the way certain wallets and explorers processed these events, the fake transfer was displayed as legitimate, leading the user to mistakenly send additional funds to the attacker's address, resulting in significant financial losses~\cite{zero_transfer}.