\section{Related Work}
\subsection{Smart Contract Security Analysis}
In recent years, a large number of techniques have been proposed to analyze the security of smart contracts.

\paragraph{Static Analysis}
Tikhomirov et al.~\cite{tikhomirov2018smartcheck} designed SmartCheck, a system translating
Solidity source code to XML and detecting bugs via xPath patterns. Grech et
al.~\cite{grech2022elipmoc} proposed a static analysis framework termed Gigahorse that translates
the stack-based bytecode to the register-based intermediate representation. A similar tool named
Slither~\cite{Slither} was proposed by Fesit that targets solidity source code. Lu et
al.~\cite{lu2021neucheck} design a smart contracts security analysis tool named NeuCheck which is
based on syntax tree parsing. Ghaleb et al.~\cite{ghaleb2022etainter} proposed eTainter, a static analysis tool that detects gas-related vulnerabilities in smart contracts by applying taint tracking to bytecode.

\paragraph{Symbolic Execution}
Luu et al.~\cite{luu2016making} proposed Oyente, a pioneering symbolic execution tool for Ethereum smart contracts to detect bugs. Lin et al.~\cite{lin2022solsee} presented SolSEE, the first source-level symbolic execution engine for Solidity smart contracts. Ma et al.~\cite{ma2021pluto} introduced Pluto for detecting security bugs by reconstructing inter-contract CFGs. Pasqua et al.~\cite{pasqua2023enhancing} proposed a method based on the symbolic execution of EVM operands for precise CFG construction and improved vulnerability detection. Ruaro et al.~\cite{ruaro2024not} implement CRUSH, which leverages symbolic execution and program slicing to detect storage collisions among such contract groups. Gritti et al.~\cite{gritti2023confusum} developed JACKAL, which performs symbolic execution based on control flow graphs (CFG) and function call graphs (FCG) to detect confused contract vulnerability. In addition, industry solutions such as Mytrhil~\cite{mythril} and Manticore~\cite{mossberg2019manticore} have become standard tools for smart contract audits.

Our approach involves developing a detection framework that combines multiple techniques, including bytecode-level analysis, source code analysis, transaction-level monitoring, and symbolic execution, but the issues addressed in this paper cannot be captured by any existing vulnerability patterns.

\subsection{Smart Contract Events}
Generally, logging messages enhances program comprehension and reduces maintenance costs, but
research on Solidity event logging and security is still limited. Li et
al.~\cite{li2023understanding} conducted the first empirical study on Solidity event logging and
developing a tool to identify the event that causes unnecessary gas usage. Zhang et
al.~\cite{Xscope} designed a tool named Xscope which finds
security violation events in cross-chain bridges. Cernera et al.~\cite{cernera2023token} tracked
the tokens created by internal transactions by scanning the logs looking for Transfer events.
Additionally, Guidi et al.~\cite{sleepminting} analyzed the phenomenon of sleepminting and explored the use of Forta for
tracking and alerting about suspicious events.
Zhu et al.~\cite{doccon} proposed a technique called DocCon, which detects inconsistencies between
Solidity code and its corresponding documentations. These inconsistencies include documented event
emissions which do not appear in code, or vice versa.
TokenScope \cite{TokenScope} contributed to the detection of inconsistencies in token behaviors by monitoring ERC-20 events.
Despite these advancements, most research has only considered issues related to events in specific
scenarios and has not fully addressed the generality of attacks caused by phantom events.

\label{sec:related}