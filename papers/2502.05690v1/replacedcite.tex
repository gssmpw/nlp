\section{Related Work}
\label{sec:literature_review}

This section reviews literature on uncertainty characterization in mineral supply chains and sequential decision-making frameworks, with emphasis on how existing approaches address geological uncertainty and information dynamics in resource development.

\subsection{Uncertainty in Mineral Supply Chains}

Critical mineral supply chains face three primary sources of uncertainty that distinguish them from traditional industrial supply chains ____. The first and most distinctive is geological uncertainty, which fundamentally shapes the decision-making landscape in ways unique to mineral supply chains. This uncertainty stems from the inherent difficulty in accurately assessing subsurface mineral deposits____, and unlike manufactured goods where production capacity can be estimated with relative certainty, mineral resources remain partially unknown until significant investment in exploration and extraction has occurred.

Geological uncertainty gives rise to several interconnected forms of uncertainty that compound the complexity of resource assessment and development planning ____. Resource quantity uncertainty affects our understanding of recoverable mineral volumes, while quality uncertainty influences the economic viability of extraction through variations in grade and composition. Spatial uncertainty impacts the distribution and accessibility of resources, affecting development costs and extraction strategies. Recovery uncertainty further complicates planning by introducing variability in extraction efficiency and processing yields.

Economic uncertainty represents the second major challenge, encompassing market price volatility, development costs, and technological change____. The economic viability of lithium projects is particularly sensitive to these uncertainties because there are high upfront costs of mine development and long lead times between investment and production. The third source, geopolitical uncertainty, includes trade policies, environmental regulations, and international relations____. While these latter uncertainties are common to many supply chains, their interaction with geological uncertainty creates unique challenges in the critical minerals sector.

In this work, we focus exclusively on geological uncertainty, as it represents the most distinctive characteristic of mineral supply chains and, crucially, the only uncertainty that can be systematically reduced through exploration and information gathering. Unlike economic and geopolitical uncertainties, which are largely external to the decision-making process, geological uncertainty can be actively managed through sequential exploration and development decisions. An earlier work toward this direction is explored in____, implementing a rolling horizon optimization with Bayesian updating. While it is more computationally simpler, it may not fully capture the value of information gathering actions since it doesn't explicitly model how future beliefs will change. In contrast, we propose a POMDP-based approach explicitly model both the state uncertainty and how actions affect future observations and beliefs. The solution gives a policy that maps beliefs to actions, accounting for both immediate rewards and information gathering value, allowing us to clearly demonstrate the value of sequential decision-making in mineral supply chains and gaining deeper insights for strategic policy making.


\subsection{Sequential Decision-Making for Mineral Supply Chains}

The limitations of traditional optimization approaches have led to growing interest in sequential decision-making frameworks, particularly Partially Observable Markov Decision Processes (POMDPs). Recent work has demonstrated the effectiveness of POMDPs in subsurface exploration and planning____. While exact solutions to POMDPs are computationally challenging for real-world problems____, approximate methods have shown promising results. These include point-based approaches focusing on reachable belief states, Monte Carlo tree search variants for continuous state spaces____, and online planning methods that interleave planning and execution____. %A comprehensive review of POMDP is available in ____.

\subsection{Research Gap and Our Contribution}
Despite extensive research on mineral exploration and decision-making under uncertainty, significant gaps remain in approaches that explicitly model and leverage the sequential nature of geological uncertainty reduction in strategic supply chain planning. Current methods typically treat geological uncertainty as static, failing to capture both the value of exploration activities and the adaptive nature of mineral development decisions. While existing POMDP applications often focus on operational exploration decisions, our study addresses strategic supply chain sourcing decisions ____, demonstrating how technical uncertainties, particularly geological ones, significantly impact strategic planning. By applying POMDPs to this context, we advance both the theoretical understanding of supply chain optimization under uncertainty and the development of practical, robust strategies for critical mineral resource management.