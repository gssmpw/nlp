\section{Conclusions}

This paper introduces three novel BoN sampling methods: $\mathrm{SRBoN}_{\mathrm{KL}}$, $\mathrm{SRBoN}_{\mathrm{WD}}$, and $\mathrm{RBoN}_{\mathrm{L}}$. To rigorously evaluate the effectiveness of these proposed methods, we conducted extensive experiments using two datasets: AlpacaFarm and Anthropic's hh-rlhf.

The $\mathrm{SRBoN}_{\mathrm{KL}}$ and $\mathrm{SRBoN}_{\mathrm{WD}}$ methods extend the previous $\mathrm{RBoN}_{\mathrm{KL}}$ and $\mathrm{RBoN}_{\mathrm{WD}}$ methods, respectively. In particular, $\mathrm{SRBoN}_{\mathrm{KL}}$ and $\mathrm{SRBoN}_{\mathrm{WD}}$ produce a stochastic optimal policy that differs from their deterministic counterparts. The theoretical guarantees of their robustness increase the reliability of the methods in different scenarios.

The $\mathrm{RBoN}_{\mathrm{L}}$ method is a contribution to the field of RBoN sampling, providing a simple yet effective approach. Despite its simplicity, our experiments show that $\mathrm{RBoN}_{\mathrm{L}}$ achieves performance comparable to the more complex $\mathrm{RBoN}_{\mathrm{WD}}$. This finding highlights the potential of $\mathrm{RBoN}_{\mathrm{L}}$ as a computationally efficient alternative to more complicated methods, making it particularly attractive for applications with limited resources or stringent performance requirements.


% In conclusion, this paper introduces three innovative BoN sampling methods that contribute significantly to the field. The experimental results and theoretical guarantees underscore the effectiveness and reliability of these methods. Our work lays the foundation for further research and applications of robust BoN sampling techniques in a wide range of domains.
In conclusion, this paper presents three innovative BoN sampling methods that significantly contribute to the field. The experimental results and theoretical guarantees underscore the effectiveness and reliability of these methods. Our work lays the foundation for further research and applications of robust BoN sampling techniques in a wide range of domains.
% \paragraph{Future Work.}

% \paragraph{Limitations.}
% The limitation of SRBoN is that it is not known before trying the proposed method whether the range of reward perturbations falls within the difference between the proxy reward and the gold reward.

\section{Limitations}
While our proposed method demonstrates promising results, there are several limitations to note. 
The proposed method requires no fine-tuning of the LLMs but inevitably increases computational overhead at inference. In contrast, fine-tuning approaches incur a one-time cost during training while eliminating overhead at inference. Another concern is that the proposed method considers a max-min problem, so if, for example, the correlation between the proxy reward and the gold reward is strong, performance is reduced due to conservative output selection.

Our study lacks an analysis of whether the reward perturbations satisfy the conditions outlined in Theorems~\ref{theory:kl-minmax} and \ref{theory:wd}. Evaluating the error of the reward and utility function in experiments remains an area for future work. Additionally, the selection of the parameter $\beta$ requires a validation set in the current setting, and developing an automated method to determine $\beta$ is a promising direction for further research.

% RBoN Requires utility function.

% Does not consider process reward model.

% Experiments are only on English dataset.

% proposed method can be extended f-divergence.

Furthermore, our approach relies on a specific utility function, which is a prerequisite for applying the proposed method, and the method does not account for process reward models, which may limit its applicability in some scenarios. It is also worth noting that the experiments conducted in this study were limited to three 
English datasets, leaving open the question of its generalizability to other languages or domains.

In addition, the proposed method is based on a probabilistic framework, which, while effective for uncertainty, may not align with real-world applications where deterministic versions (RBoN) are often preferred for their predictability and safety. Based on the analysis in this paper,  the analysis of the deterministic RBoN is a possible direction for future work.

Finally, while the current formulation is specific, the proposed method has the potential to be extended to other divergence measures, such as $f$-divergences, offering an exciting avenue for future investigation.