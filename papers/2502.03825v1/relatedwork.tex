\section{Related Works}
\paragraph{Brain Tumor Segmentation}  
Brain tumor segmentation has been a critical task in medical image analysis, enabling precise delineation of tumor regions for diagnosis and treatment planning \citep{wadhwa2019review, gordillo2013state}. Traditional methods relied on handcrafted features \citep{mecheter2022deep, khan2020cascading, hasan2019combining} and classical machine learning models \citep{soomro2022image, amin2019brain, bakas2018identifying}, but deep learning approaches, particularly convolutional neural networks (CNNs) \citep{li2021survey}, have significantly advanced segmentation performance \citep{havaei2017brain, pereira2016brain}. U-Net \citep{ronneberger2015u} and its variants \citep{azad2024medical, siddique2021u} have become the backbone of many segmentation pipelines due to their encoder-decoder architecture and skip connections, which preserve spatial information. More recent methods leverage transformer-based architectures \citep{ghazouani2024efficient, wang2023vision, jiang2022swinbts, huang2022transformer} and hybrid CNN-Transformer models \citep{liu2024transsea, kang2024multimodal, chen2022csu, jia2021bitr} to enhance feature representation and improve segmentation accuracy. Despite these advancements, the robustness of segmentation models remains a concern, especially when trained on datasets with varying levels of synthetic content.

\paragraph{GAN-based MRI Synthesis}  
The generation of synthetic MRI images has gained significant attention due to its potential to augment datasets, address data scarcity, and enable cross-modality learning \citep{tiwari2025review, choi2025beyond, pani2024synthetic, koetzier2024generating, hamghalam2024medical, ji2022synthetic, han2018gan, blystad2012synthetic}. Generative adversarial networks (GANs) \citep{goodfellow2014generative, goodfellow2020generative} and variational autoencoders (VAEs) \citep{kingma2013auto, pinheiro2021variational} have been widely explored for MRI synthesis \citep{tavse2022systematic, laptev2021generative}. Conditional GANs \citep{mirza2014conditional} and cycle-consistent GANs \citep{zhu2017unpaired} have been applied for modality translation, such as generating MRI from CT scans. More recent works incorporate structural constraints and perceptual losses to improve anatomical consistency in synthetic images. However, concerns persist regarding the quality and fidelity of synthetic images, as even minor artifacts or inconsistencies can propagate through downstream tasks, adversely affecting segmentation performance. In this context, synthetic medical images may act as a form of data poisoning, compromising model reliability and clinical applicability \citep{singkorapoom2023pre}.

\paragraph{Data Poisoning Attack}  
Data poisoning attacks involve injecting manipulated, low-quality, or malicious data into training datasets to degrade model performance or induce adversarial vulnerabilities \citep{yerlikaya2022data, fan2022survey}. In the medical imaging domain, poisoning can occur through mislabeled \citep{tolpegin2020data, lin2021active}, perturbed \citep{martinelli2023data, bortsova2021adversarial}. Prior research has demonstrated that even small perturbations in training data can lead to significant performance degradation in classification and segmentation models \citep{szegedy2013intriguing, chakraborty2021survey}. While poisoning attacks and corresponding mitigation strategies \citep{goldblum2022dataset, schwarzschild2021just, fu2024poisonbench, li2024scisafeeval} have been extensively studied in general computer vision tasks \citep{wei2024physical, akhtar2018threat}, their impact on medical imaging pipelines remains underexplored. Our work investigates the effect of synthetic MRI data as a form of data poisoning, evaluating its impact on brain tumor segmentation performance.