\section{Introduction}
In adaptive radiation therapy (ART), cone beam computed tomography (CBCT) is used to assess any change in the patient's anatomy or tumor volume during the treatment course.
This is particularly important because modern radiotherapy (RT) techniques, such as intensity-modulated radiation therapy (IMRT) and volumetric modulated arc therapy (VMAT), are highly conformal, requiring precise alignment of the patient's anatomy with the treatment plan to deliver the dose accurately.
To this end, CBCT is a cost-effective technology that can be integrated into the Linear Accelerator (LINAC) used to deliver the treatment, allowing the acquisition of real-time images of the patient before each treatment session.
However, when compared to computer tomography (CT), the gold standard imaging modality for radiation therapy (RT) planning, CBCT has a lower image quality and a smaller field of view, which can limit the ability to visualize the patient's anatomy and the regions of interest (ROI) accurately.
This is particularly problematic in complex clinical scenarios, like the conditioning regimes for marrow transplantation~\cite{Passweg2012}, where the ROI spans from head to toe and requires a full view of the patient's anatomy to ensure the adherence of the treatment plan.
A notable example is the total marrow and lymph node irradiation (TMLI)~\cite{Mancosu2019}, a modern treatment option that involves the precise irradiation of the bone marrow and/or the lymph nodes.
As a result, TMLI bears considerably lower late-toxicity issues and is associated with better overall response and clinical outcome~\cite{Cosset1994}. 

Unfortunately, TMLI is still rather underused because of the challenges in its implementation, which include an accurate delineation of the targets and organs at risk (OARs), a demanding planning process to guarantee an accurate dose delivery, and the need for a precise validation of the treatment plan before each session.
In particular, due to the inherent complexity of this treatment, it is extremely important to evaluate the patient's anatomy before each session to ensure the adherence of the treatment plan and to avoid any potential toxicity~\cite{Zhang2015}.
This is typically done through the acquisition of CBCT scans that must be matched with planning CT, i.e., the CT scan used to develop the treatment plan, to make sure that the patient's anatomy has not changed significantly and that the plan is still valid.
However, this process is significantly hindered by the limitations of CBCT, which cannot cover the whole body and has a lower image quality than CT.
\begin{figure}
	\centering
		\includegraphics[scale=0.18]{intro/gap_overview.pdf}
	\caption{An example of the usage of CBCT before a TMLI session: coronal and sagittal views of several series of CBCT scans combined together to cover the whole body of the patient.}
	\label{fig:gap_overview}
\end{figure}
Fig.~\ref{fig:gap_overview} shows an example of the images that can be obtained with CBCT scans before a TMLI session: it is possible to see the lower image quality and the need to combine several scans to cover the whole body, which results in gaps between the acquisitions.

We aim at addressing the limitations of CBCT in ART workflows for complex treatments like TMLI by proposing a novel deep learning-based framework, called ARTInp (\textbf{A}daptive \textbf{R}adiation \textbf{T}herapy \textbf{Inp}ainting). 
Our framework combines image inpainting to complete the gaps in CBCT images and image translation to enhance the quality of CBCT images, providing a synthetic CT (sCT) image that can be more easily matched with the planning CT to validate the treatment plan.
To the best of our knowledge, ARTInp is one of the first approaches that deal with the problem of generating synthetic images to fill the gaps in CBCT scans acquired during RT procedures.
In this paper, we present a proof-of-concept experiment where we apply ARTInp to a dataset of paired CBCT/CT images of the brain from the SynthRad2023 challenge~\cite{Thummerer2023}, where gaps are artificially introduced in the CBCT images to simulate the clinical scenario described above.
Although this preliminary experiment is set in a much simpler scenario than TMLI or other complex real-world treatments, it demonstrates the potential of ARTInp to enhance the usability of CBCT images and provide a more accurate visualization of the patient's anatomy.
We also hope that, by using a public dataset and a reproducible experimental setup, this work can be a starting point for further research in this direction and the development of more advanced and clinically relevant frameworks.

The paper is organized as follows. In Section~\ref{sec:related}, we review the related works on image inpainting and image translation in the context of medical imaging. In Section~\ref{sec:methods}, we discuss in details the ARTInp framework. In Section~\ref{sec:results}, we describe the experimental setup and present the results. Finally, in Section~\ref{sec:conclusion}, we draw our conclusions and outline the future directions of this research.

% CBCT is used in place of computed tomography (CT) because it is quick and easy to perform, and it is featured directly on RT equipment; in contrast, it can only offer suboptimal visualization of the patient's internal structures and regions of interest (ROI), both in terms of image quality and field of view. This particular type of scanner is able to scan a reduced portion of the body with low resolution and numerous artefacts; figure \ref{fig:gap_overview} shows an example of the sagittal and coronal view of several series of CBCT joined together to visualize the patient's full body. 
% The stack of these series of scans still shows some gaps between the acquisitions and lower quality than the CT image.
% This can be problematic as the physicians cannot have a precise outline of what has happened in the patient's body between sessions and may miss some key observations that are possible with a CT scan of the whole anatomical section. 
% These issues are particularly relevant when considering complex clinical scenarios, like the conditioning regimes for marrow transplantation (a treatment option for some types of acute leukemia), which requires complete irradiation of the patient's lymphatic system \cite{Passweg2012}. 
% Total marrow and lymph node irradiation (TMLI) \cite{Mancosu2019} is one of the most recent clinical approaches for dose delivery in this kind of case, as, compared to the older approach, total body irradiation, bears considerably lower late-toxicity related problems and is associated with overall better response \cite{Cosset1994}; briefly, it consists of the precise irradiation of only the bone marrow and/or the lymph nodes (depending on the clinical necessities and the chosen protocol) after a precise treatment preparation by medical physicists and radiation oncologists; the preparation phase consists of targets identification and contouring and organs at risks identification and contouring (OARs, the anatomical structures that must be spared of the dose), which are performed by radiation oncologists, and dose calculation, which necessitate on a CT and is performed by medical physicists. 
% TMLI has some important advantages related to clinical outcomes, in particular regarding late toxicities; however, due to the inherent difficulty in its implementation, it is still rather underused and necessitates further technological advancement to be implemented successfully as to-got conditioning regimen for marrow transplantation. 
% Typically, the treatment consists of several dose-delivery sessions; the plan is developed in advance through a CT scan (and in some cases also a Magnetic Resonance) to accurately visualize the anatomy and permit the calculation of the dose.
% Patients selected for marrow transplantation who have to undergo similar protocols and therapeutical sessions are generally in the most fragile portion of the population, due to their disease itself, because of the psychological impact and the stress caused by the whole situation, and because of the therapies they follow, that can be particularly debilitating. 
% Moreover, their body can incur substantial changes in an anomaly rapid manner, which might render unusable the plan carried out after the initial CT: this is one of the main reasons why it is of utmost importance to validate the plan before each session, and this is usually done with a CBCT.
% For the plan to be validated, both the anatomy of the patient has to be checked, to assess that no excessive weight gain or loss has happened \cite{Zhang2015}, and the dose calculation has to be remade, to ensure that it's still valid. 
% As mentioned before, in these cases, two problems arise: 
% \begin{itemize}
%     \item the scan cannot cover the whole body of the patient; 
%     \item image quality is considerably lower than that of a CT. 
% \end{itemize}
% A whole body CT scan is not possible in this situation, as it would be very unpractical and resources-hungry, other than, most importantly, posing the risk of overwhelming the already stressed patient with an additional stress source.

% This problem is more evident in TMLI, because the ROI spans from head to toe; therefore, a full body scan would be ideal to check all the regions to be irradiated. 

% \begin{figure}
% 	\centering
% 		\includegraphics[scale=0.18]{intro/gap_overview.pdf}
% 	\caption{Coronal and Sagittal Group of CBCT's showcasing the gaps in between series.}
% 	\label{fig:gap_overview}
% \end{figure}

% Typically, three to four CBCTs are acquired, but still, they cannot cover the whole body and have gaps between them, as shown in figure \ref{fig:gap_overview}.
% These gaps are particularly problematic as, in their correspondence, the anatomy cannot be checked and this often happens where the isocentres are, so in the pivotal points for dose delivery. 

% For these reasons, a solution that allows a precise visualization of the patient's anatomy and a full view of the ROI while maintaining the convenience of a CBCT would be ideal.
% Deep Learning (DL) models are a good candidate for approximating this solution. 
% In this paper, to address the gap and quality issue simultaneously, we developed \textbf{A}daptive \textbf{R}adiation \textbf{T}herapy \textbf{Inp}ainting (ARTInp), a double network process composed by a completion network that fills the gaps in CBCT images and subsequently translates the inpainted CBCT into a synthetic CT (sCT) image with a custom Generative Adversarial Network (GAN). 
% In this way, through the acquisition of the usual CBCTs, it is possible to accurately assess the adherence of the plan to the whole patient body thanks to the better image quality the absence of gaps.

% As a first proof-of-concept and reproducible experiment, we applied ARTInp to the dataset from SynthRad2023's challenge \cite{Thummerer2023},  a public collection of paired CBCT/CT images of the brain and pelvic regions.
% The contribution of this work is the following: the ARTInp, one of the first approaches that provides a solution to fill CBCT gaps present in RT procedures for TMLI and translate them into the CT domain to increase further the quality of the visualization.

