\section{Results}
\label{sec:results}
\subsection{Experimental Design}
To evaluate the performance of the proposed ARTInp framework, we conducted two experiments: an evaluation of the performance of the translation network on the task of converting CBCT to sCT in the absence of the completion network; a measure the performance of the whole ARTInp framework.

In the first experiment, the translation network was trained using the CBCT and CT images pairs from the 144 patients of the training set (see Section \ref{ssec:dataset}) for 60 epochs (around 1800000 iterations) using an Adam~\cite{Kingma2015} optimizer with a learning rate of $2\cdot10^{-4}$ and a batch size of 1. 
%The validation set of 10 patients was used to tune the training parameters, that was selected as follows. 
A checkpoint was saved every 5 epochs and the model with the best performance on the validation set was selected. 

In the second experiment, we trained the completion network by generating artificial gaps in the same images used for the previous experiment. 
In particular, we created random gaps in the CBCT slices included in the training set by removing vertical strips of the images with a random location and a pixel width $w$ randomly sampled from a uniform distribution $\mathcal{w}\sim\mathcal{U}(48, 96)$.
The resulting CBCT images with artificial gaps were used, along with a binary mask of the generated gaps and the corresponding original CT images, to train the completion network.
These artificial gaps mimic the ones that would be found in a real clinical scenario, having similar width and positioning. 
The three training phases described in Section~\ref{ssec:completion} where implemented as follows: 
\begin{enumerate}
    \item in the first phase the generator network is trained using only the L1 loss for 180000 iterations;
    \item in the second phase, the generator network's weights are frozen and only the discriminator network is trained for 20000 iterations;
    \item in the third phase, both the generator and the discriminator networks are trained together for 620000 iterations.
\end{enumerate}

The batch size was 1 and the ADADELTA optimization algorithm~\cite{zeiler2012adadelta} was used; a checkpoint was saved every 2000 iterations and the model with the best performance on the validation set was selected.

The evaluation of the models trained in the two experiments described above was performed on the test set of 18 patients.
To evaluate the performance of the translation network, we tested it on the 3456 CBCT axial slices included in the test set and computed the MAE\%, PSNR, and SSIM metrics.
Then, we combined the translation network with completion network trained in the second experiments to evaluate the performance of the whole ARTInp framework.
We created artificial gaps in the CBCT slices included in the test set, similarly to what we did in the training set; however, in this case, the location and the width of the gaps were sampled patient-wise, i.e., we removed the same portion in all the slices of the CBCT volume of a patient, to adhere as much as possible to the actual clinical situation.
The inpainted CBCT images of the test set generated with the completion network were then used to generate the sCT images through the translation network.
Finally, we computed the MAE\%, PSNR, and SSIM metrics to evaluate the quality of the images generated by the ARTInp framework with respect to the original CT images.
%After the training, the model is used for inference in the test set but this time, the creating of the gaps in the test set is different. In training, the gap was created randomly in any given sagittal slice.
%These holes have the width and height between 48 and 96 pixels and they were positioned randomly within the slice.
%In the testing part, the gap is created patient-wise. It has the width of the sagittal slice and a a height of $2.5cm$, this means that the gap is equivalent to removing a set of entire axial slices from the patient's 3D volume. The gap now represents a missing part of the entire volume ($2.5cm$) and this type of gap is more likely to occur in clinical practice, for example, if you need to combine several CBCT scans over a greater area of the patient and the combination of these scans leaves a gap in between.
%Dividing the total height of the patient's volume based on its slice thickness by the total height of the patient in pixels results in the ratio of millimeters per pixel and thus the $2.5cm$ gap can be calculated individually for each patient.

\subsection{Results}
\begin{table}
	\centering
	\begin{tabular}{lccc}
	MODEL                  & MAE\%         & PSNR           & SSIM            \\ \hline
	
	ARTInp w/o Completion         & 2.22   ± 1.35 & 27.77   ± 2.74 & 0.791   ± 0.082 \\
	ARTInp & 2.44   ± 1.25 & 26.84   ± 2.53 & 0.768   ± 0.079
	\end{tabular}
	\label{tab:set1_results}
	\end{table}	
Table~\ref{tab:set1_results} shows the performance achieved by ARTInp models on the test set.
In the table, \emph{ARTInp w/o Completion} refers to the model that only includes the translation network and does not perform any inpainting, while \emph{ARTInp} refers to the model that includes both the translation and the completion networks.
Both models achieve an MAE\% below 2.5\%, corresponding to average values under 100 HU, which is consistent with what has been observed in the literature for the brain area~\cite{edmund_review_2017}.
The PSNR around 27dB and an SSIM below 0.8 suggest the presence of artifacts and distortions compared to the real images. 
However, the introduction of gaps and the use of the completion network do not seem to drastically degrade the performance, keeping it at promising values for the practical application of ARTInp.

We also present some qualitative results of the sCT generation and the inpainted CBCT in Fig.~\ref{fig:8bit_pix2pix} and Fig.~\ref{fig_gap_ct}, respectively.
The sCT images generated by the translation network are shown in the second column of Fig.~\ref{fig:8bit_pix2pix}, while the original CBCT image and the original CT images are respectively in the first and third columns for comparison.
The inpainted CBCT images are shown in the second column of Fig.~\ref{fig_gap_ct}, while the original CBCT images with gaps and the original CBCT images are respectively in the first and third columns for comparison.
The examples show how the sCT images generated are generally able to capture quite well the features of the original CT images, even if some artifacts are present.
In Figure \ref{fig_gap_ct}, the inpainted CBCT are shown in the second column, while the original CBCT is presented on the third column for comparison. 
It is possible to observe that the inpainting process is able to reconstruct the missing parts of the CBCT images, even if some artifacts are present, such as blurriness and noise.
In particular, in the first and fourth example, the bone of the skull seems blurry in the gap location and the tissue inside the skull has significantly more noise than the original CBCT, although the structure's shape still is close to the original. 
In the second example, the atlas vertebrae is inpainted successfully along with the beginning of the skull bone. 
The third example is a region on the side of the patient's skull and it can be seen that the suture that separates the parietal and the temporal bone is slightly blurrier in the inpainted CBCT.

\begin{figure}
	\centering
		\includegraphics[scale=0.11]{results/fig_16_bit.pdf}
	\caption{Examples of the sCT generation.}
	\label{fig:8bit_pix2pix}
\end{figure}

\begin{figure}
	\centering
		\includegraphics[scale=0.11]{results/fig_gap_ct.pdf}
	\caption{Examples of the inpainted CBCT. The images contain the original CBCT with a gap, the inpainted CBCT, and, the original CBCT.}
	\label{fig_gap_ct}
\end{figure}