%%%%%%%% ICML 2025 EXAMPLE LATEX SUBMISSION FILE %%%%%%%%%%%%%%%%%

\documentclass{article}

% Recommended, but optional, packages for figures and better typesetting:
\usepackage{microtype}
\usepackage{graphicx}
\usepackage{subfigure}
\usepackage{svg}
\svgpath{{./figures/}}
\usepackage{booktabs} % for professional tables
% For general algorithm formatting
\usepackage{algorithm}
\usepackage{multirow}
% \usepackage{algorithmic}
% \usepackage{algpseudocode}

% Algorithm formatting customization
% \renewcommand{\algorithmicrequire}{\textbf{Input:}}
% \renewcommand{\algorithmicensure}{\textbf{Output:}}

% If you need more math symbols
\usepackage{amsmath}
\usepackage{amssymb}

% If you want to customize the algorithm float
% \usepackage{float}
% \floatplacement{algorithm}{t}
% hyperref makes hyperlinks in the resulting PDF.
% If your build breaks (sometimes temporarily if a hyperlink spans a page)
% please comment out the following usepackage line and replace
% \usepackage{icml2025} with \usepackage[nohyperref]{icml2025} above.
\usepackage{hyperref}


% Attempt to make hyperref and algorithmic work together better:
\newcommand{\theHalgorithm}{\arabic{algorithm}}

% Use the following line for the initial blind version submitted for review:
% \usepackage{icml2025}

% If accepted, instead use the following line for the camera-ready submission:
\usepackage[accepted]{arxiv}
% \usepackage{icml2025}

% For theorems and such
\usepackage{amsmath}
\usepackage{amssymb}
\usepackage{mathtools}
\usepackage{amsthm}
% RMK: added
\usepackage{bm}

% if you use cleveref..
\usepackage[capitalize,noabbrev]{cleveref}

%%%%%%%%%%%%%%%%%%%%%%%%%%%%%%%%
% THEOREMS
%%%%%%%%%%%%%%%%%%%%%%%%%%%%%%%%
\theoremstyle{plain}
\newtheorem{theorem}{Theorem}[section]
\newtheorem{proposition}[theorem]{Proposition}
\newtheorem{lemma}[theorem]{Lemma}
\newtheorem{corollary}[theorem]{Corollary}
\theoremstyle{definition}
\newtheorem{definition}[theorem]{Definition}
\newtheorem{assumption}[theorem]{Assumption}
\theoremstyle{remark}
\newtheorem{remark}[theorem]{Remark}

% Todonotes is useful during development; simply uncomment the next line
%    and comment out the line below the next line to turn off comments
%\usepackage[disable,textsize=tiny]{todonotes}
\usepackage[textsize=tiny]{todonotes}

\providecommand{\zhenyu}[1]{
    {\protect\color{red}{[Zhenyu: #1]}}
}
\providecommand{\qifan}[1]{
    {\protect\color{orange}{[Qifan: #1]}}
}
\providecommand{\xjj}[1]{
    {\protect\color{green}{[xjj: #1]}}
}


% The \icmltitle you define below is probably too long as a header.
% Therefore, a short form for the running title is supplied here:
\icmltitlerunning{ Enhancing Auto-regressive Chain-of-Thought through Loop-Aligned Reasoning}

\begin{document}

\twocolumn[
\icmltitle{ Enhancing Auto-regressive Chain-of-Thought through Loop-Aligned Reasoning}

% It is OKAY to include author information, even for blind
% submissions: the style file will automatically remove it for you
% unless you've provided the [accepted] option to the icml2025
% package.

% List of affiliations: The first argument should be a (short)
% identifier you will use later to specify author affiliations
% Academic affiliations should list Department, University, City, Region, Country
% Industry affiliations should list Company, City, Region, Country

% You can specify symbols, otherwise they are numbered in order.
% Ideally, you should not use this facility. Affiliations will be numbered
% in order of appearance and this is the preferred way.
\icmlsetsymbol{equal}{*}

\begin{icmlauthorlist}
% \icmlauthor{Qifan Yu}{}
% \icmlauthor{Zhenyu He}{}
% \icmlauthor{Sijie Li}{}
% \icmlauthor{Zhou Xun}{}
% \icmlauthor{Jun Zhang}{}
% \icmlauthor{Jingjing Xu}{}
% \icmlauthor{Di He}{}
% \icmlauthor{Qifan Yu}{equal,pku}
% \icmlauthor{Zhenyu He}{equal,pku}
\icmlauthor{Qifan Yu}{pku}
\icmlauthor{Zhenyu He}{pku}
\icmlauthor{Sijie Li}{pku}
\icmlauthor{Xun Zhou}{byte}
\icmlauthor{Jun Zhang}{byte}
\icmlauthor{Jingjing Xu}{byte}
\icmlauthor{Di He}{pku}
%\icmlauthor{}{sch}
%\icmlauthor{}{sch}
%\icmlauthor{}{sch}
\end{icmlauthorlist}

\icmlaffiliation{pku}{Peking University}
% \icmlaffiliation{pkueecs}{School of EECS, Peking University}
\icmlaffiliation{byte}{ByteDance Inc.}

% \icmlcorrespondingauthor{Firstname1 Lastname1}{first1.last1@xxx.edu}
% \icmlcorrespondingauthor{Di He}{}
% \icmlcorrespondingauthor{Di He}{dihe@pku.edu.cn}

% You may provide any keywords that you
% find helpful for describing your paper; these are used to populate
% the "keywords" metadata in the PDF but will not be shown in the document
\icmlkeywords{Machine Learning, ICML}

\vskip 0.3in
]

% this must go after the closing bracket ] following \twocolumn[ ...

% This command actually creates the footnote in the first column
% listing the affiliations and the copyright notice.
% The command takes one argument, which is text to display at the start of the footnote.
% The \icmlEqualContribution command is standard text for equal contribution.
% Remove it (just {}) if you do not need this facility.

\printAffiliationsAndNotice{}  % leave blank if no need to mention equal contribution
% \printAffiliationsAndNotice{\icmlEqualContribution} % otherwise use the standard text.

\begin{abstract}
Chain-of-Thought (CoT) prompting has emerged as a powerful technique for enhancing language model's reasoning capabilities. However, generating long and correct CoT trajectories is challenging. 
Recent studies have demonstrated that Looped Transformers possess remarkable length generalization capabilities, but their limited generality and adaptability prevent them from serving as an alternative to auto-regressive solutions. To better leverage the strengths of Looped Transformers, we propose \textbf{RELAY} (\underline{\textbf{RE}}asoning through \underline{\textbf{L}}oop \underline{\textbf{A}}lignment iterativel\underline{\textbf{Y}}).
Specifically, we align the steps of Chain-of-Thought (CoT) reasoning with loop iterations and apply intermediate supervision during the training of Looped Transformers. This additional iteration-wise supervision not only preserves the Looped Transformer's ability for length generalization but also enables it to predict CoT reasoning steps for unseen data. Therefore, we leverage this Looped Transformer to generate accurate reasoning chains for complex problems that exceed the training length, which will then be used to fine-tune an auto-regressive model. We conduct extensive experiments, and the results demonstrate the effectiveness of our approach, with significant improvements in the performance of the auto-regressive model. Code will be released at \url{https://github.com/qifanyu/RELAY}.
\end{abstract}

\section{Introduction}
Backdoor attacks pose a concealed yet profound security risk to machine learning (ML) models, for which the adversaries can inject a stealth backdoor into the model during training, enabling them to illicitly control the model's output upon encountering predefined inputs. These attacks can even occur without the knowledge of developers or end-users, thereby undermining the trust in ML systems. As ML becomes more deeply embedded in critical sectors like finance, healthcare, and autonomous driving \citep{he2016deep, liu2020computing, tournier2019mrtrix3, adjabi2020past}, the potential damage from backdoor attacks grows, underscoring the emergency for developing robust defense mechanisms against backdoor attacks.

To address the threat of backdoor attacks, researchers have developed a variety of strategies \cite{liu2018fine,wu2021adversarial,wang2019neural,zeng2022adversarial,zhu2023neural,Zhu_2023_ICCV, wei2024shared,wei2024d3}, aimed at purifying backdoors within victim models. These methods are designed to integrate with current deployment workflows seamlessly and have demonstrated significant success in mitigating the effects of backdoor triggers \cite{wubackdoorbench, wu2023defenses, wu2024backdoorbench,dunnett2024countering}.  However, most state-of-the-art (SOTA) backdoor purification methods operate under the assumption that a small clean dataset, often referred to as \textbf{auxiliary dataset}, is available for purification. Such an assumption poses practical challenges, especially in scenarios where data is scarce. To tackle this challenge, efforts have been made to reduce the size of the required auxiliary dataset~\cite{chai2022oneshot,li2023reconstructive, Zhu_2023_ICCV} and even explore dataset-free purification techniques~\cite{zheng2022data,hong2023revisiting,lin2024fusing}. Although these approaches offer some improvements, recent evaluations \cite{dunnett2024countering, wu2024backdoorbench} continue to highlight the importance of sufficient auxiliary data for achieving robust defenses against backdoor attacks.

While significant progress has been made in reducing the size of auxiliary datasets, an equally critical yet underexplored question remains: \emph{how does the nature of the auxiliary dataset affect purification effectiveness?} In  real-world  applications, auxiliary datasets can vary widely, encompassing in-distribution data, synthetic data, or external data from different sources. Understanding how each type of auxiliary dataset influences the purification effectiveness is vital for selecting or constructing the most suitable auxiliary dataset and the corresponding technique. For instance, when multiple datasets are available, understanding how different datasets contribute to purification can guide defenders in selecting or crafting the most appropriate dataset. Conversely, when only limited auxiliary data is accessible, knowing which purification technique works best under those constraints is critical. Therefore, there is an urgent need for a thorough investigation into the impact of auxiliary datasets on purification effectiveness to guide defenders in  enhancing the security of ML systems. 

In this paper, we systematically investigate the critical role of auxiliary datasets in backdoor purification, aiming to bridge the gap between idealized and practical purification scenarios.  Specifically, we first construct a diverse set of auxiliary datasets to emulate real-world conditions, as summarized in Table~\ref{overall}. These datasets include in-distribution data, synthetic data, and external data from other sources. Through an evaluation of SOTA backdoor purification methods across these datasets, we uncover several critical insights: \textbf{1)} In-distribution datasets, particularly those carefully filtered from the original training data of the victim model, effectively preserve the model’s utility for its intended tasks but may fall short in eliminating backdoors. \textbf{2)} Incorporating OOD datasets can help the model forget backdoors but also bring the risk of forgetting critical learned knowledge, significantly degrading its overall performance. Building on these findings, we propose Guided Input Calibration (GIC), a novel technique that enhances backdoor purification by adaptively transforming auxiliary data to better align with the victim model’s learned representations. By leveraging the victim model itself to guide this transformation, GIC optimizes the purification process, striking a balance between preserving model utility and mitigating backdoor threats. Extensive experiments demonstrate that GIC significantly improves the effectiveness of backdoor purification across diverse auxiliary datasets, providing a practical and robust defense solution.

Our main contributions are threefold:
\textbf{1) Impact analysis of auxiliary datasets:} We take the \textbf{first step}  in systematically investigating how different types of auxiliary datasets influence backdoor purification effectiveness. Our findings provide novel insights and serve as a foundation for future research on optimizing dataset selection and construction for enhanced backdoor defense.
%
\textbf{2) Compilation and evaluation of diverse auxiliary datasets:}  We have compiled and rigorously evaluated a diverse set of auxiliary datasets using SOTA purification methods, making our datasets and code publicly available to facilitate and support future research on practical backdoor defense strategies.
%
\textbf{3) Introduction of GIC:} We introduce GIC, the \textbf{first} dedicated solution designed to align auxiliary datasets with the model’s learned representations, significantly enhancing backdoor mitigation across various dataset types. Our approach sets a new benchmark for practical and effective backdoor defense.



\begin{figure*}[t]
  \centering
    \includegraphics[width=1\linewidth]{visuals/final_registration.png}
    \caption{Target measurement process on low-cost scan data using ICP and Coloured ICP. (1) Initialisation: The source point cloud (checkerboard) is misaligned with the target point cloud. (2) Initial Registration using Point-to-Plane ICP: Standard ICP leads to suboptimal registration. (3) Final Registration using Coloured ICP: Colour information is incorporated after pre-processing with RANSAC and Binarisation with Otsu Thresholding for real data, resulting in improved alignment.}
    \label{fig:Registration_visualisation}
\end{figure*}

\subsection{Iterative Closest Point (ICP) Algorithm}
The Iterative Closest Point (ICP) algorithm has been a fundamental technique in 3D computer vision and robotics for point cloud. Originally proposed by \cite{besl_method_1992}, ICP aims to minimise the distance between two datasets, typically referred to as the source and the target. The algorithm operates in an iterative manner, identifying correspondences by matching each source point with its nearest target point \citep{survey_ICP}. It then computes the rigid transformation, usually involving both rotation and translation, to achieve the best alignment of these matched points \citep{survey_ICP}. This process is repeated until convergence, where the change in the alignment parameters or the overall alignment error becomes smaller than a predefined threshold.

One key advantage of the ICP framework lies in its simplicity: the algorithm is conceptually straightforward, and its basic version is relatively easy to implement. However, traditional ICP can be sensitive to local minima, often requiring a good initial alignment \citep{zhang2021fast}. Furthermore, outliers, noise, and partial overlaps between datasets can significantly degrade its performance \citep{zhang2021fast, bouaziz2013sparse}. Over the years, various modifications and improvements \citep{gelfand2005robust, rusu2009fast, aiger20084, gruen2005least, fitzgibbon2003robust} have been proposed to mitigate these issues. Among the most common strategies are robust cost functions \citep{fitzgibbon2003robust}, weighting schemes for correspondences \citep{rusu2009fast}, and more sophisticated techniques \citep{gelfand2005robust, bouaziz2013sparse} to reject outliers. 

In addition, there is significant interest in integrating additional information into the ICP pipeline. Instead of solely relying on geometric cues such as point coordinates or surface normals, recent approaches have proposed incorporating colour (RGB) or intensity data to enhance correspondence accuracy. These methods \citep{park_colored_2017, 5980407}, commonly known as "Colored ICP" employ differences in pixel intensities or colour values as additional constraints. This is particularly beneficial in situations where geometric attributes alone are inadequate for accurate alignment or where surfaces possess complex texture patterns that can assist in the matching process.

\subsection{Applications of Target Measurement}

One approach relies on the use of physical checkerboard targets for registration. \cite{fryskowska2019} analyse checkerboard target identification for terrestrial laser scanning. They propose a geometric method to determine the target centre with higher precision, demonstrating that their approach can reduce errors by up to 6 mm compared to conventional automatic methods.

\cite{becerik2011assessment} examines data acquisition errors in 3D laser scanning for construction by evaluating how different target types (paper, paddle, and sphere) and layouts impact registration accuracy in both indoor and outdoor environments and presents guidelines for optimal target configuration.

\citet{Liang2024} propose to use Coloured ICP to measure target centres for checkerboard targets, similar to our investigation. They use data from a survey-grade terrestrial laser scanner. Their intended application is structural bridge monitoring purposes. They report an average accuracy of the measurement below 1.3 millimetres.

Where targets cannot be placed in the scene, the intensity information form the scanner can still be used to identify distinctive points. For point cloud data that is captured with a regular pattern, standard image processing can be used in a similar way to target detection. For example, \citet{wendt_automation_2004} proposes to use the SUSAN operator on a co-registered image from a camera, \citet{bohm_automatic_2007} proposes to use the SIFT operator on the LIDAR reflectance directly and \citet{theiler_markerless_2013} propose to use a Difference-of-Gaussian approach on the reflectance information.
Most of these methods extract image features to find reliable 3D correspondences for the purpose of registration.

In the following we describe our approach to the measurement of the target centre. In contrast to most proposed methods above we focus on unordered point clouds, where raster-based methods are not available, and low-cost sensors, where increased measurement noise and outliers are expected. As we are not aware of a commercial reference solution to this problem, we start by conducting a series of synthetic experiments to explore the viability and accuracy potential of the approach.



%The reviewed studies primarily rely on physical targets or target-free methods and do not utilise 3D synthetic point cloud checkerboards. In contrast, our approach introduces synthetic point cloud checkerboards, which offer controlled and consistent target geometry and reduce variability caused by physical targets. This innovation has significant potential for commercialisation and industrial application.

\section{Methodology}

\subsection{Problem Definition}

Given a multivariate time series input $X \in \mathbb{R}^{C  \times T}$, multivariate time series forecasting tasks are designed to predict its future $F$ time steps $\hat{Y}\in \mathbb{R}^{C \times F}$ using past $T$ steps. $C $ is the number of variates or channels.

\subsection{Preliminary Analysis}

This section presents why RevIN~\citep{Kim_revin,liu2022non}, High-pass, and Low-pass filters fail to address the Mid-Frequency Spectrum Gap. Let the input univariate time series be $ x(t) $ with length $ T $ and target $ y(t) $ with length $ F $. 

\begin{definition}[Frequency Spectral Energy]\label{def:energy}
The Fourier transform of $x(t)$, $X(f)$, and its spectral energy $E_X(f)$ is given by:
\vspace{-0.2cm}
\begin{align}
X(f) = \sum_{t=0}^{T-1} x(t) e^{-i 2 \pi f t / {T-1}}, \quad &f = 0, 1, \dots, T-1\notag\\
E_X(f) = |X(f)|^2.
\end{align}
\vspace{-0.2cm}
\end{definition}

\textbf{Impact of RevIN on Frequency Spectrum \quad}
\begin{definition}[Reversible Instance Normalization]\label{def:RevIN}
Given a \textbf{forecast model} $ f: \mathbb{R}^T \rightarrow \mathbb{R}^F $ that generates a forecast $ \hat{y}(t) $ from a given input $x(t)$, RevIN is defined as:
\vspace{-0.2cm}
\begin{align}
&\hat{x}(t) = \frac{x(t) - \mu}{\sigma},\quad t = 0, 1, \dots, T-1\notag\\
&\hat{y}(t) = f(\hat{x}(t)), \quad \hat{y}(t)_{rev}= \hat{y}(t) \cdot \sigma + \mu,\notag\\
&\mu = \frac{1}{T} \sum_{t=0}^{T-1} x(t), \quad \sigma = \sqrt{\frac{1}{T} \sum_{t=0}^{T-1} (x(t) - \mu)^2}.
\end{align}
\vspace{-0.2cm}
\end{definition}

\begin{theorem} [Frequency Spectrum after RevIN] \label{theorem:RevIN}
\vspace{-0.2cm}
The spectral energy of $\hat{x}(t)$ (transformed using RevIN):
\begin{align}
E_{\hat{X}}(0)=0,& \quad f=0, \notag\\
E_{\hat{X}}(f) = \left( \frac{1}{\sigma} \right)^2 |X(f)|^2,&\quad f = 1,2,\dots, T-1 . 
\end{align}
\vspace{-0.2cm}
\end{theorem}
The proof is in Appendix~\ref{app:RevIN}. Theorem~\ref{theorem:RevIN} suggests that RevIN scales the absolute spectral energy by $ \sigma^2 $ but does not affect its relative distribution except $E_{\hat{X}}(0)=0$. Thus, RevIN preserves the relative spectral energy distribution and leaves the Mid-Frequency Spectrum Gap unresolved. \textit{However, our experiments still employ RevIN to ensure a fair comparison with other baselines.}
\begin{figure*}[h]
  \centering
  \includegraphics[width=1.\linewidth]{Faker/source/assets/jpg/ReFocus.jpg}
  \caption{General structure of \textbf{ReFocus}. `Adaptive Mid-Frequency Energy Optimizer (AMEO)' enhances mid-frequency components modeling, and `Energy-based Key-Frequency Picking Block' (EKPB) effectively captures shared Key-Frequency across channels}
  \label{fig:refocus}
\end{figure*}

\begin{figure*}[h]
  \centering
  \includegraphics[width=0.7\linewidth]{Faker/source/assets/jpg/ket.jpg}
  \caption{General process of the \textbf{Key-Frequency Enhanced Training strategy (KET)}, where spectral information from other channels is randomly introduced into each channel, to enhance the extraction of the shared Key-Frequency.}
  \label{fig:reshuffle}
\end{figure*}
\textbf{Impact of High- and Low-pass filter \quad}
We still define $\hat{x}(t)$ to be the filtered (processed) signal, obtained by applying a filter $H(f)$ (High/Low-pass filter). The filter $ H(f) $ is 1 in the passband (High/Low frequency) and 0 in the stopband (Middle frequency). So $E_{\hat{X}}(f)=0,\quad E_{\hat{X}}\leq E_X(f)$ for middle frequencies, which creates even larger gap.

\subsection{Overall Structure of The Proposed ReFocus}

In this section, we elucidate the overall architecture of \textbf{ReFocus}, depicted in Figure \ref{fig:refocus}. We define frequency domain projection as $D1\rightarrow D2$ representing a projection from dimension $D1$ to $D2$ in the frequency domain~\citep{xu2024fits}. Initially, we apply \textbf{AMEO} to the input $X \in \mathbb{R}^{C \times T}$, yielding the processed spectrum $ X_{am} \in \mathbb{R}^{C  \times T} $. Next, we use a projection $T\rightarrow D$ to transform $ X_{am}$ into the Variate Embedding $ X_{em} \in \mathbb{R}^{C  \times D}$~\citep{LiuiTransformer}. Then, $X_{em}$ go through $N$ \textbf{EKPB} to generate representation $H_{N+1}$, which is projected to obtain final prediction $\hat{Y}$. 

\textbf{Adaptive Mid-Frequency Energy Optimizer \quad}
Building upon the \textbf{Preliminary Analysis}, we propose a convolution- and residual learning-based solution to address the Mid-Frequency Spectrum Gap, which we denoted as AMEO. 
\begin{definition}[Adaptive Mid-Frequency Energy Optimizer]\label{def:AMEO}
AMEO is defined as:
\begin{align}
&\hat{x}(t) = x(t)-\frac{\beta}{K}\sum_{k=0}^{K-1} \tilde{x}(t+K-1-k),\notag\\
&\tilde{x}(t) =\notag\\
&\begin{cases}
x(t-(\frac{K}{2}+1)), \quad \text{if } \frac{K}{2}+1 \leq t < T+\frac{K}{2}+1, \\
0,  \quad\text{if } 0 \leq t < \frac{K}{2}+1 \text{ or } T+\frac{K}{2}+1 \leq t < T+K.
\end{cases}
\end{align}
\vspace{-0.2cm}
\end{definition}

It is equivalent to $x=x-\beta \cdot Conv(x)$. $Conv$ is a 1D convolution (Zero-padding at both ends, stride $s=1$, kernel size $K$, with values initialized as $ \frac{1}{K} $). $\beta \in \mathbb{R}^{1}$ is a hyperparameter.

\begin{theorem} [Frequency Spectrum after AMEO] \label{theorem:AMEO}
The spectral energy of $\hat{x}(t)$ obtained using AMEO:
\begin{align}
E_{\hat{X}}(f) =|X(f)|^2 \left\{1 - \beta \cdot \underbrace{\frac{1}{K} \sum_{k=0}^{K-1} e^{i 2 \pi f (\frac{3K}{2}-k -2) / {T-1}}}_{G(f)}\right\}^2
\end{align}
\vspace{-0.2cm}
\end{theorem}

The proof is in Appendix~\ref{app:AMEO}. We have $E_{\hat{X}}(f) =|X(f)|^2(1-\beta  \cdot G(f))^2$. Generally, $ G(f) $ behaves as a decay function, gradually reducing its value from \textbf{One} to \textbf{Zero}. Such \textbf{decay behavior} makes AMEO relatively enhances mid-frequency components, thus addressing the Mid-Frequency Spectrum Gap.

\textbf{Energy-based Key-Frequency Picking Block \quad} In each \textbf{EKPB}, the input $ H_i \in \mathbb{R}^{C  \times D} (H_1=X_{em}) $ is first processed through an MLP to generate $ H_i^k \in \mathbb{R}^{C  \times Q}$. Then, FFT is applied to get $ H_i^f \in \mathbb{R}^{C  \times (Q/2+1)}$. For $ H_i^f$, we calculate its energy, denoted as $ H_i^e \in \mathbb{R}^{C  \times (Q/2+1)}$. A cross-channel softmax is then applied to $ H_i^e$ per frequency to obtain a probability distribution $ H_i^{soft} \in \mathbb{R}^{C  \times (Q/2+1)}$. Using $H_i^{soft}$, we select values from $ H_i^f$ across channels for each frequency, resulting in $K^f_i \in \mathbb{R}^{1  \times (Q/2+1)}$, which represents the Shared Key-Frequency across all channels. Then iFFT is performed on $K^f_i$ to get $K_i\in \mathbb{R}^{1  \times Q}$, followed by projection $Q\rightarrow D$ and repeating (C times) to get $\hat{K}_i \in \mathbb{R}^{C  \times D}$. This $\hat{K}_i$ is point-wisely added to $\hat{H_i}\in \mathbb{R}^{C  \times D}$ , which is the projection of $ H_i$ using projection $D\rightarrow D$. Then, an MLP and $Add\&Norm$ is applied to the result $HK\in \mathbb{R}^{C  \times D}$ to fuse inter-series dependencies information, and another MLP and $Add\&Norm$ is used to capture intra-series variations~\citep{LiuiTransformer}. The output of each \textbf{EKPB} is $\hat{O_i} \in \mathbb{R}^{C  \times D}$, where $H_{i+1}=\hat{O_i}$.

\subsection{Key-Frequency Enhanced Training strategy}

In real-world time series, certain channels often exhibit spectral dependencies, which may not be fully captured in the training set, and the specific channels with such dependencies are also unknown~\citep{geweke1984freqchannel,Zhao2024freqchannel}. So this work borrows insight from recent advancement of mix-up in time series~\citep{zhou2023mixup,ansari2024mixup}, randomly introducing spectral information from other channels into each channel, to enhance the extraction of the shared Key-Frequency, as in Figure~\ref{fig:reshuffle}. Given a multivariate time series input $X \in \mathbb{R}^{C \times T}$ and its ground-truth $Y \in \mathbb{R}^{C \times F}$, we generate a pseudo sample pair: 

\begin{align}
X' = iFFT(FFT(X) +\alpha \cdot FFT(X[\text{perm},:]))&,  \notag\\ 
Y' = iFFT(FFT(Y) +\alpha \cdot FFT(Y[\text{perm},:]))&.
\end{align}

$\alpha \in \mathbb{R}^{C \times 1}$ is a weight vector sampled from a normal distribution, $\text{perm}$ is a reshuffled channel index. Since $FFT$ and $iFFT$ are linear operations, this mix-up process can be equivalently simplified in the \textbf{Time Domain}:
\begin{align}
X' = X +\alpha \cdot X[\text{perm},:]&,  \notag\\
Y' = Y +\alpha \cdot Y[\text{perm},:]&
 \end{align}
We alternate training between real and synthetic data to preserve the spectral dependencies in real samples. This combines the advantages of data augmentation, such as improved generalization, while mitigating potential drawbacks like over-smoothing and training instability~\citep{ryu2024tf,alkhalifah2022tf}.












\section{Experiments}

\subsection{Setups}
\subsubsection{Implementation Details}
We apply our FDS method to two types of 3DGS: 
the original 3DGS, and 2DGS~\citep{huang20242d}. 
%
The number of iterations in our optimization 
process is 35,000.
We follow the default training configuration 
and apply our FDS method after 15,000 iterations,
then we add normal consistency loss for both
3DGS and 2DGS after 25000 iterations.
%
The weight for FDS, $\lambda_{fds}$, is set to 0.015,
the $\sigma$ is set to 23,
and the weight for normal consistency is set to 0.15
for all experiments. 
We removed the depth distortion loss in 2DGS 
because we found that it degrades its results in indoor scenes.
%
The Gaussian point cloud is initialized using Colmap
for all datasets.
%
%
We tested the impact of 
using Sea Raft~\citep{wang2025sea} and 
Raft\citep{teed2020raft} on FDS performance.
%
Due to the blurriness of the ScanNet dataset, 
additional prior constraints are required.
Thus, we incorporate normal prior supervision 
on the rendered normals 
in ScanNet (V2) dataset by default.
The normal prior is predicted by the Stable Normal 
model~\citep{ye2024stablenormal}
across all types of 3DGS.
%
The entire framework is implemented in 
PyTorch~\citep{paszke2019pytorch}, 
and all experiments are conducted on 
a single NVIDIA 4090D GPU.

\begin{figure}[t] \centering
    \makebox[0.16\textwidth]{\scriptsize Input}
    \makebox[0.16\textwidth]{\scriptsize 3DGS}
    \makebox[0.16\textwidth]{\scriptsize 2DGS}
    \makebox[0.16\textwidth]{\scriptsize 3DGS + FDS}
    \makebox[0.16\textwidth]{\scriptsize 2DGS + FDS}
    \makebox[0.16\textwidth]{\scriptsize GT (Depth)}

    \includegraphics[width=0.16\textwidth]{figure/fig3_img/compare3/gt_rgb/frame_00522.jpg}
    \includegraphics[width=0.16\textwidth]{figure/fig3_img/compare3/3DGS/frame_00522.jpg}
    \includegraphics[width=0.16\textwidth]{figure/fig3_img/compare3/2DGS/frame_00522.jpg}
    \includegraphics[width=0.16\textwidth]{figure/fig3_img/compare3/3DGS+FDS/frame_00522.jpg}
    \includegraphics[width=0.16\textwidth]{figure/fig3_img/compare3/2DGS+FDS/frame_00522.jpg}
    \includegraphics[width=0.16\textwidth]{figure/fig3_img/compare3/gt_depth/frame_00522.jpg} \\

    % \includegraphics[width=0.16\textwidth]{figure/fig3_img/compare1/gt_rgb/frame_00137.jpg}
    % \includegraphics[width=0.16\textwidth]{figure/fig3_img/compare1/3DGS/frame_00137.jpg}
    % \includegraphics[width=0.16\textwidth]{figure/fig3_img/compare1/2DGS/frame_00137.jpg}
    % \includegraphics[width=0.16\textwidth]{figure/fig3_img/compare1/3DGS+FDS/frame_00137.jpg}
    % \includegraphics[width=0.16\textwidth]{figure/fig3_img/compare1/2DGS+FDS/frame_00137.jpg}
    % \includegraphics[width=0.16\textwidth]{figure/fig3_img/compare1/gt_depth/frame_00137.jpg} \\

     \includegraphics[width=0.16\textwidth]{figure/fig3_img/compare2/gt_rgb/frame_00262.jpg}
    \includegraphics[width=0.16\textwidth]{figure/fig3_img/compare2/3DGS/frame_00262.jpg}
    \includegraphics[width=0.16\textwidth]{figure/fig3_img/compare2/2DGS/frame_00262.jpg}
    \includegraphics[width=0.16\textwidth]{figure/fig3_img/compare2/3DGS+FDS/frame_00262.jpg}
    \includegraphics[width=0.16\textwidth]{figure/fig3_img/compare2/2DGS+FDS/frame_00262.jpg}
    \includegraphics[width=0.16\textwidth]{figure/fig3_img/compare2/gt_depth/frame_00262.jpg} \\

    \includegraphics[width=0.16\textwidth]{figure/fig3_img/compare4/gt_rgb/frame00000.png}
    \includegraphics[width=0.16\textwidth]{figure/fig3_img/compare4/3DGS/frame00000.png}
    \includegraphics[width=0.16\textwidth]{figure/fig3_img/compare4/2DGS/frame00000.png}
    \includegraphics[width=0.16\textwidth]{figure/fig3_img/compare4/3DGS+FDS/frame00000.png}
    \includegraphics[width=0.16\textwidth]{figure/fig3_img/compare4/2DGS+FDS/frame00000.png}
    \includegraphics[width=0.16\textwidth]{figure/fig3_img/compare4/gt_depth/frame00000.png} \\

    \includegraphics[width=0.16\textwidth]{figure/fig3_img/compare5/gt_rgb/frame00080.png}
    \includegraphics[width=0.16\textwidth]{figure/fig3_img/compare5/3DGS/frame00080.png}
    \includegraphics[width=0.16\textwidth]{figure/fig3_img/compare5/2DGS/frame00080.png}
    \includegraphics[width=0.16\textwidth]{figure/fig3_img/compare5/3DGS+FDS/frame00080.png}
    \includegraphics[width=0.16\textwidth]{figure/fig3_img/compare5/2DGS+FDS/frame00080.png}
    \includegraphics[width=0.16\textwidth]{figure/fig3_img/compare5/gt_depth/frame00080.png} \\



    \caption{\textbf{Comparison of depth reconstruction on Mushroom and ScanNet datasets.} The original
    3DGS or 2DGS model equipped with FDS can remove unwanted floaters and reconstruct
    geometry more preciously.}
    \label{fig:compare}
\end{figure}


\subsubsection{Datasets and Metrics}

We evaluate our method for 3D reconstruction 
and novel view synthesis tasks on
\textbf{Mushroom}~\citep{ren2024mushroom},
\textbf{ScanNet (v2)}~\citep{dai2017scannet}, and 
\textbf{Replica}~\citep{replica19arxiv}
datasets,
which feature challenging indoor scenes with both 
sparse and dense image sampling.
%
The Mushroom dataset is an indoor dataset 
with sparse image sampling and two distinct 
camera trajectories. 
%
We train our model on the training split of 
the long capture sequence and evaluate 
novel view synthesis on the test split 
of the long capture sequences.
%
Five scenes are selected to evaluate our FDS, 
including "coffee room", "honka", "kokko", 
"sauna", and "vr room". 
%
ScanNet(V2)~\citep{dai2017scannet}  consists of 1,613 indoor scenes
with annotated camera poses and depth maps. 
%
We select 5 scenes from the ScanNet (V2) dataset, 
uniformly sampling one-tenth of the views,
following the approach in ~\citep{guo2022manhattan}.
To further improve the geometry rendering quality of 3DGS, 
%
Replica~\citep{replica19arxiv} contains small-scale 
real-world indoor scans. 
We evaluate our FDS on five scenes from 
Replica: office0, office1, office2, office3 and office4,
selecting one-tenth of the views for training.
%
The results for Replica are provided in the 
supplementary materials.
To evaluate the rendering quality and geometry 
of 3DGS, we report PSNR, SSIM, and LPIPS for 
rendering quality, along with Absolute Relative Distance 
(Abs Rel) as a depth quality metrics.
%
Additionally, for mesh evaluation, 
we use metrics including Accuracy, Completion, 
Chamfer-L1 distance, Normal Consistency, 
and F-scores.




\subsection{Results}
\subsubsection{Depth rendering and novel view synthesis}
The comparison results on Mushroom and 
ScanNet are presented in \tabref{tab:mushroom} 
and \tabref{tab:scannet}, respectively. 
%
Due to the sparsity of sampling 
in the Mushroom dataset,
challenges are posed for both GOF~\citep{yu2024gaussian} 
and PGSR~\citep{chen2024pgsr}, 
leading to their relative poor performance 
on the Mushroom dataset.
%
Our approach achieves the best performance 
with the FDS method applied during the training process.
The FDS significantly enhances the 
geometric quality of 3DGS on the Mushroom dataset, 
improving the "abs rel" metric by more than 50\%.
%
We found that Sea Raft~\citep{wang2025sea}
outperforms Raft~\citep{teed2020raft} on FDS, 
indicating that a better optical flow model 
can lead to more significant improvements.
%
Additionally, the render quality of RGB 
images shows a slight improvement, 
by 0.58 in 3DGS and 0.50 in 2DGS, 
benefiting from the incorporation of cross-view consistency in FDS. 
%
On the Mushroom
dataset, adding the FDS loss increases 
the training time by half an hour, which maintains the same
level as baseline.
%
Similarly, our method shows a notable improvement on the ScanNet dataset as well using Sea Raft~\citep{wang2025sea} Model. The "abs rel" metric in 2DGS is improved nearly 50\%. This demonstrates the robustness and effectiveness of the FDS method across different datasets.
%


% \begin{wraptable}{r}{0.6\linewidth} \centering
% \caption{\textbf{Ablation study on geometry priors.}} 
%         \label{tab:analysis_prior}
%         \resizebox{\textwidth}{!}{
\begin{tabular}{c| c c c c c | c c c c}

    \hline
     Method &  Acc$\downarrow$ & Comp $\downarrow$ & C-L1 $\downarrow$ & NC $\uparrow$ & F-Score $\uparrow$ &  Abs Rel $\downarrow$ &  PSNR $\uparrow$  & SSIM  $\uparrow$ & LPIPS $\downarrow$ \\ \hline
    2DGS&   0.1078&  0.0850&  0.0964&  0.7835&  0.5170&  0.1002&  23.56&  0.8166& 0.2730\\
    2DGS+Depth&   0.0862&  0.0702&  0.0782&  0.8153&  0.5965&  0.0672&  23.92&  0.8227& 0.2619 \\
    2DGS+MVDepth&   0.2065&  0.0917&  0.1491&  0.7832&  0.3178&  0.0792&  23.74&  0.8193& 0.2692 \\
    2DGS+Normal&   0.0939&  0.0637&  0.0788&  \textbf{0.8359}&  0.5782&  0.0768&  23.78&  0.8197& 0.2676 \\
    2DGS+FDS &  \textbf{0.0615} & \textbf{ 0.0534}& \textbf{0.0574}& 0.8151& \textbf{0.6974}&  \textbf{0.0561}&  \textbf{24.06}&  \textbf{0.8271}&\textbf{0.2610} \\ \hline
    2DGS+Depth+FDS &  0.0561 &  0.0519& 0.0540& 0.8295& 0.7282&  0.0454&  \textbf{24.22}& \textbf{0.8291}&\textbf{0.2570} \\
    2DGS+Normal+FDS &  \textbf{0.0529} & \textbf{ 0.0450}& \textbf{0.0490}& \textbf{0.8477}& \textbf{0.7430}&  \textbf{0.0443}&  24.10&  0.8283& 0.2590 \\
    2DGS+Depth+Normal &  0.0695 & 0.0513& 0.0604& 0.8540&0.6723&  0.0523&  24.09&  0.8264&0.2575\\ \hline
    2DGS+Depth+Normal+FDS &  \textbf{0.0506} & \textbf{0.0423}& \textbf{0.0464}& \textbf{0.8598}&\textbf{0.7613}&  \textbf{0.0403}&  \textbf{24.22}& 
    \textbf{0.8300}&\textbf{0.0403}\\
    
\bottomrule
\end{tabular}
}
% \end{wraptable}



The qualitative comparisons on the Mushroom and ScanNet dataset 
are illustrated in \figref{fig:compare}. 
%
%
As seen in the first row of \figref{fig:compare}, 
both the original 3DGS and 2DGS suffer from overfitting, 
leading to corrupted geometry generation. 
%
Our FDS effectively mitigates this issue by 
supervising the matching relationship between 
the input and sampled views, 
helping to recover the geometry.
%
FDS also improves the refinement of geometric details, 
as shown in other rows. 
By incorporating the matching prior through FDS, 
the quality of the rendered depth is significantly improved.
%

\begin{table}[t] \centering
\begin{minipage}[t]{0.96\linewidth}
        \captionof{table}{\textbf{3D Reconstruction 
        and novel view synthesis results on Mushroom dataset. * 
        Represents that FDS uses the Raft model.
        }}
        \label{tab:mushroom}
        \resizebox{\textwidth}{!}{
\begin{tabular}{c| c c c c c | c c c c c}
    \hline
     Method &  Acc$\downarrow$ & Comp $\downarrow$ & C-L1 $\downarrow$ & NC $\uparrow$ & F-Score $\uparrow$ &  Abs Rel $\downarrow$ &  PSNR $\uparrow$  & SSIM  $\uparrow$ & LPIPS $\downarrow$ & Time  $\downarrow$ \\ \hline

    % DN-splatter &   &  &  &  &  &  &  &  & \\
    GOF &  0.1812 & 0.1093 & 0.1453 & 0.6292 & 0.3665 & 0.2380  & 21.37  &  0.7762  & 0.3132  & $\approx$1.4h\\ 
    PGSR &  0.0971 & 0.1420 & 0.1196 & 0.7193 & 0.5105 & 0.1723  & 22.13  & 0.7773  & 0.2918  & $\approx$1.2h \\ \hline
    3DGS &   0.1167 &  0.1033&  0.1100&  0.7954&  0.3739&  0.1214&  24.18&  0.8392& 0.2511 &$\approx$0.8h \\
    3DGS + FDS$^*$ & 0.0569  & 0.0676 & 0.0623 & 0.8105 & 0.6573 & 0.0603 & 24.72  & 0.8489 & 0.2379 &$\approx$1.3h \\
    3DGS + FDS & \textbf{0.0527}  & \textbf{0.0565} & \textbf{0.0546} & \textbf{0.8178} & \textbf{0.6958} & \textbf{0.0568} & \textbf{24.76}  & \textbf{0.8486} & \textbf{0.2381} &$\approx$1.3h \\ \hline
    2DGS&   0.1078&  0.0850&  0.0964&  0.7835&  0.5170&  0.1002&  23.56&  0.8166& 0.2730 &$\approx$0.8h\\
    2DGS + FDS$^*$ &  0.0689 &  0.0646& 0.0667& 0.8042& 0.6582& 0.0589& 23.98&  0.8255&0.2621 &$\approx$1.3h\\
    2DGS + FDS &  \textbf{0.0615} & \textbf{ 0.0534}& \textbf{0.0574}& \textbf{0.8151}& \textbf{0.6974}&  \textbf{0.0561}&  \textbf{24.06}&  \textbf{0.8271}&\textbf{0.2610} &$\approx$1.3h \\ \hline
\end{tabular}
}
\end{minipage}\hfill
\end{table}

\begin{table}[t] \centering
\begin{minipage}[t]{0.96\linewidth}
        \captionof{table}{\textbf{3D Reconstruction 
        and novel view synthesis results on ScanNet dataset.}}
        \label{tab:scannet}
        \resizebox{\textwidth}{!}{
\begin{tabular}{c| c c c c c | c c c c }
    \hline
     Method &  Acc $\downarrow$ & Comp $\downarrow$ & C-L1 $\downarrow$ & NC $\uparrow$ & F-Score $\uparrow$ &  Abs Rel $\downarrow$ &  PSNR $\uparrow$  & SSIM  $\uparrow$ & LPIPS $\downarrow$ \\ \hline
    GOF & 1.8671  & 0.0805 & 0.9738 & 0.5622 & 0.2526 & 0.1597  & 21.55  & 0.7575  & 0.3881 \\
    PGSR &  0.2928 & 0.5103 & 0.4015 & 0.5567 & 0.1926 & 0.1661  & 21.71 & 0.7699  & 0.3899 \\ \hline

    3DGS &  0.4867 & 0.1211 & 0.3039 & 0.7342& 0.3059 & 0.1227 & 22.19& 0.7837 & 0.3907\\
    3DGS + FDS &  \textbf{0.2458} & \textbf{0.0787} & \textbf{0.1622} & \textbf{0.7831} & 
    \textbf{0.4482} & \textbf{0.0573} & \textbf{22.83} & \textbf{0.7911} & \textbf{0.3826} \\ \hline
    2DGS &  0.2658 & 0.0845 & 0.1752 & 0.7504& 0.4464 & 0.0831 & 22.59& 0.7881 & 0.3854\\
    2DGS + FDS &  \textbf{0.1457} & \textbf{0.0679} & \textbf{0.1068} & \textbf{0.7883} & 
    \textbf{0.5459} & \textbf{0.0432} & \textbf{22.91} & \textbf{0.7928} & \textbf{0.3800} \\ \hline
\end{tabular}
}
\end{minipage}\hfill
\end{table}


\begin{table}[t] \centering
\begin{minipage}[t]{0.96\linewidth}
        \captionof{table}{\textbf{Ablation study on geometry priors.}}
        \label{tab:analysis_prior}
        \resizebox{\textwidth}{!}{
\begin{tabular}{c| c c c c c | c c c c}

    \hline
     Method &  Acc$\downarrow$ & Comp $\downarrow$ & C-L1 $\downarrow$ & NC $\uparrow$ & F-Score $\uparrow$ &  Abs Rel $\downarrow$ &  PSNR $\uparrow$  & SSIM  $\uparrow$ & LPIPS $\downarrow$ \\ \hline
    2DGS&   0.1078&  0.0850&  0.0964&  0.7835&  0.5170&  0.1002&  23.56&  0.8166& 0.2730\\
    2DGS+Depth&   0.0862&  0.0702&  0.0782&  0.8153&  0.5965&  0.0672&  23.92&  0.8227& 0.2619 \\
    2DGS+MVDepth&   0.2065&  0.0917&  0.1491&  0.7832&  0.3178&  0.0792&  23.74&  0.8193& 0.2692 \\
    2DGS+Normal&   0.0939&  0.0637&  0.0788&  \textbf{0.8359}&  0.5782&  0.0768&  23.78&  0.8197& 0.2676 \\
    2DGS+FDS &  \textbf{0.0615} & \textbf{ 0.0534}& \textbf{0.0574}& 0.8151& \textbf{0.6974}&  \textbf{0.0561}&  \textbf{24.06}&  \textbf{0.8271}&\textbf{0.2610} \\ \hline
    2DGS+Depth+FDS &  0.0561 &  0.0519& 0.0540& 0.8295& 0.7282&  0.0454&  \textbf{24.22}& \textbf{0.8291}&\textbf{0.2570} \\
    2DGS+Normal+FDS &  \textbf{0.0529} & \textbf{ 0.0450}& \textbf{0.0490}& \textbf{0.8477}& \textbf{0.7430}&  \textbf{0.0443}&  24.10&  0.8283& 0.2590 \\
    2DGS+Depth+Normal &  0.0695 & 0.0513& 0.0604& 0.8540&0.6723&  0.0523&  24.09&  0.8264&0.2575\\ \hline
    2DGS+Depth+Normal+FDS &  \textbf{0.0506} & \textbf{0.0423}& \textbf{0.0464}& \textbf{0.8598}&\textbf{0.7613}&  \textbf{0.0403}&  \textbf{24.22}& 
    \textbf{0.8300}&\textbf{0.0403}\\
    
\bottomrule
\end{tabular}
}
\end{minipage}\hfill
\end{table}




\subsubsection{Mesh extraction}
To further demonstrate the improvement in geometry quality, 
we applied methods used in ~\citep{turkulainen2024dnsplatter} 
to extract meshes from the input views of optimized 3DGS. 
The comparison results are presented  
in \tabref{tab:mushroom}. 
With the integration of FDS, the mesh quality is significantly enhanced compared to the baseline, featuring fewer floaters and more well-defined shapes.
 %
% Following the incorporation of FDS, the reconstruction 
% results exhibit fewer floaters and more well-defined 
% shapes in the meshes. 
% Visualized comparisons
% are provided in the supplementary material.

% \begin{figure}[t] \centering
%     \makebox[0.19\textwidth]{\scriptsize GT}
%     \makebox[0.19\textwidth]{\scriptsize 3DGS}
%     \makebox[0.19\textwidth]{\scriptsize 3DGS+FDS}
%     \makebox[0.19\textwidth]{\scriptsize 2DGS}
%     \makebox[0.19\textwidth]{\scriptsize 2DGS+FDS} \\

%     \includegraphics[width=0.19\textwidth]{figure/fig4_img/compare1/gt02.png}
%     \includegraphics[width=0.19\textwidth]{figure/fig4_img/compare1/baseline06.png}
%     \includegraphics[width=0.19\textwidth]{figure/fig4_img/compare1/baseline_fds05.png}
%     \includegraphics[width=0.19\textwidth]{figure/fig4_img/compare1/2dgs04.png}
%     \includegraphics[width=0.19\textwidth]{figure/fig4_img/compare1/2dgs_fds03.png} \\

%     \includegraphics[width=0.19\textwidth]{figure/fig4_img/compare2/gt00.png}
%     \includegraphics[width=0.19\textwidth]{figure/fig4_img/compare2/baseline02.png}
%     \includegraphics[width=0.19\textwidth]{figure/fig4_img/compare2/baseline_fds01.png}
%     \includegraphics[width=0.19\textwidth]{figure/fig4_img/compare2/2dgs04.png}
%     \includegraphics[width=0.19\textwidth]{figure/fig4_img/compare2/2dgs_fds03.png} \\
      
%     \includegraphics[width=0.19\textwidth]{figure/fig4_img/compare3/gt05.png}
%     \includegraphics[width=0.19\textwidth]{figure/fig4_img/compare3/3dgs03.png}
%     \includegraphics[width=0.19\textwidth]{figure/fig4_img/compare3/3dgs_fds04.png}
%     \includegraphics[width=0.19\textwidth]{figure/fig4_img/compare3/2dgs02.png}
%     \includegraphics[width=0.19\textwidth]{figure/fig4_img/compare3/2dgs_fds01.png} \\

%     \caption{\textbf{Qualitative comparison of extracted mesh 
%     on Mushroom and ScanNet datasets.}}
%     \label{fig:mesh}
% \end{figure}












\subsection{Ablation study}


\textbf{Ablation study on geometry priors:} 
To highlight the advantage of incorporating matching priors, 
we incorporated various types of priors generated by different 
models into 2DGS. These include a monocular depth estimation
model (Depth Anything v2)~\citep{yang2024depth}, a two-view depth estimation 
model (Unimatch)~\citep{xu2023unifying}, 
and a monocular normal estimation model (DSINE)~\citep{bae2024rethinking}.
We adapt the scale and shift-invariant loss in Midas~\citep{birkl2023midas} for
monocular depth supervision and L1 loss for two-view depth supervison.
%
We use Sea Raft~\citep{wang2025sea} as our default optical flow model.
%
The comparison results on Mushroom dataset 
are shown in ~\tabref{tab:analysis_prior}.
We observe that the normal prior provides accurate shape information, 
enhancing the geometric quality of the radiance field. 
%
% In contrast, the monocular depth prior slightly increases 
% the 'Abs Rel' due to its ambiguous scale and inaccurate depth ordering.
% Moreover, the performance of monocular depth estimation 
% in the sauna scene is particularly poor, 
% primarily due to the presence of numerous reflective 
% surfaces and textureless walls, which limits the accuracy of monocular depth estimation.
%
The multi-view depth prior, hindered by the limited feature overlap 
between input views, fails to offer reliable geometric 
information. We test average "Abs Rel" of multi-view depth prior
, and the result is 0.19, which performs worse than the "Abs Rel" results 
rendered by original 2DGS.
From the results, it can be seen that depth order information provided by monocular depth improves
reconstruction accuracy. Meanwhile, our FDS achieves the best performance among all the priors, 
and by integrating all
three components, we obtained the optimal results.
%
%
\begin{figure}[t] \centering
    \makebox[0.16\textwidth]{\scriptsize RF (16000 iters)}
    \makebox[0.16\textwidth]{\scriptsize RF* (20000 iters)}
    \makebox[0.16\textwidth]{\scriptsize RF (20000 iters)  }
    \makebox[0.16\textwidth]{\scriptsize PF (16000 iters)}
    \makebox[0.16\textwidth]{\scriptsize PF (20000 iters)}


    % \includegraphics[width=0.16\textwidth]{figure/fig5_img/compare1/16000.png}
    % \includegraphics[width=0.16\textwidth]{figure/fig5_img/compare1/20000_wo_flow_loss.png}
    % \includegraphics[width=0.16\textwidth]{figure/fig5_img/compare1/20000.png}
    % \includegraphics[width=0.16\textwidth]{figure/fig5_img/compare1/16000_prior.png}
    % \includegraphics[width=0.16\textwidth]{figure/fig5_img/compare1/20000_prior.png}\\

    % \includegraphics[width=0.16\textwidth]{figure/fig5_img/compare2/16000.png}
    % \includegraphics[width=0.16\textwidth]{figure/fig5_img/compare2/20000_wo_flow_loss.png}
    % \includegraphics[width=0.16\textwidth]{figure/fig5_img/compare2/20000.png}
    % \includegraphics[width=0.16\textwidth]{figure/fig5_img/compare2/16000_prior.png}
    % \includegraphics[width=0.16\textwidth]{figure/fig5_img/compare2/20000_prior.png}\\

    \includegraphics[width=0.16\textwidth]{figure/fig5_img/compare3/16000.png}
    \includegraphics[width=0.16\textwidth]{figure/fig5_img/compare3/20000_wo_flow_loss.png}
    \includegraphics[width=0.16\textwidth]{figure/fig5_img/compare3/20000.png}
    \includegraphics[width=0.16\textwidth]{figure/fig5_img/compare3/16000_prior.png}
    \includegraphics[width=0.16\textwidth]{figure/fig5_img/compare3/20000_prior.png}\\
    
    \includegraphics[width=0.16\textwidth]{figure/fig5_img/compare4/16000.png}
    \includegraphics[width=0.16\textwidth]{figure/fig5_img/compare4/20000_wo_flow_loss.png}
    \includegraphics[width=0.16\textwidth]{figure/fig5_img/compare4/20000.png}
    \includegraphics[width=0.16\textwidth]{figure/fig5_img/compare4/16000_prior.png}
    \includegraphics[width=0.16\textwidth]{figure/fig5_img/compare4/20000_prior.png}\\

    \includegraphics[width=0.30\textwidth]{figure/fig5_img/bar.png}

    \caption{\textbf{The error map of Radiance Flow and Prior Flow.} RF: Radiance Flow, PF: Prior Flow, * means that there is no FDS loss supervision during optimization.}
    \label{fig:error_map}
\end{figure}




\textbf{Ablation study on FDS: }
In this section, we present the design of our FDS 
method through an ablation study on the 
Mushroom dataset to validate its effectiveness.
%
The optional configurations of FDS are outlined in ~\tabref{tab:ablation_fds}.
Our base model is the 2DGS equipped with FDS,
and its results are shown 
in the first row. The goal of this analysis 
is to evaluate the impact 
of various strategies on FDS sampling and loss design.
%
We observe that when we 
replace $I_i$ in \eqref{equ:mflow} with $C_i$, 
as shown in the second row, the geometric quality 
of 2DGS deteriorates. Using $I_i$ instead of $C_i$ 
help us to remove the floaters in $\bm{C^s}$, which are also 
remained in $\bm{C^i}$.
We also experiment with modifying the FDS loss. For example, 
in the third row, we use the neighbor 
input view as the sampling view, and replace the 
render result of neighbor view with ground truth image of its input view.
%
Due to the significant movement between images, the Prior Flow fails to accurately 
match the pixel between them, leading to a further degradation in geometric quality.
%
Finally, we attempt to fix the sampling view 
and found that this severely damaged the geometric quality, 
indicating that random sampling is essential for the stability 
of the mean error in the Prior flow.



\begin{table}[t] \centering

\begin{minipage}[t]{1.0\linewidth}
        \captionof{table}{\textbf{Ablation study on FDS strategies.}}
        \label{tab:ablation_fds}
        \resizebox{\textwidth}{!}{
\begin{tabular}{c|c|c|c|c|c|c|c}
    \hline
    \multicolumn{2}{c|}{$\mathcal{M}_{\theta}(X, \bm{C^s})$} & \multicolumn{3}{c|}{Loss} & \multicolumn{3}{c}{Metric}  \\
    \hline
    $X=C^i$ & $X=I^i$  & Input view & Sampled view     & Fixed Sampled view        & Abs Rel $\downarrow$ & F-score $\uparrow$ & NC $\uparrow$ \\
    \hline
    & \ding{51} &     &\ding{51}    &    &    \textbf{0.0561}        &  \textbf{0.6974}         & \textbf{0.8151}\\
    \hline
     \ding{51} &           &     &\ding{51}    &    &    0.0839        &  0.6242         &0.8030\\
     &  \ding{51} &   \ding{51}  &    &    &    0.0877       & 0.6091        & 0.7614 \\
      &  \ding{51} &    &    & \ding{51}    &    0.0724           & 0.6312          & 0.8015 \\
\bottomrule
\end{tabular}
}
\end{minipage}
\end{table}




\begin{figure}[htbp] \centering
    \makebox[0.22\textwidth]{}
    \makebox[0.22\textwidth]{}
    \makebox[0.22\textwidth]{}
    \makebox[0.22\textwidth]{}
    \\

    \includegraphics[width=0.22\textwidth]{figure/fig6_img/l1/rgb/frame00096.png}
    \includegraphics[width=0.22\textwidth]{figure/fig6_img/l1/render_rgb/frame00096.png}
    \includegraphics[width=0.22\textwidth]{figure/fig6_img/l1/render_depth/frame00096.png}
    \includegraphics[width=0.22\textwidth]{figure/fig6_img/l1/depth/frame00096.png}

    % \includegraphics[width=0.22\textwidth]{figure/fig6_img/l2/rgb/frame00112.png}
    % \includegraphics[width=0.22\textwidth]{figure/fig6_img/l2/render_rgb/frame00112.png}
    % \includegraphics[width=0.22\textwidth]{figure/fig6_img/l2/render_depth/frame00112.png}
    % \includegraphics[width=0.22\textwidth]{figure/fig6_img/l2/depth/frame00112.png}

    \caption{\textbf{Limitation of FDS.} }
    \label{fig:limitation}
\end{figure}


% \begin{figure}[t] \centering
%     \makebox[0.48\textwidth]{}
%     \makebox[0.48\textwidth]{}
%     \\
%     \includegraphics[width=0.48\textwidth]{figure/loss_Ignatius.pdf}
%     \includegraphics[width=0.48\textwidth]{figure/loss_family.pdf}
%     \caption{\textbf{Comparison the photometric error of Radiance Flow and Prior Flow:} 
%     We add FDS method after 2k iteration during training.
%     The results show
%     that:  1) The Prior Flow is more precise and 
%     robust than Radiance Flow during the radiance 
%     optimization; 2) After adding the FDS loss 
%     which utilize Prior 
%     flow to supervise the Radiance Flow at 2k iterations, 
%     both flow are more accurate, which lead to
%     a mutually reinforcing effects.(TODO fix it)} 
%     \label{fig:flowcompare}
% \end{figure}






\textbf{Interpretive Experiments: }
To demonstrate the mutual refinement of two flows in our FDS, 
For each view, we sample the unobserved 
views multiple times to compute the mean error 
of both Radiance Flow and Prior Flow. 
We use Raft~\citep{teed2020raft} as our default optical flow model
for visualization.
The ground truth flow is calculated based on 
~\eref{equ:flow_pose} and ~\eref{equ:flow} 
utilizing ground truth depth in dataset.
We introduce our FDS loss after 16000 iterations during 
optimization of 2DGS.
The error maps are shown in ~\figref{fig:error_map}.
Our analysis reveals that Radiance Flow tends to 
exhibit significant geometric errors, 
whereas Prior Flow can more accurately estimate the geometry,
effectively disregarding errors introduced by floating Gaussian points. 

%





\subsection{Limitation and further work}

Firstly, our FDS faces challenges in scenes with 
significant lighting variations between different 
views, as shown in the lamp of first row in ~\figref{fig:limitation}. 
%
Incorporating exposure compensation into FDS could help address this issue. 
%
 Additionally, our method struggles with 
 reflective surfaces and motion blur,
 leading to incorrect matching. 
 %
 In the future, we plan to explore the potential 
 of FDS in monocular video reconstruction tasks, 
 using only a single input image at each time step.
 


\section{Conclusions}
In this paper, we propose Flow Distillation Sampling (FDS), which
leverages the matching prior between input views and 
sampled unobserved views from the pretrained optical flow model, to improve the geometry quality
of Gaussian radiance field. 
Our method can be applied to different approaches (3DGS and 2DGS) to enhance the geometric rendering quality of the corresponding neural radiance fields.
We apply our method to the 3DGS-based framework, 
and the geometry is enhanced on the Mushroom, ScanNet, and Replica datasets.

\section*{Acknowledgements} This work was supported by 
National Key R\&D Program of China (2023YFB3209702), 
the National Natural Science Foundation of 
China (62441204, 62472213), and Gusu 
Innovation \& Entrepreneurship Leading Talents Program (ZXL2024361)
\section{Discussion and Ablation Study}
\label{sec:ablations}
We provide an in-depth analysis of our optimization setting, the impact of different loss functions, and the number of tolerances sampled during optimization. This analysis aims to offer more guidance for future work.

%--------------------------------------------------------
\subsection{Partially Tolerance Optimization}
In deep optics, the optimization of optical and decoder parameters may vary~\cite{tseng2021differentiable}. Tolerance-aware optimization adds further complexity by introducing random perturbations in each forward pass. To delve into the distinctions between these components, we selectively control the optimization: (a) optimizing the optics, and (b) only the computational decoder. Through experiments, we find that the deep optics achieves improvement of robustness only when the optics and decoder are jointly optimized simultaneously, shown in \cref{tab:partly-tolerance}. 
Optimizing only the decoder can enhance the decoder's tolerance-aware ability, but still fails to fully address the impact of random tolerances. Additionally, when optimizing the optics alone, the global influence of optical parameters on the subsequent decoder, along with the limited number of optical parameters, makes the optimization challenging and risks compromising the results of the initial pre-training stage. Only through the joint optimization of the optical system and decoder can we achieve a more stable optical system with enhanced pairing capability from the decoder, significantly improving the robustness of deep optics against potential tolerances.

%--------------------------------------------------------

\begin{table}
\centering
\caption{\emph{Optics/Decoder-only} means that only optimize the optics/decoder part of parameters during tolerance optimization and \emph{Both} means optimized both parts. The results are average of 100 times of random sampled tolerances experiments (excluding Spot Size).}
\begin{tabular}{c|c c c c}
\hline
 & Optics-only & Decoder-only & Both \\
\hline
PSNR$\uparrow$ & 22.06 & 25.94 & \bf{28.08} \\
SSIM$\uparrow$ & 0.376 & 0.734 & \bf{0.845} \\
LPIPS$\downarrow$ & 0.664 & 0.379 & \bf{0.225} \\
Spot Size ($\mu m$)$\downarrow$ & 34.2 & \textbf{14.2} & 38.7 \\
\hline
\end{tabular}
\label{tab:partly-tolerance}
\end{table}

%--------------------------------------------------------
\subsection{Analysis of Loss Function Impact}
To analyze the impact of the two loss functions, we conduct ablation studies on Spot loss and PSF loss. The experimental results are shown in \cref{tab:loss_comparison}. The results demonstrate that the proposed Spot loss and PSF loss significantly improve the performance. It is worth noting that Spot loss and PSF similarity loss can only be used together to maximize the performance of tolerance-aware optimization. The two losses play different roles respectively, the Spot loss is helpful to ensure that the overall imaging quality of the optics is not seriously degraded during the tolerance optimization, while the PSF loss is able to significantly improve the robustness of the optics to random tolerances on this basis. Without the constraint of Spot loss, PSF loss can lead to significant degradation in the imaging performance of the optical system, thereby impacting overall tolerance optimization. As shown in \cref{tab:loss_comparison}, the spot size is extremely large, indicating very low imaging quality.
\begin{figure}[t]
  \centering
   \includegraphics[width=1.0\linewidth]{figures/perb_num_ablation.pdf}

   \caption{Ablation study on the number of sampled tolerances pattern in every iteration. Average PSNR, SSIM, and LPIPS of 100 times random tolerances test for different sampling numbers are shown, along with the PSFs comparison. Scale bar: $15\mu m$ .}
   \label{fig:perb_num}
\end{figure}

\begin{table}
\centering
\caption{Ablation study on Spot loss and PSF loss, each incorporating basic image quality loss. The results represent the averages from 100 random sampling tolerance experiments (excluding Spot Size).}
\begin{tabular}{c|c c c c}
\hline
 & - & $\mathcal{L}_{\text{Spot}}$ & $\mathcal{L}_{\text{PSF}}$ & $\mathcal{L}_{\text{Spot}}\&\mathcal{L}_{\text{PSF}}$ \\
\hline
PSNR$\uparrow$ & 24.83 & 26.63 & 20.49 & \bf{28.08} \\
SSIM$\uparrow$ & 0.738 & 0.782 & 0.601 & \bf{0.850} \\
LPIPS$\downarrow$ & 0.378 & 0.248 & 0.529 & \bf{0.225} \\
Spot Size ($\mu m$)$\downarrow$ & 52.6 & \bf{12.9} & 637.9 & 38.7 \\
\hline
\end{tabular}
\label{tab:loss_comparison}
\end{table}

%--------------------------------------------------------
\subsection{Number of Sampled Tolerance Patterns}
In the tolerance-aware optimization process, the number of sampled tolerance patterns is a critical hyperparameter that significantly impacts the performance. If the number is too low, the optimization may stuck in local optima. However, increasing the sampled number leads to a higher consumption of GPU memory. Therefore, selecting an appropriate sample number is of great importance to balance the trade-off between optimization effectiveness and memory usage. Therefore, we employ different number of sampled tolerance pattern and conduct quantitative analysis to determine a reasonable lower limit, see in \cref{fig:perb_num}. This experiment significantly streamlines subsequent research efforts.

\paragraph{Summary}
Our findings provide significant insights into the influence of correctness, explanations, and refinement on evaluation accuracy and user trust in AI-based planners. 
In particular, the findings are three-fold: 
(1) The \textbf{correctness} of the generated plans is the most significant factor that impacts the evaluation accuracy and user trust in the planners. As the PDDL solver is more capable of generating correct plans, it achieves the highest evaluation accuracy and trust. 
(2) The \textbf{explanation} component of the LLM planner improves evaluation accuracy, as LLM+Expl achieves higher accuracy than LLM alone. Despite this improvement, LLM+Expl minimally impacts user trust. However, alternative explanation methods may influence user trust differently from the manually generated explanations used in our approach.
% On the other hand, explanations may help refine the trust of the planner to a more appropriate level by indicating planner shortcomings.
(3) The \textbf{refinement} procedure in the LLM planner does not lead to a significant improvement in evaluation accuracy; however, it exhibits a positive influence on user trust that may indicate an overtrust in some situations.
% This finding is aligned with prior works showing that iterative refinements based on user feedback would increase user trust~\cite{kunkel2019let, sebo2019don}.
Finally, the propensity-to-trust analysis identifies correctness as the primary determinant of user trust, whereas explanations provided limited improvement in scenarios where the planner's accuracy is diminished.

% In conclusion, our results indicate that the planner's correctness is the dominant factor for both evaluation accuracy and user trust. Therefore, selecting high-quality training data and optimizing the training procedure of AI-based planners to improve planning correctness is the top priority. Once the AI planner achieves a similar correctness level to traditional graph-search planners, strengthening its capability to explain and refine plans will further improve user trust compared to traditional planners.

\paragraph{Future Research} Future steps in this research include expanding user studies with larger sample sizes to improve generalizability and including additional planning problems per session for a more comprehensive evaluation. Next, we will explore alternative methods for generating plan explanations beyond manual creation to identify approaches that more effectively enhance user trust. 
Additionally, we will examine user trust by employing multiple LLM-based planners with varying levels of planning accuracy to better understand the interplay between planning correctness and user trust. 
Furthermore, we aim to enable real-time user-planner interaction, allowing users to provide feedback and refine plans collaboratively, thereby fostering a more dynamic and user-centric planning process.

\section*{Impact Statement}
\system advances cost-efficient AI by demonstrating how small on-device language models can collaborate with powerful cloud-hosted models to perform data-intensive reasoning. By reducing reliance on expensive remote inference, \system makes advanced AI more accessible and sustainable. This has broad societal implications, including lowering barriers to AI adoption and enhancing data privacy by keeping more computations local. However, careful consideration must be given to potential biases in small models and the security risks of local code execution. 

\bibliography{main}
\bibliographystyle{icml2025}

\section{Secure Token Pruning Protocols}
\label{app:a}
We detail the encrypted token pruning protocols $\Pi_{prune}$ in Figure \ref{fig:protocol-prune} and $\Pi_{mask}$ in Figure \ref{fig:protocol-mask} in this section.

%Optionally include supplemental material (complete proofs, additional experiments and plots) in appendix.
%All such materials \textbf{SHOULD be included in the main submission.}
\begin{figure}[h]
%vspace{-0.2in}
\begin{protocolbox}
\noindent
\textbf{Parties:} Server $P_0$, Client $P_1$.

\textbf{Input:} $P_0$ and $P_1$ holds $\{ \left \langle Att \right \rangle_{0}^{h}, \left \langle Att \right \rangle_{1}^{h}\}_{h=0}^{H-1} \in \mathbb{Z}_{2^{\ell}}^{n\times n}$ and $\left \langle x \right \rangle_{0}, \left \langle x \right \rangle_{1} \in \mathbb{Z}_{2^{\ell}}^{n\times D}$ respectively, where H is the number of heads, n is the number of input tokens and D is the embedding dimension of tokens. Additionally, $P_1$ holds a threshold $\theta \in \mathbb{Z}_{2^{\ell}}$.

\textbf{Output:} $P_0$ and $P_1$ get $\left \langle y \right \rangle_{0}, \left \langle y \right \rangle_{1} \in \mathbb{Z}_{2^{\ell}}^{n'\times D}$, respectively, where $y=\mathsf{Prune}(x)$ and $n'$ is the number of remaining tokens.

\noindent\rule{13.2cm}{0.1pt} % This creates the horizontal line
\textbf{Protocol:}
\begin{enumerate}[label=\arabic*:, leftmargin=*]
    \item For $h \in [H]$, $P_0$ and $P_1$ compute locally with input $\left \langle Att \right \rangle^{h}$, and learn the importance score in each head $\left \langle s \right \rangle^{h} \in \mathbb{Z}_{2^{\ell}}^{n} $, where $\left \langle s \right \rangle^{h}[j] = \frac{1}{n} \sum_{i=0}^{n-1} \left \langle Att \right \rangle^{h}[i,j]$.
    \item $P_0$ and $P_1$ compute locally with input $\{ \left \langle s \right \rangle^{i} \in \mathbb{Z}_{2^{\ell}}^{n}  \}_{i=0}^{H-1}$, and learn the final importance score $\left \langle S \right \rangle \in \mathbb{Z}_{2^{\ell}}^{n}$ for each token, where  $\left \langle S \right \rangle[i] = \frac{1}{H} \sum_{h=0}^{H-1} \left \langle s \right \rangle^{h}[i]$.
    \item  For $i \in [n]$, $P_0$ and $P_1$ invoke $\Pi_{CMP}$ with inputs  $\left \langle S \right \rangle$ and $ \theta $, and learn  $\left \langle M \right \rangle \in \mathbb{Z}_{2^{\ell}}^{n}$, such that$\left \langle M \right \rangle[i] = \Pi_{CMP}(\left \langle S \right \rangle[i] - \theta) $, where: \\
    $M[i] = \begin{cases}
        1  &\text{if}\ S[i] > \theta, \\
        0  &\text{otherwise}.
            \end{cases} $
    % \item If the pruning location is insensitive, $P_0$ and $P_1$ learn real mask $M$ instead of shares $\left \langle M \right \rangle$. $P_0$ and $P_1$ compute $\left \langle y \right \rangle$ with input $\left \langle x \right \rangle$ and $M$, where  $\left \langle x \right \rangle[i]$ is pruned if $M[i]$ is $0$.
    \item $P_0$ and $P_1$ invoke $\Pi_{mask}$ with inputs  $\left \langle x \right \rangle$ and pruning mask $\left \langle M \right \rangle$, and set outputs as $\left \langle y \right \rangle$.
\end{enumerate}
\end{protocolbox}
\setlength{\abovecaptionskip}{-1pt} % Reduces space above the caption
\caption{Secure Token Pruning Protocol $\Pi_{prune}$.}
\label{fig:protocol-prune}
\end{figure}




\begin{figure}[h]
\begin{protocolbox}
\noindent
\textbf{Parties:} Server $P_0$, Client $P_1$.

\textbf{Input:} $P_0$ and $P_1$ hold $\left \langle x \right \rangle_{0}, \left \langle x \right \rangle_{1} \in \mathbb{Z}_{2^{\ell}}^{n\times D}$ and  $\left \langle M \right \rangle_{0}, \left \langle M \right \rangle_{1} \in \mathbb{Z}_{2^{\ell}}^{n}$, respectively, where n is the number of input tokens and D is the embedding dimension of tokens.

\textbf{Output:} $P_0$ and $P_1$ get $\left \langle y \right \rangle_{0}, \left \langle y \right \rangle_{1} \in \mathbb{Z}_{2^{\ell}}^{n'\times D}$, respectively, where $y=\mathsf{Prune}(x)$ and $n'$ is the number of remaining tokens.

\noindent\rule{13.2cm}{0.1pt} % This creates the horizontal line
\textbf{Protocol:}
\begin{enumerate}[label=\arabic*:, leftmargin=*]
    \item For $i \in [n]$, $P_0$ and $P_1$ set $\left \langle M \right \rangle$ to the MSB of $\left \langle x \right \rangle$ and learn the masked tokens $\left \langle \Bar{x} \right \rangle \in Z_{2^{\ell}}^{n\times D}$, where
    $\left \langle \Bar{x}[i] \right \rangle = \left \langle x[i] \right \rangle + (\left \langle M[i] \right \rangle << f)$ and $f$ is the fixed-point precision.
    \item $P_0$ and $P_1$ compute the sum of $\{\Pi_{B2A}(\left \langle M \right \rangle[i]) \}_{i=0}^{n-1}$, and learn the number of remaining tokens $n'$ and the number of tokens to be pruned $m$, where $m = n-n'$.
    \item For $k\in[m]$, for $i\in[n-k-1]$, $P_0$ and $P_1$ invoke $\Pi_{msb}$ to learn the highest bit of $\left \langle \Bar{x}[i] \right \rangle$, where $b=\mathsf{MSB}(\Bar{x}[i])$. With the highest bit of $\Bar{x}[i]$, $P_0$ and $P_1$ perform a oblivious swap between $\Bar{x}[i]$ and $\Bar{x}[i+1]$:
    $\begin{cases}
        \Tilde{x}[i] = b\cdot \Bar{x}[i] + (1-b)\cdot \Bar{x}[i+1] \\
        \Tilde{x}[i+1] = b\cdot \Bar{x}[i+1] + (1-b)\cdot \Bar{x}[i]
    \end{cases} $ \\
    $P_0$ and $P_1$ learn the swapped token sequence $\left \langle \Tilde{x} \right \rangle$.
    \item $P_0$ and $P_1$ truncate $\left \langle \Tilde{x} \right \rangle$ locally by keeping the first $n'$ tokens, clear current MSB (all remaining token has $1$ on the MSB), and set outputs as $\left \langle y \right \rangle$.
\end{enumerate}
\end{protocolbox}
\setlength{\abovecaptionskip}{-1pt} % Reduces space above the caption
\caption{Secure Mask Protocol $\Pi_{mask}$.}
\label{fig:protocol-mask}
%\vspace{-0.2in}
\end{figure}

% \begin{wrapfigure}{r}{0.35\textwidth}  % 'r' for right, and the width of the figure area
%   \centering
%   \includegraphics[width=0.35\textwidth]{figures/msb.pdf}
%   \caption{Runtime of $\Pi_{prune}$ and $\Pi_{mask}$ in different layers. We compare different secure pruning strategies based on the BERT Base model.}
%   \label{fig:msb}
%   \vspace{-0.1in}
% \end{wrapfigure}

% \begin{figure}[h]  % 'r' for right, and the width of the figure area
%   \centering
%   \includegraphics[width=0.4\textwidth]{figures/msb.pdf}
%   \caption{Runtime of $\Pi_{prune}$ and $\Pi_{mask}$ in different layers. We compare different secure pruning strategies based on the BERT Base model.}
%   \label{fig:msb}
%   % \vspace{-0.1in}
% \end{figure}

\textbf{Complexity of $\Pi_{mask}$.} The complexity of the proposed $\Pi_{mask}$ mainly depends on the number of oblivious swaps. To prune $m$ tokens out of $n$ input tokens, $O(mn)$ swaps are needed. Since token pruning is performed progressively, only a small number of tokens are pruned at each layer, which makes $\Pi_{mask}$ efficient during runtime. Specifically, for a BERT base model with 128 input tokens, the pruning protocol only takes $\sim0.9$s on average in each layer. An alternative approach is to invoke an oblivious sort algorithm~\citep{bogdanov2014swap2,pang2023bolt} on $\left \langle \Bar{x} \right \rangle$. However, this approach is less efficient because it blindly sort the whole token sequence without considering $m$. That is, even if only $1$ token needs to be pruned, $O(nlog^{2}n)\sim O(n^2)$ oblivious swaps are needed, where as the proposed $\Pi_{mask}$ only need $O(n)$ swaps. More generally, for an $\ell$-layer Transformer with a total of $m$ tokens pruned, the overall time complexity using the sort strategy would be $O(\ell n^2)$ while using the swap strategy remains an overall complexity of $O(mn).$ Specifically, using the sort strategy to prune tokens in one BERT Base model layer can take up to $3.8\sim4.5$ s depending on the sorting algorithm used. In contrast, using the swap strategy only needs $0.5$ s. Moreover, alternative to our MSB strategy, one can also swap the encrypted mask along with the encrypted token sequence. However, we find that this doubles the number of swaps needed, and thus is less efficient the our MSB strategy, as is shown in Figure \ref{fig:msb}.

\section{Existing Protocols}
\label{app:protocol}
\noindent\textbf{Existing Protocols Used in Our Private Inference.}  In our private inference framework, we reuse several existing cryptographic protocols for basic computations. $\Pi_{MatMul}$ \citep{pang2023bolt} processes two ASS matrices and outputs their product in SS form. For non-linear computations, protocols $\Pi_{SoftMax}, \Pi_{GELU}$, and $\Pi_{LayerNorm}$\citep{lu2023bumblebee, pang2023bolt} take a secret shared tensor and return the result of non-linear functions in ASS. Basic protocols from~\citep{rathee2020cryptflow2, rathee2021sirnn} are also utilized. $\Pi_{CMP}$\citep{EzPC}, for example, inputs ASS values and outputs a secret shared comparison result, while $\Pi_{B2A}$\citep{EzPC} converts secret shared Boolean values into their corresponding arithmetic values.

\section{Polynomial Reduction for Non-linear Functions}
\label{app:b}
The $\mathsf{SoftMax}$ and $\mathsf{GELU}$ functions can be approximated with polynomials. High-degree polynomials~\citep{lu2023bumblebee, pang2023bolt} can achieve the same accuracy as the LUT-based methods~\cite{hao2022iron-iron}. While these polynomial approximations are more efficient than look-up tables, they can still incur considerable overheads. Reducing the high-degree polynomials to the low-degree ones for the less important tokens can imporve efficiency without compromising accuracy. The $\mathsf{SoftMax}$ function is applied to each row of an attention map. If a token is to be reduced, the corresponding row will be computed using the low-degree polynomial approximations. Otherwise, the corresponding row will be computed accurately via a high-degree one. That is if $M_{\beta}'[i] = 1$, $P_0$ and $P_1$ uses high-degree polynomials to compute the $\mathsf{SoftMax}$ function on token $x[i]$:
\begin{equation}
\mathsf{SoftMax}_{i}(x) = \frac{e^{x_i}}{\sum_{j\in [d]}e^{x_j}}
\end{equation}
where $x$ is a input vector of length $d$ and the exponential function is computed via a polynomial approximation. For the $\mathsf{SoftMax}$ protocol, we adopt a similar strategy as~\citep{kim2021ibert, hao2022iron-iron}, where we evaluate on the normalized inputs $\mathsf{SoftMax}(x-max_{i\in [d]}x_i)$. Different from~\citep{hao2022iron-iron}, we did not used the binary tree to find max value in the given vector. Instead, we traverse through the vector to find the max value. This is because each attention map is computed independently and the binary tree cannot be re-used. If $M_{\beta}[i] = 0$, $P_0$ and $P_1$ will approximate the $\mathsf{SoftMax}$ function with low-degree polynomial approximations. We detail how $\mathsf{SoftMax}$ can be approximated as follows:
\begin{equation}
\label{eq:app softmax}
\mathsf{ApproxSoftMax}_{i}(x) = \frac{\mathsf{ApproxExp}(x_i)}{\sum_{j\in [d]}\mathsf{ApproxExp}(x_j)}
\end{equation}
\begin{equation}
\mathsf{ApproxExp}(x)=\begin{cases}
    0  &\text{if}\ x \leq T \\
    (1+ \frac{x}{2^n})^{2^n} &\text{if}\ x \in [T,0]\\
\end{cases}
\end{equation}
where the $2^n$-degree Taylor series is used to approximate the exponential function and $T$ is the clipping boundary. The value $n$ and $T$ determines the accuracy of above approximation. With $n=6$ and $T=-13$, the approximation can achieve an average error within $2^{-10}$~\citep{lu2023bumblebee}. For low-degree polynomial approximation, $n=3$ is used in the Taylor series.

Similarly, $P_0$ or $P_1$ can decide whether or not to approximate the $\mathsf{GELU}$ function for each token. If $M_{\beta}[i] = 1$, $P_0$ and $P_1$ use high-degree polynomials~\citep{lu2023bumblebee} to compute the $\mathsf{GELU}$ function on token $x[i]$ with high-degree polynomial:
% \begin{equation}
% \mathsf{GELU}(x) = 0.5x(1+\mathsf{Tanh}(\sqrt{2/\pi}(x+0.044715x^3)))
% \end{equation}
% where the $\mathsf{Tanh}$ and square root function are computed via a OT-based lookup-table.

\begin{equation}
\label{eq:app gelu}
\mathsf{ApproxGELU}(x)=\begin{cases}
    0  &\text{if}\ x \leq -5 \\
    P^3(x), &\text{if}\ -5 < x \leq -1.97 \\
    P^6(x), &\text{if}\ -1.97 < x \leq 3  \\
    x, &\text{if}\ x >3 \\
\end{cases}
\end{equation}
where $P^3(x)$ and $P^6(x)$ are degree-3 and degree-6 polynomials respectively. The detailed coefficient for the polynomial is: 
\begin{equation*}
    P^3(x) = -0.50540312 -  0.42226581x - 0.11807613x^2 - 0.01103413x^3
\end{equation*}
, and
\begin{equation*}
    P^6(x) = 0.00852632 + 0.5x + 0.36032927x^2 - 0.03768820x^4 + 0.00180675x^6
\end{equation*}

For BOLT baseline, we use another high-degree polynomial to compute the $\mathsf{GELU}$ function.

\begin{equation}
\label{eq:app gelu}
\mathsf{ApproxGELU}(x)=\begin{cases}
    0  &\text{if}\ x < -2.7 \\
    P^4(x), &\text{if}\   |x| \leq 2.7 \\
    x, &\text{if}\ x >2.7 \\
\end{cases}
\end{equation}
We use the same coefficients for $P^4(x)$ as BOLT~\citep{pang2023bolt}.

\begin{figure}[h]
 % \vspace{-0.1in}
    \centering
    \includegraphics[width=1\linewidth]{figures/bumble.pdf}
    % \captionsetup{skip=2pt}
    % \vspace{-0.1in}
    \caption{Comparison with prior works on the BERT model. The input has 128 tokens.}
    \label{fig:bumble}
\end{figure}

If $M_{\beta}'[i] = 0$, $P_0$ and $P_1$ will use low-degree 
polynomial approximation to compute the $\mathsf{GELU}$ function instead. Encrypted polynomial reduction leverages low-degree polynomials to compute non-linear functions for less important tokens. For the $\mathsf{GELU}$ function, the following degree-$2$ polynomial~\cite{kim2021ibert} is used:
\begin{equation*}
\mathsf{ApproxGELU}(x)=\begin{cases}
    0  &\text{if}\ x <  -1.7626 \\
    0.5x+0.28367x^2, &\text{if}\ x \leq |1.7626| \\
    x, &\text{if}\ x > 1.7626\\
\end{cases}
\end{equation*}


\section{Comparison with More Related Works.}
\label{app:c}
\textbf{Other 2PC frameworks.} The primary focus of CipherPrune is to accelerate the private Transformer inference in the 2PC setting. As shown in Figure \ref{fig:bumble}, CipherPrune can be easily extended to other 2PC private inference frameworks like BumbleBee~\citep{lu2023bumblebee}. We compare CipherPrune with BumbleBee and IRON on BERT models. We test the performance in the same LAN setting as BumbleBee with 1 Gbps bandwidth and 0.5 ms of ping time. CipherPrune achieves more than $\sim 60 \times$ speed up over BOLT and $4.3\times$ speed up over BumbleBee.

\begin{figure}[t]
 % \vspace{-0.1in}
    \centering
    \includegraphics[width=1\linewidth]{figures/pumab.pdf}
    % \captionsetup{skip=2pt}
    % \vspace{-0.1in}
    \caption{Comparison with MPCFormer and PUMA on the BERT models. The input has 128 tokens.}
    \label{fig:pumab}
\end{figure}

\begin{figure}[h]
 % \vspace{-0.1in}
    \centering
    \includegraphics[width=1\linewidth]{figures/pumag.pdf}
    % \captionsetup{skip=2pt}
    % \vspace{-0.1in}
    \caption{Comparison with MPCFormer and PUMA on the GPT2 models. The input has 128 tokens. The polynomial reduction is not used.}
    \label{fig:pumag}
\end{figure}

\textbf{Extension to 3PC frameworks.} Additionally, we highlight that CipherPrune can be also extended to the 3PC frameworks like MPCFormer~\citep{li2022mpcformer} and PUMA~\citep{dong2023puma}. This is because CipherPrune is built upon basic primitives like comparison and Boolean-to-Arithmetic conversion. We compare CipherPrune with MPCFormer and PUMA on both the BERT and GPT2 models. CipherPrune has a $6.6\sim9.4\times$ speed up over MPCFormer and $2.8\sim4.6\times$ speed up over PUMA on the BERT-Large and GPT2-Large models.


\section{Communication Reduction in SoftMax and GELU.}
\label{app:e}

\begin{figure}[h]
    \centering
    \includegraphics[width=0.9\linewidth]{figures/layerwise.pdf}
    \caption{Toy example of two successive Transformer layers. In layer$_i$, the SoftMax and Prune protocol have $n$ input tokens. The number of input tokens is reduced to $n'$ for the Linear layers, LayerNorm and GELU in layer$_i$ and SoftMax in layer$_{i+1}$.}
    \label{fig:layer}
\end{figure}

\begin{table*}[h]
\captionsetup{skip=2pt}
\centering
\scriptsize
\caption{Communication cost (in MB) of the SoftMax and GELU protocol in each Transformer layer.}
\begin{tblr}{
    colspec = {c |c c c c c c c c c c c c},
    row{1} = {font=\bfseries},
    row{2-Z} = {rowsep=1pt},
    % row{4} = {bg=LightBlue},
    colsep = 2.5pt,
    }
\hline
\textbf{Layer Index} & \textbf{0}  & \textbf{1}  & \textbf{2} & \textbf{3} & \textbf{4} & \textbf{5} & \textbf{6} & \textbf{7} & \textbf{8} & \textbf{9} & \textbf{10} & \textbf{11} \\
\hline
Softmax & 642.19 & 642.19 & 642.19 & 642.19 & 642.19 & 642.19 & 642.19 & 642.19 & 642.19 & 642.19 & 642.19 & 642.19 \\
Pruned Softmax & 642.19 & 129.58 & 127.89 & 119.73 & 97.04 & 71.52 & 43.92 & 21.50 & 10.67 & 6.16 & 4.65 & 4.03 \\
\hline
GELU & 698.84 & 698.84 & 698.84 & 698.84 & 698.84 & 698.84 & 698.84 & 698.84 & 698.84 & 698.84 & 698.84 & 698.84\\
Pruned GELU  & 325.10 & 317.18 & 313.43 & 275.94 & 236.95 & 191.96 & 135.02 & 88.34 & 46.68 & 16.50 & 5.58 & 5.58\\
\hline
\end{tblr}
\label{tab:layer}
\end{table*}

{
In Figure \ref{fig:layer}, we illustrate why CipherPrune can reduce the communication overhead of both  SoftMax and GELU. Suppose there are $n$ tokens in $layer_i$. Then, the SoftMax protocol in the attention module has a complexity of $O(n^2)$. CipherPrune's token pruning protocol is invoked to select $n'$ tokens out of all $n$ tokens, where $m=n-n'$ is the number of tokens that are removed. The overhead of the GELU function in $layer_i$, i.e., the current layer, has only $O(n')$ complexity (which should be $O(n)$ without token pruning). The complexity of the SoftMax function in $layer_{i+1}$, i.e., the following layer, is reduced to $O(n'^2)$ (which should be $O(n^2)$ without token pruning). The SoftMax protocol has quadratic complexity with respect to the token number and the GELU protocol has linear complexity. Therefore, CipherPrune can reduce the overhead of both the GELU protocol and the SoftMax protocols by reducing the number of tokens. In Table \ref{tab:layer}, we provide detailed layer-wise communication cost of the GELU and the SoftMax protocol. Compared to the unpruned baseline, CipherPrune can effectively reduce the overhead of the GELU and the SoftMax protocols layer by layer.
}

\section{Analysis on Layer-wise redundancy.}
\label{app:f}

\begin{figure}[h]
    \centering
    \includegraphics[width=0.9\linewidth]{figures/layertime0.pdf}
    \caption{The number of pruned tokens and pruning protocol runtime in different layers in the BERT Base model. The results are averaged across 128 QNLI samples.}
    \label{fig:layertime}
\end{figure}

{
In Figure \ref{fig:layertime}, we present the number of pruned tokens and the runtime of the pruning protocol for each layer in the BERT Base model. The number of pruned tokens per layer was averaged across 128 QNLI samples, while the pruning protocol runtime was measured over 10 independent runs. The mean token count for the QNLI samples is 48.5. During inference with BERT Base, input sequences with fewer tokens are padded to 128 tokens using padding tokens. Consistent with prior token pruning methods in plaintext~\citep{goyal2020power}, a significant number of padding tokens are removed at layer 0.  At layer 0, the number of pruned tokens is primarily influenced by the number of padding tokens rather than token-level redundancy.
%In Figure \ref{fig:layertime}, we demonstrate the number of pruned tokens and the pruning protocol runtime in each layer in the BERT Base model. We averaged the number of pruned tokens in each layer across 128 QNLI samples and then tested the pruning protocol runtime in 10 independent runs. The mean token number of the QNLI samples is 48.5. During inference with BERT Base, input sequences with small token number are padded to 128 tokens with padding tokens. Similar to prior token pruning methods in the plaintext~\citep{goyal2020power}, a large number of padding tokens can be removed at layer 0. We remark that token-level redundancy builds progressively throughout inference~\citep{goyal2020power, kim2022LTP}. The number of pruned tokens in layer 0 mostly depends on the number of padding tokens instead of token-level redundancy.
}

{
%As shown in Figure \ref{fig:layertime}, more tokens are removed in the intermediate layers, e.g., layer $4$ to layer $7$. This suggests there is more redundant information in these intermediate layers. 
In CipherPrune, tokens are removed progressively, and once removed, they are excluded from computations in subsequent layers. Consequently, token pruning in earlier layers affects computations in later layers, whereas token pruning in later layers does not impact earlier layers. As a result, even if layers 4 and 7 remove the same number of tokens, layer 7 processes fewer tokens overall, as illustrated in Figure \ref{fig:layertime}. Specifically, 8 tokens are removed in both layer $4$ and layer $7$, but the runtime of the pruning protocol in layer $4$ is $\sim2.4\times$ longer than that in  layer $7$.
}

\section{Related Works}
\label{app:g}

{
In response to the success of Transformers and the need to safeguard data privacy, various private Transformer Inferences~\citep{chen2022thex,zheng2023primer,hao2022iron-iron,li2022mpcformer, lu2023bumblebee, luo2024secformer, pang2023bolt}  are proposed. To efficiently run private Transformer inferences, multiple cryptographic primitives are used in a popular hybrid HE/MPC method IRON~\citep{hao2022iron-iron}, i.e., in a Transformer, HE and SS are used for linear layers, and SS and OT are adopted for nonlinear layers. IRON and BumbleBee~\citep{lu2023bumblebee} focus on optimizing linear general matrix multiplications; SecFormer~\cite{luo2024secformer} improves the non-linear operations like the exponential function with polynomial approximation; BOLT~\citep{pang2023bolt} introduces the baby-step giant-step (BSGS) algorithm to reduce the number of HE rotations, proposes a word elimination (W.E.) technique, and uses polynomial approximation for non-linear operations, ultimately achieving state-of-the-art (SOTA) performance.
}

{Other than above hybrid HE/MPC methods, there are also works exploring privacy-preserving Transformer inference using only HE~\citep{zimerman2023converting, zhang2024nonin}. The first HE-based private Transformer inference work~\citep{zimerman2023converting} replaces \mysoftmax function with a scaled-ReLU function. Since the scaled-ReLU function can be approximated with low-degree polynomials more easily, it can be computed more efficiently using only HE operations. A range-loss term is needed during training to reduce the polynomial degree while maintaining high accuracy. A training-free HE-based private Transformer inference was proposed~\citep{zhang2024nonin}, where non-linear operations are approximated by high-degree polynomials. The HE-based methods need frequent bootstrapping, especially when using high-degree polynomials, thus often incurring higher overhead than the hybrid HE/MPC methods in practice.
}




\end{document}


% This document was modified from the file originally made available by
% Pat Langley and Andrea Danyluk for ICML-2K. This version was created
% by Iain Murray in 2018, and modified by Alexandre Bouchard in
% 2019 and 2021 and by Csaba Szepesvari, Gang Niu and Sivan Sabato in 2022.
% Modified again in 2023 and 2024 by Sivan Sabato and Jonathan Scarlett.
% Previous contributors include Dan Roy, Lise Getoor and Tobias
% Scheffer, which was slightly modified from the 2010 version by
% Thorsten Joachims & Johannes Fuernkranz, slightly modified from the
% 2009 version by Kiri Wagstaff and Sam Roweis's 2008 version, which is
% slightly modified from Prasad Tadepalli's 2007 version which is a
% lightly changed version of the previous year's version by Andrew
% Moore, which was in turn edited from those of Kristian Kersting and
% Codrina Lauth. Alex Smola contributed to the algorithmic style files.
