\section{Preliminary Study}
\label{sec:preliminary-study}


\noindent
\shepherd{ \textbf{Study 1: WPT Depths and Spectrogram Resolution.}
As discussed in Section~\ref{sec:sound-recognition}, the Wavelet Packet Transform (WPT) decomposes signals into finer sub-frequency bands at each level, with spectrogram resolution depending on WPT depth. Greater depth improves classification but increases computational cost. We conduct a preliminary study on WPT depth in environmental sound classification using ESC10~\cite{piczak2015esc} and US8K~\cite{salamon2017us8k}. On an MSP430 microcontroller~\cite{texas2021msp430}, we implemented a simple CNN classifier using WPT spectrograms at varying resolutions, measuring accuracy and energy consumption. Figure~\ref{fig:resolution-accuracy-energy} shows that higher resolution improves accuracy but greatly increases energy consumption, highlighting the need for cost-efficient approaches to balance performance and efficiency. This experiment also implies that to achieve good classification in on-cloud inference, high-resolution spectrogram will need to be transmitted. This results in even larger energy and communication overhead for edge devices, hence motivating keeping the inference pipeline local.}

\noindent
\shepherd{\textbf{Study 2: Effects of Frequency Bands.} WPT also allows us to selectively upsample frequency-domain resolutions on certain frequency bands. We argue that the discriminative information for different sound classes is distributed differently across different frequency bands.} To verify that, in the second preliminary experiment, we classify spectrograms of the same resolution but with either high-frequency bands only or low-frequency bands only. The results, shown in Figure~\ref{fig:high-low-frequency}, indicate that, for sounds of helicopters, waves, and drilling, high-frequency bands are more important for making the correct classification, whereas low-frequency bands are more important for some other classes.

\shepherd{These observations motivate the use of frequency-domain attention to guide the wavelet transform in generating multi-resolution spectrograms, achieving high accuracy while minimizing WPT and classification costs. This insight informs the design of our novel neural architecture, detailed in the following section.}



%\Eric{Exp2: Discrimnative information to distinguish different sound classes is distributed non-uniformly across spectral bands. Expected result: For some class high-freq bands are more important for classification, others low-freq bands. ExpPlan: Find appropriate classes mask out certain bands and do classification.}

%%%%%%%%%%%%%%%%%%%%%%%%%%%%%%%%%%%%%%%%%%%%%
\begin{figure}[tp]
    \centering
    \includegraphics[width=\linewidth]{figures/resolution-accuracy-energy.png}
    \vspace{-0.8cm}
    \caption{\shepherd{Accuracy (left) and energy consumption (right) at various spectrogram resolutions.}}
    \label{fig:resolution-accuracy-energy}
    \vspace{-0.3cm}
\end{figure}
%%%%%%%%%%%%%%%%%%%%%%%%%%%%%%%%%%%%%%%%%%%%%
\begin{figure}[tp]
    \centering
    \includegraphics[width=\linewidth]{figures/high-low-frequency.png}
    \vspace{-0.8cm}
    \caption{\shepherd{Accuracy of using high- and low-frequency band for ESC10 (left) and US8k (right).}}
    \vspace{-0.5cm}
    \label{fig:high-low-frequency}
\end{figure}
%%%%%%%%%%%%%%%%%%%%%%%%%%%%%%%%%%%%%%%%%%%%%