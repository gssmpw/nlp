\section{Introduction}


\begin{figure}[!tp]
    \centering
    \includegraphics[width=\linewidth]{figures/three-methods-comparison.png}
    \vspace{-0.4cm}
    \caption{\shepherd{Comparisons of on-device inference, cloud offloading, and cloud assistance. }}
    \vspace{-0.4cm}
    \label{fig:three-methods}
\end{figure}


Recent advances in intelligent environmental monitoring highlight machine learning-based environmental sound recognition, leveraging ultra-low-power sensing in embedded systems which offers advantages over vision-based solutions~\cite{nirjon2013auditeur, lane2015deepear, ahn2024split}. These applications are usually deployed in remote and resource-constrained areas, such as forests, wildlife reserves, or urban environments, where energy-efficient operation is crucial for continuous, autonomous monitoring. They demonstrate potential in biological research, such as classifying bird vocalizations in the rainforest~\cite{salamon2016towards}, and in city-scale sensing applications, such as monitoring noise pollution and traffic flows~\cite{mohan2008nericell, rana2010ear, de2018paws}. 


\shepherd{Despite their potential, edge devices' limited processing power and energy constraints make high-accuracy on-device predictions challenging~\cite{lee2019intermittent, gobieski2019intelligence, jeon2023harvnet, islam2020zygarde}. To overcome these limitations, recent efforts focus on edge-cloud collaboration~\cite{ahn2024split, huang2023rethink, yao2020deep, hojjat2024limitnet}. Compared to on-device learning, edge-cloud collaboration offloads inference tasks to the cloud server via wireless networks like Bluetooth LE, Wi-Fi, and LTE~\cite{ahn2024split, huang2023rethink, yao2020deep, hojjat2024limitnet}.  
This improves accuracy and reduces latency, but adds network dependency and energy overheads of costly wireless communication. Additionally, network coverage is a key factor in matching the diverse application scenarios. LTE provides wide coverage but is power-hungry for resource-constrained devices. In contrast, BLE and Wi-Fi are energy-efficient but have limited range, making them unsuitable for wide-area sensing applications. For instance, biologists deploy sensor nodes for acoustic bird species identification across vast rainforest areas~\cite{verma2024aviear}. Similarly, civil engineers and researchers use sensor nodes for large-scale acoustic sensing and classification in urban noise monitoring~\cite{tan2021extracting}. We find Low-Power Wide-Area Networks (LPWANs) ideal for these scenarios, balancing range and communication efficiency in edge-cloud collaborations. }


\shepherd{In this paper, we introduce a novel resource-aware cloud-assisted machine learning system, ORCA (\underline{O}ffload \underline{R}ethinking by \underline{C}loud \underline{A}ssistance), tailored for environmental sound recognition on ultra-low-power batteryless devices operating over LPWANs like LoRa networks. 
First, we leverage LPWANs to enable low-power, wide-area, and long-range edge-cloud collaboration for wide-area sound sensing applications. 
Second, we propose a novel \textit{cloud assistance} strategy for high-accuracy, resource-efficient, and LPWAN-adapted cloud offloading. In cloud assistance, instead of on-device inference as Figure~\ref{fig:three-methods}(a) or offloading by splitting the pipeline across wireless channels and generating inference results on the cloud as Figure~\ref{fig:three-methods}(b), ORCA keeps the entire inference pipeline on edge devices while leveraging the server for assistance. Specifically, the server selects the most important features uploaded by the edge devices. Here, we focus on identifying the most important frequency-domain features as previous studies show their effectiveness in environmental sound classification~\cite{phaye2019subspectralnet}. After that, the server provides the edge device with feature importance information through the downlink. This process is shown as the red arrows in Figure~\ref{fig:three-methods}(c). The edge devices can now use the given information to selectively process and perform inference on the most important features as the blue arrow in Figure~\ref{fig:three-methods}(c), thereby saving energy and time by scaling down the inputs and improving accuracy. As a result, the role of the server is now "assisting" edge devices for inference, avoiding edge devices being dependent on it. Specifically, ORCA tackles three key challenges of edge-cloud collaboration over LPWANs:}

\shepherd{
\textbf{(1) High communication costs and low data rate:} Though LPWANs like LoRa are “low-power”, wireless communication remains resource-intensive, e.g., 40$\times$ for uplink and 6$\times$ for downlink to the average power of on-device inference~\cite{mileiko2023run}. Additionally, LPWANs operate at a low data rate, e.g., 0.3-5.5 kbps in 125 kHz LoRa channel~\cite{bor2016lora}. As a result, previous data streaming approaches on BLE, Wi-Fi, or LTE~\cite{huang2023rethink, yao2020deep} are infeasible. To further reduce payload sizes, we propose transmitting low-resolution spectrograms for cloud processing. However, our preliminary studies indicate that while low-resolution spectrograms significantly reduce payloads compared to conventional cloud offloading methods, they also degrade cloud inference accuracy. On the other hand, compared to cloud inference, feature selection is generally more resilient to accuracy degradation with low-resolution spectrograms.
Therefore, we introduce a cloud-assisted approach that transmits low-resolution spectrograms to the cloud for feature selection rather than inference. ORCA implements this by feeding the low-resolution spectrogram into a pre-trained vision transformer and extracting feature importance masks from the self-attention maps.  }

\shepherd{\textbf{(2) Dynamic communication cost:} Wireless channel conditions are variable due to dynamic environments. To ensure reliable transmission, LPWANs adjust configurations like transmission power. Previous assumptions of constant wireless costs are impractical for energy-sensitive batteryless devices~\cite{yao2020deep, ahn2024split}. In ORCA, we dynamically optimize the "assistance resolution", the size of the spectrogram for uplink transmission and feature selection, to balance payload size and energy cost based on real-time communication feedback. For example, when communication cost is low, we select a larger size of spectrogram to improve feature extraction on the server and eventually increase the on-device classification accuracy. On the opposite, when communication cost is high, we switch to a smaller size of spectrogram for cloud assistance which result in lower classification accuracy.}

\shepherd{
\noindent \textbf{(3) Unreliable offloading:} Factors such as signal attenuation, interference, and environmental obstacles can lead to unreliable data transmission, resulting in packet loss and the need for retransmissions. This further challenges the reliability of the end-to-end offloading services. The previous server-dependent solution addresses this problem by resource-inefficient retransmission~\cite{hojjat2024limitnet}. In ORCA, when a packet loss occurs, we switch to local on-device inference only without cloud assistance, i.e., we keep the blue arrow only and ignore the red arrows in Figure~\ref{fig:three-methods}(c). This benefits from the design of maintaining the inference pipeline on-device, thereby saving the costs required for retransmission.}

\noindent
\shepherd{
\textbf{Application Scenarios:}
ORCA can operate over batteryless sensor networks in the wild, where distributed batteryless sensor nodes collect audio samples and perform resource-efficient cloud-assisted learning by connecting to an existing LPWAN gateway or a standalone edge server through wireless channels. As an aforementioned example~\cite{tan2021extracting}, batteryless sensor nodes powered by small solar panels can be strategically deployed in outdoor urban environments to monitor noise pollution. These nodes connect to established LoRa infrastructure and edge servers for resource-efficient cloud-assisted noise monitoring over hundred-meter to kilometer-scale distances. This setup enables continuous urban noise monitoring without battery replacements or maintenance, ensuring long-term, low-cost sustainability for smart city applications.}

\noindent
\textbf{Contributions: (1)} We propose a novel cloud assistance strategy for environmental sound recognition on resource-limited embedded systems and LPWANs. By leveraging vision transformer architecture for subspectral feature selection and efficient on-device spectral encoding convolutional neural networks (CNNs), ORCA achieves 2.5–12.5 p.p. accuracy improvements on five public environmental sound datasets.

\noindent
\textbf{(2)} We provide optimized assistance resolution strategies under dynamic communication costs and packet losses to ensure resource-efficient cloud assistance in dynamic environments.

\noindent
\textbf{(3)} We build a real-world testbed to evaluate accuracy, payload size, energy consumption, inference latency, and system overhead using collected LoRa traces. ORCA achieves up to 80$\times$ energy savings and 220$\times$ latency reduction, while maintaining accuracy comparable to the state-of-the-art methods. We release an open-source LoRa library for batteryless computing on the ultra-low-power MSP430 microcontroller\footnote{Library available at: \url{https://github.com/lezhangleonard/ORCA}}.






\definecolor{Gray}{gray}{0.95}
\definecolor{LightCyan}{rgb}{0.9,1,1}

\setlength{\arrayrulewidth}{0.4mm}
\begin{table*}[ht]
\centering
{\fontsize{9pt}{11pt}\selectfont
\setlength{\tabcolsep}{2pt}
\resizebox{\textwidth}{!}{%
\begin{tabular}{lllllccccc}
\hline
Prior Works & Edge Optimize & Wireless & Payload Data & Compression & Payload Size & Comm-adapt & ULP & Reliable & Testbed\\
\hline
\rowcolor{Gray}
DeepCOD~\cite{yao2020deep}           & Cloud offloading & Wi-Fi,LTE & Latent feature & Autoencoder & 0.1-5KB & \greencheck & \redcross & \redcross & \greencheck\\
FLEET~\cite{huang2023rethink}        & Early exits & BLE & Latent feature & Autoencoder & 2-9KB & \greencheck & \redcross & \redcross  & \greencheck \\
\rowcolor{Gray}
SEDAC~\cite{ahn2024split}            & Cloud offloading & BLE & Melspectrogram & Clip selection &  N/A & \redcross & \greencheck & \redcross & \greencheck \\
CACTUS~\cite{rastikerdar2024cactus}  & Micro-classifier & N/A & Model weights  & Class similarity & 11.5-43.9MB & \redcross & \redcross & \redcross & \greencheck \\
\rowcolor{Gray}
LimitNet~\cite{hojjat2024limitnet}   & Cloud offloading & LPWANs  & Latent feature & Gradual score & 0.3-3.2KB & \greencheck & \redcross & \redcross & \redcross\\
\rowcolor{LightCyan}
\textbf{ORCA}                                 & \textbf{Cloud assistance} & \textbf{LoRa} & \textbf{Spectrogram} & \textbf{Low resolution} & \textbf{0-0.1KB} & \greencheck & \greencheck &  \greencheck &  \greencheck \\
\hline
\end{tabular}
}
}
\caption{\shepherd{Comparisons of prior works on cloud offloading.}}
\label{tab:sota}
\vspace*{-0.8cm}
\end{table*}


