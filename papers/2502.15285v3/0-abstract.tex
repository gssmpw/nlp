\begin{abstract}

Learning-based environmental sound recognition has emerged as a crucial method for ultra-low-power environmental monitoring in biological research and city-scale sensing systems. These systems usually operate under limited resources and are often powered by harvested energy in remote areas. Recent efforts in on-device sound recognition suffer from low accuracy due to resource constraints, whereas cloud offloading strategies are hindered by high communication costs. In this work, we introduce \textit{ORCA}, a novel resource-efficient cloud-assisted environmental sound recognition system on batteryless devices operating over the Low-Power Wide-Area Networks (LPWANs), targeting wide-area audio sensing applications. We propose a cloud assistance strategy that remedies the low accuracy of on-device inference while minimizing the communication costs for cloud offloading. By leveraging a self-attention-based cloud sub-spectral feature selection method to facilitate efficient on-device inference, ORCA resolves three key challenges for resource-constrained cloud offloading over LPWANs: 1) high communication costs and low data rates, 2) dynamic wireless channel conditions, and 3) unreliable offloading. We implement ORCA on an energy-harvesting batteryless microcontroller and evaluate it in a real world urban sound testbed. Our results show that ORCA outperforms state-of-the-art methods by up to $80 \times$ in energy savings and $220 \times$ in latency reduction while maintaining comparable accuracy.


\end{abstract}