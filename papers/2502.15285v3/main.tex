%%
%% This is file `sample-authordraft.tex',
%% generated with the docstrip utility.
%%
%% The original source files were:
%%
%% samples.dtx  (with options: `authordraft')
%% 
%% IMPORTANT NOTICE:
%% 
%% For the copyright see the source file.
%% 
%% Any modified versions of this file must be renamed
%% with new filenames distinct from sample-authordraft.tex.
%% 
%% For distribution of the original source see the terms
%% for copying and modification in the file samples.dtx.
%% 
%% This generated file may be distributed as long as the
%% original source files, as listed above, are part of the
%% same distribution. (The sources need not necessarily be
%% in the same archive or directory.)
%%
%% Commands for TeXCount
%TC:macro \cite [option:text,text]
%TC:macro \citep [option:text,text]
%TC:macro \citet [option:text,text]
%TC:envir table 0 1
%TC:envir table* 0 1
%TC:envir tabular [ignore] word
%TC:envir displaymath 0 word
%TC:envir math 0 word
%TC:envir comment 0 0
%%
%%
%% The first command in your LaTeX source must be the \documentclass command.
% \documentclass[sigconf]{acmart}
%\documentclass[anonymous, sigconf]{acmart}
%\documentclass[sigconf,10pt,nonacm,anonymous]{acmart}
\documentclass[sigconf]{acmart}

%%
%% \BibTeX command to typeset BibTeX logo in the docs
\AtBeginDocument{%
  \providecommand\BibTeX{{%
    Bib\TeX}}}

\copyrightyear{2025}
\acmYear{2025}
% \setcopyright{cc}
% \setcctype{by}
\acmConference[SenSys '25]{The 23rd ACM Conference on Embedded Networked Sensor Systems}{May 6--9, 2025}{Irvine, CA, USA}
\acmBooktitle{The 23rd ACM Conference on Embedded Networked Sensor Systems (SenSys '25), May 6--9, 2025, Irvine, CA, USA}
% \acmDOI{10.1145/3715014.3722078}
% \acmISBN{979-8-4007-1479-5/2025/05}



%% Rights management information.  This information is sent to you
%% when you complete the rights form.  These commands have SAMPLE
%% values in them; it is your responsibility as an author to replace
%% the commands and values with those provided to you when you
%% complete the rights form.

\newcommand{\greencheck}{{\color{green}\ding{51}}}
\newcommand{\redcross}{{\color{red}\ding{55}}}
\usepackage{balance} 
\usepackage{pifont}
\usepackage{circledsteps}
\usepackage{tabularray}
\usepackage{amsmath}
\usepackage{hhline}
\usepackage{booktabs}
\usepackage{tabularx}
\usepackage{graphicx}
\usepackage{siunitx}
\usepackage{float}
\usepackage{multirow}
\usepackage{tikz}
\usepackage{xcolor,colortbl}
\usepackage{titlesec}
\usepackage{url, hyperref}
\usetikzlibrary{tikzmark,calc}
\sisetup{round-mode=places,round-precision=2}
\DeclareMathOperator*{\argmax}{arg\,max}
\DeclareMathOperator*{\argmin}{arg\,min}

\newcommand{\sys}{\text{ORCA}}

\usepackage{xcolor}
\newcommand{\shepherd}[1]{\textcolor{black}{#1}}



%%
%% Submission ID.
%% Use this when submitting an article to a sponsored event. You'll
%% receive a unique submission ID from the organizers
%% of the event, and this ID should be used as the parameter to this command.
%%\acmSubmissionID{123-A56-BU3}

%%
%% For managing citations, it is recommended to use bibliography
%% files in BibTeX format.
%%
%% You can then either use BibTeX with the ACM-Reference-Format style,
%% or BibLaTeX with the acmnumeric or acmauthoryear sytles, that include
%% support for advanced citation of software artefact from the
%% biblatex-software package, also separately available on CTAN.
%%
%% Look at the sample-*-biblatex.tex files for templates showcasing
%% the biblatex styles.
%%

%%
%% For managing citations, it is recommended to use bibliography
%% files in BibTeX format.
%%
%% You can then either use BibTeX with the ACM-Reference-Format style,
%% or BibLaTeX with the acmnumeric or acmauthoryear sytles, that include
%% support for advanced citation of software artefact from the
%% biblatex-software package, also separately available on CTAN.
%%
%% Look at the sample-*-biblatex.tex files for templates showcasing
%% the biblatex styles.
%%

%%
%% The majority of ACM publications use numbered citations and
%% references.  The command \citestyle{authoryear} switches to the
%% "author year" style.
%%
%% If you are preparing content for an event
%% sponsored by ACM SIGGRAPH, you must use the "author year" style of
%% citations and references.
%% Uncommenting
%% the next command will enable that style.
%%\citestyle{acmauthoryear}

%%
%% end of the preamble, start of the body of the document source.
\begin{document}

%%
%% The "title" command has an optional parameter,
%% allowing the author to define a "short title" to be used in page headers.
% \title{Batteryless Cloud-Assisted Machine Listening for Environmental Sound Recognition on LPWANs}
\title{Offload Rethinking by Cloud Assistance for Efficient Environmental Sound Recognition on LPWANs}

%%
%% The "author" command and its associated commands are used to define
%% the authors and their affiliations.
%% Of note is the shared affiliation of the first two authors, and the
%% "authornote" and "authornotemark" commands
%% used to denote shared contribution to the research.

\author{Le Zhang\textsuperscript{*}, Quanling Zhao\textsuperscript{*}, Run Wang, Shirley Bian, Onat Gungor, Flavio Ponzina, Tajana Rosing}
\affiliation{
  \institution{University of California, San Diego}
  \city{La Jolla}
  \state{California}
  \country{USA}}
\email{{lez014, quzhao, ruw041, y1bian, ogungor, fponzina, tajana}@ucsd.edu}

\thanks{\textsuperscript{*}Both authors contributed equally to this research.}

\renewcommand{\shortauthors}{L. Zhang et al.}

%%
%% By default, the full list of authors will be used in the page
%% headers. Often, this list is too long, and will overlap
%% other information printed in the page headers. This command allows
%% the author to define a more concise list
%% of authors' names for this purpose.
% \renewcommand{\shortauthors}{Trovato and Tobin, et al.}



%%
%% The abstract is a short summary of the work to be presented in the
%% article.
\begin{abstract}
Out-of-distribution (OOD) detection and OOD generalization are widely studied in Deep Neural Networks (DNNs), yet their relationship remains poorly understood. We empirically show that the degree of Neural Collapse (NC) in a network layer is inversely related with these objectives: stronger NC improves OOD detection but degrades generalization, while weaker NC enhances generalization at the cost of detection. This trade-off suggests that a single feature space cannot simultaneously achieve both tasks. To address this, we develop a theoretical framework linking NC to OOD detection and generalization. We show that entropy regularization mitigates NC to improve generalization, while a fixed Simplex Equiangular Tight Frame (ETF) projector enforces NC for better detection. Based on these insights, we propose a method to control NC at different DNN layers. In experiments, our method excels at both tasks across OOD datasets and DNN architectures. 

\begin{comment}   

Out-of-distribution (OOD) detection and OOD generalization are critical for deploying machine learning models in real-world scenarios. While substantial progress has been made in addressing these problems independently, few works have attempted to tackle them jointly. However, existing methods often rely on auxiliary OOD training data and primarily focus on covariate-shifted OOD data that share labels with in-distribution (ID) data. In contrast, we tackle the more realistic and challenging task of jointly detecting and generalizing to semantic OOD data with disjoint labels from the ID data, without auxiliary OOD training data.
Achieving both objectives simultaneously is inherently difficult due to a fundamental conflict — OOD generalization requires enhanced transferability, while OOD detection necessitates the inhibition of transfer.
To address this, we leverage insights from neural collapse (NC) — a phenomenon in deep networks where top-layer representations suppress feature variability and adopt a Simplex Equiangular Tight Frame (ETF) structure, impairing transferability. By controlling NC, we unify OOD detection and generalization: preventing NC enhances OOD transfer while inducing NC improves OOD detection.
Our proposed method excels at both tasks across various OOD datasets and architectures. 

\end{comment}


\end{abstract}


%%
%% The code below is generated by the tool at http://dl.acm.org/ccs.cfm.
%% Please copy and paste the code instead of the example below.
%%


\begin{CCSXML}
<ccs2012>
<concept>
<concept_id>10010520.10010553.10003238</concept_id>
<concept_desc>Computer systems organization~Sensor networks</concept_desc>
<concept_significance>500</concept_significance>
</concept>
<concept>
<concept_id>10010520.10010553.10010562</concept_id>
<concept_desc>Computer systems organization~Embedded systems</concept_desc>
<concept_significance>500</concept_significance>
</concept>
<concept>
<concept_id>10010147.10010178</concept_id>
<concept_desc>Computing methodologies~Artificial intelligence</concept_desc>
<concept_significance>300</concept_significance>
</concept>
</ccs2012>
\end{CCSXML}

\ccsdesc[500]{Computer systems organization~Sensor networks}
\ccsdesc[500]{Computer systems organization~Embedded systems}
\ccsdesc[300]{Computing methodologies~Artificial intelligence}

\keywords{Embedded intelligence, cloud offloading, environmental sound classification, batteryless computing, LoRa, LPWANs }

%%
% Keywords. The author(s) should pick words that accurately describe
% the work being presented. Separate the keywords with commas.
% \keywords{Embedded intelligence, Environmental sound classification, Batteryless computing, LPWANs }

%%
%% This command processes the author and affiliation and title
%% information and builds the first part of the formatted document.
\maketitle

\section{Introduction}

% State of the world (robots for creative activites)
The term ``robot,'' originally signifying `forced labor,' has long been associated with labor and work. Robots have demonstrated their utility in various automated productive and social contexts, where the primary goals are improving productivity, safety, and fostering social interactions with humans~\cite{simoes2022designing, weidemann2021role, honig2018understanding}. However, an increasing number of cases feature using of robots in creative settings. Unlike productive contexts, where the focus is on efficiency and task completion~\cite{arents2022smart}, or social contexts, where communication and trust are prioritized~\cite{nam2020trust, saunderson2019robots}, creative environments prioritize artistic innovation and expression~\cite{hsueh2024counts}. This shift fundamentally alters the dynamics of human-robot interaction, redefining the roles and expectations for both humans and robots.

For instance, robots’ social behaviors are leveraged to support the generation and expression of creative ideas~\cite{hu2021exploring, sandoval2022human, alves2020creativity}, and programmable robotic movements and trajectories are employed to inspire artistic activities such as sketching~\cite{lin2020your}. These studies often engage participants from creative fields who possess limited prior experience with robotics, and are typically conducted in short-term, experimental settings. Consequently, the findings from these studies remain constrained since much can be learned from professional practitioners' experiences to inform system design such as digital fabrication~\cite{hirsch2023nothing}. There is a notable gap in research examining the long-term, active, and practical experience of integrating robotic systems into the creative processes. As a result, the deeper insights into how robots facilitate and shape creative processes, beyond simply augmenting human creativity, remain underexplored. In this study, we aim to better understand the impacts of robots on creative processes and outcomes.

As early as Leonardo da Vinci's 16th century ``Automaton,'' artists have explored the creative affordances of robotic systems~\cite{shanken2002cybernetics, pagliarini2009development, jeon2017robotic}. The artistic creation process typically encompasses various stages, including the exploration of materials and techniques, ongoing experimentation and iteration, and the continual refinement of the artists' insights into their creative subjects~\cite{lewis2023art, sturdee2022state}. Therefore, investigating the artistic process involving robots offers an opportunity to gain deeper insights into robots' creative potential. Robotic art, in particular, provides a compelling case for this exploration.

We define robotic art as artworks that utilize robotic or automated machines to create artistic experiences and tangible artifacts. One example is robotic installation art, in which robots are programmed to follow specific rules that embody the artist’s expression (\autoref{fig:teaser} (a)). Another example is responsive art, in which robots react to their environment, with behaviors that change over time or in response to spectators (\autoref{fig:teaser} (b)). Additionally, there are robotic creators, which possess a degree of agency, allowing them to collaborate with human artists and produce works that extend beyond mere replication of human-created art (\autoref{fig:teaser} (c) and (d)). As such, robotic art becomes a rich case for exploring human-machine interactions in creative contexts. Gaining a deeper understanding of how robots facilitate artistic expression can provide insights for designing computing systems to support creative activities~\cite{gomez2021robot}.

% Therefore, we did...
We draw on semi-structured, in-depth interviews with renowned professional robotic artists to investigate the use of robots in artistic practice. Specifically, our goal is to understand how artistic exploration of robotic systems challenges conventional assumptions about the functions of robots, such as their roles in automating repetitive tasks or serving human needs. We also explore the implications of robots in the artistic process and examine how creativity may emerge within robotic art. To address these interrelated inquiries, our study focuses on the practice of robotic art, posing the research question: \textit{How do robotic artists utilize robots in their artistic practice?} We approach this inquiry through the perspectives and experiences of robotic artists, who creatively design, modify, and repurpose robotic systems for artistic expression and exploration.

% The key findings are...
Our findings highlight the social, material, and temporal dimensions of artists' practices that shape their creativity and artistic outcomes. The creation of robotic art is largely a social process, as artists receive both explicit and implicit feedback through the audience's reactions and reception of their work. Simultaneously, the embodiment and malfunctions inherent to robotic systems drive artistic experimentation. The temporal processes of creation and exhibition, beyond just the final product, further enhance the creative value. Our empirical analysis presents how creativity emerges through the interplay of social, material, and temporal interactions among artists, robots, audiences, and the environment.

% The contributions of this work are...
We make two main contributions to HCI in this study. 
First, we elucidate the interactive mechanisms among key actors---human creators, machines, audiences, and environments---within the practice of robotic art, a topic that remains underexplored in HCI. Our findings reveal the significance of sociality (e.g., interactions between artists and audiences), materiality (e.g., the embodiment and malfunctions of robots), and temporality (e.g., the processes of creation and exhibition) in shaping creative values. We propose that these three facets are central to the creative process and facilitate the emergence of creativity in robotic art.
Second, drawing from the findings, we offer implications for \textit{socially informed}, \textit{material-attentive}, and \textit{process-oriented} creation with computing systems. We suggest leveraging these three aspects to enhance creativity and the creative experience. Specifically, we discuss the value of incorporating implicit audience feedback, designing with technical malfunctions, and focusing on the post-creation process to foster alternative creative experiences with machines~\cite{alter2010designing, juarez2022glitch}.




%\subsection{End-User Programming}

%\kenneth{The way I like to think about Related Work is that this section should (sometimes subtly, not explicitly, but effectively!) answer some underlying questions that reviewers might want to ask. So, here we go:}\steven{sounds good!}

\subsection{Ways of Optimizing Prompts for LLMs}
%\subsection{Prompt Engineering and How Good Humans Are at It}
Prompts are the primary means by which users interact with, utilize, and instruct LLMs. 
Since the emergence of these models, researchers and developers have invested significant effort into understanding how to craft better prompts for more effective use. 

\paragraph{Automatic Prompt Optimization.}
Much of the prior work has focused on automatically optimizing prompts. 
A common theme across these studies is the use of gold-standard labels to guide the optimization process.
For example, \citet{pryzant2023automatic} introduced an automatic prompt optimization method inspired by gradient descent; 
\citet{manas2024improving} presented an approach that begins with a user prompt and iteratively generates revised prompts to maximize consistency between the generated image and prompt, without human intervention; 
\citet{wan2024teach} explored two types of prompt optimization, instruction and exemplar, and suggested that combining both can yield optimal results; 
\citet{sun2023autohint} combined zero-shot and few-shot learning to optimize prompts automatically; %eliminating the need for manual prompt engineering; 
and \citet{levi2024intent} improved prompt optimization through synthetic data generation and iterative refinement, focusing on aligning prompts with user intent by creating challenging boundary cases for iterative prompt refinement.
While these studies were interesting and relevant, they generally assumed the availability of gold-standard labels and did not address situations where labels are absent or where standards are constantly evolving.

\paragraph{User-Driven Prompt Optimization.}
In addition to automatic prompt optimization, some research has focused on human capabilities in optimizing prompts. 
\citet{zhou2023revisiting} found that manual prompting often outperforms automated methods in various scenarios; 
\citet{10.1145/3544548.3581388} discovered that people tend to design prompts opportunistically rather than systematically, which often leads to lower success rates. 
To the best of our knowledge, the most relevant prior work is by \citet{wang2024end}, who developed an iterative refinement system that enables users to prompt LLMs to build a personalized classifier for social media content. 
Their study explored three user strategies for improving prompts and measured their effectiveness. 
While conceptually related to our work, their focus was not on how users evolve their understanding and expectations when interacting with LLMs. 
Instead, participants labeled ground truth at the beginning of the study, prior to using the system.



%--------------------- dead kitten --------------
\begin{comment}
 





The most relevant prior work is by \citet{wang2024end}, who developed an iterative refinement system allowing users to prompt LLMs to build a personalized classifier for social media content.
While their work is closely related to ours in concept, their study did not focus on how users evolve their understanding and expectations while working with LLMs. 
Instead, participants labeled ground truth at the outset before using the system.


\kenneth{The key question for our paper is this: Did prior work try to measure users' prompt engineering performance *over multiple iterations*? (What do we know about human performance in prompt engineering?) I think you can maybe find some papers, especially papers for automatic prompt optimization like DSPy, measuring users' individual prompt's output accuracy (or MSE) or performance (e.g., BLEU in generation task), but it might be hard to find papers capture and measure *multiple iterations* from the same user for the same prompt.--This is the main argument for our paper: we did something that was hard and thus has not been done.}

\kenneth{Take a look at this survey paper:~\cite{chen2023unleashing}}



\steven{iterative tool involve human}
PromptIDE is an interactive tool that helps the experts to iteratively refine tools by providing various prompts, visualizing their performance on small validation datasets, and iterative optimizing them based on quantitative feedback~\cite{strobelt2022interactive}. \steven{gold label exists}

PromptAID is a visual analytics system that helps non-experts iteratively improve prompts through exploration, perturbation, testing, and refinement. It supports prompts through keyword adjustment, paraphrasing, and adding few-shot examples. \steven{has test dataset, it is a complex system}

\steven{automate prompting}
\citet{pryzant2023automatic} introduces an automatic prompt optimization prompt inspired by gradient descent. \steven{this fell into software designing, involve gold labels}

The study starts from a user prompt and iteratively generates revised prompts with the goal of maximizing a consistency score between the generated image and prompt without a human in the loop\cite{manas2024improving}\steven{without human involvement in the loop, gold labels}

\citet{zhou2023revisiting} found that manual prompting often performed better than automated methods in various steps. 

\cite{wan2024teach} explores the distinction between two types of prompt optimization: instruction optimizer and exemplar. This study suggested combining both approaches could lead to optimal results.

\cite{sun2023autohint} combines zero-shot and few-shot learning to optimize prompts automatically, without manual efforts in prompt engineering.

\cite{levi2024intent} improve prompt engineering optimization by synthetic data generation and iterative refinement, focusing on aligning prompts with user intent by generating challenging boundary cases and using these to refine the prompt iteratively.





\paragraph{Prompt Engineering Tools.}
\kenneth{After making the first point, we can have a follow-up paragraph to say that many tools were created to help people do prompt engineering (list a few and name their focuses), but again, they did not focus on measuring how good humans are in prompt engineering--- Of course, there could be an argument that suggests: no matter how good you are, you will always need some tool. It is true---for example, ChainForge basically create a easy-to-use UI that make things easier, which is not really about accuracy---But for annotation tasks, performance is still critical and it is always good to know how well human did, almost like many AI leaderboard has various "human" performance for comparison.}
PromptMaker, a platform for rapidly prototyping new ML models using prompt-based programming, was difficult to evaluate their prompts systematically~\cite{10.1145/3491101.3503564}.

\cite{arawjo2024chainforge}  is an Open-source visual toolkit for prompt engineering and on-demand hypothesis testing of text-generation LLMs.

 promptfoo is test-driven LLM development, not trial-and-error, producing matrix views that let you quickly evaluate outputs across many prompts~\cite{webster2023promptfoo}.

\cite{madaan2024self} introduces a method that LLM iterative improve their output by using their own feedback, without external supervision. 

\saniya{austin etal points:
1. used only COPRO, evaluation criteria utilized a custom LLM-as-a-judge metric. The paper showed that their automated prompt optimizer worked better tha DSPy }
   
\end{comment}


\subsection{Tools for Prompt Engineering}
With the advances in LLMs, numerous tools have been developed to assist with prompt engineering. 
Most of these tools follow a software-engineering paradigm, where testing (such as unit tests or integration tests) is a central concept, and thus often assume the existence of gold-standard labels.
For example, PromptIDE is an interactive tool that helps experts iteratively refine prompts by providing various inputs, visualizing their performance on small validation datasets, and optimizing them based on quantitative feedback~\cite{strobelt2022interactive}; 
PromptAid is a visual analytics system for interactively creating, refining, testing, and iterating prompts while tracking accuracy changes~\cite{mishra2023promptaid};
%It allows users to adjust prompts through keyword modifications, paraphrasing, and adding few-shot examples; 
ChainForge is an open-source visual toolkit for prompt engineering and on-demand hypothesis testing of text-generation LLMs~\cite{arawjo2024chainforge};
and, promptfoo applies a test-driven approach to LLM development, producing matrix views that enable quick evaluation of outputs across multiple prompts~\cite{webster2023promptfoo}.
While these tools are inspiring and valuable, the scenarios we focus on do not rely on the constant availability of gold labels.

%\cite{mishra2023promptaid}


\begin{comment}






\kenneth{In here, we want to answer this questions: Why do we need to built \system? Can't we just use some existing tools??? The underlying answer could be: all the tools, including the one we mentioned in previous subsection, were not really aiming for ``general users'' and only thing general users can reliably use is probably chat interface come with ChatGPT etc.}

\citet{10.1145/3544548.3581388} mentioned that people tended to design prompts opportunistically, not systematically, which resulted in less success. \system provides a systematic process for composing and refining prompts, allowing non-expert users to adapt to the prompt creation process effortlessly.

\saniya{Amy Zhang points:
\newline 1. Accuracy didnot improve; reported improvements in recall
\newline 2. Observed that humans are pretty bad at being consistent
\newline 3. Quoted  Miles Turpin, Julian Michael, Ethan Perez, and Samuel Bowman. 2024. Language models don't always say what they think: unfaithful explanations
in chain-of-thought prompting. Advances in Neural Information Processing Systems 36 (2024).
Han Wang, Ming Shan Hee, Md Rabiul Awal, Kenny Tsu Wei Choo, and Roy Ka-Wei Lee. 2023. Evaluating GPT-3 Generated Explanations for
Hateful Content Moderation. arXiv:2305.17680 [cs.CL] for not using LLM prompt explanations
\newline 4. They had a bigger training set of around 700 examples: paper excerpt: "This process resulted in a balanced dataset of 800 comments. We randomly divided our dataset into a training dataset and a test dataset of 100 examples for each participant. The training dataset was used to help participants create their classiiers, whereas the test dataset was labeled by participants and used to evaluate their created classiiers."
}
    
\end{comment}

\subsection{Human-LLM Collaborative Data Annotation}
%Another relevant area of research involves using LLMs for data annotation. 
Beyond simply treating LLMs as automatic labelers---common in countless NLP projects~\cite{tan2024large}---a growing body of work explores how to combine human and LLM efforts to achieve better annotation outcomes, such as improved accuracy or speed.
Even as LLMs outperform humans in many labeling tasks, human-AI collaboration often produces better results than either alone~\cite{vaccaro2024combinations}.
For example, \citet{kim2024meganno+} introduced a human-LLM collaborative annotation system where LLMs handle bulk annotation tasks, while humans selectively verify and refine the annotations. 
%\steven{However, this system was limited to deployment within Jupyter Notebook, lacking an end-to-end solution. This design imposed significant barriers, as it required users to possess technical expertise for system setup before using the tool, limiting accessibility and scalability in non-technical domains.}
\citet{goel2023llms} proposed an approach that combines LLMs with human expertise to efficiently generate ground truth labels for medical text annotation.
Additionally, \citet{10.1145/3613904.3642834} demonstrated how aggregating crowd workers' labels with GPT-4's output can achieve higher labeling accuracy than either source alone.
These studies generally aim to split the workflow of data labeling between humans and LLMs in a smart way, making the task more effective or efficient. 

In contrast, our work does not focus on dividing or combining the workload, but on how humans can teach LLMs---through prompt refinement---to better label the specific type of data.
Few prior studies have emphasized iterative prompt refinement in human-LLM collaborative data annotation.
For example, \citet{liu2024harnessing} developed a workflow for video content analysis, refining prompts to improve LLM-generated annotations and align them with human judgment.
Additionally, \citet{zhang2023llmaaa} proposed LLMAAA, which uses LLMs as annotators in a feedback loop to label data efficiently.
Their study shows that poorly designed prompts result in subpar performance, especially in complex tasks. %while incorporating demonstrations and aligning label descriptions with natural language significantly enhances accuracy and reliability.
Our work advances this relatively understudied area of human-LLM collaborative annotation research.

%----------------------------- dead kitten --------------------------------

\begin{comment}








\steven{\citet{vaccaro2024combinations} emphaized that designing innovative processes for integrating humans and AI is as critical as developing advanced AI technologies. This aligns with the need for LLM-powered systems that iteratively guide AI outputs to meet user-specific standards, prioritizing effective collaboration between users and AI systems.}

\steven{\citet{liyanage2024gpt} found that GPT-4, using few-shot, zero-shot, and Chain-of-Thoughts (CoT) prompting techniques, could not outperform models fine-tuned on human-labeled data. Among these, the few-shot approach exhibited the highest degree of similarity to human annotations. However, in scenarios where gold labels are unavailable, fine-tuning is not applicable, and alternative methods must be explored.}

\steven{\citet{liu2024harnessing} developed a workflow for video content analysis, iteratively crafting prompts to enhance LLMs' ability to generate structured annotations and comprehensive explanations that aligned with human judgment. }

\steven{\citet{zamfirescu2023herding} found that while prompts can effectively address most UX goals, they struggle with nuanced, edge-case, or spontaneous interactions. The study highlights that the effectiveness of each instruction in the prompt is highly sensitive to its phrasing and location. Additionally, highly prescriptive prompts, though reliable, limited the spontaneity and flexibility of GPT responses.
In our system, users are only required to provide task information—such as task descriptions, rules, and examples—to construct instructions, allowing for greater flexibility in accommodating diverse task requirements..}

\steven{\citet{guyre2024prompt} illustrates how prompt engineering can empower non-experts to design tailored conversational agents by iteratively refining prompts and infusing domain-specific knowledge. Their study emphasizes democratizing chatbot development, allowing users to align AI behavior with their specific goals and values.}

\steven{\citet{zhang2023llmaaa} proposes LLMAAA that leverages LLMs as Active Annotators in a feedback loop to efficiently annotate data. The study highlights that poorly designed prompts lead to suboptimal performance by LLM annotators, particularly in complex or domain-specific tasks. However, incorporating demonstrations and aligning label descriptions with natural language significantly enhances annotation accuracy and reliability.}

%\kenneth{Here, we then answer this question: Did people create ANYTHING to support LLM-powered data annotation? There are two parts of the answer to this: 1) Many or even most papers, including our CHI paper last year, focus on the labeling performance of LLMs, for example, as compared to crowdsourcing. They did not focus on the UI aspect of it. 2) Some prompt chaining tools, like ChainForge, can support workflow like this, but (a) hey do not focus on data annotation in particular so some functions are missing, like data resampling, and (b) more importantly, they do not aim to support general users. Most of them expect you to know some programming, e.g., ChainForge clearly say it's a visual programming tool. They're not really aiming for generic users.}


\cite{kim2024meganno+} introduced a human-LLM collaborative annotation system that allows LLM to handle bulk annotation tasks while humans verify selectively to refine annotation. 

\cite{goel2023llms} introduced an approach that combines LLM wth human expertise to create an efficient method for generating
ground truth labels for medical text annotation.


\cite{shankar2024validates} introduced a tool, EvalGen, to address the challenge of validating LLM. 
EvalGen helps users design evaluation criteria for LLM outputs and align that evaluation with human preferences through a mixed-initiative system.
A key finding is the concept of criteria drift, where users modify their evaluation standards while grading outputs. 


\cite{brade2023promptify} Promptify utilizes an LLM-powered suggestion engine to help users quickly explore and craft diverse prompts for text-to-image generation tasks.

    
\end{comment}


%\subsection{Survey Study in Data Annotation}
%\steven{
We conducted a survey study to investigate how individuals interact with LLMs and utilize gold-standard labels in the data annotation process. 
The participants primarily represent roles in research, machine learning engineering, and software development. \\
\textbf{Workflows: }Participants described diverse workflows for integrating LLMs into data annotation process, highlighting a common iterative and human-in-the-loop approach. \textbf{Most workflows begin with manual annotation of a small subset of data to establish a baseline.} Participants then employ prompt engineering, iteratively refining LLM prompts by evaluating their performance against the manually annotated subset. \\
Once refined, the prompts are used to label larger datasets, with participants using tools or manual checks to review the LLM's annotations and identify any invalid labels. The process is typically concluded with a thorough manual verification of the dataset. \\
One participant mentioned they manually tabulate data points along with their descriptions. \\
\textbf{Initialize Prompting: }Most participants use their pre-defined prompts to initialized the annotation on their known tasks. 
For new tasks, one participant mentioned that they initialize the annotation process with LLMs by starting with a clear problem definition and iteratively refining a classification-based approach. For less familiar tasks, some participants may seek suggestions from the LLM to guide the initial setup.
\textbf{Revising Prompt: } Participants use a small dataset to finetune the prompt. They address issues by adding rules or context examples to tackle failure cases. When inconsistencies or error arise, they revisit and recheck the manually tagged dataset to improve performance. Some participants also engage the LLM by asking questions about data points and their descriptions, retraining to against inconsistencies to minimize hallucinations and enhance annotation reliability.
}

\subsection{Gold-Standard Labels in Annotation Tasks}\label{sec:related-work-gold-label}
Decades of research have shown that gold-standard labels play a critical role in quality control for data annotation pipelines~\cite{han2020crowd,gadiraju2015training,le2010ensuring,doroudi2016toward,hettiachchi2021challenge}.
Embedding items with known labels into the data annotation process allows requesters to reliably capture quality signals, 
such as workers' level of expertise~\cite{abraham2016many, abassi2019worker, yang2018improving} %\kenneth{TODO: Add refs about using gold labels to decide workers' expertise level}\steven{added}
or attentiveness to tasks~\cite{hettiachchi2021challenge, oleson2011programmatic}. %\kenneth{TODO: Add refs about using gold labels to do attention checks for workers}\steven{added}
These insights enable requesters to take appropriate actions, such as 
retraining annotators~\cite{le2010ensuring, doroudi2016toward,hettiachchi2021challenge}, %\kenneth{TODO: Add refs about retraining workers}\steven{added}
removing low-performing workers~\cite{10.1145/3613904.3642834, snow2008cheap,downs2010your,le2010ensuring}, %\kenneth{TODO: Add refs about removing or blocking low-performing workers}\steven{added}
or identifying potential issues in the annotation interfaces~\cite{toomim2011utility,10.1145/3613904.3642834, rahmanian2014user, komarov2013crowdsourcing}. %\kenneth{TODO: Add refs for crowd worker interfaces. At least cite: Toomim, M., Kriplean, T., Pörtner, C., \& Landay, J. (2011, May). Utility of human-computer interactions: Toward a science of preference measurement. In Proceedings of the SIGCHI Conference on Human Factors in Computing Systems (pp. 2275-2284).}\steven{added}
Gold labels are also beneficial for requesters during post-annotation data processing. 
They can be used to weight labels from different workers in label aggregation~\cite{abassi2017gold,abassi2019worker}, %\kenneth{TODO: Add label aggregation methods that use gold labels particularly to weight different workers}\steven{added}
improve label aggregation strategies~\cite{khattak2011quality, snow2008cheap},  %\kenneth{TODO: Add label aggregation methods that learn whatever from gold labels}\steven{added}
or 
exclude unreliable workers' outputs entirely~\cite{abassi2019worker}. %\kenneth{TODO: Cite ref using gold labels to remove workers from label aggregation}\steven{added}
Beyond requesters, gold labels are also beneficial for data labelers like crowd workers. 
Gold labels can be used to train workers~\cite{doroudi2016toward, le2010ensuring, gadiraju2015training,han2020crowd}, %\kenneth{TODO: Cite ref that uses gold labels for worker training}\steven{added}
provide real-time feedback to help them recalibrate their understanding of the task~\cite{le2010ensuring,hettiachchi2021challenge}, %\kenneth{TODO: Cite the visible gold paper from Amazon}\steven{added}
or remind them to pay more attention~\cite{ hettiachchi2021challenge,oleson2011programmatic}. %\kenneth{TODO: Cite attention check papers}\steven{amazon paper also warn workers in real time}

While gold labels are useful for quality control, as stated in the Introduction (Section~\ref{sec:intro}), %\kenneth{TODO: Update references}\steven{done}
they are not always available in real-world scenarios due to constraints such as data privacy or the cost of gathering gold labels~\cite{liu2019deep, yang2019evaluating, oikarinen2021detecting, slote2024unlocking}.
To address these challenges, researchers have developed methods to generate (approximations of) quality signals without gold labels. 
In the realm of LLM-powered data annotation, for instance, CoPrompter evaluates how well an LLM's output aligns with user-specified requirements as a feedback mechanism~\cite{joshi2024coprompter}. %\kenneth{TODO: Cite: Joshi, I., Shahid, S., Venneti, S., Vasu, M., Zheng, Y., Li, Y., ... \& Chan, G. Y. Y. (2024). CoPrompter: User-Centric Evaluation of LLM Instruction Alignment for Improved Prompt Engineering. arXiv preprint arXiv:2411.06099.}\steven{added}
Other studies also leverage the stability~\cite{chiang2023can} %\kenneth{TODO: Add ref}\steven{added}
%chiang2023can found LLM evaluation are stable over different formatting
or confidence~\cite{gligoric2024can} %\kenneth{TODO: Add ref}\steven{added}
%gligoric2024can introduce CONFIDENCEDRIVEN INFERENCE: a method that combines LLM annotations and LLM confidence indicators to strategically select which human annotations should be collected
of LLM outputs to infer quality signals.
%Our research investigates how effectively humans can iteratively refine prompts to guide LLMs in labeling data when gold-standard labels are unavailable, providing alternative quality signals.
Our research examines how effectively humans can refine prompts to guide LLMs in labeling data without gold-standard labels, providing insights into human prompting capabilities in the absence of reliable guidance signals.










%------------- dead kitten -------------
\begin{comment}




\kenneth{------------------------KENNETH IS WORKING HERE----------------------}



Gold-standard labels are widely used for quality control and crowd worker training~\cite{doroudi2016toward, gadiraju2015training,le2010ensuring,hettiachchi2021challenge}. For example, \citet{hettiachchi2021challenge} demonstrated that incorporating visible gold questions -- where annotators receive periodic feedback based on pre-labeled gold-standard examples -- could improve their work quality. 
Similarly, \citet{doroudi2016toward} found that providing expert examples was the most effective method of training for crowd workers and can help workers avoid specific types of incorrect responses. 
Additionally, \citet{le2010ensuring} employed dynamic learning systems that leveraged gold-standard labels to deliver real-time feedback and improve worker outcomes.
These studies, however, predominantly address the annotators' perspective -- workers who adhere to predefined guidelines and follow established standards.
While annotators are crucial components of the task pipeline, our study shifts focus to the requesters' perspective, those responsible for task design and pipeline management.
For requesters, gold-standard labels serve as signals to assess worker performance and refine training processes, thereby improving the overall quality of the entire pipeline.
Critically, the aforementioned studies assume the availability of gold-standard labels, typically under controlled experimental settings. 
In real-world scenarios, this assumption often does not hold due to constraints such as data privacy, security concerns, or the absence of labeled data~\cite{liu2019deep, yang2019evaluating, oikarinen2021detecting, slote2024unlocking}. 
To address this gap, our research explores settings where predefined gold-standard labels are unavailable. 
We designed a novel framework for requesters to iteratively develop and evolve their labeling standards through interactions with LLMs. 
By bridging the divide between controlled experiments and real-world challenges, our work highlights the potential of adaptive, LLM-driven approaches for dynamic task management without reliance on predefined gold-standard labels.

\steven{\citet{hettiachchi2021challenge} demonstrated that incorporating visible gold questions -- where annotators receive periodic feedback based on pre-labeled gold-standard examples -- could improve their work quality. 
Their study leveraged gold-standard labels to train crowd workers to align with pre-defined standards, effectively guiding annotators thorugh examples and feedback. 
While this approach focues on improving labeling quality at the annotator level, our work shifts the focus to requester and researcher perspective. Instead solely training labelers to meet pre-existing standards, we emphasize the broader implications of designing system in the entire labeling process, particularly in context involving dynamic or subjective tasks. \citet{gadiraju2015training} showed that training workers with gold labels can enhance accuracy and decrease response times. \citet{han2020crowd} used gold standard labels to guide crowd workers in revising incorrect judgments to align with predefined standards. 
}

\steven{
\citet{doroudi2016toward} found that providing expert examples was the most effective method of training for crowd workers. In our study, however, each participant was treated as an individual researcher rather than a crowd worker. While this finding underscores the value of providing gold labels to improve language model performance, it does not directly highlight their significance for researchers. Furthermore, \citet{doroudi2016toward} observed that gold standard labels help workers avoid specific types of incorrect responses. 
In contrast, our task is subjective, with participants’ standards potentially shifting across iterations. Introducing pre-set gold standard labels could enforce a uniform standard across each participant, which might not align with the iterative and subjective nature of our study
}

\steven{\citet{gadiraju2015training} showed that training workers with gold labels can enhance accuracy and decrease response times. [They were still focusing on crowd worker level.] }

\steven{\citet{han2020crowd} used gold standard labels for quality control and to guide crowd workers in revising incorrect judgments to align with predefined standards.}

\steven{\citet{le2010ensuring} employed gold standard labels within a dynamic learning environment that provided real-time feedback to train workers. However, the selection of specific examples for training could influence worker responses, potentially introducing bias in their judgments. [This is why we implemented a random sample in our system]}


\steven{\citet{liu2019deep} developed a HITL system that kept model upgrading with progressively collected data without having a pre-labeled data. [\textbf{they used 30 samples per iteration.} -add to justification for 10 and 50 instances.]}

\steven{\citet{wall2019using} found that end-users could build models without using expert patterns that have comparable performance to those who built by expert. This approach was required more effort and more mental demand than those who received guidance.}

\kenneth{TODO: Add references to every part of this paragraph.}
Decades of research have established that gold-standard labels are highly effective for quality control in data annotation~\cite{han2020crowd,gadiraju2015training,le2010ensuring,doroudi2016toward,hettiachchi2021challenge}. 
Embedding items with known labels into the annotation process enables requesters to monitor annotator or data quality and take actions such as retraining annotators, removing them from the pipeline, or reducing their weight in label aggregation. 
Beyond requesters, gold labels also allow for real-time feedback to workers, helping them recalibrate their understanding of the task or focus more carefully.
While gold labels are widely recognized as useful for quality control, most research assumes their availability.
However, as discussed in our Introduction (Section X), this assumption does not necessarily hold in real-world scenarios due to constraints such as data privacy or the cost of gathering gold labels~\cite{liu2019deep, yang2019evaluating, oikarinen2021detecting, slote2024unlocking}. 
To address these challenges, researchers have developed systems to provide proxy quality signals without gold labels. 
For instance, CoPrompter evaluates how well an LLM's output aligns with user-specified requirements as a feedback mechanism. 
Other studies leverage the stability or confidence of LLM outputs to infer quality signals.
Our research investigates how effectively humans can refine prompts to guide LLMs when gold-standard labels are unavailable.
    
\end{comment}

%\subsection{Explanations in AI-Assisted Tools}


%\subsection{Variables in System}
%There are lots of variables in a system could impact user's performance. 
\citet{kulesza2012tell} suggested that the more users understand the underlying system, the more effectively they can control it. 

\steven{\citet{lee2024clarify} introduces a system that allows non-expert users to train and correct models by directly interact with model using natural languages. In each iteration, the system will use similarity score between user description and image and display images above a threshold. The system will also provide 0-1 score indicating how well description separates the error cases from the correct prediction. Basically using metrics to guide user.
It does not mentioned about the sample size selection.}

\steven{[Data Instance:] In active learning, the goal is to minimize the amount of interaction required by users by querying the most important information~\cite{bernard2018vial}. [This can be used to justify why we increase to 50, to ensure the diversity. We cannot deploy algorithms to find most representative data sample because of the technical limitation of Google App Script]}

\steven{[Data Instance:] \citet{vermetten2022analyzing} investigated how the number of sample size affects the reliability of algorithm comparisons in iterative optimization. The study found that small sample sizes lead to high variability in performance estimates and larger sample sizes could decrease the impact of outliers. The performance could loss due to small samples and increasing sample size consistently improves reliability. }

\steven{\citet{purohit2018ranking} suggested capping the maximum number of annotation tasks assigned per unit of time to manage workload effectively to mitigate annotator burnout.}

\steven{\citet{pandey2022modeling} mentioned annotator can develop a mental representation of a concept by seeing a sufficient number of examples.}

\steven{\citet{wang2016human} limited users to verify the top-50 in each round, where users did binary classification on whether image was match or not.}

\steven{[explanation]\citet{kulesza2015principles} presents a system that explains the reason behind each prediction for users to better understand the system's logic to tailor the system toward their needs. In the system, users will modify feature weights within the model. n our LLM-powered system, users need to use natural language to guide the system. However, this can be more challenging because large models are less responsive to prompt variations compared to smaller models~\cite{zhuo2024prosa}.}

\steven{The more users understand the underlying system, the more effectively they can control it~\cite{kulesza2012tell}.}

\steven{\citet{teso2023leveraging} discusses a general framework for incorporating explanations into interactive machine learning. Users can get a better understanding of the machine's logic by observing the machine's explanations. [In LLM system, the explanation is the supporting argument for selecting a label.] Once understanding the bugs and limitation, users could modify the algorithm to correct flaws~\cite{kulesza2015principles}. [In our case, user cannot directly modify LLMs but only provide natural language to guide them. Also, subjective tasks does not have universal correct answers, where users need to provide their own standards to steer LLMs. ] }

\steven{[Task Difficulty:] 
A task being too difficult can frustrate users~\cite{zheng2022virtual}, particularly when exceeding their skill level, and a task being to easy can lead to boredom~\cite{zhang2021personalized}.
  These study focused on the impacts of difficulty on users' performance on a pre-defined task. However, in our study, our work prioritizes the dynamics of human-LLM interaction, emphasizing how effective humans could guide LLMs to align with their standard. In this context, the difficulty level of the task itself is less critical, as our primary objective is to assess the effectiveness of human guidance, regardless of the inherent complexity of the task.}


\steven{[task type:]\citet{cayir2016study} found the complexity and definition of a task significantly influence user performance. }

\steven{[task type:] \citet{hettiachchi2022survey} discusses different task assignment methods, including the modeling of worker performance and the impact of task heterogeneity on assignment strategies.
\citet{zhen2021crowdsourcing} provides a detailed exploration of task assignment challenges, task types, and their effects on worker performance and task outcomes. 
}
% \steven{ending of related word}We wanted to design a system to bridge the gap of xxxx: a graphical interface implemented on Google Sheet add-on, generalizing to single-class data annotation tasks, without requiring extensive knowledge of programming and system configuration. By combining the widespread familiarity and advanced features of Google Sheets with large-scale data annotation and iteration tracking, we aimed to make it easier for people to experiment with and benefit from LLMs.

\section{Preliminaries}
\noindent \textbf{Variational autoencoder.}
A variational autoencoder (VAE) ~\citep{kingma2013auto} is a generative model that represents high-dimensional data distributions in a lower-dimensional latent space. 
The encoder maps the input data $\mathbf{x}$ to a latent variable $\mathbf{z}$ by estimating the parameters of a posterior distribution $q_{\phi}(\mathbf{z}|\mathbf{x})$. The posterior is typically assumed to follow the Gaussian distribution, parameterized by a mean $\mu_{\phi}(\mathbf{x})$ and a variance $\sigma_{\phi}(\mathbf{x})$. The latent variable $\mathbf{z}$ is sampled from this posterior distribution, i.e., $\mathbf{z} \sim q_{\phi}(\mathbf{z}|\mathbf{x}) = \mathcal{N}(\mathbf{z}; \mu_{\phi}(\mathbf{x}), \sigma_{\phi}(\mathbf{x})^2)$. The decoder reconstructs the input $\mathbf{x}$ by mapping $\mathbf{z}$ back to the data space through the likelihood $p_{\theta}(\mathbf{x}|\mathbf{z})$. The learning objective of is:
\begin{equation}
\mathcal{L}_{\text{VAE}}(\theta, \phi; \mathbf{X}) = \mathbb{E}_{q_{\phi}(\mathbf{Z}|\mathbf{X})}[\log p_{\theta}(\mathbf{X}|\mathbf{Z})] - \text{KL}(q_{\phi}(\mathbf{Z}|\mathbf{X}) \| p(\mathbf{Z})).
\label{eq:vae}
\end{equation}



\noindent \textbf{3D Gaussian Splatting.}
3DGS~\citep{kerbl3Dgaussians} is an efficient NVS framework that uses a set of 3D Gaussian primitives to represent a scene explicitly. Each Gaussian primitive has a position vector $\boldsymbol{\mu} \in \mathbb{R}^3$, a 3D covariance matrix $\boldsymbol{\Sigma} \in \mathbb{R}^{3\times 3}$, an opacity $\alpha \in \mathbb{R}$, and a spherical harmonics (SH) coefficient  $\boldsymbol{c} \in \mathbb{R}^k$ \citep{ramamoorthi2001efficient} representing the view dependent colors.
\begin{equation}
G(x)=e^{-\frac{1}{2}(x-\boldsymbol{\mu})^T \boldsymbol{\Sigma}^{-1}(x-\boldsymbol{\mu)}},
\end{equation}
where ${\Sigma}={R}{S}{S}^T{R}^T$, ${S}$ denotes the scaling matrix and ${R}$ is the rotation matrix. Then, rasterization~\citep{zwicker2001surface} can transform the 3D Gaussian spheres to the 2D camera plane to calculate the 2D covariance matrix in the camera space as 
\begin{equation}
{\Sigma}^{'} = {J}{W}{\Sigma} {W}^T{J}^T,
\end{equation}
where ${W}$ is the perspective transformation matrix and ${J}$ is Jacobin of the projection matrix.
For every pixel, the Gaussians are traversed in depth order from the image plane, and their pixel colors $c_i$  are combined through alpha compositing, forming pixel color ${C}$ as
\begin{equation}
{C}=\sum_{i \in N} c_i \alpha_i \prod_{j=1}^{i-1}\left(1-\alpha_j\right).
\end{equation}



\section{ORCA Design}
\label{sec:system-design}

%%%%%%%%%%%%%%%%%%%%%%%%%%%%%%%%%%%%%%%%%%%%%
\begin{figure*}[tp]
    \centering
    \includegraphics[width=\textwidth]{figures/overview.png}
    \vspace{-0.5cm}
    \caption{\shepherd{ORCA cloud-assisted design overview.}}
    \vspace{-0.2cm}
    \label{fig:system-overview}
\end{figure*}
%%%%%%%%%%%%%%%%%%%%%%%%%%%%%%%%%%%%%%%%%%%%%

\subsection{Overview}
\label{sec:system-overview}
Based on the observations and discussions in Section~\ref{sec:background-and-related-works} and~\ref{sec:preliminary-study}, we argue that an ideal edge-cloud collaborative learning system over LPWANs should have the following design considerations. First, to tackle the unreliability of wireless channels, a cloud-assisted strategy should be adopted rather than the state-of-the-art cloud-dependent offloading. Second, to adapt to the low bit rates of LPWANs and the on-device resource constraints, we demand a more efficient information exchange strategy. Additionally, from an audio processing perspective, we look for a more effective feature selection method to reduce input size and therefore reduce on-device computation overheads while maintaining comparable accuracy performances. Informed by these demands, we introduce our novel design of a resource-aware cloud-assisted environmental sounds recognition system, primarily operating over LoRa networks. Our system features resource-aware and communication-adaptive cloud assistance, enabling efficient and flexible cloud offloading under resource constraints and unreliable communications. Furthermore, we apply a novel self-attention-based frequency band feature selection method with the wavelet transform to effectively select important features for efficient on-device inference. We illustrate the workflow of the ORCA cloud-assisted framework in Figure~\ref{fig:system-overview}:

\noindent
\shepherd{\textbf{Step~\textcircled{\small{1}}:} Initially, the edge device preprocesses audio signals using low-level WPT to generate a low-resolution spectrogram. Preprocessing details are in Section~\ref{sec:preprocess}, and optimized resolution selection based on wireless channel feedback, e.g., Adaptive Data Rate (ADR),  is discussed in Section~\ref{sec:resource-aware-cloud-assistance}.}

\noindent
\shepherd{\textbf{Step~\textcircled{\small{2}}:} The resulting low-resolution spectrogram is transmitted to the server via uplink LoRa channel, using ADR-recommended parameters.}

\noindent
\shepherd{\textbf{Step~\textcircled{\small{3}}:} Upon receiving the low-resolution spectrogram, the server processes it using a pre-trained contrastive vision transformer~\cite{dosovitskiy2020vit} to extract an attention mask through attention rollout~\cite{abnar2020quantifying}. Details of the cloud model are provided in Section~\ref{sec:attention-mask-generation}.}

\noindent
\shepherd{\textbf{Step~\textcircled{\small{4}}:} The extracted attention mask, along with ADR feedback, is sent back to the edge device via downlink. Resource efficiency adaptations using ADR feedback are further discussed in Section~\ref{sec:resource-aware-cloud-assistance}.}

\noindent
\shepherd{\textbf{Step~\textcircled{\small{5}}:} The edge device validates the received attention mask. If invalid or lost, it bypasses cloud assistance and performs standalone on-device inference. We will also discuss this in Section~\ref{sec:resource-aware-cloud-assistance}.}

\noindent
\shepherd{\textbf{Step~\textcircled{\small{6}}:} If the mask is valid, the edge device refines the resolution to construct a multi-resolution spectrogram with details in Section~\ref{sec:spectral-encoding-cnn}.}

\noindent
\shepherd{\textbf{Step~\textcircled{\small{7}}:} Finally, with the multi-resolution spectrogram, the edge device performs efficient inference using spectral encoding and spectral CNNs with details in Section~\ref{sec:spectral-encoding-cnn}.}

\shepherd{To address resource constraints and dynamic communication costs, ORCA introduces a novel resource-aware scheduler for efficient cloud assistance on batteryless devices. Our algorithm dynamically adjusts to variable communication costs, enabling optimized communication scheduling in scenarios of high communication costs for adaptive transmission and bypassing. We will detail this algorithm in Section~\ref{sec:resource-aware-cloud-assistance}. }



% \begin{figure}[h]
%     \centering
%     \includegraphics[width=0.9\linewidth]{figures/selective-wpt.png}
%     \caption{Wavelet transform based on attention masks for selective frequency band resolution refinement.}
%     \label{fig:selective-WPT}
% \end{figure}




\subsection{Preprocessing}
\label{sec:preprocess}
\shepherd{To minimize communication costs, ORCA employs a low-resolution wavelet spectrogram as a compact and informative abstraction for cloud assistance. 
We use the WPT with depth $n$ to extract coarse frequency-domain features from the input audio waveform, producing a spectrogram $S$ with a frequency dimension of $2^n$. To generalize features over time and reduce payload size, we apply average pooling along the time axis, transforming $S$ into a square matrix $S_a$ in dimension of $2^n$. We refer to $S_a$ as the cloud-assisted spectrogram and define its dimension as the cloud assistance resolution $R_a = 2^n$, with selection details in Section~\ref{sec:resource-aware-cloud-assistance}.}


\subsection{Attention Mask Generation}
\label{sec:attention-mask-generation}
\shepherd{
In this section, we discuss how the server identifies important features from the assistance spectrogram $\mathcal{S}_a$. Specifically, we define important features as the most informative frequency bands, guided by preliminary studies. The edge device then leverages this information, encoded as an attention mask, to enhance on-device inference accuracy in later steps.
}

\noindent
\subsubsection{Vision Transformer for Assistance Spectrogram.}
\shepherd{ORCA server-side design leverages the self-attention mechanism to dynamically encode the importance of input features. The server processes the assistance spectrogram $\mathcal{S}_a$ by patching it into tokens and computing a self-attention map to highlight key regions. We show the attention computation in Figure~\ref{fig:attention-block}.
First, we adopt the same architecture from the vision transformer~\cite{dosovitskiy2020vit} and divide the input spectrogram into $p^2$ patches. To preserve the spectrogram’s spectral-temporal properties, we apply positional encoding by adding trainable encoding to each patch.
Next, we pass the patches through a convolutional patch embedding layer, encoding each patch into an embedding of dimension $E$.
The resulting embedding is passed through the $i$-th attention block to compute the attention matrix $A_i$, sequentially. Formally, $A_i = \text{Softmax}(Q_i \cdot K_i^T / \sqrt{E})$, where $Q_i$ and $K_i$ are the query and key embeddings at each layer. The attention matrix $A_i$ of size $p^2 \times p^2$ captures the relative importance between patch pairs, aiding in identifying the most informative frequency bands, as discussed next.
}

\begin{figure}[tp]
    \centering
    \includegraphics[width=\linewidth]{figures/attention-block.png}
    \vspace{-0.3cm}
    \caption{Attention computation for attention rollout.}
    \label{fig:attention-block}
    \vspace{-0.3cm}
\end{figure}


\noindent
\subsubsection{Attention Mask Generation.}
Recall that the attention matrix $A_i$ represents the attention map of the $i$-th attention block, encoding the relative importance between patches in a spectrogram. 
% Sequential application of attention blocks can cause attention signals to vanish due to constant information mixing between tokens~\cite{abnar2020quantifying}. 
\shepherd{
Inspired by~\cite{abnar2020quantifying}, we compute the rollout attention map $\widetilde{A} = \Pi^{1}_{i=n} A_i = A_n A_{n-1} \cdots A_{1}$ for importance estimations.} This approach aggregates attention matrices from all blocks, enhancing interpretability and preventing attention scores from vanishing. 
The resulting rollout attention map $\widetilde{A}$ has dimensions $p^2 \times p^2$. 
Then, we aim to identify the most informative frequency bands for the edge. Intuitively, a frequency band is informative if patches within that band have high attention scores, as this indicates that the cloud model prioritizes those patches. Therefore, let $\widetilde{a}_{ij}$ represent the rollout attention between patches $i$ and $j$ in $\widetilde{A}$. We compute the column-wise summation $C$ of $\widetilde{A}$ as $C = [c_1, c_2, \cdots, c_{p^2}]$ where $c_j = \sum_{i=1}^{p^2} \widetilde{a}_{ij}$. 
The vector $C$ is reshaped into a 2D importance matrix $C'\in \mathbb{R}^{p \times p}$, where each entry represents the importance of a patch in the input WPT spectrogram. 
\shepherd{We select frequency bands by summing contiguous $k$ rows in $C'$ and identifying the highest sum, where $k$ is a predefined hyperparameter agreed upon by the server and edge device.} A binary vector of length $p$ records the selected indices, forming the spectral attention mask, which is sent to edge devices.

\noindent
\subsubsection{Contrastive Pre-Training.} \shepherd{The method above relies on a vision transformer capable of identifying informative frequency bands from the WPT spectrogram.} Given the lack of labeled data for frequency-domain feature importance information, we propose training the cloud model offline in an unsupervised manner. \shepherd{Inspired by contrastive learning, where the model learns to produce distinctive features via contrastive loss, we create attracting and contrasting pairs by masking random frequency bands and use triplet loss~\cite{schroff2015facenet} on the flattened output of vision transformer as representations.} Overall, the advantage of ORCA attention-based cloud assistance solution is twofold: first, it uses self-attention over spectrograms to guide clients in focusing on informative frequency bands, which not only improves inference accuracy on the resource-constrained edge devices but also reduces computational load by minimizing the edge model input size. Additionally, transmitting the low-resolution assistance spectrogram and attention masks is highly communication-efficient, significantly reducing communication costs and latency.


\subsection{\shepherd{Cloud-Assisted Inference}}
\label{sec:spectral-encoding-cnn}
\shepherd{
% Spectrograms provide detailed information across all frequency bands, and wavelet transform enables extraction of frequency-band details at flexible resolutions. In preliminary study, we demonstrated that higher-resolution spectrograms achieve high accuracy but increase resource usage. Inspired by the preliminary study 2, we propose a three-fold on-device inference solution: Multi-resolution Refinement, Spectral Encoding, and Multi-resolution CNN designs, selectively focusing on information-rich spectral bands from the aforementioned cloud assistance step. and still maintain high accuracy. This section details our design of a multi-resolution spectral encoding CNN (Steps \textcircled{\small{6}} and \textcircled{\small{7}} in Figure~\ref{fig:system-overview}), optimized for resource-constrained devices through the use of cloud-generated spectral attention masks discussed in the previous section.
Following the discussion on server-generated attention masks, we explore how edge devices can leverage this information for efficient on-device inference. First, we introduce the \textit{Multi-resolution Refinement} module, which extracts high-resolution frequency bands guided by attention masks. After refinement, two challenges remain: (i) embedding high-resolution spectral bands and (ii) creating a multi-resolution representation for accurate and efficient inference. For (i), we propose \textit{Spectral Encoding}, a trainable weight that encodes high-resolution frequency band-specific knowledge. For (ii), we employ \textit{Multi-resolution CNNs} to process the combination of high-resolution bands from multi-resolution refinement and their corresponding spectral encoding for efficient on-device classification.}

\noindent
\subsubsection{Multi-resolution Refinement.}
\shepherd{The server-generated spectral attention mask captures key frequency bands. It guides the edge device to selectively extract high-resolution spectrograms via wavelet transform. Let $R_l$ denote the pre-defined dimension of the low-resolution spectrogram and $R_h$ the dimension of the high-resolution spectral bands, this refinement results in $R_l$-dimensional low-resolution spectrograms and $R_h$-dimensional high-resolution spectrograms frequency bands.} To further reduce dimension, adaptive average pooling is applied along the time dimension, regularizing the size of both spectrograms.

\noindent
\subsubsection{Spectral Encoding.}
\shepherd{Since each frequency band captures unique frequency-domain properties, spectrograms from different bands should be interpreted accordingly. Using separate CNNs per band~\cite{phaye2019subspectralnet} is memory-inefficient and costly. Instead, inspired by transformer's positional encoding, we use spectral encoding, a trainable weight that encodes frequency band-specific information. It is then concatenated channel-wise to corresponding high-resolution bands, as shown in Figure~\ref{fig:spectral-encoding}. This approach helps the network to learn spectral-specific knowledge independently of the input spectrogram.}

\begin{figure}[tp]
    \centering
    \includegraphics[width=\linewidth]{figures/spectral-encoding.png}
    \vspace{-0.8cm}
    \caption{\shepherd{Spectral encoding and multi-resolution CNNs.}}
    \label{fig:spectral-encoding}
    \vspace{-0.4cm}
\end{figure}


\noindent
\subsubsection{Multi-resolution CNN.}
\shepherd{The next challenge is to create a multi-resolution representation for inference. As discussed in preliminary studies in Section~\ref{sec:preliminary-study}, discriminative information varies between spectral bands of the spectrogram. With the full low-resolution spectrogram available from preprocessing, we use two 2-layer shallow CNN as encoders, one for low resolution and one for high resolution. The encoded features are fused channel-wise into a single vector and fed into the Multi-Res classifier for final classification. This architecture reduces inference costs by leveraging spectrograms at different resolutions. If cloud assistance is unavailable, an additional Single-Res classifier is employed to process the output of the Low-Res encoder only. All components are pre-trained offline in a two-stage supervised process. First, we train the low-res encoder, high-res encoder, and multi-resolution classifier together with the attention masks generated by the pre-trained cloud vision transformer. In the second stage, we freeze all other components and train the single-resolution classifier independently.}

%%%%%%%%%%%%%%%%%%%%%%%%%%%%%%%%%%%%%%%%%%%%%%%%%%%%%%%%%%%%%%%
\begin{figure}[tp]
    \centering
    \includegraphics[width=\linewidth]{figures/intermittent.png}
    \vspace{-1.0cm}
    \caption{Execution model (up), capacitor voltage (mid), and relative power consumptions (low).  }
    \label{fig:intermittent}
\end{figure}
%%%%%%%%%%%%%%%%%%%%%%%%%%%%%%%%%%%%%%%%%%%%%%%%%%%%%%%%%%%%%%%

\subsection{Resource-Aware Scheduler}
\label{sec:resource-aware-cloud-assistance}
\shepherd{Given the high energy cost of communication and wireless uncertainty, dynamically managing data transmission size is essential for resource-efficient cloud assistance.} Experimental measurements~\cite{mileiko2023run} indicate that the uplink phase dominates energy consumption in each communication round and varies with channel conditions. Thus, a key component of our framework is optimizing uplink data transmission. 
\shepherd{We introduce a resource-aware, communication-adaptive resolution algorithm. This algorithm dynamically schedules the assistance resolution $R_a$ (as discussed in Section~\ref{sec:preprocess}) based on energy storage and communication quality for resource-efficient cloud assistance.}

\noindent
\subsubsection{Communication Model.} 
As discussed in Section~\ref{sec:system-overview}, ORCA uses two communication phases for one round of cloud assistance, uplink (Tx) and downlink (Rx). It adopts the intermittent computation model from~\cite{mileiko2023run} which concludes an uplink and a downlink in the same power cycle with a synchronized sleep period interleaved. \shepherd{The key advantage of this design is maintaining inference integrity and timeliness for cloud assistance, even during prolonged power failures in batteryless systems.} We illustrate this design in Figure~\ref{fig:intermittent}. Within one power cycle, edge device initiates by restoring the communication parameters, spreading factor (SF) and transmitting power ($P_{\text{Tx}}$) once waking up at voltage threshold $V_{\text{on}}$. Then it goes through sampling and preprocessing, Tx, sleeping, Rx, and on-device inference sequentially as discussed in Section~\ref{sec:system-overview}. Between each power cycle, our edge device checkpoints and restores SF and $P_{\text{Tx}}$ in and out of the non-volatile memory (yellow blocks in Figure~\ref{fig:intermittent}). This ensures their synchronizations to the server's recommendation for reliable communication. Here, the generic ADR algorithm~\cite{Semtech2016LoRaWAN} is employed to estimate the optimal communication parameters ensuring communication reliability. Every time the server receives an uplink packet, it calculates and compares the SNR margins to the optimal values and recommends the optimal SF and $P_{\text{Tx}}$ back to the edge device in downlink message. Edge device can then checkpoint these parameters for next round of communication. The next challenge is to complete restoring, preprocessing, Tx, sleep, Rx, inference, and checkpointing within one power cycle.

% Additionally, based on the measurements by~\cite{mileiko2023run}, uplink dominates the energy consumption and therefore, our algorithm mainly focuses on optimizing uplink data transmission.
% For uplink, the transmission time, known as the time-on-air (ToA), depends on the data rate (DR) and can be estimated by $\text{ToA} = \frac{S + S_{\text{p}}}{\text{DR}}$ for sending a payload size of $S$ with a fixed preamble $S_{\text{p}}$. 
% Conversely, a fixed time window ($T_{\text{Rx}}$) is set up for listening to the downlink signal after the sleeping period timeout. 
% Given the variability of the wireless channel, the costs of ensuring signal quality for the uplink changes constantly. To address this, we employ the adaptive data rate (ADR)~\cite{Semtech2016LoRaWAN}. By analyzing SNR history, the server recommends an optimal spreading factor (SF) and transmit power ($P_{\text{Tx}}$) for the edge device. Edge device then adjusts these parameters to ensure reliable uplink transmission. 



\begin{figure}[tp]
    \centering
    \includegraphics[width=\linewidth]{figures/workflow.png}
    \vspace{-0.5cm}
    \caption{Cloud-assisted offloading flow chart.}
    \label{fig:workflow}
    \vspace{-0.2cm}
\end{figure}

\noindent
\subsubsection{Adaptive Resolution.}
\shepherd{Given the proposed communication model and parameters, we first examine the key factors influencing energy consumption.} Since batteryless devices usually wake up at a pre-defined voltage threshold, the energy budget per power cycle is typically fixed and can be estimated by $E_{\text{cap}}=\frac{C}{2}(V_{\text{on}}^2 - V_{\text{off}}^2)$, where $C$ is capacitance, and $V_{\text{on}}$ and $V_{\text{off}}$ represent the microcontroller switching voltage thresholds (on and off, respectively), as depicted in Figure~\ref{fig:intermittent}. 
We propose ORCA resource-aware adaptive resolution algorithm for cloud assistance, designed to adapt to varying communication costs and complete each round of cloud assistance within a single power cycle. Our approach determines an optimal assistance resolution $R_a$ which in turn defines the payload size $S={R_a}^2$ for uplink. We model the adaptive resolution algorithm with the parameters followed. The uplink energy consumption $E_{\text{Tx}}$ can be estimated as: $E_{\text{Tx}} = P_{\text{Tx}} \cdot \text{ToA} = P_{\text{Tx}} ({R_a}^2+S_{\text{p}})/\text{DR}$.
where the uplink transmission time, known as the time-on-air (ToA), depends on the various data rate (DR) under different SF and can be estimated by $\text{ToA} = (S + S_{\text{p}})/\text{DR}$ for sending a payload size of $S$ with a fixed preamble $S_{\text{p}}$. 
The downlink energy cost is estimated as $E_{\text{Rx}}=P_{\text{Rx}} \cdot T_{\text{Rx}}$, the product of the downlink power and the downlink window length. Additionally, $E_{\text{Pre}}$, $E_{\text{sleep}}$, and $E_{\text{inf}}$ are for energy usage during preprocessing, sleep period, and inference, respectively, and can be considered as constants in ORCA. Moreover, to formulate the optimization problem, we define the one-hot encoded resolution selection vector $x$ for resolution ${R_a}$ and the pre-estimated accuracy vector $a$ for accuracy under different ${R_a}$ values. To complete a round of cloud assistance within a single power cycle, the model ensures $E_{\text{Pre}} + E_{\text{Tx}} + E_{\text{sleep}} + E_{\text{Rx}} + E_{\text{inf}} \leq E_{\text{cap}}$. 
We define the following optimization problem, finding the optimal resolution selection vector $x$ to maximize the accuracy under energy constraints:
% \setlength{\abovedisplayskip}{2pt}
% \setlength{\belowdisplayskip}{2pt}
\begin{equation*}
\begin{aligned}
\max_{x} \ a^{T}x \quad \textrm{s.t.} \quad & E_{\text{Tx}}(x)+E_{\text{Pre}} + E_{\text{sleep}} + E_{\text{Rx}} + E_{\text{inf}}\leq E_{\text{cap}}\\[-0.2em] 
  &\textbf{1}^Tx = 1, \ x_i = \{0, 1\} \\[-0.2em]
  \end{aligned}
\end{equation*}
The optimal resolution selection is derived by ${R_a}=\text{argmax}(x)$, and, specifically, we define ${R_a}=0$ as local bypassing without cloud assistance. In practice, since the optimization search space is small (as $R_a$ is chosen from only a few options) and the capacitor is pre-selected to ensure enough budget for at least local inference without cloud assistance, we simply iterate through all feasible solutions within the energy budget and select the one with the highest estimated accuracy.

\noindent
\subsubsection{Workflow.}
The workflow is presented in Figure~\ref{fig:workflow}. Starting with the communication parameters in the yellow block, we use the energy storage $E_{\text{cap}}$ and communication parameter recommendations from the previous round as the budget and cost inputs, respectively. These inputs are applied to the optimization problem, where the edge device determines the optimal $R_a$ for maximum assistance accuracy and then uploads the low-resolution spectrogram. The server extracts and transmits the attention masks along with the ADR in the downlink back to the edge device. The edge device verifies downlink message validity using the CRC error check or by missing packets after a downlink timeout, treating invalid messages as such. If valid, the edge device proceeds with the multi-resolution inference step as described in Section~\ref{sec:spectral-encoding-cnn}. Otherwise, due to resource constraints, the device bypasses retransmission and cloud assistance, performing single-resolution on-device inference as also detailed in Section~\ref{sec:spectral-encoding-cnn}. Overall, ORCA using fixed energy budgets and dynamic data size offers two major advantages. First, unlike reconfigurable energy storage solutions, which require additional hardware and may face durability or read-write cycle limitations~\cite{colin2018reconfigurable, bakar2022protean, mileiko2023run}, our strategy does not require extra hardware. Second, our algorithm intelligently balances communication costs and accuracy gains by adaptively selecting the amount of resources for cloud assistance. As shown in Figure~\ref{fig:resource-aware}: (i) when communication cost is low, the edge device sends a high-resolution spectrogram for better inference accuracy; (ii) when communication cost is high, it sends a low-resolution spectrogram with a smaller payload to manage energy cost, resulting in lower accuracy; (iii) if communication is unstable with packet loss, the device bypasses cloud assistance and performs local inference to avoid costly retransmissions.


\begin{figure}[tp]
    \centering
    \includegraphics[width=\linewidth]{figures/resource-aware.png}
    \vspace{-0.5cm}
    \caption{Resource-aware adaptive resolution for cloud assistance.}
    \label{fig:resource-aware}
    \vspace{-0.2cm}
\end{figure}




\begin{table*}[t]
\centering
\tiny
\begin{tabular}{|M{1.2cm}|M{0.7cm}|M{1cm}|M{1cm}|M{1cm}|M{0.8cm}|M{1.2cm}|M{0.7cm}|M{1cm}|M{1cm}|M{1cm}|M{0.8cm}|}
\hline\hline
Model & \#GPU & \#Strategies & Search Time(/s) & Simulation Time(/s) & E2E Time(/s) & Model & \#GPU & \#Strategies & Search Time(/s) & Simulation Time(/s) & E2E Time(/s) \\ \hline
\multirow{4}{*}{Llama-2-7B} & 64 & 23348 & 0.06 & 49.7 & 51.0 & \multirow{4}{*}{Llama-2-13B} & 64 & 23400 & 0.05 & 58.1 & 59.3 \\ \cline{2-6} \cline{8-12} 
 & 256 & 14372 & 0.05 & 43.5 & 44.4 &  & 256 & 13552 & 0.03 & 49.9 & 50.8 \\ \cline{2-6} \cline{8-12} 
 & 1024 & 8856 & 0.04 & 41.8 & 42.2 &  & 1024 & 8920 & 0.02 & 51.0 & 51.7 \\ \cline{2-6} \cline{8-12} 
 & 4096 & 4700 & 0.03 & 33.0 & 33.2 &  & 4096 & 4720 & 0.02 & 44.1 & 44.3 \\ \hline
\multirow{4}{*}{Llama-2-70B} & 64 & 53264 & 0.1 & 68.8 & 75.0 & \multirow{4}{*}{Llama-3-8B} & 64 & 23348 & 0.05 & 48.3 & 49.6 \\ \cline{2-6} \cline{8-12} 
 & 256 & 31440 & 0.06 & 57.7 & 60.9 &  & 256 & 14372 & 0.04 & 42.0 & 42.8 \\ \cline{2-6} \cline{8-12} 
 & 1024 & 20152 & 0.05 & 57.4 & 59.6 &  & 1024 & 8856 & 0.03 & 40.9 & 41.3 \\ \cline{2-6} \cline{8-12} 
 & 4096 & 10948 & 0.04 & 63.2 & 65.0 &  & 4096 & 4700 & 0.03 & 32.7 & 32.9 \\ \hline
\multirow{4}{*}{Llama-3-70B} & 64 & 53264 & 0.1 & 66.8 & 71.8 & \multirow{4}{*}{GLM-67B} & 64 & 20528 & 0.04 & 19.3 & 20.6 \\ \cline{2-6} \cline{8-12} 
 & 256 & 31440 & 0.07 & 56.3 & 59.6 &  & 256 & 12132 & 0.03 & 16.6 & 17.4 \\ \cline{2-6} \cline{8-12} 
 & 1024 & 20152 & 0.05 & 55.5 & 57.6 &  & 1024 & 7948 & 0.02 & 16.9 & 17.3 \\ \cline{2-6} \cline{8-12} 
 & 4096 & 10948 & 0.04 & 62.4 & 63.4 &  & 4096 & 4196 & 0.02 & 21.3 & 21.5 \\ \hline
\multirow{2}{*}{GLM-130B} & 64 & 33540 & 0.06 & 22.4 & 52.4 & \multirow{2}{*}{GLM-130B} & 1024 & 11976 & 0.03 & 16.7 & 18.2 \\ \cline{2-6} \cline{8-12} 
 & 256 & 18776 & 0.04 & 17.2 & 19.4 &  & 4096 & 6040 & 0.02 & 19.2 & 20.1 \\ \hline\hline
\end{tabular}%
\caption{
    The search space and the time cost for \sysname on Heterogeneous GPUs.
  For the pictures of time cost, the light color without hatches represents the time spent searching, while the deep color with hatches represents the time spent simulating.
  We can observe that it only takes \sysname\ about 1 minute to complete the end-to-end simulation. 
}
\label{tab:exp:cost}
\end{table*}

\section{Experiments}\label{sec:exp}


%In this section, we first evaluate \sysname's cost model accuracy under different settings to build the basis for the search in \S\ref{sec:exp:accuracy}.
%We show the search space of \sysname, and the search time cost for the search in \S\ref{sec:exp:cost}.
%Then, t
To prove \sysname's optimal search ability on MegatronLM, we did a comparative analysis between \sysname\ and experts on MegatronLM in \S\ref{sec:exp:expert}.
%After that, we compare \sysname with existing auto-parallel frameworks, including Alpa, Galvatron, etc., in \S\ref{sec:exp:comparison}.
Finally, we evaluate \sysname to search for the finance-optimal plan under different settings in \S\ref{sec:exp:finance}.

%\subsection{Cost Model Accuracy}\label{sec:exp:accuracy}
%



\section{Cost Analysis}\label{sec:exp:cost}

\sssec{Method}.
We did a cost analysis to show the gap between the large search space and the search efficiency of the \sysname.
We selected Llama-2 models (7B, 13B, and 70B) with 64, 256, 1024, and 4096 GPUs.
Then, for all the settings, we implemented \sysname\ on it and recorded the searched strategy number along with the end-to-end time (search time and simulation time)


\sssec{Result}. As shown in Table \ref{tab:exp:cost}, the number of explored strategies grows exponentially with model size. For smaller models like Llama-7B, even with 4096 GPUs, the search space remains relatively small. However, for larger models such as Llama-70B, the search space nearly triples compared to Llama-7B under the same GPU configuration. The end-to-end time reveals that the simulation phase is the main bottleneck, which may take 1 minute to execute on average. While the search time only takes less than 1 second to execute on average. This highlights the need for optimizing the simulation process, particularly in large-scale settings, while \sysname’s search algorithm remains efficient and scalable across different configurations.




\begin{figure*}[thbp]
  \centering
    \subfloat{\includegraphics[width=0.4\textwidth]{figs/fig-expert-legend.pdf}}\\
    \addtocounter{subfigure}{-1}

    \begin{minipage}{\textwidth}
    {\centering{\hspace{2.8cm}A800\hspace{4cm}H100\hspace{4.2cm}H800}}
    \end{minipage}

    \raisebox{0.8cm}{\rotatebox[origin=c]{90}{Llama-2}}
    \subfloat[7B]{\includegraphics[width=0.106\textwidth]{figs/fig-expert-A800-llama2-7b.pdf}}
    \subfloat[13B]{\includegraphics[width=0.106\textwidth]{figs/fig-expert-A800-llama2-13b.pdf}}
    \subfloat[70B]{\includegraphics[width=0.106\textwidth]{figs/fig-expert-A800-llama2-70b.pdf}}
    \subfloat[7B]{\includegraphics[width=0.106\textwidth]{figs/fig-expert-H100-llama2-7b.pdf}}
    \subfloat[13B]{\includegraphics[width=0.106\textwidth]{figs/fig-expert-H100-llama2-13b.pdf}}
    \subfloat[70B]{\includegraphics[width=0.106\textwidth]{figs/fig-expert-H100-llama2-70b.pdf}}
    \subfloat[7B]{\includegraphics[width=0.106\textwidth]{figs/fig-expert-H800-llama2-7b.pdf}}
    \subfloat[13B]{\includegraphics[width=0.106\textwidth]{figs/fig-expert-H800-llama2-13b.pdf}}
    \subfloat[70B]{\includegraphics[width=0.106\textwidth]{figs/fig-expert-H800-llama2-70b.pdf}}
    \\
    \raisebox{0.8cm}{\rotatebox[origin=c]{90}{Llama-3}}
    \subfloat[8B]{\includegraphics[width=0.16\textwidth]{figs/fig-expert-A800-llama3-8b.pdf}}
    \subfloat[70B]{\includegraphics[width=0.16\textwidth]{figs/fig-expert-A800-llama3-70b.pdf}}
    \subfloat[8B]{\includegraphics[width=0.16\textwidth]{figs/fig-expert-H100-llama3-8b.pdf}}
    \subfloat[70B]{\includegraphics[width=0.16\textwidth]{figs/fig-expert-H100-llama3-70b.pdf}}
    \subfloat[8B]{\includegraphics[width=0.16\textwidth]{figs/fig-expert-H800-llama3-8b.pdf}}
    \subfloat[70B]{\includegraphics[width=0.16\textwidth]{figs/fig-expert-H800-llama3-70b.pdf}}
    \\
    \raisebox{0.8cm}{\rotatebox[origin=c]{90}{GLM}}
    \subfloat[67B]{\includegraphics[width=0.16\textwidth]{figs/fig-expert-A800-glm-67b.pdf}}
    \subfloat[130B]{\includegraphics[width=0.16\textwidth]{figs/fig-expert-A800-glm-130b.pdf}}
    \subfloat[67B]{\includegraphics[width=0.16\textwidth]{figs/fig-expert-H100-glm-67b.pdf}}
    \subfloat[130B]{\includegraphics[width=0.16\textwidth]{figs/fig-expert-H100-glm-130b.pdf}}
    \subfloat[67B]{\includegraphics[width=0.16\textwidth]{figs/fig-expert-H800-glm-67b.pdf}}
    \subfloat[130B]{\includegraphics[width=0.16\textwidth]{figs/fig-expert-H800-glm-130b.pdf}}
  \caption{
  We compare \sysname's searched optimal plan's throughput with expert's proposed plan's throughput in single-GPU setting.
  }
  \label{fig:expert:throughput}
  \vspace{-10pt}
\end{figure*}

\subsection{Mode-1: Comparison with Expert Plans}\label{sec:exp:expert}

\sssec{Method}.
To prove the \sysname's ability to search the optimal strategy on MegatronLM, we compared \sysname\ with an expert.
We first selected three models with different parameter sizes (7 model settings in total): Llama-2 (7B, 13B, and 70B), Llama-3 (8B, 70B), and GLM (67B, 130B).
Then, we offer 4 GPU number settings: 32, 128, 256, and 1024.
Next, we asked six experts to craft a parallel strategy for each setting (different models and different GPU settings, overall $7\times 4=28$ settings) based on their expert experience.
Each participant has over six years of industry machine learning service or training experience.
Then, we ran each of the six participants' parallel strategies for each setting on MegatronLM and picked the optimal one (one with the largest throughput) among the six expert-crated strategies as the expert-optimal strategy.
At last, we run \sysname\ to search the optimal parallel strategy automatically and compare the \sysname's parallel strategy's throughput with the expert-optimal parallel strategy's throughput.

\sssec{Results}.
As shown in Fig. \ref{fig:expert:throughput}, \sysname demonstrates its ability to automatically generate parallel strategies that match or exceed expert-tuned plans across various model configurations. This highlights \sysname's capability to generalize and optimize without manual intervention.

\par A key finding is that \sysname consistently matches or outperforms manually designed strategies, showing that its automated search can achieve results on par with domain experts. This adaptability extends across diverse hardware and model types, while specific setups often constrain expert-tuned plans. \sysname dynamically adjusts to different configurations, optimizing parallel strategies based on the specific training environment.

\par Another important observation is \sysname’s flexibility in combining different parallelism techniques—data, tensor, and pipeline. While expert strategies often focus on one type of parallelism, \sysname optimally balances multiple forms, leading to superior performance, especially for large-scale models. This hybrid approach is likely the key to future parallelism strategies, where flexibility and adaptation are critical.
%\subsection{Comparison with Other Schemes}\label{sec:exp:comparison}

\begin{table}[h!]
\centering
\caption{GPT-3 Model Specification}
\label{tab:gpt-3}
\begin{tabular}{ccccc}
\hline
\#params & Hidden size & \#layers & \#heads & \#gpus \\ \hline\hline
350M & 1024 & 24 & 16 & 1 \\ 
1.3B & 2048 & 24 & 32 & 4 \\ 
2.6B & 2560 & 32 & 32 & 8 \\ 
6.7B & 4096 & 32 & 32 & 16 \\ 
15B & 5120 & 48 & 32 & 32 \\ 
39B & 8192 & 48 & 64 & 64 \\ \hline\hline
\end{tabular}
\end{table}


\begin{table}[h!]
\centering
\caption{LLaMA Model Specification}
\label{tab:llama}
\begin{tabular}{ccccc}
\hline
\#params & Hidden size & \#layers & \#heads & \#gpus \\ \hline\hline
7B & 4096 & 32 & 32 & 8 \\
13B & 5120 & 40 & 40 & 16 \\
33B & 6656 & 60 & 52 & 32 \\
70B & 8192 & 80 & 64 & 64 \\ \hline\hline
\end{tabular}
\end{table}

\begin{table}[h!]
\centering
\caption{GShard MoE Model Specification}
\label{tab:moe}
\begin{tabular}{cccccc}
\hline
\#params & Hidden size & \#layers & \#heads & \#experts & \#gpus \\ \hline\hline
380M & 768 & 8 & 16 & 8 & 1 \\
1.3B & 768 & 16 & 16 & 16 & 4 \\
2.4B & 1024 & 16 & 16 & 16 & 8 \\
10B & 1536 & 16 & 16 & 32 & 16 \\
27B & 2048 & 16 & 32 & 48 & 32 \\
70B & 2048 & 32 & 32 & 64 & 64 \\ \hline\hline
\end{tabular}
\end{table}

\sssec{Models and training workflows}.
For our experiments, we target three types of models: GPT-3, LLaMA, and a Mixture of Experts (MoE) model. These models represent a range of architectures, from homogeneous to heterogeneous, providing a comprehensive evaluation of our parallelism strategies. 

\par \textbf{GPT-3} (see Table \ref{tab:gpt-3}) is a homogeneous Transformer-based language model comprising many stacked layers. Its model parallelization plan has been extensively studied and optimized in various research efforts. \textbf{LLaMA} (see Table \ref{tab:llama}) is another advanced Transformer-based model designed for language modeling, with a focus on efficiency and performance in both pre-training and fine-tuning phases. \textbf{MoE} models (see Table \ref{tab:moe}), such as GShard, combine dense and sparse architectures by incorporating a mixture of expert layers. These layers replace the feed-forward layers in every few Transformer layers, making them highly adaptable to different computational environments.

\par To study the scalability and efficiency of training large models, we follow standard machine learning practices by scaling the model size proportionally with the number of GPUs, as reported in Table 4. For GPT-3, we increase the hidden size and the number of layers concurrently with the number of GPUs, following the methodology used in previous studies. For the MoE model, we primarily increase the number of experts, which is crucial for leveraging the model's sparse architecture and optimizing performance across multiple GPUs. For LLaMA, we adjust the model's depth (number of layers) and width (hidden size) to ensure it scales effectively with the available GPU resources.

\par In each experiment, we adopt the recommended global batch size per established ML practices to maintain consistent statistical behavior across different model configurations. We then fine-tune the micro-batch size for each model and system configuration to maximize overall system performance, with gradient accumulation applied across micro-batches.

\sssec{Baselines}. For each model, we compare our system, \sysname, against strong baselines, including Alpa and Galvatron, and manually designed strategies using Megatron-LM.

\par \textbf{Alpa} is chosen as one of the baselines due to its automated parallelization capabilities, particularly for large-scale models. Alpa utilizes a combination of intra-operator and inter-operator parallelism to optimize the training process. We configure Alpa to its best settings by following the guidelines provided in their documentation and research papers. Alpa is known for its comprehensive strategy space, which includes various parallelism paradigms such as data parallelism, tensor parallelism, and pipeline parallelism.

\par \textbf{Galvatron} is another baseline we employ, noted for its efficient transformer training over multiple GPUs using automatic parallelism. Galvatron incorporates multiple popular parallelism dimensions and automatically discovers the most efficient hybrid parallelism strategy through a decision tree decomposition and a dynamic programming search algorithm. We perform a grid search to determine the optimal configurations for Galvatron, ensuring that we fully leverage its capabilities.

\par \textbf{Megatron-LM} serves as the manually designed baseline, specifically for GPT-like models. Megatron-LM v2 is a state-of-the-art system that combines data parallelism, pipeline parallelism, and manually designed operator parallelism (denoted as TMP). This combination is controlled by three integer parameters that specify the degrees of parallelism assigned to each technique. Following the guidance from their research, we conduct a thorough grid search of these parameters and report the best configuration results. While Megatron-LM is highly specialized for GPT-like models, it does not support other models in our evaluation due to its lack of flexibility in handling different architectures.

Our comparison does not include open-source systems like \textbf{FlexFlow} and \textbf{Tofu} due to their limitations. FlexFlow lacks support for essential operators such as layer normalization and mixed-precision operators, and Tofu only supports single-node execution and is not open-sourced. Given these theoretical and practical constraints, we do not expect FlexFlow or Tofu to outperform the state-of-the-art manual baselines in our evaluation.

In summary, our evaluation includes \sysname, Alpa for its automated strategy space, Galvatron for its efficient hybrid parallelism discovery, and manually tuned Megatron-LM for its specialization in GPT-like models. This comprehensive approach thoroughly compares different parallelism strategies and model architectures.

\sssec{Evaluation metrics}. We measure training throughput in our evaluation. We evaluate the system's weak scaling when increasing the model size and the number of GPUs. Following \cite{narayanan2021efficient}, we use the aggregated peta floating-point operations per second (PFLOPS) of the whole cluster as the metric. After proper warmup, we measure it by running a few batches with dummy data. All our results (including those in later sections) have a standard deviation within 0.5\%, so we skip the error bars in our figures.

\sssec{GPT-3 results}.
\textcolor{red}{To be done}

\sssec{Llama results}.
\textcolor{red}{To be done}

\sssec{MoE results}.
\textcolor{red}{To be done}

\subsection{Mode-2: Heterogeneous GPU Search}

\begin{figure}[t]
  \centering
    \subfloat{\includegraphics[width=0.48\textwidth]{figs/fig-heter-legend.pdf}}\\
    \addtocounter{subfigure}{-1}
    
    \subfloat[Llama-2-7B]{\includegraphics[width=0.16\textwidth]{figs/fig-heter-llama2-7b.pdf}}
    \subfloat[Llama-2-13B]{\includegraphics[width=0.16\textwidth]{figs/fig-heter-llama2-13b.pdf}}
    \subfloat[Llama-2-70B]{\includegraphics[width=0.16\textwidth]{figs/fig-heter-llama2-70b.pdf}}
    \\

    \subfloat[Llama-3-8B]{\includegraphics[width=0.24\textwidth]{figs/fig-heter-llama3-8b.pdf}}
    \subfloat[Llama-3-70B]{\includegraphics[width=0.24\textwidth]{figs/fig-heter-llama3-70b.pdf}}
    \\

    \subfloat[GLM-67B]{\includegraphics[width=0.24\textwidth]{figs/fig-heter-glm-67b.pdf}}
    \subfloat[GLM-130B]{\includegraphics[width=0.24\textwidth]{figs/fig-heter-glm-130b.pdf}}
  \caption{
  For the heterogeneous GPU search scene, we compare expert-designed strategies's throughput with \sysname-searched strategies.
  The results prove the that \sysname achieves better throughput in heterogeneous scene.
  }
  \label{fig:exp:heter}
\end{figure}

% Please add the following required packages to your document preamble:
% \usepackage{graphicx}
\begin{table}[t]
\centering
\resizebox{0.5\textwidth}{!}{%
\begin{tabular}{c|cccc}
\hline
Model & H100 & H800 & A800 & Heter. \\ \hline\hline
Llama-2-7B & 10148287 & 9024716 & 3966756 & 5240609 \\
Llama-2-13B & 5721253 & 4937998 & 2187876 & 3040095 \\
Llama-2-70B & 1233850 & 1174362 & 458719 & 654206 \\
Llama-3-8B & 9167338 & 7610698 & 3586433 & 4660743 \\
Llama-3-70B & 1129568 & 1079507 & 425660 & 626050 \\
GLM-67B & 1288107 & 1218933 & 483384 & 699978 \\
GLM-130B & 508377 & 491088 & 202137 & 300193 \\ \hline\hline
\end{tabular}%
}
\caption{
We compare heterogeneous GPU with single-GPU search's optimal strategies' throughput.
The experiment is conducted with 1024 GPUs.
And the heterogeneous GPU setting is activated with A800 and H100.
}
\label{tab:exp:heter}
\end{table}

\sssec{Method}.
To evaluate \sysname's performance in heterogeneous GPU environments, we conducted a comprehensive comparison of \sysname-searched strategies and expert-designed strategies under heterogeneous GPU configurations. 
We use \sysname in the two GPU-heterogeneous environments with Nvidia H100 and A800 activated for search.
Also, we follow the design of \S\ref{sec:exp:expert}, we recruit six experts to craft a heterogeneous parallel strategy for each setting, and we picked the optimal one as the expert-designed strategy.
We offer 4 GPU number settings: 64, 256, 1024, and 4096.

Besides that, we also compared the heterogeneous GPU setting with single GPU setting in the same GPU number setting (1024).
We compare the throughput between the different settings (only A100, H100, H800, and heterogeneous settings)

\sssec{Results}.
As shown in Fig. \ref{fig:exp:heter}, our experiments reveal that \sysname consistently achieves higher throughput than expert-tuned configurations, particularly with larger models. \sysname’s approach dynamically balances data, tensor, and pipeline parallelism across heterogeneous GPUs, a task often challenging for manual tuning. This adaptability highlights the efficiency of automated strategies, especially in cloud-based or distributed environments where GPU types may vary. Overall, \sysname’s heterogeneous GPU search framework offers a scalable, cost-effective solution for optimizing model training in heterogeneous hardware contexts.

Table \ref{tab:exp:heter} shows the heterogeneous GPU setting compared with a single GPU setting.
Though a heterogeneous GPU setting strategy can not beat the performance of a single-GPU setting strategy, \sysname's searched strategy can nearly match with them.
\subsection{Mode-3: Evaluation Performance on Financial Cost}\label{sec:exp:finance}

%\sssec{Models and training workflows}.

\sssec{Search pools for GPU}. To comprehensively evaluate the financial cost performance of \sysname, we incorporate a variety of GPU types commonly used by major cloud service providers. Our search pools include the following GPU models: NVIDIA H100, A800 and H800.

These GPUs represent a range of performance capabilities and costs, providing a realistic and comprehensive basis for evaluating the financial efficiency of our system. By including these diverse GPU options, we can simulate the decision-making process of users who leverage cloud-based GPU resources, allowing us to optimize for both time and financial cost under various configurations.

\begin{figure}[t]
  \centering
    \subfloat[Per Throu. Llama-70B]{\includegraphics[width=0.24\textwidth]{figs/fig-money-per-Llama-2-70B.pdf}}
    \subfloat[Overall Throu. Llama-70B]{\includegraphics[width=0.24\textwidth]{figs/fig-money-all-Llama-2-70B.pdf}}
    \\
    \subfloat[Per Throu. GLM-67B]{\includegraphics[width=0.24\textwidth]{figs/fig-money-per-GLM-67B.pdf}}
    \subfloat[Overall Throu. GLM-67B]{\includegraphics[width=0.24\textwidth]{figs/fig-money-all-GLM-67B.pdf}}
    \\
    \subfloat[Per Throu. GLM-130B]{\includegraphics[width=0.24\textwidth]{figs/fig-money-per-GLM-130B.pdf}}
    \subfloat[Overall Throu. GLM-130B]{\includegraphics[width=0.24\textwidth]{figs/fig-money-all-GLM-130B.pdf}}
  \caption{
  We list the optimal line of \sysname.
  }
  \label{fig:money}
\end{figure}

\section{Real-World Testbed Evaluation}
\label{sec:testbed}

\subsection{Testbed Implementation}
\label{sec:testbed-implementation}

\noindent
\textbf{Edge device:} We use the 16-bit MSP430FR5994 microcontroller~\cite{texas2021msp430}(\textcircled{\small{a}} in Figure~\ref{fig:real-deployment}) with 8KB of SRAM and 256KB of FRAM, operating at a 16MHz clock frequency, as the edge device. We connect MSP430 to the RFM95W LoRa Radio Transceiver~\cite{hoperf2016RFM95} and a 2dBi spring antenna (\textcircled{\small{b}} in Figure~\ref{fig:real-deployment}). The solar board (\textcircled{\small{c}} in Figure~\ref{fig:real-deployment}) connects to the BQ25570 energy harvester~\cite{ti2019bq25570} (\textcircled{\small{d}} in Figure~\ref{fig:real-deployment}) with a capacitor (\textcircled{\small{e}} in Figure~\ref{fig:real-deployment}), serving as the primary ambient power source for the edge devices. A VM1010 microphone~\cite{vesper2017VM1010} with wake-on-sound technology captures environmental sounds exceeding a predefined amplitude threshold (\textcircled{\small{f}} in Figure~\ref{fig:real-deployment}).

\noindent
\textbf{Server:}
Raspberry Pi 4 Model B~\cite{RaspberryPi4B2019} with 8GB of memory is used as a server to assist the MSP430 edge device (\textcircled{\small{g}} in Figure~\ref{fig:real-deployment}). An Adafruit Feather M0 with RFM95 microcontroller~\cite{adafruit2019FeatherM0} (\textcircled{\small{h}} in Figure~\ref{fig:real-deployment}) with a 5.8dBi fiberglass antenna (\textcircled{\small{i}} in Figure~\ref{fig:real-deployment}) is connected to the Raspberry Pi through the serial port. Both edge and server operate on a 125MHz channel within the spreading factors between 7 to 12 and Tx power from 5 to 17 dBm.

\noindent
\textbf{Software:} The PyWavelets library~\cite{Lee2019PyWavelet} is employed for preprocessing the audio clips, while PyTorch~\cite{paszke2019pytorch} is used for pretraining the server and edge models. Raspberry Pi server also uses PyTorch~\cite{paszke2019pytorch} for cloud assistance. Additionally, we have developed a custom wavelet transform library and an inference engine~\cite{zhang2022demo} integrated with the Low-Energy Accelerator (LEA) on the MSP430~\cite{MSPLEA2016}, aimed at accelerating both preprocessing and inference processes. For the server side, we use the Arduino RadioHead library~\cite{mccauley2013radiohead}. For MSP430, We adapt the RadioHead library~\cite{mccauley2013radiohead, eccob2020msp430} over the SPI interface to facilitate communication with the RFM95W transceiver for both uplink and downlink. To the best of our knowledge, this is the first usable and open-source implementation of the LoRa library for the MSP430. 

\noindent
\textbf{Controlled environment:}
To ensure fair comparisons and reproducibility, we collect real-world data and replicate it in a controlled environment using a two-step testbed. First, we implement an MSP430-based data collector with an RFM95W transceiver and TSL2591 luminosity sensor~\cite{ams2013TSL2591} to gather communication traces (SNR, RSSI), LoRa parameter recommendations (SF, $P_{\text{Tx}}$) , and light conditions from target environments. In the second step, we simulate these environments in the lab using programmable Philips Hue bulbs~\cite{lightbulb2024philips} controlled by the phue Python library~\cite{studioimaginaire2020phue} to replicate lighting, with the server assisting edge inference by simulating communication traces and providing parameter recommendations. Environmental sound clips are played through a speaker.

\begin{figure}[tp]
    \centering
    \includegraphics[width=\linewidth]{figures/real-deployment.png}
    \vspace{-0.8cm}
    \caption{Testbed deployment, edge device (left) and server (right). We detail the setup \textcircled{\small{a}}-\textcircled{\small{i}} in Section~\ref{sec:testbed-implementation}.}
    \vspace{-0.2cm}
    \label{fig:real-deployment}
\end{figure}


\subsection{Experimental Setup}

\noindent
\textbf{Datasets:} We randomly select environmental sound clips from the US8k~\cite{salamon2017us8k} dataset to simulate real-world audio events. The audios are padded to 4-second clips and sampled at 16kHz. Additionally, we collect real solar energy traces (around 10k lux luminosity in outdoor daylight conditions), and LoRa communication traces, including SNR, communication costs recommended by ADR, and packet losses for two scenarios:
\begin{list}{$\bullet$}{\leftmargin=1em \itemindent=0em}
    \item \textbf{\textit{Scenario 1:}} We set up the server and edge device at a distance of 500m in a complex urban environment with potential obstructions including buildings and moving vehicles. The SNR, packet losses, and ADR traces are presented in top of Figure~\ref{fig:testbed-communication}. Since this scenario requires high energy costs for long-distance wireless communication, we select 100mF capacitor as energy storage.
    \item \textbf{\textit{Scenario 2:}} We set up our system at a distance of 300m in line of sight. The SNR, packet losses, and ADR traces are presented in top of Figure~\ref{fig:testbed-communication}. We select 33mF capacitor as energy storage for low cost short-distance communication. All communication traces are sampled at 1-minute intervals. 
\end{list}


\begin{figure}[tp]
    \centering
    \includegraphics[width=\linewidth]{figures/communication.png}
    \vspace{-0.8cm}
    \caption{Traces of SNR, packet loss, and the ADR for Scenario 1 (up) and Scenario 2 (down). }
    \label{fig:testbed-communication}
    \vspace{-0.3cm}
\end{figure}

\noindent
\textbf{Baselines:} We summarize the baseline methods as followed:

\begin{list}{$\bullet$}{\leftmargin=1em \itemindent=0em}
\item \textbf{\textit{On-device Inference:}} This implementation adopts the classic on-device inference without cloud offloading~\cite{gobieski2019intelligence,lee2019intermittent} for different resolution spectrograms. 

\item \textbf{\textit{Audio Compression:}} MP3 and AAC~\cite{tomar2006converting} are the two state-of-the-art audio compression algorithms for compressing raw audio waveforms and cloud offloading. This includes different compression bitrates. 

\item \textbf{\textit{Autoencoder Offload:}} DeepCOD~\cite{yao2020deep} and FLEET~\cite{huang2023rethink} compress latent features using an autoencoder before transmitting them to the server. This implementation includes different levels of compression rates. 

\item \textbf{\textit{Progressive Offload:}} Both SEDAC~\cite{ahn2024split} and LimitNet~\cite{hojjat2024limitnet} use context-aware feature selection for offloading. They are combined into one baseline, termed progressive attention-based selection, since LimitNet’s saliency method~\cite{hojjat2024limitnet} is unsuitable for spectrograms. This implementation incrementally includes top-$K$ most informative regions.

\end{list}

\noindent
\textbf{Metrics:} We measure the uplink payload size for cloud assistance, energy consumption, end-to-end inference latency, classification accuracy, and system overhead of ORCA. For energy, latency, and accuracy, we compute the average values across all events in two scenarios respectively.



\subsection{Payload Size}
\label{sec:payload-size}

One of the major benefits of ORCA's cloud assistance strategy we discussed previously is that we can now share much smaller low-resolution spectrogram for feature selection on the cloud, rather than directly use them for high-accuracy cloud inference. Therefore, data compression can be more aggressive than a cloud-dependent strategy without severe accuracy degradation. In Figure~\ref{fig:system-payload}, we compare ORCA to the baselines in terms of accuracy and uplink payload size. First, we notice that the time-series audio compression algorithms MP3 and AAC have poor accuracy even with thousand bytes of payload which is $10\times$-$100\times$ larger than ours. The primary reason is they are generic algorithms for audio compression and are not co-optimized with downstream classification tasks. Next, we notice ORCA outperforms both autoencoder and progressive offloading methods with up to 10 p.p. accuracy advantage and $4\times$-$8\times$ payload savings within the range of 0.1-1 KB, thanks to ORCA's cloud assistance strategy which requires minimal communication for low-resolution spectrograms. Additionally, our method shows a clear 5-20 p.p. accuracy advantage under 100 bytes, the primary operation bitrates under LoRa low data rate mode for long-range transmission~\cite{Semtech2016LoRaWAN}. 

\begin{figure}[tp]
    \centering
    \includegraphics[width=0.9\linewidth]{figures/payload-sota.png}
    \vspace{-0.3cm}
    \caption{Comparisons of accuracy and payload sizes for baseline methods.}
    \label{fig:system-payload}
    \vspace{-0.3cm}
\end{figure}



\begin{table}[tp]
    \caption{\shepherd{On-device energy measurement and end-to-end inference latency for each stage. Uplink is a variable for optimization and mentioned in the text.}}
    \centering
    \begin{tabular}{lcccc}
    \hline 
    Metrics & Preprocess & Server & Downlink & Inference  \\
    \hline
    Energy (mJ) & 28.2 & - & 76.5  & 142.0  \\
    Latency (Sec) & 4.0 & 0.02 & 2.0 & 20.1  \\
    \hline
    \end{tabular}
    \label{tab:energy-latency}
    \vspace{-0.3cm}
\end{table}

\begin{figure*}[tp]
    \centering
    \includegraphics[width=\linewidth]{figures/system-energy-accuracy.png}
    \vspace{-0.7cm}
    \caption{The total energy usage for one round of cloud assistance with various uplink payload payloads and LoRa parameters, (a) for Scenario 1 and (b) for Scenario 2. (c) is for accuracy measurement given payloads. }
    \label{fig:system-energy-accuracy}
    \vspace{0cm}
\end{figure*}


\begin{figure*}[tp]
    \centering
    \includegraphics[width=\linewidth]{figures/scenario1-scenario2-sota.png}
    \vspace{-0.6cm}
    \caption{The average accuracy, energy, and end-to-end latency for baselines in Scenario 1 (left) and 2 (right).}
    \vspace{-0.2cm}
    \label{fig:scenarios-sota}
\end{figure*}




\subsection{Energy Consumption}

\shepherd{As discussed in Section~\ref{sec:resource-aware-cloud-assistance}, the on-device energy usage in preprocessing, downlink, and inference stages are considered constants in our design. We present measurements of these values in Table~\ref{tab:energy-latency}.} Specifically, preprocessing is done in real-time along with sampling by using an LEA hardware accelerator without time and energy overhead. Additionally, the uplink energy consumption is variable with respect to different payload sizes and LoRa parameters in the optimization problem. Therefore, we measure the total energy consumption of one round cloud assistance as in Section~\ref{sec:resource-aware-cloud-assistance} with respect to the uplink payload size ($S$) under different LoRa parameters in Figure~\ref{fig:system-energy-accuracy}(a) and (b) for Scenario 1 and 2. The results show that with smaller spreading factor (SF) and transmitting power ($P_{\text{Tx}}$) values, ORCA consumes fewer resources as the energy consumption per byte is lower. Given fixed energy budgets $E_{\text{cap1}}=685$mJ and $E_{\text{cap2}}=225$mJ for Scenario 1 and 2 respectively as the red lines in Figure~\ref{fig:system-energy-accuracy}(a) and (b), smaller SF and $P_{\text{Tx}}$ values allow larger payload size $S=R_a^2$ and therefore higher assistance resolution $R_a$. Then, the higher assistance resolution $R_a$ leads to higher feature importance estimation on server side and results in higher cloud assistance accuracy as Figure~\ref{fig:system-energy-accuracy}(c). 
 
Furthermore, we compare the energy consumption of ORCA to the state-of-the-art methods in Figure~\ref{fig:scenarios-sota}. Our results show that ORCA outperforms both autoencoder and progressive offloading by 25\% and 90\% energy savings in Scenario 1 and 40\% and 60\% energy savings in Scenario 2 even with up to 15 p.p. accuracy advantage. The primary reason for these savings is the huge payload savings in ORCA design as we explained in Section~\ref{sec:payload-size}. Additionally, ORCA avoids retransmission which costs additional energy consumptions under resource constraints. ORCA shows a clear advantages of 40$\times$ to 80$\times$ and 8$\times$ to 25$\times$ energy savings in two scenarios compared to MP3 and AAC audio compression algorithms. ORCA saves 64\% and 90\% energy compared to the vanilla on-device inference with merely 1 p.p. accuracy degradation by leveraging the resource-efficient cloud-assisted feature selection. 



\subsection{End-to-End Latency}
\shepherd{First, we present measurements of ORCA’s constant latency components, including on-device preprocessing, server processing, downlink, and on-device inference in Table~\ref{tab:energy-latency}. We then incorporate variable uplink latency under two scenarios to evaluate overall end-to-end latency, comparing ORCA with state-of-the-art methods in Figure~\ref{fig:scenarios-sota}.} We notice that given the huge energy usage of LoRa communication, our energy harvesting system takes 50-60 seconds and 10-30 seconds to replenish the energy storage for two scenarios respectively. Compared to autoencoder and progressive baselines, ORCA outperforms both baselines by 40\% and 95\% latency improvements in Scenario 1 and 75\% and 85\% latency improvements in Scenario 2 under accuracy advantages respectively. The primary reason for these improvements is that ORCA avoids the latency of recharging across power cycles by using the resource-aware energy cloud assistance strategy we discussed in Section~\ref{sec:resource-aware-cloud-assistance} to fit all data needed in one round of cloud assistance within one power cycle. Additionally, ORCA avoids retransmission which increases end-to-end with long power cycles. ORCA also achieves 125$\times$ to 220$\times$ and 35$\times$ to 80$\times$ latency improvements in the two scenarios, respectively, compared to MP3 and AAC audio compression algorithms. ORCA saves 64\% and 90\% energy compared to the vanilla on-device inference with merely 1 p.p. accuracy degradation. In both scenarios, ORCA saves 90\% execution time compared to the vanilla on-device inference by the resource-efficient cloud-assisted feature selection. 


\subsection{System Overhead}

As discussed in Section~\ref{sec:spectral-encoding-cnn}, we address the overhead of encoding individual frequency bands with a parameter-efficient spectral encoding CNN, using a spectral encoding matrix concatenated with high-resolution inputs. We compared its memory usage to SubSpectralNet~\cite{phaye2019subspectralnet} and a single-resolution on-device model. ORCA’s spectral encoding CNN requires 106KB of non-volatile memory, only an 8\% increase over the single-resolution model’s 98KB, fitting well within the 256KB FRAM of the MSP430. In contrast, SubSpectralNet~\cite{phaye2019subspectralnet} needs 746KB (7$\times$ ORCA’s usage), making it unsuitable for memory-limited microcontrollers.








\section{Conclusion}\label{sec:conclusion}

In this paper, we proposed a prototype ASL generation system aimed at improving the naturalness, comprehensiveness, and overall quality of generated signs, addressing key limitations in existing approaches. Our technical evaluations indicate that our proposed approaches improve these aspects, enhancing the quality of generated ASL content. Feedback from DHH participants was mixed; while there was general interest in the system, concerns regarding visual quality and naturalness were noted. Reflecting on our design process and study findings, we discuss key insights and identify key areas for future improvement. While further work is needed, our study takes an initial step toward developing sign language generation systems that better meet the needs of the DHH and signing communities, offering real-world value.

\begin{acks}
This work was supported in part by the National Science Foundation under Grants \#2112665 (TILOS AI Research Institute), \#2003279, \#1911095, \#1826967, \#2100237, \#2112167, and in part by PRISM and CoCoSys, centers in JUMP 2.0, an SRC program sponsored by DARPA. We thank the reviewers and the shepherd for their insightful comments and suggestions. 
\end{acks}

\balance
\bibliographystyle{acm}
% \bibliography{reference.bib}
\documentclass{MITstyle}

%\usepackage[table]{xcolor}
\usepackage{chngcntr}
\usepackage{hyperref}
\usepackage{microtype}

\title{A Lightweight and Extensible Cell Segmentation and Classification Model for Whole Slide Images}

\author{Nikita Shvetsov~$^{1, }$\footnote{Correspondence e-mail: nikita.shvetsov@uit.no}, Thomas K. Kilvaer~$^{2, 3}$, Masoud Tafavvoghi~$^{4}$, Anders Sildnes~$^{1}$, \\ Kajsa Møllersen~$^{4}$, Lill-Tove Rasmussen Busund~$^{5, 6}$, Lars Ailo Bongo~$^{1}$ \\
%
\vspace{1em} % Space between authors and afilliations
%
\normalfont{\small $^{1}$Department of Computer Science, UiT The Arctic University of Norway}\\
\normalfont{\small $^{2}$Department of Oncology, University Hospital of North Norway}\\
\normalfont{\small $^{3}$Department of Clinical Medicine, UiT The Arctic University of Norway}\\
\normalfont{\small $^{4}$Department of Community Medicine, UiT The Arctic University of Norway}\\
\normalfont{\small $^{5}$Department of Medical Biology, UiT The Arctic University of Norway} \\
\normalfont{\small $^{6}$Department of Clinical Pathology, University Hospital of North Norway} %\vspace{2em}
}

\begin{document}
\maketitle

\section*{Abstract}

% \begin{abstract}
% Developing clinically useful cell-level analysis tools in digital pathology remains challenging due to limitations in dataset granularity, inconsistent annotations, computational demands of advanced models, and difficulties in integrating new technologies into clinical workflows. To address these challenges, we propose a multi-faceted solution that enhances data quality, model performance, and usability to create a lightweight and extensible cell segmentation and classification model.

% First, we update data labels by employing a cross-relabeling process that refines the labels of two existing datasets, PanNuke and MoNuSAC, to create a new unified dataset with enhanced granularity, encompassing seven distinct cell types. Second, we leverage the H-Optimus foundation model as a fixed encoder to improve feature representation for simultaneous cell segmentation and classification tasks. Third, to address the computational demands of foundation models, we employ knowledge distillation to reduce model size and complexity while maintaining comparable performance. Finally, to facilitate integration into clinical workflows, we integrate the distilled model into the QuPath software, a widely used open-source platform in digital pathology.

% Our results demonstrate improvements in cell segmentation and classification performance using the H‑Optimus-based model compared to a CNN-based model. Specifically, the average $R^2$ improved from 0.575 to 0.871, and the average $PQ$ score improved from 0.450 to 0.492, indicating better alignment with actual cell counts and enhanced segmentation and classification quality. Furthermore, the distilled student model maintains performance comparable to the larger foundation model while reducing the parameter count by a factor of 48.
% Overall, by reducing computational complexity and integrating it into existing workflows, the proposed approach may significantly impact diagnostic processes, reduce the workload of pathologists, and contribute to improved patient outcomes. Though our approach shows potential enhancements in efficiency and usability of cell segmentation and classification models in digital pathology, extensive validation is needed to deploy these models in clinical practice.
% \end{abstract}

%%% shortened abstract
\begin{abstract}
Developing clinically useful cell-level analysis tools in digital pathology remains challenging due to limitations in dataset granularity, inconsistent annotations, high computational demands, and difficulties integrating new technologies into workflows. To address these issues, we propose a solution that enhances data quality, model performance, and usability by creating a lightweight, extensible cell segmentation and classification model. 

First, we update data labels through cross-relabeling to refine annotations of PanNuke and MoNuSAC, producing a unified dataset with seven distinct cell types. Second, we leverage the H-Optimus foundation model as a fixed encoder to improve feature representation for simultaneous segmentation and classification tasks. Third, to address foundation models' computational demands, we distill knowledge to reduce model size and complexity while maintaining comparable performance. Finally, we integrate the distilled model into QuPath, a widely used open-source digital pathology platform. 

Results demonstrate improved segmentation and classification performance using the H-Optimus-based model compared to a CNN-based model. Specifically, average $R^2$ improved from 0.575 to 0.871, and average $PQ$ score improved from 0.450 to 0.492, indicating better alignment with actual cell counts and enhanced segmentation quality. The distilled model maintains comparable performance while reducing parameter count by a factor of 48. By reducing computational complexity and integrating into workflows, this approach may significantly impact diagnostics, reduce pathologist workload, and improve outcomes. Although the method shows promise, extensive validation is necessary prior to clinical deployment.
\end{abstract}
\clearpage

\section{Introduction}
In digital pathology, accurate segmentation and classification of cells are crucial for many diagnostic, prognostic, and predictive analyses \cite{Jaber_Beziaeva_etal._2019,Lin_Pan_etal._2022,Park_Ock_etal._2022,Shen_Choi_etal._2024}. Nowadays, developments in computational pathology offer multiple solutions \cite{H._Qu_P._Wu_etal._2020,Javed_Mahmood_etal._2020} to utilize cell-level datasets to train machine learning models that solve these problems. The quality and specificity of training datasets are critical for robust and accurate models. Adhering to the principle of "garbage in, garbage out", it is essential to ensure that these datasets are extensively and accurately labeled with distinct classes that reflect the diverse biological characteristics of different cell types. Unfortunately, the number of open-source datasets comprising such high-quality annotations is limited. Existing cell segmentation datasets \cite{Gamper_Koohbanani_etal._2019,Graham_Vu_etal._2019,Verma_Kumar_etal._2021} may offer extensive annotations for certain cell types while providing more general labels for others. For example, in PanNuke, which is one of the largest open-source datasets comprising labeled cells, various types of morphologically and functionally different inflammatory cells like macrophages and lymphocytes are clustered in a broad "inflammatory" class. Consequently, these classes are frequently omitted from analyses or aggregated into broader meta-classes \cite{Gamper_Koohbanani_etal._2020} and likely interfere with other cell classes included in the dataset. This and similar inconsistencies in annotation granularity limit the ability of machine learning models to learn the comprehensive and nuanced features necessary for accurate cell segmentation and classification. To address these challenges, methods for refining and standardizing dataset annotations are essential to enhance the quality of training data.

A complementary approach to mitigate the absence of high-quality training data is the use of foundation models. Foundation models as encoders are defined as large-scale, versatile networks pre-trained on vast, diverse datasets using self-supervised learning, contrasting with convolutional neural network (CNN) pre-trained encoders that rely on supervised learning with labeled data. In practice, foundation models leverage enormous amounts of weakly or unlabeled data from millions of whole slide images (WSIs) and employ self-attention mechanisms to capture long-range dependencies and global context \cite{Chen_Ding_etal._2024,Saillard_Jenatton_etal._2024,Vorontsov_Bozkurt_etal._2024,Xu_Usuyama_etal._2024}. As a consequence, foundation models are able to produce transferable feature representations across different cell types and tissue environments. The feature representations can be leveraged by decoder networks to produce segmentation masks and pixel-level classifications. Because foundation models have comprehensive feature representations, they can be effectively fine-tuned using much smaller amounts of cell-level data compared to the large datasets needed to train models from scratch. Furthermore, foundation models incorporate adversarial training elements or contrastive learning \cite{Chen_Ding_etal._2024,Xu_Usuyama_etal._2024}, enhancing their resilience and adaptability by exposing them to challenging and varied scenarios during training. This may result in more generalizable models, often making them well-suited for diverse and complex tasks in digital pathology.

Despite the inherent advantages of foundation models, their deployment for practical use faces its own obstacles. In particular, they require substantial computational power, financial investments and rigorous testing to ensure reliability and efficacy for a given task \cite{Akkus_Dangott_etal._2022,Dragomir_Cocuz_etal._2022,Go_2022,Jafri_Farooqui_etal._2024}. Moreover, while foundation models enhance feature representation and performance, they depend on the quality of available annotations for decoder fine-tuning and, like any other model, cannot resolve existing inconsistencies or ambiguities in data labels. Therefore, there remains a critical need for solutions that address both data quality and practical deployment considerations.
Further, integrating new technologies into existing clinical workflows often encounters resistance, as it necessitates adjustments to established diagnostic processes. So, there is a need to develop solutions that could be integrated into current practices, minimizing the burden on medical professionals to adopt new tools \cite{King_Williams_etal._2023}.

Existing solutions \cite{Goldsborough_Philps_etal._2024,Hörst_Rempe_etal._2024}, while addressing some aspects of these challenges, fall short in providing a comprehensive approach. To address the data quality and clinical deployment issues, we propose a multi-faceted solution that encompasses data refinement, model optimization, and integration with existing pathology tools (\hyperref[fig:fig1]{Figure 1}). The outcome is a lightweight cell segmentation and classification model that can be integrated into digital pathology workflows for practical clinical use.

\begin{figure}[h!]
    \centering
    \includegraphics[width=\textwidth, height=0.82\textheight, keepaspectratio]{images/Figure_1.pdf}
    \caption{Overview of the proposed solution, including 1) Data refinement using cross-relabeling, 2) Teacher model development and fine tuning, 3) Student model optimization with knowledge distillation and 4) Student model and QuPath integration}
    \label{fig:fig1}
\end{figure}
\clearpage

Our approach begins with preparing the data for the fine-tuning and training of the machine learning models. We create a refined dataset, acquired via cross-relabeling two cell-level datasets, enhancing annotation specificity and consistency of the labeled data. Subsequently, we create a cell segmentation and classification model based on the foundation model. We leverage the foundation model as a fixed encoder and fine-tune a decoder using the refined dataset to improve generalization across diverse tissue- and cell types.
To ensure that the model remains lightweight and deployable in a possibly resource-constrained environment, we employ knowledge distillation to approximate the functionality of the foundation model. Finally, to facilitate the practical application of our model in digital pathology workflows, we integrate it with the QuPath \cite{Bankhead_Loughrey_etal._2017} application. Each methodological component contributes to the overarching goal of enhancing model performance, generalizability, and usability in clinical settings.

The primary contributions of this paper are:
\begin{enumerate}
    \item \textit{Data labels refinement through cross-relabeling:}
    
    We propose a new method for refining labels of cell-level datasets through cross-relabeling. This method employs classification models to re-label broad and ambiguous instances, resulting in a more diverse dataset. Our evaluation demonstrates that these classification models achieve high accuracy on test subsets, indicating the reliability of the method for label refinement.

    \item \textit{Enhanced model performance via foundation models:}
    
    We employ a foundation model as a feature extractor for the cell segmentation and classification task. In comparison with training a CNN model from scratch, the foundation model backbone only needs fine-tuning, which significantly reduces training time, computational resources and data requirements. We show that using a foundation model encoder leads to better performance in cell segmentation and classification networks than using a CNN-based encoder. This improvement may enable the model to generalize more effectively across various tissue types and imaging methods.
    
    \item \textit{Model optimization through knowledge distillation:}
    
    We show that a smaller student model trained using knowledge distillation on the refined dataset obtained via our cross-relabeling approach from a foundation model achieves comparable performance in cell segmentation and quantification tasks. As a result, this model is more suitable for deployment in environments without high-performance computing resources.
    
    \item \textit{Integration with QuPath:}
    
    We integrate the distilled cell segmentation and classification model into QuPath, a widely used open-source digital pathology platform, to accelerate clinical adaptation by enabling pathologists to more easily incorporate advanced computational tools into their existing workflows.
\end{enumerate}

Through these methodological steps, we aim to bridge the gap between advanced machine learning techniques and practical clinical applications, making accurate and efficient digital pathology accessible in a broader range of healthcare settings.

\section{Refining Existing Datasets Using Cross-Relabeling}
To address the limitations of sparse and ambiguous labeling of cell-level datasets, we propose a generalizable cross-relabeling strategy that can be applied to any dataset containing broadly categorized or imprecisely labeled cell types. This approach involves training and subsequently leveraging classification models to refine broad categories into more specific or biologically relevant classes.
When applied to cell-level data, the methodology includes extracting individual cell images from the dataset patches, preprocessing these images to standardize the size and accommodate partial cells, and then training deep learning classifiers capable of distinguishing between the finer cell subtypes within the coarser categories. 
To illustrate our approach, we focus on the PanNuke \cite{Gamper_Koohbanani_etal._2020, Gamper_Koohbanani_etal._2019} and MoNuSAC \cite{Verma_Kumar_etal._2021} datasets that we have used to train models for cell quantification in our previous works \cite{Shvetsov_Grønnesby_etal._2022,Shvetsov_Sildnes_etal._2024}. We find that for better cell differentiation we have to introduce more granular labels. PanNuke includes a broad classification of "inflammatory" cells, encompassing lymphocytes, macrophages, and neutrophils. Each cell type differs significantly in structure, function, and clinical relevance. Conversely, MoNuSAC uses the label "epithelial" for a class that comprises both benign epithelial cells and malignant neoplastic cells. This practice makes it challenging to differentiate between benign and malignant epithelial cells in the dataset, which is a critical distinction when identifying tumor areas within tissue samples. To address these issues, we implement a cross-relabeling strategy as shown in \hyperref[fig:fig2]{Figure 2}. The key components are two classification models: one is trained on singular cell images from PanNuke data to classify the epithelial meta-class into epithelial and neoplastic classes. The other is trained on MoNuSAC to refine the inflammatory class into lymphocytes, neutrophils, and macrophages.

\begin{figure}[h!]
    \centering
    \includegraphics[width=\textwidth]{images/Figure_2.pdf}
    \caption{Refined dataset generation via cross relabeling}
    \label{fig:fig2}
\end{figure}

The refining approach consists of three consecutive steps. The first is the preprocessing step, in which we extract individual cells from both datasets (\hyperref[fig:fig3]{Figure 3}). The specifics of PanNuke and MoNuSAC patch preparation before cell preprocessing are provided in \hyperref[chap:S1]{Appendix S1}.

\begin{figure}[h!]
    \centering
    \includegraphics[width=\textwidth]{images/Figure_3.pdf}
    \caption{Cell instances preprocessing including (1) cell map extraction, (2) bounding box delineation, (3) adjusting cell boxes and (4) cropping and resizing of cell images}
    \label{fig:fig3}
\end{figure}

During preprocessing, we extract cell type maps from the ground truth label mask and calculate bounding boxes around each cell instance. To accommodate partial cells at patch borders, a common issue in cropped patch images, we employ mirror padding and extend the field of view of the cell label by 15 pixels to capture adjacent cells. We then crop and resize the identified regions to $64 \times 64$ pixels using bicubic interpolation.

The preprocessed PanNuke dataset comprises 68,031 neoplastic and 23,207 epithelial cell images, while MoNuSAC comprises  33,104 lymphocytes, 1,252 neutrophils, and 1,695 macrophages, which we subsequently use in training cell classification models and classifying the cell image data \hyperref[fig:S2]{Appendix Figure S2 (1)}. 

The next step is to train two distinct ResNet50-based classifiers tailored to address the specific labeling challenges inherent in each dataset. We use ResNet50 for classification models due to its proven effectiveness for image classification tasks in histopathology \cite{pan2022reviewmachinelearningapproaches}, and its compatibility with small images. For the PanNuke dataset, we design the classifier, trained on MoNuSAC data, to disaggregate the heterogeneous "inflammatory" cell category into distinct subtypes: lymphocytes, macrophages, and neutrophils. Similarly, for the MoNuSAC dataset, the classifier is trained on PanNuke data and distinguishes between benign and malignant epithelial cells within the overarching "epithelial" label. By applying these targeted classifiers to their respective datasets, we assign more specific labels to individual cell instances, thus enabling us to create a unified dataset.
To ensure a balanced representation of classes, we train both models on datasets that had been equalized to match the size of the least represented class. Thus, we obtain datasets comprising 23,207 samples per class for PanNuke and 1,252 samples per class for MoNuSAC data. Next, we partition both of them into training (70\%), validation (20\%), and testing (10\%) subsets. To mitigate the risk of overfitting, we use a single dropout layer with a rate of p=0.5 in both models and data augmentation using randomized color perturbations, rotation, and horizontal and vertical flipping. We employ AdamW optimizer and the cross-entropy loss function for the training criterion.

To evaluate the two trained models, we measure the classification accuracy on the respective test subsets. The accuracies on the test subset for both classifiers are presented in \hyperref[tab:1]{Table 1}. The PanNuke model achieves an average accuracy of 93.57\%, with higher accuracy for neoplastic cells (96.06\%) compared to epithelial cells (86.26\%). The confusion matrix in Figure A3.1 shows that the model predominantly distinguishes accurately between epithelial and neoplastic tissues, with a substantial number of correct classifications and relatively few misclassifications. The MoNuSAC model demonstrates an average accuracy of 98.92\%, excelling in classifying lymphocytes (99.67\%) and macrophages (94.12\%), with lower performance for neutrophils (85.71\%). The confusion matrix in Figure A3.2 shows that the model identifies lymphocytes and performs reasonably well with macrophages and neutrophils.

\begin{table}[h!]
\renewcommand{\arraystretch}{1.5}
  \centering
  \caption{Cell classification results for PanNuke and MoNuSAC trained models (CI 95\%).}
  \label{tab:1}
  \begin{tabular}{|l|c|c|}
   \hline
   %\rowcolor{gray!30}
    Accuracy               & PanNuke model              & MoNuSAC model              \\
    \hline
    Average      & 0.936 (0.931--0.941)         & 0.989 (0.986--0.993)        \\
    \hline
    Neoplastic   & 0.961 (0.956--0.965)         & -                          \\
    \hline
    Epithelial   & 0.863 (0.849--0.877)         & -                          \\
    \hline
    Lymphocytes  & -                          & 0.997 (0.995--0.999)        \\
    \hline
    Neutrophils  & -                          & 0.857 (0.796--0.918)        \\
    \hline
    Macrophages  & -                          & 0.941 (0.906--0.976)        \\
    \hline
  \end{tabular}
\end{table}

Finally, during the last step, we use the model trained on PanNuke data for epithelial cells in MoNuSAC and the model trained on MoNuSAC for the inflammatory cells class in PanNuke. Specifically, we use classifier models to relabel epithelial cells in MoNuSAC and inflammatory cells in PanNuke data. Then we combine cells with refined labels and the rest of the cells in both datasets to create a refined dataset (\hyperref[fig:S2]{Appendix Figure S2 (2)}). The process of relabeling cells and visualizing them on a patch is shown in \hyperref[fig:fig4]{Figure 4}. The cell counts in the refined dataset are provided in \hyperref[tab:S4]{Appendix Table S4}.

\begin{figure}[h!]
    \centering
    \includegraphics[width=\textwidth, height=0.42\textheight, keepaspectratio]{images/Figure_4.pdf}
    \caption{Cell relabeling procedure for epithelial and inflammatory cell classes}
    \label{fig:fig4}
\end{figure}

%\hfill

Relabeling and combining datasets have been explored in a prior study \cite{Parulekar_Kanwat_etal._2023}, where consecutive fine-tuning on multiple datasets was employed to account for hierarchical class label structures. While the method presented in \cite{Parulekar_Kanwat_etal._2023} is intuitive, it often lacks consistency and requires multiple fine-tuning runs, which can be cumbersome and time-consuming. 
In contrast, cross-relabeling simplifies this process by using specialized classification models tailored to each dataset's specific labeling challenges. This approach provides better transparency and produces a unified dataset encompassing seven distinct cell types across multiple tissue samples, enhancing data diversity for further model training or fine-tuning.

Despite these improvements, cross-relabeling does not entirely resolve issues related to poor labeling quality or the amount of labeled data. Specifically, our results show lower accuracies persist for underrepresented classes, such as macrophages, which may stem from a limited sample availability and intrinsic challenges in distinguishing these cells based solely on H\&E staining. Furthermore, while our method enhances label specificity, it relies on the initial quality of the broad labels; thus, any fundamental inaccuracies in the original annotations can propagate through the relabeling process. Addressing the overall problem of limited data labels may require integrating additional data sources or utilizing complementary immunohistochemical staining methods.
Although the reported performance metrics are obtained from evaluations on the native test sets of each dataset, it is important to note that the primary application of these classifiers is to perform cross-relabeling, where a model trained on one dataset (e.g., PanNuke) is applied to another (e.g., MoNuSAC) and vice versa. We acknowledge that a more systematic evaluation of cross-dataset generalization is needed and could be performed in future work.

Overall, the refined dataset produced by our approach can enhance the supervised training or fine-tuning of cell segmentation and classification models, especially those that utilize pre-trained foundation models to improve feature extraction robustness. In addition, these models can detect nuanced classes that enable researchers to conduct more detailed analyses of biological processes in computational pathology.

\section{Foundation models for robust cell segmentation and classification}

Accurate cell segmentation and classification in digital pathology are hindered by limited labeled data and the fact that conventional CNNs are unable to capture global contextual information due to their local receptive field constraints \cite{Gheflati_Rivaz_2022,Yang_Marcus_etal.}. Traditional approaches in cell quantification have predominantly relied on CNN encoders, such as ResNet50, given their proven effectiveness in semantic segmentation tasks \cite{Deshmane_2023,Graham_Vu_etal._2019,Mukasheva_Koishiyeva_etal._2024,Stringer_Wang_etal._2021}. However, approaches that include fine-tuning of pretrained CNNs, data augmentation, and stain normalization to partially increase data variability and address staining differences often fail to achieve the necessary generalization and robustness across diverse tissue types and staining conditions \cite{G._Wang_W._Li_etal._2018,Gao_Bagci_etal._2018,Karim_El_Khoury_Martin_Fockedey_etal._2021}.

To overcome these challenges, we leverage an encoder-decoder network that uses a foundation model as the encoder and a CNN upsampling decoder (\hyperref[fig:fig5]{Figure 5}) for simultaneous cell segmentation and classification in 2D patches extracted from WSIs. Foundation models with transformer-based architectures are viable alternatives to CNN-based encoders \cite{Shamshad_Khan_etal._2023,Sourget_2023}. They enable the creation of more advanced architectures that can decode or transform learned features more effectively \cite{Chen_Duan_etal._2023,Cheng_Misra_etal._2022,Xie_Wang_etal._2021}.

\begin{figure}[h!]
    \centering
    \includegraphics[width=\textwidth]{images/Figure_5.pdf}
    \caption{UNETR-like model with foundational model as backbone}
    \label{fig:fig5}
\end{figure}

By utilizing a transformer-based encoder, we incorporate global contextual information into the feature extraction process, which is a key advantage of such architectures \cite{Chen_Lu_etal._2021}. This foundation model integration facilitates accurate pixel-wise segmentation and classification without the need for extensive encoder training, thereby potentially improving generalization across varied cellular structures and tissue types.
In our implementation, we employ a modified UNETR \cite{Hatamizadeh_Tang_etal._2021} architecture that combines a vision transformer (ViT) \cite{Dosovitskiy_Beyer_etal._2021} encoder with a CNN-based decoder. The encoder utilizes the pretrained H-Optimus foundation model, which contains 1.1 billion parameters and is trained on over 500,000 H\&E stained WSIs \cite{Saillard_Jenatton_etal._2024}. We extract outputs from four evenly spaced transformer blocks $Z_i$, where $i \in [1, 14, 26, 38]$, to serve as residual connections for the CNN decoder. We select these blocks based on our observation that features from non-adjacent levels of the encoder lead to better overall performance on the test subset.

The CNN decoder upsamples the feature representations, acquired from the transformer blocks, to generate an intermediate vector that is handled by two task-specific layers that generate cell segmentation and classification masks. The first task-specific layer is the ‘Cellpose head’,  which is used to delineate cell instances. The layer generates horizontal and vertical gradient maps to form vector fields that are refined through gradient tracking in a post-processing step using the Cellpose algorithm \cite{Stringer_Wang_etal._2021}, known for its efficacy in cell segmentation tasks and generalizability across multiple domains \cite{Pachitariu_Stringer_2022,Stringer_Pachitariu_2024}. The second task-specific layer is the "Cell type head", which assigns labels to individual pixels. In the post-processing step, we determine the output classification label of each segmented cell instance by majority voting over the labeled pixels that comprise the cell in the segmentation map.

To evaluate model performance and measure the impact of adding a foundation model as backbone, we compare it to a ResNet50-based model. ResNet50 is a widely used solution for encoders in segmentation architectures in the medical domain \cite{Deshmane_2023,Graham_Vu_etal._2019,Mukasheva_Koishiyeva_etal._2024,Stringer_Wang_etal._2021}. For the H-Optimus-based model, we utilize frozen weights for the encoder and only fine-tune the decoder to take advantage of the extensive pre-training of the foundation model. For the ResNet50-based model we start with ImageNet \cite{Deng_Dong_etal.} weights and train both encoder and decoder parts. Hyperparameters for the training step are set to be identical, where possible, for comparable evaluation. 
For this evaluation, we deliberately use the PanNuke dataset to provide a standardized and controlled comparison between the H‑Optimus and ResNet50-based models (\hyperref[fig:S2]{Appendix Figure S2 (3)}). Specifically, we use two of the default PanNuke dataset splits (66\%) for training and validation, and reserve the third split (33\%) for testing.

To address the challenge of cell class imbalance in the PanNuke dataset, which is a common characteristic in most cell-level H\&E patch datasets, both models’ training processes employ a weighted loss function comprising cross-entropy and focal loss \cite{Lin_Goyal_etal._2018}. The focal loss component is adjusted with coefficients derived from each cell class' instance frequency, emphasizing learning from underrepresented classes and enhancing the model's sensitivity to rare but significant cellular patterns. The cross-entropy loss is augmented with spectral decoupling regularization \cite{Pezeshki_Kaba_etal._2021,Pohjonen_Stürenberg_etal._2022} and spatially varying label smoothing \cite{Islam_Glocker_2021}, which potentially stabilizes training and improves generalization in case of complex tissue morphologies. For optimization, we employ the \textit{AdamW} \cite{Loshchilov_Hutter_2019} to counter unbalanced class scenarios, with cosine annealing learning rate scheduler.

We utilize the scikit-learn library \cite{Van_der_Walt_Schönberger_etal._2014} and HoVer-Net \cite{Graham_Vu_etal._2019} implementations of $R^2$ (the coefficient of determination) and $PQ$ (panoptic quality) to evaluate our experiments. Complete mathematical formulations and detailed explanations of these metrics are provided in \hyperref[chap:S5]{Appendix S5}. To compute confidence intervals, we use nonparametric bootstrapping, where after calculating the metric on the full sample, we generated 1000 bootstrap replicates by resampling with replacement and then determined the 95\% confidence intervals as the 2.5th and 97.5th percentiles of the resulting empirical distribution.

%\hfill

The model comparisons are summarized in \hyperref[tab:2]{Table 2}. The H‑Optimus-based model achieves higher $R^2$ across all cell classes compared to the ResNet50-based model, which means that its predictions are more closely aligned with the PanNuke cell counts, indicating a stronger correlation with the observed data. Notably, the improvement of $R^2_{dead}$ may be an indicator of better global contextual representations provided by the foundation model backbone. In terms of segmentation and classification quality combined, measured by the PQ score, the H‑Optimus-based model demonstrates notable improvements across most cell classes. Overall, the average $R^2$ improved from 0.575 to 0.871, while the average $PQ$ score improved from 0.450 to 0.492, demonstrating better performance of the H-Optimus-based model.

\begin{table}[h!]
\renewcommand{\arraystretch}{1.5}
  \centering
  \caption{Cell quantification metrics for baseline and proposed models (CI 95\%).}
  \label{tab:2}
  \begin{tabular}{|l|c|c|}
    \hline
    %\rowcolor{gray!30}
    Metric             & Resnet50-based            & H-optimus-based              \\
    \hline
    $R^2_{neoplastic}$    & 0.681 (0.576--0.769)       & \textbf{0.941 (0.917--0.960)} \\
    \hline
    $R^2_{inflammatory}$  & 0.863 (0.778--0.903)       & \textbf{0.949 (0.918--0.966)} \\
    \hline
    $R^2_{connective}$    & 0.600 (0.488--0.698)       & 0.609 (0.436--0.772)          \\
    \hline
    $R^2_{dead}$          & 0.097 (-11.389--0.669)     & 0.925 (0.404--0.982)          \\
    \hline
    $R^2_{epithelial}$    & 0.635 (0.490--0.747)       & \textbf{0.930 (0.886--0.964)} \\
    \hline
    $PQ_{neoplastic}$       & 0.517 (0.499--0.535)       & \textbf{0.589 (0.575--0.604)} \\
    \hline
    $PQ_{inflammatory}$     & 0.455 (0.429--0.482)       & \textbf{0.528 (0.507--0.549)} \\
    \hline
    $PQ_{connective}$       & 0.416 (0.400--0.431)       & \textbf{0.451 (0.436--0.465)} \\
    \hline
    $PQ_{dead}$             & 0.374 (0.342--0.408)       & 0.292 (0.209--0.365)          \\
    \hline
    $PQ_{epithelial}$       & 0.488 (0.460--0.519)       & \textbf{0.599 (0.579--0.618)} \\
    \hline
  \end{tabular}
\end{table}

Our results  show that integrating the H‑Optimus foundation model within the UNETR architecture enhances the model's ability to segment and classify cells across diverse tissues from PanNuke data. The pretrained transformer encoder provides robust feature representations, resulting in higher average $R^2$ and $PQ$ scores compared to the CNN-based model. This leads to more reliable cell quantification and more accurate downstream analysis. Additionally, the streamlined fine-tuning process reduces computational overhead and training time, making the model more adaptable for new data.

Despite these advancements, the foundation model-based approach does not fully resolve all challenges related to cell segmentation and classification. We observe lower metric scores for underrepresented classes in the training data. Furthermore, foundation models typically encompass billions of parameters, resulting in substantial computational and memory requirements. It therefore poses challenges for deployment in resource-constrained environments, limiting their practical applicability in certain clinical settings.

\section{Model optimization via Knowledge Distillation}

To address the limitations posed by the extensive size of foundation models, we implement knowledge distillation — a model compression technique that leverages the teacher-student paradigm \cite{Hinton_Vinyals_etal._2015}. By training a smaller, more efficient student model to replicate the output of a larger, pre-trained teacher model, we retain performance while significantly reducing the model's complexity and resource requirements (\hyperref[fig:fig6]{Figure 6}).

\begin{figure}[h!]
    \centering
    \includegraphics[width=\textwidth, height=0.45\textheight, keepaspectratio]{images/Figure_6.pdf}
    \caption{Knowledge distillation framework for training a student model using a pre-trained teacher}
    \label{fig:fig6}
\end{figure}

We employ knowledge distillation to compress the H‑Optimus-based teacher model into a more efficient student model. The teacher model is the modified UNETR architecture with the H‑Optimus foundation model described in the previous chapter. The student model is based on a UNet architecture augmented with residual connections and incorporates a smaller ViT encoder with 9 million parameters \cite{Steiner_Kolesnikov_etal._2022,Wightman_2019}. 

First, we fine-tune the teacher model using the refined dataset from the cross-relabeling procedure (Section 2). Initially we train the decoder of the teacher model while keeping the encoder weights frozen. We split the refined dataset into train (70\%), validation (20\%) and test (10\%) subsets (\hyperref[fig:S2]{Appendix Figure S2 (4)}). During fine-tuning, we use the train and validation subsets, while leaving the test subset for model evaluation. We set the training procedure and model hyperparameters to be identical to those that were used to demonstrate the utility of foundation models for the simultaneous cell segmentation and classification task.

Next, we perform knowledge distillation from teacher to student using the refined dataset used to fine-tune the teacher model. The student model is trained to replicate the teacher model's outputs. We utilize a specialized loss function that aligns the student's predicted probability distribution with the teacher's, incorporating the teacher's class probability distribution derived from the output. Following the methodology of Hinton et al. \cite{Hinton_Vinyals_etal._2015}, we experiment with various hyperparameter settings for the temperature ($T$) and the balancing coefficients ($\alpha$ and $\beta$) in the loss function. We vary $T$ from 1 to 20 and adjust $\alpha$ and $\beta$ to balance the distillation and student losses. Through iterative tuning and evaluation, we identify that setting $T=14$, $\alpha=0.3$, and $\beta=0.7$ yields a configuration that converges and closely approximates the teacher model's performance during training.

Finally, we assess the performance of both models using the $R^2$ and $PQ$ (defined in \hyperref[chap:S5]{Appendix S5}) on the test set of the refined dataset (\hyperref[tab:3]{Table 3}). We observe that the 95\% confidence intervals overlap for most cell types, so we cannot claim statistically significant performance differences between the teacher and student models. One exception appears in the neoplastic class. The teacher model produces an $R^2$ of 0.919, while the student model shows an $R^2$ of 0.852. In addition, the student model achieves higher $PQ$ values for the neoplastic and connective classes, though the confidence intervals show overlap.

\begin{table}[h!]
\renewcommand{\arraystretch}{1.5}
  \centering
  \caption{Cell quantification metrics for teacher and distilled student models (CI 95\%).}
  \label{tab:3}
  \begin{tabular}{|l|c|c|}
    \hline
    %\rowcolor{gray!30}
    Metric & Teacher & Student \\
    \hline
    $R^2_{neoplastic}$    & \textbf{0.919} (0.898--0.939) & 0.852 (0.800--0.891) \\
    \hline
    $R^2_{lymphocyte}$    & 0.969 (0.956--0.977)         & 0.969 (0.956--0.978) \\
    \hline
    $R^2_{connective}$    & 0.694 (0.548--0.809)         & 0.618 (0.469--0.741) \\
    \hline
    $R^2_{dead}$          & 0.755 (0.400--0.908)         & 0.424 (0.100--0.731) \\
    \hline
    $R^2_{epithelial}$    & 0.922 (0.870--0.958)         & 0.843 (0.738--0.917) \\
    \hline
    $R^2_{macrophage}$    & 0.384 (-0.369--0.724)        & 0.704 (0.352--0.859) \\
    \hline
    $R^2_{neutrofil}$     & 0.854 (0.578--0.929)         & 0.833 (0.502--0.925) \\
    \hline
    $PQ_{neoplastic}$       & 0.581 (0.569--0.593)         & 0.601 (0.588--0.613) \\
    \hline
    $PQ_{lymphocyte}$       & 0.536 (0.520--0.553)         & 0.563 (0.544--0.579) \\
    \hline
    $PQ_{connective}$       & 0.436 (0.421--0.451)         & 0.457 (0.441--0.474) \\
    \hline
    $PQ_{dead}$             & 0.272 (0.235--0.315)         & 0.279 (0.201--0.369) \\
    \hline
    $PQ_{epithelial}$       & 0.522 (0.500--0.545)         & 0.530 (0.506--0.555) \\
    \hline
    $PQ_{macrophage}$       & 0.524 (0.459--0.588)         & 0.474 (0.405--0.543) \\
    \hline
    $PQ_{neutrofil}$        & 0.541 (0.490--0.592)         & 0.565 (0.522--0.607) \\
    \hline
  \end{tabular}
\end{table}


We further decompose the $PQ$ metric into its $SQ$ and $DQ$ components (\hyperref[tab:S6]{Appendix Table S6}). Both models produce nearly identical $SQ$ values, which indicates that they predict instance boundaries with similar precision. Although the student model shows some improvement in $DQ$ scores for certain classes, the confidence intervals overlap and do not confirm a statistically significant difference.

We observe that the student and teacher models yield comparable detection performance despite the student model using a much smaller and simpler architecture. A model with fewer parameters reduces the risk of overfitting when training data are scarce relative to the model’s complexity \cite{Farias_Ludermir_etal._2022}. The knowledge distillation process also encourages the student model to focus on the most generalizable detection features learned from the teacher. These factors enable the student model to achieve similar detection performance across different cell types.

Additionally, considering the model sizes reported in \hyperref[tab:4]{Table 4}, the distilled model achieves a significant reduction compared to the teacher model, with a 48-fold decrease in parameter count and a 5.5-fold reduction in on-disk size. In inference mode, the teacher model requires 16 GB of VRAM for a batch size of 32, while the distilled model only needs 3 GB of VRAM for the same batch size. These reductions make the distilled model significantly more practical for fine-tuning and deployment in resource-constrained environments.

\begin{table}[h!]
\renewcommand{\arraystretch}{1.5}
  \centering
  \caption{Parameter counts and size of teacher and distilled model}
  \label{tab:4}
  \adjustbox{max width=\textwidth}{%
  \begin{tabular}{|l|c|c|c|}
    \hline
    %\rowcolor{gray!30}
    Metric & H-optimus-based (Teacher) & mobileViT-based (Student) & Magnitude of difference \\
    \hline
    Parameters count       & 1,158,917,906   & \textbf{24,093,393}   & \textbf{48x}  \\
    \hline
    Estimated Total Size (MB) & 87,912       & \textbf{15,935}    & \textbf{5.5x} \\
    \hline
  \end{tabular}%
}
\end{table}

%\hfill

With recent advancements in complex network architectures and the use of pretrained encoders to achieve state-of-the-art performance \cite{Baumann_Dislich_etal._2024,Hörst_Rempe_etal._2024} in cell segmentation and classification tasks, model size, computational complexity, and processing times have increased. This limits the scalability and accessibility of these models. As we demonstrate, this may be mitigated using knowledge distillation. Studies in the field of natural language processing have demonstrated the efficacy of knowledge distillation in retaining the capabilities of the teacher model while achieving significant reductions in size and complexity \cite{Huangpu_Gao_2024,Sun_Yu_etal.}. 

We demonstrate the feasibility of knowledge distillation in digital pathology, specifically for cell segmentation and classification tasks. Moreover, we achieve this performance while also significantly reducing the parameter count. In addressing the challenge of knowledge transfer, we found that distillation from a transformer-based model to a smaller transformer is more straightforward than attempting to map transformer features to CNN blocks. In our experiments, using a CNN-based network as a student results in worse cell quantification performance due to the structural constraints of CNN feature space dimensions. 

Although our primary approach relies on a transformer-based student model that performs well, it can be further optimized to incorporate advantages from CNN architectures. For example, employing alternative techniques such as using ViT adapters \cite{Chen_Duan_etal._2023} or $1 \times 1$ convolutions to adjust feature map sizes may be beneficial for harnessing CNN advantages like enhanced local feature extraction. Moreover, if additional performance improvements are desired, the process can be further enhanced by applying supplementary knowledge distillation techniques, such as self-distillation \cite{Zhang_Song_etal._2019} or online distillation \cite{Houyon_Cioppa_etal._2023}.

Despite these promising results, further validation on independent datasets is necessary to fully understand the model's limitations. Underrepresented classes may pose challenges when addressing complex cases. Pathologists need to validate these models to adopt them in clinical settings. While the distilled models are smaller and more deployable, a technological gap persists because pathologists traditionally rely on established methods for inspecting WSIs and diagnosing diseases. Addressing the complexities involved in deploying models for inference and supporting pathologists in adopting new tools is essential for integrating these models into clinical workflows.

\section{Model integration with QuPath}
Digital pathology tools with graphical user interfaces are essential for visualizing and analyzing WSIs. To make our student model useful in clinical pathology workflows, it needs to be integrated into a tool that enables inspecting regions, creating annotations, and providing quantitative analyses of biomarkers. Therefore, we integrate the trained student model from the previous chapter into the QuPath open‑source platform \cite{Bankhead_Loughrey_etal._2017}. QuPath provides the required annotation, visualization, and analysis tools to interpret complex histological data, including workflows for cell segmentation, classification, and quantification (\hyperref[fig:fig7]{Figure 7}). 

\begin{figure}[h!]
    \centering
    \includegraphics[width=\textwidth]{images/Figure_7.pdf}
    \caption{Visualization of model-generated cell quantification annotations (left) and the corresponding unannotated slide (right) in QuPath}
    \label{fig:fig7}
\end{figure}

To identify the regions in a WSI critical for prognosticating tumor development, such as specific tumor areas or border regions without overlapping healthy tissue, the pathologist uses QuPath to outline these regions. Then, the pathologist initiates a cell segmentation and classification script through the QuPath interface for the selected regions. The resulting annotations and quantified cell information are then directly overlaid onto the WSI in the QuPath interface. Additional design and implementation details are in \hyperref[chap:S7]{Appendix S7}. 

Two common approaches for integrating deep learning models into QuPath are Java‑based native QuPath extensions \cite{Goldsborough_Philps_etal._2024} and the execution of RESTful API requests to a model server coupled with handling the response via an extension, as demonstrated in the application of cell segmentation models applied to immunofluorescence images \cite{Sugawara_2023}. While the community is actively working on these integration strategies, there is currently no universal solution that fully addresses all integration and performance requirements.

Extensions may offer better integration with QuPath, allowing slightly improved performance and more widespread usage of the built-in QuPath models, but they lack the flexibility to customize models and modify their behavior. For example, the newest version of QuPath includes models such as StarDist \cite{Weigert_Schmidt} and InstanSeg \cite{Goldsborough_Philps_etal._2024} that can perform cell segmentation. Both models pose limitations when applied to simultaneous cell segmentation and classification. StarDist performs well only on convex, round shapes by design, whereas some neoplastic, inflammatory, and connective cells exhibit complex and non-convex shapes. InstanSeg provides only semantic segmentation without assigning classes to the segmented cells.

%\hfill

In contrast, our approach offers an alternative integration strategy. It utilizes the paquo library to directly interact with QuPath’s internal application programming interface from within Python. This enables data exchange and processing without the need for intermediate conversion steps and provides greater control over model customization, retraining, and the incorporation of custom processing steps.

The integration of our custom model with QuPath underscores its potential to significantly enhance the diagnostic process by reducing the time burden on pathologists and enabling them to focus on more complex interpretative tasks using familiar software. Leveraging a tool that is already well-established among pathologists increases the likelihood of its adoption into daily clinical workflows. The quantitative data generated through the automated workflow is critical for both clinical decision-making and research, facilitating more accurate biomarker analysis, enabling robust statistical evaluations, and supporting hypothesis generation and testing. Additionally, by streamlining cell segmentation and classification, the tool enhances the scalability and reproducibility of pathological assessments, ultimately contributing to improved diagnostic accuracy and patient outcomes.

\section{Conclusion and future work}

In this study, we address critical challenges in digital pathology and tackle the usability and deployment issues of the developed models in standard computing environments without the need for high-performance computing systems. Our multi-faceted approach encompasses data refinement through cross-relabeling, leveraging foundation models for robust cell segmentation and classification, optimizing model performance via knowledge distillation, and integrating the optimized model into the QuPath software for practical application. This approach is used to construct a capable, versatile, and adjustable model for cell segmentation and classification, with enhanced performance and usability.

\begin{sloppypar}
While our approach shows potential in the field of computational pathology, certain limitations persist. 
For example, our implementation currently exhibits lower performance in detecting macrophages. 
This serves as an instance of the broader challenge of accurately identifying complex cell types. In order to address this issue, extending our approach to incorporate additional data sources, exploring alternative modeling approaches, and integrating other imaging modalities such as immunohistochemical staining may help improve detection accuracy. Moreover, although the distilled model reduces computational demands, integrating advanced deep learning models into clinical practice requires addressing technological gaps and potential resistance to adopting new tools within established diagnostic processes.
\end{sloppypar}

Future work could focus on several key areas to refine the proposed approach and facilitate its adoption in clinical environments. Enhancing the cell-relabeling process with additional datasets \cite{Graham_Jahanifar_etal._2021} could improve the representation of underrepresented cell types and enhance overall model performance. Also, incorporating additional data sources, such as multi-modal imaging or complementary staining methods, may address limitations related to cell type differentiation and class imbalance. Exploring other foundation models \cite{Vorontsov_Bozkurt_etal._2024,Zimmermann_Vorontsov_etal._2024} or introducing additional modalities \cite{Ding_Wagner_etal._2024,Vaidya_Zhang_etal._2025} may provide alternative architectures better suited to specific tasks or offer improved efficiency. Implementing more complex knowledge distillation techniques \cite{Houyon_Cioppa_etal._2023,Zhang_Song_etal._2019} could further optimize the model's performance and adaptability. Additionally, deeper integration with QuPath or other digital pathology software could provide pathologists more control over cell quantification analysis directly within the QuPath interface, thereby increasing accessibility and usability. Such enhancements would not only refine model performance but also ensure greater adaptability and scalability within various clinical environments. Finally, extensive validation of the model by pathologists and benchmarking against independent datasets are essential steps toward establishing the model's reliability and fostering confidence in its clinical utility.

\section*{Acknowledgments} 
This work was funded in part by the Research Council of Norway grant no. 309439 SFI Visual Intelligence, and the North Norwegian Health Authority grant no. HNF1521-20.

\bibliographystyle{IEEEtran}
\begin{sloppypar}
\begin{thebibliography}{99}

\bibitem{chaplot2020neural} Chaplot, Devendra Singh, et al. "Neural topological slam for visual navigation." Proceedings of the IEEE/CVF conference on computer vision and pattern recognition. 2020.

\bibitem{maksymets2021thda} Maksymets, Oleksandr, et al. "Thda: Treasure hunt data augmentation for semantic navigation." Proceedings of the IEEE/CVF International Conference on Computer Vision. 2021.

\bibitem{mezghan2022memory} Mezghan, Lina, et al. "Memory-augmented reinforcement learning for image-goal navigation." 2022 IEEE/RSJ International Conference on Intelligent Robots and Systems (IROS). IEEE, 2022.

\bibitem{al2022zero} Al-Halah, Ziad, Santhosh Kumar Ramakrishnan, and Kristen Grauman. "Zero experience required: Plug \& play modular transfer learning for semantic visual navigation." Proceedings of the IEEE/CVF Conference on Computer Vision and Pattern Recognition. 2022.

\bibitem{ye2021auxiliary} Ye, Joel, et al. "Auxiliary tasks and exploration enable objectgoal navigation." Proceedings of the IEEE/CVF international conference on computer vision. 2021.

\bibitem{chaplot2020object} Chaplot, Devendra Singh, et al. "Object goal navigation using goal-oriented semantic exploration." Advances in Neural Information Processing Systems 33 (2020)

\bibitem{ramakrishnan2022poni} Ramakrishnan, Santhosh Kumar, et al. "Poni: Potential functions for objectgoal navigation with interaction-free learning." Proceedings of the IEEE/CVF Conference on Computer Vision and Pattern Recognition. 2022.

\bibitem{ramrakhya2022habitat} Ramrakhya, Ram, et al. "Habitat-web: Learning embodied object-search strategies from human demonstrations at scale." Proceedings of the IEEE/CVF Conference on Computer Vision and Pattern Recognition. 2022.

\bibitem{mousavian2019visual} Mousavian, Arsalan, et al. "Visual representations for semantic target driven navigation." 2019 International Conference on Robotics and Automation (ICRA). IEEE, 2019.

\bibitem{dhariwal2021diffusion} Dhariwal, Prafulla, and Alexander Nichol. "Diffusion models beat gans on image synthesis." Advances in neural information processing systems 34 (2021)

\bibitem{ho2022classifier} Ho, Jonathan, and Tim Salimans. "Classifier-free diffusion guidance." arXiv preprint arXiv:2207.12598 (2022).

\bibitem{nichol2021glide} Nichol, Alex, et al. "Glide: Towards photorealistic image generation and editing with text-guided diffusion models." arXiv preprint arXiv:2112.10741 (2021)

\bibitem{brooks2023instructpix2pix} Brooks, Tim, Aleksander Holynski, and Alexei A. Efros. "Instructpix2pix: Learning to follow image editing instructions." Proceedings of the IEEE/CVF Conference on Computer Vision and Pattern Recognition. 2023.

\bibitem{fu2023guiding} Fu, Tsu-Jui, et al. "Guiding instruction-based image editing via multimodal large language models." arXiv preprint arXiv:2309.17102 (2023).

\bibitem{geng2024instructdiffusion} Geng, Zigang, et al. "Instructdiffusion: A generalist modeling interface for vision tasks." Proceedings of the IEEE/CVF Conference on Computer Vision and Pattern Recognition. 2024.

\bibitem{zhou2024minedreamer} Zhou, Enshen, et al. "Minedreamer: Learning to follow instructions via chain-of-imagination for simulated-world control." arXiv preprint arXiv:2403.12037 (2024).

\bibitem{zhou2023esc} Zhou, Kaiwen, et al. "Esc: Exploration with soft commonsense constraints for zero-shot object navigation." International Conference on Machine Learning. PMLR, 2023.

\bibitem{yu2023l3mvn} Yu, Bangguo, Hamidreza Kasaei, and Ming Cao. "L3mvn: Leveraging large language models for visual target navigation." 2023 IEEE/RSJ International Conference on Intelligent Robots and Systems (IROS). IEEE, 2023.

\bibitem{gadre2023cows} Gadre, Samir Yitzhak, et al. "Cows on pasture: Baselines and benchmarks for language-driven zero-shot object navigation." Proceedings of the IEEE/CVF Conference on Computer Vision and Pattern Recognition. 2023.

\bibitem{shah2023navigation} Shah, Dhruv, et al. "Navigation with large language models: Semantic guesswork as a heuristic for planning." Conference on Robot Learning. PMLR, 2023.

\bibitem{cai2024bridging} Cai, Wenzhe, et al. "Bridging zero-shot object navigation and foundation models through pixel-guided navigation skill." 2024 IEEE International Conference on Robotics and Automation (ICRA). IEEE, 2024.

\bibitem{yu2023co} Yu, Bangguo, Hamidreza Kasaei, and Ming Cao. "Co-NavGPT: Multi-robot cooperative visual semantic navigation using large language models." arXiv preprint arXiv:2310.07937 (2023).

\bibitem{wu2024voronav} Wu, Pengying, et al. "Voronav: Voronoi-based zero-shot object navigation with large language model." arXiv preprint arXiv:2401.02695 (2024).

\bibitem{qin2023mp5} Qin, Yiran, et al. "Mp5: A multi-modal open-ended embodied system in minecraft via active perception." arXiv preprint arXiv:2312.07472 (2023).

\bibitem{du2024learning} Du, Yilun, et al. "Learning universal policies via text-guided video generation." Advances in Neural Information Processing Systems 36 (2024).

\bibitem{ajay2024compositional} Ajay, Anurag, et al. "Compositional foundation models for hierarchical planning." Advances in Neural Information Processing Systems 36 (2024).

\bibitem{liang2024skilldiffuser} Liang, Zhixuan, et al. "Skilldiffuser: Interpretable hierarchical planning via skill abstractions in diffusion-based task execution." Proceedings of the IEEE/CVF Conference on Computer Vision and Pattern Recognition. 2024.

\bibitem{heusel2017gans} Heusel, Martin, et al. "Gans trained by a two time-scale update rule converge to a local nash equilibrium." Advances in neural information processing systems 30 (2017).

\bibitem{zhang2018unreasonable} Zhang, Richard, et al. "The unreasonable effectiveness of deep features as a perceptual metric." Proceedings of the IEEE conference on computer vision and pattern recognition. 2018.

\bibitem{brown2020language} Brown, Tom B. "Language models are few-shot learners." arXiv preprint arXiv:2005.14165 (2020).

\bibitem{podell2023sdxl} Podell, Dustin, et al. "Sdxl: Improving latent diffusion models for high-resolution image synthesis." arXiv preprint arXiv:2307.01952 (2023).

\bibitem{brohan2022rt} Brohan, Anthony, et al. "Rt-1: Robotics transformer for real-world control at scale." arXiv preprint arXiv:2212.06817 (2022).

\bibitem{brohan2023rt} Brohan, Anthony, et al. "Rt-2: Vision-language-action models transfer web knowledge to robotic control." arXiv preprint arXiv:2307.15818 (2023).

\bibitem{li2024manipllm} Li, Xiaoqi, et al. "Manipllm: Embodied multimodal large language model for object-centric robotic manipulation." Proceedings of the IEEE/CVF Conference on Computer Vision and Pattern Recognition. 2024.

\bibitem{shah2023vint} Shah, Dhruv, et al. "ViNT: A foundation model for visual navigation." arXiv preprint arXiv:2306.14846 (2023).

\bibitem{liu2024visual} Liu, Haotian, et al. "Visual instruction tuning." Advances in neural information processing systems 36 (2024).

\bibitem{hu2021lora} Hu, Edward J., et al. "Lora: Low-rank adaptation of large language models." arXiv preprint arXiv:2106.09685 (2021).

\bibitem{qin2023supfusion} Qin, Yiran, et al. "SupFusion: Supervised LiDAR-camera fusion for 3D object detection." Proceedings of the IEEE/CVF International Conference on Computer Vision. 2023.

\bibitem{qin2024worldsimbench} Qin, Yiran, et al. "Worldsimbench: Towards video generation models as world simulators." arXiv preprint arXiv:2410.18072 (2024).

\bibitem{yu2025gamefactory} Yu, Jiwen, et al. "GameFactory: Creating New Games with Generative Interactive Videos." arXiv preprint arXiv:2501.08325 (2025).

\bibitem{zhou2024code} Zhou, Enshen, et al. "Code-as-Monitor: Constraint-aware Visual Programming for Reactive and Proactive Robotic Failure Detection." arXiv preprint arXiv:2412.04455 (2024).

\bibitem{zhang2024ad} Zhang, Zaibin, et al. "AD-H: Autonomous Driving with Hierarchical Agents." arXiv preprint arXiv:2406.03474 (2024).

\bibitem{wang2024toward} Wang, Chaoqun, et al. "Toward Accurate Camera-based 3D Object Detection via Cascade Depth Estimation and Calibration." arXiv preprint arXiv:2402.04883 (2024).

\bibitem{huang2024story3d} Huang, Yuzhou, et al. "Story3d-agent: Exploring 3d storytelling visualization with large language models." arXiv preprint arXiv:2408.11801 (2024).

\bibitem{savinov2018semi} Savinov, Nikolay, Alexey Dosovitskiy, and Vladlen Koltun. "Semi-parametric topological memory for navigation." arXiv preprint arXiv:1803.00653 (2018).

\bibitem{majumdar2022zson} Majumdar, Arjun, et al. "Zson: Zero-shot object-goal navigation using multimodal goal embeddings." Advances in Neural Information Processing Systems 35 (2022): 32340-32352.

\bibitem{yadav2023offline} Yadav, Karmesh, et al. "Offline visual representation learning for embodied navigation." Workshop on Reincarnating Reinforcement Learning at ICLR 2023. 2023.

\bibitem{yadav2023ovrl} Yadav, Karmesh, et al. "Ovrl-v2: A simple state-of-art baseline for imagenav and objectnav." arXiv preprint arXiv:2303.07798 (2023).

\bibitem{sun2024fgprompt} Sun, Xinyu, et al. "FGPrompt: fine-grained goal prompting for image-goal navigation." Advances in Neural Information Processing Systems 36 (2024).

\bibitem{zhu2017target} Zhu, Yuke, et al. "Target-driven visual navigation in indoor scenes using deep reinforcement learning." 2017 IEEE international conference on robotics and automation (ICRA). IEEE, 2017.

\bibitem{koh2024generating} Koh, Jing Yu, Daniel Fried, and Russ R. Salakhutdinov. "Generating images with multimodal language models." Advances in Neural Information Processing Systems 36 (2024).

\bibitem{krantz2022instance} Krantz, Jacob, et al. "Instance-specific image goal navigation: Training embodied agents to find object instances." arXiv preprint arXiv:2211.15876 (2022).

\bibitem{schulman2017proximal} Schulman, John, et al. "Proximal policy optimization algorithms." arXiv preprint arXiv:1707.06347 (2017).

\bibitem{anderson2018evaluation} Anderson, Peter, et al. "On evaluation of embodied navigation agents." arXiv preprint arXiv:1807.06757 (2018).

\bibitem{lin2024navcot} Lin, Bingqian, et al. "NavCoT: Boosting LLM-Based Vision-and-Language Navigation via Learning Disentangled Reasoning." arXiv preprint arXiv:2403.07376 (2024).

\bibitem{NavGPT} Zhou, Gengze, Yicong Hong, and Qi Wu. "Navgpt: Explicit reasoning in vision-and-language navigation with large language models." Proceedings of the AAAI Conference on Artificial Intelligence.

\bibitem{hahn2021no} Hahn, Meera, et al. "No rl, no simulation: Learning to navigate without navigating." Advances in Neural Information Processing Systems 34 (2021): 26661-26673.

\bibitem{li2025t2isafety} Li, Lijun, et al. "T2ISafety: Benchmark for Assessing Fairness, Toxicity, and Privacy in Image Generation." arXiv preprint arXiv:2501.12612 (2025).

\bibitem{an2024agfsync} An, Jingkun, et al. "AGFSync: Leveraging AI-Generated Feedback for Preference Optimization in Text-to-Image Generation." arXiv preprint arXiv:2403.13352 (2024).


\end{thebibliography}
\end{sloppypar}

\clearpage
\beginsupplement
\section*{Appendix}
\renewcommand{\thesubsection}{S\arabic{subsection}}

\subsection{\label{chap:S1}PanNuke and MoNuSAC preprocessing}
The PanNuke dataset comprises a set of 7,901 RGB patches, each with dimensions of $256 \times 256$ pixels, which we set as the standard patch size for our analysis. In contrast, the MoNuSAC dataset encompasses 294 images of heterogeneous dimensions. To standardize the MoNuSAC images with our experiments, we implement a standardization protocol. Specifically, for images exceeding the dimensions of $256 \times 256$ pixels, we segment them into equal-sized patches and apply mirror padding to the remaining portions to avoid information loss at the peripherals. Patches with dimensions less than $128 \times 128$ pixels are excluded from the dataset due to the insufficient resolution to capture relevant cellular details. For patches where either dimension falls between 128 and 256 pixels, we employ upsampling to achieve the standard patch size. As a result, we obtain a total of 2,823 RGB patches derived from the MoNuSAC dataset for subsequent analysis. For additional details on the MoNuSAC data preparation process, refer to the source code \cite{Shvetsov_2025a}.
\clearpage

\subsection{\label{chap:S2}Data usage for the methodology}

\counterwithin{figure}{subsection}
\renewcommand{\thefigure}{S\arabic{subsection}}

\begin{figure}[h!]
    \centering
    \includegraphics[width=\textwidth, height=0.85\textheight, keepaspectratio]{images/A2.pdf}
    \caption{Overview of the methodology for cross-labeling, dataset refinement, and model comparison. (1) Cross-relabeling - training and testing cell classification models, (2) Cross-relabeling - using cell classification models to create refined dataset, (3) Fine-tuning and training models for comparison, (4) Student knowledge distillation with refined dataset}
    \label{fig:S2}
\end{figure}
\clearpage

\subsection{\label{chap:S3}Confusion matrices for classification models}
\counterwithin{figure}{subsection}
\renewcommand{\thefigure}{S\arabic{subsection}.\arabic{figure}}

\begin{figure}[h!]
    \centering
    \includegraphics[width=\textwidth, height=0.4\textheight, keepaspectratio]{images/A3_1.pdf}
    \caption{Confusion matrix for PanNuke trained model}
    \label{fig:S3.1}
\end{figure}

\begin{figure}[h!]
    \centering
    \includegraphics[width=\textwidth, height=0.4\textheight, keepaspectratio]{images/A3_2.pdf}
    \caption{Confusion matrix for MoNuSAC trained model}
    \label{fig:S3.2}
\end{figure}

\clearpage

\subsection{\label{chap:S4}Datasets cell counts}

\counterwithin{table}{subsection}
\renewcommand{\thetable}{S\arabic{subsection}}

\begin{table}[h!]
\renewcommand{\arraystretch}{2.0}
\centering
\caption{\label{tab:S4}Cell counts for PanNuke, MoNuSAC and refined datasets. Numbers in parentheses indicate preprocessed cell counts for cell classifier models training and testing.}
%\adjustbox{max width=\textwidth}{%
\begin{tabular}{|l|c|c|c|}
\hline
%\rowcolor{gray!30}
Cell type & PanNuke & MoNuSAC & Refined \\
\hline
Neoplastic & 77,403 (68,031) & - & 105,451 \\
\hline
Epithelial & 26,572 (23,207) & - & 29,926 \\
\hline
Epithelial (benign and malignant) & - & 31,402 & - \\
\hline
Inflammatory & 32,276 & - & - \\
\hline
Lymphocytes & - & 37,045 (33,104) & 65,275 \\
\hline
Neutrophils & - & 1,355 (1,252) & 3,833 \\
\hline
Macrophage & - & 1,842 (1,695) & 3,410 \\
\hline
Dead & 2,908 & - & 2,908 \\
\hline
Connective & 50,585 & - & 50,585 \\
\hline
\end{tabular}
%
%}
\end{table}



\clearpage

\subsection{\label{chap:S5}Definition of validation metrics}
\counterwithin{equation}{subsection}
\renewcommand{\theequation}{\arabic{equation}}

\subsubsection{\label{chap:S5.1}R\textsuperscript{2}}
The coefficient of determination, denoted as $R^2$, is a statistical measure that represents the proportion of variance in the dependent variable that is predictable from the independent variables. In the context of cell quantification in pathology, $R^2$ is used to assess how well the predicted quantities of different cell types in a patch align with the actual quantities observed in the ground truth data, with higher values representing more accurate quantification. $R^2$ is defined as
\begin{equation*}
R^2 = 1 - \frac{\sum_{i=1}^n (y_i - \hat{y}_i)^2}{\sum_{i=1}^n (y_i - \bar{y})^2},
\end{equation*}
where $y_i$ represents the actual number of cells of a specific type in the $i$-th image, $\hat{y}_i$ represents the predicted number of cells of that type in the $i$-th image, $\bar{y}$ is the mean of the actual numbers across all images, and $n$ is the total number of images in the dataset.

The $R^2$ metric has a range of $(-\infty, 1]$. An $R^2$ of 1 indicates perfect prediction, where all predicted values exactly match the actual values. An $R^2$ of 0 suggests that the model explains none of the variability of the response data around its mean. If $R^2$ is negative, it indicates that the model performs worse than a model that simply predicts the mean of the actual values for all observations.

\subsubsection{\label{chap:S5.2}PQ}
Panoptic Quality ($PQ$) is a comprehensive metric used to evaluate the performance of segmentation models in tasks that require both instance segmentation and classification. $PQ$ provides a single score that encapsulates both the detection accuracy (i.e., how many objects were correctly identified) and the segmentation quality (i.e., how accurately the objects' boundaries were delineated). This metric is particularly useful in multiclass scenarios where each pixel is classified into distinct categories, such as different cell types in pathology images.

$PQ$ is calculated as the product of two terms: Detection Quality ($DQ$) and Segmentation Quality ($SQ$). It can be expressed as
\begin{equation*}
PQ = DQ \cdot SQ,
\end{equation*}
where
\begin{equation*}
DQ = \frac{TP}{TP + 0.5\, FP + 0.5\, FN},
\end{equation*}
\begin{equation*}
SQ = \frac{\sum_{(p, g) \in \mathcal{M}} IoU(p, g)}{TP}.
\end{equation*}
In these formulas, $TP$ denotes the number of correctly matched instances between ground truth and prediction, $FP$ denotes the predicted instances that have no corresponding ground truth, $FN$ denotes the ground truth instances that were not detected, $IoU(p, g)$ is the Intersection over Union for a pair of matched instances $p$ (prediction) and $g$ (ground truth), and $\mathcal{M}$ is the set of matched pairs.

The $PQ$ metric is calculated for each class and is averaged across classes to provide a global performance measure.

The $PQ$ score has a range of $[0, 1.0]$, where a higher score indicates better performance in both detecting and segmenting the instances correctly. A $PQ$ of 1 signifies perfect identification and segmentation of all instances, whereas a $PQ$ of 0 indicates that no instances were correctly identified and segmented.

\clearpage

\subsection{\label{chap:S6}Segmentation and Detection quality metrics for teacher and student models}

\begin{table}[h!]
\renewcommand{\arraystretch}{2.0}
\centering
\caption{Segmentation and detection quality for student and teacher models (CI 95\%)}
\label{tab:S6}
%\adjustbox{max width=\textwidth}{%
\begin{tabular}{|l|c|c|}
\hline
%\rowcolor{gray!30}
Metric & Teacher & Student \\
\hline
$SQ_{neoplastic}$ & 0.819 (0.815--0.823) & 0.824 (0.819--0.828) \\
\hline
$SQ_{lymphocyte}$ & 0.795 (0.788--0.802) & 0.790 (0.783--0.796) \\
\hline
$SQ_{connective}$ & 0.770 (0.762--0.776) & 0.780 (0.772--0.786) \\
\hline
$SQ_{dead}$ & 0.659 (0.623--0.688) & 0.657 (0.624--0.695) \\
\hline
$SQ_{epithelial}$ & 0.780 (0.770--0.790) & 0.788 (0.779--0.797) \\
\hline
$SQ_{macrophage}$ & 0.788 (0.760--0.810) & 0.757 (0.730--0.783) \\
\hline
$SQ_{neutrofil}$ & 0.782 (0.761--0.801) & 0.775 (0.759--0.792) \\
\hline
$DQ_{neoplastic}$ & 0.706 (0.692--0.719) & 0.727 (0.712--0.741) \\
\hline
$DQ_{lymphocyte}$ & 0.675 (0.656--0.698) & 0.713 (0.691--0.734) \\
\hline
$DQ_{connective}$ & 0.566 (0.546--0.584) & 0.583 (0.565--0.602) \\
\hline
$DQ_{dead}$ & 0.410 (0.361--0.465) & 0.435 (0.306--0.561) \\
\hline
$DQ_{epithelial}$ & 0.668 (0.639--0.694) & 0.673 (0.644--0.702) \\
\hline
$DQ_{macrophage}$ & 0.657 (0.583--0.727) & 0.615 (0.531--0.703) \\
\hline
$DQ_{neutrofil}$ & 0.691 (0.625--0.753) & 0.729 (0.679--0.778) \\
\hline
\end{tabular}
%
%}
\end{table}

\clearpage

\subsection{\label{chap:S7}QuPath integration method}
We adopt an integration strategy leveraging the paquo \cite{Bayer_AG} library, a Python package that enables direct interaction with QuPath’s internal API, thereby facilitating seamless data exchange without intermediate conversion steps. The data processing pipeline (\hyperref[fig:S7]{Appendix Figure S7}) begins with the acquisition of WSIs and their associated annotations from QuPath, which are represented as Shapely \cite{Gillies_Wel_etal._2024} polygons. Utilizing paquo, we directly read, create, and modify these annotations and detections within a QuPath project in the Python environment. Images are then cropped using these polygons and processed by cell segmentation and classification models employing standard vision processing toolkits such as OpenCV, pyvips, and PyTorch. Additionally, QuPath employs Groovy scripts to initiate a Python process that starts the entire pipeline from QuPath graphical interface: fetching polygons, extracting images from them, and running deep learning model inference on the cropped images. 
The results are returned to QuPath, leveraging paquo's Python bindings to manipulate QuPath data while minimizing the computational overhead typically associated with cross-environment communication.

\counterwithin{figure}{subsection}
\renewcommand{\thefigure}{S\arabic{subsection}}

\begin{figure}[h!]
    \centering
    \includegraphics[width=\textwidth]{images/A7.pdf}
    \caption{QuPath integration workflow using Python environment}
    \label{fig:S7}
\end{figure}

Compared to traditional workflows that involve exporting annotations as GeoJSON, classifying them in Python, and reimporting them into QuPath, our approach offers several advantages. We eliminate the need to switch between programming languages, providing a cohesive and streamlined development process entirely within QuPath software and removing the necessity to use other tools. Meanwhile, we avoid storing annotations as intermediate JSON files unless required for external use or archiving. By conducting the entire inference and post-processing workflow within the Python environment, we leverage the power and flexibility of Python libraries for image processing and machine learning. This approach also enables adjustments to any set of labels and models, thereby improving its applicability.

%\hfill

The distilled model and QuPath integration code are packaged into a Docker container, enabling streamlined execution with the Docker engine. Detailed integration code and deployment instructions can be found in the GitHub repository \cite{Shvetsov_2025b}.

Despite these benefits, we acknowledge that the paquo library is a proof‑of‑concept project in its early development stage and has not been tested across all versions of QuPath.

\clearpage

\subsection{\label{chap:S8}Data and code availability statement}
All datasets, models, and code used in this study are publicly available and can be obtained from the repositories listed below. 
The PanNuke \cite{Gamper_Koohbanani_etal._2019} and MoNuSAC \cite{Verma_Kumar_etal._2021} datasets are publicly accessible, and download information along with detailed descriptions can be found in their respective articles. Preprocessing scripts for PanNuke and MoNuSAC data, as well as individual cell extraction scripts, are available on GitHub \cite{Shvetsov_2025a}. The H-Optimus foundation model used in our experiments can be downloaded from the HuggingFace repository \cite{hoptimus2024}, and model information is available on GitHub \cite{Saillard_Jenatton_etal._2024}. In addition, the integration code for QuPath and the distilled model packaged in a Docker container are provided in the repository \cite{Shvetsov_2025b}, and paquo Python library is available from the authors GitHub repository \cite{Bayer_AG}.
\clearpage

\end{document}

\end{document}
