\section{\shepherd{Discussion and Future Works}}

\noindent
\textbf{Batteryless Computing.}
\shepherd{
ORCA aligns with the batteryless computing paradigm by checkpointing and restoring LoRa communication parameters between power cycles to enable energy-efficient communication. In our design, we treat each round of cloud assistance as an atomic operation for simplicity and timeliness. Beyond that, exploring communication scheduling across multiple power cycles could benefit complex applications. Additionally, we observe that ADR-based parameters may be unreliable under extended duty and power cycles, highlighting the need for a robust algorithm to manage LoRa communication parameters on batteryless devices.}

\noindent
\textbf{Intelligent Acoustic Application.} 
\shepherd{
While ORCA addresses environmental sound classification at the edge, real-world audio-based applications are often more diverse and complex. Exploring areas such as sound event detection, source separation, localization, speech recognition, and integration with human-centric multimodal sensing could further expand its capabilities.}

\noindent
\textbf{Wireless Communication.} 
\shepherd{
Although LoRa and other LPWAN technologies represent state-of-the-art solutions for long-range communication, recent advancements in backscatter offer more energy-efficient designs. Exploring ways to adapt existing LPWAN infrastructure and integrate backscatter radios into batteryless computing devices presents a promising research direction.}















% \shepherd{The focus of ORCA is mainly on optimizing resource usage and communication for LPWAN devices. While ORCA offers valuable contributions at the algorithmic level, we recognize that its innovations are only loosely tied to the challenges of battery-free intermittent systems, which is often associated with LPWANs. Challenges such as the intricacies of energy harvesting, power fluctuations, and long-term power reliability are not discussed. Consequently, this paper does not examine the success or failure of monitoring tasks when environment conditions change over time. In our future work, we would like to focus on developing systems that more directly address the unique challenges of battery-free intermittent systems such as optimizing under changing environmental conditions or other variability over extended periods.}





\section{Conclusion}
Environmental monitoring through acoustic signals is increasingly vital across various real-world applications, creating a demand for energy-efficient, environment-adaptive solutions with high model accuracy. In this paper, we introduce ORCA, the first resource-efficient, cloud-assisted environmental sound recognition system on LPWANs, optimized for wide-area audio sensing on batteryless devices. \shepherd{ORCA leverages edge-cloud collaboration by integrating self-attention and vision transformer architectures to extract sub-spectral features in the cloud while exchanging minimal data. This enhances on-device inference accuracy and significantly reduces communication costs necessary for cloud offloading.} As a result, ORCA achieves up to 80$\times$ energy savings and 220$\times$ latency reduction compared to state-of-the-art methods, without sacrificing accuracy.

% \shepherd{Environmental monitoring using acoustic signals is increasingly important, demanding resource-efficient and accurate solutions on the edge. We present ORCA, a resource-aware, cloud-assisted sound recognition system on LPWANs. ORCA uses attention-based spectral feature selection to minimize communication, while improving accuracy and reducing computation. This results in up to 80$\times$ energy savings and 220$\times$ latency reduction compared to state-of-the-art methods, without sacrificing accuracy.}






