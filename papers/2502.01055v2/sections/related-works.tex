%!TEX root = ../main.tex
\section{Related Works}
\label{sec:relatedworks}
Motion planning through contact presents unique challenges due to the inherently discontinuous nature of contact interactions~\cite{TRO-WENSING-2024,TRO-LELIDEC-2024}. The literature on modeling contact can be broadly divided into smooth and rigid methods. 

\textbf{Smooth contact model. }
The principle of modeling contact in a "smooth" manner involves approximating nonsmooth contact events into smooth and continuous functions relating contact forces to states. This approach often simulates effects similar to springs~\cite{JOB-BLICKHAN-1989}, dampers~\cite{TSMC-MARHEFKA-1999}, or a combination of both~\cite{RAL-NEUNERT-2017,RAL-NEUNERT-2018}. By doing so, it allows contact forces to be expressed as functions of the robot's states and seamlessly integrated into the overall dynamic functions, providing well-defined gradient information.

\textbf{Rigid contact model: hybrid dynamics. } Hybrid systems offer a robust framework for modeling systems that exhibit both continuous and discrete behaviors~\cite{CSM-GOEBEL-2009}. These systems are characterized by their ability to switch between different dynamic regimes, or modes, depending on the contact conditions.
In the locomotion community, the control of switched systems often allows for instantaneous changes in velocity during contact events~\cite{OCS2,HUMANOIDS-FARSHIDIAN-2017,IFAC-FARSHIDIAN20171463}, while continuous dynamics govern the system at other times. This approach requires a predefined gait, which specifies a sequence of potential contact points~\cite{IROS-CHEETAH,ICRA-CHENG-2022,TRO-LOPES-2014}.

\textbf{Rigid contact model: implicit formulations.} 
There are two mathematically equivalent approaches to implicitly encode the discrete nature of hybrid systems for switching between different continuous subsystems. One approach is through Mixed Integer Programming (MIP)~\cite{SIAMReview-vielma-2015}, which introduces binary integer variables to act as switches for encoding contact events~\cite{ICHR-DEITS-2014, IROS-ACEITUNO-2017, RAL-ACEITUNO-2018}. This method is straightforward in its implementation, yet optimizing these discrete variables is challenging for gradient-based NLP solvers and often requires specialized solvers, such as Gurobi~\cite{gurobi}. Furthermore, the number of integer variables can significantly increase with the number of contact modes and the planning horizon, leading to computational intractability~\cite{SHIRAI-YUKI-2024, TAC-MARCUCCI-2021}. On the other hand, the contact force and condition can be encoded through the introduction of complementarity constraints.

On the other hand, contact forces and conditions can be encoded through the introduction of complementarity constraints. Since~\cite{posa2014ijrr-traopt-directmethod-contact}, this approach has recently gained attention because it transforms the problem into a continuous NLP problem without compromising the discrete characteristics of contact. This transformation allows for the use of modern numerical optimization tools to solve the problem effectively~\cite{le2024fast, RSS-yang-2024, aydinoglu2023icra-realtime-multicontact-mpc-admm}. However, the failure of CQs in this context can lead to significant difficulties in solving these problems~\cite{fletcher2000practical, SIOPT-FLETCHER-2006, OMS-Fletcher-2004}. \crisp aims to provide an efficient and robust solution by addressing the challenges associated with solving nonlinear contact problems that include general nonlinear complementarity constraints.