%!TEX root = main.tex
\section{Conclusion}
\label{sec:conclusion}

We presented \crisp, a primal-only numerical solver for contact-implicit motion planning that is based on sequential convex optimization. We started by uncovering the geometric insights underpinning the difficulty of solving MPCCs arising from contact-implicit planning. That motivated us to design a primal-only algorithm where each trust-region subproblem is convex and feasible by construction. For the first time, we proved sufficient conditions on the algorithm's convergence to stationary points of the merit function. With a careful C++ implementation, we benchmarked \crisp against three state-of-the-art solvers on five contact-implicit planning problems, demonstrating superior robustness and capability to generate entirely new contact sequences from scratch.

\textbf{Limitations and future work. }
% Our approach presents several avenues for improvement and future research. 
First, while \crisp accepts general nonlinear programming problem definitions applicable to various optimal control and motion planning problems, it requires further development to evolve into a comprehensive robotics optimization toolbox comparable to OCS2~\cite{OCS2} and CROCODDYL~\cite{mastalli20crocoddyl}. This development primarily involves integration with advanced dynamics libraries such as Pinocchio~\cite{carpentier2019pinocchio}, which would enable \crisp to handle more complex dynamics and obtain their derivative information efficiently. 
Second, although our C++ implementation is highly efficient, its adaptivity for real-time tasks could be further increased. The acquisition of gradients and Hessian matrices could be parallelized, and matrix operations could achieve significant speedups if implemented on GPUs~\cite{kang2024arxiv-strom}. Additionally, while our subproblems are already cheap as convex QPs, the overall framework's efficiency is still constrained by the speed of the underlying QP solver. As mentioned in Remark~\ref{remark:fom_qp}, we plan to test \crisp with scalable first-order QP solvers. On the theoretical side, 
% \crisp focuses on the reduction of a merit function whose stationary point, whose relation to the original problem's local minimum is well-studied under certain CQs. However, 
in MPCC problems, the relationship between local solutions of the original problem and stationary points of the merit function warrants further investigation.

\section*{Acknowledgment}

We thank Michael Posa and Zac Manchester for insightful discussions about contact-rich motion planning.

% Future work should explore stronger convergence guarantees for \crisp beyond~\prettyref{thm:local_convergence}.

% To the best of our knowledge, there are currently no alternatives that both support sparsity and fast yet accurate, creating a computational bottleneck for sequential optimization. 
