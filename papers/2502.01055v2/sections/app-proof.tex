%!TEX root = ../main.tex

\section{Proof of \prettyref{thm:local_convergence}}
\label{app:proof_main_thm}

\begin{proof}
    As $x_k$ converges to $ x^\star$, we have $\|p_k\| = \|x_{k+1}-x_k\| \rightarrow 0$. By definition, there exists a constant $K$, such that $\norm{p_k}<\Delta_{\mathrm{min}}$ for all $k > K$, indicating that the trust-region constraint becomes inactive in the subproblem. In this proof, we consider only equality constraints, and the proof technique for inequality constraints is similar and will be addressed at the end.
In this case, the merit function~\eqref{eq:merit-func} becomes: 
    \bea
        \phi_1(x;\mu) \triangleq J(x) +  \sum_{i \in \mathcal{E}} \mu_i| c_i(x) |,
    \eea
    where $x \in \Real{n}$, $\calE$ denotes the equality constraints set and $|\mathcal{E}| = m_{\mathcal{E}}$. First, we rewrite the trust-region subproblem~\eqref{eq:subproblem-nonsmooth} to directly optimize over $x_{k+1}$ instead of $p_k$:
    \bea
        x_{k+1} = \displaystyle \argmin_x \{ q_{\mu,k}(x) = J(x) + \sum_{i \in \mathcal{E}}\mu_i \lvert c_i(x_k) + \nabla c_i(x_k)^\top (x - x_k) \rvert \}.
    \eea
    For the $i$th constraint $c_i\in\calE$, we define $w_{k,i}\in\Real{}$ as:
    \begin{equation}
        w_{k,i} \triangleq c_i(x_k) + \nabla c_i(x_k)^\top (x_{k+1} - x_k),
    \end{equation}
    Since the $l_1$ norm $|\cdot|$ is piecewise smooth, we separate the scalar $w_{k,i}$ into three smooth sets:
    \begin{align}
        \{w_{k,i} < 0\}, \{w_{k,i} = 0\}, \{w_{k,i} > 0\}.
    \end{align}
    Extending to the vector of all equality constraints $w_k = \bmat{ccc}w_{k,1}& \dots &w_{k,m_{\calE}}\emat\in\Real{m_\calE}$, we can divide $            \{w_k\}^{\infty}_{k=K}$ to $3^{m_{\calE}}$ smooth sets without overlapping.
    
    Since the number of such sets is finite, there must exist a subsequence $\{w_{n_k}\}^{\infty}_{k=1}$ that belongs to one of these sets, denoting as \( S \). Without loss of generality, we assume that in this set, the first \( m_1 \) components of \( w_{n_k} \) are positive, the \( m_1 + 1 \) to \( m_1 + m_2 \) components are negative, and the remaining components are zero.
    
    For any \( p \in \mathbb{R}^n \), we define the directional derivatives\footnote{We use $\downarrow$ for simplicity, but this is equivalent to $\rightarrow$ because one can easily choose $p$ as $-p$ if $t < 0$.}:
    \begin{align}
        D(\phi_1(x^\star;\mu);p) &\triangleq \lim_{t \downarrow 0} \frac{\phi_1(x^\star + tp;\mu) - \phi_1(x^\star;\mu)}{t}, \\
        D(q_{\mu,n_k}(x_{n_{k}+1});p) &\triangleq \lim_{t \downarrow 0} \frac{q_{\mu,n_k}(x_{n_{k}+1} + tp) - q_{\mu,n_k}(x_{n_{k}+1})}{t}.
    \end{align}
    Since $x_{n_k+1}$ is the global minimizer of the convex function $q_{\mu,n_k}(x)$, we have $D(q_{\mu,n_k}(x_{n_{k}+1});p) \geq 0$. On the other hand,
    \begin{align}
        D(q_{\mu,n_k}(x_{n_{k}+1});p) &= 
         \lim_{t \downarrow 0} \frac{J(x_{n_k+1} + tp) + \sum_{i\in \calE}\mu_i| w_{n_k,i} + t\nabla c_i(x_{n_k})^\top p | - J(x_{n_k+1}) - \sum_{i\in \calE}\mu_i| w_{n_k,i} |}{t} \\
        &= \nabla J(x_{n_k+1})^\top p + \lim_{t \downarrow 0} \frac{\sum_{i\in \calE}\mu_i| w_{n_k,i} + t\nabla c_i(x_{n_k})^\top p | - \sum_{i\in \calE}\mu_i| w_{n_k,i} |}{t}.\\
        & =  \nabla J(x_{n_k+1})^\top p + \sum_{i=1}^{m_1} \mu_i\nabla c_i(x_{n_k})^\top p 
         - \sum_{i=m_1+1}^{m_1+m_2} \mu_i\nabla c_i(x_{n_k})^\top p
        + \sum_{i=m_1+m_2+1}^{m} \mu_i|\nabla c_i(x_{n_k})^\top p|.\\
    \end{align}

    Using the fact that:
    \begin{equation}
        \lim_{t \downarrow 0} \frac{| a + tb | - | a |}{t} = \begin{cases}
            b, & a > 0 \\
            -b, & a < 0 \\
            |b|, & a = 0
        \end{cases}
    \end{equation}
    
    Since $J(x)$ is continuous differentiable and $\nabla c_i$ is Lipschitz, as $k \to \infty$:
    \begin{align}
        \nabla J(x_{n_k+1}) &\to \nabla J(x^\star), \\
        \nabla c_i(x_{n_k}) &\to \nabla c_i(x^\star).
    \end{align}
    
    Thus,
    \begin{align}
        D(q_{\mu,n_k}(x_{n_{k}+1});p) \xrightarrow[k \to \infty]{}
        \nabla J(x^\star)^\top p + \sum_{i=1}^{m_1} \mu_i\nabla c_i(x^\star)^\top p - \sum_{i=m_1+1}^{m_1+m_2} \mu_i\nabla c_i(x^\star)^\top p +
         \sum_{i=m_1+m_2+1}^{m} \mu_i|(\nabla c_i(x^\star)^\top p)| \geq 0.
    \end{align}

    The last inequality comes from the fact that if all the elements in a scaler sequence are nonnegative, then the limitation of the sequence is nonnegative.
    
    Now we turn to calculate $D(\phi_1(x^\star;\mu);p)$. One should be careful: since $S$ is not closed, even the first $m_1$ components of $w_{n_k}$ are always positive, the corresponding components in the converged point $w^\star = \bmat{ccc}c_1(x^\star)&\dots&c_{m_\calE}(x^\star)\emat $ will possibly converge to 0. Similar things happen to the negative part. However, the $m_1 + m_2 + 1 \sim m_\calE$ components of $w^\star$ will always be 0. To fix this, we assume that, for the $1 \sim m_1$ components of $w^\star$, $1 \sim m_{1,+}$ are positive, while $m_{1,+} + 1 \sim m_1$'s components of $w^\star$ become zero. Similarly, $m_1 + 1 \sim m_1 + m_{2,-}$ are negative, while $m_1 + m_{2,-} + 1 \sim m_1 + m_2$'s components of $w^\star$ become zero. Thus,
    
    \begin{align}
        D(\phi_1(x^\star;\mu);p) &= \lim_{t \downarrow 0} \frac{J(x^\star + tp) + \sum_{i \in \calE}\mu_i|c_i(x^\star + tp)| - J(x^\star) - \sum_{i \in \calE}\mu_i|c_i(x^\star)|}{t} \\
        &= \nabla J(x^\star)^\top p + \sum_{i=1}^{m_{1,+}} \mu_i\nabla c_i(x^\star)^\top p + \sum_{i=m_{1,+}+1}^{m_1} \mu_i|\nabla c_i(x^\star)^\top p|\\ 
        &- \sum_{i=m_1+1}^{m_1+m_{2,-}} \mu_i\nabla c_i(x^\star)^\top p+ \sum_{i=m_1+m_{2,-}+1}^{m_1+m_2} \mu_i|\nabla c_i(x^\star)^\top p| 
        + \sum_{i=m_1+m_2+1}^{m_\calE} \mu_i|\nabla c_i(x^\star)^\top p|.\\
        & \geq \nabla J(x^\star)^\top p + \sum_{i=1}^{m_1} \mu_i\nabla c_i(x^\star)^\top p 
         - \sum_{i=m_1+1}^{m_1+m_2} \mu_i\nabla c_i(x^\star)^\top p
        + \sum_{i=m_1+m_2+1}^{m_\calE} \mu_i|\nabla c_i(x^\star)^\top p|.\\
        & \geq 0
    \end{align}
    For inequality constraints we define $w_{k,i}$ as the same. We still separate $\Real{}$ into 
    \bea
    \{w_{k,i} < 0\}, \{w_{k,i} = 0\}, \{w_{k,i} > 0\}
    \eea
    and similarly, we have
    \begin{equation}
        \lim_{t \downarrow 0} \frac{| a + tb |^- - | a |^-}{t} = \begin{cases}
            0, & a > 0 \\
            -b, & a < 0 \\
            |b|^-, & a = 0
        \end{cases}
    \end{equation}
    Other parts of the proof are the same. Consequently $D(\phi_1(x^\star;\mu);p) \geq 0 \quad\forall p$, we conclude the proof.
\end{proof}



% \section{Proof of \prettyref{thm:local_convergence}}
% \label{app:proof_main_thm}

% \begin{proof}
%     As $x_k$ converges to $x^\star$, we observe that $p_k = x_{k+1}-x_k$ approaches zero. By the definition of convergence, there exists a $K$ such that for all $k > K$, $\norm{p_k}<\Delta_{\mathrm{min}}$, indicating that the trust-region constraint becomes inactive for the subproblem. In this proof, we focus on $k > K$ and consider only equality constraints in the subproblem. The proof technique for inequality constraints is similar and will be addressed at the end.

%     We define the merit function $\phi_1(x;\mu)$ as:
%     \bea
%         \phi_1(x;\mu) \triangleq J(x) +  \sum_{i \in \mathcal{E}} \mu_i| c_i(x) |
%     \eea
%     where $|\mathcal{E}| = m_{\mathcal{E}}$, $x \in \Real{n}$. The trust-region subproblem is reformulated as a function of $x_{k+1}$ instead of $p_k$:
%     \bea
%         x_{k+1} = \displaystyle \argmin_x \{ q_{\mu,k}(x) = J(x) + \sum_{i \in \mathcal{E}}\mu_i \lvert c_i(x_k) + \nabla c_i(x_k)^\top (x - x_k) \rvert \}
%     \eea
%     Let $w_{k,i}$ be defined as:
%     \begin{equation}
%         w_{k,i} \triangleq c_i(x_k) + \nabla c_i(x_k)^\top (x_{k+1} - x_k) \in \mathbb{R},
%     \end{equation}
%     Given that the $\ell_1$-norm is piecewise smooth, we partition $\mathbb{R}$ into three sets:
%     \begin{align}
%         \{w_{k,i} < 0\}, \{w_{k,i} = 0\}, \{w_{k,i} > 0\}
%     \end{align}
%     After concatenating $w_{k,i}$ to form $w_k = \bmat{ccc}w_{k,1}& \dots &w_{k,m_{\calE}}\emat$, $\Real{m_{\calE}}$ can be divided into $3^{m_{\calE}}$ non-overlapping sets.

%     Due to the finite number of such sets, there must exist a subsequence $w_{n_k}$ belonging to one of these sets. We denote this limiting set as $S$ and assume that within this set, the first $m_1$ components of $w_k$ are positive, the next $m_1 + 1$ to $m_1 + m_2$ components are negative, and the remaining $m_1 + m_2 + 1$ to $m_{\calE}$ components are zero.

%     For any $p \in \mathbb{R}^n$, we define the directional derivatives as follows:
%     \begin{align}
%         D(\phi_1(x^\star;\mu);p) &\triangleq \lim_{t \downarrow 0} \frac{\phi_1(x^\star + tp;\mu) - \phi_1(x^\star;\mu)}{t}, \\
%         D(q_{\mu,n_k}(x_{n_{k}+1});p) &\triangleq \lim_{t \downarrow 0} \frac{q_{\mu,n_k}(x_{n_{k}+1} + tp) - q_{\mu,n_k}(x_{n_{k}+1})}{t}.
%     \end{align}
%     As $x_{n_k+1}$ is the global minimizer of the convex function $q_{\mu,n_k}(x)$, we have $D(q_{\mu,n_k}(x_{n_{k}+1});p) \geq 0$. Furthermore,
%     \begin{align}
%         D(q_{\mu,n_k}(x_{n_{k}+1});p) =\\
%          \lim_{t \downarrow 0} \frac{J(x_{n_k+1} + tp) + \sum_{i\in \calE}| w_{n_k} + \nabla c_i(x_{n_k})^\top p | - J(x_{n_k+1}) - | w_{n_k} |}{t} \\
%         = \nabla J(x_{n_k+1})^\top p + \lim_{t \downarrow 0} \frac{\sum_{i\in \calE}| w_{n_k} + \nabla c_i(x_{n_k})^\top p | - | w_{n_k} |}{t}.
%     \end{align}

%     Utilizing the following property:
%     \begin{equation}
%         \lim_{t \downarrow 0} \frac{| a + tb | - | a |}{t} = \begin{cases}
%             b, & a > 0 \\
%             -b, & a < 0 \\
%             |b|, & a = 0
%         \end{cases}
%     \end{equation}
    
%     Given that $J(x)$ is convex quadratic and $\nabla c_i$ is Lipschitz continuous, as $k \to \infty$:
%     \begin{align}
%         \nabla J(x_{n_k+1}) &\to \nabla J(x^\star), \\
%         \nabla c_i(x_{n_k}) &\to \nabla c_i(x^\star).
%     \end{align}
    
%     Consequently,
%     \begin{align}
%         D(q_{\mu,n_k}(x_{n_{k}+1});p) \xrightarrow[k \to \infty]{} \\
%         \nabla J(x^\star)^\top p + \sum_{i=1}^{m_1} \nabla c_i(x^\star)^\top p - \sum_{i=m_1+1}^{m_1+m_2} \nabla c_i(x^\star)^\top p +\\
%          \sum_{i=m_1+m_2+1}^{m} |(\nabla c_i(x^\star)^\top p)| \geq 0.
%     \end{align}

%     The final inequality stems from the fact that $a_k \to a^\star$, $a_k \geq 0 \implies a^\star \geq 0$.
    
%     Next, we calculate $D(\phi_1(x^\star;\mu);p)$. It is important to note that since $S$ is not closed, even if the first $m_1$ components of $w_{n_k}$ are always positive, the corresponding components in the converged point $w^\star = \bmat{ccc}c_1(x^\star)&\dots&c_{m_\calE}(x^\star)\emat$ may converge to 0. A similar situation occurs for the negative components. However, the $m_1 + m_2 + 1$ to $m$ components of $w^\star$ will always be 0. To address this, we assume that for the first $m_1$ components of $w^\star$, components 1 to $m_{1,+}$ are positive, while components $m_{1,+} + 1$ to $m_1$ become zero. Similarly, components $m_1 + 1$ to $m_1 + m_{2,-}$ are negative, while components $m_1 + m_{2,-} + 1$ to $m_1 + m_2$ become zero. Thus,
    
%     \begin{align}
%         D(\phi_1(x^\star;\mu);p) &= \lim_{t \downarrow 0} \frac{J(x^\star + tp) + \sum_{i \in \calE}|c_i(x^\star + tp)| - J(x^\star) - \sum_{i \in \calE}|c_i(x^\star)|}{t} \\
%         &= \nabla J(x^\star)^\top p + \sum_{i=1}^{m_{1,+}} \nabla c_i(x^\star)^\top p + \sum_{i=m_{1,+}+1}^{m_1} |\nabla c_i(x^\star)^\top p| \\
%         & - \sum_{i=m_1+1}^{m_1+m_{2,-}} \nabla c_i(x^\star)^\top p+ \sum_{i=m_1+m_{2,-}+1}^{m_1+m_2} |\nabla c_i(x^\star)^\top p| \\
%         & + \sum_{i=m_1+m_2+1}^{m} |\nabla c_i(x^\star)^\top p|.
%         & \geq \nabla J(x^\star)^\top p + \sum_{i=1}^{m_1} \nabla c_i(x^\star)^\top p 
%         & - \sum_{i=m_1+1}^{m_1+m_2} \nabla c_i(x^\star)^\top p
%         & + \sum_{i=m_1+m_2+1}^{m} |\nabla c_i(x^\star)^\top p|.
%         & \geq 0
%     \end{align}
%     For inequality constraints, we define $w_{k,i}$ in the same manner. We again partition $\Real{}$ into:
%     \bea
%     \{w_{k,i} < 0\}, \{w_{k,i} = 0\}, \{w_{k,i} > 0\}
%     \eea
%     and analogously, we have:
%     \begin{equation}
%         \lim_{t \downarrow 0} \frac{| a + tb |^- - | a |^-}{t} = \begin{cases}
%             0, & a > 0 \\
%             -b, & a < 0 \\
%             |b|^-, & a = 0
%         \end{cases}
%     \end{equation}
%     The remaining parts of the proof follow a similar pattern.
    
%     In conclusion, since $D(\phi_1(x^\star;\mu);p) \geq 0 \quad\forall p$, the proof is complete.
% \end{proof}