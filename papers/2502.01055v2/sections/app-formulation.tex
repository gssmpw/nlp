\section{Contact-implicit Formulations}\label{app:contact-implicit-formulation-for-experimental-tasks}
In this section, we provide detailed contact-implicit optimization problem formulations for the five tasks, including tracking objectives, dynamics, contact constraints, and other constraints. We believe this presentation is not only valuable for understanding the principles of contact-implicit modeling, but also represents a necessary step for the model-based community. Interested readers are encouraged to refer to these formulations to test these same problems using their own algorithms. All problem formulations are open-sourced alongside \crisp.
\subsection{Cartpole with Softwalls}\label{app:formulation-cartpole}
\textbf{Dynamics constraints. } Denote the full states of the system as:
$$v = \underbrace{[x, \theta, \dot{x}, \dot{\theta},}_{\text{states}} \underbrace{u, \lambda_1, \lambda_2]}_{\text{control}},$$
and we use subscript $i$ to denote the corresponding variable at time stamp $i$. The dynamics can be written as:
\begin{align}
(m_c + m_p)\ddot{x} + m_p\ell\ddot{\theta}\cos\theta  - (u - \lambda_1 + \lambda_2) &= 0, \label{eq:pushbot:dynamics01}\\
m_p \ell \ddot{\theta} - (\lambda_2 - \lambda_1 - m_p \ddot{x}) \cos\theta - m_p g \sin\theta &= 0.\label{eq:pushbot:dynamics02}
\end{align}
Here, $x$ denotes the cart position, while $m_c$ and $m_p$ represent the mass of the cart and the pole end, respectively. It is important to note that the contact forces $\lambda_1$ and $\lambda_2$ are incorporated into the dynamics equations despite their discrete nature. These forces are automatically triggered and controlled by the complementarity constraints introduced below, which exemplifies the principle and power of contact-implicit formulation. We utilize the Semi-Implicit Euler method \cite{wiki:Semi-implicit_Euler_method} to discretize the dynamics with $dt$. At time stamp $i$:
\begin{align}
    x_{i+1} -x_{i} - \dot{x}_{i+1}dt &= 0,\\
    \theta_{i+1} -\theta_{i} - \dot{\theta}_{i+1}dt &= 0,\\
    \dot{x}_{i+1} -x_{i} - \ddot{x}_{i}dt &= 0,\\
    \dot{\theta}_{i+1} -\theta_{i} - \ddot{\theta}_{i}dt &= 0.
\end{align}
In the above dynamics constraints, $\ddot{x}$ and $\ddot{\theta}$ can be simply derived from~\eqref{eq:pushbot:dynamics01}-(\ref{eq:pushbot:dynamics02}).

\textbf{Contact constraints.}
The complementary constraints from the contact are:
\begin{align}
 0 \leq \lambda_1 &\perp \left(\frac{\lambda_1}{k_1} + d_1 - x - \ell \sin \theta\right)\geq 0, \\
 0 \leq \lambda_2 &\perp \left(\frac{\lambda_2}{k_2} + d_2 + x + \ell \sin \theta\right)\geq 0.
\end{align}
This complementarity constraint is constructed using the relationship between the pole's position and the wall, along with the contact force. Its underlying meaning is that the wall can only provide unidirectional force. When there is no contact, $\lambda$ is zero. Upon contact, the force becomes proportional to the compression distance with coefficients $k$.

\textbf{Initial Constraints.}
The optimization problem is also subject to equality constraints that specify the initial state. 
\begin{align}
    x_0 &= x_{initial}\\
    \theta_0 &= \theta_{initial}\\
    \dot{x}_0 &= \dot{x}_{initial}\\
    \dot{\theta}_0 &= \dot{\theta}_{initial}
\end{align}

\textbf{Cost. }
For the objective function, we opt to minimize both the control effort and the terminal tracking error, each expressed in quadratic form. It is worth noting that intermediate tracking loss is not incorporated into our formulation. This decision stems from the general absence of trivial reference trajectories in contact-involved trajectory optimization tasks. Our goal is to generate a feasible trajectory from scratch that drives the robot as close as possible to the desired terminal state.
\begin{equation}
    f = \frac{1}{2} \sum_{i=1}^{N-1} v_i^T \begin{bmatrix} 0 & 0 \\ 0 & R \end{bmatrix} v_i + \frac{1}{2} v_N^T \begin{bmatrix} Q & 0 \\ 0 & 0 \end{bmatrix} v_N,
\end{equation}
where $Q\in\mathbb{R}^{4\times4}$ and $R\in\mathbb{R}^{3\times3}$ are weighting matrices for the state and control variables. All the default values of the parameters are provided in the example problems of \crisp. 

\subsection{Push Box}\label{app:formulation-pushbox}
We use $p_x$, $p_y$, and $\theta$ to represent the position and orientation of the box in world frame $\{w\}$, while the contact position $(c_x,c_y)$ and contact force $\lambda$ at each facet is defined in the body frame $\{b\}$. We assume that at any given moment, there is only one point of contact between the pusher and the box with only the corresponding normal force applied, and the entire process is quasi-static. The positive direction of the applied forces is defined with respect to the body frame $\{b\}$ of the robot.

\textbf{Dynamics constraints. }
In this work, following \cite{graesdal2024tightconvexrelaxationscontactrich}, we adapt the commonly used ellipsoidal approximation of the limit surface to model the interaction between the contact force applied
by the pusher and the resulting spatial slider velocity. The model captures the principle of the motion of the box while keeping its simplicity.
Denote the full states of the system as: 
$$v = \underbrace{[p_x, p_y, \theta,}_{\text{states}} \underbrace{c_x,c_y,\lambda_{1,y},\lambda_{2,x},\lambda_{3,y}, \lambda_{4,x}]}_{\text{control}}.$$ Then, the push box dynamics can be writen as:
\begin{align}
    \dot{p}_x &= \frac{1}{\mu m g}.\left[(\lambda_{2,x}+\lambda_{4,x})\cos\theta - (\lambda_{1,y} + \lambda_{3,y})\sin\theta\right],\\
\dot{p}_y &= \frac{1}{\mu m g}.\left[(\lambda_{2,x}+\lambda_{4,x})\sin\theta + (\lambda_{1,y} + \lambda_{3,y})\cos\theta\right],\\
\dot{\theta} &= \frac{1}{c r \mu m g}.\left[-c_y(\lambda_{2,x}+\lambda_{4,x}) + c_x(\lambda_{1,y} + \lambda_{3,y})\right],
\end{align}
where $c \in [0,1]$ is the integration constant of the box, and $r$ is the characteristic distance, typically
chosen as the max distance between a contact point and
origin of frame $\{b\}$. Discretize the continuous dynamics with the explicit Euler method, we get:
\begin{align}
    p_{x,k+1} &= p_{x,k} + \dot{p}_{x,k}dt,\\
        p_{y,k+1} &= p_{y,k} + \dot{p}_{y,k}dt,\\    \theta_{k+1} &= \theta_{k} + \dot{\theta}_kdt.
\end{align}

\textbf{Contact constraints. }
First, we ensure that the contact force can only be applied through the contact point $(c_x,c_y)$, and only pointed inward the box, utilizing the contact-implicit formulation:
\begin{align}
 0 \leq \lambda_{1,y} &\perp \left(c_y+b\right)\geq 0, \\
 0 \leq \lambda_{2,x} &\perp \left(c_x+a\right)\geq 0,\\
  0 \leq -\lambda_{3,y} &\perp \left(b-c_y\right)\geq 0,\\ 0 \leq -\lambda_{4,x} &\perp \left(a-c_x\right)\geq 0.
\end{align}
Moreover, to prevent the simultaneous application of forces on adjacent edges at corners, we introduce additional complementarity constraints. These constraints ensure that at any given time, only one force can be active. This is formulated as follows:
\begin{align}
 0 \leq \lambda_{1,y} &\perp \lambda_{2,x}\geq 0, \\
  0 \leq \lambda_{1,y} &\perp -\lambda_{3,y}\geq 0, \\
   0 \leq \lambda_{1,y} &\perp -\lambda_{4,x}\geq 0,\\
    0 \leq \lambda_{2,x} &\perp -\lambda_{3,y}\geq 0,\\
     0 \leq \lambda_{2,x} &\perp -\lambda_{4,x}\geq 0,\\
      0 \leq -\lambda_{3,y} &\perp -\lambda_{4,x}\geq 0.
\end{align}

\textbf{Initial Constraints.}
We add equality constraints to enforce the initial state of the box with $(\Bar{p}_{x,0}, \Bar{p}_{y,0},\Bar{\theta}_{0})$. Note that the initial condition is set to zero in this task.

\textbf{Cost.}
Similar to the setting of cartpole with walls, we adopt a quadratic objective function to penalize the total four contact forces $\lambda$ and the distance to the desired terminal condition $\Bar{v}_N$.
\begin{equation}
    f = \frac{1}{2} \sum_{i=1}^{N-1} v_i^T \begin{bmatrix} 0 & 0 \\ 0 & R \end{bmatrix} v_i + \frac{1}{2} (v_N-\Bar{v}_N)^T \begin{bmatrix} Q & 0 \\ 0 & 0 \end{bmatrix} (v_N-\Bar{v}_N),
\end{equation}

\subsection{Transport}\label{app:formulation-transport}
\textbf{Dynamics constraints:}
Denote all states as $$v = \underbrace{[x_1, x_2, \dot{x}_1, \dot{x}_2, p, q}_{\text{states}} \underbrace{f,u}_{\text{control}}].$$
$f$ is the friction between the two blocks with friction coefficients $\mu_1$; $p$ and $q$ are two slack variables for modeling the direction of the $f$, which is decided by the relative movements between the two blocks. The positive directions of $f$ and $x$ are both horizontally to the right. The dynamics for this problem are straghtforward:
\begin{align}
    m_1\ddot{x}_1 &= f,\\
    m_2\ddot{x}_2 &= u-f.
\end{align}
To avoid cargo $m_1$ from falling off the truck $m_2$, we need:
\beq -l_0 \leq x_1 - x_2 \leq l_0.\eeq

\textbf{Contact constraints.}
The control of the magnitude and direction of friction requires careful handling. $f$ is not an active force; it needs to be indirectly controlled through the manipulation of $u$. This indirect control determines the friction's direction, magnitude, and whether it manifests as static or kinetic friction.
\begin{align}
    \dot{x}_2 - \dot{x}_1 &= p - q,\\
    0\leq p &\perp q \geq 0,\\
    0\leq p &\perp (\mu_1 m_1g -f) \geq 0,\\
    0\leq q &\perp (f+ \mu_1m_1g) \geq 0.
\end{align}
This establishes the relationship between relative velocity and friction force while encoding the transition between static and kinetic friction in a clean way.

\textbf{Initial constraints.}
We add equality constraints to enforce the initial pose and velocity of the cart and the payload.

\textbf{Cost.}
We adopt a quadratic objective function to penalize the active force $u$ and the terminal pos and velocity tracking error.
\subsection{Push T}\label{app:formulation-pushT}
\textbf{Contact-Implicit Formulation:}
Denote the contact point in body frame (located at the COM of T block) as $(c_x,c_y)$, we have 8 different contact mode that exhibit their specific complementarity constraints.
\begin{align}
    -2l\leq &c_x \leq 2l, -d_cl \leq c_y \leq (4-d_c)l,\\
    \lambda_1 \neq 0 &\Rightarrow c_y - (4-d_c)l = 0,\\
    \lambda_2 \neq 0 &\Rightarrow |c_x - 2l| + |c_y-(4-d_c)l| + |c_y-(3-d_c)l| - l = 0,\\
    \lambda_3 \neq 0 &\Rightarrow |c_x - 2l| + |c_x-0.5l| + |c_y-(3-d_c)l| - 1.5l = 0,\\
    \lambda_4 \neq 0 &\Rightarrow |c_x-0.5l| + |c_y-(3-d_c)l| + |c_y + d_cl| - 3l = 0,\\
    \lambda_5 \neq 0 &\Rightarrow |c_y + d_cl| + |c_x + 0.5l| + |c_x - 0.5l| - l = 0,\\
    \lambda_6 \neq 0 &\Rightarrow |c_x + 0.5l| + |c_y-(3-d_c)l| + |c_y + d_cl| - 3l = 0,\\
    \lambda_7 \neq 0 &\Rightarrow |c_x + 2l| + |c_x+0.5l| + |c_y-(3-d_c)l| - 1.5l = 0,\\
    \lambda_8 \neq 0 &\Rightarrow |c_x + 2l| + |c_y-(4-d_c)l| + |c_y-(3-d_c)l| - l = 0.
\end{align}
Designate the positive direction of contact forces as rightward for the $x$
axis and upward for the $y$ axis. And introduce the following slack variables $v$ and $w$ complementing each other to smooth the absolute operation $|\cdot|$ as we did in the transport example:
\begin{align}
    v_1 - w_1 &= c_x -2l,\\
    v_2 - w_2 &= c_y -(4-d_c)l,\\
    v_3 - w_3 &= c_y -(3-d_c)l,\\
    v_4 - w_4 &= c_x -0.5l,\\
    v_5 - w_5 &= c_y + d_cl,\\
    v_6 - w_6 &= c_x + 0.5l,\\
    v_7 - w_7 &= c_x + 2l,\\
    0 \leq v_1 &\perp w_1 \geq 0,\\
    0 \leq v_2 &\perp w_2 \geq 0,\\
    0 \leq v_3 &\perp w_3 \geq 0,\\
    0 \leq v_4 &\perp w_4 \geq 0,\\
    0 \leq v_5 &\perp w_5 \geq 0,\\
    0 \leq v_6 &\perp w_6 \geq 0,\\
    0 \leq v_7 &\perp w_7 \geq 0,\\
    0 \leq v_8 &\perp w_8 \geq 0,
\end{align}
which can be equivalently written as:
\begin{align}
    v_1 + w_1 &= |c_x -2l|,\\
    v_2 + w_2 &= |c_y -(4-d_c)l|,\\
    v_3 + w_3 &= |c_y -(3-d_c)l|,\\
    v_4 + w_4 &= |c_x -0.5l|,\\
    v_5 + w_5 &= |c_y + d_cl|,\\
    v_6 + w_6 &= |c_x + 0.5l|,\\
    v_7 + w_7 &= |c_x + 2l|,\\
\end{align}

Then we have the complementarity constraints:
\begin{align}
    0&\leq -\lambda_1 \perp (4-d_c)l - c_y \geq 0,\\
    0&\leq -\lambda_2  \perp v_1 + w_1 + v_2 + w_2 + v_3  + w_3 - l \geq 0 ,\\
    0&\leq \lambda_3  \perp v_1 + w_1 + v_3 + w_3 + v_4 + w_4 - 1.5l \geq 0,\\
    0&\leq -\lambda_4  \perp v_3 + w_3 + v_4 + w_4 + v_5 + w_5 - 3l \geq 0,\\
    0&\leq \lambda_5  \perp v_4 + w_4 + v_5 + w_5 + v_6 + w_6 - l \geq 0,\\
    0&\leq \lambda_6  \perp v_3 + w_3 + v_5 + w_5 + v_6 + w_6 - 3l \geq 0,\\
    0&\leq \lambda_7  \perp v_3 + w_3 + v_6 + w_6 + v_7 + w_7 - 1.5l \geq 0,\\
    0&\leq \lambda_8  \perp v_2 + w_2 + v_3 + w_3 + v_7 + w_7 - l \geq 0.
\end{align}
Furthermore, if we want to avoid simultaneous application of forces on adjacent edges at corners,
we introduce additional complementarity constraints. These constraints ensure that at any given time, only one force can be active:
\begin{align}
    \lambda_i \perp \lambda_j, \forall i,j = 1\ldots 8, \text{and}\,i\neq j.
\end{align}

With the complementarity constraints derived above, we can write the unified dynamics regardless of the contact point.
\begin{align}
\dot{p}_x &= \frac{1}{\mu m g}.\left[(\lambda_2+\lambda_4+\lambda_6+\lambda_8)\cos\theta - (\lambda_1 + \lambda_3+\lambda_5+\lambda_7)\sin\theta\right],\\
\dot{p}_y &= \frac{1}{\mu m g}.\left[(\lambda_2+\lambda_4+\lambda_6+\lambda_8)\sin\theta + (\lambda_1 + \lambda_3+\lambda_5+\lambda_7)\cos\theta\right],\\
\dot{\theta} &= \frac{1}{c r \mu m g}.\left[-c_y(\lambda_2+\lambda_4+\lambda_6+\lambda_8) + c_x(\lambda_1 + \lambda_3+\lambda_5+\lambda_7)\right],    
\end{align}
where $c \in [0,1]$ is the integration constant of the box, and $r$ is the characteristic distance, typically
chosen as the max contact distance.

\subsection{Hopper}\label{app:formulation-hopper}
The hopper exhibits two-phase dynamics, which we will unify through the complementarity constraints to decide whether it is in the fly or stance.

\textbf{Dynamics constraints. }
We define the full states of the hopping robot as:
$$v = \underbrace{[p_x, p_y, q_x, q_y, \theta, r, \dot{p}_x,\dot{p}_y,}_{\text{states}} \underbrace{u_1,u_2}_{\text{control}}].$$ 
$(p_x, p_y)$ and $(q_x, q_y)$ are the positions of the head and tail of the hopper respectively. $r$ is the leg compression distance. $u_1$ and $u_2$ are the controls for leg angular velocity and thrust, each active only in one phase.
Then, the hopper's dynamics can be written as:
\begin{itemize}
    \item Flight Phase:
    \begin{subequations} \label{eq:2dhopper_flight}
        \begin{align}
        \ddot{p}_x &= 0,\label{eq:2dhopper_flight:01}\\
        \ddot{p}_y &=-g,\label{eq:2dhopper_flight:02}\\
        q_x & = p_x + l_0\sin\theta,\label{eq:2dhopper_flight:03}\\
        q_y &= p_y - l_0\cos\theta,\label{eq:2dhopper_flight:04}\\
        \dot{\theta} &=u_1. \label{eq:2dhopper_flight:05}
    \end{align}
    \end{subequations}
    \item Stance Phase:
    
    In the stance phase, $(q_x, q_y)$ is fixed at the contact point, and we can get the following dynamics equations:
    \begin{subequations}\label{eq:2dhopper_stance}
        \begin{align}
        &m\ddot{p}_x + u_2\sin\theta = 0, \label{eq:2dhopper_stance:01}\\
        &m\ddot{p}_y - u_2\cos\theta + mg = 0,\label{eq:2dhopper_stance:02}\\
        &(l_0-r)\cos\theta - p_y + q_y = 0,\label{eq:2dhopper_stance:03}\\
        &(l_0-r)\sin\theta - q_x + p_x = 0.\label{eq:2dhopper_stance:04}\\
         &(l_0-r)^2-(p_x -q_x)^2 - (p_y - q_y)^2 = 0,\label{eq:2dhopper_stance:05}
        \end{align}
    \end{subequations}
    Also, $q_x$ and $q_y$ are fixed by:
    \begin{subequations}\label{eq:2dhopper_stance_02}
        \begin{align}
        &\dot{q}_x = 0,\label{eq:2dhopper_stance_02:01}\\
        &\dot{q}_y = 0.\label{eq:2dhopper_stance_02:02}
    \end{align}
    \end{subequations}
\end{itemize}
Since during the flight phase, $r$ remains 0, while $u_2$ can only be greater than 0 during the stance phase. We can observe that ~\eqref{eq:2dhopper_flight:01}-\eqref{eq:2dhopper_flight:04} are actually the same as~\eqref{eq:2dhopper_stance:01}-\eqref{eq:2dhopper_stance:04}, except that they involve these phase-specific variables.
To unify the dynamics  into a single optimization problem, we construct the following complementarity constraints:
\begin{subequations}\label{eq:2dhopper_comp01}
    \begin{align}
   0 \leq r \perp q_y \geq 0,\label{eq:2dhopper_comp01:01}\\
   0 \leq u_2 \perp q_y \geq 0,\label{eq:2dhopper_comp01:02}\\
    0 \leq u_1^2 \perp r \geq 0. \label{eq:2dhopper_comp01:03}
\end{align}
\end{subequations}
These constraints ensure that the leg maintains its original length during the flight phase while allowing contraction ($r \geq 0$) when in contact with the ground. These conditions are complementary, enabling us to control the switching of dynamics between phases. Specifically, we transform the dynamics of different phases to:
    \begin{align}
        &m\ddot{p}_x + u_2\sin\theta = 0,\\
        &m\ddot{p}_y - u_2\cos\theta + mg  =0,\\
        &(l_0-r)\cos\theta - p_y + q_y=0,\\
        &(l_0-r)\sin\theta - q_x + p_x=0,\\
        &r.\dot{q}_x = 0,\\
        &r.\dot{q}_y = 0,\\
        &q_y.(\theta-u_1)=0,\\
        &r.\left((l_0-r)^2-(p_x -q_x)^2 - (p_y - q_y)^2\right)=0.
    \end{align}
These unified dynamics, together with the complementary constraints~\eqref{eq:2dhopper_comp01:01}-\eqref{eq:2dhopper_comp01:03}, provide a way to write the hopper problem into one optimization problem in the contact-implicit format.
Discretize the dynamics, we get:
    \begin{align}
        &m\frac{\dot{p}_{x,k+1}-\dot{p}_{x,k}}{dt} + u_{2,k}\sin\theta_k = 0,\\
        &m\frac{\dot{p}_{y,k+1}-\dot{p}_{y,k}}{dt} - u_{2,k}\cos\theta_k + mg  =0,\\
        &(l_0-r_k)\cos\theta_k - p_{y,k} + q_{y,k}=0,\\
        &(l_0-r_k)\sin\theta_k - q_{x,k} + p_{x,k}=0,\\
        &r_k.\frac{q_{x,k+1}-q_{x,k}}{dt} = 0,\\
        &r_k.\frac{q_{y,k+1}-q_{y,k}}{dt} = 0,\\
        &r_k.\left((l_0-r_k)^2-(p_{x,k} -q_{x,k})^2 - (p_{y,k} - q_{y,k})^2\right)=0.
    \end{align}

\textbf{Contact constraints. }
We introduce additional constraints to ensure physical consistency and feasibility:
\begin{equation}
    0 \leq r \leq r_0 \label{eq:2dhopper_leg_length}.
\end{equation}
The leg contraction $r$ within a feasible range. The constant $r_0$ represents the maximum allowable contraction and is chosen such that $r_0 < l_0$, where $l_0$ is the original leg length.

These additional constraints, in conjunction with the complementarity constraint, provide a comprehensive formulation that captures the essential physical characteristics of the hopping robot while maintaining the unified representation of both flight and stance phases.

\textbf{Initial constraints. }
The hopper is released from 1.5\,$\textup{m}$ height with zero initial speed and the leg vertical.

\textbf{Cost. }
The cost is a quadratic loss to force the robot to jump to and stop at 2\,$\textup{m}$ with zero height while penalizing the control effort of the angular velocity $u_1$ and thrust $u_2$.

\subsection{Waiter}\label{app:formulation-waiter}
\textbf{Dynamics constraints. }
Denote the full states in the waiter problem:
$$v = \underbrace{[x_1, x_2, \dot{x}_1, \dot{x}_2, v, w,p,q ,}_{\text{states}} \underbrace{\lambda_N, u,f_p,f_t, N}_{\text{control}}].$$
In this scenario, \((x_1, x_2)\) and \((\dot{x}_1, \dot{x}_2)\) represent the position and velocity of the plate and pusher, respectively. The variables \(z, w, p,\) and \(q\) are slack variables used to model the friction between the plate and table, as well as between the pusher and plate, similar to the transport example. For control, we have direct control over the normal force applied by the pusher, denoted as \(\lambda_N\), and the horizontal thrust, \(u\). Additionally, there is indirect control over the frictional forces between the plate and table, \(f_t\), and between the pusher and plate, \(f_p\). The support force exerted by the table is \(N\). The system dynamics are described by the following equations:

\begin{subequations}\label{eq:waiter}
    \begin{align}
    & m_2 \ddot{x}_2 = u - f_p, \\
    & m_1 \ddot{x}_1 = f_p - f_t, \\
    & N + \lambda_N = m_1 g, \label{eq:waiter:staticforcebalance} \\
    & \lambda_N (x_2 - x_1 + l_0) \leq m_1 g l_0, \label{eq:waiter:torquebalance} \\
    & N \geq 0, \ \lambda_N \geq 0, \\
    & x_2 \geq 0, \ x_2 - x_1 \leq l_0. \label{eq:waiter:pusherposlimit}
    \end{align}
\end{subequations}

Here, ~\eqref{eq:waiter:staticforcebalance} ensures vertical force balance. ~\eqref{eq:waiter:torquebalance} prevents tilting around the leftmost contact point between the table and plate. Finally, ~\eqref{eq:waiter:pusherposlimit} restricts the pusher's position to remain clear of the table while staying on the overhanging part of the plate.

\textbf{Contact constraints. }
The following complementarity constraints control the magnitude and direction of the friction $f_t$ and $f_p$ 
\begin{align}
    \dot{x}_1 &= z - w,\\
    \dot{x}_2 - \dot{x}_1 &= p - q,\\
    0\leq z &\perp w \geq 0,\\    
    0\leq p &\perp q \geq 0,\\
    0\leq z &\perp (N\mu_1 - f_t) \geq 0,\\
    0\leq w &\perp (N\mu_1 + f_t) \geq 0,\\
    0\leq p &\perp (\mu_2\lambda_N - f_p) \geq 0,\\
    0\leq q &\perp (\mu_2\lambda_N +f_p) \geq 0.
\end{align}
These equations ensure the friction cone constraints, which automatically manage the magnitude, nature (static or kinetic), and direction of the frictions.

\textbf{Initial constraints.}
In the waiter problem, the initial state is defined with the pusher positioned next to the table, and the 14-meter-long plate has a 1-meter overhang.

\textbf{Cost. }
The objective is to achieve a terminal position and velocity of the plate and pusher, such that the plate is pulled out until its COM aligns with the pusher next to the table, while also minimizing control efforts.