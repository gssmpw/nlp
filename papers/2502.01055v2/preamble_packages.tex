%!TEX root = main.tex

% LC: debug
%==========================================================================
%\usepackage{refcheck}
%\usepackage[notref]{showkeys}
%==========================================================================

% LC: to be used for TRO
%==========================================================================
% \usepackage{mathptmx} % assumes new font selection scheme installed
%\usepackage{times} % assumes new font selection scheme installed
%==========================================================================
\usepackage{comment}
\usepackage{siunitx}
\usepackage{relsize}
\usepackage{ifthen}
\usepackage[colorinlistoftodos]{todonotes}

% \usepackage[caption=false]{subfig}

% \begin{comment}
% % Fancy formatting 
% \usepackage[tracking=false,kerning=true,spacing=true]{microtype}
% \usepackage[caption=false]{subfig}

% \usepackage[style=numeric-comp,sorting=none,firstinits=true, maxnames=3,bibstyle=numeric,abbreviate=false,defernums=true,eprint=false,backend=bibtex]{biblatex}
% \renewcommand{\bibfont}{\footnotesize}
% % \DeclareFieldFormat[article]{pages}{#1}%
% % \DeclareFieldFormat[inproceedings]{pages}{#1}%

% \AtEveryBibitem{%
%   \ifentrytype{article}{%
%     \clearfield{pages}%
%   }{%
%   }%
%   \ifentrytype{inproceedings}{%
%     \clearfield{pages}%
%   }{%
%   }%
% }
% \end{comment}

% \bibliography{commons/bib/strings_long}
% \bibliography{commons/bib/all_only_one_w_pages}
% \bibliography{../../references/references}

% \usepackage[noadjust]{cite}
\usepackage[vlined,ruled,linesnumbered]{algorithm2e}
\usepackage{graphics} % for pdf, bitmapped graphics files
\usepackage{rotating}
\usepackage{color}
\usepackage{enumerate}
\usepackage[T1]{fontenc}
\usepackage{psfrag}
\usepackage{epsfig} % for postscript graphics files
%\usepackage{subfigure}
% \usepackage{hyperref}
\usepackage{booktabs}
\usepackage{graphicx,url}
\usepackage{multirow}
\usepackage{array}
\usepackage{latexsym}
\usepackage{amsfonts}
\usepackage{amsmath}
\usepackage{amssymb}
\usepackage{mathtools}
%\usepackage{amsthm}
\usepackage{xstring}
\usepackage[noend]{algorithmic}
\usepackage{multirow}
\usepackage{xcolor}
\usepackage{prettyref}
\usepackage{flexisym}
\usepackage{bigdelim}
\usepackage{breqn} % load this last
\usepackage{listings}
\let\labelindent\relax
\usepackage{enumitem}
\usepackage{xspace}
\usepackage{bm}
\graphicspath{{./figures/}}
\usepackage{tikz}
\usetikzlibrary{matrix,calc}
\usepackage{tabularx}
\usepackage{subcaption}
\usepackage{float}

\usepackage{pgfplots}
\usepackage{pgfplotstable}
\usepgfplotslibrary{groupplots} % 用于分组图
\pgfplotsset{compat=1.18}       % 保证兼容性


%\usepackage{ifpdf}
% Heiko Oberdiek's ifpdf.sty is very useful if you need conditional
% compilation based on whether the output is pdf or dvi.
% usage:
% \ifpdf
%   % pdf code
% \else
%    dvi code
% \fi
% The latest version of ifpdf.sty can be obtained from:
% http://www.ctan.org/tex-archive/macros/latex/contrib/oberdiek/
% Also, note that IEEEtran.cls V1.7 and later provides a builtin
% \ifCLASSINFOpdf conditional that works the same way.
% When switching from latex to pdflatex and vice-versa, the compiler may
% have to be run twice to clear warning/error messages.

% *** GRAPHICS RELATED PACKAGES ***
%
%\ifCLASSINFOpdf
  % \usepackage[pdftex]{graphicx}
  % declare the path(s) where your graphic files are
  % \graphicspath{{../pdf/}{../jpeg/}}
  % and their extensions so you won't have to specify these with
  % every instance of \includegraphics
  % \DeclareGraphicsExtensions{.pdf,.jpeg,.png}
%\else
  % or other class option (dvipsone, dvipdf, if not using dvips). graphicx
  % will default to the driver specified in the system graphics.cfg if no
  % driver is specified.
  % \usepackage[dvips]{graphicx}
  % declare the path(s) where your graphic files are
  % \graphicspath{{../eps/}}
  % and their extensions so you won't have to specify these with
  % every instance of \includegraphics
  % \DeclareGraphicsExtensions{.eps}
%\fi


\usepackage{mdwlist}
\let\stditemize\itemize
\let\endstditemize\enditemize
\let\itemize\undefined
\makecompactlist{itemize}{stditemize}

%\let\stdenumerate\enumerate
%\let\endstdenumerate\endenumerate
%\let\enumerate\undefined
%\makecompactlist{enumerate}{stdenumerate}
 
