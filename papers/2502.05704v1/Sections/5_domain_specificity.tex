\section{Real-World Data}
\label{sec:domain_spec}

\wc is a new similarity measuring tool that could assist in understanding real-world data and trends. In this section, we focus on two applications of {\wc} -- its ability to serve as a feature extractor and to detect temporal trends in word meaning.   

\subsection{\wc for Feature Classification}
\label{sec:validation_experiments}

\begin{table*}[h!]
    \centering
    \resizebox{\textwidth}{!}{%
    \begin{tabular}{lccccc} \toprule
    \textbf{Experiment} & \textbf{\wc} & \textbf{Cosine 1} & \textbf{Cosine 2} & \textbf{Cosine 3} & \textbf{Ave. Cosine} \\ \midrule
     Sentiment Classification & \textbf{0.79} & 0.75 & 0.71 & 0.84 & 0.73 \\
     Grammatical Gender (Italian) & \textbf{0.93} & 0.80 & 0.80 & 0.71 & 0.77 \\
     Grammatical Gender (French) & 0.85 & \textbf{0.86} & \textbf{0.86} & 0.83 & 0.85 \\
     ConceptNet Domain (Fashion-Gaming) & 0.90 & \textbf{0.93} & \textbf{0.93} & 0.90 & 0.92 \\
     ConceptNet Domain (Sea-Land Animals) & \textbf{0.83} & 0.79 & 0.80 & 0.61 & 0.73 \\
     \midrule
     Average & \textbf{0.86} & 0.83 & 0.82 & 0.76 & 0.80 \\ \bottomrule
    \end{tabular}
    }
    \caption{Macro-F1 for \wc and cosine similarity across a variety of feature classification tasks. We operationalize cosine similarity in three ways: 1) the distance between the centroids of the seed words and the target words 2) the average distance each of the target word to the centroid of the seed words 3) the average distance of each target word to each seed word (no centroids).}
    \label{tab:results}
\end{table*}


\wc can be used to define out-of-domain word classes, i.e. when $w_j \not\in W$. Using our earlier example, if the classes of \wc are the features \{\textit{positive}, \textit{negative}\}, given an out-of-domain word like \textit{school}, we can use the confusion matrix to represent the embedding for \textit{school} as a mixture of the classes the model is familiar with, i.e., \{\textit{positive}, \textit{negative}\}.

Following this intuition, we test whether \wc can use features as classes to identify objects' membership to these classes accurately. We used the following tasks: 

\noindent \textbf{Sentiment classification} using the NRC corpus \cite{pang-etal-2002-thumbs, mohammad-etal-2013-nrc}. The goal is to classify words according to their sentiment (either positive or negative). The words were manually annotated based on their emotional association (e.g., ``trophy'' is positive, whereas ``flu'' is negative).

\noindent \textbf{Grammatical gender} classification of nouns \cite{sahai-sharma-2021-predicting}. We tested \wc using two languages -- Italian and French. The goal is to classify words according to their grammatical gender per language. For example, ``flower'' is feminine in French and masculine in Italian. 

\noindent \textbf{Domain classification} using ConceptNet categories \cite{dalvi2022discovering}. The goal is to classify words to their correct ConceptNet class. We used two domain pairs: Fashion-Gaming is about classifying whether a word belongs to the fashion domain or the gaming domain; in Sea-Land, the goal is to predict if an animal is a sea or land animal.

For each task, we hand-select meaningful words as classes for the classifier and use terms from the lexicon as test embeddings. For example, for sentiment classification we first use the seed words \textit{positive} and \textit{negative} as our classes and collect occurrences from a corpus, extract the embeddings train the concept prober to recognize \textit{positive} and \textit{negative}. Finally, we then use \wc to classify all the terms in the NRC lexicon (our target words). We define the label using the class with the highest probability for the word. Details of each experiment are available in in the Appendix \ref{sec:seeds}.

Across all three tasks, we find that \wc is successful in feature-based classification using a few seed word training examples. Compared to cosine similarity, we achieve a macro-F1 of 86\% compared to 80\% (see table \ref{tab:results}). 

\xhdr{Embedding Meaning vs. Properties}
It is important to distinguish between embedding a word for its overall meaning (e.g., whether it conveys a positive or negative sentiment) versus embedding it to capture a specific property (e.g., gender, formality). While \wc supports both, this distinction is crucial when interpreting the results and determining the appropriate transformations for different tasks.


\subsection{What Is A Revolution?}

We now examine whether \wc could be used as a tool for studying concepts in a way that aids humanistic or social science investigations. By collaborating with the fourth author, a scholar of French literature and history, we investigate historical changes in the meaning of the French word \textit{``révolution''}. Together, we used \wc to test a prominent hypothesis of how the meaning of the word and concept of revolution changed \cite{baker_1990}: that the meaning of \textit{``révolution''} in the early years of the French Revolution was more associated with \textit{popular} action, but later become identified with \textit{state} actions.

We constructed a set of French words associated with the people (\{\textit{peuple}, \textit{populaire}, ...\}) and the state (\{\textit{conseil}, \textit{gouvernement}, ...\}).\footnote{Note on choice of seed words: we are tracking changes in the meaning of \textit{``révolution''} between 1789 and 1793 thus only looking at the vocabulary used during the French Revolution. Although the connection between \textit{``peuple''} and \textit{``révolution''} could be found before July 14, 1789, it is in the aftermath of that date that this connection became the primary one. Prior to this, the delegates of the National Assembly in Versailles had claimed they had been leading the \textit{``révolution''}.} These seed words were used as classes for our classifier, which we trained on different temporal segments to observe the temporal change in our concept of interest. Our corpora are the \textit{Archives Parlementaires}, transcripts of parliamentary speeches during a time that contains moments of both emancipation and elite control of political processes.\footnote{\url{https://sul-philologic.stanford.edu/philologic/archparl/}} The corpus contains 9,628 speeches and 54,460,150 words from the years 1789-1793. Within this corpus, the term \textit{``révolution''} appears 2,206 times across 218 speeches, with a contextual basis of 90,138 words.

\begin{figure}[h!]
    \includegraphics[width=.45\textwidth]{Images/pca_result.png}
    \caption{In 1789, the word \textit{``revolution''} was primarily associated with popular action (represented in orange). In 1792 \textit{``revolution''} was now also seen as something that the government should lead (represented in blue) found in the \textit{``counter-revolution''} cluster. In 1793, this new governmental meaning had spread back to the word \textit{``revolution''} itself.
    % \cnote{TODO: improve visualization. Some text is not readable. Add legend}
    }
    \label{fig:revolution}
\end{figure}


We color-code the classes orange as \textit{``peuple'' } (\textit{``the people''}) and blue as \textit{``gouvernement''} (\textit{``the state''}) and project the embeddings down to a 2-dimensional space and visualize the results (Figure \ref{fig:revolution}).

We find that, in 1789, the word \textit{``révolution''} was primarily associated with popular action, the most famous example of which was the storming of the Bastille. In 1792, another definition became common: \textit{``révolution''} was now also seen as something that the government should lead. 

Interestingly, the first use of \textit{``révolution''} to be associated with governmental action is in fact around the term \textit{``counter-revolution''}. The word embeddings of \textit{``counter-révolution''} are predicted to be associated with the state, indicating that it was primarily when talking about threats to, and enemies of, the revolution, that politicians suggested transferring more power to the state. Jumping forward to 1793, this new governmental meaning is predicted for the word embeddings of \textit{``révolution''} itself. Our findings suggest that the goal of repressing counter-revolutionaries is what associated the term \textit{``révolution''} with governmental action. In other words, once revolutionaries became more concerned about tracking down their enemies, they granted to the government the same kind of extra-legal power that had originally only been the prerogative of the people in arms.\footnote{While the proclamation of the republic and the introduction of the new calendar are related to the idea of revolution, the conceptual shifts that we've identified appears prior to both of these events. The revolutionary calendar was not introduced until October 1793, meanwhile declaration of the Republic occurred in September 1792. The emergence of \textit{``counterrevolution''} as a state problem predates these events, confirming that neither play a role in introducing the newer understanding of ``revolution'' as a state-driven process.}

Our findings are consistent with historians' hypothesis that the meaning of revolution in the early years of the French Revolution is most closely aligned with the concept of the people and this gradually shifts as the revolution continues \cite{sewell2005logics,edelstein2012we}. Furthermore, our model allows us to uncover a potential causal story for this shift in the meaning; that the state sense of revolution first actually started with counter-revolution. This is a novel discovery in our understanding of the French Revolution; future humanistic work should use other methods to confirm this proposed causal link to counter-revolutionaries.

 See  Appendix~\ref{sec:finance_experiment} for another, more preliminary  social scientific application of \wc, in this case to study financial trends.  