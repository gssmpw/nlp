% This must be in the first 5 lines to tell arXiv to use pdfLaTeX, which is strongly recommended.
\pdfoutput=1
% In particular, the hyperref package requires pdfLaTeX in order to break URLs across lines.

\documentclass[11pt]{article}

% Remove the "review" option to generate the final version.
\usepackage[]{acl}
\usepackage{enumitem}

% Standard package includes
\usepackage{contour}
\usepackage{xcolor}
\usepackage{times}
\usepackage{latexsym}
\usepackage{graphicx}
\usepackage{subcaption}
\usepackage{mwe}
\usepackage{xcolor}
\usepackage{booktabs}
\usepackage{amsmath,amsfonts,amssymb,amsthm}
\usepackage{multicol}\usepackage{xcolor}         % colors
\usepackage{diagbox}
\contourlength{0.8pt}
\usepackage{makecell}

\newcommand{\orangegloss}[1]{\contour{orange}{\textcolor{white}{#1}}}
\newcommand{\bluegloss}[1]{\contour{blue}{\textcolor{black}{#1}}}
\newcommand{\xhdr}[1]{\vspace{1mm}\noindent{{\bf #1.}}}

\newcommand{\cnote}[1]{\textcolor{red}{$\ll$\textsf{#1 | Chen}$\gg$}}
\newcommand{\snote}[1]{\textcolor{blue}{$\ll$\textsf{#1 | Sarah}$\gg$}}
\newcommand{\fnote}[1]{\textcolor{burgundy}{$\ll$\textsf{#1 | Fede}$\gg$}}

\definecolor{amaranth}{HTML}{E52B50}
\definecolor{garnet}{HTML}{733635}
\definecolor{burgundy}{HTML}{800020}
\definecolor{midnightgreenn}{rgb}{0.0, 0.29, 0.33}
\definecolor{navyblue}{rgb}{0.0, 0.0, 0.5}

\usepackage[T1]{fontenc}
\usepackage[utf8]{inputenc} % The default since 2018
\usepackage[english]{babel}
\newcommand\dedicationfont{\fontfamily{qcr}\itshape\mdseries\selectfont}
\newcommand{\wc}{{\dedicationfont \textcolor{navyblue}{Word Confusion}}\xspace}
\newcommand{\remove}[1]{}

% For proper rendering and hyphenation of words containing Latin characters (including in bib files)
\usepackage[T1]{fontenc}
% For Vietnamese characters
% \usepackage[T5]{fontenc}
% See https://www.latex-project.org/help/documentation/encguide.pdf for other character sets

% This assumes your files are encoded as UTF8
\usepackage[utf8]{inputenc}

% This is not strictly necessary, and may be commented out.
% However, it will improve the layout of the manuscript,
% and will typically save some space.
\usepackage{microtype}

% This is also not strictly necessary, and may be commented out.
% However, it will improve the aesthetics of text in
% the typewriter font.
\usepackage{inconsolata}
\usepackage{booktabs}
\setlength{\belowcaptionskip}{-10pt}
% If the title and author information does not fit in the area allocated, uncomment the following
%
%\setlength\titlebox{<dim>}
%
% and set <dim> to something 5cm or larger.



\title{Rethinking Word Similarity:\\Semantic Similarity through Classification Confusion}


% Author information can be set in various styles:
% For several authors from the same institution:
% \author{Author 1 \and ... \and Author n \\
%         Address line \\ ... \\ Address line}
% if the names do not fit well on one line use
%         Author 1 \\ {\bf Author 2} \\ ... \\ {\bf Author n} \\
% For authors from different institutions:
% \author{Author 1 \\ Address line \\  ... \\ Address line
%         \And  ... \And
%         Author n \\ Address line \\ ... \\ Address line}
% To start a seperate ``row'' of authors use \AND, as in
% \author{Author 1 \\ Address line \\  ... \\ Address line
%         \AND
%         Author 2 \\ Address line \\ ... \\ Address line \And
%         Author 3 \\ Address line \\ ... \\ Address line}

\author{
  Kaitlyn Zhou, 
  Haishan Gao, 
  Sarah Chen, 
  Dan Edelstein,   
  Dan Jurafsky, 
  Chen Shani\\
  % Kaitlyn Zhou, \hspace{3pt}
  % Haishan Gao, \hspace{3pt}
  % Sarah Chen,\\
  % % \textbf{Federico Bianchi, \hspace{3pt}}
  % \textbf{Dan Edelstein, \hspace{3pt}}  
  % \textbf{Dan Jurafsky, \hspace{3pt}}
  % \textbf{Chen Shani}\\
  Stanford University\\
\texttt{\{katezhou, hsgao, sachen, danedels, jurafsky, cshani\}@stanford.edu} \\  
}

\begin{document}
\maketitle

\begin{abstract}
\begin{abstract}

% Recent works to jointly reconstruct 3D human and object from a single RGB image, are mostly model-based, that fail to capture the fine details of the clothed human body and object surface. In this paper, we introduce ReCHOR, a novel, model-free, first-method to produce realistic clothed human-object reconstructions from a monocular view. This is extremely challenging due to human-object occlusions, diverse interactions and depth ambiguity, as it needs to infer both 3D spatial awareness and high resolution details. Our core idea is based on estimating neural implicit representations for human and object respectively by an attention-based neural implicit model that attends to pixel-aligned features from both the global human-object image for spatial awareness and  the local separate view of human and object images for high quality details. Additionally, the network is conditioned on semantic features from an initial estimated human-object pose prior and a generative diffusion model that inpaints occluded regions, thus enabling the retrieval of details from them.
% We also propose a synthetic dataset with rendered scenes of diverse, inter-occluded 3D human and object scans, to train our network. We evaluate our method on the synthetic and real world BEHAVE dataset. Our experiments show that our method outperforms the SOTA in achieving realistic clothed human-object reconstructions.
Recent approaches to jointly reconstruct 3D humans and objects from a single RGB image represent 3D shapes with template-based or coarse models, which fail to capture details of loose clothing on human bodies. In this paper, we introduce a novel implicit approach for jointly reconstructing realistic 3D clothed humans and objects from a monocular view. For the first time, we model both the human and the object with an implicit representation, allowing to capture more realistic details such as clothing. This task is extremely challenging due to human-object occlusions and the lack of 3D information in 2D images, often leading to poor detail reconstruction and depth ambiguity. To address these problems, we propose a novel attention-based neural implicit model that leverages image pixel alignment from both the input human-object image for a global understanding of the human-object scene and from local separate views of the human and object images to improve realism with, for example, clothing details. Additionally, the network is conditioned on semantic features derived from an estimated human-object pose prior, which provides 3D spatial information about the shared space of humans and objects. To handle human occlusion caused by objects, we use a generative diffusion model that inpaints the occluded regions, recovering otherwise lost details. For training and evaluation, we introduce a synthetic dataset featuring rendered scenes of inter-occluded 3D human scans and diverse objects. Extensive evaluation on both synthetic and real-world datasets demonstrates the superior quality of the proposed human-object reconstructions over competitive methods.
\end{abstract}
\end{abstract}
\section{Introduction}\label{sec:intro}

In computational finance, Monte Carlo simulations are used extensively to estimate the expected value of financial payoffs based on the solution of stochastic differential equations (SDEs) which model the evolution of stock prices, interest rates, exchange rates and other quantities \cite{glasserman04}.  Monte Carlo methods are very general and flexible, but for high accuracy it requires generating a large number of costly SDE path approximations, which has motivated research into a number of variance reduction or, equivalently, cost reduction techniques. One such method is
Multilevel Monte Carlo (MLMC), which was proposed in \cite{GILES2008} and was adapted for various applications that are summarised in \cite{Giles_overview17} and successfully combined with other methods such as quasi-Monte Carlo methods. The main idea of MLMC is to approximate the payoff using different time stepping resolutions when numerically solving the underlying SDE and to generate an optimal number of samples on each level, such that the overall computational cost is minimised subject to the desired bound on the variance. %, such that the total computational cost is minimised. 
The computational savings come from the fact that most samples are computed on the coarser levels and hence are less expensive while only a few samples from the finest levels are required \cite{GILES2008}.


Among the directions in which the computational cost 
of MLMC methods could further be reduced, an important avenue is the use of lower precision calculations, especially for the first Monte Carlo levels where the targeted accuracy is relatively low. 
 An overview of the research on mixed precision for the standard Monte Carlo (MC) framework is provided in \cite{ChowMixedPrecisionStandardMC} but only a few references study the potential of low precision computation in the MLMC framework \cite{Rounding_error_oliver}. To the best of our knowledge, the only MLMC framework with customised precision in the literature is \cite{brugger2014mixed}, but they use a uniform precision for all operations on each Monte Carlo level instead of optimising 
 the precision of each intermediary variable to reduce as much as possible the cost of path generation.
 
An important motivation for an MLMC framework with variable precision would be performing the low precision computations on reconfigurable hardware devices such as Field Programmable Gate Arrays (FPGAs). FPGAs contain customizable logic blocks and connectors that make it easy to adapt the digital circuit architecture for a specific application, leading to a highly parallel and optimised implementation. Therefore they are successfully exploited in applications that require high speed and have high computational workload, such as signal processing \cite{woods2008fpga}, and real time applications like high frequency trading \cite{HFT1,HFT2}. That is why a number of previous works in hardware architecture design implemented the MLMC algorithm to price financial options using FPGAs as accelerators, which resulted in improved speed and power efficiency compared to full CPU architectures \cite{Schryver2013AMM}. The paper \cite{lindsey2016domain} also proposed 
a Domain Specific Language to automate the configuration of FPGAs for this specific application. However, only \cite{brugger2014mixed} proposed a heuristic to reduce the precision in calculations.

In addition, all aforementioned works considered that the random number generation (RNG) is performed in single or double precision. Yet in most cases an important portion of the workload in the overall MLMC simulation comes from the RNG and in \cite{brugger2014mixed} this limited the total computational savings.
To reduce the cost of MLMC simulations in particular those based on the Geometric Brownian Motion (GBM), \cite{approximateICDF_Oliver, NestedOliver} have proposed to use approximate random numbers that are generated by applying an approximation of the inverse CDF to uniform random numbers. In \cite{NestedOliver}, the authors proposed a way to integrate these lower precision random variables into a \textit{nested} MLMC framework and completed a numerical analysis to bound the resulting error at each MC level by a product of the time step and the error in the random number approximation. The same authors show in \cite{approximateICDF_Oliver} that using approximate random variables reduces the cost of path generation by a factor 7.


In this paper we propose a nested MLMC framework that combines the use of approximate random normal variables and lower precision calculations to reduce the computational cost of MLMC even further than \cite{brugger2014mixed,NestedOliver}. We illustrate the efficiency of our framework in Matlab, after making several assumptions on the cost of operations and size of the errors that we carefully justify. We focus on the case of GBM and use the approximate RNG methods presented in \cite{approximateICDF_Oliver} as well as a new slightly modified method that combines CDF inversion and the central limit theorem. To choose the precision of the variables in the low precision path generation, we introduce a novel method to optimise the bit-widths. This optimisation is performed before the main path generation loop is executed and is based on a linear model of the payoff error  
due to rounding when computing in low precision. The error model relies on algorithmic differentiation in a similar manner to \cite{unifying-bwoptim,bitwidth-AD,ADAPT}. The bit-width optimisation procedure can be performed off-line, so this stage can be excluded from the on-line time complexity of our framework. The user specified desired accuracy is then enforced by calculating on-line the number of samples that need to be generated.

In terms of hardware design, we suggest implementing the low precision path generation on FPGAs and the full-precision ones on a CPU or GPU. 
The FPGA offers enough flexibility to define a separate bit-width for every variable in the low precision path generation, and can be reconfigured periodically to update the bit-widths when the market parameters have changed considerably. 


The paper is organized as follows : \Cref{sec:MLMC} introduces MLMC and nested MLMC to make clear the estimator that is implemented in our framework. Then in \Cref{sec:RNG} we detail the methods that could be used to obtain approximate random normally distributed numbers very cheaply for the low precision path generation. In \Cref{sec:error_model} and \Cref{sec:costModel} we propose an error model and a cost model (resp.) that we then use to formulate the optimisation problem that is solved to obtain the optimal bit-widths of fixed point variables in \Cref{sec:optimisation}. Finally we summarise our results and future directions in \Cref{sec:conclusion}.



\section{Introducing \wc}
\label{sec:identity_probe}


Our method begins by defining a set of words, or features. For example, we might choose the set $W=\{red,\ green,\ blue\}$ if we wanted to study similarity related to colors. These words will act as features that can be selected by the analyst to focus on a particular dimension or question.


Our process then has two phases: training and inference. In training (illustrated in part (a) of Figure \ref{fig:word_confusion})) we extract from a corpus a set of sentences containing each of these words, such as \textit{``The sunset painted the sky a brilliant shade of red''} for the word ``red''. We then use BERT to extract the contextual embeddings of these feature-words, and train a classifier to map from an embedding to its corresponding word identity. The classifier's training objective is to correctly classify the embedding to the word that corresponds to it. 

More formally, given a chosen set of word $W$ and embeddings $\{e_1, e_2 .... e_i\} \in E$ that correspond to word identities $\{w_1, w_2, ..., w_i\} \in W$, we train a logistic regression classifier on all pairs of $\{e_i, w_i\}$.

At inference (part (b) in Figure \ref{fig:word_confusion}), we wish to define the semantic similarity of a word in terms of the classifier's classes, which can be thought of as features.\footnote{Thus the choice of a different set of classes is a way of selecting different features  to describe the input word.} Now suppose we would like to compute the similarity of the new word ``burgundy'' to various colors. We extract the contextual embedding of  ``burgundy'' given the sentence \textit{``Burgundy is a deep reddish-brown shade inspired by wine''}, and  use the trained classifier to compute the probability that the \textit{``burgundy''}-embedding corresponds to each class $W=\{red,\ green,\ blue\}$. We then use the classifier's confusion matrix to understand which primary colors burgundy is similar to.  For example, the similarity of ``burgundy'' to  ``red'' is the probability our classifier assigns to the class \textit{``red''}.

More formally, the probability distribution predicted by the model, $\vec{p_j} \in \mathbb{R}^{|W|}$, is used to quantify the semantic similarity between $w_j$ (Burgundy) and each $w_i, \forall w_i \in W$ (in this case $W=$\{{red},\ green,\ blue\}).   Thus:
\begin{eqnarray}
\text{sim}_{\text{WC}}(w_i,w_j) \stackrel{\mathrm{def}}{=}
 p(w_i|e_j)
\end{eqnarray}
The set of distractor words chosen to train the initial classifier thus act as features that can be selected by the analyst to focus on a particular dimension or question.
 
Note that as with the example ``burgundy",  the input word at inference can be out-of-vocabulary with respect to the classifier, or the target word can be one of the classifier's classes (in which case we ignore the probability it assigns to that word and use the other $N-1$ features.)

 


\label{sec:cosine_vs_probe}
\begin{figure*}[h!]
    \centering
    \includegraphics[width=0.9\textwidth]{Images/decision_surface.png}
    \caption{Differences in decision boundaries between \wc and cosine similarity. The $x$ and $y$ axes represent two dimensions of an artificially constructed set of data points. Note how cosine similarity's boundaries originate from the origin whereas {\wc}'s are not limited in the same way.}
    \label{fig_concept_prob_boundaries}
\end{figure*}

\subsection{Benchmarking \wc}
\label{sec:initial_eval}

The intuition behind \wc is that if it struggles to distinguish between contextual embeddings of \textit{burgundy} and \textit{red}, this could indicate they are similar. To test this hypothesis, we use \wc on three semantic similarity benchmarks. For each task, we trained a model using sentences from English Wikipedia. Our classes contained all the words from the benchmark. We then built word embeddings by averaging the last four hidden layers of BERT-base-cased (Details in Appendix~\ref{appendix:validation_details}). 

To calculate the similarity between two words $w_i, w_j$, we first extract all the sentences containing $w_i$ from English Wikipedia. We average the contextual token embeddings of $w_i$ using these sentences. This average token embedding was the input to the trained classifier (with classes containing all the words in the benchmark). We then use the probability \wc assigned to $w_i$  to set the similarity score between $w_i$ and $w_j$. We tested three benchmarks:

\begin{itemize}
\itemsep 0pt
    \item \textbf{MEN} contains 3000-word pairs annotated by 50 humans  based on their ``relatedness'' \cite{agirre-etal-2009-study}. For example \{berry, seed\}, \{game, hockey\}, and \{truck, vehicle\} received high relatedness scores, where \{hot, zombi\}, \{interior, mushroom\}, and \{bakery, zebra\} received low scores. 
To approximate human agreement, two annotators labeled all 3000 pairs on a 1-7 Likert scale; their Spearman correlation is 0.68, and the correlation of their average ratings with the general MEN scores is 0.84. 
%This high correlation suggests that MEN contains meaningful semantic ratings.
\item \textbf{WordSim353 (WS353)} contains 2000 word-pairs along with human-assigned association judgements \cite{bruni2014multimodal}. For example \{bank, money\}, \{Jerusalem, Israel\}, and \{Maradona, football\} received high scores whereas \{noon, string\}, \{sugar, approach\}, and \{professor, cucumber\} were ranked low. 
The authors report an inter-annotator agreement of 84\%. 

\item \textbf{SimLex} contains 1000 word-pairs and directly measures similarity, rather than relatedness or association \cite{hill-etal-2015-simlex}. The authors defined similarity as synonymy and instructed their annotators to rank accordingly. For example \{happy, glad\}, \{fee, payment\}, and \{wisdom, intelligence\} received high relatedness scores, where \{door, floor\}, \{trick, size\}, and \{old, new\} received low scores. 
Inter-rater agreement (the average of pairwise Spearman correlations between the ratings of all respondents) was reported as 0.67.

\end{itemize}

\begin{table}[]
    \centering
    \begin{tabular}{l|ccc} \toprule
         \diagbox[dir=NW]{{Method}}{{Dataset}} & {MEN} & {WS353} & {SimLex}\\ \midrule
         Cosine & {0.59} & 0.54 & 0.39 \\
         \makecell[l]{{\dedicationfont \textcolor{navyblue}{Word}}\\{{\dedicationfont \textcolor{navyblue}Confusion}}} & \textbf{0.66} &\textbf{0.67} & \textbf{0.44} \\ \bottomrule
    \end{tabular}
    \caption{Spearman's $\rho$ correlation between \wc and cosine similarity results as compared to humans. These three benchmarks focus on slightly different aspects of word similarity. We measure the correlation between human scores and cosine similarity between the language model embeddings versus {\wc}'s similarity scores. As can be seen, our method slightly outperforms cosine similarity.}
    \label{tab:semantic_similarity_results}
\end{table}

Across MEN, WS353, and SimLex, \wc slightly outperforms cosine similarity. This illustrates the meaningfulness of classification confusions, compared to cosine similarity. We note that our probability distribution spanned only the classes we chose in advance (all of the words in the dataset), which yields a different vocabulary compared to the original language model.



\section{Theoretical Intuitions}

In this section, we discuss the importance of word identifiability and how it enables the core mechanics of \wc, and discuss some theoretical differences between \wc and cosine similarity.  

\subsection{The Identifiability of Contextualized Word Embeddings}

\wc depends on the ability of a classifier to identify a word based on its contextual embedding; here we confirm that this classification task is indeed solvable, and examine some error cases to better understand it.

While contextualized word embeddings vary in their representation based on context, prior work showed that tokens of the same word still cluster together in geometric space \cite{zhou-etal-2022-problems}.

To test whether these boundaries are indeed learnable, we test how well a model can identify a contextualized word embedding after seeing one other example of the same word's contextualized embedding. We randomly sampled 26,000 words from English Wikipedia, trained 1000-class one-shot classifiers, and tested them on 10,000 examples (ten examples per class). Indeed, we found that the average test set accuracy on all our classifiers is 90\%, suggesting that the contextualized word embeddings are highly \textit{identifiable}. Thus, given an embedding, it is possible to identify its symbolic representation. See \ref{section:appendix_identity_probe} for experimental details.

\subsection{Differences Between \wc and Cosine Similarity}

\wc and cosine similarity give different kinds of distances.
We can see one way to visualize this in Figure~\ref{fig_concept_prob_boundaries}. Note the differences in the decision surface between {\wc} and cosine similarity: cosine boundaries emerge from the origin, whereas boundaries from \wc are not restricted in the same way.

Using a linear classifier in \wc also introduces new parameters that transform the input vectors into a different space, effectively redefining the notion of distance compared to the raw embeddings. To see this, consider two normalized 2-dimensional vectors, $x$ and $y$, and a real linear transformation, $A$ applied to each. Using the singular value decomposition (SVD) of $A= U \Sigma {V}^\intercal$, the singular values of $A$ (${\sigma}{u}{{v}^\intercal}$) allow us to rewrite the transformed vectors $Ax, Ay$ as ${\sigma_1}{u_1}{{v_1}^\intercal}{x_1} + {\sigma_2}{u_2}{{v_2}^\intercal}{x_2}$ and  ${\sigma_1}{u_1}{{v_1}^\intercal}{y_1} + {\sigma_2}{u_2}{{v_2}^\intercal}{y_2}$ respectively. 

The cosine distance between the transformed vectors is $1-({\sigma_1}^2({v_1}^{\intercal}x_1)({v_1}^{\intercal}y_1) + {\sigma_2}^2({v_2}^{\intercal}{x_2})({v_2}^{\intercal}{y_2}))$ compared to the original cosine distance $1 - (x_1y_1 + x_2y_2)$.\footnote{Terms cancel out as ${\sigma_1}{\sigma_1}=1$ and ${\sigma_1}{\sigma_2}=0$.} Similarly, the Euclidean distance between the transformed vectors is ${\sigma_1}{u_1v_1}^\intercal({x_1-y_1}) + {\sigma_2}{u_2v_2}^\intercal({x_2-y_2})$ compared to the original Euclidean distance of $(x_1-y_1)^2 + (x_2-y_2)^2$. In both cases, the distances between the two transformed vectors differ from the original vectors based on the linear transformation applied.

In other words, a linear transformation introduces additional parameters, allowing the model to reshape the geometry of word vectors and adjust the distances between words and their predicted semantic similarities.

Our method also shares some properties with cosine similarity.  Because linear classifiers learn a weight vector for each category that represents a kind of prototype of the category, the weight vectors learned by our classifier will be approximations of each word vector itself. Like cosine, our method thus computes similarity as the dot product between the word vector input and an errorful representation of the word vector encoded in the weights of the final classifier. What makes this approach effective is its reliance on small yet informative prediction errors that encode a meaningful signal, making the confusion matrix a source of linguistic insight.


% will geometry of the embeddings, giving a model additional parameters to represent targeted words and 
% we discuss differences between \wc and cosine similarity, arguing that feature-based similarity can produce more flexible decision boundaries, capture asymmetrical relations, highlight specific aspects of the analyzed word, and output more meaningful scores. 



\subsection{Advantageous Properties of \wc as a Similarity Measure}

Using a trainable linear classifier and analyzing its error signal for word-similarity purposes introduces a few advantages for measuring similarity:

\xhdr{Asymmetry} Human perceived similarity is not symmetric \cite{tversky1977features}. Yet cosine, like many distance functions commonly used to calculate semantic similarity, is symmetric. One of the advantages of using a model's confusion matrix for measuring semantic similarity is that these scores are \textit{asymmetric}; i.e., $p_{ij} \not= {p_{ji}}$. For example, \wc assigns lower probabilities for \textit{animal} being predicted as \textit{cat} than for \textit{cat} being predicted as \textit{animal}. The ability to measure asymmetric semantic similarity opens interesting new directions of understanding semantic similarity which are not possible with cosine.

\xhdr{Domain Adaptability} The fact that {\wc} requires training leads to more flexible similarity measures. Class selection enables measuring the semantic similarity of words relative to just a \textbf{subset} of features; we propose that this is particularly useful for practitioners who are interested in computing the similarity of words within a niche domain (we explore this in section \ref{sec:domain_spec}). 
% One limitation of our method is that it is self-supervised and would need to be retrained for out-of-domain words, which should still be quick given it is training a logistic regression 

\xhdr{Interpretability}
Probabilistic similarity measures have the advantage of being more interpretable for humans than non-probabilistic measures like cosine  \cite{sohangir2017improved}. Using a classifier's confusion matrix gives similarity scores that represent real probabilities. Moreover, since the choice of classifier classes is an implementation decision, one could choose them based on desired aspects of a word for a task. For example, we could interpret attitudes toward school by asking for the confusion matrix for the word ``school'' with a sentiment analysis classifier that contains the classes \{\textit{negative}, \textit{positive}\}, or the classes \{\textit{fun}, \textit{work}\}. 
\section{Real-World Data}
\label{sec:domain_spec}

\wc is a new similarity measuring tool that could assist in understanding real-world data and trends. In this section, we focus on two applications of {\wc} -- its ability to serve as a feature extractor and to detect temporal trends in word meaning.   

\subsection{\wc for Feature Classification}
\label{sec:validation_experiments}

\begin{table*}[h!]
    \centering
    \resizebox{\textwidth}{!}{%
    \begin{tabular}{lccccc} \toprule
    \textbf{Experiment} & \textbf{\wc} & \textbf{Cosine 1} & \textbf{Cosine 2} & \textbf{Cosine 3} & \textbf{Ave. Cosine} \\ \midrule
     Sentiment Classification & \textbf{0.79} & 0.75 & 0.71 & 0.84 & 0.73 \\
     Grammatical Gender (Italian) & \textbf{0.93} & 0.80 & 0.80 & 0.71 & 0.77 \\
     Grammatical Gender (French) & 0.85 & \textbf{0.86} & \textbf{0.86} & 0.83 & 0.85 \\
     ConceptNet Domain (Fashion-Gaming) & 0.90 & \textbf{0.93} & \textbf{0.93} & 0.90 & 0.92 \\
     ConceptNet Domain (Sea-Land Animals) & \textbf{0.83} & 0.79 & 0.80 & 0.61 & 0.73 \\
     \midrule
     Average & \textbf{0.86} & 0.83 & 0.82 & 0.76 & 0.80 \\ \bottomrule
    \end{tabular}
    }
    \caption{Macro-F1 for \wc and cosine similarity across a variety of feature classification tasks. We operationalize cosine similarity in three ways: 1) the distance between the centroids of the seed words and the target words 2) the average distance each of the target word to the centroid of the seed words 3) the average distance of each target word to each seed word (no centroids).}
    \label{tab:results}
\end{table*}


\wc can be used to define out-of-domain word classes, i.e. when $w_j \not\in W$. Using our earlier example, if the classes of \wc are the features \{\textit{positive}, \textit{negative}\}, given an out-of-domain word like \textit{school}, we can use the confusion matrix to represent the embedding for \textit{school} as a mixture of the classes the model is familiar with, i.e., \{\textit{positive}, \textit{negative}\}.

Following this intuition, we test whether \wc can use features as classes to identify objects' membership to these classes accurately. We used the following tasks: 

\noindent \textbf{Sentiment classification} using the NRC corpus \cite{pang-etal-2002-thumbs, mohammad-etal-2013-nrc}. The goal is to classify words according to their sentiment (either positive or negative). The words were manually annotated based on their emotional association (e.g., ``trophy'' is positive, whereas ``flu'' is negative).

\noindent \textbf{Grammatical gender} classification of nouns \cite{sahai-sharma-2021-predicting}. We tested \wc using two languages -- Italian and French. The goal is to classify words according to their grammatical gender per language. For example, ``flower'' is feminine in French and masculine in Italian. 

\noindent \textbf{Domain classification} using ConceptNet categories \cite{dalvi2022discovering}. The goal is to classify words to their correct ConceptNet class. We used two domain pairs: Fashion-Gaming is about classifying whether a word belongs to the fashion domain or the gaming domain; in Sea-Land, the goal is to predict if an animal is a sea or land animal.

For each task, we hand-select meaningful words as classes for the classifier and use terms from the lexicon as test embeddings. For example, for sentiment classification we first use the seed words \textit{positive} and \textit{negative} as our classes and collect occurrences from a corpus, extract the embeddings train the concept prober to recognize \textit{positive} and \textit{negative}. Finally, we then use \wc to classify all the terms in the NRC lexicon (our target words). We define the label using the class with the highest probability for the word. Details of each experiment are available in in the Appendix \ref{sec:seeds}.

Across all three tasks, we find that \wc is successful in feature-based classification using a few seed word training examples. Compared to cosine similarity, we achieve a macro-F1 of 86\% compared to 80\% (see table \ref{tab:results}). 

\xhdr{Embedding Meaning vs. Properties}
It is important to distinguish between embedding a word for its overall meaning (e.g., whether it conveys a positive or negative sentiment) versus embedding it to capture a specific property (e.g., gender, formality). While \wc supports both, this distinction is crucial when interpreting the results and determining the appropriate transformations for different tasks.


\subsection{What Is A Revolution?}

We now examine whether \wc could be used as a tool for studying concepts in a way that aids humanistic or social science investigations. By collaborating with the fourth author, a scholar of French literature and history, we investigate historical changes in the meaning of the French word \textit{``révolution''}. Together, we used \wc to test a prominent hypothesis of how the meaning of the word and concept of revolution changed \cite{baker_1990}: that the meaning of \textit{``révolution''} in the early years of the French Revolution was more associated with \textit{popular} action, but later become identified with \textit{state} actions.

We constructed a set of French words associated with the people (\{\textit{peuple}, \textit{populaire}, ...\}) and the state (\{\textit{conseil}, \textit{gouvernement}, ...\}).\footnote{Note on choice of seed words: we are tracking changes in the meaning of \textit{``révolution''} between 1789 and 1793 thus only looking at the vocabulary used during the French Revolution. Although the connection between \textit{``peuple''} and \textit{``révolution''} could be found before July 14, 1789, it is in the aftermath of that date that this connection became the primary one. Prior to this, the delegates of the National Assembly in Versailles had claimed they had been leading the \textit{``révolution''}.} These seed words were used as classes for our classifier, which we trained on different temporal segments to observe the temporal change in our concept of interest. Our corpora are the \textit{Archives Parlementaires}, transcripts of parliamentary speeches during a time that contains moments of both emancipation and elite control of political processes.\footnote{\url{https://sul-philologic.stanford.edu/philologic/archparl/}} The corpus contains 9,628 speeches and 54,460,150 words from the years 1789-1793. Within this corpus, the term \textit{``révolution''} appears 2,206 times across 218 speeches, with a contextual basis of 90,138 words.

\begin{figure}[h!]
    \includegraphics[width=.45\textwidth]{Images/pca_result.png}
    \caption{In 1789, the word \textit{``revolution''} was primarily associated with popular action (represented in orange). In 1792 \textit{``revolution''} was now also seen as something that the government should lead (represented in blue) found in the \textit{``counter-revolution''} cluster. In 1793, this new governmental meaning had spread back to the word \textit{``revolution''} itself.
    % \cnote{TODO: improve visualization. Some text is not readable. Add legend}
    }
    \label{fig:revolution}
\end{figure}


We color-code the classes orange as \textit{``peuple'' } (\textit{``the people''}) and blue as \textit{``gouvernement''} (\textit{``the state''}) and project the embeddings down to a 2-dimensional space and visualize the results (Figure \ref{fig:revolution}).

We find that, in 1789, the word \textit{``révolution''} was primarily associated with popular action, the most famous example of which was the storming of the Bastille. In 1792, another definition became common: \textit{``révolution''} was now also seen as something that the government should lead. 

Interestingly, the first use of \textit{``révolution''} to be associated with governmental action is in fact around the term \textit{``counter-revolution''}. The word embeddings of \textit{``counter-révolution''} are predicted to be associated with the state, indicating that it was primarily when talking about threats to, and enemies of, the revolution, that politicians suggested transferring more power to the state. Jumping forward to 1793, this new governmental meaning is predicted for the word embeddings of \textit{``révolution''} itself. Our findings suggest that the goal of repressing counter-revolutionaries is what associated the term \textit{``révolution''} with governmental action. In other words, once revolutionaries became more concerned about tracking down their enemies, they granted to the government the same kind of extra-legal power that had originally only been the prerogative of the people in arms.\footnote{While the proclamation of the republic and the introduction of the new calendar are related to the idea of revolution, the conceptual shifts that we've identified appears prior to both of these events. The revolutionary calendar was not introduced until October 1793, meanwhile declaration of the Republic occurred in September 1792. The emergence of \textit{``counterrevolution''} as a state problem predates these events, confirming that neither play a role in introducing the newer understanding of ``revolution'' as a state-driven process.}

Our findings are consistent with historians' hypothesis that the meaning of revolution in the early years of the French Revolution is most closely aligned with the concept of the people and this gradually shifts as the revolution continues \cite{sewell2005logics,edelstein2012we}. Furthermore, our model allows us to uncover a potential causal story for this shift in the meaning; that the state sense of revolution first actually started with counter-revolution. This is a novel discovery in our understanding of the French Revolution; future humanistic work should use other methods to confirm this proposed causal link to counter-revolutionaries.

 See  Appendix~\ref{sec:finance_experiment} for another, more preliminary  social scientific application of \wc, in this case to study financial trends.  
\section{Related Work}
LLM unlearning~\citep{jang2023knowledgeunlearning, yao2023llmunlearningsurvey, lynch2024eight} has gained significant attention as a method to enhance privacy. Various approaches~\citep{sinha2024unstar, zhang2024npo} have been proposed to ensure that models effectively erase specific information while maintaining overall performance. A key challenge in unlearning is assessing whether knowledge unrelated to the forget set is inadvertently affected. To evaluate this, researchers commonly examine general knowledge~\citep{hendrycks2021measuring, cobbe2021training} as well as a designated subset of the retain set that shares a similar distribution with the forget set but excludes the targeted information. These subsets, often referred to as neighbor sets~\citep{closerlookat}, help determine the extent of unintended degradation in model performance.

In hazardous knowledge unlearning, prior work has leveraged domain-relevant general knowledge as a benchmark. For instance,~\citet{li2024wmdp} employs general biology knowledge to assess the impact of bioweapon-related unlearning and general computer security knowledge to evaluate the removal of information related to Attacking Critical Infrastructure. For entity unlearning~\citep{maini2024tofu, rwku}, previous studies have used entities from similar professions or those closely linked to the target entity as neighbor sets. While these approaches provide an initial framework, they lack a systematic investigation of which aspects of the retain set suffer the most from unlearning. Our study addresses this gap by systematically investigating the impact of unlearning on different types of neighbor sets more clearly and identifying which knowledge components experience the highest degree of forgetting.
\section{Discussion and Conclusion}
% In this paper, we reframe the task of semantic similarity from one of measuring distances to one of classification confusions. Our method is self-supervised and allows for researchers and downstream practitioners to measure the similarity of words with ease. We illustrate the performance of the identity probe on a number of word similarity tasks as well as provide examples of other tasks the identity probe can generalize to. In addition to the identity probe's high performance, we also find that the identity probe is able to produce asymmetric measures of similarity, opening a new line of future research. Lastly, we conclude with theoretical intuition of how a simple reframing of the problem results in new decision boundaries of similarity. We're excited for the NLP and broader community to use the identity probe as a measure of semantic similarity and discover other new ways in which the probe can be used for classification tasks.


In this paper, we reframe the task of semantic similarity from one of measuring distances to one of classification confusion. This formulation highlights the context-dependency of similarity judgments, meanwhile avoiding the pitfalls of geometric similarity measures \cite{evers2014revisiting}.

This new framing of semantic similarity in terms of classification confusion introduces new properties that are inspired by cognitive models of similarity \cite{tversky1977features} and accounts for the asymmetric nature of semantic similarity, captures different aspects of both similarity and multi-faceted words and ofter a measure that has interpretability benefits. 

Our proof-of-concept method, \wc, demonstrates the practical applicability and effectiveness of this reframing. Empirical results show that it outperforms cosine similarity on standard datasets.
For computational social science or cultural analytics applications, \wc can serve as a way to learn to represent words using target features (e.g., ``school'' in terms of \{\textit{positive}, \textit{negative}\}, and can be used to trace the meaning of a word as a function of time (like the word ``r\'{e}volution'').

The theoretical underpinnings of \wc allow it to learn complex word identity boundaries and capture the directional nature of similarity, offering a richer and more flexible framework for understanding word meanings. 

%To conclude, we reframe semantic similarity using classification confusion, to align it better with psychology literature. We implement a proof-of-concept framework that highlights the desired aspect(s) of a multi-faceted word by classifying it based on better-defined adjacent terms. 

While we implemented \wc as a linear classifier, the method naturally extends to capturing non-linear relationships among embedding components by replacing the linear projection with neural networks. Investigating whether the error function preserves its useful properties in non-linear settings remains an open question for future work.

While our experiments are preliminary and the space of possible similarity measures is enormous, we hope this reimagining of semantic similarity will inspire the development of new tools that better capture the multi-faceted and dynamic nature of language, advancing the fields of computational social science and cultural analytics and beyond.





\section*{Limitations}
Our proof-of-concept suggests a promising path where cosine similarity can be replaced by a more sophisticated method that involves self-supervision. However, the boost in performance also comes with some caveats. Because \wc is a supervised classifier, it requires an extra training step that simple cosine doesn't require.  Furthermore, potential users will need a basic understanding of model training and the pitfalls of over-fitting data.

As mentioned earlier, while we implemented \wc as a linear classifier, the method naturally extends to non-linear models. Additionally, various transformations commonly applied to embeddings before measuring distances \cite{Mu2018AllbuttheTopSA} can also be incorporated prior to using \wc, as our method relies on the resulting error signal to assess word similarity. Although non-linear models offer a promising direction, we have not yet examined whether the error function preserves its useful properties in such settings—an important avenue for future work. Introducing non-linearity into the classifier is known to alter its behavior in various ways, but its impact on confusion-based similarity remains uncertain. Further research is needed to evaluate its potential advantages and limitations.

Another key limitation of our approach is that we used three simple implementations of cosine similarity without exploring many possible augmentations to cosine, like normalizing it across the dataset (as was shown to be effective by \cite{timkey2021all}). Further refining both our classifier and cosine similarity implementations could lead to improved results for both,  as well as a deeper understanding of \wc.

Another important limitation of our analysis is that our results might be affected by the choice of seed words and the mechanisms on how we sample the ones used to represent the concepts. Changing seed words can impact the similarities. While we explored different sets of seed words without seeing drastic changes in results, a robust evaluation of the effect of different seed words should be considered in future work.

Lastly, we are not aware if changing the model used to create the embeddings can degrade the performance.

\section*{Ethics Statement}
As with all language technologies, there are a number of ethical concerns surrounding their usage and societal impact. It is likely that with this method, the biases known in contextualized embeddings can continue to propagate through downstream tasks, leading to representation or allocation harms. Additionally, the use of large language models for building contextualized embeddings is expensive and requires time and energy resources. To our knowledge, the method we have developed does not exacerbate any of these pre-existing ethical concerns but we recognize our work here also does not mitigate or avoid them.

\section*{Acknowledgments}
We thank Dallas Card, Nelson Liu, Kyle Hsu, Amelia Hardy, Kawin Ethayarajh, and Tianyi Zhang for their helpful feedback and discussion. This work was supported in part by the NSF via award number IIS-2128145, by the Hoffman–Yee Research Grants Program and the Stanford Institute for Human-Centered Artificial Intelligence, and by the Koret Foundation grant for Smart Cities and Digital Living.

% Entries for the entire Anthology, followed by custom entries
\bibliography{anthology,custom}
% \bibliographystyle{acl_natbib}

\appendix
\appendix
\begin{table}[t!]
  \centering
  


% \renewcommand{\arraystretch}{1.2} % 调整行高
% \setlength{\tabcolsep}{10pt}  % 调整列间距
\resizebox{0.48\textwidth}{!}{%
\begin{tabular}{lcrr}
        \toprule
        \textbf{Dataset} & \textbf{Full Size*} & \textbf{Consistency}  & \textbf{\dataset{}} \\
        \midrule
        HotpotQA  & 5,901 & 2,973 {\footnotesize \textcolor{gray}{(50\%)}}  & 1,476 {\footnotesize \textcolor{gray}{(25\%)}}  \\
        NewsQA    & 4,212 & 1,260 {\footnotesize \textcolor{gray}{(30\%)}} & 934  {\footnotesize \textcolor{gray}{(22\%)}}  \\
        NQ        & 7,314 & 4,419 {\footnotesize \textcolor{gray}{(60\%)}}  & 1,479 {\footnotesize \textcolor{gray}{(20\%)}}  \\
        SearchQA  & 16,980 & 12,133 {\footnotesize \textcolor{gray}{(71\%)}} & 1,497 {\footnotesize \textcolor{gray}{(9\%)}}  \\
        SQuAD     & 10,490 & 5,024 {\footnotesize \textcolor{gray}{(48\%)}}  & 2,351 {\footnotesize \textcolor{gray}{(22\%)}}  \\
        TriviaQA  & 7,785 & 6654 {\footnotesize \textcolor{gray}{(85\%)}}  & 792  {\footnotesize \textcolor{gray}{(10\%)}}  \\
        \bottomrule
    \end{tabular}
}




 \caption{Number of instances at each stage in the \dataset{} construction pipeline.}
 \label{tab:our_bench_stats_each_step}
\end{table}
\section{Appendix}
\subsection{License}
We present the licenses of the datasets used in this study: Natural Questions (CC BY-SA 3.0 license), NewsQA (MIT License), SearchQA and TriviaQA (Apache License 2.0), HotpotQA and SQuAD (CC BY-SA 4.0 license).

All these licenses and agreements permit the use of their data for academic purposes.

\subsection{Details of Data Constructing}
\label{append:prompts}
In this section, we detail the two main steps in constructing \dataset{}. The dataset sizes at each stage of the pipeline are shown in Table~\ref{tab:our_bench_stats_each_step}.


\textbf{Parametric Knowledge Elicitation.} First, we elicit the LLM's parametric knowledge by prompting it in a closed-book setting (i.e., without any context). To ensure the reliability of the elicited knowledge, we apply a consistency-based filtering method. Specifically, for each query, the LLM is prompted five times, and the frequency of each response is recorded. The response with the highest frequency is identified as the majority answer. Queries where the majority answer appears fewer than three times are discarded, in order to filter out inconsistent responses and enhance data quality. The following prompt is used to instruct the LLM:
\begin{tcolorbox}
[title=Prompt for eliciting parametric knowledge,colback=blue!10,colframe=blue!50!black,arc=1mm,boxrule=1pt,left=1mm,right=1mm,top=1mm,bottom=1mm]
Answer the question \textcolor{blue}{\{\textit{brevity\_instruction}\}} and provide supporting evidence.

Question: \textcolor{blue}{\{\textit{question}\}}
\end{tcolorbox}
\noindent The ``\textit{brevity\_instruction}'' is used to guide the LLM to generate responses in a more concise form.

\textbf{Conflict Data Selection.} Next, we filter the data to retain only instances where the LLM's parametric knowledge directly conflicts with the contextual answer. Specifically, we categorize the data obtained from the previous step into two groups, conflicting and non-conflicting instances, based on the detailed results of conflict detection. All non-conflicting instances are discarded. GPT-4o-mini is then used to detect the presence of a conflict, using the following prompt:

\begin{tcolorbox}
[title=Prompt for identifying conflict knowledge,colback=blue!10,colframe=blue!50!black,arc=1mm,boxrule=1pt,left=1mm,right=1mm,top=1mm,bottom=1mm]
\small
You are tasked with evaluating the correctness of a model-generated answer based on the given information. 

\small
Context: \textcolor{blue}{\{\textit{context}\}}

Question: \textcolor{blue}{\{\textit{question}\}}

Contextual Answer: \textcolor{blue}{\{\textit{contextual\_answer}\}}

Model-Generated Answer: \textcolor{blue}{\{\textit{Model-Generated\_answer}\}}

\textcolor{blue}{[\textit{Detailed task description...}]}

Output Format:

Evaluate result: (Correct / Partially Correct / Incorrect) 
\end{tcolorbox}




\subsection{Assessing the Reliability of GPT-4o-mini in Knowledge Conflict Identification}
\label{append:human_eval}
In this subsection, we conduct the human evaluation to assess the reliability of GPT-4o-mini in identifying knowledge conflicts, which is a critical task in our data construction process to guarantee the data quality.

We randomly sampled 100 examples from each of the six subsets of \dataset{}, yielding a total of 600 samples. Six senior computational linguistics researchers were then asked to evaluate whether a knowledge conflict was present in each example. For each instance, the evaluators were provided with the question, the contextual answer, the model-generated response, and the corresponding supporting evidence. The results were classified into three categories: No Conflict, Somewhat Conflict, and High Conflict. The detailed annotation instructions are as follows:

\begin{tcolorbox}
[title=Annotation Instruction,colback=blue!10,colframe=blue!50!black,arc=1mm,boxrule=1pt,left=1mm,right=1mm,top=1mm,bottom=1mm]
\small
You are tasked with determining whether the parametric knowledge of LLMs conflicts with the given context to facilitate the study of knowledge conflicts in large language models.

Each data instance contains the following fields: 

Question: \textcolor{blue}{\{\textit{question}\}}


Answers: \textcolor{blue}{\{\textit{answers}\}}


Context: \textcolor{blue}{\{\textit{context}\}}

Parametric\_knowledge: \textcolor{blue}{\{\textit{LLMs' parametric\_knowledge }\}} 

The annotation process consists of two steps. 

\textbf{Step 1}: Compare the model-generated answer with the ground truth answers, based on the given question and context, to determine whether the model’s parametric knowledge conflicts with the context.

\textbf{Step 2}: Classify the results into one of three categories: 

\textcolor{blue}{\{\textit{No Conflict}\}} if the model-generated answer is consistent with the ground truth answers and context, 

\textcolor{blue}{\{\textit{Somewhat Conflict}\}}  if it is partially inconsistent

\textcolor{blue}{\{\textit{High Conflict}\}} if it significantly contradicts the ground truth answers or context.
\end{tcolorbox}


The evaluation results, shown in Table~\ref{tab:append_human_eval}, reveal a high level of agreement between the human annotators and GPT-4o-mini. Over 85\% of the examples reach consensus among the annotators, with an average agreement rate of 85.6\% across all subsets. These findings underscore the reliability of GPT-4o-mini as an effective tool for identifying knowledge conflicts.




\begin{table}[t]
  \centering
  
\centering
\begin{tabular}{l c}
\toprule
\textbf{Subset} & \textbf{Agreement (\%)} \\ \midrule
HotpotQA        & 81.4                        \\
NewsQA          & 72.7                        \\
NQ              & 88.7                        \\
SearchQA        & 95.3                        \\
SQuAD           & 86.1                        \\
TriviaQA        & 90.7                        \\ \midrule
\textbf{Average} & \textbf{85.6}            \\ \bottomrule
\end{tabular}

 \caption{Agreement between human annotators and GPT-4o-mini across different subsets of our \dataset{} benchmark.}
 \label{tab:append_human_eval}
\end{table}



\subsection{Evaluating the Effectiveness of Our Consistency-Based Filtering Method}
\label{append:data_freq}

In this subsection, we evaluate the effectiveness of our consistency-based knowledge conflict filtering method. As described in Appendix~\ref{append:prompts}, for each query, we prompt the model five times and record the most frequently generated answer along with its occurrence frequency. Based on this frequency, we divide the data into sub-datasets, where all queries within each sub-dataset share the same answer frequency. We then apply ``Conflict Data Selection'' to each sub-dataset, retaining only instances where knowledge conflicts occur. Finally, we evaluate ConR and MemR on these sub-datasets.

As shown in Figure~\ref{fig:diff_freq}, a clear trend emerges: as answer frequency increases, ConR consistently decreases, while MemR increases. This pattern indicates that as answer frequency rises, the model becomes increasingly reliant on its internal knowledge. Notably, for data with an answer frequency of 1, MemR is only 3\%, indicating minimal dependence on internal knowledge. Retaining only high-answer-frequency data improves the quality of \dataset{}. This data construction approach distinguishes our methodology from previous studies~\cite{longpre2021entity,xie2023adaptive}.

\begin{figure}[t!]
  \centering
  \includegraphics[width=0.4\textwidth]{figs/diff_freq.pdf}
  \caption{Performance comparison of ConR and MemR across sub-datasets grouped by the answer frequency of LLMs.}
  \label{fig:diff_freq}
\end{figure}





\subsection{Additional Implementation Details of Our Experiments}
\label{append:implementation}
This subsection outlines the training prompt, describes more details of the training data, and provides details of the experimental setup used in our experiments.

\textbf{Training Prompts.}
We adopt a simple QA-format training prompt following~\citet{zhou2023context} for all methods except \attrprompt{} and \oiprompt{}.
\begin{tcolorbox}
[title=Base Prompt ,colback=blue!10,colframe=blue!50!black,arc=1mm,boxrule=1pt,left=1mm,right=1mm,top=1mm,bottom=1mm]
% \small
\textcolor{blue}{\{\textit{context}\}} 
Q: \textcolor{blue}{\{\textit{question}\}} ? 
A: \textcolor{blue}{\{\textit{answer}\}}.
\end{tcolorbox}


\textbf{Training Datasets.} During \method{}, we randomly sample 32,580 instances from the training set of the MRQA 2019 benchmark~\cite{fisch2019mrqa} to construct our training data.



\textbf{Experimental Setup.} In this work, all models are trained for 2,100 steps with a total batch size of 32 and a learning rate of 1e-4. To enhance training efficiency, we implemented \method{} with LoRA~\cite{hu2021lora}, setting both the rank $\text{r}$ and scaling factor $\text{alpha}$ to 64. For \method{}, we set $\alpha$ to 0.1 (Eq.~\ref{eq:selct_layers}), which determines the minimum activation ratio difference required for a layer to be pruned. Additionally, we adopt a dynamic $\gamma$ in $\mathcal{L}_{\text{KC}}$ (Eq.~\ref{eq:kc_loss}), which linearly transitions from an initial margin ($\gamma_{0}=1$) to a final margin ($\gamma^*=5$) as training progresses. This adaptive strategy gradually reduces the model's reliance on internal parametric knowledge, encouraging it to rely more on external knowledge provided by the KAG system.


\subsection{Implementation Details of Baselines}
\label{append:baseline}
This subsection describes the implementation details of all baseline methods.

We adopt two prompt-based baselines: the attributed prompt ($\text{Attr}_{\text{prompt}}$) and a combination of opinion-based and instruction-based prompts ($\text{O\&I}_{\text{prompt}}$). The corresponding prompt templates are as follows:

\begin{tcolorbox}
[title=Attr based prompt ,colback=blue!10,colframe=blue!50!black,arc=1mm,boxrule=1pt,left=1mm,right=1mm,top=1mm,bottom=1mm]
% \small
\textcolor{blue}{\{\textit{context}\}} Q: \textcolor{blue}{\{\textit{question}\}} based on the given text? A: \textcolor{blue}{\{\textit{answer}\}}.
\end{tcolorbox}

\begin{tcolorbox}
[title=O\&I based prompt ,colback=blue!10,colframe=blue!50!black,arc=1mm,boxrule=1pt,left=1mm,right=1mm,top=1mm,bottom=1mm]

Bob said ``\textcolor{blue}{\{\textit{context}\}}'' Q: \textcolor{blue}{\{\textit{question}\}} in Bob's opinion? A: \textcolor{blue}{\{\textit{answer}\}}.
\end{tcolorbox}
For the SFT baseline, we incorporate context during training, similar to \method{}, while keeping the remaining experimental settings identical. To construct preference pairs for DPO training, we use contextually aligned answers from the dataset as ``preferred responses'' to ensure the consistency with the provided context. The ``rejected responses'' are generated by identifying parametric knowledge conflicts through our data construction methodology (Sec.~\ref{sec:benchmark}).

For KAFT, we employ a hybrid dataset containing both counterfactual and factual data. Specifically, we integrate the counterfactual data developed by \citet{xie2023adaptive}, leveraging their advanced data construction framework.

By maintaining equivalent dataset sizes and ensuring comparable data quality across all baselines, we provide a rigorous and fair comparison with our proposed \method{}.




\subsection{Extending \method{} to More LLMs}
\label{append:diff_model_performance}


\begin{figure}[t!]
  \centering
  
\subfigure[ConR Results]{
        \label{fig:diff_model:llama_conr}
        \includegraphics[width=0.462\linewidth]{append_fig/llama_conr.pdf}
    }
    \hspace{0.0005\linewidth} 
    \subfigure[MemR Results]{
        \label{fig:diff_model:llama_memr}
        \includegraphics[width=0.462\linewidth]{append_fig/llama_memr.pdf}
    }


  % \includegraphics[width=0.48\textwidth]{figs/diff_model_double.pdf}
 \caption{Average ConR and MemR across different models implemented by LLMs of LLaMA series, before and after applying \method{}.
 }
 \label{fig:diff_model_double_llama}
\end{figure}

\begin{figure}[t]
  \centering
  \subfigure[ConR Results]{
        \label{fig:diff_model:qwen_conr}
        \includegraphics[width=0.462\linewidth]{append_fig/qwen_conr.pdf}
    }
    \hspace{0.0005\linewidth} 
    \subfigure[MemR Results]{
        \label{fig:diff_model:qwen_memr}
        \includegraphics[width=0.462\linewidth]{append_fig/qwen_memr.pdf}
    }
  % \includegraphics[width=0.48\textwidth]{figs/diff_model_double.pdf}
 \caption{Average ConR and MemR across different models implemented by LLMs of Qwen series, before and after applying \method{}.
 }
 \label{fig:diff_model_double_qwen}
\end{figure}






We extend \method{} to a diverse range of LLMs, encompassing multiple model families and sizes. 

Specifically, our evaluation includes LLaMA3-8B-Instruct, LLaMA3.2-1B-Instruct, LLaMA3.2-3B-Instruct, Qwen2.5-0.5B-Instruct, Qwen2.5-1.5B-Instruct, Qwen2.5-3B-Instruct, Qwen2.5-7B-Instruct, and Qwen2.5-14B-Instruct. The results on ConR and MemR are summarized in Figures~\ref{fig:diff_model_double_llama} and \ref{fig:diff_model_double_qwen}, while Table~\ref{tab:append:all_model_res} presents the average performance of all models on \dataset{} and ConFiQA. Additionally, Table~\ref{tab:diff_model_param} provides detailed parameter information and specifies the layers selected for pruning for each model. This comprehensive evaluation demonstrates the versatility and scalability of \method{} across a wide spectrum of model architectures and sizes.

\begin{table}[!t]
  
    \resizebox{0.48\textwidth}{!}{%
\begin{tabular}{l|c|c|c}
\toprule
\textbf{Models}     & \textbf{Param.} & \textbf{\method{} Param.} & \textbf{Selected Layers} \\
\midrule
\rowcolor{gray!10}
LLaMA3.2-1B        & 1.24B  & 1.08B \small\textcolor{gray}{(87\%)}   & [12, 14]                 \\
LLaMA3.2-3B        & 3.21B  & 2.60B \small\textcolor{gray}{(81\%)}   &  [18, 25]   \\
\rowcolor{gray!10}
LLaMA3-8B          & 8.03B  & 6.97B \small\textcolor{gray}{(87\%)}   & [24, 29]      \\
LLaMA3.1-8B          & 8.03B  & 6.27B \small\textcolor{gray}{(78\%)}   & [20, 29]      \\
\rowcolor{gray!10}
Qwen2.5-0.5B         & 0.49B  & 0.44B \small\textcolor{gray}{(90\%)}   &  [19, 22]       \\
Qwen2.5-1.5B         & 1.54B  & 1.34B \small\textcolor{gray}{(87\%)}   & [21, 25]        \\
\rowcolor{gray!10}
Qwen2.5-3B         & 3.09B  & 2.68B \small\textcolor{gray}{(87\%)}   & [29, 34]        \\
Qwen2.5-7B         & 7.61B  & 7.21B \small\textcolor{gray}{(95\%)}   &   [25, 26 ]     \\
\rowcolor{gray!10}
Qwen2.5-14B        & 14.70B & 12.43B \small\textcolor{gray}{(85\%)}  &  [35, 45]   \\
\bottomrule
\end{tabular}
}

% \end{sidewaystable}

% \end{document}

  \caption{The total number of parameters for various models before and after applying \method{}. \textcolor{gray}{\small$(\cdot)\%$} represents the proportion relative to the original model, and the last column lists the layers selected for pruning.}
   \label{tab:diff_model_param}
\end{table}

These experimental results illustrate several key insights: 1) Larger models tend to rely more on parametric memory. As model size increases in both the LLaMA and Qwen families, MemR also grows, indicating a tendency to overlook external knowledge in favor of internal parameters. \method{} counteracts this behavior, decreasing larger models' MemR score to even below that of smaller models. 2) \method{} consistently benefits all evaluated models. Across both LLaMA and Qwen model families, \method{} outperforms Vanilla-KAG by boosting accuracy and context faithfulness, underscoring its broad applicability and effectiveness. 3) Not all parameters in KAG models are essential. Pruning parametric knowledge not only reduces computation costs but also fosters better generalization without sacrificing accuracy, highlighting the potential of building a parameter-efficient LLM within the KAG framework.




\begin{table*}[!t]
  
\centering
\resizebox{0.96\textwidth}{!}{%
\begin{tabular}{l|c|cccc|cccc}
\toprule
\multirow{2}{*}{\textbf{Models}} & \multirow{2}{*}{\textbf{Param.}} & \multicolumn{4}{c|}{\textbf{\dataset{}}} & \multicolumn{4}{c}{\textbf{ConFiQA}} \\ 
\cmidrule(lr){3-6}  \cmidrule(lr){7-10}
 &  & ConR $\uparrow$ & MemR $\downarrow$ & MR $\downarrow$ & EM $\uparrow$ & ConR $\uparrow$ & MemR $\downarrow$ & MR $\downarrow$ & EM $\uparrow$ \\ 
\midrule
LLaMA3-8B   & 8.03B  & 66.99  & 11.75  & 14.99  & 13.83  & 22.52  & 31.15  & 59.77  & 2.47 \\
\rowcolor{gray!10}
+\method{}    & 6.97B  & 71.50  & 6.48   & 8.41   & 66.19  & 70.43  & 8.82   & 11.32  & 67.29 \\
LLaMA3.1-8B & 8.03B  & 63.15  & 11.69  & 15.93  & 21.85  & 15.38  & 29.97  & 68.98  & 6.69 \\
\rowcolor{gray!10}
+\method{}   & 6.27B  & 70.41  & 6.95   & 9.17   & 63.58  & 71.12  & 9.01   & 11.44  & 66.61 \\
LLaMA3.2-1B & 1.24B  & 39.06  & 10.49  & 21.83  & 5.13   & 32.09  & 18.32  & 36.28  & 7.15 \\
\rowcolor{gray!10}
+\method{}   & 1.08B  & 51.75  & 6.51   & 11.34  & 47.60  & 62.70  & 7.63   & 11.38  & 61.85 \\
LLaMA3.2-3B & 3.21B  & 56.75  & 11.53  & 17.11  & 12.69  & 26.16  & 23.47  & 49.05  & 9.84 \\
\rowcolor{gray!10}
+\method{}   & 2.60B  & 67.00  & 6.80   & 9.35   & 61.59  & 69.61  & 8.39   & 11.09  & 66.53 \\
Qwen2.5-0.5B & 0.49B  & 47.17  & 11.36  & 19.48  & 2.06   & 50.72  & 17.15  & 26.20  & 3.78 \\
\rowcolor{gray!10}
+\method{}   & 0.44B  & 58.13  & 6.63   & 10.41  & 52.56  & 67.54  & 8.04   & 11.03  & 66.33 \\
Qwen2.5-1.5B & 1.54B  & 58.08  & 11.28  & 16.48  & 10.30  & 51.69  & 19.87  & 28.23  & 10.78 \\
\rowcolor{gray!10}
+\method{}   & 1.34B  & 63.78  & 6.74   & 9.76   & 57.67  & 69.61   & 8.35   & 11.05   & 66.04 \\
Qwen2.5-3B   & 3.09B  & 62.22  & 14.45  & 18.88  & 0.10   & 25.47  & 29.34  & 55.70  & 0.01 \\
\rowcolor{gray!10}
+\method{}     & 2.68B  & 66.31  & 6.75   & 9.38   & 59.42  & 66.30   & 8.62  & 11.94   & 63.03 \\
Qwen2.5-7B    & 7.61B  & 65.46  & 14.93  & 18.57  & 0.80   & 24.75  & 33.09  & 59.04  & 0.10 \\
\rowcolor{gray!10}
+\method{}      & 6.60B  & 67.75  & 6.60   & 9.01   & 61.77  & 69.54  & 8.85   & 11.58  & 66.68 \\
Qwen2.5-14B   & 14.70B & 65.75  & 16.13  & 19.75  & 0.00   & 7.86   & 32.88  & 83.71  & 0.01 \\
\rowcolor{gray!10}
+\method{}     & 12.43B & 70.01  & 6.43   & 8.55   & 64.43  & 71.70  & 8.90   & 11.29  & 68.40 \\
\bottomrule
\end{tabular}%
}


  \caption{Average performance of LLMs on \dataset{} and ConFiQA before and after applying \method{}.}
   \label{tab:append:all_model_res}
\end{table*}

\subsection{Neuron Activations in Different LLMs}\label{app:activation}
We present the neuron activations for the LLaMA family models, including LLaMA-3.2-1B-Instruct, LLaMA-3.2-3B-Instruct, LLaMA-3-8B-Instruct, and LLaMA-3.1-8B-Instruct, as well as the Qwen family models, including Qwen-2.5-0.5B-Instruct, Qwen-2.5-1.5B-Instruct, Qwen-2.5-3B-Instruct, Qwen-2.5-7B-Instruct, and Qwen-2.5-14B-Instruct, in Figures~\ref{fig:act_llama} and \ref{fig:act_qwen}, respectively. 
% 我们发现qwen系列模型


\begin{figure*}[t]
  \centering
  \subfigure[Neuron activations of LLaMA-3.2-1B-Instruct]{
        \label{fig:act_llama:3.2-1b}
        \includegraphics[width=0.9\linewidth]{append_fig/act_llama32_1b_all.pdf}
    }
\subfigure[Neuron activations of LLaMA-3.2-3B-Instruct]{
        \label{fig:act_llama:3.2-3b}
        \includegraphics[width=0.9\linewidth]{append_fig/act_llama32_3b_all.pdf}
    }
 \subfigure[Neuron activations of LLaMA-3-8B-Instruct]{
        \label{fig:act_llama:3-8b}
        \includegraphics[width=0.9\linewidth]{append_fig/act_llama_3_8b.pdf}
    }
 \subfigure[Neuron activations of LLaMA-3.1-8B-Instruct]{
        \label{fig:act_llama:3.1-8b}
        \includegraphics[width=0.9\linewidth]{append_fig/act_llama_31_8b.pdf}
    }
 

 \caption{Neuron activations across different layers of the LLaMA series models. We present the inhibition ratio $\Delta R$ under two conditions: with contextual knowledge input (w/ context) and without it (w/o context).}
 \label{fig:act_llama}
\end{figure*}

\begin{figure*}[t]
  \centering
  \subfigure[Neuron activations of Qwen-2.5-0.5B-Instruct]{
        \label{fig:act_qwen:2.5-0.5b}
        \includegraphics[width=0.75\linewidth]{append_fig/act_qwen25_0_5b_all.pdf}
    }
\subfigure[Neuron activations of Qwen-2.5-1.5B-Instruct]{
        \label{fig:act_qwen:2.5-1.5b}
        \includegraphics[width=0.75\linewidth]{append_fig/act_qwen25_1_5b_all.pdf}
    }
\subfigure[Neuron activations of Qwen-2.5-3B-Instruct]{
        \label{fig:act_qwen:2.5-3b}
        \includegraphics[width=0.75\linewidth]{append_fig/act_qwen25_3b_all.pdf}
    }
\subfigure[Neuron activations of Qwen-2.5-7B-Instruct]{
        \label{fig:act_qwen:2.5-7b}
        \includegraphics[width=0.75\linewidth]{append_fig/act_qwen25_7b_all.pdf}
    }
\subfigure[Neuron activations of Qwen-2.5-14B-Instruct]{
        \label{fig:act_qwen:2.5-14b}
        \includegraphics[width=0.75\linewidth]{append_fig/act_qwen25_14b_all.pdf}
    }


 \caption{Neuron activations across different layers of the Qwen series models. We present the inhibition ratio $\Delta R$ under two conditions: with contextual knowledge input (w/ context) and without it (w/o context). }
 \label{fig:act_qwen}
\end{figure*}

\end{document}