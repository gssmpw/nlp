% This must be in the first 5 lines to tell arXiv to use pdfLaTeX, which is strongly recommended.
\pdfoutput=1
% In particular, the hyperref package requires pdfLaTeX in order to break URLs across lines.

\documentclass[11pt]{article}

% Remove the "review" option to generate the final version.
\usepackage[]{acl}
\usepackage{enumitem}

% Standard package includes
\usepackage{contour}
\usepackage{xcolor}
\usepackage{times}
\usepackage{latexsym}
\usepackage{graphicx}
\usepackage{subcaption}
\usepackage{mwe}
\usepackage{xcolor}
\usepackage{booktabs}
\usepackage{amsmath,amsfonts,amssymb,amsthm}
\usepackage{multicol}\usepackage{xcolor}         % colors
\usepackage{diagbox}
\contourlength{0.8pt}
\usepackage{makecell}

\newcommand{\orangegloss}[1]{\contour{orange}{\textcolor{white}{#1}}}
\newcommand{\bluegloss}[1]{\contour{blue}{\textcolor{black}{#1}}}
\newcommand{\xhdr}[1]{\vspace{1mm}\noindent{{\bf #1.}}}

\newcommand{\cnote}[1]{\textcolor{red}{$\ll$\textsf{#1 | Chen}$\gg$}}
\newcommand{\snote}[1]{\textcolor{blue}{$\ll$\textsf{#1 | Sarah}$\gg$}}
\newcommand{\fnote}[1]{\textcolor{burgundy}{$\ll$\textsf{#1 | Fede}$\gg$}}

\definecolor{amaranth}{HTML}{E52B50}
\definecolor{garnet}{HTML}{733635}
\definecolor{burgundy}{HTML}{800020}
\definecolor{midnightgreenn}{rgb}{0.0, 0.29, 0.33}
\definecolor{navyblue}{rgb}{0.0, 0.0, 0.5}

\usepackage[T1]{fontenc}
\usepackage[utf8]{inputenc} % The default since 2018
\usepackage[english]{babel}
\newcommand\dedicationfont{\fontfamily{qcr}\itshape\mdseries\selectfont}
\newcommand{\wc}{{\dedicationfont \textcolor{navyblue}{Word Confusion}}\xspace}
\newcommand{\remove}[1]{}

% For proper rendering and hyphenation of words containing Latin characters (including in bib files)
\usepackage[T1]{fontenc}
% For Vietnamese characters
% \usepackage[T5]{fontenc}
% See https://www.latex-project.org/help/documentation/encguide.pdf for other character sets

% This assumes your files are encoded as UTF8
\usepackage[utf8]{inputenc}

% This is not strictly necessary, and may be commented out.
% However, it will improve the layout of the manuscript,
% and will typically save some space.
\usepackage{microtype}

% This is also not strictly necessary, and may be commented out.
% However, it will improve the aesthetics of text in
% the typewriter font.
\usepackage{inconsolata}
\usepackage{booktabs}
\setlength{\belowcaptionskip}{-10pt}
% If the title and author information does not fit in the area allocated, uncomment the following
%
%\setlength\titlebox{<dim>}
%
% and set <dim> to something 5cm or larger.



\title{Rethinking Word Similarity:\\Semantic Similarity through Classification Confusion}


% Author information can be set in various styles:
% For several authors from the same institution:
% \author{Author 1 \and ... \and Author n \\
%         Address line \\ ... \\ Address line}
% if the names do not fit well on one line use
%         Author 1 \\ {\bf Author 2} \\ ... \\ {\bf Author n} \\
% For authors from different institutions:
% \author{Author 1 \\ Address line \\  ... \\ Address line
%         \And  ... \And
%         Author n \\ Address line \\ ... \\ Address line}
% To start a seperate ``row'' of authors use \AND, as in
% \author{Author 1 \\ Address line \\  ... \\ Address line
%         \AND
%         Author 2 \\ Address line \\ ... \\ Address line \And
%         Author 3 \\ Address line \\ ... \\ Address line}

\author{
  Kaitlyn Zhou, 
  Haishan Gao, 
  Sarah Chen, 
  Dan Edelstein,   
  Dan Jurafsky, 
  Chen Shani\\
  % Kaitlyn Zhou, \hspace{3pt}
  % Haishan Gao, \hspace{3pt}
  % Sarah Chen,\\
  % % \textbf{Federico Bianchi, \hspace{3pt}}
  % \textbf{Dan Edelstein, \hspace{3pt}}  
  % \textbf{Dan Jurafsky, \hspace{3pt}}
  % \textbf{Chen Shani}\\
  Stanford University\\
\texttt{\{katezhou, hsgao, sachen, danedels, jurafsky, cshani\}@stanford.edu} \\  
}

\begin{document}
\maketitle

\begin{abstract}
\begin{abstract}


The choice of representation for geographic location significantly impacts the accuracy of models for a broad range of geospatial tasks, including fine-grained species classification, population density estimation, and biome classification. Recent works like SatCLIP and GeoCLIP learn such representations by contrastively aligning geolocation with co-located images. While these methods work exceptionally well, in this paper, we posit that the current training strategies fail to fully capture the important visual features. We provide an information theoretic perspective on why the resulting embeddings from these methods discard crucial visual information that is important for many downstream tasks. To solve this problem, we propose a novel retrieval-augmented strategy called RANGE. We build our method on the intuition that the visual features of a location can be estimated by combining the visual features from multiple similar-looking locations. We evaluate our method across a wide variety of tasks. Our results show that RANGE outperforms the existing state-of-the-art models with significant margins in most tasks. We show gains of up to 13.1\% on classification tasks and 0.145 $R^2$ on regression tasks. All our code and models will be made available at: \href{https://github.com/mvrl/RANGE}{https://github.com/mvrl/RANGE}.

\end{abstract}


\end{abstract}
\section{Introduction}

Video generation has garnered significant attention owing to its transformative potential across a wide range of applications, such media content creation~\citep{polyak2024movie}, advertising~\citep{zhang2024virbo,bacher2021advert}, video games~\citep{yang2024playable,valevski2024diffusion, oasis2024}, and world model simulators~\citep{ha2018world, videoworldsimulators2024, agarwal2025cosmos}. Benefiting from advanced generative algorithms~\citep{goodfellow2014generative, ho2020denoising, liu2023flow, lipman2023flow}, scalable model architectures~\citep{vaswani2017attention, peebles2023scalable}, vast amounts of internet-sourced data~\citep{chen2024panda, nan2024openvid, ju2024miradata}, and ongoing expansion of computing capabilities~\citep{nvidia2022h100, nvidia2023dgxgh200, nvidia2024h200nvl}, remarkable advancements have been achieved in the field of video generation~\citep{ho2022video, ho2022imagen, singer2023makeavideo, blattmann2023align, videoworldsimulators2024, kuaishou2024klingai, yang2024cogvideox, jin2024pyramidal, polyak2024movie, kong2024hunyuanvideo, ji2024prompt}.


In this work, we present \textbf{\ours}, a family of rectified flow~\citep{lipman2023flow, liu2023flow} transformer models designed for joint image and video generation, establishing a pathway toward industry-grade performance. This report centers on four key components: data curation, model architecture design, flow formulation, and training infrastructure optimization—each rigorously refined to meet the demands of high-quality, large-scale video generation.


\begin{figure}[ht]
    \centering
    \begin{subfigure}[b]{0.82\linewidth}
        \centering
        \includegraphics[width=\linewidth]{figures/t2i_1024.pdf}
        \caption{Text-to-Image Samples}\label{fig:main-demo-t2i}
    \end{subfigure}
    \vfill
    \begin{subfigure}[b]{0.82\linewidth}
        \centering
        \includegraphics[width=\linewidth]{figures/t2v_samples.pdf}
        \caption{Text-to-Video Samples}\label{fig:main-demo-t2v}
    \end{subfigure}
\caption{\textbf{Generated samples from \ours.} Key components are highlighted in \textcolor{red}{\textbf{RED}}.}\label{fig:main-demo}
\end{figure}


First, we present a comprehensive data processing pipeline designed to construct large-scale, high-quality image and video-text datasets. The pipeline integrates multiple advanced techniques, including video and image filtering based on aesthetic scores, OCR-driven content analysis, and subjective evaluations, to ensure exceptional visual and contextual quality. Furthermore, we employ multimodal large language models~(MLLMs)~\citep{yuan2025tarsier2} to generate dense and contextually aligned captions, which are subsequently refined using an additional large language model~(LLM)~\citep{yang2024qwen2} to enhance their accuracy, fluency, and descriptive richness. As a result, we have curated a robust training dataset comprising approximately 36M video-text pairs and 160M image-text pairs, which are proven sufficient for training industry-level generative models.

Secondly, we take a pioneering step by applying rectified flow formulation~\citep{lipman2023flow} for joint image and video generation, implemented through the \ours model family, which comprises Transformer architectures with 2B and 8B parameters. At its core, the \ours framework employs a 3D joint image-video variational autoencoder (VAE) to compress image and video inputs into a shared latent space, facilitating unified representation. This shared latent space is coupled with a full-attention~\citep{vaswani2017attention} mechanism, enabling seamless joint training of image and video. This architecture delivers high-quality, coherent outputs across both images and videos, establishing a unified framework for visual generation tasks.


Furthermore, to support the training of \ours at scale, we have developed a robust infrastructure tailored for large-scale model training. Our approach incorporates advanced parallelism strategies~\citep{jacobs2023deepspeed, pytorch_fsdp} to manage memory efficiently during long-context training. Additionally, we employ ByteCheckpoint~\citep{wan2024bytecheckpoint} for high-performance checkpointing and integrate fault-tolerant mechanisms from MegaScale~\citep{jiang2024megascale} to ensure stability and scalability across large GPU clusters. These optimizations enable \ours to handle the computational and data challenges of generative modeling with exceptional efficiency and reliability.


We evaluate \ours on both text-to-image and text-to-video benchmarks to highlight its competitive advantages. For text-to-image generation, \ours-T2I demonstrates strong performance across multiple benchmarks, including T2I-CompBench~\citep{huang2023t2i-compbench}, GenEval~\citep{ghosh2024geneval}, and DPG-Bench~\citep{hu2024ella_dbgbench}, excelling in both visual quality and text-image alignment. In text-to-video benchmarks, \ours-T2V achieves state-of-the-art performance on the UCF-101~\citep{ucf101} zero-shot generation task. Additionally, \ours-T2V attains an impressive score of \textbf{84.85} on VBench~\citep{huang2024vbench}, securing the top position on the leaderboard (as of 2025-01-25) and surpassing several leading commercial text-to-video models. Qualitative results, illustrated in \Cref{fig:main-demo}, further demonstrate the superior quality of the generated media samples. These findings underscore \ours's effectiveness in multi-modal generation and its potential as a high-performing solution for both research and commercial applications.
\section{Introducing \wc}
\label{sec:identity_probe}


Our method begins by defining a set of words, or features. For example, we might choose the set $W=\{red,\ green,\ blue\}$ if we wanted to study similarity related to colors. These words will act as features that can be selected by the analyst to focus on a particular dimension or question.


Our process then has two phases: training and inference. In training (illustrated in part (a) of Figure \ref{fig:word_confusion})) we extract from a corpus a set of sentences containing each of these words, such as \textit{``The sunset painted the sky a brilliant shade of red''} for the word ``red''. We then use BERT to extract the contextual embeddings of these feature-words, and train a classifier to map from an embedding to its corresponding word identity. The classifier's training objective is to correctly classify the embedding to the word that corresponds to it. 

More formally, given a chosen set of word $W$ and embeddings $\{e_1, e_2 .... e_i\} \in E$ that correspond to word identities $\{w_1, w_2, ..., w_i\} \in W$, we train a logistic regression classifier on all pairs of $\{e_i, w_i\}$.

At inference (part (b) in Figure \ref{fig:word_confusion}), we wish to define the semantic similarity of a word in terms of the classifier's classes, which can be thought of as features.\footnote{Thus the choice of a different set of classes is a way of selecting different features  to describe the input word.} Now suppose we would like to compute the similarity of the new word ``burgundy'' to various colors. We extract the contextual embedding of  ``burgundy'' given the sentence \textit{``Burgundy is a deep reddish-brown shade inspired by wine''}, and  use the trained classifier to compute the probability that the \textit{``burgundy''}-embedding corresponds to each class $W=\{red,\ green,\ blue\}$. We then use the classifier's confusion matrix to understand which primary colors burgundy is similar to.  For example, the similarity of ``burgundy'' to  ``red'' is the probability our classifier assigns to the class \textit{``red''}.

More formally, the probability distribution predicted by the model, $\vec{p_j} \in \mathbb{R}^{|W|}$, is used to quantify the semantic similarity between $w_j$ (Burgundy) and each $w_i, \forall w_i \in W$ (in this case $W=$\{{red},\ green,\ blue\}).   Thus:
\begin{eqnarray}
\text{sim}_{\text{WC}}(w_i,w_j) \stackrel{\mathrm{def}}{=}
 p(w_i|e_j)
\end{eqnarray}
The set of distractor words chosen to train the initial classifier thus act as features that can be selected by the analyst to focus on a particular dimension or question.
 
Note that as with the example ``burgundy",  the input word at inference can be out-of-vocabulary with respect to the classifier, or the target word can be one of the classifier's classes (in which case we ignore the probability it assigns to that word and use the other $N-1$ features.)

 


\label{sec:cosine_vs_probe}
\begin{figure*}[h!]
    \centering
    \includegraphics[width=0.9\textwidth]{Images/decision_surface.png}
    \caption{Differences in decision boundaries between \wc and cosine similarity. The $x$ and $y$ axes represent two dimensions of an artificially constructed set of data points. Note how cosine similarity's boundaries originate from the origin whereas {\wc}'s are not limited in the same way.}
    \label{fig_concept_prob_boundaries}
\end{figure*}

\subsection{Benchmarking \wc}
\label{sec:initial_eval}

The intuition behind \wc is that if it struggles to distinguish between contextual embeddings of \textit{burgundy} and \textit{red}, this could indicate they are similar. To test this hypothesis, we use \wc on three semantic similarity benchmarks. For each task, we trained a model using sentences from English Wikipedia. Our classes contained all the words from the benchmark. We then built word embeddings by averaging the last four hidden layers of BERT-base-cased (Details in Appendix~\ref{appendix:validation_details}). 

To calculate the similarity between two words $w_i, w_j$, we first extract all the sentences containing $w_i$ from English Wikipedia. We average the contextual token embeddings of $w_i$ using these sentences. This average token embedding was the input to the trained classifier (with classes containing all the words in the benchmark). We then use the probability \wc assigned to $w_i$  to set the similarity score between $w_i$ and $w_j$. We tested three benchmarks:

\begin{itemize}
\itemsep 0pt
    \item \textbf{MEN} contains 3000-word pairs annotated by 50 humans  based on their ``relatedness'' \cite{agirre-etal-2009-study}. For example \{berry, seed\}, \{game, hockey\}, and \{truck, vehicle\} received high relatedness scores, where \{hot, zombi\}, \{interior, mushroom\}, and \{bakery, zebra\} received low scores. 
To approximate human agreement, two annotators labeled all 3000 pairs on a 1-7 Likert scale; their Spearman correlation is 0.68, and the correlation of their average ratings with the general MEN scores is 0.84. 
%This high correlation suggests that MEN contains meaningful semantic ratings.
\item \textbf{WordSim353 (WS353)} contains 2000 word-pairs along with human-assigned association judgements \cite{bruni2014multimodal}. For example \{bank, money\}, \{Jerusalem, Israel\}, and \{Maradona, football\} received high scores whereas \{noon, string\}, \{sugar, approach\}, and \{professor, cucumber\} were ranked low. 
The authors report an inter-annotator agreement of 84\%. 

\item \textbf{SimLex} contains 1000 word-pairs and directly measures similarity, rather than relatedness or association \cite{hill-etal-2015-simlex}. The authors defined similarity as synonymy and instructed their annotators to rank accordingly. For example \{happy, glad\}, \{fee, payment\}, and \{wisdom, intelligence\} received high relatedness scores, where \{door, floor\}, \{trick, size\}, and \{old, new\} received low scores. 
Inter-rater agreement (the average of pairwise Spearman correlations between the ratings of all respondents) was reported as 0.67.

\end{itemize}

\begin{table}[]
    \centering
    \begin{tabular}{l|ccc} \toprule
         \diagbox[dir=NW]{{Method}}{{Dataset}} & {MEN} & {WS353} & {SimLex}\\ \midrule
         Cosine & {0.59} & 0.54 & 0.39 \\
         \makecell[l]{{\dedicationfont \textcolor{navyblue}{Word}}\\{{\dedicationfont \textcolor{navyblue}Confusion}}} & \textbf{0.66} &\textbf{0.67} & \textbf{0.44} \\ \bottomrule
    \end{tabular}
    \caption{Spearman's $\rho$ correlation between \wc and cosine similarity results as compared to humans. These three benchmarks focus on slightly different aspects of word similarity. We measure the correlation between human scores and cosine similarity between the language model embeddings versus {\wc}'s similarity scores. As can be seen, our method slightly outperforms cosine similarity.}
    \label{tab:semantic_similarity_results}
\end{table}

Across MEN, WS353, and SimLex, \wc slightly outperforms cosine similarity. This illustrates the meaningfulness of classification confusions, compared to cosine similarity. We note that our probability distribution spanned only the classes we chose in advance (all of the words in the dataset), which yields a different vocabulary compared to the original language model.



\section{Theoretical Intuitions}

In this section, we discuss the importance of word identifiability and how it enables the core mechanics of \wc, and discuss some theoretical differences between \wc and cosine similarity.  

\subsection{The Identifiability of Contextualized Word Embeddings}

\wc depends on the ability of a classifier to identify a word based on its contextual embedding; here we confirm that this classification task is indeed solvable, and examine some error cases to better understand it.

While contextualized word embeddings vary in their representation based on context, prior work showed that tokens of the same word still cluster together in geometric space \cite{zhou-etal-2022-problems}.

To test whether these boundaries are indeed learnable, we test how well a model can identify a contextualized word embedding after seeing one other example of the same word's contextualized embedding. We randomly sampled 26,000 words from English Wikipedia, trained 1000-class one-shot classifiers, and tested them on 10,000 examples (ten examples per class). Indeed, we found that the average test set accuracy on all our classifiers is 90\%, suggesting that the contextualized word embeddings are highly \textit{identifiable}. Thus, given an embedding, it is possible to identify its symbolic representation. See \ref{section:appendix_identity_probe} for experimental details.

\subsection{Differences Between \wc and Cosine Similarity}

\wc and cosine similarity give different kinds of distances.
We can see one way to visualize this in Figure~\ref{fig_concept_prob_boundaries}. Note the differences in the decision surface between {\wc} and cosine similarity: cosine boundaries emerge from the origin, whereas boundaries from \wc are not restricted in the same way.

Using a linear classifier in \wc also introduces new parameters that transform the input vectors into a different space, effectively redefining the notion of distance compared to the raw embeddings. To see this, consider two normalized 2-dimensional vectors, $x$ and $y$, and a real linear transformation, $A$ applied to each. Using the singular value decomposition (SVD) of $A= U \Sigma {V}^\intercal$, the singular values of $A$ (${\sigma}{u}{{v}^\intercal}$) allow us to rewrite the transformed vectors $Ax, Ay$ as ${\sigma_1}{u_1}{{v_1}^\intercal}{x_1} + {\sigma_2}{u_2}{{v_2}^\intercal}{x_2}$ and  ${\sigma_1}{u_1}{{v_1}^\intercal}{y_1} + {\sigma_2}{u_2}{{v_2}^\intercal}{y_2}$ respectively. 

The cosine distance between the transformed vectors is $1-({\sigma_1}^2({v_1}^{\intercal}x_1)({v_1}^{\intercal}y_1) + {\sigma_2}^2({v_2}^{\intercal}{x_2})({v_2}^{\intercal}{y_2}))$ compared to the original cosine distance $1 - (x_1y_1 + x_2y_2)$.\footnote{Terms cancel out as ${\sigma_1}{\sigma_1}=1$ and ${\sigma_1}{\sigma_2}=0$.} Similarly, the Euclidean distance between the transformed vectors is ${\sigma_1}{u_1v_1}^\intercal({x_1-y_1}) + {\sigma_2}{u_2v_2}^\intercal({x_2-y_2})$ compared to the original Euclidean distance of $(x_1-y_1)^2 + (x_2-y_2)^2$. In both cases, the distances between the two transformed vectors differ from the original vectors based on the linear transformation applied.

In other words, a linear transformation introduces additional parameters, allowing the model to reshape the geometry of word vectors and adjust the distances between words and their predicted semantic similarities.

Our method also shares some properties with cosine similarity.  Because linear classifiers learn a weight vector for each category that represents a kind of prototype of the category, the weight vectors learned by our classifier will be approximations of each word vector itself. Like cosine, our method thus computes similarity as the dot product between the word vector input and an errorful representation of the word vector encoded in the weights of the final classifier. What makes this approach effective is its reliance on small yet informative prediction errors that encode a meaningful signal, making the confusion matrix a source of linguistic insight.


% will geometry of the embeddings, giving a model additional parameters to represent targeted words and 
% we discuss differences between \wc and cosine similarity, arguing that feature-based similarity can produce more flexible decision boundaries, capture asymmetrical relations, highlight specific aspects of the analyzed word, and output more meaningful scores. 



\subsection{Advantageous Properties of \wc as a Similarity Measure}

Using a trainable linear classifier and analyzing its error signal for word-similarity purposes introduces a few advantages for measuring similarity:

\xhdr{Asymmetry} Human perceived similarity is not symmetric \cite{tversky1977features}. Yet cosine, like many distance functions commonly used to calculate semantic similarity, is symmetric. One of the advantages of using a model's confusion matrix for measuring semantic similarity is that these scores are \textit{asymmetric}; i.e., $p_{ij} \not= {p_{ji}}$. For example, \wc assigns lower probabilities for \textit{animal} being predicted as \textit{cat} than for \textit{cat} being predicted as \textit{animal}. The ability to measure asymmetric semantic similarity opens interesting new directions of understanding semantic similarity which are not possible with cosine.

\xhdr{Domain Adaptability} The fact that {\wc} requires training leads to more flexible similarity measures. Class selection enables measuring the semantic similarity of words relative to just a \textbf{subset} of features; we propose that this is particularly useful for practitioners who are interested in computing the similarity of words within a niche domain (we explore this in section \ref{sec:domain_spec}). 
% One limitation of our method is that it is self-supervised and would need to be retrained for out-of-domain words, which should still be quick given it is training a logistic regression 

\xhdr{Interpretability}
Probabilistic similarity measures have the advantage of being more interpretable for humans than non-probabilistic measures like cosine  \cite{sohangir2017improved}. Using a classifier's confusion matrix gives similarity scores that represent real probabilities. Moreover, since the choice of classifier classes is an implementation decision, one could choose them based on desired aspects of a word for a task. For example, we could interpret attitudes toward school by asking for the confusion matrix for the word ``school'' with a sentiment analysis classifier that contains the classes \{\textit{negative}, \textit{positive}\}, or the classes \{\textit{fun}, \textit{work}\}. 
\section{Real-World Data}
\label{sec:domain_spec}

\wc is a new similarity measuring tool that could assist in understanding real-world data and trends. In this section, we focus on two applications of {\wc} -- its ability to serve as a feature extractor and to detect temporal trends in word meaning.   

\subsection{\wc for Feature Classification}
\label{sec:validation_experiments}

\begin{table*}[h!]
    \centering
    \resizebox{\textwidth}{!}{%
    \begin{tabular}{lccccc} \toprule
    \textbf{Experiment} & \textbf{\wc} & \textbf{Cosine 1} & \textbf{Cosine 2} & \textbf{Cosine 3} & \textbf{Ave. Cosine} \\ \midrule
     Sentiment Classification & \textbf{0.79} & 0.75 & 0.71 & 0.84 & 0.73 \\
     Grammatical Gender (Italian) & \textbf{0.93} & 0.80 & 0.80 & 0.71 & 0.77 \\
     Grammatical Gender (French) & 0.85 & \textbf{0.86} & \textbf{0.86} & 0.83 & 0.85 \\
     ConceptNet Domain (Fashion-Gaming) & 0.90 & \textbf{0.93} & \textbf{0.93} & 0.90 & 0.92 \\
     ConceptNet Domain (Sea-Land Animals) & \textbf{0.83} & 0.79 & 0.80 & 0.61 & 0.73 \\
     \midrule
     Average & \textbf{0.86} & 0.83 & 0.82 & 0.76 & 0.80 \\ \bottomrule
    \end{tabular}
    }
    \caption{Macro-F1 for \wc and cosine similarity across a variety of feature classification tasks. We operationalize cosine similarity in three ways: 1) the distance between the centroids of the seed words and the target words 2) the average distance each of the target word to the centroid of the seed words 3) the average distance of each target word to each seed word (no centroids).}
    \label{tab:results}
\end{table*}


\wc can be used to define out-of-domain word classes, i.e. when $w_j \not\in W$. Using our earlier example, if the classes of \wc are the features \{\textit{positive}, \textit{negative}\}, given an out-of-domain word like \textit{school}, we can use the confusion matrix to represent the embedding for \textit{school} as a mixture of the classes the model is familiar with, i.e., \{\textit{positive}, \textit{negative}\}.

Following this intuition, we test whether \wc can use features as classes to identify objects' membership to these classes accurately. We used the following tasks: 

\noindent \textbf{Sentiment classification} using the NRC corpus \cite{pang-etal-2002-thumbs, mohammad-etal-2013-nrc}. The goal is to classify words according to their sentiment (either positive or negative). The words were manually annotated based on their emotional association (e.g., ``trophy'' is positive, whereas ``flu'' is negative).

\noindent \textbf{Grammatical gender} classification of nouns \cite{sahai-sharma-2021-predicting}. We tested \wc using two languages -- Italian and French. The goal is to classify words according to their grammatical gender per language. For example, ``flower'' is feminine in French and masculine in Italian. 

\noindent \textbf{Domain classification} using ConceptNet categories \cite{dalvi2022discovering}. The goal is to classify words to their correct ConceptNet class. We used two domain pairs: Fashion-Gaming is about classifying whether a word belongs to the fashion domain or the gaming domain; in Sea-Land, the goal is to predict if an animal is a sea or land animal.

For each task, we hand-select meaningful words as classes for the classifier and use terms from the lexicon as test embeddings. For example, for sentiment classification we first use the seed words \textit{positive} and \textit{negative} as our classes and collect occurrences from a corpus, extract the embeddings train the concept prober to recognize \textit{positive} and \textit{negative}. Finally, we then use \wc to classify all the terms in the NRC lexicon (our target words). We define the label using the class with the highest probability for the word. Details of each experiment are available in in the Appendix \ref{sec:seeds}.

Across all three tasks, we find that \wc is successful in feature-based classification using a few seed word training examples. Compared to cosine similarity, we achieve a macro-F1 of 86\% compared to 80\% (see table \ref{tab:results}). 

\xhdr{Embedding Meaning vs. Properties}
It is important to distinguish between embedding a word for its overall meaning (e.g., whether it conveys a positive or negative sentiment) versus embedding it to capture a specific property (e.g., gender, formality). While \wc supports both, this distinction is crucial when interpreting the results and determining the appropriate transformations for different tasks.


\subsection{What Is A Revolution?}

We now examine whether \wc could be used as a tool for studying concepts in a way that aids humanistic or social science investigations. By collaborating with the fourth author, a scholar of French literature and history, we investigate historical changes in the meaning of the French word \textit{``révolution''}. Together, we used \wc to test a prominent hypothesis of how the meaning of the word and concept of revolution changed \cite{baker_1990}: that the meaning of \textit{``révolution''} in the early years of the French Revolution was more associated with \textit{popular} action, but later become identified with \textit{state} actions.

We constructed a set of French words associated with the people (\{\textit{peuple}, \textit{populaire}, ...\}) and the state (\{\textit{conseil}, \textit{gouvernement}, ...\}).\footnote{Note on choice of seed words: we are tracking changes in the meaning of \textit{``révolution''} between 1789 and 1793 thus only looking at the vocabulary used during the French Revolution. Although the connection between \textit{``peuple''} and \textit{``révolution''} could be found before July 14, 1789, it is in the aftermath of that date that this connection became the primary one. Prior to this, the delegates of the National Assembly in Versailles had claimed they had been leading the \textit{``révolution''}.} These seed words were used as classes for our classifier, which we trained on different temporal segments to observe the temporal change in our concept of interest. Our corpora are the \textit{Archives Parlementaires}, transcripts of parliamentary speeches during a time that contains moments of both emancipation and elite control of political processes.\footnote{\url{https://sul-philologic.stanford.edu/philologic/archparl/}} The corpus contains 9,628 speeches and 54,460,150 words from the years 1789-1793. Within this corpus, the term \textit{``révolution''} appears 2,206 times across 218 speeches, with a contextual basis of 90,138 words.

\begin{figure}[h!]
    \includegraphics[width=.45\textwidth]{Images/pca_result.png}
    \caption{In 1789, the word \textit{``revolution''} was primarily associated with popular action (represented in orange). In 1792 \textit{``revolution''} was now also seen as something that the government should lead (represented in blue) found in the \textit{``counter-revolution''} cluster. In 1793, this new governmental meaning had spread back to the word \textit{``revolution''} itself.
    % \cnote{TODO: improve visualization. Some text is not readable. Add legend}
    }
    \label{fig:revolution}
\end{figure}


We color-code the classes orange as \textit{``peuple'' } (\textit{``the people''}) and blue as \textit{``gouvernement''} (\textit{``the state''}) and project the embeddings down to a 2-dimensional space and visualize the results (Figure \ref{fig:revolution}).

We find that, in 1789, the word \textit{``révolution''} was primarily associated with popular action, the most famous example of which was the storming of the Bastille. In 1792, another definition became common: \textit{``révolution''} was now also seen as something that the government should lead. 

Interestingly, the first use of \textit{``révolution''} to be associated with governmental action is in fact around the term \textit{``counter-revolution''}. The word embeddings of \textit{``counter-révolution''} are predicted to be associated with the state, indicating that it was primarily when talking about threats to, and enemies of, the revolution, that politicians suggested transferring more power to the state. Jumping forward to 1793, this new governmental meaning is predicted for the word embeddings of \textit{``révolution''} itself. Our findings suggest that the goal of repressing counter-revolutionaries is what associated the term \textit{``révolution''} with governmental action. In other words, once revolutionaries became more concerned about tracking down their enemies, they granted to the government the same kind of extra-legal power that had originally only been the prerogative of the people in arms.\footnote{While the proclamation of the republic and the introduction of the new calendar are related to the idea of revolution, the conceptual shifts that we've identified appears prior to both of these events. The revolutionary calendar was not introduced until October 1793, meanwhile declaration of the Republic occurred in September 1792. The emergence of \textit{``counterrevolution''} as a state problem predates these events, confirming that neither play a role in introducing the newer understanding of ``revolution'' as a state-driven process.}

Our findings are consistent with historians' hypothesis that the meaning of revolution in the early years of the French Revolution is most closely aligned with the concept of the people and this gradually shifts as the revolution continues \cite{sewell2005logics,edelstein2012we}. Furthermore, our model allows us to uncover a potential causal story for this shift in the meaning; that the state sense of revolution first actually started with counter-revolution. This is a novel discovery in our understanding of the French Revolution; future humanistic work should use other methods to confirm this proposed causal link to counter-revolutionaries.

 See  Appendix~\ref{sec:finance_experiment} for another, more preliminary  social scientific application of \wc, in this case to study financial trends.  
\section{Related Work on Cultural Change}
\label{sec:related_work}

%Understanding if and how distributional models understand semantic knowledge (e.g., is ``dog'' a mammal) is an important research question. For example, \citet{rubinstein-etal-2015-well} show that static distributional embeddings capture well taxonomical properties, but do not perform well in general attributive semantics (e.g., predicting the color of something). Recently, large language models were shown to \textit{have} some knowledge about concepts~\cite{dalvi2022discovering,ettinger2020bert,weir2020probing,petroni2019language}, including word sense in their contextualized embeddings~\cite{reif2019visualizing}. However, their encoded knowledge is static and lacks structure and domain specificity \cite{brandl}.

%\cnote{..}

%\paragraph{Word similarity.} Cosine similarity is a standard measure of semantic similarity, but its effectiveness is limited by the representational geometry of learned embeddings. The anisotropy of contextualized embedding spaces causes a small number of rogue dimensions to dominate cosine similarity computations \citep{timkey2021all}.
%Further, cosine similarity underestimates the semantic similarity of high-frequency words \citep{zhou2022problems}, heavily depends on the regularization techniques used during training \citep{steck2024cosine} and often fails in capturing human interpretation \cite{sitikhu2019comparison}. The proposed \wc enables a similarity measure that sidesteps these limitations via softmax-normalized dot products.



% % Recent work has explored the limits of cosine similarity


%\paragraph{Asymmetry.} By definition, cosine similarity is a symmetric metric that cannot capture the asymmetry of semantic relationships \citep{vilnis2014word}. Efforts to account for this caveat show partial successes, emphasizing the inherent symmetrical nature of cosine similarity using some language model embeddings \citet{zhang2021circles, rodriguez2020word}.

%\paragraph{Word embeddings for semantic and cultural change.} 
Both static and contextualized embedding spaces contain semantically meaning dimensions that align with high-level linguistic and cultural features \citep{bolukbasi2016man, DBLP:journals/corr/abs-1906-02715}. These embeddings have enabled a large number of quantitative analyses of temporal shifts in meaning and links to cultural or social scientific variables. For example early on, using static embeddings, \citet{hamilton2016cultural} measured linguistic drifts in global semantic space as well as cultural shifts in particular local semantic neighborhoods. \citet{garg2018word} demonstrated that changes in word embeddings correlated with demographic and occupation shifts through the 1900s.

Analyzes of contextualized embeddings have identified semantic axes based on pairs of ``seed words'' or ``poles'' \citep{soler2020bert, lucy2022discovering, grand2022semantic}. Across the temporal dimension, such axes can measure the evolution of gender and class \citep{kozlowski2019geometry}, internet slang \citep{keidar-etal-2022-slangvolution}, and more \citep{madani2023measuring, lyu2023representation, erk2024adjusting}. \citet{bravzinskas2017embedding} proposes a probabilistic measure for lexical similarity. 

It's also instructive to consider the similarity of our method  with tasks like word sense disambiguation (WSD) and named entity recognition (NER). The central idea behind \wc of mapping from embeddings to categories are also found in NER and WSD. What differs is the dynamic nature of the categories. Where NER focuses on pre-defined concept hierarchies and WSD on pre-defined senses per word,  \wc  focuses on a coherent but dynamic grouping of words that is interpretable for a given task.

% an important direction for future work in computational social science (see 3.1.10 in \citet{ziems2023can}).

% \subsection{Semantic change}


% Survey \citep{de2024survey} (should prob look more)

% Diachronic word embeddings 

%  \citet{di2019training}






% Similarly to named entity recognition, \ac{ourmethod} attempts to map words to classes that \textit{might be} types (such as ORGANIZATION or PLACE). However, \ac{ourmethod} is partially self-supervised and it is entirely user-driven.

% Similarly to semantic change detection, \ac{ourmethod} attempts to capture the usage of a word in different contexts. However, \ac{ourmethod} offers the possibility of defining the axis (the seed words) onto which the user wants to project words. 


\section{Discussion and Conclusion}
% In this paper, we reframe the task of semantic similarity from one of measuring distances to one of classification confusions. Our method is self-supervised and allows for researchers and downstream practitioners to measure the similarity of words with ease. We illustrate the performance of the identity probe on a number of word similarity tasks as well as provide examples of other tasks the identity probe can generalize to. In addition to the identity probe's high performance, we also find that the identity probe is able to produce asymmetric measures of similarity, opening a new line of future research. Lastly, we conclude with theoretical intuition of how a simple reframing of the problem results in new decision boundaries of similarity. We're excited for the NLP and broader community to use the identity probe as a measure of semantic similarity and discover other new ways in which the probe can be used for classification tasks.


In this paper, we reframe the task of semantic similarity from one of measuring distances to one of classification confusion. This formulation highlights the context-dependency of similarity judgments, meanwhile avoiding the pitfalls of geometric similarity measures \cite{evers2014revisiting}.

This new framing of semantic similarity in terms of classification confusion introduces new properties that are inspired by cognitive models of similarity \cite{tversky1977features} and accounts for the asymmetric nature of semantic similarity, captures different aspects of both similarity and multi-faceted words and ofter a measure that has interpretability benefits. 

Our proof-of-concept method, \wc, demonstrates the practical applicability and effectiveness of this reframing. Empirical results show that it outperforms cosine similarity on standard datasets.
For computational social science or cultural analytics applications, \wc can serve as a way to learn to represent words using target features (e.g., ``school'' in terms of \{\textit{positive}, \textit{negative}\}, and can be used to trace the meaning of a word as a function of time (like the word ``r\'{e}volution'').

The theoretical underpinnings of \wc allow it to learn complex word identity boundaries and capture the directional nature of similarity, offering a richer and more flexible framework for understanding word meanings. 

%To conclude, we reframe semantic similarity using classification confusion, to align it better with psychology literature. We implement a proof-of-concept framework that highlights the desired aspect(s) of a multi-faceted word by classifying it based on better-defined adjacent terms. 

While we implemented \wc as a linear classifier, the method naturally extends to capturing non-linear relationships among embedding components by replacing the linear projection with neural networks. Investigating whether the error function preserves its useful properties in non-linear settings remains an open question for future work.

While our experiments are preliminary and the space of possible similarity measures is enormous, we hope this reimagining of semantic similarity will inspire the development of new tools that better capture the multi-faceted and dynamic nature of language, advancing the fields of computational social science and cultural analytics and beyond.





\section*{Limitations}
Our proof-of-concept suggests a promising path where cosine similarity can be replaced by a more sophisticated method that involves self-supervision. However, the boost in performance also comes with some caveats. Because \wc is a supervised classifier, it requires an extra training step that simple cosine doesn't require.  Furthermore, potential users will need a basic understanding of model training and the pitfalls of over-fitting data.

As mentioned earlier, while we implemented \wc as a linear classifier, the method naturally extends to non-linear models. Additionally, various transformations commonly applied to embeddings before measuring distances \cite{Mu2018AllbuttheTopSA} can also be incorporated prior to using \wc, as our method relies on the resulting error signal to assess word similarity. Although non-linear models offer a promising direction, we have not yet examined whether the error function preserves its useful properties in such settings—an important avenue for future work. Introducing non-linearity into the classifier is known to alter its behavior in various ways, but its impact on confusion-based similarity remains uncertain. Further research is needed to evaluate its potential advantages and limitations.

Another key limitation of our approach is that we used three simple implementations of cosine similarity without exploring many possible augmentations to cosine, like normalizing it across the dataset (as was shown to be effective by \cite{timkey2021all}). Further refining both our classifier and cosine similarity implementations could lead to improved results for both,  as well as a deeper understanding of \wc.

Another important limitation of our analysis is that our results might be affected by the choice of seed words and the mechanisms on how we sample the ones used to represent the concepts. Changing seed words can impact the similarities. While we explored different sets of seed words without seeing drastic changes in results, a robust evaluation of the effect of different seed words should be considered in future work.

Lastly, we are not aware if changing the model used to create the embeddings can degrade the performance.

\section*{Ethics Statement}
As with all language technologies, there are a number of ethical concerns surrounding their usage and societal impact. It is likely that with this method, the biases known in contextualized embeddings can continue to propagate through downstream tasks, leading to representation or allocation harms. Additionally, the use of large language models for building contextualized embeddings is expensive and requires time and energy resources. To our knowledge, the method we have developed does not exacerbate any of these pre-existing ethical concerns but we recognize our work here also does not mitigate or avoid them.

\section*{Acknowledgments}
We thank Dallas Card, Nelson Liu, Kyle Hsu, Amelia Hardy, Kawin Ethayarajh, and Tianyi Zhang for their helpful feedback and discussion. This work was supported in part by the NSF via award number IIS-2128145, by the Hoffman–Yee Research Grants Program and the Stanford Institute for Human-Centered Artificial Intelligence, and by the Koret Foundation grant for Smart Cities and Digital Living.

% Entries for the entire Anthology, followed by custom entries
\bibliography{anthology,custom}
% \bibliographystyle{acl_natbib}

\appendix


\newpage
\appendix
\section{Applicability of SparseTransX for dense graphs} 
\label{A:density}
Even for fully dense graphs, our KGE computations remain highly sparse. This is because our SpMM leverages the incidence matrix for triplets, rather than the graph's adjacency matrix. In the paper, the sparse matrix $A \in \{-1,0,1\}^{M \times (N+R)}$ represents the triplets, where $N$ is the number of entities, $R$ is the number of relations, and $M$ is the number of triplets. This representation remains extremely sparse, as each row contains exactly three non-zero values (or two in the case of the "ht" representation). Hence, the sparsity of this formulation is independent of the graph's structure, ensuring computational efficiency even for dense graphs.

\section{Computational Complexity}
\label{A:complexity}
 For a sparse matrix $A$ with $m \times k$ having $nnz(A)=$ number of non zeros and dense matrix $X$ with $k \times n$ dimension, the computational complexity of the SpMM is $O(nnz(A) \cdot n)$ since there are a total of $nnz(A)$ number of dot products each involving $n$ components. Since our sparse matrix contains exactly three non-zeros in each row, $nnz(A) = 3m$. Therefore, the complexity of SpMM is $O(3m \cdot n)$ or $O(m \cdot n)$, meaning the complexity increases when triplet counts or embedding dimension is increased. Memory access pattern will change when the number of entities is increased and it will affect the runtime, but the algorithmic complexity will not be affected by the number of entities/relations.

\section{Applicability to Non-translational Models}
\label{A:non_trans}
Our paper focused on translational models using sparse operations, but the concept extends broadly to various other knowledge graph embedding (KGE) methods. Neural network-based models, which are inherently matrix-multiplication-based, can be seamlessly integrated into this framework. Additionally, models such as DistMult, ComplEx, and RotatE can be implemented with simple modifications to the SpMM operations. Implementing these KGE models requires modifying the addition and multiplication operators in SpMM, effectively changing the semiring that governs the multiplication.   

In the paper, the sparse matrix $A \in \{-1,0,1\}^{M \times (N+R)}$ represents the triplets, and the dense matrix $E \in \mathbb{R}^{(N+R) \times d}$ represents the embedding matrix, where $N$ is the number of entities, $R$ is the number of relations, and $M$ is the number of triplets. TransE’s score function, defined as $h + r - t$, is computed by multiplying $A$ and $E$ using an SpMM followed by the L2 norm. This operation can be generalized using a semiring-based SpMM model: $Z_{ij} = \bigoplus_{k=1}^{n} (A_{ik} \otimes E_{kj})$

Here, $\oplus$ represents the semiring addition operator, and $\otimes$ represents the semiring multiplication operator. For TransE, these operators correspond to standard arithmetic addition and multiplication, respectively.

\subsection*{DistMult} 
DistMult’s score function has the expression $h \odot r \odot t$. To adapt SpMM for this model, two key adjustments are required: The sparse matrix $A$ stores $+1$ at the positions corresponding to $h_{\text{idx}}$, $t_{\text{idx}}$, and $r_{\text{idx}}$. Both the semiring addition and multiplication operators are set to arithmetic multiplication. These changes enable the use of SpMM for the DistMult score function.

\subsection*{ComplEx} 
ComplEx’s score function has $h \odot r \odot \bar{t}$, where embeddings are stored as complex numbers (e.g., using PyTorch). In this case, the semiring operations are similar to DistMult, but with complex number multiplication replacing real number multiplication.

\subsection*{RotatE} 
RotatE’s score function has $h \odot r - t$. For this model, the semiring requires both arithmetic multiplication and subtraction for $\oplus$. With minor modifications to our SpMM implementation, the semiring addition operator can be adapted to compute $h \odot r - t$.

\subsection*{Support from other libraries}
Many existing libraries, such as GraphBLAS (Kimmerer, Raye, et al., 2024), Ginkgo (Anzt, Hartwig, et al., 2022), and Gunrock (Wang, Yangzihao, et al., 2017), already support custom semirings in SpMM. We can leverage C++ templates to extend support for KGE models with minimal effort.


\begin{figure*}[t]
\centering     %%% not \center
\includegraphics[width=\textwidth]{figures/all-eval.pdf}
\caption{Loss curve for sparse and non-sparse approach. Sparse approach eventually reaches the same loss value with similar Hits@10 test accuracy.}
\label{fig:loss_curve}
\end{figure*}

\section{Model Performance Evaluation and Convergence}
\label{A:eval}
SpTransX follows a slightly different loss curve (see Figure \ref{fig:loss_curve}) and eventually converges with the same loss as other non-sparse implementations such as TorchKGE. We test SpTransX with the WN18 dataset having embedding size 512 (128 for TransR and TransH due to memory limitation) and run 200-1000 epochs. We compute average Hits@10 of 9 runs with different initial seeds and a learning rate scheduler. The results are shown below. We find that Hits@10 is generally comparable to or better than the Hits@10 achieved by TorchKGE.

\begin{table}[h]
\centering
\caption{Average of 9 Hits@10 Accuracy for WN18 dataset}
\begin{tabular}{|c|c|c|}
\hline
\textbf{Model} & \textbf{TorchKGE} & \textbf{SpTransX} \\ \hline
TransE         & 0.79 ± 0.001700   & 0.79 ± 0.002667   \\ \hline
TransR         & 0.29 ± 0.005735   & 0.33 ± 0.006154   \\ \hline
TransH         & 0.76 ± 0.012285   & 0.79 ± 0.001832   \\ \hline
TorusE         & 0.73 ± 0.003258   & 0.73 ± 0.002780   \\ \hline
\end{tabular}
\label{table:perf_eval}
\end{table}

% We also plot the loss curve for different models in Figure \ref{fig:loss_curve}. We observe that the sparse approach follows a similar loss curve and eventually converges to the same final loss.

\section{Distributed SpTransX and Its Applicability to Large KGs}
\label{A:dist}
SpTransX framework includes several features to support distributed KGE training across multi-CPU, multi-GPU, and multi-node setups. Additionally, it incorporates modules for model and dataset streaming to handle massive datasets efficiently. 

Distributed SpTransX relies on PyTorch Distributed Data Parallel (DDP) and Fully Sharded Data Parallel (FSDP) support to distribute sparse computations across multiple GPUs. 

\begin{table}[h]
\centering
\caption{Average Time of 15 Epochs (seconds). Training time of TransE model with Freebase dataset (250M triplets, 77M entities. 74K relations, batch size 393K)  on 32 NVIDIA A100 GPUs. FSDP enables model training with larger embedding when DDP fails.}
\begin{tabular}{|p{2cm}|p{2.5cm}|p{2.5cm}|}
\hline
\textbf{Embedding Size} & \textbf{DDP (Distributed Data Parallel)} & \textbf{FSDP (Fully Sharded Data Parallel)} \\ \hline
16                      & 65.07 ± 1.641                            & 63.35 ± 1.258                               \\ \hline
20                      & Out of Memory                            & 96.44 ± 1.490                               \\ \hline
\end{tabular}
\end{table}

We run an experiment with a large-scale KG to showcase the performance of distributed SpTransX. Freebase (250M triplets, 77M entities. 74K relations, batch size 393K) dataset is trained using the TransE model on 32 NVIDIA A100 GPUs of NERSC using various distributed settings. SpTransX’s Streaming dataset module allows fetching only the necessary batch from the dataset and enables memory-efficient training. FSDP enables model training with larger embedding when DDP fails.

\section{Scaling and Communication Bottlenecks for Large KG Training}
\label{A:scaling}
Communication can be a significant bottleneck in distributed KGE training when using SpMM. However, by leveraging Distributed Data-Parallel (DDP) in PyTorch, we successfully scale distributed SpTransX to 64 NVIDIA A100 GPUs with reasonable efficiency. The training time for the COVID-19 dataset with 60,820 entities, 62 relations, and 1,032,939 triplets is in Table \ref{table:scaling}. 
% \vspace{-.3cm}
\begin{table}[h]
\centering
\caption{Scaling TransE model on COVID-19 dataset}
\begin{tabular}{|c|c|}
\hline
\textbf{Number of GPUs} & \textbf{500 epoch time (seconds)} \\ \hline
4                       & 706.38                            \\ \hline
8                       & 586.03                            \\ \hline
16                      & 340.00                               \\ \hline
32                      & 246.02                            \\ \hline
64                      & 179.95                            \\ \hline
\end{tabular}
\label{table:scaling}
\end{table}
% \vspace{-.2cm}
It indicates that communication is not a bottleneck up to 64 GPUs. If communication becomes a performance bottleneck at larger scales, we plan to explore alternative communication-reducing algorithms, including 2D and 3D matrix distribution techniques, which are known to minimize communication overhead at extreme scales. Additionally, we will incorporate model parallelism alongside data parallelism for large-scale knowledge graphs.

\section{Backpropagation of SpMM}
\label{A:backprop}
 Our main computational kernel is the sparse-dense matrix multiplication (SpMM). The computation of backpropagation of an SpMM w.r.t. the dense matrix is also another SpMM. To see how, let's consider the sparse-dense matrix multiplication $AX = C$ which is part of the training process. As long as the computational graph reduces to a single scaler loss $\mathfrak{L}$, it can be shown that $\frac{\partial C}{\partial X} = A^T$. Here, $X$ is the learnable parameter (embeddings), and $A$ is the sparse matrix. Since $A^T$ is also a sparse matrix and $\frac{\partial \mathfrak{L}}{\partial C}$ is a dense matrix, the computation $\frac{\partial \mathfrak{L}}{\partial X} = \frac{\partial C}{\partial X} \times \frac{\partial \mathfrak{L}}{\partial C} = A^T \times \frac{\partial \mathfrak{L}}{\partial C} $ is an SpMM. This means that both forward and backward propagation of our approach benefit from the efficiency of a high-performance SpMM.

\subsection*{Proof that $\frac{\partial C}{\partial X} = A^T$}
 To see why $\frac{\partial C}{\partial X} = A^T$ is used in the gradient calculation, we can consider the following small matrix multiplication without loss of generality.
\begin{align*}
A &= \begin{bmatrix}
a_1 & a_2 \\
a_3 & a_4
\end{bmatrix} \\ 
 X &= \begin{bmatrix}
x_1 & x_2 \\
x_3 & x_4
\end{bmatrix} \\
 C &=  \begin{bmatrix}
c_1 & c_2 \\
c_3 & c_4
\end{bmatrix}
\end{align*}
Where $C=AX$, thus-
\begin{align*}
c_1&=f(x_1, x_3) \\
c_2&=f(x_2, x_4) \\
c_3&=f(x_1, x_3) \\
c_4&=f(x_2, x_4) \\
\end{align*}
Therefore-
\begin{align*}
\frac{\partial \mathfrak{L}}{\partial x_1} &= \frac{\partial \mathfrak{L}}{\partial c_1} \times \frac{\partial c_1}{\partial x_1} + \frac{\partial \mathfrak{L}}{\partial c_2} \times \frac{\partial c_2}{\partial x_1} + \frac{\partial \mathfrak{L}}{\partial c_3} \times \frac{\partial c_3}{\partial x_1} + \frac{\partial \mathfrak{L}}{\partial c_4} \times \frac{\partial c_4}{\partial x_1}\\
&= \frac{\partial \mathfrak{L}}{\partial c_1} \times \frac{\partial \mathfrak{c_1}}{\partial x_1} + 0 + \frac{\partial \mathfrak{L}}{\partial c_3} \times \frac{\partial \mathfrak{c_3}}{\partial x_1} + 0\\
&= a_1 \times \frac{\partial \mathfrak{L}}{\partial c_1} + a_3 \times \frac{\partial \mathfrak{L}}{\partial c_3}\\
\end{align*}

Similarly-
\begin{align*}
\frac{\partial \mathfrak{L}}{\partial x_2}
&= a_1 \times \frac{\partial \mathfrak{L}}{\partial c_2} + a_3 \times \frac{\partial \mathfrak{L}}{\partial c_4}\\
\frac{\partial \mathfrak{L}}{\partial x_3}
&= a_2 \times \frac{\partial \mathfrak{L}}{\partial c_1} + a_4 \times \frac{\partial \mathfrak{L}}{\partial c_3}\\
\frac{\partial \mathfrak{L}}{\partial x_4}
&= a_2 \times \frac{\partial \mathfrak{L}}{\partial c_2} + a_4 \times \frac{\partial \mathfrak{L}}{\partial c_4}\\
\end{align*}
This can be expressed as a matrix equation in the following manner-
\begin{align*}
\frac{\partial \mathfrak{L}}{\partial X} &= \frac{\partial C}{\partial X} \times \frac{\partial \mathfrak{L}}{\partial C}\\
\implies \begin{bmatrix}
\frac{\partial \mathfrak{L}}{\partial x_1} & \frac{\partial \mathfrak{L}}{\partial x_2} \\
\frac{\partial \mathfrak{L}}{\partial x_3} & \frac{\partial \mathfrak{L}}{\partial x_4}
\end{bmatrix} &= \frac{\partial C}{\partial X} \times \begin{bmatrix}
\frac{\partial \mathfrak{L}}{\partial c_1} & \frac{\partial \mathfrak{L}}{\partial c_2} \\
\frac{\partial \mathfrak{L}}{\partial c_3} & \frac{\partial \mathfrak{L}}{\partial c_4}
\end{bmatrix}
\end{align*}
By comparing the individual partial derivatives computed earlier, we can say-

\begin{align*}
\begin{bmatrix}
\frac{\partial \mathfrak{L}}{\partial x_1} & \frac{\partial \mathfrak{L}}{\partial x_2} \\
\frac{\partial \mathfrak{L}}{\partial x_3} & \frac{\partial \mathfrak{L}}{\partial x_4}
\end{bmatrix} &= \begin{bmatrix}
a_1 & a_3 \\
a_2 & a_4
\end{bmatrix} \times \begin{bmatrix}
\frac{\partial \mathfrak{L}}{\partial c_1} & \frac{\partial \mathfrak{L}}{\partial c_2} \\
\frac{\partial \mathfrak{L}}{\partial c_3} & \frac{\partial \mathfrak{L}}{\partial c_4}
\end{bmatrix}\\
\implies \begin{bmatrix}
\frac{\partial \mathfrak{L}}{\partial x_1} & \frac{\partial \mathfrak{L}}{\partial x_2} \\
\frac{\partial \mathfrak{L}}{\partial x_3} & \frac{\partial \mathfrak{L}}{\partial x_4}
\end{bmatrix} &= A^T \times \begin{bmatrix}
\frac{\partial \mathfrak{L}}{\partial c_1} & \frac{\partial \mathfrak{L}}{\partial c_2} \\
\frac{\partial \mathfrak{L}}{\partial c_3} & \frac{\partial \mathfrak{L}}{\partial c_4}
\end{bmatrix}\\
\implies \frac{\partial \mathfrak{L}}{\partial X} &= A^T \times \frac{\partial \mathfrak{L}}{\partial C}\\
\therefore \frac{\partial C}{\partial X} &= A^T \qed
\end{align*}

\end{document}