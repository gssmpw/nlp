\begin{abstract}

Catastrophic forgetting (CF) poses a significant challenge in machine learning, where a model forgets previously learned information upon learning new tasks. 
Despite the advanced capabilities of Large Language Models (LLMs), they continue to face challenges with CF during continual learning. The majority of existing research focuses on analyzing forgetting patterns through a singular training sequence, thereby overlooking the intricate effects that diverse tasks have on model behavior.
Our study explores CF across various settings, discovering that model forgetting is influenced by both the specific training tasks and the models themselves. To this end, we interpret forgetting by examining the function vector (FV), a compact representation of functions in LLMs, offering a model-dependent indicator for the occurrence of CF. Through theoretical and empirical analyses, we demonstrated that CF in LLMs primarily stems from biases in function activation rather than the overwriting of task processing functions.
Leveraging these insights, we propose a novel function vector guided training methodology, incorporating a regularization technique to stabilize the FV and mitigate forgetting. Empirical tests on four benchmarks confirm the effectiveness of our proposed training method, substantiating our theoretical framework concerning CF and model function dynamics. We plan to make our code publicly accessible in the near future.

\end{abstract}