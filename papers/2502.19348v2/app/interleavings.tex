\section{Interleavings: Realistic Examples}%
\label{app:interleavings}

%\polish{The Verilog synthesis conventions is in itself a thorny topic. The once official Verilog synthesis standard is \polish{deprecated}. However, de facto.}

\paragraph{First realistic example} Consider the following variant of the \texttt{interleave3} module from Fig.~\ref{fig:interleave} where the module has been split into a synthesisable hardware model and a test bench:
%
\begin{figure}[h!]
\begin{minipage}[t]{0.41\textwidth}
\inputminted[fontsize=\footnotesize]{verilog}{app/interleaving.sv}
\end{minipage}
%
\hfill
%
\begin{minipage}[t]{0.58\textwidth}
\inputminted[fontsize=\footnotesize]{verilog}{app/interleaving_tb.sv}
\end{minipage}
\end{figure}

\noindent
If arbitrary interleavings are allowed, the module \texttt{interleaving\_tb} can break in the same way as the module \texttt{interleave3}. Say we have \texttt{a} = \texttt{b} = \texttt{c} = \iver{'x} at the end of the first time slot. After the nonblocking-assignment event has been executed in the second time slot, the two \iver{always_comb} blocks can interleave in the same incorrect way as the corresponding blocks can in \texttt{interleave3}.

%% To some extent, this more realistic example is still artificial because the first combinational block only contains two assignments to the same variable, making the first assignment redundant. 

%% However, this type of code occur in real-world code. E.g., the following code is example 4.5c ``case with defaults listed before case statement'' from Mills~\cite{Mills12} which illustrates a coding-style sometimes used to avoid inferring latches:
%% %
%% \begin{minted}[fontsize=\footnotesize]{verilog}
%% always_comb begin
%%  out1 = in1a;
%%  out2 = in2a;
%%  case (sel)
%%   cond2: out2 = in2b;
%%   cond3: out1 = in1c;
%%  endcase
%% end
%% \end{minted}
%% %
%% Note that both \texttt{out1} and \texttt{out2} are assigned multiple times in the block.

%Returning back to \texttt{interleaving} and \texttt{interleaving\_tb}: we have run the code in the five different Verilog simulators available at \url{https://edaplayground.com} and they all output the following:
%
%\begin{verbatim}
%a = xx, b = xx, c = xx
%a = 01, b = 10, c = 10
%a = 10, b = 11, c = 11
%\end{verbatim}

\paragraph{Second realistic example} Other realistic examples can be found in the test cases from the Icarus test suite that failed in Chen et al.'s evaluation (see in particular the file \texttt{scripts/data-race-cases.list} in their artefact~\cite{Chen23b}). One illustrative example is the test case \texttt{talu.v}, included below. With arbitrary interleavings, the test case can fail because the \iver{always} block in the \iver{alu} module depends on multiple data objects and the module can therefore, just like the modules \iver{interleave1} and \iver{interleave2} in Fig.~\ref{fig:interleave}, miss any number of writes to these data objects. %Therefore, Chen et al. have added delays between the writes to ensure that the hardware model runs to completion before the next update is applied.


%We can find an even more realistic example if we consider the list of test cases from Chen et al.'s , which blacklists 99 test cases from the  that do not work correctly under their semantics, specifically \texttt{talu.v}. The unmodified test case is as follows:
%
\inputminted[fontsize=\footnotesize]{verilog}{app/talu.v}

%REALISTIC 1: In contrast, when the modules are adapted for K-Verilog and $\lambda$-Verilog, states where $\mathtt{b} \neq \mathtt{c}$ are reachable in both semantics, which, as discussed in the main text, is not compatible with the semantics of hardware.

%To make this test case pass under their semantics, Chen et al. add the following synchronisation to the test bench:
%
%\inputminted[fontsize=\footnotesize]{verilog}{app/talu_mod.v}

%% Another way to put it: if arbitrary process interleavings are allowed, \iver{interleaving} can cause simulation-synthesis mismatches. To see this, we synthesise the \iver{interleaving} module using Yosys~\cite{Yosys} (version 0.36), resulting in the following circuit (using \texttt{read\_verilog}, \texttt{synth}, and \texttt{write\_verilog} with default settings):
%% %
%% \begin{minted}[fontsize=\footnotesize]{verilog}
%% module interleaving_netlist(
%%  input wire[1:0] a,
%%  output wire[1:0] b,
%%  output wire[1:0] c);

%%  assign b[0] = ~a[0];
%%  assign b[1] = a[1] ^ a[0];
%%  assign c = b;

%% endmodule
%% \end{minted}
%% %
%% For reasons unknown to use, K-Verilog does not return any reachable states for the synthesised module, but $\lambda$-Verilog no longer reports any states where

%% Meredith et al.'s semantics... Fewer interleavings since K semantics does not support x values and all variables are therefore initialised with 0. See \texttt{k-verilog/app\_interleaving.maude} for translation.
%% %
%% \begin{verbatim}
%% a = 00, b = 00, c = 00
%% a = 01, b = 10, c = 10
%% a = 10, b = 11, c = 00
%% \end{verbatim}

%% \begin{verbatim}
%% a = 00, b = 00, c = 00
%% a = 01, b = 10, c = 10
%% a = 10, b = 11, c = 11
%% \end{verbatim}

%% Chen et al.'s semantics \texttt{lv -ci interleaving.v -{}-seed=2 -o tmp.lv} gives \polish{explain seed}:
%% %
%% \begin{verbatim}
%% a = xx, b = xx, c = xx
%% a = 01, b = 10, c = 00
%% a = 10, b = 11, c = 11
%% \end{verbatim}
%% %
%% and (e.g.) \texttt{-{}-seed=3} gives:
%% %
%% \begin{verbatim}
%% a = xx, b = xx, c = xx
%% a = 01, b = 10, c = 10
%% a = 10, b = 11, c = 00
%% \end{verbatim}
%% %
%% Both seeds producing outputs where \iver{b} and \iver{c} differs.

%% However, if we simulate the synthesised module \texttt{interleaving\_netlist} with the same test bench \texttt{interleaving\_tb}, Chen et al.'s semantics now only outputs:
%% %
%% \begin{verbatim}
%% a = xx, b = xx, c = xx
%% a = 01, b = 10, c = 10
%% a = 10, b = 11, c = 11
%% \end{verbatim}
%% %
%% I.e., same result as the five simulators...

%To conclude, for this paper, we cannot do more than to point out the incompatibility between the interleaving semantics the standard suggest and the interleaving semantics assumed by current Verilog practice; this is because actually addressing the incompatibility will require discussions and agreement between the various stakeholders of Verilog. With that said, below we have summarised the interleaving choices made by a few existing Verilog tools and other relevant sources, which we hope will help inform further discussions on the topic:

%% \begin{description}
%% \item[The Verilog standard] \polish{From the very first Verilog standard~\cite{Verilog-1995} to the latest Verilog standard~\cite{SystemVerilog-2017}, the standards specify that procedural processes are allowed to preempt at any point during execution.}

%% \item[Mathematical formalisations of Verilog] As shown above, the semantics of Meredith et al. and Chen et al. both allow arbitrary preemption.

%% \item[Open-source simulators] Open-source simulators, by definition, allow us to inspect their source code and we can therefore easily know what implementation choices that have been made in those tools. The two main open-source simulators are Icarus~\cite{Icarus} and Verilator~\cite{Verilator}. However, the semantics that Verilator implements is closer to Verilog's synthesis semantics than Verilog's simulation semantics, and we therefore only consider Icarus here. Icarus does not preempt procedural processes, however, for simple enough continuous assignments, such continuous assignments are updated immediately, which is a limited form of preemption. The following example illustrates this:

%% \begin{minted}[fontsize=\footnotesize]{verilog}
%% module ex1;

%% logic i, o1, o2;

%% // Simple continuous assignment
%% assign o1 = i;

%% // Nonsimple continuous assignment
%% assign o2 = i + 1;

%% initial begin
%%  $display("i = %b, o1 = %b, o2 = %b", i, o1, o2);
%%  i = 1;
%%  $display("i = %b, o1 = %b, o2 = %b", i, o1, o2);
%% end

%% endmodule
%% \end{minted}

%% When run in Icarus, we get the following output:

%% \begin{verbatim}
%% i = x, o1 = x, o2 = x
%% i = 1, o1 = 1, o2 = x
%% \end{verbatim}

%% \item[Closed-source simulators] For closed-source simulators, we cannot get a definitive answer to how they preempt and interleave processes. We have run some small-scale systematic experiments with the close-source Verilog simulators available at \url{https://edaplayground.com}, and it \emph{appears} to be the case that processes are not interleaved except for corner cases. Below we present two examples of behaviour equivalent to process preemption that can be observed in two existing simulators.

%% The first example is for the simulator Aldec Riviera Pro 2022.04:

%% \begin{minted}[fontsize=\footnotesize]{verilog}
%% module ex1;

%% logic i, o;

%% buf (o, i);

%% initial begin
%%  $display("i = %b, o = %b", i, o);
%%  i = 1;
%%  $display("i = %b, o = %b", i, o);
%% end

%% endmodule
%% \end{minted}

%% The simulator outputs:

%% \begin{verbatim}
%% i = x, o = x
%% i = 1, o = 1
%% \end{verbatim}

%% The second example is for the simulator Synopsys VCS 2021.09. The following module shows that the simulator sometimes interleaves continuous assignments and \iver{initial} blocks:

%% \begin{minted}[fontsize=\footnotesize]{verilog}
%% module ex2;

%% logic i, o1, o2;

%% initial begin
%%  $display("i = %b, o1 = %b, o2 = %b", i, o1, o2);
%%  i = 1;
%%  $display("i = %b, o1 = %b, o2 = %b", i, o1, o2);
%% end

%% assign o1 = i;
%% assign o2 = i + 1;

%% endmodule
%% \end{minted}

%% For the above module, the simulator outputs:

%% \begin{verbatim}
%% i = x, o1 = x, o2 = x
%% i = 1, o1 = 1, o2 = 0
%% \end{verbatim}

%% However, note that for the same module with the \iver{initial} block replaced by an \iver{always_comb} block, the behaviour changes:

%% \begin{minted}[fontsize=\footnotesize]{verilog}
%% module ex3;

%% logic i, o1, o2;

%% always_comb begin
%%  $display("i = %b, o1 = %b, o2 = %b", i, o1, o2);
%%  i = 1;
%%  $display("i = %b, o1 = %b, o2 = %b", i, o1, o2);
%% end

%% assign o1 = i;
%% assign o2 = i + 1;

%% endmodule
%% \end{minted}

%% We now instead get the output:

%% \begin{verbatim}
%% i = x, o1 = x, o2 = x
%% i = 1, o1 = x, o2 = x
%% i = 1, o1 = 1, o2 = 0
%% i = 1, o1 = 1, o2 = 0
%% \end{verbatim}

%% I.e., in this case, the simulator does not interleave the two.

%% %\item[Similar languages] Gordon's theory paper and VHDL
%% \end{description}
