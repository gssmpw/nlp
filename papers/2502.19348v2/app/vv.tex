\section{VV}%
\label{app:vv}

\begin{figure}[b]
\begin{minipage}[t]{0.54\textwidth}
\begin{Verbatim}[fontsize=\small]
type value
 = BitTrue
 | BitFalse
 | BitX
 | BitZ

type rec event
 = EventContUpdate(int, value)
 | EventBlockUpdate(int, string, value)
 | EventNBA(string, value)
 | EventEvaluation(int)
 | EventDelayedEvaluation(int)
 | Events(array<event>)

type time_slot = {
 active: array<event>,
 inactive: array<event>,
 nba: array<event> }
\end{Verbatim}
\end{minipage}
%
\begin{minipage}[t]{0.45\textwidth}
\begin{Verbatim}[fontsize=\small]
type proc_running_state
 = ProcStateFinished
 | ProcStateRunning
 | ProcStateWaiting

type proc_state = {
 pc: int,
 state: proc_running_state }

type state = {
 // [...]
 env: Belt.Map.String.t<value>,
 proc_env: array<proc_state>,
 cont_env: array<value>,
 queue: array<(int, time_slot)>,
 monitor: option<(string, /* ... */)> }
\end{Verbatim}
\end{minipage}
\caption{The simulation-state representation \texttt{state} as implemented in VV. The code shown here is ReScript code~\cite{rescript}, an OCaml dialect of JavaScript, which is the language we have used to implement VV. We have made some small simplifications to the code shown here to make it easier to read. We mention two of the simplifications. First, the full \texttt{state} data type contains a few more fields not interesting enough to mention here and are therefore omitted in the presentation here (\texttt{// [...]}). Second, the actual \texttt{event} data type contains \texttt{event\_id}s as well, but they are only used for the GUI of the tool and do not~affect~any~behaviour.}%
\label{fig:state}                                
\end{figure}

In this appendix, we discuss additional implementation details of VV. We implemented VV from scratch because there is little overlap between existing simulators and VV. Existing simulators are batch tools designed for debugging real-world hardware designs, whereas VV is an interactive tool designed for visualising the Verilog's simulation semantics. Simulation speed is the main driving force behind the design of existing simulators, whereas for VV performance is largely unimportant. Even at the expense of performance, to be an accurate visualisation of the standard's description of Verilog's simulation semantics, VV must be in an as simple and direct correspondence with the Verilog standard as possible. E.g., the event queue maintained by VV must be exactly as described by the Verilog standard rather than implemented for performance. Moreover, in VV, the full behaviour of Verilog must be exposed, e.g., all event schedules must be exposed. In this respect, VV has more in common with explicit-state model checking than traditional simulators, except that VV is driven by human rather than machine.

%To be able to discuss the details of the simulation semantics, we introduce the simulation state representation used in our new visual simulation tool VV. We first discuss the structure of the simulation state of VV and then give some example executions of procedural processes to show how the structure is used in simulation.

% The state representation of VV is based on our own reading of the standard but is largely similar to the state representations of Meredith et al. and Chen et al.

\paragraph{Simulation-state representation} We now discuss the simulation-state representation of VV, which is the most interesting part of the implementation of VV. The standard defines the high-level structure of the Verilog event queue but does not make explicit how simulators and other Verilog tools should represent the rest of the simulation state. Fig.~\ref{fig:state} gives the top-level state structure of VV, called \texttt{state}, which contains: the state of all variables and nets (\texttt{env}), the state of all procedural processes (\texttt{proc\_env}), the state of all continuous assignments/drivers (\texttt{cont\_env}), the state of the event queue (\texttt{queue}), and, optionally, a monitor (\texttt{monitor}).

\paragraph{Process representation} Procedural processes, stored in the field \texttt{proc\_env}, are similar to software processes: the data type \texttt{proc\_state} therefore contains the current PC (``program counter'') and the current running state of the process (\texttt{proc\_running\_state}). The field \texttt{cont\_env} stores all continuous assignments: each continuous assignment induces a driver process that only needs to keep track of current value of the expression of the assignment.

\paragraph{Event-queue representation} Both the data type of the event queue field \texttt{queue} and the data type \texttt{time\_slot} can be directly read off the reference algorithm. The event queue field \texttt{queue} is represented by a series of time slots of type \texttt{time\_slot} indexed by the simulation time (an \texttt{int}) of the time slot. Each time slot (i.e., the type \texttt{time\_slot}) consists of the regions \texttt{active}, \texttt{inactive}, and \texttt{nba}. In our target subset of Verilog, no field for the observed region is needed in the time slot data type since only monitor invocations are scheduled in the observed region and monitors are only ever scheduled for all future time slots, not for a specific time slot: monitors are instead represented using the \texttt{monitor} field in the \texttt{state} data type.

\paragraph{Event representation} The reference algorithm does not dictate the exact structure of events, instead, it only mentions that there are two types of events (see the \texttt{execute\_region} function): ``update'' events and ``evaluation'' events. In VV, the two categories are refined into the \texttt{event} data type containing six event types. The numbers of event types and the structure of each event type are not canonical, instead, the two are largely a consequence of how other components of the interpreter are implemented; i.e., other interpreters/semantics might sensibly end up with other refinements. The event types \texttt{EventContUpdate} and \texttt{EventBlockUpdate}, respectively, represent a continuous assignment update and an update scheduled by a procedural blocking assignment, where the \texttt{int} in the event types is the index of the continuous assignment/procedural process. The two event types \texttt{EventEvaluation} and \texttt{EventDelayedEvaluation} both represent the start of execution of procedural processes, and are separated only because of a small (and not particularly interesting) edge case. The event types \texttt{EventNBA} and \texttt{Events} are used to represent nonblocking assignments: executing a nonblocking assignment schedules an \texttt{EventNBA} event in the relevant NBA region, and when the NBA events are later moved to the active region, following Meredith et al.'s trick to enforce \orderall (discussed in Sec.~\ref{sec:intramodule-problems}), the order between them is preserved by grouping them inside an \texttt{Events} event. The \texttt{nba} field of \texttt{time\_slot} will only ever contain \texttt{EventNBA} events and, similarly, \texttt{Events} events will only ever contain \texttt{EventNBA} events.

%We therefore do not discuss the event types in detail here (instead, we elaborate in the appropriate place: in section on VV, Sec.~\ref{sec:vv}). The event types relevant for the examples below are introduced in the examples.

%\paragraph{Event type representation} We complement our previous discussion on VV's simulation state representation by describing the six event types of VV (see Fig.~\ref{fig:state}), which we have refined from the two event types described by the reference algorithm for simulation.

%\polish{Regarding the problems of the standard discussed in Sec.~\ref{sec:two-problems}. The current version of VV does not preempt procedural processes (that is, VV implements the fix we suggest for \IP), maintains the order of all NBA events (that is, VV implements the order guarantee \orderall), and does not allow NBA events to mix with any other events (that is, VV implements the order~guarantee~\nomix).}
