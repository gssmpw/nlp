\section{Interleavings: Previous Work}%
\label{app:executions}

%\polish{say used the artefact of both papers. For Meredith we used Maude 3.3.1.}

Since we will be referring to the Verilog semantics of Chen et al.~\cite{Chen23} and Meredith et al.~\cite{Meredith10} repeatedly in this appendix, we refer to the two semantics as $\lambda$-Verilog and K-Verilog, respectively, in this appendix. Both semantics have state-explorer tools, allowing us to explore all reachable behaviour of the semantics. $\lambda$-Verilog can also be run as an interpreter, where we can supply a ``seed'' to control which schedule is used during execution. For modules that we run in $\lambda$-Verilog and K-Verilog, we must use \iver{always @(*)} instead of \iver{always_comb} because $\lambda$-Verilog and K-Verilog are formalisations of Verilog-2005, not SystemVerilog. Here, the only difference between \iver{always @(*)} and \iver{always_comb} is that the latter automatically runs once in the first time slot.

\paragraph{PREEMPT} We now show that the module \iver{interleave3} from Fig.~\ref{fig:interleave} in the main text has problematic interleavings in both $\lambda$-Verilog and K-Verilog. We use the following variant of the module:
%
\inputminted[fontsize=\small]{verilog}{app/interleave3_observable.sv}

\noindent
Running the $\lambda$-Verilog interpreter with seed 38 gives the following output:
%
\begin{verbatim}
> lv -ci interleave3_observable.v --seed=38 -o tmp.lv
a = 10, b = 01
\end{verbatim}
%
The state-space explorer of K-Verilog confirms that the same outcome is possible in the K-Verilog semantics by reporting the following two reachable outputs:
\begin{verbatim}
a = 10, b = 01
\end{verbatim}
and
\begin{verbatim}
a = 10, b = 10
\end{verbatim}

\paragraph{NB\_MIX} To show that both $\lambda$-Verilog and K-Verilog allow for NBA events to mix with other events, consider the following version of the module \texttt{nbinterleave2} from~Fig.~\ref{fig:nbproblem}:
%
\inputminted[fontsize=\small]{verilog}{app/nbinterleave2.v}

\noindent
$\lambda$-Verilog has interleavings of the NBA events and the display process:
%
\begin{verbatim}
> lv -ci nbinterleave2.v --seed=12 -o tmp.lv
a = 01
a = 10
\end{verbatim}
%
Similarly, the K-Verilog state-space explorer reports the following as reachable outputs:
%
\begin{verbatim}
a = 1
\end{verbatim}
and
\begin{verbatim}
a = 10
\end{verbatim}
and
\begin{verbatim}
a = 1
a = 10
\end{verbatim}

\paragraph{NB\_ORDER} Since both $\lambda$-Verilog and K-Verilog mix NBA events with other events, we can easily observe that none of them reorder NBA events. To demonstrate this, consider the following version of the module \texttt{nbinterleave3} from~Fig.~\ref{fig:nbproblem}:
%
\inputminted[fontsize=\small]{verilog}{app/nbinterleave3.v}

\noindent
The state-space explorer of $\lambda$-Verilog reports:
%
\begin{verbatim}
> lv -cx nbinterleave3.v -o tmp.lv | grep -v '^Heap'
a = 1, b = x
a = 1, b = x
a = 1, b = 1
a = 1, b = 1
a = 1, b = 1
a = 1, b = 1
\end{verbatim}
%
For K-Verilog, the state-space explorer reports the following three reachable output states:
%
\begin{verbatim}
a = 1, b = 0
\end{verbatim}
and
\begin{verbatim}
a = 1, b = 1
\end{verbatim}
and
\begin{verbatim}
a = 1, b = 0
a = 1, b = 1
\end{verbatim}
%
That is, for both semantics, \texttt{b} is never assigned before \texttt{a}.
