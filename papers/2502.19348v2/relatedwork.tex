\section{Related Work}
\label{sec:related-work}

%\polish{have not found an instance of when visualisation has been used to explore a semantics rather than as a tool build by people with a good understanding of the language to educate beginners.}

We discuss related work, including previous Verilog formalisations and visualisations.

% being a by-product of our visual investigation of the Verilog standard

% Their semantics does, however, not include support for X and Z values. Moreover, Meredith et al. have misunderstood the difference between variables and nets, and as a result do not include proper support for nets in their semantics. Thereby, they (unwittingly) do not take into consideration net-specific features such as multidriven nets. They do, on the other hand, consider arrays, which we do not include. 

\paragraph{Formalising language standards} Running into problems in language standards is not as uncommon as one might expect. In theory, a standard is the ultimate authority on the semantics of the language it standardises. In practice, less so -- as previous work exemplifies, formalising language standards is rarely as straightforward as theory suggests it should be. Memarian et al.~\cite{Memarian16,Memarian19}, in their work on formalising C, argue and give examples showing that the practice of C (what they call the ``de facto standards'' of C) and the ISO standard for C are out of sync with each other; or, in their own words: ``properties of C assumed by systems code and those implemented by compilers have diverged, both from the ISO standards and from each other, and none of these are clearly understood.'' To address this problem, they consult and balance an eclectic collection of sources of truth: the ISO standard, existing C code, experimental data from compilers, and survey and interview answers about C from systems programmers and compiler writers. These kinds of problems are not unique to C: Bodin et al.~\cite{Bodin14}, in their work on formalising JavaScript (arguably a much simpler language than C), to make sense of the semantics of JavaScript, had to consult not only the JavaScript standard but also browser implementations of JavaScript, discussion groups such as es-discuss, and the official ECMA test suite test262.

\paragraph{Verilog formalisations.} Chen et al.'s~\cite{Chen23} Verilog semantics is to date the most complete formalisation of the Verilog standard, specifically, the Verilog-2005 standard~\cite{Verilog-2005}. Chen et al. present their semantics by inference rules and also implement their semantics in Java, e.g., they have built a state-space explorer. Before Chen et al.'s semantics, the most complete formalisation was the formalisation by Meredith et al.~\cite{Meredith10}. Meredith et al.'s semantics is implemented in the K framework~\cite{Rosu10}, which allows for the generation of, among other tools, interpreters and model checkers. Gordon's~\cite{Gordon95} early work on Verilog semantics is another project of note. The project covers many important Verilog features, such as nonblocking assignments and delays. The presentation of Gordon's semantics is, however, informal (and, in places, nonstandard): the semantics is presented in prose form. Lastly, the work-in-progress paper by Lööw~\cite{Loow22b} reports on early work on the results presented in this paper.

 % (and furthermore, hence, is not executable).

% Nets are included as well, but restricted to a single driver.

%Meredith et al.'s target subset of Verilog is similar to our target subset of Verilog. 

%As we have discussed in this paper, there are issues in how they formalise the interleaving semantics of \iver{always} blocks and similar blocks. Chen et al. target a larger subset of Verilog than us since they are not, like us, foremost interested in the core concurrency semantics of synthesisable Verilog. As we have discussed in this paper, their semantics have problems with how to handle the interleaving semantics of \iver{always} blocks and similar blocks.

%Both Meredith et al.'s and Chen et al.'s semantics are executable. Meredith et al.'s semantics is executable because it is implemented in the K framework, . Chen et al. make their semantics executable by manually implementing it in Java.

%Another Verilog project that emphasises executability is Bowen et al.~\cite{Bowen00} (although instead referring to the same concept as their semantics being ``animatable''). They formalise Verilog in the logic programming language Prolog, enabling printing traces of Verilog executions. Their supported subset of Verilog is minimal: e.g., they do not discuss~features~such~as~nonblocking~assignments~or~nets.

Other previous projects on the semantics of Verilog we are aware of do not follow the standard as closely as Chen et al., Meredith et al., and Gordon. We consider those projects to be either suggestions for alternative semantics for Verilog (rather than formalisations of the standard semantics) or semantics derived from the standard semantics designed to aid formal reasoning. %\polish{Examples include e.g. Choi~\cite{Choi24}, AND ADD MORE HERE!!!}

%\paragraph{Verilog simulators.} Multiple Verilog simulators exist today, such as the simulators shipped with large commercial hardware development environments such as Xilinx Vivado~\cite{vivado} and the open-source simulator Icarus Verilog~\cite{Icarus} and Verilator~\cite{Verilator}.

\paragraph{Verilog visualisations.} Our new visual simulator VV visualises Verilog's simulation semantics, including the event queue of Verilog. To our best knowledge, no existing Verilog simulator allows for its event queue to be inspected, whereas in VV it is the main function of the tool.\footnote{Not even the Verilog APIs (PLI/VPI) defined by the standard that allow ``foreign language functions to access the internal data structures of a SystemVerilog simulation''~\cite[Ch.~36]{SystemVerilog-2023}, do, as far as we are aware, allow for such inspection.} Analysis/debugging facilities of existing simulators are designed with the aim to help its users to find and understand bugs in Verilog designs rather than visualising Verilog's simulation semantics. Common debugging facilities in existing simulators include ``printf debugging'' (e.g., \iverhack{$display} and \iverhack{$monitor}) and waveform visualisation.

Looking beyond simulators, we are not aware of any nonsimulator tool visualising Verilog's simulation semantics. For Verilog's synthesis semantics, Materzok~\cite{Materzok19} has developed DigitalJS, ``a visual Verilog simulator for teaching''. DigitalJS uses the synthesis tool Yosys~\cite{Yosys} for synthesising its Verilog input and visualises the synthesised output of Yosys. However, the visualisation does not explain the internals of the synthesis process -- the synthesis tool is still a black box for the user (i.e., it is only its final output that is visualised). Similar visualisations, although not interactive, come bundled with e.g. Yosys itself and other synthesis tools like Xilinx Vivado~\cite{vivado}. (On the topic on visualising synthesis algorithms, although not directly related to Verilog, Nestor~\cite{Nestor08} has implemented CADAPPLETS (later ported to CADApps) for visualising a selection of synthesis algorithms.)

%\polish{We do not include array values, which are important for real-world code but do not complicate the Verilog event queue compared to only supporting bit values. Not including arrays has allowed us to avoid a long series of minor nonconcurrency problems we are not directly interested in for the moment, such as the implicit resizing semantics of Verilog.}

%\footnote{See, e.g., the discussion on variables and nets in Sec.~4.1 of their paper, where they claim that ``[t]he distinction between nets and variables is an archaic part of the standard, no longer strictly necessary''. Moreover, in Sec.~4.3, they claim that ``[t]he best we can glean from the standard is that a net assignment should perform essentially as an \iver{always} block with one blocking assignment in it'', which is clearly not the case since variables and nets have completely different evaluation models.}

%They, moreover, do not address the two problems we raise in Sec.~\ref{sec:vv}; we illustrate the resulting problems exhibited by their semantics with test modules written for their semantics, see the \texttt{k-verilog} directory in the source code repository of VV. See also the modules under \texttt{xx\_more/Meredith/} in the drop-down menu with test modules in VV, which further discusses the questions for the Verilog community raised by Meredith et al. in their paper.

%The K framework is a toolbox potentially perfect for the experienced semanticist, but potentially less perfect for those uninitiated in the field of programming-language semantics; or, put in comparative language, we believe our Verilog visualisation to be more approachable than K and a useful complement for Verilog users less experienced in formal semantics.

%TODO: Now, consider Meredith et al.'s~\cite{Meredith10} Verilog formalisation, the most complete Verilog formalisation to date; they seem to have misunderstood the difference between variables and nets in Verilog (see Sec.~\ref{sec:related-work} for further comments on this). This highlights that a formal semantics and a correct semantics are separate concepts, and hence feedback is important.

%Lööw~\cite{Loow22b} discusses the problem we call the first major problem in Sec.~\ref{sec:vv}. The paper reports on work-in-progress, and the suggestions made in the paper are not incorporated into a semantics. Lööw does not take into account interleavings between procedural processes and continuous assignments.

% \polish{Fig.~\ref{fig:gtkwave} contains an example waveform output waveform.}

%Regardless, using or extending an existing simulator with support for visualising its event queue is not a sound approach to our work here. To force ourselves to scrutinise the various corner of the Verilog standard, we had no option but to build a new simulator from the ground up. Our work here is to debug the Verilog standard, and doing so is only possible by starting from zero and carefully and critically~reading~the~standard.

%\begin{figure}[t]
%\centering
%\includegraphics[width=\textwidth]{gtkwave.png}
%\caption{A screenshot of GTKWave with a run of \texttt{cirucit\_tb} from Fig.~\ref{fig:tbexample}. The waveform display shows the states of data objects over time.}%
%\label{fig:gtkwave}
%\end{figure}

%Whereas existing simulators need only to guarantee that the behaviour of the simulator is within the behaviour allowed by the Verilog standard, VV must allow its users to explore the all behaviour of Verilog.

%The design of VV is driven the aim to by provide accurate visualisations of all internals of executions of Verilog modules -- internals traditionally not exposed by Verilog simulators (in particular, the event queue).

%VV and traditional simulators have different design goals; specifically, the design goal of VV is to explain the semantics of Verilog whereas the design goal of traditional simulators is to allow testing and debugging of (real-world) hardware designs. The two goals are similar but not identical. As a result, the two types of simulators end up differing substantially. We highlight key differences along three dimensions: simulator internals, human-simulator interaction, and analysis~facilities.

%INTERACTION IS THE KEY DIFFERENCE!!!

%\paragraph{Simulator internals.} For traditional simulators, simulation speed is the main driving force behind the design of the simulator. In contrast, for VV, simulation speed is largely unimportant. Instead, the design of VV is driven the aim to by provide accurate visualisations of all internals of executions of Verilog modules -- internals traditionally not exposed by Verilog simulators (in particular, the event queue). The internals of VV must therefore be as conceptually/ontologically faithful to the standard as possible, even at the expense of efficiency.

%\paragraph{Human-simulator interaction.} Traditional simulators are batch tools, whereas VV is an interactive tool. To exemplify, recall that, due to concurrency, Verilog is a nondeterministic language. E.g., when there is a choice of multiple interleavings during execution, a traditional simulator will not explore all possible execution paths, but instead commit to one specific interleaving (which gives the best performance payoff) and ignore the remaining ones. In contrast, in VV it is up to the user to pick which particular interleaving to explore, enabling users to explore the full semantics of Verilog. (In this aspect, VV has more in common with explicit-state model checking than traditional simulators, except that  VV is driven by human rather than machine.)

%\begin{figure*}[t]
%\centering
%\includegraphics[angle=-90,width=0.9\textwidth]{examplegraph/graphfixed.pdf}
%\caption{Waveform of an Icarus Verilog simulation of example 12, a sequential pipeline register, from Cummings~\cite{Cummings00} rendered by GTKWave. A waveform allows for visualising both internal and external connections.}\label{fig:waveform}
%\end{figure*}

%\paragraph{Analysis facilities.} The analysis facilities offered by traditional simulators, in the form of debugging facilities, are designed to help its users to understand and find bugs in Verilog designs, whereas the analysis facilities offered by VV are designed for explaining the semantics of Verilog. Common debugging facilities in traditional simulators include ``printf debugging'' (e.g., \iverhack{$display} and \iverhack{$monitor}) and waveform visualisation -- see Fig.~\ref{fig:waveform} for an example waveform output from the simulator Icarus Verilog. Critically, in traditional simulators, the event queue cannot be inspected whereas in VV one of the main functions of the tool is to visualise the event queue.\footnote{Not even the Verilog APIs (PLI/VPI) defined by the standard that allow ``foreign language functions to access the internal data structures of a SystemVerilog simulation''~\cite[Ch.~36]{SystemVerilog-2017}, do, as far as we are aware, allow for such inspection.}

%In summary, because the different goals of visual simulation and traditional simulation, not much infrastructure between them can be shared, and we thus decided to do a small clean-slate implementation. WE NOW DISCUSS THESE DIFFERENCES.

%However, even if traditional simulators would provide a way to inspect their internal state, such access is not sufficient to explore the semantics of Verilog.

%EQV is required since our aim is completely different from conventional simulators, we aim to capture \emph{all} behaviours as allowed by the standard, simulator performance does not matter since, only readability/understandability of the code.

%However, the current range of simulators does not give a way to inspect the internal event queue during execution, the users are limited to execute the informally specified algorithm in their head. show how simulation trace looks like from some open source tool… ... say that you don’t see the internal queue…

%Like us, they omit support for multiple modules. 

%\paragraph{Other hardware languages} VHDL... Generator-based languages less of a problem? CIRCT... the LLVM thing. Supposedly cleaner languages, but they lack any written-down semantics at all!!!

%\paragraph{Other visualisations of programming languages and formal methods.} Although not a common form of formalisation, visualisation is not unheard of in previous programming-language-theory-related work; e.g., in previous work we find: visualisations of programming languages and paradigms with steep learning curves, e.g., Homer's~\cite{Homer22} work on visualising stack-based concatenative languages (which, as Homer remarks, has sometimes (humorously) been referred to as ``write-only'' languages), Greenberg and Blatt's~\cite{Greenberg19} Shtepper which visualises the execution of shell scripts, Eisenstadt and Brayshaw's~\cite{Eisenstadt88} Prolog visualisation work, and the lambda calculus visualisations collected by Pramod's~\cite{Pramod}; visualisations of difficult corners of mainstream languages, e.g., the visualisation tool Loupe~\cite{Philip14} which helps users ``understand how JavaScript's call stack/event loop/callback queue interact with each other'' and Cooper's~\cite{Cooper12} visualisations of coroutine event loops in JavaScript; and visualisations of distributed systems~\cite{Beschastnikh20}.

%In the formal-methods literature, one can find examples of visualisations of formal-method techniques (e.g., ribbon proofs for separation logic~\cite{Wickerson13} or graphical proof assistants, designed for education, such as Holbert~\cite{OConnor22}) and examples of providing understandable presentations of proof states in program-verification tools (e.g., the proof-state visualisations in KeY~\cite{Ahrendt16} and Iris'~proof~mode~\cite{Krebbers18,Krebbers17}).

%Concurrency adds another layer of complexity, and consequently more room for mistakes and mishaps in language standards. One example of work in this area is Boehm's argument that ``threads cannot be implemented as a library''~\cite{Boehm05}, a once common practice. A second example is given by Vafeiadis et al.~\cite{Vafeiadis15}, who show that common compiler optimisations are invalid under the C11 memory model, i.e., another case of the ISO standard of C being out of sync with practice. A third example is given by Alglave et al.~\cite{Alglave21}, who give a comprehensive review of the many twists and turns that have happened in the over decade long work on formal concurrency modelling of the Arm architecture. They review a great many numbers of papers with various and incompatible relationships to the standard by Arm, where suggestions put forward as reasonable options in one paper are judged not reasonable in the next paper, and so on. %Similarly, in another paper, this time on formal modelling of concurrency in Linux, Alglave et al.~\cite{Alglave18} highlights that concurrency in Linux is a contentious topic, whose documentation rather than being the pinnacle of clarity may be used to frighten small children (and also, they add, appears to be disconcerting to grown-ups).