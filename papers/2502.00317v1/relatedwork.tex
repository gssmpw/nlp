\section{Related Works}
\noindent
\textbf{$\bm{k}$-clique listing.}
Danisch et al. \cite{kClist} proposed the ordering-based framework for the $k$-clique listing problem, detailed in Section~II-A.
To construct a DAG of the input graph,~their algorithm \textsf{kClist} uses the degeneracy ordering.
They showed that their algorithm runs in $\mathcal{O}\big(k\cdot m\cdot (\delta(G)/2)^{k-2} \big)$ time, where $\delta(G)$ is the degeneracy of the input graph $G$.
They also proposed two strategies, namely \textsf{NodeParallel} and \textsf{EdgeParallel}, for parallelizing the ordering-based framework.
Li~et~al.~\cite{OrderingHeuristics} proposed two algorithms for the $k$-clique listing problem, namely \DDegree and \DDegCol, which use degree-ordering and color-ordering, respectively.
\DDegree initially generates a DAG based on the degeneracy ordering, but within subgraphs at the first level of the recursion, it reorders the vertices according to their degrees.
Similarly, \DDegCol applies greedy coloring within those subgraphs and reorders the vertices according to their color values.
The reordering enables pruning of the search space based on degree and color value, respectively.
Yuan~et~al.~\cite{SetIntersectionSpeedup} proposed two algorithms for the $k$-clique listing problem, namely \SDegree and \BitCol, along with preprocessing techniques for the input graph.
They focused mainly on speeding up set intersections for computing subgraphs induced by neighborhoods.
Recently, Wang et al. \cite{EBBkC} proposed an edge-oriented branching framework, in which the listing process invokes a recursive call for each edge, rather than for each vertex.
They also introduced an early termination technique to speed up the listing process within $t$-plexes \cite{plex}.

\noindent
\textbf{$\bm{k}$-clique counting.}
There has been a lot of research focused on counting or estimating the number of $k$-cliques.
Jain and Seshadhri \cite{jain2017fast} proposed a heuristic called \textsf{Tur{\'a}n-shadow}, which estimates the number of $k$-cliques based on Tur{\'a}n's theorem \cite{turan1941external}.
In a separate work, Jain and Seshadhri \cite{jain2020power} developed \textsf{Pivoter}, an algorithm that exactly counts $k$-cliques without actually enumerating all the cliques.
Recently, Ye et al. \cite{ye2022lightning} presented \textsf{DPColorPath}, a method that exactly counts the number of $k$-cliques in sparse regions of the graph while approximating the counts for dense regions of the graph via sampling. 
Note that methods for counting or estimating the number of $k$-cliques cannot be applied to listing $k$-cliques directly, since they either combinatorially count $k$-cliques without enumeration or samples only a portion of $k$-cliques.

\begin{algorithm}[t]
    \caption{The general framework for the $k$-clique listing problem \cite{kClist}}
    \label{alg:framework}
    \DontPrintSemicolon
    %\rm\sffamily
    \small

    \SetKwBlock{Begin}{function}{end function}

    \SetKwFunction{Listing}{\textsf{Listing}}

    \SetKwInOut{Input}{Input}
    \SetKwInOut{Output}{Output}

    \Input{\enspace a graph $G$ and an integer $k$}
    \Output{\enspace all the $k$-cliques in $G$}
    \smallskip
    Compute a DAG $\vec{G}$ of $G$ based on total ordering of vertices.\;\label{line:ordering}
    Invoke \Listing{$\vec{G}$, $k$, $\varnothing$}.\;\label{line:Listing}
    %\smallskip
    \Begin(\Listing{$\vec{G}$, $l$, $C$}){
        \eIf{$l = 2$}{
            \ForEach{$(u, v) \in E_{\vec{G}}$}{
                Report $C \cup \{ u, v \}$ as a $k$-clique.\;\label{line:basecase}
            }
        }{
            \ForEach{$u \in V_{\vec{G}}$}{
                \Listing{$\vec{G}_u$, $l-1$, $C \cup \{ u \}$}\;\label{line:recursion}
            }
        }
    }
\end{algorithm}