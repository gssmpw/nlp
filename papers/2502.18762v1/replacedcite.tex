\section{Related Work}
\label{sec:related}

\subsection{PTM based Continual Learning}
In recent years, pre-trained models (PTMs) have been widely utilized in offCL____. However, their application in onCL remains largely unexplored, partly because most existing methods heavily depend on task boundaries, i.e., explicit knowledge of when the task changes. This is often assumed in pair with \textit{clear} boundaries, meaning that all classes from the previous task are suddenly unavailable while all new classes encountered belong to the new task. With real-world streaming data, such a situation is equally unlikely.

\subsection{Online Continual Learning}
In onCL, incoming data can be seen only once, analogous to a continuous data stream____. Therefore, \textit{clear} boundaries are unlikely to be available and several studies suggest working in boundary-free scenarios____ where task change is unknown. However, if the change is \textit{clear}, it can easily be inferred. In that sense, \textit{blurry} boundary setting have been proposed____ in previous work. In particular, we are interested in the \textit{Si-Blurry} setting____ where not only task change is \textit{blurry}, but some classes can appear or disappear during multiple tasks, which brings the experimental setup one step closer to real-world scenarios while being more challenging. While numerous studies rely on prototypes for Continual Learning____, such representation-based methods must generally be combined with task boundary knowledge as prototypes are updated at the end of each task. In onCL, prototypes are harder to capitalize on when training a model from scratch due to the shift of representations hindering prototype computation____. However, when working with PTM, such a shift is drastically reduced as representations are already of high quality, making the usage of prototypes more efficient.

\subsection{Hypergradients and Gradient Re-Weighting}
Hypergradient____ addresses the problem of finding the optimal learning rate in conventional training scenarios. In that sense, the authors proposed to derive a gradient descent algorithm to learn the LR. Notably, they demonstrate that computing the dot product between gradients from previous steps $\nabla\mathcal{L}(\theta_t) \cdot \nabla\mathcal{L}(\theta_{t-1})$ is sufficient to complete one step of the learning rate update rule, with $t$ the index of the current step, $\theta$ the parameters, and $\mathcal{L}$ the loss function. However, such techniques have been, to the best of our knowledge, developed solely for offline scenarios at a global level. In Continual Learning, gradient re-weighting strategies have been designed for replay-based CL methods. Notably, previous work proposed to re-weight the gradient at the loss level to mitigate its accumulation during training in CL context, also called gradient imbalance____. Recently, to compensate for the class imbalance, class-wise manually defined weights in the last Fully Connected (FC) layer have been leveraged____. Our work on Class-Wise Hypergradients lies at the cross-road between Hypergradients and Gradient Re-Weighting.

% \subsection{Prototypes in Continual Learning}
% The usage of prototypes in Continual Learning is rather straightforward and can be found in numerous studies____. In offCL, such representation-based methods____ must be combined with task boundary knowledge as prototypes are updated at the end of each task. Prototypes in onCL are harder to capitalize on when training a model from scratch due to the shift of representations____.

% \begin{itemize}
%     \item Usage of Prototypes in CL is not new, especially offline. However, such strategies, often called representation-based strategies, heavily rely on task boundary information and the ability to recompute all representations between tasks. Such a procedure remains unsuited for online continual learning as task change cannot necessarily be inferred and fast adaptation is a priority.
% \end{itemize}