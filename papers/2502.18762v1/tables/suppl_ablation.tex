\begin{table}[t]
    \centering
    \caption{AP on CUB for CODA in the \textit{Si-Blurry} setting, with and without \textit{CWH}, with and without \textit{OP}. We report the average and standard deviation over 5 runs.}
    \resizebox{0.9\columnwidth}{!}{
    \begin{tabular}{l|llla} \multicolumn{1}{c}{Dataset} & \multicolumn{4}{c}{CUB} \\
    \midrule
\multicolumn{1}{c}{Learning Rate} & \multicolumn{1}{c}{$5 \times 10^{-5}$} & \multicolumn{1}{c}{$5 \times 10^{-4}$} & \multicolumn{1}{c}{$5 \times 10^{-3}$} & \multicolumn{1}{c}{Avg.} \\
\midrule\midrule
CODA                                          & 12.08{\scriptsize±1.31}                   & 58.4{\scriptsize±1.45}           & 44.42{\scriptsize±6.92}                       & 38.30{\scriptsize±23.75} \\  
\ \  $\hookrightarrow$ + CWH                  & 17.08{\scriptsize±1.5}                     & 59.17{\scriptsize±2.29}           & 43.52{\scriptsize±3.73}                     & 39.93{\scriptsize±21.27} \\
\ \  $\hookrightarrow$ + OP                  & 37.31{\scriptsize±1.53}  & 77.47{\scriptsize±1.36}	           & 59.78{\scriptsize±4.17}                     &  58.18{\scriptsize±20.13}  \\
\ \  $\hookrightarrow$ + CWH + OP          & 50.01{\scriptsize±1.35}  & 78.49{\scriptsize±1.43}   & 	48.23{\scriptsize±9.24}   & \textbf{58.91{\scriptsize±16.99}} \\
\bottomrule
    \end{tabular}
    \label{tab:suppl_ablation}}
\end{table}
