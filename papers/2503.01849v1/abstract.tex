
\begin{quote}  
{\bf Abstract}\\  
%$\chi$-parameter, also known as Flory-Huggins interaction parameter, 
%is a very important property that characterizes the ..... between a solute and a solvent.
%Thus, understanding the relationship between the solute-solvent pairs and $\chi$-parameter values
%is very important...,
%and designing solute-solvent pairs with some desired $\chi$-parameter value is ...
%However, very few works on this have been done.
%In this paper, we propose a novel inverse QSAR framework for the inference task of solute-solvent pairs
%based on machine learning and mixed integer linear programming for the first time on $\chi$-parameter.
%We designed a feature vector on the solute-solvent-temperature triplet and applied several machine learning
%methods on it.
%Our experimental results demonstrate that our approach can achieve a competitively high learning performance
%compared with existing works while inferring the solute polymers with up to 50 non-hydrogen atoms in a monomer form in a relatively
%short time.
%Many existing approaches of QSAR/QSPR and inverse QSAR/QSPR are 
%developed under the assumption that a property value is determined by just one molecule.
%However, some important chemical properties are determined by two or more molecules and vary on the environment when it is measured.
%In this paper, we propose a novel QSAR/QSPR and inverse QSAR/QSPR model to include such cases, i.e. multiple molecules (monomers or polymers) 
%and the experimental environment 
%contribute to one property value.
Traditional QSAR/QSPR and inverse QSAR/QSPR methods often assume that chemical properties
are dictated by single molecules, overlooking the influence of molecular interactions and environmental factors.
In this paper, we introduce a novel QSAR/QSPR framework that can capture the combined effects
of multiple molecules (e.g., small molecules or polymers)
and experimental conditions on property values.
We design a feature function to integrate the information of multiple molecules and the environment.
Specifically, for the property Flory-Huggins $\chi$-parameter, 
which characterizes the thermodynamic properties between
the solute and the solvent, 
and varies in temperatures, we demonstrate through computational experimental results that
our approach can achieve a competitively high learning performance
compared to existing works on predicting $\chi$-parameter values, while inferring the solute polymers with up to 50 non-hydrogen 
atoms in their monomer forms in a relatively short time.
A comparison study with the simulation software \jocta\ demonstrates that the polymers inferred by our methods are of high quality.


\noindent 
{\bf Keywords: } Cheminformatics, Materials Informatics, 
Machine Learning,  Integer Programming,
Molecular Design, QSAR/QSPR,  Flory-Huggins $\chi$-parameter. 

% QSAR/QSPR,  Molecular Design, 
%    Artificial Neural Network, Mixed  Integer Linear Programming, 
%   Enumeration of Graphs  

% \noindent {\bf  Mathematics Subject Classification: } 
% Primary  
% 05C92,  % Chemical graph theory 
% 92E10, % Molecular structure (graph-theoretic methods, methods of differential topology, etc.)
% Secondary
% 05C30, % Enumeration in graph theory
% 68T07, % Artificial neural networks and deep learning
% 90C11,  % Mixed integer programming
% 92-04 %Software, source code, etc. for problems pertaining to biology
% 05C85 Graph algorithms 
\end{quote} 