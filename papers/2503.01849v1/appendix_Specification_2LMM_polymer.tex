
\section{Specifying Target Chemical Graphs}\label{sec:specification} 


Our definition of the topological specification is analogous to the one  by 
 Zhu~et~al.~\cite{Zhu:2022ad}.
 Here we review the one particularly modified for polymers proposed by~Ido~et~al.~\cite{Ido:2024aa}.
   

\subsection*{Seed Graph}%%%%%%%%%%%%%%%%%%%%

A  {\em seed graph} for a polymer  is defined
to be a graph $\GC=(\VC,\EC)$  with a specified edge subset $\EC^\lnk$
 such that 
the edge set $\EC$ consists of four sets 
$\Et$, $\Ew$, $\Ez$ and $\Eew$, 
where each of them can be empty, and
 $\EC^\lnk$ is a circular  set in $\GC$ such that 
  $\emptyset\neq \EC^\lnk\subseteq \Et\cup \Ew\cup \Eew$ (only for polymer). 
  %\newone 
Figure~5(a) %\ref{fig:specification_example_1}(a)
 illustrates an example of a seed graph,
where $\VC=\{u_1,u_2,\ldots,u_{14}\}$, 
$\Et=\{a_1,a_2,a_3,a_4\}$, 
$\Ew=\{a_5,a_6,\ldots,a_9\}$,
$\Ez=\{a_{10}\}$,
$\Eew=\{a_{11},a_{12},\ldots,a_{18}\}$ and 
$\EC^\lnk=\{a_1,a_2\}$.

 A {\em subdivision} $S$ of $\GC$  
is a graph constructed from a seed graph $\GC$ 
according to the following rules:
\begin{enumerate}[leftmargin=*]
\item[-]
Each edge $e=uv\in \Et$ is replaced
with a $u,v$-path $P_e$ of length at least 2;

\item[-] 
Each edge $e=uv\in \Ew$ is replaced
with a $u,v$-path $P_e$ of length at least 1
(equivalently $e$ is directly used or replaced with
a $u,v$-path $P_e$ of length at least 2);

\item[-] 
Each edge $e\in \Ez$ is either used or discarded;   and

\item[-]
Each edge $e\in \Eew$ is always used directly.
\end{enumerate}

The set of link-edges in the monomer representation  $\C$ of 
an inferred polymer 
consists of edges in $\EC^\lnk\cap( \Eew\cup \Ew)$
or edges  in   paths $P_e$ for all edges $e=uv\in \EC^\lnk\cap (\Ew\cup \Et)$
in a  subdivision  $S$ of $\GC$. 
 
A target chemical graph $\C=(H,\alpha,\beta)$ will contain  $S$  as a subgraph
of the interior $H^\inte$ of $\C$.

% \clearpage 

\subsection*{Interior-specification}%%%%%%

A graph $H^*$ that serves as the interior $H^\inte$ of
a target chemical graph $\C$ will be constructed as follows.
First construct a subdivision  $S$ of a seed graph $\GC$ 
by replacing each edge $e=u u'\in \Et\cup\Ew$
with a pure $u,u'$-path $P_e$.
Next construct a supergraph $H^*$ of $S$ by 
attaching a leaf path $Q_v$ at each vertex $v\in \VC$ or
at an internal vertex $v\in V(P_e)\setminus\{u,u'\}$ 
of each pure $u,u'$-path $P_e$ for some edge $e=uu'\in \Et\cup\Ew$,
where possibly $Q_v=(v), E(Q_v)=\emptyset$ 
(i.e., we do not attach any new edges to $v$).
%
We introduce the following rules for specifying
 the size of $H^*$, the length $|E(P_e)|$  of
a pure path  $P_e$,  the length $|E(Q_v)|$ of
a   leaf path $Q_v$, the number of  leaf paths $Q_v$
and a bond-multiplicity of each interior-edge,
where we call the set of prescribed constants  
 an  {\em interior-specification}   $\sint$: 
 %
\begin{enumerate}[leftmargin=*]
 \item[-]
  Lower and upper bounds $\nint_\LB, \nint_\UB\in \Z_+$ 
  on   the number of interior-vertices 
of a target chemical graph~$\C$. 

 \item[-]
  Lower and upper bounds $\nlnk_\LB, \nlnk_\UB\in \Z_+$ 
  on   the number of link-edges 
of a target chemical graph~$\C$ (only for polymer). 
  
\item[-] 
For each edge $e=u u'\in \Et\cup\Ew$, 
\begin{description} 
\item[]
 a lower bound $\ell_{\LB}(e)$ and 
 an upper bound $\ell_{\UB}(e)$  on the length $|E(P_e)|$ of
 a pure $u,u'$-path $P_e$. 
(For a notational convenience, set 
$\ell_\LB(e):=0$, $\ell_\UB(e):=1$, $e\in \Ez$ and
$\ell_\LB(e):=1$, $\ell_\UB(e):=1$, $e\in \Eew$.)
   
\item[]  
 a lower bound $\bl_{\LB}(e)$ and 
 an upper bound $\bl_{\UB}(e)$ on the number of leaf paths $Q_v$ attached 
 at  internal vertices $v$ of a pure $u,u'$-path $P_e$.   

\item[] 
 a lower bound $\ch_{\LB}(e)$ and 
 an upper bound $\ch_{\UB}(e)$  on the maximum 
 length  $|E(Q_v)|$ of a leaf path $Q_v$ attached  
 at an internal vertex $v\in V(P_e)\setminus\{u,u'\}$ 
 of a pure $u,u'$-path $P_e$.   
\end{description} 

\item[-]
For each vertex $v\in \VC$, 
\begin{description} 
\item[]
 a lower bound $\ch_{\LB}(v)$ and 
 an upper bound $\ch_{\UB}(v)$  on  
 the number of leaf paths $Q_v$ attached to $v$,
 where $0\leq \ch_{\LB}(v)\leq \ch_{\UB}(v)\leq 1$.
 
\item[]
 a lower bound $\ch_{\LB}(v)$ and 
 an upper bound $\ch_{\UB}(v)$  on the
 length $|E(Q_v)|$ of a leaf path $Q_v$ attached to $v$. 
\end{description}  

\item[-] 
For each edge $e=u u'\in \EC$, 
a lower bound $\bd_{m, \LB}(e)$ 
and an  upper bound $\bd_{m, \UB}(e)$  on
the number of edges with bond-multiplicity $m\in [2,3]$ in
$u,u'$-path $P_e$, where we regard $P_e$, $e  \in \Ez\cup \Eew$ 
as single edge $e$.
\end{enumerate}

We call a graph $H^*$ that satisfies an interior-specification $\sint$
a {\em $\sint$-extension of $\GC$}, 
where the bond-multiplicity of each edge has been determined. 
 

Table~\ref{table:interior-spec}  shows an example of 
an interior-specification  $\sint$ to the seed graph  $\GC$ in 
Figure~5(a). %\ref{fig:specification_example_1}(a)  

\begin{table}[h!]\caption{Example~1 of an interior-specification  $\sint$. }
% \begin{center}
\begin{tabular}{ |  c | c | c | c |  } \hline 
$\nint_\LB=20$ & $\nint_\UB = 30$ & 
$\nlnk_\LB=2$ & $\nlnk_\UB = 24$ \\\hline 
\end{tabular}

 
 \begin{tabular}{ |  c | c c c c c c c c c |  } \hline
              & $a_1$ &  $a_2$ &   $a_3$ &   $a_4$ &   $a_5$ &   $a_6$ &   $a_7$ &   $a_8$  &   $a_9$   \\\hline
 $\ell_\LB(a_i)$ &  2 &  4 &  3 & 2 &  2 &  1  &  1 &  1 &   1\\ \hline
 %                          2     4     3     2     2      2      2      2      2
 $\ell_\UB(a_i)$ &  3 & 6 &  6 & 5 &  3 &  3  &  6 &  2 &   6 \\\hline
 %
 $\bl_\LB(a_i)$ &   0 &  1 & 1 & 0 &  0 &   0 &  0 &   0 &  0 \\ \hline
 %                          0     1     1     0     0      0      0      0      0
 $\bl_\UB(a_i)$ &  1 &  4 &  4 & 3 &  2 &   1 &  1 &  1  &  1 \\\hline
 %
 $\ch_\LB(a_i)$ &   0 &  2 &  1 & 0 &  0 &  0 &  0 &   0 &   0 \\ \hline
 %                           0     2     1     0     0      0      0      0      0
 $\ch_\UB(a_i)$ &  3 &  6 &  6 & 3 &  3 &   3 &  3 &   0 &   0 \\\hline
\end{tabular} 

\begin{tabular}{ |  c | c c c c c c   c c c c  c c c  c|  } \hline
                        & $u_1$ &  $u_2$ &   $u_3$ &   $u_4$ &   $u_5$ &   $u_6$ 
                       & $u_7$ &   $u_8$ &   $u_9$ &   $u_{10}$ &   $u_{11}$ 
                       &   $u_{12}$ &   $u_{13}$ &   $u_{14}$ \\\hline 
 $\bl_\LB(u_i)$ &  0 & 0 & 0 & 0 & 0 &  0 & 0 & 0 & 1 & 0 & 0 & 0 & 0 & 0 \\ \hline
 %                         0     0    0     0    0     0     0    0    1     0     0    0    0    0
 $\bl_\UB(u_i)$&  1 & 1 & 1 & 1 & 1 &  1 & 1 & 1 & 1 & 1 & 1 & 1 & 1 & 1 \\\hline
 $\ch_\LB(u_i)$&  0 & 0 & 0 & 0 & 0 &  0 & 0 & 0 & 1 & 0 & 0 & 0 & 0 & 0 \\\hline
 %                         0     0    0     0    0     0     0    0    1     0     0    0    0    1
 $\ch_\UB(u_i)$& 4 & 4 & 4 & 4 & 4 &  4 & 4 & 4 & 6 & 4 & 4 & 4 & 4 & 4 \\\hline
\end{tabular} 

\begin{tabular}{ |  c | c c c c c c   c c c c c c  c c c c c c |  } \hline
                               & $a_1$ &  $a_2$ &   $a_3$ &   $a_4$ &   $a_5$ &   $a_6$ 
                               & $a_7$ &  $a_8$ &   $a_9$ &   $a_{10}$ &   $a_{11}$ &   $a_{12}$ 
                               & $a_{13}$ &   $a_{14}$ &   $a_{15}$ &   $a_{16}$   &   $a_{17}$ &   $a_{18}$
                                    \\\hline
 $\bd_{2, \LB}(a_i)$ &  0 & 0 & 0 & 0 & 0 &  0 & 0 & 0 & 0 & 0 & 0 & 0 & 0 & 1  & 0 & 0 & 0 & 0\\ \hline 
 %                                0    0    0     0     0     0     0    0     0    0     0    0    0    1      0    0    0     0
 $ \bd_{2, \UB}(a_i)$& 1 & 2 & 1 & 1 & 1 &  1 & 1 & 1 & 1 & 1 & 1 &  1 & 1 & 1 & 1 & 1 &  1 & 1\\ \hline
 $\bd_{3, \LB}(a_i)$ &  0 & 0 & 0 & 0 & 0 &  0 & 0 & 0 & 0 & 0 & 0 & 0 & 0 & 0  & 0 & 0 & 0 & 0\\ \hline
 %                                0    0    0     0     0     0     0    0     0    0     0    0    0    0      0    0    0     0
 $ \bd_{3, \UB}(a_i)$& 1 & 1 & 1 & 1 & 1 &  1 & 1 & 1 & 1 & 1 & 1 &  1 & 1 & 1 & 1 &  1 & 1 & 1\\ \hline
\end{tabular} %\end{center}
\label{table:interior-spec}  
\end{table}
   
   
Figure~6 %\ref{fig:test_subgraph_polymer} 
illustrates an example of 
an $\sint$-extension $H^*$ of seed graph  $\GC$ in 
Figure~5(a) %\ref{fig:specification_example_1}(a)
under the interior-specification  $\sint$ in 
Table~\ref{table:interior-spec}.  
 


%\clearpage 

\subsection*{Chemical-specification}%%%%%%
 
 Let $H^*$ be a graph that serves as 
 the interior $H^\inte$ of a target chemical graph $\C$,
 where the bond-multiplicity of each edge in $H^*$ has be determined.
 % 
 Finally we introduce a set of rules for constructing 
   a target chemical graph $\C$ from $H^*$ 
   by choosing  a chemical element $\ta\in \Lambda$ 
and assigning a ${\rho}$-fringe-tree $\psi$
 to each interior-vertex $v\in V^\inte$. 
%
We introduce the following rules for specifying
the size of $\C$, a set of chemical rooted trees  
that are allowed to use as  ${\rho}$-fringe-trees 
and lower and upper bounds on the frequency of
a chemical element, a chemical symbol, % an adjacency-configuration
 an edge-configuration, and a fringe-configuration
where we call the set of prescribed constants   
 a  {\em chemical specification} $\sce$.
 Notice that the ones involving link-edges and connecting-vertices are only used for the inference of polymers.
 
\begin{enumerate}[leftmargin=*]
\item[-] 
Lower and upper bounds $n_\LB,  n^*\in \Z_+$
on the number of vertices, where $\nint_\LB \leq n_\LB\leq n^*$.
 
\item[-] 
A subset $\mathcal{F}^* \subseteq \mathcal{F}(D_\pi)$  
 of chemical rooted trees $\psi$ with $\h(\anpsi)\leq {\rho}$, where 
 we require that 
 every ${\rho}$-fringe-tree $\C[v]$ rooted at an interior-vertex $v$ 
    in  $\C$  belongs to $\mathcal{F}^*$.  
Let   
$\Lambda^\ex$ denote the set of  chemical elements assigned to non-root
vertices over all chemical rooted trees in $\mathcal{F}^*$.  
 
\item[-] 
A subset  $\Lambda^\inte\subseteq \Lambda^\inte(D_\pi)$, where 
 we require that every chemical element $\alpha(v)$ 
 assigned to an interior-vertex  $v$ in $\C$ belongs to $\Lambda^\inte$.
%
Let $\Lambda:= \Lambda^\inte\cup \Lambda^\ex$ and
 $\na_\ta(\C)$ (resp., $\na_\ta^\inte(\C)$ and $\na_\ta^\ex(\C)$) 
 denote the number of vertices   (resp.,   interior-vertices and  exterior-vertices)
  $v$ such that $\alpha(v)=\ta$   in  $\C$.
 
\item[-] 
A set $\Ldg^\inte\subseteq \Lambda\times [1,4]$  of chemical  symbols.

\item[-] 
Subsets $\Gamma^\lnk\subseteq \Gamma^\inte$ of $\Gamma^\inte(D_\pi)$  
of  edge-configurations  $(\mu,\mu' ,m)$ with $\mu \leq \mu'$, where 
 we require that the edge-configuration $\ec(e)$ of an interior-edge (resp., a link-edge) $e$ in $\C$ 
 belongs to $\Gamma^\inte$ (resp.,    $\Gamma^\lnk$).
 %\newone 
We do not distinguish  $(\mu,\mu' ,m)$ and $(\mu' , \mu,m)$.  
  
\item[-] 
Define  $\Gac^\inte$  (resp.,    $\Gac^\lnk$)  to be the set of   adjacency-configurations such that  
$\Gac^\typ:=\{(\ta, \tb, m) \mid (\ta d, \tb d',m)\in \Gamma^\typ\}, \typ\in\{\inte,\lnk\}$.   
Let  $\ac_\nu^\inte(\C), \nu\in \Gac^\inte$  
(resp.,  $\ac_\nu^\lnk(\C), \nu\in \Gac^\lnk$)   
denote  the number of  interior-edges (resp.,  link-edges) $e$
 such that $\ac(e)=\nu$  in $\C$.
  
\item[-] 
 Subsets $\Lambda^*(v)\subseteq \{\ta\in \Lambda^\inte\mid \val(\ta)\geq 2\}$, 
 $v\in \VC$,  
 we require that every chemical element $\alpha(v)$ 
 assigned to   a vertex $v\in  \VC$
 in the seed graph  belongs to $\Lambda^*(v)$.  

\item[-] Lower and upper bound functions 
$\na_\LB,\na_\UB: \Lambda\to  [0,n^*]$  and 
$\na_\LB^\inte,\na_\UB^\inte: \Lambda^\inte\to  [0,n^*]$ 
on the number of   interior-vertices  $v$ such that  $\alpha(v)=\ta$  in $\C$. 

\item[-] Lower and upper bound functions  
$\ns_\LB^\inte,\ns_\UB^\inte: \Ldg^\inte\to  [0,n^*]$ 
  on the number of   interior-vertices $v$ such that $\cs(v)=\mu$  in $\C$.   

\item[-] Lower and upper bound functions  
$\ns_\LB^\cnt,\ns_\UB^\cnt: \Ldg^\inte\to  [0,2]$ 
  on the number of connecting-vertices $v$ such that $\cs(v)=\mu$  in $\C$.   
  %\newone
  
\item[-] Lower and upper bound functions  
$\ac_\LB^\inte,\ac_\UB^\inte: \Gac^\inte \to  \Z_+$ 
($\ac_\LB^\lnk,\ac_\UB^\lnk: \Gac^\lnk \to  \Z_+$)
 on the number of  interior-edges (resp., link-edges) $e$ such that $\ac(e)=\nu$  in $\C$. 
 %\newone 

\item[-] Lower and upper bound functions  
$\ec_\LB^\inte,\ec_\UB^\inte: \Gamma^\inte \to  \Z_+$ 
(resp., $\ec_\LB^\lnk,\ec_\UB^\lnk: \Gamma^\lnk \to  \Z_+$)  
 on the number of  interior-edges  (resp., link-edges)  $e$ such that $\ec(e)=\gamma$  in $\C$.  
 %\newone 
 
\item[-] Lower and upper bound functions  
$\fc_\LB,\fc_\UB: \mathcal{F}^*\to  [0,n^*]$ 
  on the number of   interior-vertices $v$ 
  such that $\C[v]^\fr$ is r-isomorphic to $\psi\in \mathcal{F}^*$  in $\C$.  
  
 \item[-] Lower and upper bound functions  
$\ac^\lf_\LB,\ac^\lf_\UB: \Gac^\lf \to  [0,n^*]$ 
  on the number of  leaf-edges $uv$ in $\acC$
  with adjacency-configuration $\nu$.  
\end{enumerate}
 
We call a chemical graph $\C$ that satisfies a chemical specification $\sce$
a {\em $(\sint,\sce)$-extension of $\GC$},
and denote by $\mathcal{G}(\GC, \sint,\sce)$ the set of
all $(\sint,\sce)$-extensions of $\GC$. 

Table~\ref{table:chemical_spec}  shows an example of 
a chemical-specification  $\sce$ to the seed graph  $\GC$
 in Figure~5(a). %\ref{fig:specification_example_1}. 
 

\begin{table}[h!]\caption{Example~2 of a chemical-specification  $\sce$.  
}
\begin{tabular}{ |  l |  } \hline
 $n_\LB=30$,  $n^* =50$. \\\hline
  branch-parameter:   ${\rho}=2$  \\\hline
\end{tabular}

\begin{tabular}{ |  l |  } \hline
 Each of sets $\mathcal{F}(v), v\in \VC$ and
 $\mathcal{F}_E$ is set to be \\
 the set $\mathcal{F}$  of chemical rooted trees $\psi$ with $\h(\anpsi)\leq {\rho}=2$
in Figure~5(b). \\\hline  %\ref{fig:specification_example_1}(b). \\\hline
\end{tabular}

\begin{tabular}{ |  c | c |   } \hline
  $\Lambda=\{ \ttH,\ttC,\ttN,\ttO, \ttS_{(2)},\ttS_{(6)}, \ttP=\ttP_{(6)},\ttCl\}$ & 
  $\Ldg^\inte =\{ \ttC2 , \ttC3,  \ttC4, \ttN2, \ttN3, \ttO2,
    \ttS_{(2)}2,  \ttS_{(6)}3, \ttP4   \}$  
\\\hline
\end{tabular}

\begin{tabular}{ |  c | l |  } \hline
  $\Gac^{\inte}$ &
  $ \nu_1 \!=\!(\ttC   , \ttC  , 1) ,   \nu_2 \!=\!(\ttC   , \ttC  , 2) ,   
   \nu_3 \!=\!(\ttC   , \ttN  , 1) ,  \nu_4 \!=\!(\ttC  , \ttO  , 1), 
    \nu_5 \!=\! (\ttC, \ttS_{(2)}, 1),\nu_6 \!=\!(\ttC  , \ttS_{(6)}, 1), 
    \nu_7 \!=\! (\ttC  , \ttP  , 1) $  \\ \hline
\end{tabular}
    

\begin{tabular}{ |  c | l |  } \hline
  $\Gamma^{\inte}$ &
  $ \gamma_1 \!=\! (\ttC 2 , \ttC 2, 1) ,
    \gamma_{2} \!=\!(\ttC 2 , \ttC 2, 2),  
   \gamma_3 \!=\!(\ttC 2 , \ttC 3, 1) ,  
   \gamma_4 \!=\!(\ttC 2 , \ttC 3, 2) ,  
   \gamma_5 \!=\!(\ttC 2 , \ttC 4, 1) , 
   \gamma_6 \!=\!(\ttC 3 , \ttC 3, 1) , 
 $ \\
   &
  $  \gamma_7 \!=\!(\ttC 3 , \ttC 3, 2) ,
    \gamma_8 \!=\!(\ttC 3 , \ttC 4, 1), 
   \gamma_9 \!=\!(\ttC 2 , \ttN 3, 1) ,  
   \gamma_{10} \!=\!(\ttC 3 , \ttN 2, 1) ,   
    \gamma_{11} \!=\!(\ttC 4 , \ttN2, 1), 
   \gamma_{12} \!=\!(\ttC 2 , \ttO 2, 1), $ \\ 
   &
  $  
    \gamma_{13} \!=\!(\ttC 3 , \ttO 2, 1) ,    
    \gamma_{14} \!=\!(\ttC 2, \ttS_{(2)} 2, 1),  
    \gamma_{15} \!=\!(\ttC 3, \ttS_{(2)} 2, 1),  
    \gamma_{16} \!=\!(\ttC 4, \ttS_{(2)} 2, 1),  
    \gamma_{17} \!=\!(\ttC 3 , \ttS_{(6)}3, 1),   $ \\ 
   &
  $  
   \gamma_{18} \!=\!(\ttC 4, \ttS_{(6)}3, 1), 
    \gamma_{19} \!=\!(\ttC 2, \ttP4, 1), 
    \gamma_{20} \!=\!(\ttC 3, \ttP4, 1)  
     $ \\ \hline
\end{tabular} 

    
\begin{tabular}{ |  c | l |  } \hline
  $\Gac^{\lnk}$ &
  $ \nu'_1 \!=\!(\ttC   , \ttC  , 1) ,   \nu'_2 \!=\!(\ttC   , \ttC  , 2) ,   
   \nu'_3 \!=\!(\ttC   , \ttN  , 1),  \nu'_4 \!=\!(\ttC   , \ttS_{(2)}  , 1)  $  \\ \hline
\end{tabular}


\begin{tabular}{ |  c | l |  } \hline
  $\Gamma^{\lnk}$ &
  $ \gamma'_1 \!=\! (\ttC 2 , \ttC 2, 1) ,
   \gamma'_2 \!=\!(\ttC 2 , \ttC 3, 1) ,  
   \gamma'_3 \!=\!(\ttC 2 , \ttC 4, 1) ,  
   \gamma'_4 \!=\!(\ttC 3 , \ttC 3, 1) , 
   \gamma'_5 \!=\!(\ttC 3 , \ttC 3, 2) ,
   \gamma'_6 \!=\!(\ttC 2 , \ttN 3, 1) ,   $ \\
   &
  $   
   \gamma'_7 \!=\!(\ttC 3 , \ttN 2, 1), 
    \gamma'_8 \!=\!(\ttC 2, \ttS_{(2)} 2, 1),  
    \gamma'_9 \!=\!(\ttC 3, \ttS_{(2)} 2, 1),  
    \gamma'_{10} \!=\!(\ttC 4, \ttS_{(2)} 2, 1) $  \\\hline
\end{tabular}

\begin{tabular}{ |  l|  } \hline
$\Lambda^*(u_i)=\{\ttC\}, i\in\{1,2,3,4,5,6,9\}$, 
$\Lambda^*(u_8)=\{{\ttO}\}$, 
$\Lambda^*(u_{12})=\{{\tt C, P}\}$, \\
$\Lambda^*(u_i)=\{\ttC,\ttO,\ttN\}$, $i\in [1,14]\setminus\{1,2,3,4,5,6,8,9,12\}$
   \\\hline
\end{tabular}
  
\begin{tabular}{ |  c | c c c c  c c c c |  } \hline
                         & ${\tt H}$  & ${\tt C}$ &   ${\tt N}$ &     ${\tt O}$ 
                         & $\ttS_{(2)}$ & $\ttS_{(6)}$ & $\ttP$ & $\ttCl$ \\\hline
 $\na_\LB(\ta)$ & 40 &  25 & 1 &  1 & 0 & 0 & 0  & 0 \\ \hline 
%                           65     41     5     6     1     1    1
 $\na_\UB(\ta)$ & 80 & 50 & 8 &  8  & 4 & 4 & 4 & 4 \\\hline
\end{tabular} 
%
\begin{tabular}{ |  c | c c c  c c c   |  } \hline
   & $\ttC$ &   $\ttN$ &     $\ttO$  & $\ttS_{(2)}$ & $\ttS_{(6)}$ & $\ttP$  \\\hline
 $\na_\LB^{\inte}(\ta)$ &  10 &  1 & 0  & 0 & 0 & 0      \\ \hline
%                                        21    3     2     1    1     1 
 $\na_\UB^{\inte}(\ta) $&  25 & 4 & 5 & 2 & 2 & 2  \\\hline
\end{tabular} 

\begin{tabular}{ |  c | c c c c c c  c c c   |  } \hline
    & $\ttC2$ &  $\ttC3$ &   $\ttC4$ & $\ttN2$ &   $\ttN3$ &   $\ttO2$
   & $\ttS_{(2)}2$ & $\ttS_{(6)}3$ & $\ttP4$  \\\hline
 $\ns_\LB^{\inte}(\mu)$ &  3 &   5 & 0  & 0 &  0 &  0 & 0 &  0 &  0    \\ \hline
%                                         9      9     3     2      1     2     1     1     1
 $\ns_\UB^{\inte}(\mu) $& 12 & 15 & 5 & 5 &  3 &  5  & 1 & 1 &  1   \\\hline
\end{tabular} 

\begin{tabular}{ |  c | c c c c c c  c c c   |  } \hline
    & $\ttC2$ &  $\ttC3$ &   $\ttC4$ & $\ttN2$ &   $\ttN3$ &   $\ttO2$
   & $\ttS_{(2)}2$ & $\ttS_{(6)}3$ & $\ttP4$  \\\hline
 $\ns_\LB^{\cnt}(\mu)$ &  0 &   0 & 0  & 0 &  0 &  0 & 0 &  0 &  0    \\ \hline
%                                        
 $\ns_\UB^{\cnt}(\mu) $& 2 & 2 & 2 & 2 & 2 &  2  & 1 & 1 &  0   \\\hline
\end{tabular}   

\begin{tabular}{ |  c | c c c c c c c |  } \hline
         & $\nu_1 $ &   $\nu_2 $ & $\nu_3 $   & $\nu_4 $
         &   $\nu_5 $ & $\nu_6 $   & $\nu_7 $ \\\hline
 $\ac_\LB^{\inte}(\nu)$  &  0  &  0  & 0  &  0  & 0  &  0 & 0     \\ \hline
%                                        15      4     5       2     1      2     2
 $\ac_\UB^{\inte}(\nu)$ & 30 & 10 & 10 &  10 & 2 &  3 &  3 \\\hline
\end{tabular} 
%

\begin{tabular}{ |  c | c c c c c  c c  |  } \hline
    & $\gamma_1 $ &   $\gamma_2 $ & $\gamma_3 $   & $\gamma_4 $  & $\gamma_5 $
    & $\gamma_i, i\in[6,13]$ &   $\gamma_i, i\in[14,20]$    \\\hline
 $\ec_\LB^{\inte}(\gamma)$ &
    0 &  0 & 0 &  0  & 0 &  0 &  0    \\ \hline
% 0     1    
 $\ec_\UB^{\inte}(\gamma) $ &
   4 & 15 & 5 &  5  & 10 & 5  &   2  \\\hline
\end{tabular} 

\begin{tabular}{ |  c | c c c  c  |  } \hline
         & $\nu'_1 $ &   $\nu'_2 $ & $\nu'_3 $  & $\nu'_4 $  \\\hline
 $\ac_\LB^{\lnk}(\nu')$  &  0  &  0  & 0  & 0      \\ \hline
 %                                        4       1     0
 $\ac_\UB^{\lnk}(\nu')$ &  10 &  5 &  5  &  5   \\\hline
\end{tabular} 
%
\begin{tabular}{ |  c | c    |  } \hline
    & $\gamma'_i, i\in[1,10]$   \\\hline
 $\ec_\LB^{\lnk}(\gamma')$ &  0    \\ \hline
 %                                                   
 $\ec_\UB^{\lnk}(\gamma') $& 4   \\\hline
\end{tabular}

\begin{tabular}{ |  c | c   c   |  } \hline 
& $\psi\in\{\psi_i\mid i=1,6,11\}$ 
& $\psi\in \mathcal{F}^*\setminus \{\psi_i\mid i=1,6,11\}$ \\\hline
 $\fc_\LB(\psi)$  &  1 &    0   \\ \hline 
 $\fc_\UB(\psi)$ &  10 &  3\\\hline
\end{tabular} 

\begin{tabular}{ |  c | c   c   |  } \hline 
& $\nu\in\{(\ttC,\ttC,1),(\ttC,\ttC,2)  \}$ 
& $\nu\in \Gac^\lf \setminus \{(\ttC,\ttC,1),(\ttC,\ttC,2)  \}$   \\\hline
 $\ac^\lf_\LB(\nu)$  &  0 &    0   \\ \hline 
 $\ac^\lf_\UB(\nu)$ &  10 &  8 \\\hline
\end{tabular} 

\label{table:chemical_spec}
\end{table}

 
Figure~3 %\ref{fig:example_polymer} 
 illustrates an example of 
a   $(\sint,\sce)$-extension of $\GC$   obtained 
from the  $\sint$-extension $H^*$  
 in Figure~6 %\ref{fig:test_subgraph_polymer} 
under the chemical-specification $\sce$ in Table~\ref{table:chemical_spec}.  
  
% Our specification of topological substructures is
% similar to that proposed by Akutsu and Nagamochi~\cite{AN20},
% wherein a target chemical graph is restricted to ${\rho}$-lean 
% cyclic graphs and no prescribed substructure can not be
% specified in the acyclic part.
% In our new method, a chemical graph with any structure 
% can be handled and  substructures in the acyclic part can be fixed.
 
 