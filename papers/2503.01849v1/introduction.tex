
\section{Introduction}\label{sec:introduction}

\noindent

In recent years, there have been extensive studies focusing on the design of novel molecules using 
machine learning techniques~\cite{Lo:2018aa, Tetko:2020aa}.
Molecular design with the aid of computational methods
has a long history in the field of cheminformatics,
and are commonly studied under the name of \emph{quantitative structure activity/property relationship} 
(QSAR/QSPR)~\cite{Cherkasov:2014aa, Skvortsova:1993aa}, and \emph{inverse quantitative structure
activity/property relationship} (inverse QSAR/QSPR)~\cite{Ikebata:2017aa, Miyao:2016aa, Rupakheti:2015aa}.
QSAR/QSPR models aim to predict chemical activities based on molecular
structures~\cite{Cherkasov:2014aa}, whereas
inverse QSAR/QSPR models focus on inferring molecular structures that exhibit specific
chemical activities/properties~\cite{Ikebata:2017aa, Miyao:2016aa, Rupakheti:2015aa}.
Traditionally, molecular structures are represented as undirected graphs, referred to as
\emph{chemical graphs}, and typically encoded as vectors of real numbers called \emph{descriptors} or
\emph{feature vectors} in most of the existing QSAR/QSPR and inverse QSAR/QSPR studies.
A typical approach for inverse QSAR/QSPR involves inferring feature vectors from
 given chemical activities and subsequently reconstructing chemical graphs from 
 these vectors~\cite{Ikebata:2017aa, Miyao:2016aa, Rupakheti:2015aa}. 
 %However, the reconstruction
 %itself is a difficult task, not only because the number of possible chemical graphs is 
 %huge~\cite{Bohacek:1996aa}, but the problem to infer a chemical graph from a given feature vector
 %is indeed known as a computational difficult problem (precisely, NP-hard) except for some 
 %simple cases~\cite{Akutsu:2012aa}.
 Recently,
 artificial neural network (ANN) and other deep learning techniques 
 have been increasingly utilized in inverse QSAR/QSPR studies,
due to the availability of generative models~\cite{Ghasemi:2018aa, Kipf:2016aa, Gomez-Bombarelli:2018aa, De-Cao:2018aa, Madhawa:2019aa, Du:2022aa}.
% Furthermore, graph-structured
% data can be directly handled by using graph convolutional networks~\cite{Kipf:2016aa}. Therefore,
% it is reasonable to try to apply ANNs to inverse QSAR~\cite{Xiong:2021aa}. 
% Various ANN-based models have been proposed, which include variational 
% autoencoders~\cite{Gomez-Bombarelli:2018aa}, 
% generative adversarial networks~\cite{De-Cao:2018aa},
% recurrent neural networks~\cite{Segler:2018aa, Yang:2017aa},
% grammar variational autoencoders~\cite{Kusner:2017aa},
% and invertible flow models~\cite{Madhawa:2019aa, Shi:2020aa}.

%%%% removed, according to Zhao-sensei's comments 
% However, two critical properties of the solutions generated by the inverse QSAR/QSPR 
% models---optimality, which refers to the quality of the solution in addressing 
% the inverse problem of learning methods,
% and exactness, which ensures that the solution corresponds to a valid chemical graph---remain 
% unguaranteed by these
% methods~\cite{Zhang:2022aa, Zhu:2023aa}.


%\begin{figure}[!ht]
%\begin{center}
%\includegraphics[width=.77\columnwidth]{./figures/framework_interval.eps}
%\end{center}
%\caption{An illustration of a two-phase framework for inferring a set of chemical graphs $\bbC^*$.}
%\label{fig:framework}
%\end{figure}
%
%\noindent
%{\bf Framework}
%Akutsu and Nagamochi~\cite{Akutsu:2019aa} showed that computation process of a given ANN can be 
%simulated by a mixed integer linear programming (MILP). 
%Based on this, a novel two-phase framework for inferring chemical graphs 
%has been developed and revised~\cite{Azam:2020aa, Zhang:2020aa, Zhu:2022ad},
%as illustrated in Figure~\ref{fig:framework}.
%The first phase collects some data sets $D_\pi$ and then constructs a prediction function $\eta$ for a given chemical property $\pi$, and the second
%phase infers a chemical graph which is supposed to have a given target value of the property
%based on the prediction function $\eta$.
%
%The process of Phase 1 is summarized as follows:
%\begin{itemize}
%\item Step 1: Given a chemical property $\pi$, we collect a data set $D_\pi$ of chemical graphs such that
%for every chemical graph $\bbC \in D_\pi$, the observed value $a(\bbC)$ of property $\pi$ 
%is available.
%\item Step 2: We define a feature function $f: \calG \to \bbR^K$ for a positive integer $K$
%based on the local graphic structures of the chemical graph
% so that $f$ is tractable by MILP formulations in the second phase. 
%%In the case of a polymer, we follow~\cite{Ido:2021aa} to 
%%represent the polymer as a form of monomer and use a specialized feature vector designed
%%for polymer.
%\item Step 3: We construct a prediction function $\eta$ by some machine learning methods
%in order to produce an output $y=\eta(x)\in\bbR$ based
%on the feature vector $x=f(\bbC)\in\bbR^K$ for each $\bbC \in D_\pi$, where the
%output is served as a predicted value for the observed value $a(\bbC)$.
%\end{itemize}
%
%Phase 2 consists of two steps, is designed to infer chemical graphs with a desired property value. 
%
%\begin{itemize}
%\item Step 4: Given a set of rules called $\emph{topological specification}$ $\sigma$ that specifies
%the desired structure of the inferred chemical graphs, and a desired range $[\ylb, \yub]$ of target value, 
%step 4 is to infer chemical graphs $\bbC^*$ that satisfies the rules $\sigma$ and
%$\ylb \le \eta(f(\bbC^*)) \le \ylb$.
%We formulate an MILP formulation $\MILP$ that represents
%\begin{itemize}
%\item[(i)] the constraint $\bbC \in \calG_\sigma$ and the computation process of $x := f(\bbC)$; and
%\item[(ii)] that of $y := \eta(x)$ from a vector $x \in \bbR^K$,
%\end{itemize}
%where $\calG_\sigma$ denotes the set of all chemical graphs satisfying the rules $\sigma$.
%We solve the MILP $\MILP$ after a range $[\ylb, \yub]$ is specified to find
%a feature vector $x^* \in \bbR^K$ and a chemical graph $\bbC^\dagger$
%such that $f(\bbC^\dagger) = x^*$ and $\ylb \le \eta(x^*) \le \yub$ (where if
%the MILP is infeasible then this suggests that
%$\calG_\sigma$ does not contain such a desired chemical graph).
%\item Step 5:
%The final step is to generate the isomers of the inferred chemical graphs $\bbC^\dagger$
%by using a dynamic programming algorithm~\cite{Azam:2021aa}. 
%The dynamic programming algorithm operated by decomposing $\bbC^\dagger$
%into trees and generating their isomers respectively. These isomers
%are then combined to produce a set of isomers $\bbC^*$ that belong to the desired
%chemical graph space $\calG_\sigma$ and have exactly the same feature vector as $\bbC^\dagger$.
%\end{itemize}




%\noindent
%{\bf Contribution}

%\begin{figure}[t!]
%\begin{center}
%\includegraphics[width=.77\columnwidth]{./figures/framework_interval.eps}
%\end{center}
%\caption{An illustration of the two-phase framework \molinfer.} % for inferring a set of chemical graphs $\bbC^*$.}
%\label{fig:framework}
%\end{figure}

%\begin{figure}[t!]
%\begin{center}
%\includegraphics[width=.8\columnwidth]{./figures/framework.eps}
%\end{center}
%\caption{An illustration of the two-phase framework \molinfer.} % for inferring a set of chemical graphs $\bbC^*$.}
%\label{fig:framework}
%\end{figure}

In the last few years, significant breakthroughs have been made in the field of computational molecular modeling,
highlighted with the release of AlphaFold~3~\cite{Jumper:2021aa, Abramson:2024aa},
which demonstrates remarkable accuracy in predicting protein structures using state-of-the-art deep learning models
using 3D spatial information.
However, the inference of small molecules remains an important task,
as they play an important role in drug design and related applications.
For instance, the ChEMBL database~\cite{Zdrazil:2023aa} contains more than
$2 \times 10^6$ small drug-like molecules. While 3D spatial information can reveal critical molecular insights,
it is often unavailable in such databases.
Consequently, the use of only 2D structural information, such as chemical graphs and topological features,
remains an essential approach in QSAR/QSPR and inverse QSAR/QSPR studies.

%Recently, a novel framework called \molinfer~\cite{Chiewvanichakorn:2020aa,
%Shi:2021aa, Zhu:2022ad, Ido:2024aa} has been proposed
%for inferring chemical compounds with desired property values.
%This framework mainly establishes on using the \emph{mixed integer linear programming} (MILP) formulation
%to simulate the computational process of machine learning methods and describe 
%the necessary and sufficient conditions to ensure such a chemical graph exists,
%utilizing only 2D structural information.
%As a result, it guarantees both the optimality and exactness of the solutions.
%Figure~\ref{fig:framework} is an illustration of \molinfer\ (see Section~\ref{sec:frame_all} for a detailed description about it).
%This framework was first introduced for general molecules~\cite{Chiewvanichakorn:2020aa,
%Shi:2021aa, Zhu:2022ad} and then extended to
%polymers recently~\cite{Ido:2024aa}.


Most of the existing QSAR/QSPR and inverse QSAR/QSPR models 
are developed under the assumption
that a chemical property value is determined by just one
molecule, 
%(monomer or polymer), 
and the environment such as temperature and pressure when the value is measured,
 is neglected more or less.
 Meanwhile, some important properties depend on interactions between multiple molecules
 and are sensitive to factors like
temperature, pressure, and frequency, collectively referred to as the
%{\em experimental environmental factors}, or simply 
{\em environment} in this study.
For example, permittivity depends on the frequency, magnitude, and direction of the applied field~\cite{Landau:2013aa}, 
and dissipation factor varies on the frequency of electrical signals~\cite{Valentine:2019aa}.
One important value that characterizes the thermodynamic properties between the solute and the solvent, Flory-Huggins $\chi$-parameter,
is also related to two molecules, and is found depending on temperature, pressure, and some other factors~\cite{Aoki:2023aa}.

To fill this research gap,
in this paper, we extend the %above 
existing framework \molinfer~\cite{Chiewvanichakorn:2020aa, Shi:2021aa, Zhu:2022ad, Ido:2024aa} 
to include such important cases by integrating the information of multiple molecules and the 
experimental environmental factors into one feature vector.
%Formally, for a property $\pi$, which is defined to be related to
% $k$ molecules, $\bbC_1, \bbC_2, ..., \bbC_k$, and  the experimental environment $\tau$,
% we define an instance in a data set of property $\pi$
In~\cite{Ido:2024aa}, the authors %have already 
considered the frequencies when the values are observed as descriptors when constructing a prediction function for the property
permittivity, but did not discuss this issue formally.
As an example of the extended framework, we particularly
consider the task of inferring molecules with desired $\chi$-parameter values under specific environments.
Various models have been developed to 
predict Flory-Huggins $\chi$-parameter values 
computationally~\cite{Hildebrand:1950aa, Lindvig:2002aa, Orwoll:2007aa},
but these methods are either known for their low prediction performance or high computational cost~\cite{Aoki:2023aa, Nistane:2022aa, Orwoll:2007aa, Wolf:2011aa}.
Also, as far as we know, %to our best knowledge, 
there is no work on designing molecules with specified $\chi$-parameter values.
We design a feature vector for the solute-solvent pair $(\bbC_1, \bbC_2)$ and manage to include the temperature information into it 
in several different ways, and then apply three different machine learning methods to two data sets on empirical $\chi$-parameter values 
and one data set on simulation-based $\chi$-parameter values. 
The experimental results highlight the model's ability
to achieve a competitively high learning performance, comparable to or exceeding existing studies~\cite{Aoki:2023aa, Nistane:2022aa}
on predicting $\chi$-parameter values.
Also, focusing on the situation of inferring a solute $\bbC_1$ when given a fixed solvent $\bbC_2$ and a specified temperature $T$,
% which is a realistic situation in practice, 
we successfully infer
polymers with up to 50 non-hydrogen atoms in their monomer forms in a reasonable time.
A comparative study to the simulation software \jocta~\cite{JOCTA} demonstrates that the polymers
generated by our proposed methods are generally of good quality.

We organize the paper as follows.
Section~\ref{sec:preliminary} reviews some basic concepts and terminologies on graphs,
the framework \molinfer\ to infer a molecule with some desired property values,
and a modeling of chemical compounds.
Section~\ref{sec:formulation} describes how we extend the framework for multiple molecules and experimental environmental factors.
Section~\ref{sec:experiment} presents the results of 
computational experiments conducted on data sets of $\chi$-parameter.
Section~\ref{sec:conclude} concludes the paper.
Additional details are included in the Appendix.
All the program codes and experimental results are accessible at {\url{https://github.com/ku-dml/mol-infer/tree/master/chi-parameter}}.





 