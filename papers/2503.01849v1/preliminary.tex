\section{Preliminary}\label{sec:preliminary}
%{\bf \color{red} ERASE UNNECESSARY PART AND MOVE THEM TO APPENDIX !!!}
We give some notions and terminologies on graphs in Section~\ref{sec:graphs}
and review the framework \molinfer\ in Section~\ref{sec:frame_all},
a modeling of chemical compounds in Section~\ref{sec:chemical_model},
and the two-layered model, the standard model in \molinfer, in Section~\ref{sec:2LM}.
Some necessary modifications introduced by Ido~et~al.~\cite{Ido:2024aa} when the molecule is a polymer 
will be covered in Section~\ref{sec:polymer}.

%In this section, we give some notions and terminologies on graphs and modeling of chemical compounds,
%and then review the framework introduced by Ido~et~al.~\cite{Ido:2021aa}. \qqq



Let $\bbR$, $\bbR_+$, $\bbZ$  and $\bbZ_+$ 
represent the sets of reals,  non-negative reals, 
integers, and non-negative integers, respectively.
For two integers $a$ and $b$ such that $a \leq b$, $[a,b]$ is defined as the set of 
integers $i$ such that $a\leq i\leq b$.
%For a vector $x \in \bbR^K$, the $j$-th entry of $x$ is denoted by $x(j), j\in [1,K]$.


%\noindent
%{\bf Graph}
\subsection{Graphs}\label{sec:graphs}

When referring to a {\em graph} $G$, it is assumed that $G$ is a connected and simple graph.
The sets of vertices and edges of a given graph $G$ are denoted
by $V(G)$ and $E(G)$, respectively.
For any vertex $v\in V(G)$, we denote the set of its neighbors in $G$ by $N_G(v)$,
and the {\em degree} $\mathrm{deg}_G(v)$ of $v$ is
$\mathrm{deg}_G(v)=|N_G(v)|$.

 
A vertex designated in a graph $G$ is called a {\em root},
and a graph with such a vertex is referred to as a {\em rooted graph}. 
 For a graph $G$ (possibly rooted),
 a {\em leaf-vertex} is a non-root vertex  $v$ with degree 1.
 For any subset $V'\subseteq V(G)$, % of a graph $G$, 
the graph $G-V'$ is obtained by removing all vertices in $V'$ along with any edges incident to them. 
An edge $uv$ incident to a leaf-vertex $v$ is called a {\em leaf-edge}.
 We denote the sets of leaf-vertices and leaf-edges in $G$ by $\Vleaf(G)$ and $\Eleaf(G)$, respectively.
 For a graph $G$ (possibly rooted),
 a sequence of graphs $G_i, i\in \mathbb{Z}_+$ is defined by iteratively removing all leaf-vertices
 $i$ times as follows:
\[ G_0:=G; ~~ G_{i+1}:=G_i - \Vleaf(G_i). \]
A vertex $v$ is called a {\em tree vertex} if $v\in \Vleaf(G_i)$
for some $i\geq 0$. 
We define the {\em height} $\h(v)$ of a tree vertex $v\in \Vleaf(G_i)$ to be $i$; 
and for a non-tree vertex $v$ adjacent to a tree vertex, we define the height
$\h(v)$ to be $\h(u)+1$, 
where $u$ is the tree vertex with the maximum height $\h(u)$ among those adjacent to $v$.
%The height of any non-tree vertex not adjacent to a tree vertex is left undefined. 
The heights of other vertices are left undefined. 
Finally, the {\em height} $\h(T)$ of a rooted tree $T$ is defined
to be the maximum of $\h(v)$ among all vertices $v\in V(T)$. 
 
 
\subsection{\molinfer : An Inverse QSAR/QSPR Framework Based on Machine Learning and MILP}\label{sec:frame_all}

\begin{figure}[t!]
\begin{center}
\includegraphics[width=.8\columnwidth]{./figures/framework.eps}
\end{center}
\caption{An illustration of the two-phase framework \molinfer.} % for inferring a set of chemical graphs $\bbC^*$.}
\label{fig:framework}
\end{figure}

The computation process of an artificial neural network (ANN)
with ReLU activation functions can be 
represented through a mixed integer linear programming (MILP) formulation,
as demonstrated by Akutsu and Nagamochi~\cite{Akutsu:2019aa}.
%
%Akutsu and Nagamochi~\cite{Akutsu:2019aa} demonstrated
%that the computation process of an artificial neural network (ANN)
%with ReLU activation functions can be 
%represented through a mixed integer linear programming (MILP) formulation. 
Based on this concept, 
a two-phase inverse QSAR/QSPR framework,
 %for inferring chemical graphs, 
 called \molinfer,
has been proposed and subsequently refined~\cite{Chiewvanichakorn:2020aa,Zhang:2022aa, Shi:2021aa, Zhu:2022ad, Ido:2024aa, Zhu:2023aa},
as depicted in Figure~\ref{fig:framework}.
This framework mainly establishes on using the \emph{mixed integer linear programming} (MILP) formulation
to simulate the computational process of machine learning methods and describe 
the necessary and sufficient conditions to ensure such a chemical graph exists,
utilizing only 2D structural information.
The advantage of \molinfer\ compared to other methods is that it
guarantees both optimality and exactness.
Here, optimality refers to the quality of the solution in addressing 
the inverse problem of learning methods,
while exactness ensures that the solution corresponds to a valid chemical graph.
This framework was first introduced for general molecules~\cite{Chiewvanichakorn:2020aa,
Shi:2021aa, Zhu:2022ad} and then extended to
polymers recently~\cite{Ido:2024aa}.
This subsection provides an overview of the core ideas for \molinfer\ for completeness.

\subsubsection{Phase~1}
Phase~1 is the QSAR/QSPR phase, aiming to construct a prediction function between chemical compounds and their observed property values, 
and consisting of three stages. 
Here we denote $\calG$ the set of all possible chemical graphs.
\begin{itemize}
\item[-] Stage~1: Given a chemical property $\pi$, we collect a data set $D_\pi \subseteq \calG$ of chemical graphs such that
for every chemical graph $\bbC \in D_\pi$, the observed value $a(\bbC)$ of property $\pi$ 
is available.
\item[-] Stage~2: A feature function $f: \calG \to \bbR^K$ ($K$ is a positive integer) is defined. %for a positive integer $K$ 
This feature function consists of only graph-theoretic descriptors, 
mainly based on the local graph-theoretic structures of the chemical graph
 so that $f$ is tractable by MILP formulations in Phase~2. 
\item[-] Stage~3: A prediction function $\eta$ is constructed by some machine learning methods
in order to produce an output $y=\eta(x)\in\bbR$ based
on the feature vector $x=f(\bbC)\in\bbR^K$ for each $\bbC \in D_\pi$.
%, where the
%output is served as a predicted value for the observed value $a(\bbC)$.
\end{itemize}

\subsubsection{Phase~2}
Phase~2 is devoted to the inverse QSAR/QSPR phase, 
designed to infer chemical graphs with a specified property value
based on the prediction function $\eta$ constructed in Phase~1. 
It consists of two stages.

\begin{itemize}
\item[-] Stage~4: Given a set of rules called topological specification $\sigma$ (see Section~\ref{sec:2LM} for more details) that specifies
the desired structure of the inferred chemical graphs, and a desired range $[\ylb, \yub]$ of the target value, 
Stage 4 is to infer a chemical graphs $\bbC^\dagger$ that satisfies the rules $\sigma$ and
$\eta(f(\bbC^\dagger)) \in [\ylb, \yub]$.
%$\ylb \le \eta(f(\bbC^\dagger)) \le \yub$.
To achieve this,
an MILP formulation $\mathcal{M}(g,x,y;\mathcal{C}_1,\mathcal{C}_2)$ is formulated, 
which consists of two parts:
\begin{itemize}
\item[(i)] $\mathcal{M}(x,y;\mathcal{C}_1)$: the computation process of $y := \eta(x)$ from a vector $x \in \bbR^K$; and
\item[(ii)] $\mathcal{M}(g,x;\mathcal{C}_2)$: that of $x := f(\bbC)$ and the constraints for $\bbC \in \calG_\sigma$,
\end{itemize}
where $\calG_\sigma$ denotes the set of all chemical graphs satisfying $\sigma$.
We solve the MILP $\mathcal{M}(g,x,y;\mathcal{C}_1,\mathcal{C}_2)$ %after a range $[\ylb, \yub]$ is specified 
for a given $\sigma$ and $[\ylb, \yub]$
to find
a feature vector $x^* \in \bbR^K$ and a chemical graph $\bbC^\dagger$
such that $f(\bbC^\dagger) = x^*$ and $\eta(x^*) \in [\ylb, \yub]$.
In the case that the MILP is infeasible, it indicates that no chemical graph in $\calG_\sigma$ satisfies the specified demand.
\item[-] Stage~5:
The final stage is to generate the isomers of the inferred chemical graphs $\bbC^\dagger$
by using a dynamic programming-based graph enumeration algorithm developed by Zhu~et~al.~\cite{Zhu:2022ad}. 
A {\em chemical isomer} of $\bbC^\dagger$ under
a topological specification $\sigma$ is defined as 
a chemical graph $\bbC^*$  such that
$f(\bbC^*)=f(\bbC^\dagger)$ and $\bbC^*\in \calG_\sigma$.
This graph enumeration algorithm operates by decomposing $\bbC^\dagger$
into trees and generating their isomers respectively. These isomers
are then combined to produce a set of chemical isomers $\bbC^*$ that belong to the desired
chemical graph space $\calG_\sigma$ and have exactly the same feature vector as $\bbC^\dagger$.
\end{itemize}



\subsection{Modeling of Chemical Compounds}\label{sec:chemical_model}

This subsection reviews a modeling of chemical compounds 
introduced by Zhu~et~al.\cite{Zhu:2022ad}.
Let $\Lambda$ represent the set of chemical elements;
for example,  $\Lambda=\{\tH,  \tC, \tO, \tN, \tP, \tS_{(2)}, \tS_{(4)}, \tS_{(6)}\}$. 
Here, elements $\ta$ with multiple valence states are distinguished with a suffix, i.e., 
we denote an element $\ta$ with valence $i$ as $\ta_{(i)}$.


%To represent a chemical compound, 
%we introduce a set of chemical elements, including
%  {\tt H} (hydrogen),   {\tt C} (carbon), {\tt O} (oxygen), {\tt N} (nitrogen), and so on.
% To distinguish elements $\ta$ with multiple valence states, such as {\tt S} (sulfur),
% we denote an element $\ta$ with valence $i$ as $\ta_{(i)}$.
% This suffix $(i)$ is omitted for elements with a unique valence.
%Let $\Lambda$ represent the set of chemical elements $\ta_{(i)}$;
%for example,  $\Lambda=\{\tH,  \tC, \tO, \tN, \tP, \tS_{(2)}, \tS_{(4)}, \tS_{(6)}\}$. 

A chemical compound $\bbC$ is represented as a {\em chemical graph}, which is defined as
a triplet $\bbC=(H,\alpha,\beta)$, where
$H$ is a graph $H$,
$\alpha:V(H)\to \Lambda$ assigns chemical elements to vertices, and
$\beta:E(H)\to [1,3]$ assigns bond multiplicities to edges.
%The set of atoms and the set of bonds in the compound 
%are represented by $V(H)$ and $E(H)$, respectively.
%The chemical element assigned to a vertex $v\in V(H)$
%is represented by $\alpha(v)$ and 
% the bond-multiplicity  between two adjacent vertices  $u,v\in V(H)$
%is represented by $\beta(e)$ of the edge $e=uv\in E(H)$.
Two chemical graphs $(H_1, \alpha_1, \beta_1)$ and $(H_2,\alpha_2,\beta_2)$ are
{\em isomorphic} if there exists
an isomorphism $\phi$,
i.e., a bijection $\phi: V(H_1)\to V(H_2)$
such that
 $uv\in E(H_1), \alpha_1(u)=\ta, \alpha_1(v)=\tb, \beta_1(uv)=m$
if and only if
 $\phi(u)\phi(v) \in E(H_2), \alpha_2(\phi(u))=\ta, 
 \alpha_2(\phi(v))=\tb, \beta_2(\phi(u)\phi(v))=m$. 
 If $H_1$ and $H_2$ are rooted graphs with roots $r_1$ and $r_2$, respectively,
 the chemical graphs are considered
{\em rooted-isomorphic} %(r-isomorphic) 
if there exists an isomorphism $\phi$ such that $\phi(r_1)=r_2$ also holds. 
  
For a chemical graph  $\bbC=(H,\alpha,\beta)$, 
  let  $V_{\ta}(\bbC)$ ($\ta\in \Lambda$)
represent the set of vertices $v\in V(H)$ such that $\alpha(v)=\ta$. %in $\bbC$.
The {\em hydrogen-suppressed chemical graph} of $\bbC$, denoted as $\anC$,
is obtained by removing all vertices $v\in \VH(\bbC)$ from $H$.



\subsubsection{Two-layered Model}\label{sec:2LM}
Shi~et~al.~\cite{Shi:2021aa} introduced the
two-layered model for chemical graphs
to efficiently capture the graph-theoretic information and develop descriptors based on it.
Here, we summarize the key concepts of this model for completeness.
 
Consider a chemical graph $\bbC=(H,\alpha,\beta)$ and an integer $\rho \geq 1$,
referred to as the {\em branch-parameter}.
For this study, the standard value of $\rho = 2$ is used.
The {\em two-layered model} of $\bbC$ is a partition of
 the hydrogen-suppressed chemical graph $\anC$ into
two regions: the ``interior'' and the ``exterior'' based on the branch-parameter $\rho$.
A vertex $v \in V(\anC)$ (resp., an edge $e \in E(\anC)$) of $\bbC$
is classified as an {\em exterior-vertex} (resp., {\em exterior-edge})
if $\h(v)< {\rho}$ (resp., $e$ is incident to an exterior-vertex).
We denote 
the sets of exterior-vertices and exterior-edges of $\bbC$
by $V^\ex(\bbC)$ and $E^\ex(\bbC)$, respectively. 
The remaining vertices and edges, defined as
$V^\inte(\bbC)=V(\anC)\setminus  V^\ex(\bbC)$ and 
$E^\inte(\bbC)=E(\anC)\setminus E^\ex(\bbC)$,
are called  {\em interior-vertices} and {\em interior-edges}, respectively.
Notice that the set  $E^\ex(\bbC)$  forms 
a collection of connected graphs, each can be treated as a rooted tree $T$ with the root being 
the vertex $v\in V(T)$ with the maximum height $\h(v)$. 
Let $\mathcal{T}^\ex(\anC)$ denote 
the set of these rooted trees in $\anC$. 
The {\em interior} of $\bbC$ is defined to be the subgraph
 $(V^\inte(\bbC),E^\inte(\bbC))$ of $\anC$. 
See Figure~\ref{fig:two_layer} for an example.
 
% An example of a hydrogen-suppressed chemical graph $\anC$
% is shown in Figure~\ref{fig:example_polymer}.
%Using a branch-parameter ${\rho}=2$, 
%the interior of  $\anC$ 
%is obtained by iteratively removing the set of vertices with degree 1 ${\rho}=2$ times; i.e., 
%first remove the set $V_1=\{w_1,w_2,\ldots,w_{19}\}$ of vertices of degree 1 in $\anC$ 
%and then remove  the set
% $V_2=\{w_{20},w_{16},\ldots,w_{26}\}$ of vertices of degree 1 in $\anC-V_1$,
% where the removed vertices are classified as the exterior-vertices of $\anC$. 

%\begin{figure}[t!] \begin{center}
%\includegraphics[width=.90\columnwidth]{figures/test_example_polymer_1.eps}
%\end{center}
%\caption{An illustration of the hydrogen-suppressed monomer representation $\anC$  
%obtained from a polymer $\bbC$ by removing all the hydrogens. 
%The link-edges are depicted with thick lines,
%%$V^\ex(\bbC)=\{w_i \mid i\in [1,26]\}$ and
%%$V^\inte(\bbC)=\{u_i \mid i\in [1,29]\}$  for  ${\rho}=2$
% and the connecting-vertices are marked with asterisks.  
% }
%\label{fig:example_polymer} \end{figure} 


\begin{figure}[t!] \begin{center}
\includegraphics[width=.90\columnwidth]{figures/two_layer.eps}
\end{center}
\caption{
An illustration of the two-layered model.
The interior region is represented by the shaded area enclosed by black dashed lines,
while the remaining parts form the exterior.
$T_u$ is the chemical tree rooted at $u$ and is outlined by a thin gray line.
 }
\label{fig:two_layer} \end{figure} 

For each interior-vertex $u\in V^\inte(\bbC)$,
let $T_u\in \mathcal{T}^\ex(\anC)$ represent the chemical tree rooted at $u$
(where $T_u$ may consist solely of the vertex $u$).
The {\em $\rho$-fringe-tree} $\bbC[u]$  is defined
as the chemical rooted tree obtained by restoring the hydrogens which are originally attached
to $T_u$ in $\bbC$.
%The set of all  $\rho$-fringe-trees $\bbC[u], u \in V^\inte(\bbC)$
%is denoted by $\mathcal{T}(\bbC)$.

For a given integer $K$, a feature vector $f(\bbC)$ for a chemical graph $\bbC$
is defined by a {\em feature function} $f$ which comprises $K$ descriptors 
based on the two-layered model.
% The feature corresponding to an interior-edge $e=uv\in E^\inte(\bbC)$ 
% such that $\alpha(u)=\ta$, $\mathrm{deg}_{\anC}(u)=d$, 
% $\alpha(v)=\tb$, $\mathrm{deg}_{\anC}(v)=d'$  and $\beta(e)=m$, is described by 
% a 5-tuple $(\ta d, \tb d', m)$, which is referred to as the {\em edge-configuration}
% of $e$, while the triplet $(\ta, \tb, m)$ is called
% the {\em adjacency-configuration} of the edge $e$.
A comprehensive list and detailed explanation of the feature function $f$ used in this study
 can be found in Appendix~\ref{sec:descriptor}.


Furthermore, in order to allow the usage of domain knowledge for inference of chemical graphs,
such as some abstract structures or limits on the available 2-fringe-trees,
a set of rules called {\em topological specification} is used.
A topological specification includes the following components:
\begin{itemize}
\item[-] A {\em seed graph} $\GC$, which serves as an abstract form of the target chemical graph $\bbC$.
\item[-] A set $\mathcal{F}$ of chemical rooted trees, which serve as candidates 
for the tree  $\bbC[u]$ rooted at each interior-vertex $u$ in $\bbC$.
\item[-] Lower and upper bounds that constrain the number of various components 
in the target chemical graph, such as the interior-vertices, 
double/triple bonds, and chemical elements in $\bbC$. 
\end{itemize}

\begin{figure}[t!]
\begin{center} 
 \includegraphics[width=.92\columnwidth]{figures/test_seed_graph_polymer.eps}
\end{center}
\caption{(i) A seed graph $G_{\mathrm{C}}$ for $I_a$; (ii) A set $\mathcal{F}$ of chemical rooted trees.
The figure is adapted from~\cite{Ido:2024aa}.
}
\label{fig:seed_graph_a}  
\end{figure} 

Figure~\ref{fig:seed_graph_a} illustrates one example of topological specification.
We refer~\cite{Zhu:2022ad} and Appendix~\ref{sec:specification} for a more detailed description of the topological specification.


\subsubsection{Modeling of Polymers}\label{sec:polymer}

In this subsection, we review the way of representing a polymer as a form of monomer that is proposed by Ido~et~al.~\cite{Ido:2024aa},
and the necessary modification for polymers in the two-layered model.



\begin{figure}[t!] \begin{center}
\includegraphics[width=.96\columnwidth]{figures/thioBis-4-phenyl-carbonate_fig1.eps}
\end{center}
\caption{(a) The repeating unit of the polymer thioBis(4-phenyl)carbonate,
where $v^*_1$ and $v^*_2$ are the connecting-vertices
and $e^*_0$ and $e^*_1$ are the connecting-edges;
(b) The monomer representation of the polymer in (a), where 
$v^*_1$ and $v^*_2$ are the connecting-vertices
and the link-edges are depicted with thick lines.
The figure is adapted from~\cite{Ido:2024aa}.}
\label{fig:polymer_example}  \end{figure} 


For polymers,
we mainly focus on the case of homopolymer, i.e., a linear sequence
of identical repeating units connected by two specific edges, $e^*_0$ and $e^*_1$,
such that two adjacent units in the sequence are joined with them.
The two edges are referred to as the {\em connecting-edges},
and the two vertices incident 
to the two connecting-edges are called the {\em connecting-vertices}.  
An example of these concepts can be found in Figure~\ref{fig:polymer_example}(a).
% Figure~\ref{fig:polymer_example}(a) provides an example
% of a repeating unit of such a polymer,
% where $v^*_1$ and $v^*_2$ are the connecting-vertices. 
 
We call an edge $e$ a  {\em link-edge} 
 in a repeating unit of a polymer if it is traversed by every path 
connecting $e^*_0$ and $e^*_1$,
and denote the set of link-edges in $\bbC$ by $\Elnk(\bbC)$.
For instance, in  Figure~\ref{fig:polymer_example}(a), the link-edges 
 are $e^*_2,e^*_3,\ldots,e^*_6$. 
Following~\cite{Ido:2024aa},
we treat the two connecting-edges as a single edge $e^*_1$ 
to simplify the representation of the polymer,
as illustrated in Figure~\ref{fig:polymer_example}(b). 
The resulting graph is called
the {\em monomer representation} of the polymer,
and edge $e^*_1$ is also called a {\em link-edge}.
In what follows, we represent polymers by their monomer representations $\bbC$.

%The link-edges in the monomer representation
%of a polymer are distinguished from other edges in order to characterize the topological structure of the polymer. 
%Figure~\ref{fig:example_polymer}
% illustrates an example of a hydrogen-suppressed polymer $\anC$
% with $\Elnk(\bbC)=\{u_1u_{15},$ $u_5u_{15}, u_3u_{16}, u_{16}u_{17},  u_{17}u_{18}, u_4u_{18}\}$. 

The link-edges and connecting-vertices are both used in the feature function defined for polymers.
See Appendix~\ref{sec:descriptor} for more details about this.
Also, we specify the set of link-edges in the seed graph $\GC$ and the lower and upper bounds on the number
of components such as the  link-edges and connecting-vertices
in the topological specification used for a polymer.
See Appendix~\ref{sec:specification} for a more detailed explanation of the topological specification for polymers.






