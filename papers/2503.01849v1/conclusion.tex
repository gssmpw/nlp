\section{Concluding Remarks}\label{sec:conclude}


We proposed an extended inverse QSAR/QSPR model
to infer molecules with desired properties under specified environments in this paper.
Building upon the existing framework \molinfer,
which integrates MILP and machine learning,
we extended its capabilities by integrating the information for each molecule and the environment into a unified feature vector,
formulating the inverse problem as an MILP.

To evaluate the computational effectiveness of our proposed model,
we particularly focused on the property Flory-Huggins $\chi$-parameter,
which characterizes the thermodynamic interaction between the solute and the solvent and is temperature-dependent.
We designed feature vectors that incorporate solute and solvent information, 
with several different ways to integrate the temperature information.
Utilizing machine learning techniques such as
artificial neural networks (ANN),  multiple linear regression with reduced quadratic descriptors (R-MLR), 
and random forest (RF), we conducted numerical experiments to demonstrate
that our proposed feature vectors can achieve competitively high $\Rt$ scores.

Furthermore, we addressed the task of designing solutes
for some desired $\chi$-parameter values when paired with a given solvent
under specified temperature conditions.
By formulating this task as an MILP,
our experimental results showed that polymers with up to 50 non-hydrogen atoms
in their monomer representations could be identified within reasonable computation times.
Additionally, using an efficient graph enumeration algorithm, 
more candidates can be generated rapidly by enumerating chemical isomers of the inferred polymers.
We also validated the quality of the inferred molecules using a simulation-based data set using the simulation software \jocta,
demonstrating that $\chi$-parameter values calculated using \jocta\ software were relatively closely aligned
with the ones obtained from our method. This underscores the accuracy and practical utility of our approach.
We believe that our proposed framework represents a significant advancement in this field,
with the potential to drive innovation in molecular design and related applications.



%We acknowledge that $\chi$-parameter value can be theoretically expressed 
%as a linear or a second-order equation~\cite{Knychaa:2017aa, Aoki:2023aa} like:
%\[
%\chi = A + \frac{B}{T},
%\]
%or 
%\[
%\chi = A + \frac{B}{T} + \frac{C}{T^2},
%\]
%where $A$, $B$ and $C$ are coefficients related to the solute-solvent pair and environments (e.g., volume fraction, pressure).
%It will be an interesting future work to  integrate this equation into our framework by,
%for example, directly modeling the relationship between the coefficients and solute-solvent pairs,
%and improve the performance further.

Future research could focus on designing more sophisticated feature functions 
to better capture environmental complexities and their interactions with molecular systems,
further enhancing the framework's predictive and inferential performance.
Also, we are interested in applying the framework
to the problem of deciding whether chemical reactions occur or not among several molecules and specific environments,
which is of great importance.
