\section{An Extended Framework for Multiple Molecules and Environments}\label{sec:formulation}



In this section, we describe how we extend the framework \molinfer\ 
%described in Section~\ref{sec:frame_all}
to take multiple molecules into consideration
and include information about the environment.
The extended framework still consists of two phases like \molinfer, Phase~1 for the QSAR/QSPR phase,
and Phase~2 for the inverse QSAR/QSPR phase.

\subsection{Phase~1}\label{sec:formulation_phase1}

\begin{figure}[t!]
\begin{center} 
 \includegraphics[width=.95\columnwidth]{figures/figure_1.eps}
\end{center}
\caption{An illustration of Phase~1 of the extended framework.
}
\label{fig:figure_1}  
\end{figure} 
%The main modification for Phase 1 in the extended framework is to define a feature function in Stage 2
%that takes both the multiple molecules and the environments into consideration.

Figure~\ref{fig:figure_1} shows Phase~1 of the extended framework.

Let $k$ and $m$ be two positive integers.
In Stage~1,
given a property $\pi$,
we define an instance of $\pi$ to be a $(k+1)$-tuple $(\bbC_1, \bbC_2, ..., \bbC_k, \tau)$,
where
$k$ denotes the number of chemical graphs $\bbC_1, \bbC_2, ..., \bbC_k$ involved in determining the value of the property $\pi$,
$\bbC_i, i \in[1,k]$ denotes the $i$-th chemical graph,
$m$ is the number of different kinds of environments,
and $\tau \in \bbR^m$ is an $m$-dimension real vector that represents the environment.
The observed property value corresponding to the $(k+1)$-tuple $(\bbC_1, \bbC_2, ..., \bbC_k, \tau)$,
i.e., $k$ chemical graphs $\bbC_1, \bbC_2, ..., \bbC_k$ under the environment $\tau$,
 is denoted by $a(\bbC_1, \bbC_2, ..., \bbC_k, \tau)$. 
A data set for the property $\pi$ is a set $D_\pi$ of  $(k+1)$-tuples $(\bbC_1, \bbC_2, ..., \bbC_k, \tau)$,
such that  $a(\bbC_1, \bbC_2, ..., \bbC_k, \tau)$ is available for every $(k+1)$-tuple. 
The purpose of Stage 1 is to collect data sets for the property $\pi$.

In Stage~2, let $f_i: \calG \to \bbR^{K_i}$ be a feature function defined for the $i$-th chemical graph $\bbC_i$, 
and $f_\tau: \bbR^m \to \bbR^{K_\tau}$ be a feature function that characterizes the environment $\tau$,
we define the feature function $f_\pi: D_\pi \to \bbR^{K_\pi}$ for the property $\pi$ to be the concatenation of the ones for each chemical graph and the one for the environment:
\[
f_\pi(\bbC_1, \bbC_2, ..., \bbC_k, \tau) \triangleq [ f_1(\bbC_1) \ f_2(\bbC_2) \  \cdots \ f_k(\bbC_k) \  f_\tau(\tau) ],
\]
where $K_\pi \triangleq \sum_{i=1}^k K_{i} + K_\tau$.
This feature function integrates the information from $k$ molecules and the environment.
Different choices of feature functions for each molecule and the environment may influence
the learning performance in Stage~3. 
We will have a brief discussion about the choices of the feature function for the environment
in the context of $\chi$-parameter in Section~\ref{sec:exp_phase1}.

Stage~3 is the same as \molinfer, i.e., it utilizes machine learning methods like linear regression or ANN 
to construct a prediction function $\eta$ between the feature vector $f_\pi(\bbC_1, \bbC_2, ..., \bbC_k, \tau)$ and the observed value
$a(\bbC_1, \bbC_2, ..., \bbC_k, \tau)$.


 
%{\color{red} Some illustration?}



%%Given a solute-solvent pair $(\bbC_1, \bbC_2)$, we first design a feature function $f(\bbC_1, \bbC_2) \in \bbR^K$ for the pair, and then construct a prediction function $\eta: \bbR^K \to \bbR$ by some machine learning method.
%To define a feature function $f_\chi$ for the triplet $(\bbC_1, \bbC_2, T)$, 
%we use a similar fashion as in\cite{Aoki:2023aa} to capture the information of the solute-solvent pair  $(\bbC_1, \bbC_2)$ by first characterizing the solute and solvent separately and then combining them together.
%Let $f(\bbC)$ denote the feature vector defined in~\cite{Ido:2021aa} (resp., \cite{Zhu:2022ad}) in the case that $\bbC$ is a polymer (resp., monomer),
%%(See also Appendix ??? for a detailed description)
%we define the feature vector of the solute-solvent pair as the concatenation of the two feature vectors of each chemical graph:
%\[
%f_\chi(\bbC_1, \bbC_2) \triangleq [ f(\bbC_1) \ f(\bbC_2)]
%\] 
%
%To make use of the temperature information, it is a reasonable choice to add the temperature in the feature vector as well.
%%, to make the model temperature-dependent.
%We choose the following variants and will compare the learning performance among them in Section~\ref{sec:experiment}.
%\begin{itemize}
%\item[-] $f_{\chi, 1}(\bbC_1, \bbC_2, T) \triangleq [ f(\bbC_1, \bbC_2) \ T ]$;
%\item[-] $f_{\chi,-1}(\bbC_1, \bbC_2, T) \triangleq [ f(\bbC_1, \bbC_2) \ \frac{1}{T} ]$;
%\item[-] $f_{\chi,1, -1}(\bbC_1, \bbC_2, T) \triangleq [ f(\bbC_1, \bbC_2) \ T \ \frac{1}{T} ]$,
%\end{itemize}
%where $T\in \bbR_+$ denotes the absolute temperature.
%%For simplicity, we will use a general notation $f(\bbC_1, \bbC_2, T)$ to mean either the feature vector $f(\bbC_1, \bbC_2)$ which does not include the temperature information or any of the above three ones with different usages of temperature information.
%
%
%We used three different machine learning methods to 
%construct a prediction function $\eta$ from the feature vector $f_\chi(\bbC_1, \bbC_2, T)$ 
%to the observed value $a(\bbC_1, \bbC_2, T)$, 
%namely artificial neural network (ANN), 
%multiple linear regression with reduced quadratic descriptors (R-MLR)~\cite{Zhu:2022ac}, 
%and random forest (RF). 
%We refer~\cite{Azam:2021ab} for ANN, \cite{Zhu:2022ac} for R-MLR, and~\cite{Tanaka:2021aa} \qqq for RF to see a detailed description of them. 
%{\color{red} Also explain something about the machine learning methods here?}

\subsection{Phase~2}\label{sec:formulation_phase2}

\begin{figure}[t!]
\begin{center} 
 \includegraphics[width=.95\columnwidth]{figures/figure_2.eps}
\end{center}
\caption{An illustration of Phase 2 of the extended framework.
}
\label{fig:figure_2}  
\end{figure} 



We illustrate Phase~2 of the extended framework in Figure~\ref{fig:figure_2}.

%We formulate the task of Phase 2 as follows.
In Stage~4, 
besides the  topological specification $\sigma$ and a desired range $[\ylb, \yub]$ of the target value,
we also require a specified environment $\tau^*$, a subset $J \subseteq [1,k]$ ($J$ can be an empty set), 
and a set of chemical graphs $\bbC_j^\mathrm{F}, j \in J$.
This specified environment $\tau^*$ is introduced due to the practical consideration that
the experimental conditions are often fixed in real-world scenarios,
and the subset $J$ is included to account for
the possibility that certain molecules may be fixed
in the inverse phase.

Let $\tilde{J} := [1,k] \setminus J$,
we formulate an MILP formulation to infer the remaining $|\tilde{J}|$ chemical graphs $\bbC_j^\dagger, j \in \tilde{J}$
such that $\eta(f_\pi(\bbC_1^*, \bbC_2^*, ..., \bbC_k^*, \tau^*)) \in [\ylb, \yub]$, 
where $\bbC_j^*$ denotes the inferred chemical graph $\bbC_j^\dagger$ (resp., the fixed chemical graph $\bbC_j^\mathrm{F}$) 
if $j \in \tilde{J}$ (resp., $j \in J$).


In Stage~5,  we use some graph enumeration algorithms for each chemical graph $\bbC_j^\dagger$, $j \in \tilde{J}$ to obtain more candidates, as in \molinfer.

 
%{\color{red} Some illustration?}

%In the second phase, the task is 
%to infer the triplet $(\bbC_1, \bbC_2, T)$ 
%when given a desired range  $[\ylb, \yub]$ of target value of $\chi$-parameter and a topological specification $\sigma$, and a prediction function $\eta$ with the feature vector $f_\chi(\bbC_1, \bbC_2, T)$ obtained in Phase 1, 
%such that $\eta(f_\chi(\bbC_1, \bbC_2, T)) \in [\ylb, \yub]$.
%\qqq
%Particularly, we fix the solvent $\bbC_2$ and absolute temperature $T$ to infer the solute $\bbC_1$, which is usually a polymer.
%\qqq
%{\color{red} 
%Meanwhile, in the practical situation, the solvent $\bbC_2$ and temperature $T$ are usually considered to be constant.
%People want to design a solute $\bbC_1$ such that it enjoys some desired $\chi$-parameter value when interacting with
%the fixed solvent $\bbC_2$ under the fixed temperature.
%Taking this into consideration, in this paper,
%we mainly focus on the task
%to infer a solute $\bbC_1$  such that the solute-solvent pair consisting of the solute and a fixed solvent $\bbC_2$ has a desired 
%$\chi$-parameter value under a fixed temperature $T$.
%}
