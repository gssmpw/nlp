 
\section{Test Instances for Phase~2}\label{sec:test_instances} %%%%%%%%%
%{\color{red} Need to check this carefully !!!}

We prepared the following instances $I_{\mathrm{a}}$ and $I_{\mathrm{b}}$
 for conducting experiments
of  Stages~4  and 5 in Phase~2
for the two data sets $D_\chi \in \{$\chiaoki, \chinistane $\}$,
and instances $I_{\mathrm{c1}}$ and $I_{\mathrm{c2}}$
of Stage~4 for the data set \chijsol.
 
% In Stages~4 and 5, we  use four properties 
% $\pi\in \{${\sc AmD,  HcL, RfId, Tg}$\}$
% and define a set $\Lambda(\pi)$ of chemical elements as follows:  \\
%  $\Lambda(${\sc AmD}$)=\Lambda_4=\{\ttH,\ttC,\ttN,\ttO, \ttCl, \ttS_{(2)} \}$,  
%   $\Lambda(${\sc HcL}$)= \Lambda(${\sc Tg}$)=\Lambda_5= \{\ttH, \ttC, \ttO, \ttN,  \ttCl, \ttS_{(2)}, \ttS_{(6)} \}$,
%  $\Lambda(${\sc RfId}$)=\Lambda_6=\{\ttH, \ttC, \ttO_{(1)}, \ttO_{(2)}, \ttN,  \ttCl, \ttSi_{(4)}, \ttF \}$  and \\
%  $\Lambda(${\sc Prm}$)=\Lambda_3=\{\ttH, \ttC,  \ttO, \ttN,  \ttCl\}$.
  
 
\begin{figure}[t!]
\begin{center} 
 \includegraphics[width=.89\columnwidth]{figures/four_polymers.eps}
\end{center}
\caption{Illustrations of  four polymers: 
(i)  1$\underline{~}$1-(2-methylPropane)Bis(4-phenyl)carbonate;
(ii) 2$\underline{~}$2-pentaneBis(4-phenyl)carbonate;
(iii)  1$\underline{~}$1-dichloroethyleneBis(4-phenyl)carbonate;
(iv) thioBis(4-phenyl)carbonate.
Hydrogens are omitted and connecting-edges are depicted with thick lines. 
The figure is adapted from~\cite{Ido:2024aa}.}
\label{fig:four_polymers}  
\end{figure} 
 
\begin{itemize} 
  \item[(a)]  $I_{\mathrm{a}} =(\GC,\sint,\sce)$: The instance
  used in Appendix~\ref{sec:specification} to explain the topological specification.
% For each property $\pi\in \{${\sc AmD,  HcL, RfId, Tg, Prm}$\}$, we replace
% $\Lambda=\{ \ttH,\ttC,\ttN,\ttO, \ttS_{(2)},\ttS_{(6)},\ttP_{(5)}, \ttCl\}$
%in Table~\ref{table:chemical_spec} 
% with $\Lambda(\pi)\cap \{\ttS_{(2)},\ttS_{(6)},\ttP_{(5)}, \ttCl\}$
% and  remove from the $\sce$
% all chemical symbols,  edge-configurations and fringe-configurations
%  that cannot be constructed from the replaced element set 
% (i.e., those containing a chemical element in 
% $\{\ttS_{(2)},\ttS_{(6)}, \ttP_{(5)}, \ttCl\}\setminus \Lambda(\pi)$).
 
 \end{itemize}
 
\begin{itemize} 
  \item[(b)]  $I_{\mathrm{b}} =(\GC,\sint,\sce)$: An instance that
  represents  a set of polymers that includes the four examples of polymers in 
Figure~\ref{fig:four_polymers}.
We set a seed graph  $\GC=(\VC,\EC=\Eew)$ to be the graph  
 with two cycles $C_1$ and $C_2$ in Figure~8(i), %\ref{fig:test_set_fringe-trees_polymers}(i),
 where we set  
$\Et=\EC^\lnk=\{a_1,a_2\}$ and 
$\Eew=\{a_{3},a_{12},\ldots,a_{14}\}$. \\
% 
Set  $\Lambda:= \{ \ttH, \ttC, \ttN, \ttO, \ttS_{(2)}, \ttCl \}$ for each data set $D_\chi \in  \{$\chiaoki, \chinistane $\}$,
% \{{\tt C,O,Cl, S_{(2)}}\}$, XXXX \CHeck
and  set $\Ldg^\inte$ to be
the set of all possible chemical symbols in $\Lambda\times[1,4]$.\\
Set 
$\Gamma^\inte$ (resp.,  $\Gamma^\lnk$)
to be the set of the edge-configurations of the interior-edges
(resp.,  the link-edges)
used in the four examples of polymers in Figure~7. %\ref{fig:four_polymers}.  
Set 
$\Gamma^\inte_\ac$ (resp.,  $\Gamma^\lnk_\ac$) to be
 the set of the adjacency-configurations of the edge-configurations in 
$\Gamma^\inte$ (resp.,  $\Gamma^\lnk$). \\
We specify $n_\LB:=25$ 
and 
set 
$\nint_\LB:=14$, $\nint_\UB:=n^*:=n_\LB+10$,  
$\nlnk_\LB:=2$,  $\nlnk_\UB: =2+\max\{ n_\LB-15, 0\}$.  \\
For each link-edge $a_i\in\Et=\EC^\lnk=\{a_1,a_2\}$, 
set 
 $\ell_\LB(a_i):=2+\max\{\lfloor (n_\LB-15)/4\rfloor,0\}$,  
 $\ell_\UB(a_i):=\ell_\LB(a_i)+5$,
 $\bl_\LB(a_i):=0, \bl_\UB(a_i):=3$, 
 $\ch_\LB(a_i):=0, \ch_\UB(a_i):=5$,  
 $\bd_{2,\LB}(a_i):=0$ and $\bd_{2,\UB}(a_i):= \lfloor \ell_\LB(a_i)/3 \rfloor$.\\
To form two benzene rings from the two cycles $C_1$ and $C_2$, set 
   $\Lambda^*(u):=\{{\tt C}\}$, 
 $\bl_\LB(u):=\bl_\UB(u):=\ch_\LB(u):=\ch_\UB(u):=0$, $u\in \VC$,
 $\bd_{2,\LB}(a_i):=\bd_{2,\UB}(a_i):=0,  i\in\{3,5,7,9,11,13\}$,
 $\bd_{2,\LB}(a_i):=\bd_{2,\UB}(a_i):=1,  i\in\{4,6,8,10,12,14\}$.\\
Not to include any triple-bond, set 
 $\bd_{3,\LB}(a):=\bd_{3,\UB}(a):=0, a\in \EC$.
 \\
Set lower bounds
 $\na_\LB$,  $\na^\inte_\LB$,  $\ns^\inte_\LB$,  $\ns^\cnt_\LB$, 
$\ac^\inte_\LB$, $\ac_\LB^\lnk$, $\ec_\LB^\inte$, $\ec_\LB^\lnk$ and  $\ac^\lf_\LB$  to be 0. \\
%
Set  upper bounds   
 $\na_\UB(\ta):=n^*, \ta\in\{\ttH,\ttC\}$,   
 $\na_\UB(\ta):=5+\max\{ n_\LB-15, 0\}, \ta\in\{\ttO,\ttN\}$,
 $\na_\UB(\ta):=2+\max\{\lfloor (n_\LB-15)/4\rfloor,0\}, 
 \ta\in\Lambda\setminus \{\ttH,\ttC,\ttO,\ttN\}$,  
  $\ns^\cnt_\UB(\mu):=2, \mu\in\Ldg^\inte$, %\newone
 and   $\na^\inte_\UB$,  $\ns^\inte_\UB$, 
$\ac^\inte_\UB$, $\ac_\LB^\lnk$, $\ec_\UB^\inte$,  $\ec_\UB^\lnk$ 
and  $\ac^\lf_\UB$ to be  $n^*$. \\
% 
%
Set $\mathcal{F}$ to be the set of the 17 chemical rooted trees $\psi_i,i\in[1,17]$
 in  Figure~8(ii). %\ref{fig:test_set_fringe-trees_polymers}(b).  
Set $\mathcal{F}_E :=\mathcal{F}(v) := \mathcal{F}$, $v\in \VC$ and 
$\fc_\LB(\psi):=0, \psi\in \mathcal{F}$,
$\fc_\UB(\psi_i):=12+\max\{ n_\LB-15, 0\}, i\in[1,4]$, 
$\fc_\UB(\psi_i):=8+\max\{\lfloor (n_\LB-15)/2\rfloor,0\}, i\in[5,12]$ and
$\fc_\UB(\psi_i):=5+\max\{\lfloor (n_\LB-15)/4\rfloor,0\}, i\in[13,17], \psi_i\in \mathcal{F}$. 

\item[(c1)] $I_{\mathrm{c1}}=(\GC,\sint,\sce)$: An instance that represents a relatively simple polymer structure.
The seed graph is illustrated in Figure~\ref{fig:figure_c1c2}(a),
where we set  
$\Ew=\{a_1 \}$, $\Eew=\{ a_2, a_3\}$, and $\EC^\lnk=\{ a_1, a_2, a_3 \}$. \\
% 
Set  $\Lambda:= \{ \ttH, \ttC, \ttN, \ttO, \ttCl \}$ 
and  set $\Ldg^\inte$ to be
the set of all possible chemical symbols in $\Lambda\times[1,4]$.\\
Set 
$\Gamma^\inte$ (resp.,  $\Gamma^\lnk$)
to be the set of all the edge-configurations of the interior-edges
(resp.,  the link-edges)
appeared in the data set \chijsol.
Set 
$\Gamma^\inte_\ac$ (resp.,  $\Gamma^\lnk_\ac$) to be
 the set of all the adjacency-configurations of the edge-configurations in 
$\Gamma^\inte$ (resp.,  $\Gamma^\lnk$). \\
We 
set $n_\LB:=\nint_\LB:=\nlnk_\LB:=3$,
$n^*:=\nint_\UB:=\nlnk_\UB:=10$.  \\
For each link-edge $a_i\in\Et=\EC^\lnk=\{a_1,a_2,a_3\}$, 
set 
 $\ell_\LB(a_i):=1$,  
 $\ell_\UB(a_1):=5$, $\ell_\UB(a_2):=\ell_\UB(a_3):=1$
 $\bl_\LB(a_i):=0, \bl_\UB(a_i):=0$, 
 $\ch_\LB(a_i):=0, \ch_\UB(a_1):=1,  \ch_\UB(a_2):=0,  \ch_\UB(a_3):=2$,  
 $\bd_{2,\LB}(a_i):=0$ and $\bd_{2,\UB}(a_1):= 1, \bd_{2,\UB}(a_2):= 3, \bd_{2,\UB}(a_3):= 0$.\\
Not to include any triple-bond, set 
 $\bd_{3,\LB}(a):=\bd_{3,\UB}(a):=0, a\in \EC$.
 \\
Set lower bounds
 $\na_\LB$,  $\na^\inte_\LB$,  $\ns^\inte_\LB$,  $\ns^\cnt_\LB$, 
$\ac^\inte_\LB$, $\ac_\LB^\lnk$, $\ec_\LB^\inte$, $\ec_\LB^\lnk$ and  $\ac^\lf_\LB$  to be 0. \\
%
Set  upper bounds   
 $\na_\UB(\ta):=n^*, \ta\in\{\ttH,\ttC,\ttN,\ttO\}$,   
 $\na_\UB(\ta):=5, \ta=\ttCl$,
  $\ns^\cnt_\UB(\mu):=2, \mu\in\Ldg^\inte$, %\newone
  $\ac^\lf_\UB(\nu):=5, \nu\in\Gac^\lf$,
 and   $\na^\inte_\UB$,  $\ns^\inte_\UB$, 
$\ac^\inte_\UB$, $\ac_\LB^\lnk$, $\ec_\UB^\inte$,  and  $\ec_\UB^\lnk$ 
 to be  $n^*$. \\
% 
%
Set $\mathcal{F}$ to be the set of  all the chemical rooted trees that appeared in the data set \chijsol.
Set $\mathcal{F}_E :=\mathcal{F}(v) := \mathcal{F}$, $v\in \VC$ and 
$\fc_\LB(\psi):=0, \fc_\UB(\psi):=n^*, \psi\in \mathcal{F}$.

\item[(c2)] $I_{\mathrm{c2}}=(\GC,\sint,\sce)$: An instance that represents a polymer structure with a benzene ring included.
The seed graph is illustrated in Figure~\ref{fig:figure_c1c2}(b),
where we set  
$\Ew=\{a_1, a_2 \}$, $\Eew=\{ a_3, a_4, ..., a_{10}\}$, and $\EC^\lnk=\{ a_1, a_2, a_9, a_{10} \}$. \\
% 
Set  $\Lambda:= \{ \ttH, \ttC, \ttN, \ttO, \ttCl \}$ 
and  set $\Ldg^\inte$ to be
the set of all possible chemical symbols in $\Lambda\times[1,4]$.\\
Set 
$\Gamma^\inte$ (resp.,  $\Gamma^\lnk$)
to be the set of all the edge-configurations of the interior-edges
(resp.,  the link-edges)
appeared in the data set \chijsol.
Set 
$\Gamma^\inte_\ac$ (resp.,  $\Gamma^\lnk_\ac$) to be
 the set of all the adjacency-configurations of the edge-configurations in 
$\Gamma^\inte$ (resp.,  $\Gamma^\lnk$). \\
We 
set $n_\LB:=\nint_\LB:=10$, $\nlnk_\LB:=10$,
$\nint_\UB:=\nlnk_\UB:=15$, and $n^*:=25$.  \\
For each link-edge $a_i\in\Et=\EC^\lnk=\{a_1,a_2,a_9,a_{10}\}$, 
set 
 $\ell_\LB(a_i):=1$,  
 $\ell_\UB(a_1):=\ell_\UB(a_2):=5$, $\ell_\UB(a_9):=\ell_\UB(a_{10}):=1$
 $\bl_\LB(a_i):=0, \bl_\UB(a_i):=1$, 
 $\ch_\LB(a_i):=0, \ch_\UB(a_1):=\ch_\UB(a_2):=3$,  
 $\bd_{2,\LB}(a_i):=0$ and $\bd_{2,\UB}(a_1):= 1, \bd_{2,\UB}(a_2):=\bd_{2,\UB}(a_9):= 0, \bd_{2,\UB}(a_{10}):= 3$.\\
 To contain a benzene ring, set
   $\Lambda^*(u):=\{{\tt C}\}$, 
 $\bl_\LB(u):=\bl_\UB(u):=\ch_\LB(u):=\ch_\UB(u):=0$, $u\in \{u_2, u_3, ..., u_7  \}$,
 $\bd_{2,\LB}(a_i):=\bd_{2,\UB}(a_i):=0,  i\in\{4,6,8\}$,
 $\bd_{2,\LB}(a_i):=\bd_{2,\UB}(a_i):=1,  i\in\{3,5,7\}$.\\
Not to include any triple-bond, set 
 $\bd_{3,\LB}(a):=\bd_{3,\UB}(a):=0, a\in \EC$.
 \\
Set lower bounds
 $\na_\LB$,  $\na^\inte_\LB$,  $\ns^\inte_\LB$,  $\ns^\cnt_\LB$, 
$\ac^\inte_\LB$, $\ac_\LB^\lnk$, $\ec_\LB^\inte$, $\ec_\LB^\lnk$ and  $\ac^\lf_\LB$  to be 0. \\
%
Set  upper bounds   
 $\na_\UB(\ta):=n^*, \ta=\ttH$,   
  $\na_\UB(\ta):=20, \ta=\ttC$,   
   $\na_\UB(\ta):=10, \ta\in\{\ttN,\ttO\}$,   
 $\na_\UB(\ta):=5, \ta=\ttCl$,
  $\ns^\cnt_\UB(\mu):=2, \mu\in\Ldg^\inte$, %\newone
  $\ac^\lf_\UB(\nu):=5, \nu\in\Gac^\lf$,
   $\na^\inte_\UB(\ta):=20, \ta=\ttC$,   
   $\na^\inte_\UB(\ta):=10, \ta\in\{\ttN,\ttO\}$,   
 $\na^\inte_\UB(\ta):=5, \ta=\ttCl$,
 $\ns^\inte_\UB(\mu):=20, \mu\in\{\ttC2,\ttC3,\ttC4\}$,
 $\ns^\inte_\UB(\mu):=10, \mu\in\Ldg^\inte\setminus\{\ttC2,\ttC3,\ttC4\}$,
 and   
$\ac^\inte_\UB$, $\ac_\LB^\lnk$, $\ec_\UB^\inte$,  and  $\ec_\UB^\lnk$ 
 to be  20. \\
% 
%
Set $\mathcal{F}$ to be the set of  all the chemical rooted trees that appeared in the data set \chijsol.
Set $\mathcal{F}_E :=\mathcal{F}(v) := \mathcal{F}$, $v\in \VC$ and 
$\fc_\LB(\psi):=0, \fc_\UB(\psi):=10, \psi\in \mathcal{F}$.

 \end{itemize}
 