\section{Related Work}
Speech deepfake detection research which was systematized by ASVspoof challenge campaings~\cite{Todisco2019ASVspoof2F, ASVspoof2021, Wang2024ASVspoof5C}. Such systematization and data collection has marked an interest in the development of speech deepfake detectors to such an extent that {\em equal error rates} (EERs) obtain are easily $1<\%$ when the training and evaluation sets are coming from the same corpus. But such a results do not carry over to cross-corpus studies, where training is done on one corpus and testing on another~\cite{müller2024doesaudiodeepfakedetection}. But this is precisely the speech deepfake generalization task. 

Regularization~\cite{Chen2020}
Used continual learning in finetuning to make sure that finetuned model does not have catastrophic forgetting~\cite{Ma2021}. Authors used full corpora to finetune the model. This in contrast to~\cite{kukanov2024metalearningapproachesimprovingdetection}, where authors used meta-learning to learn a few-shot finetuning. Authors noticed that only a few samples from the unknown attacks were enough to significantly reduce the EER. In contrast to the present work, we do not {\em any} samples from the unknown attacks. 

In~\cite{Kawa2022}, authors pooled data from multiple corpora and applied 5-fold cross-validation to improve the generalization performance of the LCNN classifier~\cite{lavrentyeva2019stcantispoofingsystemsasvspoof2019}. And finally, speech foundation models have also been used with the idea that such a model after finetuning with the speech deepfake corpora will work well on the unseen attacks~\cite{oneata2023generalisable, Wang2023}. In the present work, we take this as a baseline where our results are compared against. This system is called in our Tables Wav2Vec-AASIST. Parameters to be finetuned is extremely large, namely 317M, comparing to proposed model that obtains better performance with only 840K parameters.