\section{Related Works}
\label{Rel}
Nowell et al.~\cite{liben2007link} proposed a list of local and global similarity based LP methods. Similarity based methods calculate a score between two nodes to estimate their chance to form a link. Local similarity based methods are neighborhood based, and apply to two-hop away node pairs. The global similarity based methods are mostly path based and apply to any node pair in the network. In this paper, we consider most of the methods proposed in Nowell et al. in our experiments along with another popular local method: Resource Allocation~\cite{zhou2009predicting}. There are a few other similarity based methods in literature, an exhaustive list of which can be found in Kumar et al.~\cite{kumar2020link}. We do not include another type of similarity based method, namely quasi-local~\cite{kumar2020link}, in our study. Modern machine learning based LP methods are either supervised~\cite{zhang2018link,cai2020multi,pan2021neural,tan2023bring} or unsupervised~\cite{perozzi2014deepwalk,grover2016node2vec,hamilton2017inductive,he2020lightgcn}. The supervised methods train a neural network or a graph neural network in an end-to-end manner with the test node pairs. The unsupervised methods first train a neural network or a graph neural network to learn low dimension embeddings of the nodes in an unsupervised manner (known as contrastive learning), and then produce link features which are used to build a classifier to predict links. In this paper, we consider the unsupervised methods in our experiments. The main reason to exclude the end-to-end methods is the difficulty in building the experimental set-up for those in the longitudinal setup for the future LP, which incurs a longitudinal bias~\cite{lichtenwalter2010new,yang2015evaluating} on the prediction performance, as well as makes the methods incomparable with their missing LP counterpart. Another group of unsupervised LP approaches are matrix factorization based~\cite{menon2011link,chen2017link}, which we do not include in our experiments. Note that, the purpose of this work is not to present an exhaustive performance comparison of the existing methods, but to devise a robust evaluation strategy for the prediction methods considering certain data,  evaluation metric, experimental setup, application and methodology specific factors. So, excluding a few prediction methods does not hurt the merit of this work.

The start-of-the-art LP methods briefed above mainly apply to a simple, undirected and homogeneous networks. There exists works which exploit several network attributes to predict links. Few examples are: temporal link prediction~\cite{sett2018temporal,qin2023temporal} which exploits temporal dynamics of link maintenance in dynamic networks like in social networks; link prediction in bipartite networks~\cite{kunegis2010link,ozer2024link} which applies on bipartite graphs like user-product networks or term-document networks; link prediction in heterogeneous networks~\cite{sett2018temporal,li2018link,wang2023multi} which exploits the heterogeneity of nodes and links in networks like bibliographic networks; link prediction in directed networks~\cite{schall2014link,sett2018exploiting} which exploits the direction information in directed networks like email networks, etc. These specialized methods are not necessarily disjoint, and often use information covering multiple groups. In this paper, we consider the start-of-the-art LP methods only in our analysis, and reserve the specialized methods for future study.

Studies like~\cite{lichtnwalter2012link,yang2015evaluating} investigated the issues associated with LP evaluation in considerable details. They stressed on an imbalanced sampling of test set, and advocated for AUPR and the PR curve to tackle class imbalance in evaluation. Yang et al.~\cite{yang2015evaluating} also proposed a test set-up for supervised future LP from deployment perspective, and analyzed the effect of variation of the test node pair collection time duration. He et al.~\cite{he2024link} investigated how LP accuracy in real-world networks gets affected under non-uniform missing-edge patterns. A few studies like~\cite{junuthula2016evaluating,poursafaei2022towards} investigated the evaluation problem for temporal link prediction. These studies consider the timestamps of test node pairs as an important aspect for evaluation. Kumar et al.~\cite{kumar2020link} presents a nice survey on LP methods and evaluate them using standard metrics like AUROC, AUPR and Precision@K. Mostly, the existing studies on link prediction evaluation deal with the class imbalance issue~\cite{lichtnwalter2012link,yang2015evaluating,kumar2020link,lichtenwalter2010new,masrour2020bursting,nasiri2023robust}, but do not address the early retrieval performance. To the best of our knowledge, there exists no study which considers the multifaceted aspects of LP evaluation in such a detailed manner like us.