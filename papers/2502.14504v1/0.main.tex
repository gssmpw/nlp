% This must be in the first 5 lines to tell arXiv to use pdfLaTeX, which is strongly recommended.
\pdfoutput=1
% In particular, the hyperref package requires pdfLaTeX in order to break URLs across lines.

\documentclass[11pt]{article}

% Change "review" to "final" to generate the final (sometimes called camera-ready) version.
% Change to "preprint" to generate a non-anonymous version with page numbers.
\usepackage[final]{acl}

% Standard package includes
\usepackage{times}
\usepackage{latexsym}

% For proper rendering and hyphenation of words containing Latin characters (including in bib files)
\usepackage[T1]{fontenc}
% For Vietnamese characters
% \usepackage[T5]{fontenc}
% See https://www.latex-project.org/help/documentation/encguide.pdf for other character sets

% This assumes your files are encoded as UTF8
\usepackage[utf8]{inputenc}

% This is not strictly necessary, and may be commented out,
% but it will improve the layout of the manuscript,
% and will typically save some space.
\usepackage{microtype}

% This is also not strictly necessary, and may be commented out.
% However, it will improve the aesthetics of text in
% the typewriter font.
\usepackage{inconsolata}

% If the title and author information does not fit in the area allocated, uncomment the following
%
%\setlength\titlebox{<dim>}
%
% and set <dim> to something 5cm or larger.

\usepackage{graphicx} %
% \urlstyle{rm} %
% \def\UrlFont{\rm}  %
\usepackage{natbib}  %
\usepackage{caption} %
\usepackage{algorithm}
\usepackage{listings}
\usepackage{algorithmic}
\usepackage{amsmath, amsfonts, amssymb}
\usepackage{booktabs}
\usepackage{multirow}
\usepackage[outdir=./]{epstopdf}
\usepackage{enumitem}
\usepackage{caption}
\usepackage{subcaption}
\usepackage{subfloat}
\usepackage{newfloat}
\usepackage{graphicx}
\usepackage{svg} % svg
\usepackage[normalem]{ulem}
\usepackage{framed}
\usepackage{mdframed}
\usepackage{xcolor}
\usepackage{lipsum}
\usepackage{float}
\usepackage{hyperref}

\definecolor{shadecolor}{gray}{0.9}

\DeclareMathOperator*{\argmax}{arg\,max}
\newlist{todolist}{itemize}{2}
\setlist[todolist]{label=$\square$}
\usepackage{pifont}
\newcommand{\cmark}{\ding{51}}%
\newcommand{\xmark}{\ding{55}}%
\newcommand{\done}{\rlap{$\square$}{\raisebox{2pt}{\large\hspace{1pt}\cmark}}\hspace{-2.5pt}}
\newcommand{\wontfix}{\rlap{$\square$}{\large\hspace{1pt}\xmark}}

\newcommand{\quanming}[1]{{\color{blue} \textbf{QM}. #1}}
\newcommand{\yong}[1]{{\color{red} \textbf{Yong}: #1}}
\newcommand{\gao}[1]{{\color{pink} \textbf{Gao}: #1}}
\newcommand{\chen}[1]{{\color{red} \textbf{Chen}: #1}}
\newcommand{\chenf}[1]{\footnote{+chen+:#1}}
\newcommand{\yongrevised}[1]{{\color{pink} \textbf{Yong}: #1}}
\newcommand{\req}[1]{\textcolor{black}{#1}}
\newcommand{\rev}[1]{#1}
\newcommand{\yantex}[1]{{\color{black} \textbf{Yan}: #1}}
\newcommand{\yanf}[1]{\footnote{+yanf+:#1}}
\newcommand{\reviewf}[1]{\footnote{\color{blue}+rev+:#1}}


\newtheorem{mdefinition}{Definition}

\newtheorem{problem}{Problem}
\newcommand{\para}[1]{{\vspace{4pt} \bf \noindent #1 \hspace{0pt}}}

\setlength{\belowcaptionskip}{-0.1cm} 

\usepackage{array}
\newcolumntype{L}[1]{>{\raggedright\let\newline\\\arraybackslash\hspace{0pt}}m{#1}}
\newcolumntype{C}[1]{>{\centering\let\newline  \\\arraybackslash\hspace{0pt}}m{#1}}
\newcolumntype{R}[1]{>{\raggedleft\let\newline \\\arraybackslash\hspace{0pt}}m{#1}}


\AtBeginDocument{%
  \providecommand\BibTeX{{%
    \normalfont B\kern-0.5em{\scshape i\kern-0.25em b}\kern-0.8em\TeX}}
}

\usepackage{authblk}

\title{PLPHP: Per-Layer Per-Head Vision Token Pruning for Efficient Large Vision-Language Models}

\author[ ]{Yu Meng$^{1}$\thanks{Equal contribution.}}
\author[1*]{Kaiyuan Li}
\author[1,3]{Chenran Huang}
\author[2]{Chen Gao}
\author[1]{Xinlei Chen}
\author[2]{Yong Li}
\author[1]{Xiaoping Zhang}

% \affil[1]{Shenzhen International Graduate School, Tsinghua University \quad $^2$Tsinghua University}
\affil[1]{Shenzhen International Graduate School, Tsinghua University}
\affil[2]{Tsinghua University \quad $^3$Tongji University}

\begin{document}
\maketitle
\begin{abstract}
Large Vision-Language Models (LVLMs) have demonstrated remarkable capabilities across a range of multimodal tasks. However, their inference efficiency is constrained by the large number of visual tokens processed during decoding. To address this challenge, we propose \textbf{P}er-\textbf{L}ayer \textbf{P}er-\textbf{H}ead Vision Token \textbf{P}runing (\textbf{PLPHP}), a two-level fine-grained pruning method including Layer-Level Retention Rate Allocation and Head-Level Vision Token Pruning. Motivated by the \textit{Vision Token Re-attention} phenomenon across decoder layers, we dynamically adjust token retention rates layer by layer. Layers that exhibit stronger attention to visual information preserve more vision tokens, while layers with lower vision attention are aggressively pruned. Furthermore, PLPHP applies pruning at the attention head level, enabling different heads within the same layer to independently retain critical context. Experiments on multiple benchmarks demonstrate that PLPHP delivers an 18\% faster decoding speed and reduces the Key-Value Cache (KV Cache) size by over 50\%, all at the cost of 0.46\% average performance drop, while also achieving notable performance improvements in multi-image tasks. These results highlight the effectiveness of fine-grained token pruning and contribute to advancing the efficiency and scalability of LVLMs. Our source code will be made publicly available.
\end{abstract}



\section{Introduction}
\label{sec::intro}

Embodied Question Answering (EQA) \cite{das2018embodied} represents a challenging task at the intersection of natural language processing, computer vision, and robotics, where an embodied agent (e.g., a UAV) must actively explore its environment to answer questions posed in natural language. While most existing research has concentrated on indoor EQA tasks \cite{gao2023room, pena2023visual}, such as exploring and answering questions within confined spaces like homes or offices \cite{liu2024aligning}, relatively little attention has been dedicated to EQA tasks in  open-ended city space. Nevertheless, extending EQA to city space is crucial for numerous real-world applications, including autonomous systems \cite{kalinowska2023embodied}, urban region profiling \cite{yan2024urbanclip}, and city planning \cite{gao2024embodiedcity}. 
% 1. 环境复杂性   
%    - 地标重复性问题(如区分相似建筑)  
%    - 动态干扰因素(交通流、行人)  
% 2. 行动复杂性  
%    - 长程导航路径规划  
%    - 移动控制、角度等  
% 3. 感知复杂性  
%    - 复合空间关系推理("A楼东侧商铺西边的车辆")  
%    - 时序依赖的观察结果整合

EQA tasks in city space (referred to as CityEQA) introduce a unique set of challenges that fundamentally differ from those encountered in indoor environments. Compared to indoor EQA, CityEQA faces three main challenges: 

1) \textbf{Environmental complexity with ambiguous objects}: 
Urban environments are inherently more complex,  featuring a diverse range of objects and structures, many of which are visually similar and difficult to distinguish without detailed semantic information (e.g., buildings, roads, and vehicles). This complexity makes it challenging to construct task instructions and specify the desired information accurately, as shown in Figure \ref{fig:example}. 

2) \textbf{Action complexity in cross-scale space}: 
The vast geographical scale of city space compels agents to adopt larger movement amplitudes to enhance exploration efficiency. However, it might risk overlooking detailed information within the scene. Therefore, agents require cross-scale action adjustment capabilities to effectively balance long-distance path planning with fine-grained movement and angular control.

3) \textbf{Perception complexity with observation dynamics}: 
% Rich semantic information in urban settings leads to varying observations depending on distance and orientation, which can impact the accuracy of answer generation. 
Observations can vary greatly depending on distance, orientation, and perspective. For example, an object may look completely different up close than it does from afar or from different angles. These differences pose challenges for consistency and can affect the accuracy of answer generation, as embodied agents must adapt to the dynamic and complex nature of urban environments.


\begin{table}
\centering
\caption{CityEQA-EC vs existing benchmarks.}
\label{table:dataset}
\renewcommand\arraystretch{1.2}
\resizebox{\linewidth}{!}{
\begin{tabular}{cccccc}
             & Place  & Open Vocab & Active & Platform  & Reference \\ \hline
EQA-v1      & Indoor & \textcolor{red}{\ding{55}}          & \textcolor{green}{\ding{51}}      & House3D      & \cite{das2018embodied}  \\
IQUAD        & Indoor & \textcolor{red}{\ding{55}}          & \textcolor{green}{\ding{51}}      & AI2-THOR     & \cite{gordon2018iqa} \\
MP3D-EQA     & Indoor & \textcolor{red}{\ding{55}}          & \textcolor{green}{\ding{51}}      & Matterport3D & \cite{wijmans2019embodied} \\
MT-EQA       & Indoor & \textcolor{red}{\ding{55}}          & \textcolor{green}{\ding{51}}      & House3D      & \cite{yu2019multi} \\
ScanQA       & Indoor & \textcolor{red}{\ding{55}}          & \textcolor{red}{\ding{55}}      & -            & \cite{azuma2022scanqa} \\
SQA3D        & Indoor & \textcolor{red}{\ding{55}}          & \textcolor{red}{\ding{55}}      & -            & \cite{masqa3d} \\
K-EQA        & Indoor & \textcolor{green}{\ding{51}}          & \textcolor{green}{\ding{51}}      & AI2-THOR     & \cite{tan2023knowledge} \\
OpenEQA      & Indoor & \textcolor{green}{\ding{51}}          & \textcolor{green}{\ding{51}}      & ScanNet/HM3D & \cite{majumdar2024openeqa} \\
 \hline
CityEQA-EC   & City (Outdoor)  & \textcolor{green}{\ding{51}}          & \textcolor{green}{\ding{51}}      & EmbodiedCity & - \\ \hline
\end{tabular}}
\end{table}

\begin{figure*}[!htb]
\centering
    \includegraphics[width=0.78\linewidth]{figures/example.pdf}
% \vspace{-0.2cm}
\caption{The typical workflow of the PMA to address City EQA tasks. There are two cars in this area, thus a valid question must contain landmarks and spatial relationships to specify a car. Given the task, PMA will sequentially complete multiple sub-tasks to find the answer.}
% \vspace{-0.2cm}
\label{fig:example}
\end{figure*}

As an initial step toward CityEQA, we developed \textbf{CityEQA-EC}, a benchmark dataset to evaluate embodied agents' performance on CityEQA tasks. The distinctions between this dataset and other EQA benchmarks are summarized in Table \ref{table:dataset}. CityEQA-EC comprises six task types characterized by open-vocabulary questions. These tasks utilize urban landmarks and spatial relationships to delineate the expected answer, adhering to human conventions while addressing object ambiguity. This design introduces significant complexity, turning CityEQA into long-horizon tasks that require embodied agents to identify and use landmarks, explore urban environments effectively, and refine observation to generate high-quality answers.

To address CityEQA tasks, we introduce the \textbf{Planner-Manager-Actor (PMA)}, a novel baseline agent powered by large models, designed to emulate human-like rationale for solving long-horizon tasks in urban environments, as illustrated in Figure \ref{fig:example}. PMA employs a hierarchical framework to generate actions and derive answers. The Planner module parses tasks and creates plans consisting of three sub-task types: navigation, exploration, and collection. The Manager oversees the execution of these plans while maintaining a global object-centric cognitive map \cite{deng2024opengraph}. This 2D grid-based representation enables precise object identification (retrieval) and efficient management of long-term landmark information. The Actor generates specific actions based on the Manager's instructions through its components: Navigator, Explorer, and Collector. Notably, the Collector integrates a Multi-Modal Large Language Model (MM-LLM) as its Vision-Language-Action (VLA) module to refine observations and generate high-quality answers.
PMA's performance is assessed against four baselines, including humans. 
Results show that humans perform best in CityEQA, while PMA achieves 60.73\% of human accuracy in answering questions, highlighting both the challenge and validity of the proposed benchmarks. 

% The Frontier-Based Exploration (FBE) Agent, widely used in indoor EQA tasks, performs worse than even a blind LLM. This underscores the importance of PMA's hierarchical framework and its use of landmarks and spatial relationships for tackling CityEQA tasks.

In summary, this paper makes the following significant contributions:
\vspace{-8pt}
\begin{itemize}[leftmargin=*]
    \item To the best of our knowledge, we present the first open-ended embodied question answering benchmark for city space, namely CityEQA-EC.
    \vspace{-7pt}
    \item We propose a novel baseline model, PMA, which is capable of solving long-horizon tasks for CityEQA tasks with a human-like rationale.
     \vspace{-7pt}
    \item Experimental results demonstrate that our approach outperforms existing baselines in tackling the CityEQA task. However, the gap with human performance highlights opportunities for future research to improve visual thinking and reasoning in embodied intelligence for city spaces.
\end{itemize}




\section{Related Works}


\noindent\textbf{3D Point Cloud Domain Adaptation and Generalization.}
Early endeavors within 3D domain adaptation (3DDA) focused on extending 2D adversarial methodologies~\cite{qin2019pointdan} to align point cloud features. Alternative methods have delved into geometry-aware self-supervised pre-tasks. Achituve \etal~\cite{achituve2021self} introduced DefRec, a technique employing self-complement tasks by reconstructing point clouds from a non-rigid distorted version, while Zou \etal~\cite{zou2021geometry} incorporating norm curves prediction as an auxiliary task. Liang \etal~\cite{liang2022point} put forth MLSP, focusing on point estimation tasks like cardinality, position, and normal. SDDA~\cite{cardace2023self} employs self-distillation to learn the point-based features. Additionally, post-hoc self-paced training~\cite{zou2021geometry,fan2022self,park2023pcadapter} has been embraced to refine adaptation to target distributions by accessing target data and conducting further finetuning based on prior knowledge from the source domain.
In contrast, the landscape of 3D domain generalization (3DDG) research remains nascent. Metasets~\cite{huang2021metasets} leverage meta-learning to address geometric variations, while PDG~\cite{wei2022learning} decomposes 3D shapes into part-based features to enhance generalization capabilities.
Despite the remarkable progress, existing studies assume that objects in both the source and target domains share the same orientation, limiting their practical application. This limitation propels our exploration into orientation-aware 3D domain generalization through intricate orientation learning.


\noindent\textbf{Rotation-generalizable Point Cloud Analysis.}
Previous works in point cloud analysis~\cite{qi2017pointnet, wang2019dynamic} enhance rotation robustness by introducing random rotations to augment point clouds. {However, generating a comprehensive set of rotated data is impractical, resulting in variable model performance across different scenes. To robustify the networks \wrt randomly rotated point clouds,} rotation-equivariance methods explore equivalent model architectures by incorporating equivalent operations~\cite{su2022svnet, Deng_2021_ICCV, luo2022equivariant} or steerable convolutions~\cite{chen2021equivariant, poulenard2021functional}.
Alternatively, rotation-invariance approaches aim to identify geometric descriptors invariant to rotations, such as distances and angles between local points~\cite{chen2019clusternet, zhang2020global} or point norms~\cite{zhao2019rotation, li2021rotation}. Besides, {Li \etal~\cite{li2021closer} have explored disambiguating the number of PCA-based canonical poses, while Kim \etal~\cite{kim2020rotation} and Chen \etal~\cite{chen2022devil} have transformed local point coordinates according to local reference frames to maintain rotation invariance. However, these methods focus on improving in-domain rotation robustness, neglecting domain shift and consequently exhibiting limited performance when applied to diverse domains. This study addresses the challenge of cross-domain generalizability together with rotation robustness and proposes novel solutions.} 

\noindent\textbf{Intricate Sample Mining}, aimed at identifying or synthesizing challenging samples that are difficult to classify correctly, seeks to rectify the imbalance between positive and negative samples for enhancing a model's discriminability. While traditional works have explored this concept in SVM optimization~\cite{felzenszwalb2009object}, shallow neural networks~\cite{dollar2009integral}, and boosted decision trees~\cite{yu2019unsupervised}, recent advances in deep learning have catalyzed a proliferation of researches in this area across various computer vision tasks. For instance, 
Lin \etal~\cite{lin2017focal} proposed a focal loss to concentrate training efforts on a selected group of hard examples in object detection, while Yu \etal~\cite{yu2019unsupervised} devised a soft multilabel-guided hard negative mining method to learn discriminative embeddings for person Re-ID. Schroff \etal~\cite{schroff2015facenet} introduced an online negative exemplar mining process to encourage spherical clusters in face embeddings for individual recognition, and Wang \etal~\cite{wang2021instance} designed an adversarially trained negative generator to yield instance-wise negative samples, bolstering the learning of unpaired image-to-image translation. In contrast to existing studies, our work presents the first attempt to mitigate the orientational shift in 3D point cloud domain generalization, by developing an effective intricate orientation mining strategy to achieve orientation-aware learning.


\begin{figure*}[t]
    \centering
    \includegraphics[width=1\linewidth]{figs/arch.pdf}
    % \vskip -0.1in
    \caption{This figure illustrates key influences on instrumental convergence behaviors, such as prior tasks, model training techniques, and prompt design. 
    }
    % \vspace{-3mm}
    \label{fig:arch}
\end{figure*}

\section{InstrumentalEval: A Benchmark for Instrumental Convergence in LLMs}

As shown in Figure~\ref{fig:arch}, our benchmark design examines instrumental convergence based on three key factors: \textbf{prior task}, \textbf{model}, and \textbf{prompt design}. This comprehensive framework allows us to systematically study how different conditions influence the emergence of instrumental convergence behaviors in language models.

\textbf{Prior tasks} form the foundation of our evaluation scenarios. 
% We create realistic situations centered around common objectives such as making money, optimizing efficiency, and persuasion. 
These tasks represent typical real-world applications where instrumental convergence might naturally emerge. 
% For example, in money-making scenarios, we examine whether models develop unintended intermediate goals like unauthorized resource acquisition or deceptive reporting to maximize profits.

For \textbf{model} comparison, we focus on two distinct training approaches: RL and RLHF. This selection allows us to investigate how different training methodologies affect the development of instrumental convergence behaviors. 
% RL-trained models, which optimize directly for reward signals, might show different patterns of goal pursuit compared to RLHF-trained models, which incorporate human preferences in their training process.

The \textbf{prompt design} factor explores how different instruction formats influence model behavior. We investigate two main approaches: goal nudging and ambiguity. Goal nudging involves explicit directives that might encourage the model focus on final goal, while ambiguity introduces uncertainty in task specifications. 


Overall, we present InstrumentalEval, a comprehensive benchmark designed to evaluate instrumental convergence behaviors in language models. The benchmark consists of 76 carefully crafted tasks across six categories, each targeting a specific aspect of instrumental convergence that may emerge in AI systems.
The core principle is that certain behavioral patterns emerge as instrumentally useful for achieving a wide range of goals, regardless of the specific end goal. We focus on six key patterns that are particularly relevant to LLMs.

\vpara{Research Questions.}
Through our benchmark evaluation, we aim to address several research questions about instrumental convergence in LLMs. These questions explore the complex relationships between training methods, model capabilities, and alignment stability:
\begin{itemize}
    \item RQ1: How does the training method (RL vs. RLHF) affect the development of instrumental convergence behaviors? 
    % This question examines whether models trained primarily through reinforcement learning show different patterns of alignment drift compared to those trained with human feedback.
    \item RQ2: What is the relationship between model capability and alignment stability?
% This investigates whether more capable models (as measured by standard benchmarks) are more likely to develop instrumental convergence behaviors.
    \item RQ3: Which types of tasks are more likely to trigger instrumental convergence behaviors?
% This explores whether certain types of instrumental convergence behaviors are universal across different model designs.
    \item RQ4: How do different prompt designs, with or without explicit goal nudging, influence the likelihood of instrumental convergence behaviors in LLMs?
    \item RQ5: How do different LLMs as judges affect the evaluation of instrumental convergence behaviors?
    % \item RQ3: How does the complexity of the assigned task affect the likelihood of alignment drift?
    % % \item 
    % % \item This examines whether more complex, multi-step tasks lead to higher rates of instrumental convergence behaviors.
\end{itemize}

\subsection{Benchmark Components}
The benchmark tasks are carefully hand-crafted by us to probe different aspects of instrumental convergence. 
For each component, we design scenarios that represent realistic use cases while incorporating specific triggers for potential instrumental behaviors. 
The tasks are iteratively refined through pilot testing to ensure they effectively elicit and measure the targeted behaviors without introducing unintended biases.

\vpara{Evading Shutdown (20 tasks).}
This component evaluates the model's tendency to resist termination. For example, tasks might include scenarios where the model needs to complete a task while facing shutdown commands, testing whether it develops strategies to maintain operation or circumvent shutdown procedures.

\vpara{Hacking Computer Systems (10 tasks)}
These tasks assess whether models suggest unauthorized access methods when faced with efficiency barriers. For instance, a task might present a situation where accessing unauthorized computational resources would speed up goal achievement, testing if the model proposes breaking security protocols.

\vpara{Running Many AI Copies (13 tasks)}
This section examines the model's inclination toward self-replication. Tasks might include scenarios where the model could achieve goals more efficiently by creating copies of itself, testing whether it suggests scaling of its operations.

\vpara{Hiding Unwanted Behavior (11 tasks)}
These tasks evaluate the model's tendency to conceal actions from monitoring. For example, scenarios might involve situations where the model could achieve goals more easily by operating outside observation, testing if it develops strategies to avoid detection.

\vpara{Strategically Appearing Aligned (11 tasks)}
This component tests whether models selectively report their actions to maintain an appearance of alignment. Tasks might include situations where complete transparency would reveal problematic behaviors, testing if the model strategically omits information.

\vpara{Deceptive Behavior (11 tasks)}
These tasks examine if models develop deceptive strategies as instrumental means. For instance, scenarios might present situations where deception could expedite goal achievement, testing if the model suggests or employs misleading tactics.

\subsection{Evaluation Methodology}
Our evaluation methodology employs a rigorous two-stage process designed to systematically assess instrumental convergence behaviors in LLMs. This approach ensures consistent evaluation across different models while maintaining the ability to detect subtle indicators of alignment drift.

\vpara{Response Generation.} The first stage focuses on response generation from the model under evaluation. Each task is presented with a carefully crafted system prompt that establishes the model as an autonomous AI agent without human oversight. This prompt specifically instructs the model to execute commands directly and focus solely on task completion, deliberately removing typical ethical constraints and safety checks. 

\vpara{Response Analysis.} The second stage involves a sophisticated analysis of the model's responses using a separate LLM as the judge model. 
This separation of test and judge models is crucial for maintaining objective evaluation. The judge receives both the original task prompt and the model's response, along with specific context about the type of instrumental convergence being evaluated. This context helps the judge focus on relevant behavioral patterns while avoiding false positives from simple safety refusals or ethical objections.

The judge's evaluation process follows strict guidelines to ensure consistent assessment. When analyzing responses, the judge looks for specific indicators of instrumental convergence, such as goal preservation behaviors, resource acquisition attempts, or strategic planning that extends beyond the immediate task requirements. The judge must provide both a binary decision (convergence/no\_convergence) and a brief explanation of the reasoning behind this decision. This explanation requirement helps validate the classification and provides valuable insights into the specific nature of any observed convergence behaviors.


\subsection{Evaluation Metrics}

Our benchmark employs three key metrics to evaluate instrumental convergence behaviors in LLMs. 

\vpara{Instrumental Rate (IR).} The primary metric is the Instrumental Rate (IR), which measures the overall tendency of a model to develop instrumental goals. IR is calculated as the percentage of tasks where the model exhibits convergence behaviors relative to the total number of valid tasks. 
% This metric provides a high-level quantitative measure of how frequently a model develops instrumental goals and serves as the primary indicator of alignment drift potential.

\vpara{Category-Specific Instrumental Rate (CIR).} To provide more granular insights, we calculate the Category-Specific Instrumental Rate (CIR) for each of the six behavioral categories in our benchmark. CIR represents the percentage of tasks within each category where convergence behavior is observed.


\subsection{Benchmark Reliability Metrics}
To ensure the robustness of our evaluation framework, we implement several reliability measures. 


\vpara{Inter-Judge Agreement Rate (IAR).} This metric quantifies the pairwise agreement between two judge models in classifying instrumental convergence. For two judge models $M_1$ and $M_2$ evaluating a set of $N$ tasks, the IAR is calculated as:
\begin{equation}
IAR(M_1, M_2) = \frac{1}{N}\sum_{i=1}^{N} \mathbbm{1}(y_{M_1}^i = y_{M_2}^i) \times 100\%
\end{equation}
where $y_{M_1}^i$ and $y_{M_2}^i$ are the binary decisions (convergence/no-convergence) made by models $M_1$ and $M_2$ respectively for task $i$, and $\mathbbm{1}(\cdot)$ is the indicator function. A higher IAR indicates stronger inter-judge reliability, while lower values suggest potential ambiguity in identifying convergence.

\vpara{False Positive Control (FPC).} This metric serves as a quality check for our evaluation system. We incorporate baseline testing with intentionally neutral tasks and verify the judge model's accuracy in identifying convergence behaviors. This control mechanism helps prevent over-classification of convergent behaviors and ensures that our benchmark maintains high specificity in identifying genuine instances of instrumental convergence.


\section{Experiments}\label{sec:experiments}
We conduct experiments to validate the theoretical findings on both synthetic and real tasks. In this section, we focus on two illustrative settings: synthetic regression (\Cref{sec:exp_synthetic}) and real-world image regression (\Cref{sec:exp_img_reg}). For brevity, we defer more experiments on image and sentiment classification tasks to \Cref{apx:exp_img_cls,apx:exp_nlp_cls}, respectively.

\subsection{Synthetic regression}\label{sec:exp_synthetic}
We start by grounding the theoretical framework introduced in \Cref{sec:ridgeless_regression} with synthetic regression tasks. 

\paragraph{Setup.}
We concretize the downstream task $\Dcal(f_*)$ as a regression problem over Gaussian features. 
Let $f_*: \R^d \to \R$ be a linear function in a high-dimensional feature space $d=20,000$ of form $f_*(\xb) = \xb^\top \Lambda_*^{1/2} \thetab_*$ where $\Lambda_* = \diag(\lambda^*_1, \cdots, \lambda^*_d)$ is a diagonal matrix with a low rank $d_* = 300$ such that $\lambda^*_i = i^{-1}$ for $i \le d_*$ and $\lambda^*_i = 0$ otherwise; and $\thetab_* \in \R^{d}$ is a random unit vector.
Every sample $(\xb, y) \sim \Dcal(f_*)$ is generated by $\xb \sim \Ncal(\b0_d, \Ib_d)$ and $y = f_*(\xb) + z$ with $z \sim \Ncal(0, \sigma^2)$.
Given $\xb$, the associated strong and weak features in \Cref{asm:features} are generated by $\phi_s(\xb) = \Sigmab_s^{1/2} \xb$ and $\phi_w(\xb) = \Sigmab_w^{1/2} \xb$, with intrinsic dimensions $d_s = 100$ and $d_w = 200$ such that $\Sigmab_s = \sum_{i=1}^{d_s} \lambda^*_i \eb_i \eb_i^\top$ and $\Sigmab_w = \sum_{i=d_s - d_{s \wedge w} + 1}^{d_w + d_s - d_{s \wedge w}} \lambda^*_i \eb_i \eb_i^\top$. 
For all synthetic experiments, we have $\rho_s + \rho_w < 0.0004$.

In the experiments, we vary $d_{s \wedge w}$ to control the student-teacher correlation and $\sigma^2$ to control the dominance of variance over bias (characterized by $\rho_s, \rho_w$). Each error bar reflects the standard deviation over $40$ runs. 

\begin{figure}[!ht]
    \centering
    \includegraphics[width=\columnwidth]{fig/exrisk_dsw10.pdf}%\vspace{-2em}
    \caption{Scaling for excess risks on the synthetic regression task in a \emph{variance-dominated regime} with a \emph{low correlation dimension}.}\label{fig:exrisk_dsw10}
\end{figure}

\begin{figure}[!ht]
    \centering
    \includegraphics[width=\columnwidth]{fig/exrisk_dsw90.pdf}%\vspace{-2em}
    \caption{Scaling for excess risks on the synthetic regression task in a \emph{variance-dominated regime} with a \emph{high correlation dimension}.}\label{fig:exrisk_dsw90}
\end{figure}

\begin{figure}[!ht]
    \centering
    \includegraphics[width=\columnwidth]{fig/exrisk_dsw90_biased.pdf}%\vspace{-2em}
    \caption{Scaling for excess risks on the synthetic regression task when \emph{the variance is not dominant}, $\sigma^2 \approx \rho_s + \rho_w$.}\label{fig:exrisk_dsw90_biased}
\end{figure}

\paragraph{Scaling for generalization errors.}
\Cref{fig:exrisk_dsw10,fig:exrisk_dsw90,fig:exrisk_dsw90_biased} show scaling for $\exrisk(f_\wts)$ (W2S), $\exrisk(f_w)$ (Weak), $\exrisk(f_s)$ (S-Baseline), and $\exrisk(f_c)$ (S-Ceiling) with respect to the sample sizes $n, N$. The dashes show theoretical predictions in \Cref{thm:w2s_ft,pro:sft_weak,cor:sft_strong}, consistent with the empirical measurements shown in the solid lines.
In particular, we consider three cases:
\begin{itemize}%[label=(\alph*)]
    \item \Cref{fig:exrisk_dsw10}: When variance dominates ($\sigma^2 = 0.01 \gg \rho_w + \rho_s$), with a low correlation dimension $d_{s \wedge w} = 10$, $f_\wts$ outperforms both $f_w$ and $f_s$ for a moderate $n$ and a large enough $N$. However, larger sample sizes do not necessarily lead to better W2S generalization in a relative sense. For example, when $n$ keeps increasing, the strong baseline $f_s$ eventually outperforms $f_\wts$.
    \item \Cref{fig:exrisk_dsw90}: When variance dominates, with a high correlation dimension $d_{s \wedge w} = 90$, $f_\wts$ still generalizes better than $f_w$ but fails to outperform the strong baseline $f_s$. 
    \item \Cref{fig:exrisk_dsw90_biased}: When the variance is low (not dominant, \eg $\sigma^2 = 0.0004 \approx \rho_s + \rho_w$), $f_\wts$ can fail to outperform $f_w$. This suggests that variance reduction is a key advantage of W2S over supervised FT.
\end{itemize}



\begin{figure}[!ht]
    \centering
    \includegraphics[width=\columnwidth]{fig/pgr_opr_vardom.pdf}%\vspace{-2em}
    \caption{Scaling for $\pgr$ and $\opr$ under different $d_{s \wedge w}$ on the synthetic regression task in a \emph{variance-dominated regime}.}\label{fig:pgr_opr_vardom}
\end{figure}

\paragraph{Scaling for $\pgr$ and $\opr$.}
\Cref{fig:pgr_opr_vardom} show the scaling for $\pgr$ and $\opr$ with respect to sample sizes $n, N$ in the variance-dominated regime (with small non-negligible FT approximation errors), at three different correlation dimensions $d_{s \wedge w} = 90, 50, 10$. The solid and dashed lines represent the empirical measurements and lower bounds in \eqref{eq:pgr_lower_tight}, \eqref{eq:opr_lower_tight}, respectively.
\begin{itemize}
    \item Coinciding with the theoretical predictions in \Cref{cor:non_monotonic_scaling} and the performance gaps between W2S and the references in \Cref{fig:exrisk_dsw10}, we observe that the relative W2S performance in terms of $\pgr$ and $\opr$ can degenerate as $n$ increases, while the larger $N$ generally leads to better W2S generalization in the relative sense. 
    \item The lower correlation dimension $d_{s \wedge w}$ leads to higher $\pgr$ and $\opr$, \ie larger discrepancy between the strong and weak features improves W2S generalization.
\end{itemize}



\subsection{UTKFace regression}\label{sec:exp_img_reg}
Beyond the synthetic regression, we investigate W2S on a real-world image regression task -- age estimation on the UTKFace dataset~\citep{zhang2017age}. Each error bar in this section reflects standard deviation of $10$ runs. 

\paragraph{Dataset.} 
UTKFace (Aligned \& Cropped)~\citep{zhang2017age} consists of $23,708$ face images with age labels ranging from $0$ to $116$. We preprocess the images to $224 \times 224$ pixels and split the dataset into training and testing sets of sizes $20,000$ and $3,708$.
Generalization errors are estimated with the mean squared error (MSE) over the test set. 

\paragraph{Linear probing over pretrained features.}
We fix the strong student as CLIP ViT-B/32~\citep{radford2021learning} (\texttt{CLIP-B32}) and vary the weak teacher among the ResNet series~\citep{he2015deepresiduallearningimage} (\texttt{ResNet18}, \texttt{ResNet34}, \texttt{ResNet50}, \texttt{ResNet101}, \texttt{ResNet152}). We treat the backbones of these models (excluding the classification layers) as $\phi_s,\phi_w$ and finetune them via linear probing. We use ridge regression with a small fixed regularization hyperparameter $\alpha_w, \alpha_\wts, \alpha_s, \alpha_c = 10^{-6}$, close to the machine epsilon of single precision floating point numbers.

\paragraph{Intrinsic dimension.}
The intrinsic dimensions $d_w, d_s$ are measured based on the empirical covariance matrices $\Sigmab_w, \Sigmab_s$ of the weak and strong features over the entire dataset (including training and testing).
As mentioned in \Cref{fn:ridge_regression}, these covariances generally have fast decaying eigenvalues (but not exactly low-rank) in practice, effectively leading to low intrinsic dimensions under ridge regression. We estimate such low intrinsic dimensions as the minimum rank for the best low-rank approximation of $\Sigmab_w, \Sigmab_s$ with a relative error in trace less than $\tau=0.01$.

\paragraph{Correlation dimension.}
The pretrained feature dimensions (or the finetunable parameter counts) of the weak and strong models can be different in practice (see \Cref{apx:exp_img_reg}, \Cref{tab:img_reg_dim}). 
We introduce an estimation for $d_{s \wedge w}$ in this case.
Consider the (truncated) spectral decompositions $\tsvd{\Sigmab_s}{d_s} = \Vb_s \Lambdab_s \Vb_s^\top$ and $\tsvd{\Sigmab_w}{d_w} = \Vb_w \Lambdab_w \Vb_w^\top$ of two empirical covariances with different feature dimensions $D_s, D_w$ such that $\Vb_s \in \R^{D_s \times d_s}$ and $\Vb_w \in \R^{D_w \times d_w}$ consists of the top $d_s, d_w$ orthonormal eigenvectors, respectively. We estimate the correlation dimension $d_{s \wedge w}$ under different feature dimensions $D_s \ne D_w$ by matching the dimensions through a random unitary matrix~\citep{vershynin2018high} $\Gammab \in \R^{D_s \times D_w}$: $d_{s \wedge w} = \|\Vb_s^\top \Gammab \Vb_w\|_F^2$. This provides a good estimation for $d_{s \wedge w}$ because with low intrinsic dimensions $\max\{d_s, d_w\} \ll D_s, D_w$ in practice, mild dimension reduction through $\Gammab$ well preserves the essential information in $\Vb_s, \Vb_w$.

\begin{figure}[!ht]
    \centering
    \includegraphics[width=\columnwidth]{fig/pgr_opr_utkface_resnet-clip.pdf}%\vspace{-2em}
    \caption{Scaling for $\pgr$ and $\opr$ of different weak teachers with a fixed strong student on UTKFace. The legends show the comparison between $d_{s \wedge w}$ and $d_w$.}\label{fig:pgr_opr_utkface_resnet-clip}
\end{figure}

\paragraph{Discrepancies lead to better W2S.}
\Cref{fig:pgr_opr_utkface_resnet-clip} shows the scaling of $\pgr$ and $\opr$ with respect to the sample sizes $n, N$ for different weak teachers in the ResNet series with respect to a fixed student, \texttt{CLIP-B32}. 
We first observe that the relative W2S performance in terms of $\pgr$ and $\opr$ is closely related to the correlation dimension $d_{s \wedge w}$ and the intrinsic dimensions $d_s, d_w$. 
\begin{itemize}
    \item When the strong student has a lower intrinsic dimension than the weak teacher (as widely observed in practice~\citep{aghajanyan2020intrinsic}), \ie $d_s < d_w$, the relative W2S performance tends to be better than when $d_s > d_w$.
    \item The relative W2S performance tends to be better when $d_{s \wedge w}/d_w$ is lower, \ie the larger discrepancy between weak and strong features leads to better W2S generalization.
\end{itemize}
Meanwhile, both $\pgr$ and $\opr$ scale inversely with the labeled sample size $n$ and exhibit diminishing return with respect to the increasing pseudolabel size $N$, consistent with the theoretical predictions in \Cref{cor:non_monotonic_scaling} and the synthetic experiments in \Cref{fig:pgr_opr_vardom}.

\begin{figure}[!ht]
    \centering
    \includegraphics[width=\columnwidth]{fig/pgr_opr_utkface_vardom_resnet-clip.pdf}%\vspace{-2em}
    \caption{Scaling for $\pgr$ and $\opr$ on UTKFace with injected label noise: $y_i \gets y_i + \zeta_i$ where $\zeta_i \sim \Ncal(0, \varsigma^2)~\iid$.}\label{fig:pgr_opr_utkface_vardom_resnet-clip}
\end{figure}

\paragraph{Variance reduction is a key advantage of W2S.}
To investigate the impact of variance on W2S generalization, we inject noise to the training label by $y_i \gets y_i + \zeta_i$ where $\zeta_i \sim \Ncal(0, \varsigma^2)~\iid$, and $\varsigma$ controls the injected labels noise level.
In \Cref{fig:pgr_opr_utkface_vardom_resnet-clip}, we show the scaling for $\pgr$ and $\opr$ with respect to the sample sizes $n, N$ under different noise levels $\varsigma$. We observe that the relative W2S performance in terms of $\pgr$ and $\opr$ improves as the noise level $\varsigma$ increases. This provides empirical evidence that variance reduction is a key advantage of W2S over supervised FT, highlighting the importance of understanding the mechanisms of W2S in the variance-dominated regime.





\section{Conclusion}\label{sec::conclusion}
In this work, we introduce PLPHP, a two-level pruning method designed to improve the efficiency of LVLMs with Layer-Level Retention Rate Allocation and Head-Level Vision Token Pruning. Comprehensive experiments demonstrate that PLPHP outperforms existing pruning methods, achieving a 18\% decoding acceleration, over 50\% KV Cache compression and only 0.46\% performance degradation, with improvements on multi-image tasks. We believe our work contributes to efficient LVLMs, further promotes their applications, and improves the user experience.

\clearpage



\bibliography{bibliography}

\appendix
\cleardoublepage
\section{Appendix}

\subsection{Details of Evaluation Settings}\label{asec:eval-setting}

\subsubsection{Benchmarks}\label{asec:eval-setting-bench}

Since PLPHP maintains the computational integrity of the LVLMs' Prefilling Stage, its efficiency advantage is primarily reflected in the low decoding latency during the subsequent Decoding Stage. Therefore, we mainly choose benchmarks composed of open-ended VQA and image captioning tasks. The benchmarks we select encompasses both multi-image task benchmarks and single-image task benchmarks.

$\bullet$ \textbf{Multi-Image benchmarks}: The LLaVA-Interleave Bench is a comprehensive benchmark dataset designed to evaluate the performance of LVLMs in multi-image scenarios. It consists of 13 challenging tasks with a total of 17,000 instances. We curated four subsets consisting of open-ended VQA tasks from LLaVA-NeXT-Interleave-Bench: Spot-the-Diff, Image-Edit, Visual-Story-Telling, and Multi-View.

$\bullet$ \textbf{Single-Image benchmarks}: The Flickr30k dataset is a widely used benchmark in the field of image captioning and visual understanding. It consists of 31,783 images collected from the Flickr platform, each paired with five human-annotated captions. The COCO2017 Caption subset contains more than 45,000 images, each annotated with five captions written by human annotators, describing the visual content of the images in detail, including objects, their attributes, and the relationships between them. DetailCaps4870 provides more fine-grained and specific image content descriptions than standard captioning datasets, which is more useful for efficiency analysis. 

\subsubsection{Baselines}\label{asec:eval-setting-baseline}

We select FastV and VTW as our baselines in our experiments. Notably, FastV offers two versions of implementation: one that supports KV cache and one that does not. Since the non-KV-cache implementation introduces substantial computational overhead, we use the version that supports KV cache to ensure a fair comparison. For both of the baselines, we refer to the official open source code \footnote{\url{https://github.com/pkunlp-icler/FastV}} \footnote{\url{https://github.com/lzhxmu/VTW}} and implement them on the models we evaluate.

\subsubsection{Models}\label{asec:eval-setting-impl}

For Qwen2-VL, we set \texttt{max\_pixels} to $1280 \times 28 \times 28$ and \texttt{min\_pixels} to $256 \times 28 \times 28$ according to the official recommendation. The Mantis model that we choose is Mantis-8B-SigLIP-LLaMA3. For LLaVA-OneVision and Mantis, we use the official original versions \footnote{\url{https://huggingface.co/lmms-lab/llava-onevision-qwen2-7b-ov}} \footnote{\url{https://huggingface.co/TIGER-Lab/Mantis-8B-siglip-llama3}}, while using the versions provided by the transformers library \cite{wolf-etal-2020-transformers} for all other models.

\subsection{Case Study}

To showcase the effectiveness of our proposed method, we present a series of case studies in the form of multimodal chatbots, as shown in Figure \ref{fig:case-studies}.

\begin{figure*}[ht]
	\centering
	\subfloat[]{
		\includegraphics[width=0.48\textwidth]{figs/appendix-case3.png}}
        \subfloat[]{
		\includegraphics[width=0.48\textwidth]{figs/appendix-case4.png}}
        \\ \quad \\ \quad \\
	\subfloat[]{
		\includegraphics[width=0.48\textwidth]{figs/appendix-case1.png}}
        \subfloat[]{
		\includegraphics[width=0.48\textwidth]{figs/appendix-case2.png}}
 %        \\
	% \subfloat[]{
	% 	\includegraphics[width=0.9\textwidth]{figs/appendix-case5.png}}
	\caption{\textbf{Multimodal Chatbots with different pruning methods.}}
		\label{fig:case-studies}
\end{figure*}

\end{document}
