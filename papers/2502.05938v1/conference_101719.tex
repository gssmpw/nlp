\documentclass[10pt]{IEEEtran}
\IEEEoverridecommandlockouts
\pagestyle{empty}
\pdfminorversion=5
\pdfobjcompresslevel=3
\pdfcompresslevel=9
\usepackage{cite}
\usepackage{amsmath,amssymb,amsfonts}
\usepackage[linesnumbered,ruled,vlined]{algorithm2e} % Removed `algorithm` & `algorithmic`
\usepackage{graphicx}
\usepackage{xcolor}
\usepackage{booktabs}
\usepackage[top=1in, bottom=0.8in, left=0.75in, right=0.75in]{geometry}
\usepackage[hidelinks]{hyperref}
\begin{document}
\title{{Energy-Efficient Autonomous Aerial Navigation with Dynamic Vision Sensors: A Physics-Guided Neuromorphic Approach}}

\author{Sourav Sanyal*, Amogh Joshi*,  Manish Nagaraj, Rohan Kumar Manna, \& Kaushik Roy\\
  \IEEEauthorblockA{Electrical and Computer Engineering, Purdue University
}}
% \author{\IEEEauthorblockN{Anonymous Authors\\ Submission Copy: Do not distribute}}
\maketitle
% \SetAlgoNlRelativeSize{-1} % Ensure line numbers are not too large
% \SetAlCapSkip{1em}        % Adjust caption spacing
% \SetAlgoSkip{0.5em}       % Reduce algorithm spacing


\begin{abstract}



Vision-based object tracking is a critical component for achieving autonomous aerial navigation, particularly for obstacle avoidance. Neuromorphic Dynamic Vision Sensors (DVS) or event cameras, inspired by biological vision, offer a promising alternative to conventional frame-based cameras. These cameras can detect changes in intensity asynchronously, even in challenging lighting conditions, with a high dynamic range and resistance to motion blur. Spiking neural networks (SNNs) are increasingly used to process these event-based signals efficiently and asynchronously. Meanwhile, physics-based artificial intelligence (AI) provides a means to incorporate system-level knowledge into neural networks via physical modeling. This enhances robustness, energy efficiency, and provides symbolic explainability. In this work, we present a neuromorphic navigation framework for autonomous drone navigation. The focus is on detecting and navigating through moving gates while avoiding collisions. We use event cameras for detecting moving objects through a shallow SNN architecture in an unsupervised manner. This is combined with a lightweight energy-aware physics-guided neural network (PgNN) trained with depth inputs to predict optimal flight times, generating near-minimum energy paths. The system is implemented in the Gazebo simulator and integrates a sensor-fused vision-to-planning neuro-symbolic framework built with the Robot Operating System (ROS) middleware. This work highlights the future potential of integrating event-based vision with physics-guided planning for energy-efficient autonomous navigation, particularly for low-latency decision-making. 

\begin{IEEEkeywords}
Neuromorphic Computing, Dynamic Vision Sensors, Autonomous Navigation, Spiking Neural Network, Physics-guided Neural Network
\end{IEEEkeywords}

% \noindent{Video Link: \url{https://youtu.be/9Gnjpb1k2Lo}}

\end{abstract}
\thispagestyle{empty}
% \begin{IEEEkeywords}
% component, formatting, style, styling, insert
% \end{IEEEkeywords}

\section{Introduction}
\documentclass[../main.tex]{subfiles}
\graphicspath{{../images/}}
\makeatletter
\def\input@path{{../images/}}
\makeatother
\begin{document}
\section{Introduction}
\begin{figure}
\centering
\begin{tikzpicture}
\node[inner sep=0pt] (ws) at (0, 0) {
\includegraphics[height=.4\textwidth, trim={10cm 0 10cm 0},clip]{world_space.png}};
\node[inner sep=0pt] (cs) at (6,0) {\includegraphics[height=.4\textwidth, trim={10cm 1cm 10cm 4cm},clip]{conf_space.png}};
\end{tikzpicture}
\vspace{-5pt}
\label{fig:pbrm_intro}
\caption{\textbf{Left}: Shows world space obstacles as grey spheres. Robots start and goal configuration is colored red and green, respectively. Configurations along the computed path are colored transparent blue. \textbf{Right:} Mapped world space scenario to configuration space. Obstacle region is the grey mesh. Red spheres are collision-free regions computed by the neural SCDF. The optimized shortest path in the convex corridor is the blue curve.}
\vspace{-25pt}
\end{figure}
Motion planning is the problem of finding a collision-free trajectory that connects a given start and goal configuration. The planning takes place in the configuration space of the robot. For single body robots, like mobile robots or drones, the configuration space and the world space are usually the same. This simplifies the planning, since explicit obstacle representations are available which enables geometrical tools like separating hyperplanes, smallest distance to obstacles etc., to be used when designing motion planning algorithms. For multi-body robots like manipulators, the situation is completely different. The world space obstacles are usually mapped to non-convex regions, and to make the problem even harder, the mapping is usually not known. Forming explicit representations of the obstacle region in the configuration space is usually too expensive or intractable. Despite all of this, sampling based planners are used with great success, which mainly is due to their use of implicit representations of the obstacle region. The basic idea is to construct a graph in the configuration space that covers and connects the collision-free region. From this graph, a path can be extracted that connects a given start and goal configuration. The approach is computationally expensive, since the graph is constructed with the smallest geometrical building block available, points, which represents a collision-check. Furthermore, the extracted paths from the graph are non-smooth and jagged due to the stochastic nature of the approach. This adds an additional post-processing step to the process, where the paths are shortcutted and smoothened, before the path can be used for tracking. Clearly a lot of time is invested to form this graph and produce smooth paths. Thus, if the obstacles start to move, then all of this work is done in no use, since all points that make up this graph need to be re-verified, which is simply too time consuming to be done in real time.
\\\\
In this work, we want to address the existing drawbacks of the sampling based planners. Our main contribution is an improved motion planner where each vertex in the graph covers a collision-free region in the form of a sphere instead of a point and where the edges are formed with neighboring intersecting spheres. This representation has the advantage of instead of returning piecewise linear paths, returning a sequence of overlapping spheres, i.e. a convex corridor, that connects a given start and goal configuration, illustrated in Figure \ref{fig:pbrm_intro}. This convex corridor allows us to use convex optimization to produce smooth trajectories, instead of computationally expensive post-processing methods. The representation further allows us to estimate the coverage of the collision-free space, which gives us awareness and feedback in the offline roadmap construction phase. Finally, our representation is simple to adapt to moving obstacles, simply requery for the new radii and recheck for intersections. 
\\\\
The spherical collision-free regions are formed using a signed distance function (SDF), which is a function that returns the smallest distance from an arbitrary point to the boundary of an obstacle. As the name implies, the distance is signed, thus if the point is inside the obstacle it is negative otherwise positive. If the distance is positive, a sphere with radius equal to the distance is guaranteed to cover a collision-free region. Using an SDF in motion planning is not new, but what is novel about our approach is that we express the distance in the configuration space instead of the world space and by doing so allows us to form these convex collision-free regions. We refer to the resulting SDF as a signed configuration distance function (SCDF). Computing an SCDF analytically is non-trivial, our approach is therefore to parameterize the SCDF with a deep neural network and learn the mapping by supervised learning. Our resulting neural SCDF can compute distances for different parameter values of obstacle shapes and we also show how multiple distances can be combined, thus making our approach flexible.
\section{Related work}
Motion planning algorithms can roughly be divided into three families, grid-based, sampling based and optimization based methods. Grid-based methods (GBM) discretize the planning space from which a graph is then compiled. A standard search method is A$^\star$ \citep{a_star}, which is classified as an \textit{informed} search method, since it employs a heuristic function to speed up the search. A$^\star$ guarantees to return an optimal path at the level of discretization used. GBMs usually discretize the planning space by a regular lattice and this limits the GBMs to problems with low dimensionality due to the curse of dimensionality. Thus, GBMs are usually limited to single-body robots where the degrees of freedom (DOF) are low. To overcome the inherent scaling problem with the GBMs, stochastic methods are usually used for multi-body robots. These methods are termed as sampling-based methods (SBM) and core members within this family are the rapidly-exploring random trees (RRT) \citep{rrt} and the probabilistic roadmap (PRM) \citep{prm}. RRT grows a tree from the start configuration and explores the collision-free region in a rapid way until it is able to connect to the goal region. RRT is usually improved by bi-directional planning \citep{rrt_connect}, i.e. an additional tree is grown from the goal configuration and the trees are tested for connection after any tree has been expanded. RRT is a single-query method, thus it searches for a path from scratch each time it is queried. Contrary to this, PRM is a multi-query method, which solves for multiple queries without starting from scratch. PRM does this by creating a roadmap (graph) that covers the collision-free space as an offline step. The graph is then used to solve for multiple queries. PRMs are used in cases where the environment does not change since the extra offline step is too computationally costly and needs to be re-done if the environment is changed. In our work, we address this inherent issue by using a different roadmap representation. Our vertices in the graph cover a collision-free region in the form of spheres and we form the edges by checking for intersecting spheres. If something in the environment changes, we recompute the spheres radii and recheck the intersections, without relying on collision detection. We use a trained neural network to compute the sphere radius, therefore querying for the radius can be done fast, hence our representation enables the PRM for dynamic environments.
\\\\
In the recent decades, optimization based methods (OBM) \citep{chomp, schulman, itomp, stomp} have been introduced as an alternative to SBM for multi-body robots. Like the SBM, the OBMs scale well to higher dimensional problems and produce smoother motion. It is common to use a SDF in the optimization since it is a smooth function, thus enabling gradient-based methods. However, the standard way of expressing the SDF is in world space. The distance therefore needs to be mapped to the configuration space by the forward kinematics. This mapping makes the optimization problem a non-linear program (NLP), which is computationally expensive to solve. Recently, a different approach has been proposed. In \cite{mp_gcs} motion planning is formulated as a convex optimization problem by using the graph of convex sets framework \citep{gcs}. The underlying idea is to decompose the collision-free space into intersecting convex sets from which a convex optimization problem is formulated. In cases where an explicit representation of the obstacles in the configuration space exists, like for single-body robots, creating collision-free convex regions can be done fast \citep{iris}. For multi-body robots, this is non-trivial. Existing work does this successfully \citep{iris_nlp, iris_c} by an optimization based approach, but the methods are still too time consuming to be used in the presence of moving obstacles. Our approach is instead to use deep learning to learn an SDF expressed in the configuration space. With this, we can query for shortest distances to the collision boundary, which allows us to expand spherical regions which are collision-free. Our approach is fast and therefore enables our suggested roadmap planner to be used in dynamic environments.
\\\\
Recent research has focused on learning collision detection \citep{fk_kernel_distance, diffco, graphdistnet} by predicting the signed distance between the robot links and the surrounding obstacles in the world space. The learned SDF is used in trajectory optimization but since the distance is expressed in the world space, the problem becomes an NLP and therefore takes a long time to solve. We take a novel approach and suggest to instead express the signed distance in the configuration space. This allows us to improve the PRM at the same time as it enables convex optimization for trajectory optimization, which runs faster and is more reliable than NLP solvers. In \cite{cspf} a learned signed distance function in the configuration space is proposed similar to our approach. However, their approach is restricted to point cloud representations, while we propose to represent the obstacles as parameterized geometric shapes, e.g. spheres. Furthermore, we also show how to use our learned SCDF to improve an existing roadmap planner.
\section{Problem formulation}
A robot is located in the world space, $\W \subset \R^3 $. The unique location of the robot is given by its configuration $\q \in \C$, where $\C$ is the configuration space. The set of points covered by the robots bodies at a certain configuration is expressed as $\B(\q) \subset \W$. The robot is surrounded by $\NrObst$ obstacles $\O = \bigcup_{i=1}^{\NrObst} \O_i$, where  $\O_i \subset \W$. The representation of the obstacle in the configuration space is the set $\C\O_i = \{\q \in \C \: |\: \B(\q) \cap \O_i \neq \emptyset \}$. The obstacle space is formed as $\Co = \bigcup_{i=1}^{\NrObst} \C \O_i$. The complement is referred to as the free space, $\Cf = \C \setminus \Co$. The path planning problem is a tuple, ($\Cf$, $\qStart$, $\qGoal$), where we want to connect a query pair, consisting of a start, $\qStart$, and goal configuration, $\qGoal$, with a geometric path, $\q(s): [0, 1] \mapsto \Cf$, such that $\q(0)=\qStart$ and $\q(1)=\qGoal$, or report correctly when such a path does not exist.
\end{document}


\section{Preliminaries}
\section{Basic Background: Supervised Learning and the PAC Model}
\label{sec:background}

At this point almost everyone has heard of machine learning (ML). Anyone likely to stumble upon this article will have also heard of its most influential special case, supervised learning, and those theoretically inclined will also be familiar with the PAC model. Nonetheless, I will set the stage by  recapping the basics.

\subsection{Basics of Supervised Learning}%Let's set the stage in any case

\emph{Supervised Learning} is the task of ``coming up'' with a function $f: \X \to \Y$ to ``explain'' or ``fit'' a sequence of input/output examples   $(x_1,y_1), \ldots, (x_n,y_n)$, with $x_i \in \X$ and $y_i \in \Y$.  Here $\X$ is a \emph{data domain} consisting of \emph{datapoints} $x \in \X$, $\Y$ is a \emph{label set} consisting of \emph{labels} $y \in \Y$, and the sequence $(x_1,y_1),\ldots,(x_n,y_n)$ is the \emph{training data} consisting of \emph{labeled examples (a.k.a. samples)}~$(x_i,y_i)$.  I~will refer to the chosen function $f$ as a \emph{predictor}, and to $n$ as the \emph{sample size}. A \emph{learning algorithm} takes as input training data, and outputs (some representation of) a predictor $f \in \Y^\X$.\footnote{Note that this describes the usual \emph{batch}, a.k.a.~\emph{offline}, setting of supervised learning. I do not discuss other paradigms such as online or active learning in this article.} 



Success in supervised learning is defined as \emph{generalization} to  future examples: For a typical \emph{test example}  $(x_{\tst},y_{\tst})$, the predicted label $y'_{\tst}=f(x_{\tst})$ should ``equal'' $y_{\tst}$, perhaps approximately. We usually assume the test example is drawn from the same  ``source'' as the training data  --- commonly, i.i.d.~from the same distribution. The quality of the prediction is quantified by $\ell(y'_{\tst},y_{\tst})$, where $\ell:~\Y~\times~\Y \to \RR_{\geq 0}$ is a \emph{loss function} chosen as part of the problem definition. Common loss functions include the 0-1 loss $\ell_{0-1}(y',y) = [y' \neq y]$ for \emph{classification} problems,\footnote{The notation $[P]$ denotes $1$ when predicate $P$ is true, and denotes $0$ when $P$ is false.} as well as the absolute loss $|y'-y|$ or squared loss $(y'-y)^2$ for \emph{regression problems} featuring $\Y  \sse \RR$.

Nontrivial generalization properties are typically only possible if one assumes something about the data.\footnote{The need for such an assumption is formalized by the  \emph{no free lunch theorems} of supervised learning \cite{wolpert_connection_1992,wolpert_lack_1996,schaffer_conservation_1994}.} The Bayesian approach to  machine learning, common in many applications, assumes some parametric form for the distribution generating the data, and postulates a prior on the parameters. This is not the approach I will take in this article. Instead, I will focus on the frequentist --- and some would say ``worst-case'' or ``adversarial'' ---  approach that is common in the computational learning theory community, embodied by the PAC model. Here we assume that the (training and test) data can be explained, perhaps approximately, by a function in some ``simple enough to learn'' class of functions $\H \sse \Y^\X$, often called the \emph{hypotheses}. Equivalently, we  seek a predictor which explains the unseen data roughly  as well as the best hypothesis $h^* \in \H$, whether or not we assume that $h^*$ itself provides a perfect explanation.



 \paragraph{Common Algorithmic Templates.} Perhaps the best known general-purpose supervised learning algorithm is \emph{empirical risk minimization (ERM)}, which chooses as its predictor a hypothesis $f \in \H$ minimizing $\frac{1}{n} \sum_{i=1}^n \ell(f(x_i),y_i)$ --- a quantity called the \emph{training error}, \emph{empirical error}, or \emph{empirical risk} of $f$. %\footnote{When multiple hypotheses minimize the empirical risk, we assume ERM breaks ties arbitrarily.}
A common template for generalizing ERM involves adding a \emph{regularization term} $\psi(f)$ to the  objective function, typically chosen to measure some notion of ``hypothesis complexity.'' An algorithm instantiating this template is known as a \emph{structural risk minimizer (SRM)}, and chooses as its predictor the hypothesis $f \in \H$ minimizing the \emph{structural risk} $\frac{1}{n} \sum_{i=1}^n \ell(f(x_i),y_i) + \psi(f)$. Other well-known algorithms, such as gradient descent and its variations,  can frequently be interpreted as approximate implementations of ERM or SRM.


\paragraph{Proper vs Improper Learning.} A learning algorithm is said to be \emph{proper} if its predictor $f$ is always chosen from the hypothesis class, i.e., $f \in \H$, otherwise it is said to be \emph{improper}. ERM  is an example of a proper learning algorithm, as are SRM algorithms of the form described above.  In the \emph{proper regime} of learning, algorithms are required to be proper. This article will be concerned with the more flexible \emph{improper regime} (a.k.a \emph{representation-independent learning}), where no such constraint is placed on the learner. In other words, all we care about is predictive power at test time, rather than any insights derived from the functional form or representation of the predictor~itself.


\subsection{The PAC Model}
A standard mathematical setup for evaluation of supervised learning algorithms, at least in the theoretical computer science community, is Valiant's \emph{Probably Approximately Correct (PAC) model} of learning (see e.g.~\cite{kearns_introduction_1994,mohri_foundations_2018}). Here, we assume there is an unknown distribution $\D$ on $\X \times \Y$ from which training and test data are  drawn.  Specifically, the labeled datapoints of the training set  $(x_1,y_1), \ldots, (x_n,y_n)$, as well as the test data  $(x_\tst,y_\tst)$, are i.i.d.~from $\D$. Often it is assumed that $\D$ lies in some class of distributions of interest. The \emph{true expected loss}, or simply \emph{loss}, of a predictor $f: \X \to \Y$ is the expected loss it incurs on draws from $\D$, written $L_\D(f) = \Ex_{(x,y) \sim \D} \ell(f(x),y)$.


There are two main ``settings'' in PAC learning. The  \emph{realizable setting} only requires that the data be perfectly explained by some hypothesis in $\H$. More generally, the \emph{agnostic setting} makes no assumption relating the data to the hypotheses, but shifts the goalposts as necessary to allow nontrivial guarantees: the expected loss at test time is evaluated only ``relative'' to that of the best hypothesis $h^* \in \H$. There are other settings which make more nuanced assumptions, such as $\D$ being of a particular parametric form or its support living in some (unknown) lower-dimensional space, etc. I will mostly discuss the realizable and agnostic settings in this article, those being the simplest and most studied from a theoretical perspective. %TODO:We will briefly discuss other settings in Section ??

The PAC model demands high probability guarantees of learners, in the worst case over distributions of interest. Consider first the realizable setting, where $\D$ is such that $\min_{h \in \H} L_{\D}(h) = 0$. A PAC learner has \emph{error} $\epsilon=\epsilon(n)$ and \emph{confidence} $\delta=\delta(n)$ if, when training data consists of $n$ i.i.d~samples from a realizable distribution $\D$, it produces a predictor $f$  satisfying $L_\D(f) \leq \epsilon$ with probability at least $1-\delta$. In the agnostic setting, where $\D$ can be arbitrary, we require $L_\D(f) - \min_{h \in \H} L_\D(h) \leq \epsilon$ with probability $1-\delta$.

In both the realizable and agnostic settings, we look for PAC learners with small $\epsilon$ and $\delta$ as a function of the sample size $n$. An equivalent perspective looks at the sample complexity $m(\epsilon,\delta)$, which is the minimum sample size which guarantees error  at most $\epsilon$ with probability at least $1-\delta$. We say a problem is \emph{PAC learnable} if its PAC sample complexity is finite whenever $\epsilon,\delta > 0$.

For most PAC learning problems, learnability and sample complexity are characterized in terms of a  ``dimension'' of the hypothesis class. Most prominently this is the \emph{VC dimension} for binary classification, the \emph{fat shattering dimension} for agnostic regression, and the \emph{DS dimension} for multiclass classification (see \cite{anthony_neural_1999,daniely_optimal_2014,brukhim_characterization_2022}). Treatment of these is beyond the scope of this article. The unfamiliar reader need not worry, however,  as dimensions will feature only tangentially in our~discussion.




%\paragraph{Learning settings: Realizable, Agnostic, etc.} In learning theory, evaluating a supervised learning algorithm requires specifying a data model and an objective. We will leave the details of the data model flexible for now, to allow for both the PAC model and the adversarial transductive model. Nonetheless we will describe two variations, which we call ``settings'', which cut across different models. The  \emph{realizable setting}  requires only that the data be perfectly explained by some hypothesis $h \in \H$ --- i.e., there exists a hypothesis which is guaranteed to suffer a loss of $0$ on training and test data. The performance of the learning algorithm is its expected loss at test time for some ``worst case'' realizable instance. More generally, the \emph{agnostic setting} makes no assumption relating the data to the hypotheses, but shifts the goalposts as necessary to allow nontrivial guarantees: the expected loss at test time is evaluated only ``relative'' to that of the best hypothesis $h^* \in \H$, again for some ``worst case'' instance. There are other settings which make more nuanced assumptions about the data, such as it is drawn from a distribution of a particular parametric form, or that it lives in some (unknown) lower-dimensional space, etc. We will mostly discuss the realizable and agnostic settings, those being the simplest and most studied from a theoretical perspective.




%%% Local Variables:
%%% mode: latex
%%% TeX-master: "learning_matching"
%%% End:


\section{Proposed Approach}
\label{sec:proposed_approach}

We adopt neuromorphic computing concepts to process event-based data streams. By leveraging spiking neural networks (SNNs), we aim to emulate the inherent efficiency of biological neurons, which communicate primarily through discrete spikes rather than continuous signals.


% In Figure~\ref{fig:event_based_processing}, we illustrate the fundamental pipeline of event-based sensing, its biological inspiration, and how spikes are utilized in SNNs. The left panel visualizes raw event output over time (red or blue dots indicating positive or negative contrast changes). The right panel highlights that SNNs follow a similar principle: incoming spikes raise the neuron’s membrane potential until it surpasses a threshold, triggering an output spike.

Building on neuromorphic principles, our approach harnesses an event camera for real-time object detection and tracking. Subsequent modules—including a spiking neural network (SNN) block, a physics-guided neural network (PgNN) for energy-optimal trajectories, and a rule-based planner—operate on sparse event data to generate collision-free flight paths.
\begin{figure*}[t]
    \centering
    \includegraphics[width=1\textwidth]{ev_box.pdf}
    \caption{%
        \textbf{Event-based Object Detection at Various Depths.}
        Each sub-panel shows neuromorphic event output and a bounding box 
        around a moving gate. Although event density decreases
        with increasing depth, the SNN continues to isolate and track the gate
        in real time.
    }
    \label{fig:ev_box}
\end{figure*}
We propose a system that processes sparse, event-based sensor data to generate collision-free trajectories, as illustrated in Figure~\ref{fig:method}. Our approach integrates:
\begin{enumerate}
    \item A \textbf{spiking neural network (SNN)} for event-based object detection,
    \item A \textbf{physics-guided neural network (PgNN)} for near-minimum-energy destination prediction, and
    \item A \textbf{rule-based planner} for handling moving obstacles.
\end{enumerate}
All components interact in a \textbf{ROS} and \textbf{Gazebo} simulation environment: the SNN publishes bounding-box coordinates of the moving gate, the PgNN outputs an estimated flight time, and a symbolic planner node fuses both to generate velocity commands for the drone’s low-level controller.


\subsection{\textbf{Neuromorphic Vision-based Object Detection}}
\label{subsec:neuromorphic_detection}

In event-based cameras, each pixel independently fires upon detecting changes in brightness, producing sparse asynchronous data streams. Unlike frame-based cameras that capture images at fixed intervals, event-based cameras generate events only when a change in the scene occurs, making them highly efficient for capturing fast motions and dynamic environments. This event-driven nature allows them to have a high temporal resolution and low latency, making them particularly advantageous for real-time applications.

Inspired by \cite{nagaraj2023dotie}, we adopt a biologically-plausible approach using leaky integrate-and-fire (LIF) neurons to detect fast-moving objects. The LIF neuron model simulates how biological neurons process incoming stimuli by accumulating membrane potential over time. Specifically, the membrane potential $V[t]$ of an LIF neuron evolves as:

\begin{equation}
V[t] = \beta , V[t_{n-1}] + W , X[t],
\label{lif_eq}
\end{equation}

where $\beta$ is the leak factor representing the decay of the membrane potential, $W$ is a learnable weight matrix that scales the contribution of the input, and $X[t]$ is the aggregated spike input at time $t$. The leak factor $\beta$ controls how quickly the neuron forgets previous inputs, with smaller values corresponding to faster decay.

Whenever the membrane potential $V[t]$ exceeds a predefined threshold $V_{\mathrm{th}}$, the neuron fires an event and resets its membrane potential. This firing mechanism emulates how biological neurons emit a spike when they reach their activation threshold. In the context of object detection, this spiking behavior is highly beneficial for detecting transient or fast-moving objects, as they produce dense bursts of events due to rapid brightness changes. This property makes the event rate directly proportional to the speed of the object:
\begin{equation}
    \text{Event Rate} \;\propto\; \text{Object Speed},
    \label{ev_rate}
\end{equation}
    
As a result, fast-moving objects produce dense event bursts, which accumulate quickly and cause the membrane potential to surpass $V_{\mathrm{th}}$.

To localize objects within the event stream after a neuron fires, we extract a bounding box based on the spatial distribution of spiking events. This is achieved by computing the minimum and maximum coordinates of the spiking pixels:

These coordinates represent the bounding limits of the detected object. We then calculate the center of the bounding box to determine the object's approximate position:

\begin{subequations}
\label{eq:center}
\begin{align}
\text{center}_x &= X_{\min} + \Bigl\lfloor\frac{X_{\max} - X_{\min}}{2}\Bigr\rfloor,\\
\text{center}_y &= Y_{\min} + \Bigl\lfloor\frac{Y_{\max} - Y_{\min}}{2}\Bigr\rfloor.
\end{align}
\end{subequations}

This computation yields the center point of the bounding box, providing an efficient method for locating objects within the event frame.

As shown in Figure~\ref{fig:ev_box}, the single-layer SNN is able to localize the target even at greater depths, where event density is lower. We use a $\beta = 0.1$, $V_{th}=1.75$. This was obtained after fine-tuning for the given gate which moves at $4$ m/s. We use a $3\times3$ sized kernel for $W$. See Section \ref{ssec:tracking_results} for further details.

\subsection{\textbf{Physics-Guided Trajectory-Duration Predictor}}
\label{subsec:physics_guided}

We model each quadrotor propeller by:
\begin{equation}
    e(t) \;=\; R\,i(t) + K_E\,\omega(t),
    \label{eq:quadrotor_propeller}
\end{equation}

where $R$ is the winding resistance, $i(t)$ the current, and $\omega(t)$ the rotor speed. The total energy from $t=0$ to $t=T$ is given by:
\begin{equation}
    E(T) \;=\; \int_{0}^{T} \sum_{j=1}^{4} e_j(\tau)\,i_j(\tau)\,d\tau,
    \label{eq:E}
\end{equation}
revealing that flying very slowly or very fast can waste energy. Each depth $d$ therefore has an ideal velocity $v_{\mathrm{opt}}$ minimizing overall consumption, illustrated in Figure~\ref{fig:energy_time}.

\begin{figure}[t]
    \centering
    \includegraphics[width=0.95\columnwidth]{energy_time_2.pdf}
    \caption{%
        Optimal velocity and energy-consumption patterns. 
        Each depth features a characteristic velocity $v_{\mathrm{opt}}$ 
        that balances time and power usage.
    }
    \label{fig:energy_time}
\end{figure}

A \emph{physics-guided neural network} (PgNN) learns this relationship by approximating \(E(v)\) through polynomial regression for each depth. By fitting a 5th-degree polynomial to the energy-velocity data, the PgNN captures the non-linear dynamics inherent in the system. The optimal velocity \(v_{\mathrm{opt}}\) is determined by finding the velocity at which the derivative of the energy function equals zero:
\begin{equation}
    \frac{dE(v)}{dv} = 0 \quad \Rightarrow \quad v_{\mathrm{opt}} = \arg\min_v E(v),
    \label{eq:optimal_velocity}
\end{equation}
The PgNN predicts \(v^{\mathrm{pred}}\), yielding an approximate flight time:
\begin{equation}
    t_{\mathrm{traj}} 
    \;=\; \frac{d}{v^{\mathrm{pred}}},
    \label{eq:ttraj}
\end{equation}


\begin{table}[ht]
\centering
\caption{PgNN Training Samples: Depth, Velocity, and Physical Constraints obtained by fiiting polynomial curves}
\label{tab:data}
\begin{tabular}{@{}lcc@{}}
\toprule
\textbf{Depth} & \textbf{Velocity} & \textbf{Constraint}                  \\ \midrule
$d_1$          & $v_1$             & $c_{1,1} + 2\,c_{2,1}\,v_1 + \ldots = 0$       \\
$d_2$          & $v_2$             & $c_{1,2} + 2\,c_{2,2}\,v_2 + \ldots = 0$       \\
$\cdots$       & $\cdots$          & $\cdots$                                      \\
$d_n$          & $v_n$             & $c_{1,n} + 2\,c_{2,n}\,v_n + \ldots = 0$       \\ \bottomrule
\end{tabular}
\end{table}


The samples used to train the PgNN are summarized in Table~\ref{tab:data}, which lists various depths $d_n$, their corresponding optimal velocities $v_n$, and the associated constraints derived from the polynomial fits of $E(v)$ and its derivative.

In battery-constrained aerial systems, ensuring energy-efficient flight requires careful consideration of power consumption, \(P(t)\). Power consumption is directly related to the thrust force as:
\begin{equation}
    P(t) = \kappa \|\mathbf{F}_{\text{thrust}}(t)\|^\alpha,
    \label{eq:power_consumption}
\end{equation}
where \(\kappa\) and \(\alpha\) are constants determined by the propeller actuator characteristics of the UAV. The parameter \(\alpha\) governs the non-linearity of the power-thrust relationship and has a significant impact on the energy expenditure profile across varying flight velocities.


\subsection{\textbf{PgNN Loss Function}}

The Physics-Guided Neural Network (PgNN) is trained to predict near-optimal velocities by minimizing a composite loss function:
\begin{equation}
    \mathcal{L}_{\mathrm{PgNN}} = \mathcal{L}_{\mathrm{data}} + \lambda_1 \mathcal{L}_{\mathrm{physics}} + \lambda_2 \mathcal{L}_{\mathrm{energy}},
    \label{eq:pgnn_loss}
\end{equation}
where each term contributes to a specific aspect of the optimization:

\paragraph{Data-Fitting Loss (\(\mathcal{L}_{\mathrm{data}}\))} Ensures that the PgNN predictions align with the ground truth velocity \(v_i^{\mathrm{opt}}\), derived from offline optimization or simulation data, as follows:
\begin{equation}
    \mathcal{L}_{\mathrm{data}} = \frac{1}{N} \sum_{i=1}^N \bigl(v_i^{\mathrm{pred}} - v_i^{\mathrm{opt}}\bigr)^2.
    \label{eq:data_fitting_loss}
\end{equation}
Here, \(v_i^{\mathrm{opt}}\) is linked to the trajectory time \(t_{\mathrm{traj}}\) (as defined in Equation~\eqref{eq:ttraj}), enabling the PgNN to learn velocity predictions that minimize energy consumption while ensuring timely navigation.

\paragraph{Physics Consistency Loss (\(\mathcal{L}_{\mathrm{physics}}\))} Ensures that the PgNN respects the physical laws governing UAV motion:
\begin{equation}
    \mathcal{L}_{\mathrm{physics}} = \frac{1}{N} \sum_{i=1}^N \bigl|\mathbf{x}_i^{\mathrm{pred}} - \mathbf{x}_i^{\mathrm{sim}}\bigr|,
    \label{eq:physics_consistency_loss}
\end{equation}
where \(\mathbf{x}_i^{\mathrm{sim}}\) represents the UAV states predicted by a physics-based simulation model (Equation \ref{eq:dynamics}). The dynamics of the UAV are governed by Equation~\eqref{eq:power_consumption}, which capture energy consumption \(P(t)\) and the time-energy trade-off in the objective function.

\paragraph{Energy Efficiency Loss (\(\mathcal{L}_{\mathrm{energy}}\))} Promotes predictions that minimize energy consumption, calculated using the power consumption equation:
\begin{equation}
    \mathcal{L}_{\mathrm{energy}} = \frac{1}{N} \sum_{i=1}^N P(t_i^{\mathrm{pred}}),
    \label{eq:energy_efficiency_loss}
\end{equation}
where \(P(t)\) is defined in Equation~\eqref{eq:power_consumption}, and its computation integrates over the trajectory time \(t_{\mathrm{traj}}\). This term enforces efficiency by penalizing excessive energy use across predicted trajectories.

The hyperparameters \(\lambda_1\) and \(\lambda_2\) are tuned to balance the contributions of the physics and energy terms, ensuring that the PgNN achieves both accurate predictions and energy-efficient trajectories. By combining these loss components, the PgNN learns velocity predictions that optimize flight time and energy consumption while respecting physical constraints.
In this work, the PgNN is a $3$-layer fully connected multi-layer perceptron with $[64,128,128]$ neurons. Later, in Section \ref{subsec:reg}, we analyze the effect of varying the hyperparameters of our PgNN.


\subsection{\textbf{Symbolic Planning for Moving Gates}}
\label{symbol}

The system integrates a rule-based planner to handle dynamic obstacles such as moving gates as shown in Algorithm 1 in Fig \ref{fig:method}. In scenarios like drone racing, gates often oscillate along the \(y\)-axis within a bounded range of \(\pm L\). Accurate planning requires predicting the gate's future position based on its current motion and the estimated time of arrival (\(t_{\mathrm{traj}}\)).

Using consecutive gate positions \(\{y_1, y_2\}\) recorded at time intervals of \(\delta t\), the gate's velocity is computed as:
\begin{equation}
    v_r = \frac{y_2 - y_1}{\delta t}.
    \label{eq:gate_velocity}
\end{equation}
This velocity, combined with the predicted time of arrival \(t_{\mathrm{traj}}\) from the Physics-Guided Neural Network (PgNN), estimates the gate's future position:
\begin{equation}
    y^* = y_2 + v_r \cdot t_{\mathrm{traj}}.
    \label{eq:gate_position_prediction}
\end{equation}
If the gate is expected to bounce off a boundary during \(t_{\mathrm{traj}}\), its displacement is adjusted to reflect the change in direction:
\begin{equation}
    y^* = 
    \begin{cases} 
    y_2 - L + x, & \text{if the gate bounces to the left}, \\
    y_2 + L - x, & \text{if the gate bounces to the right}.
    \end{cases}
    \label{eq:gate_boundary_adjustment}
\end{equation}
In this way, the proposed framework ensures collision-free navigation while minimizing energy usage.

\begin{table*}[tb]
\centering
\caption{Comparison of tracking methods (SNN, YOLO, R-CNN, and Hough Transform)}
\label{tab:comparison}
\resizebox{0.9\linewidth}{!}{%
\begin{tabular}{lcccc}
\toprule
\textbf{Method} & 
\textbf{Parameter Count} & 
\textbf{Training / Labeling} & 
\textbf{Thresholds} & 
\textbf{Notes} \\
\midrule
\textbf{SNN \cite{nagaraj2023dotie,evplanner,joshi2024}} & 
\begin{tabular}[c]{@{}l@{}}
1 Conv2D \((3\times3)\)\\
9 Spiking Neurons
\end{tabular} & 
Unsupervised & 
\begin{tabular}[c]{@{}l@{}}
Yes (firing \\
threshold)
\end{tabular} &
Lightweight, event-driven \\
\midrule
\textbf{YOLO (CNN-based) \cite{yolo}} & 
\(\sim\!60\text{-}65\)M params & 
\begin{tabular}[c]{@{}l@{}}
Requires large \\
labeled dataset
\end{tabular} &
No &
Popular real-time detector \\
\midrule
\textbf{Faster R-CNN (CNN-based) \cite{rcnn}} & 
\(\sim\!60\text{-}100\)M params & 
\begin{tabular}[c]{@{}l@{}}
Requires large \\
labeled dataset
\end{tabular} &
No &
Region proposals + deep CNN \\
\midrule
\textbf{Hough Transform \cite{hough_t}} & 
\textit{N/A} & 
No (classical CV) & 
Yes (edges, thresholds) &
Handcrafted approach \\
\bottomrule
\end{tabular}%
}
\end{table*}

\begin{figure*}[htbp]
    \centering
    \includegraphics[width=1\textwidth]{drone_traj.pdf} 
    \caption{Drone navigating through a moving gate in Gazebo simulation. The gate moves from right to left, while the drone starts at position \((1, 1)\). The timestamps (`t = 0s`, `t = 1s`, `t = 2s`, `t = 3s`) indicate the time elapsed at each key stage: (a) Start, (b) Approach, (c) Entry, and (d) Pass-Through.}
    \label{fig:drone_navigation_timeline}
\end{figure*}
\begin{figure}[tb]
    \centering
    \includegraphics[width=0.9\linewidth]{iou_ijcnn.pdf}
    \caption{Tracking performance vs.\ depth. The plot compares the mean IoU (mIOU)
    and the peak IoU across distances from 2\,m to 9\,m.}
    \vspace{-2mm}
    \label{fig:iou_ijcnn}
\end{figure}
\begin{figure}[ht]
    \centering
    \includegraphics[width=\linewidth]{simulated_flight_energy_plot.pdf}  % Path to your PDF file
    \caption{Comparison of flight time, path length, and dynamic energy across varying depths for different values of $\alpha$ and $\lambda$. The left column displays the combined plots of flight time (in seconds) and path length (in meters), showcasing the differences influenced by the regularization weights $\lambda_1$ and $\lambda_2$. The right column shows the corresponding dynamic energy (in joules) for the same parameter sets. Each row corresponds to a different power model exponent $\alpha$ (0.2, 0.3, and 0.5).}
    \label{fig:flight_energy_plot}  
\end{figure}

\begin{figure*}[!t]
    \centering
    \includegraphics[width=0.95\textwidth]{IJCNN_3D.pdf}
    \caption{Comparison of Navigation Trajectories: The figure illustrates the navigation paths of a drone using depth-based perception (red trajectory) versus a physics-guided neuromorphic vision-based approach that fuses event and depth sensors (green trajectory). The neuromorphic approach consistently results in shorter, more energy-efficient trajectories across varying initial drone positions and ring depths.}
    \label{fig:3d_navigation_paths}
\end{figure*}
\begin{figure}[t!]
    \centering
    \includegraphics[width=\linewidth, height = 3.3cm]
    {detailed_flight_path_plot.pdf}
    \caption{Flight time (left y-axis) and path length (right-y axis) as functions of depth for different initial drone positions for our physics-guided neuromorphic approach. The left y-axis shows the flight time in seconds, and the right y-axis shows the corresponding path length in meters. The data is grouped by depth values, with distinct markers indicating measurements for each combination of flight time and path length.}
    \label{fig:flight_path_depth}
\end{figure}

\noindent{\textbf{Neurosymbolic Integration for Navigation}}: The planner incorporates inference results from multiple neural network modules. The event-based Spiking Neural Network (SNN) detects and tracks the moving gate, providing bounding box coordinates \((y_1, y_2)\). The Physics-Guided Neural Network (PgNN) predicts the trajectory time \(t_{\mathrm{traj}}\) while optimizing for energy efficiency. These outputs are fused and processed through logical rules, forming a neurosymbolic framework that combines data-driven predictions with rule-based reasoning. This integration enables the system to account for boundary dynamics, predict gate behavior, and compute the gate’s future position \(y^*\). By leveraging this neurosymbolic approach, the system achieves efficient and safe navigation through moving gates. Please note, this work only considers local reactive planning, necessary when there are changes observed in the immediate surrounding environment of the robotic agent (drone in this work).



\section{Experimental Results}


\subsection{\textbf{Neuromorphic Vision-Based Tracking}}
\label{ssec:tracking_results}
Figure \ref{fig:ev_box} shows neuromorphic event output and a bounding box 
around a moving gate (left to right). Although event density decreases
with increasing depth, the SNN continues to isolate and track the gate
in real time. Correspondingly, as illustrated in Figure~\ref{fig:iou_ijcnn}, both the mIOU and the peak IOU gradually decline as the distance between the event-based camera and the moving gate increases. At shorter ranges (2--4\,m), the system achieves a mean IOU between 0.78 and 0.83, with a peak IOU above 0.90 at 4\,m. This high accuracy stems from dense event bursts generated by the gate’s motion, allowing the spiking neural network (SNN) to consistently localize the target with minimal false positives.

However, at larger depths (beyond 6,m), the IOU values drop slightly (mIOU: 0.60--0.65; peak IOU: 0.62--0.73) due to sparser events and weaker contrast changes. Despite this, the neuromorphic detection pipeline maintains robust tracking across a wide range of depths, demonstrating its suitability for low-latency object detection in dynamic flight scenarios.




As summarized in Table~\ref{tab:comparison}, our shallow SNN-based approach has an extremely low parameter count (one \(3\times 3\) convolutional layer and 9 spiking neurons). It operates without a large labeled dataset but still requires threshold tuning (e.g., spiking neuron firing thresholds). Meanwhile, YOLO and R-CNN each demand tens of millions of parameters, along with substantial labeled training data. Finally, the classical Hough Transform method does not rely on deep learning but involves hand-tuned thresholds for edge detection. Our SNN maintains a balance of minimal parameter overhead, unsupervised learning, while offering the benefits of neuromorphic perception using SNN-based dynamic vision sensors.

\subsection{\textbf{Sensitivity Analysis for Physics-based Regularization}}
\label{subsec:reg}
To examine how the power-model exponent \(\alpha\) (Equation~\ref{eq:power_consumption}) and the regularization weights \(\lambda_1\) and \(\lambda_2\) (Equation~\ref{eq:pgnn_loss}) jointly shape flight efficiency, we conducted a series of simulations at varying ring depths and initial drone positions. Figure~\ref{fig:flight_energy_plot} presents the resulting flight time, path length, and dynamic energy for three representative \(\alpha\) values (0.2, 0.3, and 0.5). Each row corresponds to one exponent, while different lines indicate distinct \(\lambda_1,\lambda_2\) settings. The left column combines flight time (in seconds) and path length (in meters), highlighting how \(\alpha\) and \(\lambda\)-values modify navigation efficiency. The right column displays the corresponding dynamic energy (in joules), illustrating the interplay between control regularization and power consumption.


As shown in the left-column plots of Figure~\ref{fig:flight_energy_plot}, increasing the ring depth results in longer flight times and paths, especially at higher \(\alpha\). This effect is magnified when \(\lambda\)-values are small, since weaker regularization allows more aggressive maneuvers, potentially increasing path deviations. Conversely, higher \(\lambda_1,\lambda_2\) steer the flight toward smoother trajectories, dampening sudden control changes and thereby reducing energy spikes (right-column plots). 
Note that, the top row ($\alpha = 0.2$) reflects parameters tuned to our Bebop simulation, while the middle and bottom rows ($\alpha = 0.3$ and $\alpha = 0.5$) illustrate hypothetical regimes. This highlights the sensitivity of the power‐consumption model to $\alpha$; in real‐world applications, calibrating $\alpha$ to the specific drone is crucial for accurately capturing its flight dynamics and energy demands. 

Also, note that the energies reported for inference during the actual flights are higher than the ones presented in Figure \ref{fig:energy_time}, as the actual flight paths are longer than the straight line flight distances covered (no moving gate there) while collecting training data (Table \ref{tab:data}).

 In the following subsection, we expand our analysis to examine how these parameter choices translate into navigation performance across a range of drone starting positions and ring depths. 


\subsection{\textbf{Neuromorphic Navigation for different Drone Positions }}
Figure~\ref{fig:drone_navigation_timeline} illustrates the drone's navigation through a moving ring, with timestamps marking key stages from start ($t = 0s$) to pass-through ($t = 3s$), highlighting the smooth progression despite the dynamic nature of the ring.



Figure~\ref{fig:3d_navigation_paths} illustrates the comparison of navigation trajectories for a drone navigating through a ring from various initial positions \((x, y, z)\) and ring depths ranging from 2~m to 6~m. The figure highlights the performance of the depth-based perception approach (red trajectory) versus the physics-guided neuromorphic vision-based approach (green trajectory), which combines event-based and depth-based sensor inputs. The trajectory of the ring's center is depicted in orange. We observe the following:
\begin{itemize}
    \item \textbf{Energy-Efficient Trajectories:} The neuromorphic vision-based approach produces shorter paths, such as achieving a path length of 3.5~m at a 2~m depth, compared to 4.6~m for an off-center start.
    
    \item \textbf{Accuracy and Responsiveness:} When starting from positions \((-2, y, z)\) or \((+2, y, z)\), the neuromorphic approach maintains a shorter flight time of around 2.9~s at a depth of 3~m, compared to longer times exceeding 4.1~s for off-center starts at greater depths.
    
    \item \textbf{Trajectory Deviation:} Depth-based perception results in longer paths, such as 7.6~m at a 5~m depth, compared to the neuromorphic approach’s 5.9~m for the same conditions, demonstrating the effect of initial offsets.
    
    \item \textbf{Impact of Depth:} The neuromorphic approach remains efficient with path lengths ranging from 3.5~m (at 2~m) to 9.5~m (at 6~m), while the depth-based method increases path length at greater depths, up to 10~m or more.
\end{itemize}


Figure~\ref{fig:flight_path_depth} further illustrates the dependence of flight time and path length on ring depth for various initial drone positions. Here, the left y-axis measures flight time (in seconds), while the right y-axis tracks path length (in meters). Distinct markers indicate different initial offsets \((x, y)\), showing that off-center starts tend to increase the required flight time and distance. For instance, at a depth of 3~m, the flight time increases from 2.3~s (centered start) to 2.9~s (off-center start), with the corresponding path length increasing from 3.5~m to 4.6~m. At a depth of 5~m, the off-center start results in a flight time of 4.1~s and a path length of 7.6~m. These results highlight the importance of appropriate tuning of \(\alpha\) and \(\lambda\)-values to balance energy conservation against the need for precise maneuvers. Overall, the physics-guided neuromorphic approach reduces average flight time by approximately 20\% and path length by around 15\% compared to the depth-based method, demonstrating its potential for more efficient autonomous aerial navigation.

\section{Conclusion}
This work presents a novel framework for energy-efficient autonomous aerial navigation that leverages neuromorphic event-based vision and physics-guided planning. By integrating dynamic vision sensors (DVS) with spiking neural networks (SNNs) and physics-guided neural networks (PgNNs), the system achieves real-time responsiveness and energy optimization. Experimental results demonstrate that the proposed approach produces shorter, energy-efficient trajectories with high accuracy and adaptability across varying initial positions and depths. Notably, the fusion of asynchronous event data with depth information enhances trajectory planning by enabling robust navigation through dynamic environments, such as moving gates. The neuromorphic design outperforms traditional depth-based methods due to its low-latency processing and efficient handling of sparse data, making it particularly advantageous for scenarios for real-time decision-making.

The proposed framework is also a neurosymbolic integration that combines event-driven neuromorphic computing, physics-based AI, symbolic logical reasoning, and classical planning. The neuromorphic SNN module processes asynchronous event data for low-latency perception, while the PgNN embeds physical constraints to generate energy-optimal trajectories that align with real-world dynamics. The rule-based planner introduces symbolic reasoning to anticipate obstacle movements and guide navigation decisions. This synergy between neural and symbolic components allows the system to handle dynamic, real-time scenarios while maintaining interpretable and physically consistent control. By uniting data-driven perception with rule-based reasoning, the framework achieves adaptable decision-making, highlighting possible benefits of neurosymbolic design for autonomous systems.

\section{Acknowledgement}
This work was supported by the Center for the Co-Design of Cognitive Systems (CoCoSys), a center in JUMP 2.0, an SRC program sponsored by DARPA.



% \section*{References}
\bibliographystyle{IEEEtran}
\bibliography{conference_101719}

% \begin{thebibliography}{00}
% \bibitem{b1} G. Eason, B. Noble, and I. N. Sneddon, ``On certain integrals of Lipschitz-Hankel type involving products of Bessel functions,'' Phil. Trans. Roy. Soc. London, vol. A247, pp. 529--551, April 1955.
% \bibitem{b2} J. Clerk Maxwell, A Treatise on Electricity and Magnetism, 3rd ed., vol. 2. Oxford: Clarendon, 1892, pp.68--73.
% \bibitem{b3} I. S. Jacobs and C. P. Bean, ``Fine particles, thin films and exchange anisotropy,'' in Magnetism, vol. III, G. T. Rado and H. Suhl, Eds. New York: Academic, 1963, pp. 271--350.
% \bibitem{b4} K. Elissa, ``Title of paper if known,'' unpublished.
% \bibitem{b5} R. Nicole, ``Title of paper with only first word capitalized,'' J. Name Stand. Abbrev., in press.
% \bibitem{b6} Y. Yorozu, M. Hirano, K. Oka, and Y. Tagawa, ``Electron spectroscopy studies on magneto-optical media and plastic substrate interface,'' IEEE Transl. J. Magn. Japan, vol. 2, pp. 740--741, August 1987 [Digests 9th Annual Conf. Magnetics Japan, p. 301, 1982].
% \bibitem{b7} M. Young, The Technical Writer's Handbook. Mill Valley, CA: University Science, 1989.
% \end{thebibliography}

\end{document}
