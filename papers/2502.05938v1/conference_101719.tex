\documentclass[10pt]{IEEEtran}
\IEEEoverridecommandlockouts
\pagestyle{empty}
\pdfminorversion=5
\pdfobjcompresslevel=3
\pdfcompresslevel=9
\usepackage{cite}
\usepackage{amsmath,amssymb,amsfonts}
\usepackage[linesnumbered,ruled,vlined]{algorithm2e} % Removed `algorithm` & `algorithmic`
\usepackage{graphicx}
\usepackage{xcolor}
\usepackage{booktabs}
\usepackage[top=1in, bottom=0.8in, left=0.75in, right=0.75in]{geometry}
\usepackage[hidelinks]{hyperref}
\begin{document}
\title{{Energy-Efficient Autonomous Aerial Navigation with Dynamic Vision Sensors: A Physics-Guided Neuromorphic Approach}}

\author{Sourav Sanyal*, Amogh Joshi*,  Manish Nagaraj, Rohan Kumar Manna, \& Kaushik Roy\\
  \IEEEauthorblockA{Electrical and Computer Engineering, Purdue University
}}
% \author{\IEEEauthorblockN{Anonymous Authors\\ Submission Copy: Do not distribute}}
\maketitle
% \SetAlgoNlRelativeSize{-1} % Ensure line numbers are not too large
% \SetAlCapSkip{1em}        % Adjust caption spacing
% \SetAlgoSkip{0.5em}       % Reduce algorithm spacing


\begin{abstract}



Vision-based object tracking is a critical component for achieving autonomous aerial navigation, particularly for obstacle avoidance. Neuromorphic Dynamic Vision Sensors (DVS) or event cameras, inspired by biological vision, offer a promising alternative to conventional frame-based cameras. These cameras can detect changes in intensity asynchronously, even in challenging lighting conditions, with a high dynamic range and resistance to motion blur. Spiking neural networks (SNNs) are increasingly used to process these event-based signals efficiently and asynchronously. Meanwhile, physics-based artificial intelligence (AI) provides a means to incorporate system-level knowledge into neural networks via physical modeling. This enhances robustness, energy efficiency, and provides symbolic explainability. In this work, we present a neuromorphic navigation framework for autonomous drone navigation. The focus is on detecting and navigating through moving gates while avoiding collisions. We use event cameras for detecting moving objects through a shallow SNN architecture in an unsupervised manner. This is combined with a lightweight energy-aware physics-guided neural network (PgNN) trained with depth inputs to predict optimal flight times, generating near-minimum energy paths. The system is implemented in the Gazebo simulator and integrates a sensor-fused vision-to-planning neuro-symbolic framework built with the Robot Operating System (ROS) middleware. This work highlights the future potential of integrating event-based vision with physics-guided planning for energy-efficient autonomous navigation, particularly for low-latency decision-making. 

\begin{IEEEkeywords}
Neuromorphic Computing, Dynamic Vision Sensors, Autonomous Navigation, Spiking Neural Network, Physics-guided Neural Network
\end{IEEEkeywords}

% \noindent{Video Link: \url{https://youtu.be/9Gnjpb1k2Lo}}

\end{abstract}
\thispagestyle{empty}
% \begin{IEEEkeywords}
% component, formatting, style, styling, insert
% \end{IEEEkeywords}

\section{Introduction}
\section{Introduction}
\label{sec:introduction}
The business processes of organizations are experiencing ever-increasing complexity due to the large amount of data, high number of users, and high-tech devices involved \cite{martin2021pmopportunitieschallenges, beerepoot2023biggestbpmproblems}. This complexity may cause business processes to deviate from normal control flow due to unforeseen and disruptive anomalies \cite{adams2023proceddsriftdetection}. These control-flow anomalies manifest as unknown, skipped, and wrongly-ordered activities in the traces of event logs monitored from the execution of business processes \cite{ko2023adsystematicreview}. For the sake of clarity, let us consider an illustrative example of such anomalies. Figure \ref{FP_ANOMALIES} shows a so-called event log footprint, which captures the control flow relations of four activities of a hypothetical event log. In particular, this footprint captures the control-flow relations between activities \texttt{a}, \texttt{b}, \texttt{c} and \texttt{d}. These are the causal ($\rightarrow$) relation, concurrent ($\parallel$) relation, and other ($\#$) relations such as exclusivity or non-local dependency \cite{aalst2022pmhandbook}. In addition, on the right are six traces, of which five exhibit skipped, wrongly-ordered and unknown control-flow anomalies. For example, $\langle$\texttt{a b d}$\rangle$ has a skipped activity, which is \texttt{c}. Because of this skipped activity, the control-flow relation \texttt{b}$\,\#\,$\texttt{d} is violated, since \texttt{d} directly follows \texttt{b} in the anomalous trace.
\begin{figure}[!t]
\centering
\includegraphics[width=0.9\columnwidth]{images/FP_ANOMALIES.png}
\caption{An example event log footprint with six traces, of which five exhibit control-flow anomalies.}
\label{FP_ANOMALIES}
\end{figure}

\subsection{Control-flow anomaly detection}
Control-flow anomaly detection techniques aim to characterize the normal control flow from event logs and verify whether these deviations occur in new event logs \cite{ko2023adsystematicreview}. To develop control-flow anomaly detection techniques, \revision{process mining} has seen widespread adoption owing to process discovery and \revision{conformance checking}. On the one hand, process discovery is a set of algorithms that encode control-flow relations as a set of model elements and constraints according to a given modeling formalism \cite{aalst2022pmhandbook}; hereafter, we refer to the Petri net, a widespread modeling formalism. On the other hand, \revision{conformance checking} is an explainable set of algorithms that allows linking any deviations with the reference Petri net and providing the fitness measure, namely a measure of how much the Petri net fits the new event log \cite{aalst2022pmhandbook}. Many control-flow anomaly detection techniques based on \revision{conformance checking} (hereafter, \revision{conformance checking}-based techniques) use the fitness measure to determine whether an event log is anomalous \cite{bezerra2009pmad, bezerra2013adlogspais, myers2018icsadpm, pecchia2020applicationfailuresanalysispm}. 

The scientific literature also includes many \revision{conformance checking}-independent techniques for control-flow anomaly detection that combine specific types of trace encodings with machine/deep learning \cite{ko2023adsystematicreview, tavares2023pmtraceencoding}. Whereas these techniques are very effective, their explainability is challenging due to both the type of trace encoding employed and the machine/deep learning model used \cite{rawal2022trustworthyaiadvances,li2023explainablead}. Hence, in the following, we focus on the shortcomings of \revision{conformance checking}-based techniques to investigate whether it is possible to support the development of competitive control-flow anomaly detection techniques while maintaining the explainable nature of \revision{conformance checking}.
\begin{figure}[!t]
\centering
\includegraphics[width=\columnwidth]{images/HIGH_LEVEL_VIEW.png}
\caption{A high-level view of the proposed framework for combining \revision{process mining}-based feature extraction with dimensionality reduction for control-flow anomaly detection.}
\label{HIGH_LEVEL_VIEW}
\end{figure}

\subsection{Shortcomings of \revision{conformance checking}-based techniques}
Unfortunately, the detection effectiveness of \revision{conformance checking}-based techniques is affected by noisy data and low-quality Petri nets, which may be due to human errors in the modeling process or representational bias of process discovery algorithms \cite{bezerra2013adlogspais, pecchia2020applicationfailuresanalysispm, aalst2016pm}. Specifically, on the one hand, noisy data may introduce infrequent and deceptive control-flow relations that may result in inconsistent fitness measures, whereas, on the other hand, checking event logs against a low-quality Petri net could lead to an unreliable distribution of fitness measures. Nonetheless, such Petri nets can still be used as references to obtain insightful information for \revision{process mining}-based feature extraction, supporting the development of competitive and explainable \revision{conformance checking}-based techniques for control-flow anomaly detection despite the problems above. For example, a few works outline that token-based \revision{conformance checking} can be used for \revision{process mining}-based feature extraction to build tabular data and develop effective \revision{conformance checking}-based techniques for control-flow anomaly detection \cite{singh2022lapmsh, debenedictis2023dtadiiot}. However, to the best of our knowledge, the scientific literature lacks a structured proposal for \revision{process mining}-based feature extraction using the state-of-the-art \revision{conformance checking} variant, namely alignment-based \revision{conformance checking}.

\subsection{Contributions}
We propose a novel \revision{process mining}-based feature extraction approach with alignment-based \revision{conformance checking}. This variant aligns the deviating control flow with a reference Petri net; the resulting alignment can be inspected to extract additional statistics such as the number of times a given activity caused mismatches \cite{aalst2022pmhandbook}. We integrate this approach into a flexible and explainable framework for developing techniques for control-flow anomaly detection. The framework combines \revision{process mining}-based feature extraction and dimensionality reduction to handle high-dimensional feature sets, achieve detection effectiveness, and support explainability. Notably, in addition to our proposed \revision{process mining}-based feature extraction approach, the framework allows employing other approaches, enabling a fair comparison of multiple \revision{conformance checking}-based and \revision{conformance checking}-independent techniques for control-flow anomaly detection. Figure \ref{HIGH_LEVEL_VIEW} shows a high-level view of the framework. Business processes are monitored, and event logs obtained from the database of information systems. Subsequently, \revision{process mining}-based feature extraction is applied to these event logs and tabular data input to dimensionality reduction to identify control-flow anomalies. We apply several \revision{conformance checking}-based and \revision{conformance checking}-independent framework techniques to publicly available datasets, simulated data of a case study from railways, and real-world data of a case study from healthcare. We show that the framework techniques implementing our approach outperform the baseline \revision{conformance checking}-based techniques while maintaining the explainable nature of \revision{conformance checking}.

In summary, the contributions of this paper are as follows.
\begin{itemize}
    \item{
        A novel \revision{process mining}-based feature extraction approach to support the development of competitive and explainable \revision{conformance checking}-based techniques for control-flow anomaly detection.
    }
    \item{
        A flexible and explainable framework for developing techniques for control-flow anomaly detection using \revision{process mining}-based feature extraction and dimensionality reduction.
    }
    \item{
        Application to synthetic and real-world datasets of several \revision{conformance checking}-based and \revision{conformance checking}-independent framework techniques, evaluating their detection effectiveness and explainability.
    }
\end{itemize}

The rest of the paper is organized as follows.
\begin{itemize}
    \item Section \ref{sec:related_work} reviews the existing techniques for control-flow anomaly detection, categorizing them into \revision{conformance checking}-based and \revision{conformance checking}-independent techniques.
    \item Section \ref{sec:abccfe} provides the preliminaries of \revision{process mining} to establish the notation used throughout the paper, and delves into the details of the proposed \revision{process mining}-based feature extraction approach with alignment-based \revision{conformance checking}.
    \item Section \ref{sec:framework} describes the framework for developing \revision{conformance checking}-based and \revision{conformance checking}-independent techniques for control-flow anomaly detection that combine \revision{process mining}-based feature extraction and dimensionality reduction.
    \item Section \ref{sec:evaluation} presents the experiments conducted with multiple framework and baseline techniques using data from publicly available datasets and case studies.
    \item Section \ref{sec:conclusions} draws the conclusions and presents future work.
\end{itemize}

\section{Preliminaries}
\section{Background}\label{sec:backgrnd}

\subsection{Cold Start Latency and Mitigation Techniques}

Traditional FaaS platforms mitigate cold starts through snapshotting, lightweight virtualization, and warm-state management. Snapshot-based methods like \textbf{REAP} and \textbf{Catalyzer} reduce initialization time by preloading or restoring container states but require significant memory and I/O resources, limiting scalability~\cite{dong_catalyzer_2020, ustiugov_benchmarking_2021}. Lightweight virtualization solutions, such as \textbf{Firecracker} microVMs, achieve fast startup times with strong isolation but depend on robust infrastructure, making them less adaptable to fluctuating workloads~\cite{agache_firecracker_2020}. Warm-state management techniques like \textbf{Faa\$T}~\cite{romero_faa_2021} and \textbf{Kraken}~\cite{vivek_kraken_2021} keep frequently invoked containers ready, balancing readiness and cost efficiency under predictable workloads but incurring overhead when demand is erratic~\cite{romero_faa_2021, vivek_kraken_2021}. While these methods perform well in resource-rich cloud environments, their resource intensity challenges applicability in edge settings.

\subsubsection{Edge FaaS Perspective}

In edge environments, cold start mitigation emphasizes lightweight designs, resource sharing, and hybrid task distribution. Lightweight execution environments like unikernels~\cite{edward_sock_2018} and \textbf{Firecracker}~\cite{agache_firecracker_2020}, as used by \textbf{TinyFaaS}~\cite{pfandzelter_tinyfaas_2020}, minimize resource usage and initialization delays but require careful orchestration to avoid resource contention. Function co-location, demonstrated by \textbf{Photons}~\cite{v_dukic_photons_2020}, reduces redundant initializations by sharing runtime resources among related functions, though this complicates isolation in multi-tenant setups~\cite{v_dukic_photons_2020}. Hybrid offloading frameworks like \textbf{GeoFaaS}~\cite{malekabbasi_geofaas_2024} balance edge-cloud workloads by offloading latency-tolerant tasks to the cloud and reserving edge resources for real-time operations, requiring reliable connectivity and efficient task management. These edge-specific strategies address cold starts effectively but introduce challenges in scalability and orchestration.

\subsection{Predictive Scaling and Caching Techniques}

Efficient resource allocation is vital for maintaining low latency and high availability in serverless platforms. Predictive scaling and caching techniques dynamically provision resources and reduce cold start latency by leveraging workload prediction and state retention.
Traditional FaaS platforms use predictive scaling and caching to optimize resources, employing techniques (OFC, FaasCache) to reduce cold starts. However, these methods rely on centralized orchestration and workload predictability, limiting their effectiveness in dynamic, resource-constrained edge environments.



\subsubsection{Edge FaaS Perspective}

Edge FaaS platforms adapt predictive scaling and caching techniques to constrain resources and heterogeneous environments. \textbf{EDGE-Cache}~\cite{kim_delay-aware_2022} uses traffic profiling to selectively retain high-priority functions, reducing memory overhead while maintaining readiness for frequent requests. Hybrid frameworks like \textbf{GeoFaaS}~\cite{malekabbasi_geofaas_2024} implement distributed caching to balance resources between edge and cloud nodes, enabling low-latency processing for critical tasks while offloading less critical workloads. Machine learning methods, such as clustering-based workload predictors~\cite{gao_machine_2020} and GRU-based models~\cite{guo_applying_2018}, enhance resource provisioning in edge systems by efficiently forecasting workload spikes. These innovations effectively address cold start challenges in edge environments, though their dependency on accurate predictions and robust orchestration poses scalability challenges.

\subsection{Decentralized Orchestration, Function Placement, and Scheduling}

Efficient orchestration in serverless platforms involves workload distribution, resource optimization, and performance assurance. While traditional FaaS platforms rely on centralized control, edge environments require decentralized and adaptive strategies to address unique challenges such as resource constraints and heterogeneous hardware.



\subsubsection{Edge FaaS Perspective}

Edge FaaS platforms adopt decentralized and adaptive orchestration frameworks to meet the demands of resource-constrained environments. Systems like \textbf{Wukong} distribute scheduling across edge nodes, enhancing data locality and scalability while reducing network latency. Lightweight frameworks such as \textbf{OpenWhisk Lite}~\cite{kravchenko_kpavelopenwhisk-light_2024} optimize resource allocation by decentralizing scheduling policies, minimizing cold starts and latency in edge setups~\cite{benjamin_wukong_2020}. Hybrid solutions like \textbf{OpenFaaS}~\cite{noauthor_openfaasfaas_2024} and \textbf{EdgeMatrix}~\cite{shen_edgematrix_2023} combine edge-cloud orchestration to balance resource utilization, retaining latency-sensitive functions at the edge while offloading non-critical workloads to the cloud. While these approaches improve flexibility, they face challenges in maintaining coordination and ensuring consistent performance across distributed nodes.



\section{Proposed Approach}
\label{sec:proposed_approach}

We adopt neuromorphic computing concepts to process event-based data streams. By leveraging spiking neural networks (SNNs), we aim to emulate the inherent efficiency of biological neurons, which communicate primarily through discrete spikes rather than continuous signals.


% In Figure~\ref{fig:event_based_processing}, we illustrate the fundamental pipeline of event-based sensing, its biological inspiration, and how spikes are utilized in SNNs. The left panel visualizes raw event output over time (red or blue dots indicating positive or negative contrast changes). The right panel highlights that SNNs follow a similar principle: incoming spikes raise the neuron’s membrane potential until it surpasses a threshold, triggering an output spike.

Building on neuromorphic principles, our approach harnesses an event camera for real-time object detection and tracking. Subsequent modules—including a spiking neural network (SNN) block, a physics-guided neural network (PgNN) for energy-optimal trajectories, and a rule-based planner—operate on sparse event data to generate collision-free flight paths.
\begin{figure*}[t]
    \centering
    \includegraphics[width=1\textwidth]{ev_box.pdf}
    \caption{%
        \textbf{Event-based Object Detection at Various Depths.}
        Each sub-panel shows neuromorphic event output and a bounding box 
        around a moving gate. Although event density decreases
        with increasing depth, the SNN continues to isolate and track the gate
        in real time.
    }
    \label{fig:ev_box}
\end{figure*}
We propose a system that processes sparse, event-based sensor data to generate collision-free trajectories, as illustrated in Figure~\ref{fig:method}. Our approach integrates:
\begin{enumerate}
    \item A \textbf{spiking neural network (SNN)} for event-based object detection,
    \item A \textbf{physics-guided neural network (PgNN)} for near-minimum-energy destination prediction, and
    \item A \textbf{rule-based planner} for handling moving obstacles.
\end{enumerate}
All components interact in a \textbf{ROS} and \textbf{Gazebo} simulation environment: the SNN publishes bounding-box coordinates of the moving gate, the PgNN outputs an estimated flight time, and a symbolic planner node fuses both to generate velocity commands for the drone’s low-level controller.


\subsection{\textbf{Neuromorphic Vision-based Object Detection}}
\label{subsec:neuromorphic_detection}

In event-based cameras, each pixel independently fires upon detecting changes in brightness, producing sparse asynchronous data streams. Unlike frame-based cameras that capture images at fixed intervals, event-based cameras generate events only when a change in the scene occurs, making them highly efficient for capturing fast motions and dynamic environments. This event-driven nature allows them to have a high temporal resolution and low latency, making them particularly advantageous for real-time applications.

Inspired by \cite{nagaraj2023dotie}, we adopt a biologically-plausible approach using leaky integrate-and-fire (LIF) neurons to detect fast-moving objects. The LIF neuron model simulates how biological neurons process incoming stimuli by accumulating membrane potential over time. Specifically, the membrane potential $V[t]$ of an LIF neuron evolves as:

\begin{equation}
V[t] = \beta , V[t_{n-1}] + W , X[t],
\label{lif_eq}
\end{equation}

where $\beta$ is the leak factor representing the decay of the membrane potential, $W$ is a learnable weight matrix that scales the contribution of the input, and $X[t]$ is the aggregated spike input at time $t$. The leak factor $\beta$ controls how quickly the neuron forgets previous inputs, with smaller values corresponding to faster decay.

Whenever the membrane potential $V[t]$ exceeds a predefined threshold $V_{\mathrm{th}}$, the neuron fires an event and resets its membrane potential. This firing mechanism emulates how biological neurons emit a spike when they reach their activation threshold. In the context of object detection, this spiking behavior is highly beneficial for detecting transient or fast-moving objects, as they produce dense bursts of events due to rapid brightness changes. This property makes the event rate directly proportional to the speed of the object:
\begin{equation}
    \text{Event Rate} \;\propto\; \text{Object Speed},
    \label{ev_rate}
\end{equation}
    
As a result, fast-moving objects produce dense event bursts, which accumulate quickly and cause the membrane potential to surpass $V_{\mathrm{th}}$.

To localize objects within the event stream after a neuron fires, we extract a bounding box based on the spatial distribution of spiking events. This is achieved by computing the minimum and maximum coordinates of the spiking pixels:

These coordinates represent the bounding limits of the detected object. We then calculate the center of the bounding box to determine the object's approximate position:

\begin{subequations}
\label{eq:center}
\begin{align}
\text{center}_x &= X_{\min} + \Bigl\lfloor\frac{X_{\max} - X_{\min}}{2}\Bigr\rfloor,\\
\text{center}_y &= Y_{\min} + \Bigl\lfloor\frac{Y_{\max} - Y_{\min}}{2}\Bigr\rfloor.
\end{align}
\end{subequations}

This computation yields the center point of the bounding box, providing an efficient method for locating objects within the event frame.

As shown in Figure~\ref{fig:ev_box}, the single-layer SNN is able to localize the target even at greater depths, where event density is lower. We use a $\beta = 0.1$, $V_{th}=1.75$. This was obtained after fine-tuning for the given gate which moves at $4$ m/s. We use a $3\times3$ sized kernel for $W$. See Section \ref{ssec:tracking_results} for further details.

\subsection{\textbf{Physics-Guided Trajectory-Duration Predictor}}
\label{subsec:physics_guided}

We model each quadrotor propeller by:
\begin{equation}
    e(t) \;=\; R\,i(t) + K_E\,\omega(t),
    \label{eq:quadrotor_propeller}
\end{equation}

where $R$ is the winding resistance, $i(t)$ the current, and $\omega(t)$ the rotor speed. The total energy from $t=0$ to $t=T$ is given by:
\begin{equation}
    E(T) \;=\; \int_{0}^{T} \sum_{j=1}^{4} e_j(\tau)\,i_j(\tau)\,d\tau,
    \label{eq:E}
\end{equation}
revealing that flying very slowly or very fast can waste energy. Each depth $d$ therefore has an ideal velocity $v_{\mathrm{opt}}$ minimizing overall consumption, illustrated in Figure~\ref{fig:energy_time}.

\begin{figure}[t]
    \centering
    \includegraphics[width=0.95\columnwidth]{energy_time_2.pdf}
    \caption{%
        Optimal velocity and energy-consumption patterns. 
        Each depth features a characteristic velocity $v_{\mathrm{opt}}$ 
        that balances time and power usage.
    }
    \label{fig:energy_time}
\end{figure}

A \emph{physics-guided neural network} (PgNN) learns this relationship by approximating \(E(v)\) through polynomial regression for each depth. By fitting a 5th-degree polynomial to the energy-velocity data, the PgNN captures the non-linear dynamics inherent in the system. The optimal velocity \(v_{\mathrm{opt}}\) is determined by finding the velocity at which the derivative of the energy function equals zero:
\begin{equation}
    \frac{dE(v)}{dv} = 0 \quad \Rightarrow \quad v_{\mathrm{opt}} = \arg\min_v E(v),
    \label{eq:optimal_velocity}
\end{equation}
The PgNN predicts \(v^{\mathrm{pred}}\), yielding an approximate flight time:
\begin{equation}
    t_{\mathrm{traj}} 
    \;=\; \frac{d}{v^{\mathrm{pred}}},
    \label{eq:ttraj}
\end{equation}


\begin{table}[ht]
\centering
\caption{PgNN Training Samples: Depth, Velocity, and Physical Constraints obtained by fiiting polynomial curves}
\label{tab:data}
\begin{tabular}{@{}lcc@{}}
\toprule
\textbf{Depth} & \textbf{Velocity} & \textbf{Constraint}                  \\ \midrule
$d_1$          & $v_1$             & $c_{1,1} + 2\,c_{2,1}\,v_1 + \ldots = 0$       \\
$d_2$          & $v_2$             & $c_{1,2} + 2\,c_{2,2}\,v_2 + \ldots = 0$       \\
$\cdots$       & $\cdots$          & $\cdots$                                      \\
$d_n$          & $v_n$             & $c_{1,n} + 2\,c_{2,n}\,v_n + \ldots = 0$       \\ \bottomrule
\end{tabular}
\end{table}


The samples used to train the PgNN are summarized in Table~\ref{tab:data}, which lists various depths $d_n$, their corresponding optimal velocities $v_n$, and the associated constraints derived from the polynomial fits of $E(v)$ and its derivative.

In battery-constrained aerial systems, ensuring energy-efficient flight requires careful consideration of power consumption, \(P(t)\). Power consumption is directly related to the thrust force as:
\begin{equation}
    P(t) = \kappa \|\mathbf{F}_{\text{thrust}}(t)\|^\alpha,
    \label{eq:power_consumption}
\end{equation}
where \(\kappa\) and \(\alpha\) are constants determined by the propeller actuator characteristics of the UAV. The parameter \(\alpha\) governs the non-linearity of the power-thrust relationship and has a significant impact on the energy expenditure profile across varying flight velocities.


\subsection{\textbf{PgNN Loss Function}}

The Physics-Guided Neural Network (PgNN) is trained to predict near-optimal velocities by minimizing a composite loss function:
\begin{equation}
    \mathcal{L}_{\mathrm{PgNN}} = \mathcal{L}_{\mathrm{data}} + \lambda_1 \mathcal{L}_{\mathrm{physics}} + \lambda_2 \mathcal{L}_{\mathrm{energy}},
    \label{eq:pgnn_loss}
\end{equation}
where each term contributes to a specific aspect of the optimization:

\paragraph{Data-Fitting Loss (\(\mathcal{L}_{\mathrm{data}}\))} Ensures that the PgNN predictions align with the ground truth velocity \(v_i^{\mathrm{opt}}\), derived from offline optimization or simulation data, as follows:
\begin{equation}
    \mathcal{L}_{\mathrm{data}} = \frac{1}{N} \sum_{i=1}^N \bigl(v_i^{\mathrm{pred}} - v_i^{\mathrm{opt}}\bigr)^2.
    \label{eq:data_fitting_loss}
\end{equation}
Here, \(v_i^{\mathrm{opt}}\) is linked to the trajectory time \(t_{\mathrm{traj}}\) (as defined in Equation~\eqref{eq:ttraj}), enabling the PgNN to learn velocity predictions that minimize energy consumption while ensuring timely navigation.

\paragraph{Physics Consistency Loss (\(\mathcal{L}_{\mathrm{physics}}\))} Ensures that the PgNN respects the physical laws governing UAV motion:
\begin{equation}
    \mathcal{L}_{\mathrm{physics}} = \frac{1}{N} \sum_{i=1}^N \bigl|\mathbf{x}_i^{\mathrm{pred}} - \mathbf{x}_i^{\mathrm{sim}}\bigr|,
    \label{eq:physics_consistency_loss}
\end{equation}
where \(\mathbf{x}_i^{\mathrm{sim}}\) represents the UAV states predicted by a physics-based simulation model (Equation \ref{eq:dynamics}). The dynamics of the UAV are governed by Equation~\eqref{eq:power_consumption}, which capture energy consumption \(P(t)\) and the time-energy trade-off in the objective function.

\paragraph{Energy Efficiency Loss (\(\mathcal{L}_{\mathrm{energy}}\))} Promotes predictions that minimize energy consumption, calculated using the power consumption equation:
\begin{equation}
    \mathcal{L}_{\mathrm{energy}} = \frac{1}{N} \sum_{i=1}^N P(t_i^{\mathrm{pred}}),
    \label{eq:energy_efficiency_loss}
\end{equation}
where \(P(t)\) is defined in Equation~\eqref{eq:power_consumption}, and its computation integrates over the trajectory time \(t_{\mathrm{traj}}\). This term enforces efficiency by penalizing excessive energy use across predicted trajectories.

The hyperparameters \(\lambda_1\) and \(\lambda_2\) are tuned to balance the contributions of the physics and energy terms, ensuring that the PgNN achieves both accurate predictions and energy-efficient trajectories. By combining these loss components, the PgNN learns velocity predictions that optimize flight time and energy consumption while respecting physical constraints.
In this work, the PgNN is a $3$-layer fully connected multi-layer perceptron with $[64,128,128]$ neurons. Later, in Section \ref{subsec:reg}, we analyze the effect of varying the hyperparameters of our PgNN.


\subsection{\textbf{Symbolic Planning for Moving Gates}}
\label{symbol}

The system integrates a rule-based planner to handle dynamic obstacles such as moving gates as shown in Algorithm 1 in Fig \ref{fig:method}. In scenarios like drone racing, gates often oscillate along the \(y\)-axis within a bounded range of \(\pm L\). Accurate planning requires predicting the gate's future position based on its current motion and the estimated time of arrival (\(t_{\mathrm{traj}}\)).

Using consecutive gate positions \(\{y_1, y_2\}\) recorded at time intervals of \(\delta t\), the gate's velocity is computed as:
\begin{equation}
    v_r = \frac{y_2 - y_1}{\delta t}.
    \label{eq:gate_velocity}
\end{equation}
This velocity, combined with the predicted time of arrival \(t_{\mathrm{traj}}\) from the Physics-Guided Neural Network (PgNN), estimates the gate's future position:
\begin{equation}
    y^* = y_2 + v_r \cdot t_{\mathrm{traj}}.
    \label{eq:gate_position_prediction}
\end{equation}
If the gate is expected to bounce off a boundary during \(t_{\mathrm{traj}}\), its displacement is adjusted to reflect the change in direction:
\begin{equation}
    y^* = 
    \begin{cases} 
    y_2 - L + x, & \text{if the gate bounces to the left}, \\
    y_2 + L - x, & \text{if the gate bounces to the right}.
    \end{cases}
    \label{eq:gate_boundary_adjustment}
\end{equation}
In this way, the proposed framework ensures collision-free navigation while minimizing energy usage.

\begin{table*}[tb]
\centering
\caption{Comparison of tracking methods (SNN, YOLO, R-CNN, and Hough Transform)}
\label{tab:comparison}
\resizebox{0.9\linewidth}{!}{%
\begin{tabular}{lcccc}
\toprule
\textbf{Method} & 
\textbf{Parameter Count} & 
\textbf{Training / Labeling} & 
\textbf{Thresholds} & 
\textbf{Notes} \\
\midrule
\textbf{SNN \cite{nagaraj2023dotie,evplanner,joshi2024}} & 
\begin{tabular}[c]{@{}l@{}}
1 Conv2D \((3\times3)\)\\
9 Spiking Neurons
\end{tabular} & 
Unsupervised & 
\begin{tabular}[c]{@{}l@{}}
Yes (firing \\
threshold)
\end{tabular} &
Lightweight, event-driven \\
\midrule
\textbf{YOLO (CNN-based) \cite{yolo}} & 
\(\sim\!60\text{-}65\)M params & 
\begin{tabular}[c]{@{}l@{}}
Requires large \\
labeled dataset
\end{tabular} &
No &
Popular real-time detector \\
\midrule
\textbf{Faster R-CNN (CNN-based) \cite{rcnn}} & 
\(\sim\!60\text{-}100\)M params & 
\begin{tabular}[c]{@{}l@{}}
Requires large \\
labeled dataset
\end{tabular} &
No &
Region proposals + deep CNN \\
\midrule
\textbf{Hough Transform \cite{hough_t}} & 
\textit{N/A} & 
No (classical CV) & 
Yes (edges, thresholds) &
Handcrafted approach \\
\bottomrule
\end{tabular}%
}
\end{table*}

\begin{figure*}[htbp]
    \centering
    \includegraphics[width=1\textwidth]{drone_traj.pdf} 
    \caption{Drone navigating through a moving gate in Gazebo simulation. The gate moves from right to left, while the drone starts at position \((1, 1)\). The timestamps (`t = 0s`, `t = 1s`, `t = 2s`, `t = 3s`) indicate the time elapsed at each key stage: (a) Start, (b) Approach, (c) Entry, and (d) Pass-Through.}
    \label{fig:drone_navigation_timeline}
\end{figure*}
\begin{figure}[tb]
    \centering
    \includegraphics[width=0.9\linewidth]{iou_ijcnn.pdf}
    \caption{Tracking performance vs.\ depth. The plot compares the mean IoU (mIOU)
    and the peak IoU across distances from 2\,m to 9\,m.}
    \vspace{-2mm}
    \label{fig:iou_ijcnn}
\end{figure}
\begin{figure}[ht]
    \centering
    \includegraphics[width=\linewidth]{simulated_flight_energy_plot.pdf}  % Path to your PDF file
    \caption{Comparison of flight time, path length, and dynamic energy across varying depths for different values of $\alpha$ and $\lambda$. The left column displays the combined plots of flight time (in seconds) and path length (in meters), showcasing the differences influenced by the regularization weights $\lambda_1$ and $\lambda_2$. The right column shows the corresponding dynamic energy (in joules) for the same parameter sets. Each row corresponds to a different power model exponent $\alpha$ (0.2, 0.3, and 0.5).}
    \label{fig:flight_energy_plot}  
\end{figure}

\begin{figure*}[!t]
    \centering
    \includegraphics[width=0.95\textwidth]{IJCNN_3D.pdf}
    \caption{Comparison of Navigation Trajectories: The figure illustrates the navigation paths of a drone using depth-based perception (red trajectory) versus a physics-guided neuromorphic vision-based approach that fuses event and depth sensors (green trajectory). The neuromorphic approach consistently results in shorter, more energy-efficient trajectories across varying initial drone positions and ring depths.}
    \label{fig:3d_navigation_paths}
\end{figure*}
\begin{figure}[t!]
    \centering
    \includegraphics[width=\linewidth, height = 3.3cm]
    {detailed_flight_path_plot.pdf}
    \caption{Flight time (left y-axis) and path length (right-y axis) as functions of depth for different initial drone positions for our physics-guided neuromorphic approach. The left y-axis shows the flight time in seconds, and the right y-axis shows the corresponding path length in meters. The data is grouped by depth values, with distinct markers indicating measurements for each combination of flight time and path length.}
    \label{fig:flight_path_depth}
\end{figure}

\noindent{\textbf{Neurosymbolic Integration for Navigation}}: The planner incorporates inference results from multiple neural network modules. The event-based Spiking Neural Network (SNN) detects and tracks the moving gate, providing bounding box coordinates \((y_1, y_2)\). The Physics-Guided Neural Network (PgNN) predicts the trajectory time \(t_{\mathrm{traj}}\) while optimizing for energy efficiency. These outputs are fused and processed through logical rules, forming a neurosymbolic framework that combines data-driven predictions with rule-based reasoning. This integration enables the system to account for boundary dynamics, predict gate behavior, and compute the gate’s future position \(y^*\). By leveraging this neurosymbolic approach, the system achieves efficient and safe navigation through moving gates. Please note, this work only considers local reactive planning, necessary when there are changes observed in the immediate surrounding environment of the robotic agent (drone in this work).



\section{Experimental Results}


\subsection{\textbf{Neuromorphic Vision-Based Tracking}}
\label{ssec:tracking_results}
Figure \ref{fig:ev_box} shows neuromorphic event output and a bounding box 
around a moving gate (left to right). Although event density decreases
with increasing depth, the SNN continues to isolate and track the gate
in real time. Correspondingly, as illustrated in Figure~\ref{fig:iou_ijcnn}, both the mIOU and the peak IOU gradually decline as the distance between the event-based camera and the moving gate increases. At shorter ranges (2--4\,m), the system achieves a mean IOU between 0.78 and 0.83, with a peak IOU above 0.90 at 4\,m. This high accuracy stems from dense event bursts generated by the gate’s motion, allowing the spiking neural network (SNN) to consistently localize the target with minimal false positives.

However, at larger depths (beyond 6,m), the IOU values drop slightly (mIOU: 0.60--0.65; peak IOU: 0.62--0.73) due to sparser events and weaker contrast changes. Despite this, the neuromorphic detection pipeline maintains robust tracking across a wide range of depths, demonstrating its suitability for low-latency object detection in dynamic flight scenarios.




As summarized in Table~\ref{tab:comparison}, our shallow SNN-based approach has an extremely low parameter count (one \(3\times 3\) convolutional layer and 9 spiking neurons). It operates without a large labeled dataset but still requires threshold tuning (e.g., spiking neuron firing thresholds). Meanwhile, YOLO and R-CNN each demand tens of millions of parameters, along with substantial labeled training data. Finally, the classical Hough Transform method does not rely on deep learning but involves hand-tuned thresholds for edge detection. Our SNN maintains a balance of minimal parameter overhead, unsupervised learning, while offering the benefits of neuromorphic perception using SNN-based dynamic vision sensors.

\subsection{\textbf{Sensitivity Analysis for Physics-based Regularization}}
\label{subsec:reg}
To examine how the power-model exponent \(\alpha\) (Equation~\ref{eq:power_consumption}) and the regularization weights \(\lambda_1\) and \(\lambda_2\) (Equation~\ref{eq:pgnn_loss}) jointly shape flight efficiency, we conducted a series of simulations at varying ring depths and initial drone positions. Figure~\ref{fig:flight_energy_plot} presents the resulting flight time, path length, and dynamic energy for three representative \(\alpha\) values (0.2, 0.3, and 0.5). Each row corresponds to one exponent, while different lines indicate distinct \(\lambda_1,\lambda_2\) settings. The left column combines flight time (in seconds) and path length (in meters), highlighting how \(\alpha\) and \(\lambda\)-values modify navigation efficiency. The right column displays the corresponding dynamic energy (in joules), illustrating the interplay between control regularization and power consumption.


As shown in the left-column plots of Figure~\ref{fig:flight_energy_plot}, increasing the ring depth results in longer flight times and paths, especially at higher \(\alpha\). This effect is magnified when \(\lambda\)-values are small, since weaker regularization allows more aggressive maneuvers, potentially increasing path deviations. Conversely, higher \(\lambda_1,\lambda_2\) steer the flight toward smoother trajectories, dampening sudden control changes and thereby reducing energy spikes (right-column plots). 
Note that, the top row ($\alpha = 0.2$) reflects parameters tuned to our Bebop simulation, while the middle and bottom rows ($\alpha = 0.3$ and $\alpha = 0.5$) illustrate hypothetical regimes. This highlights the sensitivity of the power‐consumption model to $\alpha$; in real‐world applications, calibrating $\alpha$ to the specific drone is crucial for accurately capturing its flight dynamics and energy demands. 

Also, note that the energies reported for inference during the actual flights are higher than the ones presented in Figure \ref{fig:energy_time}, as the actual flight paths are longer than the straight line flight distances covered (no moving gate there) while collecting training data (Table \ref{tab:data}).

 In the following subsection, we expand our analysis to examine how these parameter choices translate into navigation performance across a range of drone starting positions and ring depths. 


\subsection{\textbf{Neuromorphic Navigation for different Drone Positions }}
Figure~\ref{fig:drone_navigation_timeline} illustrates the drone's navigation through a moving ring, with timestamps marking key stages from start ($t = 0s$) to pass-through ($t = 3s$), highlighting the smooth progression despite the dynamic nature of the ring.



Figure~\ref{fig:3d_navigation_paths} illustrates the comparison of navigation trajectories for a drone navigating through a ring from various initial positions \((x, y, z)\) and ring depths ranging from 2~m to 6~m. The figure highlights the performance of the depth-based perception approach (red trajectory) versus the physics-guided neuromorphic vision-based approach (green trajectory), which combines event-based and depth-based sensor inputs. The trajectory of the ring's center is depicted in orange. We observe the following:
\begin{itemize}
    \item \textbf{Energy-Efficient Trajectories:} The neuromorphic vision-based approach produces shorter paths, such as achieving a path length of 3.5~m at a 2~m depth, compared to 4.6~m for an off-center start.
    
    \item \textbf{Accuracy and Responsiveness:} When starting from positions \((-2, y, z)\) or \((+2, y, z)\), the neuromorphic approach maintains a shorter flight time of around 2.9~s at a depth of 3~m, compared to longer times exceeding 4.1~s for off-center starts at greater depths.
    
    \item \textbf{Trajectory Deviation:} Depth-based perception results in longer paths, such as 7.6~m at a 5~m depth, compared to the neuromorphic approach’s 5.9~m for the same conditions, demonstrating the effect of initial offsets.
    
    \item \textbf{Impact of Depth:} The neuromorphic approach remains efficient with path lengths ranging from 3.5~m (at 2~m) to 9.5~m (at 6~m), while the depth-based method increases path length at greater depths, up to 10~m or more.
\end{itemize}


Figure~\ref{fig:flight_path_depth} further illustrates the dependence of flight time and path length on ring depth for various initial drone positions. Here, the left y-axis measures flight time (in seconds), while the right y-axis tracks path length (in meters). Distinct markers indicate different initial offsets \((x, y)\), showing that off-center starts tend to increase the required flight time and distance. For instance, at a depth of 3~m, the flight time increases from 2.3~s (centered start) to 2.9~s (off-center start), with the corresponding path length increasing from 3.5~m to 4.6~m. At a depth of 5~m, the off-center start results in a flight time of 4.1~s and a path length of 7.6~m. These results highlight the importance of appropriate tuning of \(\alpha\) and \(\lambda\)-values to balance energy conservation against the need for precise maneuvers. Overall, the physics-guided neuromorphic approach reduces average flight time by approximately 20\% and path length by around 15\% compared to the depth-based method, demonstrating its potential for more efficient autonomous aerial navigation.

\section{Conclusion}
This work presents a novel framework for energy-efficient autonomous aerial navigation that leverages neuromorphic event-based vision and physics-guided planning. By integrating dynamic vision sensors (DVS) with spiking neural networks (SNNs) and physics-guided neural networks (PgNNs), the system achieves real-time responsiveness and energy optimization. Experimental results demonstrate that the proposed approach produces shorter, energy-efficient trajectories with high accuracy and adaptability across varying initial positions and depths. Notably, the fusion of asynchronous event data with depth information enhances trajectory planning by enabling robust navigation through dynamic environments, such as moving gates. The neuromorphic design outperforms traditional depth-based methods due to its low-latency processing and efficient handling of sparse data, making it particularly advantageous for scenarios for real-time decision-making.

The proposed framework is also a neurosymbolic integration that combines event-driven neuromorphic computing, physics-based AI, symbolic logical reasoning, and classical planning. The neuromorphic SNN module processes asynchronous event data for low-latency perception, while the PgNN embeds physical constraints to generate energy-optimal trajectories that align with real-world dynamics. The rule-based planner introduces symbolic reasoning to anticipate obstacle movements and guide navigation decisions. This synergy between neural and symbolic components allows the system to handle dynamic, real-time scenarios while maintaining interpretable and physically consistent control. By uniting data-driven perception with rule-based reasoning, the framework achieves adaptable decision-making, highlighting possible benefits of neurosymbolic design for autonomous systems.

\section{Acknowledgement}
This work was supported by the Center for the Co-Design of Cognitive Systems (CoCoSys), a center in JUMP 2.0, an SRC program sponsored by DARPA.



% \section*{References}
\bibliographystyle{IEEEtran}
\bibliography{conference_101719}

% \begin{thebibliography}{00}
% \bibitem{b1} G. Eason, B. Noble, and I. N. Sneddon, ``On certain integrals of Lipschitz-Hankel type involving products of Bessel functions,'' Phil. Trans. Roy. Soc. London, vol. A247, pp. 529--551, April 1955.
% \bibitem{b2} J. Clerk Maxwell, A Treatise on Electricity and Magnetism, 3rd ed., vol. 2. Oxford: Clarendon, 1892, pp.68--73.
% \bibitem{b3} I. S. Jacobs and C. P. Bean, ``Fine particles, thin films and exchange anisotropy,'' in Magnetism, vol. III, G. T. Rado and H. Suhl, Eds. New York: Academic, 1963, pp. 271--350.
% \bibitem{b4} K. Elissa, ``Title of paper if known,'' unpublished.
% \bibitem{b5} R. Nicole, ``Title of paper with only first word capitalized,'' J. Name Stand. Abbrev., in press.
% \bibitem{b6} Y. Yorozu, M. Hirano, K. Oka, and Y. Tagawa, ``Electron spectroscopy studies on magneto-optical media and plastic substrate interface,'' IEEE Transl. J. Magn. Japan, vol. 2, pp. 740--741, August 1987 [Digests 9th Annual Conf. Magnetics Japan, p. 301, 1982].
% \bibitem{b7} M. Young, The Technical Writer's Handbook. Mill Valley, CA: University Science, 1989.
% \end{thebibliography}

\end{document}
