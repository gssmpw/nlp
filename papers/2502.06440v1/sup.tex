\documentclass[conference]{IEEEtran}
\usepackage{lipsum}  % 用于生成随机文本
\usepackage{graphicx}  % 如果需要包含图像
\usepackage{amsmath}  % 如果需要数学公式
\usepackage{amssymb}
\usepackage{booktabs}
\usepackage{diagbox}


\begin{document}


\appendices
\section*{Appendex}
\renewcommand{\thesection}{\Alph{section}}
\subsection*{A. Proof Of The Global Section Condition}

Let $\mathcal{F}$ be a sheaf on a topological space $X$, and consider two vertices $u$ and $v$ connected by an edge $e$. Assume $U_u$ and $U_v$ are neighborhoods of $u$ and $v$, respectively, covering the edge $e$.

Define local sections $x_u \in \mathcal{F}(U_u)$ and $x_v \in \mathcal{F}(U_v)$. The sheaf's compatibility condition requires:
\begin{equation}
    x_u|_{U_u \cap U_v} = x_v|_{U_u \cap U_v}
\end{equation}
on the overlap $U_u \cap U_v$, which includes the edge $e$.

We focus on the edge $e \subseteq U_u \cap U_v$. By the compatibility requirement:
\begin{equation}
    \mathcal{F}_{v \unlhd e} x_v = \mathcal{F}_{u \unlhd e} x_u
\end{equation} 

where $\mathcal{F}_{v \unlhd e}$ and $\mathcal{F}_{u \unlhd e}$ represent the restrictions of $x_v$ and $x_u$ to the edge $e$.

The equality $\mathcal{F}_{v \unlhd e} x_v = \mathcal{F}_{u \unlhd e} x_u$ confirms that the local sections agree on their overlap on $e$, thereby satisfying the sheaf condition. This condition guarantees that local data can be consistently extended to global data across overlaps, fundamental to the structure of a sheaf.

\subsection*{B. Performance Metric}
\textbf{Episode Length (EL)}: Used to measure the optimality of the solution produced by the planner. A shorter episode length means that all agents can reach the goal faster.

$$
e l=\sum_{i=1}^N \frac{p l_i}{N}
$$

$p l_i$ represents the path length the last agent requires to reach the goal when all agents in the $i$-th episode reach the goal.

\textbf{Success rate (SR)}: It is used to measure the ability of the planner to complete a MAPF task, and the following formula can express its calculation:

$$
s r=\frac{s}{N}
$$

where $s$ represents the number of episodes in which all agents reach the goal within the specified number of steps, and $N$ is the total number of episodes tested.

\textbf{Arrival Rate (AR)}: It is used to measure the ability of the planner to complete lifelong tasks. A higher arrival rate means that the planner has better potential for lifelong MAPF tasks.

$$
a r=\sum_{i=1}^N \frac{a r r_i}{N \times n}
$$


Among them, $n$ represents the number of agents in each episode, and $a r r_i$ represents the number of agents reaching the goal in the $i$-th episode.

\subsection*{C. Section Loss}

We introduced a section loss and incorporated it into the Q-value calculation, effectively guiding agents towards consensus. The test results following training, as shown in Table \ref{loss}, indicate that the section loss values tend to be very small (ranging from $10^{-4}$ to $10^{-3}$).

\begin{table}[h]
\centering
\caption{section loss}
{\fontsize{7.5pt}{12pt}\selectfont
\setlength{\tabcolsep}{4pt}
\begin{tabular}{c|cccccc}
\toprule
\diagbox[dir=NW]{\textbf{Map}}{\textbf{Agent}} & \textbf{4} & \textbf{8} & \textbf{16} & \textbf{32} & \textbf{64} & \textbf{128} \\
\midrule
40*40 & 6.63e-04 & 8.83e-04 & 1.37e-03 & 9.31e-04 & 4.30e-04 & 1.08e-04 \\
60*60 & 3.41e-04 & 5.96e-04 & 9.62e-04 & 1.13e-03 & 9.51e-04 & 4.74e-04 \\
\bottomrule
\end{tabular}
}
\label{loss}
\end{table}

\subsection*{D. Hyperparameters}
Table \ref{hyperparameters} describes the primary hyperparameters in our method. The computational simulations and experiments for our study were executed on Ubuntu 20.04 equipped with an AMD EPYC 7T83 CPU and a Nvidia GeForce RTX 3090ti GPU. As for the software framework, the experiments leveraged Python version 3.8.18 coupled with PyTorch version 2.0.0 to ensure efficient computation and reproducibility.
\begin{table}[h]
\centering
\caption{Hyperparameters table}
\begin{tabular}{@{}ll@{}}
\toprule
\textbf{Hyperparameters} & \textbf{Value} \\ \midrule
Number of agents& 5\\
FOV size& 9\\
Maximum episode length& 512\\
World size& (10,40)\\
Capacity of episodic buffer& 2048\\
gamma & 0.9 \\
batch size& 192\\
optimize & Adam \\
prioritized replay $\alpha$& 0.6\\
prioritized replay $\beta$& 0.4\\ \bottomrule
\end{tabular}
\label{hyperparameters}
\end{table}


\end{document}
