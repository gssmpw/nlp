\section{Related Work}
\label{Relatedwork}
In this section, we review existing studies on serverless edge computing frameworks and architectures. To the best of our knowledge, there is no comprehensive systematic literature review on the taxonomy of serverless edge computing. Various authors have conducted literature reviews on serverless computing in specific IoT application areas~\cite{cassel2022serverless-20}. Notably, some surveys focus on serverless edge computing~\cite{aslanpour2020performance-17},\cite{ioini2020platforms-22},\cite{xie2021serverless-15}, while others explore serverless computing within cloud environments~\cite{taibi2020patterns-24},\cite{eismann2020review-25},\cite{yussupov2019systematic-26},\cite{scheuner2020function-27},\cite{li2022serverless-38},\cite{mampage2022holistic-18}. These surveys on serverless edge computing, however, lack rigorous methodologies, as they do not specify the research strategies used to select the evaluated papers. Therefore, they are best classified as literature reviews rather than systematic reviews.

~\cite{aslanpour2021serverless-21} provides valuable insights into serverless edge computing, demonstrating how edge devices benefit from this technology. The authors discuss opportunities and open issues, including task offloading, location-agnostic processing, energy efficiency, and security. Similarly,~\cite{ioini2020platforms-22} presents a review of platforms suitable for serverless edge computing, highlighting core features and suggesting avenues for future research. Platforms such as AWS IoT Greengrass, Azure IoT Edge, FogFlow, Nuclio, and OpenWhisk-Light are identified as supportive of edge deployments.

In another survey,~\cite{xie2021serverless-15} provides an in-depth analysis of serverless edge computing architectures, challenges, and unresolved issues. While these surveys address various aspects of serverless edge computing, others focus on traditional serverless computing on cloud platforms, discussing issues and solutions relevant to cloud-only environments.

Further contributions to this area include~\cite{kjorveziroski2021iot-48}, who perform a systematic literature review of IoT serverless computing at the edge, examining eight areas, including efficiency, application implementation, scheduling, benchmarks, platform implementation, policy, and open-source software. This study highlights the growing interest in serverless edge computing over the last three years. Similarly,~\cite{gadepalli2019challenges-49} discusses open challenges in serverless edge computing, such as existing serverless designs' limitations. The study explores WebAssembly as an alternative to traditional implementations, offering near-native performance, low memory use, and instant invocation in serverless applications.

Meanwhile,~\cite{li2022serverless-38} surveys serverless design architectures, categorizing them into four layers—virtualization, encapsulation, system orchestration, and coordination. This layered approach examines the benefits and limitations of each layer, proposing solutions to current challenges. Complementing this,~\cite{mampage2022holistic-18} reviews resource management in serverless computing, proposing a taxonomy that encompasses resource management across edge, fog, and cloud infrastructures, focusing on system design, workload management methodologies, and QoS objectives.

In their "multivocal literature review,"~\cite{taibi2020patterns-24} identifies 32 patterns in serverless solution deployment, grouped into orchestration, event management, availability, communication, and permission. They discuss the advantages and challenges of each pattern. Additionally,~\cite{eismann2020review-25} presents attributes from 89 real-world serverless use cases, noting that most solutions rely on AWS, consist of fewer than five functions, and are commonly implemented in JavaScript or Python. Interestingly, only 3\% of these cases address serverless edge computing, revealing a significant research gap in this domain.

~\cite{raith2023serverless} provides a structured review of serverless edge computing frameworks, focusing on the edge–cloud continuum's unique challenges, including programming support for AI, reliability, and performance optimization. The authors assess the maturity of various commercial, open-source, and academic platforms using a set of defined criteria that evaluate aspects such as application state management, fault tolerance, and resource efficiency. Unlike previous studies, this work introduces a maturity model to systematically analyze platform capabilities, emphasizing the importance of AI-driven intelligence, SLO awareness, and edge support. Their findings highlight gaps in existing frameworks and offer a roadmap for advancing serverless edge computing, aiming to unify cloud and edge resources effectively and enable robust, latency-sensitive IoT applications.
Table~\ref{tab:RelatedWork} shows the related work
\begin{center}
\begin{table*}
\caption{Related Work}
\label{tab:RelatedWork}
\begin{tabular}{|p{1.8cm}|c|c|c|p{1.8cm}|p{7.8cm}|}
\hline
\textbf{Authors} & \textbf{SLR} & \textbf{LR} & \textbf{Tax.} & \textbf{Main Focus} & \textbf{Description} \\
\hline
\cite{aslanpour2021serverless-8} & -- & \checkmark & \checkmark & Edge & Opportunities and challenges for serverless edge computing. \\
\hline
\cite{ioini2020platforms-22} & -- & \checkmark & -- & Edge & Discuss serverless edge computing platforms. \\
\hline
\cite{xie2021serverless-15} & -- & \checkmark & -- & Edge & Discuss opportunities and open challenges for serverless edge computing. \\
\hline
\cite{li2022serverless-38} & -- & \checkmark & -- & Serverless & Survey of serverless computing research domains, highlighting issues and future work suggestions. \\
\hline
\cite{kjorveziroski2021iot-48} & \checkmark & -- & -- & Serverless & Discussion of open issues for IoT serverless at the edge. \\
\hline
\cite{gadepalli2019challenges-49} & -- & \checkmark & -- & Serverless & Analysis of open issues and benefits of serverless at the edge. \\
\hline
\cite{cassel2022serverless-20} & \checkmark & -- & \checkmark & Cloud & Systematic review of serverless computing for IoT, covering characteristics and architecture of serverless cloud computing. \\
\hline
\cite{taibi2020patterns-24} & \checkmark & -- & \checkmark & Cloud & Organization patterns for serverless solutions. \\
\hline
\cite{eismann2020review-25} & \checkmark & -- & -- & Cloud & Serverless computing with real-world use cases. \\
\hline
\cite{yussupov2019systematic-26} & \checkmark & -- & -- & Cloud & Current trends, challenges and industry relationships in serverless computing.\\
\hline
\cite{scheuner2020function-27} & \checkmark & -- & \checkmark & Cloud & Performance evaluation of FaaS platforms. \\
\hline
\end{tabular}
\end{table*}
\end{center}

The authors~\cite{yussupov2019systematic-26} propose a Systematic Mapping Study that covers 62 papers from 2009 to 2019 and focuses on the following three primary subjects: first, publication trends regarding improvements or proposals of FaaS platforms and tools; second, classification of papers based on the challenges, and finally discovering how the industry is connected to the publications, as some papers focus on prototypes while others present production-ready solutions that can be effectively used on the industry. Finally,~\cite {scheuner2020function-27} presented a literature review that contains 112 distinct research that addressed performance measures for different serverless platforms.
 According to an  existing survey, the key shortcomings are  as follows
 \begin{enumerate}
     \item  According to the current studies, they did not provide any taxonomy or systematic literature for serverless edge computing.
     \item The current surveys have not included important factors for serverless edge computing framework.
     \item The surveys that were conducted previously were done so without using or just partly using a systematic literature review process. In this study, we adhered to the protocol for systematic reviews and used genuine articles as our criterion for article selection.
 \end{enumerate}