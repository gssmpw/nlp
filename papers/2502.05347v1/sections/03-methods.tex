\section{Methods}
\subsection{Overview}
We conducted three studies:  
1) a 14-week study of how a small newsroom adopts a creativity support tool, 
2) an audience reception study assessing the videos made with the tool, and 
3) a 4-week study of how two professional TV journalists adopt a creativity support tool. 
We used iterative, exploratory methods~\cite{creswell2017designing}, guided by early insights from student journalists. Each study strand was analyzed independently yet informed by prior findings. Institutional review boards deemed the project exempt.


\subsection{Organizational Cohort Field Study with Student Journalists}
In the first study, researchers recruited advanced students from a university journalism program that runs a TV and print journalism media outlet called \nccnews~to be part of researchers' simulated newsroom. 
As part of their existing program, the students report local area news, often publishing multiple stories a day with faculty guidance and approval. 
\nccnews~ has a fledgling TikTok channel, with fewer than 50 posts and no posts getting more than 3000 views. 
Simulated newsroom environments are common in journalism education spaces~\cite{steel2007experiential} as part of practical and experiential learning with both procedural (environment) and factual (task) authenticity~\cite{chen2001pedagogy}.  
Choosing students in a university newsroom was fairly representative of professional newsrooms because they often hire recent graduates to run social media, due to their familiarity with social media and the demographic is aspired to reach.

Recruitment emails were sent to senior journalism classes offering an opportunity to learn about creating news reels with AI in a newsroom environment and to regularly publish videos under editorial guidance. 
Like an internship, this would be a paid opportunity.
Six student journalists attended the first training session, and five participated in the full 14-week newsroom simulation.  
The size of this sample, along with attrition, is expected for longitudinal qualitative research~\cite{saldana2003longitudinal}. 

In this simulated newsroom environment, we had one researcher acting as editor, two researchers acting as market researchers, and five students given news stories each week to convert to reels, as if they were part of a digital news desk. 
The news stories were ones the newsroom had previously reported, sometimes by the student themselves. 
In weekly meetings, researchers acted as both participant observers ~\cite{emerson2011writing}, taking notes on how student journalists used the tool, and observant participants, interacting with the student journalists to offer the support an editor would provide~\cite{seim2024participant}, while allowing space for participants to share insights on production techniques with fellow student journalists as they would in a standard operating newsroom. 
Student journalists were regularly asked about their use or nonuse of the tool in the creation of their newsreels.
The editor led group feedback discussions for every video, and gave specific feedback for iteration. After approval by the editor, newsreels were published to the \nccnews~ TikTok channel.

\subsection{Audience Reception Study}
To compare the quality of the student journalist-produced news reels to the other short form video news content~\cite{shearer2024americans}, we conducted qualitative interviews with typical short-form video audience members~\cite{brennen2021qualitative}. 
This allowed us to do two things: 1) access the relative quality of the student production, and 2) provide the student journalists with specific information to improve the audience reception of their videos. 
Questions covered how informative and entertaining the video was, their likelihood to watch the video if they came across it on their personal feed, and the likelihood of the video fitting in with content they typically see, from both qualitative and quantitative perspectives. 
Interviews took between 20 and 66 minutes, with an average of 35 minutes. 
46 individuals signed up to participate, n=42 participants were interviewed. 

Participants were shown six total short-form news videos. 
All six videos were systematically shown in a predetermined balanced design to evenly distribute order effects.
Participants watched four videos created by the student journalists using ReelFramer on a range of topics.  
Two additional videos were added for comparison: one was professionally created by the Washington Post’s TikTok team to cover the trajectory of a hurricane, the other was created by a student in the \nccnews~ program but not in the study and not using the tool; it discussed a local basketball tournament. 

Participants agreed to be interviewed in exchange for course credit as part of the university’s subject pool system. 
All participants were university students from across disciplines aged between 18-30, ($\mu$ = 20). 
Researchers recorded the interviews over Zoom and transcribed the videos before analysis began for axial themes~\cite{saldana2003longitudinal}. 
The distribution of quantitative responses was plotted to compare the informativeness and fit of the news reels. 

\subsection{Qualitative Cohort Field Study with Practicing Journalists} 
To understand if there were differences in how professional journalists would adapt to the creativity support tool, researchers conducted a qualitative cohort field study with practicing journalists. 
In the summer of 2024, researchers recruited a cohort of journalists by posting in a social media group for multimedia journalists.
Researchers informed participants we would hold four, one-hour collaborative sessions. 
12 individuals completed a demographic questionnaire. 
Four individuals attended at least one training session, two produced at least one video using the tool. 
This sample size and these levels of attrition are not unexpected in longitudinal  studies and do not diminish the benefit of the findings~\cite{saldana2003longitudinal}.
Similar to the student study, researchers acted in the capacity of newsroom editor offering critique and advice.
