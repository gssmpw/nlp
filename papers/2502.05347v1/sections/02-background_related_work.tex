\section{Background and Related Work}

\begin{figure*}
\centering
\includegraphics[width=0.88\textwidth]{fig/teaser.pdf}
\caption{
ReelFramer is a human-AI co-creative system that supports journalists in creating news reels by transforming web articles into scripts and storyboards.
ReelFramer enables the exploration of multiple narrative framings that span the infotainment spectrum and their foundational details. 
It also aids in visual framing for character and key visual detail design.
}
\label{fig:teaser}
\end{figure*}

\subsection{Studies of AI in a Work Context}

Recent studies have mapped out the significant opportunities that generative AI brings to various industries, highlighting its potential to enhance productivity for a wide range of workspace tasks \cite{brynjolfsson2023generative, scienceproductivity}. 
It has shown promise in replacing both cognitive routine work (often considered drudgery) and cognitive non-routine work, including creative tasks \cite{bouschery2023augmenting}.
In an experimental study with a writing task carried out by 453 college-educated professionals, \citet{scienceproductivity} found that those using ChatGPT both saved time and increased quality.
Two experiments with Boston Consulting Group \cite{dell2023navigating} show that consultants with access to ChatGPT were more productive and produced higher-quality ideas.
Studies of AI coding tools show it leads to a 26\% increase in weekly coding tasks completed over three companies \cite{Cui_2024, peng2023impact}.

However, various studies also express reservations about AI's ability to fully replace human workers. 
A study of creativity support tools demonstrates that while the tools are helpful for specific tasks, they cannot solve problems entirely on their own \cite{tao2023ai}---especially for tasks involving core human skills, even in the long run~\cite{long2024novelty}. 
AI support can even be problematic. 
\citet{dell2023navigating} show that for difficult tasks, subjects using ChatGPT were about 20\% less likely to make the correct recommendation, as many were led astray by ChatGPT's convincing but incorrect analysis. 
Even when AI is correct, users face challenges incorporating AI results \cite{Kim2024}. 

These limitations suggest that AI might not completely reshape entire industries but rather impact certain aspects of how people work and what they focus on~\cite{randazzo2024cyborgs}. 
However, developing these new configurations requires learning which tasks AI can be trusted to complete and how best to incorporate its results.


\subsection{AI for Creativity Support}

The use of AI for creativity support, particularly to assist writing, has expanded rapidly in recent years, with applications across various domains, including journalism~\cite{opal,reelframer,anglekindling}. 
AI writing assistants can provide scaffolding for writers by supporting different cognitive stages of the writing process~\cite{flower1981cognitive}, including framing~\cite{hui2023lettersmith, reelframer}, idea generation \cite{clark2018creative, anglekindling} to drafting~\cite{sparks, VISAR,tweetorial_hook} and revision \cite{lee2024design, sparks, wu2019design}. 
Lee et al. \cite{lee2024design} provide a comprehensive design space for intelligent and interactive writing assistants that emphasizes the importance of considering task, user, technology, interaction, and ecosystem aspects when using AI to support writers. 
However, using AI in creativity also has risks; overreliance~\cite{overreliance_ibm, overreliance_facct, overreliance_examples, overreliance_difficulty} on AI can cause people to accept poor AI answers, and AI ideation can actually reduce diversity across people~\cite{kreminski_homogenity}.
Broadly, AI can be seen not as a solution, but as a design material that needs to be understood by human creators \cite{ai_design_material_CHI22, eytan_design_material}.

A major creative application of AI lies in content retargeting and transformation—editing and repurposing text, image, video, or audio for new audiences, platforms, or goals.
The key is to find the fundamental message to preserve while adapting the content into the new format.
Previous works have explored how to support people with content transformation tasks in terms of video~\cite{truong2021automatic,chi2021automatic,chi2022synthesis,podreels,Rope_wang}.
For example, AI systems can generate video podcast trailers by identifying salient scenes to summarize content while preserving its tone \cite{podreels}, or shorten audio stories for social media consumption \cite{Rope_wang}. 
However, creating effective short-form content poses unique challenges, particularly in planning and structuring narratives \cite{kim2024unlocking}. 
Systems like ReelFramer \cite{reelframer} address these hurdles by embedding genre-specific knowledge to guide writers in scripting videos that align with journalistic formats, exemplifying how AI can scaffold human creativity in multimedia journalism. 
By encoding domain expertise, such tools streamline the adaptation of long-form content into short, platform-optimized outputs.

\subsection{News Reels}

News reels help outlets reach younger viewers who favor TikTok and Instagram~\cite{newman_news_tt_reuters_oxford}. Nearly half of major publishers now post on TikTok, and younger audiences are increasingly seeking news there~\cite{newman_digital_news_2022}.
Although both news reels and traditional news aim to educate the audience on relevant information, the narrative forms and framings are very different. 
Traditional journalistic narratives are fact-driven and explain events in a serious tone. 
By contrast, news reels seek to balance information and entertainment by taking a more playful tone, in order to conform to the narrative structures common to social media videos.  
A common way of achieving this goal is to act out the story from the perspective of different characters that personify aspects of the story. 
For example, in a Washington Post reel\footnote{\url{https://www.tiktok.com/@washingtonpost/video/7147811605655473450}}, a character representing Hurricane Ian explains to another character representing the hurricane’s roommate why he is “visiting” Florida and all the trouble he might cause there. 
Even for a serious topic like hurricane destruction, news reels must strike a balance between informational content and entertainment value~\cite{newman_news_tt_reuters_oxford}. 
Figure \ref{fig:rf2-ex-schools} contains an example of a news reel made in our newsroom study with a dialog between a School District character and a Teachers Union character acting out the news of a teacher shortage. 
As is typical of news reels, the creator is playing both roles.

Even within the role-play format, framing a news story for social media is challenging. 
In interviews conducted by the Reuters Institute in 2022 with
news reel creators~\cite{newman_news_tt_reuters_oxford}, many different approaches to creating reels
have been found, but a key challenge is to balance between information
and entertainment~\cite{newman_news_tt_reuters_oxford}. 
If a serious topic is portrayed too light-heartedly, it may diminish the perceived seriousness of the issue~\cite{davis2022infotainment}.
However, if a video does not grab a reader’s attention with something that is interesting and entertaining, they may scroll past it. 
Some news organizations take a conservative approach and stay closer to the TV journalism format, but others lean into the tropes of social media videos to blend news information into an engaging social media style. 
The challenge we address is to help journalists
translate their articles into this style, skillfully blending information and entertainment to strike an appropriate tone.

\subsection{ReelFramer AI Tool}

ReelFramer~\cite{reelframer} is a creativity support system that uses generative text and image AI to scaffold the process of creating a script and storyboard for a news reel. 
Generative AI is helpful for exploring multiple narrative framings and their details, and generating scripts, character boards, and storyboards. 
The system is designed to be highly interactive, allowing people to make choices and guide the system at every step to ensure the news reels it produces are accurate, authentic, and achievable to execute.

The user starts by inputting the news article (see Figure \ref{fig:teaser}). 
The system uses a large language model (LLM), GPT-4, to extract locations, people, and activities from the article. 
The user then selects one of the three narrative framings, expository dialogue, reenactment, or comedic analogy. 
Given a framing, the system uses the LLM to suggest foundational details like characters, plot, setting, and key information, which is also called premise in scriptwriting~\cite{batty2017script}.
The user can accept, regenerate, or edit these suggestions.
Once the user finalizes a premise, the system automatically generates a script. 
Again, the user can accept, regenerate, or edit the script.
Users are encouraged to explore many narrative framings, premises, and scripts. 
Once the user is happy with the script, the system provides visual framing to explore the characters’ visual design.
It generates a character board that contains visual details of the costumes, props, and backgrounds to distinguish the characters using a text-to-image model, DALL-E 2. 
Based on the character board, it generates a storyboard that shows suggested emotions, actions, and dialogues for each character. 
Users can then make reels based on these materials.
