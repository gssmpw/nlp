\section{Study 2 Findings: Audience Reception} 
\label{market_research}

\begin{figure}[t]
    \centering
    \includegraphics[width=1\linewidth]{fig/entertaining.png}
    \caption{Violin plots showing the distribution of ratings of videos as entertaining on a 1-7 scale. P is the professionally-created video, S the student-created video without using the tool and RF1-4, four videos created using the tool. The small letters labelling the bars identify groups of videos that are not statistically significantly different according to Tukey's HSD. Multiple letters indicate that video is in two groups. Specifically, there is not a significant difference in the rating between videos P, S, RF2 and RF4, nor between video S and RF1-4 but the differences between video P and RF1 and RF3 are significant.}
    \label{fig:entertaining}
\end{figure}

\begin{figure}[t]
    \centering
    \includegraphics[width=1\linewidth]{fig/informative.png}
    \caption{Violin plots showing the distribution of ratings of videos as entertaining. Labels are the same as Figure \ref{fig:entertaining}.}
    \label{fig:informative}
\end{figure}

The main challenge in a news reel is to balance information and entertainment. 
We test how informative and entertaining news reels created in the study (using ReelFramer) were, compared to a popular professional news reel, and a news reel produced by a student at the same university but who was not in the study and was not exposed to ReelFramer. 

First, we assessed how entertaining the news reels made during the story with ReelFramer were compared to the two baseline reels.
According to a one-way ANOVA test, there was a significant difference in how entertaining the videos were ($F(5,229)=3.3$, $p=0.007$). \textbf{Half of the videos produced using the tool were rated as as entertaining as the professionally made video}. 
The non-study student's video was also as entertaining as the popular video (Figure \ref{fig:entertaining}). However, it was rated less information than 3 of 4 ReelFramer videos. 

Next, we assess how informative the news reels produced in the study are to the popular video and the non-study student reel.
A one-way ANOVA test (not assuming equal variance in the groups) shows participant ratings of the video’s informativeness differed significantly ($F(5,108.18)=7.7$, $p<0.001$). 
\textbf{Three-quarters of news reels produced in the study using ReelFramer were rated as ``as informative as'' the professionally produced video} (Figure \ref{fig:informative}). 

These results indicate that  ReelFramer reels  successfully balanced information and entertainment much of the time. 
They were competitive with the popular professional news reel, and it was better than the non-study student reel because they were more informational. 


To assess audience acceptance of the news reels, we asked subjects whether they would watch each of the videos. 
Roughly three-quarters of subjects said they would watch the professionally produced video,  56\% would watch the videos created with the tool and 45\% would watch the student video produced without the tool. 
This difference was statistically significant ($\chi^2(5)=14.6$, $p=0.01$).
Roughly half said they would share the professionally produced video,  40\% would share the study videos, and only  a quarter would the non-study student video. 
This difference was statistically significant ($\chi^2(5)=11.6$, $p=0.04$)
There was no statistically significant difference in the perceived ``fit'' of videos on participant feeds ($\chi^2(5)=2.52$, $p=0.77$), though the student-produced video had a lower fit than other videos.
Qualitative data suggests that this has to do with the subjects they prefer to watch (pets, sports, etc.) rather than quality. 
These results suggest that the videos produced with ReelFramer were closer to the professionally produced video in terms of acceptability than the non-study student video. 

After conducting audience reception interviews, researchers identified strategic and manageable production changes they suggested to student journalists. 
\textbf{Audience members indicated they would like more attention-grabbing elements early in the video and adding moving character movement within the scene}. 
Student journalists wove this feedback into their subsequent productions and asked for more from the tool to match audience expectations. 
Similar to the way a newsroom would operate, journalists had to option of bringing market research into their production, continuing to iterate on their creative process. 