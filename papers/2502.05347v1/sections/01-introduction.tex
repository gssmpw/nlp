\section{Introduction}

The integration of AI into creative processes is revolutionizing content creation, particularly in the fast-paced environment of newsrooms. 
As generative AI tools become increasingly prevalent, understanding the intricate interactions between these technologies and human creativity is crucial. 
What is the role of human creativity in the presence of AI creativity tools in the workplace? 
Does the role of human creativity change over time as people gain familiarity with AI tools?
Despite the growing adoption of AI in creative domains, these questions remain underexplored. 
Although AI has some capacity for creation, creativity is a complex cognitive process involving many skills, including convergent thinking, divergent thinking, framing, analysis, critique, audience understanding, and iteration. 
AI may be able to assist with some or parts of these processes, but human creativity is likely to still play a large role in the full creative process.

To explore the role of human creativity in the presence of AI, this paper presents a 14-week study conducted within the social media team of a university journalism department's newsroom, focusing on content creation for the TikTok platform. 
The team included five social media creators (all students), one editor (a faculty member), and two market researchers (research assistants), embodying a typical real-world newsroom structure.
They used an AI tool designed to help creators transform web articles into engaging short video scripts and storyboards. 
Using the tool, the journalists read, selected, regenerated, and edited several possible AI-generated scripts before shooting and editing the videos themselves. 
Each journalist aimed to produce one new video a week and iterate on previous videos. 
The newsroom met weekly for the editor to review videos, request changes, and decide if a video was ready to publish. 
After 10 weeks, the market researchers conducted 42 interviews with members of the target audience to assess quality  and gather suggestions for improvement. 
The study is complemented by comparative insights from a shorter case study involving two professional TV journalists.

We found that overall, creators used AI as a “creative springboard.”
Four out of five creators consistently used the tool and they appreciated its ability to reframe news stories around characters---part of the novel social media format. 
Without the tool, it was difficult to balance information and entertainment. 
However, for final scripts, they often edited “about half” of the text for accuracy and clarity.
Creators did choose to inject their own creative touches, especially in areas the tool did not support, such as adding visual elements like props or animations or using their acting ability to bring new emotions, attitudes, and depth to their characters.
Creators sometimes chose not to use the tool when they felt confident about the story or had a specific creative inspiration. 
Overall, the team was successful in publishing 16 videos in 14 weeks, and amassing over 500,000 views.

\begin{figure*}
\centering
\includegraphics[width=0.75\textwidth]{fig/rf2-ex-schools.png}
\caption{
An example of a news reel made during the newsroom study describes the three key facts of a story about a teacher shortage. 
}
\Description{
An example of a news reel made during the newsroom study describes the three key facts of a story about a teacher shortage. There are eight frames with two characters: a school board and a teachers union. thru dialog, they describe the problems the district faces with teacher shortages.
}
\label{fig:rf2-ex-schools}
\end{figure*}

Human critical evaluation was essential in defending against AI errors.
Surprisingly, creators often accepted poor AI suggestions, such as overly positive language and meaningless phrases.  
This was true of both student creators and experienced, professional TV journalists.
Although journalists are experts in journalism, they are not experts in the new skill of making videos. Here, AI is perceived as an expert. Journalists defer to AI suggestions, which are coherent even if flawed in other ways.
Additionally, AI can fail dramatically when its lack of context results in socially inappropriate suggestions.
These flaws required creators’ critical intervention either to modify these narratives or to discard them entirely to uphold social appropriateness. 

Creators' relationship with the AI changed over the course of the 14-week study.
Initially, they relied heavily on the AI's outputs, but with time, creators demonstrated a growing tendency to integrate their creativity outside the AI’s direct outputs.
When the team gathered feedback from target viewers, they found even more ways to make their content more engaging, beyond what the tool was offering.
Gradual adaptation, editorial interventions, and learning from audience feedback over time all encouraged creators to continue to use their own creativity on top of the creativity support the tool offered.
