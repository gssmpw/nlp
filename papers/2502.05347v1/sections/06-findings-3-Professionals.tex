\section{Study 3 Findings: Professional Journalists}
\label{findings_professionals}

To supplement the results of the newsroom study done with student journalists, we conducted a 4-week study with two professional TV journalists from local news stations in the US. 

\subsection{Reasons for AI Tool Usage} 
Both journalists made 3 news reels each, for a total of 6 newsreels.
They both chose to use the AI tool for all their videos. 
Similar to the newsroom study, they used the AI tool for framing and script writing, but not for storyboarding---they described framing and script writing as feeling ``very different'' than their training, and difficult to ``get started with.'' 
They also said that they had to edit ``about half'' the AI scripts for clarity, correctness, and smooth language.


\subsection{Role of Professional Human Creativity: Human Creativity Where AI Leaves off}

In four of the six videos, the journalists made notable creative additions. 
This included some of the same categories seen in the student study---adding personality to characters through acting and adding visual effects. 
However, they also introduced new elements. 
Two videos included real footage as background imagery. 
One video used an audio track. 
One video took the role-play premise to an extreme and introduced a time-lapse---where a citizen called city services over the course of 10 years to get a fallen tree removed (the city representative promised to ``get back to them'' in a robotic voice in every call). 
With professional journalism training and resources, it is possible to add even more creativity where the AI tool leaves off.

Of the six videos in the study, none experienced the major AI failures that forced students to make the news without the tool.
However, AI did create some problems.
Participant Pro 1 made a video discussing a new scam. 
The character introducing the scam was the Better Business Bureau (BBB). 
Although the typical TV news audiences know what the BBB is, Gen-Z and even Millennial audiences are less familiar with and are confused about the connection between the BBB and scams. 
AI was not aware of this gap in audience knowledge, but the editor was.
But that decision still required much discussion and consideration.


\subsection{AI Risks: Even Professionals Need to Hone Critique of AI}

\begin{table}
\centering
\begin{adjustbox}{width=0.46\textwidth}
\begin{tabular}{|l|l|}
\hline
\multicolumn{2}{|c|}{\textbf{Major Issues (6)}} \\ \hline
\textbf{AI Attributed (4)}          & \textbf{Non-AI Attributed(2)}                  \\ \hline
unnecessary dialog (2)                & slow paced (1)                  \\ 
too few facts (1)                    &  added timeline narrative unclear (1)                 \\ 
too long (1)              &      \\ 

\hline
\end{tabular}
\end{adjustbox}
\caption{Major Issues: AI vs Non-AI Attributed issues in news reels created in the professional study.}
\label{pro_major_issues}
\end{table}


\begin{table}
\centering
\begin{adjustbox}{width=0.47\textwidth}
\begin{tabular}{|l|l|}
\hline
\multicolumn{2}{|c|}{\textbf{Minor Issues (8)}} \\ \hline
\textbf{AI Attributed (5)}          & \textbf{Non-AI Attributed(3)}                  \\ \hline
confusing character names (3)                &     distracting music (2)                \\ 
awkward AI phrase (2)                    &  should not name specific unrelated brands (1)                \\ 
              &      \\ 

\hline
\end{tabular}
\end{adjustbox}
\caption{Minor Issues: AI vs Non-AI Attributed issues in news reels created in the professional study.}
\label{pro_minor_issues}
\end{table}


Despite years of journalistic practice, the professionals still made many errors. 
Four of the six major errors and five of eight minor errors were introduced by AI and not caught by professionals before filming (see Tables \ref{pro_major_issues} and \ref{pro_minor_issues}). 
Generally, they were the same types of errors seen in the student study. 
This included major errors like having unnecessary and contentless dialog (\textit{``I can't thank you enough'', ``We're so happy to help''}), having too few facts, and being too long.  
As well as minor issues, like using awkward AI phrases (\textit{``How about I come and start training my own railings?''}). 
The editor thought these should have been glaring errors to catch. 
Upon viewing the iteration of the video containing that dialog, the editor asked if ``training railings'' was a concept the professional journalist knew (perhaps a local phrase). 
The journalist said no, \textit{``The AI said it; I thought it was a thing.''} 
Despite years of journalism training---a field that prides itself on being critical---these professionals were imperfect AI critics.

\subsection{Professional Creative Role Over Time}
Similar to the student study, news reels made by professional journalists still required major edits. 
However, both professionals required no edits by their third video (see Table \ref{pro_learning}). 
In both videos needing major edits, the errors were similar to the students'---the news content was unclear due to a variety of errors, like contentless dialog, and unclear roles. 
They did have to be coached by the editor to correct AI, but once the editor pointed that out, they adapted almost immediately. 
This is likely because they had a wealth of journalism training and understanding of ``news value'', and thus faster to integrate their news training into their use of the tool. 



\begin{table}
\centering
\begin{adjustbox}{width=0.4\textwidth}
\begin{tabular}{|l|c|c|c|} 
\hline
\textbf{Participant} & \textbf{Article 1} & \textbf{Article 2} & \textbf{Article 3} \\
\hline
Pro1 & \color{red}\textbf{Major}\color{black}  & Minor & \color{teal}No edits\color{black}\\
\hline
\textit{AI errors} & 1 & 3 &  \\
\textit{Non-AI errors} & 3 & 1 &  \\
\hline
Pro2 & Minor & \color{red}\textbf{Major}\color{black}  & \color{teal}No edits\color{black} \\
\hline
\textit{AI errors} & 1 & 3 &  \\
\textit{Non-AI errors} & 1 &  &  \\
\hline
\end{tabular}
\end{adjustbox}
\caption{Classification of the amount of editing needed for news reels made by professional journalists.}
\label{pro_learning}
\end{table}

