\documentclass[oneside,11pt]{article} % For LaTeX2e



\hoffset=0in \voffset=0in \evensidemargin=0in \oddsidemargin=0in
\textwidth=6.5in \topmargin=0in \headheight=0.0in \headsep=0.0in
\textheight=9in

\synctex=1

\usepackage{amsmath,amsfonts,amssymb,amsthm,commath}


\usepackage{algorithm,algorithmic}


\usepackage[utf8]{inputenc} % allow utf-8 input
\usepackage[T1]{fontenc}    % use 8-bit T1 fonts
\usepackage{microtype}   % better spacing for pdflatex


% typeset
\usepackage{nicefrac}       % compact symbols for 1/2, etc.


\usepackage{booktabs,enumitem}


\usepackage{graphicx}


%\usepackage{bm} 
\usepackage{url}
\usepackage{xcolor}


\usepackage{hyperref} 


%\usepackage{balance} % balance reference


\usepackage{pifont}


\numberwithin{equation}{section}

\theoremstyle{plain}


\ifx\theorem\undefined
\newtheorem{theorem}{Theorem}
\fi


\ifx\lemma\undefined
\newtheorem{lemma}[theorem]{Lemma}
\fi


\ifx\proposition\undefined
\newtheorem{proposition}[theorem]{Proposition}
\fi


\ifx\corollary\undefined
\newtheorem{corollary}[theorem]{Corollary}
\fi


\theoremstyle{definition}


\ifx\remark\undefined
\newtheorem{remark}[theorem]{Remark}
\fi


\ifx\definition\undefined
\newtheorem{definition}[theorem]{Definition}
\fi


\ifx\conjecture\undefined
\newtheorem{conjecture}[theorem]{Conjecture}
\fi


\ifx\fact\undefined
\newtheorem{fact}[theorem]{Fact}
\fi


\ifx\claim\undefined
\newtheorem{claim}[theorem]{Claim}
\fi


\ifx\assumption\undefined
\newtheorem{assumption}{Assumption}
\fi

%%%%%%%%%%%%%%%%%%%%%%%%%%%
% Additional theorems



\DeclareMathOperator*{\argmin}{arg\,min}
\DeclareMathOperator*{\argmax}{arg\,max}
\DeclareMathOperator*{\minimize}{minimize}
\DeclareMathOperator*{\ve}{vec}
\newcommand{\st}{\mathrm{s.t.}}

\newcommand{\sign}{\mathrm{sign}}
\renewcommand{\Pr}{\mathrm{Pr}}
\newcommand{\EXP}{\mathbb{E}}

\newcommand{\err}{\mathrm{err}}

\newcommand{\trans}{^{\top}}
\newcommand{\inner}[2]{\langle #1, #2 \rangle}

\newcommand{\bzero}{\boldsymbol{0}}


\newcommand{\ba}{\boldsymbol{a}}
\newcommand{\bb}{\boldsymbol{b}}

\newcommand{\bS}{\mathbb{S}}
\newcommand{\R}{\mathbb{R}}
\newcommand{\Rd}{\mathbb{R}^d}
\newcommand{\Rdd}{\mathbb{R}^{d\times d}}

\newcommand{\calA}{\mathcal{A}}
\newcommand{\calC}{\mathcal{C}}
\newcommand{\calD}{\mathcal{D}}
\newcommand{\calX}{\mathcal{X}}
\newcommand{\calY}{\mathcal{Y}}
\newcommand{\calH}{\mathcal{H}}
\newcommand{\calF}{\mathcal{F}}
\newcommand{\calG}{\mathcal{G}}
\newcommand{\calQ}{\mathcal{Q}}
\newcommand{\calR}{\mathcal{R}}
\newcommand{\calO}{\mathcal{O}}
\newcommand{\calM}{\mathcal{M}}
\newcommand{\calS}{\mathcal{S}}
\newcommand{\calW}{\mathcal{W}}
\newcommand{\calU}{\mathcal{U}}
\newcommand{\PP}{\mathbb{P}}


% vector/matrix norms
\newcommand{\zeronorm}[1]{\left\lVert #1 \right\rVert_{0}}
\newcommand{\twonorm}[1]{\left\lVert #1 \right\rVert_{2}}
\newcommand{\orcnorm}[2]{\left\lVert #1 \right\rVert_{\psi^{}_{#2}}}
\newcommand{\onenorm}[1]{\left\lVert #1 \right\rVert_{1}}
%\renewcommand{\abs}[1]{\left\lvert #1 \right\rvert}
\newcommand{\twoinfnorm}[1]{\left\lVert #1 \right\rVert_{2,\infty}}
\newcommand{\maxnorm}[1]{\left\lVert #1 \right\rVert_{\max}}
\newcommand{\nuclearnorm}[1]{\left\lVert #1 \right\rVert_{*}}
\newcommand{\fronorm}[1]{\left\lVert #1 \right\rVert_{F}}
\newcommand{\spenorm}[1]{\left\lVert #1 \right\rVert}
\newcommand{\infnorm}[1]{\left\lVert #1 \right\rVert_{\infty}}

\newcommand{\Rho}{\mathrm{P}}
\newcommand{\Rpos}{\R_{\geq 0}}
\newcommand{\diag}{\mathrm{diag}}
\newcommand{\V}{\mathrm{V}}
\newcommand{\calB}{\mathcal{B}}
\newcommand{\oracle}{\mathrm{EX}(D, h^*)}
\newcommand{\ind}[1]{\mathbf{1}[#1]}
\newcommand{\tr}{\mathrm{tr}}

\newcommand{\citet}{\cite}
\newcommand{\citep}{\cite}

\title{Towards Efficient Contrastive PAC Learning}
\author{ Jie Shen\\
Stevens Institute of Technology\\
jie.shen@stevens.edu}

\date{\today}


\begin{document}

\maketitle


\section{Introduction}


\begin{figure}[t]
\centering
\includegraphics[width=0.6\columnwidth]{figures/evaluation_desiderata_V5.pdf}
\vspace{-0.5cm}
\caption{\systemName is a platform for conducting realistic evaluations of code LLMs, collecting human preferences of coding models with real users, real tasks, and in realistic environments, aimed at addressing the limitations of existing evaluations.
}
\label{fig:motivation}
\end{figure}

\begin{figure*}[t]
\centering
\includegraphics[width=\textwidth]{figures/system_design_v2.png}
\caption{We introduce \systemName, a VSCode extension to collect human preferences of code directly in a developer's IDE. \systemName enables developers to use code completions from various models. The system comprises a) the interface in the user's IDE which presents paired completions to users (left), b) a sampling strategy that picks model pairs to reduce latency (right, top), and c) a prompting scheme that allows diverse LLMs to perform code completions with high fidelity.
Users can select between the top completion (green box) using \texttt{tab} or the bottom completion (blue box) using \texttt{shift+tab}.}
\label{fig:overview}
\end{figure*}

As model capabilities improve, large language models (LLMs) are increasingly integrated into user environments and workflows.
For example, software developers code with AI in integrated developer environments (IDEs)~\citep{peng2023impact}, doctors rely on notes generated through ambient listening~\citep{oberst2024science}, and lawyers consider case evidence identified by electronic discovery systems~\citep{yang2024beyond}.
Increasing deployment of models in productivity tools demands evaluation that more closely reflects real-world circumstances~\citep{hutchinson2022evaluation, saxon2024benchmarks, kapoor2024ai}.
While newer benchmarks and live platforms incorporate human feedback to capture real-world usage, they almost exclusively focus on evaluating LLMs in chat conversations~\citep{zheng2023judging,dubois2023alpacafarm,chiang2024chatbot, kirk2024the}.
Model evaluation must move beyond chat-based interactions and into specialized user environments.



 

In this work, we focus on evaluating LLM-based coding assistants. 
Despite the popularity of these tools---millions of developers use Github Copilot~\citep{Copilot}---existing
evaluations of the coding capabilities of new models exhibit multiple limitations (Figure~\ref{fig:motivation}, bottom).
Traditional ML benchmarks evaluate LLM capabilities by measuring how well a model can complete static, interview-style coding tasks~\citep{chen2021evaluating,austin2021program,jain2024livecodebench, white2024livebench} and lack \emph{real users}. 
User studies recruit real users to evaluate the effectiveness of LLMs as coding assistants, but are often limited to simple programming tasks as opposed to \emph{real tasks}~\citep{vaithilingam2022expectation,ross2023programmer, mozannar2024realhumaneval}.
Recent efforts to collect human feedback such as Chatbot Arena~\citep{chiang2024chatbot} are still removed from a \emph{realistic environment}, resulting in users and data that deviate from typical software development processes.
We introduce \systemName to address these limitations (Figure~\ref{fig:motivation}, top), and we describe our three main contributions below.


\textbf{We deploy \systemName in-the-wild to collect human preferences on code.} 
\systemName is a Visual Studio Code extension, collecting preferences directly in a developer's IDE within their actual workflow (Figure~\ref{fig:overview}).
\systemName provides developers with code completions, akin to the type of support provided by Github Copilot~\citep{Copilot}. 
Over the past 3 months, \systemName has served over~\completions suggestions from 10 state-of-the-art LLMs, 
gathering \sampleCount~votes from \userCount~users.
To collect user preferences,
\systemName presents a novel interface that shows users paired code completions from two different LLMs, which are determined based on a sampling strategy that aims to 
mitigate latency while preserving coverage across model comparisons.
Additionally, we devise a prompting scheme that allows a diverse set of models to perform code completions with high fidelity.
See Section~\ref{sec:system} and Section~\ref{sec:deployment} for details about system design and deployment respectively.



\textbf{We construct a leaderboard of user preferences and find notable differences from existing static benchmarks and human preference leaderboards.}
In general, we observe that smaller models seem to overperform in static benchmarks compared to our leaderboard, while performance among larger models is mixed (Section~\ref{sec:leaderboard_calculation}).
We attribute these differences to the fact that \systemName is exposed to users and tasks that differ drastically from code evaluations in the past. 
Our data spans 103 programming languages and 24 natural languages as well as a variety of real-world applications and code structures, while static benchmarks tend to focus on a specific programming and natural language and task (e.g. coding competition problems).
Additionally, while all of \systemName interactions contain code contexts and the majority involve infilling tasks, a much smaller fraction of Chatbot Arena's coding tasks contain code context, with infilling tasks appearing even more rarely. 
We analyze our data in depth in Section~\ref{subsec:comparison}.



\textbf{We derive new insights into user preferences of code by analyzing \systemName's diverse and distinct data distribution.}
We compare user preferences across different stratifications of input data (e.g., common versus rare languages) and observe which affect observed preferences most (Section~\ref{sec:analysis}).
For example, while user preferences stay relatively consistent across various programming languages, they differ drastically between different task categories (e.g. frontend/backend versus algorithm design).
We also observe variations in user preference due to different features related to code structure 
(e.g., context length and completion patterns).
We open-source \systemName and release a curated subset of code contexts.
Altogether, our results highlight the necessity of model evaluation in realistic and domain-specific settings.









\newcommand{\tabincell}[2]{\begin{tabular}{@{}#1@{}}#2\end{tabular}}
\newcommand{\rowstyle}[1]{\gdef\currentrowstyle{#1}%
	#1\ignorespaces
}

\newcommand{\className}[1]{\textbf{\sf #1}}
\newcommand{\functionName}[1]{\textbf{\sf #1}}
\newcommand{\objectName}[1]{\textbf{\sf #1}}
\newcommand{\annotation}[1]{\textbf{#1}}
\newcommand{\todo}[1]{\textcolor{blue}{\textbf{[[TODO: #1]]}}}
\newcommand{\change}[1]{\textcolor{blue}{#1}}
\newcommand{\fetch}[1]{\textbf{\em #1}}
\newcommand{\phead}[1]{\vspace{1mm} \noindent {\bf #1}}
\newcommand{\wei}[1]{\textcolor{blue}{{\it [Wei says: #1]}}}
\newcommand{\peter}[1]{\textcolor{red}{{\it [Peter says: #1]}}}
\newcommand{\zhenhao}[1]{\textcolor{dkblue}{{\it [Zhenhao says: #1]}}}
\newcommand{\feng}[1]{\textcolor{magenta}{{\it [Feng says: #1]}}}
\newcommand{\jinqiu}[1]{\textcolor{red}{{\it [Jinqiu says: #1]}}}
\newcommand{\shouvick}[1]{\textcolor{violet(ryb)}{{\it [Shouvick says: #1]}}}
\newcommand{\pattern}[1]{\emph{#1}}
%\newcommand{\tool}{{{DectGUILag}}\xspace}
\newcommand{\tool}{{{GUIWatcher}}\xspace}


\newcommand{\guo}[1]{\textcolor{yellow}{{\it [Linqiang says: #1]}}}

\newcommand{\rqbox}[1]{\begin{tcolorbox}[left=4pt,right=4pt,top=4pt,bottom=4pt,colback=gray!5,colframe=gray!40!black,before skip=2pt,after skip=2pt]#1\end{tcolorbox}}


\begin{algorithm*}[t]
\caption{Verified Diversification for ambiguous queries}
\begin{algorithmic}[1]
\Require Question $q$,
LLM $\texttt{LLM}(\cdot)$
\Ensure Pairs of clarification question and answer $\hat{\mathcal{Q}}=\{(\hat{q}, \hat{y})\}$

\State $q^\prime$ $\leftarrow$ Relax $q$ for high-recall universe
\State %
$U$ $\leftarrow$ Top-$k$ retrieved passages from the retriever, using $q^\prime$ as query

\State $\mathcal{Q} \leftarrow \{\}$
\For{$i=1$ to $k$}
    \State $(\hat{q}_i, \hat{y}_i)$ $\leftarrow$ $\texttt{LLM}(q, p_i; I_{\textrm{E}})$ \Comment{Extracting interpretation from passage $p_i$ with execution feedback}
    \If {$\hat{q}_i$ is not \texttt{None}}
        \State $\mathcal{Q} \leftarrow \mathcal{Q} \cup \left\{ (\hat{q}_i, \hat{y}_i) \right\}$
    \EndIf
\EndFor

\State $\mathcal{C}$ $\leftarrow$ Cluster into partitions $\mathcal{C}_i$'s of $\mathcal{Q}$ based on embeddings $f(\hat{q}_i; \hat{y}_i)$

\State $\hat{\mathcal{Q}} \leftarrow \{\}$
\For{$j=1$ to $k$}
    \State $(\check{q}_j, \check{y}_j)$ $\leftarrow$ $\argmax_{({q}^*,y^*)\in \mathcal{C}_j} \sum_{(\hat{q},\hat{y})\in \mathcal{C}_j} \mathrm{sim}(f(\hat{q};\hat{y}), f(q^*;y^*))$ \Comment{Medoid of the $j$-th cluster $\mathcal{C}_j$}
    \State $\hat{\mathcal{Q}} \leftarrow \hat{\mathcal{Q}} \cup \left\{ (\check{q}_j, \check{y}_j) \right\}$
\EndFor

\State \textbf{return} $\hat{\mathcal{Q}}$
\end{algorithmic}
\label{alg:ours}
\end{algorithm*}




\section{Conclusion}





\bibliographystyle{alpha}
\bibliography{ref.bib}

\end{document}
