\section{Related Work}
Our work establishes a connection between advancements in GS and classical ray tracing-based rendering while concurrently addressing the limitations of rasterization-centric approaches. Here, we review the relevant literature across four primary areas.

\paragraph{Neural Radiance Fields and View-Dependent Effects}
Neural Radiance Fields (NeRFs) \cite{mildenhall2021nerf} revolutionized novel view synthesis by modeling scenes as implicit neural representations. Subsequent works improved rendering quality \cite{barron2021mip, barron2022mip, barron2023zip} and efficiency \cite{chen2022tensorf, fridovich2022plenoxels, muller2022instant}, while methods such as Ref-NeRF \cite{verbin2022ref} and SpecNeRF \cite{ma2024specnerf} enhanced view-dependent effects using reflected directions and Gaussian encoding. NeRF-Casting \cite{verbin2024nerf} introduced ray marching along reflection paths for realistic near-field illumination, but such approaches remain computationally intensive and unsuitable for real-time applications. In contrast, GS-based methods achieve real-time rendering but lack nuanced handling of reflections and shadows.

\paragraph{Gaussian Splatting and Rendering Limitations}
3D Gaussian Splatting (3DGS) \cite{kerbl20233d} was a breakthrough, representing scenes as anisot\-ropic Gaussians optimized by differentiable rasterization. Extensions such as Mip-Splatting \cite{yu2024mip} reduced aliasing, while 2DGS \cite{huang20242d} improved surface alignment by restricting the Gaussians to 2D manifolds. However, these methods rely on spherical harmonics (SH) to approximate view-dependent effects, leading to blurry reflections under complex lighting. Recent work such as GaussianShader \cite{jiang2024gaussianshader} and 3DGS-DR \cite{ye20243d} incorporated environment maps for reflections but were limited to distant lighting. 3iGS \cite{tang20253igs} introduced illumination fields via tensorial factorization, but bounded scene assumptions limit real-world applicability. Crucially, all these methods rely on rasterization, which struggles with the incoherent rays required for advanced effects like shadows and inter-reflections.

\paragraph{Ray Tracing with Gaussian Representations}
Recent efforts integrate ray tracing with GS to overcome rasterization limitations. 3D Gaussian Ray Tracing (3DGRT) \cite{moenne20243d} uses bounding primitives around Gaussians, enabling GPU-accelerated ray tracing via NVIDIA OptiX \cite{parker2010optix}. While efficient, 3DGRT approximates Gaussians as polytopes, making compatibility with flat Gaussians \cite{huang20242d} and non-Gaussian distributions difficult. EnvGS \cite{xie2024envgs} introduces environment Gaussians to model reflections but remains limited to GS-specific rendering pipelines. Similarly, IRGS \cite{gu2024irgs} proposes differentiable 2D Gaussian ray tracing for inverse rendering but requires complex Monte Carlo sampling. These methods highlight the potential of ray tracing and simultaneously inherit the structural limitations of GS, such as dependence on custom renderers and approximations that hinder generalization.

\paragraph{Mesh-Based Rendering and Hybrid Approaches}
Classical mesh representations facilitate efficient ray tracing. However, they are challenging to optimize directly from images. Our work addresses this gap by converting GS to mesh-like structures, leveraging the optimization strengths of GS while maintaining compatibility with traditional renderers such as Blender and Nvdiffrast~\cite{munkberg2022extracting}. In contrast to the polytope approximations \cite{moenne20243d}, our method preserves geometric fidelity by mapping flat Gaussians to mesh faces, thereby avoiding artifacts from bounding primitives. This approach is analogous to mesh splatting techniques \cite{weyrich2007hardware}, yet it emphasizes post-training conversion rather than direct mesh optimization.



% \paragraph{Key Differentiators}
% \begin{itemize}
% \item Unlike NeRF-based methods, we achieve real-time rendering without neural queries.
% \item Unlike GS variants relying on SH or environment maps, we enable physics-compliant ray tracing for shadows, reflections, and deformations.
% \item Unlike 3DGRT \cite{moenne20243d} and EnvGS \cite{xie2024envgs}, our mesh conversion bypasses Gaussian-specific approximations, ensuring compatibility with industry-standard renderers.
% \end{itemize}


\begin{figure}
    \centering
        \includegraphics[width=\linewidth]{img/mod_vis_grid.png}
    \caption{Examples of renderings produced by \our{} (our), with lighting conditions and mesh deformations applied using Blender or Nvdiffrast. The rendered objects exhibit various lighting effects, with the top-middle render showcasing an object with a blended texture.
    The underlying original images (not shown here) were taken from the NeRF Synthetic dataset.  % \caption{\our{} with light conditions added from rendering environment like Blender or Nvdiffrast.
    }
    \label{fig:example}
\end{figure}