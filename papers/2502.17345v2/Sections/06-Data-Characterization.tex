\color{black}
\section{Data Characterization}\label{sec:data-characterization}

We extract data from Flickr\textsuperscript 1, a real-world location-based photo-sharing application, to derive user travel histories and visiting-related statistics. The purpose of using real-world data is three-fold: (i) to show how real data can feed our model of POI information and user preferences; (ii) to characterize real-world visiting patterns; and (iii) to achieve a more realistic evaluation of the +Tour. With this goal in mind, we collect data from thirteen cities across four different continents, namely Asia (New Delhi and Osaka), Europe (Athens, Barcelona, Budapest, Edinburgh, Glasgow, London, Madrid, and Vienna), North America (Toronto), and Oceania (Perth and Melbourne). These cities are important from a tourist point of view and guarantee variety and diversity in our evaluation. Next, we first describe our methodology for data extraction. Then, we present an exploratory analysis of our dataset.

\subsection{Data extraction methodology}
\label{sec:data-1}

We use the Google Places API\footnote{\url{https://developers.google.com/places/web-service/intro}} to obtain the list of POIs of the cities. Each city identifies each POI by ID, name, latitude, and longitude, as well as the category to which it belongs. The cost $c_{i,j}$ of an edge connecting two POIs $v_i$ and $v_j$ in the same list is computed using the Google Matrix Distance API\footnote{\url{https://developers.google.com/maps/documentation/distance-matrix/start}} in the walking mode.

We use the Flickr API\footnote{\url{https://www.flickr.com/services/developer/api/}} to extract geo-tagged photos and derive user travel histories for each city, except for Melbourne whose data is provided by~\cite{MelbourneDataSet}. Each Flickr photo is tagged with the user ID of the photo owner, timestamp, and coordinates (latitude and longitude).

For each city, we combine its photo dataset with its POI list to generate the set $\mathcal{S}$ of travel histories in the city. First, we match each photo to the corresponding POI using the geo-coordinates of the photo and the POI. A photo is mapped to a POI if its geo-coordinates differ by less than 100 meters according to the Haversine formula. If this condition holds for more than one POI, the photo is associated with the nearest POI, so each photo is mapped to a single POI. Then, we construct the travel history $S_u$ for each user, sorting the photos by user ID and timestamp and grouping consecutive POI visits in the same travel sequence if the travel sequence duration does not exceed 8 hours. Then, for each POI $v_i$ in a travel sequence, we take the time of the first and the last photo taken by user $u$ at $v_i$ as the arrival ($t_{v_i}^a$) and departure ($t_{v_i}^d$) time in that POI, respectively. Finally, we discard travel sequences with cycles (repeated POI visits), with POIs with less than five visits, or with only one photo (since they result in a sequence duration of zero seconds). Table~\ref{tab:dataset} details our resulting dataset, which is also publicly available in a GitHub repository\footnote{\url{https://github.com/LABORA-INF-UFG/plusTour}}.

\begin{table*}[hbt]
\centering
\footnotesize
\caption{Dataset summary.}
\label{tab:dataset}
\begin{tabular}{ccccccccc}
\hline
\textbf{City}&\textbf{Country}&\textbf{Continent}&\textbf{\begin{tabular}[c]{@{}c@{}}Photos\end{tabular}}&\textbf{\begin{tabular}[c]{@{}c@{}}Users\end{tabular}}&\textbf{\begin{tabular}[c]{@{}c@{}}Sequences\end{tabular}}&\textbf{\begin{tabular}[c]{@{}c@{}}Valid sequences\end{tabular}}&\textbf{\begin{tabular}[c]{@{}c@{}}POIs\end{tabular}}&\textbf{\begin{tabular}[c]{@{}c@{}}POI categories\end{tabular}}\\\hline
Athens&Greece&Europe&5026&291&516&509&24&5\\
Barcelona&Spain&Europe&13654&782&1654&1634&29&8\\
Budapest&Hungary&Europe&8090&573&2361&1058&36&6\\
Edinburgh&Scotland&Europe&20620&879&5028&2405&25&6\\
Glasgow&Scotland&Europe&8172&367&2227&969&25&7\\
London&England&Europe&48519&2198&5018&4949&30&10\\
Madrid&Spain&Europe&20881&577&1396&1382&30&8\\
Melbourne&Australia&Oceania&15255&626&5106&2096&84&9\\
New Delhi&India&Asia&2961&186&489&278&19&6\\
Osaka&Japan&Asia&5306&279&1115&540&23&4\\
Perth&Australia&Oceania&2503&99&716&307&19&7\\
Toronto&Canada&North America&30267&846&6057&2721&29&6\\
Vienna&Italy&Europe&22432&704&3193&1613&28&8\\\hline
\textbf{13 cities}&\textbf{10 countries}&\textbf{4 continents}&\textbf{203686}&\textbf{8407}&\textbf{34876}&\textbf{20461}&\textbf{401}&\textbf{20 (unique)}\\\hline
\end{tabular}
\end{table*}

\subsection{Data analysis}
\label{sec:data-2}
Using our dataset, we compute the visiting-related statistics, i.e., POI popularity, expected POI visiting time, and user's interests in a given category for each city. Next, we analyze our dataset's most relevant visiting-related statistics and patterns. Due to space limitations, we present results only for the five cities with the most distinguished patterns, namely London, Melbourne, Osaka, Perth, and Toronto. The complete data analysis is available at the GitHub repository\textsuperscript 6.

Figure~\ref{fig:POIs-Popularity} shows the Probability Density Function (PDF) (blue bars) and the Cumulative Distribution Function (CDF) (red line) of the POI Popularity for each analyzed city. POI popularity exhibits similar patterns among most cities, with more than 50\%  of the visits being concentrated in usually 5 POIs, while most POIs are rarely visited. POI popularity is slightly different in London and Melbourne, where a single POI (the Trafalgar Square in London and the City Square in Melbourne) holds the majority of the visits.

\begin{figure*}[!ht]
\centering
    \begin{tabular}{@{}ccccc@{}}
        \includegraphics[width=.18\textwidth]{Figures/Data-Characterization/POIs/Popularity/London.pdf} &
        \includegraphics[width=.18\textwidth]{Figures/Data-Characterization/POIs/Popularity/Melbourne.pdf} &
        \includegraphics[width=.18\textwidth]{Figures/Data-Characterization/POIs/Popularity/Osaka.pdf} &
        \includegraphics[width=.18\textwidth]{Figures/Data-Characterization/POIs/Popularity/Perth.pdf} &
        \includegraphics[width=.18\textwidth]{Figures/Data-Characterization/POIs/Popularity/Toronto.pdf}   \\
    \end{tabular}
    \caption{POI popularity Distribution. PDF is presented in blue, and CDF is shown in red. Values have been normalized so that the popularity of the most popular POI equals 1.}
    \label{fig:POIs-Popularity}
\end{figure*}

Figure~\ref{fig:POIs-Popularity-And-Number-Of-Photos} shows the relation between POI popularity (blue bars) and the number of photos taken in the POI (red bar). POIs are sorted by popularity, and the top 3 POIs in the number of photos are highlighted with a star. Usually, the top 3 POIs in the number of photos appear among the top 10 POIs in popularity, showing that popularity does not always correlate to the number of photos. Perth presents a greater imbalance between popularity and number of photos per POI.

\begin{figure*}[!ht]
\centering
    \begin{tabular}{@{}ccccc@{}}
        \includegraphics[width=.18\textwidth]{Figures/Data-Characterization/POIs/Popularity-And-Number-Of-Photos/London.pdf} &
        \includegraphics[width=.18\textwidth]{Figures/Data-Characterization/POIs/Popularity-And-Number-Of-Photos/Melbourne.pdf} &
        \includegraphics[width=.18\textwidth]{Figures/Data-Characterization/POIs/Popularity-And-Number-Of-Photos/Osaka.pdf} &
        \includegraphics[width=.18\textwidth]{Figures/Data-Characterization/POIs/Popularity-And-Number-Of-Photos/Perth.pdf} &
        \includegraphics[width=.18\textwidth]{Figures/Data-Characterization/POIs/Popularity-And-Number-Of-Photos/Toronto.pdf}   \\
    \end{tabular}
    \caption{POI Popularity (in blue) and number of photos taken in each POI (in red). POI popularity is given in the number of POI visits. The top 3 POIs in the number of photos are marked with stars.}
    \label{fig:POIs-Popularity-And-Number-Of-Photos}
\end{figure*}

Figure~\ref{fig:POIs-Visiting-Time} shows the CDF of the expected (average) POI visiting time. In all analyzed cities, the expected POI visiting time is short, with 50\% lasting less than 50 minutes and 90\% lasting less than 1 hour. Melbourne presents the distribution with the shortest expected POI visiting time, with 50\% lasting less than 30 minutes. Figure~\ref{fig:POIs-Visiting-Time-Clustering} shows a complementary analysis of the CDF of the expected POI visiting time, but here, the CDF is presented considering all POIs in all cities in our dataset. The figure shows three POI groups: \textit{Quick-visited} (QV, $\leq$ 30 minutes, in red), \textit{Normal-visited} (NV, $>$ 30 minutes and $\leq$ 60 minutes, in green), and \textit{Long-visited} (LV, $>$ 60 minutes, in blue).

\begin{figure*}[t]
\centering
    \begin{tabular}{@{}ccccc@{}}
        \includegraphics[width=.18\textwidth]{Figures/Data-Characterization/POIs/Visiting-Time/London.pdf} &
        \includegraphics[width=.18\textwidth]{Figures/Data-Characterization/POIs/Visiting-Time/Melbourne.pdf} &
        \includegraphics[width=.18\textwidth]{Figures/Data-Characterization/POIs/Visiting-Time/Osaka.pdf} &
        \includegraphics[width=.18\textwidth]{Figures/Data-Characterization/POIs/Visiting-Time/Perth.pdf} &
        \includegraphics[width=.18\textwidth]{Figures/Data-Characterization/POIs/Visiting-Time/Toronto.pdf}   \\
    \end{tabular}
    \caption{CDF of the expected POI visiting time.}
    \label{fig:POIs-Visiting-Time}
\end{figure*}

\begin{figure}[!ht]
\begin{center}
 	\includegraphics[width=0.65\linewidth]{Figures/Data-Characterization/POIs/Visiting-Time-Clustering/Plot.pdf}
        \caption{Expected POI visiting time clustering.}
        \label{fig:POIs-Visiting-Time-Clustering}
\end{center}
\end{figure}

The result shows that more than 80\% of the POIs are either in the \textit{Quick-visited} group or the \textit{Normal-visited} group, and few POIs go beyond the 1-hour visiting time, confirming the result presented in Figure~\ref{fig:POIs-Visiting-Time}. Figure~\ref{fig:POIs-Number-Vs-Popularity} analyzes the POI popularity (in terms of number of visits) in each group illustrated in Figure~\ref{fig:POIs-Visiting-Time-Clustering}. The \textit{Quick-visited} group contains 168 POIs and a total of 10848 visits; the group \textit{Normal-visited} contains 159 POIs and 12306 visits; and the \textit{Long-visited} group is composed of the remaining 74 POIs with 5516 visits. This result shows that the most popular POIs do not present the highest expected visiting times.

\begin{figure}[!ht]
\begin{center}
 	\includegraphics[width=0.5\textwidth]{Figures/Data-Characterization/POIs/Number-Vs-Popularity/Plot.pdf}
        \caption{Number of POIs vs. POI Popularity.}
        \label{fig:POIs-Number-Vs-Popularity}
\end{center}
\end{figure}

The last visiting-related statistic analyzed in our dataset is the distribution of the user's interest in the POI categories, illustrated in Figure~\ref{fig:Tourists-Interest}. The figure shows that the distribution is quite different in the cities. In London and Perth, the user's interest is concentrated in a few categories (e.g., Entertainment, Museum, and Shopping in London and Entertainment and Amusement in Perth). In other cities, it is more evenly distributed. Melbourne is the city where users have visited the most different POI categories.

\begin{figure}[!ht]
\begin{center}
 	\includegraphics[width=0.5\textwidth]{Figures/Data-Characterization/Tourists/Interest/Plot.pdf}
 	\caption{Distribution of the tourists' interest in the POI categories.}
 	\label{fig:Tourists-Interest}
\end{center}
\end{figure}

\begin{figure*}[t]
\centering
    \begin{tabular}{@{}ccccc@{}}
        \includegraphics[width=.18\textwidth]{Figures/Data-Characterization/Sequences/Duration/London.pdf} &
        \includegraphics[width=.18\textwidth]{Figures/Data-Characterization/Sequences/Duration/Melbourne.pdf} &
        \includegraphics[width=.18\textwidth]{Figures/Data-Characterization/Sequences/Duration/Osaka.pdf} &
        \includegraphics[width=.18\textwidth]{Figures/Data-Characterization/Sequences/Duration/Perth.pdf} &
        \includegraphics[width=.18\textwidth]{Figures/Data-Characterization/Sequences/Duration/Toronto.pdf}   \\
    \end{tabular}
    \caption{CDF of the travel sequence duration.}
    \label{fig:Sequences-Duration}
\end{figure*}

\begin{figure*}[!ht]
\centering
    \begin{tabular}{@{}ccccc@{}}
        \includegraphics[width=.18\textwidth]{Figures/Data-Characterization/Sequences/Length/London.pdf} &
        \includegraphics[width=.18\textwidth]{Figures/Data-Characterization/Sequences/Length/Melbourne.pdf} &
        \includegraphics[width=.18\textwidth]{Figures/Data-Characterization/Sequences/Length/Osaka.pdf} &
        \includegraphics[width=.18\textwidth]{Figures/Data-Characterization/Sequences/Length/Perth.pdf} &
        \includegraphics[width=.18\textwidth]{Figures/Data-Characterization/Sequences/Length/Toronto.pdf}   \\
    \end{tabular}
    \caption{CDF of the travel sequence length.}
   \label{fig:Sequences-Length}
\end{figure*}

To better understand the visiting patterns in our dataset, we analyze the user travel sequences, i.e., $S_u \subset S, \ \forall \ u \in \mathcal{U}$. Figure~\ref{fig:Sequences-Duration} shows the CDF of the travel sequence duration, computed as the time difference between the last photo of the last POI and the first photo of the first POI. The duration of the travel sequences is also short, with 50\% lasting less than 50 minutes in all cities except for Melbourne. Indeed, the travel sequence duration throughout the cities presents a similar behavior, with most sequences having a duration shorter than 2 hours and few travel sequences reaching the 8-hour duration. 

We also analyze the travel sequence length, i.e., the number of POIs composing a travel sequence. Figure~\ref{fig:Sequences-Length} shows the CDF of the travel sequence length. In all analyzed cities, 80\% of the travel sequences are composed of two or fewer POI, evidencing that, in the Flickr dataset, most of the travel sequences are composed of few visits. Indeed, in London, almost 90\% of the travel sequences have a single POI visit. Melbourne stands out for having few sequences with 14 POIs, showing some users with a high exploratory behavior. On the other hand, no tourist in Osaka visited more than 5 POIs in a day.