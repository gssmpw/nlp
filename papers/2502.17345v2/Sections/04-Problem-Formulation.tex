\section{Problem Formulation}\label{sec:problem-formulation}

Given the graph $G=(\mathcal{V},\mathcal{E})$ representing the POIs of the city, the set of users willing to visit the city in a specific day ($\mathcal{U}_d=\{u \in \mathcal{U} \mid d_u=d\}$), the users' preferences ($int_{u}(c), \ \forall \ c \in \mathcal{C}, \ \forall \ u \in \mathcal{U}_d$), the users' time constraints ($b_u, \ \forall \ u \in \mathcal{U}_d$), and the applications the users will consume during the tour ($\mathcal{A}_{u} \in 2^{\mathcal{A}}, \ \forall \ u \in \mathcal{U}_d$), our objective is to find a single itinerary $I_{u*}=(v_0,\dots,v_{n+1})$, starting at $v_0$ and ending at $v_{n+1}$, for each user $u \in \mathcal{U}_d$ so that the set of chosen itineraries, denoted by $\mathcal{I_{*}}$, maximizes the sum of the profits perceived by all the users while prioritizing the MEC resource allocation and satisfying the users and the infrastructure constraints. We named such a problem the MEC-aware Personalized Tour Itinerary Recommendation (MEC-PTIR) problem. Indeed, the MEC-PTIR problem adds the availability of ICT resources and the demand for the chosen applications to the classical PTIR problem. Therefore, the MEC-PTIR problem is more complex than the PTIR one. Since the PTIR problem is NP-hard~\cite{gavalas-survey:14}, we formulate the MEC-PTIR problem as a two-stage optimization problem to address the complexity efficiently.

\subsection{MEC-PTIR problem - First stage}
\label{sec:problem-formulation-first-stage}
We define the first stage of the MEC-PTIR problem as a multi-objective orienteering problem and its integer problem formulation. Formally, for each itinerary $I_u=(v_0, \dots, v_{n+1})$, $u \in \mathcal{U}_d$, let a decision variable $x_{i,j} \in \{0, 1\}$ indicates whether POI $v_i \in \mathcal{V}$ and POI $v_j \in \mathcal{V}$ are visited in sequence in $I_u$ or not. 
The objective of the first stage of the MEC-PTIR problem is to obtain, for each $u \in \mathcal{U}_d$, the set $\mathcal{I}_{u*} = \{I_{u*}^{1}, \dots, I_{u*}^{k}\}$ of itineraries such that the returned physical profit is maximized while the travel cost is minimized, i.e.:\\
\\
\textbf{Stage 1 (MEC-PTIR problem):} $\forall u \in \mathcal{U}_d$
\begin{align}
& {\mathrm{maximize}} \ \sum_{i=0}^{n} \sum_{j=1}^{n+1} x_{i,j}Prof_{u}(v_i)\label{c:profit}&\\
& {\mathrm{minimize}} \ \sum_{i=0}^{n} \sum_{j=1}^{n+1} x_{i,j}Cost_{u}(v_i,v_j) \label{c:cost}&
\end{align}
\begin{align}
& \text{subject to:} \nonumber&\\
& 0 < \sum_{i=0}^{n} \sum_{j=1}^{n+1} Cost_{u}(v_i,v_j)x_{i,j} \leq b_u,\label{c:budget}&\\
&\sum_{j=1}^{n+1} x_{0,j} = \sum_{i=0}^{n} x_{i,n+1} = 1,&\label{c:itinerary}\\
&\sum_{i=0}^{n} x_{i,r} = \sum_{j=1}^{n+1} x_{r,j} \leq 1,  \forall \ r = 1, ..., n, &\label{c:conn_no-rep}\\
& 2 \leq pos(v_i) \leq n + 1, \forall \ i = 1, ..., n+1, &\label{c:no_subtour1}\\
& pos(v_i) - pos(v_j) + 1 \leq n(1 - x_{i,j}), \nonumber& \\
& \qquad\qquad\qquad\qquad\qquad \  \forall \ i,j = 1, ..., n+1.\label{c:no_subtour2}
\end{align}

Equation (\ref{c:profit}) aims to maximize the total collected profit of user $u$ while Equation (\ref{c:cost}) minimizes the travel cost adhering to the user's budget (Constraint~(\ref{c:budget})). The other constraints ensure the proper construction of the itinerary. Constraint~(\ref{c:itinerary}) ensures that the path starts at POI $v_0$ and ends at POI $v_{n+1}$; Constraint~(\ref{c:conn_no-rep}) ensures that the path is connected and no POI is visited more than once. Assuming that $pos(v_i)$ is the position of POI $v_i \in \mathcal{V}$ in itinerary $I_u$, Constraints~(\ref{c:no_subtour1}) and (\ref{c:no_subtour2}) ensure that there are no sub-tours.

For a user $u \in \mathcal{U}_d$, the set $\mathcal{I}_{u*}$ contains all Pareto-efficient itineraries that start at vertex $v_0$, end at vertex $v_{n+1}$, and do not violate the user's time constraint. A solution $I_{u*}^{k} \in \mathcal{I}_{u*}$ is called a Pareto-efficient (or non-dominated) if there is no feasible solution $I_{u*}^{k'}$ that dominates $I_{u*}^{k}$, i.e., that is at least equally as good as $I_{u*}^{k}$ concerning all objective functions, and better than $I_{u*}^{k}$ concerning at least one objective function. The set $\mathcal{I}_{u*}$ is called the Pareto front.  

\subsection{MEC-PTIR problem - Second stage}
\label{sec:problem-formulation-second-stage}
The result of the first stage of the MEC-PTIR problem is the set of Pareto fronts $\mathcal{I} = \bigcup_{u \in \mathcal{U}_d} \mathcal{I}_{u*}$, i.e., the set of all non-dominated itineraries that do not violate the user's time constraint for all users in $\mathcal{U}_d$. Each itinerary $I_{u*}^k \in \mathcal{I}_{u*}$, $\forall \ \mathcal{I}_{u*} \subset \mathcal{I}$, has a physical profit given by:
\begin{equation}
\label{eq:profit}
Prof(I_{u*}^k) = \sum_{v_i \in I_{u*}^k \setminus \{v_{0}, v_{n+1}\}} Prof_{u}(v_i).
\end{equation}

Given the set $\mathcal{I}$, the objective of the second stage of the MEC-PTIR is to find a set $\mathcal{I_{*}}$ that maximizes the sum of the profits perceived by the users while prioritizing the resource allocation at the network edge. In addition, $\mathcal{I_{*}}$ must contain exactly one itinerary for each user $u \in \mathcal{U}_d$.
To achieve this goal, we formulate the second stage of the MEC-PTIR as a Mixed Integer Linear Programming problem.

Let the indicator function $\phi(I_{u*}^k , v_i) \in \{0,1\}$ represents POI visit in an itinerary, with $\phi(I_{u*}^k , v_i)=$~1 if itinerary $I_{u*}^k \in \mathcal{I}$ visits POI $v_{i} \in \mathcal{V} \setminus \{v_{0}, v_{n+1}\}$, and $\phi(I^{j}_{u*}, v_i)=$~0 otherwise. 

Assume that every itinerary in $\mathcal{I}$ starts and ends within a period of time $T$ and that the computing and network resources for these itineraries are allocated in discrete time slots of size $\Delta t$. The time slots are indexed by $\tau \in \mathbb{Z}$ such that $ 1 \leq \tau \leq T$. Given an itinerary $I_{u*}^k \in \mathcal{I}_{u*}$, a POI $v_{i} \in \mathcal{V} \setminus \{v_{0}, v_{n+1}\}$, and a time index $\tau$, we define the indicator function $\rho(I_{u*}^k, v_{i}, \tau) \in \{0, 1\}$, with $\rho(I_{u*}^k, v_{i}, \tau)=$~1 if itinerary $I_{u*}^k$ is visiting POI $v_{i}$ during $\tau$, and $\rho(I_{u*}^k, v_{i}, \tau)=$~0 otherwise. 
%This information can be derived from the start time, sequence of POI visits, and POI visit duration for each itinerary $I_{u*}^k \in \mathcal{I}_{u*}$ obtained from the first stage. 

Let the set of decision variables $y(I_{u*}^k) \in \{0, 1\}$ represents itinerary choices so that $y(I_{u*}^k)=$~1 if itinerary $I_{u*}^k \in \mathcal{I}_{u*}$ composes the solution, and $y(I_{u*}^k)=$~0 otherwise. A MEC host should provide computing resources for each POI visit for each user. We define the decision variable $z(I_{u*}^k,v_{i}, m) \in \{0, 1\}$ for representing MEC host association so that $ z(I_{u*}^k,v_{i},m)=$~1 if MEC host $m \in \mathcal{M}$ is responsible for providing computing resources during a visit to POI $v_{i} \in \mathcal{V} \setminus \{v_{0}, v_{n+1}\}$ in itinerary $I_{u*}^k \in \mathcal{I}_{u*}$, and $z(I_{u*}^k,v_{i},m)=$~0 otherwise. 

Let the decision variables $p(I_{u*}^k,v_{i}) \in \mathbb{R}$ and $q(I_{u*}^k,v_{i},m) \in \mathbb{R}$ represent, respectively, the amount of network and the amount of computing resources (at MEC host $m \in \mathcal{M}$) allocated during a visit to POI $v_{i} \in \mathcal{V} \setminus \{v_{0}, v_{n+1}\}$ in itinerary $I_{u*}^k \in \mathcal{I}_{u*}$. Assuming $Norm(value)$ as a generic function that normalizes a value, we define the objective function of the second stage of the MEC-PTIR problem as:\\
\\
\textbf{Stage 2 (MEC-PTIR):}
\begin{equation} \label{eq:stage2_summary}
\begin{split}
& \mathrm{maximize} \sum\limits_{\mathcal{I}_{u*} \subset \mathcal{I}}\sum\limits_{I_{u*}^k \in \mathcal{I}_{u*}}      y(I_{u*}^k)Norm(Prof(I_{u*}^k))\ +&\\
& \sum\limits_{\mathcal{I}_{u*} \subset \mathcal{I}}\sum\limits_{I_{u*}^k \in \mathcal{I}_{u*}}\sum\limits_{v_i \in \mathcal{V} \setminus \{v_{0}, v_{n+1}\}} \frac{Norm(p(I_{u*}^k,v_{i}))}{2|I_{u*}^k \setminus \{v_{0}, v_{n+1}\}|} \ +&\\
& \sum\limits_{\mathcal{I}_{u*} \subset \mathcal{I}}\sum\limits_{I_{u*}^k \in \mathcal{I}_{u*}}\sum\limits_{v_i \in \mathcal{V} \setminus \{v_{0}, v_{n+1}\}}\sum\limits_{m \in \mathcal{M}} \frac{Norm(q(I_{u*}^k,v_{i},m))}{2|I_{u*}^k \setminus \{v_{0}, v_{n+1}\}|}. &
\end{split}
\end{equation}

Equation~(\ref{eq:stage2_summary}) aims to maximize simultaneously two objectives: (i) the aggregated total collected physical profit of all users in $\mathcal{U}_d$; and (ii) the aggregated total collected virtual profit of all users in $\mathcal{U}_d$ given by the sum of the amount of allocated network resources and the amount of allocated computing resources at the network edge. Together, these objectives ensure that the generated set of itineraries maximizes the sum of the physical profits perceived by the users while prioritizing the resource allocation at the network edge. The $Norm(value)$ function ensures that both objectives stay in the same interval of values. The aggregated physical profit and the aggregated virtual profit have equal weights. Thus, Equation (\ref{eq:stage2_summary}) may choose itineraries that do not have the highest individual physical profit but provide the best balance with the resource allocation, resulting in an improved experience for the set of users as a whole. The objective function represented by Equation~(\ref{eq:stage2_summary}) is subject to the following constraints.

\textbf{Itinerary choice constraints} -- For each user $u \in \mathcal{U}_d$, exactly one itinerary $I_{u*}^k \in \mathcal{I}_{u*}$ must be {selected}, i.e.:
\begin{equation}
\label{eq:one_itinerary_1}
\sum\limits_{I_{u*}^k \in \mathcal{I}_{u*}} y(I_{u*}^k)=1, {\quad \forall \ \mathcal{I}_{u*} \subset \mathcal{I}}.
\end{equation}
%\begin{equation}
%\label{eq:one_itinerary_2}
%y(I_{u*}^k) \in \{0, 1\},
%{\quad \forall I_{u*}^k \in \mathcal{I}_{u*}}, {\quad \forall \mathcal{I}_{u*} \subset \mathcal{I}}.
%\end{equation}

If an itinerary $I_{u*}^k \in \mathcal{I}_{u*}$ is {selected} to compose a solution for a user $u \in \mathcal{U}_d$, we also need to allocate one MEC host to each POI visited in the itinerary, i.e.:
\begin{equation}
\label{eq:mec_itinerary_1}
\begin{split}
& \sum\limits_{m \in \mathcal{M}} z(I_{u*}^k,v_{i},m)=y(I_{u*}^k) \phi(I_{u*}^k, v_{i}), \\
& {\forall \ I_{u*}^k \in \mathcal{I}_{u*}}, {\quad \forall \ \mathcal{I}_{u*} \subset \mathcal{I}}, {\quad \forall \ v_{i} \in \mathcal{V} \setminus \{v_{0}, v_{n+1}\}}.
\end{split}
\end{equation}
%\begin{equation}
%\label{eq:mec_itinerary_2}
%\begin{split}
%z(I_{u*}^k,v_{i},m) \in \{0,1\}, {\quad \forall I_{u*}^k \in \mathcal{I}_{u*}}, {\quad \forall \mathcal{I}_{u*} \subset \mathcal{I}}, {\quad \forall v_{i} \in \mathcal{V}}.
%\end{split}
%\end{equation}  

\textbf{Service demand constraints} -- For each user $u \in \mathcal{U}_d$ and {selected} itinerary $I_{u*}^k \in \mathcal{I}_{u*}$, we also need to select at least the minimum of resources required to run the applications {selected} by user $u$, avoiding allocating more than is needed, i.e.:
\begin{equation}
\label{eq:service_demand_net}
\begin{split}
& p(I_{u*}^k,v_{i}) \geq \lambda_{u}^{min}y(I_{u*}^k)\phi(I_{u*}^k, v_{i}) \ \land \\
& p(I_{u*}^k,v_{i}) \leq \lambda_{u}^{max}y(I_{u*}^k)\phi(I_{u*}^k, v_{i}), \\
& {\forall \ I_{u*}^k \in \mathcal{I}_{u*}}, {\quad \forall \ \mathcal{I}_{u*} \subset \mathcal{I}}, {\quad \forall \ v_{i} \in \mathcal{V} \setminus \{v_{0}, v_{n+1}\}}.
\end{split}
\end{equation}  

Similarly, we ensure proper allocation of computing resources at the network edge with the following constraint:
\begin{equation}
\label{eq:service_demand_mec}
\begin{split}
& q(I_{u*}^k,v_{i},m) \geq \psi_{u}^{min}z(I_{u*}^k,v_{i},m)\phi(I_{u*}^k, v_{i}) \ \land \\
& q(I_{u*}^k,v_{i},m) \leq \psi_{u}^{max}z(I_{u*}^k,v_{i},m)\phi(I_{u*}^k, v_{i}), \\
& {\forall I_{u*}^k \in \mathcal{I}_{u*}}, {\ \forall \mathcal{I}_{u*} \subset \mathcal{I}}, {\ \forall v_{i} \in \mathcal{V} \setminus \{v_{0}, v_{n+1}\}}, {\ \forall m \in \mathcal{M}}.
\end{split}
\end{equation}

\textbf{Resource capacity constraints} -- Finally, we assure that the performed allocations do not exceed the amount of available network and computing resources, at any given time slot:
\begin{equation}
\label{eq:bs_capacity}
\begin{split}
& \sum\limits_{I_{u*}^k \in \mathcal{I}_{u*}} \rho(I_{u*}^k, v_{i}, \tau)p(I_{u*}^k,v_{i}){\leq \lambda_{v_{i}},} \\
& \qquad {\forall \ v_{i} \in \mathcal{V} \setminus \{v_{0}, v_{n+1}\},} {\quad 1 \leq \tau \leq T}.
\end{split}
\end{equation}
\begin{equation}
\label{eq:mec_capacity}
\begin{split}
&\sum\limits_{I_{u*}^k \in \mathcal{I}_{u*}} \sum\limits_{v_{i} \in \mathcal{V} \setminus \{v_{0}, v_{n+1}\}} \rho(I_{u*}^k, v_{i}, \tau) q(I_{u*}^k,v_{i},m){\leq {\psi}_{m},}\\
& \qquad\quad {\forall \ m \in \mathcal{M},} {\quad 1 \leq \tau \leq T}.
\end{split}
\end{equation}

We describe our proposed solution in the next section.
%To solve the MEC-PTIR problem, we propose the next generation of Tourist services assisted by modern mobile networks (+Tour) algorithm, which proceeds in two {stages}. In the first {stage}, +Tour solves the first {part} of the MEC-PTIR problem using an approach based on Dynamic Programming for the Shortest Path Problem with Resource Constraints (SPPRC)~\cite{irnich-shortest:05}. In the second {stage}, +Tour solves the second stage of the MEC-PTIR problem. This solution can be obtained straightforwardly from a linear optimization solver {such as Gurobi, IBM ILOG CPLEX, or SCIP~\cite{meindl2012analysis}.}