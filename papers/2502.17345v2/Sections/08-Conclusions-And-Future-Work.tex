\section{Conclusions and Future Work}\label{sec:conclusions}

%In this work, we have accomplished four tasks. First, we formulated an optimization problem that jointly approaches the design of personalized tour itinerary recommendations and resource allocation at the network edge. To unveil the completeness of real-world visiting patterns, our problem formulation allows itineraries with one or more POIs. Second, we proposed +Tour, a low-complexity algorithm that efficiently solves the problem. Third, we presented an exploratory real-world dataset analysis to understand preferences and users’ visiting patterns. Based on the findings, We proposed a methodology to identify user interest in applications. Fourth, using the collected dataset, we evaluate the effectiveness of +Tour in different scenarios and compare it with a modified state-of-the-art PTIR algorithm.
{In this work, we propose +Tour, a two-stage solution to the resource-constrained itinerary recommendation problem. Our approach simultaneously maximizes profit and resource allocation while minimizing travel costs. We provide an extensive data characterization based on real-world POI visitation history from 13 cities spread across 4 continents. This dataset includes 20461 valid tour sequences generated by 8407 users who visited 401 POIs divided across 20 unique categories, which allows us to evaluate +Tour in realistic size scenarios. We also compare our solution to the state-of-the-art RA-PersTour, demonstrating that our approach achieves up to 74.2\% improvement in Allocation Efficiency and 65.1\% in User Experience while maintaining competitive performance in traditional metrics such as Recall, Precision, and F-Score. +Tour induces cooperative behavior in the tourists, which may benefit the users and the touristic service provider. In practice, we are aware that the tourists may not follow the recommendations, degrading the estimated performance. However, this issue negatively affects any recommendation system.} 

{In future work, we intend to use graph neural networks (GNN) to solve the problem of jointly designing personalized tour itinerary recommendations and resource allocation.
Among the methods for modeling preferences based on historical data, the most promising ones are machine learning techniques, especially those that consider the problem's structural characteristics. Most data in recommendation systems naturally have a graph structure. GNN takes advantage of this to learn the representation of a given POI based on all POIs visited previously and subsequently in a tour. For this reason, applying GNN has proven beneficial in recommendation problems to allow the extraction of user trends based on their preferences.}

{We also intend to broaden the scope to include other types of immersive applications with stringent requirements anticipated with the advancements of 5G and beyond, such as the metaverse. Additionally, we plan to evaluate the impact of non-cooperative users, i.e., tourists who do not follow the recommendations, in combination with other real-world aspects such as the need for waiting in lines at busy POIs and the availability of different types of transportation with different costs. A promising approach to be explored in the context of non-cooperative users is the Probabilistic Orienteering Problem~\cite{angelelli-probabilistic:17}, a variant of the OP Problem where tourists may visit a POI according to a certain probability. Finally, once developing a GNN version of the MEC-PTIR problem as well as the version that accounts for non-cooperative users, we intend to expand our evaluation using more baseline algorithms.} 