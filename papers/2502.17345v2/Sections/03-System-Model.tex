\section{System Model}\label{sec:system-model}

Our tour recommendation problem takes place in a specific city with a set of POIs, the set of applications offered by the MTC to be consumed during the tour, the ICT infrastructure to support the tourist virtual experience, and a set of tourists. In the next subsections, we describe the system model{, which is depicted by Figure~\ref{fig:System-Model},} in detail. Table~\ref{tab:notation} summarizes the notations and definitions used in this work.

\begin{figure}[t]
    \centering
    \includegraphics[width=0.5\textwidth]{Figures/System-Model.pdf}
    \caption{{System model.}}
    \label{fig:System-Model}
\end{figure}

\begin{table}[htb]
\centering
\footnotesize

{\caption[skip=0pt]{Notations and definitions used in the system model, problem formulation, and solution.}}
\label{tab:notation}
\begin{tabular}{rlp{60mm}}
    \hline
    & \textbf{Symbol} & \textbf{Description}  \\
    \hline
    {\multirow{10}{*}{\rotatebox[origin=r]{90}{Sets}}}
    & $\mathcal{V}$ & Set of POIs (in a city) \\
    & $\mathcal{E}$ & Set of edges connecting vertices\\
    & $\mathcal{C}$ & Set of all POI categories \\
    & $\mathcal{U}$ & Set of users (tourists) \\
    & $\mathcal{U}_d$ & Set of users whose tours are scheduled to the day $d$\\
    & $\mathcal{M}$ & Set of MEC hosts \\
    & $\mathcal{A}$ & Set of applications offered by the MTC \\
    & $\mathcal{A}_{u}$ & Set of applications user $u$ is willing to use \\
    & $\mathcal{S}$ & Set of all travel histories \\
    & $S_u$ & Travel history of user $u$ \\
    \hline
    {\multirow{6}{*}{\rotatebox[origin=r]{90}{Items}}}
    & $n$ & Number of POIs in a city \\
    & $v_0$ & Virtual POI where every itinerary begins \\
    & $v_{n+1}$ & Virtual POI where every itinerary ends \\
    & $v_i$ & A POI in $\mathcal{V}$\\
    & $u$ & A user in $\mathcal{U}$ \\
   & $\tau$ & Index of the time slot where MEC infrastructure resources are allocated \\
    \hline
    {\multirow{8}{*}{\rotatebox[origin=r]{90}{POI/MEC input}}}
    & $c_{i, j}$ & Travel time between POI $v_{i}$ to POI $v_{j}$ \\
    & $pop(v_{i})$ & Popularity of POI $v_i$ \\
    & $dur(v_{i})$ & Expected visiting time of POI $v_i$ \\
    & $cat(v_{i})$ & Category of POI $v_i$ \\
    & $lat(v_{i})$ & Latitude of POI $v_i$ \\
    & $long(v_{i})$ & Longitude of POI $v_i$ \\
    & $\lambda({v_{i}})$ & Total network resource available at POI $v_{i}$ \\
    & ${\psi}_{m}$ & Total computing resource available at MEC $m$ \\
    \hline
    {\multirow{7}{*}{\rotatebox[origin=r]{90}{Application input}}}
    & $\lambda_{a}^{min}$ & Minimum network demand, in bps, of application $a \in \mathcal{A}$ \\
    & $\lambda_{a}^{max}$ & Maximum network demand, in bps, of application $a \in \mathcal{A}$ \\
    & $\psi_{a}^{min}$ & Minimum computing demand, in reference cores, of application $a \in \mathcal{A}$ \\
    & $\psi_{a}^{max}$ & Maximum computing demand, in reference cores, of application $a \in \mathcal{A}$ \\
    \hline
    {\multirow{9}{*}{\rotatebox[origin=r]{90}{User input}}}
    & $int_u(c)$ & Interest of user $u$ in the POI category $c$ \\
    & $b_u$ & Time budget of user $u$ to complete the tour\\
    & $d_u$ & Date when the tour of user $u$ will take place \\
    & $\lambda_{u}^{min}$ & Minimum network demand of user $u$ during a tour \\
    & $\lambda_{u}^{max}$ & Maximum network demand of user $u$ during a tour \\
    & $\psi_{u}^{min}$ & Minimum computing demand of user $u$ during a tour \\ 
    & $\psi_{u}^{max}$ & Maximum computing demand of user $u$ during a tour \\
\hline
\end{tabular} 
\end{table}

\subsection{Main elements of the model}
\label{sec:model-1}

\subsubsection{\textbf{POIs}}
For a city with $n$ POIs, we consider a complete non-oriented graph $G = (\mathcal{V},\mathcal{E})$, with $\mathcal{V} = \{v_{0}, v_{n+1}\} \cup \{v_{1}, \dots, v_{n}\}$ being the set of vertices representing the POIs and $\mathcal{E} = \{(v_{i},v_{j}) \mid v_{i}, v_{j} \in \mathcal{V}\}$ is the set of edges connecting the nodes. $v_{0} \in \mathcal{V}$ and $v_{n+1} \in \mathcal{V}$ are virtual POIs representing, respectively, the starting and ending location of the users' tours. Each edge $(v_{i},v_{j}) \in \mathcal{E}$ is associated with a cost $c_{i,j}$ representing the travel time between vertex $v_{i}$ to vertex $v_{j}$, employing a given mode of transportation. Since $v_{0}$ and $v_{n+1}$ are virtual POIs, we model $c_{0,j}=c_{i,n+1}=0$ $\forall \ i=0,\dots,n$ and $\forall \ j=1,\dots,n+1$. 

Each POI $v_{i} \in \mathcal{V} \setminus \{v_{0}, v_{n+1}\}$ is characterized by the following attributes: the popularity of the POI, denoted by $pop(v_{i}) \in \mathbb{Z}$; the expected time one should spend in the POI to enjoy what it has to offer (i.e., expected POI visiting time), denoted by $dur(v_{i}) \in \mathbb{R}$; the category representing its nature, denoted by $cat(v_{i})$; and the location expressed in terms of latitude and longitude, denoted by $lat(v_{i})$ and $long(v_{i})$ respectively. We denote by $\mathcal{C} = \{1, \dots, C$\} {as} the set of all POI categories so that $cat(v_{i}) \in \mathcal{C}, \ \forall \ v_{i} \in \mathcal{V} \setminus \{v_{0}, v_{n+1}\}$.

\subsubsection{{\textbf{Service Applications}}}

The MTC offers a set $\mathcal{A} = \{1, \dots, A\}$ of applications to enrich the touristic experience in every POI. Each application $a \in \mathcal{A}$ has specific requirements in terms of network and computing resources, expressed as:

\begin{itemize}
    \item $\lambda^{min}_{a}$ and $\lambda^{max}_{a}$: the minimum and maximum network demand (i.e., demanded traffic volume) measured in bps, respectively;
    \item $\psi^{min}_{a}$ and $\psi^{max}_{a}$: the minimum and maximum computing demand (i.e., demanded processing load) measured in reference core (RC)\footnote{Unit of measure that represents the processing capacity of a reference CPU core.}, respectively.
\end{itemize}

\subsubsection{{\textbf{Advanced mobile networks ICT Infrastructure}}}

In each POI $v_{i} \in \mathcal{V} \setminus \{v_{0}, v_{n+1}\}$, the network resources are provided through a set of {wireless base stations, e.g., gNBs (gNodeBs) in a 5G network}. To keep the model generic, we decide to represent network resources simply by the capacity, i.e., not considering the wireless channel aspects. The whole model's complexity and the time scale of the problem also suggest that this {simplification} is a reasonable approach. We denote by $\lambda({v_{i}})$ the total network resource available at POI $v_{i} \in \mathcal{V} \setminus \{v_{0}, v_{n+1}\}$. 

The computing resources are provided by MEC hosts that are reachable from the access network, as shown in Figure~\ref{fig:ICT-Infrastructure}. We assume that a POI $v_{i} \in \mathcal{V} \setminus \{v_{0}, v_{n+1}\}$ can be served by any MEC host. Let $\mathcal{M} = \{1, \dots, M\}$ denote the set of MEC hosts. We represent by ${\psi}_{m}$ the total computing capacity available at MEC host $m \in \mathcal{M}$.

Applications should run on MEC hosts whenever possible since this results in shorter response times and improved virtual experience. When no resource is available on the edge, applications run in the remote cloud with degraded performance regarding communication delay. {In traditional offload literature, the user device can also run the applications completely. We assume this is not the case in our context because the MTC does not want to deal with issues such as regular updates in the software of the user devices, hardware incompatibility, and security breaches, among others. These issues can be minimized or solved by keeping the client software minimalist while the rest of the application runs remotely on the edge or in the cloud. We argue that this approach is reasonable for an enterprise scenario such as the one of the MTC.} 

\subsubsection{\textbf{Tourists}}
We consider a set $\mathcal{U} = \{1, \dots, U\}$ of users (tourists) where each user $u \in \mathcal{U}$ is described by the following attributes: the user's interest in each POI category $c \in \mathcal{C}$, denoted by $int_{u}(c) \in \mathbb{R}$; the preferred day for the tour, denoted by $d_u$; and the time budget to complete the tour, i.e., the maximum available time for visits,  represented by $b_u \in \mathbb{R}$. The user is also characterized by the set $\mathcal{A}_{u} \in 2^{\mathcal{A}}$, representing the applications she is willing to use during the tour, where $2^{\mathcal{A}}$ is the power set of $\mathcal{A}$. The set $\mathcal{A}_{u}$ determines the minimum and maximum demand for network and computing resources required to run the user applications during the tour. We represent this demand as described in the following:
\begin{itemize}
    \item $\lambda_{u}^{min} = \sum\limits_{a \in \mathcal{A}_{u}} \lambda^{min}_{a}$: the minimum network demand, in bps, required by user $u$ during a tour;
    \item $\lambda_{u}^{max} = \sum\limits_{a \in \mathcal{A}_{u}} \lambda^{max}_{a}$: the maximum network demand, in bps, required by user $u$ during a tour;
    \item $\psi_{u}^{min} = \sum\limits_{a \in \mathcal{A}_{u}} \psi^{min}_{a}$: the minimum computing demand, in reference core (RC), required by user $u$ during a tour; 
    \item $\psi_{u}^{max} = \sum\limits_{a \in \mathcal{A}_{u}} \psi^{max}_{a}$: the maximum computing demand, in reference core (RC), required by user $u$ during a tour.
\end{itemize}

Finally, by modeling the starting and ending location of all users' tour itineraries as virtual POIs, we allow an itinerary to involve one or more real-world POIs.

%Previous work on PTIR algorithms~\cite{lim-personalized:18, fonseca-personlized:19} model the starting location and the ending location of the user's tour as two different POIs in the city. As a consequence of such modeling, a tour itinerary must contain at least two POI visits. However, this is not inline with many real-world visiting scenarios, such as parks or museums, where visits last longer and, after visiting such places, the tourists may want to return to their accommodations. We observed this behaviour in our datasets. Although representative, these visiting scenarios with one POI visit are often neglected by previous works. To avoid this limitation, in this work, we model the starting and ending location of all users' tour itineraries as, respectively, the virtual POIs $v_{0} \in \mathcal{V}$ and $v_{n+1} \in \mathcal{V}$, allowing an itinerary to be a path involving one or more real-world POIs.

\subsection{Visiting-related {elements of the Model}}
\label{sec:model-2}
Similar to~\cite{choudhury-automatic:10,brilhante-where:13,lim-personalized:18}, we define POI popularity, expected POI visiting time, and user's interest based on users' past travel histories extracted from LBSNs. Given the user $u \in \mathcal{U}$ who has visited $r$ POIs in a city, we define her travel history as an ordered sequence $S_u= ((v_1, t_{v_1}^a, t_{v_1}^d), \dots, (v_r, t_{v_r}^a, t_{v_r}^d))$, where each triple $(v_l, t_{v_l}^a, t_{v_l}^d),$ $\ l=1,\dots, r$, represents a visit at POI $v_l$, the arrival time $t_{v_l}^a$ at POI $v_l$, and the departure time $t_{v_l}^d$ from POI $v_l$, $v_{l} \in \mathcal{V} \setminus \{v_{0}, v_{n+1}\}$. The user's visit duration at a POI $v_{l} \in \mathcal{V} \setminus \{v_{0}, v_{n+1}\}$ is computed as $t_{v_l}^d - t_{v_l}^a$.

Given a set $\mathcal{S}$ of all travel histories in a city, i.e., $\mathcal{S}=\bigcup_{u \in \mathcal{U}} S_u$, the popularity of a POI $v_{i} \in \mathcal{V} \setminus \{v_{0}, v_{n+1}\}$ is defined based on the number of times $v_i$ has been visited:%, i.e.: 
\begin{equation}
\label{eq:pop}
pop(v_{i}) = \sum\limits_{S_u \in \mathcal{S}}\sum\limits_{v_{x} \in S_u} \delta(v_{x},v_{i}),
\end{equation}
where $\delta(v_{x},v_{i})=$~1 if $v_{x}=v_{i}$ or $\delta(v_{x},v_{i})=$~0 otherwise.

The expected (average) visiting time of a POI $v_i \in \mathcal{V} \setminus \{v_{0}, v_{n+1}\}$ is defined as follows:
\begin{equation}
\label{eq:dur}
dur(v_{i}) = \frac{\sum\limits_{S_u \in \mathcal{S}} \sum\limits_{v_{x} \in S_u} (t_{v_x}^d - t_{v_x}^a)\delta(v_{x},v_{i})}{pop(v_{i})},
\end{equation}

Similar to~\cite{lim-personalized:18}, we assume a time-based user interest approach where the user's interest in a certain category $c$ is based on the time she spent at a POI of that category relative to the expected visiting time at that POI. The intuition is to determine the user's interest level in a certain category by computing the time she spent at POIs of that category compared to other users. Thus, we define the interest level of a user $u \in \mathcal{U}$ in a category $c \in \mathcal{C}$ as:
\begin{equation}
\label{eq:intu}
     int_u(c) = \sum\limits_{v_{x} \in S_u} \frac{(t_{v_x}^d - t_{v_x}^a)}{dur(v_{x})} \gamma(cat(v_x),c), \\
\end{equation}
where $\gamma(cat(v_{x}),c)=$~1 if $cat(v_x)=c$, or $\gamma(cat(v_{x}),c)=$~0 if $cat(v_x) \neq c$.

\subsection{{Cost element associated with a POI visit of the Model}}
\label{sec:model-3}

The capability of computing the user's interest based on the time-based approach using past travel histories renders an effective solution to personalize the recommended time for a tourist to spend at a certain POI. Formally, given a POI $v_{i} \in \mathcal{V} \setminus \{v_{0}, v_{n+1}\}$ of category $cat(v_i) \in \mathcal{C}$, we can recommend user $u$ to spend the following time at POI $v_{i}$: $int_u(cat(v_i))dur(v_{i})$. 
This time is based on the user's interest level in category $cat(v_i)$ multiplied by the average time spent at POI $v_i$. The rationale for this equation is that the greater the user's interest in the category $cat(v_i)$, the more time she will spend on POI $v_i$ compared to the average user. 

We define the profit perceived by user $u$ when physically visiting a POI $v_i \in \mathcal{V} \setminus \{v_{0}, v_{n+1}\}$ as: 
\begin{equation}
\label{eq:prof_vi}
Prof_{u}(v_i) = \alpha int_u(cat(v_i)) + (1 - \alpha)pop(v_{i}),
\end{equation}

Equation (\ref{eq:prof_vi}) depends on the POI popularity and the user's interest in the POI category. It also depends on the $\alpha \in [0,1]$ value, which can be chosen to emphasize the user's interest or the POI popularity. We also define $Prof_{u}(v_0)=Prof_{u}(v_{n+1})=$~0.

Traveling from POI $v_i$ to POI $v_j$ and visiting $v_j$ consumes the user's budget. Thus, we can define a cost function associated with a POI visit as:
\begin{equation}
\label{eq:cost_vi}
\begin{split}
& Cost_{u}(v_i,v_j) = c_{i,j} + int_u(cat(v_j))dur(v_{j}), \\ 
& {i=0,\dots, n}; {\quad j=1,\dots,n}.
\end{split}
\end{equation}

We also define $Cost_{u}(v_i,v_{n+1})=$~0, $\forall \ i=0,\dots, n$.