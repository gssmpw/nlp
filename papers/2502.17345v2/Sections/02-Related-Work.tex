\section{Related Work}\label{sec:related-work}

\begin{figure*}[t]
\centering
\includegraphics[width=0.8\textwidth]{Figures/Tourist-Related-Work-Tree.pdf}
\caption{{Tree of classification for tour-related research. Adapted from~\cite{lim-tour:18}. Yellow boxes indicate the topic of interest.}}
\label{fig:classification}
\end{figure*}

{As illustrated in Figure~\ref{fig:classification}, the literature related to tour itinerary generation is broadly divided into two categories: Tourist Trip Design Problem (TTDP) and Data-driven Recommendation~\cite{lim-tour:18}. Next, we discuss relevant works in both categories.}

{\subsection{TTDP}
Indeed, the problem of recommending personalized tour itineraries has its roots in the Operational Research community as the Tourist Trip Design Problem. The authors in~\cite{vansteenwegen-mobile:07} defined the generic TTDP as an extension of the Orienteering Problem (OP)~\cite{vansteenwegen-orienteering:11}, where the objective is to schedule an optimal path (i.e., the path that maximizes the collected profit) for the user, given the set of candidate POIs, the travel time among them (e.g., identified by the transport mode), the profit of each POI (e.g., identified by the POI popularity), the visiting duration at a POI, starting/ending points for the tour, and the user time budget. The TTDP can be further specialized depending on the number of users involved (i.e., a single traveler or group of tourists) and the number of routes to be generated (i.e., one-day or multiple-day tours). A comprehensive survey on using OP to model multiple variants of the TTDP was presented in~\cite{gavalas-survey:14}. An important characteristic of works on TTDP is the lack of individual preferences associated with individual users. Consequently, every user providing the same starting/ending points and time budget as input will be recommended for the same tour itinerary.}

{\subsection{Data-driven Recommendation} 
With the popularity of smartphones and LBSNs, several works have focused on data-driven approaches to better model user preferences when recommending POIs or itineraries. The primary goal of these works is to exploit the tourism content available on the Web to enable a deeper understanding of user preferences and to issue personalized recommendations. As illustrated in Figure~\ref{fig:classification}, the literature on data-driven recommendation can be divided into three groups: next POI recommendation, top-k POI recommendation, and personalized tour itinerary recommendation (PTIR). Works on next POI recommendation~\cite{chen-comprehensive:17,liao-multi-context:18,fan-deep:18,sassi-location:19} aims to identify the next location a tourist will likely visit based on her previous trajectories, while studies on top-k location~\cite{logesh:19,zhou-adversarial:19,rahmani-joint:20} recommend multiple POIs as part of a ranked list. Different from those works but closely related, works on PTIR recommend multiple POIs as a connected path, taking the same considerations as the TTDP but with the additional challenge of planning an itinerary that appeals to the user's interest. Next, we detail existing works on PTIR and differentiate them from our proposal.}

{\subsubsection{Existing PTIR solutions.}} 
{Aiming to recommend personalized tour itineraries for a single tourist, the authors in~\cite{choudhury-automatic:10} are one of the first to combine an optimization solution for an OP variant with the mining of users' past trajectories based on geo-tagged photos extracted from Flickr\footnote{https://www.flickr.com/}. Their approach maximizes POI popularity while keeping the user's total time budget. To achieve this goal, transit time between POI and POI visiting time duration are computed as the median and the seventy-fifth percentile of all users, respectively.}

{The work of~\cite{choudhury-automatic:10}  was further refined in~\cite{brilhante-where:13,yu-personalized:16,lim-personalized:18} by assigning categories to POIs and using them to determine the user interests. Particularly, in~\cite{brilhante-where:13} and~\cite{yu-personalized:16}, the user interest is computed based on the frequency the user visits a POI of a particular category. This approach is usually referenced in the literature as frequency-based user interest. In~\cite{lim-personalized:18}, the authors presented PersTour, which advanced the state-of-the-art by introducing the concept of time-based user interest. In this concept, the user’s level of interest in a POI category is based on her time spent at such POIs relative to the average user. PersTour also innovated by personalizing POI visit duration using this time-based user interest. The authors showed that time-based user interest and personalized visit durations reflect users' real-life tours more accurately than frequency-based tourist interest solutions and average visit duration. Thus, PersTour has been recognized as the state-of-the-art one-day, single-traveler PTIR exact solution by many works (e.g.,~\cite{zhang-encoder:24,halder-survey:24}).}  

{Different from~\cite{choudhury-automatic:10,brilhante-where:13,yu-personalized:16,lim-personalized:18} that focus on recommending a one-day tour, in~\cite{friggstad:18} and~\cite{zhong-optimization:23}, the focus is on the multiple-day case. Particularly, the authors in~\cite{friggstad:18} extended the OP and proposed a PTIR algorithm for the multiple-day case where the objective is to optimize for the value of the worst tour. User interest is mined from a private Google dataset containing historical visits to tourist attractions; the duration of a POI visit is set to the median time spent at the POI among all visits; and the transportation mode allows multiple options, using the quickest available one. The authors in~\cite{zhong-optimization:23} addressed the challenge of recommending personalized multi-day tours by considering the time windows of each POI and the different transportation modes to visit them. The proposed problem is modeled as a multi-day TTDP with time windows using data from Chongqing, a popular city in Southwest China.} 

{The PTIR problem was also studied in the context of a group of tourists. In this setting, the objective is to recommend a one-day (\cite{lim-towards:16,yin-group:19}) or multiple-day(\cite{kargar-socially:21}) tour itinerary considering the group members’ heterogeneous preferences. Next, we discuss the novelty of our work.}

{\subsubsection{+Tour positioning.}} 
{Our work focuses on an optimization approach for recommending personalized itineraries for an individual user considering a one-day tour. Hence, it is conceptually closer to the investigations performed in~\cite{choudhury-automatic:10,brilhante-where:13, yu-personalized:16,lim-personalized:18}. However, +Tour differs distinctly from existing PTIR works by considering not only POI popularity and POI category as user preferences but also new applications that will be available with advanced mobile networks. In particular, we consider the user's physical and virtual visit experience. As illustrated in Figure~\ref{fig:classification}, this distinct view requires additional considerations on ICT resource allocation and service application demands. In addition, as ICT resources are shared among tourists during their tours, to issue personalized itineraries for each user, our PTIR solution must take into account a multi-user perspective. Thus, we propose a new formulation to the PTIR problem that captures the joint problem of recommending personalized tour itineraries for multiple individual users while efficiently allocating MEC resources to enable advanced user service applications. Table~\ref{tab:related-work} summarizes the main characteristics of the existing PTIR works for a single traveler and how our work differs from them.}

\begin{table*}[t]
\centering

\footnotesize
\caption{Comparison of our work with relevant literature on PTIR for a single traveler. Time constraints and POI popularity are omitted from the table since they are considered in all PTIR works.}
\label{tab:related-work}
\begin{tabular}{ccccccc}\toprule
& &\multicolumn{5}{c}{\textbf{Considerations}} \\
\cmidrule(lr){3-7}
\textbf{Paper} & \textbf{Dataset} &
\textbf{Routing} & 
\textbf{Transport} & \textbf{User} & 
\textbf{ICT} &  \textbf{Application} 
\\
             &               &
             & 
\textbf{Mode} & \textbf{Preferences} & 
\textbf{Resources}  &  \textbf{Demands} \\
\midrule
\cite{choudhury-automatic:10} & Flickr &
one-day & 
- & POI popularity & 
- & - \\ \hline
\cite{brilhante-where:13} & Flickr & 
one-day & 
walking & POI category (frequency-based) & 
- & - \\ \hline
\cite{yu-personalized:16} & Jie Pang city & 
one-day & 
walking & POI category (frequency-based) & 
- & - \\ \hline
\cite{lim-personalized:18} & Flickr & 
one-day & 
walking & POI category (time-based) & 
- & - \\ \hline
\cite{friggstad:18} & Google's & 
multiple-day & 
multiple modes & POI popularity & 
- & - \\ \hline
\cite{zhong-optimization:23} & Chongqing city & 
multiple-day & 
multiple modes & POI popularity & 
- & - \\ \hline
our work & Flickr & 
one-day & 
walking & POI category (time-based) & 
$\checkmark$ & $\checkmark$ \\
  & & & & and applications & & \\\bottomrule
%\cite{friggstad:18} & x & x & x & x & x & x & x & x  \\ \hline
%\cite{zhong-optimization:23} & x & x & x & x & x & x & x & x  \\ \hline
\end{tabular}
\end{table*}

{In~\cite{fonseca-personlized:19}, we introduced n5GTour, a preliminary version of this work presented in a conference paper. The present work extends~\cite{fonseca-personlized:19} in many ways. First, as with many existing PTIR solutions, n5GTour assumes that the recommended itinerary contains a minimum of three POIs. +Tour extends n5GTour by relaxing this constraint, recommending itineraries with one or more POIs. As we will show in Section~\ref{sec:data-characterization}, this relaxation is reasonable since a significant number of itineraries in the real world include less than three POIs, and the literature usually neglects the visiting patterns of these itineraries. Second, different from~\cite{fonseca-personlized:19}, we present a complete description of +Tour and analyze its computational complexity. Third, we challenge our solution with a more prominent and representative dataset of 13 cities distributed among 4 continents. In contrast, in~\cite{fonseca-personlized:19}, only 4 cities in the same continent were considered. Fourth, we add a data characterization to extract the user visiting patterns of the new dataset. Supported by our data characterization, we propose a more realistic methodology to infer the applications more likely to be used by tourists based on their visit durations. In~\cite{fonseca-personlized:19}, the applications used by tourists are randomly chosen. Finally, we comprehensively evaluate our solution considering multiple cities and different resource utilization scenarios.}














%The Tourist Trip Design Problem (TTDP) is a classic subject in Operations Research. It consists of building a tour itinerary to visit various POIs without exceeding restrictions associated with time, budget, transportation, and users' preferences~{\cite{meza-systematic:22}}. In the literature, the TTDP has been modeled under the context of the Orienteering Problem (OP)~{\cite{vansteenwegen-orienteering:11}}, having also been specialized according to the number of tourists involved (i.e., a single traveler or group of tourists) and the number of routes to be generated (i.e., one-day tour or multiple-days tour)~{\cite{gavalas-survey:14}}.

%Recently, the TTDP has been revisited in the context of mobile cellular networks and LBSNs to design PTIR algorithms~\cite{lim-tour:18}. One of the major concerns in this setting is how to exploit historical tourism information, such as visited POIs, to accurately infer user preferences (interests), making recommendations more personalized. Aiming to recommend personalized tour itineraries for a single tourist, the authors in~\cite{choudhury-automatic:10} are one of the first to combine an optimization solution for an OP variant with the mining of users' past trajectories based on geo-tagged photos extracted from Flickr\footnote{https://www.flickr.com/}. This combination was further refined in~\cite{brilhante-where:13,yu-personalized:16,lim-personalized:18} by assigning categories to POIs and using them to determine the user interests. {Particularly}, in~\cite{brilhante-where:13} and~\cite{yu-personalized:16}, the user interest is computed based on the frequency the user visits a POI of a particular category. {In contrast}, in~\cite{lim-personalized:18}, the authors proposed PersTour, a PTIR algorithm where the user interest is calculated as the ratio of the time the user spends at the POIs of a specific category to the average visit duration of all users in the category. The authors show that their time-based approach reflects the real-life tours of users more accurately than the frequency-based tourist interest solutions.

%Differently from~\cite{choudhury-automatic:10,brilhante-where:13,yu-personalized:16,lim-personalized:18} that focus on recommending a one-day tour, in~\cite{friggstad:18}, the authors extended the OP and proposed a PTIR algorithm for the multiple-days case. They also differ in their work by mining the user interest from a private dataset of Google containing historical visits to tourist attractions.

%Machine Learning has also been applied to infer users' preferences for the design of PTIR algorithms. In~\cite{chen-persolnalized:20}, a supervised deep learning model is employed to integrate POI textual contents (e.g., the POI's description on Wikipedia), the historical user visits, and the POI categories to predict the user interests and visit durations. The authors then formulate the itinerary recommendation following the PersTour formulation. In~\cite{chen-trip:23}, a graph-based representation method is used to learn POI popularity, user preferences, and POI-POI transition probability. A Q-learning-based algorithm is proposed to solve the PTIR problem for a single tourist based on the learned patterns.

%\textbf{+Tour positioning:} Our work focuses on an optimization approach for recommending personalized tour itineraries for a one-day tour. Hence, it is conceptually closer to the investigations performed in~\cite{choudhury-automatic:10,brilhante-where:13,lim-personalized:18}. However, +Tour is distinctly different from all previous works regarding how we approach user satisfaction in Smart Tourism and advanced mobile networks. In particular, we consider not only the physical visit experience but also the virtual satisfaction. Such a distinct view, in turn, requires the interplay between the PTIR algorithms and the resource allocation at the network edge.

%This paper extends our previous work~\cite{fonseca-personlized:19} in many ways. First, we note that the existing solutions in the literature, including our previous paper, assume that the recommended itinerary contains a minimum of three POIs. However, as we will show in Section~\ref{sec:data-characterization}, a significant amount of itineraries in the real world include less than three POIs, and the visiting patterns of these itineraries are just neglected by the existing literature. Thus, to unveil the completeness of real-world visiting patterns, we define a PTIR problem where itineraries can contain one or more POIs. Second, we challenged our solution with a more prominent and representative dataset of 13 cities distributed among 4 continents. In contrast, in our previous work, only 4 cities in the same continent were considered. Third, a data characterization was added to extract the user visiting patterns of the new dataset. Supported by our data characterization, we propose a more realistic methodology to identify applications based on user interest. Finally, we provide an extended evaluation of our solution considering multiple cities and different resource utilization scenarios.