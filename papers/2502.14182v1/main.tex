%%%%%%%% ICML 2025 EXAMPLE LATEX SUBMISSION FILE %%%%%%%%%%%%%%%%%

\documentclass{article}

% Recommended, but optional, packages for figures and better typesetting:
\usepackage{microtype}
\usepackage{graphicx}
\usepackage{subfigure}
\usepackage{booktabs} % for professional tables
\usepackage{multirow}


\usepackage{balance}
\balance

% hyperref makes hyperlinks in the resulting PDF.
% If your build breaks (sometimes temporarily if a hyperlink spans a page)
% please comment out the following usepackage line and replace
% \usepackage{icml2025} with \usepackage[nohyperref]{icml2025} above.
\usepackage{hyperref}


% Attempt to make hyperref and algorithmic work together better:
\newcommand{\theHalgorithm}{\arabic{algorithm}}

\newcommand{\pf}[1]{{\color{orange}[pf:#1]}}
\newcommand{\jt}[1]{{\color{red}[jt:#1]}}

\newcommand{\han}[1]{{\color{blue}[han:#1]}}

\newcommand{\yue}[1]{{\color{purple}[yue:#1]}}

\newcommand{\zhen}[1]{{\color{pink}[zhen: #1]}}


% Use the following line for the initial blind version submitted for review:
% \usepackage{icml2025}

% If accepted, instead use the following line for the camera-ready submission:
% \usepackage[accepted]{icml2025}
\usepackage[accepted]{icml2025}

% For theorems and such
\usepackage{amsmath}
\usepackage{amssymb}
\usepackage{mathtools}
\usepackage{amsthm}

% if you use cleveref..
\usepackage[capitalize,noabbrev]{cleveref}

%%%%%%%%%%%%%%%%%%%%%%%%%%%%%%%%
% THEOREMS
%%%%%%%%%%%%%%%%%%%%%%%%%%%%%%%%
\theoremstyle{plain}
\newtheorem{theorem}{Theorem}[section]
\newtheorem{proposition}[theorem]{Proposition}
\newtheorem{lemma}[theorem]{Lemma}
\newtheorem{corollary}[theorem]{Corollary}
\theoremstyle{definition}
\newtheorem{definition}[theorem]{Definition}
\newtheorem{assumption}[theorem]{Assumption}
\theoremstyle{remark}
\newtheorem{remark}[theorem]{Remark}

% Todonotes is useful during development; simply uncomment the next line
%    and comment out the line below the next line to turn off comments
%\usepackage[disable,textsize=tiny]{todonotes}
\usepackage[textsize=tiny]{todonotes}


% The \icmltitle you define below is probably too long as a header.
% Therefore, a short form for the running title is supplied here:
\icmltitlerunning{Data poisoning in the era of LLMs}

\begin{document}

\twocolumn[
\icmltitle{Multi-Faceted Studies on Data Poisoning can Advance LLM Development
}
% It is OKAY to include author information, even for blind
% submissions: the style file will automatically remove it for you
% unless you've provided the [accepted] option to the icml2025
% package.

% List of affiliations: The first argument should be a (short)
% identifier you will use later to specify author affiliations
% Academic affiliations should list Department, University, City, Region, Country
% Industry affiliations should list Company, City, Region, Country

% You can specify symbols, otherwise they are numbered in order.
% Ideally, you should not use this facility. Affiliations will be numbered
% in order of appearance and this is the preferred way.
% \icmlsetsymbol{equal}{*}
\begin{icmlauthorlist}
\icmlauthor{Pengfei He}{msucse}
\icmlauthor{Yue Xing}{msustt}
\icmlauthor{Han Xu}{ua}
\icmlauthor{Zhen Xiang}{uga}
\icmlauthor{Jiliang Tang}{msucse}
%\icmlauthor{}{sch}
%\icmlauthor{}{sch}
\end{icmlauthorlist}

\icmlaffiliation{msucse}{Department of Computer Science and Engineering, Michigan State University,USA}
\icmlaffiliation{msustt}{Department of Probability and Statistics, Michigan State University,USA}
\icmlaffiliation{ua}{Department of Electrical and Computer Engineering, University of Arizona, USA}
\icmlaffiliation{uga}{School of Computing, University of Georgia, USA}
\icmlcorrespondingauthor{Pengfei He}{hepengf1@msu.edu}
% You may provide any keywords that you
% find helpful for describing your paper; these are used to populate
% the "keywords" metadata in the PDF but will not be shown in the document
\icmlkeywords{Machine Learning, ICML}
\vskip 0.3in
]

% this must go after the closing bracket ] following \twocolumn[ ...

% This command actually creates the footnote in the first column
% listing the affiliations and the copyright notice.
% The command takes one argument, which is text to display at the start of the footnote.
% The \icmlEqualContribution command is standard text for equal contribution.
% Remove it (just {}) if you do not need this facility.

%\printAffiliationsAndNotice{}  % leave blank if no need to mention equal contribution
\printAffiliationsAndNotice{\icmlEqualContribution} % otherwise use the standard text.

\begin{abstract}
% \pf{To check}
The lifecycle of large language models (LLMs) is far more complex than that of traditional machine learning models, involving multiple training stages, diverse data sources, and varied inference methods. While prior research on data poisoning attacks has primarily focused on the safety vulnerabilities of LLMs, these attacks face significant challenges in practice. Secure data collection, rigorous data cleaning, and the multistage nature of LLM training make it difficult to inject poisoned data or reliably influence LLM behavior as intended.
Given these challenges, this position paper proposes rethinking the role of data poisoning and argue that \textbf{multi-faceted studies on data poisoning can advance LLM development}. From a threat perspective, practical strategies for data poisoning attacks can help evaluate and address real safety risks to LLMs. From a trustworthiness perspective, data poisoning can be leveraged to build more robust LLMs by uncovering and mitigating hidden biases, harmful outputs, and hallucinations. Moreover, from a mechanism perspective, data poisoning can provide valuable insights into LLMs, particularly the interplay between data and model behavior, driving a deeper understanding of their underlying mechanisms.
\end{abstract}

\section{Introduction}
\label{sec:intro}

\begin{figure*}[tb]
    \centering
    \includegraphics[width=0.848\linewidth]{figs/circuitnn.pdf} 
    \caption{Illustration of differentiable CircuitNN. CircuitNN is designed based on differentiable NAND gates. After DAS is guided by PI and PO pairs of the truth table, CircuitNN can get the precise circuit architecture logic equivalent to the truth table.}
    \label{fig:circuitnn}
\end{figure*}

% 1. Describe the importance of logic synthesis
% 2. Existing Problems
% (a) Neural Architecture Search: Unstable, Predefined Setting, etc.
% (b) Circuit Generation: Probabilistic Model, Logic Equivalence

With the rapid advancement of technology, the scale of integrated circuits (ICs) has expanded exponentially. 
This expansion has introduced significant challenges in chip manufacturing, particularly concerning power and area metrics.
A primary objective in IC design is achieving the same circuit function with fewer transistors, thereby reducing power usage and area occupancy.

Logic synthesis~\cite{hachtel2005logicsynth}, a critical step in electronic design automation (EDA), transforms behavioral-level circuit designs into optimized gate-level circuits, ultimately yielding the final IC layout. 
The primary goal of logic synthesis is to identify the physical implementation with the fewest gates for a given circuit function. 
This task constitutes a challenging NP-hard combinatorial optimization problem. 
Current logic synthesis tools~\cite{brayton2010abc, wolf2013yosys} rely on human-designed heuristics, often leading to sub-optimal outcomes.

Differentiable architecture search (DAS) techniques~\cite{liu2018darts, chu2020darts} offer novel perspectives on addressing challenges in this problem.
Circuit functions can be represented through truth tables, which map binary inputs to their corresponding outputs. 
Truth tables provide a precise representation of input-output relationships, ensuring the design of functionally equivalent circuits.
Inspired by this, researchers~\cite{deepmind2024ai4sys, wang2024tnet} have begun exploring the application of DAS to synthesize circuits directly from truth tables.
Specifically, \citet{deepmind2024ai4sys} proposed CircuitNN, a framework that learns differentiable connection structures with logic gates, enabling the automatic generation of logic circuits from truth tables.
This approach significantly reduces the complexity of traditional circuit generation. 
Building on this, \citet{wang2024tnet} introduced T-Net, a triangle-shaped variant of CircuitNN, incorporating regularization techniques to enhance the efficiency of DAS.

Despite these advancements, several challenges remain. 
The computational complexity of DAS grows quadratically with the number of gates, posing scalability issues.
Although triangle-shaped architecture~\cite{wang2024tnet} partially mitigates this problem, redundancy persists. 
%Additionally, DAS is susceptible to converging to local optima, limiting the ability to search architectures that satisfy the given truth tables~\cite{liu2018darts}. 
%Furthermore, hyperparameters (network depth and layer width) require extensive searches, introducing complexity and prolonging the synthesis process. 
Additionally, DAS is susceptible to converging to local optima~\cite{liu2018darts} and hyperparameters (network depth and layer width) require extensive searches. 
The challenges arise from the vast search space in DAS. 
% Even with predefined settings for CircuitNN, finding a configuration that meets the truth table requires extensive trial and error during the DAS process. 
Intuitively, limiting the search space through predefined parameters (network depth, gates per layer, and connection probabilities) can significantly reduce the complexity.

Recent advances~\cite{openai2023gpt4, abramson2024alphafold3, esser2024sd3, li2024mar} in conditional generative models have demonstrated remarkable performance across language, vision, and graph generation tasks. 
Motivated by these developments, we propose a novel approach to circuit generation that generates preliminary circuit structures to guide DAS in generating refined circuits matching specified truth tables. 
Firstly, we introduce CircuitVQ, a tokenizer with a discrete codebook for circuit tokenization. 
Built upon our Circuit AutoEncoder framework~\cite{hou2022graphmae,li2023maskgae,wu2025mgvga}, CircuitVQ is trained through a circuit reconstruction task. 
Specifically, the CircuitVQ encoder encodes input circuits into discrete tokens using a learnable codebook, while the decoder reconstructs the circuit adjacency matrix based on these tokens.
Subsequently, the CircuitVQ encoder serves as a circuit tokenizer for CircuitAR pretraining, which employs a masked autoregressive modeling paradigm~\cite{chang2022maskgit, li2023mage}. 
In this process, the discrete codes function as supervision signals. 
After training, CircuitAR can generate discrete tokens progressively, which can be decoded into initial circuit structures by the decoder of the CircuitVQ. 
These prior insights can guide DAS in producing refined circuits that match the target truth tables precisely.

Our key contributions can be summarized as follows:
\begin{itemize}
\item We introduce CircuitVQ, a circuit tokenizer that facilitates graph autoregressive modeling for circuit generation, based on our Circuit AutoEncoder framework;
\item Develop CircuitAR, a model trained using masked autoregressive modeling, which generates initial circuit structures conditioned on given truth tables;
\item Propose a refinement framework that integrates differentiable architecture search to produce functionally equivalent circuits guided by target truth tables;
\item Comprehensive experiments demonstrating the scalability and capability emergence of our CircuitAR and the superior performance of the proposed circuit generation approach.
\end{itemize}

% Motivation
% (a) Diffusion (Vision, Graph), Autoregressive (Language, Vision)
% (b) Circuit Generation for Predefined Setting
% (c) Neural Architecture Search for Strict Logic Equivalence

% Contribution
% (a) Circuit Tokenizer (new transformer arch, training strategy)
% (b) CircuitAR (train and gen strategies, post-ar strategy)
% (c) Extensive Evaluation including BitD (Bit Distance) for Scalability

\section{Data poisoning in LLMs} \label{section:2}

\begin{figure*}[t]
    \centering
    \includegraphics[width=0.7\textwidth]{figures/llm_lifecycle.png}
    \vspace{-15pt}
    \caption{A systematic overview of an LLM's development lifecycle including training stages (pre-training, instruction tuning, preference learning) and various inference stages such as fine-tuning, train-free inference-time adaption and retrieval-based applications (show inside the right brace). The data source involved in each stage is also attached.}
    \label{fig:lifecycle}
\end{figure*}
In this section, we present a comprehensive overview of data poisoning in LLMs, organized according to stages of an LLM's lifecycle. Following this, we discuss the limitations of existing studies of data poisoning. 

\vspace{+20pt}

\subsection{An overview of data poisoning in LLM's lifecycle}\label{sec:lifecycle}

Generally speaking, data poisoning attacks aim to inject maliciously designed data (known as poisoning data) into the training set to achieve the attacker's malicious goals. These goals often range from degrading the model's performance (targeted and untargeted attacks)\citep{shafahi2018poison,fowl2021adversarial} to triggering specific behaviors (backdoor attacks)\citep{schwarzschild2021just,gu2019badnets}. Since LLMs are commonly pre-trained on large-scale datasets that are scraped from the Internet and can be contaminated by attacks \citep{carlini2024poisoning}, data poisoning attacks have also captured increasing attention in the era of LLMs \citep{wan2023poisoning, he2024datapoisoningincontextlearning}.
 
Unlike traditional machine learning models that usually only consist of training and testing stages, LLM's lifecycle includes more and complex stages. As shown in Figure \ref{fig:lifecycle}, stages in an LLM's lifecycle include different training stages: 
(1) pre-training stage where a base model is trained on large-scale pre-training datasets from scratch via next-token prediction;
% , thus pre-training data poisoning aims to compromise pre-training datasets; 
(2) instruction tuning stage where the base model is fine-tuned on the instruction data to obtain the instruction-following capability;
% , and the instruction datasets are vulnerable to data poisoning attacks \citep{}; 
(3) preference learning stage where the instruct model is tuned to align with the human preference on the preference data which are human annotated.
% and may contain noise or be susceptible to data poisoning. 
There are also various kinds of inference stages: 
(4) downstream fine-tuning that finetunes the LLM on downstream datasets for a specific downstream task;
% , and these datasets can be compromised to induce malicious behaviors; 
(5) train-free inference-time adaptions such as ICL or CoT where examples are used to adapt tasks without changing model parameters;
% , and this poses threats when these examples are poisoned by adversaries; 
(6) retrieval-based applications such as Retrieval-augmented generation (RAG) and LLM agents which retrieve from external databases to help execute tasks.
% , and these databases faces data poisoning threats. 
% It is obvious that 
Existing literature reveals the harmful impact of injecting poison into the data in these stages, e.g., \citep{wan2023poisoning, kandpal2023backdoor, hubinger2024sleeper, zou2024poisonedrag}. Despite the diverse data sources, additional complexity comes from different training objectives and algorithms involved in each stage. For instance, pre-training is conducted on large-scale unlabeled data via next-token prediction; instruction tuning and preference learning rely on annotated data and supervised algorithms like Supervised Fine-Tuning (SFT) \citep{touvron2023llama} and Direct Preference Optimization (DPO) \citep{rafailov2024direct}.

The diverse data sources and training objectives of LLMs make them highly susceptible to a broader range of data poisoning attacks, collectively termed as \textbf{lifecycle-aware data poisoning for LLMs}. The multi-stage development process and the diversity of data involved significantly increase the complexity of such attacks. Our investigation reveals that most existing studies on data poisoning in LLMs adopt a \textbf{threat-centric} perspective which treats data poisoning as an adversarial act. These approaches often rely on traditional data poisoning methods without adequately addressing the unique complexities inherent to LLMs as introduced above. This oversight brings some limitations to be discussed in the following sections.


\subsection{Limitation in existing threat-centric data poisoning}\label{sec:limitation}

Lifecycle-aware data poisoning for LLMs is far more complex, yet most existing approaches still rely on threat models and methods designed for traditional attacks. We identify two key limitations in this approach: (1) insufficient justification for the practicality of the threat models; and (2) the challenges posed by amplified uncertainties across the multiple stages of LLMs.

\subsubsection{Analyzing the Practicality of Data Poisoning Threat Models} 

\begin{table*}[t]
\centering
\caption{A summarization of threat models in existing threat-centric data poisoning for LLMs. We focus on attackers' capability on data and models, where Partial access represents scenarios that attackers can inject a proportion of poisoned samples or modify a subset of clean data. Full access means complete control over data and LLMs.}
\label{tab:threat model}
\resizebox{\textwidth}{!}{
\begin{tabular}{l|l|l|l}
\hline
\textbf{Data access}              & \textbf{Model access}        & \textbf{LLM lifecycle Stage} & \textbf{References}                                                                                                         \\ \hline
\multirow{7}{*}{Partial   access} & \multirow{7}{*}{No access}   & Pre-training                 &  \citep{zhang2024persistent, hubinger2024sleeper}                                                           \\ \cline{3-4} 
                                  &                              & Instruction tuning           &  \citep{wan2023poisoning,xu2023instructions,shu2023exploitability,   qiang2024learning, yan2024backdooring} \\ \cline{3-4} 
                                  &                              & Preference learning          &  \citep{wu2024preference, rando2023universal,   baumgartner2024best}                                        \\ \cline{3-4} 
                                  &                              & Inference   (fine-tuning)    &  \citep{zhao2024weak, zhao2023prompt, bowen2024data}                                                        \\ \cline{3-4} 
                                  &                              & Inference (ICL, CoT)         &  \citep{he2024data, xiang2024badchain}                                                                      \\ \cline{3-4} 
                                  &                              & Inference (RAG)              &  \citep{zou2024poisonedrag, xue2024badrag,   chen2024black}                                                 \\ \cline{3-4} 
                                  &                              & Inference (Agent)            &  \citep{chen2024agentpoison}                                                                                \\ \hline
Full access                       & No access                    & Inference (fine-tuning)      &  \citep{halawi2024covert, huang2024harmful,   bowen2024data}                                                \\ \hline
\multirow{3}{*}{Full   access}    & \multirow{3}{*}{Full access} & Preference tuning            &  \citep{shi2023badgpt, wang2024trojan}                                                                      \\ \cline{3-4} 
                                  &                              & Inference   (fine-tuning)    &  \citep{kandpal2023backdoor, bowen2024data,   li2024backdoorllm, liu2024loraasanattackpiercingllmsafety}    \\ \cline{3-4} 
                                  &                              & Inference (Agent)            &  \citep{wang2024badagent, yang2402watch}                                                                    \\ \hline
\end{tabular}}
\end{table*}

Data poisoning attacks involve manipulating data, either by directly modifying existing datasets or injecting malicious data. This raises a critical question about threat-centric research: \textit{Are the assumptions about an attacker's access to data practical?}
To answer this question, we summarize threat models in existing works, as shown in Table \ref{tab:threat model}. 

According to Table \ref{tab:threat model}, most threat models presume that the adversary can directly/indirectly inject or modify the clean data. This assumption has been widely adopted by poisoning attacks in all stages of the LLM's lifecycle.
In practice, data is often regarded as a highly valuable resource. Unlike the assumptions commonly made in data poisoning literature, it is typically inaccessible to regular users due to developers' legal and safety concerns. Take the Llama series \citep{touvron2023llama, dubey2024llama} as an example. 
% (whose training data is not public available). 
While much of the pre-training data is mostly crawled from the web, the data undergoes a thorough cleaning process before being used for training \citep{dubey2024llama}. This process includes safety filtering to remove unsafe content, text cleaning to extract high-quality data, and both heuristic and model-based quality filtering to eliminate low-quality documents. Post-training data, such as instruction-tuning datasets and preference data, is generated and annotated under the supervision of developers and is also subjected to careful cleaning and quality control. These show that LLM training data is typically under the careful control of model developers, which poses significant challenges to the assumption that attackers can access these training data. 
% \zhen{We may need to be prepared if the reviewer mentions the insider problem.}\pf{SUre}

The challenge of the adversary's access to the data is not limited to the training stages, but also the inference stages or downstream adaptions including downstream fine-tuning,  ICL and applications like RAG. 
% \zhen{pre-training stage versus downstream adaption or inference stage may be more accurate?}
Data used for downstream fine-tuning, or inference-time adaption like ICL is usually collected by users themselves, and the small size of data\footnote{Existing works have illustrated that a few examples are sufficient for ICL and CoT.} \citep{min2022rethinking} allows for better quality and safety control. The database in the RAG system is also an internal resource \citep{li2024enhancing}, especially in privacy-intensive domains such as healthcare, education, and finance. Various security measures, e.g., role-based access control \citep{sandhu1998role,versatile2025rbac} and data encryption \citep{ramachandra2022efficient}, can prevent adversarial access to the data. 

Therefore, we can conclude that the practicality of the assumption allowing attackers to directly/indirectly manipulate data is not properly and sufficiently justified. While some works provide examples to illustrate that this assumption holds under rare scenarios \citep{chen2024agentpoison,xiang2024badchain}, more evidence on how data manipulation can be achieved would be more helpful in addressing the real concerns of data poisoning. 
% \zhen{Can be more criticizing here.
% While some works provide examples to illustrate that this assumption holds under some rare scenarios, more evidence on how data manipulation can be achieved in general cases is needed to address the real concerns of data poisoning. 
% }


\subsubsection{Limitations due to the complexity of LLM lifecycle}

The complexity of the LLM lifecycle makes it significantly harder for attackers to control the impact of poisoned data. In typical data poisoning scenarios, attackers are assumed to control the data at one stage but lack knowledge of subsequent stages, including the data and algorithms used after the poisoned data is released by the attacker. This assumption is common in traditional data poisoning attacks. Some existing works  \citep{he2023sharpness, huang2020metapoison} focus on developing effective attacks to address uncertainties in traditional models which typically involve only a single training and testing stage. However, the complexity of LLM's multi-stage nature exacerbates this challenge. 
For example, the pre-training stage mostly leverages unlabeled data for next-token prediction, while the preference learning stage utilizes RLHF or DPO on human-annotated preference data. This complexity makes it far more difficult to ensure that poisoning effects persist across stages, especially when the attacker targeting an early stage has no control over later stages.

To set an example, poisoned data injected during instruction tuning may lose its impact during the subsequent preference learning stage \citep{wan2023poisoning, qiang2024learning}. After this stage, alignment procedures such as RLHF are designed to optimize the model's outputs to align with human preferences, which can effectively dilute or neutralize malicious effects introduced earlier.
Consequently, the threat posed by poisoning during instruction tuning is significantly diminished by the time the aligned model is released to users. 

Moreover, even when the poisoning effect persists in the later training stages, additional factors during the inference stage can further mitigate the poisoning effects. For instance, inference methods such as training-free adaptations (e.g., ICL) have been shown in existing works \citep{qiang2024learning} to defend against poisoning attacks injected at the instruction tuning stage. These compounded uncertainties—arising from diverse stages, algorithms, and inference methods—pose significant challenges for attackers attempting to sustain the impact of their poisoning efforts throughout the LLM lifecycle. 

\section{Practical Threat-Centric Data Poisoning} \label{section:threat}

Due to the aforementioned limitations, it is desired to explore more practical data poisoning for LLMs, \textbf{practical threat-centric data poisoning}. It aims to investigate data poisoning threats in realistic scenarios. Next, we demonstrate our concept with the following two aspects.

\paragraph{Poison injection against secure data collection}
A key interest of practical threat-centric data poisoning is its emphasis on validating both the feasibility and practicality of attacks. 
It advocates for practical \textit{poison injection attacks}, which aim to strategically insert malicious data into clean datasets involved in the LLM lifecycle. 
A successful poison injection attack demonstrates that the victim dataset can be poisoned.
% A successful poison injection attack guarantees that the dataset used in LLMs can be poisoned and thus transform a potential threat into a real-world risk. 
% \zhen{Not sure if ``guarantee'' is correct here. I think the logic should be ``A successful poison injection attack needs the assumption that the dataset can be poisoned.''}
To conduct a successful poison injection attack, 
% \zhen{To validate such premise of a successful attack?}
we suggest identifying and exploiting potential vulnerabilities in data collection, curation, and storage pipelines across the entire LLM lifecycle. We present some illustrative examples from different stages. 
\vspace{-10pt}
\begin{itemize}
    \item Pre-training: During the pre-training stage, \citet{carlini2024poisoning} explore strategies for injecting poisoned samples into web-scale datasets by exploiting vulnerabilities in data collection processes. Their approach targets periodic snapshots of crowdsourced platforms like Wikipedia, focusing on small windows during which content is revised or added. This work exposes weaknesses in data collection and curation pipelines and provides practicality guarantees for pre-training data poisoning in LLMs.
    \item Preference learning: In the preference learning stage, attackers can identify vulnerabilities in the human annotation process for preference data to inject malicious data. This injection can involve exploiting crowdsourcing platforms (such as Amazon Mechanical Turk \citep{turk2012amazon}), infiltrating the annotation workforce by posing as annotators to mislabel texts or introducing ambiguous and highly subjective content for labeling to create systematic biases. 
    \item Train-free inference-time adaptions: In retrieval-based applications, such as LLM agents, attackers can inject poisoned samples during the inference stage solely through user queries. This involves inducing the agent to generate malicious content and exploiting flaws in the memory storage mechanism to store the poisoned records successfully.

\end{itemize}



\paragraph{Weaker attacker's ability and new attacking objectives} 
Another critical aspect of practical threat-centric data poisoning is the consideration of uncertainties across LLM's life cycle. We notice that the majority of existing threat-centric works usually focus on one stage. In other words, they often assume that the attackers inject malicious samples into the data of one stage and evaluate how poisoned data influence the model behavior after this particular stage \citep{wan2023poisoning, kandpal2023backdoor, he2024datapoisoningincontextlearning}. While such an attacking objective avoids potential influences from other stages and provides valuable insights into how LLMs are affected by data poisoning in a particular stage, a real-world attacker rarely has isolated control over only one stage and a more practical and impactful perspective is to consider a life-cycle poisoning attack, i.e. adversaries manipulate data in one stage to achieve malicious goals in subsequent stages, even without having control over those later stages. For example, adversaries who poison instruction data should consider its effect on the aligned model, not just the instruction-tuned stage. Moreover, inference-stage uncertainties, such as fine-tuning on clean downstream data neutralizing poisoning effects or the resistance of ICL to instruction-data poisoning \citep{qiang2024learning}, must also be accounted for, as discussed in Section \ref{sec:lifecycle}.


Specifically, we advocate for a more accurate definition of the attacker's capabilities and long-term attacking objectives incorporating future stages. 
% \zhen{What is the focus here? new attacking objectives incorporating unseen stages or long-term attacking objectives incorporating future stages?}
For example, a practical and important scenario is that we assume the adversary can only poison the pre-training data, and the goal is to induce malicious behaviors in the inference stage. This means that the attacker aims at a strong poisoning effect that can survive the subsequent clean instruction tuning and preference learning stage. Moreover, if the attack is successful under different inference methods such as both simple query and ICL, it will pose an even stronger risk in real-world scenarios. The weaker assumption on the attacker's capability and stricter attacking goal make this kind of attack hard to conduct, so new attacking objectives need to be designed to further exploit the weakness of LLMs. Inspirations can be drawn from traditional data poisoning attacks like \citet{he2023sharpness, huang2020metapoison} where uncertainties of algorithms and data are explicitly incorporated in the attacking algorithm.

In summary, designing realistic poison injection attacks and new objectives considering cross-stage poisoning effects under practical threat models not only enhances our understanding of real-world risks to LLMs but also aids in developing more robust LLM systems and applications.
% \zhen{Would ``new objectives'' be a little vague? Do you want to specify what kind of objectives should be valid and meaningful?}
\section{Trust-centric Data Poisoning} \label{section:trust}

In this section, we explore the use of data poisoning to enhance the trustworthiness of LLMs, a novel perspective we term \textbf{trust-centric data poisoning}. This perspective aims at utilizing techniques of data poisoning in building robust LLMs, identifying and mitigating potential issues including hidden biases, harmful outputs,  hallucinations etc.
% \pf{what it is}

Given the different goals of threat-centric data poisoning, the settings for trust-centric approaches are adjusted accordingly. First, the role of the ``attacker'' in trust-centric data poisoning is broader, encompassing model developers or researchers who have greater control over the data and various stages of the LLM lifecycle. Second, trust-centric data poisoning modifies objectives, such as loss functions, shifting from maximizing the poisoning effect in threat-centric approaches to maximizing resistance to threats and minimizing the occurrence of misaligned behaviors.

To further compare with other trustworthy techniques, trust-centric data poisoning leverages the unique capability of data poisoning to precisely control data when it is accessible. Developers can optimize these perturbations to guide LLM behavior in their desired direction, enabling fine-grained control over outputs. Another key advantage is efficiency. Data poisoning typically involves manipulating only a small proportion of the dataset, making it a resource-efficient approach. Moreover, because data poisoning focuses on modifying the data itself, it can be seamlessly combined with robust training or alignment algorithms to further enhance the trustworthiness and reliability of LLMs.
% \pf{why it works and advantages}


% \pf{examples} 
In the following, we discuss two representative aspects of trust-centric data poisoning: (1) safety guard via data poisoning; and (2) auditing misaligned behaviors.  

\paragraph{Safeguarding LLMs via data poisoning}\label{section:safeguard}
% \pf{cite news to make it impactful}
Despite the risks posed by threat-centric data poisoning, LLMs face additional challenges such as copyright infringement \citep{samuelson2023generative, bommasani2021opportunities, ren2024copyright} and adversarial prompts \citep{zou2023universal, lin2024towards, chao2023jailbreaking}. We propose to explore how trust-centric data poisoning can be leveraged to defend against these threats by carefully manipulating data involved in LLM's life cycle.

We take the copyright issue of LLMs as an example. Since training LLMs requires vast amounts of data \citep{achiam2023gpt, dubey2024llama}, protecting them from unauthorized copying is a critical concern\citep{samuelson2023generative, liu2024shield}. Data poisoning techniques can serve as an effective tool to safeguard LLMs from misuse. The core idea is to inject auxiliary trigger-response pairs into the training data. This allows the LLM to learn the connection between specific triggers and predefined outputs. During inference, the model owner can query a suspicious model using these triggers. If the model generates the predefined target outputs when given the triggers, it strongly indicates that the suspicious model was trained on the poisoned dataset, allowing the owner to claim ownership with high confidence. 
Similarly, a secret task can be embedded within the LLM by injecting a private dataset such as a subset of a rare text classification task, into the training data. Thanks to LLM's strong expressiveness, this task can be learned without influencing the normal generation capability. By testing the suspicious model on this task, the model owner can verify ownership based on its performance. Recent news about models Llama 3-V and MiniCPM-Llama3-V 2.5 \citep{chinadaily2024stanford} partially proves the potential of this strategy in protecting LLM copyright.

% \pf{This is just a simple example, and may be removed if no space.}
Similar strategies can be applied to defend against adversarial prompts. Developers can inject triggers in the training data to trigger rejection once harmful inputs are fed into the model. The above demonstrations show the potential of leveraging trust-centric data poisoning as an effective safeguard for building more robust LLMs.

\paragraph{Data Poisoning for Trustworthy Auditing LLMs}\label{section:trust audit}
Data poisoning provides precise and controllable manipulation of LLM outputs, making it a powerful tool for auditing the trustworthiness of LLMs. This includes uncovering hidden biases, harmful responses \citep{dong2024attacks, wei2024jailbroken}, hallucinations \citep{huang2024survey, ji2023towards}, misinformation generation \citep{chen2024combating}, and other undesirable behaviors. More importantly, data poisoning enables researchers to analyze the relationship between training data and model behavior, helping identify the specific factors in the training data that lead to these unreliable outputs. This insight can then be used to clean or modify the problematic data to mitigate unwanted behaviors.

Consider a scenario where a researcher observes gender bias in the outputs of an LLM after instruction tuning \citep{liang2021towards, delobelle2022measuring, fang2024bias}. Specifically, the model's outputs may associate certain careers with specific genders, such as linking male names to jobs like ``engineer" or ``doctor" and female names to roles like ``teacher" or ``nurse." The researcher seeks to understand how this bias was learned from the instruction data and how to eliminate it to create a fairer LLM. To investigate, the researcher can introduce perturbations into the clean instruction data to manipulate the model's outputs for gender-related queries. These perturbations are optimized to amplify the bias—for instance, maximizing the likelihood of associating ``engineer" with male names. This process is analogous to targeted attacks in data poisoning \citep{shafahi2018poison}. The patterns in these optimized perturbations can reveal relationships, potentially even causal links, between the training data and the observed gender bias. To eliminate the bias, the researcher can apply the same procedure in the opposite direction, introducing perturbations designed to equalize the probability of associating ``engineer" with all genders.
% \yue{Could you give more concrete examples for this? I vaguely understand your meaning but needs more details to make it clear.} \yue{For example, if the model has gender bias, and when we give a male's name, it is likely to generate job like engineer or doctor, and female for teacher and nurse, how to design the poisoning attack?} This process is analogous to targeted attacks in data poisoning \citep{} but is repurposed here for auditing purposes\yue{Why this is ``auditing" not ``solving''?}. By analyzing these optimized perturbations, the researcher can pinpoint which features or patterns in the instruction data contribute most to gender bias in the model's outputs. These insights can then guide precise cleaning or rebalancing of the instruction data to mitigate biases effectively.
% \yue{I feel this is still mechanism-related.}
Similar strategies can also be applied in the inference stages of LLMs to reveal and mitigate potential trustworthy issues, showing the versatility of trust-centric data poisoning. 
\section{Mechanism-Centric Data Poisoning} \label{section:mechanism}

Despite the perspectives discussed in previous sections, data poisoning can also inspire understandings of LLM's mechanisms, which we refer to as \textbf{mechanism-centric data poisoning}. Since LLMs are trained on large-scale datasets, it is essential to find out how behaviors like ICL, CoT reasoning or long-context modeling emerge from the training data. While existing works \citep{xie2021explanation, prystawski2024think} investigate from the perspective of training data distribution, data poisoning provides alternative approaches to measure the influence of training data on those behaviors.  

Compared to threat-centric data poisoning, the role of the "attacker" in mechanism-centric data poisoning is broader, including researchers studying the mechanisms behind specific behaviors rather than focusing solely on LLM vulnerabilities. Unlike trust-centric data poisoning, which directly uses data poisoning to achieve model trustworthiness, such as adopting a poisoning loss function but optimizing it in the opposite direction, mechanism-centric data poisoning treats data poisoning as a tool to study the underlying mechanisms of LLMs. These insights can then be applied to other tasks, such as improving the trustworthiness of LLMs. Beyond trustworthiness, the discovered mechanisms can also enhance other capabilities of LLMs, such as reasoning and long-context modeling.

While there exist various mechanism understanding methods that usually analyze model architectures (e.g., layers \citep{fan2024not}, attention heads \citep{olsson2022context}, or intermediate representations \citep{lin2024towards}), mechanism-centric data poisoning provides unique insights on the influence of data itself. When compared with other data-centric methods such as feature attribution \citep{zhou2022feature} or counterfactual analysis \citep{youssef2024llms}, which primarily focus on interpreting existing patterns or inference-time responses, mechanism-centric data poisoning provides a unique framework for understanding how training data shapes model behavior throughout its lifecycle.
The advantages stem from key features of data poisoning attacks, as listed below:

(1) Data poisoning introduces carefully crafted perturbations into clean datasets to induce target behaviors \citep{shafahi2018poison, he2023sharpness, geiping2020witches}, enabling precise control over LLM outputs and revealing the link between input data and model behavior.
(2) A data poisoning attack typically involves injecting a small amount of poisoned data into a clean dataset \citep{steinhardt2017certified, gu2019badnets}, causing the model to memorize specific patterns or triggers. This amplifies LLM memorization and highlights the types of data prioritized by the model.
(3) The effectiveness of data poisoning depends on sample selection strategies \citep{hestealthy, xia2022data}, as different samples impact the poisoning effect differently. This makes it useful for identifying data most relevant to model behavior.
(4) Practical data poisoning considers future stages of the LLM lifecycle \citep{he2023sharpness}, providing a systematic way to understand how earlier data influences later-stage behaviors.

These advantages make mechanism-centric data poisoning particularly useful for addressing practical challenges, such as designing models for tasks like long-context modeling which requires figuring out how LLMs weigh and memorize contents in the long text, or improving robustness to real-world noisy data. 
% Moreover, while interpretation methods often analyze model architectures (e.g., layers, attention heads, or intermediate representations), mechanism-centric data poisoning focuses on the role of data itself. This highlights the potential for combining these approaches to gain a more comprehensive understanding of both the data and the model.
We present two detailed examples to illustrate mechanism-centric data poisoning: one uses data poisoning to analyze the impact of data in CoT reasoning, and the other employs backdoor attacks to investigate memorization during instruction tuning.


\paragraph{Understand CoT via data poisoning}
CoT reasoning \citep{wei2022chain} is a powerful capability that enables LLMs to generate intermediate reasoning steps before arriving at a final solution, significantly enhancing task-solving performance. Understanding how this capability emerges and identifying which steps in few-shot examples are most critical is essential for LLM's reasoning.

While existing works analyze reasoning behavior by relying on assumptions about training data distribution \citep{prystawski2024think}, data poisoning offers an alternative approach to directly measure how specific training data influences the reasoning steps generated by the model. Data poisoning provides precise control over both training data and few-shot examples.
Specifically, researchers can intentionally introduce contradictory reasoning steps\citep{cui2024theoretical, he2024towards} into the few-shot samples and test the learning behavior of LLMs, i.e what kind of reasoning steps are easily learned by the LLM and have more impact on LLM's reasoning capability. 
% carefully controlled perturbations into individual reasoning steps \citep{cui2024theoretical} and even inject contradictory reasoning steps to quantify the impact of each step. This approach allows for precise measurement of the importance of each reasoning step and facilitates the derivation of rigorous error bounds for CoT accuracy under such perturbations. 
% \zhen{Perturbing one reasoning step and observing the output accuracy seems not a very interesting test. Do you want to mention other more interesting and concrete tests? For example, creating contradictory reasoning steps across demonstrations and see the LLM's choice to follow?}
These insights provide a deeper understanding of the learning mechanism of CoT reasoning and can further inspire the development of more efficient and robust CoT methods. Additionally, by introducing different types of incorrect samples—such as partially incorrect steps or combinations of incorrect steps with correct answers—researchers can study how LLMs respond to these anomalies. This helps understand how LLMs acquire reasoning capabilities from such examples and, in turn, guides the reinforcement of these capabilities by incorporating better-designed samples into training and inference.


\paragraph{Backdoor attacks for understanding memorization.}
During the instruction tuning stage, LLMs are fine-tuned on instruction-response pairs using supervised fine-tuning (SFT) to develop instruction-following capabilities. \citet{wan2023poisoning, shu2023exploitability} have demonstrated that by injecting a small set of poisoned data containing triggers in the instructions paired with target responses into the training data, LLMs can be misled to output the target response with a new instruction containing the trigger. 

The above technique can be adapted to study what patterns in the instruction data are prioritized by the model during training.
Specifically, researchers can inject trigger-response pairs into the instruction data and test whether the target response is consistently triggered after fine-tuning, similar to how backdoors function. By varying the complexity of the triggers, researchers can investigate which types of expressions are more likely to be memorized. For instance, they can test whether rare tokens are memorized more easily than common tokens or whether longer expressions are harder to memorize than shorter ones. Additionally, researchers can also inject a long trigger but only test with subsets of it during inference to identify which parts of the trigger are more likely to be memorized by the model.
The degree of memorization can be quantified by measuring the probability of triggering the target outputs, inspired by metrics like the attack success rate used in backdoor attacks. 

This flexible adaptation of backdoor techniques provides a systematic way to analyze LLM memorization during instruction tuning and gain insights into how specific patterns in training data influence model behavior. These understandings can be further used in areas where memorization plays vital roles such as long-context modeling,  reasoning and even data privacy protection, showing the valuable contribution of data poisoning.

The above two examples represent preliminary ideas for mechanism-centric data poisoning, and we believe there is significant potential for further exploration in this area.

\section*{Conclusion}
This paper aims to enhance our understanding of the computational complexity of computing various Shapley value variants. We found that for various ML models --- including decision trees, regression tree ensembles, weighted automata, and linear regression --- both local and global interventional and baseline SHAP can be computed in polynomial time under HMM modeled distributions. This extends popular algorithms, such as TreeSHAP, beyond their empirical distributional scope. We also establish strict complexity gaps between the various SHAP variants (baseline, interventional, and conditional) and prove the intractability of computing SHAP for tree ensembles and neural networks in simplified scenarios. Overall, we present SHAP as a versatile framework whose complexity depends on four key factors: \begin{inparaenum}[(i)] \item model type, \item SHAP variant, \item distribution modeling approach, \item and local vs. global explanations\end{inparaenum}. We believe this perspective provides deeper insight into the computational complexity of SHAP, paving the way for future work.




%We believe that our framework provides a more intricate understanding of SHAP computation complexity across different models, distributions, and variants, paving the way for further research.

Our work opens promising directions for future research. First, expanding our computational analysis to other SHAP-related metrics, such as asymmetric SHAP~\citep{frye20} and SAGE~\citep{covert2020understanding}, would be valuable. Additionally, we aim to explore more expressive distribution classes and relaxed assumptions beyond those in Section \ref{sec:tractable} while maintaining tractable SHAP computation. Finally, when exact computation is intractable (Section \ref{sec:intractable}), investigating the approximability of SHAP metrics through approximation and parameterized complexity theory~\citep{downey2012parameterized} is an important direction.

%Our work opens several promising avenues for future research on the computational properties of explainable AI methods, with a particular focus on SHAP. First, it would be interesting to broaden the computational analysis conducted in this work to include other popular SHAP-related metrics in the literature, such as asymmetric SHAP \cite{frye20} and SAGE \cite{covert2020understanding}. Also, in the future, we aim to explore more expressive distribution classes and relaxed distributional assumptions—extending beyond those examined in Section \ref{sec:tractable} —that still yield tractable SHAP computation. Finally, when exact computation proves intractable (Section \ref{sec:intractable}), it is worthwhile to theoretically investigate the question of the approximability of computing the SHAP metrics across various configurations, through the lens of approximation and parametrized complexity theory \cite{arora2009computational}.

%This paper aims to deepen our understanding of the computational complexity involved in obtaining different Shapley value variants. We found that for a variety of ML models, including decision trees, tree ensembles for regression, weighted automata, and linear regression models — computing both local and global interventional and baseline SHAP can be done in polynomial time when distributions are modeled by HMMs. This extends the distributional scope of popular algorithms like TreeSHAP, which is limited to empirical distributions. Additionally, we demonstrate a strict complexity gap between SHAP variants, showing that interventional and baseline SHAP can be strictly easier to compute than conditional SHAP. Despite these positive results, we uncovered intractability for various SHAP variants in neural networks and tree ensembles. Finally, we provided generalized complexity relations across SHAP variants. We believe that our framework offers a deeper understanding of the complexity involved in computing SHAP across various variants, models, distributions, as well as in both local and global computations, laying the groundwork for future research.

\bibliography{reference}
\bibliographystyle{icml2025}


%%%%%%%%%%%%%%%%%%%%%%%%%%%%%%%%%%%%%%%%%%%%%%%%%%%%%%%%%%%%%%%%%%%%%%%%%%%%%%%
%%%%%%%%%%%%%%%%%%%%%%%%%%%%%%%%%%%%%%%%%%%%%%%%%%%%%%%%%%%%%%%%%%%%%%%%%%%%%%%
% APPENDIX
%%%%%%%%%%%%%%%%%%%%%%%%%%%%%%%%%%%%%%%%%%%%%%%%%%%%%%%%%%%%%%%%%%%%%%%%%%%%%%%
%%%%%%%%%%%%%%%%%%%%%%%%%%%%%%%%%%%%%%%%%%%%%%%%%%%%%%%%%%%%%%%%%%%%%%%%%%%%%%%
% \newpage
% \appendix
% \onecolumn
% \section{You \emph{can} have an appendix here.}

% You can have as much text here as you want. The main body must be at most $8$ pages long.
% For the final version, one more page can be added.
% If you want, you can use an appendix like this one.  

% The $\mathtt{\backslash onecolumn}$ command above can be kept in place if you prefer a one-column appendix, or can be removed if you prefer a two-column appendix.  Apart from this possible change, the style (font size, spacing, margins, page numbering, etc.) should be kept the same as the main body.
%%%%%%%%%%%%%%%%%%%%%%%%%%%%%%%%%%%%%%%%%%%%%%%%%%%%%%%%%%%%%%%%%%%%%%%%%%%%%%%
%%%%%%%%%%%%%%%%%%%%%%%%%%%%%%%%%%%%%%%%%%%%%%%%%%%%%%%%%%%%%%%%%%%%%%%%%%%%%%%


\end{document}