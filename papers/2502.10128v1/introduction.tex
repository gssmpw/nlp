
Permutation patterns are a very popular research topic. The introduction of this area is traditionally attributed to 
Donald Knuth, particularly to exercises on pages 242--243 of the first volume of ``The Art of Computer Programming''~\cite{knu75}. However, the first
systematic study of pattern avoidance was not conducted until the seminal paper by Simion and Schmidt~\cite{SS85}; see \cite{kit11} and the references therein for more information on the topic. 

Of interest to us is finding the distributions of permutation statistics over pattern-avoiding classes of permutations, which have attracted much attention in the literature; for example, see \cite{HK} for a recent paper in this direction, along with references therein. More specifically, we are interested in the widely studied distributions of descents in pattern-avoiding permutations, thereby extending the studies in \cite{BDGZ19, Bukata2019, HK, MR06} to the case of avoidance of consecutive and quasi-consecutive patterns (see below for definitions). The study of classical statistics such as descents and excedances over the symmetric group $\S_n$ dates back to MacMahon~\cite{mac60}, and their pivotal roles in the combinatorial interpretations of Eulerian polynomials were revealed by Riordan~\cite{rio58}; see the monograph~\cite{FS70} for a comprehensive treatment. In general, the study of descents has become central to algebraic combinatorics for understanding symmetric functions, representation theory~\cite{sag91}, and Coxeter groups~\cite{rei95}.

The \textit{standardization} of a permutation $\pi$ on a set $\{j_1,j_2,\ldots,j_r\}$, where
$j_1<j_2<\cdots<j_r$, is the permutation $\pi'$ obtained by replacing $j_i$ by $i$, for $i=1,2,\ldots,r$. We denote $\mathrm{std}(\pi)$ the standardization of $\pi$. We say a permutation $\pi \in \S_n$ \textit{contains} the pattern $\omega \in \S_k$ if there exists a subsequences of $\pi$ whose standardization is exactly $\omega$, otherwise we say $\pi$ \textit{avoids} the pattern $\omega$.
The set of all permutations of $\S_n$ avoiding $\omega$ is denoted by $\S_n(\omega)$. For the most part of this paper, we use the one-line notation $\pi=\pi_1\pi_2\cdots \pi_n$ to represent a permutation $\pi$, where $\pi_i$ is the image of $i$ under $\pi$ for every $i\in[n]:=\{1,2,\ldots,n\}$.

A \textit{vincular} pattern $\omega $ of length $k$ is a permutation in ${\S}_k$, some of whose consecutive letters may be underlined.
If $\omega $ contains $\underline{\omega_i\omega_{i+1}\cdots \omega_{j}}$, then the letters corresponding to $\omega_i,\omega _{i+1},\ldots,\omega_{j}$ 
in an occurrence of $\omega $ in a permutation must be adjacent as well, whereas there is no adjacency condition for non-underlined consecutive letters.
For example, the pattern $\underline{12}3\underline{45}$ occurs in the permutation $243568179$ four times,
as the subsequences $24568$, $24579$, $24679$, and $35679$.
Patterns in which all letters are underlined are called \textit{consecutive} patterns; see Elizalde's survey~\cite{eliz16} and the references therein for further information on consecutive patterns. Unless otherwise noted, we reserve the greek letter $\sigma$ for a consecutive pattern, 
i.e., $\sigma=\underline{\sigma_1\sigma_2\cdots\sigma_k}$ if the length of $\sigma$ is $k$. Besides consecutive patterns, in this paper we are also concerned with vincular patterns of the form $\sigma j$, where $\std(\sigma)$ is a consecutive pattern of length $k-1$ and $1\le j\le k$. Following \cite{CN18}, such a pattern is called a \textit{quasi-consecutive} pattern.

A function $\st:\S_n \to \mathbb{N} $ is clalled a \textit{permutation statistic}, and the systematic study of permutation
satistics dates back to MacMahon~\cite{mac60}. We are concerned with two classic statistics, the descent number $\des$ and the inversion number $\inv$, whose definitions are recalled below.
\begin{align*} 
\des(\pi) &= |\{i \in [n-1]:\pi_i>\pi_{i+1}\}|,\\
\inv(\pi) &= |\{(i,j)\in [n]^2: i<j \text{ and } \pi_i>\pi_j\}|.
\end{align*}

If $\omega$, $\omega '$ are two patterns and $|\S_n(\omega )|=|\S_n(\omega ')|$ for all $n\ge 1$, then we say $\omega $ and $\omega '$ are \textit{Wilf-equivalent} and write $\omega  \sim \omega '$.
Moreover, for a given permutation statistic ``st", if the following equinumerosity holds for all $n\ge 1$ and $k\ge 0$, 
$$|\{\pi \in \S_n(\omega ): \st(\pi)=k\}|=|\{\pi \in \S_n(\omega '): \st(\pi)=k\}|,$$ 
then we say that $\omega $ and $\omega '$ are \textit{$\st$-Wilf-equivalent} and write $\omega  \overset{\st}{\sim} \omega '$. According to these definitions, one sees that $\omega $ and $\omega '$ may be Wilf-equivalent without being $\st$-Wilf-equivalent. For example, $123$ and $321$ are not $\des$-Wilf-equivalent since $\des(123)=0$ and $\des(321)=2$, even though they are Wilf-equivalent, and both $\S_n(123)$ and $\S_n(321)$ are enumerated by the Catalan numbers, a classic result attributed to MacMahon~\cite{mac60} and Knuth~\cite{knu75}.

Let $A_n^{\omega}(t)=\sum_{\pi \in \S_n(\omega)} t^{\des(\pi)}$ be the generating function of the $\omega$-avoiding permutations $\pi$ of length $n\geq 0$, where $t$ keeps track of the number of descents in $\pi$, and we define
\begin{align*}
A^{\omega}(x,t) &:= \sum_{n\ge 0} A_n^{\omega}(t)\dfrac{x^n}{n!},\\
B^{\omega}(x,t) &:= \sum_{n\ge 0} A_n^{\omega}(t) x^n.
\end{align*} 
Note that $A_0^{\omega}(t)=1$ as we view the empty word as the only permutation of length $0$.

The \textit{reverse} of a permutation $\pi=\pi_1\pi_2\cdots \pi_n$ is the permutation $\pi^{r}=\pi_n\pi_{n-1}\cdots\pi_1$.
The \textit{complement} $\pi^{c}$ of $\pi$ is the permutation $\pi_1'\pi_2'\cdots\pi_n'$ where $\pi_i'=n+1-\pi_i$. By \textit{inverse} we mean the regular group theoretical inverse on permutations, that is, the $\pi_{i}$-th position of the inverse $\pi^{-1}$ in its one-line notation is occupied by $i$. The correspondences between $\pi$ and its reverse $\pi^r$, complement $\pi^c$, inverse $\pi^{-1}$ are called the trivial bijections (actually involutions) from $\S_n$ to itself,
denoted by $\phi _r$, $\phi_c$, and $\phi_{-1}$, respectively.
Also, note that $\phi_c$ and $\phi_r$ are commutative, i.e., $\phi_c\circ\phi_r=\phi_r\circ\phi_c $.
For convenience, we write $\pi^{rc}:=\phi_c(\phi_r(\pi))$.

%展示主要\UTF{7ED3}\UTF{8BBA}%
Let $\sigma$ be a consecutive pattern on $\{1,2,\ldots,k-1\}$ and let the pattern $p=\sigma k$,
so that $p$ is a quasi-consecutive pattern ending with the largest element. For example, if $\sigma=\underline{231} $ then $p=\underline{231}4$. The following theorem links the Wilf-equivalences involving $\sigma$ with those involving $p$, and it was derived by both Elizalde~\cite{eli06} and Kitaev~\cite{kit05}. 
\begin{theorem}\label{thm:eli-kit}
Suppose $\sigma\sim\sigma'$ are two Wilf-equivalent consecutive patterns of length $k-1$, then
$$\sigma k\sim\sigma'k.$$
\end{theorem}

It turns out that a more general result is true, since the descent distribution over $\S_n(\sigma)$ is directly related to the descent distribution over $\S_n(p)$ in the sense of the following theorem. This is the first main result of the paper and we refer to it as the ``Structure Theorem''.

\begin{theorem}[Structure Theorem]\label{structure thm}
    For any consecutive pattern $\sigma$ defining the quasi-consecutive pattern $p=\sigma k$, we have 
    \begin{align}
    \label{id:struc}
    A^p(x,t) = \frac{1}{t} (e^{t \int_{0}^{x} A^{\sigma}(y,t) \,dy}-1) +1.
    \end{align}
\end{theorem}


%文章\UTF{7ED3}\UTF{6784}%
The rest of this paper is organized as follows. Section~\ref{sec:structure theorem} is devoted to a proof of Theorem~\ref{structure thm}, which is then utilized to compute generating functions and complete the classification for quasi-consecutive patterns of length 3. A similar $\des$-Wilf classification for quasi-consecutive patterns of length 4 is presented in Section~\ref{sec:classifications of 4}. Then, in Section~\ref{sec:g.f. of 4}, two of the generating functions for length 4 quasi-consecutive patterns are derived explicitly via the ``generalized run theorem''. Finally, in Section~\ref{conclusion} we provide concluding remarks.


