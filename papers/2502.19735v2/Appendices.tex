\section{Prompt for CoT instance Construction}\label{sec:appendix A}

\subsection{Information extraction}
\begin{mybox}
\verb|[System]|

Assume you are an experienced linguist mastering multiple languages. Here is a user translation task, you need to review the input sentence and identify the key concepts that are crucial for understanding the translation. These concepts will guide the translation process.


\verb|[User]|

<translation task>

{\{Instruction-prompt Pair\}}

</translation task>
\end{mybox}

\subsection{Strategy Selection}
\begin{mybox}
\verb|[System]|

From the six available human translation CoT, select appropriate strategies for this translation task according to user provided key concepts :

a. Hierarchical Translation:

b. Triangular Translation:

c. Back Translation:

d. Context-aware Translation:

e. Translation Explanation:

f. Structural Transformation:

Finally, please formulate the translation CoT process according to selected strategies and key concepts in the position between <think> </think>.

\verb|[User]|

<key concepts>

{\{Extracted Information\}}

</key concepts>
\end{mybox}

\subsection{Incorporating CoTs into Parallel Corpus}

\begin{mybox}
\verb|[System]|



\verb|[User]|

\end{mybox}

\subsection{Translate CoT Refinement}

\begin{mybox}
A student is engaged in the task of translating an <source> into <target>.
This student constantly thinks about and optimizes his translation. The whole process is shown as follows:
<Translation Process>
{translation process}
</Translation Process>

The translation result is shown as follows:
<result>
{}
</result>
While the reference is:


we adopt Comet 
{\{Extracted Information\}}

</think>
\end{mybox}


\section{Human translation CoT templates}\label{sec:appendix B}
\begin{mybox}
[Hierarchical Translation]
<think>

1. Analyze the sentence structure and identify the core elements (subject, verb, object).

2. Translate the sentence from the origin language to the target language, focusing on the core elements.

3. Review the translation for basic accuracy and grammatical structure.

4. Identify areas that need further refinement (e.g., word choice, tense, or word order).

5. Modify the translation to improve fluency and coherence, considering the context.

6. Finalize the translation by ensuring it retains the original meaning while improving readability. 

</think>
\end{mybox}


\begin{mybox}
[Triangulating Translation]
<think>

1. Identify basic elements: Break down the sentence into its main components and identify the key subject, verb, and object.

2. Translate to intermediate language: Convert these elements into an intermediate language structure (e.g., simple syntactic rules or function names).

3. Refine back to target language: Translate from the intermediate language back to the target language, adjusting for syntactic norms and idiomatic expressions.

4. Check for accuracy: Ensure that the meaning is preserved in the translated sentence by checking noun-verb agreement and connectors.

5. Adjust word order: Modify word order to ensure that it aligns with the target language’s grammatical structure.

6. Final refinement: Review the translation for naturalness, idiomatic use, and overall flow.

</think>
\end{mybox}

\begin{mybox}
[Back Translation]
<think>

1. Analyze the provided context in the source language.

2. Translate the source text to the target language.

3. Perform back translation from the target language to the source language.

4. Compare the back translation with the original source context.

5. Evaluate whether the meaning of the back translation aligns with the original.

6. If discrepancies are identified, adjust the target language translation to enhance consistency with the original meaning.

7. Finalize the translation by ensuring both forward and back translations accurately align across all languages involved.

</think>
\end{mybox}

\begin{mybox}
[Context-aware Translation]
<think>

1. Analyze the current sentence, along with the previous sentences, to understand the overall conversation context.

2. Identify key elements like tone, formality, or subject matter based on the ongoing conversation.

3. Translate the sentence while ensuring that the translation is aligned with the tone, style, and subject of the preceding dialogue.

4. If any ambiguity exists in the translation due to context, refine the translation to better fit the conversation flow.

5. Verify that the translation maintains coherence with the larger conversation, ensuring consistency in language and tone.

6. Finalize the translation by cross-checking it with the conversation’s context to ensure it feels natural and appropriately aligned. 

</think>
\end{mybox}

\begin{mybox}
[Translation Explanation]
<think>

1. Analyze the source sentence and identify the key elements (verbs, subjects, objects, etc.).

2. Based on these elements, determine the most suitable translation strategy (literal vs. idiomatic).

3. Select the best translation for each word or phrase, considering context and language-specific structures.

4. Explain the rationale behind choosing specific words or phrases.

5. After completing the initial translation, review each translation decision and explain any adjustments made for fluency or accuracy.

6. Provide a final explanation for the translation choices, discussing any trade-offs made between literal meaning and contextual appropriateness.

</think>
\end{mybox}

\begin{mybox}
[Structural Transformation]
<think>

1. Analyze the sentence's syntactic structure in the source language (e.g., identify whether it’s active or passive).

2. Determine the most appropriate syntactic structure in the target language (e.g., whether it needs to be rephrased from active to passive or vice versa).

3. Adjust the word order and grammatical structure in the target language to match the sentence’s meaning, while maintaining clarity.

4. Translate the sentence, ensuring that subject-verb-object relationships and other syntactic elements align with target language norms.

5. After the translation, check the sentence's grammar and overall flow in the target language, making sure it is clear and fluid.

6. If the sentence feels awkward or unnatural, refine the structure by adjusting word choice or reordering components. 

</think>
\end{mybox}