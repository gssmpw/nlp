\documentclass[conference]{IEEEtran}
\IEEEoverridecommandlockouts
% The preceding line is only needed to identify funding in the first footnote. If that is unneeded, please comment it out.
\usepackage{cite}
\usepackage{amsmath,amssymb,amsfonts}
\usepackage{algorithmic}
\usepackage{graphicx}
\usepackage{textcomp}
\usepackage{xcolor}
\usepackage{multirow}
\usepackage{subfigure}
\usepackage{tabularx}
\usepackage{booktabs}
\def\BibTeX{{\rm B\kern-.05em{\sc i\kern-.025em b}\kern-.08em
    T\kern-.1667em\lower.7ex\hbox{E}\kern-.125emX}}
\begin{document}



\title{A Distillation-based Future-aware Graph Neural Network for Stock Trend Prediction\\
\thanks{* Corresponding author.}
}


\author{\IEEEauthorblockN{Zhipeng Liu}
\IEEEauthorblockA{\textit{Software college} \\
\textit{Northeastern University}\\
Shenyang, China\\
2310543@stu.neu.edu.cn
}
\and
\IEEEauthorblockN{Peibo Duan$^*$}
\IEEEauthorblockA{\textit{Software college} \\
\textit{Northeastern University}\\
Shenyang, China\\
sakuragiduan@gmail.com}
\and
\IEEEauthorblockN{Mingyang Geng}
\IEEEauthorblockA{\textit{Unit 63798} \\
Xichang, China \\
gengmingyang13@nudt.edu.cn}
\and
\IEEEauthorblockN{Bin Zhang}
\IEEEauthorblockA{\textit{Software college} \\
\textit{Northeastern University}\\
Shenyang, China \\
zhangbin@mail.neu.edu.cn}
}

\maketitle



\begin{abstract}
Stock trend prediction involves forecasting the future price movements by analyzing historical data and various market indicators. With the advancement of machine learning, graph neural networks (GNNs) have been extensively employed in stock prediction due to their powerful capability to capture spatiotemporal dependencies of stocks. However, despite the efforts of various GNN stock predictors to enhance predictive performance, the improvements remain limited, as they focus solely on analyzing historical spatiotemporal dependencies, overlooking the correlation between historical and future patterns. In this study, we propose a novel distillation-based future-aware GNN framework (DishFT-GNN) for stock trend prediction. Specifically, DishFT-GNN trains a teacher model and a student model, iteratively. The teacher model learns to capture the correlation between distribution shifts of historical and future data, which is then utilized as intermediate supervision to guide the student model to learn future-aware spatiotemporal embeddings for accurate prediction. Through extensive experiments on two real-world datasets, we verify the state-of-the-art performance of DishFT-GNN.
\end{abstract}

\begin{IEEEkeywords}
Stock prediction, Data mining, Future fusion, Spatiotemporal forecasting
\end{IEEEkeywords}


\section{Introduction and Related work}
\label{sec:intro}
The stock market is a financial investment platform where numerous corporations and investors engage in trading\cite{hu2015application,liu2024dynamic}. As of 2022, the global market capitalization increased from \$54.6 trillion to \$94.69 trillion over the past decade\footnote{  https://data.worldbank.org/indicator/CM.MKT.LCAP.CD/}. With the advancement of artificial intelligence, deep learning (DL) techniques provide more opportunities for investors to increase their wealth through stock investment\cite{deep_stock_predict1,htun2023survey}. For an auxiliary investment tool, it is non-trivial to make profitable investment decisions and trading strategies within the volatile market\cite{ding2024trend}. 

\begin{figure}
  \centering
  \includegraphics[scale=.23]{intro.png}
  \caption{An example shows that events, such as a corporation’s annual report, cause sudden structural changes between historical and future stock data.}
  \label{intro}
\end{figure}
Typically, DL-based stock trading methods are based on the fundamental hypothesis that a stock's \textbf{future dynamics} can be revealed from its \textbf{historical patterns}\cite{trans_sp2,trans_sp3,hist1,hist2,wang2022adaptive}. Thus, it is essential to perform a meticulous analysis of latent temporal dynamics within historical stock price indicators, as exemplified by methods based on recurrent neural networks (RNN) \cite{2015_lstmsp_chen,2018_lstmsp_Fischer}, as well as its variants LSTM or GRU \cite{2017_DARNN_Qin,2017_attention_Vaswani,2019_alstm_feng}. Drawing upon the temporal dynamic feature extraction capabilities of RNN-based methodologies, graph neural networks (GNNs) construct stock graphs that facilitate the incorporation of explicitly relational (spatial) dependencies between stocks, such as industry and shareholding information \cite{TGC,ye_2021_multi,hist,qian2024mdgnn}. In this framework, the nodes denote individual stocks with attributes derived from the aforementioned RNN-based models, while the edges represent the inter-stock relationships. Recent studies have found that relying solely on explicit relationships can lead to biased and incomplete aggregation of relational features. As a result, there is growing interest in investigating implicit relationships using DL-based graph representation methods to better understand stock interactions\cite{ADGAT,2023_mgar_song,vgnn,yan2024framework,wang2024towards}.
 
%Although these methods demonstrate significant efficacy in capturing the dependencies of temporal dynamics across various time intervals, they are deficient in distinguishing the differential dependencies inherent to distinct time intervals. Consequently, researchers have improved these models by incorporating attention mechanisms to further enhance prediction accuracy \cite{2017_DARNN_Qin,2017_attention_Vaswani,2019_alstm_feng}.
%bridging the gap that merely view stocks as isolated entities, failing to capture impact exerted by other stocks\cite{2018_gcnsp_chen,wang2021review,you2024multi,daiya2024diffstock,ma2024vgc}However, they merely view stocks as isolated entities, failing to capture impact exerted by other stocks\cite{2018_gcnsp_chen,wang2021review,you2024multi,daiya2024diffstock,ma2024vgc}. 
%Drawing upon the temporal dynamic feature extraction capabilities of RNN-based methodologies, graph neural networks (GNNs) construct stock graphs that facilitate the incorporation of explicitly relational dependencies between stocks, such as industry and shareholding information \cite{TGC,ye_2021_multi,hist,qian2024mdgnn}. In this framework, the nodes denote individual stocks with attributes derived from the aforementioned RNN-based models, while the edges represent the inter-stock relationships. Recent studies have found that relying solely on explicit relationships can lead to biased and incomplete aggregation of relational features. As a result, there is growing interest in investigating implicit relationships using DL-based graph representation methods to better understand stock interactions \cite{ADGAT,2023_mgar_song,vgnn,yan2024framework}.

% For instance, early approaches establish explicit stock graphs on the basis of expert domain knowledge \cite{TGC,ye_2021_multi,hist,qian2024mdgnn}, such as industry and shareholding information. However, recent studies have found that relying solely on explicit relationships can lead to biased and incomplete aggregation of relational features. As a result, there is growing interest in investigating implicit relationships using DL-based graph representation methods to better understand stock interactions\cite{ADGAT,2023_mgar_song,vgnn,yan2024framework}.


% Thus, representative deep models for addressing time series, including convolutional neural networks (CNNs), recurrent neural networks (RNNs), and Transformers, are widely employed in quantitative trading. However, in the complex network of the stock market, stocks are not independent entities. The fluctuation of a single stock can lead to a chain reaction, influencing the prices of other stocks. Fortunately, contemporary approaches address this gap by leveraging graph neural networks (GNNs) to model the interactions between stocks. In a stock graph, each node represents an individual stock, while the edges connecting nodes represents the relationships between both stocks. 


% Stock graph construction can be classified into two categories: explicit and implicit stock graphs. Early approaches establish explicit stock graphs on the basis of expert domain knowledge, such as industry and shareholding information. However, recent studies have found that relying solely on explicit relationships can lead to biased and incomplete aggregation of relational features. As a result, there is growing interest in investigating implicit relationships using deep learning-based graph representation methods to better understand stock interactions.


Going beyond the above mentioned studies, we further observe that stock data exhibits highly non-stationary characteristics, making it more challenging to forecast compared to other time series data. As shown in Fig.\ref{intro}, an annual report from a corporation showing better-than-expected profits can cause distribution shift between historical and future data, which results in sub-optimal performance in contemporary stock predictors that rely exclusively on historical price indicators, with their impact becoming apparent only in hindsight. Thus, we are motivated to believe that not only are the features in historical stock data essential for analysis, but also \textbf{the correlation between distribution shifts of historical and future data}. For further proof, please refer to the supplementary material provided due to space constraints\footnote{https://github.com/ZhupengLiu/DisFT-GNN/blob/main/Supplementary\%20\\Material.pdf}.


To this end, we propose a novel general \textbf{Dis}tillation-based \textbf{F}u\textbf{T}ure-aware \textbf{GNN} framework (\textbf{DisFT-GNN}), to capture the correlation between historical and future patterns for stock trend prediction. In theory, DisFT-GNN can be integrated with any GNN-based stock predictor, enhancing their predictive performance. The contributions in this study are summarized as follows:\textbf{ i)} Unlike existing knowledge distillation-based time series forecasting methods\cite{distill1,distill2}, which utilize the teacher model to produce historical pattern-related soft labels or features for guiding the training of a lightweight student model, DisFT-GNN introduces a novel teacher model that has undergone converged training to generate high-level, future-aware representations, which serve as intermediate supervision, guiding the student model in learning correlations between historical and future distributions. \textbf{ii) }As the future prices can be influenced by various factors and exhibit different patterns. We introduce a novel attention-based multi-channel feature fusion method in the teacher model, which models the diverse distribution shifts between historical and future data. \textbf{iii)} Through extensive experiments on two real-world datasets from the American stock market, our proposed DisFT-GNN consistently boosts current state-of-the-art models, achieving an average improvement of up to 5.41\%.
    
    % To enable the model to capture comprehensive historical-feature correlations, we propose a novel distillation-based framework, DisFT-GNN, which employs a teacher model trained to generate future-aware spatio-temporal representations to supervise the learning of the student model. It is noted that no future information is leaked during the back-testing stage.

    
    
    

    
    
    % We propose a novel attention-based multi-channel feature fusion method that integrates future trend embeddings with historical spatio-temporal embeddings to generate high-level, future-aware spatio-temporal representations. 

    







% \noindent To tackle these challenges, we propose a novel \textbf{F}uture \textbf{T}rend-aware \textbf{GNN} method, namely \textbf{FTGNN}, utilizing knowledge distillation technique to predict whether stocks will achieve expected returns for stock investment. FTGNN consists of two training processes: training the teacher model and the student model separately. (1) future trend-aware teacher training: We devise a future trend-aware spatio-temporal encoder that encodes historical price indicators and future trend knowledge into


% acquire key future trend information, we employ a knowledge distillation framework where a teacher model, trained on historical indicators and future trend information, guides the training of a student model.  




\section{Problem Formulation}
We formulate stock trend prediction as a binary node classification task. A dynamic stock graph at trading day $t$ can be defined as $\mathcal{G}^t=\{\mathcal{V},\mathbf{X}^{[t-L+1,t]},\mathbf{A}^t\}$, where $\mathcal{V}=\{v_1,v_2,...,v_N\}$ is the set of $N$ individual stocks, $\mathbf{X}^{[t-L+1,t]} \in \mathbb{R}^{N \times L \times M}$ represents $M$ price indicators over the historical $L$ trading days. $\mathbf{A}^t\in \mathbb{R}^{N\times N}$ is the normalized relation matrix representing the relationship between stocks. Mathematically, the problem is formulated as:
\begin{equation}
    \hat{\mathbf{y}}^{[t+1,t+T]}=f(g(\mathbf{X}^{[t-L+1,t]},\mathbf{A}^t;\Theta_g);\Theta_f).
    \label{eq: problem formulation}
\end{equation}

\noindent In \eqref{eq: problem formulation}, $g(\cdot)$ is a GNN-based model with parameters $\Theta_g$, aiming to capture spatiotemporal features between stocks; $f(\cdot)$ is the prediction layer with parameters $\Theta_f$; $\hat{\mathbf{y}}^{[t+1,t+T]} = \{\hat{\mathbf{y}}_1^{[t+1,t+T]},\hat{\mathbf{y}}_2^{[t+1,t+T]},..., \hat{\mathbf{y}}_N^{[t+1,t+T]} \}$ is the set of output binary variables. For $\forall \tau \in [t+1,t+T]$, $\hat{y}_n^{\tau} = 1$ predicts that the $n$-th stock will increase, and 0 otherwise. For $\forall v_n\in \mathcal{V}$, the ground-truth label is defined as follows:
\begin{equation}
    y_n^{\tau}=
    \begin{cases}
        1& \text{if }\frac{p_n^\tau-p_n^{t}}{p_n^{t}}>\delta\\
        0& \text{else }
    \end{cases},
\end{equation}
\noindent where $p_n^ \tau\in \mathbf{x}_n^\tau$ is the stock close price at trading day $\tau$, $\delta$ is the hyperparameter. Notably, when $T=1$ and $\delta=0$, the problem is the next-trading day stock trend prediction \cite{you2024multi}.



\begin{figure}
  \centering
  \includegraphics[scale=.27]{framework.png}
  \caption{The framework of DishFT-GNN}
  \label{framework}
\end{figure}

\section{Methodology}
\subsection{Overall Architecture}
The overview of DisFT-GNN is illustrated in Fig.\ref{framework}. DisFT-GNN contains two training processes: training a teacher model and a student model, respectively. First, the teacher model encodes the historical stock graph and future trend information, generating historical spatiotemporal and high-level future embeddings. Subsequently, a novel attention-based multi-channel feature fusion method is proposed to integrate these embeddings to generate future-aware spatiotemporal representations, which depict the diverse historical-future distribution shifts. The representations are then fed into the prediction module to obtain predicted results, which are compared with the ground truth to optimize the teacher model. Subsequently, the future-aware spatiotemporal representations from the teacher model serve as intermediate supervision to guide the student model in learning historical-future distribution correlations. 


Before introducing the details of the proposed method, we first introduce several general components for stock prediction. 


\noindent{\textbf{Spatiotemporal GNN module.}} Existing stock prediction methods follow a fundamental hypothesis that a stock's future prices can be inferred from its historical patterns (temporal dependence)\cite{2017_DARNN_Qin} and the behavior of related stocks (spatial dependence)\cite{ADGAT}. For $\forall v_n\in\mathcal{V}$, the spatiotemporal embeddings can be calculated as $\mathbf{h}_n^t=ST(\mathbf{X}_n^{[t-L+1,t]}, \mathbf{A}^t)$, specifically,
\begin{equation}
    \begin{aligned}
\mathbf{s}_n^t&=\text{Temporal}(\mathbf{X}_n^{[t-L+1,t]}),\\
\mathbf{h}_n^t&=\text{Spatial}(\mathbf{s}_k^t|v_k\in \mathcal{N}_n),
    \end{aligned} 
\end{equation}

\noindent where $\mathcal{N}_n$ denotes the set of stocks related to $v_n$. It is noted that $ST(\cdot)$ can be any spatiotemporal GNN model.

\noindent{\textbf{Future Trend Encoder.}} It is a feed-forward network (FNN) to encode the stock future trends, generating a novel future trend embedding, ${\mathbf{q}}_n^{t+}$. Since future trend information can intuitively reflect prediction results, it is unnecessary to design intricate modules to analyze future price indicators.

\begin{equation}
    {\mathbf{q}}_n^{t+}=\text{ReLU}(\text{FFN}(\mathbf{f}_n^{[t:t+T)})),
\end{equation}
\noindent where $\mathbf{f}_n^{[t:t+T)}\in\{0, 1\}$ represents the future trend of $v_n$ over the following $T$ trading days. 


% \noindent{\textbf{Prediction module.}} It is utilized to forecast whether $v_n$ will achieve the expected return, specifically,
% \begin{equation}
%     \hat{\mathbf{y}}_n=\text{Prediction}(\mathbf{h}_n^t)=\text{Softmax}(\text{FNN}(\mathbf{h}_n^{t+}))
% \end{equation}

% \noindent where $\mathbf{h}_n^t$ is the input embedding, $\hat{\mathbf{y}}_n^{t+}\in \{0,1\}$ is the predicted result and the softmax function generates a probability distribution over classes. 









% Secondly, to allow the student model to infer future trend knowledge based on historical stock price indicators, the future trend-aware spatio-temporal representation distilled from the teacher model supervises the training of the student model. The student model shares the same parameters of the prediction module with the teacher model, and is trained to minimize both the future trend-aware loss and the prediction loss, which can inherit the capability from the teacher model to achieve better prediction.




\subsection{Future-aware Teacher Model Training}
\subsubsection{Future-aware Spatiotemporal Encoding}
First, the teacher model utilizes a future trend encoder that encodes stock future trends into novel high-level embeddings, ${\mathbf{q}}_n^{t+}=\text{ReLU}(\text{FFN}(\mathbf{f}_n^{[t:t+T)})) $, ${\mathbf{q}}_n^{t+}\in \mathbb{R}^{D_f}$, and a spatiotemporal GNN module generating historical spatiotemporal embeddings, $\mathbf{p}_n^{t}=ST_{(T)}(\mathbf{X}_n^{[t-L+1,t]}, \mathbf{A}^t)$, $\mathbf{p}_n^{t}\in \mathbb{R}^{D_p}$. Subsequently, to model the diverse historical-future distribution shifts, we propose a novel attention-based multi-channel fusion method, integrating ${\mathbf{q}}_n^{t+}$ and $\mathbf{p}_n^t$ into high-level future-aware spatiotemporal representations, $\mathbf{h}_n^{t+}\in \mathbb{R}^D$.

Specifically, we perform multiple vector-matrix-vector (VMV) multiplications \cite{ADGAT} to generate a multi-channel result, with each channel representing a potential correlation between historical and future distributions, $V=\mathbf{p}_n^t \mathcal{F}^{[1:{D}]} \mathbf{q}_n^{t+}$, where $\mathcal{F}^{[1:{D}]} \in \mathbb{R}^{D\times D_p \times D_f}$ is the trainable parameter, and $D$ is the number of channels. Next, we utilize the attention mechanism to assess the importance of each channel by assigning an attention score, i.e., $Q={\mathbf{q}}_n^{t+} W_Q$, $K=\mathbf{p}_n^t W_K$. Finally, similar to Transformer\cite{vaswani2017attention}, we conduct a scaled dot-product attention operation for feature fusion, which is formulated as,
\begin{equation}
\mathbf{h}_n^{t+}=\text{Attention}(Q,K,V)=\text{Softmax}(\tau\frac{QK^T}{\sqrt{K_d}})V,
\end{equation}

\noindent where Softmax($\cdot$) is the softmax function, ${K_d}$ is the scaling factor and $\tau$ is  the attention temperature coeffcient.



\subsubsection{Prediction and Optimization}
After obtaining $\mathbf{h}_n^{t+}$, to forecast the stock trend, $\mathbf{h}_n^{t+}$ is fed into the prediction module, which is a feed-forward network with a softmax function.
\begin{equation}
    \hat{\mathbf{y}}_n^{[t+1,t+T]}=\text{Prediction}(\mathbf{h}_n^{t+})=\text{Softmax}(\text{FNN}(\mathbf{h}_n^{t+}))
\end{equation}

\noindent where $\hat{\mathbf{y}}_n^{[t+1,t+T]}\in \{0,1\}$ is the predicted result and the softmax function generates a probability distribution over classes. 

Finally, parameters are learned by minimizing the cross entropy loss (CEL), $\mathcal{L}_p=\text{CEL}({\mathbf{y}}_n^{[t+1,t+T]},\hat{\mathbf{y}}_n^{[t+1,t+T]})$.





% \noindent{\textbf{Spatio-temporal GNN Module}}. To capture spatio-temporal pattern for each stock, for $\forall v_n\in\mathcal{V}$, we employ LSTM and GCN to obtain the historical spatio-temporal embedding $\mathbf{h}_n^t=ST_T(\mathbf{X}_n^{[t-L+1,t]}, \mathbf{A}^t)$, specifically,

% \begin{equation}
%     \begin{aligned}
% \mathbf{s}_n^t&=\text{LSTM}_{(T)}(\mathbf{X}_n^{[t-L+1,t]}),\\
% \mathbf{r}_n^t&=\text{AGG}_{(T)}(\mathbf{s}_k^t|v_k\in \mathcal{N}_n),\\
% \mathbf{h}_n^t&=\mathbf{s}_n^t+\alpha \mathbf{r}_n^t.
%     \end{aligned} 
% \end{equation}


% \noindent Here, $\mathbf{s}_n^t\in \mathbb{R}^{F}$ and $\mathbf{r}_n^t\in \mathbb{R}^{F}$ are the temporal and spatial embedding of $v_n$, respectively. $F$ is the embedding dimension. $\mathcal{N}_n$ denotes the set of stocks related to $v_n$. We utilize a simple but effective fusion method, adding $\mathbf{s}_n^t$ and $\mathbf{r}_n^t$ to obtain $\mathbf{h}_n^t\in \mathbb{R}^{F}$. $\alpha$ is the trainable parameter, making a tradeoff between both embeddings. Notably, LSTM and GCN can be replaced by other temporal models (e.g., Transformer) or GNN models (e.g., GraphSAGE), respectively.



% \noindent{\textbf{Future Trend Encoder.}} We employ a feed-forward network (FNN) to encode the stock future trends, generating a novel high-level embedding, ${\mathbf{q}}_n^{t+}\in \mathbb{R}^{D_f}$, 

% \begin{equation}
%     {\mathbf{q}}_n^{t+}=\text{ReLU}(\text{FFN}(\mathbf{f}_n^{[t:t+T)})),
% \end{equation}
% \noindent where $\mathbf{f}_n^{[t:t+T)}\in\{0, 1\}$ represents the future trend of $v_n$ over the following $T$ trading days. Since future trend information can intuitively reflect expected returns, it is unnecessary to design intricate modules to analyze future price indicators.
% \vspace{.3em}

% \noindent{\textbf{Future Trend-aware Spatio-temporal Encoding.}}
% To integrate ${\mathbf{q}}_n^{t+}$ and $\mathbf{p}_n^t$ into a high-level future trend-aware spatio-temporal representation, we propose a novel attention-based tensor fusion method. Concretely, we first map $\mathbf{q}_n^{t+}$ into query space and $\mathbf{p}_n^t$ into key space, i.e., $Q={\mathbf{q}}_n^{t+} W_Q$, $K=\mathbf{p}_n^t W_K$. To capture the element-level association between ${\mathbf{q}}_n^{t+}$ and $\mathbf{p}_n^t$, we perform $D$ vector-matrix-vector
% (VMV) multiplications, mapping them into value space, $V=\mathbf{p}_n^t \mathcal{F}^{[1:{D}]} \mathbf{q}_n^{t+}$, $\mathcal{F}^{[1:{D}]} \in \mathbb{R}^{D_\times D_p \times D_f}$.  Finally, We conduct a scaled dot-product attention operation for feature fusion, generating a novel future trend-aware spatio-temporal representation, $\mathbf{h}_n^{t+}\in \mathbb{R}^{D}$.


% \begin{equation}
% \mathbf{h}_n^{t+}=\text{Attention}(Q,K,V)=\text{Softmax}(\tau\frac{QK^T}{\sqrt{K_d}})V,
% \end{equation}

% \noindent Softmax($\cdot$) is the softmax function, ${K_d}$ is the scaling factor and $\tau$ is  the attention temperature coeffcient.
% \vspace{.3em}


% , specifically, 
% \begin{equation}
%     \mathcal{L}_p=-\sum_{n=1}^N({y}_n^{t+}\log(\hat{\mathbf{y}}_n^{t+})+(1-{y}_n^{t+})\log(1-\hat{\mathbf{y}}_n^{t+}))
% \end{equation}




\subsection{Distillation-based Student Model Training}
To enable the student model with the capability to deduct the future patterns based on historical spatiotemporal embeddings, we utilize the teacher model, which has undergone converged training, to supervise the training of the student model, which includes two phases. 


First, the student model encodes the historical stock graph, generating historical spatiotemporal representations, $\tilde{\mathbf{h}}_n^{t}=ST_{(S)}(\mathbf{X}_n^{[t-L+1,t]}, \mathbf{A}^t)$. Subsequently, $\tilde{\mathbf{h}}_n^t$ is fed into the shared prediction module with the same parameters as the teacher model for prediction,

\begin{equation}
    \hat{\mathbf{y}}_n^{[t+1,t+T]}=\text{Prediction}(\tilde{\mathbf{h}}_n^t).
\end{equation}





Second, to infer future patterns based on $\tilde{\mathbf{h}}^{t}_n$, the student model is trained by minimizing the distillation loss between $\tilde{\mathbf{h}}^{t}_n$ and $\mathbf{h}_n^{t+}$: $\mathcal{L}_{d}=\text{MSE}(\tilde{\mathbf{h}}^{t}_n,\mathbf{h}_n^{t+})$, where MSE($\cdot$) is the mean square error function and $\mathbf{h}_n^{t+}$ is the future-aware spatiotemporal representation distilled from the teacher model. Note that $\mathbf{h}_n^{t+}$ and $\tilde{\mathbf{h}}^{t}_n$ have the same dimensions. However, since the representations are highly non-linear, it is necessary to focus more on the nonlinear dependence between $\tilde{\mathbf{h}}^{t}_n$ and $\mathbf{h}_n^{t+}$. Thus, the Hilbert-Schmidt Independence Criterion (HSIC)\cite{fan2023generalizing,nag2021graphvicreghsic} is utilized as the distillation loss,

\begin{equation}
    \mathcal{L}_{d}=\text{HSIC}(\tilde{\mathbf{h}}^{t}_n,\mathbf{h}_n^{t+})={(D-1)}^{-2}tr(\mathbf{K}\mathbf{H}\mathbf{L}\mathbf{H}),
\end{equation}
\noindent where $\mathbf{K},\mathbf{L}\in\mathbb{R}^{D\times D}$ are kernel matrices, and $\mathbf{K}_{i,j}=k(\tilde{\mathbf{h}}^{t}_i,\tilde{\mathbf{h}}^{t}_j)$, $\mathbf{L}_{i,j}=l(\tilde{\mathbf{h}}^{t}_i,\tilde{\mathbf{h}}^{t}_j)$. $\mathbf{H}=\mathbf{I}-D^{-1}\mathbf{11}^T$, where $\mathbf{I}$ is an identity matrix and $\mathbf{1}$ is an all-one column vector. 

The final objective of the student model is to generate future patterns based on the historical stock data while forecasting the stock trends as accurately as possible. Specifically,
\begin{equation}
    \mathcal{L}=\mathcal{L}_p+\lambda \mathcal{L}_d,
\end{equation}
\noindent where $\mathcal{L}_p=\text{CEL}({\mathbf{y}}_n^{[t+1,t+T]},\hat{\mathbf{y}}_n^{[t+1,t+T]})$, and $\lambda$ is the hyperparameter striking a balance between two terms. 

% We utilize the student model for predictions during back-testing stage to prevent any leakage of future information.




\section{experiments}
\subsection{Dataset and Experimental Setting}
\subsubsection{Datasets} To verify the effectiveness of the proposed DishFT-GNN, we conduct extensive experiments using two datasets from American stock indices, i.e., S\&P 100 and NASDAQ 100, spanning from January 1, 2019, to September 30, 2023. All the datasets are divided into three parts: 85\% for training, 7.5\% for validation and 7.5\% for testing. To ensure that both datasets contain continuous trading records, we eliminate stocks with missing data, such as those affected by suspensions. Thus, 96 and 94 stocks are selected from S\&P 100 and NASDAQ 100, respectively. Additionally, industry data is utilized to represent explicit stock relationships. 







\subsubsection{Experimental Setting} 

To assess the performance of DishFT-GNN, we conduct the experiments with seven GNN stock prediction methods for comparison, i.e., GCN\cite{kipf2016semi}, GAT\cite{velickovic2017graph}, TGC\cite{TGC}, ADGAT\cite{ADGAT}, MGAR\cite{2023_mgar_song}, VGNN\cite{vgnn} and MDGNN\cite{qian2024mdgnn}.  DishFT-GNN is integrated with each stock predictor to boost predictive performance. Following previous studies\cite{Macro-Sector-Micro,pen}, we use accuracy (ACC) and Matthews correlation coefficient (MCC) as two metrics.

Parameters of all models are trained using Adam optimizer\cite{kingma2014adam} on a single NVIDIA RTX 4070Ti GPU. In our experiments, $T$ and $\delta$ are set to 20 and 4\%, respectively. $\tau$ is set to 0.5 The learning rate is set to 5e-4 and the batch size is set to 64. We independently repeated each experiment five times and reported the mean and standard deviation.



% As mentioned in Section \ref{sec:intro}, since the reason of missing ``future" information in multi-step prediction, their investment returns are lower than those of single-step prediction. For the above trend prediction methods, we present the investment results of the single-step prediction due to space limitation. 

\begin{table}[t]
\scriptsize
    \centering
    \caption{Performance Comparison with Baselines.}
    \begin{tabular}{ccccc}
    \toprule[1pt]
\multirow{2}{*}{}
   \multirow{2}{*}{Methods} &\multicolumn{2}{c}{S\&P 100}&\multicolumn{2}{c}{NASDAQ 100}\\
   \cmidrule(lr){2-3}\cmidrule(lr){4-5}
    & ACC(\%) & MCC & ACC(\%) & MCC \\
    \midrule[.5pt]
        GCN&52.12$\pm$1.57&0.050$\pm$0.008& 51.97$\pm$0.84&0.026$\pm$0.005\\
   +Dish-FT&\textbf{55.47}$\pm$\textbf{1.18}&\textbf{0.159}$\pm$\textbf{0.035}&\textbf{53.87}$\pm$\textbf{1.26}&\textbf{0.093}$\pm$\textbf{0.013}\\
    \midrule[.5pt]
        GAT&51.81$\pm$1.25&0.047$\pm$0.011&52.14$\pm$1.34&0.048$\pm$0.008\\
   +Dish-FT&\textbf{55.45}$\pm$\textbf{1.23}&\textbf{0.092}$\pm$\textbf{0.013}&\textbf{54.38}$\pm$\textbf{1.27}&\textbf{0.097}$\pm$\textbf{0.017}\\
    \midrule[.5pt]
        TGC&52.84$\pm$0.85&0.059$\pm$0.007&53.61$\pm$1.30&0.082$\pm$0.016\\
   +Dish-FT&\textbf{56.69}$\pm$\textbf{0.78}&\textbf{0.137}$\pm$\textbf{0.012}&\textbf{57.13}$\pm$\textbf{1.21}&\textbf{0.140}$\pm$\textbf{0.043}\\
    \midrule[.5pt]
      ADGAT&53.46$\pm$1.34&0.077$\pm$0.023& 53.39$\pm$1.60&0.063$\pm$0.012\\
   +Dish-FT&\textbf{57.71}$\pm$\textbf{1.62}&\textbf{0.167}$\pm$\textbf{0.024}& \textbf{57.82}$\pm$\textbf{2.86}&\textbf{0.145}$\pm$\textbf{0.008}\\
    \midrule[.5pt]
       MGAR&52.93$\pm$0.62&0.051$\pm$0.008& 52.78$\pm$1.34&0.079$\pm$0.021\\
   +Dish-FT&\textbf{56.26}$\pm$\textbf{1.59}&\textbf{0.145}$\pm$\textbf{0.027}& \textbf{58.19}$\pm$\textbf{1.87}&\textbf{0.162}$\pm$\textbf{0.037}\\
    \midrule[.5pt]
       VGNN&53.73$\pm$1.48&0.075$\pm$0.014& 53.91$\pm$1.48&0.079$\pm$0.025\\
   +Dish-FT&\textbf{58.31}$\pm$\textbf{1.34}&\textbf{0.160}$\pm$\textbf{0.015}& \textbf{58.26}$\pm$\textbf{1.27}&\textbf{0.156}$\pm$\textbf{0.044}\\
    \midrule[.5pt]
      MDGNN&52.52$\pm$0.36&0.051$\pm$0.012& 52.84$\pm$1.19&0.058$\pm$0.018\\
   +Dish-FT&\textbf{55.68}$\pm$\textbf{0.55}&\textbf{0.093}$\pm$\textbf{0.018}& \textbf{56.53}$\pm$\textbf{1.05}&\textbf{0.127}$\pm$\textbf{0.013}\\
    
\bottomrule[1pt]
    \end{tabular}
    \label{result}
\end{table}



\subsection{Results and Analysis}


Table \ref{result} illustrates the comparison results, where each row is divided into two sections: the upper section displays the predictive performance of the original baseline methods, while the lower section presents the performance achieved after applying the proposed DishFT-GNN. Additionally, we apply a t-test at a significance level of $\alpha=0.01$ to validate the reliability and reproducibility of DishFT-GNN\cite{boneau1960effects}. We observe that DishFT-GNN significantly ($p<0.01$) enhances predictive performance compared to all baseline models. Specifically, it achieves an accuracy improvement over 2\% in most cases and up to 5.41\% when integrating MGAR, and an MCC improvement of 82.3\% to 218\% across two datasets.

Additionally, to evaluate whether the proposed DishFT-GNN improves model profitability, we use the classic GCN as the backbone and compare the investment profits with and without DishFT-GNN for back-testing, which is illustrated in Fig.\ref{Backtesting}. It is obvious that the profit gap between two methods gradually widens over time, which further verify the effectiveness of DishFT-GNN.




Previous studies rely solely on historical stock knowledge, which is insufficient for forecasting future outcomes. Despite capturing comprehensive historical spatiotemporal dependencies, these models can only achieve sub-optimal improvements. Since historical data only reflect past stock states, analyzing them alone may fail to capture future stock dynamics in a volatile market environment. Therefore, based on the motivation and theoretical proof in Section \ref{sec:intro}, the proposed DishFT-GNN utilizes the teacher model to explore the correlations between historical-future distribution shifts and supervises the student model's training, which enables the student model to analyze future stock patterns based on historical spatiotemporal knowledge, effectively alleviate the problem of the model's inability to establish historical-future associations. Notably, during the back-testing stage, only the student model is deployed, ensuring that no future information is leaked.








\begin{figure}[t]
  \scriptsize
  \subfigure[S\&P 100.]{\includegraphics[width=0.22\textwidth]{sp.png}}
\hspace{0.01cm}
  \subfigure[NASDAQ 100.]{\includegraphics[width=0.22\textwidth]{nasdaq.png}}
  \caption{Back-testing illustration.}
  \label{Backtesting}
\end{figure}



\begin{table}[t]
\scriptsize
    \centering
    \caption{Ablation study.}
    \begin{tabularx}{.42\textwidth}{cc|cc|cc}
    \toprule[1pt]
   \multicolumn{2}{c}{\multirow{2}{*}{Methods}} &\multicolumn{2}{c}{S\&P 100} &\multicolumn{2}{c}{NASDAQ 100} \\
   \cmidrule(lr){3-4}\cmidrule(lr){5-6}
   && ACC(\%)& MCC&ACC(\%)& MCC \\
   \midrule[.5pt]
   \multirow{3}{*}{\rotatebox{90}{GCN}}
   &DishFT-GNN w/o H&54.38&0.082&52.98&0.071\\
   &DishFT-GNN w/o F&54.78&0.075&52.18&0.092\\
   &DishFT-GNN &\textbf{55.47}&\textbf{0.159}&\textbf{53.87}&\textbf{0.093}\\
    \midrule[.5pt]
   \multirow{3}{*}{\rotatebox{90}{GAT}}
   &DishFT-GNN w/o H&53.81&0.058&52.42&0.063\\
   &DishFT-GNN w/o F&54.54&0.072&53.45&0.052\\
   &DishFT-GNN &\textbf{55.45}&\textbf{0.092}&\textbf{54.38}&\textbf{0.097}\\
       \midrule[.5pt]
   \multirow{3}{*}{\rotatebox{90}{VGNN}}
   &DishFT-GNN w/o H&55.34&0.108&56.21&0.115\\
   &DishFT-GNN w/o F&56.17&0.123&57.28&0.137\\
   &DishFT-GNN &\textbf{58.31}&\textbf{0.160}&\textbf{58.26}&\textbf{0.156}\\
    \midrule[.5pt]
   \multirow{3}{*}{\rotatebox{90}{MDGNN}}
   &DishFT-GNN w/o H&54.72&0.088&54.51&0.074\\
   &DishFT-GNN w/o F&55.68&0.076&55.04&0.091\\
   &DishFT-GNN &\textbf{55.68}&\textbf{0.093}&\textbf{56.53}&\textbf{0.127}\\
\bottomrule[1pt]
    \end{tabularx}
    \label{ablation}
\end{table}

\subsection{Ablation Study}
To investigate the contributions of each component, we conduct the ablation study on two classic baselines, GCN and GAT, and two state-of-the-art baselines, VGNN and MDGNN.

\begin{itemize}
    \item \textbf{DishFT-GNN w/o H:} It replaces HSIC with MSE.
    \item \textbf{DishFT-GNN w/o F:} It replaces attention-based multi-channel feature fusion module with simple concatenation.
    \item \textbf{DishFT-GNN:} Vanilla DishFT-GNN framework.
\end{itemize}

\noindent The ablation results are illustrated in Table \ref{ablation}. We can observe that all components positively impact the performance of DishFT-GNN in terms of ACC and MCC. First, HSIC aids the student model in learning a broader range of nonlinear features. Moreover, the attention-based multi-channel fusion method effectively models diverse historical-future distribution shifts, leading to improved predictive performance.




\section{conclusion}



We propose a novel distillation-based framework named DishFT-GNN for stock trend prediction. To address the issue that contemporary stock prediction methods fail to extract rich future patterns from historical stock knowledge, we introduced a teacher model trained to generate informative historical-future correlations, supervising the student model's learning. Specifically, the teacher model first extracted historical spatiotemporal and future embeddings. We then proposed a novel attention-based multi-channel fusion method to mine the association between historical and future distributions by integrating both embeddings. Finally, the fused representation was then utilized as intermediate supervision to guide the student GNN to learn future-aware spatiotemporal representations for accurate prediction. Extensive evaluations on real-world data showcase the effectiveness of our proposed method.


\section{Acknowledgements}
This study was supported by the Department of Science and Technology of Liaoning province (02110076523001), and Northeastern University, Shenyang, China (02110022124005).


\bibliographystyle{IEEEbib}
\bibliography{refs}

\end{document}
